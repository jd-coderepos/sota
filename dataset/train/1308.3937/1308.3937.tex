













\documentclass[runningheads,a4paper]{llncs}

\usepackage{eso-pic}
\AddToShipoutPicture*{
\put(133,160){
\scriptsize
Proceedings of WLPE 2013, {\tt arXiv:1308.2055}, August 2013.
}}



\usepackage{rotating}
\usepackage{amssymb}
\usepackage{amsmath}
\usepackage{url}
\usepackage[all]{xy}
\usepackage{color}
\usepackage{rotating}
\usepackage{listings}

\usepackage{multicol}
\usepackage{subfig}


\usepackage{cancel}
\lstset{language=Prolog}
\renewcommand{\labelitemi}{}



\makeatletter
\newcommand{\xRightarrow}[2][]{\ext@arrow 0359\Rightarrowfill@{#1}{#2}}
\makeatother


\usepackage{graphics}
\usepackage{graphicx}
\usepackage{wrapfig}
\usepackage{enumitem}

\newcommand{\mxor}{\operatorname{xor}}
\newcommand{\xor}{\texttt{xor}}
\newcommand{\unify}{\texttt{unify}}
\newcommand{\unary}{\texttt{unary}}
\newcommand{\true}{\texttt{true}}
\newcommand{\diff}{\texttt{diff}}
\newcommand{\allDifferent}{\texttt{allDiff}}
\newcommand{\sumAllDifferent}{\texttt{SumAllDiff}}
\newcommand{\permutation}{\texttt{permutation}}
\newcommand{\permPair}{\texttt{permPair}}
\newcommand{\sumNums}{\texttt{sumNums}}
\newcommand{\sumBits}{\texttt{sumBits}}
\newcommand{\comparator}{\texttt{comparator}}
\newcommand{\uadder}{\texttt{uadder}}
\newcommand{\lexleq}{\texttt{lexleq}}
\newcommand{\pairwise}{\texttt{pairwise}}
\newcommand{\block}{\texttt{block}}
\newcommand{\lessEQ}{\texttt{leq}}
\newcommand{\plus}{\texttt{int\_plus}}
\newcommand{\satelite}{\textsc{SatELite}}


\newcommand{\set}[1]{\left\{
      \begin{array}{l}\!\!#1\!\!\end{array}
      \right\}}
\newcommand{\sset}[2]{\left\{~#1  \left|
      \begin{array}{l}#2\end{array}
    \right.     \right\}}
\newcommand\tuple[1]{\langle #1 \rangle}
\newcommand\qf{\hspace{5mm}}
\newcommand{\prop}{\mathit{prop}}
\newcommand{\solv}{\mathit{solv}}
\newcommand{\elim}{\mathit{elim}}
\newcommand{\vars}{\mathit{vars}}
\newcommand{\FF}{\mathbf{F}}

\newenvironment{SProg}
     {\begin{small}\begin{tt}\begin{tabular}[c]{l}}{\end{tabular}\end{tt}\end{small}}
\newcommand{\qin}{\hspace*{0.15in}}

\newcommand{\BB}{{\cal B}}
\newcommand{\LL}{{\cal L}}
\newcommand{\EE}{{\cal E}}
\newcommand{\RR}{{\cal R}}
\newcommand{\Arith}{{\cal A}}
\renewcommand{\iff}{\mathit{iff}}
\newcommand{\false}{\mathit{false}}
\renewcommand{\true}{\mathit{true}}

\def\ll{\mathbf{[\![}}
\def\rr{\mathbf{]\!]}}
\def\prop#1{\ll {#1}\rr}

\newcommand{\ignore}[1]{}



\newcommand{\mike}[1]{\marginpar{\sc Mike}\textcolor{blue}{#1}}


\newcommand{\bee}{\textsf{BEE}}










\begin{document}
\title{Compiling Finite Domain
  Constraints to\\ SAT with \bee: the Director's Cut}  


\author{Michael Codish\and Yoav Fekete\and Amit Metodi}

\institute{ Department of Computer Science, Ben-Gurion University,
  Israel }

\maketitle
\label{firstpage}


\begin{abstract}
\bee\ is a compiler which facilitates solving finite domain
  constraints by encoding them to CNF and applying an underlying SAT
  solver. In \bee\ constraints are modeled as Boolean functions which
  propagate information about equalities between Boolean literals.
  This information is then applied to simplify the CNF encoding of the
  constraints. We term this process \emph{equi-propagation}.
A key factor is that considering only a small fragment of a
  constraint model at one time enables to apply stronger, and even
  complete reasoning to detect equivalent literals in that
  fragment. Once detected, equivalences propagate to simplify the
  entire constraint model and facilitate further reasoning on other
  fragments.
\bee\ is described in several recent papers: \cite{Metodi2011},
  \cite{bee2012} and \cite{jair2013}.
In this paper, after a quick review of \bee, we elaborate on two
  undocumented details of the implementation: the hybrid encoding of
  cardinality constraints and complete equi-propagation. We then
  describe on-going work aimed to extend \bee\ to consider binary
  representation of numbers.
\end{abstract}

\section{Introduction}


\bee\ (Ben-Gurion Equi-propagation Encoder) is a tool which applies to
encode finite domain constraint models to CNF. \bee\ was first
introduced in~\cite{bee2012} and is further described in
\cite{jair2013}.  During the encoding process, \bee\ performs
optimizations based on equi-propagation~\cite{Metodi2011} and partial
evaluation to improve the quality of the target CNF.
\bee\ is implemented in (SWI) Prolog and can be applied in conjunction
with any SAT solver. It can be downloaded from~\cite{bee2012web} where
one can also find examples of its use. This version of \bee\ is
configured to apply the CryptoMiniSAT solver \cite{Crypto} through a
Prolog interface \cite{satPearl}. CryptoMiniSAT offers direct support
for \texttt{xor} clauses, and \bee\ can be configured to take
advantage of this feature.


A main design choice of \bee\ is that integer variables are
represented in the unary order-encoding (see,
e.g.~\cite{baker,BailleuxB03}) which has many nice properties when
applied to small finite domains.
In the \emph{order-encoding}, an integer variable  in the domain
 is represented by a bit vector .
Each bit  is interpreted as  implying that 
is a monotonic non-increasing Boolean sequence.
For example, the value 3 in the interval  is represented in
5 bits as .


It is well-known that the order-encoding facilitates the propagation
of bounds. Consider an integer variable  with
values in the interval .  To restrict  \pagebreak to take values in the
range  (for ), it is sufficient to assign
 and  (if ). The variables  and
 for  and  are then determined 
and , respectively, by \emph{unit propagation}.  For example, 
given , assigning  and  propagates
to give , signifying that
.
We observe an additional property of the order-encoding for
: its ability to specify that a variable cannot
take a specific value  in its domain by equating two
variables: .
This indicates that the order-encoding is well-suited not only to
propagate lower and upper bounds, but also to represent integer
variables with an arbitrary, finite set, domain.
For example, given , equating  imposes
that . Likewise  and  impose that 
and . Applying these equalities to  gives,

(note the repeated literals),
signifying that .


The order-encoding has many additional nice features that can be
exploited to simplify constraints and their encodings to CNF. To
illustrate one, consider a constraint of the form 
where \texttt{A} and \texttt{B} are integer values in the range
between 0 and 5 represented in the order-encoding. At the bit level
(in the order encoding) we have:  and
.  The constraint is satisfied precisely
when . Equi-propagation derives
the equations  and instead of
encoding the constraint to CNF, we apply the substitution indicated by
, and remove the constraint which is a tautology given .


\section{Compiling Constraints with \bee}

\bee\ is a constraint modeling language similar to the subset of
FlatZinc~\cite{miniZinc2007} relevant for finite domain constraint
problems. The full language is presented in Table \ref{tab:beeSyntax}.
Boolean constants ``'' and ``'' are viewed as (integer)
values ``1'' and ``0''.  Constraints are represented as (a list of)
Prolog terms. Boolean and integer variables are represented as Prolog
variables and may be instantiated when simplifying constraints.
In Table \ref{tab:beeSyntax},  and  (possibly
with subscripts) denote a Boolean literal and a vector of literals,
 (possibly with subscript) denotes an integer variable,
and  (possibly with subscript) denotes an integer
constant.  On the right column of the table are brief explanations
regarding the constraints. The table introduces 26 constraint
templates.

Constraints (1-2) are about variable declarations: Booleans and
integers. Constraint (3) expresses a Boolean as an integer value.
Constraints (4-8) are about Boolean (and reified Boolean)
statements. The special cases of Constraint (5) for
 and
 facilitate the
specification of clauses and of \texttt{xor} clauses (supported
directly in the CryptoMiniSAT solver~\cite{Crypto}).
Constraint (8) specifies that sorting a bit pair 
(decreasing order) results in the pair . This is a
basic building block for the construction of sorting networks
\cite{Batcher68} used to encode cardinality (linear Boolean)
constraints during compilation as described in~\cite{AsinNOR11}
and~\cite{DBLP:conf/lpar/CodishZ10}.
Constraints (9-14) are about integer relations and operations.
Constraints (15-20) are about linear (Boolean, Pseudo Boolean, and
integer) operations.
Constraints (21-26) are about lexical orderings of Boolean and integer
arrays.  










\begin{table}[t]\footnotesize
  \centering
\begin{tabular}{rlll}
\hline\hline
\multicolumn{4}{l}{\bf\small Declaring Variables}\\
\hline
(1)& & &declare Boolean \texttt{X}
\\
(2)& & & declare integer \texttt{I}, 
\\
(3)&     &
          &
          \\
\hline

\multicolumn{4}{l}{\bf\small Boolean (reified) Statements~ 
       \hfill }\\
\hline
(4)&     ~or~ &
          &
           ~or~ \\
(5)&     &
          &
          \\
(6)&     &
          &
          \\
(7)&     &
          &
          \\
(8)& &&
          \\
\hline

\multicolumn{4}{l}{\bf\small Integer relations (reified)
                    \hfill }\\
\multicolumn{4}{l}{\bf\small \& arithmetic \hfill
            ~, 
               }\\
    \hline
(9)&     &
          &
          \\
(10)&     &
          &
          \\
(11)&     &
          &
          \\

(12)& &
      &
      \\
(13)& &
      &
      \\
(14)& &
      &
      \\
\hline

\multicolumn{4}{l}{\bf\small  Linear Constraints~
            \hfill }\\
    \hline
(15)&     &
          &
          \\
(16)& &
      &
          \\
(17)& &
      &
          \\

(18)& &
      &
          \\
(19)& &
      &
          \\

(20)& &
      &
          \\
\hline


\multicolumn{4}{l}{\bf\small  Lexical Order}\\
    \hline

(21)&      &&
              (lex order)\\
(22)&      &&
              (lex order)\\
(23)&      &&
             \\
(24)&      &&
             \\

(25)&      &&
              (lex order)\\
(26)&      &&
              (lex order)\\



\hline\hline
\end{tabular}
  \caption{Syntax of \bee\ Constraints. }
  \label{tab:beeSyntax}
\end{table}



The compilation of a constraint model to a CNF
using \bee\ goes through three phases:
\textbf{(1)}~Unary bit-blasting: integer variables (and constants) are
  represented as bit vectors in the order-encoding.
\textbf{(2)}~Constraint simplification: three types of actions are applied:
  equi-propagation, partial evaluation, and decomposition of
  constraints.  Simplification is applied repeatedly until no rule is
  applicable.
\textbf{(3)}~CNF encoding: the best suited encoding technique is applied to
  the simplified constraints.






Bit-blasting is implemented through Prolog unification. Each
declaration of the form  triggers a
unification . To ease
presentation we assume that integer variables are represented in a
positive interval starting from~ but there is no such limitation in
practice (\bee\ supports also negative integers).
\bee\ applies ad-hoc equi-propagators.  Each constraint is associated
with a set of ad-hoc rules.  The novelty is that the approach is not
based on CNF, as in previous works (for example
\cite{chu-min2003}, \cite{sateliteEenB05}, 
\cite{HeuleJarvisaloBiere2011}, and \cite{Manthey2012}),
but rather driven by the bit
blasted constraints that are to be encoded to CNF.
For example, Figure~\ref{fig:rules2} illustrates the rules for the
\plus\ constraint.
\begin{wrapfigure}[12]{r}{72mm}\small
\vspace{-5mm}
\begin{tabular}{|c|c|}
\hline
  \multicolumn{2}{|c|}{\small  where ,}\\
  \multicolumn{2}{|c|}{\small , and
                                }\\
\hline
  {if } & {then propagate} \\
\hline
\hline
,  & \\
\hline
,  & \\
\hline
,  &  \\
\hline
,  &  \\
\hline
 &  \\
\hline
 &   \\
\hline 
\end{tabular}
\caption{Ad-hoc rules for  }
\label{fig:rules2}
\end{wrapfigure}
For an integer , we
write:  to denote the equation ,  to denote the
equation ,  to denote the equation , and
 to denote the pair of equations .
We view  as if padded with sentinel cells
such that all cells ``to the left of''  take value 1 and all
cells ``to the right of''  take the value 0. This facilitates the
specification of the ``end cases'' in the formalism.
The first four rules of Figure~\ref{fig:rules2} capture the
standard propagation behavior for interval arithmetic. The last two
rules apply when one of the integers in the relation is a
constant. There are symmetric cases when replacing the role of  and
.

When an equality of the form  (between a variable and a literal
or a constant) is detected, then equi-propagation is implemented by
unifying  and  and applies to all occurrences of  thus
propagating to other constraints involving .

Decomposition is about replacing complex constraints (for example
about arrays) with simpler constraints (for example about array
elements). Consider, for instance, the constraint
. It is decomposed to a list of
 constraints applying a straightforward divide and
conquer recursive definition. At the base case, if \texttt{As=[A]}
then the constraint is replaced by a constraint of the form
\texttt{int\_eq(A,Sum)} which equates the bits of  and
, or if  then it is replaced by
.
In the general case \texttt{As} is split into two halves, then
constraints are generated to sum these halves, and then an additional
 constraint is introduced to sum the two sums.

CNF encoding is the last phase in the compilation of a constraint
model. Each of the remaining simplified (bit-blasted) constraints is
encoded directly to a CNF. These encodings are standard and similar to
those applied in various tools such as Sugar~\cite{sugar2009}.




\section{Cardinality Constraints in \bee}

Cardinality Constraints take the form 
where the  are Boolean literals,  is a constant, and the
relation  might be any of .  There is a
wide body of research on the encoding of cardinality to CNF. We focus
on those using sorting networks. For example, the presentations in
\cite{EenS06}, \cite{CodishFFS11}, and \cite{AsinNOR09,AsinNOR11}
describe the use of odd-even sorting networks to encode pseudo Boolean
and cardinality constraints to Boolean formula.
We observe that for applications of this type, it suffices to apply
``selection networks''~\cite{Knuth73} rather than sorting
networks. Selection networks apply to select the  largest
elements from  inputs.  In~\cite{Knuth73}, Knuth shows a simple
construction of a selection network with  size whereas,
the corresponding sorting network is of size .
Totalizers~\cite{BailleuxB03} are similar to sorting networks except
that the merger for two sorted sequences  involves a direct
encoding with  clauses instead of   clauses. Totalizers have been shown to give better encodings when
cardinality constraints are not excessively large.
\bee\  enables the user to select encodings based on sorting networks,
totalizers or a hybrid approach which is further detailed below.


Consider the constraint  in a
context where  is a list of  Boolean literals and
integer variable  defined as
. \bee\ applies a divide and conquer
strategy. If , the constraint is trivial and satisfied by
unifying . If  and  then
 and the constraint is decomposed to
. In the general case, where
, the constraint is decomposed as follows where 
and  are a partitioning of  such that
, , and
:
 
This decomposition process continues as long as there remain
 and when it terminates the model contains
only \texttt{comparator} and \texttt{int\_plus} constraints.
The interesting discussion is with regards to the 
constraints where \bee\ offers two options and depending on this
choice the original \texttt{bool\_array\_sum\_eq} constraint then
takes the form either of a sorting network~\cite{Batcher68} or of a
totalizer~\cite{BailleuxB03}.
So, consider a  constraint  where
,  and
 represent integer variables in the
order encoding.
A unary adder leads to a direct encoding of the sum of two unary
numbers. It involves   clauses where  is the size of
the inputs and as a circuit it has ``depth'' 1. The encoding introduces
the following clauses where  and :

An alternative encoding for  is obtained by
means of a recursive decomposition based on the so called odd-even
merger from Batcher's construction~\cite{Batcher68}. It leads to an
encoding with  clauses where  is the size of the inputs
and as a circuit it has ``depth'' . The decomposition is as
follows (ignoring the base cases) where ,
,  and  are partitions
of  and  to their odd and even positioned
elements, ,  are new unary variables
defined with the appropriate domains, and where
 signifies a set of comparator
constraints and is defined as :
 
In addition to the encodings based on unary adders (direct) and
mergers (recursive decomposition), \bee\ offers a combination of the
two which we call ``hybrid''.  The intuition is simple: in the hybrid
approach we perform recursive decomposition as for odd-even mergers,
but only so long as the resulting CNF is predetermined to be
smaller than the corresponding unary adder. So, it is just like a
merger except that the base case is a unary adder.
Before each decomposition of , \bee\ evaluates the
benefit (in terms of CNF size) of decomposing the constraint as a
merger and takes the smaller of the two.


\begin{figure}[t]
\begin{minipage}{0.47\linewidth}
  \includegraphics[width=1.0\linewidth,keepaspectratio]{pos_imgs/AMHclauses.png}
\end{minipage}
\qquad
\begin{minipage}{0.47\linewidth}
\includegraphics[width=1.0\linewidth,keepaspectratio]{pos_imgs/AMHvars.png}
\end{minipage}
\caption {Relative size of CNF encodings for cardinality: adders,
  hybrid \& mergers. On the left number of clauses, and on the right
  number of added variables.}
 \label{fig:size}
\end{figure}




Figure~\ref{fig:size} depicts the size of CNF encodings for the
constraint  where
. The left graph illustrates the number
of clauses in the three encodings. The unary adder has fewest number
of clauses for inputs of size 7 or less. The hybrid encoding is always
just slightly smaller than the merger. Each time a merger is
decomposed to an adder it is just about of the same number of clauses.
In contrast, in the right graph we see that the encoding never
introduces fresh variables, and as the size of the input increases so
does the benefit of the hybrid approach in number of added variables.










Now let us consider the constraint
 where  is a list
of  Boolean literals and  is a constant.  Assume as before that
 and  are a partitioning of
 such that , ,
and .
A naive decomposition might proceed as follows:\pagebreak
8mm]
    \xRightarrow[]{\mathtt{decompose}}
    \fbox{}
\end{array}
\small \fbox{}
    \xRightarrow[]{\mathtt{decomp.}}
    \fbox{}
(T_1 + T_2 \leq k)    \leftrightarrow     
  (T_1 \leq T_3) \wedge  (T_2+T_3=k)

  \label{eq:cep}
  \varphi' =
         \varphi\land
         \sset{e_{ij}\leftrightarrow (x_i\leftrightarrow x_j)}
              {0\leq i<j\leq n}
\small \fbox{}
    \xRightarrow[]{\mathtt{decompose}}
    \fbox{}
\small \fbox{}
    \xRightarrow[\mathtt{C>0}]{\mathtt{decompose}}
    \fbox{}

  \label{eq:sq}
  \bigwedge_{\tiny{
  \begin{array}{l}
    1\leq i\leq n\\  1\leq j\leq m
  \end{array}}
  } Z_{ij}\leftrightarrow A_i\wedge B_j
Z_{nj}\ldots Z_{1j} \underbrace{0\ldots 0}_{j-1}\sum_{i=1}^n {{x_i}\over{10*y_i+z_i}}=1 such that each digit value
is used between 1  and  times. 
Table~\ref{tab:fractions} depicts experimental results comparing two
encodings of : Both
techniques sum the columns in the matrix \texttt{As} (where the rows
are binary numbers). The \emph{binary approach} repeatedly reduces
triplets of bits in a column to a pair of bits: one in the same column
(the sum bit), and one in the next (the carry bit). This is a standard
``'' reduction. The alternative, \emph{unary approach} is
defined in terms of the unary core of \bee.
One may note that the unary approach typically: gives slightly slower
compilation times (there is more to optimize), smaller encoding sizes
(equi-propagation kicks in), and significantly faster SAT solving
times (it pays off) (see footnote [\ref{machine}] for details on
machine).


\begin{table}[t]\small
  \centering
  \begin{tabular}{|l|rrrr|rrrr|}
\hline
           & \multicolumn{4}{c|}{summing with full adders} 
           & \multicolumn{4}{c|}{summing in the unary core}\\
\hline
       &comp.&clauses&vars&sat&comp.&clauses&vars&sat\\
\hline
  3 & 0.05&25492 & 4354&2.72   & 0.26&23793 &4556 &1.39\\
  4 & 0.13&56125 & 9556&11.19  & 0.50&47743 &9078 &0.56\\
  5 & 0.23&98712 &16551&59.4   & 0.77&78607 &14703&55.65\\
  6 & 0.38&164908&27283&844.91 & 1.01&118850&21977&5.13\\
  7 & 0.76&247082&40572&~~ & 1.87&164451&30125&36.83\\
  8 & 1.29&363323&59183&~~~& 2.14&221262&40196&2653.68\\
\hline
\end{tabular}

\caption{Comparison of encodings for the -fractions problem
  (comp. and sat times in sec. with 4 hour timeout marked as )}
\label{tab:fractions}
\end{table}


\paragraph{\bf Squaring:~~}

Consider the special case of multiplication
 specifying that  where
we introduce two additional optimizations. First, consider the
variables  introduced in Equation~\ref{eq:sq}, we
have  and hence equi-propagation applies to remove the
redundant variables. The result of this is illustrated in
Figure~\ref{fig:square}\subref{fig:square2}. In
Figure~\ref{fig:square}\subref{fig:square3} we reorder the bits in the
columns, as if, letting the bits drop down to the baseline.  Second,
consider the ``columns'' in the
 constraint. Each
variable of the form  with  in a column occurs
twice. So, both can be removed and one inserted back in the column to
the left. This is illustrated in
Figure~\ref{fig:square}\subref{fig:square4} where we highlight the
move of the two  instances. These optimizations reduce the
size of the CNF and are applied both in the binary and in the unary
encodings. 



To evaluate the encoding of  for the
special case when , we consider the application of \bee\
to model and solve the number partitioning problem, also known as
\texttt{CSPLIB 049}.\footnote{See
  \url{http://www.cs.st-andrews.ac.uk/~ianm/CSPLib/prob/prob049/index.html}}
Here, one should finding a partition of numbers  into
two sets  and  such that:  and  have the same cardinality,
the sum of numbers in   equals the sum of numbers in , and
the sum of the squares of the numbers in  equals the sum of the
squares of the numbers in . 

\begin{figure}
  \centering
  
  \subfloat[][Encoding sizes (number of variables): unary and binary
              approach, with and without CEP]{
\includegraphics[width=0.45\linewidth,keepaspectratio]{pos_imgs/npvc.png}
\label{tab1}}
\subfloat[][SAT solving  time (sec.)]{
\includegraphics[width=0.45\linewidth,keepaspectratio]{pos_imgs/nptc.png}
\label{tab2}}

\subfloat[][Encoding size, CEP minus without (\# clauses)]{
\includegraphics[width=0.45\linewidth,keepaspectratio]{pos_imgs/npc.png}
\label{tab3}}
\subfloat[][Encoding size, CEP minus without (\# vars)]{
\includegraphics[width=0.45\linewidth,keepaspectratio]{pos_imgs/npv.png}
\label{tab4}}



\caption{Number Partitioning in \bee, encoding binary arithmetic.}
\label{fig:compareNumPartition}
\end{figure}
 
Figure \ref{fig:compareNumPartition} depicts our results. We consider
four settings. The first two are the binary and unary approaches
described above where buckets of bits of the same binary weight are
summed using full adders or sorting networks respectively. In the
other two settings, we apply complete equi-propagation per individual
constraint (on binary numbers), on top of the ad-hoc rules implemented
in \bee.
Figure~\ref{fig:compareNumPartition}\subref{tab1} illustrates the size
of the encodings (number of CNF variables) for each of the four
settings in terms of the instance size. The two top curves coincide
and correspond to the unary encodings which create slightly larger
CNFs. However note that the unary core of \bee\ with its ad-hoc (and
more efficient) implementation of equi-propagation, detects all of the
available equi-propagation. There is no need to apply CEP. The bottom
two curves correspond to the binary encodings and illustrate that CEP
detects further optimizations beyond what is detected using \bee.

Figure~\ref{fig:compareNumPartition}\subref{tab2} details the SAT
solving times. Here we ignore the compilation times (which are high
when using CEP) and focus on the quality of the obtained CNF. The
graph indicates a clear advantage to the unary approach (where CEP
is not even required).
The average solving time using the unary approach approach (without
CEP) is 270 (sec) vs 1503 (sec) using the binary approach (with
CEP). This is in spite of the fact that unary approach involves larger
CNF sizes.

Figures~\ref{fig:compareNumPartition}\subref{tab3} and \subref{tab4}
further detail the effect of CEP in the binary and unary encodings
depicting the numbers of clauses and of variables reduced by CEP in
both techniques. The smaller this number, the more equi-propagation
performed ad-hoc by \bee.
In both graphs the lower curve corresponds to the encodings based on
the unary core indicating that this is the one of better quality. See
footnote [\ref{machine}] for details on machine.





\section{Conclusion}

We have detailed two features of \bee\ not described in previous
publications. These concern the hybrid approach to encode cardinality
constraints and the procedure for applying complete
equi-propagation. We have also described our approach to enhance the
unary kernel of \bee\ for binary numbers. Our approach is to rely as
much as possible on the implementation of equi-propagation on unary
numbers to build the task of equi-propagation for binary numbers. We
have illustrated the power of this approach when encoding binary
number multiplication. The extension of \bee\ for binary numbers is
ongoing and still requires a thorough experimentation to evaluate its
design.

\newpage


\newcommand{\noopsort}[1]{}
\begin{thebibliography}{10}

\bibitem{AsinNOR09}
R.~As\'{\i}n, R.~Nieuwenhuis, A.~Oliveras, and E.~Rodr\'{\i}guez-Carbonell.
\newblock Cardinality networks and their applications.
\newblock In O.~Kullmann, editor, {\em SAT}, volume 5584 of {\em Lecture Notes
  in Computer Science}, pages 167--180. Springer, 2009.

\bibitem{AsinNOR11}
R.~As\'{\i}n, R.~Nieuwenhuis, A.~Oliveras, and E.~Rodr\'{\i}guez-Carbonell.
\newblock Cardinality networks: a theoretical and empirical study.
\newblock {\em Constraints}, 16(2):195--221, 2011.

\bibitem{BailleuxB03}
O.~Bailleux and Y.~Boufkhad.
\newblock Efficient {CNF} encoding of {B}oolean cardinality constraints.
\newblock In F.~Rossi, editor, {\em CP}, volume 2833 of {\em LNCS}, pages
  108--122, Kinsale, Ireland, 2003. Springer.

\bibitem{Batcher68}
K.~E. Batcher.
\newblock Sorting networks and their applications.
\newblock In {\em AFIPS Spring Joint Computing Conference}, volume~32 of {\em
  AFIPS Conference Proceedings}, pages 307--314, Atlantic City, NJ, USA, 1968.
  Thomson Book Company, Washington D.C.

\bibitem{CodishFFS11}
M.~Codish, Y.~Fekete, C.~Fuhs, and P.~Schneider-Kamp.
\newblock Optimal base encodings for pseudo-{B}oolean constraints.
\newblock In P.~A. Abdulla and K.~R.~M. Leino, editors, {\em TACAS}, volume
  6605 of {\em Lecture Notes in Computer Science}, pages 189--204. Springer,
  2011.

\bibitem{satPearl}
M.~Codish, V.~Lagoon, and P.~J. Stuckey.
\newblock Logic programming with satisfiability.
\newblock {\em TPLP}, 8(1):121--128, 2008.

\bibitem{cmps:ijcai13}
M.~Codish, A.~Miller, P.~Prosser, and P.~J. Stuckey.
\newblock Breaking symmetries in graph representation.
\newblock In {\em Proceedings of IJCAI}, 2013.
\newblock (to appear).

\bibitem{DBLP:conf/lpar/CodishZ10}
M.~Codish and M.~Zazon-Ivry.
\newblock Pairwise cardinality networks.
\newblock In E.~M. Clarke and A.~Voronkov, editors, {\em LPAR (Dakar)}, volume
  6355 of {\em LNCS}, pages 154--172. Springer, 2010.

\bibitem{baker}
J.~M. Crawford and A.~B. Baker.
\newblock Experimental results on the application of satisfiability algorithms
  to scheduling problems.
\newblock In B.~Hayes-Roth and R.~E. Korf, editors, {\em AAAI}, volume~2, pages
  1092--1097, Seattle, WA, USA, 1994. AAAI Press / The MIT Press.

\bibitem{sateliteEenB05}
N.~E{\'e}n and A.~Biere.
\newblock Effective preprocessing in {SAT} through variable and clause
  elimination.
\newblock In F.~Bacchus and T.~Walsh, editors, {\em SAT}, volume 3569 of {\em
  Lecture Notes in Computer Science}, pages 61--75. Springer, 2005.

\bibitem{EenS06}
N.~E{\'e}n and N.~S{\"o}rensson.
\newblock Translating pseudo-{Boolean} constraints into {SAT}.
\newblock {\em JSAT}, 2(1-4):1--26, 2006.

\bibitem{Garnick93}
D.~K. Garnick, Y.~H.~H. Kwong, and F.~Lazebnik.
\newblock Extremal graphs without three-cycles or four-cycles.
\newblock {\em Journal of Graph Theory}, 17(5):633--645, 1993.

\bibitem{HeuleJarvisaloBiere2011}
M.~Heule, M.~J{\"a}rvisalo, and A.~Biere.
\newblock Efficient {CNF} simplification based on binary implication graphs.
\newblock In K.~A. Sakallah and L.~Simon, editors, {\em SAT}, volume 6695 of
  {\em Lecture Notes in Computer Science}, pages 201--215. Springer, 2011.

\bibitem{Knuth73}
D.~E. Knuth.
\newblock {\em The Art of Computer Programming, Volume III: Sorting and
  Searching}.
\newblock Addison-Wesley, 1973.

\bibitem{chu-min2003}
C.~Li.
\newblock Equivalent literal propagation in the {DLL} procedure.
\newblock {\em Discrete Applied Mathematics}, 130(2):251--276, 2003.

\bibitem{Manthey2012}
N.~Manthey.
\newblock Coprocessor 2.0 - a flexible {CNF} simplifier - (tool presentation).
\newblock In A.~Cimatti and R.~Sebastiani, editors, {\em SAT}, volume 7317 of
  {\em Lecture Notes in Computer Science}, pages 436--441. Springer, 2012.

\bibitem{Marques-SilvaJL10}
J.~Marques-Silva, M.~Janota, and I.~Lynce.
\newblock On computing backbones of propositional theories.
\newblock In H.~Coelho, R.~Studer, and M.~Wooldridge, editors, {\em ECAI},
  volume 215 of {\em Frontiers in Artificial Intelligence and Applications},
  pages 15--20. IOS Press, 2010.
\newblock Extended
  version:\url{http://sat.inesc-id.pt/~mikolas/bb-aicom-preprint.pdf}.

\bibitem{bee2012web}
A.~Metodi.
\newblock \bee.
\newblock \url{http://amit.metodi.me/research/bee/}, 2012.

\bibitem{bee2012}
A.~Metodi and M.~Codish.
\newblock Compiling finite domain constraints to {SAT} with {{\textsf{BEE}}}.
\newblock {\em TPLP}, 12(4-5):465--483, 2012.

\bibitem{Metodi2011}
A.~Metodi, M.~Codish, V.~Lagoon, and P.~J. Stuckey.
\newblock Boolean equi-propagation for optimized {SAT} encoding.
\newblock In J.~H.-M. Lee, editor, {\em CP}, volume 6876 of {\em LNCS}, pages
  621--636, ~, 2011. Springer.

\bibitem{jair2013}
A.~Metodi, M.~Codish, and P.~J. Stuckey.
\newblock Boolean equi-propagation for concise and efficient {SAT} encodings of
  combinatorial problems.
\newblock {\em J. Artif. Intell. Res. (JAIR)}, 46:303--341, 2013.

\bibitem{miniZinc2007}
N.~Nethercote, P.~J. Stuckey, R.~Becket, S.~Brand, G.~J. Duck, and G.~Tack.
\newblock Minizinc: Towards a standard {CP} modeling language.
\newblock In C.~Bessiere, editor, {\em CP2007}, volume 4741 of {\em Lecture
  Notes in Computer Science}, pages 529--543, Providence, RI, USA, 2007.
  Springer-Verlag.

\bibitem{Schneider96}
J.~J. Schneider.
\newblock Searching for backbones -- an efficient parallel algorithm for the
  traveling salesman problem.
\newblock {\em Comput. Phys. Commun}, 1996.

\bibitem{Crypto}
M.~Soos.
\newblock {CryptoMiniSAT}, v2.5.1.
\newblock \url{http://www.msoos.org/cryptominisat2}, 2010.

\bibitem{sugar2009}
N.~Tamura, A.~Taga, S.~Kitagawa, and M.~Banbara.
\newblock Compiling finite linear {CSP} into {SAT}.
\newblock {\em Constraints}, 14(2):254--272, 2009.

\end{thebibliography}


\end{document}
