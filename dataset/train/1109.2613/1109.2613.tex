\documentclass[journal, letterpaper]{IEEEtran}
\IEEEoverridecommandlockouts
\usepackage[dvips]{graphicx}
\usepackage{color}
\usepackage{amssymb,amsmath}
\usepackage{epsfig}
\usepackage{algorithm2e}
\usepackage[compress]{cite}



\begin{document}
\title{Whether and Where to Code in the Wireless Packet Erasure Relay Channel}
\author{\IEEEauthorblockN{Xiaomeng Shi, \textit{Student Member, IEEE}, Muriel M\'{e}dard, \textit{Fellow, IEEE}, and Daniel E. Lucani, \textit{Member, IEEE}}
\thanks{Manuscript received on August 15, 2011, and revised on March 18, 2012.}
\thanks{Xiaomeng Shi (xshi@mit.edu) and Muriel M\'{e}dard (medard@mit.edu) are with the Research Laboratory of Electronics, Massachusetts Institute of Technology, MA, USA. Daniel E. Lucani (dlucani@fe.up.pt) is with the Instituto de Telecomunica\c c\~oes, DEEC Faculdade de Engenharia, Universidade do Porto, Portugal.}
\thanks{The authors acknowledge the financial support of the Interconnect Focus Center, one of the six research centers funded under the Focus Center Research Program, a Semiconductor Research Corporation program. This work is also supported by the NSERC Postgraduate Scholarship (PGS) issued by the Natural Sciences and Engineering Research Council of Canada, and in part by the Funda\c c\~ ao para a Ci\^ encia e a Tecnologia under project PEst-OE/EEI/LA0008/2011.}
}

\maketitle
\begin{abstract}
The throughput benefits of random linear network codes have been studied extensively for wirelined and wireless erasure networks. It is often assumed that all nodes within a network perform coding operations. In energy-constrained systems, however, coding subgraphs should be chosen to control the number of coding nodes while maintaining throughput. In this paper, we explore the strategic use of network coding in the wireless packet erasure relay channel according to both throughput and energy metrics. In the relay channel, a single source communicates to a single sink through the aid of a half-duplex relay. The fluid flow model is used to describe the case where both the source and the relay are coding, and Markov chain models are proposed to describe packet evolution if only the source or only the relay is coding. In addition to transmission energy, we take into account coding and reception energies. We show that coding at the relay alone while operating in a rateless fashion is neither throughput nor energy efficient. Given a set of system parameters, our analysis determines the optimal amount of time the relay should participate in the transmission, and where coding should be performed.
\end{abstract}
\begin{IEEEkeywords}
Random linear network coding, wireless relay channel, packet delivery energy
\end{IEEEkeywords}


\section{Introduction}\label{sec:introduction}
Network coding, although initially introduced as a theoretical tool in the field of network information theory \cite{ahlswede2000network}, has been made practical by the use of random linear network codes (RLNC) \cite{chou2003practical, ho2006random}, and has been shown to offer throughput, delay, energy and other advantages over classical store-and-forward strategies. To minimize the amount of centralized control, RLNC is often performed at the source as well as all intermediate nodes within a transmission subgraph. Because coding operations and data reception by intermediate nodes can have non-trivial energy costs, resource constrained networks, such as wireless body area networks (WBANs), could potentially benefit from strategies that allow a reduction in the number of coding nodes, while maintaining the benefits of network coding. This paper studies the strategic use of network coding in a three-node wireless packet erasure relay channel, as illustrated by Figure~\ref{fig:model}(a), with an emphasis on whether and where to code when the relay operates in half-duplex mode. We propose Markov chain models to characterize the system performance in terms of throughput and packet delivery energy, thus providing a way to find the optimal fraction of time for which the relay should participate in the transmission. We show, through numerical analysis, that coding at the relay alone while operating in a rateless fashion is neither throughput nor energy efficient, while coding at the source alone has performances close to the case where coding is performed at both nodes. The decision to code is based on packet erasure probabilities, transmission energy, as well as energy spent on reception and coded packet generation.
\begin{figure}[t!]
  \centering
  \includegraphics[width=3.2in]{./graphics/model.eps}
  \caption{Single relay unicast network, with corresponding flow hypergraph.  represents the packet transmission success probabilities between  and .}
  \label{fig:model}
\end{figure}


Although seemingly simple, the analysis of the three-node network can offer insights to more complicated systems with more source or sink nodes. Two such examples are WBAN and advanced LTE cellular networks \cite{3gpp}. In a WBAN, the topology is almost always star-shaped: data are uploaded in a converge-cast sense to a central base station (BS) \cite{latre2011survey,kwak2009study}. Depending on the relative location of a sensor on the human body, it may be useful in terms of energy efficiency to deploy a relay around the shoulder, in direct line of sight with both the front and back of the body. To the best of our knowledge, the throughput and energy tradeoffs in this case have not been studied before. On the other hand, in advanced LTE systems, data are transmitted in both directions, with an emphasis on download in a broadcast sense from a central BS to individual user equipments. Currently relaying is being considered as an improvement tool, for example, for coverage of high data rates, and for temporary network deployment \cite{3gpp}. Here the relay is to wirelessly connect to the radio-access network, and may either function as a smart repeater, or have control of its own cell. For both WBANs and advanced LTE systems, the introduction of network coding and the insertion of a relay may bring energy or throughput gains. As a starting point, we consider the three-node packet erasure relay channel, assuming physical layer designs are readily available on point-to-point links. Further extensions of this setup can involve additional source nodes, as in a WBAN, additional sink nodes, as in an LTE system, and additional relay nodes, a scenario applicable to both examples.

In evaluating system performance, we parametrically model the total energy consumption, taking into account transmission and reception energies as well as processing energy required to generate coded packets. The inclusion of energies for data reception, idle listening, and protocol overheads allows a more comprehensive understanding of the implications of inserting a relay and using network coding. Our goal is to characterize the throughput and energy performances of the system when a relay is added to a point-to-point link and nodes are network coding capable. Our previous work studied the energy advantages of network coding in half-duplex WBANs with a star-topology and showed that when reception energy is taken into account, total energy use could be reduced \cite{shi2011both}. The gains are expected to grow with the number of nodes.

The remaining of the paper is organized as follows. Section~\ref{sec:relatedWork} summarizes some of the previous work related to the wireless relay channel and the application of network coding in such settings. Differently from previous works that focus on joint channel and network coding for optimal throughput analysis, in this paper, we assume a physical layer design is available, and network coding is inserted into the network layer, independently of the source and channel codes employed. Section~\ref{sec:systemModel} details the assumptions made and the system modeled. Section~\ref{sec:analysis} then discusses three separate cases, depending on the coding locations. For each case, we try to characterize the expected completion time and the expected completion energy of transmitting a given number of packets from the source to the destination. Through Markov chain analysis, we provide a framework for evaluating bounds on the system performance when coding is conducted at the relay only, and for determining the system performance when coding is conducted at the source only. Section~\ref{sec:simulations} then establishes the key results through numerical evaluations, showing that, for a wide range of parameters, coding at the relay only is not throughput or energy efficient. Section~\ref{sec:conclusion} concludes the paper with discussions on future work.

\section{Related Work}\label{sec:relatedWork}
A relay channel models the problem where two nodes communicate through the help of one or more relays. This setup is common in multihop wireless networks such as sensor networks, where transmission power is limited, or in decentralized ad hoc networks, where nodes can only communicate with their immediate neighbors. Relays can overhear transmissions to the destination, owing to the broadcast advantage of the wireless medium. Recently there has been a renewed interest in the classical relay channel \cite{van1968transmission, cover1979capacity}, motivated by the potential to achieve cooperative diversity, and thus better capacity bounds \cite{laneman2004cooperative, kramer2005cooperative,lai2006three, dana2006capacity}. Schemes such as amplify-and-forward, decode-and-forward, and compress-and-forward have been proposed and studied extensively in terms of capacity, outage, energy efficiency, and optimal power allocation schemes \cite{yao2005energy, zhao2007improving}. Much of the analysis has focused on the fundamental performance limits at the physical layer and on the transmission of a single data packet. The introduction of network coding into the relay channel has also focused mostly on joint channel-network code design, with or without limited processing complexity constraints \cite{hausl2006iterative, xiao2007network, zhang2009channel, yang2007network, tuninetti2005throughput, tuninetti2004processing}. In larger networks, however, network coding typically resides in higher layers of the protocol stack, independently of physical layer implementations.

In this paper, we assume network coding takes place at the protocol layer, independently of source and channel codes employed. Such an assumption on the separation of channel and network codes may not necessarily be capacity achieving, but allows the introduction of network coding into existing systems.

The use of RLNC in wireless erasure networks under packetized operations is first studied by Lun et al. \cite{lun2008coding}, and extended to a scheduling framework by Traskov et al. \cite{traskovThesis}. Other schemes that employ network coding in a relay setup includes the MORE protocol \cite{chachulski2007trading}, which performs RLNC at the source only to reduce the amount of coordination required by multiple relay nodes, and the COPE protocol, which employs RLNC at the relay only in a 2-way relay channel to improve reliability, taking advantage of opportunistic listening and coding \cite{katti2006xors}. Fan et al. also proposed a network coding based cooperative multicast scheme to show that significant throughput gains can be achieved when network coding is performed at the relay only \cite{fan2009reliable}; one assumption in this work is that feedback is available from both the destination and the relay to the source after each packet reception. In practical systems, feedback can be costly in terms of both throughput and energy, depending on the underlying hardware architecture \cite{shi2011both}.

In this paper, we explore rateless transmissions, where the acknowledgement for successful reception is sent only once by the destination when the transmission of all available data is completed. As described in the introduction, we also take into account the energy spent on reception and packet processing in addition to the energy required to transmit them. Furthermore, we assume that a sufficiently large field is used for network coding operations, such that transmissions of non-innovative packets from the source can be neglected. In terms of energy use, we make the simple assumption that coding energy stays constant as field size increases, and show that, the decision to code depends on the dominating energy term (transmission, reception, or code generation). The tradeoff between energy budget for the transmission of linearly dependent packets when field size is small and the energy budget for code generation when field size is large is discussed in \cite{angelopoulos2011energy}.

\section{System Model}\label{sec:systemModel}
We represent the data flow through the relay channel using a hypergraph, as shown in Figure~\ref{fig:model}(a). A hypergraph is a generalization of a graph: a broadcast link is represented by a hyperarc between a single start node and a set of end nodes, and a multiple access link is represented by a hyperarc between a set of start nodes and a single end node \cite{lun2008coding}. A wireless relay channel consists of a source node , a relay node , and a sink node . Source  has  packets of the same length to transmit to . It broadcasts to both  and , while the relay  assists the transmission by either forwarding the original packet, or computing linear combinations of received packets before forwarding the ensuing mixtures.

We assume transmissions occur in a rateless fashion, with minimal feedback:  and  take turns to transmit, until  acknowledges that it has received enough degrees of freedom (dof) to recover the original  data packets. Here we use dofs to represent linearly independent packets. Such rateless operations are often desirable in systems where feedback can be costly in terms of energy or delay.

Our model considers packetized operations, independently of the physical layer implementation of the system. As such, erroneous packets are dropped, and channel losses are measured by a time-averaged erasure rate. This separation of channel and network coding follows from the assumption that physical layer designs are already available for the underlying point-to-point link, with a relay being inserted for performance improvements. The transmission success rates are assumed to be  between  and ,  between  and , and  between  and . Nodes operate in half-duplex mode, where a node cannot transmit and receive at the same time. To avoid interferences and collisions in a contention based scheme, we consider a time-division framework, where  and  share the use of the wireless medium. A genie scheduler allocates the wireless medium to the source  fraction of the total time, , and allocates the wireless medium to the relay the remaining  fraction of time. One possible implementation of such a genie-aided scheduler is to share the same randomness at  and . Figure~\ref{fig:model}(b) illustrates the maximum flow on each possible link in this network model, computed directly from the transmission success rates and the time-sharing constant .

In terms of memory, let both  and  contain  units, but assume  contains  units only, where .  uses its memory as a queue: arriving packets are stored; if  is already full, newly arrived packets are discarded. If  does not perform coding, it sends to  a packet from its memory directly and drops this packet from the queue; if  performs RLNC before forwarding, it sends to  a linear combination of stored packets, where each is weighted by a random number chosen uniformly from a finite field . In this paper,  is assumed to be sufficiently large, such that linear combinations thus generated are linearly independent from each other with high probability. Reference \cite{angelopoulos2011energy} discusses the tradeoff between field size and energy use in more details.

To evaluate the amount of energy spent to deliver successfully a packet from  to , we define four different energy terms.  represents the transmission energy per packet, where transmission occurs at either  or .  represents the reception energy per packet at . The relay therefore pays for being on and listening to the broadcast from the source.  represents energy for generating a coded packet; it should be a function of , since the complexity of network coding operations depends on field size and generation size, which is the number of packets coded together. Nonetheless, in this paper,  is assumed to be constant, representing a maximum allowable value. Lastly,  represents the amount of energy spent by  to listen to the final acknowledgement from . Note that all energy terms are defined relative to  or . It is assumed that the destination  represents a base station without power or energy constraints.


\section{Network coding in the Wireless Relay Channel}\label{sec:analysis}
In this section, three different cases are examined: RLNC at both  and , RLNC at  alone, and RLNC at  alone. For the first case, a fluid flow model is used to analyze the achievable rate, packet delivery energy, and the ratio of these two metrics. For the latter cases, we propose Markov chain models to characterize the expected completion time and the expected completion energy of transmitting  packets from  to . We also offer a brief discussion on the use of systematic codes, which will be studied in future works.

\subsection{Coding at Both the Source  and the Relay }\label{subsec:rs}

When RLNC is performed at both  and , we can use the fluid flow model by Lun et al. \cite{lun2008coding, traskov2008scheduling} to study the rateless transmission of network coded packets through the relay channel. A packet is considered innovative when it carries a new dof to a node. In the relay channel,  injects innovative packets into the hyperarc  in  fraction of the total transmission time, while  injects innovative packets into the arc  in  fraction of the total transmission time. Packet transmissions form innovative flows in this setup because both  and  perform RLNC over a large number of packets. Each mixture is an additional dof relative to . The amount of innovative flow is limited by the packet erasure probabilities. Assuming flow conservation at , the maximum achievable rate  from  to  can be derived by solving the following mathematical programming problem analytically using Fourier-Motzkin elimination \cite{traskovThesis}.


\subsubsection{Packet-Level Capacity Bound}\label{subsucsec:traskov} .
This case is equivalent to maximizing , making the optimization linear. A closed-form solution can be found:
\begin{itemize}
 \item Case 1: , then
  4pt]
    \alpha ^*  & = \frac{p_{rd}}{p_{rd}+p_{sr}(1-p_{sd})}\,.
  
    R^* &       = p_{sd}\,, \quad \alpha^*=1\,. \label{eq:alpha}
  
    P_{\{m,k,l\}\rightarrow\{m+1,k,l\}}   &= \frac{n-m-k-l}{n}p_{sd}(1-p_{sr})\alpha \label{eq:r1}\\
    P_{\{m,k,l\}\rightarrow\{m,k+1,l\}}   &= \frac{n-m-k-l}{n}p_{sd}p_{sr}\alpha \label{eq:r2}\\
    P_{\{m,k,l\}\rightarrow\{m,k,l+1\}}   &= \frac{n-m-k-l}{n}p_{sr}(1-p_{sd})\alpha \label{eq:r3}\\
    P_{\{m,k,l\}\rightarrow\{m-1,k+1,l\}} &= \frac{m}{n}p_{sr}\alpha\label{eq:r4} \\
    P_{\{m,k,l\}\rightarrow\{m,k+1,l-1\}} &= \frac{l}{n}p_{sd}\alpha+\mathbf{I}_{l>0}p_{rd}(1-\alpha)\label{eq:r5}

    P_{\{m,k,l\}\rightarrow\{m,k,l\}}   &=  \left[\frac{m}{n}(1-p_{sr})+\frac{l}{n}(1-p_{sd})
     + \frac{k}{n} \right. \notag \\ & \left. + \frac{n-m-k-l}{n}(1-p_{sd})(1-p_{sr})\right]\alpha \notag \\
     & \hspace{.1in} + (1-\mathbf{I}_{l>0}p_{rd})(1-\alpha) \label{eq:r6}

T_{i} &= \frac{1}{1-P_{ii}} \left\{ 1 + \sum_{j\neq i} P_{ij} T_{j} \right\}, \quad T_{0} = 0\,, i\neq 0.

 \bar{T} &= (\mathbf{I} - P_{\backslash 0})^{-1} \mathbf{1}\,.\label{eq:T}

    P_{\{m,k,l\}\rightarrow\{m+1,k,l\}}   &= \alpha p_{sd}(1-p_{sr}) + \alpha p_{sd}p_{sr} \mathbf{I}_{k+l=x} \label{eq:s1}\\
    P_{\{m,k,l\}\rightarrow\{m,k+1,l\}}   &= \alpha p_{sd}p_{sr} \mathbf{I}_{k+l<x} \label{eq:s2}\\
    P_{\{m,k,l\}\rightarrow\{m,k,l+1\}}   &= \alpha p_{sr}(1-p_{sd}) \mathbf{I}_{k+l<x} \label{eq:s3}\\
    P_{\{m,k,l\}\rightarrow\{m,k,l\}}     &= \alpha p_{sr}(1-p_{sd})\mathbf{I}_{k+l=x} \notag \\ & \hspace{-0.5in}+ \alpha (1-p_{sr})(1-p_{sd})+ (1-\alpha)\mathbf{I}_{k+l=0} \label{eq:s4}\\
    P_{\{m,k,l\}\rightarrow\{m+1,k-1,l\}} &= (1-\alpha) \frac{k}{l+k} \mathbf{I}_{k>0} \label{eq:s5}\\
    P_{\{m,k,l\}\rightarrow\{m+1,k,l-1\}} &= (1-\alpha) \frac{l}{l+k} p_{rd}\mathbf{I}_{l>0} \label{eq:s6}\\
    P_{\{m,k,l\}\rightarrow\{m,k,l-1\}}   &= (1-\alpha) \frac{l}{l+k} (1-p_{rd}) \mathbf{I}_{l>0} \label{eq:s7}

In the case where  has not received any packet successfully but is chosen to transmit, the slot is assumed to be wasted. Assuming independent packet losses, the state transition probabilities are computed as described below.

First, when ,  broadcasts with probability , and the following can occur.

\begin{enumerate}
    \item If  receives the transmitted mixture, but  does not, , i.e., Eq.~\eqref{eq:s1};
    \item if both  and  receive the transmitted mixture, and , , i.e., Eq.~\eqref{eq:s2};
    \item if both  and  receive the transmitted mixture, and , , i.e., Eq.~\eqref{eq:s1};
    \item if  receives the transmitted mixture, but  does not, and , the mixture is stored in memory, , i.e., Eq.~\eqref{eq:s3};
    \item if  receives the transmitted mixture, but  does not, and , the mixture is dropped, , i.e., Eq.~\eqref{eq:s4};
    \item if neither  nor  receives the packet, , i.e., Eq.~\eqref{eq:s4}.
\end{enumerate}

\noindent  transmits coded packets with probability , and the following state transitions can occur.
\begin{enumerate}
    \item If  has no unique mixture to share, , and , , i.e., Eq.~\eqref{eq:s4};
    \item if  has no unique mixture to share, , and , , , i.e., Eq.~\eqref{eq:s5}; observe that since a packet is dropped from 's memory after being sent,  decrements by 1 since the dof is no longer shared, while  increments by 1 since this dof becomes unique to ;
    \item if  has a unique mixture to share, , and
        \begin{itemize}
        \item a unique mixture is sent,  receives successfully,
            , i.e., Eq.~\eqref{eq:s6};
        \item a unique mixture is sent, transmission is unsuccessful,
            , i.e., Eq.~\eqref{eq:s7};
        \item , a shared mixture is sent,  receives successfully,
            , i.e., Eq.~\eqref{eq:s5};
        \item , a shared mixture is sent, transmission is unsuccessful,
            , i.e., Eq.~\eqref{eq:s5}.
        \end{itemize}
\end{enumerate}

Transmission terminates when , and . Again, a virtual terminating state  can be appended, such that , and . With this addition, the Markov chain has one recurrent state only. Figure~\ref{fig:mc_n2_sOnly} gives a sample Markov chain when . The states can be indexed linearly starting from  as state 0, to  as the last state, which corresponds to state number 14 in this example.

\begin{figure}[t!]
  \centering
  \includegraphics[width=2.7in]{./graphics/mc_random_n2_sOnly.eps}
  \caption{Markov chain model, coding at the source  only,  with added terminating state . The number of packets to send at  is ; the amount of memory at the relay  is .}
  \label{fig:mc_n2_sOnly}
\end{figure}

Our goal is to find the value of  that minimizes either the expected completion time , or the expected completion energy . Unlike the coding at  only case, the Markov chain now contains cyclic paths in addition to loops. The expected first passage time starting from different states is . Here, the invertibility of  is guaranteed because   has entries less than  on the main diagonal, and  is a lower Hessenberg matrix with non-zero entries on the main diagonal \cite{horn1990matrix}. Once the value of  that minimizes  is found, we can compute the associated , where . The only difference is the value of , since coding is now performed at the source : .

\subsection{Use of Systematic Codes}\label{subsec:systematic}

Systematic network codes are an attractive alternative to non-systematic random linear network codes, since they often reduce computation complexity and energy use, while maintaining the innovation of independent flows \cite{lucani2010systematic}. With a systematic code at  only,  can first broadcast the uncoded packets one by one in order, then compute random linear mixtures for all remaining packets transmitted from .  performs the store-and-forward function always. In a sufficiently large finite field, since every packet sent by  is innovative with respect to  and , if we view each uncoded packet as an innovative mixture, the state evolution under this setup is the same as the case where full RLNC is performed at  only. If systematic coding is performed at  and RLNC is performed at , the system gives the same performance as the case where RLNC is performed at both nodes. Another possibility is to perform systematic coding at both  and .  first broadcasts uncoded packets one by one in order. It then computes a random linear mixture of all  packets whenever a transmission opportunity becomes available.  acts as a size  queue. When the relay has the opportunity to transmit, it examines the next packet in the queue. If this packet is uncoded,  transmits the uncoded packet directly. If this packet is coded,  linearly combines all data it has in memory before sending out the mixture to . The system performance under this setup should be upper-bounded by the full coding case, and lower-bounded by the coding at  only case. The analysis of this additional systematic phase is non-trivial, so we leave its description and discussion to a later time.



\section{Numerical Results}\label{sec:simulations}

This section compares the performance of the three schemes discussed in Section~\ref{sec:analysis} under different channel conditions. We first consider the coding at  only and coding at  only cases and examine the expected transmission completion times per data packet. We then compare the three cases in terms of achievable throughput, computed as the inverse of average completion time, and packet delivery energy.

\subsection{RLNC at the Relay  Only}

Figure~\ref{fig:rOnly_n} plots the expected completion time per transmitted data packet as a function of  for different values of , when , , and . Recall that  represents the number of data packets to be transmitted by the source to the destination. The optimal  value that achieves the lowest  is indicated by a large dot on each curve.

In this figure, when ,  listens but does not transmit. If , the expected number of transmissions per data packet is 2. This is the solution to the ARQ scheme when , where each packet is retransmitted until successfully received at . When , the expected number of transmissions per data packet is 3. Observe that, since  is unused and  does not code,  simply retransmits one of the two uncoded data packets each round, until both are received at . This scenario is similar to the coupon collector's problem with 2 coupons, except packet erasures need to be taken into account. When 2 coupons are to be collected, the expected number of trials until success is . If divided by  and normalized by the number of packets, this solution leads to the value of 3, the value on the curve , at , in Figure~\ref{fig:rOnly_n}.

Another observation from this figure is that, as  increases, the expected completion time  increases as well. This increase comes from transmissions by . Since  randomly chooses one from  packets to transmit, a packet to be transmitted would have been received by  or  already with non-zero probability. This effect is especially significant towards the end of the transmission, when  has collected most of the dofs. In addition, the optimal  values, which correspond to the horizontal coordinates of the large dots, first decrease in value as  goes from 1 to 5, then increase in value as  increases to 20. This effect indicates that a tradeoff exists between the use of the relay and the amount of wasted retransmissions by the source.

Although not explicitly shown here, we can plot and compare the expected completion time per data packet when the channel between  and  varies. It can be observed that when  increases,  is used a larger fraction of the time, with  becoming 1 if  is larger than , similar to the coding at both nodes case discussed in Section~\ref{subsec:rs}.

From the above numerical evaluations, we can conclude that RLNC at the relay  only while operating in a rateless fashion is not an efficient transmission scheme in terms of throughput. Figure~\ref{fig:rOnly_n} shows that using ARQ without coding (, ) achieves the best expected completion time, or the best throughput. However, one issue with the  case is that each data packet, when transmitted successfully, requires an acknowledgement from , i.e., . Such frequent feedbacks are not energy efficient. If , even though the effect of the  term is mitigated by amortization over a larger , the large increase in the value of  shown in Figure~\ref{fig:rOnly_n} indicates that coding at the relay only is the most energy efficient when . In Section~\ref{subsec:comparison}, we shall compare the energy use of this particular case with other schemes.

\begin{figure}[t!]
  \centering
  \includegraphics[width=3.2in]{./graphics/rOnly_n.eps}
  \caption{Coding at the relay  only, expected completion time per packet   vs. , as  changes in value; , , . The optimal  is labeled with a large dot on each curve.}
  \label{fig:rOnly_n}
\end{figure}

\subsection{RLNC at the Source  Only}

Figure \ref{fig:sOnly_n} plots the expected completion time per data packet as a function of , when  and  vary.  is the number of data packets to be transmitted by , and  is the amount of memory available at  to store received mixtures. Unlike the coding at  case, here  decreases as  becomes larger, because each packet sent by  is innovative relative to  and , and as more packets are combined, the probability that a mixture sent by  is innovative becomes larger. In addition to reducing , another advantage of coding  packets together at  is that the cost for feedback can be amortized over a large number of data packets. Also observe from this figure that as low as  units of memory suffices to achieve the expected completion time of the full memory case (i.e., ).


\begin{figure}[t!]
\centering
  \includegraphics[width=3.2in]{./graphics/sOnly_n_x.eps}
  \caption{Coding at the source  only, expected completion time per packet   vs. , as  and  changes in value; , , . The optimal  is labeled with a large dot on each curve.}
  \label{fig:sOnly_n}
\end{figure}
\begin{figure}[t!]
\centering
  \includegraphics[width=3.2in]{./graphics/sOnly_psrprd.eps}
  \caption{Coding at the source  only, expected completion time per packet  vs. , as  and  change; , , .}
  \label{fig:sOnly_psrprd}
\end{figure}

Figure~\ref{fig:sOnly_psrprd} plots the expected completion time per data packet as a function of , for different , , and . Comparison among curves (1), (4) and (5) show that  should be given more time to transmit when the tandem link from  to  through  is more reliable than the direct link between  and . Comparison between (3) and (4), however, show that  should not be used if the channel between  and  sees large packet losses, even if the channel between  and  is relatively reliable. This observation echoes the decision of not using the relay in the full coding case, as given by Eq.~\eqref{eq:alpha}, and discussed in Section~\ref{subsec:rs}. Moreover, comparison among curves (2), (3), and (4) show that the optimal value of  is a function of channel conditions.

\subsection{Comparisons}\label{subsec:comparison}

Figure~\ref{fig:comparison_psd} compares the maximum achievable rates of three cases: coding at  only as discussed in Section~\ref{subsec:r}, coding at  only as discussed in Section~\ref{subsec:s}, and coding at both  and  as discussed in Section~\ref{subsec:rs}. Figure~\ref{fig:comparison_psd_alpha} plots the corresponding  values that achieve these rates. For the coding at  and coding at  cases, the metric being plotted is the inverse of the optimal expected transmission completion time per data packet (). This inverse corresponds to the throughput  of the systems under discussion. For the case where coding is performed at both  and , the achievable rate is computed using Equations (1) and (2).

When RLNC is performed at  only, as previous discussions have suggested, it is more desirable to mix fewer number of packets; since packets retransmitted from  are uncoded, a larger fraction of the repetitions are wasted. In Figure~\ref{fig:comparison_psd}, the achievable rates are given for two different values of . When , coding is not performed, hence the transmission degenerates into a routing scheme:  and  retransmit a single packet until an acknowledgement is received from . Observe from Figure~\ref{fig:comparison_psd_alpha} that when the channel between  and  is poor (e.g. ), the route through  is preferred (), otherwise  is not used (). When ,  still performs network coding, but only as the sum of two packets. Recall the assumption that all mixed packets transmitted by  are innovative relative to ; the second curve (`') in Figure~\ref{fig:comparison_psd} is therefore an upper bound on the actual system throughput, reconfirming that coding at  only is not throughput efficient.

When RLNC is performed at  only, Figure~\ref{fig:comparison_psd} shows that more than  of the rate attained by the coding at both nodes scheme can be achieved. Here the achievable rates are plotted for only one set of channel realizations, with  and . The exact amount of coding gain depends on the reliability of all three links in the relay channel. Also observe that the performance gap decreases as the channel between  and  becomes more reliable. Moreover, Figure~\ref{fig:comparison_psd_alpha} shows that, when coding at  only, transmissions from  are not required after  becomes reasonably good (e.g., ). This is because transmissions from  follow a randomized scheme, leading to redundant repetitions that do not contribute additional dof to .

\begin{figure}[t!]
  \centering
    \includegraphics[width=3.3in]{./graphics/comparison_ET_ack_1_1_1_1_R.eps}
    \caption{Achievable throughput as a function of , ; , .}
    \label{fig:comparison_psd}
\end{figure}
\begin{figure}[t!]
\centering
  \includegraphics[width=3.3in]{./graphics/comparison_ET_ack_1_1_1_1_alpha.eps}
  \caption{Optimal  corresponding to throughput values in Figure~\ref{fig:comparison_psd}; , .}
  \label{fig:comparison_psd_alpha}
\end{figure}
\begin{figure}[t!]
  \centering
    \includegraphics[width=3.3in]{./graphics/comparison_ET_ack_1_1_1_1_E.eps}
    \caption{Packet delivery energy  as a function of , corresponding to the optimal  in Figure~\ref{fig:comparison_psd}; , , , , , .}
    \label{fig:comparison_ET_ack_1_1_1_1_E}
 \end{figure}
\begin{figure}[t!]
\centering
  \includegraphics[width=3.3in]{./graphics/comparison_ET_ack_1_1_1_1_ER.eps}
  \caption{Packet delivery energy per throughput rate  as a function of , corresponding to the optimal  in Figure~\ref{fig:comparison_psd}; , , , , , .}
  \label{fig:comparison_ET_ack_1_1_1_1_ER}
\end{figure}

In maximizing throughput, coding as much as possible while fully using the relay  seems to be the optimal strategy, followed by coding at  alone. However, assuming that both coding and listening costs power, such approaches may pay higher costs in terms of energy. As discussed in Section~\ref{sec:analysis}, our analysis enables the derivation of total energy costs. For example, if , , ,  are identically 1, Figure~\ref{fig:comparison_ET_ack_1_1_1_1_E} plots the packet deliver energy corresponding to the optimal  in Figure~\ref{fig:comparison_psd_alpha}, while Figure~\ref{fig:comparison_ET_ack_1_1_1_1_ER} plots this energy consumption scaled by the maximum achievable rate. The different energy terms have been chosen assuming that coding and listening consumes energy on the same scale as transmission. Such assumptions are valid in systems where just having the circuitry turned on constitutes the most significant portion of energy use. Other ranges of values are also possible, as we have discussed in \cite{shi2011both}, depending on the underlying physical layer hardware implementations.

It is easy to see from these figures that when  is low, coding at both  and  is the most throughput and energy efficient, while coding at  alone provides a compromise between throughput and energy use; under better channel conditions, however, not coding () and not using the relay () require energy, while achieving equally good throughputs. At , the energy cost for the successful delivery of one data packet is 2 when coding is conducted at both  and : one on transmission, and one on coding. On the other hand, the energy cost for coding at  only, assuming , is : one for transmission, one for coding, and  for listening to transmission termination acknowledgement. Moreover, the energy cost for coding at  only, assuming , is : according the coupon collector's problem, on average 3 units of energy are spent on transmitting the 2 packets, and one unit of energy is spent on receiving the acknowledgement. Lastly with simple ARQ (), two units of energy are spent on each successfully delivered packet.
\begin{figure}[t!]
\centering
  \includegraphics[width=3.3in]{./graphics/comparison_EE_ack_1_1_1_1_alpha.eps}
  \caption{Optimal  corresponding to packet delivery energy values in Figure~\ref{fig:comparison_EE_ack_1_1_1_1_E}; , , , , , .}
  \label{fig:comparison_EE_ack_1_1_1_1_alpha}
\end{figure}
\begin{figure}[t!]
  \centering
    \includegraphics[width=3.3in]{./graphics/comparison_EE_ack_1_1_1_1_E.eps}
    \caption{Minimum packet packet delivery energy  as a function of , , , , , , .}
    \label{fig:comparison_EE_ack_1_1_1_1_E}
\end{figure}
Under the same channel conditions and system parameters as given in Figure~\ref{fig:comparison_ET_ack_1_1_1_1_E}, optimizing for energy use leads to a very different set of  values, plotted in Figure~\ref{fig:comparison_EE_ack_1_1_1_1_alpha}. The corresponding optimal packet delivery energies are shown in Figure~\ref{fig:comparison_EE_ack_1_1_1_1_E}. Observe that the decision to turn off the relay  entirely comes at smaller  values. This is because  consumes energy in listening to incoming packets from  as well as sending outgoing packets to . The energy cost of using  is the same as retransmitting twice from . In addition, since  shares the use of the wireless medium with , having  turned on reduces the rate at which packets can be transmitted from . With these two effects combined,  is used only at small  values. Another result of the energy tradeoff between  and  observable from these two figures is that even though the optimal packet deliver energy curve is continuous,  sees a jump for each of the coding strategies.

Similar energy and throughput curves can be evaluated when energy parameters , ,  and  take on different ranges. In practical systems, depending on the underlying circuit implementation, one or more of these energy terms can dominate over the others, and the optimal transmission schedule could be very different from the ones shown above. Nonetheless, our analysis enables robust decision making to determine when and where to code in a wireless packet erasure relay channel.


\section{Conclusion}\label{sec:conclusion}
We propose Markov chain models to analyze the throughput and packet delivery energy performances of network coding strategies in the wireless packet erasure relay channel. The evolution of innovative packets are tracked when either or both the source and the relay perform random linear network coding. We show through numerical evaluations that using a random code at the relay alone is neither throughout nor energy efficient, while coding at the source alone can provide a good tradeoff between throughput and energy use. We also show that only a very small amount of memory is required at the relay when coding is performed at the source only. Although we do not attempt to categorize explicitly the optimal network coding strategies in the relay channel under different system parameters, we provide a framework for deciding whether and where to code, taking into account of throughput maximization and energy depletion. Future work will consider the use of systematic codes, which have been mentioned in this paper but not studied in detail. A natural extension of the three-node relay channel is a star-shaped network, where nodes can  act as relays for their neighbors. A direct generalization of our given framework does not seem tractable, nonetheless it is clear from our short analysis that the problem of choosing an optimal coding subgraph is very important when practical constraints, such as energy, are taken into account.




\bibliographystyle{IEEEtran}
\bibliography{references}

\begin{IEEEbiography}[{\includegraphics[width=1in,height=1.25in,clip,keepaspectratio]{./graphics/shirley}}]
{Xiaomeng Shi}
is currently pursuing a Ph.D. degree in Electrical Engineering and Computer Science at the Massachusetts Institute of Technology (MIT), Cambridge, USA. She received a B.Eng. degree in Electrical and Computer Engineering in 2005 from the University of Victoria, Victoria, BC, Canada, and an S.M. degree in Electrical Engineering in 2008 from MIT. Her research interests include network coding, energy efficient protocol design in wireless networks, and signal processing.
\end{IEEEbiography}

\begin{IEEEbiography}[{\includegraphics[width=1in,height=1.25in,clip,keepaspectratio]{./graphics/medardnews2}}]
{Muriel M\'edard} is a Professor of Electrical Engineering at MIT. She was previously an Assistant Professor in the ECE Department at UIUC and a Staff Member at MIT Lincoln Laboratory. She received B.S. degrees in EECS, in Mathematics, and in Humanities, as well as M.S. and Sc D. degrees in EE, all from MIT. She has served as an Associate Editor for the Optical Communications and Networking Series of the IEEE Journal on Selected Areas in Communications, the IEEE Transactions on Information Theory and the OSA Journal of Optical Networking. She has served as a Guest Editor for the IEEE Journal of Lightwave Technology, the IEEE Transactions on Information Theory
(twice), the IEEE Journal on Selected Areas in Communications and the IEEE Transactions on Information Forensic and Security. She serves as an associate editor for the IEEE/OSA Journal of Lightwave Technology. She is a member of the Board of Governors of the IEEE Information Theory Society and serves as the President. She has served as TPC co-chair of ISIT, WiOpt and CONEXT. She was awarded the 2009 IEEE Communication Society and Information Theory Society Joint Paper Award , the 2009 IEEE William R. Bennett Prize in the Field of Communications, and the 2002 IEEE Leon K. Kirchmayer Prize Paper Award. She was co-winner of the 2004 MIT Harold E. Edgerton Faculty Achievement Award. In 2007, she was named a Gilbreth Lecturer by the National Academy of Engineering. Professor M\'edard's research interests are in the areas of network coding and reliable communications, particularly for optical and wireless networks.
\end{IEEEbiography}

\begin{IEEEbiography}[{\includegraphics[width=1in,height=1.25in,clip,keepaspectratio]{./graphics/dlucani}}]
{Daniel E. Lucani} is an Assistant Professor at the Faculty of Engineering of the University of Porto and a member of the Instituto de Telecomunica\c c\~oes (IT). He received his B.S.  (\textit{summa cum laude}) and M.S. (with honors) degrees in Electronics Engineering from Universidad Sim\'on Bol\'ivar, Venezuela in 2005 and 2006, respectively, and the Ph.D. degree in Electrical Engineering from the Massachusetts Institute of Technology (MIT) in 2010. His research interests lie in the general areas of communications and networks, network coding, information theory and their applications to highly volatile wireless sensor networks, satellite and underwater networks, focusing on issues of robustness, reliability, delay, energy, and resource allocation. Prof. Lucani was a visiting professor at MIT. He is the general co-chair of the Network Coding Applications and Protocols Workshop (NC-Pro 2011) and has also served as reviewer for high impact international journals and conferences, such as, the IEEE Journal of Selected Areas in Communications, IEEE Transactions on Information Theory, IEEE Transactions on Communications, IEEE Transactions on Wireless Communications, IEEE Journal of Oceanic Engineering.
\end{IEEEbiography}



\end{document}
