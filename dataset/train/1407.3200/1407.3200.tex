\documentclass{article}[11pt,letter]


\usepackage[T1]{fontenc}
\usepackage[english]{babel}
\usepackage[cmex10]{amsmath}
\usepackage[utf8]{inputenc}
\usepackage{amssymb}
\usepackage{amsthm}
\usepackage{authblk}
\usepackage{mathtools}
\usepackage{xspace}
\usepackage[symbol*]{footmisc}
\usepackage[hyperfootnotes=false,breaklinks,bookmarks=false]{hyperref}
\usepackage{algorithmic}
\usepackage{algorithm}
\usepackage{enumerate}
\usepackage{color}
\usepackage{times}
\usepackage{fullpage}


\renewcommand{\algorithmiccomment}[1]{\bgroup\hfill//~#1\egroup}

\bibliographystyle{abbrv}

\newtheorem{definition}{Definition}[section]
\newtheorem{lemma}[definition]{Lemma}
\newtheorem{theorem}[definition]{Theorem}
\newtheorem{proposition}[definition]{Proposition}
\newtheorem{example}[definition]{Example}
\newtheorem{observation}[definition]{Observation}
\newtheorem{corollary}[definition]{Corollary}
\newcommand{\bigo}{\mathcal{O}}
\newcommand{\state}{\mathcal{S}}
\newcommand{\inedg}{\mathrm{in}}
\newcommand{\outedg}{\mathrm{out}}
\newcommand{\ie}{{\it i.e.}\xspace}
\newcommand{\eg}{{\it e.g.}\xspace}
\newcommand{\load}{\mathcal{L}}
\newcommand{\wload}{\mathcal{W}}
\newcommand{\cload}{\mathcal{C}}
\newcommand{\dt}{\Delta t}
\newcommand{\expl}{\mathrm{expl}}
\newcommand{\walk}{\mathcal{E}}
\newcommand{\mset}[1]{\{\!\{#1\}\!\}}
\newcommand{\const}{\mathrm{const}}
\newcommand{\closedrange}[2]{[#1\,..\,#2]}
\newcommand{\halfrange}[2]{[#1\,..\,#2)}
\newcommand\defeq{\stackrel{\mathclap{\normalfont{\mbox{\text{\tiny{def}}}}}}{=}}
\newcommand{\stab}{t_{s}}
\newcommand{\per}{t_{p}}
\newcommand{\etal}{{\it et~al.}}
\newcommand{\tokens}{\mathit{tokens}}
\newcommand{\pointer}{\mathit{pointer}}



\DefineFNsymbolsTM{myfnsymbols}{\textasteriskcentered *
  \textdagger    \dagger
  \textdaggerdbl \ddagger
  \textsection   \mathsection
  \textbardbl    \|\textparagraph \mathparagraph
}\setfnsymbol{myfnsymbols}

\DeclareMathOperator*{\avg}{\mathbf{avg}}



\usepackage{setspace}   \newcommand{\NOTE}[1]{\marginpar{\setstretch{0.43}\textcolor{blue}\newline\textcolor{red}{\tiny\bf #1}}}



\begin{document}
\author[1]{Jérémie Chalopin}
\author[1]{Shantanu Das}
\author[2]{Paweł Gawrychowski}
\author[3]{Adrian Kosowski}
\author[1]{Arnaud Labourel}
\author[4]{Przemys\l{}aw Uznański\thanks{Part of the work was done while the author was affiliated with LIF,  CNRS and Aix-Marseille University (supported by the Labex Archimède and by the ANR project MACARON (ANR-13-JS02-0002)).}}

\iffalse
\author{Jérémie Chalopin\inst{1}
\and
Shantanu Das\inst{1}
\and
Paweł Gawrychowski\inst{2}
\and
Adrian Kosowski\inst{3}
\and
Arnaud Labourel\inst{1}
\and
Przemys\l{}aw Uznański\inst{4}}
\fi





\affil[1]{LIF, CNRS and Aix-Marseille University, France}
\affil[2]{Max-Planck-Institut f\"{u}r Informatik, Saarbr\"ucken, Germany}
\affil[3]{Inria – LIAFA – Paris Diderot University, France}
\affil[4]{Helsinki Institute for Information Technology HIIT, Department of Computer Science, Aalto University, Finland}

\date{}

\title{Lock-in Problem for Parallel Rotor-router Walks}
\maketitle


\begin{abstract}
The \emph{rotor-router} model, also called the \emph{Propp machine}, was introduced as a deterministic alternative to the random walk. In this model, a group of identical tokens are initially placed at nodes of the graph. Each node maintains a cyclic ordering of the outgoing arcs, and during consecutive turns the tokens are propagated along arcs chosen according to this ordering in round-robin fashion.
The behavior of the model is fully deterministic. Yanovski \etal (2003) proved that a single rotor-router walk on any graph with  edges and diameter  stabilizes to a traversal of an Eulerian circuit on the set of all  directed arcs on the edge set of the graph, and that such periodic behaviour of the system is achieved after an initial transient phase of at most  steps.

The case of multiple parallel rotor-routers was studied experimentally, leading Yanovski~\etal\ to the conjecture that a system of  parallel walks also stabilizes with a period of length at most  steps. In this work we disprove this conjecture, showing that the period of parallel rotor-router walks can in fact, be superpolynomial in the size of graph. On the positive side, we provide a characterization of the periodic behavior of parallel router walks, in terms of a structural property of stable states called a \emph{subcycle decomposition}.
This property provides us the tools to efficiently detect whether a given system configuration corresponds to the transient or to the limit behavior of the system. Moreover, we provide polynomial upper bounds of  and  on the number of steps it takes for the system to stabilize. Thus, we are able to predict any future behavior of the system using an algorithm that takes polynomial time and space. In addition, we show that there exists a separation between the stabilization time of the single-walk and multiple-walk rotor-router systems, and that for some graphs the latter can be asymptotically larger even for the case of  walks.
\end{abstract}

\section{Introduction}

Dynamical processes occurring in nature provide inspiration for simple, yet powerful distributed algorithms. For example, the \emph{heat equation}, which describes real-world processes such as heat and particle diffusion, also proves useful when designing schemes for load-balancing and token rearrangement in a discrete graph scenario. In the diffusive model of load-balancing on a network, each node of the network is initially endowed with a certain load value, and in each step it distributes a fixed proportion of its load evenly among its neighbors. Given that such a balancing operation is performed for load which is infinitely divisible (so-called \emph{continuous diffusion}), in the long term the distribution of load converges on a degree-regular network to uniform over all nodes. When load is composed of indivisible unit tokens, the continuous diffusion process is no longer practicable. It is, however, possible to design randomized schemes in which the \emph{expected} value of load of each node at each moment of time corresponds precisely to the value of its load in the corresponding continuous diffusion process. This may be achieved, for instance, by allowing each token of load to follow an independent random walk on the network, as well as by applying more refined techniques admitting stronger concentration of the load distribution, cf.~\cite{DBLP:conf/focs/SauerwaldS12}. Such methods are stochastic in their very nature, and it is natural to ask whether there exist \emph{deterministic} methods which mimic this type of stochastic load balancing behavior? The answer is affirmative, with the natural candidate process being the so-called \emph{rotor-router} model.

Formally, the rotor-router mechanism is represented by an undirected anonymous graph . Initially, a set of identical tokens is released on a vertex of the graph. At discrete, synchronous steps, the tokens are propagated according to the deterministic round robin rule, where after sending each token, the pointer is advanced to the next exit port in the fixed cyclic ordering. Such a mechanism has been proposed as a viable alternative to stochastic and random-walk-based processes in the context of load balancing problems~\cite{BerenbrinkKKMU14,DBLP:journals/cpc/CooperS06,DBLP:journals/cpc/DoerrF09}, exploration of graphs \cite{DBLP:journals/siamcomp/AfekG94,DereniowskiKPU14,Frae70,GR,KosowskiP14}, and stabilization of distributed processes \cite{BHKKR09,BhattEGT02,PhysRevLett.77.5079,YanovskiWB03}.

The resemblance between the rotor-router token distribution mechanism and stochastic balancing processes based on continuous diffusion is at least twofold, in that: (1) the number of tokens on each node for the rotor-router process has a bounded discrepancy with respect to that in the continuous diffusion process~\cite{DBLP:journals/corr/BerenbrinkKKMU14,DBLP:conf/cocoon/ShiragaYKY14}, and (2) when performing time-averaging of load over sufficiently long time intervals, the observed load averages for all nodes in the rotor-router process converge precisely to their corresponding value for the continuous diffusion process.

By contrast to time-averaged load, for any \emph{fixed} moment of time, the deterministic rotor-router process and the stochastic approaches exhibit important differences. A stochastic load balancing process based on tokens following random walks leads the system towards a ``heat death'' stochastic state, which is completely independent of the starting configuration. By contrast, the rotor-router process is a deterministic process on a graph and its limit behavior may be much more interesting.\footnote{Perhaps the first work to highlight the importance of differences between limit properties of deterministic and stochastic variants of a token-based discrete process on a graph was that of M.~Kac, in the setting of statistical mechanics (cf.~\cite{thompson2015mathematical}[Section 1.9] for a comprehensive discussion).} The number of possible configurations of a rotor-router system is finite, hence, after a transient initial phase, the process must stabilize to a cyclic sequence of states which will be repeated ever after. Natural questions arise, concerning the eventual structural behavior observed in this limit cycle of the rotor-router system, the length of the limit cycle, and the duration of the stabilization phase leading to it. So far, the only known answer concerned the case when only a single token is operating in the entire system. Yanovski \etal~\cite{YanovskiWB03} showed that such a single token stabilizes within a polynomial number of steps to periodic behavior, in which it performs a traversal of some Eulerian cycle on the directed version of the network graph.

In this work, we provide a complete structural characterization of the limit behavior of the rotor-router for an \emph{arbitrary} number  of tokens. The obtained characterization shows that the rotor-router mechanism provides a way of self-organizing tokens, initially spread out arbitrarily over a graph, into balanced groups, each of which follows a well-defined walk in some part of the network graph. The practical implications of our result may be seen as twofold. On the one hand, when viewing the rotor-router as a load-balancing process, we obtain a better understanding of its limit behavior. On the other hand, when considering each of the tokens as a walker in the graph, we show that the rotor-router may prove to be a viable strategy for perpetual graph exploration, with possible applications in so-called network patrolling problems.

\subsection{Related Work}

\paragraph{Load balancing.}
The rotor-router mechanism of token distribution has been considered in problems of balancing workload among network nodes for specific network topologies. In this context, each token is considered as a unit-length task to be performed by one of the processors in a network of computers. Cooper and Spencer~\cite{DBLP:journals/cpc/CooperS06} studied load balancing with parallel rotor walks in -dimensional grid graphs and showed a constant bound on the discrepancy between the number of tokens at a given node  in the rotor-router model and the expected number of tokens at  in the random-walk model. The structural properties of the distribution of tokens for a rotor-router system on the -dimensional grid were considered by Doerr and Friedrich~\cite{DBLP:journals/cpc/DoerrF09}. Akbari and Berenbrink~\cite{DBLP:conf/spaa/AkbariB13} proved an upper bound of  on the load-balancing discrepancy for hypercubes, and for tori of constant dimensions, they showed that the discrepancy is bounded by a constant. For general -regular graphs, a bound of  on the discrepancy of the rotor-router mechanism with respect to continuous diffusion follows from the general framework of~\cite{RSW98}, where  is the eigenvalue gap of the graph, under the assumption that a sufficient number of self-loops are present at each node of the graph. This discrepancy bound has recently been improved to  in~\cite{DBLP:journals/corr/BerenbrinkKKMU14}.

\paragraph{Graph exploration.}
The walks of tokens following fixed local rules at nodes provide a local mechanism of graph exploration. Each token, starting at a node of the graph, moves at each step to one of the adjacent nodes, until it has explored all the nodes of the graph. Such mechanisms tend to be location-oblivious and robust, displaying resilience to changes in network topology. In some cases, we may require the token to periodically visit all nodes, \eg, with the goal of monitoring the network or distributing updates.

One basic exploration technique relies on multiple tokens, each of which follows an independent random walk: at each step, each token chooses one of the arcs incident to the current node uniformly at random and traverses it. The performance of such parallel random walks have been analyzed by Alon~\etal~\cite{A08}, Efremenko and Reingold~\cite{DBLP:conf/approx/EfremenkoR09}, and Els\"asser and Sauerwald~\cite{DBLP:journals/tcs/ElsasserS11}, who have demonstrated that in terms of the expected time until all nodes have been visited by at least one token (i.e., the cover time), parallelization brings about a speedup of between  and  when running random walks with  parallel tokens.

The rotor-router mechanism has also been studied in the context of graph exploration, sometimes under the name of {\em Edge Ant Walks\/}~\cite{Wagner99distributedcovering,YanovskiWB03}, and in the context of traversing a maze and marking edges with pebbles, \eg~in~\cite{BhattEGT02}. Cover times of rotor-router systems have been investigated by Wagner~\etal~\cite{Wagner99distributedcovering} who showed that starting from an arbitrary initial configuration\footnotemark[1], a single token following the rotor-router rule explores all nodes of a graph on  nodes and  edges within  steps. Later, Bhatt~\etal~\cite{BhattEGT02} showed that after at most  steps, the token continues to move periodically along an Eulerian cycle of the (directed symmetric version of the) graph. Yanovski \etal~\cite{YanovskiWB03} and Bampas~\etal~\cite{BHKKR09} studied the stabilization time and showed that the token starts circulating in the Eulerian cycle within  steps, in the worst case, for a graph of diameter . Studies of the rotor router system for specific classes of graphs were performed in~\cite{FS10}. While all these studies were restricted to static graphs, Bampas \etal~\cite{BGKKR09}  considered the time required for the rotor-router to stabilize to a new Eulerian cycle after an edge is added or removed from the graph.\footnotetext[1]{A configuration is defined by: the cyclic order of outgoing arcs, the initial pointers at the nodes, and the current location of the token.}

\looseness-1
Studies of the parallel (\ie, multiple token) rotor-router were performed by Yanovski \etal~\cite{YanovskiWB03} and Klasing~\etal~\cite{DBLP:conf/podc/KlasingKPS13}, and the speedup of the system due to parallelization was considered for both worst-case and best-case scenarios.  In \cite{DereniowskiKPU14}, Dereniowski~\etal\ establish bounds on the minimum and maximum possible cover time for a worst-case initialization of a -rotor-router system in a graph  with  edges and diameter ,  as   and  respectively.  In \cite{KosowskiP14}, Kosowski and Pająk provided a more detailed analysis of the speedup for specific classes of graphs, providing tight bounds of cover-time speed-up for all values of  for degree-restricted expanders, random graphs, and constant-dimensional tori. For hypercubes, they resolve the question precisely, except for values of  much larger than .



\subsection{Our Results}


In this work we provide a structural characterization of the limit behavior of the rotor-router model with multiple tokens. Yanovski \etal~\cite{YanovskiWB03} conjectured that the rotor-router system enters a short sequence of states (of length at most ), which repeats cyclically ever after. We start this work by disproving this conjecture. In fact, we display an example of a starting configuration which admits a limit cycle with a period of superpolynomial length () with respect to the size of the graph. Our example is similar to the construction presented by Kiwi~\etal~\cite{Kiwi94nopolynomial} to prove the existence of super-polynomial periods for chip firing games on graphs (although the rules of chip firing games are only very loosely related to those of the rotor-router).

By contrast, it turns out the fact that the rotor-router admits long limit cycles does not signify that the limit behavior of the rotor-router should be perceived as a ``disordered'' discrete dynamical system. The long period in our counterexample comes from the system being composed from many smaller parts, each of which exhibits a small (but different) period length. We show that for any limit sequence of states in the rotor-router model, the graph can be partitioned into arc-disjoint directed Eulerian cycles, with each token in the limit periodically traversing arcs of one particular cycle. We name such behavior a \emph{subcycle decomposition}, the exact properties of which are described in Section~\ref{sec:lock-in}. To complement the lower bound, we provide an upper bound of  on the period of parallel rotor walks in its limit behavior. This upper bound asymptotically almost matches the lower bound from our example.

There are several consequences of our structural characterization of the limit behavior of the rotor-router.
First, we show that it is possible to determine efficiently whether the system has already stabilized (\ie, reached a configuration that will repeat itself) or not. This detection is based on the analysis of the properties of stable states, that is, of how the tokens arriving at a node are distributed into groups leaving on different outgoing arcs. The main point of this analysis is the observation that the cumulative number of tokens entering a vertex  (over the time period ) is equal to the cumulative number of tokens leaving vertex  (over time ).

Next, by defining an appropriate potential of a system and showing its monotonicity, we can give a polynomial bound on a number of steps necessary for a system with an arbitrary initialization to reach a periodic configuration. We provide an upper bound of , together with examples of graphs with initial configuration having just 2 tokens that require  steps. This analysis is presented in Section~\ref{sec:stab}. The obtained polynomial upper bound means that the rotor-router is an efficient means of self-organizing tokens so as to perform a periodic traversal of the edges of the graph.

Finally, Section~\ref{sec:simulation} is dedicated to showing how the previous results can be applied in a constructive way with regard to efficient simulation of a rotor-router system.  We show how the properties of subcycle decomposition can be applied to provide a way to preprocess any starting configuration in a way that makes it possible to answer queries of certain type in a polynomial time. This shows that a structural characterization of the rotor-router system is not only an important as a theoretical tool for understanding the limit behavior of the system, but it also as a practical tool for solving certain problems related to the rotor-router system.

As a complementary result, we show for the single-token rotor-router howto efficiently compute the Eulerian traversal cycle on which the token would be locked-in, faster than by running the process directly. A naive simulation would take  time, but by using the structural properties of a single token walk together with application of efficient data structures we show how to preprocess the input graph in time  such that we can answer queries about token position at any given time , in  time per query.



\vspace{-0.2cm}

\section{Model and Preliminaries}

Let  be an undirected connected graph with  nodes,  edges and \emph{diameter} . Let  be the number of tokens. The digraph  is the directed version of  created by replacing every edge  with two directed arcs  and . We will refer to the undirected links in graph  as \emph{edges} and to the directed links in the graph  as \emph{arcs}. Given a vertex , we will denote its set of incoming arcs by  and outgoing arcs by .
Each vertex  of  is equipped with a fixed ordering of all its outgoing arcs .

The precise definition of the rotor-router model on the system defined by  is as follows:\\ A \emph{state} at the current time step   is a tuple:

where  is an arc outgoing from node , which is referred to as \emph{the current port pointer at node} , and  is the number of tokens at any given node.  For an arc , let  denote the arc after the arc  in the cyclic order .
During each step, each node  distributes in round-robin fashion all of its tokens, using the following algorithm:\\cload_{t_1}^{t_2}(v) \defeq \sum_{t \in \halfrange{t_1}{t_2}} \load_t(v),\ \cload_{t_1}^{t_2}(e) \defeq \sum_{t \in \halfrange{t_1}{t_2}} \load_t(e).\sum_{e \in \outedg(v)} \cload_{t+1}^{t+i+1}(e) = \cload_{t+1}^{t+i+1}(v) = \sum_{e \in \inedg(v)} \cload_{t}^{t+i}(e).\sum_{e \in \outedg(v)} (\cload_{t+1}^{t+i+1}(e))^2 \le \sum_{e \in \inedg(v)} (\cload_{t}^{t+i}(e))^2,\Phi_{i}(\state_{t+1}) = \sum_v \sum_{e \in \outedg(v)} (\cload_{t+1}^{t+i+1}(e))^2 \le \sum_v \sum_{e \in \inedg(v)} (\cload_{t}^{t+i}(e))^2 = \Phi_{i}(\state_{t}).
\label{eq:dec}
\forall_{e \in \inedg(v)}  \forall_{t \in \halfrange{T}{T+\dt}} \load_{t}(e) = \load_{t+1}(M_v(e)). 
\label{eq:roundfair}
\forall_{e\in \outedg(v)} \wload_t(e) \in \left\{\left\lfloor \frac{\wload_t(v)}{\deg(v)} \right\rfloor,\left\lceil\frac{\wload_t(v)}{\deg(v)}\right\rceil \right\}
\label{eq:b}\mathcal{B}(\wload_t) \defeq 2m\cdot\left( \avg_{e \in \vec{E}} \wload_t(e)\right)^{\!2},\max_{e,e' \in \vec{E}} (\wload_t(e) - \wload_t(e')) \ge \max_{e,e' \in \vec{E}} (\wload_{t+1}(e) - \wload_{t+1}(e')).\Phi(\wload_t) - \Phi(\wload_{t+1}) \ge \frac{(x-4D-1)(x-1)}{4D} \ge \frac{x^2}{16D}.\Phi(\wload_{t+1})-\mathcal{B}(\wload_0) \le (\Phi(\wload_t)-\mathcal{B}(\wload_0))\left(1-\frac{1}{8mD}\right).\Phi(\wload_T)-\mathcal{B}(\wload_0) \le k^2 \cdot \left(1-\frac{1}{8mD}\right)^{16mD\ln k} < k^2 (1/e)^{2 \ln k}  = 1,
\label{eq:sum_of_pot}
 \sum_{i=1}^{3m}\mathcal{B}( \cload_0^{i} )  \le \sum_{i=1}^{3m} \Phi_i(\state_t) \le 150m^2 D^2+\sum_{i=1}^{3m}\mathcal{B}( \cload_0^{i} ).

\left\lceil \frac{\Delta \tau_v}{2(v-2)}\right\rceil - \left\lfloor \frac{\Delta \tau_v}{2(n-v-3)}\right\rfloor \geq 3,\label{eq:lbound}

&\Delta \tau_v \geq \left(\frac{1}{2(v-2)} - \frac{1}{2(n-v-3)}\right)^{-1} > 2 \frac{(v-2)(n-v-3)}{n-2v} >\\&> 2 \frac{(n/3-2)(n/2-3)}{n-2v} = \frac{(n-6)^2}{6}\cdot\frac{1}{n/2 - v}.

&\tau(\lfloor n/2\rfloor) \geq \sum_{a = 1}^{n/18 - 1} \Delta\tau_{\lfloor n/2 \rfloor  - 3a} \geq \frac{(n-6)^2}{6} \cdot \sum_{a = 1}^{n/18 - 1} \frac{1}{3a+1} >\\&> \frac{(n-6)^2}{18} \sum_{a = 2}^{n/18} a^{-1} > \frac{(n-6)^2}{18}(\ln n - 5).

\left\lceil \frac{\Delta \tau_v}{2M' + 2(v-2)}\right\rceil - \left\lfloor \frac{\Delta \tau_v}{2M' + 2(N-v-3)}\right\rfloor \geq 3,

\Delta \tau_v \geq \frac{M'^2}{2} \frac{1}{N/2 - v}

\tau(\lfloor N/2\rfloor) \geq \frac{M'^2}{6}(\ln N - 5).
 \walk_{\infty} = \underbracket{e_1,\ldots,e_{a_1}}_{\walk_1},\underbracket{e_{a_1+1},\ldots,e_{a_2}}_{\walk_2},\underbracket{e_{a_2+1},\ldots,e_{a_3}}_{\walk_3},\ldots a_1^2 + a_2^2 > a_1^2 + a_2^2 - 2(a_1-a_2-1)  = (a_1-1)^2 + (a_2+1)^2,\forall_{  t,t' : \halfrange{t}{t'} \subseteq \halfrange{T}{T+\dt} } \forall_{e \in \inedg(v)} \cload_{t}^{t'}(e) = \cload_{t+1}^{t'+1}(M_v(e)).\forall_{  t,t' : \halfrange{t}{t'} \subseteq \halfrange{T}{T+\dt} } \mset{\cload_{t}^{t'}(e)}_{e\in \vec{E}} =& \bigcup_{v \in V} \mset{\cload_{t}^{t'}(e)}_{e\in \inedg(v)} = \bigcup_{v \in V} \mset{\cload_{t+1}^{t'+1}(e)}_{e\in \outedg(v)} =\\ =& \mset{\cload_{t+1}^{t'+1}(e)}_{e\in \vec{E}},
 \cload_t^{t'}(e_1) - \cload_t^{t'}(e_2) \ge 2.
c(e_1) &= \cload_t^{t'}(e_1)-1,\\
c(e_2) &= \cload_t^{t'}(e_2)+1,\\
c(e) &= \cload_t^{t'}(e) \text{ for every other } e.
\forall_v \sum_{e \in \inedg(v)} \cload_t^{t'}(e) = \sum_{e \in \inedg(v)} c(e).\Phi_i(\state_t) = \sum_v \sum_{e \in \outedg(v)}  \cload_t^{t'}(e)^2 > \sum_v \sum_{e \in \outedg(v)} c(e)^2 = \sum_v \sum_{e \in \inedg(v)} c(e)^2 \ge \Phi_i(\state_{t+1}),
\forall_{0 \le i \le \dt} \cload_T^{T+i}(e_{(\pi_1)}) \ge \cload_T^{T+i}(e_{(\pi_2)}) \ge \ldots \ge \cload_T^{T+i}(e_{(\pi_d)}),
\cload_T^{T+i_1}(e_1) < \cload_T^{T+i_1}(e_2),\cload_T^{T+i_2}(e_1) > \cload_T^{T+i_2}(e_2).\cload_{T+i_1}^{T+i_2}(e_1) - \cload_{T+i_1}^{T+i_2}(e_2) \ge 2,
\forall_{0 \le i \le \dt} \cload_{T+1}^{T+i+1}(f_{(\sigma_1)}) \ge \cload_{T+1}^{T+i+1}(f_{(\sigma_2)}) \ge \ldots \ge \cload_{T+1}^{T+i+1}(f_{(\sigma_d)}).

\label{eq:bad_discrp}
\exists_{T' \in \closedrange{T}{T+\tau}}  |\cload_{T'}^{T+\tau+1}(e_1) - \cload_{T'}^{T+\tau+1}(e_2)| \ge 2.

\label{eq:cycle1}
\cload_{t}^{t+\ell}(e_1) = \frac{\ell}{|\vec{E}_1|} \sum_{e \in \vec{E}_1} \load_T(e),

\label{eq:cycle2}
\cload_{t}^{t+\ell}(e_2) = \frac{\ell}{|\vec{E}_2|} \sum_{e \in \vec{E}_2} \load_T(e).
|\cload_T^{T+2\ell}(e_1) - \cload_T^{T+2\ell}(e_2)| = \left|\frac{2\ell}{|\vec{E}_1|}\cdot\sum_{e \in \vec{E}_1} \load_t(e) - \frac{2\ell}{|\vec{E}_2|}\cdot\sum_{e \in \vec{E}_2} \load_t(e)\right| = 2 \cdot |\cload_T^{T+\ell}(e_1) - \cload_T^{T+\ell}(e_2)|.\label{eq:shifting}\cload_{t}^{t+\ell}(e_1) = \cload_{t}^{t+\ell}(e_2).\sum_{i=1}^{3m} \Phi_{i}(\state_T) = \sum_{i=1}^{3m} \Phi_{i}(\state_{T+2m^2}).
\label{eq:bad_discrp2}
\exists_{t\in \closedrange{T}{T+\tau}}  \left|\cload_{t}^{T+\tau+1}(e_1) - \cload_{t}^{T+\tau+1}(e_2)\right| \ge 2.

(a_0,a_1,\ldots) &= (\load_T(e_1),\load_{T+1}(e_1),\ldots),\\
(b_0,b_1,\ldots) &= (\load_T(e_2),\load_{T+1}(e_2),\ldots),\\
(c_0,c_1,\ldots) &= (a_0-b_0,a_1-b_1,\ldots).

\label{eq:sequence_property}
\forall_{\closedrange{i}{j} \subseteq \halfrange{0}{2m^2}} \text{ if } (j-i<3m) \text{ then } |c_i +c_{i+1}+\ldots+c_{j}| \le 1,
\forall_{\closedrange{i}{j} \subseteq \halfrange{0}{\tau}} |c_{i}+c_{i+1}+\ldots+c_{j}| \le 1
\label{eq:bad_discrp3}
\exists_{\eta \in \closedrange{0}{\tau}} |c_\eta+ c_{\eta+1} + \ldots + c_{\tau}| \ge 2.
\forall_{e,e' \in \mathcal{P}} \ d(e,e') \le 4D+1\label{eq:b'}\mathcal{B}(\wload_t) \defeq \sum_{\mathcal{P}}|\mathcal{P}|\cdot\left( \avg_{e \in \mathcal{P}} \wload_t(e)\right)^{\!2},\mathcal{B}(\wload_t) = \frac{k^2}{2m} = \const,\mathcal{B}(\wload_t) = \frac{k_1^2+k_2^2}{m} = \const.\max_{\mathcal{P}} \max_{e,e' \in \mathcal{P}} (\wload_t(e) - \wload_t(e'))= x.\Phi(\wload_t) - \Phi(\wload_{t+1}) \ge \frac{(x-4D-1)(x-1)}{4D}.\sum_{j = 0}^{h-1} |\wload_t(e_j)-\wload_t(e_{j+1})| \ge x.\sum_{i\le I} \delta_i \ge x-2D-1.
&\sum_{e \in \inedg(v_i)} (\wload_t(e))^2 = (\wload_t(e_{i1}))^2+(\wload_t(e_{i2}))^2+\sum_{e \in \inedg(v_i)\setminus\{e_{i1},e_{i2}\}} (\wload_t(e))^2 \ge\\
&\ge \left(\wload_t(e_{i1})-\left\lfloor\frac{\delta_i}{2}\right\rfloor\right)^2+\left(\wload_t(e_{i2})+\left\lfloor\frac{\delta_i}{2}\right\rfloor\right)^2 + \sum_{e \in \inedg(v_i)\setminus\{e_{i1},e_{i2}\}} (\wload_t(e))^2 \ge\\
&\ge \sum_{e \in \outedg(v_i)} (\wload_{t+1}(e))^2,
\sum_{e \in \inedg(v_i)} (\wload_t(e))^2 - \sum_{e \in \outedg(v_i)} (\wload_{t+1}(e))^2 \ge (\wload_t(e_{i1}))^2+(\wload_t(e_{i2}))^2-\left(\wload_t(e_{i1})-\left\lfloor\frac{\delta_i}{2}\right\rfloor\right)^2--\left(\wload_t(e_{i2})+\left\lfloor\frac{\delta_i}{2}\right\rfloor\right)^2=
2\left(\wload_t(e_{i1})-\wload_t(e_{i2})\right)\left\lfloor\frac{\delta_i}{2}\right\rfloor-2\left\lfloor\frac{\delta_i}{2}\right\rfloor^2 \ge 2 \left\lceil\frac{\delta_i}{2}\right\rceil\left\lfloor\frac{\delta_i}{2}\right\rfloor \ge \frac{\delta_i^2-1}{2},\Phi(\wload_t) - \Phi(\wload_{t+1}) \ge \sum_{i \le I} \frac{\delta_i^2-1}{2} \ge \left(\sum_{i \le I} \frac{\delta_i^2}{2}\right)  - \frac{I}{2} \ge \frac{(\sum_{i \le I} \delta_i)^2}{2I}-\frac{I}{2} \ge \ge\frac{(x-2D-1)^2}{4D} - D=\frac{(x-4D-1)(x-1)}{4D}.\sum_{e \in \mathcal{P}} \wload_t(e)^2 \le \frac{|\mathcal{P}|}{2} \cdot (\avg_{e \in \mathcal{P}}(\wload_t(e))+x/2)^2+\frac{|\mathcal{P}|}{2} \cdot (\avg_{e \in \mathcal{P}}(\wload_t(e)) - x/2)^2.\Phi(\wload_t) \le \mathcal{B}(\wload_0) + \sum_{\mathcal{P}} |\mathcal{P}|\left(\frac{x}{2}\right)^2.

\end{proof}

\end{document}
