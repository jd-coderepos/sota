





\documentclass[letterpaper,12pt, onecolumn, nodraft]{IEEEtran}

















































\hyphenation{op-tical net-works semi-conduc-tor IEEEtran}

\usepackage{graphicx,latexsym,epsfig,amssymb,amsmath,subfigure}
\usepackage{setspace}
\doublespace
\begin{document}


\title{A Novel Cyber-Insurance Model for Internet Security}




\author{Ranjan Pal, Leana Golubchik, and Konstantinos Psounis\\University of Southern California\\Email: \{rpal, leana, kpsounis\}@usc.edu}












\maketitle
\begin{abstract}
Internet users such as individuals and organizations are subject to different types of epidemic risks such as worms, viruses, and botnets. To reduce the probability of risk, an Internet user generally invests in self-defense mechanisms like antivirus and antispam software. However, such software does not completely eliminate risk. Recent works have considered the problem of residual risk elimination by proposing the idea of cyber-insurance. In reality, an Internet user faces risks due to security attacks as well as risks due to non-security related failures (e.g., reliability faults in the form of hardware crash, buffer overflow, etc.) . These risk types are often indistinguishable by a naive user. However, a cyber-insurance agency would most likely insure risks only due to security attacks. In this case, it becomes a challenge for an Internet user to choose the right type of cyber-insurance contract as standard optimal contracts, i.e., contracts under security attacks only, might prove to be sub-optimal for himself.

In this paper, we address the problem of analyzing cyber-insurance solutions when a user faces risks due to both, security as well as non-security related failures. We propose \emph{Aegis}, a novel cyber-insurance model in which the user accepts a fraction \emph{(strictly positive)} of loss recovery on himself and transfers rest of the loss recovery on the cyber-insurance agency. We mathematically show that given an option, Internet users would prefer Aegis contracts to traditional cyber-insurance contracts, under all premium types. This result firmly establishes the non-existence of traditional\footnote{Traditional cyber-insurance contracts are those which do not operate on non-security related losses in addition to security related losses, and do not give a user that option of being liable for a fraction of insurer advertised loss coverage. In Section \ref{sec-related} we cite recent important papers on traditional cyber-insurance and differentiate their work from our cyber-insurance model.} cyber-insurance markets when Aegis contracts are offered to users. We also derive an interesting counterintuitive result related to the Aegis framework: we show that an increase(decrease) in the premium of an Aegis contract \emph{may not} always lead to decrease(increase) in its user demand. In the process, we also state the conditions under which the latter trend and its converse emerge. Our work proposes a new model of cyber-insurance for Internet security that extends all previous related models by accounting for the extra dimension of non-insurable risks. Aegis also incentivizes Internet users to take up more personal responsibility for protecting their systems. 

\emph{Keywords:} Aegis, risks, insurable, non-insurable
\end{abstract}
\IEEEpeerreviewmaketitle
\section{Introduction} \label{sec-intro}
The Internet has become a fundamental and an integral part of our daily lives. Billions of people nowadays are using the Internet for various types of applications. However, all these applications are running on a network, that was built under assumptions, some of which are no longer valid for today's applications, e,g., that all users on the Internet can be trusted and that there are no malicious elements propagating in the Internet. On the contrary, the infrastructure, the users, and the services offered on the Internet today are all subject to a wide variety of risks. These risks include denial of service attacks, intrusions of various kinds, hacking, phishing, worms, viruses, spams, etc. In order to counter the threats posed by the risks, Internet users\footnote{The term `users' may refer to both, individuals and organizations.} have traditionally resorted to antivirus and anti-spam softwares, firewalls, and other add-ons to reduce the likelihood of being affected by threats. In practice, a large industry (companies like \emph{Symantec, McAfee,} etc.) as well as considerable research efforts are centered around developing and deploying tools and techniques to detect threats and anomalies in order to protect the Internet infrastructure and its users from the resulting negative impact.

In the past one and half decade, protection techniques from a variety of computer science fields such as cryptography, hardware engineering, and software engineering have continually made improvements. Inspite of such improvements, recent articles by Schneier \cite{sch} and Anderson \cite{ranr}\cite{amr} have stated that it is impossible to achieve a 100\% Internet security protection. The authors attribute this impossibility primarily to four reasons: 
\begin{itemize}
\item New viruses, worms, spams, and botnets evolve periodically at a rapid pace and as a result it is extremely difficult and expensive to design a security solution that is a panacea for all risks.
\item The Internet is a distributed system, where the system users have divergent security interests and incentives, leading to the problem of `misaligned incentives' amongst users. For example, a rational Internet user might well spend \LL - dd\ge01 - \theta\theta\theta1 -\theta\thetaWw_{0} + vvL_{S}L_{NS}I(L_{S})0\le I(L_{S}) \le L_{S}L_{S}L_{NS}[0,v]PP\theta\thetaP = (1 + \lambda)E(I(L_{S}))\lambda \theta\,\epsilon\,[0,1]\theta = 0.6ABCDug()L_{S}L_{NS}\alpha\alpha\betaf_{S}(L_{S})f_{NS}(L_{NS})g()A1B1C1(\lambda = 0)(\lambda > 0)E(W)\thetaI(L_{S}) = L_{S}\theta = 1E(I(L_{S})) = \alpha\cdot\int_{0}^{v}L_{S}\cdot f_{S}(L_{S})dL_{S} = Pu'(w_{0} + v - L_{NS} - P) > u'(w_{0} + v - P)\, \forall L_{NS} > 0\frac{dE(W)}{d\theta} < 0\theta = 1\theta\lambda > 0\frac{dE(W)}{d\theta} < 0\theta\blacksquareXYXYX \le_{ST} YVaR[X;p] \le VaR[Y; p]p\,\epsilon\,[0,1]VaR[X;p]F_{X}^{-1}(p)L_{NS}XYXYX \le_{ST} YVaR[X;p] \le VaR[Y; p]p\,\epsilon\,[0,1]VaR[X;p]F_{X}^{-1}(p)f_{S}(L_{S})\betaL_{NS}\int_{0}^{v}u'(w_{0} + v - L_{NS} - \theta(P))(-P)(1 - \alpha - \beta)\cdot f_{NS}(L_{NS})dL_{NS}u'(w_{0} + v - L_{NS} - \theta(P))L_{NS}u(W)uL_{NS}\frac{dE(W)}{d\theta}\theta\blacksquare\theta = 0E(W)L_{NS}L_{NS}[u(w_{0} + v - L_{NS} - \theta(P)) - u(w_{0} + v - L_{NS})]L_{NS}L_{NS}\blacksquare\thetaP = \lambda'E(L)\lambda'(1 + \lambda)ww_{0} + v\theta[0,1]\lambda \ge 1L = L_{S} + L_{NS}\lambda E(L) = \alpha \int_{0}^{v}L\cdot f_{S}(L)dL\theta = 1E\theta0 \le \theta \le 1\theta\theta = \theta^{*}W(x)WL = xW'(L) = -(1 - \theta) \le 0U'' < 0CUUA'(W)WU' > 0U'' < 0AUUWAA(W) = -\frac{U''(W)}{U'(W)}RR(W) = -\frac{WU''(W)}{U'(W)}\frac{d\theta^{*}}{d\lambda'}\frac{d\theta^{*}}{d\lambda'} \le 0\frac{d\theta^{*}}{d\lambda'}  \ge 0w\lambda'FUC\frac{d\theta^{*}}{d\lambda'} \ge 0\rho\,\epsilon\,\mathbb{R}\frac{d\theta^{*}}{d\lambda'} < 0\rho\,\epsilon\,\mathbb{R}L\,\epsilon\,[0,w]F(\cdot)L\frac{d\theta^{*}}{d\lambda'} = -\frac{E'_{\theta\lambda'}}{E''_{\theta}}\frac{d\theta^{*}}{d\lambda} \le 0E''_{\theta} < 0X + Y \ge M + N\frac{d\theta^{*}}{d\lambda'} \ge 0\frac{d\theta^{*}}{d\lambda'} \le 0\blacksquare\rho\,\epsilon\,\mathbb{R} - \{0\}\lambda\theta\thetaw\lambda'FUC\frac{d\theta^{*}}{d\lambda'} \ge 0R(W) > 1\frac{d\theta^{*}}{d\lambda'} < 0R(W) \le 1W\,\epsilon\,[W(w), W(0)]L\,\epsilon\,[0,w]R(W) > 1W-A(W(L))(w_{0} - L) < 0L\,\epsilon\,[0,w)L \le w_{0}L\,\epsilon\,[0,w]-\int_{L}^{w}\rho(L - \lambda'E(L))dF(x) < 0\int_{L}^{w}(x - \lambda'E(L))dF(x))(0,w)R(W) \le 1WL\,\epsilon\,[0,w]UC\frac{d\theta^{*}}{d\lambda'} \ge 0FR(W)\blacksquare$. 

\emph{Implication of Theorem 5.} The theorem implies that above a certain level of the degree of relative risk averseness, a user prefers Aegis contracts even if there is an increase in contract premiums. \\
\emph{Intuition Behind Theorem 5.} The coefficient of relative risk aversion is measured relative to the wealth of a user and thus more his wealth, lesser would be his concerns about losing money due to paying more cyber-insurance premiums, and not getting coverage on being affected by a non-security failure. The intuition is similar for the case when below a certain threshold of relative risk averseness, users reduce their demand for Aegis contracts. 

\section{Related Work}	\label{sec-related}
The field of cyber-insurance in networked environments has been fueled
by recent results on the amount of individual user self-defense investments
in the presence of network externalities\footnote{
	An externality  is a positive (external benefits) or negative (external costs) impact on an user not directly involved in an economic transaction.}.
The authors in \cite{gccr}\cite{jaw}\cite{leb5}\cite{leb4}\cite{mybm}\cite{oom} mathematically show
that Internet users invest too little in self-defense mechanisms relative
to the socially efficient level, due to the presence of network externalities.
These works highlight the role of positive externalities in preventing
users from investing optimally in self-defense. Thus, one challenge to
improving overall network security lies in incentivizing end-users to
invest in a sufficient amount of self-defense in spite of the
positive externalities they experience from other users in the network.
In response to this challenge, the works in \cite{leb5}\cite{leb4} modeled
network externalities and showed that a tipping phenomenon is possible,
i.e., in a situation where the level of self-defense is low, if a certain
fraction of population decides to invest in self-defense mechanisms,
then a large cascade of adoption in security features could be triggered,
thereby strengthening the overall Internet security. However, these
works did not state how the tipping phenomenon could be realized in
practice. In a series of recent works \cite{leb3}\cite{leb}, Lelarge and Bolot
have stated that under conditions of
no \emph{information asymmetry} \cite{wik}\cite{hv} between the insurer and
the insured, cyber-insurance \emph{incentivizes} Internet user investments
in self-defense mechanisms, thereby paving the path to triggering a cascade
of adoption. They also showed that investments in both self-defense mechanisms
and insurance schemes are quite inter-related in maintaining a socially
efficient level of security on the Internet. In a follow up work on joint self-defense and cyber-insurance investments, the authors in \cite{pg} show that Internet users invest more efficiently in self-defense investments in a cooperative environment when compared to a non-cooperative one, in relation to achieving a socially efficient level of security on the Internet. 

In spite of Lelarge and Bolot highlighting the role of cyber-insurance for
networked environments in incentivizing increasing of user security investments,
it is common knowledge that the market for cyber-insurance has not yet
blossomed with respect to its promised potential. Most recent works \cite{rabohme} \cite{ssfw} have
attributed this to
(1) \emph{interdependent security} (i.e., the effects of security investments
of a user on the security of other network users connected to it),
(2) \emph{correlated risk} (i.e., the risk faced by a user due to risks
faced by other network users),
and (3) \emph{information asymmetries} (i.e., the asymmetry between the insurer
and the insured due to one having some specific information about its risks
that the other does not have). In a recent work \cite{plg}, the authors have designed mechanisms to overcome the market existence problem due to information asymmetry, and show that a market for cyber-insurance exists in a single cyber-insurer setting.

However, none of the above mentioned works related to cyber-insurance address the scenario where a user faces risks due to security attacks as well as due to non-security related failures. The works consider attacks that occur due to security lapses only. In reality, an Internet user faces both types of risks and cannot distinguish between the types that caused a loss. Under such scenarios, it is not obvious that users would want to rest the full loss recovery liability to a cyber-insurer. We address the case when an Internet user faces risks due to both security as well as non-security problems, and show that users always prefer to rest some liability upon themselves, thus de-establishing the market for traditional cyber-insurance.
However, the Aegis framework being a type of a cyber-insurance framework also faces problems identical to the traditional cyber-insurance framework, viz., that of interdependent security, correlated risk, and information asymmetry. 

\section{Conclusion} \label{sec-conslusion}
In this paper we proposed Aegis, a novel cyber-insurance model in which an Internet user accepts a fraction (strictly positive) of loss recovery on himself and transfers the rest of the loss recovery on the cyber-insurance agency. Our model is specifically suited to situations when a user  cannot distinguish between similar types of losses that arise due to either a security attack or a non-security related failure. We showed that given an option, Internet users would prefer Aegis contracts to traditional cyber-insurance contracts, under all premium types. The latter result firmly establishes the non-existence of traditional cyber-insurance markets when Aegis contracts are offered to users. Furthermore, the Aegis model incentivizes risk-averse Internet users to invest more in taking care of their own systems than simply rest the entire coverage liability upon a cyber-insurer. We also derived two interesting counterintuitive results related to the Aegis framework, i.e., we showed that an increase (decrease) in the premium of an Aegis contract \emph{may not} always lead to a decrease (increase) in its user demand. As part of future work, we plan to investigate adverse selection and information asymmetry issues in Aegis contracts. 

\newpage
\bibliography{alluvion}
\bibliographystyle{plain}





\end{document}
