\label{sec:program}
The size of the DPLL formalization 
is approximately 5500  lines of Minlog code. The extracted program comes to 300 lines of code as a Minlog term and  600 lines of Haskell code. In the following we present two versions of our extracted solver: one optimized with $\forallnc$ quantifiers which we shall refer to as the $\forallnc$ solver, and the other without these optimizations which we shall refer to as the $\forall$ solver.
  
The $\forall$ solver takes a CNF formula $\Delta$ represented as a list of
clauses as input, and produces either a model of $\Delta$ or a
derivation of its unsatisfiability.
Models are represented as functions from literals to booleans. An
algebraic data type for DPLL derivations is automatically generated
from its inductive definition in Minlog. It has five constructors, one
for each of the DPLL rules in Definition~\ref{def:derivable-DPLL}:
\begin{verbatim}
      data Algdpll = CConflict Valu For
                   | CElim Valu For Cla Lit Algdpll
                   | CUnit Valu For Lit Algdpll
                   | CRed Valu For Cla Lit Algdpll
                   | CSplit Valu For Lit Algdpll Algdpll
        deriving (Show, Read, Eq, Ord)
\end{verbatim}
Each constructor takes a formula and a valuation as arguments. The
formula itself never changes during the proof and is only part of the algebra for the purpose of proving correctness and does not play a role in any computation. While the valuation changes
during the proof search, these changes can be captured by indicating
which literal was added by the $\Unit$ and $\Split$ rules, thus making
the valuation redundant as well.  We added $\NC$-quantifiers to the
definition by hand in order to remove redundant
computational content, resulting in
\begin{verbatim}
      data Algdpll = CConflict
                   | CElim Cla Lit Algdpll
                   | CUnit Lit Algdpll
                   | CRed Cla Lit Algdpll
                   | CSplit Lit Algdpll Algdpll
        deriving (Show, Read, Eq, Ord)
\end{verbatim}


The control structure of the program closely follows the structure
of the case distinctions and proofs by induction performed in the
proof. Lemmas invoked during the proof are extracted separately and
called as procedures. Since the proof is by general induction along a
measure, the main body of the program is using general recursion along
the same measure.
























