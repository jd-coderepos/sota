\section{Experiments}

\subsection{Experimental Setup}







Our framework requires specifying 
(1) a set of building block molecules , (2) a target molecule dataset , (3) a reaction dataset , (4) a retrosynthetic planning algorithm , and (5) a backward reaction model , a reference backward reaction model , a forward reaction model . The details of the components are as follows.


\textbf{Dataset.}
For building blocks , we use all of 231M commercially available molecules present in \emph{eMolecules}.\footnote{\url{http://downloads.emolecules.com/free/2019-11-01/}} Note that chemists can choose another set of molecules for  depending on their own circumstances, such as a financial budget. For the target molecules , we choose synthesizable molecules from  and reactions in the United States Patent Office (USPTO) database \citep{lowe2012extraction}. To this end, we follow the procedure described by \citet{chen2020retro} and then obtain 299202 target molecules. For the reaction dataset , we use reactions extracted from USPTO, following training/validation/test splits by \citet{chen2020retro}.

\textbf{Retrosynthetic planning algorithm.}
We use \textsc{Retro*} and \textsc{Retro*-0} \citep{chen2020retro} as the search algorithm  in our framework. \textsc{Retro*-0} denotes its variant not relying on the value function, which estimates the cost to synthesize a given molecule, proposed by \citet{chen2020retro}. 

\textbf{Model.}
We use template-based 2-layer MLPs \cite{segler2017neural} for parameterizing backward reaction model , reference backward reaction model , and forward reaction model . More specifically, the models predict a plausible reaction template among a pre-defined set of templates , i.e., they can be considered as classification models. 
We use RDChiral \citep{coley2019rdchiral} for extracting reaction templates from the USPTO database, and it results in 380k reaction templates. We use the Morgan fingerprint \citep{rogers2010extended}, of radius 2 with 2048 bits, as an input of MLPs.

\begin{table*}[ht!]
\caption{Performance of backward reaction model and retrosynthetic planning in the USPTO dataset. The backward reaction model is evaluated using \textsc{Top-1} and \textsc{Top-10} exact match accuracy (\%). Retrosynthetic planning is evaluated using \textsc{Succ. Rate }, \textsc{Succ. Rate }, \textsc{Length}, \textsc{Time}, and \textsc{Cost}.  denotes the limit of backward reaction model calls. \textsc{Length} is the number of reactions in a route. \textsc{Time} is measured by the number of backward reaction model calls, with a hard limit of 500. The experimental results of \textsc{Greedy DFS}, \textsc{MCTS}, and \textsc{DFPN-E} are from \citet{chen2020retro}.  The best results are marked in bold. We use brackets to report the relative gains over each counterpart that does not use our framework.}
\label{tab:1_main}
\vspace{-0.1in}
\begin{center}
\begin{sc}
\resizebox{\textwidth}{!}{
\begin{tabular}{l cc ccccc}
\toprule
& \multicolumn{2}{c}{Reactions} & \multicolumn{5}{c}{Reaction pathways} \\
\cmidrule(lr){2-3} \cmidrule(lr){4-8}
Algorithm & Top-1  & Top-10  &\thead{Succ. rate  \\ } & \thead{Succ. rate  \\ } & Length  & Time  & Cost  \\
\midrule
Greedy DFS & - & - & - & 22.63 & - & 388.15  &  - \\
MCTS & - & - & - & 33.68 & - & 370.51  &  - \\
DFPN-E & - & - & - & 55.26 & - & 279.67  &  - \\
\midrule
Retro*-0 & \textbf{44.53} & 72.71 & 27.37 & 79.47 & 11.21 & 208.09  &  19.40 \\
\rowcolor{Gray}Retro*-0 + ours      & 44.03 & 73.14      & 57.37   & \textbf{96.32} &  \textbf{7.69} & \textbf{96.22}  &  \textbf{11.66} \\
\rowcolor{Gray} & (-1.12\%) & (+0.59\%) & (+109.62\%) & (+21.20\%) & (-31.40\%) & (-53.76\%) & (-39.90\%) \\
\midrule
Retro* & \textbf{44.53} & 72.71 & 44.21 & 86.84 & 9.71 & 157.11  &  15.33 \\
\rowcolor{Gray}Retro* + ours        & 44.03 & \textbf{73.15} & \textbf{57.89} & 91.05 & 8.74 & 100.15  &  15.23 \\
\rowcolor{Gray} & (-1.12\%) & (+0.61\%) & (+30.94\%) & (+4.85\%) & (-9.99\%) & (-36.25\%) & (-0.65\%) \\
\bottomrule
\end{tabular}}
\end{sc}
\end{center}
\vskip -0.1in
\end{table*} \begin{figure}[t]
\vspace{0.1in}
    \centering
    \includegraphics[width=0.45\textwidth]{figure/2_main_graph.png}
    \caption{Success rate (\%) under varying limits of backward reaction model calls. Our framework outperforms the best baselines, \textsc{Retro*-0}, regardless of the limits.}
    \label{fig:main_graph}
\end{figure} \begin{table}[t]
\caption{Ablation study on augmentation via the forward reaction model. \textsc{Search} indicates the type of search algorithm . \textsc{Aug} indicates the augmentation via the forward reaction model.  denotes the limit of backward reaction model calls.} 
\label{tab:3_ablation_aug}
\vskip 0.1in
\begin{center}
\begin{sc}
\begin{tabular}{lc ccc}
\toprule
& &
\multicolumn{3}{c}{Succ. Rate } \\
\cmidrule(lr){3-5}
Search & Aug &  &  &  \\
\midrule
Retro*-0 &         & 46.84 & 81.05 & 92.63 \\
Retro*-0 &  & \textbf{50.00} & \textbf{83.16} & \textbf{93.16} \\
\midrule
Retro* &         & \textbf{51.58} & 82.11 & 88.95 \\
Retro* &  & \textbf{51.58} & \textbf{83.68} & \textbf{90.00} \\
\bottomrule
\end{tabular}
\end{sc}
\end{center}
\vskip -0.1in
\end{table} \textbf{Training.}
The reference backward reaction model  and forward reaction model  are trained using reactions in the dataset  and then frozen before conducting our framework. Instead of training a reference backward reaction model from scratch, we use the backward reaction model trained by \citet{chen2020retro}\footnote{\url{https://github.com/binghong-ml/retro\_star}} for our reference backward model . The forward reaction model  is trained with a learning rate of 0.001 for 100 epochs. The parameters of the backward reaction model  are initialized to that of the reference backward reaction model . 
During the self-improving procedure in our framework, the backward reaction model  is trained with a learning rate of 0.0001 for 20 epochs.
Adam optimizer \citep{kingma2014adam} is used with a mini-batch of size 1024 for training all the models. We iterate our overall procedure three times.











\textbf{Filtering thresholds.}
In our framework, there exists two thresholds: (1)  for removing unrealistic reactions from success routes via a reference backward reaction model, (2)  for rejecting unconfident augmented reaction generated by a forward reaction model. We set both thresholds  as 0.8. Namely, we filter out reactions of which likelihood under the corresponding model is under 0.8.


















\begin{figure*}[t]
\vspace{0.1in}
\centering
\begin{subfigure}{0.45\textwidth}
\includegraphics[width=0.95\linewidth]{figure/4_iter_acc.png}
\caption{\textsc{Top-10} accuracy under multiple iteration}
\end{subfigure}
\begin{subfigure}{0.45\textwidth}
\includegraphics[width=0.95\linewidth]{figure/4_iter_succ.png}
\caption{Success rate under multiple iteration}
\end{subfigure}
\caption{We repeat our procedure multiple times and investigate the performance of the backward reaction model and the retrosynthetic planning. Iteration 0 is vanilla \textsc{Retro*-0} or \textsc{Retro*}, which our framework is not applied yet. As we iterate our framework, we can further improve the performance of retrosynthetic planning while maintaining the reliability of the backward reaction model.}
\label{fig:iter}
\end{figure*} \begin{table*}[ht!]
\caption{Experimental results of our framework with different filtering thresholds  in reaction extraction step using the reference backward model. As we do not filter out any reactions from success routes, we suffer performance degradation in \textsc{Top-1} and \textsc{Top-10} accuracy, as unrealistic reactions can be included in the success routes. If we filter out unrealistic reactions using log-likelihood under the reference backward model, we can improve the performance of retrosynthetic planning while maintaining that of the backward reaction model. We report mean and standard deviation across five independent runs.}
\label{tab:2_ablation_delta}
\begin{center}
\begin{sc}


\begin{tabular}{l cc ccccc}
\toprule
& \multicolumn{2}{c}{Reactions} & \multicolumn{5}{c}{Reaction pathways} \\
\cmidrule(lr){2-3} \cmidrule(lr){4-8}
\thead{Filtering \\ Threshold } & Top-1  & Top-10  &\thead{Succ. rate  \\ ()} & \thead{Succ. rate  \\ ()} & Length  & Time  & Cost \\
\midrule
0.9  & 44.45 \stdv{0.01} & 73.29 \stdv{0.01} & 48.42 \stdv{0.88} & \textbf{92.21} \stdv{0.70} & 8.77 \stdv{0.11} & 130.66 \stdv{1.69} & 13.23 \stdv{0.40} \\
0.8  & \textbf{44.52} \stdv{0.01} & \textbf{73.30} \stdv{0.01} & 47.68 \stdv{1.23} & 92.00 \stdv{0.21} & \textbf{8.74} \stdv{0.08} & \textbf{129.70} \stdv{2.15} & \textbf{13.01} \stdv{0.18} \\
0.7  & 44.50 \stdv{0.02} & 73.28 \stdv{0.01} & 45.79 \stdv{0.58} & 91.47 \stdv{0.84}  & 8.82 \stdv{0.21} & 131.91 \stdv{1.84} & 13.15 \stdv{0.38} \\
0.6  & 44.48 \stdv{0.01} & 73.28 \stdv{0.01} & 44.11 \stdv{0.52} & 90.95 \stdv{1.02}  & 8.92 \stdv{0.28} & 130.14 \stdv{1.02} & 13.55 \stdv{0.62} \\
0.5  & 44.46 \stdv{0.01} & 73.26 \stdv{0.00} & 43.79 \stdv{0.61} & 91.05 \stdv{0.67}  & 8.95 \stdv{0.20} & 130.39 \stdv{2.69} & 13.24 \stdv{0.43} \\
0    & 41.04 \stdv{0.02} & 71.94 \stdv{0.01} & \textbf{54.32} \stdv{0.39} & 83.58 \stdv{0.39}  & 9.91 \stdv{0.09} & 149.16 \stdv{1.90} & 18.23 \stdv{0.18} \\
\bottomrule
\end{tabular}

\end{sc}
\end{center}
\vskip -0.1in
\end{table*} \textbf{Evaluation.}
We evaluate retrosynthetic planning for 190 target molecules in a limited time budget (i.e., the number of calls of backward reaction model ) following \citet{chen2020retro}. In the limit, we measure the success rate, the average time of planning, the average length, and the cost of discovered routes. 
Note that we consider the cost of a route as the summation of negative log-likelihoods of reactions in the route following \citet{chen2020retro}, i.e., .
We evaluate the backward reaction model using widely-used top- exact match accuracy in the test split of the reaction dataset . 

\textbf{Baselines.}
To demonstrate the effectiveness of our framework, we compare our method with existing retrosynthetic planning frameworks such as \textsc{Greedy DFS}, \textsc{MCTS} \cite{segler2018planning}, \textsc{DFPN-E} \cite{kishimoto2019depth}, \textsc{Retro*}, and \textsc{Retro*-0} \cite{chen2020retro}. We note that the considered baselines focus on designing efficient search algorithms for retrosynthetic planning, while our main contribution is the training of the backward reaction model towards maximizing the performance of retrosynthetic planning. 





\subsection{Main Result}
We report the results for evaluating the performance of our framework and baselines in Table \ref{tab:1_main} and Figure \ref{fig:main_graph}. We observe that combining our framework with \textsc{Retro*} and \textsc{Retro*-0} significantly outperforms the baselines, including the original \textsc{Retro*} and \textsc{Retro*-0}. For example, \textsc{Retro*-0+ours} achieves the success rate of  with a computation limit of , while \textsc{Retro*} achieves  as the best baseline. We also observe that \textsc{Retro*-0+ours} and \textsc{Retro*+ours} outperforms \textsc{Retro*-0} and \textsc{Retro*} significantly in terms of other evaluation metrics, e.g., the length and the cost of discovered reaction pathways.\footnote{If a reaction pathway is failed to be found, the length and the cost are set to two times of maximum length and cost among the ground-truth pathways of , respectively, and the time is set to the limit of backward reaction model calls, i.e., 500.} Such a result demonstrates the effectiveness of our framework for retrosynthetic planning. 

\begin{figure*}[t!]
\vspace{0.1in}
\centering
\begin{subfigure}{0.44730671739\textwidth}
\centering
\includegraphics[width=0.9\linewidth]{figure/3_case_study_ours.png}
\caption{Reaction pathway from \textsc{Retro*-0 + Ours}}
\end{subfigure}
\begin{subfigure}{0.50269328261\textwidth}
\centering
\includegraphics[width=0.9\linewidth]{figure/3_case_study_baseline.png}
\caption{Reaction pathway from \textsc{Retro*-0}}
\end{subfigure}
\caption{Reaction pathways from the same target molecule, searched by (a) \textsc{Retro*-0 + Ours} and (b) \textsc{Retro*-0}. In the example, our reaction pathway has a shorter length, which implies that our solution has better quality than that from \textsc{Retro*-0}, as shorter reaction pathways are easier to be conducted in laboratories.}
\label{fig:case_study}
\end{figure*} On the other hand, our backward reaction model does not suffer from a drop in \textsc{Top-} accuracy. Instead, they even outperform the backward reaction model trained on the reaction dataset  via supervised learning in terms of \textsc{Top-10} accuracy. 
We understand that such an improvement in the \textsc{Top-10} accuracy comes from "diverse" solution candidates generated by our backward reaction model, which is encouraged by being trained on a large variety of samples, e.g., augmented reactions.
Intriguingly, we also observe that \textsc{Retro*-0+ours} performs similarly with \textsc{Retro*+ours} without the help of an additional value function for guiding the search algorithm, i.e., \textsc{Retro*}. We hypothesize that the performance of the search algorithm may saturate when the backward reaction model appropriately adapts to the search algorithm. This highlights the importance of training an appropriate backward reaction model for retrosynthetic planning. 



















\subsection{Ablation Study}

Next, we conduct ablation studies on our framework to investigate the effect of (1) removing the reaction augmentation procedure, (2) varying the number of iterations, and (3) varying the filtering threshold  for realistic reactions. 
For the ablation studies on (1) and (3), our overall procedure is conducted a single time, i.e., we do not iterate our framework multiple times.


\textbf{Effectiveness of reaction augmentation.}
To evaluate whether if the proposed reaction augmentation is effective, we evaluate our framework without the reaction augmentation scheme. As shown in Table \ref{tab:3_ablation_aug}, our framework achieves a higher success rate for finding the reaction pathway for the target molecules. This validates the effectiveness of our reaction augmentation scheme based on the forward reaction model. 




\textbf{Number of iterations.} 
In Figure \ref{fig:iter}, we report the performance of the backward reaction model and the success rate for finding the reaction pathways over multiple iterations. The result demonstrates that iterating our framework improves the success rate of finding a reaction pathway while maintaining the accuracy of the backward reaction model. This indeed validates that our framework is effective without compromise in the ability to model realistic reactions.



\textbf{Filtering threshold .}
\label{ablation:filter}
To recognize the effectiveness of filtering out unrealistic reactions within success routes, we compare the performance of our framework with varying thresholds for filtering out reactions.\footnote{In this experiment, we do not include the augmented reaction data via the forward reaction model.} 
As shown in Table \ref{tab:2_ablation_delta}, the performance of our backward reaction model is (a) improved when using the threshold, i.e., using Thr0 instead of Thr0, and (b) robust to the change of hyper-parameter, i.e., varying the threshold from . In particular, setting \textsc{Filtering Threshold}  to 0, the top-1 exact match accuracy of our backward reaction model degrades from 44.52\% to 41.04\%. On the other hand, if we filter out unrealistic reactions by the reference backward reaction model, our backward reaction model retains its reliability and enjoys the improved performance in retrosynthetic planning without compromise.


\subsection{Case Study}
In Figure \ref{fig:case_study}, we compare reaction pathways found by \textsc{Retro*-0+ours} and \textsc{Retro*-0} for the same target molecule. In the example, we observe that both frameworks can find realistic reaction pathways for the given target molecule. However, the reaction pathway searched by \textsc{Retro*-0+ours} is preferable according to our predefined criteria, i.e., the number of reactions required for execution.