The amalgamation of typed objects allows to combine objects of different types provided that they agree on a common subtype. 
This concept is already known in the context of different types of Petri net processes, such as open net processes \cite{BCEH01}
and algebraic high-level processes \cite{EG11}, which can be seen as special kinds of typed objects. 
In this section, we introduce a general definition for the amalgamation of typed objects. Moreover, we extend the concept to 
the amalgamation of positive nested conditions and their solutions. 

As  required for amalgamation, we discuss under which conditions morphisms can be composed via a span of restriction morphisms. 
Two morphisms $g_B$ and $g_C$ ``agree'' in a morphism $g_D$, if $g_D$ can be constructed as a common restriction and can be used 
as a composition interface for $g_B$ and $g_C$ as in \autoref{def:agreement}.

\begin{definition}[Agreement and Amalgamation of Typed Objects]\label{def:agreement}
Given a span $TG_B \stackrel{tg_{DB}}{\longleftarrow} TG_D \stackrel{tg_{DC}}{\longrightarrow} TG_C$, 
with $tg_{DB}, tg_{DC} \in \M$ and typed objects $G_B \stackrel{g_{B}}{\rarr} TG_B$, $G_C \stackrel{g_{C}}{\rarr} TG_C$ and $G_D \stackrel{g_{D}}{\rarr} TG_D$. 
We say $g_{B},g_{C}$ \emph{agree} in $g_{D}$, if $g_{D}$ is a restriction of $g_{B}$ and $g_{C}$, i.e.,~$Restr_{tg_{DB}}(g_{B}) = g_{D} = Restr_{tg_{DC}}(g_{C})$.

Given pushout (1) below with all morphisms in \M and typed objects $g_B, g_C$ agreeing in $g_D$. 
A morphism $g_A : G_A \rarr TG_A$ is called \emph{amalgamation} of $g_B$ and $g_C$ over $g_D$, written $g_A = g_B +_{g_D} g_C$, 
if the outer square is a pushout and $g_B, g_C$ are restrictions of $g_A$.

	\xcmatrix{@R-3ex@C9ex}{
		& & G_D \ar[d]^{g_D} \ar[ddll] \ar[ddrr]
		& &\\
		& & TG_D \ar[dl]^{tg_{DB}} \ar[dr]_{tg_{DC}}
		& & \\
		G_B \ar[r]_{g_B} \ar[ddrr]
		& TG_B \ar[dr]^{tg_{BA}} \ar@{}[rr]|{(1)}
		& 
		& TG_C \ar[dl]_{tg_{CA}}
		& G_C \ar[l]^{g_C} \ar[ddll] \\
		& & TG_A 
		& & \\
		& & G_A \ar[u]^{g_A}
	}
\end{definition}




\autoref{fact:amalgamation} is essentially based on the horizontal VK property.

\begin{fact}[Amalgamation of Typed Objects]\label{fact:amalgamation}
Given pushout (1) with all morphisms in \M as in \autoref{def:agreement}.
\begin{description} 
	\item \textbf{Composition.}
		Given $g_B,g_C$ agreeing in $g_D$, then there exists a unique amalgamation $g_A = g_B +_{g_D} g_C$. 
	\item \textbf{Decomposition.}
		Vice versa, given $g_A: G_A \rarr TG_A$, there are unique restrictions $g_B, g_C,$ and $g_D$ of $g_A$ such that $g_A = g_B +_{g_D} g_C$.
\end{description}
Here and in the following, uniqueness means uniqueness up to isomorphism.
\end{fact}

\begin{proof}
Given $g_B, g_C$ agreeing in $g_D$, we have that the upper two trapezoids are pullbacks. Now we construct $G_A$ as pushout over $G_B$ and $G_C$ via $G_D$, such that the outer diamond is a pushout. This leads to a unique induced morphism $g_A: G_A \rarr TG_A$, such that the diagram commutes and via the horizontal VK property we get that the lower two trapezoids are pullbacks and therefore $g_A = g_B +_{g_D} g_C$.

Vice versa, we can construct $G_B, G_C, G_D$ as restrictions such that the trapezoids become pullbacks, where $g_A: G_A \rarr TG_A$ and $TG_A, TG_B, TG_C, TG_D$ are given such that (1) is a pushout with \M-morphisms only. Then the horizontal VK property implies that the outer diamond is a pushout and $g_A$ is unique because of the universal property and $g_A = g_B +_{g_D} g_C$.

The uniqueness (up to isomorphism) of the amalgamated composition and decomposition constructions follows from uniqueness of pushouts
and pullpacks up to isomorphism.
\end{proof}

\begin{example}[Amalgamation of Typed Objects]\label{ex:amalgamation-objects}
	Figure~\ref{fig:amalgamation-objects} shows a pushout of type graphs $TG_A$, $TG_B$, $TG_C$ and $TG_D$. 
	\begin{description}
		\item \textbf{Composition.}
			Consider the typed graphs $G_B$, $G_C$ and $G_D$ typed over $TG_B$, $TG_C$ and $TG_D$, respectively. Graph $G_D$, containing 
			the same nodes as $G_B$ and $G_C$ and no edges, is the common restriction of $G_B$ and $G_C$. So, the type morphisms $g_B$ and $g_C$
			agree in $g_D$, which by \autoref{fact:amalgamation} means that there is an amalgamation $g_A = g_B +_{g_D} g_C$. It can be obtained
			by computing the pushout of $G_B$ and $G_C$ over $G_D$, leading to the graph $G_A$ that contains the \texttt{b}-edges of $G_B$ as well
			as the \texttt{c}-edges of $G_C$. The type morphism $g_A$ is induced by the universal property of pushouts, mapping all edges
			in the same way as $g_B$ and $g_C$.
		\item \textbf{Decomposition.}
			Vice versa, consider the graph $G_A$ typed over $TG_A$. We can restrict $G_A$ to the type graphs $TG_B$ and $TG_C$, leading
			to typed graphs $G_B$ and $G_C$, containing only the \texttt{b}- respectively \texttt{c}-edges of $G_A$. Restricting the graphs
			$G_B$ and $G_C$ to type graph $TG_D$, we get in both cases the graph $G_D$ that contains no edges, and we have that $g_A = g_B +_{g_D} g_C$.
	\end{description}
	
	\begin{figure}[htb]\centering
	\includegraphics[width=0.5\textwidth]{figures/Amalgamation-Graphs}
	\caption{Amalgamation of typed graphs}\label{fig:amalgamation-objects}
	\end{figure}
\end{example}

We already defined the restriction of positive nested conditions (\autoref{def:restrAC}) and their solutions \linebreak (\autoref{def:restrSolution}). Now we want to consider the case that we have two conditions, which have a common restriction and can be amalgamated.


\begin{definition}[Agreement and Amalgamation of Positive Nested Conditions]\label{def:agreement-amalgamation-conditions}
	Given a pushout (1) below with all morphisms in \M. 
	Two positive nested conditions $ac_{P_B}$ typed over $\TG_B$ and $ac_{P_C}$ typed over $\TG_C$ \emph{agree} in $ac_{P_D}$
	typed over $\TG_D$ if $ac_{P_D}$ is a restriction of $ac_{P_B}$ and $ac_{P_C}$.
	
	Given $ac_{P_B}$ and $ac_{P_C}$ agreeing in $ac_{P_D}$ then a positive nested condition 
	$ac_{P_A}$ typed over $TG_A$ is called \emph{amalgamation} of $ac_{P_B}$ and $ac_{P_C}$ over
	$ac_{P_D}$, written $ac_{P_A} = ac_{P_B} +_{ac_{P_D}} ac_{P_C}$, if $ac_{P_B}$ and $ac_{P_C}$ are restrictions of $ac_{P_A}$
	and $t_{PA} = t_{PB} +_{t_{PD}} t_{PC}$.
	In particular, we have $true_A = true_B +_{true_D} true_C$, short $true = true +_{true} true$.

	\xcmatrix{@R-3ex@C9ex}{
		& & P_D \ACright{ac_{P_D}} \ar[d]^{t_{PD}} \ar[ddll] \ar[ddrr]
		& &\\
		& & TG_D \ar[dl]^{tg_{DB}} \ar[dr]_{tg_{DC}}
		& & \\
		P_B \ACleft{ac_{P_B}} \ar[r]_{t_{PB}} \ar[ddrr]
		& TG_B \ar[dr]^{tg_{BA}} \ar@{}[rr]|{(1)}
		& 
		& TG_C \ar[dl]_{tg_{CA}}
		& P_C \ACright{ac_{P_C}} \ar[l]^{t_{PC}} \ar[ddll] \\
		& & TG_A 
		& & \\
		& & P_A \ACright[4.5ex,-0.5ex]{ac_{P_A}} \ar[u]^{t_{PA}}
	}
\end{definition}

In the following \autoref{fact:amalgamation-conditions}, we give a construction for the amalgamation of positive nested conditions and in \autoref{thm:ACviaRestr} for the corresponding solutions.

\begin{fact}[Amalgamation of Positive Nested Conditions]\label{fact:amalgamation-conditions}
	Given a pushout (1) as in \autoref{def:agreement-amalgamation-conditions} with all morphisms in \M. 
	\begin{description}
		\item \textbf{Composition.} 
			If there are positive nested conditions $ac_{P_B}$ and $ac_{P_C}$ typed over $TG_B$ and $TG_C$, respectively,
			agreeing in $ac_{P_D}$ typed over $TG_D$, then there exists a unique positive nested condition $ac_{P_A}$ typed over $TG_A$ such that 
			$ac_{P_A} = ac_{P_B} +_{ac_{P_D}} ac_{P_C}$.
		\item \textbf{Decomposition.}
			Vice versa, given a positive nested condition $ac_{P_A}$ typed over $TG_A$, there are unique restrictions $ac_{P_B}$, $ac_{P_C}$ and $ac_{P_D}$ 
			of $ac_{P_A}$ such that $ac_{P_A} = ac_{P_B} +_{ac_{P_D}} ac_{P_C}$.
	\end{description}
	The amalgamated composition and decomposition constructions are unique up to isomorphism.
	
\end{fact}

\begin{remark}\label{rem:amalgamation-conditions}
	Given an amalgamation $ac_{P_A} = ac_{P_B} +_{ac_{P_D}} ac_{P_C}$ of positive nested conditions, we can conclude from the proof of \autoref{fact:amalgamation-conditions} (see \autoref{sec:appendixB}) that we also have corresponding amalgamations in each
	level of nesting.
\end{remark}

\begin{figure}[htb]\centering
\includegraphics[width=0.9\textwidth]{figures/Amalgamation-Conditions}
\caption{Amalgamation of positive nested conditions}\label{fig:amalgamation-conditions}
\end{figure}


\begin{example}[Amalgamation of Positive Nested Conditions]\label{ex:amalgamation-conditions}
	Figure~\ref{fig:amalgamation-conditions} shows a pushout of typed graphs $TG_A$, $TG_B$, $TG_C$ and $TG_D$, and four positive nested conditions
	$ac_{P_A}$, $ac_{P_B}$, $ac_{P_C}$ and $ac_{P_D}$ typed over $TG_A$, $TG_B$, $TG_C$ and $TG_D$, respectively. For simplicity, the figure
	contains only the type morphisms of the $P$s, but there are also corresponding type morphisms for the $C$s, mapping all \texttt{b}-edges to
	\texttt{b} and all \texttt{c}-edges to \texttt{c}. 
	There is $ac_{P_A} = \bigvee_{i \in \{1,2\}} ac_{C_{i,A}}$ with $ac_{C_{i,A}} = \exists(a_{i,A}, true)$ for $i = 1,2$, and $ac_{P_B}$,
	$ac_{P_C}$ and $ac_{P_D}$ have a similar structure.
	\begin{description}
		\item \textbf{Composition.}
			We have that $t_{P_D}$ is a common restriction of $t_{P_B}$ and $t_{P_C}$, and also that $a_{i,D}$ is a common
			restriction of $a_{i,B}$ and $a_{i,C}$ for $i=1,2$. Thus, $ac_{P_D}$ is a common restriction of $ac_{P_B}$ and $ac_{P_C}$
			which means that $ac_{P_B}$ and $ac_{P_C}$ agree in $ac_{P_D}$. So by \autoref{fact:amalgamation-conditions} there
			exists an amalgamation $ac_{P_A} = ac_{P_B} +_{ac_{P_D}} ac_{P_C}$, and according to \autoref{rem:amalgamation-conditions}
			it can be obtained as amalgamation of its components. This means that we have an amalgamation 
			$t_{P_A} = t_{P_B} +_{t_{P_D}} t_{P_C}$ with pushout of the $P$s as shown in \autoref{fig:amalgamation-conditions},
			as well as amalgamations of the corresponding type morphisms of the $C$s, leading to the
			pushouts depicted in \autoref{fig:amalgamation-conditions} by dotted arrows for the $C_1$s and by dashed arrows for the $C_2$s.
			The morphisms $a_{1,A}$ and $a_{2,A}$ are obtained by the universal property of pushouts.
		\item \textbf{Decomposition.}
			The other way around, considering the condition $ac_{P_A}$, we can construct the restrictions $ac_{P_B}$ and $ac_{P_C}$ by deleting
			the \texttt{c}- respectively \texttt{b}-edges. Then, restricting $ac_{P_B}$ and $ac_{P_C}$ to $TG_D$ by deleting all remaining edges,
			we obtain the same condition $ac_{P_D}$ such that $ac_{P_A} = ac_{P_B} +_{ac_{P_D}} ac_{P_C}$.
	\end{description}
\end{example}


In order to answer the question, under which conditions such amalgamated positive nested conditions are satisfied, we need to define an amalgamation of their solutions. Afterwards, we show in the proof of \autoref{thm:ACviaRestr} that a composition of two solutions via an interface leads to a unique amalgamated solution and that a given solution for an amalgamated positive nested condition is the amalgamation of its unique restrictions.

\begin{definition}[Agreement and Amalgamation of Solutions for Positive Nested Conditions]\label{def:agreement-amalgamation-solution}
	Given pushout (1) below with all morphisms in \M, an amalgamation of typed objects $g_A = g_B +_{g_D} g_C$,
	and an amalgamation of positive nested conditions $ac_{P_A} = ac_{P_B} +_{ac_{P_D}} ac_{P_C}$ 
	with corresponding matches $p_A = p_B +_{p_D} p_C$.
	\begin{enumerate}
		\item Two solutions $Q_B$ for $p_B \vDash ac_{P_B}$ and $Q_C$ for $p_C \vDash ac_{P_C}$ \emph{agree} in a solution $Q_D$ for $p_D \vDash ac_{P_D}$,
			if $Q_D$ is a restriction of $Q_B$ and $Q_C$.
		
		\item Given solutions $Q_B$ for $p_B \vDash ac_{P_B}$ and $Q_C$ for $p_C \vDash ac_{P_C}$ agreeing in a solution $Q_D$ for $p_D \vDash ac_{P_D}$, 
			then a solution $Q_A$ for $p_A \vDash ac_{P_A}$ is called \emph{amalgamation} of
			$Q_B$ and $Q_C$ over $Q_D$, written $Q_A = Q_B +_{Q_D} Q_C$, if $Q_B$ and $Q_C$ are restrictions of $Q_A$.
	\end{enumerate}

	\xcmatrix{@R-3ex}{
		& P_A \ACleft{ac_{P_A}} \ar[dr]_{p_A}
		&&&& 
		& P_C \ACright{ac_{P_C}} \ar[lllll]^{p_{CA}} \ar[dl]^{p_C}
		& \\
		&& G_A \ar[dr]_{g_A}
		&&
		& G_C \ar[lll]^{g_{CA}} \ar[dl]^{g_C}
		&& \\
		&&& TG_A \ar@{}[dr]|{(1)}
		& TG_C \ar[l]_{tg_{CA}}
		&&& \\
		&&& TG_B \ar[u]^{tg_{BA}}
		& TG_D \ar[l]_{tg_{DB}} \ar[u]_{tg_{DC}}
		&&& \\
		&& G_B \ar[uuu]_{g_{BA}} \ar[ur]^{g_B}
		&&
		& G_D \ar[lll]^{g_{DB}} \ar[uuu]_{g_{DC}} \ar[ul]_{g_D}
		&& \\
		& P_B \ACleft{ac_{P_B}} \ar[uuuuu]_{p_{BA}} \ar[ur]^{p_B}
		&&&& 
		& P_D \ACright{ac_{P_D}} \ar[lllll]^{p_{DB}} \ar[uuuuu]_{p_{DC}} \ar[ul]_{p_D}
		& 
	}
\end{definition}

\begin{remark}\label{rem:agreement-amalgamation-solution}
	Note that by assumption $g_A = g_B +_{g_D} g_C$ in the definition above we already have a pushout over the $G$s, and by 
	$ac_{P_A} = ac_{P_B} +_{ac_{P_D}} ac_{P_C}$ we also have a pushout over the $P$s. 
\end{remark}

\begin{theorem}[Amalgamation of Solutions for Positive Nested Conditions]\label{thm:ACviaRestr}
Given pushout (1) as in \autoref{def:agreement-amalgamation-solution} with all morphisms in \M, 
an amalgamation of typed objects $g_A = g_B +_{g_D} g_C$,
and an amalgamation of positive nested conditions $ac_{P_A} = ac_{P_B} +_{ac_{P_D}} ac_{P_C}$ 
with corresponding matches $p_A = p_B +_{p_D} p_C$.
\begin{description}
	\item \textbf{Composition.}
		Given solutions $Q_B$ for $p_B \vDash ac_{P_B}$ and $Q_C$ for $p_C \vDash ac_{P_C}$ agreeing in a solution $Q_D$ for $p_D \vDash ac_{P_D}$,
		then there is a solution $Q_A$ for $p_A \vDash ac_{P_A}$ constructed as 
		amalgamation $Q_A = Q_B +_{Q_D} Q_C$.
	\item \textbf{Decomposition.}
		Given a solution $Q_A$ for $p_A \vDash ac_{P_A}$, then there are 
		solutions $Q_B$, $Q_C$ and $Q_D$ for \linebreak
		$p_B \vDash ac_{P_B}$, $p_C \vDash ac_{P_C}$ 
		and $p_D \vDash ac_{P_D}$, respectively, which are constructed as
		restrictions $Q_B$, $Q_C$ and $Q_D$ of $Q_A$ such that
		$Q_A = Q_B +_{Q_D} Q_C$.
\end{description}
The amalgamated composition and decomposition constructions are unique up to isomorphism.
\end{theorem}








\begin{remark}\label{rem:amalgamation-solutions}
	From the proof of \autoref{thm:ACviaRestr} (see \autoref{sec:appendixB}) we can conclude that for a given amalgamation of solutions $Q_A = Q_B +_{Q_D} Q_C$, we also
	have corresponding amalgamations of its components.
\end{remark}