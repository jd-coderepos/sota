The amalgamation of typed objects allows to combine objects of different types provided that they agree on a common subtype. 
This concept is already known in the context of different types of Petri net processes, such as open net processes \cite{BCEH01}
and algebraic high-level processes \cite{EG11}, which can be seen as special kinds of typed objects. 
In this section, we introduce a general definition for the amalgamation of typed objects. Moreover, we extend the concept to 
the amalgamation of positive nested conditions and their solutions. 

As  required for amalgamation, we discuss under which conditions morphisms can be composed via a span of restriction morphisms. 
Two morphisms  and  ``agree'' in a morphism , if  can be constructed as a common restriction and can be used 
as a composition interface for  and  as in \autoref{def:agreement}.

\begin{definition}[Agreement and Amalgamation of Typed Objects]\label{def:agreement}
Given a span , 
with  and typed objects ,  and . 
We say  \emph{agree} in , if  is a restriction of  and , i.e.,~.

Given pushout (1) below with all morphisms in \M and typed objects  agreeing in . 
A morphism  is called \emph{amalgamation} of  and  over , written , 
if the outer square is a pushout and  are restrictions of .

	\xcmatrix{@R-3ex@C9ex}{
		& & G_D \ar[d]^{g_D} \ar[ddll] \ar[ddrr]
		& &\\
		& & TG_D \ar[dl]^{tg_{DB}} \ar[dr]_{tg_{DC}}
		& & \\
		G_B \ar[r]_{g_B} \ar[ddrr]
		& TG_B \ar[dr]^{tg_{BA}} \ar@{}[rr]|{(1)}
		& 
		& TG_C \ar[dl]_{tg_{CA}}
		& G_C \ar[l]^{g_C} \ar[ddll] \\
		& & TG_A 
		& & \\
		& & G_A \ar[u]^{g_A}
	}
\end{definition}




\autoref{fact:amalgamation} is essentially based on the horizontal VK property.

\begin{fact}[Amalgamation of Typed Objects]\label{fact:amalgamation}
Given pushout (1) with all morphisms in \M as in \autoref{def:agreement}.
\begin{description} 
	\item \textbf{Composition.}
		Given  agreeing in , then there exists a unique amalgamation . 
	\item \textbf{Decomposition.}
		Vice versa, given , there are unique restrictions  and  of  such that .
\end{description}
Here and in the following, uniqueness means uniqueness up to isomorphism.
\end{fact}

\begin{proof}
Given  agreeing in , we have that the upper two trapezoids are pullbacks. Now we construct  as pushout over  and  via , such that the outer diamond is a pushout. This leads to a unique induced morphism , such that the diagram commutes and via the horizontal VK property we get that the lower two trapezoids are pullbacks and therefore .

Vice versa, we can construct  as restrictions such that the trapezoids become pullbacks, where  and  are given such that (1) is a pushout with \M-morphisms only. Then the horizontal VK property implies that the outer diamond is a pushout and  is unique because of the universal property and .

The uniqueness (up to isomorphism) of the amalgamated composition and decomposition constructions follows from uniqueness of pushouts
and pullpacks up to isomorphism.
\end{proof}

\begin{example}[Amalgamation of Typed Objects]\label{ex:amalgamation-objects}
	Figure~\ref{fig:amalgamation-objects} shows a pushout of type graphs , ,  and . 
	\begin{description}
		\item \textbf{Composition.}
			Consider the typed graphs ,  and  typed over ,  and , respectively. Graph , containing 
			the same nodes as  and  and no edges, is the common restriction of  and . So, the type morphisms  and 
			agree in , which by \autoref{fact:amalgamation} means that there is an amalgamation . It can be obtained
			by computing the pushout of  and  over , leading to the graph  that contains the \texttt{b}-edges of  as well
			as the \texttt{c}-edges of . The type morphism  is induced by the universal property of pushouts, mapping all edges
			in the same way as  and .
		\item \textbf{Decomposition.}
			Vice versa, consider the graph  typed over . We can restrict  to the type graphs  and , leading
			to typed graphs  and , containing only the \texttt{b}- respectively \texttt{c}-edges of . Restricting the graphs
			 and  to type graph , we get in both cases the graph  that contains no edges, and we have that .
	\end{description}
	
	\begin{figure}[htb]\centering
	\includegraphics[width=0.5\textwidth]{figures/Amalgamation-Graphs}
	\caption{Amalgamation of typed graphs}\label{fig:amalgamation-objects}
	\end{figure}
\end{example}

We already defined the restriction of positive nested conditions (\autoref{def:restrAC}) and their solutions \linebreak (\autoref{def:restrSolution}). Now we want to consider the case that we have two conditions, which have a common restriction and can be amalgamated.


\begin{definition}[Agreement and Amalgamation of Positive Nested Conditions]\label{def:agreement-amalgamation-conditions}
	Given a pushout (1) below with all morphisms in \M. 
	Two positive nested conditions  typed over  and  typed over  \emph{agree} in 
	typed over  if  is a restriction of  and .
	
	Given  and  agreeing in  then a positive nested condition 
	 typed over  is called \emph{amalgamation} of  and  over
	, written , if  and  are restrictions of 
	and .
	In particular, we have , short .

	\xcmatrix{@R-3ex@C9ex}{
		& & P_D \ACright{ac_{P_D}} \ar[d]^{t_{PD}} \ar[ddll] \ar[ddrr]
		& &\\
		& & TG_D \ar[dl]^{tg_{DB}} \ar[dr]_{tg_{DC}}
		& & \\
		P_B \ACleft{ac_{P_B}} \ar[r]_{t_{PB}} \ar[ddrr]
		& TG_B \ar[dr]^{tg_{BA}} \ar@{}[rr]|{(1)}
		& 
		& TG_C \ar[dl]_{tg_{CA}}
		& P_C \ACright{ac_{P_C}} \ar[l]^{t_{PC}} \ar[ddll] \\
		& & TG_A 
		& & \\
		& & P_A \ACright[4.5ex,-0.5ex]{ac_{P_A}} \ar[u]^{t_{PA}}
	}
\end{definition}

In the following \autoref{fact:amalgamation-conditions}, we give a construction for the amalgamation of positive nested conditions and in \autoref{thm:ACviaRestr} for the corresponding solutions.

\begin{fact}[Amalgamation of Positive Nested Conditions]\label{fact:amalgamation-conditions}
	Given a pushout (1) as in \autoref{def:agreement-amalgamation-conditions} with all morphisms in \M. 
	\begin{description}
		\item \textbf{Composition.} 
			If there are positive nested conditions  and  typed over  and , respectively,
			agreeing in  typed over , then there exists a unique positive nested condition  typed over  such that 
			.
		\item \textbf{Decomposition.}
			Vice versa, given a positive nested condition  typed over , there are unique restrictions ,  and  
			of  such that .
	\end{description}
	The amalgamated composition and decomposition constructions are unique up to isomorphism.
	
\end{fact}

\begin{remark}\label{rem:amalgamation-conditions}
	Given an amalgamation  of positive nested conditions, we can conclude from the proof of \autoref{fact:amalgamation-conditions} (see \autoref{sec:appendixB}) that we also have corresponding amalgamations in each
	level of nesting.
\end{remark}

\begin{figure}[htb]\centering
\includegraphics[width=0.9\textwidth]{figures/Amalgamation-Conditions}
\caption{Amalgamation of positive nested conditions}\label{fig:amalgamation-conditions}
\end{figure}


\begin{example}[Amalgamation of Positive Nested Conditions]\label{ex:amalgamation-conditions}
	Figure~\ref{fig:amalgamation-conditions} shows a pushout of typed graphs , ,  and , and four positive nested conditions
	, ,  and  typed over , ,  and , respectively. For simplicity, the figure
	contains only the type morphisms of the s, but there are also corresponding type morphisms for the s, mapping all \texttt{b}-edges to
	\texttt{b} and all \texttt{c}-edges to \texttt{c}. 
	There is  with  for , and ,
	 and  have a similar structure.
	\begin{description}
		\item \textbf{Composition.}
			We have that  is a common restriction of  and , and also that  is a common
			restriction of  and  for . Thus,  is a common restriction of  and 
			which means that  and  agree in . So by \autoref{fact:amalgamation-conditions} there
			exists an amalgamation , and according to \autoref{rem:amalgamation-conditions}
			it can be obtained as amalgamation of its components. This means that we have an amalgamation 
			 with pushout of the s as shown in \autoref{fig:amalgamation-conditions},
			as well as amalgamations of the corresponding type morphisms of the s, leading to the
			pushouts depicted in \autoref{fig:amalgamation-conditions} by dotted arrows for the s and by dashed arrows for the s.
			The morphisms  and  are obtained by the universal property of pushouts.
		\item \textbf{Decomposition.}
			The other way around, considering the condition , we can construct the restrictions  and  by deleting
			the \texttt{c}- respectively \texttt{b}-edges. Then, restricting  and  to  by deleting all remaining edges,
			we obtain the same condition  such that .
	\end{description}
\end{example}


In order to answer the question, under which conditions such amalgamated positive nested conditions are satisfied, we need to define an amalgamation of their solutions. Afterwards, we show in the proof of \autoref{thm:ACviaRestr} that a composition of two solutions via an interface leads to a unique amalgamated solution and that a given solution for an amalgamated positive nested condition is the amalgamation of its unique restrictions.

\begin{definition}[Agreement and Amalgamation of Solutions for Positive Nested Conditions]\label{def:agreement-amalgamation-solution}
	Given pushout (1) below with all morphisms in \M, an amalgamation of typed objects ,
	and an amalgamation of positive nested conditions  
	with corresponding matches .
	\begin{enumerate}
		\item Two solutions  for  and  for  \emph{agree} in a solution  for ,
			if  is a restriction of  and .
		
		\item Given solutions  for  and  for  agreeing in a solution  for , 
			then a solution  for  is called \emph{amalgamation} of
			 and  over , written , if  and  are restrictions of .
	\end{enumerate}

	\xcmatrix{@R-3ex}{
		& P_A \ACleft{ac_{P_A}} \ar[dr]_{p_A}
		&&&& 
		& P_C \ACright{ac_{P_C}} \ar[lllll]^{p_{CA}} \ar[dl]^{p_C}
		& \\
		&& G_A \ar[dr]_{g_A}
		&&
		& G_C \ar[lll]^{g_{CA}} \ar[dl]^{g_C}
		&& \\
		&&& TG_A \ar@{}[dr]|{(1)}
		& TG_C \ar[l]_{tg_{CA}}
		&&& \\
		&&& TG_B \ar[u]^{tg_{BA}}
		& TG_D \ar[l]_{tg_{DB}} \ar[u]_{tg_{DC}}
		&&& \\
		&& G_B \ar[uuu]_{g_{BA}} \ar[ur]^{g_B}
		&&
		& G_D \ar[lll]^{g_{DB}} \ar[uuu]_{g_{DC}} \ar[ul]_{g_D}
		&& \\
		& P_B \ACleft{ac_{P_B}} \ar[uuuuu]_{p_{BA}} \ar[ur]^{p_B}
		&&&& 
		& P_D \ACright{ac_{P_D}} \ar[lllll]^{p_{DB}} \ar[uuuuu]_{p_{DC}} \ar[ul]_{p_D}
		& 
	}
\end{definition}

\begin{remark}\label{rem:agreement-amalgamation-solution}
	Note that by assumption  in the definition above we already have a pushout over the s, and by 
	 we also have a pushout over the s. 
\end{remark}

\begin{theorem}[Amalgamation of Solutions for Positive Nested Conditions]\label{thm:ACviaRestr}
Given pushout (1) as in \autoref{def:agreement-amalgamation-solution} with all morphisms in \M, 
an amalgamation of typed objects ,
and an amalgamation of positive nested conditions  
with corresponding matches .
\begin{description}
	\item \textbf{Composition.}
		Given solutions  for  and  for  agreeing in a solution  for ,
		then there is a solution  for  constructed as 
		amalgamation .
	\item \textbf{Decomposition.}
		Given a solution  for , then there are 
		solutions ,  and  for \linebreak
		,  
		and , respectively, which are constructed as
		restrictions ,  and  of  such that
		.
\end{description}
The amalgamated composition and decomposition constructions are unique up to isomorphism.
\end{theorem}








\begin{remark}\label{rem:amalgamation-solutions}
	From the proof of \autoref{thm:ACviaRestr} (see \autoref{sec:appendixB}) we can conclude that for a given amalgamation of solutions , we also
	have corresponding amalgamations of its components.
\end{remark}