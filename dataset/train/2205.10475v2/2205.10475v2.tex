\section{Experimental Setup}
\label{appendix:exp}
\subsection{Implementation Details} 
\label{sec:implementation}

\paragraph{Model Architecture}
We leverage the Generalized Language Model (GLM)~\cite{du2021all} as our base language model pretrained on autoregressive blank infilling objectives. GLM follows an adaptive encoder-decoder architecture. It improves the pretrain-finetune consistency via cloze-style finetuning. GLM adopts the Byte Pair Encoding~\cite{radford2019language}, covering 50,257 tokens. In this work, we leverage the models in four different scales: 110M, 220M, 2B, and 10B~\footnote{https://github.com/THUDM/GLM}. The 110M model is pretrained over English Wikipedia and BookCorpus, and the others are pretrained over the Pile corpora~\cite{pile}. The Pile corpora are regarded as the similar corpora for training GPT-3. GLM outperforms T5 on text summarization, which shares a similar nature with structure prediction tasks. Compared to GPT-3, GLM is a bidirectional model and is able to perform autoregressive generation.

\paragraph{Structure Pretraining Procedure} 
\paragraph{Task-Agnostic Pretraining} We conduct the pretraining on 8 NVIDIA DGX-A100 machines using an Adam optimizer with a 5e-6 learning rate and 0.1 weight decay. We train the model with batch size 4 per GPU for 3 epochs and use the checkpoint of the last iteration.

\paragraph{Multi-Task Training} We conduct the multi-task training on 8 NVIDIA DGX-A100 machines using an Adam optimizer with a 5e-6 learning rate and 0.1 weight decay. We train the model with batch size 4 per GPU for 6 epochs and use the checkpoint with the best performance on the corresponding validation set for each dataset.

\paragraph{Inference} During the inference, length penalty and minimum target length are the most important hyperparameters. Length penalty is a float between 0 and 1 to control the GLM's generation length. The larger the length penalty is, the longer the generation length is. In general, for entity prediction tasks (e.g., NER, SRL, event extraction), a larger length penalty is used. For entity and relation prediction or triple prediction tasks (e.g., JER and OIE), a smaller one is used. For other tasks that require a specific number of output triples (e.g., relation classification, intent detection, factual probe), we trim the generation results according to the requirements of different tasks. We show details of the task-specific hyperparameters in Appendix~\ref{sec:appendixoie} to Appendix~\ref{sec:appendixid}.

\paragraph{Pretraining Data}
We apply the task-agnostic pretraining data presented in Sec.~\ref{sec:genp} during structure pretraining. A small portion of T-REx~\cite{elsahar2018t} is used in the factual probe task~\cite{petroni2020context}. To avoid the test leakage, we (\expandafter{\romannumeral1}) sample a small portion of the T-REx as our pretraining data, and (\expandafter{\romannumeral2}) remove samples that appeared in the T-REx dataset of the factual probe task from the pretraining data. We integrate WebNLG 2.1 and 3.0 into a WebNLG dataset. For OPIEC, We use its OPIEC-clean version. Similar to T-REx, we also sample a portion of the OPIEC for our pretraining due to its large size.

The following sections introduce the dataset formats, comparison methods, and training details for all 10 structure prediction tasks. We show additional input and output examples on all datasets in Table~\ref{tab:datasetexamples}.

\begin{table}[t]
\centering
\resizebox{\linewidth}{!}
  {
  \begin{tabular} {l | l | r r r}
    \toprule
    \multirow{2}{*}{{\bf Task}} & \multirow{2}{*}{{\bf Dataset}} & \multicolumn{3}{l}{{\bf \#Sents}} \\
    & & Train & Dev & Test \\
    \hline
    \multirow{4}{*}{{\small \bf Open information extraction}}
                                    & OIE2016                           &  2,278  &    571 &     589 \\
                                    & WEB                               &       - &      - &     920 \\
                                    & NYT                               &       - &    300 &     149 \\
                                    & PENN                              &       - &      - &      51 \\
    \hline
    \multirow{2}{*}{{\small \bf Relation classification}}
                                    & TACRED                            & 	68,124  & 22,631 &   15,509 \\
                                    & FewRel 1.0                            & 56,000  &      1,120 &  --  \\
    \hline
    \multirow{4}{*}{{\small \bf Joint entity and relation extraction}}    
                                    & CoNLL04                           &     922 &    231 &    288 \\
                                    & ADE                               &   3,845 &    --  &    427 \\
                                    & NYT                               &  56,195 &  5,000 &  5,000 \\
                                    & ACE2005                           &   7,477 &  1,789 &  1,517 \\
    \hline
    \multirow{2}{*}{{\small \bf Event extraction}}
                                    & ACE2005 Trigger                     &  11,178 &    649 &    642 \\
                                    & ACE2005 Argument                     &   4,450 &    531 &    612 \\
    \hline
                    {\small \bf Coreference resolution}
                                    & CoNLL12                           &   3,991 &  2,359 &  2,421 \\
    \hline
    \multirow{2}{*}{{\small \bf Intent detection}}
                                    & ATIS                              &  4,478  &    500 &     893 \\
                                    & SNIPS                             & 13,084  &    700 &     700 \\
    \hline
    \multirow{4}{*}{{\small \bf Semantic role labeling}}
                                    & CoNLL05                           &  39,832 &  3,206 &    --  \\
                                    & CoNLL05 WSJ                       &  39,832 &  3,206 &  5,221 \\
                                    & CoNLL05 Brown                     &  39,832 &  3,206 &    779 \\
                                    & CoNLL12                           &  89,549 & 32,397 & 21,499 \\
    \hline
    \multirow{4}{*}{{\small \bf Named entity recognition}}
                                    & CoNLL03                           &  14,041 &  3,250 &  3,453 \\
                                    & OntoNotes                         &  59,924 &  8,528 &  8,262 \\
                                    & GENIA                             &  14,824 &  1,855 &  1,854 \\
                                    & ACE2005                           &   7,299 &    971 &  1,060 \\
    \hline
    {\small \bf Dialogue state tracking}                                & MultiWOZ 2.1                      & 62,367  &  7,371 &   7,368 \\
    \hline
    \multirow{2}{*}{{\small \bf Factual probe}}
                                    & Google-RE                         &      -  &      - &     552 \\
                                    & T-REx                             &      -  &      - &   3,403 \\
    \bottomrule
  \end{tabular}
  }
\caption{{\small Statistics of downstream datasets.}}
\label{tab:statistics}
\end{table}


 
\subsection{Open Information Extraction}
\label{sec:appendixoie}
For OIE, given a sentence, we are asked to extract triples consisting of arguments and predicates. An example of the input and output format is as follows.

\paragraph{Input and Output Format}
\begin{enumerate*}
    \item[] {\bf Input} Born in 1951 in Tbilisi, Iago is a Georgian artist.
    \item[] {\bf Output} (Iago; Born in; 1951) (Iago; is a; Georgian artist)
\end{enumerate*}

\noindent where we extract arguments (e.g., Iago) and their predicates (e.g., Born in) in the form of triples as outputs from the input sentence.

\paragraph{Datasets} We evaluate the performance of the compared approaches on OIE benchmark datasets including {\bf OIE2016}~\cite{stanovsky2016creating}, a dataset converted from QA-SRL~\cite{he2015question} based on Newswire and Wikipedia; three datasets transformed from news corpus, including {\bf NYT}~\cite{mesquita2013effectiveness}, {\bf WEB}~\cite{mesquita2013effectiveness}, {\bf PENN}~\cite{xu2013open}. The statistics of all datasets are shown in Table~\ref{tab:statistics}. 

\paragraph{Comparison Methods} We compare our method \method\ to the following OIE systems presented in ~\cite{stanovsky2018supervised}: (\expandafter{\romannumeral1}) ClausIE~\cite{del2013clausie}: which leverages linguistic and grammatical knowledge to split clauses in a sentence and identify their roles, (\expandafter{\romannumeral2}) OpenIE 4~\footnote{\label{ft:openie51}\tiny\url{https://github.com/dair-iitd/OpenIE-standalone}}: which integrates SRLIE~\cite{christensen2011analysis} and Relnoun~\cite{pal2016demonyms} systems, (\expandafter{\romannumeral3}) PropS~\cite{stanovsky2016getting}: which focuses on prepositional phrases structure in sentences for OIE, (\expandafter{\romannumeral4}) RnnOIE~\cite{stanovsky2018supervised}: a supervised recurrent neural network based approach that views the OIE task as a sequence tagging problem. We additionally compare to MAMA using BERT from \cite{wang2020language}, which proposes to leverage knowledge stored in attention matrices for OIE.

\paragraph{Training Details}
We train our model on the OIE2016 training set for 5 epochs during multi-task finetuning. The per GPU batch size is 4. During inference, for OIE2016, we choose a length penalty of 0.8. WEB, NYT, and PENN only contain the test sets. For these datasets, we use a length penalty of 0.5 and trim the outputs to only contain the first triple. We adopt the preprocessed OIE datasets provided by \citet{stanovsky2018supervised}.

\paragraph{Additional Results}
A detailed comparison between \method and compared approaches is shown in Table~\ref{tab:oie}. On OIE2016, NYT, and PENN datasets, \method presents significant improvements compared to the OIE systems. While on WEB, PropS~\cite{stanovsky2016getting} outperforms our method. The reason is that the arguments of WEB are very short and concise (e.g., ``( google ; assimilates ; youtube )''), which aligns better with the phrase extraction paradigm of PropS. We also observe that finetuning hurts the performance on WEB and PENN. This is because we are only able to finetune \method on the OIE2016 training set and this can lead to overfitting.

\subsection{Relation Classification}
Given head and tail entities in the target sentence, we seek to identify the relation between them. An example is as below.

\paragraph{Input and Output Format}
\begin{enumerate*}
    \item[] {\bf Input} The 1976 Thomas Cup was the tenth edition of Thomas Cup, the world championship of men's international team badminton (its female counterpart is the Uber Cup). The relationship between Uber Cup and badminton is
    \item[] {\bf Output} (Uber Cup; sport; badminton)
\end{enumerate*}

\noindent where ``Uber Cup'' and ``badminton'' are the corresponding head and tail entities, and ``sport'' is a relation from a predefined category. In addition, we augment the input sentence with the task-specific suffix ``The relationship between [head entity] and [tail entity] is'' following \cite{paolini2021structured} as shown in the above example.

\paragraph{Datasets} We evaluate on FewRel~\cite{han2018fewrel} and TACRED~\cite{zhang2017tacred}. 
\begin{itemize}
    \item {\bf FewRel} is a few-shot N-way K-shot relation classification dataset for meta learning. For all 100 relations, train (64 relations), validation (16 relations), and test set (20 relations) are constructed accordingly. We report the results on the dev set.
    \item {\bf TACRED} is a large-scale benchmark including over 100K samples and 41 relation types. We select the checkpoint on the dev set, and report results on the test set.
\end{itemize}
We show the dataset statistics in Table~\ref{tab:statistics}. 
We use F1 to evaluate the results. We parse every relation type and the corresponding head and tail entities from every original sample and formulate it as the input and output format as shown in the above example.

\paragraph{Comparison Methods} The compared models are as follows: (\expandafter{\romannumeral1}) BERT-PAIR~\cite{gao2019fewrel} is a sentence-level pairwise model that optimizes the similarity between sentences with the same relation. (\expandafter{\romannumeral2}) BERT+Matching the Blanks (MTB)~\cite{soares2019matching} proposes continual pretraining for relations over a large-scale entity-linked corpus. (\expandafter{\romannumeral3}) TANL~\cite{paolini2021structured} is a sequence-to-sequence generation model using task-augmented natural languages.  

\paragraph{Training Details} 
We train our model on training sets of TACRED and FewRel for 20 epochs during multi-task finetuning respectively. The per GPU batch size is 4. As shown in the above input and output format, we use the prompt of ``The relationship between [head entity] and [tail entity] is'' to query the model to generate the relation. For the zero-shot setting, the model is also provided with the prefix ``( [head entity];'' to generate the relation and tail entity. The prediction is correct only if both the relation and tail entity are correct. The length penalty equals 0.5.

\paragraph{Additional Results}
We show the results in Table~\ref{tab:rc}. \method outperforms all supervised methods on both TACRED and FewRel. We find that our task-agnostic pretraining can significantly help improve relation classification. This is vital to few-shot settings (FewRel), where \method can achieve almost perfect F1 scores. We also notice that multi-task \method outperforms all compared approaches except for the 5-5 FewRel setting.

\subsection{Factual Probe}
Given an input sentence, and a gold head entity and relation, the task is to predict the missing tail entity in the following.
\paragraph{Input and Output Format}
\begin{enumerate*}
    \item[] {\bf Input} Daniel Bowen, born in 1970, is a Melbourne resident best known as the author of the blog, Diary of an Average Australian.
    \item[] {\bf Output} (Daniel Bowen; date\_of\_birth; 1970)
\end{enumerate*}

\noindent where ``(Daniel Bowen; date of birth; '' is provided in the output, and the model is asked to generate ``1970)''.

\paragraph{Datasets} We use the Google-RE dataset consisting of 3 relations (``place of birth'', ``place of death'', and ``date of birth''), and T-REx with 41 relations from the LAMA benchmark~\cite{petroni2019language}. The task is evaluated using mean precision at one (P@1).

\paragraph{Comparison Methods} We compare to the following approaches: (\expandafter{\romannumeral1}) LAMA~\cite{petroni2019language} uses only the head and relation to form the query without using the oracle context, and (\expandafter{\romannumeral2}) LAMA-Oracle~\cite{petroni2020context} takes (at most) five gold sentences as additional context in the query. Both methods are based on BERT and the query is constructed based on natural language templates. For example, the Wikidata relation ``place\_of\_birth'' is aligned with a template ``was born in''.

\paragraph{Training Details} 
As the factual probe task is usually performed without training sets, we only report \method's results in the zero-shot and multi-task setting (without finetuning). We follow the task format of LAMA-Oracle~\cite{petroni2020context}, which appends the query to five oracle context sentences. We use the relation labels (e.g., ``place of birth'') rather than the templates (e.g., ``was born in'') as they align better with our task-agnostic pretraining. Note that we have removed the T-REx data in the LAMA benchmark from the pretraining data.

\paragraph{Additional Results}
Table~\ref{tab:fr} shows the results. \method significantly outperforms compared approaches, which is attributed to the larger model size and knowledge-intensive task-agnostic pretraining. Multi-task setting actually hurts the performance due to the difference between the relation schema of downstream datasets and task-agnostic pretraining datasets. 

\subsection{Joint Entity and Relation Extraction}
\label{appendix:jer}
The goal of the task is to extract entities and relations (with their type information) from a given sentence. We formulate the task as two unit tasks: the first task is entity prediction to generate the entities, while the second task is relation prediction to generate the relations. Our task formalization is different compared to traditional JER, where our two unit tasks are independent. An example is as follows.
\paragraph{Input and Output Format}
\begin{enumerate*}
    \item[] {\bf Input} Blackstone already holds a 50 percent stake in the two parks that make up Universal Orlando.
    \item[] {\bf Entity Output} (Blackstone; instance of; organization) (parks; instance of; organization) (Universal Orlando; instance of; organization)
    \item[] {\bf Relation Output} (Blackstone; employer; parks)
\end{enumerate*}

\noindent where the entity output contains entity predictions and relation output contains relation predictions. The entity mentions are detected following the procedure below.

\paragraph{Evaluation Details} 
The conventional entity prediction evaluation is based on extractive span matching. To ensure a fair comparison in situations where there are multiple entities with the same surface, we adopt the following strategy: we match the spans of the generated entities from left to right in the original sentence when they are first mentioned. If there are duplicated entities, they are matched sequentially. For example, the first generated one matches the first mention span, while the second one matches the second mention span, etc. This strategy applies to all tasks that involve entity mention detection such as named entity recognition.

\paragraph{Datasets}
We experiment on the following datasets:

\begin{itemize}
    \item The {\bf CoNLL04}~\citep{roth-yih-2004-linear} dataset: CoNLL04 consists of four types of entities (``location'', ``organization'', ``person'', ``other'') and five types of relations (``work for'', ``kill'', ``organization based in``, ``live in'', ``located in'') on sentences taken from WSJ, AP, etc, containing 922 samples for training, 231 samples for validating, and 288 samples for testing. We use the same split as in~\cite{gupta-etal-2016-table}. We use the same type names of entities and relations with TANL~\cite{paolini2021structured}.
    \item The {\bf ADE}~\citep{GURULINGAPPA2012885} dataset: ADE contains annotated documents for drug-related adverse effects over medical case reports corpus. It consists of two entity types (``Adverse-Effect'' and ``Drug'') and one relation type (``(Has-)Adverse-Effect''). We use the same type names as in \cite{paolini2021structured}.
    \item The {\bf NYT}~\citep{10.1007/978-3-642-15939-8_10} dataset: NYT is a distantly-supervised joint entity and relation extraction dataset based on New York Times corpus. The dataset consists of three entity types (``PER'', ``ORG'', and ``LOC'') and 24 Freebase relations. We use a preprocessed version of this dataset from~\cite{yu2020jointer} and use the same type names with TANL~\cite{paolini2021structured}.
    \item The {\bf ACE2005}~\citep{walker2005ace} dataset: ACE2005 is based on the ACE 2005 Multilingual Training Corpus. We use a preprocessed version of this dataset in \cite{luan2019general}. We make use of seven entity types and six relation types with the same type names as in TANL.
\end{itemize}
The dataset statistics are shown in Table~\ref{tab:statistics}.

\paragraph{Comparison Methods}
We compare our method \method on the four datasets to the following JER methods: (\expandafter{\romannumeral1}) SpERT~\cite{DBLP:journals/corr/abs-1909-07755}: The BERT-based model first conducts named entity recognition formulated as sequence tagging, and performs relation classification between recognized entities; (\expandafter{\romannumeral2}) DyGIE~\cite{luan2019general}: The general information extraction framework organizes dynamic spans into graphs; (\expandafter{\romannumeral3}) MRC4ERE~\cite{zhao-etal-2020-asking}: The model formulates the joint entity and relation extraction task as machine reading comprehension; (\expandafter{\romannumeral4}) RSAN~\cite{yuan2020relation}: The work presents a relation-specific attention network to jointly extract entities and relations; (\expandafter{\romannumeral5}) TANL~\cite{paolini2021structured}: It is a sequence-to-sequence extraction model using augmented natural languages.

\paragraph{Training Details}
We train our model on JER training sets during multi-task finetuning for (\expandafter{\romannumeral1}) 10 epochs on CoNLL04, (\expandafter{\romannumeral2}) 10 epochs on ADE, (\expandafter{\romannumeral3}) 3 epochs on NYT, and (\expandafter{\romannumeral4}) 10 epochs on ACE2005. We employ less number of epochs on NYT, as its size is much larger compared to other datasets. We find that the relation prediction task and the entity prediction task need different length penalties. Therefore, we split the training sets corresponding to the two tasks. We choose a length penalty of 0.8 for entity prediction and 0.3 for relation prediction during inference. We use the same evaluation scripts as in \cite{paolini2021structured}. As multi-task training and finetuning for 10 folds on ADE is too expensive for \method 10B, only the first split of ADE is included in the multi-task training and finetuning.
In the ablation study (Sec.~\ref{sec:ablation}), we also present a model variant with entity and relation augmentation. For this setting, we augment the output with entity boundary information. For example, for the example in Figure~\ref{fig:overview}, ``([Iago]; instance of; person) ([Iago]; city\_of\_birth; [Tbilisi])'' are the augmented outputs. 

\paragraph{Additional Results}
Table~\ref{tab:jer} presents the results. \method outperforms or is competitive compared to state-of-the-art supervised approaches. We also find the multi-task \method performs competitively with previous task-specific approaches on both ADE and NYT. This indicates that multi-task trained models are cost-effective alternatives to per-task finetuned models.

\subsection{Named Entity Recognition}
Compared to joint entity and relation extraction, named entity recognition only focuses on predicting the entities and their corresponding types in the target sentence. We show an example below.

\paragraph{Input and Output Format}
\begin{enumerate*}
    \item[] {\bf Input} What we need to do is to make sure that state boards, number one, have adequate funding.
    \item[] {\bf Output} (we; instance of; human) (state; instance of; geographical entity) (state boards; instance of; organization)
\end{enumerate*}

\noindent where the head entities of these triples are the entity mentions in the given sentence, and tail entities are from a predefined list of entity types.

\paragraph{Datasets}
We experiment on the following datasets:
\begin{itemize}
    \item The {\bf CoNLL03}~\citep{tjong-kim-sang-de-meulder-2003-introduction} dataset: CoNLL03 (English) data was taken from the Reuters Corpus, containing 14,041 training samples, 3,250 validating samples and 3,453 testing samples. It consists four entity types (``LOC'', ``ORG'', ``PER'', and ``MISC''). We use the preprocessed version of this dataset from~\cite{li2020unified}.
    \item The {\bf OntoNotes}~\citep{pradhan-etal-2013-towards} dataset: OntoNotes contains 59,924 training samples, 8,528 validating samples, and 8,262 testing samples. It consists 18 different entity types (e.g., ``ORG'', ``PER''). We use the preprocessing scripts provided by \cite{luan2019general}.
    \item The {\bf GENIA}~\citep{10.5555/1289189.1289260} dataset: GENIA consists of compiled and annotated biomedical literature, which contains 14,824 training samples, 1,855 validating samples, and 1,854 testing samples. It consists five entity types (``DNA'', ``RNA'', ``cell\_line'', ``cell\_type'', and ``protein''). We use a preprocessed version of this dataset~\cite{li2020unified}.
    \item The {\bf ACE2005}~\citep{walker2005ace} dataset: ACE2005 contains 7,299 training samples, 971 validating samples, and 1,060 testing samples. Note that it is also processed based on the ACE2005 corpus but with different data splits compared to that of the ACE2005 JER dataset. It includes seven entity types. We use the preprocessed version of this dataset in \cite{li2020unified}, and exclude this dataset in the \method's multi-task setting due to its overlap with the ACE2005 JER dataset.
\end{itemize}

\paragraph{Comparison Methods}
We compare our method \method on the four datasets to the following NER methods: (\expandafter{\romannumeral1}) BERT-MRC~\cite{li2020unified}: this method formulates NER as a machine reading comprehension problem, (\expandafter{\romannumeral2}) BERT-MRC+DSC~\cite{li2020dice}: this model is a dice-loss enhanced version of BERT-MRC, (\expandafter{\romannumeral3}): Cloze-CNN~\cite{baevski2019cloze}: the model leverages cloze-style pretraining on convolutional neural networks for natural languages, (\expandafter{\romannumeral4}) GSL~\cite{athiwaratkun2020augmented}: the method uses augmented language for intent detection, slot filling, and named entity recognition, (\expandafter{\romannumeral5}) BiaffineLSTM~\cite{yu2020named}: the model transforms NER into dependency parsing using biaffine LSTMs, (\expandafter{\romannumeral6}) TANL~\cite{paolini2021structured}: it presents a sequence-to-sequence extraction approach using augmented natural languages.

\paragraph{Training Details} 
We train our model on NER training sets for 15 epochs on every dataset during multi-task finetuning with early stopping. The per GPU batch size is 4. We choose a length penalty of 0.8 during inference. Since some datasets may contain null predictions, we set the minimum target length to 0. 

\paragraph{Additional Results}
Table~\ref{tab:ner} shows the results. \method achieves comparable performance to task-specific supervised approaches, except for OntoNotes. We suppose that OntoNotes contains a relatively large number of entity types, making it more challenging for models to use labels for considering their semantic meanings. On GENIA and ACE2005, \method outperforms state-of-the-art task-specific methods.

\subsection{Semantic Role Labeling}
In semantic role labeling, we seek to identify the corresponding arguments in the form of spans (or the semantic roles) given a certain predicate. Consider an example as follow.

\paragraph{Input and Output Format}
The predicate is marked in the input. The model then yields arguments according to the predicate in the output with their corresponding argument types from a predefined set.
\begin{enumerate*}
    \item[] {\bf Input} Scotty [ accepted ] the decision with indifference and did not enter the arguments.
    \item[] {\bf Output} (Scotty; instance of; subject) (decision; instance of; object)
\end{enumerate*}

\noindent where ``[ accepted ]'' is the given predicate, and arguments such as ``Scotty'', ``the decision'' and their corresponding types are generated in the form of triples.

\paragraph{Datasets}
We experiment on the following datasets: CoNLL05 WSJ, CoNLL05 Brown \citep{carreras-marquez-2005-introduction} and CoNLL12 \citep{pradhan-etal-2013-towards}. Table~\ref{tab:statistics} shows the dataset statistics.

\begin{itemize}
    \item The {\bf CoNLL05 WSJ} and {\bf CoNLL05 Brown} datasets: CoNLL05 WSJ and CoNLL05 Brown datasets share the same train and validation splits. They have different test sets. For CoNLL05 WSJ and CoNLL05 Brown, the corresponding test datasets are taken from the WSJ and Brown corpus respectively. The datasets consist of seven different types including ``V'' (verb), ``A0'' (subject), ``A1'' (object), ``A2'', ``A3'', ``AM-MOD'', and ``AM-NEG''. We use the same type names as in ~\cite{paolini2021structured}. 
    \item The {\bf CoNLL12} dataset: CoNLL12 dataset is built upon OntoNotes dataset including 39 argument types. We leverage the same type names as in ~\cite{paolini2021structured}.
\end{itemize}

\paragraph{Comparison Methods}
We compare our method \method on the datasets to the following SRL models: (\expandafter{\romannumeral1}) Dep and Span~\cite{li2019dependency}: this model formulates semantic role labeling as an end-to-end dependency parsing task, (\expandafter{\romannumeral2}) BERT SRL~\cite{shi2019simple}: it is a sequence-tagging version of BERT, (\expandafter{\romannumeral3}) TANL~\cite{paolini2021structured}: this is a sequence-to-sequence extraction model using augmented natural languages. 

\paragraph{Training Details}
During multi-task finetuning, for CoNLL05, the model is trained on CoNLL05 WSJ's training set, and evaluated on both CoNLL05 WSJ and CoNLL05 Brown test sets. We train \method on CoNLL05 WSJ and CoNLL12 for 5 epochs respectively. The per GPU batch size equals 4. The length penalty is 0.8. 

\paragraph{Evaluation Details} Sentences with multiple target predicates are duplicated during data preprocessing. So, each sentence is only related to one target predicate that is marked by ``[]''. We adopt the same evaluation scripts as in ~\cite{paolini2021structured}.

\paragraph{Additional Results}
Table~\ref{tab:srl} shows the results. Both multi-task and multi-task finetuned \method outperform task-specific models by a large margin. An important reason is that PropBank~\cite{kingsbury2003propbank} is included in the multi-task training. The knowledge of PropBank transfers well to other SRL datasets. We find that the performance gain is significant since the large-scale model has the capacity to capture the PropBank knowledge. We also observe a minor performance drop from multi-task to multi-task finetuned \method on CoNLL05 WSJ and CoNLL12 datasets. This might be attributed to overfitting. Besides, the issue can be relieved if a better hyperparameter combination is used in the multi-task finetuning setting. 

\subsection{Event Extraction}
This task contains two sequential subtasks: (\expandafter{\romannumeral1}) event triggers identification and classification: this subtask first identifies the trigger words in target sentences that refer to certain types of events, and (\expandafter{\romannumeral2}) trigger arguments identification and classification: this subtask then extracts arguments from the target sentences that can be mapped to certain roles in the event from (\expandafter{\romannumeral1}).

\paragraph{Input and Output Format}
\begin{enumerate*}
    \item[] {\bf Trigger Input} But the Saint Petersburg summit ended without any formal declaration on Iraq .
    \item[] {\bf Trigger Output} (summit; instance of; meet)
    \item[] {\bf Argument Input} But the Saint Petersburg [ summit ] ended without any formal declaration on Iraq .
    \item[] {\bf Argument Output} (Saint Petersburg; instance of; place)
\end{enumerate*}

\noindent where ``summit'' is an extracted trigger and ``meet'' is its corresponding trigger event. Then, based on the ``summit'' event, we can further extract the role of ``place'' in this event as ``Saint Petersburg''.

\paragraph{Datasets}
We experiment on the ACE2005 dataset. For detailed dataset statistics, please refer to Table~\ref{tab:statistics}.
\begin{itemize}
    \item The {\bf ACE2005} dataset~\citep{walker2005ace}: ACE2005 contains 33 types of event triggers, and each of them corresponds to a set of argument roles. We follow the preprocessing in TANL~\cite{paolini2021structured} and use the same evaluation scripts. 
\end{itemize}

\paragraph{Comparison Methods}
We compare our method \method on the dataset to the following methods: (\expandafter{\romannumeral1}) J3EE~\cite{nguyen2019one}: This method presents a joint model based on a recurrent neural network to first extract mention spans for triggers and arguments and then perform pairwise classification, (\expandafter{\romannumeral2}) DyGIE++~\cite{wadden2019entity}: the method leverages BERT for sequence tagging to identify mention spans and then classify each mention with triggers in pair for argument roles, (\expandafter{\romannumeral3}) TANL~\cite{paolini2021structured}: this is a sequence-to-sequence extraction approach using augmented natural languages.

\paragraph{Training Details} 
We train our model on ACE2005 event trigger and argument training sets for 20 epochs during multi-task finetuning. The per GPU batch size is 4. During inference, we choose a length penalty of 0.8. The argument prediction task requires triggers as input to make predictions. An example is shown in Table~\ref{tab:datasetexamples}. For the argument prediction task, we first generate all trigger predictions using our 10B model. If there is more than one trigger in a sentence, we will duplicate the sentence to make sure that every sample corresponds to a single trigger.
The ACE2005 dataset is processed similarly to the named entity recognition task.

\paragraph{Additional Results}
Table~\ref{tab:ee} presents the results. \method is competitive with the state-of-the-art task-specific supervised models on trigger identification and classification, as well as the argument identification task. In the meantime, \method outperforms the comparison methods on the argument classification task. 

\subsection{Coreference Resolution}
The coreference resolution aims to identify and cluster mentions in a document that refers to the same entity. An example is as follows.

\paragraph{Input and Output Format}
\begin{enumerate*}
    \item[] {\bf Input} And deterrents don't work well when an enemy values your death more than his life.
    \item[] {\bf Output} (an enemy; refer to; his)
\end{enumerate*}

\noindent where ``an enemy'' appears as the target entity and ``his'' is the mention it refers to. ``refer to'' is provided as part of the output triple.

\paragraph{Datasets}
We experiment on the CoNLL12 \citep{pradhan-etal-2013-towards} dataset constructed from OntoNotes corpus. The dataset statistics are presented in Table~\ref{tab:statistics}. 

\paragraph{Comparison Methods}
We compare our method \method on the dataset to the following methods: (\expandafter{\romannumeral1}) Higher-order c2f-coref~\cite{lee2018higherorder}: this method proposes a fully differentiable formulation of coreference resolution via iterative refinement using attention mechanism, (\expandafter{\romannumeral2}) BERT+c2f-coref~\cite{joshi2019bert}: this model replaces original recurrent neural network backbone in ~\cite{lee2018higherorder} with BERT, (\expandafter{\romannumeral3}) CorefQA+SpanBERT~\cite{wu-etal-2020-corefqa}: the model formulates coreference resolution as question answering, (\expandafter{\romannumeral4}) TANL~\cite{paolini2021structured}: this is a sequence-to-sequence extraction model using augmented natural languages.

\paragraph{Training Details} 
We train our model on CoNLL12 coreference resolution training set for 40 epochs during multi-task finetuning. The per GPU batch size is 4. During inference, we choose a length penalty of 0.8. CoNLL12 coreference resolution has different evaluation metrics compared to other structure prediction tasks, including: (\expandafter{\romannumeral1}) MUC: a link-based metric that reflects the minimum number of missing mentions in the response chain~\cite{moosavi2016coreference}, (\expandafter{\romannumeral2}) B: a single-mention based metric which computes the macro F1 of all entity mentions, and (\expandafter{\romannumeral3}) CEAF: a similarity metric based on the assumption that the coreference map should be one-to-one.
Due to the limited maximum sequence length of language models, the dataset is chunked with a fixed size of 512 during data preprocessing. Following TANL~\cite{paolini2021structured}, only intra-chunk coreferences are preserved. We also use the same evaluation scripts with ~\cite{paolini2021structured}.

\paragraph{Additional Results}
Table~\ref{tab:coref} shows the results. \method presents better results compared to TANL and classic task-specific supervised approaches. However, \method fails when compared with the state-of-the-art coreference method. The main reason is that this task requires task-specific model architectures. In the meantime, we argue that it is promising to employ a unified framework for multiple structure prediction tasks.

\subsection{Dialogue State Tracking}
We are presented with a dialogue between a user and an agent to identify what information is known given a list of slots by the end of each round of the conversation. An example is as follows.

\paragraph{Input and Output Format}
\begin{enumerate*}
    \item[] {\bf Input} [User]: I would like a taxi from Saint Johns College to Pizza Hut Fen Ditton. [Agent]: What time do you want to leave and when you want to arrive? [User]: I want to leave after 17:15.
    \item[] {\bf Output} ([User]; taxi arrive by; not given) ([User]; taxi departure; Saint Johns College) ([User]; taxi destination; Pizza Hut Fen Ditton) ([User]; taxi leave at; 17:15)
\end{enumerate*}

\noindent in which by the end of this conversation, we know that the user wants to get to Pizza Hut Fen Ditton from Saint Johns College, leaving at 17:15, while the taxi's arrival time is unknown. The slots ``taxi arrive by'', ``taxi departure'', ``taxi destination'', and ``taxi leave at'' are provided for the output.

\paragraph{Datasets} We use the MultiWOZ 2.1 \cite{budzianowski2018large, ramadan2018large, eric2019multiwoz, zang2020multiwoz}, which is a daily dialogue dataset for task-oriented conversations. We follow the preprocessing in \citep{wu2019transferable}. Following TANL~\cite{paolini2021structured}, the ``police'' and ``hospital'' slots are excluded from the training set as they are absent in the test set. The resulting training set contains 7,904 samples. Dataset statistics are presented in Table~\ref{tab:statistics}.

\paragraph{Comparison Methods} We compare our method with: (\expandafter{\romannumeral1}) TRADE~\cite{wu2019transferable}: It is a transferable multi-domain generative dialogue state tracking model, (\expandafter{\romannumeral2}) SimpleTOD \cite{hosseiniasl2020simple}: This is a state-of-the-art task-specific model for dialogue state tracking based on GPT-2 \cite{radford2019language}. In addition, we also compare our method with TANL.

\paragraph{Training Details}
We finetune for 20 epochs. The maximum sequence length is 512, and the per GPU batch size is 4. Given a domain and all possible slots, \method generates triples regarding the slots: if the information is not yet provided, the tail should be ``not given''. We use the same type names with ~\cite{paolini2021structured}.

\paragraph{Additional Results}
Table~\ref{tab:dst} shows the results. While \method does not outperform the state-of-the-art SimpleTOD, it is still competitive compared to the task-specific supervised models. This demonstrates the effectiveness of \method in dealing with different structure prediction tasks under the same architecture.

\subsection{{Intent Detection}}
\label{sec:appendixid}
Intent detection identifies the user's intent in the conversation with the agent based on a predefined list of slots. It resonates with the classical sentence classification task. Below is an example.

\paragraph{Input and Output Format}
\begin{enumerate*}
    \item[] {\bf Input} Show flight and prices from Kansas City to Chicago next Wednesday arriving in Chicago by 7 pm.
    \item[] {\bf Output} (intent; is; flight and airfare)
\end{enumerate*}

\noindent where our prediction is the ``flight and airfare''. The head entity ``intent'' and predicate ``is'' are given for all outputs.

\paragraph{Datasets}
We use two datasets, the ATIS dataset \cite{hemphill1990atis}, which contains flight and airline-related conversation and queries, and the SNIPS dataset \cite{coucke2018snips}, which consists of daily queries from the interaction between the users and dialogue agents.
The dataset statistics are shown in Table~\ref{tab:statistics}.

\paragraph{Comparison Methods}
We compare our method to SF-ID~\cite{haihong2019novel} and TANL~\cite{paolini2021structured} in this task. 

\paragraph{Training Details}
We formulate the label of every sample as ``(intent; is; [label])''. We parse every intent from every original sample and formulate it into the input and output format as shown above. We finetune for 20 epochs. The maximum sequence length is 512, and the per GPU batch size is 4. We report F1 for this task.
            
\paragraph{Additional Results}
Table~\ref{tab:id} shows the results. \method is comparable to task-specific supervised approaches on both ATIS and SNIPS datasets. For TANL, we produce the results using the released code~\footnote{\tiny\url{https://github.com/amazon-research/tanl}}.

\begin{table*}[]
    \centering
    \small
\renewcommand\tabcolsep{15.3pt}
    \begin{tabular}{@{}llllllll@{}} 
\toprule
\multicolumn{2}{l}{}                 & OIE2016 & WEB  & NYT  & PENN  \\ \midrule
\multicolumn{2}{l}{ClausIE~\cite{del2013clausie}}          & 58.8    & 44.9 & 29.6 & 34.6  \\
\multicolumn{2}{l}{OpenIE 4}         & 59.6    & 55.7 & 38.3 & 42.6  \\
\multicolumn{2}{l}{PropS~\cite{stanovsky2016getting}}            & 55.6    & 58.9 & 37.2 & 39.1  \\
\multicolumn{2}{l}{RnnOIE~\cite{stanovsky2018supervised}}           & 67.0    & 58.1 & 28.3 & 34.5  \\
\multicolumn{2}{l}{MAMA~\cite{wang2020language}}             & 36.6    & 54.3 & 32.9 & 33.0  \\\midrule
\multirow{3}{*}{ \bf \method} & zero-shot  & 28.1         & 43.8         & 28.9       & 51.0        \\
                                  & multi-task  & 71.2         & 50.8         & 43.6       & 54.5        \\
                                  & w/ finetune & 71.3         & 49.1         & 45.0       & 45.1        \\ \bottomrule
\end{tabular}
\caption{{Results on open information extraction.}}  \label{tab:oie}
\renewcommand\tabcolsep{7pt}
    \begin{tabular}{@{}lllllll@{}}
\toprule
   &                 & \multicolumn{1}{c}{\multirow{2}{*}{TACRED}} & \multicolumn{4}{c}{FewRel 1.0}            \\ 
&                       & \multicolumn{1}{c}{}                        & 5-1      & 5-5      & 10-1     & 10-5     \\ \midrule
\multicolumn{2}{l}{BERT~\cite{soares2019matching}}     & 70.1                                        & 88.9     & -        & 82.8     & -        \\
\multicolumn{2}{l}{BERT+MTB~\cite{soares2019matching}} & 71.5                                        & 90.1     & -        & 83.4     & -        \\
\multicolumn{2}{l}{DG-SpanBERT~\cite{chen2020efficient}}            & 71.5                                        & -        & -        & -        & -        \\
\multicolumn{2}{l}{BERT-PAIR~\cite{gao2019fewrel}}              &                                             & 85.7     & 89.5     & 76.8     & 81.8     \\
\multicolumn{2}{l}{NLI-DeBERTa~\cite{sainz2021label}}           & 73.9      & & & & \\
\multicolumn{2}{l}{TANL~\cite{paolini2021structured}}                   & 71.9                                        & 93.6±5.4 & 97.6±3.2 & 82.2±5.1 & 89.8±3.6 \\
\multicolumn{2}{l}{TANL (multitask)~\cite{paolini2021structured}}       & 69.1                                        & -        & -        & -        & -        \\ \midrule
\multirow{3}{*}{ \bf \method} & zero-shot  & 36.1         & 72.4±6.9     & 70.8±8.0   & 67.6±4.5   & 66.4±6.3     \\
                                  & multi-task  & 74.9         & 93.6±6.0     & 96.4±4.2   & 92.2±6.4   & 94.6±3.6     \\
                                  & w/ finetune & 76.8         & 98.4±2.8     & 100±0.0    & 97.8±2.0   & 99.8±0.6      \\ \bottomrule
\end{tabular}
\label{tab:rc}
\caption{{Results on relation classification.}}  \label{tab:rc}
\renewcommand\tabcolsep{37.5pt}
    
\begin{tabular}{@{}p{2cm}lll@{}}
\toprule
\multicolumn{2}{l}{}            & Google-RE & T-Rex \\ \midrule
\multicolumn{2}{l}{LAMA~\cite{petroni2019language}} & 10.5      & 32.3  \\ 
\multicolumn{2}{l}{LAMA-Oracle~\cite{petroni2020context}} & 74.3      & 66.0  \\ \midrule
\multirow{2}{*}{\bf \method} & zero-shot  &  97.9        &  85.0\\
                                  & multi-task  &  90.3        &  71.0\\\bottomrule
\end{tabular}
\label{tab:fr}
\caption{{Results on factual probe.}}  \label{tab:fr}
\renewcommand\tabcolsep{6.9pt}
    
\begin{tabular}{@{}llllllllll@{}}
\toprule
\multicolumn{2}{l}{\multirow{2}{*}{}}   & \multicolumn{2}{c}{CoNLL04} & \multicolumn{2}{c}{ADE} & \multicolumn{2}{c}{NYT} & \multicolumn{2}{c}{ACE2005} \\ \cmidrule(l){3-10} 
\multicolumn{2}{l}{}              & Ent          & Rel          & Ent        & Rel        & Ent        & Rel        & Ent         & Rel         \\ \midrule
\multicolumn{2}{l}{SpERT~\cite{DBLP:journals/corr/abs-1909-07755}}               & 88.9         & 71.5         & 89.3       & 78.8       &            &            &             &             \\
\multicolumn{2}{l}{DyGIE~\cite{luan2019general}}               &              &              &            &            &            &            & 88.4        & 63.2        \\
\multicolumn{2}{l}{MRC4ERE~\cite{zhao-etal-2020-asking}}             & 88.9         & 71.9         &            &            &            &            & 85.5        & 62.1        \\
\multicolumn{2}{l}{RSAN~\cite{yuan2020relation}}                &              &              &            &            &            & 84.6       &             &             \\
\multicolumn{2}{l}{TANL~\cite{paolini2021structured}}                & 89.4         & 71.4         & 90.2       & 80.6       & 94.9       & 90.8       & 88.9        & 63.7        \\
\multicolumn{2}{l}{TANL (multitask)~\cite{paolini2021structured}}    & 90.3         & 70.0         & 91.2       & 83.8       & 94.7       & 90.7       & -           & -           \\ \midrule
\multirow{3}{*}{ \bf \method} & zero-shot  & 48.3         & 25.8         & 60.7       & 10.6       & 60.5       & 28.6       & 31.8        & 5.3         \\
                                  & multi-task  & 88.4         & 72.8         & 90.5       & 83.6       & 95.4       & 93.7       & 90.2        & 58.9        \\
                                  & w/ finetune & 90.7         & 78.3         & 91.1       & 83.8       & 95.9       & 93.3       & 90.0        & 66.8       \\ \bottomrule
\end{tabular}
\label{tab:jer}
\caption{{Results on joint entity and relation extraction.} }
  \label{tab:jer}
\renewcommand\tabcolsep{11.5pt}
    
\begin{tabular}{@{}llllll@{}}
\toprule
\multicolumn{2}{l}{} & CoNLL03 & OntoNotes & GENIA & ACE2005 \\ \midrule
\multicolumn{2}{l}{BERT-MRC~\cite{li2020unified}}               & 93.0    & 91.1      & -     & 86.9  \\
\multicolumn{2}{l}{BERT-MRC+DSC~\cite{li2020dice}}           & 93.3    & 92.1      &       &       \\
\multicolumn{2}{l}{Cloze-CNN~\cite{baevski2019cloze}}              & 93.5    &           &       &       \\
\multicolumn{2}{l}{GSL~\cite{athiwaratkun2020augmented}}                    & 90.7    & 90.2      &       &       \\
\multicolumn{2}{l}{BiaffineLSTM~\cite{yu2020named}}          & 93.5     & 91.3      & 80.5      & 85.4  \\
\multicolumn{2}{l}{TANL~\cite{paolini2021structured}}                   & 91.7    & 89.8      & 76.4  & 84.9  \\
\multicolumn{2}{l}{TANL (multitask)~\cite{paolini2021structured}}       & 91.7    & 89.4      & 76.4  & -     \\ \midrule
\multirow{3}{*}{ \bf \method} & zero-shot  & 44.4         & 42.5     & 47.2       & 28.1        \\
                                  & multi-task  & 93.1         & 87.6         & 80.2       & -        \\
                                  & w/ finetune & 93.0         & 87.8         & 80.8       & 86.9        \\ \bottomrule
\end{tabular}
\label{tab:ner}
\caption{{Results on named entity recognition.}}  \label{tab:ner}

\end{table*}

\begin{table*}[]
    \centering
    \small
\renewcommand\tabcolsep{12.5pt}
    
\begin{tabular}{@{}lllll@{}}
\toprule
\multicolumn{2}{l}{}                         & CoNLL05 WSJ     & CoNLL05 Brown & CoNLL12 \\ \midrule
\multicolumn{2}{l}{Dep and Span~\cite{li2019dependency}}             & 86.3            & 76.4          & 83.1    \\
\multicolumn{2}{l}{BERT SRL~\cite{shi2019simple}}                 & 88.8            & 82.0          & 86.5    \\
\multicolumn{2}{l}{TANL~\cite{paolini2021structured}}                     & 89.3            & 82.0          & 87.7    \\
\multicolumn{2}{l}{TANL (multitask)~\cite{paolini2021structured}}         & 89.1            & 84.1          & 87.7    \\\midrule
\multirow{2}{*}{ \bf \method} 
                                  & multi-task  & 95.5         & 92.0         & 97.2       \\
                                  & w/ finetune & 95.2         & 92.1         & 96.0       \\ \bottomrule
\end{tabular}
\label{tab:srl}
\caption{Results on semantic role labeling.}  \label{tab:srl}
\renewcommand\tabcolsep{8pt}
    
\begin{tabular}{@{}llllll@{}}
\toprule
\multicolumn{2}{l}{}                       & Trigger Id & Trigger Cl & Argument Id & Argument Cl \\ \midrule
\multicolumn{2}{l}{J3EE~\cite{nguyen2019one}}                   & 72.5       & 69.8       & 59.9        & 52.1        \\
\multicolumn{2}{l}{DyGIE++~\cite{wadden2019entity}}                &            & 69.7       & 55.4        & 52.5        \\
\multicolumn{2}{l}{TANL~\cite{paolini2021structured}}                   & 72.9       & 68.4       & 50.1        & 47.6        \\
\multicolumn{2}{l}{TANL (multitask)~\cite{paolini2021structured}}       & 71.8       & 68.5       & 48.5        & 48.5        \\ \midrule
\multirow{2}{*}{ \bf \method} 
                                  & multi-task  & 72.7         & 69.2         & 67.5      & 63.9       \\
                                  & w/ finetune & 73.5         & 69.8         & 59.4       & 56.2        \\ \bottomrule
\end{tabular}
\label{tab:eventextractionappendixace}
\caption{{Results on event extraction (ACE2005).}}  \label{tab:ee}
\renewcommand\tabcolsep{13.4pt}
    \begin{tabular}{@{}llllll@{}}
\toprule
\multicolumn{2}{l}{}         & \multicolumn{4}{c}{CoNLL12}                                                                                            \\ \midrule
\multicolumn{2}{l}{}         & \multicolumn{1}{c}{MUC} & \multicolumn{1}{c}{B} & \multicolumn{1}{c}{CEAF} & \multicolumn{1}{c}{Avg. F1} \\
\multicolumn{2}{l}{Higher-order c2f-coref~\cite{lee2018higherorder}} & 80.4                    & 70.8                      & 67.6                               & 73                          \\
\multicolumn{2}{l}{BERT+c2f-coref~\cite{joshi2019bert}}         & 81.4                    & 71.7                      & 68.8                               & 73.9                        \\
\multicolumn{2}{l}{CorefQA+SpanBERT~\cite{wu-etal-2020-corefqa}}       & 86.3                    & 77.6                      & 75.8                               & 79.9                        \\
\multicolumn{2}{l}{TANL~\cite{paolini2021structured}}                   & 81.0                    & 69.0                      & 68.4                               & 72.8                        \\
\multicolumn{2}{l}{TANL (multitask)~\cite{paolini2021structured}}       & 78.7                    & 65.7                      & 63.8                               & 69.4                        \\ \midrule
\multirow{2}{*}{ \bf \method} 
                                  & multi-task  & 63.9        & 57.7       & 60.2       & 60.6\\
                                  & w/ finetune & 74.9        & 71.3       & 73.1       & 73.1\\ \bottomrule
\end{tabular}
\label{tab:coref}
\caption{{Results on coreference resolution.}}  \label{tab:coref}
\renewcommand\tabcolsep{54.5pt}
    
\begin{tabular}{@{}p{2cm}ll@{}}
\toprule
\multicolumn{2}{l}{}         & MultiWOZ 2.1 \\ \midrule
\multicolumn{2}{l}{TRADE~\cite{wu2019transferable}}                  & 45.6         \\
\multicolumn{2}{l}{SimpleTOD~\cite{hosseiniasl2020simple}}              & 55.7         \\
\multicolumn{2}{l}{TANL~\cite{paolini2021structured}}                   & 50.5         \\
\multicolumn{2}{l}{TANL (multitask)~\cite{paolini2021structured}}       & 51.4         \\ \midrule
\multirow{2}{*}{\bf \method} 
                                  & multi-task  & 53.5         \\
                                  & w/ finetune & 54.2         \\ \bottomrule
\end{tabular}
\label{tab:dst}
\caption{{Results on dialogue state tracking.}}  \label{tab:dst}
\renewcommand\tabcolsep{37.2pt}
    \begin{tabular}{@{}p{2cm}lll@{}}
\toprule
\multicolumn{2}{l}{\multirow{2}{*}{}}   &  ATIS & SNIPS \\ \midrule
\multicolumn{2}{l}{SF-ID~\cite{haihong2019novel}}            & 97.8 & 97.4  \\
\multicolumn{2}{l}{TANL~\cite{paolini2021structured}}        & 97.6 & 98.7  \\\midrule
\multirow{2}{*}{\bf \method} 
                                  & multi-task  & 97.3      & 97.4         \\
                                  & w/ finetune & 97.8     & 97.3     \\ \bottomrule
\end{tabular}
\label{tab:id}
\caption{{Results on intent detection.}}  \label{tab:id}


\end{table*}

\begin{table*}[]
\resizebox{\linewidth}{!}{
\begin{tabular}{@{}p{2.7cm}lp{10.5cm}p{10.5cm}@{}}
\toprule
\footnotesize
\textbf{Task}                        & \textbf{Dataset} & \textbf{Input} 
                                                        & \textbf{Output} 
                                                        \\ \midrule
Open Information Extraction          & OIE2016          & \quad oie oie2016: But for now, at least, {\color{black}Americans} are far better at {\color{black}making} {\color{black}PCs and the software that {\color{black}runs} them}.
                                                        &({\color{orange}Americans}; {\color{blue}making}; {\color{orange}PCs and the software that runs them}) ({\color{orange}PCs}; {\color{blue}runs}; {\color{orange}software})
                                                        \\
                                     & WEB              & \quad oie web: Finally {\color{black}google} {\color{black}bought} {\color{black}youtube}.
                                                        &({\color{orange}google}; {\color{blue}bought}; {\color{orange}youtube})
                                                        \\
                                     & NYT              & \quad oie nyt: Now in its 58th final, the {\color{black}United States} is {\color{black}pursuing} a 30th {\color{black}cup} title.
                                                        & ({\color{orange}United States}; {\color{blue}pursuing}; {\color{orange}cup})
                                                        \\
                                     & PENN             &\quad oie penn: {\color{black}Samsung} already {\color{black}owns} {\color{black}korea first advertising co.}, that country's largest agency.
                                                        & ( {\color{orange}Samsung}; {\color{blue}owns}; {\color{orange}korea first advertising co.} )
                                                        \\ \midrule
Relation Classification              & TACRED           & \quad rc tacred: Donald Wildmon , the founder and head of the American Family Association , is asking its members to petition Congress to end all funding for PBS . The relationship between Donald Wildmon and American Family Association is               
                                                        &  ( Donald Wildmon; {\color{blue}employee of}; American Family Association )               
                                                        \\
                                     & FewRel 1.0       & \quad rc fewrel: Boott was elected an Associate Fellow of the American Academy of Arts and Sciences in 1835 . The relationship between Boott and American Academy is              
                                                        &  ( Boott; {\color{blue}member of}; American Academy )               
                                                        \\ \midrule
Factual Probe                       & Google-RE        & \quad fp google-re: Eldon Coombe (born c 1941) is a Canadian curler from Ottawa, Canada.               
                                                        &  (Eldon Coombe; date of birth; {\color{orange}1941})               
                                                        \\
                                     & T-REx            & \quad fp t-rex: Kurt Schwertsik (born 25 June 1935, Vienna) is an Austrian contemporary composer.              
                                                        &  (Kurt Schwertsik; place of birth; {\color{orange}Vienna})              
                                                        \\ \midrule
Joint Entity and Relation Extraction & CoNLL04          & \quad jer conll04: An art exhibit at the {\color{black} Hakawati Theatre} in {\color{black} Arab} east {\color{black} Jerusalem} was a series of portraits of {\color{black} Palestinians} killed in the rebellion .
                                                        & ( {\color{orange} Hakawati Theatre}; instance of; {\color{orange} organization} ) ( {\color{orange} Arab}; instance of; {\color{orange} other} ) ( {\color{orange} Jerusalem}; instance of; {\color{orange} location} ) ( {\color{orange} Palestinians}; instance of; {\color{orange} other} ) ( {\color{orange} Hakawati Theatre}; {\color{blue} organization based in}; {\color{orange} Jerusalem} )
                                                        \\
                                     & ADE              & \quad jer ade: {\color{black} Lethal anuria} complicating high dose {\color{black} ifosfamide} chemotherapy in a breast cancer patient with an impaired renal function .
                                                        & ( {\color{orange} Lethal anuria}; instance of; {\color{orange} disease} ) ( {\color{orange} ifosfamide}; instance of; {\color{orange} drug} ) ( {\color{orange} Lethal anuria}; {\color{blue} effect}; {\color{orange} ifosfamide} )
                                                        \\
                                     & NYT              & \quad jer nyt: Mary L. Schapiro , who earlier this year became the new head of {\color{black} NASD} , was more amenable to fashioning a deal to the New York Exchange 's liking than her predecessor , {\color{black} Robert R. Glauber} .
                                                        & ( {\color{orange} NASD}; instance of; {\color{orange} organization} ) ( {\color{orange} Robert R. Glauber}; instance of; {\color{orange} human} ) ( {\color{orange} Robert R. Glauber}; {\color{blue} company}; {\color{orange} NASD} )
                                                        \\
                                     & ACE2005          & \quad jer ace2005: The {\color{black} Davao Medical Center} , a regional {\color{black} government} {\color{black} hospital} , recorded 19 deaths with {\color{black} 50} wounded .
                                                        & ( {\color{orange} Davao Medical Center}; instance of; {\color{orange} organization} ) ( {\color{orange} government}; instance of; {\color{orange} geographical entity} ) ( {\color{orange} hospital}; instance of; {\color{orange} organization} ) ( {\color{orange} 50}; instance of; {\color{orange} human} ) ( {\color{orange} hospital}; {\color{blue} part of}; {\color{orange} government} )
                                                        \\ \midrule
Named Entity Recognition             & CoNLL03          & \quad ner conll03: {\color{black} Japan} began the defence of their {\color{black} Asian Cup} title with a lucky 2-1 win against {\color{black} Syria} in a Group C championship match on Friday .
                                                        & ( {\color{orange} Japan}; instance of; {\color{orange} location} ) ( {\color{orange} Asian Cup}; instance of; {\color{orange} miscellaneous} ) ( {\color{orange} Syria}; instance of; {\color{orange} location} )
                                                        \\
                                     & OntoNotes        & \quad ner ontonotes: Relevant departments from {\color{black} Beijing Municipality} promptly activated emergency contingency plans .
                                                        & ( {\color{orange} Beijing Municipality}; instance of; {\color{orange} country city state} )
                                                        \\
                                     & GENIA            & \quad ner genia: {\color{black} Human T and B lymphocytes} demonstrate an early and transient hyperpolarization after ligand binding .
                                                        & ( {\color{orange} Human T and B lymphocytes}; instance of; {\color{orange} cell type} )
                                                        \\
                                     & ACE2005          & \quad ner ace2005: BEGALA {\color{black} Dr . Palmisano} , again , thanks for staying with {\color{black} us} through the break .
                                                        & ( {\color{orange} Dr}; instance of; {\color{orange} human} ) ( {\color{orange} Dr . Palmisano}; instance of; {\color{orange} human} ) ( {\color{orange} us}; instance of; {\color{orange} human} )
                                                        \\ \midrule
Semantic Role Labeling               & CoNLL05 WSJ      & \quad srl conll05: But while {\color{black} the New York Stock Exchange} did {\color{black} n't} [ fall ] apart Friday as the Dow Jones Industrial Average plunged 190.58 points -- most of it in the final hour -- it barely managed to stay this side of chaos .
                                                        & ( {\color{orange} the New York Stock Exchange}; instance of; {\color{orange} second argument} ) ( {\color{orange} n't}; instance of; {\color{orange} negation} )
                                                        \\
                                     & CoNLL05 Brown    & \quad srl conll05: {\color{black} His father} [ tried ] {\color{black} to make the food a topic} .
                                                        & ( {\color{orange} His father}; instance of; {\color{orange} first argument} ) ( {\color{orange} to make the food a topic}; instance of; {\color{orange} second argument} )
                                                        \\
                                     & CoNLL12          & \quad srl conll12: Dear viewers , {\color{black} the China News program} {\color{black} will} [ end ] {\color{black} here} .
                                                        & ( {\color{orange} the China News program}; instance of; {\color{orange} second argument} ) ( {\color{orange} will}; instance of; {\color{orange} modal} ) ( {\color{orange} here}; instance of; {\color{orange} location} )
                                                        \\ \midrule
Event Extraction                     & ACE2005 Trigger  & \quad ee ace2005 trg: The European Union held a {\color{black} summit} in Brussels.
                                                        & ( {\color{orange} summit}; instance of; {\color{orange} meet} )
                                                        \\
                                     & ACE2005 Argument & \quad ee ace2005 arg: The European Union held a [ summit ] in {\color{black} Brussels}.
                                                        & ( {\color{orange} Brussels}; instance of; {\color{orange} place} )
                                                        \\ \midrule
Coreference Resolution               & CoNLL12          & \quad cr conll12: And deterrents does n't work terribly well when {\color{black} an enemy} values your death more than {\color{black} his} life .
                                                        & ( {\color{orange} an enemy}; refer to; {\color{orange} his} )
                                                        \\ \midrule
Dialogue State Tracking              & MultiWOZ 2.1     & \quad dst multiwoz: [User]: I am looking for a place to to stay that has {\color{black}cheap} {\color{black}price range} it should be in a {\color{black}type} of {\color{black}hotel}. [Agent]: Okay , do you have a specific area you want to stay in? [User]: No , I just need to make sure it s cheap. Oh, and I {\color{black}need parking}. [Agent]: I found 1 cheap hotel for you that include parking. Do you like me to book it? [User]: Yes, please. {\color{black}6} {\color{black}people} {\color{black}3} nights starting on {\color{black}Tuesday}.               
                                                        &   ([User]; {\color{black}hotel area}; {\color{orange}not given}) ([User]; {\color{black}hotel book day}; {\color{orange}Tuesday}) ([User]; {\color{black}hotel book people}; {\color{orange}6}) ([User]; {\color{black}hotel book stay}; {\color{orange}3}) ([User]; {\color{black}hotel internet}; {\color{orange}not given}) ([User]; {\color{black}hotel name}; {\color{orange}not given}) ([User]; {\color{black}hotel parking}; {\color{orange}yes}) ([User]; {\color{black}hotel price range}; {\color{orange}cheap}) ([User]; {\color{black}hotel stars}; {\color{orange}not given}) ([User]; {\color{black}hotel type}; {\color{orange}hotel})
              
                                                        \\ \midrule
Intent Detection                     & ATIS             & \quad id atis: Please give me a list of all the {\color{black}flights} between Dallas and Baltimore and their {\color{black}cost}.            
                                                        &   (intent; is; {\color{orange}flight and airfare})              
                                                        \\
Intent Detection                     & SNIPS            & \quad id snips: Play the song little robin redbreast.             
                                                        &  (intent; is; {\color{orange}play music})               
                                                        \\ \bottomrule
\end{tabular}
}
\caption{{Input and output examples for every dataset.}}
\label{tab:datasetexamples}
\end{table*} 
\section{Error Analysis}

We analyze the errors of {\method} 10B multi-task in recall on the CoNLL04 (JER) dataset. We specifically investigate the errors in the relation outputs. Table~\ref{tab:errmain} shows the error cases. We find that most errors are caused by minor differences between ground truth entities and predicted entities from entity outputs. For example, the predicated entity has almost the same span as the ground truth entity (e.g., ``U.S.'' and ``the U.S.''). Besides, we observe some false-positive errors that are due to the noise in the datasets. In such cases, our predictions are reasonable while they are missing due to the incompleteness of human annotations.
\begin{table*}[]
\small
\resizebox{\linewidth}{!}{
\begin{tabular}{@{}llp{12cm}p{7cm}p{7cm}@{}}
\toprule
{\bf Error type}                    & {\bf Percentage}          & {\bf Input}                                                                                                                                                                   & {\bf Ground Truth}                                        & {\bf Ours Prediction}                           \\ \midrule
\multirow{2}{*}{Close Entity}       & \multirow{2}{*}{65.3\%}   & {\sl Locations containing suitable federally owned land were listed as : Fort Wainwright annex , Fairbanks , Alaska ;}                                                        & (Fort Wainwright annex ; located in ; Fairbanks)          & (Fort Wainwright annex ; located in ; Alaska)   \\ \midrule
\multirow{2}{*}{Totally Missing}    & \multirow{2}{*}{26.4\%}   & {\sl Judith C. Toth says she returned for a fourth term in Maryland 's House of Delegates because she couldn 't find a better job .}                                          & (House of Delegates ; organization based in ; Maryland)   & (Judith C. Toth ; lives in ; Maryland)          \\ \midrule
\multirow{2}{*}{Wrong Relation}     & \multirow{2}{*}{ 4.2\%}   & {\sl After buying the shawl for \$1 , 600 , Darryl Breniser of Blue Ball , said the approximately 2-by-5 foot shawl was worth the money .}                                    & (Darryl Breniser ; lives in ; Blue Ball)                  & (Darryl Breniser ; works for ; Blue Ball)       \\ \midrule
\multirow{2}{*}{Different Focus}    & \multirow{2}{*}{ 1.7\%}   & {\sl An architect of President Nixon 's unsuccessful executive-privilege Watergate defense is a top prospect for the post of U.S. solicitor in the new Bush administration .} & ( Bush ; lives in ; U.S. )                                & ( Nixon ; lives in ; U.S. )                     \\ \bottomrule
\end{tabular}}
\caption{{\small Analysis of recall errors of \method\ on CoNLL04 joint entity and relation extraction task. For each error type, we list the percentage of missing triples caused by this particular type of error, and an example of this type of error taken from the CoNLL04 dataset.}}
\label{tab:errmain}
\end{table*} 

\section{Updates from V1 to V2}

\begin{itemize}
    \item The \method multi-task results were updated in Table~\ref{tab:allres}, Table~\ref{tab:zero-shot}, Figure~\ref{fig:scaling}, Table~\ref{tab:jer}, Table~\ref{tab:srl}, Table~\ref{tab:ee}, and Table~\ref{tab:id}. We corrected the evaluation of the following tasks: joint entity and relation extraction, semantic role labeling, event extraction, and intent detection.
\end{itemize}

%
 