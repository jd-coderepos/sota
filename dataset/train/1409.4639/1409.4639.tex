\documentclass[letter, 10pt, conference]{ieeeconf}      

\IEEEoverridecommandlockouts                              \overrideIEEEmargins




\usepackage{amsmath}
\usepackage{amssymb}
\usepackage{graphicx}
\usepackage{xspace}
\usepackage{enumerate}
\usepackage{color}
\usepackage{epstopdf}
 
\newtheorem{theorem}{Theorem}[section]
\newtheorem{lemma}[theorem]{Lemma}
\newtheorem{corollary}[theorem]{Corollary}
\newtheorem{propo}[theorem]{Proposition}
\newtheorem{assu}[theorem]{Assumption}
\newtheorem{defn}[theorem]{Definition}
\newtheorem{algo}[theorem]{Algorithm}
\newtheorem{exa}{Example}[section]
\newtheorem{remark}[theorem]{Remark}
\newtheorem{condition}[theorem]{Condition}

\title{Ultimate Boundedness of Droop Controlled Microgrids \\with
  Secondary Loops}

\author{Rahmat Heidari, Maria M.  Seron, Julio
 H. Braslavsky\thanks{ Priority Research Centre for Complex Dynamic Systems
   and Control (CDSC), School of Electrical Engineering and Computer
   Science, The University of Newcastle, Callaghan NSW 2308,
   Australia {\tt\scriptsize heidari.rahmat@gmail.com, 
     maria.seron@newcastle.edu.au}} \thanks{ Australian Commonwealth Scientific and Industrial
   Research Organisation (CSIRO),  Energy Flagship,
   PO box 330, Newcastle, NSW 2300, Australia {\tt\scriptsize
     julio.braslavsky@csiro.au}} }

\newcommand{\R}{\mathbb{R}}
\newcommand{\C}{\mathbb{C}}
\newcommand{\F}{\mathbb{F}}
\newcommand{\X}{\mathcal{X}}
\newcommand{\Y}{\mathcal{Y}}
\newcommand{\Z}{\mathbb{Z}}
\newcommand{\Rb}{\mathcal{R}}
\newcommand{\Sb}{\mathcal{S}}
\newcommand{\Tb}{\mathcal{T}}
\newcommand{\U}{\mathcal{U}}
\newcommand{\A}{\mathsf{A}}
\newcommand{\B}{\mathcal{B}}
\newcommand{\bc}{\mathsf{b}}
\newcommand{\bb}{\mathbf{b}}
\newcommand{\V}{\mathcal{V}}
\newcommand{\W}{\mathcal{W}}
\newcommand{\M}{\mathcal{M}}
\newcommand{\I}{\mathcal{I}}
\newcommand{\J}{\mathcal{J}}
\newcommand{\Ja}{\mathbf{J}}
\newcommand{\la}{\mathbf{l}}
\newcommand{\Ou}{\mathcal{O}}
\newcommand{\vv}{\hat v^\natural}
\newcommand{\rank}{\mathrm{rank}}
\newcommand{\ie}{i.e.}
\newcommand{\Ie}{That is,\xspace}
\newcommand{\eg}{p.ej.\xspace}
\newcommand{\integs}{\mathbb{Z}}
\newcommand{\nats}{\mathbb{N}}
\newcommand{\natso}{\mathbb{N}_0}
\newcommand{\Reg}{\mathcal{R}}
\newcommand{\T}{^T}
\newcommand{\hpl}{\mathcal{P}}
\newcommand{\setU}{\mathcal{U}}
\newcommand{\ul}[1]{\underline{#1}}
\newcommand{\spn}{\mathrm{Span}}
\newcommand{\col}[1]{\mathrm{col}[#1]}
\DeclareMathOperator{\re}{\mathfrak{Re}}
\DeclareMathOperator{\im}{\mathfrak{Im}}
\newcommand{\mat}{\mathrm{Mat}\,}
\DeclareMathOperator{\img}{\mathrm{img}}
\newcommand{\dcm}{\mathrm{DCM}}
\newcommand{\mcm}{\mathrm{MCM}}
\newcommand{\inprod}[1]{\left\langle #1 \right\rangle}
\newcommand{\diag}{\mathrm{diag}}
\newcommand{\srad}{\boldsymbol{\rho}}
\newcommand{\id}{\mathrm{I}}
\newcommand{\s}{N}
\newcommand{\CL}{\textsc{cl}}
\newcommand{\g}{\mathbf{g}}
\newcommand{\spec}{\boldsymbol{\sigma}}
\newcommand{\sw}{\sigma}
\newcommand{\cle}{\preceq}
\newcommand{\bw}{\mathbf{w}}
\newcommand{\dfn}{\doteq}
\newcommand{\rb}{\mathbf{z}}
\newcommand{\Ind}{\underline{\s}}\newcommand{\cea}{CEA\xspace}
\newcommand{\fb}{\mathrm{FB}}
\newcommand{\ri}{\mathrm{\scriptscriptstyle RI}}
\newcommand{\E}{E}
\renewcommand{\H}{H}
\newcommand{\dist}{\mathrm{d}}
\newcommand{\blk}{\mathrm{blk}}
\newcommand{\EOS}{\hfill}\newcommand{\md}{\underline{\dist}}
\renewcommand{\dim}{\mathrm{d}}
\newcommand{\bnd}{\mathbf{b}}
\newcommand{\mer}{\hfill}

\newcommand{\1}{\mathbf{1}}
\newcommand{\0}{\mathbf{0}}






 
\begin{document}
\maketitle
\thispagestyle{empty}
\pagestyle{empty}


\begin{abstract}
  In this paper we study theoretical properties of inverter-based
  microgrids controlled via primary and secondary loops.
  Stability of these microgrids has been the subject of a number of 
  recent studies.  Conventional approaches based on standard hierarchical
  control rely on time-scale separation between primary and secondary
  control loops to show local stability of equilibria. In this paper 
  we show that (i) frequency regulation can be ensured
  without assuming time-scale separation and, (ii) ultimate boundedness
  of the trajectories starting inside a region of the state space can be
  guaranteed under a condition on the inverters power injection
  errors. The trajectory ultimate bound can be computed by simple
  iterations of a nonlinear mapping and provides a certificate of the
  overall performance of the controlled microgrid.
\end{abstract}

\section{Introduction}
In the last decade, the need to mitigate the environmental impacts of
coal-fired electricity generation has stimulated a gradual transition
from large centralised energy grids towards small-scale distributed
generation (DG) of power \cite{Ustun2011}.  A common operating regime
for DG is to form microgrids before being connected to the main energy
grid.  A microgrid is a small-scale power system consisting of a
collection of DG units, loads and local storage, operating together
with energy management, control and protection devices and associated
software \cite{Lasseter01,PLT09}.

Control strategies are indispensable to provide stability in
microgrids \cite{PMM06}. Recently, hierarchical control for microgrids
has been proposed in order to standardise their operation and
functionalities \cite{GVMdVC11,Bidram2012}. In this hierarchical
approach, three main control levels are defined to manage voltage and
frequency stability and regulation, and power flow and economic
optimisation. In this paper we focus on the primary and secondary
control levels, which are the main parts of the automatic control
system for the microgrid.

The primary control level deals with the local control loops of the DG
sources.  Many of these sources generate either variable frequency AC
power or DC power, and are interfaced with an AC grid via power
electronic DC/AC inverters.  For inductive lines, inverters are
typically controlled to emulate the droop characteristic of
synchronous generators.  Conventionally, the frequency-active power
(or ``-P'' ) droop control \cite{CDA93} is adopted as the
decentralised control strategy for the autonomous active power sharing
at primary level.  Because standard droop control is a purely
proportional control strategy, the secondary control level has the
task of compensating for frequency steady-state errors induced by the
primary control layer.  Although the secondary control level is
conventionally implemented in a centralised fashion, several recent
works have suggested distributed control implementations
\cite{SVG12,ASDJ12,SATS12}.

Stability and convergence properties of droop-controlled networks of
inverters and loads have recently been the focus of the detailed
analyses that highlight the dynamic properties of the power system
\cite{AiG13,ASDJ12,JWSP-FD-FB:12u}.
For example, in \cite{JWSP-FD-FB:12u}, the authors present a necessary 
and sufficient condition for the existence of a unique and locally
exponentially stable steady state equilibrium for a droop-controlled
network.  The paper also proposes a distributed secondary-control
scheme to dynamically regulate the network frequency to a nominal
value while maintaining proportional power sharing among the
inverters, and without assuming time-scale separation between primary
and secondary control loops.  This is in contrast with more
conventional analyses which rely on time-scale separation and do not
discuss stability properties beyond local results around equilibrium
points \cite{JWSP-FD-FB:12u}.



In this paper{\footnote{Preprint. Original version submitted to AuCC`14.}
 we analyse ultimate boundedness of the states of an
inverter-based purely inductive microgrid with decentralised droop
control and secondary control systems.
The network of our study is inherently decentralised as no communication between
neighbouring droop controllers is needed.
Our first contribution is a structured nonlinear model for a microgrid
with embedded primary and secondary control levels.  
By performing a suitable change of coordinates, we show how the
stability analysis for the controlled system is decoupled into a
linear system stability problem, and that of characterising
ultimate boundedness of the trajectories of a perturbed nonlinear
subsystem around steady-state solutions. Our second and main
contribution is then to establish stability properties of the original
nonlinear system by exploiting this model separation.  The linear
analysis shows that frequency regulation
is ensured without the need for time-scale separation.  For the
perturbed nonlinear subsystem, we show that ultimate boundedness of
the trajectories starting inside a region of the state space is
guaranteed under a condition on the power injection errors for the
inverters. The ultimate bounds for the trajectories can be computed by
iterating a well-specified nonlinear map, which provides key
certificates for the overall performance of the controlled microgrid.





\emph{Notation and Definitions:} Let  and 
be the -dimensional vectors of unit and zero entries.  Let 
and  be 
index sets of inverter buses and edges, respectively.  For a matrix
, , ,  and  denote
its -th row, -th column, rows  to , and -th entry,
respectively.  Denote by  the \emph{incidence}
matrix of a directed graph such that  if the node  is
the source of the edge  and  if the node  is the
sink node of the edge ; all other entries are zero.  The
\emph{Laplacian} matrix is  where
, ,  denoting the pure imaginary -th line admittance and
 denoting the bus voltage magnitude.  For connected graphs, . The entries of the  vector function
 contain the scalar
function  applied to  in the same order
as the entries in the matrix . The symbol  denotes the
Kronecker product of matrices.   denotes the set of real -vectors with nonnegative
components.  denotes the set of positive integers.
Inequalities and absolute values are taken componentwise.
A nonnegative vector function  is said to
be componentwise non-increasing (CNI) if whenever
 and , then .




\section{Decentralised Droop Control Model}
We start by presenting our model of an inverter-based microgrid under
decentralised droop control, and then analyse its structure to reveal
important modal characteristics of the underlying linear part of the
system.  The model is essentially a weighted graph where each node
represents a common-voltage point of power injection, and branches
represent microgrid node-interconnecting lines
\cite{JWSP-FD-FB:12u,AiG13a}.

The standard primary droop control at each inverter  in the
microgrid is such that the deviation in frequency  from
a nominal rated frequency  is proportional to the power
injection  in the following way:

where  is the droop controller coefficient, 
 is the inverter power injection error between
the inverter nominal injection setpoint  and the bus load
, and  is the frequency of 
the voltage signal at the -th inverter.  By assuming purely (loseless)
inductive lines, the power injection to each bus has the form

with ,  denoting the pure
imaginary -th line admittance and  denoting the bus voltage
magnitude. We make the standard \emph{decoupling approximation}
\cite{ZhH13} where all voltage magnitudes  are constant so that
the power injection is considered a function of only the phase angles,
that is, . 


The droop controller \eqref{eq:01} results in a static error in the steady
state frequency.  In \cite{AiG13a}, it is shown that as long as the
network state trajectories remain in a specified region, then the
controller in \eqref{eq:01} ensures network synchronisation to the
average frequency error

where the last equality follows from the fact that  for purely inductive lines.

We observe that  if and only if 
or equivalently ,
that is, the nominal injections are balanced. 
As discussed in~\cite{FD-JWSP-FB:14a}, it is not possible to achieve 
balanced nominal power injections since they depend on generally unknown and
variable load demand. Also, selecting the droop coefficients 
arbitrary large to make  small is not realistic.
Thus, complementary control action is required to eliminate  or 
at least reduce the frequency error ; for example, 
by including additional secondary control inputs  to each inverter
bus as follows:

for each  with .
As shown in \cite{AiG13} and  discussed here in Section 
\ref{subsec:wsyncbndd}, the parameter  in \eqref{eq:04}
can be tuned to reduce the frequency error. 

\begin{assu}
  In this paper we take all the droop coefficients as well as all the secondary
  control coefficients to be identical, that is,  and  for all 
  .
\end{assu}
The above assumption leads to having a simplified expression for the
average frequency error which is


Let  where .
Then, from the definitions of the incidence matrix 
and the Laplacian matrix  introduced in Notation and Definitions above, the
system \eqref{eq:03}--\eqref{eq:04} can be expressed as

where , , 
,  and the 
matrices
2mm]
		\frac{-1}{dk} L  & \frac{-e}{dk} I_n
	\end{bmatrix},
	H=\begin{bmatrix} 
		\frac{-1}{d} B Y  \2mm] \frac{1}{dk} P^*
	\end{bmatrix}

	\label{eq:08}
	M \doteq \diag \{\mu_1, \dots, \mu_n\}, \quad
U \doteq [u_1\; \dots \; u_n].

	\label{eq:11}
	\lambda_{2i-1,2i}&=
	  \begin{smallmatrix} 
	    -\frac{e+\mu_i k \mp R_i}{2dk}
	  \end{smallmatrix},
	R_i \doteq 
	  \begin{smallmatrix} 
	    \sqrt{4\mu_i k + (e-\mu_i k)^2},  i \in \I
	  \end{smallmatrix}
	\\
	\label{eq:12}
	[v_1\; v_2]&=\left[\begin{smallmatrix}
	   1 & \frac{k}{e} \\ 0 & 1 \end{smallmatrix}\right] \otimes u_1, \\
	\label{eq:13}
	[v_{2i-1}\;v_{2i}]
	&= \left[\begin{smallmatrix}
		\frac{e+dk\lambda_{2i-1}}{\mu_i} & \frac{e+dk\lambda_{2i}}{\mu_i} \\
		-1 & -1
	\end{smallmatrix}\right] \otimes u_i, 
\; i \in \I-\{1\}
  
  	\label{eq:15}
  	\begin{bmatrix}
  		(-1/d)L-\lambda_i I_n  & (-1/d)I_n \\
  		(-1/dk)L & (-e/dk - \lambda_i)I_n
  	\end{bmatrix}
  	\begin{bmatrix} v_{i,\theta} \\ v_{i,p} \end{bmatrix}
  	= \0_{2n}
  	
  	\label{eq:16}
  	v_{i,p}=-(L+d\lambda_i I_n) v_{i,\theta},
  
  	\label{eq:17}
  	\left(-L+(e+dk\lambda_i)(L+d\lambda_i I_n)\right) v_{i,\theta} = \0_n.
  
  	(-L+Le) u_1 = (e-1) L u_1 = \0_n,
  
  	[-L+\underbrace{(e+dk(-e/dk))}_0(L+d(-e/dk) I_n)] u_1 (k/e) \\
  	=-L u_1 (k/e) = \0_n
  
  	v_{2,p}&=-(L+d(-e/dk) I_n) u_1 (k/e) \\
  	&= -L u_1 (k/e) + (e/k)u_1(k/e) = u_1
  
  	\label{eq:18}
  	\lambda&=-\frac{e+\mu k \mp R}{2dk},\quad
  	R=\sqrt{4\mu k + (e-\mu k)^2} \\
  	\label{eq:19}
  	v&=\begin{bmatrix}
  		 u (e+dk\lambda)/\mu  \\ -u
  	\end{bmatrix}=
  	\begin{bmatrix}
  		 v_\theta \\ v_p
  	\end{bmatrix}.
  
  	\notag &(-L+(e+dk\lambda)(L+d\lambda I_n)) u \\
  	\notag &=-L u + (e + dk \lambda) (Lu + d \lambda u) \\
  	\notag &=-\mu u + (e + dk \lambda) (\mu u + d \lambda u) \\
  	\notag &=-\mu u + (e + dk \lambda) (\mu + d \lambda) u \\
  	\notag &=-\mu u + (\frac{e-\mu k \pm R}{2})
  			    (\frac{-e+\mu k \pm R}{2k}) u \\
\notag &=-\mu u + \frac{1}{4k} ( R^2 - (e-\mu k)^2) u \\
  	\notag &=-\mu u + \frac{1}{4k} ( 4 \mu k + (e-\mu k)^2) - (e-\mu k)^2) u \\
  	 &=-\mu u + \frac{1}{4k} ( 4 \mu k) u = \0_n.
  	\label{eq:20}

    \notag v_p &= -(L+d \lambda I_n)v_\theta
    \\\notag &= -(Lu + d \lambda u) \frac{(e+dk\lambda)}{\mu} \\
    \notag &= -(\mu u + d \lambda u) \frac{(e+dk\lambda)}{\mu} \\
  	\label{eq:21} 
  	&= -\underbrace{(\mu + d \lambda) (e+dk\lambda)}_{\mu} \frac{u}{\mu}
  	= -u.
  
	\Lambda &\doteq \diag \{ \lambda_1,\dots,\lambda_{2n} \}, \quad
	V \doteq \begin{bmatrix} v_1 & \dots & v_{2n} \end{bmatrix}.

	\label{eq:22}
	\dot{z}=\Lambda z + V^{-1} H \mathbf{f} + V^{-1} \bar{P}

	\label{eq:23}
	V^{-1}H = \Gamma U_H, \quad
	V^{-1}\bar{P} = -\Gamma U_P,

	\label{eq:24}
	\Gamma&=\mathrm{diag}\{
	  \begin{smallmatrix}
	     e-1,-\lambda_2,\lambda_3,-\lambda_4,\dots,
			\lambda_{2n-1},-\lambda_{2n}
	  \end{smallmatrix}\}, \\
	\label{eq:25}
	U_H&= u_h \otimes \left[\begin{smallmatrix} 1 \\ 1 \end{smallmatrix}\right]
, \qquad 
U_P= u_p \otimes \left[\begin{smallmatrix} 1 \\ 1 \end{smallmatrix}\right]
\\
	\label{eq:27}
    u_h &= R^{-1} U^{-1} B Y
, \quad
u_p = R^{-1} U^{-1} P^*

  	-\frac{e+\mu_i k \mp \sqrt{4\mu_i k + (e-\mu_i k)^2}}{2dk} <
        0 \quad &\iff \\
  	\mp \sqrt{4\mu_i k + (e-\mu_i k)^2} < e+\mu_i k \quad &\iff \\
  	4\mu_i k + (e-\mu_i k)^2 < (e+\mu_i k)^2 \quad &\iff \\
  	4\mu_i k < 4\mu_i k e \quad &\iff
  	1 < e ,
  
	\label{eq:28}
	\left[\begin{matrix} \dot{z}_1 \\ \dot{z}_2 \end{matrix}\right]
	&=\left[\begin{matrix} 
		0 & 0 \\ 0 & \lambda_2 
	 \end{matrix}\right]
	 \left[\begin{matrix} z_1 \\ z_2 \end{matrix}\right]
	+\left[\begin{matrix} (e-1)/(de) \\ 1/(dk) \end{matrix}\right]
	 \begin{matrix} \frac{(\sum_{i=1}^n P_i^*)}{n} \end{matrix}
	\\
	\label{eq:29}
	\dot{\hat{z}} &= \hat{\Lambda} \hat{z}+ \hat{\Gamma} (
        \hat{U}_H \mathbf{f} - \hat{U}_P )

  \label{eq:Lhat}
  &\hat{\Lambda}=\diag(\lambda_3,\lambda_4, \dots,\lambda_{2n}), \\
  \label{eq:Ghat}
  &\hat{\Gamma}=\diag(\lambda_3,-\lambda_4,\dots,\lambda_{2n-1},-\lambda_{2n}), \\
  &\hat{U}_H=[U_H]_{(3:2n,:)}, \quad \hat{U}_P=[U_P]_{(3:2n)}. \label{eq:Uhat}

  	\label{eq:33}
	V=\left[\begin{smallmatrix} V_\theta \\ V_p \end{smallmatrix}\right]
	=\left[\begin{smallmatrix}
		u_1 & (k/e) u_1 & \dots \\
		0   & u_1 & \dots
	\end{smallmatrix}\right],
  
	\label{eq:34}
	[\theta_i-\theta_j]_{i,j\in\J} = B^T V_\theta z.
  
  	\label{eq:32}
  	z_1 = (\sum_{i=1}^n \theta_i)/n - k z_2/e. \quad

  	\label{eq:321}
    \dot{z}_1=\underbrace{(\sum_{i=1}^n \dot{\theta}_i)/n}_{\omega_{sync}} 
    - k \dot{z}_2/e  =(e-1)(\sum_{i=1}^n P_i^*)/(nde).
  
  	\label{eq:wsync_ss}
  	{\omega_{sync}}_{ss}=\frac{(\sum_{i=1}^n P_i^*)d \epsilon}{nd(1+d\epsilon)}
  
    \label{eq:1}
    |\hat{\Gamma}(\hat{U}_H \mathbf{f}-\hat{U}_P)|\le |\hat{\Gamma}\hat{U}_H|
    F(\hat{z}) +  |\hat{\Gamma} \hat{U}_P|, 
  
  \label{eq:2}
  F(\hat{z}) \doteq  \frac{(|B^T V_\theta||z|)^3}{6},

	\label{eq:36}
	f(\theta_i-\theta_j) 
	\le \frac{(|{[B^T]}_{(i,:)} V_\theta||z|)^3}{6} 
	 \doteq F_i(\hat{z}),
  
	\label{eq:37}
	\mathbf{f}=[f(\theta_i-\theta_j)]_{i,j\in\J}
	\le \frac{(|B^T V_\theta||z|)^3}{6} \doteq F(\hat{z}).
  
   \notag T(\hat{z}) &\doteq |\hat{\Lambda}|^{-1} (|\hat{\Gamma}
   \hat{U}_H| F(\hat{z}) + |\hat{\Gamma} \hat{U}_P|)
   \\
   \label{eq:38}
   &= |\hat{U}_H| F(\hat{z}) + |\hat{U}_P|,
  
	\label{eq:35}
	T(\bar{z}) < \bar{z}

  	\label{eq:ContCond}
  	{u_p}_{(i+1)}^2 < \frac{4 g_i^3}{27 \gamma_i}

  	\label{eq:39}
    \left[\begin{smallmatrix}
      	T_{2i-1}(\hat{z}) \\ T_{2i}(\hat{z}) 
    \end{smallmatrix}\right]
    &= t_i(\hat{z}) \otimes 
    \left[\begin{smallmatrix} 1 \\ 1 \end{smallmatrix}\right], 
    \\
    \notag
    t_i(\hat{z}) &= |{u_h}_{(i+1,:)}| F(\hat{z}) + |{u_p}_{(i+1)}|
    \\
    \label{eq:391}
    &= |{u_h}_{(i+1,:)}| \frac{(|B^T V_\theta||z|)^3}{6} +
    |{u_p}_{(i+1)}|.
  
  	\left[\begin{smallmatrix}
		T_{2i-1}(\bar{z}) \\ T_{2i}(\bar{z}) 
	\end{smallmatrix}\right]
	&= t_i(\bar{z}) \otimes 
	\left[\begin{smallmatrix} 1 \\ 1 \end{smallmatrix}\right]
	<
	\left[\begin{smallmatrix} \
		\bar{z}_{2i-1} \\ \bar{z}_{2i} 
	\end{smallmatrix}\right],
  
  	\left[\begin{smallmatrix}
		T_{2i-1}(\bar{z}) \\ T_{2i}(\bar{z}) 
	\end{smallmatrix}\right]
	&= t_i(\bar{z}) \otimes 
	\left[\begin{smallmatrix} 1 \\ 1 \end{smallmatrix}\right]
	<
	\bar{z}_{2i} \otimes 
	\left[\begin{smallmatrix} 1 \\ 1 \end{smallmatrix}\right],
  
	\label{eq:40}
	t_i(\bar{z}) 
	= |{u_h}_{(i+1,:)}| \frac{(|B^T V_\theta||z|)^3}{6} + |{u_p}_{(i+1)}| 
	< \bar{z}_{2i}
  
\label{eq:41}
        \bar{t}_i(\zeta) &=|{u_h}_{(i+1,:)}| \frac{(|B^T V_\theta|G)^3}{6}
        \zeta^3 + |{u_p}_{(i+1)}| < g_i \zeta
  
	\label{eq:43}
        \gamma_i \zeta^3 - g_i \zeta + |{u_p}_{(i+1)}| < 0
  
    \Delta_i = \gamma_i (4  g_i^3-27 \gamma_i {u_p}_{(i+1)}^2) > 0,
  
  \label{eq:3}
  u_p(i+1) =\ell_{i}(P_1^*, \dots, P_n^*)

B &= \left[\begin{smallmatrix}
		 1 & 1 \\ -1 & 0 \\ 0 & -1
	\end{smallmatrix}\right],  \quad
	Y = \left[\begin{smallmatrix}
		 2 & 0 \\ 0 & 5
	\end{smallmatrix}\right],\quad
	L = \left[\begin{smallmatrix}
		 7 & -2 & -5 \\
    	-2 &  2 &  0 \\
    	-5 &  0 &  5
	\end{smallmatrix}\right], \
Using the above data, the system matrices and its eigenvalue-eigenvectors
from \eqref{eq:07}, \eqref{eq:11}--\eqref{eq:13}, with ,  
and  giving , are
2mm]
	\Lambda &=\diag \{\begin{smallmatrix}
		0,& -2,& -0.6641,& -3.9770,& -0.9126,& -12.4463 
		\end{smallmatrix}\}, \

Next, to form the transformed system \eqref{eq:22} with matrices
\eqref{eq:23}, the required matrices \eqref{eq:24}--\eqref{eq:28} are

It can be seen that  yields the first two rows of
 equal to zero, which confirms that the two subsystems

are decoupled.


Partitioning  as in \eqref{eq:33} with , 
the line phases are computed from \eqref{eq:34} to be

The function  in Lemma~\ref{lem:CNIfcn} is obtained from
\eqref{eq:2} as

Then, from \eqref{eq:38}, \eqref{eq:39}--\eqref{eq:391} the nonlinear
mapping  is

with
	

With the selection of  and , the
scalar inequalities \eqref{eq:41} to satisfy the contractivity
condition~\eqref{eq:35} are

for arbitrary .
The inverter power injection setpoints, through the linear
functions~\eqref{eq:3}, then need to satisfy the scalar inequalities

for the system to be ultimately bounded.  With regard to these
inequalities, one can run a nonlinear optimisation on  and 
to maximise the upper bounds  and .  The nonlinear
optimisation

yields  which in turn lead to
.

Take, for instance, . The contractivity conditions
\eqref{ex:01}--\eqref{ex:02} are then satisfied


The next step is to find . For each 
function, the  domain for which  is the
interval between the two positive roots of the polynomial .  For  we have
 and
 which yields

Then, the  domain that satisfies both conditions is the
intersection of these intervals, that is,

Now we just need to select a starting point  from this
interval, compute the associated  and iteratively calculate
the ultimate bound of the system.  From \cite{HaS13}, the ultimate
bound can be computed by first taking ,
 and then iterating,
 for . Since
, the ultimate bound is obtained as
.

Let .  Then
.  The resulting ultimate
bound on the  states is

We can interpret this ultimate bound 
on the line phases  
as follows

where the  entries are irrelevant since the first two columns
of  are zero.
The above bounds on the phase differences is validated as can be seen in
Fig.~\ref{fig:subplot}(a).

The next variable derived from this simulation is the average frequency 
error as in \eqref{eq:wsync1}. Corollary~\ref{cor:wsync} proves that
this frequency converges to the steady state frequency 
as in \eqref{eq:wsync_ss}. It is also notable that this steady state average
frequency static error is reliant on , that is, by decreasing 
 we obtain a smaller .
For  and  the obtained values are

The convergency of  to 
 for  is depicted in Fig.~\ref{fig:subplot}(b).

\begin{figure}[h]
\begin{center}
\includegraphics[width=0.5\textwidth]{subplot03.eps}
\caption{(a) Line phases, 
         (b) convergence of  to  }
\label{fig:subplot}
\end{center}
\end{figure}










\section{Conclusions}

We have analysed theoretical properties of inverter-based microgrids
controlled via primary and secondary loops. We have shown that
frequency regulation is ensured without
the need for time separation, and that ultimate boundedness of the
trajectories starting inside a region of the state space is guaranteed
under a condition on the inverters power injection errors. The
trajectory ultimate bound can be computed by simple iterations of a
nonlinear mapping and provides a certificate of the overall
performance of the controlled microgrid.  
Future work includes the derivation of design procedures based on 
the provided analysis, the extension of the results to more general controller
parameters and structures as well as relaxing some of the modelling
assumptions.


\bibliographystyle{plain}
\bibliography{DP_biblio}

\end{document}
