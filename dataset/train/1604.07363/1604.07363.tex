\documentclass[journal]{IEEEtran}

\usepackage[sort]{cite}
\usepackage{tikz}
\usepackage[cmex10]{amsmath}
\interdisplaylinepenalty=2500
\usepackage{float}
\usepackage{framed}
\usepackage{graphics} 			\usepackage{epsfig} 				\usepackage{amssymb}  			\usepackage{url}

\usepackage[caption=false,font=footnotesize]{subfig}

\usepackage[ruled]{algorithm}
\usepackage{algorithmicx}
\usepackage{algpseudocode}

\providecommand{\e}[1]{\ensuremath{\times 10^{#1}}}

\makeatletter
\algnewcommand{\LineComment}[1]{\Statex \hskip\ALG@thistlm \parbox[t]{\dimexpr\linewidth-\ALG@thistlm}{\textit{#1}\strut}}
\makeatother

\makeatletter
\newenvironment{breakablealgorithm}
  {\begin{center}
     \refstepcounter{algorithm}\hrule height.8pt depth0pt \kern2pt\renewcommand{\caption}[2][\relax]{{\raggedright\textbf{\ALG@name~\thealgorithm} ##2\par}\ifx\relax##1\relax \addcontentsline{loa}{algorithm}{\protect\numberline{\thealgorithm}##2}\else \addcontentsline{loa}{algorithm}{\protect\numberline{\thealgorithm}##1}\fi
       \kern2pt\hrule\kern2pt
     }
  }{\kern2pt\hrule\relax \end{center}
  }
\makeatother



\hyphenation{op-tical net-works semi-conduc-tor}

\begin{document}
\title{Efficient estimation of probability of conflict between air traffic using Subset Simulation}


\author{
				Chinmaya Mishra, Simon Maskell, Siu-Kui Au and Jason F. Ralph\\
				\{c.mishra, smaskell, siukuiau, jfralph\}@liv.ac.uk\\
				Department of Electrical Engineering and Electronics, University of Liverpool, United Kingdom
				}



\markboth{}{Shell \MakeLowercase{\textit{et al.}}: Bare Demo of IEEEtran.cls for IEEE Journals}



\maketitle


\begin{abstract}
This paper presents an efficient method for estimating the probability of conflict between air traffic within a block of airspace. Autonomous Sense-and-Avoid is an essential safety feature to enable Unmanned Air Systems to operate alongside other (manned or unmanned) air traffic. The ability to estimate probability of conflict between traffic is an essential part of Sense-and-Avoid. Such probabilities are typically very low. Evaluating low probabilities using naive Direct Monte Carlo generates a significant computational load. This paper applies a technique called Subset Simulation. The small failure probabilities are computed as a product of larger conditional failure probabilities, reducing the computational load whilst improving the accuracy of the probability estimates. The reduction in the number of samples required can be one or more orders of magnitude. The utility of the approach is demonstrated by modeling a series of conflicting and potentially conflicting scenarios based on the standard Rules of the Air.
\end{abstract}

\begin{IEEEkeywords}
Probability of conflict, air traffic, Subset Simulation, Direct Monte Carlo, Metropolis Hastings, Sense-and-Avoid
\end{IEEEkeywords}



\section{Introduction}
\IEEEPARstart{F}{uture} autonomous operations of Unmanned Air Systems (UAS) within densely populated airspace require an automated Sense-and-Avoid (SAA) system~\cite{angelov2012sense}. A key element within the Sense-and-Avoid (SAA) topic is Conflict Detection and Resolution (CD\&R)~\cite{angelov2012sense}. A conflict occurs when the separation between any aircraft or obstacle reduces below a minimum distance. Such a situation could~ in the worst case~ generate a collision between air vehicles but even in the absence of an actual collision it will violate the mandated Rules of the Air, and may give rise to an air ‘incident’. Such incidents must be reported as soon as possible to the local Air Traffic Service Unit (ATSU)~\cite{ICAOannex13}. 

Initial work on CD\&R can be found in robotics where the collision avoidance problem has been treated as a path planning task~\cite{moravec1980obstacle} and an early approach to the collision avoidance problem involved using artificial potential fields~\cite{khatib1986real}. Such methods are suitable for scenarios where movement of the vehicles may be relatively slow, restricted in space or in scope. However, over the following decades the increased use of UAS has created demand for autonomous CD\&R solutions which are suitable for the more dynamic aerospace environment. A large number of CD\&R methods have been proposed during this period and comprehensive surveys have been conducted by Kuchar and Yang~\cite{kuchar2000review}, Krozel et al.~\cite{krozel1997conflict}, Warren~\cite{warren1997medium} and Zeghal~\cite{zeghal1998review}. Kuchar and Yang have proposed a taxonomy of methods useful in identifying gaps and directing future efforts within the SAA community~\cite{kuchar2000review}. More recently, Albaker and Rahim have presented an up to date survey of CD\&R methods for UAS~\cite{albaker2009survey}. The work presented in this paper can be categorized as a Conflict Detection method that assumes non-cooperative sensor technology.

The CD\&R methods are broadly categorized as cooperative and non-cooperative. Cooperative methods assume that traffic shares relevant information via radio, data link or by contacting ground based ATSU. These methods are dependent on cooperative equipment such as Transponders and/or Automatic Dependent Surveillance-Broadcast (ADS-B) that are carried on-board the aircraft. This equipment declares the current state of the aircraft to nearby traffic. If the potential for a conflict is identified the situation will be resolved by coordinating maneuvers between the traffic, often via two-way radio communications. The maneuvers are dictated by following a set of customary rules that determine the ‘right-of-way’ for each aircraft. These are based on existing Visual Flight Rules (VFR) within the civil aviation domain~\cite{ICAOannex2}. In VFR, it is the flight crew's responsibility to maintain safe separation with traffic. In the absence of visual information (due to limited visibility caused by bad weather), the flight crew must rely on external information. In such situations, Instrument Flight Rules (IFR) are used with the ATSU monitoring traffic separation using Radar and then directing the flight crew so as to maintain safe separation. Alternatively, on larger aircraft, a Traffic Alert Collision Avoidance System (TCAS)~\cite{kuchar2007traffic} can be used. The TCAS system provides Resolution Advisories (RA) to flight crews of conflicting traffic in the form of maneuvers to be followed to resolve the conflict. In each case, a potential conflict is resolved in accordance with the rules given by the local aviation authority for the airspace within which the aircraft are operating; such as the Federal Aviation Authority (FAA) in the US~\cite{authority2015FAA} or the Civil Aviation Authority (CAA) in the UK~\cite{authority2015cap}. The rules stated by most aviation authorities are based on the rules outlined by the International Civil Aviation Organization (ICAO)~\cite{ICAOannex1to18}. When a conflict type is identified the appropriate resolution maneuver is executed. For example, when aircraft are approaching each other head-on the rules will say that both aircraft maneuver to their right. All traffic involved with the conflict must cooperate for a successful resolution~\cite{mishra2013doing}. Each of these methods assumes that all aircraft involved in the potential conflict are sharing information and behaving in accordance with the accepted Rules of the Air. 

In contrast, non-cooperative methods assume that no information related to the current state or future intent of traffic has been shared (i.e. there is no flight plan exchange or radio/data link). This is a far more challenging problem since information related to traffic state and intentions must be measured or inferred from the behavior of non-cooperative aircraft. Normally, this will be due to the lack of appropriate technology on-board the aircraft: for example, a lightweight commercial of-the-shelf (COTS) UAS, obtained by the general public and used for recreational purposes. Problems occur when these aircraft are operated within non-segregated airspace. This type of airspace contains aircraft (manned or unmanned) that adhere to the Rules of the Air and expect traffic to do so as well. The lack of cooperative technology on-board a lightweight UAS prevents awareness of traffic and increases the risk of a midair collision. This problem needs to be addressed due to the increased number of near miss incidents involving such UAS operating within non-segregated airspace~\cite{WashpostAug2015NearMiss}. The problem of the lack of information is addressed by using on-board sensors. Information related to state of traffic is obtained from observations using sensors such as Radar, Lidar and/or cameras. For example, Mcfadyen et al. have considered using visual predictive control with a spherical camera model to create a collision avoidance controller~\cite{mcfadyen2013aircraft}. Recently, Huh et al. have proposed a vision based Sense-and-Avoid framework that utilizes a camera to detect and avoid approaching airborne intruders~\cite{HUH2010}. A collision avoidance system that uses a combination of Radar and electro-optical sensors have been prototyped and tested by Accardo et al~\cite{Accardo2013}. Measurement data obtained from sensors are inherently noisy. This gives rise to uncertainties in the observed state and predicted motion of the non-cooperative aircraft. In an environment where future trajectories are uncertain, the likelihood of a conflict is an essential metric. Obtaining an accurate estimate for the Probability of Conflict (), given the sensor data, is a key parameter required to resolve traffic conflicts. This paper provides a method to calculate the  metric that is more efficient than the standard approach of using Direct Monte Carlo (DMC) methods.

Probabilistic methods for conflict resolution requiring the calculation of metrics like the Probability of Conflict () have been discussed in~\cite{kuchar2000review}. Nordlund and Gustafsson~\cite{Nordlund2011} noted the huge number of simulations required to get sufficient reliability for small risks and suggested an approach that reduced the three dimensional problem to a one dimensional integral along piecewise straight paths~\cite{nordlund2008probabilistic,Lindsten200965}.
More recently, Jilkov et al. have extended a method developed by Blom and Bakker~\cite{blom2002conflict} and estimated  using multiple models for aircraft trajectory prediction~\cite{jilkov2014improved}. Many probabilistic methods involve the use of Monte Carlo methods where uncertainties exist and Monte Carlo methods can be found in existing CD\&R methods~\cite{jilkov2014improved,chryssanthacopoulos2011accounting,wolf2011aircraft,belkhouche2013modeling,blom2006free,Watkins2003Stochastic,prandini2000probabilistic,Krozel1997Decision}. Unfortunately, for scenarios where the expected  is low, a Monte Carlo method will require a very large number of simulations to estimate  with any accuracy. To reduce the computational cost associated with Monte Carlo methods, Prandini et al. have estimated the risk of conflict using the Interacting Particle System (IPS) method~\cite{prandini2011air}. This method fixes a set of initial conditions of the aircraft and alters reducing subsets of the propagated trajectories to satisfy the intermediate thresholds; this assumes that the predicted trajectories are non-deterministic with the probability of conflict being associated with outliers in the propagation, not outliers in the initial conditions. If, however, the trajectory is deterministic (or near-deterministic), then IPS is unable to provide improved computational efficiency relative to direct (Monte Carlo) sampling. This paper proposes the use of the Subset Simulation method~\cite{au2001estimation} to avoid this problem and allows the initial conditions to be adjusted as the subsets are navigated. Subset Simulation approaches the problem of reducing the computational load associated with calculating low probabilities by focusing the simulation towards the rare regions of interest within the probability distribution function (pdf). The regions of interest correspond to the events which may lead to conflict between traffic. 

Originally, Au and Beck proposed Subset Simulation as a method for computing small failure probabilities as a result of (larger) conditional failure probabilities~\cite{au2001estimation}. The method was proposed in Civil Engineering to compute probabilities of structural failure and identify associated failure scenarios~\cite{au2003subset}. The focus of their work was on understanding the risk to structures posed by seismic activity. This paper modifies the methods developed by Au and Beck~\cite{IVAN} and demonstrates that they can significantly reduce the computational load required to estimate the value of  for air traffic within a block of airspace by reducing the number of samples required. The proposed method is applied to a set of conflicting and potentially conflicting test scenarios based on the Rules of the Air specified by aviation authorities. Since these scenarios are standard engagements considered by aviation authorities, they could also be used as a benchmark for comparison against future methods. The  during some scenarios is low; despite this, it is essential to provide an approximation this metric due to the catastrophic nature of a collision.

The paper is structured as follows: sections~\ref{sec:DMC} and~\ref{sec:mh} describe the Direct Monte Carlo (DMC) and Metropolis Hastings (MH) methods respectively. The Subset Simulation theory is based on a combination of DMC and MH methods. Section~\ref{sec:SS} describes Subset Simulation. Section~\ref{sec:SS_app} then describes the application of Subset Simulation to the estimation of  between air traffic in non-cooperative scenarios. Section~\ref{sec:results} presents simulation results of estimating  between air traffic for conflicting and potentially conflicting non-cooperative scenarios. Section~\ref{sec:acc_eff} analyzes the efficiency and accuracy of estimating the  using Subset Simulation and Direct Monte Carlo. Finally, section~\ref{sec:conclusion} concludes the paper.

\section{Direct Monte Carlo}
\label{sec:DMC}

The Direct Monte Carlo (DMC) method is a sampling method that can be used to characterize a distribution of interest. The objective of this section is to estimate the probability of a type of event to occur. Therefore the DMC method is used as a `statistical averaging' tool, where the probability of failure  is estimated as the ratio of failure responses to the total number of trials~\cite{IVAN}.

A set of  independent identically distributed (i.i.d) inputs  are drawn from the proposal distribution  of the input parameter space. The proposal distribution can be any known distribution that can be sampled from. We choose a Normal distribution that is centered at the mean  and has a variance of . A set of system responses are observed , where  is the system process. The occurrence of a failure event  is indicated when a scalar quantity  (threshold) is exceeded. The number of samples that exceed the threshold is . Therefore the probability of failure is estimated as . Such an approach is suitable for large probabilities (such as ) where a small number of samples can be used to estimate the probability. However for small probabilities (such as the tail region of the pdf, where ) a significantly large number of samples must be drawn to accurately estimate the probability. This is illustrated by the following example.

\subsection{Estimating probability of drawing samples from region }
\begin{algorithm}[!t]\caption{Determine distance between samples X and C}
\label{alg:h}
\begin{algorithmic}[1]
	\Function{h}{,}
        \State 
\State 
				
	\State \textbf{return }	
	\EndFunction
\end{algorithmic}
\end{algorithm}

\begin{algorithm}[!t]\caption{Direct Monte Carlo}
\label{alg:dmc}
\begin{algorithmic}[1]
	\Function{DMC}{, , }
	
\State 
	
\For{\texttt{}}
\State 
        \State 
        \State 
				
\State  \Call{H}{, }
\If {}
					\State 
				\EndIf
      \EndFor
			
\State 
	\State \textbf{return }	
	\EndFunction
\end{algorithmic}
\end{algorithm}

\begin{figure}[!t]\centering
	\subfloat[Direct Monte Carlo with 100 samples]{\includegraphics[width=\columnwidth]{direct_monte_carlo_100.eps}
	\label{fig:dmc_100}}
	\\
	\subfloat[Direct Monte Carlo with  samples]{\includegraphics[width=\columnwidth]{direct_monte_carlo_10e4_lowres.eps}
	\label{fig:dmc_10e4}}\caption{The probability of drawing samples from the region  is estimated using Direct Monte Carlo. Fig.~\ref{fig:dmc_100} estimates the  with 100 samples. Fig.~\ref{fig:dmc_10e4} estimates the  with  samples.}
	\label{fig:DMC_100_10e4}
\end{figure}

Fig.~\ref{fig:DMC_100_10e4} shows a  square centered at . The region  is a circle with radius , centered at  within this square. The objective is to estimate the probability of drawing samples from this region. The probability distribution of the overall area is represented by a Gaussian distribution centered at . A set of  samples  are drawn where each sample is a vector; . The  and  values of each sample are the x-coordinate and y-coordinates of the position respectively. To clarify,  where  and . The distance between the position of each sample and center of circle  is  as defined by Algorithm~\ref{alg:h}. To clarify, the distance between sample  and  is . Algorithm~\ref{alg:dmc} is used to estimate the probability of drawing samples from the region .

Fig.~\ref{fig:dmc_100} shows  samples drawn from the distribution. Note no samples are drawn from the area . The probability is estimated . The number of samples are increased to . Fig.~\ref{fig:dmc_10e4} shows some samples are drawn from the region  and the probability is estimated  This illustrates that Direct Monte Carlo requires a significantly large number of samples to estimate the probability of drawing samples from the region .



This method estimates  by attempting to realize the entire pdf centered at  that includes the area F. As the area  reduces the number of samples required to estimate  increases making such an approach computationally demanding. A different algorithm is needed. 
\section{Metropolis Hastings}
\label{sec:mh}

Metropolis-Hastings (MH) is a Markov Chain Monte Carlo (MCMC) method used to characterize a distribution of interest by sampling from a known distribution. We refer to this distribution of interest as the target distribution. The MH algorithm originates from the Metropolis algorithm first used in statistical Physics by Metropolis and co-workers (Metropolis et al, 1953)\cite{metropolis1953equation}. Hastings proposed a generalized form of this algorithm leading to the Metropolis Hastings (MH) algorithm~\cite{hastings1970monte}.

The MH method generates samples from the proposal distribution  by starting from a seed value . A chain of  samples is then generated, starting with . The sample  is generated from the current sample  using the following steps~\cite{IVAN}:

\begin{enumerate}
  \item Generate a candidate sample .
  \item Calculate an acceptance ratio:  
	\item Draw a sample  from a uniform distribution [0,1]	
  \item Set 
						
	\item Repeat steps 1 to 4 until  samples have been generated.
\end{enumerate}

\noindent The function  defines the target density for the input sample. While, , this process is guaranteed to accept samples from  that leads to the realization of the target distribution~\cite{robert2004metropolis}. To help ensure that all regions of the target density are explored, multiple seeds can be used to generate multiple chains of samples in parallel~\cite{IVAN}.

\subsection{Drawing samples from the region }

\begin{figure}[h]\centering
	\includegraphics[width=\columnwidth]{metropolis_hastings.eps}
\caption{Drawing samples from the region F using Metropolis Hastings algorithm to generate chains of conditional samples. The initial samples used as seeds are drawn using Direct Monte Carlo.}
	\label{fig:mh}
\end{figure}

The Metropolis Hastings method is defined in algorithm~\ref{alg:mh} and it is applied to the example of estimating the probability of drawing samples from region  as shown in the previous section. The covariance of the proposal  is a  identity matrix  and the covariance of the distribution of interest  where  is the radius of the region . For this example , therefore .

Fig.~\ref{fig:mh} illustrates the chains of samples generated by the Metropolis Hastings algorithm. This figure shows 10 samples drawn from the proposal distribution using the DMC method. These samples are seeds . The MH algorithm is applied using the seeds . Each seed generates a chain of 10 samples. Note that many sample chains do not reach the region . It is clear that it might be more efficient to generate more samples for chains with seeds that are closer to the region  since they have higher likelihood of generating samples that are within the region  or closer to the region . Subset Simulation achieves this and is described in the next section.

\begin{algorithm}\caption{Generate conditional chains of samples using Metropolis Hastings algorithm}
\label{alg:mh}
\begin{algorithmic}[1]
\Function{MH}{, , , }
	\State 
	\For{\texttt{}} \textit{ For each seed}	
		\State 	\textit{Select seed sample}
		\For{\texttt{}}
			
\LineComment{Generate Candidate sample }
			\State 
			\State 
			
\LineComment{Calculate acceptance ratio}

			\State 
			\State 
			\State 
			
\State 
		\EndFor	
	\EndFor

\State \textbf{return }

\EndFunction
\end{algorithmic}
\end{algorithm}

\section{Subset Simulation}
\label{sec:SS}
Subset Simulation (SS) is based on a combination of Direct Monte Carlo (DMC) and Metropolis Hastings (MH) methods as described in sections~\ref{sec:DMC} and~\ref{sec:mh} respectively. It calculates the probability of rare events occurring as the product of the probabilities of less-rare events. Such an approach is less computationally expensive than either DMC or MH alone. A general outline of the SS method is presented in this paper and the interested reader is referred to~\cite{IVAN} for more details.

Subset Simulation generates a Complimentary Cumulative Distribution Function (CCDF) of the response quantity of interest . The probability of failure  can be directly estimated from the CCDF. This CCDF is constructed by generating samples that satisfy a series of intermediate thresholds  that divide the space into  nested regions. These thresholds are adaptively defined as the simulation progresses. This is described later on in this section. The threshold  is the required failure threshold  (). The intermediate thresholds allow the probability of failure to be estimated using a classical conditional structure given by



Samples are generated to satisfy the threshold for each level. The total number of levels  is dependent on the magnitude of the target probability . Subset Simulation uses `level probability'  to control how quickly the simulation reaches the target event of interest~\cite{IVAN}. The target probability is used to approximate the number of levels  required by evaluating . To clarify, if the target probability is  and  then the total number of levels required will be .

\subsection{Level 0}

Subset Simulation begins at level  with Direct Monte Carlo (DMC) sampling from the entire region of interest. A set of  samples  are drawn from a proposal distribution  (as described in section~\ref{sec:DMC}). The set of output responses  are evaluated . The function  defines the system response to the input sample. In the context of SS, the responses  are also known as the quantity of interest. The set  is sorted in descending order to create the set . The input samples  are reordered  and correspond to the sorted quantity of interest . To clarify,  is the input sample that generates the largest output . A CCDF is generated by plotting  against the probability intervals . The probability intervals  are generated using the following equation:



\begin{table*}[!t] \centering
\resizebox{2\columnwidth}{!}{\begin{tabular}{cccccccccc}
\multicolumn{1}{c}{} & \multicolumn{1}{c}{Level 0} & \multicolumn{1}{c}{} & & Level  & & & & Level  & \\ \cline{1-10}
\multicolumn{1}{|c|}{} & \multicolumn{1}{c|}{} & \multicolumn{1}{c|}{} & \multicolumn{1}{|c|}{}& \multicolumn{1}{c|}{}& \multicolumn{1}{|c|}{} & \multicolumn{1}{c|}{.....} & \multicolumn{1}{|c|}{}& \multicolumn{1}{c|}{} & \multicolumn{1}{c|}{}\\ \cline{1-10}
\multicolumn{1}{|c|}{} & \multicolumn{1}{c|}{} & \multicolumn{1}{c|}{} & & & & & & & \\ \cline{1-3}
\multicolumn{1}{|c|}{\vdots} & \multicolumn{1}{c|}{\vdots} & \multicolumn{1}{c|}{\vdots} & & & & & &                       &                       \\ \cline{1-3}
\multicolumn{1}{|c|}{} & \multicolumn{1}{c|}{} & \multicolumn{1}{c|}{}   	 & & & & & & & \\ \cline{1-6}
\multicolumn{1}{|c|}{} & \multicolumn{1}{c|}{}    & \multicolumn{1}{c|}{}   & \multicolumn{1}{c|}{} & \multicolumn{1}{c|}{} & \multicolumn{1}{c|}{} &  & & & \\ \cline{1-6}
\multicolumn{1}{|c|}{\vdots}    & \multicolumn{1}{c|}{\vdots}    & \multicolumn{1}{c|}{\vdots}   & \multicolumn{1}{c|}{\vdots} & \multicolumn{1}{c|}{\vdots} & \multicolumn{1}{c|}{\vdots} &  &  & &                       \\ \cline{1-6}
\multicolumn{1}{|c|}{}    & \multicolumn{1}{c|}{}    & \multicolumn{1}{c|}{}   & \multicolumn{1}{c|}{} & \multicolumn{1}{l|}{} & \multicolumn{1}{l|}{} & & & & \\ \cline{1-10}
                             &                          & \multicolumn{1}{c|}{}   & \multicolumn{1}{c|}{} & \multicolumn{1}{c|}{} & \multicolumn{1}{c|}{} &  \multicolumn{1}{c|}{.....}&  \multicolumn{1}{c|}{} & \multicolumn{1}{c|}{} & \multicolumn{1}{c|}{}\\ \cline{4-10}
                             &                          & \multicolumn{1}{c|}{}   & \multicolumn{1}{c|}{\vdots} & \multicolumn{1}{c|}{\vdots} & \multicolumn{1}{c|}{\vdots} & \multicolumn{1}{c|}{.....} & \multicolumn{1}{c|}{\vdots} & \multicolumn{1}{c|}{\vdots} &  \multicolumn{1}{c|}{\vdots}      \\ \cline{4-10}
                             &                          & \multicolumn{1}{c|}{} & \multicolumn{1}{c|}{} & \multicolumn{1}{c|}{} & \multicolumn{1}{c|}{} &   \multicolumn{1}{c|}{.....} & \multicolumn{1}{c|}{} & \multicolumn{1}{c|}{} & \multicolumn{1}{c|}{}\\ \cline{4-10}
                             &                          &                         &                       &                       & \multicolumn{1}{c|}{} & \multicolumn{1}{c|}{.....} & \multicolumn{1}{c|}{} & \multicolumn{1}{c|}{} & \multicolumn{1}{c|}{} \\ \cline{7-10} 
                             &                          &                         &                       &                       &                       & \multicolumn{1}{c|}{} & \multicolumn{1}{c|}{\vdots} & \multicolumn{1}{c|}{\vdots} & \multicolumn{1}{c|}{\vdots} \\ \cline{8-10}
                             &                          &                         &                       &                       &                       & \multicolumn{1}{c|}{} & \multicolumn{1}{c|}{} & \multicolumn{1}{c|}{} & \multicolumn{1}{c|}{} \\ \cline{8-10} 
														
\end{tabular}
}
\caption{}
\label{table:general_subset_multilevel}
\end{table*}

\noindent The vector of probability intervals  is concatenated with the sorted quantity of interest  and their respective samples  as illustrated in table~\ref{table:general_subset_multilevel} by the column titled `Level 0'. 
The set of probability intervals  are plotted against  to generate the CCDF. Level 0 makes it possible to accurately approximate CCDF values from  to . Typically the region of interest within the pdf is outside this range (since SS is typically used to realize rare events). To explore probabilities below , further levels of simulation must be conducted.

\subsection{Level }

The subsequent levels of SS where,  explore the rarer regions of the probability distribution. This is achieved by generating multiple chains of conditional samples using the MH method as discussed in the previous section. The number of chains and number of samples per chain are  and  respectively. They are determined as





\noindent Each level of subset simulation maintains  samples (). The response values of conditional samples generated for the current level  must not exceed the intermediate threshold  for this level. This threshold is determined by



\noindent The intermediate threshold for level  is . To clarify the intermediate threshold is the  element of the sorted set of response values . The set of seeds  are used to generate samples for the current level  are samples generated from the previous level () are defined by

\begin {equation}
	s_{j}^{(i)} = \tilde{X}_{n}^{(i-1)}
\label{eq:seeds}

A = \begin{bmatrix}
	1 & \Delta T & \frac{1}{2}\Delta T^{2} & 0 & 0 & 0\\	
	0 &  1 & \Delta T & 0 & 0 & 0\\	
	0 &  0 & 1 & 0 & 0 & 0 \\
	0 &  0 & 0 & 1 & \Delta T & \frac{1}{2}\Delta T^{2} \\
	0 &  0 & 0 & 0 & 1 & \Delta T \\
	0 &  0 & 0 & 0 & 0 & 1 
\end{bmatrix}
\label{eq:transition_matrix}

	Z = HU(K) + [w_{x},w_{y}]'
	\label{eq:measurement}

H = \begin{bmatrix}
1 & 0 & 0 & 0 & 0 & 0\\	
0 & 0 & 0 & 1 & 0 & 0\\	
\end{bmatrix}
 \label{eq:measurement_matrix} 

w_{x} \sim{~} \mathcal{N}(0,\sigma_{x})
\label{eq:noise_x}

w_{y} \sim{~} \mathcal{N}(0,\sigma_{y})
\label{eq:noise_y}

\hat{U}(K+1) = A\hat{U}(K)
\label{eq:kalman_dyn}

\hat{S}(K+1) = A\hat{S}(K)A^{T} + Q
\label{eq:kalman_error_predict}

	Q = \begin{bmatrix}		
			 Q_{\sigma}\frac{\sigma_{a_{x}}^{2}}{\Delta T} & 0 \\
			 0 & Q_{\sigma}\frac{\sigma_{a_{y}}^{2}}{\Delta T}
		\end{bmatrix}

	Q_{\sigma} = \begin{bmatrix}
	\frac{1}{20}\Delta T^{5} & \frac{1}{8}\Delta T^{4} & \frac{1}{6}\Delta T^{3} \\	
	\frac{1}{8}\Delta T^{4} & \frac{1}{3}\Delta T^{3} & \frac{1}{2}\Delta T^{2} \\	
	\frac{1}{6}\Delta T^{3} & \frac{1}{2}\Delta T^{2} & \Delta T
	\end{bmatrix} 

G = \hat{S}(K+1)H^{T}([H\hat{S}(K+1)H^{T}]+R)^{-1}
\label{eq:kalman_gain}

R = \begin{bmatrix}
\sigma_{x}^{2} & 0 \\	
0 & \sigma_{y}^{2} 
\end{bmatrix} 

\hat{U}(K+1) = \hat{U}(K+1)+G\{Z(K)-[H\hat{U}(K+1)]\}
\label{eq:kalman_est}

\hat{S}(K+1) = [I - GH]\hat{S}(K+1)
\label{eq:kalman_cov}
	
	P_{n+1}^{(i)} = p_{0}^i\frac{N - n}{N}	\quad n = 0,...,(N-1)
	\label{eq:prob_intervals_mod}


\noindent where ,  and .

Fig.~\ref{fig:headon_level0_seeds} highlights the trajectories of level 0 samples selected as seeds to generate level 1 conditional samples. Fig.~\ref{fig:headon_level1} shows the trajectories of the conditional samples generated in level 1.
The set  contains  seeds; one for each chain. Each chain generates  samples. This maintains the total number of samples as  for each level. The MH method uses an indicator  (as shown in algorithm~\ref{alg:conflict_samples_MH}) to ensure the miss-distance  between the Observer's trajectory  and Intruder trajectory  of the proposed sample  is less than the intermediate threshold  set by equation~\ref{eq:thresholds}. If  then the proposed sample is rejected and the current sample of the Intruder is maintained.

The miss-distances  of the conditional samples  generated in level 1 are determined and sorted in descending order  using the same method as level 0. The input samples  are reordered  to correspond to the sorted miss-distances . The probability intervals  for the current level are generated and plotted against  to construct a CCDF. Fig.~\ref{fig:headon_level1_CCDF} shows the CCDF generated up to level 1. Note the miss-distances of the samples used as seeds from the previous level 0 (that are highlighted in Fig.~\ref{fig:headon_level0_seeds_CCDF}) are discarded and replaced with the miss-distances of the conditional samples generated in level 1. This illustrates that the samples used as seeds are discarded and replaced with the conditional samples generated in the current level. This process is repeated as SS progresses to higher levels until the condition  is satisfied or the maximum number of levels is reached as defined in algorithm~\ref{alg:SS_pc}. Fig.~\ref{fig:headon_level3_conflict} shows the trajectories of the conflicting samples encountered in level 3. However the condition  had not been satisfied. This required SS to proceed to level 4 and generate conditional samples that satisfy the condition  as shown in Fig.~\ref{fig:headon_level4_conflict}. The CCDF generated up to level 4 is shown in Fig.~\ref{fig:headon_level4_conflict_CCDF}. The CCDF is used to estimate the  as shown in Fig.~\ref{fig:headon_level4_conflict_CCDF_zoom}.
This process is repeated through out the duration of the simulation to determine the probability of conflict for each time-step using samples from the prediction of the Intruder's estimate  and covariance .

\begin{algorithm}[!t]
\caption{Determine Probability of Conflict using SS and DMC}
\label{alg:pc_ss_dmc}
\begin{algorithmic}[1]
\State  \textit{Initialize Observer}
\State  \textit{Initialize Intruder}
\State  \textit{Initialize Intruder Estimate}
\State  \textit{Initialize Intruder Covariance}

\State  \textit{Measurement counter}

\For{\texttt{}}
	
\State  \textit{Propagate Observer}
		
\State  \textit{Propagate Intruder}
		
\State  \textit{Flag to indicate new measurement}
		\If {} \textit{Conduct Intruder position measurement}
			\State 
				\State  \textit{Set flag to indicate that new measurement is available for Kalman filter Update}
				\State  \textit{ Reset measurement counter}
		\EndIf
		\State  \textit{Increment measurement counter}
		
\LineComment{Predict/Update estimate of Intruder with Kalman filter}
		\State  \Call{KF}{, }
		
\LineComment{Estimate Probability of Conflict using Subset Simulation}
		
\State 
		
		\Call{PC\_SS}{, , , , , , , , , }
\LineComment{Estimate Probability of Conflict using Direct Monte Carlo}
		
\State 
		
		\Call{PC\_DMC}{}
	\EndFor
\end{algorithmic}
\end{algorithm}

\section{Results}
\label{sec:results}

The Subset Simulation method has been tested and compared with the Direct Monte Carlo (DMC) method to estimate the probability of conflict  between the Observer and Intruder by simulating the scenarios shown in Fig.~\ref{fig:potential_conflict_scenarios}. The Observer and Intruder were modeled as points with nearly constant velocity in a geometric configuration based on the three different types of conflict shown in Fig.~\ref{fig:conflict_types}. The  metric was estimated as an average of 50 Monte Carlo simulation during the Head-on and Overtaking conflicts as shown in figures~\ref{fig:potential_headon} and~\ref{fig:intruderovertaking_potential} respectively. The tests were repeated with varying lateral separations . 

The following Subset Simulation parameters were used for all scenarios: ; Level probability: ;  ; ; ; Observer minimum separation threshold . Algorithm~\ref{alg:pc_ss_dmc} defines the simulation conducted.

The number of samples used for each level of SS remain constant. However the number of levels required at a given time-step vary depending on the magnitude of . Therefore the total number of samples  required to realize a conflict at a given time-step varies as a function of time-step. In the interest of a fair comparison of the computational effort between the two methods, an equal number of samples are evaluated for both methods. The estimation using DMC is conducted with  samples, where  is the number of samples that are used in the SS method at the same time-step. To clarify, if the SS method reaches level  = 4 to satisfy the conflict condition for estimating the  at time-step , then  samples have been used by the SS method. Therefore DMC estimates the  for the same time-step with 500 samples only.




\begin{figure*}\centering	
	\subfloat[ during Head-on conflict with 0 m Lateral separation]{\includegraphics[trim={0 0cm 0 0cm},clip,width=0.66\columnwidth]{Head_on_pass_0_MC_50.eps}
	\label{fig:headonpass_SS_DMC_latsep_0}}
	\hfill 
	\subfloat[ during Head-on conflict with 100m Lateral separation]{\includegraphics[trim={0 0cm 0 0cm},clip,width=0.66\columnwidth]{Head_on_pass_100_MC_50.eps}
	\label{fig:headonpass_SS_DMC_latsep_100}}	
	\hfill 
	\subfloat[ during Head-on conflict with 152.4m Lateral separation]{\includegraphics[trim={0 0cm 0 0cm},clip,width=0.66\columnwidth]{Head_on_pass_152_MC_50.eps}
	\label{fig:headonpass_SS_DMC_latsep_152}}
	\hfill
	\subfloat[ during Head-on pass with 500m Lateral separation]{\includegraphics[trim={0 0cm 0 0cm},clip,width=0.66\columnwidth]{Head_on_pass_500_MC_50.eps}
	\label{fig:headonpass_SS_DMC_latsep_500}}
	\hfill
	\subfloat[ during Head-on pass with 1000m Lateral separation]{\includegraphics[trim={0 0cm 0 0cm},clip,width=0.66\columnwidth]{Head_on_pass_1000_MC_50.eps}
	\label{fig:headonpass_SS_DMC_latsep_1000}}	
	\hfill
	\subfloat[ during Head-on pass with 1100m Lateral separation]{\includegraphics[trim={0 0cm 0 0cm},clip,width=0.66\columnwidth]{Head_on_pass_1100_MC_50.eps}
	\label{fig:headonpass_SS_DMC_latsep_1100}}
	\caption{The estimated  using the Subset Simulation and Direct Monte Carlo methods during the Head-on pass as shown in Fig.~\ref{fig:potential_headon} with varying lateral separation .
	}
	\label{fig:headon_set1}	
\end{figure*}

\subsection{Estimation of  for Head-on Pass scenario}

The Intruder and Observer parameters used for the Head-on pass scenario are as follows: The Intruder and Observer maintain a constant speed of 150 knots (77.17ms). The Observer maintains a constant heading of ; the Intruder maintains a constant heading of . The Observer's minimum separation threshold is . The Longitudinal separation is 

Figures~\ref{fig:headonpass_SS_DMC_latsep_0},~\ref{fig:headonpass_SS_DMC_latsep_100} and~\ref{fig:headonpass_SS_DMC_latsep_152} show the estimation of  for the Head-on pass scenario using SS and DMC methods with lateral separations of 0m, 100m and 152m respectively. The scenarios are conflicting because the geometric configuration and initial conditions of both the Observer and Intruder are conflicting and remain as such throughout the duration of the simulation. When  the Intruder and Observer are approaching each other the estimated  increases. This is as expected because a conflict is imminent. Both estimation methods show approximately the same  as expected, since the first level of the SS method is DMC sampling. At this stage the conflict region of the pdf is large and the probability of drawing a sample which leads to a conflict is high. The conflict occurs at s due to the loss of separation between the Observer and Intruder. Fig.~\ref{fig:headonpass_SS_DMC_latsep_152} shows the estimation of  with lateral separation . This is a conflicting scenario since the Intruder skims Observer's protected boundary at  as the Observer and Intruder pass each other. The oscillations during  are due to . This is a borderline situation. 

The Intruder and Observer pass each other at s. The  estimated by both methods is still  until s where the Intruder has exited the Observer's protected zone. At this stage the Observer and Intruder have receding relative velocities and are moving away from each other.  is expected to reduce at this stage as shown in the log- plot. The conflict region of the pdf reduces since both Intruder and Observer are moving away from each other. The SS method estimates the  as being close to zero at an order of magnitude of . The lowest probability which can be realized is . This is due to a maximum level restriction imposed in the simulation. In such instances the probability of conflict can be considered to be less than the order of . At this stage the DMC method draws the same number of samples as SS but is unable to find conflicting samples and estimates . This is because the region of conflict within the pdf has reduced and the probability of drawing a conflicting sample is rare. This requires the DMC method to draw and evaluate a larger number of samples at this stage before a conflicting sample is drawn from the rare region of conflict within the pdf. The SS method is able to obtain the conflicting samples from the rare region of the pdf by generating samples conditionally in such a way that the samples satisfy the intermediate thresholds leading to the rare region using the MH method. Each subset level corresponds to an intermediate threshold.  This progressive feature of the SS method allows a more efficient approach to reach the rare `tail' region of the pdf.

As the lateral separation of the scenario is increased, the expected  decreases. The scenario is simulated with a lateral separation of 500m, 1000m and 1100m as shown in figures~\ref{fig:headonpass_SS_DMC_latsep_500},~\ref{fig:headonpass_SS_DMC_latsep_1000} and~\ref{fig:headonpass_SS_DMC_latsep_1100} respectively. These are non-conflicting scenarios. The figures show abrupt variations in . These are caused by the Monte Carlo nature of our algorithm. Note that, since the sampling frequency is high relative to the thickness of the line in the figure, the variations in  are particularly readily perceived. The conflict region of the pdf is smaller than the previous scenarios. The SS estimation method is able to estimate low  throughout the duration of the simulation, whereas with an equivalent number of samples the DMC method is unable to find conflicting or near conflicting samples of the Intruder in most instances. Fig.~\ref{fig:headonpass_SS_DMC_latsep_500} shows abrupt variations in the  estimated by the DMC method when  where the estimate tends to zero. These are instances where the DMC method is unable for find any conflicting samples and estimates .

\begin{figure}\centering	
	\subfloat[Head-on conflict scenario with 1000m Lateral separation before head-on pass.]{\includegraphics[width=\columnwidth]{headon_pass_1000m_beforeConflict_lowres.eps}
	\label{fig:headonpass_SS_DMC_latsep_1000_before}}
	\\
	\subfloat[Head-on conflict scenario with 1000m Lateral separation after pass.]{\includegraphics[width=\columnwidth]{headon_pass_1000m_afterConflict_lowres.eps}
	\label{fig:headonpass_SS_DMC_latsep_1000_after}}	
	\caption{SS and DMC trajectories for Head-on pass with lateral separation 1000m}
	\label{fig:headonpass_1000m_snapshot}	
\end{figure}

Figures~\ref{fig:headonpass_SS_DMC_latsep_1000_before} and~\ref{fig:headonpass_SS_DMC_latsep_1000_after} show the trajectories of the samples evaluated by SS and DMC methods at an instance before and after the Intruder and Observer pass each other respectively. The progressive nature of the SS method can be observed as a concentration of trajectories leading to the conflict trajectory. In contrast the DMC method has drawn the same number of samples (most are overlapping) without realizing any conflicts.

\subsection{Estimation of  for Intruder Overtaking Observer}

The scenario parameters used are as follows: The Intruder speed is  and the Observer speed is . Both Intruder and Observer maintain a constant heading of a constant heading of . The longitudinal distance  between the Intruder and Observer is .

Both SS and DMC methods have been applied to the Overtaking scenario as shown in Fig.~\ref{fig:intruderovertaking_potential}. Similar to the previous scenario, the SS method is able to obtain samples from the rare conflicting region of the pdf consistently throughout the duration of the simulation for this scenario. As the lateral separation increases, the  decreases (as expected). Figures~\ref{fig:intruderOvertaking_1000} and~\ref{fig:intruderOvertaking_1100} show the  when the lateral separation is 1000m and 1100m respectively. The change in  is less abrupt compared to the 100m lateral separation after the Intruder as passed the Observer when s. The  is approximately the same throughout the duration of the simulation. This is because the increased lateral separation includes samples with low turn rates in the conflict category and these are common enough to be drawn by the DMC method and SS method. With low lateral separation the conflicting samples will need high turn rates. These are rare and are realized by using SS method. In contrast the DMC method is unable to realize them. Also throughout the simulation, the relative change in angle of the Intruder from the Observer's perspective reduces as the lateral separation is increased. The conflicting samples can have lower turn rates despite the Intruder having passed the Observer. Such samples are common and can be realized by both methods.

\begin{figure*}\centering
\subfloat[ during Intruder overtaking Observer conflict with 0m Lateral separation]{\includegraphics[trim={0 0cm 0 0cm},clip,width=0.66\columnwidth]{IntruderOvertaking_0_MC_50.eps}
	\label{fig:intruderOvertaking_0}}	
	\hfill
	\subfloat[ during Intruder overtaking Observer conflict with 100m Lateral separation]{\includegraphics[trim={0 0cm 0 0cm},clip,width=0.66\columnwidth]{IntruderOvertaking_100_MC_50.eps}
	\label{fig:intruderOvertaking_100}}
	\hfill
	\subfloat[ during Intruder overtaking Observer conflict with 152m Lateral separation]{\includegraphics[trim={0 0cm 0 0cm},clip,width=0.66\columnwidth]{IntruderOvertaking_152_MC_50.eps}
	\label{fig:intruderOvertaking_152}}
	\hfill	
	\subfloat[ during Intruder overtaking Observer with 500m Lateral separation]{\includegraphics[trim={0 0cm 0 0cm},clip,width=0.66\columnwidth]{IntruderOvertaking_500_MC_50.eps}
	\label{fig:intruderOvertaking_500}}		
	\hfill
	\subfloat[ during Intruder overtaking Observer with 1000m Lateral separation]{\includegraphics[trim={0 0cm 0 0cm},clip,width=0.66\columnwidth]{IntruderOvertaking_1000_MC_50.eps}
	\label{fig:intruderOvertaking_1000}}	
	\hfill
	\subfloat[ during Intruder overtaking Observer with 1100m Lateral separation]{\includegraphics[trim={0 0cm 0 0cm},clip,width=0.66\columnwidth]{IntruderOvertaking_1100_MC_50.eps}
	\label{fig:intruderOvertaking_1100}}
	\caption{The  is estimated using the SS and DMC methods during the Intruder Overtaking the Observer scenario as shown in Fig.~\ref{fig:intruderovertaking_potential} with varying lateral separation .}
	\label{fig:intruderOvertaking_set1}	
\end{figure*}

\section{Accuracy and Efficiency of Subset Simulation}
\label{sec:acc_eff}

A range of magnitudes of probabilities have been evaluated within the simulated scenarios shown in the previous section. This section analyzes the accuracy and efficiency of using the Subset Simulation and Direct Monte Carlo methods to estimate probabilities at each of a number of orders of magnitude. In order for a fair comparison to be conducted -- a common phase within a simulation scenario must be found where both methods are able to realize conflicting samples and estimate the probability of conflict. 

The first order of magnitude considered for comparison is . A suitable phase to conduct the comparison is at  during the Head-on scenario with lateral separation  and longitudinal separation  where a conflict is inevitable. At this phase  and both methods estimate a similar probability of conflict. This is as expected since the probability is large enough to generate sufficient conflicting samples in the first level of Subset Simulation and it does not progress to higher levels of Subset Simulation. The first level of Subset Simulation is Direct Monte Carlo so the performance is the same.

The second order of magnitude considered is . This probability needs to be lower than  where . Such phases occur frequently in the Head-on pass and Overtaking scenarios, typically when  as shown in figures~\ref{fig:headon_set1} and~\ref{fig:intruderOvertaking_set1} respectively. Note, during such phases the Subset Simulation method is able to obtain conflicting samples and provide a good estimate for . However, the Direct Monte Carlo method fails to find conflicting samples and is unable to estimate the probability of conflict accurately (other than in a trivial case,  that is inaccurate). For example the Head-on pass scenarios in Fig.~\ref{fig:headon_set1} shows abrupt changes in  in some cases from a magnitude of  to  at approximately  as the Observer and Intruder pass each other. This change in magnitude of probability is very large and abrupt (steep). The magnitude  is very rare. For such probabilities the Subset Simulation method is able to obtain conflicting samples and estimate the  but Direct Monte Carlo method fails to obtain conflicting samples and results in estimating . The Direct Monte Carlo method requires a large number of samples to estimate probabilities of such magnitude (). This might not be practical due to limited simulation resources. Therefore, this order of magnitude of probability is impractical for comparison since although the Subset Simulation method is able to find conflicting samples and estimate the , the Direct Monte Carlo method is unable to find conflicting samples and fails to estimate the .

\begin{figure}\centering
	\includegraphics[width=\columnwidth]{Head_on_pass_20KM_1000_MC_1.eps}
	\caption{Head-on pass scenario with 1000m lateral separation and 20km longitudinal separation}
	\label{fig:headon_long}
\end{figure}

In order to find a phase where  can be evaluated by both methods the simulation of the Head-on pass scenario with lateral separation of 1000m was repeated once with increased longitudinal separation  for an increased period of . This allowed the change in  to occur less abruptly. Fig.~\ref{fig:headon_long} shows  estimated by Subset Simulation and Direct Monte Carlo methods during this scenario. Note, during the period , there are frequent abrupt variations in the  estimated by the Direct Monte Carlo method as zero. These are phases where the method was unable to find a conflicting sample and estimated the probability of conflict as zero. A suitable phase for  is at  where the probability of conflict estimated by Subset Simulation has reduced to approximately ; (). This satisfies the  criteria. Also, it is the last phase after which the frequency of the Direct Monte Carlo method finding conflicting samples to estimate the  diminishes. In other words, it is the last phase where both methods are able to generate conflicting samples to estimate the probability of conflict for a comparison to be conducted.

The accuracy and efficiency are compared by calculating the coefficient of variance (c.o.v.)  for estimating the probabilities of conflict  and  using both Subset Simulation and Direct Monte Carlo methods for varying samples sizes . The mean  and standard deviation  is calculated over 50 Monte Carlo runs. The sample intervals for Direct Monte Carlo are  and the sample intervals for Subset Simulation are . Note, that  is the number of samples at each level of Subset Simulation. The total number of levels can vary for each Monte Carlo run of Subset Simulation. This causes a total number of samples to vary for each Monte Carlo run. To allow a fair comparison an average of the total number of samples for each Monte Carlo run of Subset Simulation is used. 



\begin{figure*}\centering	
	\subfloat[Coefficient of variance for varying number of samples using SS and DMC methods for estimating ]{\includegraphics[trim={0 0cm 0 0cm},clip,width=\columnwidth]{SS_prop_vs_DMC_MC_scenario_20_2016_3_31_22_9.eps}
	\label{fig:SS_vs_DMC_scenario_2}}	
	\hfill
	\subfloat[Coefficient of variance for varying number of samples using SS and DMC methods for estimating ]{\includegraphics[trim={0 0cm 0 0cm},clip,width=\columnwidth]{SS_prop_vs_DMC_MC_scenario_2000_2016_3_31_18_5.eps}
	\label{fig:SS_vs_DMC_scenario_140}}	
	\hfill
	\caption{A comparison of accuracy and efficiency between DMC and SS methods for estimating the  during the Head-on pass scenario.}
	\label{fig:headonpass_1100m_snapshot}	
\end{figure*}

The c.o.v. for estimating  using Subset Simulation and Direct Monte Carlo methods at varying sample sizes  is shown by Fig.~\ref{fig:SS_vs_DMC_scenario_2}. Note both methods have similar c.o.v. as the average sample size increases. This is expected since the probability is large enough to be realized in level 0 of Subset Simulation that is Direct Monte Carlo. In Fig.~\ref{fig:SS_vs_DMC_scenario_140} the c.o.v. of Subset Simulation for the lower probability of conflict  becomes significantly lower than the c.o.v. of DMC as the average number of samples is increased. A point of comparison between both methods can be made where the number of samples . Note that the c.o.v for Direct Monte Carlo is approximately  and the c.o.v for Subset Simulation is approximately . Also note that in order for the DMC method to achieve similar c.o.v as the Subset Simulation method it must use  samples. Therefore the Subset Simulation estimates probabilities of magnitude  approximately 10 times more accurately than the Direct Monte Carlo method while using a fraction of the samples (approximately ) that are required by the Direct Monte Carlo method to achieve similar levels of accuracy. 

\section{Conclusion}
\label{sec:conclusion}

This paper has demonstrated the utility of the Subset Simulation method to estimate the Probability of Conflict () between air traffic within a block of airspace during conflicting and potentially conflicting scenarios based on the Rules of the Air defined by the International Civil Aviation Organization. These scenarios can be used to conduct benchmarks for comparing future algorithms. The Subset Simulation method has demonstrated the ability to seek samples from the rare conflict region of interest in an effort to estimate the probability of conflict with lower computational effort than Direct Monte Carlo method. For the equivalent number of samples, the Direct Monte Carlo method fails to consistently obtain samples from the region of interest within the probability distribution function. 

This paper has also demonstrated the ability of Subset Simulation to estimate low probability of conflict (of magnitude ) approximately 10 times more accurately than the Direct Monte Carlo method while using approximately  of the total samples used by the Direct Monte Carlo method to achieve the same level of accuracy as Subset Simulation. This has been demonstrated at a phase during a potentially conflicting scenario based on the Rules of the Air. This example situation has demonstrated that the Subset Simulation method is able to estimate low probabilities more accurately than Direct Monte Carlo method while using less samples than the Direct Monte Carlo method. We conclude that Subset Simulation method is more accurate and efficient than the Direct Monte Carlo method for estimating low probability of conflict between air traffic.  

The Subset Simulation method is scalable to involve multiple Intruders where the  is estimated for each Intruder. This would be useful for the resolution stage, where Intruders can be prioritized based on the respective  and an optimized resolution maneuver determined to minimize the new  after the resolution maneuver. A more efficient method of estimating the  would be to modify the SS method further to use Sequential Monte Carlo Samplers instead of Markov Chain Monte Carlo~\cite{del2006sequential}. This will allow the implementation to be parallelized in the seed selection stage and will give rise to improved statistical efficiency. We plan to investigate such improvements in future work.


\section*{Acknowledgments}
The authors would like to thank Matteo Fasiolo, Fl\'avio De Melo, Elias Griffith and James Wright for their contributions. This work was supported by the Engineering and Physical Sciences Research Council (EPSRC) Doctoral Training Grant.

\bibliography{conflict_detection_ref}{}
\bibliographystyle{ieeetr}

\end{document}
