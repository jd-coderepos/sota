\pdfoutput=1  \documentclass[]{article}  
\usepackage{url,float}
\usepackage{graphicx}
\usepackage{amsmath}
\usepackage{amsfonts}
\usepackage{amssymb}
\usepackage{latexsym}




\newcommand{\hide}[1]{}

\newcommand{\ABox}{
\raisebox{3pt}{\framebox[6pt]{\rule{6pt}{0pt}}}
}
\newenvironment{proof}{{\bf Proof:}}{\hfill\ABox}


\newtheorem{theorem}{{\bf Theorem}}
\newtheorem{corollary}{Corollary}
\newtheorem{lemma}{Lemma}


\newcommand{\lemlab}[1]{\label{lemma:#1}}
\newcommand{\thmlab}[1]{\label{thm:#1}}
\newcommand{\corlab}[1]{\label{cor:#1}}
\newcommand{\deflab}[1]{\label{def:#1}}
\newcommand{\tablab}[1]{\label{tab:#1}}
\newcommand{\figlab}[1]{\label{fig:#1}}
\newcommand{\seclab}[1]{\label{sec:#1}}


\newcommand{\lemref}[1]{\ref{lemma:#1}}
\newcommand{\thmref}[1]{\ref{thm:#1}}
\newcommand{\corref}[1]{\ref{cor:#1}}
\newcommand{\defref}[1]{\ref{def:#1}}
\newcommand{\secref}[1]{\ref{sec:#1}}
\newcommand{\figref}[1]{\ref{fig:#1}}
\newcommand{\tabref}[1]{\ref{tab:#1}}




{\makeatletter
 \gdef\xxxmark{\expandafter\ifx\csname @mpargs\endcsname\relax \expandafter\ifx\csname @captype\endcsname\relax \marginpar{xxx}\else
       xxx \fi
   \else
     xxx \fi}
 \gdef\xxx{\@ifnextchar[\xxx@lab\xxx@nolab}
 \long\gdef\xxx@lab[#1]#2{{\bf [\xxxmark #2 ---{\sc #1}]}}
 \long\gdef\xxx@nolab#1{{\bf [\xxxmark #1]}}
\gdef\turnoffxxx{\long\gdef\xxx@lab[##1]##2{}\long\gdef\xxx@nolab##1{}}}





\def\R{{\mathbb{R}}}
\def\D{{\Delta}}
\def\c{{\chi}}
\def\s{{\sigma}}


\newcommand{\squeezelist}{\setlength{\itemsep}{0pt}}



\title{A Note on Solid Coloring\\of Pure Simplicial Complexes
} 

\author{Joseph O'Rourke\thanks{Department of Computer Science, Smith College, Northampton, MA
      01063, USA.
      \protect\url{orourke@cs.smith.edu}.}
}

\begin{document}
\maketitle

\begin{abstract}
We establish a simple generalization of a known result in the plane.
The simplices in any pure simplicial complex in  may be
colored with  colors so that no two simplices that share a
-facet
have the same color.
In  this says that any planar map all of whose faces
are triangles may be 3-colored,
and in  it says that tetrahedra in a collection
may be ``solid 4-colored'' so that no two glued face-to-face receive
the same color.
\end{abstract}

\section{Introduction}
\seclab{Introduction}
The famous 4-color theorem says that the regions of any planar map
may be colored with four colors such that no two regions that share a
positive-length border receive the same color.
A lesser-known special case is that if all the regions are triangles,
three colors suffice.
For the purposes of generalization, this can be phrased as building a
planar
object by gluing triangles edge-to-edge, and then 3-coloring the
triangles.
Because the coloring constraint in this formulation
only applies to triangles adjacent the
dual
graph---whose nodes are triangles and whose arcs join triangle nodes
that share a whole edge---slightly more general objects can be 3-colored:
\emph{pure} (or \emph{homogenous})\footnote{
  Pure/homogenous means that there are no dangling edges or isolated
  vertices, and in general, no pieces of dimension less than  that
  are not part of a simplex of dimension .  So the complex is a
  collection of -simplices glued facet-to-facet.
}
\emph{simplicial complexes}
in , whose dual graph may have several components, with
independent colorings. 
See Figure~\figref{TriComplex}.
\begin{figure}[htbp]
\centering
\includegraphics[width=0.75\linewidth]{Figures/TriComplex}
\caption{A triangle complex and its dual graph .}
\figlab{TriComplex}
\end{figure}


For simplicity, we will call such a complex a \emph{triangle complex},
its analog in  a \emph{tetrahedron complex},
and the generalization a \emph{-simplex complex}.
We permit these complexes to contain an infinite number of simplices;
e.g., tilings of space by simplices are such complexes.
The main result of this note is:
\begin{theorem}
A -simplex complex may be -colored in the sense that each
simplex may be colored with one of  colors
so that any pair of simplices that share
a -facet receive different colors.
\thmlab{main}
\end{theorem}
One can think of the whole volume of each simplex being colored---so
``solid coloring'' of tetrahedra in .
Although I have not found this result in the literature,
it is likely known, as its proof is not difficult---essentially,
remove one simplex and induct.
Consequently, this
note should be considered expository,
and I will describe proofs in more detail than in a research announcement.
Perhaps more interesting than the result itself 
are the many related questions in Section~\secref{Beyond}.

\section{Triangle Complexes}
\seclab{TriangleComplexes}
Let  be the dual graph of a triangle complex, and
let  be the maximum degree of nodes of .
For triangle complexes, .
Let  be the chromatic number of .
An early result of Brooks~\cite{b-ocnn-41}
says that
 for any graph .
For duals of triangle complexes, this theorem only
yields , the 4-color theorem for triangle complexes.
We now proceed to establish  in three stages:
\begin{enumerate}
\item We first prove it for finite triangle complexes.
\item We then apply a powerful result of deBruijn and  Erd\H{o}s to
  extend
the result to infinite complexes.
\item We formulate a second proof for infinite complexes that does not
invoke deBruijn-Erd\H{o}s.
\end{enumerate}
The primary reason for offering two proofs is that related questions
raised
in Section~\secref{Beyond} may benefit from more than one proof
approach.

\subsection{Finite Triangle Complexes}
Let  be a triangle complex containing a finite number of triangles,
and  its dual graph.
Let  be the convex hull of , i.e., the boundary of the
smallest
convex polygon enclosing .
The proof is by induction on the number of triangles, with
the base case of one triangle trivial.
\begin{description}
\item[Case 1.] There is a triangle  with at least one edge  on
  .
Then  is \emph{exposed}
(i.e., not glued to another
triangle of the complex),
and  has at most degree 2 in .
Remove  to produce complex , 3-color  by induction,
put back , and color it with a color distinct from the colors of
its
at most 2 neighbors in .
\item[Case 2.] No triangle has an edge on .  Let  be any vertex
  of , and let  be the most counterclockwise (ccw) triangle incident
  to .
See Figure~\figref{TriangleHull}.
\begin{figure}[htbp]
\centering
\includegraphics[width=0.75\linewidth]{Figures/TriangleHull}
\caption{Triangle  has an exposed edge .}
\figlab{TriangleHull}
\end{figure}
Then the ccw edge  of  incident to  is exposed.
Then---just as in the previous case---remove
,
3-color by induction, put  back colored with a color not used by
its
at most two neighbors.
\end{description}
This simple induction argument establishes  for finite triangle complexes.

\subsection{deBruijn-Erd\H{o}s}

The result of deBruijn and  Erd\H{o}s is this~\cite{ed-cpigp-51}:
\begin{theorem}
If a graph  has the property that any finite subgraph is
-colorable,
then  is -colorable itself.
\thmlab{deB-E}
\end{theorem}


This immediately extends the result just proved to infinite triangle
complexes.
Note that the induction proof presented fails for infinite complexes, 
because it is possible that every
triangle
has degree 3 in  for infinite complexes,
for example, in a triangular tiling.

\subsection{Proof based on }
The alternative proof in some sense ``explains'' why a triangle complex
is
3-colorable: because it does not contain  as a subgraph.
Of course we could obtain this indirectly by using the above proof and
conclude that  could not be a subgraph (because it needs 4
colors),
but establishing it directly gives additional insight.

We rely here on this result,
obtained independently by several researchers
(Borodin and Kostochka, Catlin, and Lawrence, as reported in~\cite{s-nubcn-02}):
\begin{lemma}
If  does not contain any  as a subgraph,
,
then 

\lemlab{Lawrence}
\end{lemma}
We will now show that  is not a subgraph of  for triangle
complexes,
which, because  and , then implies

and so (because  is an integer), .

\begin{lemma}
.
\lemlab{notK4}
\end{lemma}
\begin{proof}
\emph{Sketch.}
We only sketch the argument, because in the Appendix we prove
more formally the extension to , including .

\begin{figure}[htbp]
\centering
\includegraphics[width=0.5\linewidth]{Figures/TriK3}
\caption{Triangles forming .}
\figlab{TriK3}
\end{figure}
If  is a subset of , then  must be as well.
The only configuration of triangles that realizes  is that shown
in Figure~\figref{TriK3}: the three triangles share and surround a vertex (labeled
1 in the figure).  Now consider attempting to
extend this to  by gluing another triangle to the only uncovered
edge of , edge .  Its apex, call it
,
must lie below , but because  lies above ,
the new triangle  cannot share the edges
 and , which it must to be adjacent to the other two
triangles.  Therefore,  cannot occur in , and we have established the claim.
\end{proof}

And as we argued above, Lemmas~\lemref{Lawrence} and~\lemref{notK4}
together
imply that : triangle complexes are 3-colorable.

\section{Tetrahedron Complexes}
Again we follow the same procedure as above, although we will defer
consideration of  to general -simplex complexes to the
Appendix, Section~\secref{Kd2}.
Now  is a finite tetrahedron complex,  its dual graph, and 
the convex hull of , the boundary of a convex polyhedron.
Again the proof is by induction.  Although we could repeat the
structure
of the proof for triangle complexes, we opt for an argument that more
easily
generalizes to  dimensions.

Let  be a vertex of the hull , and let  be the subset of
 of tetrahedra incident to .
Let  be the convex hull of .
If there is a tetrahedron  with at least one face  lying
on , then  has at most 3 neighbors in .  Remove ,
4-color the smaller complex , put  back, and color it with a
color not used for its at most 3 neighbors.
Note that it could well be that the face  lies on  because
 and  coincide at .  But having  on  is not the
crucial fact; if it is on , it is exposed, and induction then
applies.

If no tetrahedron in  has a face on , then there must be a
tetrahedron
 that has an edge  on  
(in fact, there must be at least three such tetrahedra).
See Figure~\figref{TetraHull}.
\begin{figure}[htbp]
\centering
\includegraphics[width=0.5\linewidth]{Figures/TetraHull}
\caption{No tetrahedron has a face on .}
\figlab{TetraHull}
\end{figure}


Let  be the subset of those tetrahedra in  that share .
Let  be the convex hull of these tetrahedra.
It must be that at least one tetrahedron has a face on .
The tetrahedra sharing  are angularly sorted about , and we can
select
the most ccw one (which might be the same as the most cw one if
).
So we have identified a tetrahedron with an exposed face, and
induction applies and establishes the result:
finite tetrahedon complexes have .
Infinite tetrahedon complexes follow from 
Theorem~\thmref{deB-E}.
And we could now work backward to conclude that  cannot contain
 as a subgraph.

\section{-Simplex Complexes}
We repeat the outline just employed.
The only difficult part is showing that in a finite -simplex
complex , there must be a simplex with an exposed facet.\footnote{
   We use \emph{facet} for a -dimensional face, and
   \emph{face} for any smaller dimensional face.
}
Then induction goes through just as before.

Say that a convex hull  of points in  dimensions is \emph{full-dimensional}
if  is not contained in a -dimensional flat (hyperplane)
for any .

Let  be a vertex of the hull , and let  be the subset of
 of simplices incident to .
Let  be the convex hull of ;
this is a -polytope that contains .
If there is a simplex  with at least one
-dimensional facet  
contained in , then  has at most  neighbors in ,
and induction establishes that  may be -colored.

So suppose that no simplex in  has a -dimensional facet on .
Let .
We must have , because otherwise  would bound a single
simplex,
and all of its facets would be on  and so exposed.
We know  is full dimensional because it contains -simplices.
Let  be a simplex that has a -dimensional face  in
, such that
 is maximal among all simplices with faces in .
We claim that there must be another simplex  that also has a face  in
, where .
For suppose otherwise, that is, suppose that all simplices in 
share .  Then, because  is full-dimensional,
one of these simplices  must have a vertex  not part of 
on 
(otherwise all simplices lie in the flat containing ).
But then  has a face (the hull of  and )
on  of dimension larger than ,
contradicting the choice of .

So  has a face on , and  does not share .
Let  be all the simplices in  that share
, and
let  be the convex hull of .
Because we know that ,
.

Now the argument is repeated:  is full-dimensional because it
includes at least one -simplex .
If some simplex in  has a -dimensional facet on
,
we have identified an exposed face.
Otherwise, we select some simplex  with a face  on ,
and separate out into  all the simplices sharing .
 must have at least one fewer simplex than does ,
following the same reasoning.

Continuing in this manner, we identify smaller and smaller subsets of
:

via repeated convex hulls ,
and eventually either identify a simplex with a -dimensional
facet on
the corresponding hull , or reach a set of one simplex, which
has all of its facets exposed.
So there is always a simplex with an exposed facet:

\begin{lemma}
Any finite -simplex complex contains a simplex with an 
exposed  -dimensional
facet.
\lemlab{exposed}
\end{lemma}
\noindent
Given the nearly obvious nature of
this lemma, it seems likely there is a less labored proof
that identifies an exposed simplex more directly.

This lemma then proves Theorem~\thmref{main} for finite complexes,
and deBruijn-Erd\H{o}s establishes it for infinite complexes.
Again we may now conclude that  cannot be a subgraph
of ,
where
we use the notation  for the dual graph of a -simplex
complex.
A geometric proof of this non-subgraph result is offered in the
Appendix.
With that, we obtain an alternative proof of  Theorem~\thmref{main},
which we restate in slightly different notation:
\begin{theorem}
The dual graph  of a -simplex complex in 
has chromatic number .
\end{theorem}
\begin{proof}
Lemma~\lemref{notKd2} tells us that  is not a subgraph of
,
with .
We have that  because each -simplex has  facets.
Therefore we have 

for .
Therefore
Lemma~\lemref{Lawrence} applies, and yields

Now we can see that 

by expanding  and :

Thus  is strictly less than , which, because  is an
integer,
implies .
\end{proof}



\section{Beyond Simplices}
\seclab{Beyond}
One can ask for analogs of Theorem~\thmref{main} for complexes composed
of shapes beyond simplices.
In the plane, a natural generalization is a complex built from
convex quadrilaterals glued edge-to-edge.
These complexes sometimes need four colors, as
the example in 
Figure~\figref{ConvexQuads} shows.
\begin{figure}[htbp]
\centering
\includegraphics[width=0.5\linewidth]{Figures/ConvexQuads}
\figlab{ConvexQuads}
\caption{A convex quadrilateral complex that needs four colors
\protect\cite[Fig.3a]{sw-rpt3c-00}.
}
\end{figure}
One does not need the 4-color theorem for this restricted class,
even without the convexity assumption:
there must exist a quadrilateral in a quadrilateral complex
with an exposed edge, and
4-coloring follows by induction.
Complexes built from pentagons 
can be proved
4-colorable by modifying the Kempe-chain argument;\footnote{
   I owe this observation to Sergey Norin,
   \url{http://mathoverflow.net/questions/49743/4-coloring-maps-of-pentagons}.
} 
so again
the full 4-color theorem is not needed here.



Sibley and Wagon proved in~\cite{sw-rpt3c-00}
the beautiful result: if the convex quadrilaterals are all parallelograms,
then three colors suffice
(essentially because there must be a parallelogram with two exposed edges).
In particular, Penrose rhomb tilings (their original interest) are 3-colorable.
Even more restrictive is requiring that the parallelograms be
rectangles.  Here with a student I proved  in~\cite{go-cobb-03} that such
\emph{rectangular brick} complexes of genus 0 are 2-colorable.
It is easily seen that complexes of genus 1 or greater might need
three colors (surround a hole with an odd cycle).


We also explored generalizations to  in~\cite{go-cobb-03}.
Somewhat surprisingly, genus-0 complexes built from \emph{orthogonal
bricks} (rectangular boxes in 3D) are again 2-colorable.
We also established that genus-1 orthogonal brick complexes are
3-colorable,
and conjectured that the same result holds for arbitrary genus.
I am aware of no substantive results on complexes built from
parallelopipeds (aside from the observation in~\cite{go-cobb-03} that
four colors are sometimes necessary), 
a natural generalization of the Sibley-Wagon result.\footnote{
  Our attempted proof in~\cite{go-cobb-03} for zonohedra is flawed.
}
One could also generalize convex quadrilaterals to convex hexahedra
(distorted cubes).
All of these generalizations seem unexplored.


\section{Appendix: }
\seclab{Kd2}
Here we establish that  without appeal
to deBruijn-Erd\H{o}s.
We partition the argument into four lemmas, the first three of which
show that there is essentially only one configuration that
achieves , the analog of the configuration in
Figure~\figref{TriK3}.
The fourth lemma then shows that  cannot be achieved.

Let , , and  be -simplices.
Suppose  and  share a -facet.
We will represent each simplex by the set of its vertex labels,
with distinct labels representing distinct points in .
When specifically referring to the point in space corresponding
to label , we'll use .
Let 
,
with  their shared -facet.
Under these circumstances, the following lemma holds:

\begin{lemma}
If  shares a -facet with  and 
a -facet
with  (and so the three simplices form  in the dual),
then the  vertices of  are among the  vertices
of :
 cannot include a vertex that is not a vertex of 
either  or .
\lemlab{K3.d2}
\end{lemma}
\begin{proof}
Suppose to the contrary that  includes a new vertex labeled .
For  to share a -facet with ,
it needs to match  of the  vertices of .
But it cannot match the facet  because
that is already covered by .
Without loss of generality, let us assume that  includes
vertex  but excludes vertex  with .
So the  vertices of  are

Now comparison to ,

shows that it is not possible for  to match  of the 
vertices of  (as it must to share a -facet):
the two only share  labels: 

This contradiction establishes the claim.
\end{proof}

\begin{lemma}
Suppose  -simplices are glued together so that their dual
graph is .
Then all the simplices together include only  vertices.
\lemlab{Kd1.d2}
\end{lemma}
\begin{proof}
Let  be the simplices. 
By Lemma~\lemref{K3.d2},  together include only 
vertices,
the  vertices of .
But then repeating the argument for  for each 
yields the same conclusion.
\end{proof}

\begin{figure}[htbp]
\centering
\includegraphics[width=0.75\linewidth]{Figures/TetraLemma}
\caption{Four tetrahedra whose dual forms .}
\figlab{TetraLemma}
\end{figure}



We continue to study the  configuration in the above
lemma.  Let us specialize to  to make the situation clear.
We have four tetrahedra glued together to form ,
and Lemma~\lemref{Kd1.d2} says they have altogether 5 vertices.
Because , only one of the possible combinations
of the labels  is missing among the four tetrahedra.
Without loss of generality, we can say that  is missing,
and that our four tetrahedra have these labels:






Our next claim is that  lies to the same
side of the plane determined by the face  as does .
Refer to Figure~\figref{TetraLemma}.

Let  be the plane containing the vertices with labels
, , and .
Let  be the open halfspace bound by 
and exterior to the tetrahedron ,
and  the analogous open halfspace including tetrahedron .
The claim is that . The other three tetrahedra can each be viewed as the hull of  and
one of the three faces of the  tetrahedron above
the base: , , and .
Because a tetrahedron can only be formed by a point above
each of these faces, we have that



So  must lie in the intersection of these three halfspaces,
which is a cone apexed at  that is strictly above
the base plane .  See again Figure~\figref{TetraLemma}.
And therefore ,
as claimed.

We now repeat this argument for -simplices, where
the logic is identical but is perhaps obscured by the notation.

The configuration of  -simplices forming  in
Lemma~\lemref{Kd1.d2} uses only  vertices.
Because , only one of the combinations
of  labels is missing, which we take to be 
 without loss of generality.
So the labels of the  simplices are:






\begin{lemma}
In the configuration of  simplices forming  labeled as
just detailed above,  lies in ,
the same halfspace in which  lies.
\lemlab{Kd1}
\end{lemma}
\begin{proof}
 is the flat containing the ``base'' of the first simplex in the
list above, . The remaining  simplicies in the list share the facets of 
incident to , each including .
Thus  is above each of those facets, i.e., it lies in the
corresponding  halfspaces:



And therefore  lies in the intersection of all these halfspaces,
which is a cone apexed at  and lying strictly above .
Therefore  is in .
\end{proof}

Completing the argument is now straightforward.

\begin{lemma}

\lemlab{notKd2}
\end{lemma}
\begin{proof}
Assume to the contrary that  is a subgraph of .
Then  must be also.
Using the notation of Lemma~\lemref{Kd1},
that lemma establishes that in a configuration that
realizes , vertex  lies in 
.
Because  is the only facet of
the simplex   not yet covered by
another simplex, the last simplex  must have labels
.
And therefore .
But this is a contradiction, as it is saying that  must lie
strictly to both sides of .
\end{proof}






\bibliographystyle{alpha}
\bibliography{/Users/orourke/bib/geom/geom}
\end{document}
