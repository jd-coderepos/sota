\documentclass[oneside,10pt]{article}


\topmargin -0.5in
\textheight 9 true in       \textwidth 6.25 true in
\oddsidemargin .0in    \evensidemargin .0in

\usepackage{times,url,mathrsfs}
\usepackage{amsmath}
\usepackage{amsfonts}
\usepackage{graphicx}
\usepackage{subfigure}
\newtheorem{lemma}{Lemma}
\newtheorem{proof}{Proof}


\newcommand{\dataset}{{\cal D}}
\newcommand{\fracpartial}[2]{\frac{\partial #1}{\partial  #2}}



\begin{document}

\title{The Optimal Quantile Estimator for Compressed Counting}


\author{ Ping Li \hspace{0.01in} (pingli@cornell.edu) \hspace{0.2in} \\
       Cornell University,  Ithaca, NY 14853
  }
\date{}
\maketitle


\begin{abstract}
Compressed Counting (CC)\footnote{The results  were initially drafted in Jan 2008, as part of a report for private communications with several theorists. That report was later filed to arXiv\cite{Article:Li_CC_v0}, which, for shortening the presentation, excluded the content of the optimal quantile estimator. }
was recently proposed for very efficiently computing the (approximate) th frequency moments of data streams, where . Several estimators were reported including the {\em geometric mean} estimator, the {\em harmonic mean} estimator, the {\em optimal power} estimator, etc. The {\em geometric mean} estimator is particularly interesting for theoretical purposes. For example, when , the complexity of CC (using the {\em geometric mean} estimator) is , breaking the well-known large-deviation bound . The case  has  important applications, for example, computing entropy of data streams.

For practical purposes, this study proposes the {\em optimal quantile} estimator. Compared with previous estimators, this estimator is computationally more efficient and is also more accurate when .

\end{abstract}


\section{Introduction}

\textbf{\em Compressed Counting (CC)}\cite{Article:Li_CC,Article:Li_CC_v0} was very recently proposed for efficiently computing the th frequency moments, where , in data streams. The underlying technique of CC is {\em maximally skewed stable random projections}, which significantly improves the well-know algorithm based on {\em symmetric stable random projections}\cite{Article:Indyk_JACM06,Proc:Li_SODA08}, especially when . CC boils down to a statistical estimation problem and various estimators have been  proposed\cite{Article:Li_CC,Article:Li_CC_v0}. In this study, we present an estimator based on the {\em optimal quantiles}, which is computationally more efficient and significantly more accurate when , as long as the sample size is not too small.


One direct application of CC is to estimate entropy of data streams. A recent trend is to approximate entropy using frequency moments and estimate frequency moments using {\em symmetric stable random projections}\cite{Proc:Zhao_IMC07,Proc:Harvey_FOCS08}.  \cite{Report:Li_CC_entropy} applied CC to estimate entropy and demonstrated huge improvement (e.g., 50-fold) over previous studies.\\



CC was recently presented at {\em  MMDS 2008: Workshop on Algorithms for Modern Massive Data Sets}. Slides are available at \url{http://www.stanford.edu/group/mmds/slides2008/li.pdf}.


\subsection{The Relaxed Strict Turnstile Data Stream Model }

Compressed Counting (CC) assumes a {\em relaxed strict Turnstile} data stream model. In the {\em Turnstile} model\cite{Article:Muthukrishnan_05}, the input  stream ,  arriving sequentially describes the underlying signal , meaning
 where the increment  can be either positive (insertion) or negative (deletion). Restricting   at all    results in the {\em strict Turnstile} model, which suffices for describing most natural phenomena. CC constrains  only at the  we care about; however, when at , CC allows  to be arbitrary.

Under the {\em relaxed strict Turnstile} model, the th frequency moment of a data stream  is defined as

When , it is obvious that one can compute  trivially, using a simple counter.  When , however, computing  exactly requires  counters.

\subsection{Maximally-skewed Stable Random Projections}

Based on {\em maximally skewed stable random projections}), CC provides an very efficient mechanism for approximating .  One first generates a random matrix , whose entries are i.i.d. samples of a -skewed -stable distribution with scale parameter 1, denoted by .

By property of stable distributions\cite{Book:Zolotarev_86,Book:Samorodnitsky_94}, entries of the resultant projected vector  are i.i.d. samples of a -skewed -stable distribution whose scale parameter is the  frequency moment of  we are after:


The skewness parameter . CC recommends  , i.e., maximally-skewed, for the best performance.

In real implementation, the linear projection  is conducted {\em incrementally}, using the fact that the {\em Turnstile} model is also linear. That is, for every incoming , we update  for  to .   This procedure is similar to that of {\em symmetric stable random projections}\cite{Article:Indyk_JACM06,Proc:Li_SODA08}; the difference is the distribution of the elements in .

\section{The Statistical Estimation Problem and Previous Estimators}

CC boils down to a statistical estimation problem. Given  i.i.d. samples, , estimate the scale parameter .

Assume  i.i.d. samples . Various estimators were proposed in \cite{Article:Li_CC,Article:Li_CC_v0}, including the {\em geometric mean} estimator, the {\em harmonic mean} estimator, the {\em maximum likelihood} estimator, the {\em optimal quantile} estimator. Figure \ref{fig_comp_var_factor} compares their asymptotic variances along with the asymptotic variance of the {\em geometric mean} estimator for {\em symmetric stable random projections}\cite{Proc:Li_SODA08}.



\begin{figure}[h]
\begin{center}
\includegraphics[width = 3.5 in]{comp_var_factor.eps}
\end{center}
\vspace{-0.3in}
\caption{Let  be an estimator of  with asymptotic variance . We plot the  values for the {\em geometric mean} estimator,  the {\em harmonic mean} estimator (for ), the {\em optimal power} estimator (the lower dashed curve), and the {\em optimal quantile} estimator, along with the  values for the {\em geometric mean} estimator for {\em symmetric stable random projections} in \cite{Proc:Li_SODA08} (``symmetric GM'', the upper dashed curve). When , CC achieves an ``infinite improvement'' in terms of the asymptotic variances.
}\label{fig_comp_var_factor}
\end{figure}


\subsection{The geometric mean estimator, , for ,  ()}
 is unbiased and has exponential tail bounds for all .

\subsubsection{The harmonic estimator, , for }
 has exponential tail bounds.

\subsection{The maximum likelihood estimator, , for  only}
 has exponential tail bounds.

\subsection{The optimal power estimator, , for , ()} 


When ,   and  has exponential tail bounds.

 becomes the {\em harmonic mean} estimator when , the {\em arithmetic mean} estimator when , and the {\em maximum likelihood} estimator when .

\section{The Optimal Quantile Estimator}

Because  belongs to the location-scale family (location is zero always), one can estimate the scale parameter  simply from the sample qantiles.

\subsection{A General Quantile Estimator}
Assume ,  to .  One possibility is to use the -quantile of the absolute values, i.e.,

where


Denote , where . Note that when , . Denote the probability density function of  by , the probability cumulative function by , and the inverse cumulative function by .

We can analyze the asymptotic (as ) variance of , presented in Lemma \ref{lem_q_var}.
\begin{lemma}\label{lem_q_var}

\textbf{Proof:} \ \ The proof directly follows from known statistical results on sample quantiles, e.g., \cite[Theorem
9.2]{Book:David}, and the ``delta'' method.

\noindent using the fact that

\end{lemma}


We can choose  to minimize the asymptotic variance factor,
, which is apparently a convex function of , although there appears no simple algebraic method to prove it (except when ).

We denote the optimal quantile estimator as .

\subsection{The Optimal Quantiles}

The optimal quantiles, denoted by , has to be determined by numerical procedures, using the simulated probability density functions for stable distributions. We used the \textbf{fBasics} package in \textbf{R}. We, however, found those functions had numerical problems when  and .

For all other estimators, we have not noticed any numerical issues even when  or . Therefore, we do not consider there is any numerical instability for CC, as far as the method itself is concerned.



Table \ref{tab_oq} presents the numerical results, including ,  , and the variance of  (without the  term). The variance factor is also plotted in Figure \ref{fig_comp_var_factor}, indicating significant improvement over the geometric mean estimator when .

\begin{table}[h]
\caption{\small
 }
\begin{center}{\scriptsize
\begin{tabular}{l l l l}
\hline \hline
 &  &Var  & \\\hline
0.20 &0.180& 1.39003806& 0.05561700\\
0.30 &0.167& 1.21559359& 0.11484008\\
0.40 &0.151& 1.00047427& 0.2720723\\
0.50 &0.137& 0.76653704& 0.4522449\\
0.60 &0.127& 0.53479789& 0.7406894\\
0.70 &0.116& 0.32478420& 1.231919\\
0.80 &0.108& 0.15465894& 2.256365\\
0.85 &0.104& 0.08982992& 3.296870\\
0.90 &0.101& 0.04116676& 5.400842\\
0.95 &0.098& 0.01059831 &1.174773\\
0.96 &0.097& 0.006821834 & 14.92508\\
0.97 &0.096& 0.003859153 &20.22440\\
0.98   &0.0944& 0.001724739&    30.82616\\
0.989 & 0.0941& 0.0005243589& 56.86694\\
1.011 & 0.8904& 0.0005554749& 58.83961\\
1.02 & 0.8799& 0.001901498&  32.76892\\
1.03  & 0.869& 0.004424189& 22.13097\\
1.04 &0.861& 0.008099329& 16.80970\\
1.05 & 0.855& 0.01298757& 13.61799\\
1.10 &0.827& 0.05717725& 7.206345\\
1.15 &0.810& 0.1365222& 5.070801\\
1.20 &0.799& 0.2516604& 4.011459\\
1.30 &0.784& 0.5808422& 2.962799\\
1.40 &0.779& 1.0133272& 2.468643\\
1.50 &0.778& 1.502868& 2.191925\\
1.60 &0.785 &1.997239& 2.048035\\
1.70 & 0.794 &2.444836& 1.968536\\
1.80 &0.806 &2.798748& 1.937256\\
1.90 &0.828 &3.019045& 1.976624\\
2.00 &0.862 &3.066164& 2.097626\\
\hline\hline
\end{tabular}
}
\end{center}
\label{tab_oq}
\end{table}


\subsection{Comments on the Optimal Quantile Estimator}

The optimal quantile estimator has at least two advantages:
\begin{itemize}
\item When the sample size  is not too small (e.g., ),  is more accurate then , especially for .
\item  is computationally more efficient.
\end{itemize}

The disadvantages are:
\begin{itemize}
\item For small samples (e.g., ),  exhibits bad behaviors when .
\item Its theoretical analysis, e.g., variances and tail bounds, is based on the density function of skewed stable distributions, which do not have closed-forms. The tail bound bounds can be obtained similarly using the method developed in \cite{Article:Li_arXiv_sym_oq}.
\item The important parameters,  and , are obtained from the numerically-computed density functions. Due to the numerical difficulty in those functions, we can only obtain  and  values for  and .
\end{itemize}

\section{Conclusion}

Compressed Counting (CC) dramatically improves {\em symmetric stable random projections}, especially when , and has important applications in data streams computations such as entropy estimation.

CC boils down to a statistical estimation problem. We propose the optimal quantile estimator, which considerably improves the previously proposed geometric mean estimator when , at least asymptotically. For practical purposes, this estimator should be very useful. However, for theoretical purposes,  it can not replace the geometric mean estimator.



\appendix
{\small
\begin{thebibliography}{10}

\bibitem{Book:David}
Herbert~A. David.
\newblock {\em Order Statistics}.
\newblock John Wiley \& Sons, Inc., New York, NY, second edition, 1981.

\bibitem{Proc:Harvey_FOCS08}
Nicholas J.~A. Harvey, Jelani Nelson, and Krzysztof Onak.
\newblock Sketching and streaming entropy via approximation theory.
\newblock In {\em FOCS}, 2008.

\bibitem{Article:Indyk_JACM06}
Piotr Indyk.
\newblock Stable distributions, pseudorandom generators, embeddings, and data
  stream computation.
\newblock {\em Journal of ACM}, 53(3):307--323, 2006.

\bibitem{Article:Li_CC}
Ping Li.
\newblock Compressed counting.
\newblock {\em CoRR}, abs/0802.2305, 2008.

\bibitem{Article:Li_arXiv_sym_oq}
Ping Li.
\newblock Computationally efficient estimators for dimension reductions using
  stable random projections.
\newblock {\em CoRR}, abs/0806.4422, 2008.

\bibitem{Proc:Li_SODA08}
Ping Li.
\newblock Estimators and tail bounds for dimension reduction in 
  () using stable random projections.
\newblock In {\em SODA}, pages 10 -- 19, 2008.

\bibitem{Article:Li_CC_v0}
Ping Li.
\newblock On approximating frequency moments of data streams with skewed
  projections.
\newblock {\em CoRR}, abs/0802.0802, 2008.

\bibitem{Report:Li_CC_entropy}
Ping Li.
\newblock A very efficient scheme for estimating entropy of data streams using
  compressed counting.
\newblock Technical report, Department of Statistical Science, Cornell
  University, 2008.

\bibitem{Article:Muthukrishnan_05}
S.~Muthukrishnan.
\newblock Data streams: Algorithms and applications.
\newblock {\em Foundations and Trends in Theoretical Computer Science},
  1:117--236, 2 2005.

\bibitem{Book:Samorodnitsky_94}
Gennady Samorodnitsky and Murad~S. Taqqu.
\newblock {\em Stable Non-Gaussian Random Processes}.
\newblock Chapman \& Hall, New York, 1994.

\bibitem{Proc:Zhao_IMC07}
Haiquan Zhao, Ashwin Lall, Mitsunori Ogihara, Oliver Spatscheck, Jia Wang, and
  Jun Xu.
\newblock A data streaming algorithm for estimating entropies of od flows.
\newblock In {\em IMC}, San Diego, CA, 2007.

\bibitem{Book:Zolotarev_86}
Vladimir~M. Zolotarev.
\newblock {\em One-dimensional Stable Distributions}.
\newblock American Mathematical Society, Providence, RI, 1986.

\end{thebibliography}

}



\end{document}
