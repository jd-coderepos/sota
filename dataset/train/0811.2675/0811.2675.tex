\documentclass[secthm]{elsart}

\usepackage{graphicx}
\usepackage{amsmath, amssymb}
\usepackage{mathrsfs}
\usepackage{color}
\usepackage{colortbl}



\usepackage{ifpdf}
\ifpdf
\usepackage[pdftitle={Instructions for use of the document class
    elsart},pdfauthor={Simon Pepping},pdfsubject={The preprint document class elsart},pdfkeywords={instructions for use, elsart, document class},pdfstartview=FitH,bookmarks=true,bookmarksopen=true,breaklinks=true,colorlinks=true,linkcolor=blue,anchorcolor=blue,citecolor=blue,filecolor=blue,menucolor=blue,pagecolor=blue,urlcolor=blue]{hyperref}
\else
\usepackage[breaklinks=true,colorlinks=true,linkcolor=blue,anchorcolor=blue,citecolor=blue,filecolor=blue,menucolor=blue,pagecolor=blue,urlcolor=blue]{hyperref}
\fi



\newcommand{\C}{\circ}
\newcommand{\To}{\longrightarrow}
\newcommand{\Map}[3]{#1\, :\, #2\To #3}
\newcommand{\A}{\alpha}
\newcommand{\B}{\beta}
\newcommand{\LL}{\mathcal{L}}
\newcommand{\J}{\mathcal{J}}
\newcommand{\D}{\mathcal{D}}
\newcommand{\R}{\mathcal{R}}
\newcommand{\HH}{\mathcal{H}}
\newcommand{\lcm}{\textrm{lcm}}
\newcommand{\Real}{\mathbb{R}}
\newcommand{\Rat}{\mathbb{Q}}
\newcommand{\Nat}{\mathbb{N}}
\newcommand{\Int}{\mathbb{Z}}
\newcommand{\Comp}{\mathbb{C}}
\newcommand{\abs}[1]{\left\vert#1\right\vert}
\newcommand{\set}[1]{\left\{#1\right\}}
\newcommand{\Set}[2]{\set{#1\ \vert\ #2}}
\renewcommand{\atop}[2]{\genfrac{}{}{0pt}{}{#1}{#2}}
\newtheorem{obs}[thm]{Observation}
\newtheorem{algo}[thm]{Algorithm}
\definecolor{darkmagenta}{rgb}{0.5,0,0.5}



\begin{document}

\pagestyle{headings}

\begin{frontmatter}

\title{Characterizations of probe interval graphs}

\author{Shamik Ghosh\corauthref{cor}}
\corauth[cor]{Corresponding author.}
\ead{sghosh@math.jdvu.ac.in},
\author{Maitry Podder}
\ead{maitry\_podder@yahoo.co.in}

\address{Department of Mathematics, Jadavpur University, Kolkata - 700 032, India.}

\author{Malay K. Sen}
\ead{senmalay@hotmail.com}

\address{Department of Mathematics, North Bengal University,
   Darjeeling, West Bengal, India, Pin - 734 430.}



\begin{abstract}
In this paper we obtain several characterizations of the adjacency matrix of a probe interval graph. In course of this study we describe an easy method of obtaining interval representation of an interval bipartite graph from its adjacency matrix. Finally, we note that if we add a loop at every probe vertex of a probe interval graph, then the Ferrers dimension of the corresponding symmetric bipartite graph is at most . 
\end{abstract}

\begin{keyword}
Interval graph, Interval bipartite graph, Probe interval graph, Ferrers bigraph, Ferrers dimension, Adjacency matrix of a graph.
\end{keyword}

\end{frontmatter}

\section{Introduction}

An undirected graph  is an {\em{interval graph}} if it is the intersection graph of a family of intervals on the real line in which each vertex is assigned an interval and two vertices are adjacent if and only if their corresponding intervals intersect. The study of interval graphs was spearheaded by Benzer \cite{B} in course of his studies of the topology of the fine structure of genes. Since then interval graphs and their various generalizations were studied thoroughly. Also advances in the field of molecular biology, and genetics in particular, solicited the need for a new model. In \cite{Z}, Zhang introduced another generalization of interval graphs called probe interval graphs, in an attempt to aid a problem called cosmid contig mapping, a particular component of the physical mapping of DNA. A {\em{probe interval graph}} is an undirected graph  in which the set of vertices  can be partitioned into two subsets  and  (called probes and nonprobes respectively) and there is an interval (on the real line) corresponding to each vertex such that vertices are adjacent if and only if their corresponding intervals intersect and at least one of the vertices belongs to . Now several research works are continuing on this topic and some special classes of it \cite{BL,CK,GL,JS,MWZ,LS}. In fact, Golumbic and Trenk have devoted an entire chapter on probe interval graphs in their recent book \cite{GT}. Moreover, motivated by the definition of probe interval graphs, genrally, the concept of probe graph classes has been introduced. Given a class of graphs , a graph  is a {\em probe graph of}  if its vertices can be partitioned into a set  of probes and an independent set  of nonprobes such that  can be extended to a graph of  by adding edges between certain nonprobes. In this way, many more probe graph classes have been defined and widely investigated, eg., probe split graphs, probe chordal graphs, probe tolerance graphs, probe threshold graphs and others \cite{BGL,CKKLP,RB,RLB}.

Among all such studies nothing has been said about the nature of adjacency matrices of probe interval graphs until now. In this paper we will present three characterizations of the adjacency matrix of a probe interval graph. The first one is in this section and two others are in section 3. In section 2, we describe an easy method of obtaining interval representation of an interval bipartite graph from its adjacency matrix. Moreover, we prove that if we add a loop at every probe vertex of a probe interval graph, then the Ferrers dimension of the corresponding symmetric bipartite graph is at most .

We first note that any interval graph  is a probe interval graph with probes  and nonprobes , where  is any independent set (possibly singleton) of  and . Certainly the converse is false, as  is a probe interval graph which is not an interval graph. So probe interval graphs generalize the class of interval graphs. Further generalizations lead to the following concepts. An undirected graph  is an {\em{interval split graph}} \cite{Br} if the set of vertices  are partitioned into two subsets  and  such that the subgraph induced by  is an interval graph and  is an independent set. Every probe interval graph is an interval split graph, as  is an independent set and the subgraph induced by  is an interval graph. Again interval bipartite graphs (cf.~\S 2) are generalized to interval -graphs. An undirected graph with a proper coloring by  colors is an {\em{interval -graph}} \cite{Br} if each vertex corresponds to an interval on the real line so that two vertices are adjacent if and only if their corresponding intervals intersect and they are of different colors. Brown \cite{Br} showed that any -chromatic probe interval graph is an interval -graph. Also, since interval -graphs are weakly chordal\footnote{An undirected graph  is {\em weakly chordal} if neither  nor its complement  contains an induced cycle of length greater than .} and hence perfect,\footnote{An undirected graph  is {\em perfect} if for every induced subgraph  of , the chromatic number of  is equal to its maximal clique size.} we have that probe interval graphs are also weakly chordal and perfect. While comparing two graphs discussed earlier, Brown \cite{Br} made the comment:
``there are interval split graphs which are not interval -graphs (e.g.,  or any cycle of length greater than or equal to ). The converse is not known -- but has not received much attention.'' The following example shows that there are interval -graphs which are not interval split graphs.

\begin{exmp} {\em Consider the graph , which is an interval -graph. But it is not an interval split graph, since it has only  independent sets, namely,  and . For each such choice, the other  vertices induce the subgraph  which is not an interval graph.}

\begin{figure}[h]
\begin{center}
\includegraphics*[scale=0.4]{abs6.eps}
\end{center}
\end{figure}
\end{exmp}

Now the class of probe interval graphs lies in the intersection of the class of interval split graphs and the class of interval -graphs but there are examples (Brown presented one such in \cite{Br}) which are both interval split graphs and interval -graphs but not probe interval graphs. Thus the following is an interesting open problem to study.

\begin{prob}
Characterize the class of graphs which are both interval split graphs and interval -graphs.
\end{prob}

Regarding forbidden subgraph characterizations, Brown \cite{Br} showed that interval -graphs and hence probe interval graphs are ATE-free\footnote{An {\em asteroidal triple of edges} (ATE) in an undirected graph  is a set of three edges such that for any two there exists a path in  that avoids the neighborhood of the third edge.}, while Sheng \cite{LS} characterized cycle free probe interval graphs in terms of six forbidden subgraphs (trees). Among the other characterizations, Brown \cite{Br} and Zhang \cite{Z} generalized the well known \cite{GH,G} result that an undirected graph is an interval graph if and only if its maximal cliques are consecutively ordered\footnote{A set of distinct induced subgraphs  of a graph  is {\em consecutively ordered} when for each , if  and , then .}. Brown \cite{Br} proved that if  is an undirected graph with an independent set , then  is a probe interval graph with probes  and nonprobes  if and only if  has an edge cover of quasi-cliques\footnote{A {\em quasi-clique}  in a probe interval graph  is a set of vertices with all vertices of  are adjacent to each other and any vertex of  is adjacent to all vertices of .} that can be consecutively ordered. 

In the following we shall present the first characterization of adjacency matrices of probe interval graphs (cf. Observation \ref{o:char1}), which is simple and immediate. Let  be a simple undirected graph. If we replace\footnote{This replacement is equivalent to add a loop at each vertex of .} all principal diagonal elements\footnote{which are  in the adjacency matrix of .} of the adjacency matrix of  by , then this matrix is known as the {\em \label{'augmented'} augmented adjacency matrix} of . Let  be a symmetric  matrix with 's in the principal diagonal. Then  is said to satisfy the {\em quasi-linear ones property} if 's are consecutive right to and below the principal diagonal. It is known \cite{MR} that  is an interval graph if and only if rows and columns of the augmented adjacency matrix of  can be suitably permuted (using the same permutation for rows and columns) so that it satisfies the quasi-linear ones property.

\begin{defn} 
{\em Let  be a symmetric -matrix with 's in the principal diagonal. Suppose  contains a principal submatrix\footnote{We call a square submatrix  of  {\em{principal}} if the principal diagonal elements of  are also principal diagonal elements of .}  which is an identity matrix. Denote all the zeros of  by . Then  is said to satisfy the {\em{quasi-x-linear ones property}} if every  right to the principal diagonal has only  and  to its right and every  below the principal diagonal has only  and  below it.}  
\end{defn}

\vspace{1em} Now from the definition of a probe interval graph  it follows that the graph obtained by adding edges to  between the pairs of nonprobes whose intervals intersect is an interval graph with the same assignment of intervals to the vertices as in . Conversely, let  be an interval graph and . Then the graph obtained by removing all the edges between any two vertices belonging to  from  is a probe interval graph with probes  and nonprobes . Thus for an undirected graph  with an independent set , if adding edges between some vertices of  make it an interval graph, then the graph  must be a probe interval graph with probes  and nonprobes . This simple observation leads to the following characterization of probe interval graphs:

\vspace{2em}\begin{obs}\label{o:char1}
Let  be an undirected graph with an independent set . Let  be the augmented adjacency matrix of . Then  is a probe interval graph with probes  and nonprobes  if and only if rows and columns of  can be suitably permuted (using the same permutation for rows and columns) in such a way that it satisfies the quasi-x-linear ones property. 
\end{obs}

\vspace{-1.5em} \section{Interval representations of interval bipartite graphs}

\vspace{-1.25em} An {\em{interval bipartite graph}} (in short, {\em interval bigraph}) is a bipartite graph  with bipartation , representable by assigning each vertex  an interval  (on the real line) so that two vertices  and  are adjacent if and only if  \cite{HKM}. Since  and  are independent sets in , here we only consider the submatrix of the adjacency matrix of  consisting of the rows corresponding to one partite set and the columns corresponding to the other. This submatrix is known as the {\em{biadjacency matrix}} of . A bipartite graph  is an interval bigraph if and only if the rows and columns of the biadjacency matrix of  can be (independently) permuted so that each  can be replaced by  or  in such a way that every  has only 's to its right and every  has only 's below it. Such a partition of zeros in the biadjacency matrix of  is called an {\em{R-C partition}} of it \cite{SDRW}. Again a -matrix  has the {\em generalized linear ones property} if it has a stair partition\footnote{A {\em stair partition} of a matrix is a partition of its positions into two sets  by a polygonal path from the upper left to the lower right, such that the set  [] is closed under leftward or downward [respectively, rightward or upward] movement \cite{SDW}.}  such that the 's in  are consecutive and appear leftmost in each row, and the 's in  are consecutive and appear topmost in each column. For the biadjacency matrix  of a bipartite graph  this property is equivalent to having an R-C partition, i.e.,  is an interval bigraph if and only if the rows and columns of  can be (independently) permuted so that the resulting matrix has the generalized linear ones property \cite{SDW}. Now there are many methods \cite{Mu,SDRW,W} of obtaining interval representation of an interval bigraph when the R-C partition of its biadjacency matrix is given. We present here another one for further use. 

\begin{defn}
{\em Let  be an interval bigraph with the biadjacency matrix , which is in R-C partition form. We insert some rows and columns in , each of which has all the entries  except the principal diagonal element which is  such that  is enhanced to a square matrix in which each  is right to the principal diagonal and each  is below the principal diagonal. Now replace each  right to  by  and each  below  by  and the rest by . This matrix, say,  is called a {\em diagonalized} form of  and the above process of obtaining  from  will be called a {\em{diagonalization}} of . We denote the bigraph whose biadjacency matrix is  by }\footnote{Note that  is also an interval bigraph, as  is still in R-C partition form and  is an induced subgraph of .}.
\end{defn}

An easy method of diagonalization is as follows. In the stair partition of , if a step, parallel to rows [columns], is lying through  columns [respectively, rows], then insert  rows [respectively, columns] (as described previously) just above [respectively, after] the step. Accordingly we get a diagonalized matrix  whose number of rows (as well as columns) is the sum of number of rows and columns of . For practical purpose the number of insertions of rows and columns can be reduced as the following example shows.

\begin{exmp}\label{exmp:diag}
{\em Consider the following biadjacency matrix  of an interval bigraph:}

\vspace{-1.5em}

{\em A diagonalization of  is given by} 

\end{exmp}

\vspace{-1em}Now we present an algorithm to obtain an interval representation of an interval bigraph .

\begin{algo} \label{alg:big1}
{\em {\small\bf Input:}\ Diagonalized matrix  (of order  (say)), where  is the biadjacency matrix (in R-C partition form) of an interval bigraph .\\
{\small\bf Step I:}\ For each , define  and , where in the  row the last  appears in the  column on or after the principal diagonal of .\\
{\small\bf Step II:}\ For each , define  and , where in the  column the last  appears in the  row on or after the principal diagonal  \mbox{of }.\\
{\small\bf Output:}\ The closed intervals  and , which are corresponding to the  row and  column of  respectively.}
\end{algo}

Using the above algorithm in the case of the interval bigraph considered in Example \ref{exmp:diag}, we have

Finally removing newly inserted rows and columns we get


\begin{prop}
Algorithm \ref{alg:big1} provides an interval representation of an interval bigraph .
\end{prop} 

\vspace{-1.5em}\begin{pf*}{Proof.}
Let  be an interval bigraph with biadjacency matrix  is in R-C partition form. Let us denote the vertex corresponding to the  row [ column] of  by  [respectively, ]. Now suppose \footnote{For convenience, an entry of a matrix corresponding to, say, the vertex  in the row and the vertex  in the column will be denoted by, simply, .}. If , then by Algorithm \ref{alg:big1},  and so . Thus  contains  and hence it is nonempty. Again if , then by Algorithm \ref{alg:big1}, . So  which implies  as it contains . Next let . Since  is diagonalized, . But then by Algorithm \ref{alg:big1},  and so , i.e., . Similarly, if , then  and by Algorithm \ref{alg:big1}, it follows that , i.e., . Therefore Algorithm \ref{alg:big1} provides an interval representation of  and hence of , as  is an induced subgraph of . \hfill 
\end{pf*}

\section{Probe interval graphs}

Let  be an undirected graph with an independent set . Let . We construct a bipartite graph  with the partite sets  and  and two vertices  and  are adjacent in  if and only if either  (in ) or  (i.e.,  and  are adjacent in ). That is,  is a bipartite graph whose biadjacency matrix is the submatrix  of the augmented adjacency matrix (cf.~page \pageref{'augmented'}) of  consisting of all the columns, but only the rows corresponding to all the vertices of . Henceforth we refer this graph as .

We note that if  is a probe interval graph with probes  and nonprobes , then the bipartite graph  is necessarily an interval bigraph by the same assignment of intervals to the vertices as in . But the following example shows that the above necessary condition is not sufficient.

\begin{exmp}\label{exmp:at}
{\em Consider the following graph, say, .  is not\footnote{Note that  is a probe interval graph with probes  and nonprobes .} a probe interval graph with probes  and nonprobes  as neither  nor the graph  (the graph obtained by joining the edge  to ) is an interval graph.} 

\vspace{1em}
\begin{figure}[h]
\begin{center}
\includegraphics*[scale=0.45]{abs7.eps}
\end{center}
\end{figure}

\vspace{1em}{\em But the biadjacency matrix of the bipartite graph  has an R-C partition showing that  is an interval bigraph.}

\end{exmp}

\begin{thm}\label{t:char1}
An undirected graph  with an independent set  is a probe interval graph with probes  and nonprobes  if and only if 
\begin{enumerate}
\item[{\bf (1)}] the bigraph  is an interval bigraph and
\item[{\bf (2)}] there exists an R-C partition of the biadjacency matrix of  which does not contain the following submatrix for any  and :\\label{eq:cfg}
\begin{array}{cc|cccccc}
\multicolumn{3}{c}{} & p && q && n\\ \cline{3-8}
p &&& 1 && 1 && R\\
q &&& 1 && 1 && C
\end{array}
\begin{array}{cc|cccc}
\multicolumn{3}{c}{} & q && p \\ \cline{3-6}
p &&&  && 1 \\
q &&& 1 &&  
\end{array}\begin{array}{cc|cccc}
\multicolumn{3}{c}{} & q && p \\ \cline{3-6}
p &&& 1 && 1 \\
q &&& 1 && 1 
\end{array}\begin{array}{cc|cccc}
\multicolumn{3}{c}{} & r && p \\ \cline{3-6}
r &&& 1 && R \\
p &&& C && 1 \\
q &&& 1 && 1
\end{array}\label{eq:rj}
r_j=\left\{\begin{array}{l}
\min \Set{a_i}{p_in_j=C} - 1\\
\infty,\ \ \textrm{ if there is no  in the column of \footnotemark .} 
\end{array}\right .
\label{eq:lj}
l_j=\left\{\begin{array}{l}
\max \Set{b_i}{p_in_j=R} + 1\\
0,\ \ \textrm{ if there is no  in the column of .}
\end{array}\right .
\begin{array}{cc|cccccc}
\multicolumn{3}{c}{} & p_i && p_j && n_k\\ \cline{3-8}
p_i &&& 1 && 0 && R\\
p_j &&& 0 && 1 && C
\end{array}\max \Set{b_i}{p_in_k=R}\ <\ \min \Set{a_j}{p_jn_k=C}.\label{eq:kit}
p_tp_i=C\ \textrm{ whenever }\ p_tp_k=C.
\left(\begin{array}{cc}
1 & 0\-0.25em]
1 & 0
\end{array}
\right).-1em]

Now if this assignment of colors provides a -coloring of the vertices of , then we have nothing to prove. If not, then there exist couples of the forms:
-0.25em]
R & 1
\end{array}
\right) \hspace{0.5in}
\textrm{ or } \hspace{0.5in}\left(\begin{array}{cc}
1 & C\
in the biadjacency matrix of  where none of the zeros ( or ) belongs to the submatrix . Also since vertices of  is properly -colored (by  or ), at least one of the two rows of these couples must corresponds to a nonprobe (i.e., these couples cannot lie fully in the submatrix ). So the following three cases may arise for couples of the first type (containing 's):

where  and . The first case implies the existence of

in the biadjacency matrix, say,  of  which is not possible. The second one again forces the following in :

where  or . Clearly . Suppose . But then we have  and consequently the couple 

in , which is a contradiction. So finally we consider the last one. In this case we get the submatrix

in  which is of the form (\ref{eq:cfg}) and so is forbidden as we mentioned at the beginning of the proof. The proof for the couples of other type (contaning 's) is similar and hence omitted. Therefore  is bipartite and there is a bipartation of it which yields an R-C partition of .

Conversely, let the conditions (1) and (2) be satisfied. So we have the bigraph  is an interval graph, the graph  is bipartite and there is a bipartation of  which gives an R-C partition of . We show that such an R-C partition of  cannot contain any submatrix of the form (\ref{eq:cfg}). Then it will follow that  is a probe interval graph by Theorem \ref{t:char1}. 

Now if the R-C partition of  has a submatrix of the form (\ref{eq:cfg}), then we have the following submatrix:

in the biadjacency matrix of , where .\footnote{Denoting all the vertices of one partite set of  by  and those of the other by  such that this yields the R-C partition of .} But  cannot be either  or  as we have the following couples in the above submatrix:

This contradiction proves that the above R-C partition cannot contain any submatrix of the form (\ref{eq:cfg}), as required.\hfill 
\end{pf*}

Finally, in the following we show that if we add a loop at every probe vertex of a probe interval graph, then the Ferrers dimension of the corresponding symmetric bipartite graph is at most . We know that an interval bigraph is of Ferrers dimension at most , but the converse is not true. Below we show that the property of being of Ferrers dimension at most  is also a sufficient criterion for a graph to be an interval graph.

\begin{prop}
An undirected graph  is an interval graph if and only if the Ferrers dimension of the corresponding bipartite graph , whose biadjacency matrix is the augmented adjacency matrix of , is at most .
\end{prop}

\begin{pf*}{Proof.}
From the quasi-linear ones property of the augmented adjacency matrix of an interval graph it is clear that the 's in the upper triangle and those in the lower triangle form two Ferrers digraphs whose union is , the complement of . This proves the direct part.

Conversely, Cogis \cite{C} proved that a bipartite graph  is of Ferrers dimension at most  if and only if its associated graph  is bipartite. In fact he proved that if  has nontrivial components  and has  as its isolated vertices, then there is a -coloring  of  so that  and  are two Ferrers bigraphs whose union is . Clearly, if there is no isolated vertex, i.e., , then  is decomposed into disjoint Ferrers bigraphs. 

Let  and  be two Ferrers bigraphs whose union is , i.e., . Let  be the augmented adjacency matrix of  and so the biadjacency matrix of . Let . Then  and the couple

in the matrix  shows that the two 's at positions  and  are adjacent in . Let  so that . Thus every  in the matrix  belongs to a non-trivial component of , which implies that  has no isolated vertex. Hence, as noted earlier, the two Ferrers bigraphs  and  are disjoint. So  is an interval bigraph and consequently by Theorem 1 of \cite{SSW} we have  is an interval graph.\hfill 
\end{pf*}

Let  be a probe interval graph. Let us add a loop at every probe vertex of  and denote the graph thus obtained by .

\begin{cor}
Let  be a probe interval graph. Then the Ferrers dimension of the bipartite graph, whose biadjacency matrix is the adjacency matrix of , is at most .
\end{cor}

\begin{pf*}{Proof.}
Let  be a probe interval graph with probes  and nonprobes . Let  be the interval graph with the same assignment of intervals to all the vertices as in . Let  be the bipartite graph whose biadjacency matrix is the adjacency matrix of \footnote{Note that, in the adjacency matrix of ,  for each probe vertex  of .} and  be the bipartite graph whose biadjacency matrix is the augmented adjacency matrix of . Then by the above theorem, we have  is of Ferrers dimension at most  and so  for some Ferrers bigraph  and  such that  is complete. Also the bipartite graph, whose biadjacency matrix is the following matrix, is a Ferrers bigraph, say, .


Thus we have , as required.\hfill 
\end{pf*}

\begin{ack}
The authors are grateful to the learned referees for their meticulous reading and valuable suggestions which have definitely improved the paper.
\end{ack}

\begin{thebibliography}{10}\label{bibliography}

\bibitem{B}
S. Benzer, \emph{On the topology of the genetic fine structure}, Proc. Nat. Acad. Sci. USA \textbf{45} (1959), 1607---1620.

\bibitem{BGL}
A. Berry, M. C. Golumbic and M. Lipshteyn, \emph{Recognizing chordal probe graphs and cycle-bicororable graphs}, SIAM J. Discrete Math., \textbf{21} (2007), 573--591.

\bibitem{Br}
D. E. Brown, \emph{Variations on interval graphs}, Ph.D. Thesis, Univ. of Colorado at Denver, USA, 2004.

\bibitem{BL}
D. E. Brown and J. R. Lundgren, \emph{Bipartite probe interval graphs, circular arc graphs, and interval point bigraphs}, Aust. J. Combin., \textbf{35} (2006), 221-236.

\bibitem{CK}
G. J. Chang, T. Kloks and S.-L. Peng, \emph{Probe interval bigraphs}, Electronic notes in Disc. Math., \textbf{19} (2005), 195--201.

\bibitem{CKKLP}
M.-S. Chang, T. Kloks, D. Kratsch, J. Liu and S.-L. Peng, \emph{On the recognition of probe graphs of some self-complementary classes of perfect graphs}, Proceedings of the 11th. International Computing and Combinatorics Conference (COCOON), Kunming, China, Lecture Notes in Comput. Sci., \textbf{3595}, pp. 808--817, Springer, Berlin, 2005.

\bibitem{C}
O. Cogis, \emph{A characterization of digraphs with Ferrers dimention 2}, Rapport de Recherche, \textbf{19}, G. R. CNRS no. 22, Paris, 1979.

\bibitem{GH}
P. C. Gilmore and A. J. Hoffman, \emph{A characterization of comparability graphs and of interval graphs}, Canad. J. Math., \textbf{16} (1964), 539--548.

\bibitem{G}
M. C. Golumbic, Algorithmic graph theory and perfect graphs, Annals of Disc. Math., \textbf{57}, Elsevier Sci., USA, 2004.

\bibitem{GL}
M. C. Golumbic and M. Lipshteyn, \emph{Chordal probe graphs}, Disc. App. Math., \textbf{143} (2004), 221--237.

\bibitem{GT}
M. C. Golumbic and A. Trenk, \emph{Tolerence Graphs}, Cambridge studies in advanced mathematics, Cambridge University Press, 2004.

\bibitem{HKM}
F. Harary, J. A. Kabell and F.~R. McMorris, \emph{Interval bigraphs}, Comment. Math. Univ. Carolina, \textbf{23} (1984), 739--745.

\bibitem{JS}
J. Johnson and J. Spinrad, \emph{A polynomial time recognition algorithm for probe interval graphs}, Proc. 12th annual ACM - SIAM Symposium on Disc. Algorithms, Washington D.C., 2001.

\bibitem{MWZ}
F. McMorris, C. Wang and P. Zhang, \emph{On probe interval graphs}, Disc. App. Math., \textbf{88} (1998), 315--324.

\bibitem{MR}
B. G. Mirkin and S. N. Rodin, \emph{Graphs and Genes}, Springer-Verlag, New York, 1984.

\bibitem{Mu}
H. Mller, \emph{Recognizing interval digraphs and interval bigraphs in polynomial time}, Discrete Applied Math., \textbf{78} (1997), 189--205.\\
(Erratum, \href{http://www.comp.leeds.ac.uk/hm/publ.html}{\texttt{http://www.comp.leeds.ac.uk/hm/publ.html}})

\bibitem{R}
J. Riguet, {\emph Les relations des Ferrers}, C. R. Acad. Sci. Paris, \textbf{232} (1951), 1729.

\bibitem{RB}
H. N. de Ridder and V. B. Le, {\emph Probe split graphs}, Discrete Math. Theor. Comput. Sci., \textbf{9} (2007), 207--238 (electronic).

\bibitem{RLB}
H. N. de Ridder, V. B. Le and D. Bayer, {\emph On probe classes of graphs}, Electronic Notes in Disc. Math., \textbf{27} (2006), 21.

\bibitem{SDRW}
M. Sen, S. Das, A. B. Roy and D. B. West, \emph{Interval digraphs: an analogue of interval graphs}, J. Graph Theory, \textbf{13} (1989), 189--202.

\bibitem{SDW}
M. Sen, S. Das, and D. B. West, \emph{Circular-arc digraphs: a characterization}, J. Graph Theory, \textbf{13} (1989), 581--592.

\bibitem{SSW}
M. Sen, B. K. Sanyal, and D. B. West, \emph{Representing digraphs using intervals or circular arcs}, Disc. Math., \textbf{47} (1995), 235--245.

\bibitem{LS}
L. Sheng, \emph{Cycle free probe interval graphs}, Congressus Numerantium, \textbf{140} (1999), 33--42.

\bibitem{W}
D. B. West, \emph{Short proofs for interval digraphs}, Disc. Math. \textbf{178} (1998), 287--292. 

\bibitem{Z}
P. Zhang, \emph{Probe interval graphs and their application to physical mapping of DNA}, Manuscript, 1994.

\end{thebibliography}

\end{document} 
