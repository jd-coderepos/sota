\documentclass[11 pt] {article}










\usepackage{graphicx}
\usepackage{latexsym}
\usepackage{amsfonts}
\usepackage{amsmath}
\usepackage{enumerate}


\newcommand\blackslug{\hbox{\hskip 1pt \vrule width 4pt height 8pt depth 1.5pt
        \hskip 1pt}}
\newcommand\bbox{\hfill \quad \blackslug \medbreak}
\def\d{\hbox{-}}
\def\c{\hbox{-}\cdots\hbox{-}}
\def\l{,\ldots,}

\DeclareMathOperator{\Mark}{Mark}
\DeclareMathOperator{\Move}{Move}
\DeclareMathOperator{\Unmark}{Unmark}
\DeclareMathOperator{\Explore}{Explore}
\DeclareMathOperator{\State}{State}


\newtheorem{theorem}{}[section]
\newtheorem{lemma}[theorem]{}

\newcounter{claim}

\newcommand{\Proof}{\setcounter{claim}{0}\noindent{\bf Proof.}\ \ }

\newenvironment{claim}[1][]
{\refstepcounter{claim}\vspace{1ex}\noindent{(\it\arabic{claim}){#1}{}}\it}{\vspace{1ex}}
\newcommand{\fpc}{This proves~(\arabic{claim}).}

\sloppy

\title{Coloring perfect graphs with no balanced skew-partitions}
\author{Maria Chudnovsky\thanks{Columbia University, New
    York. Partially supported by NSF grants DMS-1001091 and
    IIS-1117631.}~, Nicolas Trotignon\thanks{CNRS, LIP, ENS Lyon,
    INRIA, Universit\'e de Lyon.}~, Th\'eophile Trunck\thanks{ENS Lyon,
    LIP, INRIA, Universit\'e de Lyon.}\\ and Kristina Vu\v
  skovi\'c\thanks{School of Computing, University of Leeds, and
    Faculty of Computer Science (RAF), Union University, Belgrade,
    Serbia.  Partially supported by EPSRC grant EP/H021426/1 and
    Serbian Ministry of Education and Science projects 174033 and
    III44006.  \newline The second and third authors are partially
    supported by \emph{Agence Nationale de la Recherche} under
    reference \textsc{anr 10 jcjc 0204 01}. The three last authors are
    partially supported by PHC Pavle Savi\'c grant 2010-2011, jointly
    awarded by EGIDE, an agency of the French Minist\`ere des Affaires
    \'etrang\`eres et europ\'eennes, and Serbian Ministry of Education
    and Science.}}


\begin{document}
\maketitle

\begin{abstract}
  We present an  algorithm that computes a maximum stable set
  of any perfect graph with no balanced skew-partition.  We present
   time algorithm that colors them.
\end{abstract}


\section{Introduction}

A graph  is \emph{perfect} if every induced subgraph  of 
satisfies .  In the 1980's, Gr\"ostchel,
Lov\'asz, and Schrijver~\cite{gls:color} described a polynomial time
algorithm that colors any perfect graph.  A graph is \emph{Berge} if
none of its induced subgraphs, and none of the induced subgraphs of its
complement, is an odd chordless cycle on at least five vertices.
Berge~\cite{berge:61} conjectured in the 1960s that a graph is Berge
if and only if it is perfect.  This was proved in 2002 by Chudnovsky,
Robertson, Seymour and Thomas~\cite{CRST}.  Their proof relies on a
\emph{decomposition theorem}: every Berge graph is either in some
simple basic class, or has some kind of decomposition.  In 2002,
Chudnovsky, Cornu\'ejols, Liu, Seymour and Vu\v
skovi\'c~\cite{chudnovsky.c.l.s.v:reco} described a polynomial time
algorithm that decides whether any input graph is Berge.  The method
used in \cite{gls:color} to color perfect graphs (or equivalently
by~\cite{CRST}, Berge graphs) is based on the ellipsoid method,
and so far no purely combinatorial method is known.  In particular, it
is not known whether the decomposition theorem from~\cite{CRST} may be
used to color Berge graphs in polynomial time.

This question contains several potentially easier questions.  Since
the decomposition theorem has several outcomes, one may wonder
separately for each of them whether it is helpful for coloring.  The
basic graphs are all easily colorable, so the problem is with the
decompositions.  One of them, namely the balanced skew-partition,
seems to be hopeless.  The other ones (namely the -join, the
complement -join and the homogeneous pair) seem to be more useful
for coloring and we now explain the first step in this direction.
Chudnovsky~\cite{thesis,trigraphs} proved a decomposition theorem for
Berge graphs that is more precise than the theorem from~\cite{CRST}.
Based on this theorem, Trotignon~\cite{nicolas:bsp} proved an even
more precise decomposition theorem, that was used by Trotignon and
Vu\v skovi\'c~\cite{nicolas.kristina:2-join} to devise a polynomial
algorithm that colors Berge graphs with no balanced skew-partition,
homogeneous pair nor complement -join.  This algorithm focuses on
the -join decompositions.  Here, we strengthen this result by
constructing a polynomial time algorithm that colors Berge graphs with
no balanced skew-partition.



Our algorithm is based directly on \cite{thesis,trigraphs}, and a few
results from~\cite{nicolas:bsp} and \cite{nicolas.kristina:2-join} are
used.  It should be pointed out that the method presented here is
significantly simpler and shorter than \cite{nicolas.kristina:2-join},
while proving a more general result.  This improvement is mainly due
to the use of \emph{trigraphs}, that are graphs where some edges are
left ``undecided''.  This notion introduced by
Chudnovsky~\cite{thesis,trigraphs} helps a lot to handle inductions,
especially when several kinds of decompositions appear in an arbitrary
order.

It is well known that an  algorithm that computes a maximum
weighted stable set for a class of perfect graphs closed under
complementation, yields an  algorithm that computes an
optimal coloring.  See for instance~\cite{KrSe:colorP},
\cite{schrijver:opticomb} or Section~\ref{sec:color} below.  This
method, due to Gr\"ostchel, Lov\'asz, and Schrijver, is quite
effective and combinatorial.  Hence, from here on we just focus on an
algorithm that computes a maximum weighted stable set.  Also, in what
follows, in order to keep the paper as readable as possible, we
construct an algorithm that computes the weight of a maximum weighted
stable set, but does not output a set.  However, all our methods are
clearly constructive, so our algorithm may easily be turned into an
algorithm that actually computes the desired stable set.


Our algorithm may easily be turned into a \emph{robust} algorithm,
that is an algorithm that takes any graph as an input, and outputs
either a stable set on  vertices and a partition of  into 
cliques of  (so, a coloring of the complement), or some polynomial
size certificate proving that  is not
``Berge-with-no-balanced-skew-partition''.  In the first case, we know
by the duality principle that the stable set is a maximum one, and the
clique cover is an optimal one, even if the input graph is not in the
class ``Berge-with-no-balanced-skew-partition''.  This feature is
interesting, because our algorithm is faster than the fastest one (so
far) for recognizing Berge graphs (Berge graphs can be recognized in
 time~\cite{chudnovsky.c.l.s.v:reco}, and determining
whether a Berge graph has a balanced skew partition can be done in
 time~\cite{nicolas:bsp,ChHaTrVu:2-join}).  However, it should be
pointed out that the certificate is not just and odd hole or antihole,
or a balanced skew-partition.  If the algorithm fails to find a stable
set at some point, the certificate is a decomposition tree, one leaf
of which satisfies none of the outputs of the decomposition theorem
for Berge graphs with no balanced skew-partitions; and if the
algorithm fails to find a clique cover, the certificate is a matrix on
 rows showing that the graph is not perfect (see
Section~\ref{sec:color}).  This really certifies that a graph is not
in our class, but is maybe not as desirable as a hole, antihole, or
balanced skew-partition.


In Section~\ref{sec:def}, we give all the definitions and state some
known results. In Section~\ref{sec:blocks}, we define a new class of
Berge trigraphs called , and we prove a decomposition theorems
for trigraphs from .  In Section~\ref{sec:BlockDec}, we define
\emph{blocks} of decomposition. In Section~\ref{sec:bas}, we
show how to recognize all basic trigraphs, and find maximum weighted
stable sets for them. In
Section~\ref{sec:decAlpha}, we describe blocks of decomposition that
allow us to compute the maximum weight of a weighted stable set.  In
Section~\ref{sec:computeAlpha}, we give the main algorithm for
computing the maximum weight of a stable set in time .  

Results in the next sections are not needed to prove our main result.
We include them because they are of indepedent interest (while on the
same subject).  In Section~\ref{sec:color}, we describe the classical
algorithm that colors a perfect graph with a stable set oracle.  We
include it because it is hard to extract it from the deeper material that
surrounds it in \cite{gls:color} or~\cite{KrSe:colorP}.  In
Section~\ref{sec:ext}, we show that Berge trigraphs with no
balanced skew partitions admit extreme decompositions, that are
decompositions one block of decomposition of which is a basic
trigraph.  We do not need them here, but extreme decompositions are
sometimes very useful, in particular to prove properties by induction.
In Section~\ref{sec:end} we give an algorithm for finding extreme decompositions in a
trigraph (if any).  In Section~\ref{sec:enlarge}, we state several
open questions about how this work could be generalized to larger
classes of graphs.

We now state our main result (the formal definitions are given in the
next section, and the proof at the end of Section~\ref{sec:color}). In
complexity of algorithms,  stands for the number of the vertices of
the input graph.

\begin{theorem}
  \label{th:colorM}
  There is an  time algorithm that colors any Berge graph with
  no balanced skew-partition.
\end{theorem}




\section{Trigraphs}
\label{sec:def}

For a set , we denote by  the set of all subsets of
 of size~2. For brevity of notation an element  of  is also denoted by  or . A {\em trigraph} 
consists of a finite set , called the {\em vertex set} of ,
and a map ,
called the {\em adjacency function}.


Two distinct vertices of  are said to be {\em strongly adjacent} if
, {\em strongly antiadjacent} if , and
{\em semiadjacent} if . We say that  and  are
{\em adjacent} if they are either strongly adjacent, or semiadjacent;
and {\em antiadjacent} if they are either strongly antiadjacent, or
semiadjacent. An \emph{edge} (\emph{antiedge}) is a pair of adjacent
(antiadjacent) vertices. If  and  are adjacent (antiadjacent),
we also say that  is {\em adjacent (antiadjacent) to} , or that
 is a {\em neighbor (antineighbor)} of . Similarly, if  and
 are strongly adjacent (strongly antiadjacent), then  is a {\em
  strong neighbor (strong antineighbor)} of . Let  be the
set of all strongly adjacent pairs of ,  the set of all
strongly antiadjacent pairs of , and  the set of all
semiadjacent pairs of . Thus, a trigraph  is a graph if
 is empty. A pair  of distinct
vertices is a \emph{switchable pair} if , a
\emph{strong edge} if  and a \emph{strong antiedge} if
.  An edge  (antiedge, strong edge, strong
antiedge, switchable pair) is \emph{between} two sets  and  if  and  or if 
and .


Let  be a trigraph. The \emph{complement}  of  is a
trigraph with the same vertex set as , and adjacency function
. For , let  denote the
set of all vertices in  that are adjacent to
. Let  and . We say that  is {\em strongly complete} to
 if  is strongly adjacent to every vertex of ;  is {\em
 strongly anticomplete} to  if  is strongly antiadjacent to
every vertex of ;  is {\em complete} to  if  is adjacent
to every vertex of ; and  is {\em anticomplete} to  if  is
antiadjacent to every vertex of . For two disjoint subsets 
of ,  is {\em strongly complete (strongly anticomplete,
 complete, anticomplete)} to  if every vertex of  is strongly
complete (strongly anticomplete, complete, anticomplete) to . A
set of vertices  \emph{dominates (strongly
 dominates)}  if for all , there exists
 such that  is adjacent (strongly adjacent) to .


A {\em clique} in  is a set of vertices all pairwise adjacent, and
a {\em strong clique} is a set of vertices all pairwise strongly
adjacent. A {\em stable set} is a set of vertices all pairwise
antiadjacent, and a {\em strongly stable set} is a set of vertices all
pairwise strongly antiadjacent. For  the trigraph
{\em induced by  on } (denoted by ) has vertex set ,
and adjacency function that is the restriction of  to . Isomorphism between trigraphs is defined in the natural
way, and for two trigraphs  and  we say that  is an {\em
 induced subtrigraph} of  (or  {\em contains  as an induced
 subtrigraph}) if  is isomorphic to  for some . Since in this paper we are only concerned with the induced subtrigraph
containment relation, we say that \emph{ contains~} if 
contains  as an induced subtrigraph. We denote by 
the trigraph .


Let  be a trigraph. A \emph{path}  of  is a sequence of
distinct vertices  such that either , or for ,  is adjacent to  if  and
 is antiadjacent to  if . Under these
circumstances,  and we say that  is a
path {\em from  to }, its {\em interior} is the set
, and the {\em length} of  is
. We also say that  is a \emph{-edge-path}. Sometimes,
we denote  by .  Observe that, since a graph is also a
trigraph, it follows that a path in a graph, the way we have defined
it, is what is sometimes in literature called a chordless path.


A {\em hole} in a trigraph  is an induced subtrigraph  of 
with vertices  such that , and for ,  is adjacent to  if  or
; and  is antiadjacent to  if . The
{\em length} of a hole is the number of vertices in it. Sometimes we
denote  by . An {\em antipath} ({\em antihole})
in  is an induced subtrigraph of  whose complement is a path
(hole) in .


A {\em semirealization} of a trigraph  is any trigraph  with
vertex set  that satisfies the following: for all , if  then , and
if  then .  Sometimes we will describe
a semirealization of  as an {\em assignment of values} to
switchable pairs of , with three possible values: ``strong edge'',
``strong antiedge'' and ``switchable pair''.  A {\em realization} of
 is any graph that is semirealization of  (so, any
semirealization where all switchable pairs are assigned the value
``strong edge'' or ``strong antiedge'').  For , we denote by  the realization of  with edge set ,  so in  the switchable pairs in  are assigned
the value ``edge'', and those in  the value
``antiedge''. The realization  is called the {\em
  full realization} of~.

Let  be a trigraph. For , we say that  and
 are {\em connected} ({\em anticonnected}) if the graph
 () is
connected. A {\em connected component} (or simply \emph{component}) of
 is a maximal connected subset of , and an {\em anticonnected
component} (or simply \emph{anticomponent}) of  is a maximal
anticonnected subset of .


A trigraph  is {\em Berge} if it contains no odd hole and no odd
antihole. Therefore, a trigraph is Berge if and only if its complement
is. We observe that  is Berge if and only if every realization
(semirealization) of  is Berge.


\subsection{Basic trigraphs}

A trigraph  is {\em bipartite} if its vertex set can be partitioned 
into two strongly stable sets. Every realization of a bipartite 
trigraph is a bipartite graph, and hence every bipartite trigraph is Berge, 
and so is the complement of a bipartite trigraph.

A trigraph  is a {\em line trigraph} if the full realization of 
is the line graph of a bipartite graph and every clique of size at
least  in  is a strong clique. The following is an easy fact
about line trigraphs.

\begin{theorem}
\label{linetrig}
If  is a line trigraph, then every realization of  is a line
graph of a bipartite graph.  Moreover, every semirealization of  is
a line trigraph.
\end{theorem}

\Proof From the definition, the full realization  of  is a line
graph of a bipartite graph . Let . Define
 as follows. For every , let
 be the common end of  and  in . Then  has
degree 2 in  because every clique of size at least  in  is a
strong clique. Let  and  be its neighbors. Now remove
 from , and replace it by two new vertices, ,
 such that  is only adjacent to , and 
to . Then  is bipartite and  is the line graph of
.  Hence, the first statement holds and the second follows
(because the full realization of a semirealization is a realization).
\bbox


Note that this implies that every line trigraph is Berge and so is the
complement of a line trigraph. Let us now define the trigraph analogue
of the double split graph (first defined in~\cite{CRST}), namely the
{\em doubled trigraph}.  A \emph{good partition} of a trigraph  is
a partition  of  (possibly,  or
) such that:

\begin{itemize}
\item Every  component of  has at most two vertices, and every
  anticomponent of  has at most two vertices.
\item No switchable pair of  meets both  and . 
\item For every component  of , every anticomponent  of
  , and every vertex  in , there exists at most
  one strong edge and at most one strong antiedge between  and
   that is incident to .
\end{itemize}

A trigraph is \emph{doubled} if it has a good partition.  Doubled
trigraphs could also be defined as induced subtrigraphs of double
split trigraphs (see~\cite{trigraphs} for a definition of double split
trigraphs, we do not need it here).  Note that doubled trigraphs are
closed under taking induced subtrigraphs and complements (because  is a good partition of some trigraph  if and only if 
is a good partition of ).  A \emph{doubled graph} is any
realization of a doubled trigraph. We now show that:

\begin{theorem}
\label{bicotri} 
If  is a doubled trigraph, then every realization of  is a
doubled graph.  Moreover, every semirealization of  is a doubled
trigraph.
\end{theorem}

\Proof The statement about realizations is clear from the
definition. Let  be a doubled trigraph, and  a good
partition of .  Let  be a semirealization of .  It is easy
to see that  is also a good partition for  (for instance,
if a switchable pair  of  is assigned value ``antiedge'',
then  and  become components of , but they still
satisfy the requirement in the definition of a good partition).
This proves the statement about semirealizations.  \bbox

Note that this implies that every doubled trigraph is Berge, because
every doubled graph is Berge.  Note that doubled graphs could be
defined equivalently as induced subgraphs of \emph{double split
  graphs} (see \cite{CRST} for a definition of double split graphs, we
do not need the definition here).

A trigraph is \emph{basic} if it is either a bipartite trigraph, the
complement of a bipartite trigraph, a line trigraph, the complement of
a line trigraph or a doubled trigraph.  The following sums up the
results of this subsection.

\begin{lemma}
  \label{l:presBas}
  Basic trigraphs are Berge, and are closed under taking induced
  subtrigraphs, semirealizations, realizations and complementation.
\end{lemma}

\subsection{Decompositions}

We now describe the decompositions that we need to state the
decomposition theorem. First, a {\em -join} in a trigraph  is a
partition  of  such that there exist disjoint sets
 satisfying:

\begin{itemize}
\item  and ;
\item  and  are non-empty;
\item no switchable pair meets both  and ;
\item every vertex of  is strongly adjacent to every vertex of
  , and every vertex of  is strongly adjacent to every
  vertex of ;
\item there are no other strong edges between  and ; 
\item for  ; and 
\item for , if , then the full realization of
 is not a path of length two joining the members of  and .
\end{itemize}

In these circumstances, we say that 
is a \emph{split} of . The -join is \emph{proper} if
for , every component of  meets both  and
. Note that the fact that a -join is proper does not depend on
the particular split that is chosen. A \emph{complement -join} of a
trigraph  is a -join of .  More specifically, a
{\em complement -join} of a trigraph  is a partition 
of  such that  is a -join of ; and
 is a {\em split} of this complement -join
if it is a split of the respective -join in the complement, i.e.\
 is strongly complete to  and strongly anticomplete
to ,  is strongly complete to , and  is strongly
complete to  and strongly anticomplete to .


\begin{theorem}
 \label{l:par2Join} Let  be a Berge trigraph and  a split of a proper -join of . Then all paths
with one end in , one end in  and interior in , for
, have lengths of the same parity.
\end{theorem}

\Proof Otherwise, for , let  be a path with one end in
, one end in  and interior in , such that  and
 have lengths of different parity. They form an odd hole, a
contradiction. \bbox


Our second decomposition is the balanced skew-partition. Let 
be disjoint subsets of . We say the pair  is {\em
 balanced} if there is no odd path of length greater than  with
ends in  and interior in , and there is no odd antipath of
length greater than  with ends in  and interior in . A {\em
 skew-partition} is a partition  of  so that  is not
connected and  is not anticonnected. A skew-partition  is {\em balanced} if the pair  is. 
Given a balanced skew-partition ,  is a
\emph{split of } if  and  are disjoint
non-empty sets, , ,  is strongly
anticomplete to , and  is strongly complete to . Note
that for every balanced skew-partition, there exists at least one
split.


The two decompositions we just described generalize some
decompositions used in~\cite{CRST}, and in addition all the
``important'' edges and non-edges in those graph decompositions are
required to be strong edges and strong antiedges of the trigraph,
respectively.  We now state several technical lemmas.


A trigraph is called {\em monogamous} if every vertex of it belongs to
at most one switchable pair. We are now ready to state the
decomposition theorem for Berge monogamous trigraphs. This is
Theorem~3.1 of \cite{trigraphs}.
\begin{theorem} 
\label{noMjointri}
Let  be a monogamous Berge trigraph. Then one of the following holds:
\begin{itemize}
\item  is basic;
\item  or  admits a proper -join; or 
\item  admits a balanced skew-partition.
\end{itemize}
\end{theorem}




When  is a skew-partition of a trigraph , we say that  is
a \emph{star cutset} of  if at least one anticomponent of  has size~1.
The following is Theorem~5.9 from~\cite{thesis}.

\begin{theorem}
 \label{starcutset}
 If a Berge trigraph admits a star cutset, then it admits a balanced
 skew-partition.
\end{theorem}


Let us say that  is a \emph{homogeneous set} in
a trigraph  if , and every vertex of  is
either strongly complete or strongly anticomplete to .

\begin{theorem}
\label{lemma}
Let  be a trigraph and let  be a homogeneous set in , 
such that some vertex of  is strongly complete to , and
some vertex of  is strongly anticomplete to . Then
 admits a balanced skew-partition.
\end{theorem}

\Proof Let  be the set of vertices of  that are 
strongly anticomplete to , and  the set of vertices of 
 that are strongly complete to . Let . Then 
 is a star cutest of  (since  and  
are non-empty and strongly anticomplete to each other), and so  
admits a balanced skew-partition by~\ref{starcutset}. \bbox


We also need the following (this is an immediate corollary of Theorem 5.13 in 
\cite{thesis}):
\begin{theorem}
\label{onepair}
Let  be a Berge trigraph.
Suppose that there is a partition of  into four nonempty sets 
, such that  is strongly anticomplete to , and 
is strongly complete to . If  is balanced then  admits a balanced 
skew-partition.
\end{theorem}




\section{Decomposing trigraphs from }
\label{sec:blocks}

Let  be a trigraph, denote by  the graph with vertex set
 and edge set  (the switchable pairs of ).  The
connected components of  are called the \emph{switchable
  components} of .  Let  be the class of Berge
trigraphs such that the following hold:
\begin{itemize}
\item Every switchable component of  has at most two edges (and
therefore no vertex has more than two neighbors in ).
\item Let  have degree two in , denote its
neighbors by  and~. Then either  is strongly complete to
 in , and  is strongly adjacent to
 in  (in this case we say that  and the switchable component
that contains  are {\em heavy}), or  is
strongly anticomplete to  in , and 
is strongly antiadjacent to  in  (in this case we say that 
and the switchable component that contains  are {\em light}).
\end{itemize} 

Observe that  if and only if ; also  is light in  if and only if  is heavy in
. 



\begin{theorem}\label{2joinform}
  Let  be a trigraph from  with no balanced
  skew-partition, and let  be a split of a
  -join  in . Then the following hold:
 \begin{enumerate}[(i)]
 \item  is a proper -join;
 \item every vertex of  has a neighbor in , ;
 \item every vertex of  has an antineighbor in , ;
 \item every vertex of  has an antineighbor in , ;
 \item every vertex of  has a neighbor in ,
  ;
 \item every vertex of  has a neighbor in ,
  ;
\item if , then  and ,
  ;
\item \label{size}, .
 \end{enumerate}
\end{theorem}

\Proof 
Note that by \ref{starcutset}, neither  nor  can have
a star cutset.  

To prove , we just have to prove that every component of 
meets both  and , . Suppose for a contradiction that
some connected component  of  does not meet  (the other
cases are symmetric). If there is a vertex  then
for any vertex , we have that  is a star
cutset that separates  from , a contradiction.  So,
. If  then pick any vertex  and
 in . Then  is a star cutset that
separates  from . So, . Hence, there exists some
component of  that does not meet , so by the same argument
as above we deduce  and the unique vertex of  has no
neighbor in . Since , there is a vertex  in
. Now  where  is a star cutset that
separates  from , a contradiction.


To prove , just notice that if some vertex in  has no
neighbor in , then it forms a component of  that does not
meet one of . This is a contradiction to .

To prove  and , consider a vertex  strongly
complete to  (the other cases are symmetric). If  then  is a star cutset that
separates  from . So  and  because .  But now  is
a homogeneous set, strongly complete to  and strongly
anticomplete to , and so  admits a balanced skew-partition
by~\ref{lemma}, a contradiction.

To prove  and , consider a vertex  strongly
anticomplete to  (the other cases are symmetric). By
,  has a neighbor in , and so . But now
 is a star cutset in
, a contradiction.

To prove , suppose that  and  (the
other cases are symmetric). By  and , and since
,  is both complete and anticomplete to
. This implies that the unique vertex of  is semiadjacent
to every vertex of , and therefore, since . Since , we deduce that , and,
since , the unique vertex of  is either
strongly complete or strongly anticomplete to , which is a contradiction because  is strongly complete to
 and strongly anticomplete to .

To prove , we may assume by  that ,
so suppose for a contradiction that .  Let  be the vertices in  respectively.  By , 
is an antiedge.  Also,  is adjacent to , for otherwise, there is
a star cutset centered at  that separates  from . Similarly,
 is adjacent to .  Since the full realization of  is not
a path of length 2 from  to , we know that  is a switchable
pair.  But this contradicts~\ref{l:par2Join}.
\bbox



Let  be a vertex of degree two in , and let  be the
neighbors of  in . Assume that  is light. We call a
vertex  an {\em -appendage} of
 if there exist  such that:
\begin{itemize}
\item  is a path;
\item  is strongly anticomplete to ;
\item  is strongly anticomplete to ; and
\item  has no neighbors in  except possibly  (i.e.\
  there is no switchable pair containing  in  except possibly
  ).
\end{itemize}

A -appendage is defined similarly. If  is a heavy vertex of ,
then  is an -appendage of  in  if and only if  is an
-appendage of  in .

The following is an analogue of~\ref{noMjointri} for trigraphs in .  It can 
be easily deduced from~\cite{thesis}, but for the reader's convenience we include a
short proof, whose departure point is~\ref{noMjointri}.

\begin{theorem}
  \label{structure}
  Every trigraph in  is either basic, or admits a
  balanced skew-partition, a proper -join, or a proper -join
  in the complement.
\end{theorem}

\Proof For , let  be the number of
vertices of degree two in . The proof is by induction on
. If , then the result follows
from~\ref{noMjointri}. Now let  and let  be a
vertex of degree two in . Let  be the two neighbors of
 in . By passing to the complement if necessary, we may
assume that  is light.

Let  be the trigraph obtained from  by making  strongly
adjacent to . If  has no -appendages, then no further changes
are necessary; set . Otherwise choose an -appendage
 of , and let  be as in the definition of an -appendage;
set  and make  semiadjacent to 
in ; set .

If  then clearly  and . Suppose that . If  is adjacent to
both  and , then  is an odd hole in
. Thus no vertex of  is adjacent to both  and . In
particular, no antihole of length at least  of  goes through
 and .  Also, there is no odd hole that goes through  and
.  Hence  is in . Moreover,  (we
remind the reader that  is the only possible neighbor of  in
).

Inductively, one of the outcomes of~\ref{structure} holds for .
We consider the following cases, and show that in each of them, one of
the outcomes of~\ref{structure} holds for .

\noindent{\bf Case 1:}  is basic.

Suppose first that  is bipartite.  We claim that  is bipartite.
Let  where  and  are disjoint strongly stable
sets. The claim is clear if  has no -appendage, so we may
assume that . We may assume that ; then . Then  and  are strongly stable sets of
 with union , and thus  is bipartite.

Suppose  is a line trigraph.  First observe that no clique of size
at least three in  contains  or . So, if ,
then clearly  is a line trigraph. So assume that . Note that the full realization of  is obtained from the
full realization of  by subdividing twice the edge . Since no
vertex of  is adjacent to both  and , it follows that  is a
line trigraph (because line graphs are closed under subdividing an
edge whose ends have no common neighbors, and line graphs of bipartite
graphs are closed under subdividing twice such an edge).

Suppose  is bipartite, and let  be a partition of
 into two strong cliques of . We may assume that . Assume first that . Since  is the unique strong
neighbor of  in , it follows that , so  contains
 and , a contradiction. Thus we may assume that . Since
 is the unique strong neighbor of  in , it follows that
, and  is strongly anticomplete to . Let  be the set of strong neighbors of  in , and  the set of strong antineighbors of  in . Since , it follows that . If either  or , then  admits a balanced
skew-partition by~\ref{lemma}, so we may assume that  and
. Since no vertex of  is adjacent to both  and ,
it follows that . Now if  or  then  is bipartite and we proceed as above, otherwise
 is a clique cutset of  of size , which is a
star cutset in , and hence  admits a balanced skew-partition
by~\ref{starcutset}.

Next assume that  is a line trigraph. Since  is a
switchable pair in  and  is strongly anticomplete to , it follows that  is strongly complete to
 else there would be in  a
clique of size  with a switchable pair. Since  is
a line trigraph, it follows that for every triangle  of  and a
vertex ,  has at least one strong neighbor
in . If  are adjacent, then
 is a triangle and  has no strong neighbor in it, and
hence  is a strongly stable set. But now,
 form a partition of  into
two strongly stable sets of .  So  is bipartite and we proceed as
above.

Finally, suppose that  is doubled and let  be a good
partition of~.  If  is empty or has a unique anticomponent,
then  is bipartite.  Hence, we may assume that  contains two
strongly adjacent vertices  and~.  If there exist  and
 such that  and  are anticomponents
of , then every vertex of  has at least two strong
neighbors, a contradiction because of .  It follows that ,
say, is an anticomponent of .  If  has a single component or
is empty, then  is the complement of a bipartite trigraph.  Hence
we may assume that  has at least two components.  Therefore, 
is a star cutset of  centered at .  This is handled in the next
case.

\noindent{\bf Case 2:}  admits a balanced skew-partition.

Let  be a balanced skew-partition of . If , let ; and if , let .
Then  is not connected. We claim that if some anticomponent 
of  is disjoint from , then  admits a balanced skew-partition. Since  is complete to  in , some component  of
 is disjoint from , and hence  is a component of
 as well.  We may assume w.l.o.g. that  is disjoint from 
(this is clearly the case if , and if
 we may assume w.l.o.g. that ).  Now, in ,  is strongly complete to ,  is strongly anticomplete to , and thus
 is a skew-partition of  and .  Since  is a
balanced skew-partition of , the pair  is balanced in ;
consequently~\ref{onepair} implies that  admits a balanced skew-partition. This proves the claim.

Thus we may assume that no such  exists, and therefore  has
exactly two anticomponents,  and , and  and . Since  is the unique strong neighbor of  in , it
follows that . Since  is anticomplete to ,
we deduce that . Let  be the component
of  containing  and . Suppose that 
does not have a strong neighbor in .  Then , and since
,  is strongly anticomplete to . We may assume
that  is not bipartite, since otherwise  satisfies one of the
outcomes of the theorem we are proving. Then  contains an odd cycle
, which must be in  or  (since  is strongly
anticomplete to ).  Since ,  must contain at
least one strong edge, say . But then  is a star cutest
in  separating  from a vertex of . So
by~\ref{starcutset},  has a balanced skew-partition. Therefore we
may assume that  has at least one strong neighbor in~.

Let . Let  be the set of strong neighbors of  in
. Then  is a star cutset in 
separating  from , unless  is the unique strong neighbor of
. In this case  is a star cutset separating  from
, unless . Now suppose that  has a
neighbor  (that is in fact a strong neighbor since ). Then  is a star cutset separating  from , unless , in which case  is
bipartite. So we may assume that  has no neighbor in .  Now,
either  is bipartite, or  has an odd cycle. But in this later
case, the cycle is in  and any strong edge of it (which exists
since ) forms a star cutset separating  from the
rest of the cycle.  Therefore, by~\ref{starcutset},  has a balanced
skew-partition.

\noindent{\bf Case 3:}  admits a proper -join.

Let  be a split of a proper -join of
. We may assume that . Then . If  let , and otherwise let . We may assume that
 is not a proper -join of , and
hence w.l.o.g.\  and . Then . Since  is the unique strong neighbor of  in , it
follows that . By Case~2, we may assume that  does not
admit a balanced skew-partition, and therefore~\ref{2joinform} implies
that  is anticomplete to .  Note that since ,
 is the only switchable pair in  that involves .  Let  be
the set of strong neighbors of  in  in . It follows from
the definition of a proper -join that .  We may
assume that  does not admit a balanced skew-partition, and hence
by~\ref{2joinform}, every -join of  is proper.  So either  is a split of a proper
-join in , or  and the full realization of  is a path of length two joining the members of  and
. Let this path be  where  and . Since  has no neighbor in  except possibly ,
it follows that  is an -appendage of . In particular, . Since , it follows that
,  and . But then ,
contrary to the fact that  is a split of a
proper -join of .

\noindent{\bf Case 4:}  admits a proper -join.

Let  be a split of a proper -join in
. First suppose that . Then we may
assume that . Since no vertex of  is
adjacent to both  and , it follows w.l.o.g. that ,  and .  Since  is the unique strong neighbor
of  in , it follows that  and .  But
now  is a split of a
-join in . By Case 2 we may assume that  does not admit a
balanced skew-partition, and hence this -join is proper
by~\ref{2joinform}. But then we may proceed as in Case 3.  Therefore
we may assume that .

We may assume that  is
not a split of a proper -join in , and therefore , and  (up to
symmetry). Since  is the unique strong neighbor of  in , and
since  are both non-empty, we deduce that ,
and so we may assume that . Since , it
follows that  and .  Since , it follows that .  Since  is strongly
antiadjacent to  and semiadjacent to  in , we deduce that . But now, if  has a neighbor  in  (which is therefore a
strong neighbor), then  is a star cutset
in , and if  is strongly anticomplete to  in , then it follows
from the definition of a proper -join that  is a 
homogeneous set in~. In both cases, by~\ref{starcutset} and
\ref{lemma}, respectively, we deduce that  admits a balanced
skew-partition.
\bbox


\section{Blocks of decomposition}
\label{sec:BlockDec}

The way we use decompositions for computing stable sets in
Section~\ref{sec:decAlpha} requires building blocks of decomposition
and asking several questions on the blocks. To do that
we need to ensure that the blocks of decomposition are still in our class.
 
A set  is a \emph{fragment} of a trigraph  if
one of the following holds:
\begin{enumerate}
\item\label{i:2J}  is a proper -join of ;
\item\label{i:C2J}  is a proper complement -join of .
\end{enumerate}

Note that a fragment of  is a fragment of . We now
define the \emph{blocks of decomposition } with respect to some
fragment .  A -join is \emph{odd} or \emph{even} according to
the parity of the lengths of the paths described in~\ref{l:par2Join}.

If  is a proper odd -join and , then let
 be a split of . We build the
block of decomposition  as follows. We start with . We then add two new \emph{marker vertices} 
and  such that  is strongly complete to ,  is strongly
complete to ,  is a switchable pair, and there are no other 
edges between  and .  Note that 
is a switchable component of .  We call it the \emph{marker
  component of~}.   

If  is a proper even -join and , then let
 be a split of . We build the
block of decomposition  as follows. We start with . We then add three new \emph{marker vertices}
,  and  such that  is strongly complete to ,  is
strongly complete to ,  and  are switchable pairs, and
there are no other edges between  and .  Again,
 is called the \emph{marker component of~}.


If  is a proper odd complement -join and , then
let  be a split of . We build
the block of decomposition  as follows. We start with . We then add two new \emph{marker vertices}
 and  such that  is strongly complete to , 
is strongly complete to ,  is a switchable pair, and
there are no other edges between  and .  Again,  is called the \emph{marker component of~}.

If  is a proper even complement -join and , then
let  be a split of . We build
the block of decomposition  as follows. We start with . We then add three new \emph{marker vertices}
,  and  such that  is strongly complete to ,
 is strongly complete to ,  is strongly complete to
,  and  are switchable pairs,  is a strong edge, and
there are no other edges between  and .  Again,
 is called the \emph{marker component of~}.


\begin{theorem}
 \label{l:stayBerge}
 If  is a fragment of a trigraph  from  with no
 balanced skew-partition, then  is a trigraph from~.
\end{theorem}

\Proof From the definition of , it is clear that every vertex of
 is in at most one switchable pair, or is heavy, or is light.
So, to prove that , it remains only to prove
that  is Berge.

Let  and  is a proper -join of . Let
 be a split of . 

Suppose first that  has an odd hole .
Assume that the vertices of  are consecutive in ,
then  is a path  with one end in , the
other one in  and interior in . A hole of  is obtained
by adding to  a path with one end in , the other one in ,
and interior in . By~\ref{l:par2Join}, this hole is odd, a
contradiction. Thus the marker vertices are not consecutive in ,
and since  has no neighbors in , we deduce
that . Now a hole of the same length as  is obtained in
 by possibly replacing  and/or  by some vertices  and , chosen to be antiadjacent (this is possible
by~\ref{2joinform}).

Suppose now that  has an odd antihole . Since an antihole of length~5 is also a hole, we may assume that
 has length at least~7. So, in , any pair of vertices has a
common neighbor. It follows that at most one of  is in ,
and because of its degree,  is not in . 
An antihole of same length as  is obtained in
 by possibly replacing  or  by some vertices  or , a contradiction. 

Note that the case when  has a complement -join follows by complementation.

\bbox


\begin{theorem}\label{stab}
  If  is a fragment of a trigraph  from  with no
  balanced skew-partition, then the block of decomposition  has
  no balanced skew-partition.
\end{theorem}

\Proof To prove this, we suppose that  has a balanced skew-partition  with a split .  From this,
we find a skew-partition in .  Then we use~\ref{onepair} to prove
the existence of a \emph{balanced} skew-partition in .  This gives
a contradiction that proves the theorem.

Let  and  is a proper -join of
. Let  be a split of .

Since the marker vertices in ,  and  have no common strong
neighbor and  has no strong neighbor, there are up to symmetry two
cases:  and , or  and . Note that when  is even, the marker vertex  must
be in  because it is adjacent to  and has no strong neighbor.

Assume first that  and  are both in . Then  is a split of a skew-partition  in~.  The pair  is
balanced in  because it is balanced in .  Hence,
by~\ref{onepair},  admits a balanced skew-partition, a
contradiction.  


Thus not both  and  are in , and so  and . In this case,  is a split of a skew-partition  in .  The pair  is
balanced in  because it is balanced in .  Hence,
by~\ref{onepair},  admits a balanced skew-partition, a
contradiction.   

The case when  has a complement -join follows by complementation
\bbox

\section{Handling basic trigraphs}
\label{sec:bas}

Our next goal is to compute maximum strong stable sets.  We need to
work in weighted trigraphs for the sake of induction. So, throughout
the remainder of the paper, by ``trigraph'' we mean a trigraph with
weights on the vertices.  Weights are numbers from  where  is
either the set  of non-negative real numbers or the set
 of non negative integers.  The statements of the
theorems will be true for  but the algorithms are to
be implemented with . Note that we view a trigraph
where no weight is assigned to the vertices as a weighted trigraph all
of whose vertices have weight~1.  Observe that a set of vertices of a
trigraph is a strong stable set if and only if it is a stable set of
its full realization.



\begin{theorem}
  \label{th:decBas}
  There is an  algorithm whose input is a trigraph and whose
  output is either the true statement `` is not basic'', or the
  name of a basic class in which  is and the maximum weight of a
  strong stable set of .
\end{theorem}

\Proof For each basic class, we provide an at most  time
algorithm that decides whether a trigraph  belongs to the class,
and if so, computes a maximum weighted strong stable set.

For bipartite trigraphs, we construct the full realization  of .
It is easy to see that  is bipartite if and only if  is
bipartite, and deciding whether a graph is bipartite can be done in
linear time by the classical Breadth First Search.  If  is
bipartite, a maximum weighted stable set of  (which is a maximum
weighted strong stable set of ) can be computed in time ,
see~\cite{schrijver:opticomb}.


For complements of bipartite trigraph, we proceed similarly: we first
take the complement  of the input trigraph , and then
recognize whether the full realization of  is
bipartite. We then compute the maximum weighted clique in
. All this can clearly be done in  time.

For line trigraphs, we compute the full realization , and test
whether  is a line graph of a bipartite graph by a classical
algorithm from~\cite{lehot:root} or~\cite{roussopoulos:linegraphe}.
Note that these algorithms also provide a graph  such that
.  In time  we can check that every clique of size at
least~3 in  is a strong clique so we can decided whether  is a
line trigraph.  If so, a maximum stable set in  can be computed in
time  by computing a maximum weighted matching
(see~\cite{schrijver:opticomb}) in a bipartite graph  such that .

For complements of line trigraphs, we proceed similarly for the
recognition except that we work with the full realization of .  And
computing a maximum weighted strong stable set is easy: compute the
full realization  of , then compute a bipartite graph  such
that  (this exists because by~\ref{linetrig},
line trigraphs are closed under taking realizations) and compute a
maximum weighted stable set in  (note that such a set is an
inclusion-wise maximal set of pairwise adjacent edges in , and there are
linearly many such sets).  This is a maximum weighted strong stable
set in~.

For doubled trigraphs, the situation is slightly more complicated,
because we do not know how to rely on classical results.  But for one
who starts from scratch (with no knowledge of matching theory for
instance), they are in fact the easiest basic graphs to handle.  To
decide whether a graph  is doubled, we may use the list of
minimally non-doubled graphs described in~\cite{alexeevFK:doubled}.
This list is made of 44 graphs on at most 9 vertices, so it yields an
 time recognition algorithm.  We propose here something
faster, and which also works for trigraphs.

If a partition  of the vertices of a trigraph is given,
deciding whether it is good can be done by a brute force checking of
all items from the definition in time .  And if an edge 
from  is given, reconstructing the good partition is easy: all
the vertices strongly antiadjacent to  and  go into , and all
the vertices strongly adjacent to at least one of  or  go into
.  So, by checking all edges , one can guess one that is in
, then reconstruct , and therefore test in time 
whether a trigraph  has a good partition  such that 
contains at least one edge.  Similarly, one can test in time 
whether a trigraph  has a good partition  such that 
contains at least one antiedge.  We are left with the recognition of
doubled trigraphs such that all good partitions are made of one strong
stable set and one strong clique.  These are in fact graphs (there is
no switchable pair), and are known as \emph{split graphs} (in fact,
double split graphs were named after split graphs).  They can be
recognized in linear time, see~\cite{hammerS:split} where it it shown
that by looking at the degrees, one can easily output a partition of a
graph into a clique and a stable set, if any such partition exists.

Now, we know that  is a doubled graph, and we look for a maximum
weighted strong stable set in .  To do so, we compute the full
realization  of~.  So, by~\ref{bicotri} ,  is a doubled graph,
and in fact,  is good partition for~.  We then compute a
maximum weighted stable set in  (that is bipartite), in 
(that is complement of bipartite), and all stable sets made of a
vertex from  together with its non-neighbors in .  One of these
is a maximum weighted stable set of , and so a strong one in .
\bbox

\section{Keeping track of }
\label{sec:decAlpha}

In this section, we define several blocks of decompositions that allow
us to compute maximum strong stable sets.  From here on, 
denotes the weight of a maximum weighted strong stable set of .

In what follows,  is a trigraph from~ with no balanced
skew-partition,  is an fragment of  and  (so
 is also fragment of ).  To compute , it is not enough to
consider the blocks  and  (as defined in
Section~\ref{sec:BlockDec}) separately.  Instead, we need to enlarge
 slightly, to encode information from . In this section, we define four different kinds
of gadgets, named , \dots,  and for , we prove that  may easily be computed from
.  We sometimes have to define different gadgets for
handling the same situation.  This is because in
Section~\ref{sec:computeAlpha} (namely to prove~\ref{expBas}), we need
that gadgets preserve being basic, and depending on the basic class
under consideration, we need to use different gadgets.  Note that the
gadgets are not class-preserving (some of them introduce balanced
skew-partitions).  In this section, this is not a problem, but in the
next section, this makes things a bit more complicated.



\subsection{Complement -join}
\label{ss:c2j}

If  is a proper complement -join of  then let ,
, and let  be a split of .  We build the gadget  as follows.  We
start with . We then add two new \emph{marker vertices} ,
, such that  is strongly complete to ,  is
strongly complete to  and  is a strong edge.  We give
weights  and  to
 and  respectively.  We set .


\begin{lemma}
  \label{alphaC2J}
  If  is a proper complement -join of , then 
  is Berge and .
\end{lemma}
\Proof Since  is a semirealization of an induced subtrigraph of
the block  as defined in Section~\ref{sec:BlockDec}, it is clearly
Berge by~\ref{l:stayBerge}.

Let  be a maximum weighted strong stable set in . If , then  is also a strong stable set in , so .  If  and , then  is a strong
stable set in  of weight , so .  If  and , then
 is a strong stable set in  of
weight , so .  If  and , then , so .
In all cases, we proved that .

Conversely, let .  If
, then by considering any maximum strong stable set
of , we see that .  So we
may assume that  and let  be a maximum
weighted strong stable set in .  If  and , then  is also a strong stable set in , so .  If  and , then , where  is a maximum weighted strong stable in
, is also a strong stable set in  of same weight as , so
.  If  and , then  where  is a maximum weighted stable in 
is also a strong stable set in  of same weight as , so .  In all cases, we proved that .  \bbox



\subsection{-join}
\label{ss:2j}

In~\cite{nicolas.kristina:2-join}, an NP-hardness result is proved,
that suggests that the -join is maybe not the most convenient tool to
compute maximum stable sets.  It seems that to use them, we really
need to take advantage of Bergeness in some way.  This is done here by
proving several inequalities.

If  is a -join of  then let ,  and let
      be a split of .
We define , ,  and .  Let  be the weight function on .  When 
is an induced subtrigraph of , or a subset of , 
denotes the sum of the weights of vertices in .


\begin{lemma}
  \label{l:4cases}
  Let  be a maximum weighted strong stable set of . Then exactly
  one of the following holds:

  \begin{enumerate}
  \item\label{i:4c1} , ,  is a maximum weighted strong stable set of  and ;
  \item\label{i:4c2} , ,  is a maximum weighted strong stable set of  and ;
  \item\label{i:4c3} , ,  is a maximum weighted strong stable set of
     and ;
  \item\label{i:4c4} , ,  is a maximum weighted strong stable set of
     and .
  \end{enumerate}
\end{lemma}

\Proof
  Follows directly from the definition of a -join.
\bbox

We need several inequalities that say more about how strong stable
sets and -joins overlap.  These lemmas are proved
in~\cite{nicolas.kristina:2-join} in the context of graphs.  The
proofs are the same for trigraphs, but for the sake of
completeness, we rewrite them.

\begin{lemma}
  \label{l:ineqbasic}
  .
\end{lemma}

\Proof The inequalities  are trivially true. Let  be a maximum
weighted strong stable set of .  We have:
  
\bbox

\begin{lemma}
  \label{l:ineqOdd}
  If  is an odd -join of , then .
\end{lemma}

\Proof Let  be a strong stable set of  of weight
 and  a strong stable set of  of weight .
In the bipartite trigraph , we denote by 
(resp.\ ) the set of those vertices of  for which there
exists a path in  joining them to some vertex of
 (resp.\ ).  Note that from the definition, ,  and there are no edges 
between  and .  We claim that
, and  is strongly anticomplete to . 
Suppose not. then there exists a  path   in  from a vertex of 
 to a vertex of .  We may assume that  is minimal 
with respect to this property, and so the  interior of  is in ; 
consequently  is of even length because  is bipartite.  
This contradicts the assumption that  is odd.  Now we set:

  \begin{itemize}
  \item ;
  \item .
  \end{itemize}

  From all the definitions and properties above,  and  are
  strong stable sets and  and .  So, .  \bbox


\begin{lemma}
  \label{l:ineqEven}
  If  is an even -join of , then
  .
\end{lemma}

\Proof Let  be a strong stable set of  of weight
 and  a strong stable set of  of
weight .  In the bipartite trigraph , we
denote by  (resp.\ ) the set of those vertices of 
for which there exists a path  in  joining them to
a vertex of  (resp.\ ).  Note that from the
definition, , ,
and  is strongly anticomplete to .  We claim that  and  is
strongly anticomplete to . Suppose not, then there is a path 
in  from a vertex of  to a vertex of .  We may assume that  is minimal with respect to this
property, and so the interior of  is in ; consequently it is
of odd length because  is bipartite.  This contradicts
the assumption that  is even.  Now we set:

  \begin{itemize}
  \item ;
  \item .
  \end{itemize}

  From all the definitions and properties above,  and  are
  strong stable sets and  and .
  So, .  \bbox






We are now ready to build the gadgets. 

If  is a proper odd -join of , then we build the
gadget  as follows. We start with . We then add four
new \emph{marker vertices} , , , , such that  and
 are strongly complete to ,  and  are strongly
complete to , and  is a strong edge.  We give weights
, , 
and  to , ,  and  respectively.
Note that by~\ref{l:ineqbasic} and~\ref{l:ineqOdd}, all the weights
are non-negative.

We define another gadget of decomposition  for the same
situation, as follows. We start with . We then add three new
\emph{marker vertices} , , , such that  and  are
strongly complete to ,  is strongly complete to , and
 and  are strong edges.  We give weights ,  and 
to ,  and  respectively.  Note that by~\ref{l:ineqbasic},
all the weights are non-negative.


\begin{lemma}
  \label{l:oddblock3}
  If  is a proper odd -join of , then  and
   are Berge, and .
\end{lemma}

\Proof Suppose that  contains an odd hole .  Since an odd
hole has no strongly dominated vertex, it contains at most one of  and at most one of .  Hence,  is an odd hole of some
semirealization of the block  (as defined in
Section~\ref{sec:BlockDec}). This contradicts~\ref{l:stayBerge}.
Similarly,  contains no odd antihole, and therefore, it is
Berge.  The proof that  is Berge is similar.

Let  be a strong stable set in  of weight .  We build
a strong stable set in  by adding to  one the
following (according to the outcome of~\ref{l:4cases}): ,
, , or .  In each case, we obtain
a strong stable set of  with weight .
This proves that .

Conversely, let  be a stable set in  with weight
.  We may assume that  is
one of , , , or , and
respectively to these cases, we construct a strong stable set of 
by adding to  a maximum weighted strong stable set of the
following: ,  , , or
.  We obtain a strong stable set in  with weight
, showing that  . This completes the proof for . 

Let us now prove the equality for .  Let  be a strong
stable set in  of weight .  We build a strong stable set
in  by adding to  one the following (according
to the outcome from~\ref{l:4cases}): , , , or
.  In each case, we obtain a strong stable set of  with weight .  This proves that .

Conversely, let  be a stable set in  with weight
.  By~\ref{l:ineqOdd}, , so we may assume that 
is one of , , , or , and
respectively to these cases, we construct a strong stable set of 
by adding to  a maximum weighted strong stable set of the
following: , , , or
.  We obtain a strong stable set in  with weight
, showing that . This completes the proof for .
\bbox


If  is a proper even -join of  and , ,
then let  be a split of . We
build the gadget  as follows.  We start with
.  We then add three new \emph{marker vertices} , , 
such that  is strongly complete to ,  is strongly complete
to , and  is strongly adjacent to  and has no other
neighbors.  We give weights , , and  to ,  and  respectively.  Note that
by~\ref{l:ineqbasic} and~\ref{l:ineqEven}, these weights are non-negative.

\begin{lemma}
  \label{l:evenblock4}
  If  is a proper even -join of , then  is
  Berge and .
\end{lemma}
\Proof 
Clearly,  is Berge, because it is a semirealization of the
block  as defined in Section~\ref{sec:BlockDec}, which is Berge
by~\ref{l:stayBerge}.  

Let  be a strong stable set in  of weight .
We build a strong stable set in  by adding to 
one the following (according to the outcome of~\ref{l:4cases}):
, , , or .  In each case, we obtain a
strong stable set of  with weight .  This proves that .

Conversely, let  be a strong stable set in  with weight
.  We may assume that  is one of
, , , or , and respectively to these
cases, we construct a strong stable set of  by adding to  a maximum weighted strong stable set of the following: , , , or .  We obtain a
strong stable set in  with weight , showing that .\bbox

\section{Computing }
\label{sec:computeAlpha}

We are ready to describe our main algorithm, that computes a maximum
weighted stable set.  The main difficulty is that blocks of
decompositions as defined in Section~\ref{sec:BlockDec} have to be used
in order to stay in the class, while gadgets as defined in
Section~\ref{sec:decAlpha} have to be used for computing .
Our idea is to use blocks in a first stage, and to replace them by
gadgets in a second stage.  To transform a block into a gadget (this
operation is called an \emph{expansion}), one needs to erase a
switchable component, and to replace it by some vertices with the
appropriate weights.  Two kinds of information are needed.  The first
one is the type of decomposition that is originally used and the
weights; this information is encoded into what we call a
\emph{prelabel}.  The second one is the type of basic class in which
the switchable component ends up (because not all gadgets preserve
being a basic class); this information is encoded into what we call a
\emph{label}.  Note that the prelabel is known right after decomposing
a trigraph, while the label becomes known much later, when the
decomposition is fully processed.  Let us make all this formal.

Let  be a switchable component of a trigraph  from .  A
\emph{prelabel} for  is one of the following: 

\begin{itemize}
\item ``Complement odd -join", 
  where ,  and  are integers, if  is
  a switchable pair.

\item  ``Odd -join",  where ,
  ,  and  are integers,
if  is switchable pair and no vertex of  is complete to .

\item  ``Complement even -join",  where
  ,  and  are integers, if  is a heavy 
  component.

\item ``Even -join",
   where ,
  ,  and  are integers, if  is a light 
  component.

\end{itemize} 

We remark that certain types of switchable components are ``eligible''
for both the first and second type of prelabel.

A prelabel should be thought of as ``the decomposition from which the
switchable component has been built''.  When  is a trigraph and
 is a set of switchable components of , a
\emph{prelabeling} for  is a function that associates
to each  a prelabel.  It is important to notice that
 is just a set of switchable component, so that some
switchable components may have no prelabel. 

What follows is slightly ambiguous when we talk about ``the basic
class containing the trigraph '', because some trigraphs may be
members of several basic classes (typically, small trigraphs, complete
trigraphs, independent trigraphs and a few others).  But this is not a
problem; if a trigraph belongs to several basic classes, our algorithm
chooses one such class arbitrarily, and the output is correct.  We
choose not to make this too formal and heavy, so this is not mentioned
explicitly in the descriptions of the algorithms.  For doubled graphs,
there is one more ambiguity.  Let  be a doubled graph, and 
 a good partition of .  A switchable pair  of  is a
\emph{matching pair} if  and an \emph{antimatching pair}
if .  In some small degenerate cases, a switchable pair of
a doubled graph may be a matching and an antimatching pair according
to the good partition under consideration, but once a good partition
is fixed, there is no ambiguity.  Again, this is not a problem: when a
pair is ambiguous, the algorithm chooses arbitrarily one particular
good partition.  


Let  be a switchable component of a trigraph from .  A
\emph{label} for  is a pair  such that  is a
prelabel and  is one of the following: ``bipartite'', ``complement
of bipartite'', ``line'', ``complement of line'',
``doubled-matching'', ``doubled-antimatching''.  We say that 
\emph{extends}~.  The tag added to extend a prelabel of a
switchable component  should be thought of as ``the basic class in
which  ends up when the trigraph is fully decomposed''.  When 
is a trigraph and  is a set of switchable components of , a
\emph{labeling} for  is a function that associates to
each  a label.  Under these circumstances we say that
 is {\em labeled}.  As with prelabels, switchable components not in
 receive no label. 


Let  be a labeled trigraph,  a set of switchable components
of  and  a labeling for .  The
\emph{expansion} of  is the trigraph obtained
from  after performing for each  with label  the
following operation:

\begin{enumerate}
\item If ``Complement odd -join'',  for some  (so  is a switchable pair ):
  transform  into a strong edge, give weight  to 
   and weight  to .



\item If  ``Odd -join'',  for some  (so  is a switchable pair
  ): transform  into a strong edge, and:
  \begin{itemize}  
  \item If  is equal to one of ``bipartite'', ``complement of
    line'', or ``doubled-matching'', then add a vertex , a vertex
    , make  strongly complete to , make
     strongly complete to , and give weights
    , ,  and  to , ,  and
     respectively.
  \item If  is equal to one of ``complement of bipartite'',
    ``line'' or ``doubled-antimatching'', then add a vertex , make
     strongly complete to , and
    give weights ,  and  to ,  and 
    respectively.
  \end{itemize}

\item If ``Complement even -join'',  for some  (so  is made of two switchable pairs
   and  and  is heavy): delete the vertex , and give weight 
   to  and   weight  to .

\item If  ``Even -join'',  for some  (so  is made of two
  switchable pairs  and , and  is light): transform  
  and  into strong edges, and give weights , , and  to ,  and  respectively.

\end{enumerate}

The expansion should be thought of as ``what is obtained if one uses a
gadget as defined in Section~\ref{sec:decAlpha} instead of a
block of decomposition as defined in Section~\ref{sec:BlockDec}''.  

\begin{theorem}
  \label{expBas}
  Suppose that  is trigraph that is in a basic class  with name ,
   is a set of switchable components of  and  is a
  labeling for  such that for all  with label ,
  one of the following holds:
  \begin{itemize}
  \item  where  is ``bipartite'', ``complement of
    bipartite'', ``line" or "complement of line''; or
  \item  ``doubled'',  is a matching pair of  and  ``doubled-matching''; or 
 \item  ``doubled'',  is an antimatching pair of  and  ``doubled-antimatching''. 
 \end{itemize}

 Then the expansion of  is a basic trigraph.
\end{theorem}

\Proof From our assumptions,  is basic.  So, it is enough to prove
that expanding one switchable component  preserves being basic, and
the result then follows by induction on . Let  be the
expansion.  For several cases from the definition of expansions
(namely items 1, 3 and 4), expanding just means possibly transforming
some switchable pairs into strong edges, and possibly deleting a
vertex.  From~\ref{l:presBas}, this preserves being basic. Hence, in
the argument below, we just study item 2 from the definition of
expansions, and thus we may assume that  is a switchable pair
. 

It is easy to check that expansion as defined in item~2 preserves
being bipartite and being complement bipartite; so if ``bipartite'', ``complement of bipartite'', then we are done.

Suppose that ``line'', and so  is a line trigraph.  Let  be
the full realization of , and  a bipartite graph such that .  So  is an edge  in , and  is an edge .  Since  is a line trigraph, it follows that every clique of
size at least  in  is a strong clique, and so  and  have
no common neighbors in . Therefore all the neighbors of  except
 are edges incident with , and not with .  Let  be
the graph obtained from  by adding a pendant edge  at .  We
observe that  is isomorphic to the full realization of 
(the edge  yields the new vertex ), and therefore  is a
line trigraph.


Next suppose that ``complement of line'', so  is the complement
of a line trigraph.  Since every clique of size at least  in
 is a strong clique, it follows that . Assume that there exist  such that
 is adjacent to . Since  is a line trigraph, and
if  is a semiedge then  is a clique of size  in
, it follows that  is strongly adjacent to  in
. Let  be a bipartite graph such that the full realization of
 is .  Then in  no two of the edges 
share and end, and yet  shares an end with all three of them, a
contradiction.  This proves that  (and
symmetrically ) is a strongly stable set in
. As , then  is bipartite, and so a
previous argument shows that  is basic.

 
So we may assume that  is a doubled trigraph with a good partition
. If  is a matching pair of , then adding the
vertices  to , produces a good partition of .  If
 is an antimatching pair of , then adding the vertex  to
 produces a good partition of . Thus in all cases  is basic
and the theorem holds.  \bbox


Let  be a trigraph,  a set of switchable components of ,
 a labeling of  and  the expansion of .  Let .  We define the  \emph{expansion
   of } as follows. Start with  and perform the following for  every . 

\begin{enumerate}
\item If ``Complement odd -join'',  for some  (so  is a switchable pair ), do not change .


\item If  ``Odd -join'',  for some  (so  is a switchable pair
  ): 
  \begin{itemize}  
  \item If  is equal to one of ``bipartite'', ``complement of
    line'', or ``doubled-matching'', do: 
    if  then add  to , and if  then add  to .

\item If  is equal to one of ``complement of bipartite'',
    ``line'' or ``doubled-antimatching'', do:  
    if  then add  to .
\end{itemize}

\item If ``Complement even -join'',  for some  (so  is made of two switchable pairs
   and  and  is heavy), do: if , then remove  from .

\item If ``Even -join'',  for some  (so  is made of two
  switchable pairs  and , and  is light), do not change .

\end{enumerate}




\begin{theorem}
  \label{l:expDec}
  With the notation as above, if  is a proper (complement)
  -join of  with split , then
   is a proper (complement) -join of  with split
  , with same parity as  (note that the notion of parity makes sense for , since
   is Berge by~\ref{alphaC2J}, \ref{l:oddblock3} and~\ref{l:evenblock4}).
\end{theorem}

\Proof
Follows easily from the definitions. 
\bbox


\begin{theorem}
  \label{mainAlg}
  There exists an algorithm with the following specification.
  \begin{description}
  \item[Input:] A triple  such that  is a
    trigraph in  with no balanced skew-partition,  is
    a set of switchable components of  and  is a
    prelabeling for .
  \item[Output:] A labeling  for  that
    extends , and a maximum weighted strong stable set of the
    expansion of .
    \item[Running time:]  
  \end{description}
\end{theorem}

\Proof We describe a recursive algorithm.  The first step of the
algorithm is to use~\ref{th:decBas} to check whether  is basic.
Note that if  is a doubled trigraph, the algorithm
from~\ref{th:decBas} also outputs which switchable pair is a
matching-pair, and which switchable pair is an antimatching pair.

Suppose first that  is in a basic class with name~ (this is the
case in particular when ).  We extend the prelabeling
 into a labeling  as follows: if ``doubled'',
then we append  to every label and otherwise, for each  with label , we add ``doubled-matching'' (resp.\
``doubled-antimatching'') to  when  is a matching (resp.\
antimatching) pair.  It turns out that the labeling that we obtain
satisfies the requirements of~\ref{expBas}, so the expansion  of
 is basic, and by running the algorithm
from~\ref{th:decBas} again for , we obtain a maximum weighted
strong stable set of  in time .  So, as claimed, we may
output a labeling  for  that extends , and a maximum weighted strong stable set of the expansion of 

Suppose now that  is not basic.  Since  is in  and has
no balanced skew-partition, by~\ref{structure}, we know that  has a
-join or the complement of a -join.  In~\cite{ChHaTrVu:2-join},
an  time algorithm for computing a -join in any input graph
is described.  In fact 2-joins as defined in~\cite{ChHaTrVu:2-join}
are sligthly different from the ones we use: they are not required to
satisfy the last item in our definition of a 2-join (this item ensures
to no side of the 2-join is a path of length exactly~2).  But the
method from Theorem~4.1 in~\cite{ChHaTrVu:2-join} shows how to handle
these kinds of requirements  with no additional time.  It is easy to
adapt this method to the detection of a -join in a trigraph (also
Section~\ref{sec:end} of the present article gives a similar
algorithm).  So we can find the decomposition that we need in time
. We then compute the blocks  and  as defined in
Section~\ref{sec:BlockDec}.  Note that every member of  is a
switchable pair of exactly one of  or .  We call  (resp.\ ) the set formed by the members of  that are in  (resp.\ ). Let  be the marker switchable
component used to create the block .  Observe that for every  there exists a vertex  such that .  The same is true for .  So, the prelabeling
 for  naturally yields a prelabeling  for  and a prelabeling  for
 (each  receives the same prelabel
it has in , similarly for ).  In what follows,
\emph{the decomposition} refers to the decomposition that was used to
build  and , (so one of ``complement odd -join'',
``complement even -join'', ``odd -join'' or ``even -join'')
and we use our standard notation for a split of the decomposition.


Up to symmetry, we may assume that . By~\ref{l:stayBerge},  are trigraphs from ,
and by~\ref{stab}, they have no balanced skew-partition.

Let  be the marker switchable component that was used to create
block .  We set .  We now
build a prelabeling  for  as follows.  All
switchable components in  keep the prelabel that they have
in .  The marker component  receives the following
prelabel:
\begin{itemize}
\item If the decomposition is a complement odd -join, then
  recursively compute
  ,  and
  , and define the prelabel of  as
  ``Complement odd -join'', .
  Observe that in this case .

\item If the decomposition is an odd -join, then recursively compute , ,  and  and define the prelabel of  as ``Odd -join'',
  . Observe that in
  this case  
  and no vertex of  is strongly complete to .


\item If the decomposition is a complement even -join, then
  recursively compute
  ,  and
  , and define the prelabel of  as
  ``Complement even -join'', .  
  Observe that in  this case  and  is light.

\item If the decomposition is an even -join, then recursively compute
  , ,  and
   and define the prelabel of  as
  ``Even -join'', .
 Observe that in  this case  and  is heavy.

\end{itemize}

Now,  has a prelabeling .  We
recursively run our algorithm for .

We obtain an extension  of  and a maximum
weighted strong stable set of the expansion  of .

We use  to finish the construction of , 
using for each  the same extension as we have in
 for extending .  Hence, now, we have an
extension  of .  Let  be the expansion of
.

Observe now that by~\ref{l:expDec},  is precisely a gadget for
, as defined in Section~\ref{sec:decAlpha}.  Hence, 
may be recovered from , as explained in one
of~\ref{alphaC2J}, \ref{l:oddblock3}, or \ref{l:evenblock4}.

Hence, the algorithm works  correctly when it returns  and the 
maximum weight of a strong stable set that we have just computed.

\medskip

\noindent{\bf Complexity analysis:} By the way we construct our
blocks of decomposition, we have  and
by~\ref{2joinform}(\ref{size}) we have . Recall that we have assumed that .

Let  be the complexity of our algorithm. For each kind of
decomposition we perform at most four recursive calls on the small
size, namely , and one recursive call for the big side .
So we have  when the graph is basic and otherwise
, where  is the
constant arising from the complexity of finding a -join or a
complement -join and finding  in basic trigraphs.


We now prove that there exists a constant  such that .  Our proof is by induction on .  We show that there exists a
constant  such that the induction step of our induction goes
through for all  (this argument, and in particular , does
not depend on ).  The base case our induction is therefore graphs
that are either basic or that have most  vertices.  For them, 
clearly exists.

We write the proof of the induction step only when the decomposition
under consideration is an even -join (possibly in the complement).
The proof for the odd -join is similar.  We set .  We
have  for all  and 
satisfying .

Let us define .  We show
that there exists a constant  such that for all  and all
 such that ,
.  By the induction hypothesis, this proves our claim. A
simple computation yields:






Since  is positive, we have .  So,  is decreasing,
and it is easy to see that if  is large enough, it is positive for
 and negative for .  Now 
is minimum for  or .  Since
 where  is a polynomial with , if  is large enough, then  is positive.  Also
. Again, if  is large enough,
 is positive.  Hence, there exists a
constant  such that for all , .  This means
that our algorithm runs in time . \bbox


\begin{theorem}
  \label{t:Strigraph}
  A maximum weighted strong stable set of a trigraph  in  with no
  balanced skew-partition can be computed in time .
\end{theorem}
\Proof Run the algorithm from \ref{mainAlg} for .  \bbox


\begin{theorem}
  \label{th:alphaG}
  A maximum weighted stable set of a Berge graph with no balanced skew-partition can be computed in time .
\end{theorem}
\Proof Follows from \ref{t:Strigraph} and the fact that a Berge graph may
be seen as a trigraph from .  \bbox

\section{Coloring perfect graphs with an algorithm for stable sets}
\label{sec:color}

Gr\"ostchel, Lov\'asz and Schrijver~\cite{gls:color} proved that the
ellipsoid method yields a polynomial time algorithm that optimally
colors any input perfect graph.  However, so far, no purely
combinatorial method is known.  But, one is known (also due to
Gr\"otschel, Lov\'asz and Schrijver), under the assumption that a
subroutine for computing a maximum stable set is available.  The goal
of this section is to present this algorithm, because it is hard to
extract it from the deeper material that surrounds it in
\cite{gls:color} or~\cite{KrSe:colorP}.


In what follows,  denotes the number of vertices of the graph under
consideration.  We suppose that  is a subclass of perfect
graphs, and there is an  algorithm  that computes a
maximum weighted stable set and a maximum weighted clique for any
input graph in .


\begin{theorem}[Lov\'asz \cite{lovasz:nh}]
  \label{th:lovasz}
  A graph is perfect if and only if its complement is perfect.
\end{theorem}

\begin{lemma}
  \label{l:compS}
  There is an algorithm with the following specification:
  \begin{description}
  \item[Input: ] A graph  in , and a sequence
     of maximum cliques of  where .
  \item[Output: ] A stable set of  that intersects each ,
    . 
  \item[Running time: ] 
  \end{description} 
\end{lemma}

\Proof
  By  we mean here the maximum \emph{cardinality} of a
  clique in~.  Give to each vertex  the weight .  Note that this weight is possibly zero.  With
  Algorithm , compute a maximum weighted stable set  of~.

  Let us consider the graph  obtained from  by replicating
   times each vertex . So each vertex  in  becomes a
  stable set  of size  in  and between two such stable sets
  ,  there are all possible edges if  and
  no edges otherwise. Note that vertices of weight zero in 
  are not in . Note also that  may fail to be in ,
  but it is easily seen to be perfect.  By replicating  times
  each vertex  of , we obtain a stable set  of  of
  maximum cardinality.

  By construction,  can be partitioned into  cliques of size
   that form an optimal coloring of 
  because .  Since by Theorem~\ref{th:lovasz}
   is perfect, .  So, in ,  intersects
  every , .
\bbox


\begin{theorem}
  \label{th:color}
  There exists an algorithm of complexity  whose input is
  a graph from  and whose output is an optimal coloring of .
\end{theorem}

\Proof
  We only need to show how to find a stable set  intersecting all
  maximum cliques of , since we can apply recursion to  (by giving weight~0 to vertices of ).  Start with . At
  each iteration, we have a list of  maximum cliques  and we compute by the algorithm in Lemma~\ref{l:compS} a stable
  set  that intersects every , .  If
   then  intersects every
  maximum clique, otherwise we can compute a maximum clique 
  of  (by giving weight~0 to vertices of~).  This
  will eventually find the desired stable set, the only problem being the
  number of iterations.  We show that this number is bounded by .

  Let  be the incidence matrix of the cliques .
  So the columns of  correspond to the vertices of  and each
  row is a clique (we see  as row vector).  We prove by induction
  that the rows of  are independent.  So, we assume that the rows
  of  are independent and prove that this holds again for .

  The incidence vector  of  is a solution to 
  but not to .  If the rows of 
  are not independent, we have .  Multiplying by , we obtain .  Multiplying by , we
  obtain , so , a
  contradiction.

  So the matrices  cannot have more than 
  rows. Hence, there are at most  iterations.
\bbox

\subsection*{Proof of Theorem~\ref{th:colorM}}

\Proof An  time algorithm exists for the maximum weighted
stable set  by~\ref{th:alphaG}, so an  time coloring
algorithm for the same class exists by~\ref{th:color}.  \bbox

\section{Extreme decomposition}
\label{sec:ext}

In this section, we prove that non-basic trigraphs in our class
actually have extreme decompositions. They are decompositions whose
one block of decomposition is basic. Note that this is non-trivial in
general, since in~\cite{nicolas.kristina:2-join} an example
is given, showing that Berge graphs in general do not necessarily have
extreme -joins.  Extreme decompositions are sometimes very useful
for proofs by induction.

In fact, we are not able to prove that any trigraph in our class has
an extreme -join or complement -join; to prove such a statement,
we have to include a new decomposition, the homogeneous pairs, in our
set of decompositions.  Interestingly this decomposition is not new,
it has been used in several variants of Theorem~\ref{noMjointri}.

A {\em proper homogeneous pair} of a trigraph  is a pair of
disjoint nonempty subsets  of , such that if 
denote respectively the sets of all strongly -complete and strongly
-anticomplete vertices and  are defined similarly, then:
\begin{itemize}
\item  and ;
\item 
(and in particular every vertex in  has a neighbor and an 
antineighbor in  and vice versa); and 
\item the four sets , , , 
 are all nonempty.
\end{itemize}

In these circumstances, we say that  is a \emph{split} of the homogeneous
pair.

A way to prove the existence of an extreme decomposition is to
consider a ``side'' of a decomposition and to minimize it, to obtain
what we call an \emph{end}.  But for homogeneous pairs, the two sides
(which are  and  with our usual
notation) are not as symmetric as the two sides of a -join, so we
have to decide which side is to be minimized.  We decide to minimize
the side .  To make all this formal, we therefore have to
distinguish between a \emph{fragment}, which is any side of any
decomposition, and a \emph{proper fragment} which is a side to be
minimized, and therefore cannot be the side  of a homogeneous pair.  All definitions are formally given below.

First we modify our definition of a fragment to include
homogeneous pairs. From here on, A set  is a
\emph{fragment} of a trigraph  if one of the following holds:

\begin{enumerate}
\item\label{i:2J}  is a proper -join of ;
\item\label{i:C2J}  is a proper complement -join of ;
\item there exists a proper homogeneous pair  of  such that
 or ). 
\end{enumerate}

A set  is a \emph{proper fragment} of a trigraph  if
one of the following holds:
\begin{enumerate}
\item\label{i:2J}  is a proper -join of ;
\item\label{i:C2J}  is a proper complement -join of ;
\item\label{i:HP} there exists a proper homogeneous pair  of  such that
.
\end{enumerate}


An \emph{end} of  is a proper fragment  of  such that no
proper induced subtrigraph of  is a proper fragment of .

Note that a proper fragment of  is a proper fragment of
, and an end of  is an end of
. Moreover a fragment in  is still a fragment in
. We have already defined the blocks of decomposition of
a -join or complement--join.  We now define the blocks of
decomposition of a homogeneous pair.

If  where  is a split of a proper
homogeneous pair  of , then we build the block of
decomposition as follows. We start with . We then add
two new \emph{marker vertices}  and  such that  is strongly
complete to ,  is strongly complete to ,  is a
switchable pair, and there are no other edges between  and .  Again,  is called the \emph{marker component of~}.



If  where  is a
split of a proper homogeneous pair  of , then we build the
block of decomposition  with respect to  as follows. We
start with . We then add two new \emph{marker vertices}  and
 such that  is strongly complete to ,  is strongly
complete to ,  is a switchable pair, and there are no
other edges between  and .  Again,
 is called the \emph{marker component of }.

\begin{theorem}
 \label{l:stayBergeExt}
 If  is a fragment of a trigraph  from  with no
 balanced skew-partition, then  is a trigraph from~.
\end{theorem}

\Proof From the definition of , it is clear that every vertex of
 is in at most one switchable pair, or is heavy, or is light.
So, to prove that , it remains only to prove
that  is Berge.

If the fragment come from a -join or the complement of a -join,
we have the result by~\ref{l:stayBerge}.

If  and  is a proper homogeneous pair of ,
then let  be a hole or an antihole in . Passing to the
complement if necessary, we may assume that  is a hole. If it
contains the two markers , it must be a cycle on four vertices,
or it must contain two strong neighbors of  in , and two strong
neighbors of  in , so  has length~6. Hence, we may assume
that  contains at most one of , so a hole of the same length
in  is obtained by possibly replacing  or  by some vertex of
 or . Hence,  has even length.


If there exists a proper homogeneous pair  of  such that , then since every vertex of  has a
neighbor and an antineighbor in , we see that every realization of
 is an induced subgraph of some realization of . It follows
that  is Berge.  \bbox

\begin{theorem}\label{stabExt}
  If  is a fragment of a trigraph  from  with no
  balanced skew-partition, then the block of decomposition  has
  no balanced skew-partition.
\end{theorem}

\Proof To prove this, we suppose that  has a balanced skew-partition  with a split .  From this,
we find a skew-partition in .  Then we use~\ref{onepair} to prove
the existence of a \emph{balanced} skew-partition in .  This gives
a contradiction that proves the theorem.

If the fragment come from a -join or the complement of a -join,
we have the result by~\ref{stab}.

If  and  is a homogeneous
pair of , then  let  be a split of .
Because  is a switchable pair, the markers  and  have no
common neighbor and  dominates , there is up to symmetry only
one case:  and . Since  is complete to
, and  is anticomplete to , it follows that .

Now  is a split of a skew-partition  in
. The pair  is balanced in  because it is balanced
in .  Hence, by~\ref{onepair},  admits a balanced skew-partition, a contradiction.

If 
 and  is a proper homogeneous pair
of , then  let  be a split of .  
Because  is a switchable pair we may assume, using symmetry and
complementation that  and . If , then 
is a split of a skew-partition in , and if  , then

is a split of a skew-partition in .  In both cases, the pair
 is balanced in  because it is balanced in .
Hence, by~\ref{onepair},  admits a balanced skew-partition, a
contradiction. \bbox


\begin{theorem} 
 \label{l:extreme}
 If  is an end of a trigraph  from  with no balanced
 skew-partition, then the block of decomposition  is basic.
\end{theorem}

\Proof Let  be a trigraph from  with no balanced
skew-partition and  an end of . By~\ref{l:stayBergeExt}, we know
that  and by~\ref{stabExt}, we know that  has no
balanced skew-partition. By~\ref{structure}, it is enough to show that
 has no proper -join and no proper complement -join.

Passing to the complement if necessary, we may assume that one of the
following three statements hold:
\begin{itemize}
\item  and  is a proper homogeneous pair of ;
\item  is a proper even -join of ;
\item  is a proper odd -join of .
\end{itemize}

\noindent {\bf Case 1:}  where  is a proper
homogeneous pair of . Let  be a split of . 


Suppose  admits a proper -join . Let
 be a split of . Because 
is a switchable pair we may assume that  are both in . As
 strongly dominates  we may assume that  and
, so . Since  is strongly complete
to , , and analogously .
By~\ref{stabExt} and~\ref{2joinform},  and , and because
, every vertex from  has a neighbor and an
antineighbor in  and vice versa. Now  is
a split of a proper homogeneous pair of . Because ,
 is strictly included in , a contradiction.

Because  is also a homogeneous pair of , by the
same argument as above,  cannot admit a proper complement -join.


\noindent {\bf Case 2:}  is a proper even -join
 of , where . Let  be a split of
.

Suppose that  admits a proper -join . Let
 be a split of . Since
 and  are switchable pairs, we may assume that . Now we claim that  is a proper
-join of  and  is strictly included in , which gives a
contradiction. Note that because of the definition of a -join and
the fact that  has no strong neighbor,  cannot only be  and hence,  is strictly included in .  Since  has
no strong neighbor, we have . Since  and  have no
common strong neighbor in , there are up to symmetry three
cases: either , , or , ,
or .

If  and , then  is a split of a  -join of .

If  and , then  is
a split of a -join of .

If  and , then  is a split of a  -join of
.

By~\ref{stabExt} and~\ref{2joinform} each of these -joins is proper, and we have a
contradiction.

Suppose  admits a proper complement -join . Because
 has no strong neighbor we get a contradiction.



\noindent {\bf Case 3:}  is a proper odd -join
 of , where . Let  be a split of
.

Suppose  admits a proper -join . Let
 be a split of . Since
 is a switchable pair, we may assume that . Now we
claim that  is a proper -join of ,
obtaining a contradiction, because  cannot be only  (by
the definition of a -join), so  is strictly included in .
Because  and  have no common strong neighbor in  there
are up to symmetry three cases: either , , or
, , or .

If  and , then  is
a split of a  -join of .

If  and , then  is a
split of a  -join of .

If  and , then  is a split of a  -join of .

By~\ref{stabExt} and~\ref{2joinform} each of these -joins is proper, and we have a
contradiction.


Suppose  admits a proper complement -join . Let
 be a split of .
Because  is a switchable pair, we may assume that .
Because  and  have no common strong neighbor we may assume that
,  and . If  and  are
not empty, then  is a split of a proper homogeneous
pair of  and  is strictly included in , a
contradiction (note that by~\ref{stabExt} and~\ref{2joinform},
, , and each vertex from  has a
neighbor and an antineighbor in  and vice versa). If  is
not empty and  is empty, then  is a split of a
proper -join of  (the -join is proper by~\ref{stabExt}
and~\ref{2joinform}). If  is empty, then  is a
split of a proper complement -join of  (again, it is proper
by~\ref{stabExt} and~\ref{2joinform}).  \bbox

\section{Means to an end}
\label{sec:end}

The goal of this section is to describe a polynomial time algorithm
that outputs an end (defined in Section~\ref{sec:ext}) of an input
trigraph (if any).  To do so, one may rely on existing algorithms for
detecting -joins and homogeneous pairs.  The fastest one is
in~\cite{ChHaTrVu:2-join} for -joins and~\cite{HaMaMo:HP} for
homogeneous pairs.  But there are several problems with this approach.
First, all the classical algorithms work for graphs, not for
trigraphs.  They are easy to convert into algorithms for trigraphs,
but it is hard to get convinced by that without going through all the
algorithms.  Worse, most of the algorithms output a fragment, not an
end.  In fact, for the -join, an algorithm
from~\cite{ChHaTrVu:2-join} does output a minimal set  such that
 is a -join, but there still could be a
homogeneous pair inside .  So, we prefer to write our own
algorithm, even if most ideas are from existing work.



Our algorithm looks for a proper fragment .  Because all the
technical requirements in the definitions of -joins and homogeneous
pairs are a bit messy, we introduce a new notion.  A \emph{weak
  fragment} of a trigraph  is a set  such that
there exist disjoint sets , , , , , ,
,  satisfying:

\begin{itemize}
\item ;
\item  ;
\item  is strongly complete to  and strongly
  anticomplete to ;
\item  is strongly complete to  and strongly
 anticomplete to ;
\item  is strongly anticomplete to ;
\item  is strongly complete to ;
\item  and ;
\item  and , ;
\item and at least one of the following statement:

\begin{itemize}
\item , , and ,  or
\item , or
\item .
\end{itemize}
\end{itemize}


In these circumstances, we say that  is a \emph{split} for .
Given a weak fragment we say it is of type {\em homogeneous pair}
if , , and , of 
type {\em -join} if , and
of type {complement -join} if . 
Note that a weak fragment may be simultaneously a -join fragment and
a complement -join fragment (when ). 

\begin{theorem}
\label{weakstruct}
If  is a trigraph from  with no balanced
skew-partition, then  is a weak fragment of  if and only if 
is a proper fragment of .
\end{theorem}

\Proof If  is a proper fragment, then it is clearly a weak fragment (the
conditions  and  are satisfied
when  is a side of a -join by~\ref{2joinform}).  Let us prove the
converse. Let  be a weak fragment, and let 
 be a split for . If 
is of type -join  or complement -join, then  it is proper
by~\ref{2joinform}. Thus we may assume that  is
of type homogeneous pair, and so
, , and .
Since all 4 sets  are non-empty, it remains to check the
following:
\begin{itemize}
\item[(i)] Every vertex of  has a neighbor and antineighbor
in . 
\item[(ii)]  and .
\end{itemize}
 

Suppose (i) does not hold. By passing to  if necessary,
we may assume that some  is strongly complete to
. Since  is not a star cutset
in  by~\ref{starcutset}, it follows that . Now every
vertex of  is strongly complete to , and so, by the same
argument, , contradicting the assumption that . Therefore (i) holds.

To prove (ii) assume that . Since , it follows that .
By (i)  every vertex
of  is semi-adjacent to the unique vertex of , which is impossible
since  and . Therefore (ii) holds.
\bbox

A -tuple  of vertices from a trigraph  is
\emph{proper} if:

\begin{itemize}
\item , , ,  are pairwise distinct;
\item ;
\item .
\end{itemize}

A proper -tuple  is \emph{compatible} with a
weak fragment  if there is a split
 for  such that ,
,  and .

We use the following notation. When  is a vertex of a trigraph ,
 denotes the set of the neighbors of , 
denotes the set of the antineighbors of ,  the set of the
strong neighbors of , and  the set of vertices  such
that .

\begin{theorem}\label{l:forcing}
  Let  be a trigraph and  a proper
  -tuple of .  There is an  time algorithm that given a
  set  of size at least  such that , outputs a weak fragment  compatible with  and
  such that , or outputs the true statement ``There
  exists no weak fragment  compatible with  and such that ''.

  Moreover, when  is outputted, it is minimal with respect to
  these properties, meaning that  for every weak
  fragment  satisfying the properties.
\end{theorem}

\Proof
\begin{table}{\small\label{algoM}
\begin{description}
\item{\bf Input:}  a  set of vertices of a trigraph  and a proper -tuple  such
  that  and .

\item{\bf Initialization:}

; ; 
;  ;\\
;


Vertices  are left unmarked. For the other vertices of :

\rule{1em}{0ex} for every vertex ;

\rule{1em}{0ex} for every vertex ;

\rule{1em}{0ex} for every vertex ;

\rule{1em}{0ex}Every other vertex of  is marked by ;

\rule{1em}{0ex}; ;



\item{\bf Main loop:}


\textbf{While} there exists a vertex  marked

\textbf{Do} 

\item{\bf Function Explore(x):}


\rule{1em}{0ex}\textbf{If} 
 \textbf{and}  \textbf{then}

\rule{2em}{0ex}2, ;

\rule{1em}{0ex}\textbf{If}  \textbf{and}
2 \textbf{then}  
;

\rule{1em}{0ex}\textbf{If}  \textbf{and}
2 \textbf{then}\\
\rule{2em}{0ex}{\textbf{Output} {\it No weak fragment is found}}, 
\textbf{Stop};


\rule{1em}{0ex}\textbf{If}  \textbf{then}  
, ;

\rule{1em}{0ex}\textbf{If}  \textbf{then}  
, ;

\rule{1em}{0ex}\textbf{If}  \textbf{and}
 \textbf{then} 

\rule{2em}{0ex}2, ;

\rule{1em}{0ex}\textbf{If}  \textbf{and} 2 \textbf{then} 
;

\rule{1em}{0ex}\textbf{If}  \textbf{and} 2 \textbf{then} \\
\rule{2em}{0ex}{\textbf{Output} {\it No weak fragment is found}}, 
\textbf{Stop};

\item{\bf Function Move(Y):}

\rule{1em}{0ex}{\it This function just moves a subset  from  to . }

\rule{1em}{0ex}\textbf{If}  \textbf{then} \\
\rule{2em}{0ex}{\textbf{Output} {\it No weak fragment is found}}, 
\textbf{Stop};

\rule{1em}{0ex}{;  ;   ;  ;}
\end{description}

\caption{Procedure used in Theorem~\ref{l:forcing}\label{algoM}}}
\end{table}
We use the procedure described in Table~\ref{algoM}.  It
tries to build a weak fragment , starting with  and .  Then,
several forcing rules are implemented, stating that some sets of
vertices must be moved from  to .  The variable ``State''
contains the type of the weak fragment that is being considered.  At the
beginning, it is ``Unknown''.   The following
properties are easily checked to be invariant during all the
execution of the procedure (meaning that they are satisfied after each
call to Explore):

\begin{itemize}
\item  and  form a partition of ,  and
  .

\item For all unmarked , and all ,  is not a
  switchable pair.  

\item All unmarked vertices belonging to  have the same neighborhood in , namely 
  (and  is a strong neighborhood).

\item All unmarked vertices belonging to  have the same neighborhood in , namely  (and 
  is a strong neighborhood).

\item All unmarked vertices belonging to  have the same neighborhood in , namely .

\item All unmarked  vertices belonging to  not adjacent to  
nor  are strongly anticomplete to .

\item For every weak fragment  such that  and
  , we have that  and
  .
\end{itemize}

By the last item all moves from  to  are necessary.  Hence, if
some vertex in  is strongly adjacent to  and , any weak
fragment compatible with  that contains  must be a complement
-join fragment.  This is why the variable State is assigned value
2 and all vertices of  are moved to .  Similarly, if some vertex in  is strongly
antiadjacent to  and , any weak fragment compatible with 
that contains  must be a -join fragment.  This is why the variable
State is assigned value 2 and all vertices of  are moved to .

When 2 and a vertex in  is
discovered to be strongly antiadjacent to  and , there is a
contradiction with the definition of the complement -join, so the
algorithm must stop.  When 2 and a vertex
in  is discovered to be strongly adjacent to  and , there
is a contradiction with the definition of the -join, so the
algorithm must stop.  When the function Move tries to move  or
 in  (this may happen if some vertex in  is semiadjacent to 
or ), then  cannot be contained in any
fragment compatible with .

If the process does not stop for all the reasons above, then all
vertices of  have been explored and therefore are unmarked.  So, if
, at the end, , is a weak fragment compatible with .
More specifically,  is a split for the
weak fragment .



Since all moves from  to  are necessary, the fragment is minimal
as claimed.  This also implies that if , then no desired
fragment exists, in which case, the algorithm outputs that no weak
fragment exists.

\noindent\textbf{Complexity Issues:} The neighborhood and
antineighborhood of a vertex in  is considered at most once.  So,
globally, the process requires  time.  \bbox


\begin{theorem}
  \label{th:detect}
  There exists an  time algorithm whose input is a trigraph
   from  with no balanced skew-partition, and whose output
  is an end  of  (if any such end exists) and the block .
\end{theorem}

\Proof Recall that by~\ref{weakstruct}, the weak fragments of  are
its proper fragments.  We first describe an  time algorithm,
and then we explain how to speed it up.  We assume that 
for otherwise no proper fragment exists.  For all proper -tuple  and for all pairs of vertices  of , we apply~\ref{l:forcing} to .  This method detects for each  and each  a
proper fragment compatible with , containing , and minimal
with respect to these properties (if any).  Among all these fragments,
we choose one with minimum cardinality, this is an end.  Once the end is
given, it is easy to know the type of decomposition that is used and
to build the corresponding block (in particular, by~\ref{l:par2Join},
one may test by just checking one path whether a -join is odd or
even).  Let us now explain how to speed this up.

We look for -joins and homogeneous pairs separately.  We describe
an  time procedure that outputs a -join weak fragment, an
 time procedure that outputs a complement -join weak
fragment, and an  time procedure that outputs a homogeneous
pair weak fragment.  Each of them outputs a fragment of minimum
cardinality among all fragments of its respective kind.  Hence, a
fragment of minimum cardinality chosen among the three is an end.

Let us first deal with -joins.  A set  of proper -tuples
is \emph{universal} if for every proper -join with split , there exists  such that , , , .  Instead of testing all -tuples as in the  time
algorithm above, it is obviously enough to restrict the search to a
universal set of -tuples.  As proved in~\cite{ChHaTrVu:2-join},
there exists an algorithm that generates in time  a universal
set of size at most  for any input graph.  It is easy to
obtain a similar algorithm for trigraphs.

The next idea for -joins is to apply the method from
Table~\ref{algoM} to  for all 's instead of
 for all 's.  As we explain now, this
finds a -join compatible with  when there
is one.  For suppose  is such a -join.  If  contains
a vertex  whose neighborhood (in ) is different from , then by~\ref{2joinform},  has at least one neighbor in
.  Hence, when the loop considers ,
the method from Table~\ref{algoM} moves some new vertices in .  So,
at the end,  and the -join is detected.  So, the method
fails to detect a -join only when  has degree~2 and  is a path while a -join compatible with  exists, with 
in the same side as .  In fact, since all vertices  are
tried, this is a problem only if this failure occurs for every
possible , that is if the -join we look for has one side made of
, , and a bunch of vertices  of degree~2
all adjacent to  and .  But in this case, either one of the
's is strongly complete to  and it is the center of
star cutset, or all the 's are adjacent to at least one of  by a switchable pair.  In this last case, all the 's are
moved to  when we run the method from Table~\ref{algoM}, so the
-join is in fact detected.

Complement -joins are handled by the same method in the complement.

Let us now consider homogeneous pairs.  It is convenient to define
\emph{weak homogeneous pairs} exactly as proper homogeneous pairs,
except that we require that ``,  and '' instead of `` and ''.  A theorem similar
to~\ref{l:forcing} exists, where the input of the algorithm is a graph
, a triple  and a set  that contains  but not , and the output is a weak
homogeneous pair  such that , ,  and , and such that  is
complete to  and anticomplete to , if any such weak homogeneous
pair exits.  As in~\ref{l:forcing}, the running time is  and
the weak homogeneous pair is minimal among all possible weak
homogeneous pairs.  This is proved in~\cite{everett.k.r:findingHP}.

As for -joins, we define the notion of a universal set of triples
.  As proved in~\cite{HaMaMo:HP}, there exists an
algorithm that generates in time  a universal set of size at
most  of triples for any input graph.  It is very easy to
obtain a similar algorithm for trigraphs.  As in the -join case, we
apply the analogue of ~\ref{l:forcing} to all vertices  instead of
all pairs .  The only problem is when after the call to the
analogue of~\ref{l:forcing}, we have a weak and non-proper homogeneous
pair (so ).  But then, it can be checked that the
trigraph has a star cutset or a star cutset in the complement.  \bbox


\section{Enlarging the class: open questions}
\label{sec:enlarge}

 The class  of Berge graphs for which we are able to compute maximum
 stable sets, namely Berge graphs with no balanced skew-partitions, has a
 strange disease: it is not closed under taking induced subgraphs.
  But from an algorithmic point of view,
 since we are able to do the computations with weights on the
 vertices, we can simulate ``taking an induced subgraph'' by putting
 weight zero on the vertices that we want to delete.

This suggests that in fact, we work on the more general class  of graphs that are  induced subgraphs of some graph in
.  The class  is closed under taking
induced subgraphs so it must be defined by a list of forbidden induced
subgraphs.  We leave the following questions open: what are the
forbidden induced subgraphs for ?  One could think that
 is in fact the class of all Berge graphs, but it is not
the case as shown by the graph  represented
in Figure~\ref{fig:contrex4}.  The graph  is Berge and admits an obvious
balanced skew-partition.  Moreover,  Robertson, Seymour and Thomas proved
that a Berge graph that contains  as an induced subgraph also
admits a balanced skew-partition, see~\cite{seymour:how}, page~78.
So,  is not in  and  might be the smallest example of
a Berge graph not in .


\begin{figure}
  \begin{center}
    \includegraphics{contrex-4.pdf}
    \caption{A graph with a balanced skew-partition\label{fig:contrex4}}
  \end{center}
\end{figure}

Here are more questions on  and .
For any graph  in , is there a graph  in
 whose size is polynomial in the size of  and such
that  is an induced subgraph of ?  If yes, or when yes, can we
compute  from  in polynomial time?  


\section*{Acknowledgment}

Thanks to Antoine Mamcarz for useful discussions on how to detect an
end of a trigraph. Thanks to Fabien de Montgolfier for having
suggesting to us that the complexity analysis in the proof of
Theorem~\ref{mainAlg} should exist.  The work on this paper began when
the first and last author were visiting LIAFA under the generous
support of Universit\'e Paris~7.

\begin {thebibliography}{9}
\bibitem{alexeevFK:doubled}
N.~Alexeev, A.~Fradkin, and I.~Kim.
\newblock Forbidden induced subgraphs of double-split graphs.
\newblock {\em SIAM Journal on Discrete Mathematics}, 26:1--14, 2012.


\bibitem{berge:61}
C.~Berge.
\newblock F{\"a}rbung von {G}raphen, deren s{\"a}mtliche bzw.~deren ungerade
  {K}reise starr sind.
\newblock Technical report, Wissenschaftliche Zeitschrift der
  Martin-Luther-Universit{\"a}t Halle-Wittenberg,
  Mathematisch-Naturwissenschaftliche Reihe 10, 1961.

\bibitem{ChHaTrVu:2-join}
P.~Charbit, M.~Habib, N.~Trotignon, and K.~Vu{\v s}kovi{\'c}.
\newblock Detecting 2-joins faster.
\newblock {\em Journal of discrete algorithms}, 17:60--66, 2012.


\bibitem{trigraphs} 
M.~Chudnovsky.
\newblock Berge trigraphs.
\newblock {\em Journal of Graph Theory}, 53(1):1--55, 2006.


\bibitem{thesis} M.~Chudnovsky.
\newblock {\em Berge trigraphs and their applications}.
\newblock PhD thesis, Princeton University, 2003.

\bibitem{chudnovsky.c.l.s.v:reco}
M.~Chudnovsky, G.~Cornu{\'e}jols, X.~Liu, P.~Seymour, and K.~Vu{\v s}kovi{\'c}.
\newblock Recognizing {B}erge graphs.
\newblock {\em Combinatorica}, 25:143--186, 2005.
 
\bibitem{CRST}
 M.~Chudnovsky, N.~Robertson, P.~Seymour, and R.~Thomas.
\newblock The strong perfect graph theorem.
\newblock {\em Annals of Mathematics}, 164(1):51--229, 2006.

\bibitem{everett.k.r:findingHP}
H.~Everett, S.~Klein, and B.~Reed.
\newblock An algorithm for finding homogeneous pairs.
\newblock {\em Discrete Applied Mathematics}, 72(3):209--218, 1997.

\bibitem{DBLP:conf/latin/2012}
D. Fern{\'a}ndez-Baca, editor.
\newblock {\em LATIN 2012: Theoretical Informatics - 10th Latin American
  Symposium, Arequipa, Peru, April 16--20, 2012. Proceedings}, volume 7256 of
  {\em Lecture Notes in Computer Science}. Springer, 2012.


\bibitem{gls:color}
M.~Gr{\"o}stchel, L.~Lov{\'a}sz, and A.~Schrijver.
\newblock {\em Geometric Algorithms and Combinatorial Optimization}.
\newblock Springer Verlag, 1988.


\bibitem{HaMaMo:HP}
M.~Habib, A.~Mamcarz, and F.~de~Montgolfier.
\newblock Algorithms for some -join decompositions.
\newblock In Fern{\'a}ndez-Baca \cite{DBLP:conf/latin/2012}, pages 446--457.


\bibitem{hammerS:split}
P.L. Hammer and B.~Simeone.
\newblock The splittance of a graph.
\newblock {\em Combinatorica}, 1(3):275--284, 1981.

\bibitem{KrSe:colorP}
J.~Kratochv{\'i}l and A.~Seb{\H o}.
\newblock Coloring precolored perfect graphs.
\newblock {\em Journal of Graph Theory}, 25:207--215, 1997.



\bibitem{lehot:root}
P.G.H. Lehot.
\newblock An optimal algorithm to detect a line graph and output its root
  graph.
\newblock {\em Journal of the Association for Computing Machinery},
  21(4):569--575, 1974.

\bibitem{lovasz:nh}
L.~Lov{\'a}sz.
\newblock Normal hypergraphs and the perfect graph conjecture.
\newblock {\em Discrete Mathematics}, 2:253--267, 1972.

\bibitem{roussopoulos:linegraphe}
N.D. Roussopoulos.
\newblock A max  algorithm for determining the graph {} from its
  line graph {}.
\newblock {\em Information Processing Letters}, 2(4):108--112, 1973.

\bibitem{seymour:how}
P.~Seymour.
\newblock How the proof of the strong perfect graph conjecture was found.
\newblock {\em Gazette des Math\'ematiciens}, 109:69--83, 2006.

\bibitem{schrijver:opticomb}
A.~Schrijver.
\newblock {\em Combinatorial Optimization, Polyhedra and Efficiency}, volume A,
  B and C.
\newblock Springer, 2003.

\bibitem{nicolas:bsp}
N.~Trotignon.
\newblock Decomposing {B}erge graphs and detecting balanced skew partitions.
\newblock {\em Journal of Combinatorial Theory, Series B}, 98:173--225, 2008.


\bibitem{nicolas.kristina:2-join}
N.~Trotignon and K.~Vu{\v s}kovi{\'c}.
\newblock Combinatorial optimization with 2-joins.
\newblock {\em Journal of Combinatorial Theory, Series B}, 102(1):153--185,
  2012.
\end{thebibliography}



\end{document}
