\documentclass{sig-alternate}
\let\proof\undefined
\let\endproof\undefined
\usepackage{amsthm,amssymb,amsmath}
\usepackage{graphics}
\usepackage{tikz}
\usepackage[plainpages=false,pdfpagelabels,colorlinks=true,citecolor=blue,hypertexnames=false]{hyperref}
\usepackage{color}

\newcommand{\bN} { {\mathbb{N}}}
\newcommand{\bC} { {\mathbb{C}}}
\newcommand{\bQ} { {\mathbb{Q}}}
\newcommand{\bZ} { {\mathbb{Z}}}
\newcommand{\bR} { {\mathbb{R}}}
\newcommand{\bF} { {\mathbb{F}}}
\newcommand{\bK} { {\mathbb{K}}}
\newcommand{\bE} { {\mathbb{E}}}
\newcommand{\bO} { {\mathbb{O}}}
\newcommand{\abs}[1]{\lvert#1\rvert}

\newcommand{\acf} { \overline{{\mathbb{F}(x)}}}
\newcommand{\cont}{\operatorname{Cont}}
\newcommand{\lc}{\operatorname{lc}}
\newcommand{\rrem}{\operatorname{rrem}}
\newcommand{\de} { {\delta}}
\newcommand{\simq}{{\sim_{q^\bZ}}}

\newcommand{\si} { {\sigma}}
\newcommand{\lde} { {\ell \de}}
\newcommand{\lsi} { {\ell \si}}
\newcommand{\ok} { {\overline{K}}}
\newcommand{\pa}{\partial}
\newcommand{\den}{\operatorname{den}}
\newcommand{\disp}{\operatorname{disp}}
\newcommand{\qdisp}{\operatorname{qdisp}}
\newcommand{\num}{\operatorname{num}}
\newcommand{\rk}{\operatorname{rank}}
\newcommand{\cres}{\operatorname{cres}}
\newcommand{\dres}{\operatorname{dres}}
\newcommand{\qres}{\operatorname{qres}}
\newcommand{\Tr}{{\operatorname{Tr}}}
\newcommand{\EOP} { {\hfill }}
\newcommand{\highlight}{\color{highlight}}
\newcommand{\blue}{\color{blue}}
\newcommand{\ie}{{\it i.e.}}
\newcommand{\bark}{{\overline{k}}}
\newcommand{\aut}{{\text{Aut}_{\bE}(\bE(y))}}
\newcommand{\QED}{{ }}
\newcommand{\bv}{\bf v}

\newcommand{\red}{\color{red}}
\newcommand{\green}{\color{green}}

\newtheorem{theorem}{Theorem}[section]
\newtheorem{lemma}[theorem]{Lemma}

\newtheorem{definition}[theorem]{Definition}
\newtheorem{fact}[theorem]{Fact}
\newtheorem{prop}[theorem]{Proposition}
\newtheorem{corollary}[theorem]{Corollary}
\newtheorem{example}[theorem]{Example}
\newtheorem{xca}[theorem]{Exercise}
\newtheorem{algo}[theorem]{Algorithm}

\newtheorem{remark}[theorem]{Remark}

\newfont{\mycrnotice}{ptmr8t at 7pt}
\newfont{\myconfname}{ptmri8t at 7pt}
\let\crnotice\mycrnotice \let\confname\myconfname 
\permission{Permission to make digital or hard copies of all or part of this work for personal or
classroom use is granted without fee provided that copies are not made or distributed for profit or
commercial advantage and that copies bear this notice and the full citation on the first page.
Copyrights for components of this work owned by others than ACM must be honored.
Abstracting with credit is permitted. To copy otherwise, or republish,
to post on servers or to redistribute to lists, requires prior specific permission and/or a fee.
Request permissions from permissions@acm.org.}

\conferenceinfo{ISSAC'16,}{July 20--22, 2016,  Waterloo, Canada.\\
{\mycrnotice{Copyright is held by the owner/author(s). Publication rights licensed to ACM.}}}
\copyrightetc{ACM \the\acmcopyr}
\crdata{978-1-4503-3435-8/15/07\ ...\LxI\mathbf{k}(x)IR[x]R\mathbf{k}\mathbf{k}Lxfa_0,\dots,a_r\xi \in \bCa_r(\xi)=0a_0,\dots,a_ra_ra_ra_r\paf(n) \mapsto f(n + 1)u_0,u_1\in  \bQu \colon
\bN \to \bQu(0)=u_0u(1)=u_1Luuu_0,u_1\bZ(n+2)L(1+16n)^2LT(1+16n)^264T(1+16n)^264TLLT64TLLL\in \bZ[x][\pa]\bZ[x][\pa]\langle L \rangle = \bQ(x)[\pa]LL\bQ(x)[\pa]\langle L \rangle\bZ[x][\pa]\cont(L) := \langle L \rangle \cap
\bZ[x][\pa]\bZ[x][\pa]\bZ[x][\pa]L\bZ[x][\pa]\cont(L)\{L, \tilde T\}\bQ[x][\pa]\langle L \rangle \cap \bQ[x][\pa]\bQ[x][\pa]pppRRR[x][\pa; \sigma, \delta]\sigma: R[x] \rightarrow R[x]R\delta: R[x] \rightarrow R[x]\sigmaRR[x][\pa]\pa p = \sigma(p) \pa + \delta(p)p \in R[x].L \in R[x][\pa]\ell_0, \ldots, \ell_r \in R[x]\ell_r \neq 0r\ell_rL\deg_{\pa}(L)\lc_{\pa}(L)R[x][\pa; \sigma, \delta]R[x][\pa]\sigma\deltaSR[x][\pa]SR[x][\pa]\cdot SQ_RRQ_{R}(x)[\pa]R[x][\pa]L \in R[x][\pa]LQ_{R}(x)[\pa] L \cap R[x][\pa]\cont(L)\mathbb{K}[x]R[x]\mathbb{K}L \in R[x][\pa]p\lc_{\pa}(L)R[x]pLnP \in Q_{R}(x)[\pa]kw, v \in R[x]\gcd(p, w) = 1R[x]PpLR[x]PLppLkk \in \bNpLp\cont(L)(1 + 16n)^2LT(1 + 16n)^2L\bZ[x][\pa]\pa x = x \pa + 1L = x (x - 1) \pa - 1(1 {-} x) \pa^2 {-} 2 \pa {=} \left(\frac{1}{x}\pa \right) LxLp\mathbb{K}[x]R[x]L \in R[x][\pa]p \in R[x]LkpLR[x]p_iR[x]\gcd(p_i, \sigma^{k}(p)) = 1R[x]i = 0,1,kd_k \geq 1\lc_\pa(P) = \sigma^k \left(w/(vp)\right)w,vR[x]\gcd(w,p)=1pLR[x]p_i, q_i \in R[x]\gcd(p_i q_i, \sigma^{k}(p)) = 1R[x]i = 0, \ldots, kd_k \geq 1\tilde{P} = \left( \prod_{i = 0}^k q_i \right) P\tilde{p_i} = p_i \left( \prod_{i = 0}^k q_i \right) / q_ii = 0, \ldots, k\gcd(\tilde{p}_i, \sigma^{k}(p)) = 1R[x]i = 0, \ldots, k\tilde{P}pLR[x]R[x][\pa]R=\mathbf{k}[t]\mathbf{k}\sigmaRR[x]\sigma(x) = a x + ba, bRa\prec\left\{x^i \pa^j \mid i, j \in \bN \right\}PR[x][\pa]cP\prec\pa^i Pc a^ica^iL \in R[x][\pa]M_k(L)\cont(L)R[x]k\cont(L)LM_k(L)M_kL \in R[x][\pa]r > 0c \in Rp_1,p_m \in R[x] \setminus RT \in R[x][\pa]kLT \in \cont(L)a, b \in Rp_i^{d_i}Ld_i > k_ii = 1 \ldots mL \in R[x][\pa]r > 0k \in \bNk \geq rTL\deg_{\pa}(T) = k\deg_{x}(\lc_{\pa}(T)) = \min \{  \deg_{x}(\lc_{\pa}(Q)) \mid Q \in M_k(L) \setminus \{0\}\}.\pa^i TLi \in \bN\lc_{\pa}(T) = a ga \in Rg \in R[x]F \in \cont(L)jj \ge k\si^{j - k}(g)\lc_{\pa}(F)R[x] t = \lc_{\pa}(T)d < \deg_{x}(t)Q \in \cont(L)\deg_{x}(\lc_{\pa}(Q)) = d\deg_{\pa}(Q) = kQ\pa^i Qi \in \bNR[x]s {\in} R {\setminus} \{0\}q, h {\in} R[x]h=0\deg_x(h) < dhs T - q Qk\cont(L)ds t = q \lc_{\pa}(Q)\deg_x(q)d < \deg_{x}(t)R[x]\si^{r-k}(q)\lc_{\pa}(L)R[x]i \in \{1 \ldots m \}p_i\si^{r-k}(q)R[x]p^{k_i+1}L\lc_{\pa}(F) = u fu \in RfR[x]\pa^{j - k} Ta \si^{j - k}(g)v \in R \setminus \{0\}p \in R[x]R[x]\si^{j - k}(g) \mid fIR[x][\pa]a \in RIai \in \bNa\sigma\deltaI : a^\inftyL \in R[x][\pa]r>0TL\lc_{\pa}(T) {=} a ga \in RgR[x]TM_kk \in \bNTkJ = ( R[x][\pa] \cdot M_k ) : a^{\infty}F \in Jj \in \bNa^j F \in R[x][\pa] \cdot M_kF \in Q_{R}(x)[\pa] LF \in \cont(L)\cont(L) = \cup_{i = r}^{\infty} M_iM_i \subseteq M_{i + 1}M_i \subseteq J \text{ for all } i \geq kii = kM_k \subseteq JiF \in M_{i + 1} \backslash M_{i}\deg_{\pa}(F) = i + 1\lc_{\pa}(F) = p \si^{i + 1 - k}(g)p \in R[x]\lc_{\pa}(a F) = \lc_{\pa}(p \pa^{i + 1 - k} T)a F - p \pa^{i + 1 - k} T \in M_{i}a F \in R[x][\pa] \cdot M_iM_i {\subset} Ja F \in R[x][\pa] \cdot JJF \in JQ_R[x]R[x]L {\in} R[x][\pa]r>0p {\in} R[x]\lc_{\pa}(L)pLQ_R[x]pLR[x]rP \in Q_R(x)[\pa]pLQ_R[x]PkPLa_i \in R[x], b_i \in Ri = 0, \ldots, k + rw,v \in R[x]\gcd(w, p)=1b = \text{lcm}(b_0, b_1, \ldots, b_{k + r})RP' = b Pp\gcd(bw, p) = 1R[x]P'pkpQ_R[x]pR[x]pTR[x]M_kkTM_k(L)LR[x][\pa]kM_kR[x]^{k+1}R[x]Q_{R}(x)[\pa]F, G \in Q_{R}(x)[\pa]G \neq 0Q, R \in Q_{R}(x)[\pa]\deg_{\pa}(R) < \deg_{\pa}(G)F = Q G + RRFG\rrem(F, G)F \in R[x][\pa]kF \in M_kF \in Q_R[x][\pa] L\rrem(F, L)=0A(k+1) \times rQ_R(x)AAR[x]R[x]N_kR[x]F = \sum_{i=0}^k f_i \pa^i \in R[x]FM_k\rrem(F, L)=0(f_k, \ldots, f_0)N_k\phi\phi(f_k, \ldots, f_0) \in N_k\rrem\left( F, L \right)=0FM_k\phi\phiR[x]M_kR[x]R[x]M_kRRLk \in \bZ^+[\pa^k] P\pa^kPI_kR[x]I_kk\cont(L)\sigma(I_k) \subset I_{k+1}L \in R[x][\pa]kM_k\cont(L)\{B_1, \ldots, B_\ell \}R[x]k\langle [ \pa^k ] B_1, \ldots,  [ \pa^k ] B_\ell \rangle \subseteq I_kf \in I_kf = \lc_{\pa}(F)F \in M_k\deg_{\pa}(F) = kM_k\{B_1, \ldots, B_\ell \}R[x]h_1, \ldots, h_\ell \in R[x]f  = h_1 \left( [ \pa^k ] B_1 \right) + \cdots + h_\ell \left( [ \pa^k ] B_\ell \right).f \in \langle [ \pa^k ] B_1, \ldots,  [ \pa^k ] B_\ell \rangleL \in R[x][\pa]kM_k\cont(L)LskSM_ksTM_kTkt = \lc_\pa(T)\deg(t)=\deg(s)usxvtut-vsu T - v Sk\pa\deg_x(t)ut=vsSL R[x][\pa]\{B_1, \ldots, B_{\ell}\}k\cont(L)kL\{b_1, \ldots, b_{\ell}\}kI_k\cont(L)\bar{I}_kI_kQ_R[x]Q_R[x]c_1,c_\ell \in R[x]s \in R[x]T=c_1 B_1 + \cdots + c_\ell B_\ellL\lc_\pa(T)=sasx\cont(L)R[x][\pa] \cdot M_kaaRR[x][\pa]IR[x][\pa]cRJyR[x][\pa]I : c^{\infty}J \cap R[x][\pa]yc\paL \in R[x][\pa]\pa x = (x + 1) \pa\pa x = x \pa + 1\cont(L)kLR[x]M_kTa\lc_{\pa}(T)x(R[x][\pa] \cdot M_k) : a^{\infty}R[x][\pa]\pa x = (x + 1) \pa\pa x = x \pa + 1R[x][\pa]R[x]R[x]TQ_R[x]\bQ[t][n][\pa]\pa n = (n + 1) \pa2M_2LM_2T_1\lc_{\pa}(T_1) = (2 + t) n\cont(L) = (\bQ[t][n][\pa] \cdot M_2 ) : (2 + t)^{\infty}\{ L, T_2 \}\bZ[n][\pa]3M_3LM_3\{L, \tilde{T} \}\tilde T\lc_{\pa}(\tilde{T}) {=} 1\tilde T\bZ[x][\pa]\pa x = x \pa + 1L = x \pa^2 - (x + 2) \pa + 2 \in \bZ[x][\pa]4M_4LM_4\{L, \pa L, T \}T = \pa^4 - \pa^3\lc_{\pa}(T) = 1TT64\lc_\pa(L)64\tilde{T}L(a_n)_{n \ge 0}(b_n)_{n \ge 0}\bZn(n! a_n b_n)_{n \ge 0}\bZnL(n! a_n b_n)_{n \ge 0}\cont(L)n1L \in R[x][\pa]QL\lc_\pa(Q)=c \, gc\lc_\pa(Q)xgQLcLLrTkc_T \in Rg = p_1^{e_1-k_1} \cdots p_s^{e_m-k_m}.gL \in R[x][\pa]r >0Ic_TLIRRLIT_1T_2k_1k_2k_1 \ge k_2c_1+c_2=0T_1+\pa^{k_1-k_2}T_2(c_1+c_2) \si^{k_1-r}(g).T_1+\pa^{k_1-k_2}T_2c_1 + c_2IRIc\prec\left\{x^i\pa^j \mid i, j \in \bN \right\}P \in R[x][\pa]PP\prec\operatorname{HT}(P)L \in R[x][\pa]r>0k \ge rR[x][\pa] \cdot M_k {=} R[x][\pa] \cdot M_{k + 1}\si(I_k) {=} I_{k + 1}.\si(I_k) = I_{k + 1}M_k \subset M_{k + 1}M_{k + 1} \subset R[x][\pa] \cdot M_kT \in M_{k + 1} \setminus M_k\lc_{\pa}(T) \in \si(I_k)F \in M_k\si(\lc_{\pa}(F)) = \lc_{\pa}(T)T - \pa F \in M_kT \in R[x][\pa] \cdot M_kR[x][\pa] \cdot M_{k + 1}= R[x][\pa] \cdot M_kI_{k + 1} \subseteq \si(I_k)\si(I_k) \subseteq I_{k + 1}\mathcal{B}R[x]M_k\mathcal{B}R[x][\pa] \cdot M_k\precx^{\ell_1} \pa^{m_1} {\prec} x^{\ell_2} \pa^{m_2}m_1 {<} m_2m_1 {=} m_2\ell_1 {<} \ell_2\deg_{\pa}(P) {\leq} kP \in \mathcal{B}M_kk\mathcal{G}R[x][\pa] \cdot \mathcal{B}\prec\deg_{\pa}(G) \leq kG \in \mathcal{G}p {\in} I_{k + 1} {\setminus} \{0\}T \in M_{k + 1} \setminus  M_k\lc_{\pa}(T) {=} pT {\in} R[x][\pa] \cdot M_{k + 1}T {\in} R[x][\pa] \cdot M_kT\mathcal{G}\operatorname{HT}(V_G G) \preceq \operatorname{HT}(T)\deg_{\pa}(V_G G) \leq k + 1V_G Gk+1\lc_\pa(V_G G) = a_G  \, \si^{k  + 1 - d_G}(\lc_{\pa}(G))a_GR[x]d_GG\deg_\pa(T)=k+1\si^{k - d_G}(\lc_{\pa}(G)) = \lc_\pa\left(\pa^{k-d_G} G\right)\si^{k - d_G}(\lc_{\pa}(G)) \in I_k\si^{-1}(p) \in I_kI_{k+1}  \subset \si(I_k)I_j = \si^{j-\ell}(I_\ell)j \ge \ell\cont(L)=R[x][\pa]\cdot M_\ellI_jI_\ellL \in R[x][\pa]r>0\ellM_\ell\cont(L)\cont(L)I_\ell\ell\cont(L)\mathbf{G}I_\ellf \in \mathbf{G}xF\cont(L)\lc_\pa(F)=fFL\cont(L)Sj=\deg_\pa(S)\lc_\pa(S)I_jj \ge \ell\sigma^{j-\ell}(I_\ell)=I_j\si^{\ell - j}(\lc_\pa(S))I_\ellc_S \in RgR[x]\si^{\ell - j}(\lc_\pa(S)) = c_S \si^{\ell - r}(g)\si^{\ell - j}(\lc_\pa(S)) \in I_\ellc_S \si^{\ell - r}(g)F\sigma^{r - \ell}\left(f\right) = c_F \, g,c_F \in Rf = c_F \si^{\ell - r}(g)\mathbf{G}I_\ellf\mathbf{G}c_S \si^{l - r}(g)I_\ellfc_F \mid c_Sc_S \mid c_Fc_Sc_FFLL \in R[x][\pa]\pa x = (x + 1) \pa\pa x = x \pa + 1L\mathcal{A}\cont(L)\ell\mathcal{A}R[x]\mathcal{B}M_\ell\mathcal{B}^\prime = \{ B \in \mathcal{B} \mid \deg_\pa(B)=\ell\}\mathbf{G}\left\langle \left\{ \lc_\pa(B) \mid B \in \mathcal{B}^\prime \right\}  \right\rangle.f\mathbf{G}xu_B\in R[x]f = \sum_{B \in \mathcal{B}^\prime} u_B \lc_\pa(B).\sum_{B \in \mathcal{B}^\prime} u_B B(a_n)_{n \ge 0}(b_n)_{n \ge 0}c_n = n! a_n b_nL \in \bZ[n][\pa]\cont(L) = R[x][\pa]\cdot M_{14}I_{14}n {+} 14 T14\lc_\pa(T) = n + 14c_n\alpha_i \in \bZ[n]i = 1, \ldots, 14a_nb_n14c_n = n! a_n b_nL \in \bZ[n][\pa]10\lc_\pa(L) {=} (n + 9) \alpha\alpha {\in} \bZ[n]\deg_{n}(\alpha) {=} 20c_n\beta\beta_i \in \bZ[n]i = 1, \ldots, 10TL14n + 14c_nn c_n = \gamma_1 c_{n - 1} + \cdots + \gamma_{14} c_{n - 14},\gamma_i \in \bZ[n]L \in R[x][\pa]TL\lc_\pa(L)xk_1, \ldots, k_m\cont(L)d\cont(L)Ra_i \in Rf_{i}(x) \in R[x]i = 0, 1, \ldots, kLR\gcd(a_0, a_1, \ldots, a_k) = 1RPQR[x][\pa]PQRP QA\si : A \rightarrow A\si\delta : A \rightarrow AI \subseteq A\si\delta\si(I) \subseteq I\delta(I) \subseteq I\chi |_{A}: A \rightarrow A/I\chi(\pa) = \tilde{\pa}\tilde{\si}\tilde{\delta}\tilde{\si}\si\deltapRI = \langle p \rangleR[x]I\si\delta\chi |_{R[x]}: R[x] \rightarrow R[x]/I\chi(\pa) = \tilde{\pa}\si^{-1}(I)\subset Ip f \in If \in R[x]\si^{-1}(p f) = p \si^{-1}(f) \in I\tilde{\si}A/IR[x]/II(R[x]/I)[\tilde{\pa}; \tilde{\si}, \tilde{\delta}]R[x]/I\tilde{\si}PLR\chi(P) \chi(L) \neq 0\chi(P L) \neq 0 \chiP LRL \in R[x][\pa]pRLRp \mid \lc_{\pa}(L)pppPPL \in R[x][\pa]p_i \in R[x]\gcd(p_i, p) = 1R[x]i = 0, \ldots, kd_k \geq 1d = \max_{0 \leq i \leq k} d_iP_1 = p^d PcP_1\pa\gcd(p_0, \ldots, p_k)\gcd(p_i, p) = 1i = 0, \ldots, k.P_1=cP_2.P_2P_1P_2R c P_2 L = p^d PL\gcd(c, p) = 1PL \in R[x][\pa]pP_2L\papRP_2LR\bZ[n][\pa]\bZ \binom{4 n}{n}+ 3^n3\lc_{\pa}(L)3R[x][\pa]RRR'$}
\begin{thebibliography}{10}

\bibitem{Abramov2006}
S.~A. Abramov, M.~Barkatou, and M.~van Hoeij.
\newblock Apparent singularities of linear difference equations with polynomial
  coefficients.
\newblock {\em AAECC}, 117--133, 2006.

\bibitem{Abramov1999}
S.~A. Abramov and M.~van Hoeij.
\newblock Desingularization of linear difference operators with polynomial
  coefficients.
\newblock In {\em Proc.\ of ISSAC'99}, 269--275, New York, NY,
  USA, 1999. ACM.

\bibitem{Barkatou2015}
M.~A. Barkatou and S.~S. Maddah.
\newblock Removing apparent singularities of systems of linear differential
  equations with rational function coefficients.
\newblock In {\em Proc.\ of ISSAC'15}, 53--60, New York, NY,
  USA, 2015. ACM.

\bibitem{Weispfenning1993}
T.~Becker and V.~Weispfenning.
\newblock {\em Gr\"{o}bner Bases, A Computational Approach to Commutative
  Algebra}.
\newblock Springer-Verlag, New York,
  USA, 1993.

\bibitem{Bronstein1996}
M.~Bronstein and M.~Petkov\v{s}ek.
\newblock An introduction to pseudo-linear algebra.
\newblock {\em Theoretical Computer Science}, pages 3 --33, 1996.

\bibitem{Chen2013}
S.~Chen, M.~Jaroschek, M.~Kauers, and M.~F. Singer.
\newblock Desingularization explains order-degree curves for {O}re operators.
\newblock In {\em Proc.\ of ISSAC'13}, 157--164, New York, NY,
  USA, 2013. ACM.

\bibitem{Chen2016}
S.~Chen, M.~Kauers, and M.~F. Singer.
\newblock Desingularization of {O}re operators.
\newblock {\em J. Symb.\ Comput.}, 74:617--626, 2016.

\bibitem{Zhang2009}
R.~C. Churchill and Y.~Zhang.
\newblock Irreducibility criteria for skew polynomials.
\newblock {\em Journal of Algebra}, 322:3797--3822, 2009.

\bibitem{Salvy1998}
F.~Chyzak and B.~Salvy.
\newblock Non-commutative elimination in {O}re algebras proves multivariate
  identities.
\newblock {\em J. Symb.\ Comput.\ }, 1998.


\bibitem{George2015}
G. Labahn et~al.
\newblock Workshop on symbolic combinatorics and computational differential
  algebra.
\newblock {{\tt http://www.fields.utoronto.ca/video-archive/event/411/2015}}.

\bibitem{David182}
D.~Grayson, M.~Stillman, and D.~Eisenbud.
\newblock Macaulay2doc -- macaulay2 documentation.

\bibitem{Max2013}
M.~Jaroschek.
\newblock {\em Removable Singularities of Ore Operators}.
\newblock PhD thesis, RISC-Linz, Johannes Kepler Univ., 2013.

\bibitem{Kapur1988}
A.~Kandri-Rody and D.~Kapur.
\newblock Computing a {G}r\"{o}bner basis of a polynomial ideal over a {E}uclidean
  domain.
\newblock {\em J. Symb.\ Comput.\ }, pages 37--57, 1988.

\bibitem{Weispfenning1990}
A.~Kandri-Rody and V.~Weispfenning.
\newblock Non-commutative {G}r\"obner bases in algebras of solvable type.
\newblock {\em J.\ Symb.\ Comput.\ }, pages 1--26, 1990.

\bibitem{Christoph2009}
C.~Koutschan.
\newblock {\em Advanced Applications of the Holonomic Systems Approach}.
\newblock PhD thesis, Johannes Kepler University Linz, 2009.

\bibitem{Christoph2010}
C.~Koutschan.
\newblock {\em Holonomic{F}unctions ({U}ser's {G}uide)}.
\newblock RISC Report Series, Johannes Kepler Univ., 2010.

\bibitem{Kovacic1972}
J.~Kovacic.
\newblock An {E}isenstein criterion for noncommutative polynomials.
\newblock {\em Proceedings of the American Mathematical Society}, 34, 1972.

\bibitem{Johannes2011}
J.~Middeke.
\newblock {\em A computational view on normal forms of matrices of {O}re
  polynomials}.
\newblock PhD thesis, Johannes Kepler University Linz, 2011.

\bibitem{Saito1999}
M.~Saito, B.~Sturmfels, and N.~Takayama.
\newblock {\em Gr\"{o}bner Deformations of Hypergeometric Differential Equations.}
\newblock Springer-Verlag, New York, USA, 1999.

\bibitem{Arne2013}
A.~Storjohann.
\newblock {\em Algorithms for Matrix Canonical Forms}.
\newblock PhD thesis, Swiss Federal Institute of Technology Zurich, 2013.

\bibitem{Tsai2000}
H.~Tsai.
\newblock Weyl closure of a linear differential operator.
\newblock {\em J.\ Symb.\ Comput.\ }, pages 747--775, 2000.

\end{thebibliography}


\end{document}
