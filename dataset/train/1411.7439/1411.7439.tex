\documentclass[letterpaper, 11pt, onecolumn]{ieeeconf}  \IEEEoverridecommandlockouts                              

\overrideIEEEmargins                                      


\usepackage{graphics} \usepackage{amsmath} \usepackage{amssymb}  \usepackage[noadjust,sort]{cite}

\usepackage{array,arydshln}

\DeclareSymbolFont{bbold}{U}{bbold}{m}{n}
\DeclareSymbolFontAlphabet{\mathbbold}{bbold}
\newcommand{\onev}{\mathbbold{1}}

\title{\LARGE \bf
Output Feedback Stabilization of Switched Linear Systems with Limited Information
}
\author{Masashi Wakaiki and Yutaka Yamamoto\thanks{
This work was supported by The Kyoto University Foundation.
M. Wakaiki and Y. Yamamoto are with the Department of Applied Analysis and Complex
Dynamical Systems, Graduate School of Informatics, Kyoto University, Kyoto
606-8501, Japan
(e-mail:{\tt  \ masashiwakaiki@ece.ucsb.edu};
{\tt \ yy@i.kyoto-u.ac.jp}).}}

\newtheorem{theorem}{Theorem}[section]
\newtheorem{corollary}[theorem]{Corollary}
\newtheorem{proposition}[theorem]{Proposition}
\newtheorem{lemma}[theorem]{Lemma}

\newtheorem{definition}[theorem]{Definition}
\newtheorem{assumption}[theorem]{Assumption}
\newtheorem{example}[theorem]{Example}
\newtheorem{problem}[theorem]{Problem}

\newtheorem{remark}[theorem]{Remark}



\usepackage{graphicx}

\usepackage{mathtools}
\mathtoolsset{showonlyrefs=true}
\mathtoolsset{centercolon}

\newcommand{\afterequation}{\vskip 3pt}

\newcommand{\norm}[1]{\lVert #1 \rVert}
\newcommand{\Norm}[1]{\left\lVert #1 \right\rVert}
\newcommand{\abs}[1]{\lvert #1 \rvert}
\newcommand{\Abs}[1]{\left\lvert #1 \right\rvert}
\renewcommand{\Pr}{\mathcal P}
\newcommand{\oneto}[1]{[#1]}


\begin{document}


\maketitle
\thispagestyle{empty}
\pagestyle{empty}
\begin{abstract}                We propose an encoding and control strategy for 
the stabilization of switched systems with limited information, 
supposing that the controller is given for each mode.
Only the quantized output and the active mode of the plant 
at each sampling time are transmitted to the controller. 
Due to switching,
the active mode of the plant may be different from that of the controller
in the closed-loop system.
Hence if switching occurs, 
the quantizer must recalculate a bounded set containing the estimation error
for quantization at the next sampling time.
We establish the global asymptotic stability
under a slow-switching assumption on dwell time and average dwell time.
To this end, we construct multiple discrete-time Lyapunov functions with
respect to the state and the size of the bounded set.
\end{abstract}

\section{Introduction}
Digital devices such as samplers, quantizers, and communication channels
play an indispensable role in low-cost, intelligent control systems.
This has motivated researchers to study control problems 
with limited information
due to sampling and quantization, 
as surveyed in~\cite{Nair2007,Hepsanha2007, Ishii2012}.
On the other hand, many systems encountered in practice have switching among
several modes of operation.
The stabilization problem of switched systems has also been studied 
extensively; see the book \cite{Liberzon2003Book},
the survey~\cite{Shorten2007, Lin2009}, and many references therein.

Both sampling/quantization and switching are
discrete-time dynamics and often appear in control systems simultaneously.
The authors of
\cite{Nair2003, Ling2010, Xiao2010, Xu2013} have studied 
quantized control for Markov jump discrete-time systems.
In \cite{Cetinkaya2012}, the stabilization of
Markov jump systems with uniformly sampled mode information is investigated.
However,  for switched systems with deterministic switching signals, 
most works deal with sampling/quantization and switching separately.
Based on the result in \cite{Liberzon2003Automatica},
our previous work~\cite{WakaikiMTNS2014} has developed 
an output encoding strategy for switched system
under an average dwell-time condition~\cite{Hespanha1999CDC} but
have not considered sampling.

The following difficulty arises from partial knowledge of the switching signal
due to sampling:
Switching can lead to the mismatch of the active modes between
the plant and the controller.
Accordingly, 
we need to prepare for another encoding strategy in case switching occurs.
For the quantization at the next sampling time, an encoding strategy
after a switch happens
must include the estimation of intersample information, e.g., the state behavior
in the sampling interval, from the transmitted data.

For switched systems with sampling and quantization, 
{\em state} feedback stabilization
has been studied under a slow-switching assumption
in \cite{Liberzon2014,Wakaiki2014IFAC}.
By contrast,
we assume that the information on 
the quantized {\em output} and the active mode of the plant
is transmitted to the controller at each sampling time.
The objective of this paper is to
develop an encoding and control strategy
achieving global asymptotic stabilization for given state feedback gains. 
The detection of switching within each sampling interval requires
a dwell-time assumption. On the other hand, we also use an average dwell-time
assumption for the convergence of the state to the origin.

Our proposed method can be seen as the extension of
\cite{Liberzon2014} from state feedback to output feedback
and also
that of \cite{Liberzon2003} from non-switched systems to switched systems.
A data-rate bound derived from our result is that from~\cite{Liberzon2003}
maximized over all the subsystems.



We organize this paper as follows. In Section II, first we show the switched 
linear system and the information structure we consider. After placing 
some basic assumptions, we state the main result.
Section III is devoted to the so-called ``zooming-out'' stage, whose objective
is to measure the output adequately.
In Section IV, we provide the encoding and control strategy 
that makes the state converges to the origin, and 
obtain a bound on the set in which the estimation error can reach when
a switch occurs.
In Section V, we show that the Lyapunov stability is achieved.
and Section VI contains a numerical example. 
Finally we conclude this paper in Section VII.








{\it Notation:~}
Let  be the set of non-negative integers.
For ,  is the largest integer not 
greater than .

Let  and  denote 
the smallest and the largest eigenvalue of .
Let  denote the transpose of .

The Euclidean norm of  is
denoted by . 
The Euclidean induced norm of  is defined by
.
For ,
its maximum norm is , and
the corresponding induced norm of  is given by
.



\section{Output Stabilization of Switched Systems with
Limited Information}
\subsection{Switched Systems and Information Structure}
Consider the switched linear system

where  is the state,
 is the control input, and
 is the output.
For a finite index set ,
the function  is right-continuous and
piecewise constant.
We call  {\em switching signal} and
the discontinuities of  {\em switching times}.
Let  stand for their number in the interval . 


To generate the control input , we can use
the following information on the output  and the switching signal :

{\sl Sampling:~}
Let  be the sampling period.
The output  and the switching signal  are measured only at
sampling times
 .

{\sl Quantization:~}
Pick an odd positive number .
The measured output  is encoded by an integer in
.
This encoded output and the sampled switching signal 
are transmitted to the controller.

For the Lyapunov stability in Section V, we take  to be odd.
Fig.~\ref{fig:SSSQOF} shows the closed loop we consider.
 \begin{figure}[tb]
 \centering
 \includegraphics[width = 7cm,clip]{Closed_loop.pdf}
 \caption{Sampled-data switched system with quantized output feedback}
 \label{fig:SSSQOF}
 \end{figure}


\subsection{Main Result}
Our first assumption is the stabilizability and observability of each subsystem.
\begin{assumption}
\label{ass:system}
{\em
For every ,  is stabiliable and

is observable.
We choose  so that
 is Hurwitz.
For all . 
the sampling time  is not pathological.
}
\end{assumption}
The non-pathological sampling time implies that
 is observable in the discrete-time sense,
which is used for state reconstruction.


Next we assume that the switching signal  has
the following slow-switching properties:
\begin{assumption}
\label{ass:switching_time}
{\em
{\sl Dwell time:~}
Every interval between two switches is not smaller than
the sampling period . That is, 
 if .

{\sl Average dwell time~\cite{Hespanha1999CDC}:~}
There exist  and  such that

for all .
}
\end{assumption}

Switching signals in
Assumption \ref{ass:switching_time} are called 
{\em hybrid dwell-time} signals~\cite{Vu2011, Liberzon2014}.
The assumption on dwell time is necessary for the
detection of a switch between sampling times,
while that on average dwell time is used in the proof that the state
converges to the origin.

Furthermore, we extend the quantization assumption for 
systems with a single mode in \cite{Liberzon2003} to
switched systems.
\begin{assumption}
\label{ass:quantization}
{\em
Let  be the smallest natural number such that
 defined by

has full column rank.
Let  be a left inverse of .
Then 

for all .
}
\end{assumption}

Assumption \ref{ass:quantization} gives a lower bound on the available data rate
implicitly, and
\eqref{eq:N_condition} is the data-rate bound from \cite{Liberzon2003} 
maximized over the individual modes.
This assumption is used for finer quantization when a switch does not occur.
Note that as  becomes small,
 and 
converge to  and  respectively, but that  does not have full column 
rank in general
if .
Therefore
the left side of \eqref{eq:N_condition} may not decrease 
as  tends to zero.

If , then  and .
Hence \eqref{eq:N_condition} is consistent to
the data rate assumption in the state feedback case \cite{Liberzon2014}.


The main result shows that
global asymptotic stabilization is possible if
the average dwell time is sufficiently large.
\begin{theorem}
\label{thm:stability_theorem2}
{\em
Consider the switched system \eqref{eq:SLS}, and let 
Assumptions \ref{ass:system}, \ref{ass:switching_time}, and 
\ref{ass:quantization} hold.
If the average dwell time  in \eqref{eq:ADT_cond} 
is larger than a certain value, then
there exists an output encoding that achieves the following stability
for every  and every :

\noindent
{\sl Convergence to the origin:~}


\noindent
{\sl Lyapunov stability:~}
To every , there corresponds  such that
if , then 
for .
}
\end{theorem}

A constructive proof of Theorem \ref{thm:stability_theorem2} is given 
in the next sections.
We obtain a sufficient condition \eqref{eq:ADT_final_condition} on 
in the proof.
As in \cite{Liberzon2003, WakaikiMTNS2014, Liberzon2014},
we show the convergence to the origin 
by dividing the proof into the ``zooming-out'' stage and
the ``zooming-in'' stage. 



\section{Zooming-out Stage}


The objective of the ``zooming-out'' stage is to generate 
an upper bound on the estimation error of the state.
We have to obtain such a bound by using the quantized output
and the switching signal at each sampling time.

Define  by




At this stage, we set the control input .
Assume that the average dwell time  satisfies


Pick  and , and define

for . We construct the encoding function  by 

The following theorem is used 
for the reconstruction of the state:
\begin{theorem}
\label{prop:property_of_ADT}
{\em
If the average dwell time  in \eqref{eq:ADT_cond} 
satisfies \eqref{eq:DTC_zo}, then
there exists an integer  such that

for . 
Such  satisfies , where
 depends on  and  in \eqref{eq:ADT_cond}
but not on  itself.
}
\end{theorem}

To prove Theorem \ref{prop:property_of_ADT}, 
we use the following property of average dwell time:
\begin{lemma}[\cite{WakaikiMTNS2014}]
\label{lem:ADT_upperbound}
{\em
Fix an initial time .
Suppose that  satisfies \eqref{eq:ADT_cond}.
Let , and choose an integer  such that

There exists  such that 
.
}
\end{lemma}



\begin{proof}[Proof of Theorem 3.1.]
The growth rate of  in \eqref{eq:j_def_zo} is 
larger than that of  for arbitrary switching. Hence
there is an integer  such that 
for all , which leads to \eqref{eq:zooming_out_QY_SD}.

Let  be an integer satisfying \eqref{eq:N_ADTcond} with 
 in place of .
Since ,
Lemma \ref{lem:ADT_upperbound} shows that

for some .
The interval  contains  sampling times. Thus
we have an integer  satisfying
\eqref{eq:zooming_out_QY_SD} and \eqref{eq:zooming_out_SS_SD}
for , and .
\end{proof}



In conjunction with the dwell-time assumption, 
\eqref{eq:zooming_out_SS_SD} shows that
the active mode of the plant does not change in 
.
We can therefore reconstruct  by using
 in \eqref{eq:WpDef} and the output at 
:

The rest of the procedure is the same as in the non-switched 
case \cite{Liberzon2003}.
Combining
\eqref{eq:zooming_out_QY_SD} and \eqref{eq:x_j0}, 
we obtain

It follows that

Define the estimated state  at  by

Then the error  satisfies

This completes the ``zooming-out'' stage.



\section{Zooming-in Stage}
Here we construct an encoding and control strategy for the
convergence of the state to the origin.
Since the size  of the quantization region increases
after a switch occurs, the term ``zooming-in'' may be misleading.
However, in order to contrast the ``zooming-out'' phase in the
previous section, 
we call the stage in this section the ``zooming-in'' stage as in
\cite{Liberzon2003Automatica,Brockett2000}.

Let  be the initial time of the zooming-in stage
or the time at which the upper bound  of the estimation error is updated. 
Let  and  be the estimated state and the estimation error ,
respectively.
Assume that  and 
.


\subsection{Basic encoding and control method}
If no switch happens, then
we can use the encoding and control method
for systems with a single mode in \cite{Liberzon2003}.
However, after a switch occurs, a modified upper bound on 
the estimation error is needed for the next quantized measurement.
We shall obtain the upper bound in 
Section IV. C. 1).
In this subsection, assuming that the state estimate  and
the error bound 
are derived, we
briefly state the encoding and control method 
because it will be needed in the sequel.

Let .
If no switch occurs in ,
we set , and otherwise we define 
by the minimum integer in the interval  such that 
.
We generate the state estimate  and the output estimate  by

for , and set
the control input 

Since

it follows that  satisfies

If , then

For ,
divide the hypercube 

into  equal boxes and assign a number in 
to each divided box by a certain one-to-one mapping.
The encoder sends to the decoder 
the number  of the divided box containing 
, and then
the decoder generates  equal to the center of the box
with number .
If  lies on the boundary on several boxes, then
we can choose any one of them.


\subsection{Non-switched case}
The calculation of an upper bound  on the estimation error
is dependent of whether 
a switch occurs in an interval with length .
Let us first study the case without switching in the interval
, i.e., the case 
.

\subsubsection{Calculation of an error bound}
An upper bound  on  can be
obtained in the same way as in \cite{Liberzon2003}.
We therefore omit the details of the calculation here. 



Define

From the result in \cite{Liberzon2003}, 
if we appropriately determine  at 
from the transmitted data ,
then we obtain
, where
 is defined by

Note that  for every  
by \eqref{eq:N_condition}.

\subsubsection{Decrease rate of multiple Lyapunov functions}
Here we construct a discrete-time Lyapunov function  of mode  
with respect to  and .
The calculation below 
is similar to that in the state feedback case~\cite{Liberzon2014}, but
we sketch it for completeness.

For simplicity of notation, we write  instead of
.
We obtain an upper bound of  using .

First we obtain  from  and
.
Since

for ,
it follows from \eqref{eq:e_deq} that

where  and  are defined by

Recall that  is Hurwitz by Assumption \ref{ass:quantization}.
To every positive definite matrix , 
there correponds a positive definite matrice  such that

Fix 
for each . Similarly to \cite{Liberzon2014},
define the Lyapunov function  by



Pick .
A simple computation gives
-5pt]
&\qquad \qquad
+ 
\left( \frac{\kappa_p}{\kappa_p -1}
\frac{\|\bar A_p^{\top} P_p \bar B_p\|^2}{\lambda_{\min}(Q_p)} +
\| \bar B_p^{\top}P_p \bar B_p \|
\right)  | e(k_0\tau_s)|^2 \notag \\
&\quad \leq
-\alpha_p |x(k_0\tau_s) |^2 
 +\beta_p E_{k_0}^2,
\label{eq:x_Lyapnov}

\alpha_p &= \frac{1}{\kappa_p} \lambda_{\min}(Q_p) \\
\beta_p &= {\sf n}
\left( \frac{\kappa_p}{\kappa_p -1}
\frac{\|\bar A_p^{\top} P_p \bar B_p\|^2}{\lambda_{\min}(Q_p)} +
\| \bar B_p^{\top}P_p \bar B_p \|
\right).

V_p(k_0+\eta_p)
&\leq
\left(
1 - \frac{\alpha_p}{\lambda_{\max}(P_p)}
\right) x(k_0\tau_s)^{\top}P_p x(k_0\tau_s) 
+ \left(\frac{\beta_p}{\rho_p} + \theta_p^2 \right) \rho_p E_{k_0}^2.

\label{eq:rho_cond}
\rho_p > \frac{\beta_p}{1 - \theta_p^2}.

\label{eq:nu_p_def}
\nu_p := 
\max
\left\{
1 - \frac{\alpha_p}{\lambda_{\max}(P_p)},~~
\frac{\beta_p}{\rho_p} + \theta_p^2
\right\} < 1,

\label{eq:Lyapunov_nonswitched}
V_p(k_0+\eta_p) \leq \nu_p V_p(k_0).

&H_{p,q} = (A_q - A_p) + (B_q - B_p)K_p \label{eq:Hpq} \\
&\bar \delta_{p,q}(k) = \max_{0\leq \tau \leq \tau_s}
\left\|
\int^{\tau_s}_{\tau}
e^{A_q(\tau_s - t)}H_{p,q} e^{(A_p+B_pK_p)(t+(k-1)\tau_s )} dt 
\right\|_{\infty} \notag\\
&\bar \gamma_{p,q}'(k) = \max_{0\leq \tau \leq \tau_s}
\left\| e^{A_q(\tau_s - \tau)} e^{A_p((k-1)\tau_s + \tau)}
 \right\|_{\infty}. \notag

|e((k_0+k) \tau_s)|_{\infty}
\leq
\bar \delta_{p,q}(k) |\xi (k_0\tau_s)|_{\infty} 
+ \bar \gamma_{p,q}'(k) E_{k_0} =: E_{k_0+k}. \label{eq:Ek0_k_def}

e(T) = e^{A_p(T-k_0\tau_s)} e(k_0\tau_s).

\label{eq:x_de_switch}
\dot x = A_qx + B_qK_p \xi,

\dot e 
= A_q e + H_{p,q} \xi \label{eq:e_eq_after_switch}

\label{eq:xi_with_switch}
\xi (t) = e^{(A_p+B_pK_p)(t-k_0\tau_s)} \xi(k_0\tau_s)

e((k_0+k) \tau_s) 
=
e^{A_q(\tau_s - \tau)} e^{A_p((k-1)\tau_s + \tau)} e(k_0\tau_s)+
\int^{\tau_s}_{\tau}
e^{A_q(\tau_s - t)}H_{p,q} e^{(A_p+B_pK_p)(t+(k-1)\tau_s )} dt 
\cdot \xi(k_0\tau_s), 

\dot \zeta = A_p \zeta + B_pu,\qquad \zeta(k_0\tau_s) = \zeta_0.

\dot e_{\zeta} = 
A_q e_{\zeta} + (A_q - A_p) \zeta + (B_q - B_p) K_p \xi,

e_{\zeta}((k_0+\eta_p)\tau_s) = 
F_{1}(T) e_{\zeta}(k_0\tau_s) + 
F_{2}(T) \zeta_0 +
F_{3}(T) \xi(k_0\tau_s)

\xi((k_0+\eta_p)\tau_s) := \zeta((k_0+\eta_p)\tau_s),

E_{k_0+\eta_p} = 
\max_{T \in I_{0}}
\|F_1(T)\|_{\infty} \cdot E_{\zeta} + 
\max_{T \in I_{0}}
\|F_2(T)\|_{\infty} \cdot |\zeta_0|_{\infty} +
\max_{T \in I_{0}}
\|F_3(T)\|_{\infty} \cdot |\xi(k_0\tau_s)|_{\infty},

\label{eq:e_bound_mismatch}
E_{k_0+k} 
\leq \bar \delta_{p,q}(k)|x (k_0\tau_s)| + 
\bar \gamma_{p,q}(k) E_{k_0}.

|\xi(k_0\tau_s)|_{\infty} \leq  |x(k_0\tau_s)|_{\infty} + |e(k_0\tau_s)|_{\infty}
\leq |x(k_0\tau_s)| + E_{k_0}.

\bar \alpha_{p,q}(k) &=
\max_{0 \leq \tau \leq \tau_s}
\left\|
e^{A_q(\tau_s - \tau)}e^{(A_p+B_pK_p)\tau(k)}
+ 
\int^{\tau_s }_{\tau}
e^{A_q(\tau_s  - t)}
B_qK_p e^{(A_p+B_pK_p) (t + (k-1)\tau_s)} dt
\right\|.  \\
\bar \beta_{p,q} (k)
&=
\sqrt{\sf n} 
\max_{0 \leq \tau \leq \tau_s}
\biggr\|
e^{A_q(\tau_s - \tau)}
\int^{\tau(k)}_{0}
e^{(A_p+B_pK_p)(\tau(k) - t)}
B_pK_p e^{A_p t} dt \notag \\
&\qquad \qquad \qquad \qquad \qquad -
\int^{\tau_s }_{\tau}
e^{A_q(\tau_s  - t)}
B_qK_p e^{(A_p+B_pK_p) (t + (k-1)\tau_s)} dt
\biggr\|.

\label{eq:x_bound_mismatch}
|x((k_0+k)\tau_s)| 
\leq \bar \alpha_{p,q}(k) |x(k_0\tau_s)| +
\bar \beta_{p,q}(k) E_{k_0}.

\label{eq:xT}
x(T)
=
e^{(A_p+B_pK_p)\tau(k)} x(k_0\tau_s) +\int^{\tau(k)}_{0}
e^{(A_p+B_pK_p)(\tau(k) - t)}
B_pK_p e^{A_p t} dt \cdot
e(k_0\tau_s). 

x((&k_0+k)\tau_s) 
=
e^{A_q(\tau_s - \tau)} x(T) +
\int^{\tau_s }_{\tau}
e^{A_q(\tau_s  - t)}
B_qK_p e^{(A_p+B_pK_p) (t +(k-1)\tau_s)} dt  \cdot
(x(k_0\tau_s) - e(k_0\tau_s)), \label{eq:xk0_k_tau_s}

V_q(k_0+k) 
\leq
\frac{
2(\lambda_{\max}(P_q) \bar \alpha_{p,q}^2 +
\rho_q \bar \delta_{p,q}^2 )}{\lambda_{\min}(P_p)}
x(k_0\tau_s)^{\top}P_q x(k_0\tau_s) +
\frac{2(\lambda_{\max}(P_q)\bar \beta_{p,q}^2
+ \rho_q  \bar \gamma_{p,q}^2) }{\rho_p}
\rho_p E_{k_0}^2. \label{eq:Vq_K0+k}

\label{eq:Lyapunov_switched}
V_q(k_0+k) \leq \bar \nu_{p,q}(k) V_p(k_0),

\bar \nu_{p,q}(k) &= 
\max
\biggl\{
\frac{
2(\lambda_{\max}(P_q) \bar \alpha_{p,q}(k)^2 +
\rho_q \bar \delta_{p,q}(k)^2) }{\lambda_{\min}(P_p)},~~
\frac{2(\lambda_{\max}(P_q)\bar \beta_{p,q}(k)^2
+ \rho_q  \bar \gamma_{p,q}(k)^2) }{\rho_p}
\biggr\}.

\label{eq:nu_bnu_def}
\nu = \max_{p \in \mathcal{P}} \nu_p,\quad
\bar \nu = \max_{p\not=q} \max_{1 \leq k \leq \eta_p}\bar \nu_{p,q}(k).

\label{eq:ADT_final_condition}
\tau_a > \left( 1+
\frac{ \log \bar \nu}{\log(1/\nu)} 
\right)\eta\tau_s,

\psi_i = \left\lfloor \frac{k_i - k_{i-1} - 1}{\eta_{\sigma(k_{i-1}\tau_s)}} \right\rfloor, \quad
\ell_i = k_{i-1} + \psi_i \eta_{\sigma(k_{i-1}\tau_s)}

\label{eq:ceiling_func_property}
\frac{k-n+1}{n} \leq  \left\lfloor \frac{k}{n}\right\rfloor
\leq \frac{k}{n}

\label{eq:V_switching}
V_{\sigma(k_i\tau_s)}(k_i) 
\leq \bar \nu V_{\sigma(\ell_i\tau_s)}(\ell_i )
\leq \bar \nu \nu^{\psi_i}V_{\sigma(k_{i-1}\tau_s)}(k_{i-1})

\psi_{r+1}= \left\lfloor \frac{M- k_{r} }{\eta_{\sigma(k_{r}\tau_s)}} \right\rfloor, \quad
\ell_{r+1} = k_{r} + \psi_{r+1} \eta_{\sigma(k_{r}\tau_s)}.

\label{V_final}
V_{\sigma (M\tau_s)}(M) 
\leq \hspace{-1.5pt} 
\hat{\nu} \nu^{\psi_{r+1}}V_{\sigma (k_r \tau_s)}(k_r)
~\text{for some .}

\label{eq:c_def}
\psi = \sum_{i=1}^{r+1} \psi_i.

\label{eq:V_N_bound}
V_{\sigma(M\tau_s)}(M) \leq \hat \nu \bar \nu^{r} \nu^{\psi}V_{\sigma(k_0\tau_s)}(k_0).

\psi
&\geq 
\frac{M - k_0 + 1}{\eta} - (r+1), \label{eq:psi_inequality}

V_{\sigma(M\tau_s)}(M) \leq 
\hat \nu \nu^{1/\eta - 1}  \cdot
\bar \nu^r \nu^{(M-k_0)/\eta - r}
V_{\sigma(k_0\tau_s)}(k_0).

\bar \nu^r \nu^{(M-k_0)/\eta - r}
\leq
\bar \nu^{N_0}\nu^{-N_0}
\cdot
\left(
\bar \nu^{\tau_s / \tau_a}
\nu^{1/\eta - \tau_s/ \tau_a}
\right)^{M- k_0}.

\lim_{M\to \infty}x(M\tau_s) = \lim_{M\to \infty}\xi(M\tau_s) =
0.

|x(M\tau_s + \tau)| \leq L_1 |x(M\tau_s)|
+ L_2 |\xi(M\tau_s)|
\qquad (0 <  \tau < \tau_s)

\sigma(t) = \sigma(n_0\tau_s) =: p

C_{\max} e^{\max_{p \in \mathcal{P}} \|A_p\|_{\infty} 
(m\eta - 1)\tau_s} \delta < \mu_0,

|x(n_0\tau_s)|_{\infty} \leq E_{n_0} \leq 
\max_{p \in \mathcal{P}} 
\|W^{\dagger}_p\|_{\infty} \cdot \mu_{m\eta - 1}
=:\bar{E}_0.

|x((n_0+\eta_p)\tau_s)|_{\infty} &\leq
e^{\max_{p \in \mathcal{P}} \|A_p\|_{\infty} \tau_s} 
\left\|e^{A_p(\eta_p-1)\tau_s} \right\|_{\infty} 
E_{n_0} (=: E_{n_0+\eta_p}) \\
&\leq 
e^{\max_{p \in \mathcal{P}} \|A_p\|_{\infty} \tau_s} \cdot
\max_{p \in \mathcal{P}} 
\left\|e^{A_p(\eta_p-1)\tau_s}
\right\|_{\infty} \cdot \bar{E}_0
=: \bar E.

\Lambda_{\max} = \max_{p\in \mathcal{P}} (
{\sf n} \lambda_{\max}(P_p) + \rho_p).

\label{eq:V_bound_forLyapnouvS}
V_{\sigma(M\tau_s )}(M) \leq \hat \nu \bar \nu^{N_{\sigma}(M\tau_s,k_0\tau_s)} 
\nu^{\psi}\Lambda_{\max} \bar E^2.

\label{eq:x_bound_Lyapunov_afterT_0}
|x(t)|_{\infty} < \varepsilon \qquad  (t \geq T_0).

\label{eq:x_bound_Lyapunov_beforeT_0}
|x(t)|_{\infty} < \varepsilon \qquad  (t \leq T_0),

C_{\max} e^{\max_{p \in \mathcal{P}} \|A_p\|_{\infty} T_0} \delta &\leq 
\frac{1}{\sf p}
\min_{p \in \mathcal{P}} \min_{0\leq k \leq \eta_p-1} 
\| C_p e^{A_pk\tau_s}\|_{\infty} \cdot
(\min_{p \in \mathcal{P}} \theta_p)^{\lfloor T_0 / \min_{p \in \mathcal{P}}
\eta_p \rfloor} \\
&\qquad
\times \min_{p \in \mathcal{P}}\left( 
\left\|e^{A_p(\eta_p - 1)\tau_s}\right\|_{\infty} \cdot \|W^{\dagger}_p\|_{\infty}
\right)\mu_0,

e^{\max_{p \in \mathcal{P}} \|A_p\|_{\infty} T_0} \delta 
<\varepsilon,

A_1 = 
\begin{bmatrix}
0 & -1 \\ -1 & -2
\end{bmatrix},\quad
&B_1 = 
\begin{bmatrix}
1 \\ -1
\end{bmatrix},\quad
C_1 = 
\begin{bmatrix}
1 & 1
\end{bmatrix}\\
A_2 = 
\begin{bmatrix}
1 & 2 \\ -2 & -1
\end{bmatrix},\quad
&B_2 = 
\begin{bmatrix}
-2 \\ 1
\end{bmatrix},\quad
C_2 = 
\begin{bmatrix}
1 & -1
\end{bmatrix}.

The feedback gains of each mode are
 and .
Note that  and 
have an unstable pole  and ,
respectively.
The sampling period  and
the partition number  of the quantizer are
 and .

We took  and  in \eqref{eq:Lyapnov_equation} and
the parameters of  and  in \eqref{eq:nu_p_def}
as follows: 
, 
, , 
, and . These were chosen by
trial and error.
We see from \eqref{eq:ADT_final_condition} that
if , our encoding and control strategy
achieves the global asymptotic stabilization.

A time response in the interval  was calculated for
, , and .
The switching signal was chosen so that 
the dwell time  and
\eqref{eq:ADT_cond}
holds with  and the average dwell time .
Fig.~\ref{fig:simulation}
shows the Euclidean norm of
the state  and the state estimate . 
In this simulation, the ``zooming-out'' stage finished at  but we observe that 
the system was not controlled until . The reason is that
the state estimate is zero at the end of ``zooming-out'' stage; see \eqref{eq:ES_Initial}.
This leads to no control input in
the initial period of ``zooming-in'' stage as in
the ``zooming-out'' stage.

If the state of the plant is accessible, i.e., , then we see from 
\cite[Assumption 3]{Liberzon2014} that if the number of symbols in the 
quantizer is not smaller than , then 
the encoding and control strategy in \cite{Liberzon2014}
stabilizes the plant.
On the other hand, the counterpart in the output feedback case from
\eqref{eq:N_condition} is .
Hence in this example, if we consider only stabilization
of systems with sufficiently large average-dwell time property, then 
the use of the output with lower dimension than the dimension of the state
has the advantage in terms of data rate.
 \begin{figure}[t]
 \centering
 \includegraphics[width = 7cm,bb= 35 20 700 510,clip]{Ex1_st4.pdf}
 \caption{The Euclidean norm of the state  and the estimated state }
 \label{fig:simulation}
 \end{figure}




\section{Concluding Remarks}
We have studied the problem of stabilizing 
a switched linear system with limited information: 
the quantized output and active mode at each sampling time.
We have supposed that the controller is given
and have examined the intersample behavior of the estimation error
for the encoding strategy after the detection of switching.
Using multiple discrete-time Lyapunov functions, we have achieved
global asymptotic stabilization under
the hybrid dwell-time assumption.
The data-rate bound used here is
the maximum among the bounds of the individual subsystems
that are from the earlier work. 


\begin{thebibliography}{10}
\providecommand{\url}[1]{#1}
\csname url@samestyle\endcsname
\providecommand{\newblock}{\relax}
\providecommand{\bibinfo}[2]{#2}
\providecommand{\BIBentrySTDinterwordspacing}{\spaceskip=0pt\relax}
\providecommand{\BIBentryALTinterwordstretchfactor}{4}
\providecommand{\BIBentryALTinterwordspacing}{\spaceskip=\fontdimen2\font plus
\BIBentryALTinterwordstretchfactor\fontdimen3\font minus
  \fontdimen4\font\relax}
\providecommand{\BIBforeignlanguage}[2]{{\expandafter\ifx\csname l@#1\endcsname\relax
\typeout{** WARNING: IEEEtran.bst: No hyphenation pattern has been}\typeout{** loaded for the language `#1'. Using the pattern for}\typeout{** the default language instead.}\else
\language=\csname l@#1\endcsname
\fi
#2}}
\providecommand{\BIBdecl}{\relax}
\BIBdecl

\bibitem{Nair2007}
G.~N. Nair, F.~Fagnani, S.~Zampieri, and R.~J. Evans, ``Feedback control under
  data rate constraints: An overview,'' \emph{Proc. IEEE}, vol.~95, pp.
  108--137, 2007.

\bibitem{Hepsanha2007}
J.~P. Hespanha, P.~Naghshtabrizi, and Y.~Xu, ``{A survey of rrecent results in
  networked control systems},'' \emph{Proc. IEEE}, vol.~95, pp. 138--162, 2007.

\bibitem{Ishii2012}
H.~Ishii and K.~Tsumura, ``{Data rate limitations in feedback control over
  network},'' \emph{IEICE Trans. Fundamentals}, vol. E95-A, pp. 680--690, 2012.

\bibitem{Liberzon2003Book}
D.~Liberzon, \emph{Switching in Systems and Control}.\hskip 1em plus 0.5em
  minus 0.4em\relax Birkh\"auser, Boston, 2003.

\bibitem{Shorten2007}
R.~Shorten, F.~Wirth, O.~Mason, K.~Wulff, and C.~King, ``{Stability criteria
  for switched and hybrid systems},'' \emph{SIAM Review}, vol.~49, pp.
  545--592, 2007.

\bibitem{Lin2009}
H.~Lin and P.~J. Antsaklis, ``{Stability and stabilizability of switched linear
  systems: a survey of recent results},'' \emph{IEEE Trans. Automat. Control},
  vol.~54, pp. 308--322, 2009.

\bibitem{Nair2003}
G.~N. Nair, S.~Dey, and R.~J. Evans, ``Infinmum data rates for stabilising
  markov jump linear systems,'' in \emph{Proc. 42nd IEEE CDC}, 2003.

\bibitem{Ling2010}
Q.~Ling and H.~Lin, ``Necessary and sufficient bit rate conditions to stabilize
  quantized markov jump lineaer systems,'' in \emph{Proc. ACC 2010}, 2010.

\bibitem{Xiao2010}
N.~Xiao, L.~Xie, and M.~Fu, ``{Stabilization of Markov jump linear systems
  using quantized state feedback},'' \emph{Automatica}, vol.~46, pp.
  1696--1702, 2010.

\bibitem{Xu2013}
Q.~Xu, C.~Zhang, and G.~E. Dullerud, ``{Stabilization of Markovian jump linear
  systems with log-quantized feedback},'' \emph{J. Dynamic Systems, Meas,
  Control}, vol. 136, pp. 1--10 (031\,919), 2013.

\bibitem{Cetinkaya2012}
A.~Cetinkaya and T.~Hayakawa, ``{Feedback control of switched stochastic
  systems using uniformly sampled mode information},'' in \emph{ACC 2012},
  2012.

\bibitem{Liberzon2003Automatica}
D.~Liberzon, ``Hybrid feedback stabilization of systems with quantized
  signals,'' \emph{Automatica}, vol.~39, pp. 1543--1554, 2003.

\bibitem{WakaikiMTNS2014}
M.~Wakaiki and Y.~Yamamoto, ``{Quantized output feedback stabilization of
  switched linear systems},'' in \emph{Proc. MTNS 2014}, 2014.

\bibitem{Hespanha1999CDC}
J.~P. Hespanha and A.~S. Morse, ``Stability of swithched systems with average
  dwell-time,'' in \emph{Proc. 38th IEEE CDC}, 1999.

\bibitem{Liberzon2014}
D.~Liberzon, ``Finite data-rate feedback stabilization of switched and hybrid
  linear systems,'' \emph{Automatica}, vol.~50, pp. 409--420, 2014.

\bibitem{Wakaiki2014IFAC}
M.~Wakaiki and Y.~Yamamoto, ``{Quantized feedback stabilization of sampled-data
  switched linear systems},'' in \emph{Proc. 19th World Congress of IFAC},
  2014.

\bibitem{Liberzon2003}
D.~Liberzon, ``On stabilization of linear systems with limited information,''
  \emph{IEEE Trans. Automat. Control}, vol.~48, pp. 304--307, 2003.

\bibitem{Vu2011}
L.~Vu and D.~Liberzon, ``Supervisory control of uncertain linear time-varying
  systems,'' \emph{IEEE Trans. Automat. Control}, vol.~56, pp. 27--42, 2011.

\bibitem{Brockett2000}
R.~W. Brockett and D.~Liberzon, ``Quantized feedback stabilization of linear
  systems,'' \emph{IEEE Trans. Automat. Control}, vol.~45, pp. 1279--1289,
  2000.

\end{thebibliography}

\end{document}
