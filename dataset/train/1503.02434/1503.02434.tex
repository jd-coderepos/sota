\documentclass[a4paper,english,10pt]{article}

\usepackage[utf8]{inputenc}
\usepackage[a4paper]{geometry}
\usepackage{cite}
\usepackage{enumerate,hyperref,nameref}
\usepackage[inline]{enumitem}
\usepackage[algo2e,noline,tworuled]{algorithm2e}
\usepackage{amsmath,amsthm,amsfonts,amssymb,stmaryrd,mathtools,esvect}
\usepackage{tikz,graphicx}
\usetikzlibrary{backgrounds,calc,fit,shapes,positioning,arrows,intersections}

\theoremstyle{plain}\newtheorem{theorem}{Theorem}
\theoremstyle{plain}\newtheorem{corollary}[theorem]{Corollary}
\theoremstyle{plain}\newtheorem{proposition}[theorem]{Proposition}
\theoremstyle{plain}\newtheorem{lemma}[theorem]{Lemma}
\theoremstyle{plain}\newtheorem{definition}{Definition}
\theoremstyle{plain}\newtheorem{remark}{Remark}
\theoremstyle{plain}\newtheorem{example}[remark]{Example}

\makeatletter
\let\orgdescriptionlabel\descriptionlabel
\renewcommand*{\descriptionlabel}[1]{\let\orglabel\label
  \let\label\@gobble
  \phantomsection
  \edef\@currentlabel{\ignorespaces #1\unskip}\let\label\orglabel
  \orgdescriptionlabel{#1}}
\makeatother

\DeclareSymbolFont{symbolsC}{U}{txsyc}{m}{n}
\SetSymbolFont{symbolsC}{bold}{U}{txsyc}{bx}{n}
\DeclareFontSubstitution{U}{txsyc}{m}{n}
\DeclareMathSymbol{\multimapboth}{\mathrel}{symbolsC}{"13}

\newcommand{\?}[1]{}

\newcommand{\defeq}{\triangleq}
\newcommand{\defiff}{\stackrel{\triangle}{\iff}}
\newcommand{\mono}{\rightarrowtail}
\newcommand{\epi}{\twoheadrightarrow}
\newcommand{\emb}{\mathbin{\begin{tikzpicture}[baseline=-.6ex]\draw[right hook->] (0,0) -- (2ex,0);\end{tikzpicture}}}
\newcommand{\pemb}{\mathbin{\begin{tikzpicture}[baseline=-.6ex]\draw[right hook-left to] (0,0) -- (2ex,0);\end{tikzpicture}}}
\newcommand{\pmap}{\rightharpoonup}
\newcommand{\face}[1]{\langle #1 \rangle}
\newcommand{\dom}{\textsf{\textit{dom}}}
\newcommand{\cod}{\textsf{\textit{cod}}}
\newcommand{\rng}{\textsf{\textit{img}}}
\newcommand{\ar}{\textsf{\textit{ar}}}
\newcommand{\prnt}{\textsf{\textit{prnt}}}
\newcommand{\ctrl}{\textsf{\textit{ctrl}}}
\newcommand{\link}{\textsf{\textit{link}}}
\newcommand{\src}{\mathsf{src}}

\newcommand{\mbb}[1]{\mathbb{#1}}
\newcommand{\mcl}[1]{\mathcal{#1}}
\newcommand{\mfk}[1]{\mathfrak{#1}}
\newcommand{\mtt}[1]{\mathtt{#1}}
\newcommand{\mrm}[1]{\mathrm{#1}}
\newcommand{\msf}[1]{\mathsf{#1}}

\newcommand{\esf}[2]{#1^\mathsf{#2}}
\newcommand{\ephi}[1]{\esf{\phi}{#1}}

\newcommand\restr[2]{{\left.\kern-\nulldelimiterspace#1\vphantom{\big|}\right|_{#2}}}
\newcommand\corestr[2]{{\left.\kern-\nulldelimiterspace#1\vphantom{\big|}\right|^{#2}}}

\newcommand{\prt}{\mbb}
\newcommand{\eprt}[1]{\prt P^\mathsf{\kern1pt#1}}
\newcommand\embrestr[3][\?]{{\left.\kern-\nulldelimiterspace#3\vphantom{\big|}\right|_{#1,#2}}}
\newcommand{\nbrel}{\multimapboth}
\newcommand{\adjto}{\mathrel{{\circ}\!{\to}}}

\hyphenation{bi-graph bi-graphs bi-graph-i-cal}
\hypersetup{
	colorlinks=true,
	linkcolor=black,
	citecolor=black,
	filecolor=black,
	urlcolor=black,
	pdftitle={Distributed execution of bigraphical reactive systems},
	pdfauthor={Alessio Mansutti and Marino Miculan and Marco  Peressotti},
	pdfsubject={Concurrency Theory},
	pdfkeywords={Concurrent and distributed graph transformations, Models of graph transformation, Formal graph languages, Multi-agent systems}
}

\title{Distributed execution of bigraphical reactive systems\thanks{
	This work is partially supported by MIUR PRIN project 2010LHT4KM, \emph{CINA}.}}

\author{
	\begin{tabular}{ccccc}
	Alessio Mansutti&\qquad& Marino Miculan&\qquad& Marco  Peressotti\\
	\small\href{mailto:alessio.mansutti@gmail.com}{\tt alessio.mansutti@gmail.com}
	&&
	\small\href{mailto:marino.miculan@uniud.it}{\tt marino.miculan@uniud.it}
	&&
	\small\href{mailto:marco.peressotti@uniud.it}{\tt marco.peressotti@uniud.it}
	\end{tabular}\-.8ex]
	\small	Department of Mathematics and Computer Science\
		(V_G, E_G, \ctrl_G, \prnt_G, \link_G):\face{n_G,X_G}\to\face{m_G,Y_G}
	
	\label{eq:def-emb-ord}
	\phi \sqsubseteq \psi \defiff \forall x \in \dom(\phi)\, 
	\phi(x) \neq \perp \implies \psi(x) = \phi(x)
\text{.}
	\phi \sqcap \psi \defeq \lambda x.
	\begin{cases}
		\phi(x) & \text{if } \phi(x) = \psi(x) \\
		\perp & \text{otherwise} 
	\end{cases}

	\phi \sqcup \psi \defeq \lambda x.
	\begin{cases}
		\phi(x) & \text{if } \phi(x) \neq \perp \\
		\psi(x) & \text{if } \psi(x) \neq \perp \\
		\perp &  \text{otherwise} 
	\end{cases}

		\psi \sqsubseteq \alpha 
		\implies 
		\psi = \varnothing 
		\lor
		\exists!s \in n_G \uplus X_G.\psi(s) = \emptyset
		\text{.}
	\tag{}\label{def:nlge-5a}
		(x \in \link^{-1}_G(e) \land 
		\esf{\rho}{p}(x)\not\subseteq 
		\link^{-1}_H(\esf{\rho}{e}(e))) \lor (x \not\in \link^{-1}_G(e) \land \esf{\rho}{p}(x)\subseteq \link^{-1}_H(\esf{\rho}{e}(e)))
	
	At\left(\embrestr{\{Q\}}{\phi}\right) = \left\{
		\alpha \in At 
	\,\middle|\, 
		\alpha \sqsubseteq \embrestr{\{Q\}}{\phi}
	\right\}

	\alpha \equiv \beta 
	\defiff 
	\alpha \cong \beta 
	\text{ and }
	\forall \gamma \in At_{G,H}\setminus\{\alpha,\beta\}
	\alpha \sqcup \gamma \iff \beta \sqcup \gamma

where  whenever there are two
bigraph isomorphisms  and  s.t.~.
It is easy to check that this relation is an equivalence
and hence defines quotients for atom graphs i.e.~an effective
compression for messages and, in general, structures based
on atom graphs.
A lossless compression requires atoms bo be decorated with their
multiplicity (and any list of additional user provided
properties often found in some extensions of bigraphs).

\subsection{Adequacy}
Reactions change the current bigraph and can be though as resetting
the embedding engine with the latter then checking and updating its
state coherently.
Reworded, reactions are perturbations the embedding engine has to 
stabilize from and restoring the equilibrium produces traffic over 
the network. Traffic stops only when the equilibrium is reached 
i.e.~the machine stabilizes.
\begin{theorem}[Completeness]
	When the system is stable, every embedding can
	be found in the collection of available embeddings
	of some process.
\end{theorem}

By causally ordered communication we can assume, w.l.o.g.,
that the system stabilized before the last reaction. Then
completeness is equivalent to the fact that for each 
there is some  s.t.~
where  is the set of partial embeddings
whose atoms are in 
\begin{lemma}
	If  is a partial embedding for  then 
	there is a process  s.t.~.
\end{lemma}
\begin{proof}
	The proof is given by induction on the size of
	.
	If  then the embedding is local to  and hence
	.
	Otherwise, let  for
	. By inductive hypothesis each
	.
	By connectedness hypothesis there is at least one process 
	 reachable by each .
	Messages are routed to all, and only, 
	the processes that can benefit from or contribute to them,
	in particular to . All edges that are locally checked
	and ancestor checked are added while messages travel the network.
	We only have to prove that there is always a process that can 
	add each edge along the paths to . By Theorem~\ref{thm:small-checks}, the only cases left are ancestor checkable. We conclude by
	Definition~\ref{def:adj-routing} and by
	Corollary~\ref{cor:ancestor-checkable}.
\end{proof}

\begin{lemma}
	\label{lem:ancestor-checkable}
	Let , , 
	,  two
	atoms, and  be the roots above  and  respectively. 
	If  is the process to receive/compute  and  earlier then
	at least one of the following is true:
	\begin{enumerate}[label=(\alph*)]\itemsep=0pt
		\item
				holds the least ancestor of  and ;
		\item
			 holds both  and ;
		\item
			 holds either  or  and the
			process holding the other sent the embedding.
	\end{enumerate}
	Let , , ,
	and . There is a process  
	that holds  and an ancestor of .
\end{lemma}
\begin{proof}[Proof (Sketch)]
	Atoms for guest sites and roots are dispatched following  only.
	Atoms for host ports are dispatched following both  and .
\end{proof}

\begin{corollary}[Ancestor checks]
	\label{cor:ancestor-checkable}
	For any two ancestor checkable atoms involving host ports, guest roots or sites there is a process that computes their edge before the system
	stabilize.
\end{corollary}
\begin{proof}[Proof (Sketch)]
	The process receiving/computing the atoms for guest sites and roots
	earlier checks them by looking at his piece of the shared bigraph
	and at the adjacency witness used to dispatch the message (i.e.~which child or sibling root was used by the sender process to route the message).
	Likewise, a process holding the image of a site checks whether a received
	inner name sits below it.
\end{proof}

\begin{theorem}[Soundness]
	If the system stabilizes then each embedding
	in the collection of available ones is valid w.r.t.~the
	current bigraph.
\end{theorem}
\begin{proof}
	Effects of reactions are computed locally to each 
	embedding engine and then propagated through the network.
	Propagation stops as soon as it stops producing changes
	in each . By network connectedness and stabilization
	of the machine each invalid	embedding is eventually computed
	and removed by \hyperref[algo:retract]{\withhold}. Embeddings are added only by
	\hyperref[algo:suggest]{\publish} which filters out candidate
	and partial embeddings.
\end{proof}

\subsection{Complexity} 

The arity of the set of all embeddings of  into  is 
in  since, in the worst case, guest and host
encode two finite sets with a root for each element. On the other hand,
by Proposition~\ref{prop:almost-atomic}, the same set is described by
families in  or, following the representation used by the algorithm,
by a suitable graph on . Because elements of 
are essentially pairs from  the spatial
complexity of the graph representation is in 
without any particular encoding. The same bound holds for the size of
each message sent on the overlay network. 
However, a process sends over the network
only nodes and edges it adds or removes from his  and messages
are dispatched on the base of their semantic adjacency. Therefore,
between two reactions, every edge travels a link at most once
(either inside a suggest or retract message). 

\begin{lemma}
	The number of links in   is in .
\end{lemma}
\begin{proof}
	The number  of links in  is bounded
	by the size of  since the  is a quotient
	of . Hence, the worst case network is 
	where  is the finest possible
	partition (i.e~each component is assigned a distinct process).
Except for the clique induced by roots and handles,
	 is a directed acyclic graph where
	each vertex has at most 
	 
	outgoing edges and therefore is bounded by the maximal arity 
	occurring the given signature  which is a fixed parameter
	of the D-BAM, hence a constant.
	The remaining case is given by the clique of roots and handles;
	their outgoing degree may exceed  but their topology 
	can be easily reorganized to into a tree that satisfies the 
	bound and the above reasoning.
	Therefore,  is bounded by the number of components of .
\end{proof}

The algorithm generates, in the \emph{worst case scenario}, as much
traffic as a centralized one in its \emph{best case scenario}.
\begin{theorem}
	\label{thm:traffic}
	The traffic generated over  while finding all the 
	available embedding,
	between two reactions, is in .
\end{theorem}
This scenario corresponds to bigraphs and partitions forcing information to
traverse all the network. In fact, the algorithm sends atoms only to
processes that can effectively benefit from it and hence their propagation
is stopped as soon as possible while retaining correctness and completeness.

In a typical scenario guests are fixed over time (hence a constant) and
 outmatches  by orders of magnitude. Moreover, embeddings
unaffected by a reactions are not recomputed.

\section{Conclusions and future work}\label{sec:concl}

In this paper we have presented a D-BAM, an abstract machine for
executing BRSs in a distributed environment.  The core novelty of
this machine is an algorithm for computing bigraph embeddings in a
distributed environment where the host bigraph is spread across
several cooperating peers.  Differently from existing algorithms
\cite{gdbh:implmatch,mp:memo14,sevegnani2010sat}, this one is
completely decentralized and does not to have a complete view of the
global state in any process in the system; hence it can scale to
handle bigraphs too large to reside on a single process/machine.

As in any distributed system, the complexity of our algorithm is
rendered by the number and the size of exchanged messages (i.e., the
\emph{network footprint}).  On one hand, the number of messages needed
for computing an embedding is linearly bounded by the size of the
embedded bigraph (which usually is constant during execution) and the
depth of the parent map of the host.  The worst case
(Theorem~\ref{thm:traffic}) is when the overlay network of processes
is a list, and atoms have to traverse it entirely. This case happens
for bigraphs and embeddings that can be seen as ``pathological'' in
the context of BRS; this suggests to consider different encodings of
the model into the BRS in order to improve locality of reactions.  On
the other hand, the size of messages depends on internal isomorphisms
in the guest and host bigraphs: these symmetries yield a combinatorial
explosion of the possible embeddings, leading to larger messages to be
exchanged between processes. This is mitigated by the heuristics
presented in Section~\ref{sec:heu}.  A possible future work is to
perform a formal analysis of locality and isomorphisms and their
impact in the context of smoothed complexity.

When a reaction is applied, it alters the distributed state and
inherently invalidates some of the partial embeddings computed by each
process.  Consistency of the state is guaranteed by reactions being
wrapped inside distributed transactions, but invalidated embeddings
are an unnecessary burden. To this end, we used a retraction
mechanisms as an \emph{asynchronous distributed garbage collection};
moreover, embeddings that are not affected by a reaction are not
recomputed.  We think that this approach is a good trade-off between
performance and consistency.  In fact, other solutions can be
implemented; for instance, invalidated embeddings can be collected
during the reaction commit phase; this offers the highest consistency
(the set of available embeddings will never contain invalid ones) at
the cost of slower reactions. On the other extreme of the spectrum,
invalidated embeddings are collected only when an inconsistency is
found by some process. Reactions are as fast as in presence of
asynchronous retractions but process data structures are heavily
polluted by invalid embeddings resulting in a higher rate of
aborted transactions i.e.~failed reactions.

An interesting feature of the bigraphical framework is that, given a
bigraph and a redex, we can calculate the \emph{minimal contexts}
(called IPOs)
needed to complete the bigraph in order to match the given redex.
Leveraging this property, a different, ``semi-distributed''
implementation of the bigraphical abstract machine has been proposed
in \cite{mmp:dais14}.  According to this algorithm, a process willing
to perform a rewrite has to
\begin{enumerate*}[label=\em(\arabic*)]
	\item
		collect a (suitable) view of the host
		bigraph from its neighbour processes; 
	\item 
		compute locally all the
		embeddings (i.e.~all possible reactions for 
		the given rewriting rule);
	\item
		apply the execution policy and start 
		a distributed rewriting inside a transaction.
\end{enumerate*} 
The existence of minimal contexts provide a bound to
the view a process has to collect at step 1. However, this bound is
outmatched by more substantial drawbacks, e.g.: parametric rules have
to be expanded into ground ones beforehand, and each process may end
up visiting (and copying) the entire bigraph.  Hence, we think that
the algorithm proposed in this paper outperformes the one in \cite{mmp:dais14}. 

A direct application of the distributed embedding algorithm is to
simulate, or execute, \emph{multi-agent systems}.  In
\cite{mmp:dais14} the authors propose a methodology for designing and
prototyping multi-agent systems with BRSs. Intuitively, the
application domain is modelled by a BRS and entities in its states are
divided as ``subjects'' and ``objects'' depending on their ability to
actively perform actions. Subjects are precisely the agents of the
system and reactions are reconfigurations.  This observation yields a
coherent way to partition and distribute a bigraph among the agents,
which can be assimilated to the processes of the distributed
bigraphical machine (execution policies are defined by agents desires
and goals).  Therefore, these agents can find and perform bigraph
rewritings in a truly concurrent, distributed fashion, by using the
distributed embedding algorithm presented in this paper.

Finally, we observe that the performance of the algorithm (and hence
of the D-BAM) depends on how the bigraph is partitioned and
distributed. For instance, it is easy to devise a situation in which
even relatively small guests require the cooperation of several
processes, say nearly one for each component of the guest. An
interesting line of research would be to study the relation between
guests, partitions, and performance in order to develop efficient
distribution strategies. Moreover, structured partitions lend
themselves to ad-hoc heuristics and optimizations.  As an example, the
way bigraphs are distributed among agents in \cite{mmp:dais14} takes
into account of their interactions and reconfigurations.

{\small
\begin{thebibliography}{10}

\bibitem{bgm:biobig}
G.~Bacci, D.~Grohmann, and M.~Miculan.
\newblock Bigraphical models for protein and membrane interactions.
\newblock In G.~Ciobanu, editor, {\em Proc.~MeCBIC}, volume~11 of {\em
  Electronic Proceedings in Theoretical Computer Science}, pages 3--18, 2009.

\bibitem{bmr:tgc14}
G.~Bacci, M.~Miculan, and R.~Rizzi.
\newblock Finding a forest in a tree - the matching problem for wide reactive
  systems.
\newblock In M.~Maffei and E.~Tuosto, editors, {\em Proc.~TGC}, volume 8902 of
  {\em Lecture Notes in Computer Science}, pages 17--33. Springer, 2014.

\bibitem{bdehn:fossacs06}
L.~Birkedal, S.~Debois, E.~Elsborg, T.~Hildebrandt, and H.~Niss.
\newblock Bigraphical models of context-aware systems.
\newblock In L.~Aceto and A.~Ing{\'o}lfsd{\'o}ttir, editors, {\em
  Proc.~FoSSaCS}, volume 3921 of {\em Lecture Notes in Computer Science}, pages
  187--201. Springer, 2006.

\bibitem{bghhn:coord08}
M.~Bundgaard, A.~J. Glenstrup, T.~T. Hildebrandt, E.~H{\o}jsgaard, and H.~Niss.
\newblock Formalizing higher-order mobile embedded business processes with
  binding bigraphs.
\newblock In D.~Lea and G.~Zavattaro, editors, {\em Proc.~COORDINATION}, volume
  5052 of {\em Lecture Notes in Computer Science}, pages 83--99. Springer,
  2008.

\bibitem{cooper1982:distcommit}
E.~C. Cooper.
\newblock Analysis of distributed commit protocols.
\newblock In {\em Proc.~SIGMOD}, pages 175--183. ACM, 1982.

\bibitem{dhk:fcm}
T.~C. Damgaard, E.~H{\o}jsgaard, and J.~Krivine.
\newblock Formal cellular machinery.
\newblock {\em Electronic Notes in Theoretical Computer Science}, 284:55--74,
  2012.

\bibitem{fph:gcm12}
A.~J. Faithfull, G.~Perrone, and T.~T. Hildebrandt.
\newblock {BigRed}: A development environment for bigraphs.
\newblock {\em ECEASST}, 61, 2013.

\bibitem{gdbh:implmatch}
A.~Glenstrup, T.~Damgaard, L.~Birkedal, and E.~H{\o}jsgaard.
\newblock An implementation of bigraph matching.
\newblock {\em IT University of Copenhagen}, 2007.

\bibitem{hoesgaard:thesis}
E.~H{\o}jsgaard.
\newblock {\em Bigraphical Languages and their Simulation}.
\newblock PhD thesis, IT University of Copenhagen, 2012.

\bibitem{jm:popl03}
O.~H. Jensen and R.~Milner.
\newblock Bigraphs and transitions.
\newblock In A.~Aiken and G.~Morrisett, editors, {\em POPL}, pages 38--49. ACM,
  2003.

\bibitem{kmt:mfps08}
J.~Krivine, R.~Milner, and A.~Troina.
\newblock Stochastic bigraphs.
\newblock In {\em Proc.~MFPS}, volume 218 of {\em Electronic Notes in
  Theoretical Computer Science}, pages 73--96, 2008.

\bibitem{mmp:dais14}
A.~Mansutti, M.~Miculan, and M.~Peressotti.
\newblock Multi-agent systems design and prototyping with bigraphical reactive
  systems.
\newblock In K.~Magoutis and P.~Pietzuch, editors, {\em Proc.~DAIS}, volume
  8460 of {\em Lecture Notes in Computer Science}, pages 201--208. Springer,
  2014.

\bibitem{mpm:gcm14w}
A.~Mansutti, M.~Miculan, and M.~Peressotti.
\newblock Towards distributed bigraphical reactive systems.
\newblock In R.~Echahed, A.~Habel, and M.~Mosbah, editors, {\em Proc.~GCM'14},
  page~45, 2014.
\newblock Workshop version.

\bibitem{mp:br-tr13}
M.~Miculan and M.~Peressotti.
\newblock Bigraphs reloaded: a presheaf presentation.
\newblock Technical Report UDMI/01/2013, Dept.~of Mathematics and Computer
  Science, Univ.~of Udine, 2013.

\bibitem{mp:memo14}
M.~Miculan and M.~Peressotti.
\newblock A {CSP} implementation of the bigraph embedding problem.
\newblock In T.~T. Hildebrandt, editor, {\em Proc.~MeMo}, 2014.
\newblock To appear.

\bibitem{milner:bigraphbook}
R.~Milner.
\newblock {\em The Space and Motion of Communicating Agents}.
\newblock Cambridge University Press, 2009.

\bibitem{pkss:bigactors}
E.~Pereira, C.~M. Kirsch, J.~B. de~Sousa, and R.~Sengupta.
\newblock {BigActors}: a model for structure-aware computation.
\newblock In C.~Lu, P.~R. Kumar, and R.~Stoleru, editors, {\em ICCPS}, pages
  199--208. ACM, 2013.

\bibitem{perrone:thesis}
G.~Perrone.
\newblock {\em Domain-Specific Modelling Languages in Bigraphs}.
\newblock PhD thesis, IT University of Copenhagen, 2013.

\bibitem{pdh:refine11}
G.~Perrone, S.~Debois, and T.~T. Hildebrandt.
\newblock Bigraphical refinement.
\newblock In J.~Derrick, E.~A. Boiten, and S.~Reeves, editors, {\em
  Proc.~REFINE}, volume~55 of {\em Electronic Proceedings in Theoretical
  Computer Science}, pages 20--36, 2011.

\bibitem{pdh:sac12}
G.~Perrone, S.~Debois, and T.~T. Hildebrandt.
\newblock A model checker for bigraphs.
\newblock In S.~Ossowski and P.~Lecca, editors, {\em Proc.~SAC}, pages
  1320--1325. ACM, 2012.

\bibitem{graphtransformation}
G.~Rozenberg, editor.
\newblock {\em Handbook of graph grammars and computing by graph
  transformation}, volume~1.
\newblock World Scientific, River Edge, NJ, USA, 1997.

\bibitem{sp:memo14}
M.~Sevegnani and E.~Pereira.
\newblock Towards a bigraphical encoding of actors.
\newblock In T.~T. Hildebrandt, editor, {\em Proc.~MeMo}, 2014.
\newblock To appear.

\bibitem{sevegnani2010sat}
M.~Sevegnani, C.~Unsworth, and M.~Calder.
\newblock A {SAT} based algorithm for the matching problem in bigraphs with
  sharing.
\newblock Technical Report TR-2010-311, Department of Computer Science,
  University of Glasgow, 2010.

\end{thebibliography}
}

\end{document}