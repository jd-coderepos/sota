

\documentclass[runningheads,a4paper]{llncs}

\usepackage{amsmath}
\usepackage{mathrsfs}
\usepackage{marvosym}
\usepackage{bbm}
\usepackage{etex}
\usepackage{amssymb}
\setcounter{tocdepth}{3}
\usepackage{graphicx}
\usepackage{stmaryrd}
\usepackage[all]{xy}
\usepackage{color}
\usepackage{tikz-qtree}
\usepackage{bussproofs}
\usepackage[justification=centering,labelfont=bf]{caption}


\usepackage{url}
\urldef{\mailsa}\path{grellois@pps.univ-paris-diderot.fr}
\urldef{\mailsb}\path{mellies@pps.univ-paris-diderot.fr}
\newcommand{\keywords}[1]{\par\addvspace\baselineskip
\noindent\keywordname\enspace\ignorespaces#1}

\newcommand\red[1]{#1}

\newcommand{\superbang}{\lightning} \newcommand{\modality}{\Box}
\newcommand{\colorbang}{\lightning\hspace{-.5765em}\lightning\hspace{-.5765em}\lightning}
\newcommand{\Rel}{Rel}
\newcommand{\Relinfinitary}{\underline{Rel}}
\newcommand{\with}{\&}
\newcommand{\tensor}{\otimes}
\newcommand{\lax}{m}
\newcommand{\laxzero}{m^0}
\newcommand{\laxdeux}[2]{m^2_{#1 , #2}}
\newcommand{\LAT}{\mathscr{L}}
\newcommand{\Ccategory}{\mathscr{C}}
\newcommand{\dig}[1]{\mathbf{dig}_{#1}}
\newcommand{\der}[1]{\mathbf{der}_{#1}}
\newcommand{\coldig}[1]{\mathbf{dig^{\infty}}_{#1}} \newcommand{\colder}[1]{\mathbf{der^{\infty}}_{#1}} \newcommand{\diag}[1]{\Delta_{#1}}
\newcommand{\id}[1]{1_{#1}}
\newcommand{\morph}[1]{\stackrel{#1}{\longrightarrow}}
\newcommand{\one}{1}
\newcommand{\paire}[2]{\langle #1,#2\rangle}
\newcommand{\fixrule}{fix}
\newcommand{\runtree}[2]{\textbf{run-tree}(#1,#2)}
\newcommand{\tree}{\textit{witness}}
\newcommand{\leaves}[1]{\textbf{leaves}(#1)}
\newcommand{\negation}[1]{#1^{\bot}}
\newcommand{\fixpoint}[1]{\textbf{Y}_{#1}}
\newcommand{\cardreals}{2^{\aleph_0}}


\begin{document}

\mainmatter  

\title{An infinitary model of linear logic}

\author{Charles Grellois \and Paul-Andr\'e Melli\`es}
\authorrunning{Charles Grellois \and Paul-Andr\'e Melli\`es}

\institute{Universit\'e Paris Diderot, Sorbonne Paris Cit\'e\\
Laboratoire Preuves, Programmes, Syst\`emes\\
\mailsa\\
\mailsb}

\maketitle

\begin{abstract}
In this paper, we construct an infinitary variant of the relational model of linear logic, 
where the exponential modality is interpreted as the set of finite or countable multisets. 
We explain how to interpret in this model the fixpoint operator~ as a Conway operator 
alternatively defined in an inductive or a coinductive way. We then extend the relational semantics
with a notion of color or priority in the sense of parity games. This extension enables us 
to define a new fixpoint operator~ combining both inductive and coinductive policies. 
We conclude the paper by mentionning a connection between the resulting model of
-calculus with recursion and higher-order model-checking.
\keywords{Linear logic, relational semantics, fixpoint operators, induction and coinduction,
parity conditions, higher-order model-checking.}
\end{abstract}

\abovedisplayskip = 5pt
\belowdisplayskip = 5pt
\abovedisplayshortskip = 5pt
\belowdisplayshortskip = 5pt
\section{Introduction}

In many respects, denotational semantics started in the late 1960's with Dana Scott's introduction of domains and the fundamental intuition that
-terms should be interpreted as \emph{continuous} rather than general functions between domains.
This seminal insight has been so influential in the history of our discipline that it remains 
deeply rooted in the foundations of denotational semantics more than fourty-five years later.
In the case of linear logic, this inclination for continuity means that 
the interpretation of the exponential modality

is \emph{finitary} in most denotational semantics of linear logic.
This finitary nature of the exponential modality is tightly connected to continuity
because this modality regulates the linear decomposition of the intuitionistic implication:

Typically, in the qualitative and quantitative 
coherence space semantics of linear logic, 
the coherence space  is either defined as the coherence space~ 
of \emph{finite} cliques (in the qualitative semantics) or of \emph{finite} multi-cliques (in the quantitative semantics)
of the original coherence space~.
This finiteness condition on the cliques  or multi-cliques  of the coherence space  
captures the computational intuition that, in order to reach a given position~ of the coherence space~,
every proof or program

will only explore a \emph{finite} number of copies of the hypothesis~, 
and reach at the end of the computation a specific position~ in each copy of the coherence space~.
In other words, the finitary nature of the interpretation of  is just an alternative
and very concrete way to express in these traditional models of linear logic
the continuity of proofs and programs.



In this paper, we would like to revisit this well-established semantic tradition and accomodate another equally well-established tradition,
coming this time from verification and model-checking.
We find especially important to address and to clarify an apparent antagonism between the two traditions.
Model-checking is generally interested in infinitary (typically -regular) inductive and coinductive behaviours 
of programs which lie obviously far beyond the scope of Scott continuity.
For that reason, we introduce a variant of the relational semantics of linear logic 
where the exponential modality, noted in this context

is defined as the set of \emph{finite} or \emph{countable} multisets of the set~.
From this follows that a proof or a program

is allowed in the resulting infinitary semantics to explore
a possibly countable number of copies of his hypothesis~ 
in order to reach a position in~.
By relaxing the continuity principle, this mild alteration of the original relational semantics
paves the way to a fruitful interaction between linear logic and model-checking.
This link between linear logic and model-checking is supported by the somewhat unexpected observation 
that the binary relation 

defining the fixpoint  associated to a morphism

in the familiar (and thus finitary) relational semantics of linear logic 
is defined by performing a series of explorations of the infinite binary tree

by an alternating tree automaton 
on the alphabet  defined by the binary relation~.
The key idea is to define the set of states of the automaton as  and to associate a transition 


\noindent
of the automaton to any element  of the binary relation ,
where the 's are elements of~ and the 's are elements of~~;
and to let the symbol~ accept any state .
Then, it appears that the traditional definition of the fixpoint operator  as a binary relation 
may be derived from the construction of run-trees of the tree-automaton 
on the infinitary tree .
More precisely, the binary relation  contains all the elements 
such that there exists a finite run-tree (called ) of the tree automaton 
accepting the state~ with the multi-set of states  collected at the leaves .
As far as we know, this automata-theoretic account of the traditional construction of the fixpoint operator  
in the relational semantics of linear logic is a new insight of the present paper, which we carefully develop in~\S\ref{section/finitary-fixpoint}.


\medbreak

Once this healthy bridge between linear logic and tree automata theory identified,
it makes sense to study variations of the relational semantics inspired by verification.
This is precisely the path we follow here by replacing the finitary interpretation 
of the exponential modality by the finite-or-countable one .
This alteration enables us to define an inductive as well as a coinductive fixpoint operator 
in the resulting infinitary relational semantics.
The two fixpoint operators only differ in the acceptance condition applied to the run-tree .
We carry on in this direction, and introduce a \emph{coloured} variant of the relational semantics,
designed in such a way that the tree automaton 
associated to a morphism  defines a parity tree automaton.
This leads us to the definition of an inductive-coinductive fixpoint operator~
tightly connected to the current investigations on higher-order model-checking.


\paragraph*{Related works.}
The present paper is part of a wider research project devoted to the relationship 
between linear logic, denotational semantics and higher-order model-checking.
The idea developed here of shifting from the traditional finitary relational semantics
of linear logic to infinitary variants is far from new.
The closest to our work in this respect is probably the work by Miquel \cite{these-miquel}
where stable but non-continuous functions between coherence spaces are considered.
However, our motivations are different, since we focus here on the case 
of a modality  defined by finite-or-countable multisets in~, 
which is indeed crucial for higher-order model-checking, but is not considered by Miquel.
In another closely related line of work, Carraro, Ehrhard and Salibra \cite{carraro-ehrhard-salibra}
formulate a general and possibly infinitary construction of the exponential modality~
in the relational model of linear logic.
However, the authors make the extra finiteness assumption in~\cite{carraro-ehrhard-salibra}
that the support of a possibly infinite multiset in  is necessarily finite.
Seen from that prospect, one purpose of our work is precisely to relax this finiteness condition which
appears to be too restrictive for our semantic account of higher-order model-checking
based on linear logic.
In a series of recent works,
Salvati and Walukiewicz \cite{salvati-walukiewicz2} \cite{salvati-walukiewicz3} have exhibited
a nice and promising connection between  higher-order model checking
and finite models of the simply-typed -calculus.
In particular, they establish the decidability of weak MSO properties of higher-order recursion schemes
by using purely semantic methods.
In comparison, we construct here a cartesian-closed category of sets and coloured relations
(rather than finite domains) where -regular properties of higher-order recursion schemes
(and more generally of -terms) may be interpreted semantically thanks to a colour modality.
In a similar direction, Ong and Tsukada \cite{ong-tsukada} have recently constructed
a cartesian-closed category of infinitary games and strategies with similar connections
to higher-order model-checking.
Coming back to linear logic,
we would like to mention the works by Baelde \cite{baelde} and Montelatici \cite{montelatici}
who developed infinitary variants (either inductive-coinductive or recursive) of linear logic,
with an emphasis on the syntactic rather than semantic side.
In a recent paper working like we do here at the converging point of linear logic and automata theory, 
Terui \cite{terui} uses a qualitative variant of the relational semantics of linear logic
where formulas are interpreted as partial orders and proofs as downward sets in order to
establish a series of striking results on the complexity of normalization of simply-typed -terms.
Finally, an important related question which we leave untouched here is the comparison
between our work
and the categorical reconstruction of parity games achieved by Santocanale
\cite{santocanale2,santocanale} using the notion of bicomplete category,
see also his more recent work with Fortier \cite{santocanale3}.






\paragraph*{Plan of the paper.}
We start by recalling in \S\ref{section/rel} the traditional relational model of linear logic.
Then, after recalling in \S\ref{section/fixpoints} the definition of a Conway fixpoint operator in a Seely category,
we construct in \S\ref{section/finitary-fixpoint} such a Conway operator for the relational semantics.
We then introduce in \S\ref{section/infinitary-rel} our infinitary variant of the relational semantics,
and illustrate its expressive power in \S\ref{section/inductive-and-coinductive} by defining two
different Conway fixpoint operators.
Then, we define in \S\ref{section/coloured-modality} a coloured modality for the relational semantics,
and construct in \S\ref{section/y-colore} a Conway fixpoint operator in that framework.
We finally conclude in \S\ref{section/conclusion}.

\section{The relational model of linear logic}
\label{section/rel}
In order to be reasonably self-contained, we briefly recall the relational model of linear logic.
The category  is defined as the category with finite or countable sets as objects, and with binary relations between  and 
as morphisms~.
The category  is symmetric monoidal closed, with tensor product defined as (set-theoretic) cartesian product,
and tensorial unit defined as singleton:

Its internal hom (also called linear implication)  simply defined as .
Since the object  is dualizing,
the category  is moreover -autonomous.
The category  has also finite products defined as

with the empty set as terminal object .
As in any category with finite products, there is a diagonal morphism 
for every object~, defined
as



Note that the category  has finite sums as well, since the negation  
of any object  is isomorphic to the object  itself.
All this makes  a model of multiplicative additive linear logic.
In order to establish that it defines a model of propositional linear logic, 
we find convenient to check that it satisfies the axioms
of a Seely category, as originally axiomatized by Seely \cite{seely} 
and then revisited by Bierman \cite{bierman}, see the survey \cite{models-of-linear-logic} for details.
To that purpose, recall that a \emph{finite multiset} over a set  
is a (set-theoretic) function  with finite support,
where the support of  is the set of elements of  whose image is not equal to .
The functor  is defined as



\noindent
The comultiplication and counit of the comonad are defined as the digging and dereliction morphisms below:



\noindent
In order to define a Seely category, one also needs the family of isomorphisms



\noindent
which are defined as  and



\noindent
One then carefully checks that the coherence diagrams expected of a Seely category commute.
From this follows that 

\begin{property}
The category  together with the finite multiset interpretation
of the exponential modality~ defines a model of propositional linear logic.
\end{property}



\section{Fixpoint operators in models of linear logic}\label{section/fixpoints}
We want to extend linear logic with a fixpoint rule:



\noindent
In order to interpret it in a Seely category, we need a parametrized fixpoint operator,
defined below as a family of functions



\noindent
parametrized by  and satisfying two elementary conditions,
mentioned for instance by Simpson and Plotkin in~\cite{simpson-plotkin}.
\begin{itemize}
\item \textbf{Naturality:} for any  and , the diagram:



\noindent
commutes, where the morphism  in the upper part of the diagram is defined as the composite



\item \textbf{Parametrized fixpoint property:} for any , the following diagram commutes:



\end{itemize}

These two equations are fundamental but they do not reflect all the equational properties 
of the fixpoint operator in domain theory.
For that reason, Bloom and Esik introduced the notion of \emph{Conway theory}
in their seminal work on iteration theories~\cite{bloom-esik,Bloom19961}.
This notion was then rediscovered and adapted to cartesian categories
by Hasegawa \cite{hasegawa},  by Hyland and by Simpson and Plotkin \cite{simpson-plotkin}.
Hasegawa and Hyland moreover independently established a nice correspondence
between the resulting notion of \emph{Conway fixpoint operator} and 
the notion of trace operator introduced a few years earlier
by Joyal, Street and Verity \cite{joyal-street-verity}.
Here, we adapt in the most straightforward way this notion of Conway fixpoint operator
to the specific setting of Seely categories.
Before going any further, we find useful to introduce the following notation: for every pair of morphisms



\noindent
we write  for the composite:



\noindent
A Conway operator is then defined as a parametrized fixpoint operator satisfying the two additional properties below:
\begin{itemize}
\item \textbf{Parametrized dinaturality:} for any  and , the following diagram commutes:


\item \textbf{Diagonal property:} for every morphism ,



belongs to  , since


is sent by  to a morphism of , so that



to which the fixpoint operator  can be applied, giving the morphism (\ref{eq/Y-diag}) of . This morphism is required to
coincide with the morphism , where the morphism  is defined as the composite

\end{itemize}
\noindent
Just as expected, we recover in that way the familiar notion of Conway fixpoint operator
as formulated in any cartesian category by Hasegawa, Hyland, Simpson and Plotkin:


\begin{property}
A Conway operator in a Seely category is the same thing as a Conway operator 
(in the sense of \cite{hasegawa,simpson-plotkin}) in the cartesian closed category
associated to the exponential modality by the Kleisli construction.
\end{property}




\section{A fixpoint operator in the relational semantics}
\label{section/finitary-fixpoint}

\red{
The relational model of linear logic can be equipped 
with a natural parameterized fixpoint operator  which transports any binary relation}

\red{to the binary relation}

defined in the following way:

where  is the set of ``run-trees'' defined as trees
with nodes labelled by elements of the set  and such that:
\begin{itemize}
\item the root of the tree is labelled by ,
\item the inner nodes are labelled by elements of the set ,
\item the leaves are labelled by elements of the set ,
\item and for every node labelled by an element :
\begin{itemize}
\item if  is an inner node, and letting  denote the labels 
of its children belonging to  and  the labels belonging to :

then 

\item if  is a leaf, then .
\end{itemize}
\end{itemize}
and where  is the multiset obtained by enumerating the labels of the leaves of the run-tree . 
Recall that multisets account for the number of occurences of an element, 
so that  has the same number of elements as there are leaves
in the run-tree .
Moreover,  is independent of the enumeration of the leaves, 
since multisets can be understood as abelian versions of lists.
Finally, we declare that a run-tree is \emph{accepting} when it is a finite tree.
\begin{property}
The fixpoint operator  is a Conway operator on Rel.
\end{property}

\begin{example}
\label{example/witness}
Suppose that 

where  and . Denote by  the finite multiset containing the element  with multiplicity . 
Then, for every , we have that  
since  can be obtained from the -labelled witness run-tree 
of Figure \ref{figure/witness1}, which has  internal occurrences of the element , 
and  occurrences of the element  at the leaves. The witness tree is finite, so that it is accepted.
Now, consider the relation

In that case,  is not an element of  for any 
because all run-trees are necessarily infinite, as depicted in Figure \ref{figure/badwitness}, 
and thus, none is accepting. 
As a consequence,  is the empty relation.

\begin{figure}[t]
\begin{small}
\centering
\begin{minipage}{.5\textwidth}
  \centering
\begin{tikzpicture}
\Tree [.  [.  \edge[dotted]; [.   [. ] ] ] ]
\end{tikzpicture}
  \captionof{figure}{An accepting run-tree.}
\label{figure/witness1}
\end{minipage}\begin{minipage}{.5\textwidth}
  \centering
 \begin{tikzpicture}
\tikzset{level distance=22pt}
\Tree [.  [.  \edge[dotted]; [.  [. \edge[dotted];  ] ] ] ]
\end{tikzpicture}
  \captionof{figure}{A non-accepting run-tree.}
\label{figure/badwitness}
\end{minipage}
\end{small}
\vspace{-0.5cm}
\end{figure}
\end{example}
The terminology which we have chosen for the definition of  is obviously automata-theoretic.
In fact, as we already mentioned in the introduction, this definition may be formulated
as an exploration of the infinitary tree  on the ranked alphabet 
by an alternating tree automaton associated to the binary relation .
Indeed, given an element , 
consider the alternating tree automata  where, for  and :

\begin{center}
\begin{tabular}{rcl}
 &
 &
\\
\end{tabular}
\end{center}
Note that we allow here the use of an infinite non-deterministic choice operator  in formulas describing transitions, but only with \emph{finite} alternation.
Now, our point is that  coincides with the set of run-trees of the alternating automaton 
over the infinite tree \textbf{comb} depicted in the Introduction. 
Notice that only finite run-trees are accepting: this requires that for some  the transition  contains the alternating choice , in which the exploration of the infinite branch of \textbf{comb} stops and produces an accepting run-tree. This requires in particular the existence of some  such that .




\section{Infinitary exponentials}
\label{section/infinitary-rel}
Now that we established a link with tree automata theory, it is tempting to relax the finiteness acceptance condition
on run-trees applied in the previous section.
To that purpose, however, we need to relax the usual assumption that the formulas of linear logic
are interpreted as finite or countable sets.
Suppose indeed that we want to interpret the exponential modality

as the set of finite or countable multisets, where a countable multiset
of elements of~ is defined as a function

with finite or countable support.
Quite obviously, the set

has the cardinality of the reals .
We thus need to go beyond the traditionally countable relational interpretations of linear logic.
However, we may suppose that every set  interpreting a formula has a cardinality 
below or equal .
In order to understand why, it is useful to reformulate the elements of 
as finite or infinite words of elements of~ modulo an appropriate notion of equivalence
of finite or infinite words up to permutation of letters.
Given a finite word  and a finite or infinite word , we write

when there exists a finite prefix  of  such that  is a prefix of  modulo permutation of letter.
We write

where  denotes the set of finite words on the alphabet~.
\begin{proposition}
There is a one-to-one relationship between the elements of 
and the finite or infinite words on the alphabet  modulo the equivalence relation~.
\end{proposition}
This means in particular that for every set~,
there is a surjection from the set~
of finite or infinite words on the alphabet~ to the set 
of finite or countable multisets.
An element of the equivalence class associated to a multiset is called a \emph{representation} of this multiset.
Notice that if a set~ has cardinality at most , the set~
is itself bounded by , since .
This property leads us to define the following extension of :



\begin{definition}
The category  has the sets  of cardinality at most
 as objects, and binary relations 
between  and  as morphisms .
\end{definition}
Since a binary relation between two sets  and  is a subset of ,
the cardinality of a binary relation in  is also bounded by .
Note that the hom-set  is in general of higher cardinality than ,
yet it is bounded by the cardinality of the powerset of the reals.
It is immediate to establish that:
\begin{property}
The category  is -autonomous and has finite products.
As such, it provides a model of multiplicative additive linear logic.
\end{property}
There remains to show that the finite-or-countable multiset construction 
defines a categorical interpretation of the exponential modality of linear logic.
Again, just as in the finitary case, we find convenient to check that  together 
with the finite-or-countable multiset interpretation  satisfy the axioms of a Seely category.
In that specific formulation of a model of linear logic, the first property to check is that:
\begin{property}
The finite-or-countable multiset construction  defines 
a comonad on the category .
\end{property}
The counit of the comonad is defined as the binary relation

which relates  to  for every element~ of the set~. 
In order to define its comultiplication, we need first to extend the notion of sum 
of multisets to the infinitary case, which we do in the obvious way, by extending
the binary sum of~ to possibly infinite sums in its completion .
In order to unify the notation for finite-or-countable multisets with the one for finite multisets used in Section \ref{section/rel}, 
we find convenient to denote by  the countable multiset admitting the representation 


We are now ready to describe the comultiplication 
 
of the comonad  as a straightforward generalization of the finite case:

One then defines the isomorphism

and the family of isomorphisms

indexed by the objects  of the category~ which relates 
every pair  of the set  with the finite-or-countable multiset

where the operation  maps the finite-or-countable multiset  of elements of
 to the finite-or-countable multiset   of . We define
  similarly.
We check carefully that
\begin{property}
The comonad  on the category  
together with the isomorphisms (\ref{equation/iso-seely-0}) and (\ref{equation/iso-seely-2})
satisfy the coherence axioms of a Seely category -- see \cite{models-of-linear-logic}.
\end{property}

\noindent
In other words, this comonad  over the category  induces a new and infinitary model of propositional linear logic. 
The next section is devoted to the definition of two different fixpoint operators living inside this new model.



\section{Inductive and coinductive fixpoint operators}\label{section/inductive-and-coinductive}
In the infinitary relational semantics, a binary relation 

may require a countable multiset~ of elements (or positions) of the input set~
in order to reach a position~ of the output set~. 
For that reason, we need to generalize the notion of alternating tree automata to \emph{finite-or-countable alternating tree automata}, 
a variant in which formulas defining transitions use of a possibly countable alternation operator  
and of a possibly countable non-deterministic choice operator .
The generalization of the family of automata  of \S\ref{section/finitary-fixpoint} leads to a new definition of the set , in which witness trees may have internal nodes of countable arity.
A first important observation is the following result:

\begin{property}
Given , , and , the multiset  is finite or countable.
\end{property}
An important consequence of this observation is that the definition of the Conway operator  given in Equation~(\ref{master-equation}) 
can be very simply adapted to the finite-or-countable interpretation of the exponential modality~ in the Seely category .
Moreover, in this infinitary model of linear logic, we can give more elaborate acceptation conditions, among which two are canonical:
\begin{itemize}
\item considering that any run-tree is accepting, one defines the \emph{coinductive} fixpoint on the model, which is the greatest fixpoint over .
\item on the other hand, by accepting only trees without infinite branches, we obtain the \emph{inductive} interpretation of the fixpoint,
which is the least fixpoint operator over .
\end{itemize}

It is easy to see that the two fixpoint operators are different: recall Example~\ref{example/witness},
and observe that the binary relation~ is also a relation in the infinitary semantics.
It turns out that its inductive fixpoint is the empty relation, while its coinductive fixpoint coincides with the relation

In this coinductive interpretation, the run-tree obtained by using infinitely  and never  is accepting and is the witness tree generating .


\begin{property}
The inductive and coinductive fixpoint operators over the infinitary relational model of linear logic are Conway operators on this Seely category.
\end{property}

\section{The coloured exponential modality}\label{section/coloured-modality}
In their semantic study of the parity conditions used in higher-order model-checking,
and more specifically in the work by Kobayashi and Ong \cite{kobayashi-ong},
the authors have recently discovered~\cite{coloured-tensorial-logic} that these parity conditions are secretly regulated
by the existence of a comonad  which can be interpreted in the relational semantics of linear logic
as 

where  is a finite set of integers called \emph{colours}.
The colours (or priorities) are introduced in order to regulate the fixpoint discipline:
in the immediate scope of an even colour, fixpoints should be interpreted coinductively,
and inductively in the immediate scope of an odd colour.
It is worth mentioning that the comonad~ has its comultiplication defined by the maximum operator
in order to track the maximum colour encountered during a computation:

whereas the counit is defined using the minimum colour .
The resulting comonad is symmetric monoidal and also satisfies the following key property:

\begin{property}
There exists a distributive law  between comonads.
\end{property}

\noindent
A fundamental consequence is that the two comonads can be composed 
into a single comonad  defined as follows:

The resulting \emph{infinitary} and \emph{coloured} relational semantics of linear logic
is obtained from the category  equipped with the composite comonad .

\begin{theorem}
The category  together with the comonad 
defines a Seely category and thus a model of propositional linear logic.
\end{theorem}

\section{The inductive-coinductive fixpoint operator~}
\label{section/y-colore}
We combine the results of the previous sections
in order to define a fixpoint operator  over the infinitary coloured relational model, 
which generalizes both the inductive and the coinductive fixpoint operators.
Note that in this infinitary and coloured framework,
we wish to define a fixpoint operator  which transports a binary relation

into a binary relation

To that purpose, notice that the definition given in \S\ref{section/finitary-fixpoint}
of the set  of run-trees
extends immediately to this new coloured setting, since the only change is in the set of labellings. 
Again, accepting all run-trees would lead to the coinductive fixpoint, while accepting only run-trees whose branches are finite 
would lead to the inductive fixpoint. 
We now define our acceptance condition for run-trees in the expected way, 
directly inspired by the notion of alternating parity tree automaton.
Consider a run-tree , and remark that its nodes are labelled with elements 
of . 
We call the colour of a node the first element of its label. Coloured acceptance is then defined as follows:
\begin{itemize}
\item a finite branch is accepting,
\item an infinite branch is accepting precisely when the greatest colour appearing infinitely often in the labels of its nodes is even.
\item a run-tree is accepting precisely when all its branches are accepting.
\end{itemize}
Note that a run-tree whose nodes are all of an even colour will be accepted independently of its depth, 
as in the coinductive interpretation, while a run-tree labelled only with odd colours will be accepted 
precisely when it is finite, just as in the inductive interpretation. 
We call the fixpoint operator associated with the notion of coloured acceptation 
the inductive-coinductive fixpoint operator over the infinitary coloured relational model.

\begin{theorem}
The inductive-coinductive fixpoint operator  defined over the infinitary coloured relational semantics of linear logic
is a Conway operator.
\end{theorem}

\section{Conclusion}\label{section/conclusion}

In this article, we introduced an infinitary variant of the familiar relational semantics of linear logic.
We then established that this infinitary model accomodates an inductive as well as a coinductive Conway operator .
This propelled us to define a coloured relational semantics and to define an inductive-coinductive fixpoint operator
based on a parity acceptance condition.
The authors proved recently \cite{coloured-tensorial-logic} that a recursion scheme can be interpreted in this model in such a way 
that its denotation contains the initial state of an alternating parity automaton if and only if the tree 
it produces satisifies the MSO property associated to the automaton. 
A crucial point related to the work by Salvati and Walukiewicz \cite{salvati-walukiewicz1}
is the fact that a tree satisfies a given MSO property if and only if any suitable representation 
as an infinite tree of a -term generating it also does.
We are thus convinced that this infinitary and coloured variant of the relational semantics of linear logic
will play an important and clarifying role in the denotational and compositional study of higher-order model-checking.


\bibliography{fossacs}
\bibliographystyle{plain}


\section*{Appendix: Seely categories}
A \emph{Seely category} is defined as a symmetric monoidal closed category
 with binary products , a terminal object~, and:
\begin{enumerate}\item a comonad ,
\item two natural isomorphisms 
\begin{center}
\begin{tabular}{ccc}

&
\hspace{4em}
&

\end{tabular}
\end{center}
making

a symmetric monoidal functor.
\end{enumerate}
One also asks that the coherence diagram

commutes in the category~ for all objects~ and~, and that the four following diagrams expressing the fact that the functor~
is symmetric monoidal:



commute in the category~ for all objects~ and .

\end{document}
