

This section is divided in three parts. We start with  definitions of Multi-Authority Key-Policy ABE and of CPA selective security. In the second part we present in detail our first scheme and, finally, we prove the security of this scheme under the classical BDH assumption in the selective set model.

A security parameter will be used to determine the size of the bilinear group used in the construction, this parameter represents the order of complexity of the assumption that provides the security of the scheme.
Namely, first the complexity is chosen thus fixing the security parameter, then this value is used to compute the order that the bilinear group must have in order to guarantee the desired complexity, and finally a suitable group is picked and used.

\subsection{Multi Authority KP-ABE Structure and Security}\label{KP-ABE}

  In this scheme, after the common universe of attributes and bilinear group are agreed, the authorities set up independently their master key and public parameters.
  The master key is subsequently used to generate the private keys requested by users.
  Users ask an authority for keys that embed a specific access structure, and the authority issues the key only if it judges that the access structure suits the user that requested it.
  Equivalently an authority evaluates a user that requests a key, assigns an access structure, and gives to the user a key that embeds it.
  When someone wants to encrypt, it chooses a set of attributes that describes the message (and thus determines which access structures may read it) and a set of trusted authorities.
  The ciphertext is computed using the public parameters of the chosen authorities, and may be decrypted only using a valid key for each of these authorities.
  A key with embedded access structure $\mathbb{A}$ may be used to decrypt a ciphertext that specifies a set of attributes $S$ if and only if $S \in \mathbb{A}$, that is the structure considers the set authorized.

  This scheme is secure under the classical BDH assumption in the selective set model, in terms of chosen-ciphertext indistinguishability.

  The security game is formally defined as follows.\\

  \noindent Let $\mathcal{E}=(\Set,\Enc,\Key,\Dec)$ be a MA-KP-ABE scheme for a message space $\mathcal M$, a universe of authorities $X$ and an access structure space $\mathcal{G}$ and consider the following MA-KP-ABE experiment $\mathsf{MA}$-$\mathsf{KP}$-$\mathsf{ABE}$-$\mathsf{Exp}_{\A,\E}(\lambda,U)$  for an adversary $\mathcal A$, parameter $\lambda$ and attribute universe $U$:
 \begin{description}
      \item [Init.]
        The adversary declares the set of attributes $S$ and the set of authorities $A \subseteq X$ that it wishes to be challenged upon. Moreover it selects the \textit{honest authority} $k_0 \in A$.
      \item [Setup.]
        The challenger runs the Setup algorithm, initializes the authorities and gives to the adversary the public parameters.
      \item [Phase I.]
        The adversary issues queries for private keys of any authority, but $k_0$ answers only to queries for keys for access structures $\mathbb{A}$ such that $S \notin \mathbb{A}$.
        On the contrary the other authorities respond to every query.
      \item [Challenge.]
        The adversary submits two equal length messages $m_0$ and $m_1$.
        The challenger flips a random coin $b \in \{0, 1\}$, and encrypts $m_b$ with $S$ for the set of authorities $A$.
        The ciphertext is passed to the adversary.
      \item [Phase II.]
        Phase \texttt{I} is repeated.
      \item [Guess.]
        The adversary outputs a guess $b'$ of $b$.
  \end{description}


  \begin{definition}[MA-KP-ABE Selective Security]
  The MA-KP-ABE scheme $\E$ is  CPA selective secure (or secure against   chosen-plaintext attacks) for attribute universe $U$ if for all probabilistic polynomial-time adversaries $\mathcal A$, there exists a negligible function $negl$ such that:
  $$\Pr[\mathsf{MA}\mbox{-}\mathsf{KP}\mbox{-}\mathsf{ABE}\mbox{-}\mathsf{Exp}_{\A,\E}(\lambda,U) = 1] \leq \frac{1}{2} + negl(\lambda).$$
  \end{definition}

\subsection{The Scheme}
  The scheme plans a set $X$ of independent authorities, each with their own pa\-ra\-me\-ters, and it sets up an encryption algorithm that lets the encryptor choose a set $A \subseteq X$ of authorities, and combines the public parameters of these in such a way that an authorized key for each authority in $A$ is required to successfully decrypt.\\
  Our scheme consists of three randomized algorithms ($\Set, \Key,$ $\Enc$) plus the decryption $\Dec$.
  The techniques used are inspired from the scheme of Goyal et al. in \cite{goyal2006attribute}.
  The scheme works in a bilinear group $\mathbb{G}_1$ of prime order $p$, and uses LSSS matrices to share secrets according to the various access structures.
  Attributes are seen as elements of $\mathbb{Z}_p$.

  The description of the algorithms follows.

  \paragraph*{$\Set (U, g, \mathbb{G}_1) \rightarrow (\PK_k, \MK_k).$}
    Given the universe of attributes $U$ and a generator $g$ of $\mathbb{G}_1$ each authority sets up independently its parameters.
    For $k \in X$ the Authority $k$ chooses uniformly at random $\alpha_k \in \mathbb{Z}_p$, and $z_{k, i} \in \mathbb{Z}_p$ for each $i \in U$.
    Then the public parameters $\PK_k$ and the master key $\MK_k$ are:
    $$
      \PK_k = \left(e(g,g)^{\alpha_k}, \{g^{z_{k, i}}\}_{i \in U} \right) \qquad \MK_k = \left(\alpha_k, \{z_{k, i}\}_{i \in U}\} \right)
    $$

  \paragraph*{$\Key_k(\MK_k, (M_k, \rho_k)) \rightarrow \SK_k.$}
    The key generation algorithm for the authority $k$ takes as input the master secret key $\MK_k$ and an LSSS access structure $(M_k, \rho_k)$, where $M_k$ is an $l\times n$ matrix on $\mathbb{Z}_p$ and $\rho_k$ is a function which associates rows of $M_k$ to attributes.
    It chooses uniformly at random a vector $\vec{v}_k \in \mathbb{Z}_p^n$ such that $v_{k,1} = \alpha_k$.
    Then it computes the shares $\lambda_{k,i} = M_{k,i} \vec{v}_k$ for $1 \leq i \leq l$ where $M_{k, i}$ is the $i$-th row of $M_k$.
    Then the private key $\SK_k$ is:
    $$
      \SK_k = \left\{K_{k, i} = g^{\frac{\lambda_{k,i}}{z_{k, \rho_{k}(i)}}}\right\}_{1\leq i \leq l}
    $$

  \paragraph*{$\Enc(m, S, \{\PK_k\}_{k \in A}) \rightarrow \CT.$}
    The encryption algorithm takes as input the public parameters, a set $S$ of attributes and a message $m$ to encrypt.
    It chooses $s \in \mathbb{Z}_p$ uniformly at random and then computes the ciphertext as:
    $$
      \CT = \left(S, C' = m \cdot \left(\prod_{k\in A} e(g,g)^{\alpha_k} \right)^s, \{C_{k,i} = (g^{z_{k, i}})^s\}_{k\in A, ~i \in S} \right)
    $$

  \paragraph*{$\Dec(\CT, \{\SK_k\}_{k\in A})  \rightarrow m'.$}
    The input is a ciphertext for a set of attributes $S$ and a set of authorities $A$ and an authorized key for every authority cited by the ciphertext.
    Let $(M_k, \rho_k)$ be the LSSS associated to the key $k$, and suppose that $S$ is authorized for each $k \in A$.
    The algorithm for each $k\in A$ finds $w_{k,i} \in \mathbb{Z}_p, i \in I_k$ such that
    \begin{equation}
      \label{alfak}
      \sum_{i \in I_k} \lambda_{k, i} w_{k, i} = \alpha_k
    \end{equation}
    for appropriate subsets $I_k \subseteq S$ and then proceeds to reconstruct the original message computing:
    \begin{align*}
      m' & = \frac{C'}{\prod_{k \in A} \prod_{i \in I_k} e(K_{k, i}, C_{k, \rho_{k}(i)})^{w_{k,i}}} \\
& = \frac{m \cdot \left(\prod_{k \in A} e(g,g)^{\alpha_k} \right)^s}{\prod_{k \in A} \prod_{i \in I_k} e\left(g^{\frac{\lambda_{k,i}}{z_{k, \rho_{k}(i)}}}, (g^{z_{k, \rho_{k}(i)}})^s\right)^{w_{k,i}}} \nonumber \\
& = \frac{m \cdot e(g, g)^{s(\sum_{k\in A}\alpha_k)}}{\prod_{k \in A} e(g, g)^{s \sum_{i\in I_k} w_{k, i} \lambda_{k, i}}} \nonumber \\
& \stackrel{*}{=} \frac{m \cdot e(g, g)^{s(\sum_{k\in A}\alpha_k)}}{e(g, g)^{s (\sum_{k\in A}\alpha_k)}} = m
\end{align*}
    Where $\stackrel{*}{=}$ follows from property (\ref{alfak}).

\subsection{Security}
  The scheme is proved secure under the BDH assumption (Definition \ref{BDH}) in a selective set security game in which every authority but one is supposed curious (or corrupted or breached) and then it will issue even keys that have enough clearance for the target set of attributes, while the honest one issues only unauthorized keys.
  Thus if at least one authority remains trustworthy the scheme is secure.\\
  The security  is provided by the following theorem.

  \begin{theorem}
    If an adversary can break the scheme, then a simulator can be constructed to play the Decisional BDH game with a non-negligible advantage.
  \end{theorem}

  \begin{proof}
    Suppose there exists a polynomial-time adversary $\mathcal{A}$, that can attack the scheme in the Selective-Set model with advantage $\epsilon$.
    Then a simulator $\mathcal{B}$ can be built that can play the Decisional BDH game with advantage $\epsilon/2$.
    The simulation proceeds as follows.

    \paragraph*{Init}
      The simulator takes in a BDH challenge $g, g^a, g^b, g^s, T$.
      The adversary gives the algorithm the challenge access structure $S$.

    \paragraph*{Setup}
      The simulator chooses random $r_k \in \mathbb{Z}_p$ for $k \in A \setminus \{k_0\}$ and implicitly sets $\alpha_k = -r_k b$ for $k \in A \setminus \{k_0\}$ and $\alpha_{k_0} = a b + b \sum_{k \in A \setminus \{k_0\}} r_k$ by computing:
      \begin{align*}
        e(g,g)^{\alpha_{k_0}} &= e(g^a, g^b) \prod_{k \in A \setminus \{k_0\}} (g^b, g^{r_k}) \\
e(g,g)^{\alpha_k} &= e(g^b, g^{-r_k}) \qquad \forall k \in A \setminus \{k_0\}
      \end{align*}
      Then it chooses $z_{k, i}' \in \mathbb{Z}_p$ uniformly at random for each $i \in U$, $k \in A$ and implicitly sets
      $$
        z_{k, i} =
        \begin{cases}
          z_{k, i}' &\text{ if } i \in S \\
          b z_{k, i}' &\text{ if } i \notin S
        \end{cases}
      $$
      Then it can publish the public parameters computing the remaining values as:
      $$
        g^{z_{k, i}} =
        \begin{cases}
          g^{z_{k, i}'} &\text{ if } i \in S \\
          (g^b)^{z_{k, i}'} &\text{ if } i \notin S
        \end{cases}
      $$

    \paragraph*{Phase I}
      In this phase the simulator answers private key queries.
      For the queries made to the authority $k_0$ the simulator has to compute the $K_{k_0,i}$ values of a key for an access structure $(M,\rho)$ with dimension $l \times n$ that is not satisfied by $S$.
      Therefore for the properties of an LSSS it can find a vector $\vec{y} \in \mathbb{Z}_p^n$ with $y_1 = 1$ fixed such that
      \begin{equation}\label{y-sempl}
        M_i \vec{y} = 0 \qquad \forall i \mbox{ such that } \rho(i) \in S
      \end{equation}
      Then it chooses uniformly at random a vector $\vec{v} \in \mathbb{Z}_p^n$ and implicitly sets the shares of $\alpha_{k_0} = b(a+ \sum_{k \in A \setminus \{k_0\}} r_k)$ as
      $$
        \lambda_{k_0,i} = b \sum_{j=1}^n M_{i,j} (v_j + (a + \sum_{k \in A \setminus \{k_0\}} r_k - v_1)y_j)
      $$
      Note that $\lambda_{k_0,i} = \sum_{j=1}^n M_{i,j} u_j$ where $u_j = b (v_j + (a + \sum_{k \in A \setminus \{k_0\}} r_k - v_1)y_j)$ thus \linebreak[4] ${u_1 = b (v_1 + (a + \sum_{k \in A \setminus \{k_0\}} r_k - v_1) 1) = ab + b \sum_{k \in A \setminus \{k_0\}} r_k= \alpha_{k_0} }$ so the shares are valid.
      Note also that from (\ref{y-sempl}) it follows that
      $$
        \lambda_{k_0,i} = b \sum_{j=1}^n M_{i,j}v_j \quad \forall i \mbox{ such that } \rho(i) \in S
      $$
      Thus if $i \mbox{ is such that } \rho(i) \in S$ the simulator can compute
      $$
        K_{k_0,i} = (g^b)^{\frac{\sum_{j=1}^n M_{i,j}v_j}{z_{k_0,\rho(i)}'}} = g^{\frac{\lambda_{k_0,i}}{z_{k_0, \rho(i)}}}
      $$
Otherwise, if $i \mbox{ is such that } \rho(i) \notin S$ the simulator computes
      \begin{align*}
        K_{k_0,i} &= g^{\frac{\sum_{j=1}^n M_{i,j}(v_j +(r - v_1)y_j)}{z_{k_0,\rho(i)}'}} (g^a)^{\frac{\sum_{j=1}^n M_{i,j}y_j}{z_{k_0,\rho(i)}'}}  = g^{\frac{\lambda_{1,i}}{z_{k_0, \rho(i)}}}
      \end{align*}
      Remembering that in this case $z_{k_0, \rho(i)} := b z_{k_0, \rho(i)}'$.
      Finally for the queries to the other authorities $k \in A \setminus \{k_0\}$, the simulator chooses uniformly at random a vector $\vec{t}_k \in \mathbb{Z}_p^n$ such that $t_{k, 1} = -r_k$ and implicitly sets the shares $\lambda_{k,i} = b \sum_{j=1}^n M_{i,j} t_{k,j}$ by computing
      $$
        K_{k,i} =
        \begin{cases}
          (g^b)^{\frac{\sum_{j=1}^n M_{i,j} t_{k,j}}{z_{k, \rho(i)}'}} = g^{\frac{b \sum_{j=1}^n M_{i,j} t_{k,j}}{z_{k, \rho(i)}'}} = g^{\frac{\lambda_{k,i}}{z_{k, \rho(i)}}} &\text{ if } i \in S \\
          g^{\frac{\sum_{j=1}^n M_{i,j} t_{k,j}}{z_{k, \rho(i)}'}} = g^{\frac{b \sum_{j=1}^n M_{i,j} t_{k,j}}{b z_{k, \rho(i)}'}} = g^{\frac{\lambda_{k,i}}{z_{k, \rho(i)}}} &\text{ if } i \notin S
        \end{cases}
      $$

    \paragraph*{Challenge}
      The adversary gives two messages $m_0, m_1$ to the simulator.
      It flips a coin $\mu$.
      It creates:
      \begin{align*}
        C' &= m_{\mu} \cdot T \stackrel{*}{=} m_{\mu} \cdot e(g,g)^{a b s} \\
        &=  m_{\mu} \cdot \left(e(g,g)^{(a b + b\left(\sum_{k \in A \setminus \{k_0\}} r_k\right)}\prod_{k \in A \setminus \{k_0\}} e(g,g)^{b r_k} \right)^s \nonumber \\
        C_{k,i} &= (g^s)^{z_{k, \rho(i)}'} = g^{s z_{k, \rho(i)}} \qquad k \in A, \quad i \in S
      \end{align*}
      Where the equality $\stackrel{*}{=}$ holds if and only if the BDH challenge was a valid tuple (i.e. $T$ is non-random).

    \paragraph*{Phase II}
      During this phase the simulator acts exactly as in \emph{Phase~I}.

    \paragraph*{Guess}
      The adversary will eventually output a guess $\mu'$ of $\mu$.
      The simulator then outputs $0$ to guess that $T = e(g, g)^{a b s}$ if $\mu' = \mu$; otherwise, it outputs $1$ to indicate that it believes $T$ is a random group element in $\mathbb{G}_2$.
      In fact when $T$ is not random the simulator $\mathcal{B}$ gives a perfect simulation so it holds:
      $$
        Pr\left[\mathcal{B}\left(\vec{y},T=e(g,g)^{a b s}\right)=0\right] = \frac{1}{2} + \epsilon
      $$
      On the contrary when $T$ is a random element $R \in \mathbb{G}_2$ the message $m_{\mu}$ is completely hidden from the adversary point of view, so:
      $$
        Pr\left[\mathcal{B}\left(\vec{y},T=R\right)=0\right] = \frac{1}{2}
      $$
      Therefore, $\mathcal{B}$ can play the decisional BDH game with non-negligible advantage$~\frac{\epsilon}{2}$.
  \end{proof}