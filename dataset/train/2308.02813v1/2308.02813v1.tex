
\documentclass[sigconf]{acmart}

\AtBeginDocument{\providecommand\BibTeX{{Bib\TeX}}}

\setcopyright{acmcopyright}
\copyrightyear{2023}
\acmYear{2023}
\setcopyright{acmlicensed}
\acmConference[MM '23] {Proceedings of the 31st ACM International Conference on Multimedia}{October 29--November 3, 2023}{Ottawa, ON, Canada}
\acmBooktitle{Proceedings of the 31st ACM International Conference on Multimedia (MM '23), October 29--November 3, 2023, Ottawa, ON, Canada}
\acmPrice{15.00}
\acmDOI{10.1145/3581783.3612404}
\acmISBN{979-8-4007-0108-5/23/10}


\settopmatter{printacmref=true}

\acmSubmissionID{3055}
\newcommand\hmmax{0}
\newcommand\bmmax{0}
\usepackage{xspace}
\usepackage{subcaption}
\usepackage{siunitx}
\usepackage{cleveref}
\usepackage{wrapfig}
\usepackage{stmaryrd}
\usepackage{amsfonts}
\usepackage{bm}
\usepackage{amsmath}
\usepackage{multirow}
\usepackage{graphicx}



\usepackage{multirow}
\usepackage{array}


\begin{document}


\title{Deep Image Harmonization in Dual Color Spaces}

\author{Linfeng Tan}
\orcid{0009-0003-3376-0989}
\affiliation{\institution{MoE Key Lab of Artificial Intelligence,}
  \institution{Shanghai Jiao Tong University}
  \country{China}
}
\email{tanlinfeng@sjtu.edu.cn}

\author{Jiangtong Li}
\orcid{0000-0003-3873-4053}
\affiliation{\institution{MoE Key Lab of Artificial Intelligence,}
  \institution{Shanghai Jiao Tong University}
  \country{China}
}
\email{keep_moving-Lee@sjtu.edu.cn}

\author{Li Niu}
\orcid{0000-0003-1970-8634}
\authornote{Corresponding authors.}
\affiliation{\institution{MoE Key Lab of Artificial Intelligence,}
  \institution{Shanghai Jiao Tong University}
  \country{China}
}
\email{ustcnewly@sjtu.edu.cn}

\author{Liqing Zhang}
\authornotemark[1]
\orcid{0000-0001-7597-8503}
\affiliation{\institution{MoE Key Lab of Artificial Intelligence,}
  \institution{Shanghai Jiao Tong University}
  \country{China}
}
\email{zhang-lq@cs.sjtu.edu.cn}

\begin{abstract}

Image harmonization is an essential step in image composition that adjusts the appearance of composite foreground to address the inconsistency between foreground and background. 
Existing methods primarily operate in correlated  color space, leading to entangled features and limited representation ability. 
In contrast, decorrelated color space (\emph{e.g.}, ) has decorrelated channels that provide disentangled color and illumination statistics. 
In this paper, we explore image harmonization in dual color spaces, which supplements entangled  features with disentangled , ,  features to alleviate the workload in harmonization process. 
The network comprises a  harmonization backbone, an  encoding module, and an  control module. 
The backbone is a U-Net network translating composite image to harmonized image. Three encoders in  encoding module extract three control codes independently from , ,  channels, which are used to manipulate the decoder features in harmonization backbone via  control module. 
Our code and model are available at \href{https://github.com/bcmi/DucoNet-Image-Harmonization}{https://github.com/bcmi/DucoNet-Image-Harmonization}.

\end{abstract}

\begin{CCSXML}
<ccs2012>
    <concept>
        <concept_id>10002950.10003648.10003649.10003656</concept_id>
        <concept_desc>Mathematics of computing~Stochastic differential equations</concept_desc>
        <concept_significance>300</concept_significance>
    </concept>
    <concept>
        <concept_id>10010147.10010178.10010224.10010240.10010243</concept_id>
        <concept_desc>Computing methodologies~Appearance and texture representations</concept_desc>
        <concept_significance>500</concept_significance>
    </concept>
    <concept>
        <concept_id>10010147.10010371.10010382</concept_id>
        <concept_desc>Computing methodologies~Image manipulation</concept_desc>
        <concept_significance>500</concept_significance>
    </concept>
    <concept>
       <concept_id>10010147.10010178.10010224</concept_id>
       <concept_desc>Computing methodologies~Computer vision</concept_desc>
       <concept_significance>500</concept_significance>
    </concept>
</ccs2012>
\end{CCSXML}

\ccsdesc[500]{Computing methodologies~Image manipulation}
\ccsdesc[500]{Computing methodologies~Computer vision}


\keywords{image harmonization,decorelated color space,image composition}
\maketitle

\section{Introduction}\label{section:Introduction}

\begin{figure}[ht]
  \centering
  \includegraphics[width=0.95\linewidth]{figs/sample_v2.pdf}
  \caption{
    We randomly sample 1000 pixels from 100 real images in iHarmony4~\cite{dovenet} and plot the correlation between every two channels in \textit{RGB} (\emph{resp.}, \textit{Lab}) color space in the top (\emph{resp.}, bottom) row. It can be seen that \textit{RGB} channels have strong positive correlations, while \textit{Lab} channels are decorrelated. 
  }
  \label{fig:example}
\end{figure}


Image composition~\cite{niu2021making} targets at generating a composite image by merging foreground and background. Nevertheless, the foreground and background in the obtained composite image might have appearance discrepancy, which is caused by different lighting, climate, and capture devices between foreground and background. To tackle this challenge, image harmonization~\cite{sunkavalli2010multi, xue2012understanding, dovenet, issam, CDTNet} modifies the foreground appearance to ensure its compatibility with the background.
Early traditional image harmonization methods~\cite{sunkavalli2010multi, xue2012understanding, song2020illumination, lalonde2007using} are often designed based on low-level color and illumination statistics. However, with the rapid advance of deep learning techniques, deep image harmonization methods~\cite{dovenet, issam, CDTNet, harmonizer} have become dominant and achieved impressive results.

Existing deep image harmonization methods have been developed from different aspects (\emph{e.g.}, attention mechanism, domain/style transfer, Retinex theory, color transfer) to address the appearance mismatch between foreground and background.
In detail, some works~\cite{ssam, feature_mod} explored attention mechanism to adjust the foreground features more effectively. 
Besides, some works~\cite{dovenet, bargainnet} approached image harmonization as the translation from foreground domain to background domain with additional loss to guide the domain transfer.
Moreover, some works~\cite{intrinsic, IHT} introduced Retinex~\cite{land1971lightness} theory to image harmonization tasks by decoupling an image into reflectance and illumination. 
Recently, some works~\cite{harmonizer, CDTNet} considered the balance between effectiveness and efficiency, and solved image harmonization in the form of color transfer.
Despite the success achieved by existing methods, they mainly operate in  color space to extract and adjust features.
However,   color space is a correlated color space and the entangled  features may increase the workload of existing harmonization methods.


As known to all, an image can be represented in various color spaces, such as , , or . 
These color spaces can be categorized into two groups: correlated color spaces and decorrelated color spaces. 
In correlated color spaces (\emph{e.g.}, , \textit{XYZ}),  different channels are strongly correlated and tend to change simultaneously. 
In contrast, in decorrelated color spaces (\emph{e.g.}, \textit{YUV}, ), different channels are decorrelated.
By taking  as an example decorrelated color space, \textit{L} represents lightness, \textit{a} represents the spectrum from green to red, and \textit{b} represents the spectrum from blue to yellow. 
In \Cref{fig:example}, we plot the correlation between every two channels in  (\emph{resp.}, ) color spaces in the top (\emph{resp.}, bottom) row. It can be observed that  \textit{RG}, \textit{RB}, and \textit{GB} in  color space exhibit strong positive correlations, while \textit{La}, \textit{Lb}, and \textit{ab} in  color space are decorrelated.
Considering the correlation within the  color space, the extracted  features may not effectively disentangle the independent factors of color and illumination statistics, which potentially complicates the harmonization process~\cite{dovenet,issam,rainnet,ssam}. 
However, the decorrelated~ color space contains decorrelated factors (\emph{i.e.}, lightness, orthogonal colors) in three channels, serving as a valuable complement to the entangled features extracted from  color space. 
Moreover, recent studies~\cite{liang2022inharmonious, wu2022inharmonious} on inharmonious region localization have revealed that the decorrelated color space can help identify the inharmonious region, which also motivates us to explore image harmonization in the decorrelated color space. 


Our primary insight for image harmonization is to alleviate the workload of harmonization process by supplementing the entangled  features with the disentangled , ,  features. 
To this end, we propose a novel image harmonization network in \textbf{Du}al \textbf{Co}lor Spaces (\textbf{DucoNet}). 
Our DucoNet comprises a  harmonization backbone, an  encoding module, and an  control module. 
The harmonization backbone is a U-Net network responsible for harmonizing the input composite image in the  color space. In detail, the backbone takes in the  channels and the foreground mask, producing the  channels of the harmonized image. 
The  encoding module consists of three encoders to extract the , ,  control codes from , ,  channels of the composite image independently. 
The  control module interacts with the harmonization backbone to adjust the decoder features with , ,  control codes. 
Each control code adjusts the decoder features in multiple decoder layers of the harmonization backbone. Specifically, each control code is used to generate dynamic convolution kernels~\cite{styleganv2}, which are applied to the foreground region in the decoder feature maps. The decoder feature maps manipulated using three control codes are fused to produce the harmonized image. 
Considering that , ,  channels may contribute differently to various images or even various pixels, we tend to learn pixel-wise weights for three channels when fusing the decoder feature maps manipulated using three control codes, which could also provide hints for the contributions of , ,  channels when harmonizing a specific image. 

The effectiveness of our DucoNet is verified through extensive experiments of low/high-resolution harmonization on the benchmark dataset iHarmony4~\cite{dovenet} and real composite images.
Our contribution can be summarized as follows: 1) To the best of our knowledge, we are the first to investigate image harmonization in both correlated and decorrelated color spaces. 2) We propose a novel image harmonization network in Dual Color Spaces (DucoNet) with   encoding module and control module, which supplements entangled  features with disentangled , ,  features. 3) Extensive experiments on the benchmark dataset demonstrate that our DucoNet outperforms the state-of-the-art approaches by a large margin.


\section{Related Work}

\subsection{Image Harmonization}
As a subtask in image composition~\cite{niu2021making}, image harmonization aims to create a harmonious composite image by ensuring that the appearances of foreground and background are consistent. 
In the early stage, traditional image harmonization methods~\cite{lalonde2007using} focused on adjusting the low-level illumination and color statistics of foreground to match the background.

In recent years, deep learning based harmonization methods have brought significant advance to this research field. 
Unsupervised image harmonization methods~\cite{zhu2015learning} were initially explored using adversarial learning.
With the introduction of the first large-scale image harmonization dataset iHarmony4~\cite{dovenet}, supervised image harmonization methods~\cite{charmnet, Scs-co, S2CRNet, ssh, xing2022composite,peng2022frih,bao2022deep,zhu2022image,chen2023dense,chen2022hierarchical, PHDNet, ren2022semantic} have received increasing attention. 
Among them, some works~\cite{ssam, issam, feature_mod} designed attention modules to extract background features and adjust the foreground features through channel-wise adjustment~\cite{ssam}, semantic representation~\cite{issam, dih}, and modulation-demodulation~\cite{feature_mod}.
Additionally, some works~\cite{dovenet, rainnet, bargainnet} formulated image harmonization as domain/style translation, and employed adversarial learning~\cite{dovenet}, region-aware AdaIn~\cite{rainnet}, and contrastive loss~\cite{bargainnet} to transfer the foreground into the background domain/style. 
Moreover, some works~\cite{intrinsic, IHT, HT} introduced Retinex~\cite{land1971lightness} theory to image harmonization by decomposing the harmonization task into reflectance maintenance and illumination adjustment. 
Recently, some works~\cite{harmonizer, CDTNet} treated image harmonization as color-to-color transformation~\cite{CDTNet} or image-level regression~\cite{harmonizer}, striking a good balance between effectiveness and efficiency in high-resolution image harmonization.

Existing methods mainly rely on the correlated  space to extract the background features and adjust the foreground features. However, the entangled  features may increase the workload of harmonization network and impede the harmonization performance.
Our work focuses on dual color spaces (\emph{i.e.},  and ), by using the decorrelated  color space to generate \textit{L}, \textit{a}, and \textit{b} control codes for feature manipulation in harmonization backbone. 


\subsection{Color Spaces}
There are multiple color spaces to represent images, such as , , \textit{XYZ}, which can be divided into correlated and decorrelated color spaces based on whether each color channel correlates with each other. 
The correlated color space can be directly shown in different monitors and reflect the basic physics rules, for example,  represents three primary colors of light.
However, the correlations among different color channels may prevent the critical factors to be encoded independently and complicate the color transformation~\cite{reinhard2001color}.
On the contrary, the decorrelated color space usually disentangles some critical factors (\emph{i.e.}, lightness), which may help extract the corresponding features independently.
Most works in computer vision field predominantly use  color space. 
Nevertheless, some works also utilize multiple color spaces~\cite{peng2023u, li2021underwater} to achieve the desired effect. 

For example, in underwater image enhancement~\cite{peng2023u, li2021underwater, ma2022wavelet, zhang2022underwater}, it is important to incorporate multiple color spaces to enhance model capabilities.
Among them, Peng~\emph{et al.}\cite{peng2023u} integrated , , and  color spaces into a loss function to improve the contrast and saturation of the enhanced image. 
Li~\emph{et al.}~\cite{li2021underwater} proposed a multi-color encoder to enrich the diversity of feature representations by incorporating the characteristics of , , and  color spaces into a unified structure. 
Zhang~\emph{et al.}~\cite{zhang2022underwater} studied the near-independent properties of  color space, and proposed an adaptive method to enhance the contrast and saturation in  color.
In grayscale image coloring, Wan~\emph{et al.}~\cite{wan2020automated} utilized the  color space to colorize the initialized super-pixel, and then employed the \textit{YUV} color space for color propagation to achieve a balance between efficiency and effectiveness. 
In video tracking, Lai~\emph{et al.}~\cite{lai2020mast} investigated loss designation in terms of different color spaces (\emph{e.g.}, , , and \textit{HSV}), revealing that the decorrelated color space could force models to learn more robust features.

Our work is the first deep image harmonization method using multiple color spaces. Specifically, we extract disentangled , ,  features from decorrelated  color space, to supplement the entangled  features extracted from correlated  color space. 

\subsection{Dynamic Neural Network}
Dynamic neural networks aim to dynamically adjust the model parameters or structures to cope with different conditions, which can improve the generalization and representation ability of models.

For dynamic neural networks with dynamic parameters, Chen~\emph{et al.}~\cite{dynamic_conv} were the first to propose dynamic convolution, which aggregates multiple convolution kernels based on attention weight. 
CondConv~\cite{condconv} introduced the idea of learning sample-dependent convolution kernels to replace original convolution layers, resulting in improved model performance for classification and detection tasks. 
PAC~\cite{PAC} proposed the pixel-adaptive convolution operation by combining learnable local pixel features with the filter weights to change the standard convolution operation. 
In terms of dynamic neural networks with dynamic structures, MSDNet~\cite{huang2017multi} proposed a multi-scale DenseNet with an early-exit strategy that decides when to exit the network for different samples. 
ATC~\cite{graves2016adaptive} developed an algorithm that enables recurrent neural networks to learn the number of computational steps between receiving an input and emitting an output, making previously inaccessible problems manageable.

In our  control module, inspired by StyleGANv2~\cite{styleganv2}, we use \textit{L}, \textit{a}, and \textit{b} control codes to generate dynamic convolution kernels for feature manipulation, which falls within the scope of dynamic parameters. 
This approach enables us to adjust the decoder features in the harmonization backbone using the \textit{L}, \textit{a}, and \textit{b} control codes.

\section{Method} \label{section:Method}

In this section, we will set forth to our DucoNet. 
In detail, we will first briefly introduce our overall framework in~\Cref{Overview}, and our used harmonization backbone in~\Cref{Backbone Network}.
In~\Cref{Color Style Encoder}, we will detail the process to extract the \textit{L}, \textit{a}, \textit{b} control codes.
In~\Cref{Color transfer Module}, we will describe how our  control module (-CM) exploits the \textit{L}, \textit{a}, \textit{b} control codes to adjust the decoder features in the harmonization backbone.

\begin{figure*}[ht]
  \centering
  \includegraphics[width=0.80\linewidth]{figs/framework.drawio_v6.pdf}
  \caption{The illustration of our harmonization network with Dual Color Spaces (DucoNet). 
  Given a composite image  and its foreground mask , the harmonization backbone~\cite{issam} takes \textit{RGB} channels of composite image~() concatenated with  as input, and generates the harmonized image . 
  In \textit{Lab} encoding module, three encoders extract control codes , , and  from \textit{L}, \textit{a}, and \textit{b} channels of composite image~, respectively,
  which are used to manipulate the decoder feature maps in the harmonization backbone. 
  We insert \textit{Lab} control module (\textit{Lab}-CM) into each decoder layer. For the -th decode feature map  output from the -th decoder layer, 
  we use , , and  to manipulate  independently through style blocks~\cite{styleganv2}. 
  Then, three manipulated decoder feature maps are fused as  with learnt pixel-wise weights. 
  Finally, the foreground of  and the background of  are combined as  and sent back to the decoder to produce the harmonized image .}
  \Description{}
  \label{fig:framework}
\end{figure*}


\subsection{Overview}\label{Overview}

Given a composite image  and its foreground mask , the goal of image harmonization is adjusting the foreground of  and producing the harmonized image  as output. 
Prior works~\cite{issam, CDTNet, dovenet, harmonizer, IHT} only use the composite image in the  color space as input. 
However,  color space is a correlated color space, which may increase the workload of existing methods to disentangle independent factors (\emph{e.g.}, lightness, orthogonal colors), potentially complicating the harmonization process.
Considering that the decorrelated  color space contains disentangled color and illumination statistics, we additionally use the composite image with  channels as input to help improve the  harmonization performance. 

As shown in \Cref{fig:framework}, the overall framework consists of three parts: the harmonization backbone, the  encoding module, and the  control module. 
Following previous works~\cite{issam, ssam}, the harmonization backbone uses the composite image with  channels~ concatenated with the foreground mask~ as input. We have also tried using  color space in harmonization backbone, but the results are compromised (see \Cref{section:Ablation Study}). Therefore, we still use  color space in harmonization backbone.
Considering the effectiveness and efficiency, we adopt iSSAM~\cite{issam} as our harmonization backbone, which can also be easily replaced by other harmonization backbones.
For the  encoding module, we use the composite image with  channels  concatenated with the foreground mask~ as input. 
Considering that the , , and  channels are near-independent, we process different channels , ,  using three encoders , ,  separately to obtain the corresponding , , and  control codes .
 control module uses , , and  control codes to adjust the decoder feature maps in the harmonization backbone. 
Finally, the decoder of harmonization backbone outputs the harmonized image , which is supervised by the ground-truth image  using  loss . 

\subsection{Harmonization Backbone}\label{Backbone Network}

The choice of harmonization backbones should balance effectiveness and efficiency simultaneously.
Therefore, we opt for iSSAM~\cite{issam} as our harmonization backbone, which is framed as a U-Net~\cite{ronneberger2015u} with four encoder layers and three decoder layers.
The first three encoder layers output features, which are connected with the corresponding decoder layers via skip connections to preserve the encoded information.
To tailor for image harmonization, an Spatial-Separated Attention Module~\cite{ssam} and a blending layer~\cite{issam} are inserted to the last decoder layer. For more details, please refer to iSSAM~\cite{issam}.  

As mentioned earlier, the harmonization backbone still uses the composite image with  channels~ concatenated with the foreground mask~ as input, and outputs the harmonized result .
To adjust the decoder feature maps with the , , and  control codes, each decoder feature map is sent into our -CM along with the , , and  control codes, which allows disentangled , ,  features to help produce more harmonious images. 
The details of -CM will be introduced in \Cref{Color transfer Module}.

\subsection{\textit{Lab} Encoding Module}\label{Color Style Encoder}

The  color space has been well explored in image harmonization tasks~\cite{dovenet, ssam, issam, intrinsic, IHT, rainnet, feature_mod, bargainnet, harmonizer, CDTNet}. Due to the correlation among  channels, the extracted RGB features may not disentangle the independent factors (\emph{e.g.}, lightness, orthogonal colors) effectively. 
Thus, we additionally use the decorrelated  color space to supplement  color space. 
As introduced in \Cref{section:Introduction}, , , and  channels in  color space represent lightness, the spectrum from green to red, and the spectrum from blue to yellow, respectively.

In the  color space, we attempt to obtain the control code of each channel using the respective control encoder. 
Each encoder , , and ) in the  encoding module has the same structure as the encoder of the harmonization backbone, followed by a pooling layer and a fully-connected layer. Each encoder extracts the independent feature from one channel, which serves as the control code to manipulate the decoder feature maps through our  control module (-CM).
In detail, we first convert the composite image from  color space  to  color space , and obtain three separate channels , , and . 
These three single-channel composite images are concatenated with the  and delivered to the corresponding control encoders to yield the corresponding control code.

By taking the  channel~ as an example, the \textit{L} control code  is generated through the following steps. 
We first scale the range of  to , and then concatenate it with  as input. The concatenation is sent into E to produce the feature map , which is then transformed into the  control code   through one pooling layer  and one fully connected layers .
The whole process for generating , , and  control codes can be formulated as


With three control encoders, we get three control codes , and  corresponding to three channels. 
They encode the independent factors of color and illumination statistics from the composite image in  color space, which can further provide guidance for decoder feature manipulation in our  control module.

\subsection{\textit{Lab} Control Module}\label{Color transfer Module}
Our  Control Module (-CM) aims to migrate useful information from the decorrelated  color space to the  color space, by using three control codes to manipulate the decoder feature maps in the harmonization backbone. 
Recall that our harmonization backbone has three decoder layers and the output feature map from the -th decoder layer is denoted as . We insert -CM after each decoder layer. 
For the -th decoder layer, -CM takes  along with , ,  control codes as input, producing the -enhanced decoder feature map~. Precisely, we first use three control codes to get three manipulated feature maps independently, and then fuse them using learnt pixel weights.  

\noindent\textbf{Feature Map Manipulation: }By taking the decoder feature map  from the first decoder layer as an example, we attempt to use three control codes , and  to manipulate  independently and obtain three manipulated decoder feature maps. 
In this work, we adopt the style block proposed in StyleGANv2~\cite{styleganv2}, which is essentially dynamic convolution. The style block produces dynamic convolution kernel using the control code and apply it to the decoder feature map. 

Specifically, for each color channel  from \{\textit{L, a, b}\}, we have one  base convolution kernel , and use control code  to dynamically scale the input channels of . We first project  to a scale vector  using two fully-connected layers, in which  contains the scales for each input channel. The scaling process is represented by

in which  is the -th entry in  with  enumerating the input channel, output channel, and the spatial location respectively.  is the -th entry in , representing the scale for the -th input channel. Then, we normalize  as

where  is a small constant to prevent numerical errors.  form the dynamic convolution kernel , which acts upon the decoder feature map  to produce the manipulated feature map . For more details of the style block, please refer to StyleGANv2~\cite{styleganv2}.

With three control codes, we can get three manipulated feature maps~, , . By using  P to denote the style block for the color channel , the feature map manipulation can be formulated as



\noindent\textbf{Feature Map Fusion: } Considering that , ,  channels may contribute differently to various images or even various pixels, we learn pixel-wise weights  for three channels when fusing three manipulated feature maps . Specifically, we concatenate three manipulated feature maps and send them to :

where G is constructed by a  convolution layer and a softmax layer,  are single-channel weight maps. After that, we fuse three manipulated feature maps~(, , ) using the predicted pixel-wise weights. Note that we only manipulate the foreground feature map, aiming to make it compatible with the background. Thus, the original background feature map in  is preserved. The above process is represented by

where  means element-wise product and  is the final -enhanced feature map. 

Similar steps can be applied to decoder feature maps  and  to get the -enhanced feature maps  and . The -enhanced feature maps are sent back to the decoder of the harmonization backbone to generate the final harmonized image .


\begin{figure*}[htbp]
  \centering
  \includegraphics[width=0.94\linewidth]{figs/best_results_baseline_IHD_7_v2.pdf}
  \caption{From left to right, we show the composite image ( foreground outlined  in green), the harmonized results of iSSAM~\cite{issam}, CDTNet~\cite{CDTNet}, Harmonizer ~\cite{harmonizer}, DCCF~\cite{DCCF}, our DucoNet, and the ground-truth in iHarmony4~\cite{dovenet} dataset. Best viewed in color and zoom in.}
  \Description{}
  \label{fig:baseline_IHD}
\end{figure*}

\begin{table*}[t]
    \centering
    \setlength\tabcolsep{0.9pt}
    \resizebox{\textwidth}{!}{
    \begin{tabular}{c|ccc|ccc|ccc|ccc|ccc}
    \toprule
        \multirow{2}{4em}{\textbf{Method}} & \multicolumn{3}{c|}{\textbf{All}} & \multicolumn{3}{c|}{\textbf{HCOCO}} & \multicolumn{3}{c|}{\textbf{HFlickr}} & \multicolumn{3}{c|}{\textbf{HAdobe5k}} & \multicolumn{3}{c}{\textbf{Hday2night}}  \\
        \cline{2-16}
        &\textbf{MSE } & \textbf{fMSE  } & \textbf{PSNR  } 
        &\textbf{MSE } & \textbf{fMSE  } & \textbf{PSNR  }
        &\textbf{MSE } & \textbf{fMSE  } & \textbf{PSNR  }
        &\textbf{MSE } & \textbf{fMSE  } & \textbf{PSNR  }
        &\textbf{MSE } & \textbf{fMSE  } & \textbf{PSNR  } \\ 
    \midrule
        Composite images & 172.47 & 1387.30 & 31.63 & 69.37 & 1013.27 & 33.94 & 264.35 & 1612.59 & 28.32 & 345.54 & 2137.07 & 28.16 & 109.65 & 1443.05 & 34.01 \\
        DoveNet~\cite{dovenet} & 52.36  & 549.96  & 34.75  & 36.72  & 554.55  & 35.83  & 133.14  & 823.64  & 30.21  & 52.32  & 383.91  & 34.34  & 54.05  & 1075.42  & 35.18   \\ 
        RainNet~\cite{rainnet}& 40.29  & 469.60  & 36.12  & 31.12  & 535.40  & 37.08  & 117.59  & 751.12  & 31.64  & 42.85  & 320.43  & 36.22  & 47.24  & 852.12  & 34.83   \\ 
        Instrinsic~\cite{intrinsic} & 38.71  & 400.29  & 35.90  & 24.92  & 416.38  & 37.16  & 105.13  & 716.60  & 31.34  & 43.02  & 284.21  & 35.20  & 55.53  & 797.04  & 35.96   \\ 
        IHT~\cite{IHT} & 27.89  & 295.56  & 37.94  & 14.98  & 274.67  & 39.22  & 67.88  & 471.04  & 33.55  & 36.83  & 242.57  & 37.17  & 49.67  & 736.55  & 36.38   \\ 
        iSSAM~\cite{issam} & 24.64  & 262.67  & 37.95  & 16.48  & 266.14  & 39.16  & 69.68  & 443.63  & 33.56  & 22.59  & 166.19  & 37.24  & 40.59  & 591.07  & 37.72   \\ 
        CDTNet~\cite{CDTNet} & 23.75  & 252.05  & 38.23  & 16.25  & 261.29  & 39.15  & 68.61  & 423.03  & 33.55  & 20.62  & 149.88  & 38.24  & 36.72  & 549.47  & 37.95   \\ 
        Harmonizer~\cite{harmonizer} & 24.26  & 280.51  & 37.84  & 17.34  & 298.42  & 38.77  & 64.81  & 434.06  & 33.63  & 21.89  & 170.05  & 37.64  & \textbf{33.14}  & 542.07  & 37.56   \\ 
        DCCF~\cite{DCCF} & 22.05  & 266.49  & 38.50  & 14.87  & 272.09  & 39.52  & 60.41  & 411.53  & 33.94  & 19.90  & 175.82  & 38.27  & 49.32  & 655.43  & 37.88   \\ 
        \cline{1-16}
        DucoNet & \textbf{18.47} & \textbf{212.53} & \textbf{39.17} & \textbf{12.12} & \textbf{211.25} & \textbf{40.23} & \textbf{51.71} & \textbf{353.81} & \textbf{34.65} & \textbf{17.06} & \textbf{141.55} & \textbf{38.87} & 38.70 & \textbf{527.07} & \textbf{38.11} \\
    \bottomrule
    \end{tabular}}
    \caption{Comparison of different methods with image size 256  256 on iHarmony4.  (\emph{resp.}, ) indicates that lower (\emph{resp.}, higher) values are better. The best results are highlighted in bold face. }
    \label{table:1}
\end{table*}

\begin{table}[t]
    \centering
    \setlength\tabcolsep{7.8pt}
    \resizebox{\columnwidth}{!}{
    \begin{tabular}{c|ccc}
    \toprule
        \textbf{Method} & \textbf{MSE  } & \textbf{fMSE  }  & \textbf{PSNR }  \\ 
    \midrule
         Composite images & 352.05 & 2122.37 & 28.10  \\ 
         iSSAM~\cite{issam}            & 25.03 & 168.85 & 38.29  \\
         CDTNet-256(sim)~\cite{CDTNet} & 31.15 & 195.93 & 37.65  \\
         CDTNet-256~\cite{CDTNet}      & 21.24 & 152.13 & 38.77  \\
         Harmonizer~\cite{harmonizer}  & 20.12 & 150.99 & 38.45 \\
         DCCF~\cite{DCCF}              & 21.12 & 171.17 & 38.38  \\
        \cline{1-4}
         DucoNet             & \textbf{10.94} & \textbf{80.69} & \textbf{41.37} \\
    \bottomrule
    \end{tabular}}
    \caption{Comparison of different methods with image size  on HAdobe5k.  (\emph{resp.}, ) indicates that lower (\emph{resp.}, higher) values are better. The best results are denoted in bold face.}
    \label{table:2}
\end{table}



\section{Experiments}
\subsection{Datasets and Evaluation Metrics}
\subsubsection{Dataset} 
Following previous image harmonization works, we conduct experiments on the benchmark dataset iHarmony4~\cite{dovenet} to evaluate the effectiveness of our DucoNet, where the iHarmony4~\cite{dovenet} has been widely used in supervised image harmonization. 
In detail, iHarmony4~\cite{dovenet} consists of four sub-datasets, including HFlickr, Hday2night, HCOCO, and HAdobe5K, with 73,146 samples in total.
For each sample in iHarmony4~\cite{dovenet}, it includes a composite image, its foreground mask, and the corresponding ground-truth image.

We perform both low-resolution and high-resolution image harmonization based on iHarmony4. 
For low-resolution harmonization, we conduct experiments with image size  following previous works~\cite{issam}.
For high-resolution harmonization, we follow the experimental setting in CDTNet~\cite{CDTNet}. Specifically, we perform training and testing based on the HAdobe5k dataset with image size . 
Moreover, we also evaluate our trained model on 100 high-resolution real composite images collected in CDTNet~\cite{CDTNet}. 
Since real composite images have no ground-truth image for evaluation, we present the user study results.

\subsubsection{Evaluation Metrics} 
We adopt the evaluation metrics which are commonly used in previous image harmonization works~\cite{dovenet,IHT,issam,CDTNet,harmonizer,DCCF}, including MSE (Mean-Square-Error), fMSE (foreground Mean-Square-Error), and PSNR (Peak Signal to Noise Ratio).

\subsection{Implementation Details}

Our network is implemented with PyTorch 1.10.1, optimized by Adam optimizer with initial learning rate as .
The batch size is set as 64 and we train our DucoNet for 120 epochs in total.
The learning rate decay starts at epoch 105 and epoch 115 with a decay factor of 10. 
The hardware devices used for training are Intel(R) Xeon(R) Silver 4116 CPU, with 128GB memory and two NVIDIA GeForce RTX 3090 GPUs.
More details about the implementation can be found in Supplementary.


\subsection{Comparison with Start-of-the-Art Methods}
\noindent\textbf{Low-resolution Harmonization: } We compare our method with the existing methods. 
In the low-resolution setting with image size , we compare our method with DoveNet~\cite{dovenet}, RainNet~\cite{rainnet}, Instrinsic~\cite{intrinsic}, IHT (Image Harmonization with Transformer)~\cite{IHT}, iSSAM~\cite{issam}, CDTNet~\cite{CDTNet}, Harmonizer~\cite{harmonizer}, and DCCF~\cite{DCCF}. The experiment results are copied from original papers or reproduced with the released models.

In \Cref{table:1}, we report the results on four sub test sets and the whole test set in the low-resolution setting.
For the results on the whole test set, our DucoNet outperforms the SOTA method by a large margin.
Specifically, our DucoNet achieves 15.68\% relative improvement over CDTNet~\cite{CDTNet} in terms of fMSE and 16.23\% relative improvement over DCCF~\cite{DCCF} in terms of MSE. 
Considering each sub test set, our DucoNet achieves the best results on HCOCO, HFlickr, and HAdobe5k, which indicates the generation ability our method.
On Hday2night, our method achieves the best results in terms of fMSE and PSNR, and the third best result for MSE, probably due to the small-scale training set and test set (only 311 images for training and 133 image for test). 

We further visualize the harmonized results of different methods in \Cref{fig:baseline_IHD}.
It can be seen that our method can produce more visually appealing and harmonious results, that are closer to the ground-truth real images. These visualisation results again demonstrate the effectiveness of our proposed method. 

\noindent\textbf{High-resolution Harmonization: } 
Recently, there are also a few works that focus on high-resolution image harmonization.
In the high-resolution setting with image size , we compare our  DucoNet with iSSAM~\cite{issam}, CDTNet~\cite{CDTNet}, Harmonizer~\cite{harmonizer}, DCCF~\cite{DCCF} in HAdobe5k subset with image size . 
CDTNet-256 is the CDTNet~\cite{CDTNet} model with the input size of harmonization backbone being , and CDTNet-256(sim) is a simplified version of CDTNet-256. 
The experimental results for DCCF, CDTNet-256 and CDTNet-256(sim) are copied from the corresponding paper. 
Harmonizer did not report their results in the same high-resolution setting as CDTNet~\cite{CDTNet}, so we train the corresponding models on HAdobe5k training set with image size  for fair comparison. 

In \Cref{table:2}, we report the results on HAdobe5k in the high-resolution image harmonization setting.
Our DucoNet outperforms all the baselines by a large margin in terms of all evaluation metrics in high-resolution image harmonization.
Specifically, our DucoNet achieves 45.63\% relative improvement over Harmonizer~\cite{harmonizer} in terms of MSE and achieves 46.56\% relative improvement over Harmonizer~\cite{harmonizer} in terms of fMSE. 


\subsection{Ablation Study}\label{section:Ablation Study}

As described in \Cref{section:Method}, our DucoNet consists of the harmonization backbone, the  encoding module, and the  control module (-CM). 
In this section, we demonstrate the effectiveness of each component and each color space by ablating each component or comparing with alternatives.

The results of our ablation studies are presented in \Cref{table:3}. Firstly, when only using the harmonization backbone, we compare using the input composite image with  channels (row 1) and using the input composite image with  channels (row 2).
By comparing row 1 and row 2,  we see that  channels outperforms  channels, revealing that  channels are still more suitable as the input for the U-Net structure. 
Note that although the inputs to the network are different, the loss and evaluation metrics are all calculated based on  channels for fair comparison.
In detail, when using the input composite image with , we first generate the harmonized image with  channels and then convert it into  color space for loss calculation and evaluation.

To evaluate the effectiveness of  color space for feature manipulation, we treat the  (\emph{resp.}, ) channels as a whole input in the encoding module and use a single control code in the control module, leading to the results in row 3 (\emph{resp.}, row 4). 
Comparing row 3 and row 4, we can find that the  channels are more helpful for feature manipulation, because  color space could supplement  color space with extra useful guidance.

In row 5, we study a simple way to fuse  and  features. In particular, we treat the  channels as a whole input in encoding module and send multi-scale encoder features to the decoder via skip-connection, in the same way as the backbone encoder. The obtained performance is worse than row 3,  which demonstrates the effectiveness of feature manipulation in our -CM.

Furthermore, we conduct experiments by treating each individual , ,  channel as the input in the encoding module and use the single control code in the control module (row 6 \emph{v.s.} row 7 \emph{v.s.} row 8). 
Experimental results shows the  channel is the most effective one among all three channels. 
To provide some insights for the importance of  channel, we calculate the amount of change between the foreground area of composite image and the ground-truth image for each channel (, , and ), the average amount of change in three channels are 25.90, 3.88, and 6.65 respectively over the entire test set.
The average amount of change in  channel is significantly higher than the other two channels, which corroborates that merely using  channel could achieve compelling results (row 6). 

Finally, we conduct experiments to verify the effectiveness of the pixel-wise weighting strategy.
Comparing row 9 with row 10, we can find that simply averaging the manipulated feature maps~, ,  undermines the representation ability of -CM, since , ,  channels contribute differently to the harmonization results.

\begin{table}[t]
    \centering
    \setlength\tabcolsep{3.7pt}
    \resizebox{\columnwidth}{!}{
    \begin{tabular}{c|c|c|c|ccc}
    \toprule
        \makebox[1em][c]{} & \textbf{}& \textbf{} & \textbf{Fusion} & \textbf{MSE  } & \textbf{fMSE  } & \textbf{PSNR } \\ 
    \midrule
    1&    iSSAM &  -    & - & 24.64 & 262.67 & 37.95 \\
    2&      -   & iSSAM & -  & 28.13 & 296.59 & 37.20  \\
    3&    iSSAM & E() & CM   & 19.30 & 222.22 & 38.93  \\
    4&    iSSAM & E(\textit{RGB}) & CM & 22.66 & 245.38 & 38.62   \\
    5&    iSSAM & E() & SC   & 21.76 & 243.06 & 38.47 \\
    6&    iSSAM & E(\textit{L})   & CM & 21.43 & 234.70 & 38.72   \\
    7&    iSSAM & E(\textit{a})   & CM & 23.32 & 256.34 & 38.39   \\
    8&    iSSAM & E(\textit{b})   & CM & 23.46 & 255.29 & 38.36 \\
    9&    iSSAM & E(\textit{L,a,b}) & CM-avg & 20.45 & 227.71 & 38.88 \\
    10&   iSSAM & E(\textit{L,a,b}) & CM-pix & 18.47 & 212.53 & 39.17 \\
    \bottomrule
    \end{tabular}}
    \caption{The ablation study of our DucoNet. 
    ``iSSAM" indicates using the harmonization backbone~\cite{issam} in the corresponding color space.
    ``E(\textit{Lab})", ``E(\textit{RGB})", ``E(\textit{L})", ``E(\textit{a})", ``E(\textit{a})", and ``E(\textit{L,a,b})" indicate that we treat \textit{Lab} as a whole, \textit{RGB} as a whole, only \textit{L}, only \textit{a}, only \textit{b}, and  \textit{L,a,b} separately as input in the \textit{Lab} encoding module.
    ``SC" is short for skip-connection.
    ``CM" is short for \textit{Lab}-CM.
    ``CM-avg'' indicates average fusion.
    ``CM-pix'' indicates weighted fusion with pixel-wise weights.
    }
    \label{table:3}
\end{table}

\subsection{Visualization of Weight Map}

\begin{figure}[t]
  \centering
  \includegraphics[width=0.95\linewidth]{figs/weight_v3.pdf}
  \caption{From left to right, we show the composite image ( foreground outlined  in green), the ground-truth, the harmonized results of our method, visualization of  in \textit{Lab}-CM. In composite image and ground-truth image, we also show the average value in \textit{L}, \textit{a}, \textit{b} channels within the foreground region. Best viewed in color and zoom in.}
  \Description{}
  \label{fig:weight_visual}
\end{figure}

To show the effectiveness of our proposed -CM, we visualize the weight maps  from the third decoder layer in \Cref{fig:weight_visual}. Recall that we only manipulate the foreground region of decoder feature map and the background pixel weights do not contribute to the final output. Thus, we mask out the background pixels and only show the pixel weights in the foreground region in \Cref{fig:weight_visual}, in which brighter pixel indicates higher weight.
For composite image and ground-truth, we also show the average value in , ,  channels within the foreground region, which reflects the amount of change in each channel. 

Based on \Cref{fig:weight_visual}, we observe that the learnt weight map is closely related to the amount of change in each channel. 
Recall that , , and  channels in  color space represent lightness, the spectrum from green to red, and the spectrum from blue to yellow, respectively. 
When the lightness between foreground and background in the composite image is contrastively different (row 1), the value of  channel would change greatly after harmonization, in which case the weight map  corresponding to the  channel has the largest values. 
When the foreground object has dominant color (row 3) or the lighting has color cast (row 2),  the value of the corresponding color channel (\emph{e.g.}, red, blue) would vary greatly after harmonization, in which the corresponding weight map has the largest values (\Cref{fig:weight_visual}).

\subsection{Real Composite Images}
Following previous works, we also evaluate different methods on 100 real composite images in CDTNet~\cite{CDTNet}.  The visualization results of different baseline methods are provided in the Supplementary.
Since these real composite images do not have ground-truth image, we conduct user study to compare different methods, which is also left to the Supplementary. 

\section{Conclusion}
In this paper, we have explored image harmonization in dual color spaces, where we additionally use the decorrelated color space  to relieve the burden of the harmonization process when compared with using  color space alone.
We have proposed a novel network DucoNet, which manipulates the foreground of the decoder feature maps from the harmonization backbone using the control codes from  color space. Experiments conducted on the benchmark dataset have shown that our approach significantly outperforms the state-of-the-art methods.

\begin{acks}
 The work was supported by the Shanghai Municipal Science and Technology Major / Key Project, China (Grant No. 20511100300 / 2021SHZDZX0102) and the National Natural Science Foundation of China (Grant No. 62076162).
\end{acks}

\bibliographystyle{ACM-Reference-Format}
\bibliography{acmart}

\end{document}
