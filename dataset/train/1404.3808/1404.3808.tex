
\documentclass[twocolumn]{autart} 
\usepackage{amsfonts}
\usepackage{amssymb}
\usepackage{amsmath}                                                        \usepackage{subfig}
\usepackage{epsfig}
\usepackage{graphicx}
\newtheorem{theorem}{Theorem}
\newcommand{\startproof}{\vspace{0.01cm} \noindent {\bf \em Proof: }}   
\newcommand{\finishproof}{\hfill }    
\newtheorem{assumption}{Assumption}     
\newtheorem{observation}{Observation}                                                \DeclareMathOperator*{\infy}{inf}   
\newtheorem{remark}{Remark} 
\newcommand{\eqdef}{\stackrel{\triangle}{=}}
\linespread{0.8}
\addtolength{\parskip}{-5mm}

\begin{document}
\begin{frontmatter}
\title{Robust Dynamic State Feedback Guaranteed Cost Control of Nonlinear Systems using Copies of Plant Nonlinearities}

\author[SEIT]{Obaid Ur Rehman}\ead{s.obaid.rehman@gmail.com},    \author[SEIT]{Ian R. Petersen}\ead{i.r.petersen@gmail.com}, 
\address[SEIT]{School of Engineering and Information Technology, University of New South Wales, Canberra, Australia.} 
\thanks{This research was supported by the Australian Research Council.}
\begin{abstract}
This paper presents a systematic approach to the design of a robust dynamic state feedback controller using copies of the plant nonlinearities, which is based on the use of IQCs and minimax LQR control. The approach combines a linear state feedback guaranteed cost controller and copies of the plant nonlinearities to form a robust nonlinear controller.
\end{abstract}
\end{frontmatter}
\section{INTRODUCTION}
This paper presents a new approach to the constructive design of a robust nonlinear dynamic state feedback  controller using an integral quadratic constraint (IQC) approach. The idea of using a copy of the plant nonlinearity in the controller is used previously in the literature \cite{outputfeedback_petersen,GLover_Nonlinear,Chu_Nonlinear}. However in this paper, we apply a new methodology to  construct a controller which uses linear state feedback guaranteed cost control and copies of the plant nonlinearities to form a dynamic state feedback robust nonlinear controller. This approach provides robust performance in the case where uncertainties and nonlinearities are present in the plant. 
\section{System Definition}\label{sec:system}
Consider a class of uncertain nonlinear systems described by the following state equations:
\begin{small}

\end{small}

\noindent where  is the state,  is the control input, , , ,  are the uncertainty outputs, , , ,  are the uncertainty inputs,  are the nonlinearity outputs,  are the nonlinearity inputs.
Also,

The nonlinearity inputs and outputs are related as follows:

where the nonlinear functions  satisfy the following generalized monotonicity conditions:

for all ,  and . Also,  are given symmetric matrices representing the monotonicity or global Lipschitz conditions; see \cite{outputfeedback_petersen}. 
Furthermore, we assume that

and the uncertainty in the system satisfies the following integral quadratic constraints, (see \cite{IP}):
\begin{small}

\end{small}
\begin{figure}[t]
\begin{center}
\epsfig{file=nonlinear.eps, scale=0.6}
\caption{Nonlinear system with a nonlinear dynamic state feedback  controller.}
\label{fig:nonlinear}
\end{center}
\end{figure}
\noindent for all . Here the  are given symmetric matrices and the  are given positive definite matrices. Let us define
\begin{small}

\end{small}

\noindent Hence the class of controllers considered here are nonlinear dynamic state feedback controllers which contain a copy of the plant nonlinearities (see Fig. \ref{fig:nonlinear});
\begin{small}

where

and

\end{small}

\noindent Also  is a controller gain matrix. 
The following IQCs, which follow from (\ref{eqmon1}), are satisfied:

for all . Here the , ,  are any positive definite matrices.  
Now, we first move the controller nonlinearities (\ref{eqnon2}) into the plant description and introduce new notation as follows:
\begin{small}

\end{small}

\noindent for all  and . Using the above notation and (\ref{eqnx}), a new system can be written as follows:

 and . Also, the IQCs (\ref{eqIQC2})--(\ref{eqIQC4}) for the nonlinear uncertainty terms can be written as follows:

for  , where  and  are positive definite matrices. We consider the following cost functional associated with the system (\ref{eqsystemc1}):

where  and  are positive-definite symmetric matrices. 
\begin{observation}\label{Obs1}
It is observed that the nonlinearities (\ref{eqnon1}) and (\ref{eqnon2}) satisfy the IQCs (\ref{eqIQC2})--(\ref{eqIQC4}). Hence, it follows that if the linear uncertain system (\ref{eqsystemc1}) and (\ref{eqIQC}), with the linear controller (\ref{eqnestimator}), leads to an upper bound on the cost function (\ref{eqnfcost}) then the same controller (\ref{eqnestimator}) will yield the same upper bound for the uncertain system (\ref{eqsystem}), (\ref{eqnon1}), (\ref{eqIQC1}) and (\ref{eqnon2}). Furthermore, it follows from the above discussion that the system (\ref{eqsystem}), (\ref{eqnon1}), (\ref{eqIQC1}) with controller (\ref{eqnestimator}) and (\ref{eqnon2}) will also lead to the same upper bound on the cost.
\end{observation}
\section{Main Results}\label{sec:main}
We first write the IQC (\ref{eqIQC}) in the following form which is parametrized by a set of multipliers  for :

where , . Furthermore,  where all . Note that we only consider those  which satisfy the following condition

where   represents the number of negative eigenvalues of   and  is the matrix of eigenvectors corresponding to the negative eigenvalues. 
If condition (\ref{eqcondq}) is satisfied, then there exists a matrix  such that the matrix  is a diagonal matrix 

We define 
where ,  and  . Here  are the negative eigenvalues and  are the positive eigenvalues of the matrix .
Now a change in variables is introduced as follows:


The IQCs (\ref{eqIQCM}) for a given  can now  be modified by incorporating new variables as given below (from now on we remove the argument () from the equations wherever possible for the sake of brevity):
\begin{small}
\end{small}
Hence,
\begin{small}
\end{small}  

We have  and , which imply the following relation:

Hence, we obtain

Substituting for  and  into (\ref{eqsystemc1}) gives the following dynamical system:

for all , and  where
\begin{small}

\end{small}

\noindent Also in order to deal with the  terms in (\ref{eqC1system}) we use  standard loop shifting ideas \cite{outputfeedback_petersen,Basar_book} where we require that the following condition is satisfied for all  and : 

For this purpose, we first define the following quantities:

By using the definition in (\ref{eqphis}), we define the transformed uncertainty inputs and outputs as follows:

Hence, 
and (\ref{eqnnIQC}) can be rewritten as follows for all :
\begin{small}

\end{small} 
Now, we re-write (\ref{eqC1system}) using (\ref{eqndef}) as follows:

for all , and  where
\begin{small}

\end{small}

\noindent In order to obtain a bound on the cost function (\ref{eqnfcost}), we design a guaranteed cost controller for the system (\ref{eqC2system}). The theory of guaranteed cost controllers can be found in \cite{IP}. 
In order to apply a guaranteed cost controller of the form (\ref{eqnestimator}), a parameter dependent algebraic Riccati equation is required to be solved for different values of the multipliers  and  for all   and . This Riccati equation is given below:
\begin{small}

\end{small}
where
\begin{small}

\end{small}

\noindent The parameters  and  are chosen such that the Riccati equation (\ref{eqARE}) has a positive definite solution .
\begin{assumption}\label{Ass1}
For any  and   and  satisfying conditions (\ref{eqcondq}), (\ref{eqcondD1}), the following conditions are satisfied:
\begin{enumerate}
\item The Riccati equation (\ref{eqARE}) has a symmetric nonnegative definite solution .
\end{enumerate}
\end{assumption}
\vspace{0.2mm}
If the conditions (\ref{eqcondq}), (\ref{eqcondD1}) along the Assumption \ref{Ass1} are satisfied then a controller of the form (\ref{eqnestimator}) for the system (\ref{eqC2system}) can be obtained as follows:

and hence we define .
The corresponding bound on the cost function is obtained as follows:

\begin{theorem}
Suppose there exist constants  and vectors   and  such that conditions (\ref{eqcondq}), (\ref{eqcondD1}), along with Assumption 1 are satisfied. Then:
\begin{enumerate}
\item If the controller defined by (\ref{eqARE}), (\ref{eqdefAREs}), and (\ref{eqcontrol2}) is applied to the uncertain system defined by (\ref{eqfIQC}), (\ref{eqC2system}), (\ref{eqfIQC}), then the cost functional (\ref{eqnfcost}) satisfies the bound .
\item If the controller defined by (\ref{eqARE}), (\ref{eqdefAREs}), and (\ref{eqcontrol2}) is applied to the  uncertain system defined by (\ref{eqnnIQC}), (\ref{eqC1system}),  then the cost functional (\ref{eqnfcost}) satisfies the bound .
\end{enumerate}
\end{theorem}
\vspace{0.9mm}
\startproof
The first part of the theorem follows directly from the main results of \cite{IP} [Theorem 5.3.1]. The second part of the theorem is a result of  condition (\ref{eqcondD1}) which allows for the system (\ref{eqnnIQC}), (\ref{eqC1system}) to be written in the form  (\ref{eqfIQC}), (\ref{eqC2system}) and hence application of the result in \cite{IP} [Theorem 5.3.1] to this system will result in  as noted in Observation \ref{Obs1}.
\finishproof
\begin{theorem}\label{th1}
Suppose there exist constants  and vectors   and  such that conditions (\ref{eqcondq}), (\ref{eqcondD1}), along with Assumption 1 are satisfied.  If the nonlinear controller defined by (\ref{eqnestimator}), (\ref{eqnon2}), (\ref{eqnstate}) is applied to the nonlinear uncertain system defined by (\ref{eqsystem}), (\ref{eqnon1}), (\ref{eqIQC1}), then the cost functional (\ref{eqnfcost}) satisfies the bound .
\end{theorem}
\vspace{0.25mm}
\startproof
The result directly follows from part (ii) of Theorem \ref{th1} and the construction of the IQC (\ref{eqnnIQC}), the system (\ref{eqC1system}) and the controller (\ref{eqnestimator}) along with the discussion in Observation \ref{Obs1}.
\finishproof
\section{Illustrative Example}\label{sec:example}
\vspace{0.25mm}
An example of state feedback control of axial compressor surge is considered in \cite{outputfeedback_petersen,Arcak_example,Krstic_Adaptive} and is given as follows:
\begin{small}

\end{small}

\noindent where  and  are the system states, and  is the control input. In order to obtain a nonlinearity which is monotonic and sector bounded, we add a linear function to the nonlinearity. We also add an additional uncertainty satisfying an IQC for robustness purposes. Hence, we obtain
\begin{small} 

\end{small}

\noindent where . The uncertainty input  satisfies the IQC 
 for all . We solve the algebraic Riccati equation (\ref{eqARE}) for the steady state stabilizing solution for possible values of  and  satisfying conditions (\ref{eqmon1}), (\ref{eqmon2}), (\ref{eqcondD1}) along with Assumption 1 and considering the IQCs (\ref{eqIQC1}), (\ref{eqIQC2})-(\ref{eqIQC4}). These values of  and  are chosen so that steady state cost bound (\ref{eqbound}) is minimized.   The value of bound on the cost functional (\ref{eqnfcost}) for an initial condition of  and  is obtained as  for the following values of the parameters:
\begin{small}

\end{small}
The cost bound obtained using this scheme is lower than the cost bound obtained in \cite{outputfeedback_petersen} for the same example. This is expected as in the state feedback design we assume that all states are available for measurement. The nonlinear system with the nonlinear state feedback controller has also been simulated using  the above initial conditions and by assuming . The result of the simulation is presented in Fig. \ref{fig:simulation}. It is observed that the control system performance is satisfactory.
\begin{figure}[t]
\hfill
\begin{center}
\epsfig{file=simulation_state.eps, scale=0.6}
\caption{Compressor states ,  and control input .}
\label{fig:simulation}
\end{center}
\end{figure}
\vspace{1mm}
\bibliographystyle{IEEEtran}
\begin{thebibliography}{1}
\providecommand{\url}[1]{#1}
\csname url@samestyle\endcsname
\providecommand{\newblock}{\relax}
\providecommand{\bibinfo}[2]{#2}
\providecommand{\BIBentrySTDinterwordspacing}{\spaceskip=0pt\relax}
\providecommand{\BIBentryALTinterwordstretchfactor}{4}
\providecommand{\BIBentryALTinterwordspacing}{\spaceskip=\fontdimen2\font plus
\BIBentryALTinterwordstretchfactor\fontdimen3\font minus
  \fontdimen4\font\relax}
\providecommand{\BIBforeignlanguage}[2]{{\expandafter\ifx\csname l@#1\endcsname\relax
\typeout{** WARNING: IEEEtran.bst: No hyphenation pattern has been}\typeout{** loaded for the language `#1'. Using the pattern for}\typeout{** the default language instead.}\else
\language=\csname l@#1\endcsname
\fi
#2}}
\providecommand{\BIBdecl}{\relax}
\BIBdecl

\bibitem{outputfeedback_petersen}
I.~R. Petersen, ``Robust output feedback guaranteed cost control of nonlinear
  stochastic uncertain systems via an {IQC} approach,'' \emph{IEEE Transactions
  on Automatic Control}, vol.~54, no.~6, pp. 1299--1304, 2009.

\bibitem{GLover_Nonlinear}
Y.~Chu and K.~Glover, ``Stabilization and performance synthesis for systems
  with repeated scalar nonlinearities,'' \emph{IEEE Transaction On Automatic
  Control.}, vol.~44, no.~3, pp. 484--496, March 1999.

\bibitem{Chu_Nonlinear}
Y.~Chu, ``Further results for systems with repeated scalar nonlinearities,''
  \emph{IEEE Transaction On Automatic Control.}, vol.~46, no.~3, pp.
  2031--2035, Dec 2001.

\bibitem{IP}
I.~R. Petersen, V.~A. Ugrinovskii, and A.~V. Savkin, \emph{Robust control
  design using  methods}.\hskip 1em plus 0.5em minus
  0.4em\relax London: Springer, 2000.

\bibitem{Basar_book}
T.~Basar and P.~Bernhard, \emph{H -Optimal Control and Related
  Minimax Design Problems: A Dynamic Game Approach}.\hskip 1em plus 0.5em minus
  0.4em\relax Boston, MA: Birkh\"auser, 1991.

\bibitem{Arcak_example}
M.~Arcak, M.~Larsen, and P.~Kokotovi\'{c}, ``{C}ircle and {P}opov criteria as
  tools for nonlienar control design,'' \emph{Automatica}, vol.~39, no.~5, pp.
  643--650, 2003.

\bibitem{Krstic_Adaptive}
M.~Krsti\'{c}, I.~Kanellakopoulos, and P.~Kokotovi\'{c}, \emph{Nonlinear and
  Adaptive Control Design}.\hskip 1em plus 0.5em minus 0.4em\relax New York:
  Wiley, 1995.

\end{thebibliography}

\end{document}
