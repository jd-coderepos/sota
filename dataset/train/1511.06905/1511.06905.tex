\documentclass[11pt,a4paper]{llncs}
\usepackage{amssymb}
\usepackage{url}
\usepackage{algorithm}
\usepackage{algorithmic}
\usepackage{epsfig}
\usepackage{graphicx}
\usepackage{epstopdf}
\usepackage{verbatim}
\newtheorem{thm}{Theorem}
\newtheorem{lem}[thm]{Lemma}
\renewcommand{\algorithmicforall}{\textbf{for each}}

\newcounter{ALC@tempcntr}\newcommand{\LCOMMENT}[1]{\setcounter{ALC@tempcntr}{\arabic{ALC@rem}}\setcounter{ALC@rem}{1}\item \{#1\}\setcounter{ALC@rem}{\arabic{ALC@tempcntr}}}\makeatother

\urldef{\mailsa}\path|cs14resch01002@iith.ac.in|

\begin{document}

\title{A Simple Algorithm For Replacement Paths Problem}
\title{A Simple Algorithm For Replacement Paths Problem}
\author{Anjeneya Swami Kare}
\institute{Department of Computer Science and Engineering\\
Indian Institute of Technology\\ Hyderabad, India\\ \mailsa}

\maketitle
\begin{abstract}
 Let  ( and ) be an undirected graph with positive edge weights.
 Let  be a shortest  path in . Let  be the number of edges in .
 The \emph{Edge Replacement Path}  problem is to compute a shortest  path in , for every edge
  in . The \emph{Node Replacement Path} problem is to compute a shortest
  path in , for every vertex  in .

 In this paper we present an  time and  space algorithm
 for both the problems. Where,  is the asymptotic time to compute a single source
 shortest path tree in . The proposed algorithm is simple and easy to implement.
 \end{abstract}

\keywords{Replacement Path, Replacement Shortest Path, Edge Replacement Path, Node Replacement Path, Shortest Path}

\section{Introduction}
Let  ( and ) be an undirected graph with a weight
function  on the edges.
Let  be a shortest
 path in . Let  denote the  number of edges in , also denoted by .
The total weight of the path  is denoted by ,
i.e, , where,  is the
edge . A shortest path tree (SPT) of  rooted at  
(respectively, ) is denoted by  (respectively, ).

A \emph{Replacement Shortest Path} (RSP) for the edge 
(respectively, node ) is a shortest  path in 
(respectively, ). The \emph{Edge Replacement Path} problem is to compute
RSP for all . Similarly, the \emph{Node Replacement Path}
problem is to compute RSP for all . 

Like in all existing algorithms for RSP problem, our algorithm has two phases:
\begin{enumerate}
  \item Computing shortest path trees rooted at  and ,  and  respectively.
  \item Computing RSP using  and .
\end{enumerate}
For graphs with non-negative edge weights, computing
an SPT takes  time, using the standard Dijkstra's algorithm~\cite{dijkstra} with
Fibonacci heaps~\cite{fheaps}. However, for integer weighted graphs (RAM model)~\cite{thorup},
planar graphs~\cite{planarspt} and minor-closed graphs~\cite{minorspt},
 time algorithms are known. In this paper, to compute SPTs  and  (phase 1)
we use the existing algorithms. For phase 2, we present an  time algorithm
which is simple and easy to implement.

Motivation for studying replacement paths problem is its relevance in single link (or node) recovery protocols.
Other problems which are closely related to replacement paths problem are \emph{Most Vital Edge} problem~\cite{fastermve},
\emph{Most Vital Node} problem~\cite{mostvn} and \emph{Vickrey Pricing}~\cite{vickreyprice}.
Often an algorithm for replacement paths problem is used as a subroutine in finding -simple
shortest paths between a pair of nodes.

\begin{table}[H]
\centering
\begin{tabular}{|c|c|}
\hline
  \multicolumn{2}{|c|}{\textbf{Edge Replacement Path Problem}}  \\
  \hline
  Reference & Time Complexity  \\
  \hline
  Malik et al.~\cite{kmve} (1989) &   \\
  \hline
  Hershberger and Suri~\cite{vickreyprice} (1997) &   \\
  \hline
  Nardelli et al.~\cite{fastermve} (2001) &   \\
  \hline
  Jay and Saxena~\cite{jay} (2013) &   \\
  \hline
  Lee and Lu~\cite{linearrsp} (2014) &   \\
  \hline
  This Paper &   \\
  \hline
  \multicolumn{2}{|c|}{\textbf{Node Replacement Path Problem}}  \\
  \hline
  Reference & Time Complexity  \\
  \hline
  Nardelli et al.~\cite{mostvn} (2003) &   \\
  \hline
  Jay and Saxena~\cite{jay} (2013) &   \\
  \hline
  Lee and Lu~\cite{linearrsp} (2014) &   \\
  \hline
  Kare and Saxena~\cite{asksax} (2014) &   \\
  \hline
  This Paper &   \\
  \hline
  \end{tabular}
\label{table1}
\caption{Summary of existing algorithms \protect\footnotemark for RSP problem}
\end{table}

\footnotetext{In the referenced papers, authors ignore the term , as they assume
either shortest path trees are given or restriction on the input graph class for which
linear time algorithms are known for SPT}

For the Edge Replacement Path problem Malik et al.~\cite{kmve} and Hershberger
and Suri~\cite{vickreyprice} independently gave 
time algorithms. Nardelli et al.~\cite{fastermve} gave an 
time algorithm, where  is the inverse Ackermann function.

For the Node Replacement Path problem Nardelli et al.~\cite{mostvn} gave an 
time algorithm. Kare and Saxena~\cite{asksax} gave an  time algorithm.

Jay and Saxena~\cite{jay} gave an  algorithm, where  is the diameter of the graph.
Their algorithm can be used to solve both the edge and the node
replacement path problems. They used linear time algorithms for Range Minima Query
(RMQ)~\cite{rmq} and integer sorting in their solution. A total of  instances, each of
RMQ and integer sorting has been used (with size of each instance at most ).
Recently, Lee and Lu~\cite{linearrsp} gave an  time algorithm.
Table~\ref{table1} summarises the existing algorithms for RSP problem.

In this paper, we present an  time and  space algorithm.
The asymptotic complexity of our algorithm matches that of~\cite{jay}. However, our solution does not use RMQ
and integer sorting. Our algorithm organizes the non-tree edges of the graph in a simple manner.
Moreover, an advantage of our algorithm over~\cite{jay} and~\cite{linearrsp} is that, in a single iteration both
edge and node replacement paths can be obtained, whereas in~\cite{jay} and~\cite{linearrsp} the algorithm has to
be run independently for the edge and node replacement paths. Note that, linear time algorithm for RMQ~\cite{rmq}
and the algorithm in~\cite{linearrsp} are complex to implement. The simplicity of our algorithm
makes it an ideal candidate for the RSP. In particular, for dense graphs and graphs with small diameter
()) our algorithm is optimal and matches with that of~\cite{linearrsp}.
As observed in~\cite{jay}, graphs in real world data sets have small diameter, which further adds significance
to our algorithm.

The contribution of this paper is summarized in the following theorem.

\begin{thm}
\label{thm1}
There is an algorithm for the edge and the node replacement path problems that runs in
 time using  space.
\end{thm}

The rest of the report is organised as follows. In Section~\ref{ersp} we discuss the
algorithm for the edge replacement path problem. In Seection~\ref{nrsp} we discuss the
algorithm for node replacement path problem. We conclude with Section~\ref{conc}.
\section{Edge Replacement Paths}
\label{ersp}
We start by computing shortest path trees  and . In the rest of the section
we describe the algorithm for computing RSP using  and  (phase 2).

A potential replacement path for the edge  can be
seen as a concatenation of three paths ,  and , where,  for
some ,   for some  and
 as shown in \figurename~\ref{fig1}. Here, the symbol  represents a path in .
One extreme case is when  and  (i. e.  and ) as shown in \figurename~\ref{fig1}(b).
Such a replacement path is also a potential replacement path for all the edges . The other extreme case is when 
and  (i.e.  and ) as shown in \figurename~\ref{fig1}(c).
Such a replacement path is a potential replacement path only for the edge .

\begin{figure}[!ht]
\centering
\includegraphics[width=4.0in]{CRP}
\caption{Potential replacement paths for the edge .
The zig-zag lines represent a path} \label{fig1}
\end{figure}

Consider the shortest path tree rooted at  (). When the edge 
is removed from ,  is disconnected into two sub-trees.  (sub-tree rooted at )
and  (sub-tree rooted at ). The vertex sets of  and 
determine a cut in the graph . Let  denote the set of all non-tree edges crossing the cut.
These edges are called crossing edges, i.e,
.
In order to have a replacement path, the set  needs to be nonempty. And,
any replacement path must use at least one crossing edge from .
Moreover, as we see from the Lemmas~\ref{lemmadist} and~\ref{lemmace} there exists
an RSP that uses exactly one crossing edge.



\begin{lem}
\label{lemmadist}
For all ,  =  and  = .
\end{lem}

\begin{proof}
If , then  and .
Shortest  path is fully in  and does not include
the edge . Hence,  = .

To prove , for the sake of contradiction,
let us assume that  (i.e ).
It means that,  uses the edge .
This implies .
Since ,  a contradiction.
Hence, .
\qed
\end{proof}

\begin{lem}
\label{lemmace}
For any edge , there exists a shortest  path in 
which contains exactly one edge from .
\end{lem}

\begin{proof}
Let us consider a shortest  path in  (say ) which
uses more than one crossing edge from . Let  be the last crossing edge
in . Clearly  and . By replacing the part of  from
 to , by the  path in , we get a new path which is not
longer than  and uses exactly one edge from .
\qed
\end{proof}

Using the Lemmas~\ref{lemmadist} and~\ref{lemmace}, we write the total weight of the RSP for
the edge 
as:


All the terms in the equation (\ref{ersplen}) are available in constant time
for a fixed  from  and . Let  be the crossing edge that minimizes
the RHS of the equation (\ref{ersplen}). We call that  the \emph{swap edge}. If we have the swap edge, we can report
the RSP as  in constant time. Every non-tree edge can be a potential crossing edge for every edge in . So, solving equation (\ref{ersplen}) by brute force gives us  time algorithm. In this paper we present an  time algorithm.
In the rest of the paper, we concentrate on computing the swap edge for each .
\subsection{Labeling the nodes of }
\label{labeling}
Every vertex of  is labeled with an integer value from  to , with
respect to the shortest path tree . The process of labeling is as follows:

Let  be the sub-tree rooted at the node  in .
All the nodes in the sub-tree  are labeled with the integer value .
For , all the nodes in the sub-tree 
are labeled with the integer value . See \figurename~\ref{fig2}(a) for an example labeling.

Using pre-order traversal on , we compute the labels of all the vertices in linear time.
We start pre-order traversal from the source vertex  using zero as initial label.
While visiting the children of a node recursively, the child node part of  (if any)
will be visited last with an incremented label.
Let  denote the label of a vertex  in . The following Lemma is straightforward.

\begin{lem}
\label{lemmalabel1}
A non-tree edge  if and only if  and .
In other words, for a non-tree edge , if  and 
for some , then , .
\end{lem}
\subsection{Computing Swap Edges}
We construct a directed acyclic graph which will aid us in computing the swap edges.
We call this DAG as RSP-DAG, denoted by .
The following algorithm explains the construction of the RSP-DAG.
An example RSP-DAG is shown in \figurename~\ref{fig2}(b).

\begin{algorithm}[H]
    \caption{Algorithm to construct the RSP-DAG  = .}
    \label{rspdag}
    \begin{algorithmic}[1]
    \LCOMMENT{Adding Nodes. Each node is identified by an ordered pair }
    \STATE 
    \STATE 
    \FOR{}
        \FOR{}
           \STATE 
        \ENDFOR
    \ENDFOR
    \LCOMMENT{Adding Edges}
    \FORALL{ }
        \IF{}
            \STATE 
            \STATE 
        \ENDIF
    \ENDFOR
    \end{algorithmic}
\end{algorithm}

Clearly, the number of vertices in the RSP-DAG is  and the number of edges is also
. Every node has in-degree and out-degree at most two.
The node with identifier  has zero in-degree. Nodes 
have zero out-degree (sink nodes).

\begin{figure}[!ht]
\centering
\includegraphics[width=4.8in,height=2.3in]{SPT1}
\caption{(a)An SPT rooted at . Solid lines are part of the SPT. Dashed lines represent
the non-tree edges (we omit the edge weights). Number inside the vertex circle denotes the vertex number, where as
the number above the vertex circle represents vertex label.\newline
(b)Corresponding RSP-DAG with set of non-tree edges associated with nodes} \label{fig2}
\end{figure}

For each node , we associate a set  of crossing edges.
This set includes all the non-tree edges  such that  and .
This association of crossing edges partitions the crossing edges into disjoint sets.

\begin{lem}
If the swap edge  for the tree edge  is present in
the edge set () of a node , then there exists a directed path
from the node  to the node  in the RSP-DAG.
\end{lem}
\begin{proof}
Clearly  and , otherwise,  will not be the crossing
edge for . If  is a sink node () in the RSP-DAG, then
the theorem is trivially true.

Otherwise, if we observe the way edges are added in the RSP-DAG, for the node
, two directed edges  and 
are added and from these nodes, we keep adding edges to the lower level nodes in the RSP-DAG. We will
eventually connect to the leaf node . Hence there is
a directed path from  to .
\qed
\end{proof}

Now we make a BFS traversal on the RSP-DAG starting from the node with identifier
. During the traversal, at every node, the minimum cost non-tree edge 
(cost being ) from the corresponding edge set is inserted into the edge sets of its
two children. By the end of this process, minimum cost non-tree edges in the respective
sink nodes give us the swap edges.

\begin{thm}
\label{thm2}
There is an algorithm for
the Edge Replacement Path problem that runs in  time using  space.
\end{thm}

\begin{proof}
 represents the time to compute SPTs  and .
Construction of the RSP-DAG takes  time and  space. BFS traversal on the RSP-DAG
takes  time. During the traversal at each node , we
extract the minimum cost non-tree edge from the set of size at most .
Time complexity of overall edge extraction steps is:
 = . Therefore the total time complexity is .
Space complexity is  which is the space to store the RSP-DAG.
\qed
\end{proof}

Using the linear time algorithms for SPT, for integer weighted graphs,
minor closed graphs our algorithm takes  time.

\section{Node Replacement Paths}
\label{nrsp}
When the node  is removed,
the SPT  is partitioned as:
 (sub-tree rooted at ),  (sub-tree rooted at ) and
 (the remaining forest ).
The crossing edges are denoted as:


\begin{lem}
\label{lemmadistvrp}
For all ,  = , and for all ,  = .
\end{lem}

\begin{proof}
\begin{comment}
As , the shortest  path is fully in  and does not include
the vertex . Hence,  = .

To prove , for the sake of contradiction,
let us assume that . It means that,
 uses the vertex . This implies that, .
Since ,  a contradiction.
Hence, .
\end{comment}
We omit the proof as the proof is similar to lemma~\ref{lemmadist}
\qed
\end{proof}

Using Lemma~\ref{lemmadistvrp}, the length of the RSP is written as:


Having  and , all the terms in the equations (\ref{eq:5}) and (\ref{eq:6}) are available in constant time, except the distance  for  (partial shortest path distance).
We need all the partial shortest path distances ,  and .

To compute all the partial shortest path distances, we use the technique used in~\cite{jay} and~\cite{linearrsp}.

Let  (corresponding to the vertex ) be the graph constructed from  as follows:
The vertex set of , , consists of the source vertex  and the vertices which are
part of the forest . The edge set of , , consists of the following edges:

\begin{itemize}
  \item Edges between the nodes within the forest .
        These edges will get the same edge weight as in .
  \item For every , an edge  is added whenever there
       is at least one edge from  to .
       The weight of this edge is calculated as follows:
       
  \end{itemize}
That is,


 is a graph minor of , since it can be obtained by edge contraction.
Hence, SPT,  of , rooted at  can be constructed in  time.
Moreover,  for any .
As , for any , .
Construction of  and  for all  takes a total time of  = .
 for any  is available in constant time from  of .

Instead of computing  SPTs, , for all , we compute one graph,
, where  is constructed as explained earlier.
 can be constructed from  in  time. Single source shortest path tree rooted at ,
 of  is computed in  time. 
for any  is available in constant time from  of .
Moreover, as ,
the distances  for all  are available in linear time.

To compute  for all , we use the RSP-DAG. We use the vertex labeling on  (as computed in Section~\ref{labeling}),
for a non-tree edge , 
if and only if  and . In other words, for a non-tree edge , if 
and  for some , then , for all .

Hence, the crossing edges  will be part
of edge sets associated with the vertices  in the RSP-DAG. After the BFS traversal on the RSP-DAG,
the minimum cost crossing edge (over ) for  is available in the edge set of the node  in the RSP-DAG.
We do not need to perform the BFS traversal on the RSP-DAG again, because, the data populated during the BFS traversal for the edge replacement
paths suffices.

If we have the swap edge  for the vertex ,
we can report the RSP in constant time as . Here 
is available from  if . It is constructed from SPTs  and   if
.

\begin{thm}
\label{thm3}
There is an algorithm for the
Node Replacement Path problem that runs in  time using  space.
\end{thm}

\begin{proof}
 represents the time to compute SPTs  and .
Computing the distances  for all  takes  time.
Computing  for all  using the RSP-DAG takes  time
and  space. Therefore the total time complexity is .
Space complexity is  which is the space necessary to store the RSP-DAG.
\qed
\end{proof}

Using the linear time algorithms for SPT, for integer weighted graphs,
minor closed graphs our algorithm takes  time.

Proof of Theorem~\ref{thm1} is implied from theorems~\ref{thm2} and~\ref{thm3}.


\section{Conclusions}
\label{conc}
In this paper, we propose an  time  and 
space algorithm for the Replacement Paths problem. The asymptotic complexity of our algorithm matches
with that of Jay and Saxena~\cite{jay}. However,
our algorithm does not require external algorithms RMQ and integer sorting and it is easy to implement.
An advantage of our algorithm over~\cite{jay} and~\cite{linearrsp} is that, in a single iteration
both the edge and node replacement paths can be computed, whereas in~\cite{jay} and~\cite{linearrsp}
the algorithm has to be run independently for the edge and node replacement paths.
For dense graphs and graphs with small diameter our algorithm is optimal and matches that of~\cite{linearrsp}.
\bibliographystyle{splncs}
\bibliography{RSP_IPL}
\end{document}
