\documentclass{LMCS}

\def\doi{8 (3:15) 2012}
\lmcsheading {\doi}
{1--19}
{}
{}
{Nov.~\phantom08, 2011}
{Sep.~13, 2012}
{}
 

\usepackage{enumerate}
\usepackage{hyperref}
\usepackage{tikz}


\theoremstyle{plain}
\newtheorem{claim}[thm]{Claim}
\def\eg{{\em e.g.}}
\def\cf{{\em cf.}}
\newenvironment{example}{\vspace*{1.5mm}\noindent\bf Example
\hspace{ex}\rm}{\mbox{}}

\def\IIR{\mathfrak{R}}
\def\IB{\mathbb{B}}
\def\IC{\mathbb{C}}
\def\ID{\mathbb{D}}
\def\IM{\mathbb{M}}
\def\IN{\mathbb{N}}
\def\IQ{\mathbb{Q}}
\def\IR{\mathbb{R}}
\def\II{\mathbb{I}}
\def\IZ{\mathbb{Z}}
\def\IP{\mathcal{P}}
\def\dotminus{\mbox{  \raisebox{1mm}{}\hspace{-2.2mm}}}
\def\cent{\mbox{ \hspace{-6.9pt}\raisebox{0.5pt}{}\,}}
\newcommand{\dom}{\mbox{\rm dom}}
\newcommand{\rang}{\mbox{\rm range}}
\newcommand{\bC}{\mbox{\bf C}}
\newcommand{\bF}{\mbox{\bf F}}
\newcommand{\upr}{\upharpoonright}
\newcommand{\halfup}{\upharpoonright }
\newcommand{\halfdown}{\downharpoonright }


\newcommand{\ec}{\mbox{\bf EC}}         \newcommand{\lc}{\mbox{\bf LC}}         \newcommand{\rc}{\mbox{\bf RC}}         \newcommand{\SC}{\mbox{\bf SC}}         \newcommand{\wc}{\mbox{\bf WC}}         \newcommand{\dbc}{\mbox{\bf DBC}}       \newcommand{\mc}{\mbox{\bf MC}}         \newcommand{\mmc}{\mbox{\rm -\bf MC}}   \newcommand{\ra}{\mbox{\bf RA}}         \def\delec{\mbox{-\,\bf EC}}
\def\bec{\mbox{{\rm -b}\bf EC}}
\def\cec{\mbox{{\rm -c}\bf EC}}
\def\dec{\mbox{{\rm -d}\bf EC}}
\def\eec{\mbox{-\bf EC}}


\newcommand{\cpf}{\mathbb{CPF}}     \newcommand{\ctf}{\mathbb{CTF}}     \newcommand{\cmf}{\mathbb{CMF}}     \newcommand{\mon}{\mathbb{MON}}     \newcommand{\cbv}{\mathbb{CBV}}     \newcommand{\FIN}{\mathbb{CBV}}     \newcommand{\bv}{\mathbb{BV}}       
\newcommand{\cl}{\mbox{\rm cl}}
\newcommand{\md}{\mbox{\rm mid}}
\newcommand{\Int}{\mbox{\it Int}}

\begin{document}

\title[Point-Separable Classes of Computable Curves]{Point-Separable Classes\\ of Simple Computable Planar Curves}

\author[X.~Zheng]{Xizhong Zheng\rsuper a}	
\address{{\lsuper a}Jiangsu University, Zhenjiang 212013, China, and
Arcadia University, Glenside, PA 19038, USA}	
\email{zhengx@arcadia.edu}
\thanks{{\lsuper a}The first author is supported by DFG  (446 CHV 113/266/0-1),  NSFC (10420130638) and NSFC 61070231}	

\author[R.~Rettinger]{Robert Rettinger}
\address{FernUniversit\"at Hagen, 58084 Hagen, Germany\rsuper b}	
\email{{\lsuper b}robert.rettinger@FernUni-Hagen.de }
\thanks{{\lsuper b}The second author is supported by DFG  (446 CHV 113/266/0-1) and NSFC (10420130638)}



\keywords{Computable Curves, Point separable.}
\subjclass{F.1.3}
\titlecomment{Part of the results have been presented at CCA2009 and MFCS2009.}


\begin{abstract}
\noindent
In mathematics curves are typically defined as the images of continuous real functions ({\em parametrizations}) defined on a closed interval. They can also be defined as connected one-dimensional compact subsets of points. For simple curves of finite lengths, parametrizations can be further required to be injective or even length-normalized. All of these four approaches to curves are classically equivalent. In this paper we investigate four different versions of {\em computable curves} based on these four approaches. It turns out that they are all different, and hence, we get four different classes of computable curves. More interestingly, these four classes are even {\em point-separable} in the sense that the sets of points covered by computable curves of different versions are also different. However, if we consider only computable curves of computable lengths, then all four versions of computable curves become equivalent. This shows that the definition of computable curves is robust, at least for those of computable lengths. In addition, we  show that the class of computable curves of computable lengths is point-separable from the other four classes of computable curves.
\end{abstract}

\maketitle

\section{Introduction}\label{sec-intro}

A curve is a mathematical model which describes the ``path (or locus) of a continuously moving point". Therefore, a planar curve is defined in mathematics as the image of a continuous function . Surprisingly, under this definition, a curve can be so complicated that it fills even a square (cf. \cite{Peano1890,EHMoo1900}). In fact, as shown independently by Hahn and Mazurkiewicz in about 1913, a point set is a curve if and only if it is a locally connected continuum (we ignore the mathematical details in this paper which are not related to our discussion). However, if we are interested only in the curves which do not cross themselves (i.e., {\em simple}) and have finite length (i.e., {\em rectifiable}), then the curves defined in these ways coincide with our intuition about ``curves" and they do have the ``two-sidedness'' and ``thinness" (cf \cite{Why42}). For rectifiable simple curves, the parametrizations can be required to be injective or length-normalized while the induced class of curves remains the same. Therefore, a rectifiable simple curve can be defined as any of the following: a point set of some special topological properties, the image of a continuous function, the image of an injective continuous function, or the image of a continuous function which is length-normalized.

If a point-movement is ``algorithmically determined'', then its path (the curve) should be considered ``computable''. More precisely, the notion of computable curves can be defined by the effectivization of the classical definition of curves. This naturally raises the question whether the effectivizations of these four definitions of curves mentioned above lead to the same notion of ``computable curves''? Our answer is no, even in a very strong sense. Before we can explain our answer in a more precise way, let us recall first the basic idea of how to define computability of continuous objects in general.

In computable analysis, computability over various continuous structures is typically defined by the Turing-machine-based bit model (see \cite{Ko91,Wei00,BC06}). In order to input a real number  to such a Turing machine, it must be represented by effectively convergent sequences of rational numbers (the {\em names} of ). Here, a sequence  ``converges effectively" means that  for all . A real number  is computable if it has a computable name, i.e., there is a computable sequence of rational numbers which converges to  effectively. Furthermore, a real function  is computable if there is a Turing machine which computes  in the sense that, after inputing any name of a real number  in the domain of , the machine outputs a name of . By the same principle, computability of other mathematical objects can be defined by introducing proper ``naming systems". For example, the computability of subsets of the Euclidean space \cite{BW99}, of semi-continuous functions \cite{WZ97}, of functional spaces \cite{ZW03} are all defined in this way. This approach is also called the ``effectivization" of classical mathematical definitions.

The same approach can be applied to curves as well. In this paper, we only consider plane curves. Curves in higher dimensions can be discussed in essentially the same way. Furthermore we will restrict ourselfes to rectifiable curves unless otherwise said, where a curve is {\em rectifiable} if it has a finite length.
As mentioned above, a curve can be defined as a connected and one-dimensional compact subset. Based on this approach, we can define the computable curves by means of the computability of compact subsets of Euclidean space (\cite{BW99}).  Physically, a curve records the trace of a particle motion. If the particle moves according to some algorithmically definable laws, its trace should be regarded as computable. In mathematical terms, a curve is the range of a continuous function defined on a closed interval and this function is called a parametrization of the curve. Thus, it is also natural to call a curve computable if it has a computable parametrization (see e.g.,  \cite{GLM06,GLM11}).

However, the parametrization of a curve may have various extra properties, if the curve is simple. Here a curve is called {\em simple} if it does not intersect itself, or if it has an injective parametrization. Of course, the parametrization  of a simple curve  is not necessarily injective. If  is not injective, then  retraces some parts of the curve . If a curve  is simple, then it has even an arc-length normalized parametrization. Here, a parametrization   is arc-length normalized roughly means that the function  models a particle movement along the curve  with a constant speed.

In this paper, four versions of computable curves are introduced by effectivizing the above four mathematical approaches to curves. We will see that these four versions of computable curves are all different. The difference of the curve classes defined by simple computable parametrizations and computable injective parametrizations was already shown by Gu, Lutz and Mayordomo in a recent paper \cite{GLM11}. However, in this paper we will distinguish these four versions of computable curves in a much stronger sense. Namely, the sets of points covered by the four classes of computable curves are different. In other words, different versions of computable curves can be separated by the points they cover, or they are ``point-separable" (see definition in Section \ref{sec-point-separable}).

Interestingly, the computability of the curve length plays an important role for the computability of curves. If we look only at curves of computable lengths, then the four effectivizations mentioned above are indeed equivalent. This means that the definition of computable curves is robust, at least, for curves of computable lengths.  On the other hand, Gu, Lutz and Mayordomo  constructed  in \cite{GLM11} a computable curve of non-computable length such that none of its computable parametrizations can be injective, although the curve does not intersect itself. As an open question, they asked whether {\em there exists a point which lies on a computable curve of finite length, but not on any computable curve of computable length}, i.e., if the class of computable curves of computable lengths is point-separable from the class of computable curves of finite lengths. A positive answer will be given in this paper.

Our paper is organized as follows. In Section \ref{sec-comp-curve} we will briefly recall some basic notions related to curves, give the precise definition of computable curves and then show some basic properties of computable curves. In Section \ref{sec-point-separable}, we discuss some basic facts of point-separable classes and show a technical lemma which will be used in the proof of the main theorems. Section \ref{sec-n-comp-curve} investigates the class of length-normalized computable curves and shows a significant difference between this class and the class of computable curves of computable lengths. Then it is shown that  these two  classes are point-separable. In the last Section \ref{sec-main} we prove that the four classes of computable curves mentioned above are all point-separable.


\section{Computable Curves}\label{sec-comp-curve}

In mathematics, a {\em plane curve} is defined as a subset  which is the range of a continuous function , i.e., . This continuous function  is then called a {\em parametrization} of . Here, we use, w.l.o.g., the unit interval  instead of more general closed intervals of the form . Obviously, any curve has infinitely many parametrizations. Geometrically, a curve records the path of a particle movement in the plane. If the particle never visits one position more than once, in other words, if the curve does not intersect itself (or it has an injective parametrization ), then the curve is called {\em simple}. The simple curves defined in this way are also called {\em open}, or {\em Jordan arcs}. If a curve  has a parametrization  which is injective on the interval  and fulfills the condition that , then the curve  is traditionally also called {\em simple}, but it is {\em closed}, or a {\em Jordan curve}. Equivalently, a Jordan curve is the continuous image of the unit circle. In this paper we look only at the open simple curves. But all results are true for closed simple curves as well.

For open simple curves, their lengths can be defined by means of the lengths of polygons which approximate the curves according to Jordan \cite{Jor1882}. More precisely, Let  be a simple curve and let  be an injective continuous parametrization of . Then the {\em length}  of the curve  is defined by

where  is the length of the straight line connecting the points  and , and the supremum is taken over all possible partitions  of the unit interval . The length of a curve  is denoted by . Notice that we actually defined the length  of the function . The length of a simple curve is then the length of an injective parametrization of that curve. It is well known that the length of a simple curve is independent from its (injective) representations.  A curve of finite length is traditionally called {\em rectifiable}. Not every curve, even a simple curve,  has finite length. As already mentioned above, we focus mainly on simple rectifiable curves; unless otherwise stated a curve is always meant to be simple and rectifiable in this paper.

If  is a simple rectifiable curve of the length , then there exists a bijective continuous function  such that the arc  has exactly the length . That is, the arc-length  is used as the argument of the function . Let . Then the function  is a parametrization such that the curve segment  has the length  for all . We call the parametrization  of this property {\em length-normalized} or simply {\em normalized}.  Thus, a simple rectifiable curve can have three different kinds of parametrizations---continuous, injective continuous and normalized. In addition, a curve can also be defined as a connected, one-dimensional, compact point set. By effectivizing  all these approaches to curves, we can introduce four totally different versions of computable curves.

Remember that a real function  is computable if there is a Turing machine  which transfers any name of  to a name of .  Equivalently,  is computable iff there is a computable sequence  of computable rational polygon functions which converges uniformly and effectively to  (see \cite{PR89}). Naturally, a function  is computable if all of its component functions are computable, or equivalently, if there is a Turing machine  which transfers any name of  into a tuple  of names of  respectively, where . In this case, we simply say that  computes the function . Remember also that any computable function must be continuous.

In this paper, an -neighborhood  of a point  with Cartesian coordinates  is the rectangle bounded by the lines  and . A neighborhood  is called rational if  is a rational point and  is a rational number. For a set , the -neighborhood of  is defined by . Then for any two point sets , their Hausdorff distance is defined by . Notice that, we  always have .

Now we can define the different versions of computable curves as follows.

\begin{defi}\label{Def-comp-curve}
Let  be a simple, not necessarily rectifiable, planar curve.
\begin{enumerate}[(1)]
 \item  is called {\em -computable} if there is a computable sequence  of finite sets of rational neighborhoods such that

for all , where  denotes the Hausdorff distance.

\item  is called {\em -computable} if there is a computable function  such that .

\item  is called {\em -computable} if there is an injective computable function  such that .

\item  is called {\em -computable} if  has a computable parametrization  such that the length of the curve segment  is equal to  for all .
\end{enumerate}
\end{defi}

\noindent In item (1) of the definition, the finite sets  of rational neighborhoods are also called compact covers of the curve . The union  means the union of all neighborhoods in , not the union .   The second part of condition (\ref{cond-K-comp}) means that the maximum distance from  to the boundary of the compact cover  is bounded by . W.l.o.g., we can even require that the sequence  is decreasing in the sense that  for all .  The letter  of the -computability comes from the German word ompakt (compact) due to the compact coverings.

In item (2), the letter  stands for  etracable because the parametrization  of a -computable curve  can retrace the curve . Namely, there might be some disjoint subintervals  such that . In this case,  traces the segment  of  more than once, or, we say that  is retraceable.

If the parametrization of a curve  is injective, then  records the movement of a particle with a monotone direction. The letter  in -computability stands for onotonically  directed movement or onotone paramatrization. Notice that, if we consider also closed simple curves, then the monotonicity has to exclude the endpoints of the unit interval.

Finally, if a parametrization  satisfies the condition that the length of the curve segment  is proportional to , i.e.,  for all ,  then it is  normalized. Thus, -computability stands for ormalized parametrization.

By definition \ref{Def-comp-curve}, any -computable curve must be rectifiable. However, it is known that -computable curves can have infinite lengths
(see e.g \cite{Ko98}). We will give a simple proof of this fact below by constructing a  Koch curve, which is well known to be -computable (see e.g. \cite{Kam96}). The main reason for re-proving the following result is to introduce basic curve construction techniques which will be used throughout the more involved proofs in the next sections.

\begin{thm}\label{thm-inf-length-M-curve}
There is an -computable curve  which has infinite length.
\end{thm}

\proof
We will construct a computable sequence  of rational polygons inductively and finally let  be the limiting curve of this sequence. Here, a rational polygon is simply a finite sequence  of rational points  and its (not necessarily simple) curve is the union of all line segments connecting these points in their given order. We use the term polygon to mean both the point sequence and the corresponding curve. In the following we will construct a new polygon  from  by adding new points to the sequence  without deleting the original points or changing their relative order.

Given a polygon  we can define straightforwardly its length-normalized parametrization  by


where  and


Back to the proof of our theorem, we construct the sequence  of polygons as follows:
Let . Then we define  by adding three new points  to . Thus  consists of four line segments of length  and it has a total length , i.e., .  Apparently we have  and  for all .

A similar procedure can be applied to each of the four segments of  to construct a polygon  consisting of 16 segments of the length  and hence . In addition, we have   for all . Continuing this process inductively, we can construct a computable sequence  of rational polygons\footnote{It is possible that some polygons contains non-rational points by this construction. But these points can only be algebraic. In this case these irrational points can be replaced by some close enough rational points to guarantee that the result holds as well. For the simplicity, we skip the details here. } such that

for all  and .

The second part of condition (\ref{RoEq1}) implies that the limit  exists and it is computable, and hence a continuous function which should be a parameterization of the limiting curve  . By definition of the curve length, we have  for all  because  and   for all . Therefore  has an infinite length.

It remains only to be shown that  is also injective. This follows immediately from the fact that  which can be proved by induction on . By the uniform convergence of the sequence , we conclude that , that is,  is an injective parameterization of  and hence  is an -computable curve.
\qed


Although a computable curve may have infinite length, computable rectifiable curves seem more interesting and more important. As mentioned above we will focus on computable curves of finite length in this paper and we denote by  and  the classes of all -, -, - and -computable rectifiable simple curves, respectively. By definition, it is straightforward that we have the following relationship between these four versions of computable curves.

\begin{thm}\label{Thm-comp-curve-subset}
.
\end{thm}
We will see that all four versions of  computable curves are different and hence all the subset relations above are proper.

From (\ref{def-arc-length}) it is straightforward that the length of a rectifiable -computable curve is left computable (see also Theorem of \cite{MZ2008}), where a real number  is left computable, or computably enumerable (c.e. for short), if there is an increasing computable sequence  of rational numbers which converges to . In \cite{GLM11}, Gu, Lutz and Mayordomo have shown that any rectifiable -computable curve also has a left computable length. This can be strengthened further to the -computable curves as follows.

\begin{thm}\label{thm-lc-length-K}
Any rectifiable -computable curve has left computable length.
\end{thm}
\proof If  is a rectifiable -computable curve, then there is a decreasing computable sequence  of rational compact covers of  such that  where  is a finite set of rational neighborhoods for all . Furthermore let  be the corresponding finite set of the closed coverings where each open neighborhood of  is replaced by its closure. Thus  (the union of all sets in ) is a closed rational polygon area built of rational neighborhoods (squares). Since this area contains at least one curve  such that , we can find, for each , a simple polygon  of shortest length in this area such that .
Let  be the length of  and let  be some rational approximation of  with . Thus  is a computable sequence of rational numbers. Obviously we have .

In the following we will prove that the length  of  (and hence also ) will be  arbitrarily close to . Therefore,  is left computable.

Let  be an injective (not necessarily computable) parametrization of . By definition of curve length, for any , there exists a partition  such that . Let  be the corresponding polygon. Then we have .

We try now to compare the lengths of the polygons  and  for large indices . Let

for . Let  and  for technical reasons. Consider the -neighborhoods   of , where . Notice that, the  is small enough such that   if .  Choose an index  large enough such that  and consider the rational compact cover  with  where . Remember that,  is a rational polygon in  of the shortest length such that . This, together with , implies that . In particular, we have  for all  which implies that . That is, there are  such that  for all , where  is a length-normalized parametrization of . Notice that  and  are disjoint neighborhoods but they are connected by a  subarea of  containing the curve segment  of . For the neighborhoods  and , they are also disjoint, and the only possible path in  which connects them must pass through the neighborhood . All shortcut between  and  without passing through  will have a Hausdorff distance greater than . This is generally true for any non-neighboured   and  (i.e., ). Therefore, the polygon  can connect points  only in the order  (or the reverse one). W.l.o.g., we can assume that . For any , the polygon  connects two points  and  by a straight line, while the polygon  may connect the points  and  by several linear segments. Therefore, we have  because  is in the -neighborhood  of ,  and hence . This implies that . Thefore, we can conclude that . Since  is arbitrary, we have 
\qed

The construction in the proof of Theorem \ref{thm-lc-length-K} implies immediately an equivalent characterization of -computable curves as follows.

\begin{cor}\label{cor-K-equivalence}
A rectifiable curve  is -computable if and only if there is a computable sequence  of rational polygons which converges to  effectively in the sense that .
\end{cor}
\proof
If  is a -computable curve, then there is a computable sequence  of rational compact covers of . By a construction given in the proof of Theorem \ref{thm-lc-length-K}, there is a computable sequence  of rational polygons which converges to  effectively.

On the other hand, if  is a computable sequence of rational polygons which converges to  effectively, then we have . Construct a rational compact cover  of  such that . Then  is also a rational compact cover of  such that . That is,  is -computable.
\qed

By Theorems \ref{Thm-comp-curve-subset} and \ref{thm-lc-length-K}, any rectifiable  -, - and -computable curve has also a left computable length. Ko \cite{Ko95a} constructed a ``monster curve" which is -computable (even in polynomial time) with a non-computable length. This implies that the length of a -computable curve is not necessarily computable. Our next theorem shows that the computability of the curve-length plays a very important role in the study of computable curves.

\begin{thm}\label{thm-K+compLengh}
If  is a -computable curve  with a computable length, then  must be -computable.
\end{thm}
\proof
Suppose that  is a -computable curve whose length  is a computable real number. Then there is a decreasing computable sequence  of rational compact covers of  such that  and . There is also an increasing computable sequence  of rational numbers converging to  effectively in the sense that . By the proof of Theorem \ref{thm-lc-length-K}, there exists a computable sequence  of rational polygons such that  for all  and  where . Notice that, because  is decreasing, the sequence  is increasing. Furthermore, we have also that  for all .

For each , we can find a sufficiently large index  such that  . Such an index  exists because both sequences  and  converge to the same limit . Actually we can choose the sequence  to be strictly increasing and . Thus we have .  Since  is a rational polygon, there is a computable function  such that  is a length-normalized parametrization of . Because of the conditions  and , we can choose the computable sequence of functions  such that  for all . In other words, the sequence  converges effectively and hence its limit  is also a computable function which is a length normalized parametrization of . Therefore, the curve  is  -computable.
\qed

The following corollary follows immediately from Theorem \ref{thm-K+compLengh}.

\begin{cor}\label{cor-equi-compLength}
If  is a rectifiable simple curve of computable length, then -, -, -, and -computability of  are equivalent.
\end{cor}

Thus, if we consider only curves of computable length, then it is not necessary to distinguish between -, -, - and -computability of curves. That is, the notion of ``computable curves" is quite robust at least for simple curves of computable lengths.
Therefore, we can denote simply by  the class of computable curves of computable lengths in any of these versions. Later on, we will call a curve {\em computable} (without mentioning the prefixes , ,  or ) if it is an element of .

Now let  be an -computable rectifiable curve which is not -computable (such curve exists by Theorem \ref{thm-M-N-sep}). This curve  is of course also -computable (Theorem \ref{Thm-comp-curve-subset}). By Theorem \ref{thm-K+compLengh},  does not have computable length. Therefore, there exist -,-, and -computable curves which have non-computable lengths.  For -computable curves, we can prove the same property by a direct construction as well. The construction needs the following simple fact.

\begin{prop}\label{prop-polygon-a<b}
Let  and  be any positive rational numbers and let  be a simple rational polygon of length . There is a simple rational polygon  of the length  such that . In addition, we can choose their length-normalized parameterizations  and  such that  as well.
\end{prop}
\proof
For simplicity, just consider the case . For general rational polygon  we need only look at each segment of the  and construct  in a similar way.

Choose an integer  such that . Let   for  and . We define the polygon  by replacing the segment  of  by a polygon  where . Because , we have . Apparently, we also have .

\begin{figure}[h]
\begin{center}
\begin{tikzpicture}
\draw[help lines] (-1,-1) grid (13,2);
\draw [very thick] (0,0) -- (2, 0) -- (2,1) -- (4,1) --(4,0) --(6,0)--(6,1)
--(8,1) --(8,0) --(10,0)--(10,1) --(12,1)--(12,0);

\draw[fill] (0,0) circle [radius=0.08];
\node[below] at (0,0) {};

\draw[fill] (4,0) circle [radius=0.08];
\node[below] at (4,0) {};
\draw[fill] (6,0) circle [radius=0.08];
\node[below] at (6,0) {};
\draw[fill] (6,1) circle [radius=0.08];
\node[above] at (6,1) {};
\draw[fill] (8,1) circle [radius=0.08];
\node[above] at (8,1) {};

\draw[fill] (8,0) circle [radius=0.08];
\node[below] at (8,0) {};
\draw[fill] (12,0) circle [radius=0.08];
\node[below] at (12,0) {};
\end{tikzpicture}
\end{center}
\caption{The polygon \ ()}\label{fig-q}
\end{figure}

\noindent Finally we look at the length-normalized parameterization  and . By construction, we have  and  for all . Because the length  and , we have  for all . This implies immediately that  for all .
\qed

\begin{thm}\label{thm-N-noncomp-length}
For any left computable real number , there is an -computable curve with the length .
\end{thm}
\proof Let  be a left computable real number and let  be an increasing computable sequence of rational numbers which converges to . W.l.o.g., we assume that  and  are positive. By Proposition \ref{prop-polygon-a<b}, we can construct a computable sequence  of rational polygons with  and   inductively as follows.

First, let . For any , if  is already defined with , then define a new polygon  according to the construction of Proposition \ref{prop-polygon-a<b} such that ,  and  where  is a length-normalized parameterization of . This implies that the limit  is a computable function which is a length-normalized parameterization of the limiting curve .

Furthermore, when we construct the polygon  from , we should choose the constant  (of the proof of Proposition \ref{prop-polygon-a<b}) to be smaller than one fourth of all line segments of . In addition we should also choose the extension direction of  carefully. In this way, we can prove by induction that, there is a constant  such that  for all . This concludes that  is an injective function and hence  is -computable.
\qed

In fact, many curves we are familiar with in mathematics have computable length. The following lemma gives a simple sufficient condition that a curve has computable length.

\begin{lem}\label{lem-complength-diff-function}
If an injective parametrization of a simple curve  has a computable derivative, then  has computable length.
\end{lem}
\proof Let  be a one-to-one parametrization of  such that the derivative  is computable as well. Then the arc length of  can be calculated by  which is computable and   is a computable length-normalized parametrization of .
\qed

Thus, by Lemma \ref{lem-complength-diff-function},  line segments connecting two computable points, computable polygons (connecting finitely many computable points by straight lines), computable circles, etc, all have computable length.




\section{Point Separable Classes of Curves}\label{sec-point-separable}

The main goal of this paper is to distinguish different versions of computable curves introduced in Section \ref{sec-comp-curve} in a very strong sense, i.e., by means of point separability. In this section we will introduce formally the notion of point-separability and explore some basic facts about it. Finally we show a technical lemma which are useful in the proofs of our point-separability results.

The non-equivalence of the -computability and the -computability of curves is proved by Gu, Lutz and Mayordomo in \cite{GLM11}. Actually they have shown that there is a polynomial time computable curve   which does not have any injective computable parametrization. In other words, any computable parametrization  of the curve  must be retraced in the sense that  for some disjoint subintervals . Thus, the curve  is -computable but not -computable. In the same paper, Gu, Lutz and Mayordomo asked  whether  there exists a point which lies on a computable curve of finite length but not on any computable curve of computable length?  This leads naturally to the following notion.

\begin{defi}\label{def-point-sep}
Let  and  be classes of curves.
\begin{enumerate}[(1)]
    \item A point  is called  {\em -reachable} if  lies on some curve  of the class .
    \item The class  is called {\em point-separable} from the class  if there is a -reachable point which is not -reachable.
\end{enumerate}
\end{defi}

\noindent Thus, if  and  are the classes of computable curves of finite and computable length, respectively, then, the question of Gu, Lutz and Mayordomo becomes whether  is point-separable from .

Notice that the endpoints of a computable curve are computable, so we can always extend a computable curve from one end so that it starts from the origin. Thus, for  , the -reachable points are just those points on the plane which can be accessed from the origin along some -computable curve.

If  is the class of all planar curves, then all points are -reachable. For some special classes of curves we can prove the point-separability very easily. For example, let  be the class of all rational circles (i.e., centered at rational points with rational radii) and let  be the class of all rational polygons. Then  is point separable from  and vice versa. The proof is quite simple. Given a rational circle , any rational line segment intersects the circle  in at most two points. The number of rational line segments is countable. Since the circle  contains uncountably many points, there must be points on  which do not lie on any rational line segment. Therefore,  is point separable from . The other direction can be proved similarly.

This example can be easily extended to the following proposition.

\begin{prop}\label{prop-p-sep-simple-case}
Let  and  be countable classes of curves such that for any curve  and ,  intersects  at most in countably many points. Then  is point-separable from .
\end{prop}

It makes more sense if  is point-separable from some subclass . In this case,  contains some curve which is significantly more complicated than any curve of . To prove such kind of point-separability, the following technical lemma is very useful. It is based on a simple observation that, if a curve  is not contained in another curve , then there must be a small neighborhood of some point on  which is disjoint from .

\begin{lem}\label{lem-two-curves}
Let  and  be two rectifiable, simple curves and let  be a parametrization of . If we have  for all points  and all open neighborhoods , then there exists an interval  such that .
\end{lem}
\proof
Suppose that  are rectifiable, simple curves. If  for any point  and any , then  must be a part of , i.e., . Otherwise, by the compactness of , we can find a point  in  which has positive distance from  and hence some open neighborhood of  is disjointed from  which contradicts the hypothesis.

As a rectifiable simple curve  has an injective parametrization . Its inverse function  is also continuous which maps particularly two end points of  to . Suppose w.l.o.g. that . Then we have  due to the connectedness of the curve.

Let  be the continuous function defined by . Since , we have . By the continuity of , there exist   and  such that  (we suppose w.l.o.g that ). This implies immediately that .
\qed

By Lemma \ref{lem-two-curves}, if a curve  is not contained in another curve , then there exist a point  of  and a neighborhood  which is disjoint from the curve . Particularly, if  is longer than , then  cannot be contained in . If in addition  is a rational polygon and  is a computable curve, then the point  and the number  can be even rational. Thus, just by ``checking and waiting'' we can always find effectively such a rational point  and the corresponding rational neighborhood . This idea will be used several times in the proofs of Section \ref{sec-main}. In those proofs, we are given a rational polygon  and a (-, -, - or -)computable curve . As long as we can verify that  is sufficiently different from  (and hence  is not contained in ), then we can always find a point  on  and an -neighborhood  which is disjoint from .



\section{Length-Normalized Computable Curves}\label{sec-n-comp-curve}

An -computable curve has a length-normalized computable parametrization. This type of computable curves model the particle motion of constant speed. By Theorem \ref{thm-N-noncomp-length}, an -computable curve does not necessarily have a computable length. Thus, the class  is a proper superset of . Our next result shows that the class  is different from the class  in a very strong way. Namely, for any curve ,  is either an element of , or any non-trivial segment of  is not in .

\begin{lem}\label{lem-N-C-no-containing}
If  is an -computable curve of non-computable length, then no non-trivial segment of  is a computable curve of computable length.
\end{lem}
\proof
Let  be an -computable curve of length  which is not computable. By Definition \ref{Def-comp-curve}, there is an injective computable function  such that  and  for all .

If  is a nontrivial segment of , then there are  in  such that . Suppose by contradiction that  is a computable curve of computable length . Then it must be also -computable by Theorem \ref{thm-K+compLengh}, and it has a normalized computable parametrization . Let  and  be the endpoints of . Both  and  are computable points. Because  is an injective computable function and  and , the numbers  are also computable.

Let  be the length of the segment . Then, we have  and . This implies that . Therefore  is computable which contradicts the hypothesis.
\qed

From a mathematical point of view, Lemma \ref{lem-N-C-no-containing} is quite surprising and even strange. Physically, an -computable curve  can model the algorithmic particle motion of a constant speed. In particular, if the argument  of its parametrization  is regarded as the time, the length  corresponds to the speed of the motion. Thus, an -computable curve of non-computable length is a model of a particle motion with non-computable constant speed, while its trace can be effectively determined. In this case, of course, any of its segments models also a particle motion of a non-computable constant speed.

From Lemma \ref{lem-N-C-no-containing} we can prove the following point-separable result.

\begin{thm}\label{thm-N-C-sep}
There is an -computable curve  and a point on  which is not on any computable curve of computable length. That is, the classes  and  are point-separable.
\end{thm}

\proof
Let  be an -computable curve of a non-computable length and let  be a (not necessarily effective) enumeration of all computable curves of computable length. By Lemma \ref{lem-N-C-no-containing}, the intersection  is a nowhere dense set for any . Thus, the set  is a meager set. This implies immediately that . That is, there is a point on  which is not on  for all .
\qed

Theorem \ref{thm-N-C-sep} answers, even in a stronger sense, the question of Gu, Lutz and Mayordomo \cite{GLM11} that wether there exists a point which lies on a computable curve of finite length but is not covered by any computable curve of computable length, because their notion of computable curves
is the -computable curves.


\section{Point-Separable Classes of Computable Curves}\label{sec-main}

In this section we will prove the point-separability of four versions of the computable curves. The proofs are standard finite injury priority constructions. We sketch only the main ideas, because a priority construction with complete formal details, although it is technically not difficult, will be very long and could hide the essential proof ideas. The detailed explanation about the injury priority construction can be found in \cite{Soa87}.

Remember that a function  is computable if there is a Turing machine  which computes  in the sense that  transfers any sequence  of rational numbers which converges effectively to  to a sequence  of rational points which converges effectively to . Equivalently,  is computable if and only if there is a computable sequence  of rational polygon functions  which converges to  uniformly and effectively. For technical simplicity, we can understand in this section that a Turing machine  computes a function  means that  computes a sequence  of rational polygon functions which converges to  uniformly effectively, i.e.,  for all  and .

Let  be an effective enumeration of all Turing machines such that  possibly computes a computable sequence  of rational polygon functions defined on  in the sense that . If the sequence  converges to  uniformly effectively, then  computes the function  which can be regarded as a parameterization of an -computable curve . The polygon curve defined by  is denoted by . If  doesn't compute a computable sequence of rational polygons, or the sequence doesn't converge effectively, then we say that   computes only an empty curve, i.e., . Therefore,  is an effective enumeration of all -computable curves.

Now we are ready to show that the classes  and  are point-separable. Our proof will use the following fact about the ``sweep" of a continuous function.

Let  be a continuous function, let  be a point and  be a constant. An interval  is called a -sweep of  if there is a point  in the range of  such that   and the function  travels from  to , turns back to  and then go through  and forward again. In other words, there are  with   such that ,  and .  In other words,  retraces the curve segment between  and  two times.


\begin{lem}\label{lem-sweep}
Let  and let  be a continuous function. For any constant , there can be at most finitely many -sweeps where  and .
\end{lem}

\proof
Since  is also uniformly continuous on the interval , there exists a  such that  if . Now, suppose that   is a -sweep of  for some  and  , then we have . If there is another -sweep  which is, say, inside the interval , then if forces the interval length  to be greater than . Therefore, any -sweep, no matter nested or not,  costs at least a length  of the interval .  This implies immediately that the finite interval  can contain only finitely many such sweeps.
\qed


\begin{thm}\label{Thm-K-R-sep}
There exists a rectifiable -computable curve  and a point  on  such that  does not belong to any -computable curve .
\end{thm}
\proof By Corollary \ref{cor-K-equivalence}, a rectifiable curve  is -computable iff there is a computable sequence  of rational polygons which converges to  effectively in the sense that  for all . In the following, we will construct such a computable sequence  of rational polygons which converges effectively to a curve , and at the same time we construct also a computable sequence  of rational points which converges to a point  on . Let  and  be the candidates  constructed at the stage .

Let  be an effective enumeration of all Turing machines and let  be the corresponding enumeration of all -computable curves. Thus, it suffices to guarantee that the constructed -computable curve  and the point  on  satisfy, for all , the following requirements:


We explain the strategy to satisfy a single requirement  first.

Suppose, at stage , that a rational polygon  and a point  on  are defined. In addition, we have also defined a neighborhood  which contains the point  as well as part of . The new rational polygon  will be defined by, if it is necessary, changing part of polygon of  within the neighborhood . Meanwhile, we construct a new neighborhood  which contains the new point candidate  on the polygon  such that  is disjoint from the curve . In this way, we can guarantee that the point  is on the curve , but not on the curve . That is, the requirement  is satisfied.

For simplicity, let  be the box of a side-length  centered at the point  and the polygon  contained in this box is simply the line segment .  Suppose now that  is a total function and  intersects with , otherwise, we need do nothing. Let  be the part of  in the box . If , then, by Lemma \ref{lem-two-curves}, we can find a new neighborhood  and  a point  such that  and hence the requirement  is satisfied. Notice that,  can be determined, say,  by finding a rational point  such that , for some , where  is the intersection of  with . Note that, we can always compute the curve  by the computation of  () up to  steps which is denoted by .

We consider now the case that . By Lemma \ref{lem-sweep},  can have at most finitely many -sweeps for any  and . Therefore, there must be a rational point  and an -neighborhood  of  such that  does not have a  sweep for all  and . We can find such a  by calculating  to sufficient precision, that is, by calculating  for sufficiently large . Otherwise, either  is not a total function, or  is not contained in . Here, ``calculating  to sufficient precision" means we try to find a maximum  such that the computations  all halt, and  is large enough such that the precision  is good enough to determine the ``no-sweep" case.

Suppose that we already find the rational point  such that  does not have any -sweep. Then we define the new polygon  by replacing the linear segment  by the polygon  , where . Apparently, we have .

\begin{figure}[h]
\begin{center}
\begin{tikzpicture}
\draw[help lines] (-1,-1) grid (10,2);
\draw [very thick] (-1,0) -- (4, 0) -- (0,1) -- (8,0) --(10,0);

\draw[fill] (0,0) circle [radius=0.08];
\node[below] at (0,0) {};
\draw[fill] (0,1) circle [radius=0.08];
\node[above] at (0,1) {};
\draw[fill] (4,0) circle [radius=0.08];
\node[below] at (4,0) {};
\draw[fill] (8,0) circle [radius=0.08];
\node[below] at (8,0) {};

\end{tikzpicture}
\end{center}
\caption{The polygon  which simulates a -sweep}\label{fig-K-R}
\end{figure}


After this change, the constructed new polygon  is different enough from the curve  so that, by Lemma \ref{lem-two-curves}, we can find a new neighborhood  and  a point  such that  and hence the requirement  is satisfied. For technical reasons, we should also choose the new neighborhood  small enough such that it doesn't contain any other  for .

\noindent Notice that, in the above construction, the line segment  of length  of  is replaced by  which is a polygon of two line segments of the lengths  and , respectively. Therefore, the length-increment of the new polygon can be estimated as follows:


To satisfy all requirements  simultaneously, we need the technique of the finite injury priority construction. We say that a requirement  has a higher priority than  if . At any stage , we have to construct a finite sequence  of the neighborhoods and a finite sequence of rational points  for some natural number , in addition to the rational polygon ,  such that

and that  is disjoint from the curve . The neighborhood  has to be canceled (by the fact that ) at some stage only if a new neighborhood  is redefined at the stage  for some . In this case, The requirement  is injured.  Whenever a box  is defined according to the strategy mentioned above, it is not necessary to redefine it again unless  is injured by a requirement of higher priority. By an simple induction it is not difficult to prove that any requirement  can be injured no more than  times and  needs to be redefined at most  times. Thus,  exists and  is disjointed from . Similarly,  exists too and .  Because  and the size of  converges to zero if the index  goes to infinity, the limit  exists and  is a point on the curve . Here the existence of the limit and the -computability of the limiting curve  follows from the Corollary \ref{cor-K-equivalence}. The point  belongs to all neighborhoods  and hence is disjointed from all curves . Therefore,  is a point on a -computable curve but is never covered by an -computable curve.

Finally, we can show that the limiting curve  has a finite length. Notice that, for each , the curve length can be increased by the actions for  at most  times, while it can increase at most  each times. This means that the total length-increment caused by  is bounded by . Therefore the total length of  must be finite.
\qed

In the following, we will show that the classes of -computable curves, -computable curves and -computable curves are all point-separable. Because the proofs are finite injury priority constructions similar to that of Theorem \ref{Thm-K-R-sep}, we just give sketches of the proofs.

\begin{thm}\label{thm-R-M-sep}
There exists a rectifiable -computable curve  and a point  on  such that  does not belong to any -computable curve .
\end{thm}

\proof (Sketch) We need only to construct an -computable curve  and a point  on  which satisfy, for all , the requirements

where  is a computable enumeration of all (possibly partial) computable functions . The -computable curve  is defined as the limit of a computable sequence  of rational polygons which converges to  effectively in the sense that  for all . At the same time, we also construct a computable sequence  of real functions  such that  is a computable parametrization of , and the sequence  converges effectively to a computable function  which is a parametrization of . This guarantees that  is an -computable curve. In addition, we construct a sequence  of points such that  is on the polygon  and disjoint from , and  converges to a point  on .

The sequences ,  and  are constructed in stages by a finite injury priority method. We explain the idea of how to satisfy a single requirement  only.

Suppose that, at some stage , we have defined a rectangular box  of a side length  which contains a segment of the polygon  constructed so far, where . For simplicity, let  be the box centered at the point  and let  be the line segment of the polygon  in . Suppose also that the parametrization  defined at the stage  retraces the line  three times, that is, it starts at , goes to , back to  and then goes forward to . Because the length of  is bounded by , we can always define  in this way without violating the effective convergency of the sequence .

Similar to the proof of Theorem \ref{Thm-K-R-sep}, calculating  to sufficient precision so that we can determine the following cases.

Case 1. If  is disjoint from , then we need to do nothing.

Case 2. If  intersects the box  and  closely passes the segment  only once. In this case, replace the segment  by a Z-sweep  of height : . Where  is a sufficiently small rational number.

\begin{figure}[h]
\begin{center}
\begin{tikzpicture}
\draw[help lines] (-1,-2) grid (9,2);
\draw [very thick] (-1,0) -- (0, 0)-- (8,1) --(0,-1)--(8,0) --(9,0);

\draw[fill] (0,0) circle [radius=0.08];
\node[above] at (0,0) {};
\draw[fill] (8,1) circle [radius=0.08];
\node[above] at (8,1) {};


\draw[fill] (0,-1) circle [radius=0.08];
\node[below] at (0,-1) {};
\draw[fill] (8,0) circle [radius=0.08];
\node[below] at (8,0) {};

\end{tikzpicture}
\end{center}
\caption{A Z-sweep polygon }\label{fig-R-M}
\end{figure}

Case 3.  is close to  and also has Z-sweep near . Notice that  dose not have a Z-sweep. We do nothing in this case.

In both cases 2 and 3, since the new polygon  is sufficiently different from , by Lemma \ref{lem-two-curves}, we can choose a new box  which contains part of  and choose a point  on  in the box . This new box  and the new point  can be used as witnesses for the requirement . In addition, to guarantee the finite length of the limiting curve, we should choose  if it is implemented at the stage . Then we can redefine the parametrization  of  (which contains ) such that  . This is possible because the original parametrization  traces the  three times which is very close to the Z-sweep of . This guarantees that the function sequence  converges effectively.

On the other hand, we can estimate the length of  as follows:

Therefore, the length of the polygon  differs from  by no more than . Since the requirement  will be injured at most  times and the curve can be increased due to the strategy for  at most . Thus, the limiting curve is of a finite length.

The strategy described above can be used to satisfy all requirements  simultaneously by a finite injury priority method. The detailed construction is very similar to the proof of Theorem \ref{Thm-K-R-sep} and is omitted here.
\qed

Finally, we show the difference between - and -computability of curves.

\begin{thm}\label{thm-M-N-sep}
There exists a rectifiable  -computable curve  and a point  on  such that  does not belong to any -computable curve . That is, the classes  and  are point-separable.
\end{thm}
\proof (Sketch) We will use the priority technique again to construct an -computable curve  and a point  on  such that the following requirements are satisfied

Again, we want to construct a computable sequence  of rational polygons and a computable sequence  of injective functions which converges to the curve  and the computable function , respectively, such that  is an injective parametrization of . At the same time, we also construct a sequence  of points which converges to a point  on , but  is disjoint from all .

The strategy for satisfying a single requirement  is to find a neighborhood  which is disjoint from  and which contains a segment of  and a point  on this segment. To guarantee that the curve  has a finite length, similar to the proofs of Theorem \ref{Thm-K-R-sep} and \ref{thm-R-M-sep}, we should choose  so that the length of the curve is increased at most  . For simplicity, suppose that  is a neighborhood of size  centered at  and let  be our first candidate of the witness neighborhood. Suppose in addition that the line segment  connecting  and  is the segment of  in the box . Let  be an injective computable parametrization of the (current candidate of) .

Suppose that  is an -computable curve and  is a length-normalized parametrization of . Just wait until the Turing machine  can compute   to sufficient precision. As long as  is disjoint to , we need to do nothing. If  does intersect with , but is not close to the segment , then we can apply Lemma \ref{lem-two-curves} to choose a new neighborhood  which contains part of  but is disjoint from . Otherwise, suppose that  is very close to the segment . That is, there are  such that the segment  almost coincides with . Then compute the middle point  of the segment   and check if it is close to the middle point of . If it is not the case, then  is not length-normalized and we are done. Otherwise, replace the segment  by a polygon  which double the length of the first half of the segment  (i.e. the part from  to ) by introducing small zigzags like the graph in Figure \ref{fig-M-N}.

\begin{figure}[h]
\begin{center}
\begin{tikzpicture}
\draw[help lines] (-1,-1) grid (11,1);
\draw [very thick] (0,0) -- (1/2, 0)-- (1/2,1/2) --(1,1/2)--(1,0) --
(1.5,0)--(1.5,1/2)--(2,1/2)--
(2,0) --(2.5,0)--(2.5,1/2)--(3,1/2)--
(3,0) --(3.5,0)--(3.5,1/2)--(4,1/2)--
(4,0) --(4.5,0)--(4.5,1/2)--(5,1/2)--
(5,0) --(10,0);

\draw[fill] (0,0) circle [radius=0.08];
\node[below] at (0,0) {};
\draw[fill] (5,0) circle [radius=0.08];
\node[below] at (5,-0.2) {};

\draw[fill] (10,0) circle [radius=0.08];
\node[below] at (10,-0.2) {};

\end{tikzpicture}
\end{center}
\caption{A new polygon  with doubled length of the first half segment.}\label{fig-M-N}
\end{figure}

\noindent Denote the new (whole) polygon by . At the same time, modify the function  to a new injective function  such that  is a computable parametrization of .  Now the part of  in  is different enough from the curve  and hence we can apply Lemma \ref{lem-two-curves} to find a point on  and a neighborhood  of   which is disjoint from .

By a standard priority construction, the curve  can be constructed as the effective limiting curve of a computable sequence  of rational polygons, and  has an injective computable parametrization  which is the limit of a computable sequence  of injective functions. In addition, the limit  is a point on  which is disjoint from any , if  is -computable.
\qed


\begin{thebibliography}{10}

\bibitem{BW99}
V.~Brattka and K.~Weihrauch.
\newblock Computability on subsets of {E}uclidean space {I}: Closed and compact
  subsets.
\newblock {\em Theoretical Computer Science}, 219:65--93, 1999.

\bibitem{BC06}
M.~Braverman and S.~Cook.
\newblock Computing over real numbers: Foundation for scientific computing.
\newblock {\em Notics of AMS}, 53(3):318--329, 2006.

\bibitem{Edg95}
G.~A. Edgar.
\newblock {\em Measure, Topology, and Fractal Geometry}.
\newblock Undergraduate Texts in Mathematics. Springer, 1995.

\bibitem{GLM06}
X.~Gu, J.~H. Lutz, and E.~Mayordomo.
\newblock Points on computable curves.
\newblock In {\em Proceedings of FOCS 2006}, pages 469--474. IEEE Computer
  Society Press, 2006.

\bibitem{GLM11}
X.~Gu, J.~H. Lutz, and E.~Mayordomo.
\newblock Curves that must be retraced.
\newblock {\em Information and Computation} 209(2011), 992--1006.


\bibitem{Jor1882}
C.~Jordan.
\newblock {\em Cours d'analyse de l'Ecole Polytechnique}.
\newblock Publications math{\'e}matiques d'Orsay, 1882.

\bibitem{Kam96}
H.~Kamo.
\newblock {\em Computability of Koch Curve and Koch Island}
\newblock Academic Journal 100(1996), 1--8.

\bibitem{Ko91}
K.-I. Ko.
\newblock {\em Complexity Theory of Real Functions}.
\newblock Progress in Theoretical Computer Science. Birkh{\"a}user, Boston, MA,
  1991.

\bibitem{Ko95a}
K.-I. Ko.
\newblock A polynomial-time computable curve whose interior has a nonrecursive
  measure.
\newblock {\em Theoretical Computer Science}, 145:241--270, 1995.

\bibitem{Ko98}
K.-I. Ko.
\newblock On the computability of fractal dimensions and Hausdorff measure.
\newblock {\em Annals of Pure and Applied Logic}, 93:195--216, 1998.

\bibitem{EHMoo1900}
E.~H.~Moore
\newblock On certain crinkly curves.
\newblock {\em Transactions of AMS }, 1(1900):72--90.

\bibitem{MZ2008}
Norbert Th. M\"uller and Xishun Zhao
\newblock Jordan Areas and Grids.
\newblock {\em Electr. Notes Theor. Comput. Sci.}, 2008: 191~206

\bibitem{Peano1890}
G.~Peano.
\newblock Sur une vourbe, qui remplit toute une aire plane.
\newblock {\em Mathematische Annalen}, 36(1):157--160, 1890.

\bibitem{PR89}
M.~B. Pour-El and J.~I. Richards.
\newblock {\em Computability in Analysis and Physics}.
\newblock Perspectives in Mathematical Logic. Springer-Verlag, Berlin, 1989.

\bibitem{Soa87}
R.~I~.Soare.
\newblock {\em Recursively enumerable sets and degrees. A study of computable functions
        and computably generated sets}.
\newblock Springer, Berlin Heidelberg, 1987.

\bibitem{Wei00}
K.~Weihrauch.
\newblock {\em Computable Analysis, An Introduction}.
\newblock Springer, Berlin Heidelberg, 2000.

\bibitem{WZ97}
K.~Weihrauch and X.~Zheng.
\newblock Computability on continuous, lower semi-continuous and upper
  semi-continuous real functions.
\newblock In T.~Jiang and D.~Lee, editors, {\em Computing and Combinatorics},
  volume 1276 of {\em Lecture Notes in Computer Science}, pages 166--175,
  Berlin, 1997. Springer.
\newblock Third Annual Conference, COCOON'97, Shanghai, China, August 1997.

\bibitem{Why42}
G.~T.~Whyburn.
\newblock What is a Curve?.
\newblock {\em The American Mathematical Monthly}, 49(8):493-497, 1942.

\bibitem{ZW03}
N.~Zhong and K.~Weihrauch.
\newblock Computability theory of generalized functions.
\newblock {\em J. ACM}, 50(4):469--505, 2003.

\end{thebibliography}

\end{document}
