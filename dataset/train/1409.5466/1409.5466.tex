\documentclass[11pt,a4paper]{article}

\usepackage{fullpage}
\usepackage[utf8x]{inputenc}
\usepackage{amsmath, amsthm}
\usepackage{wrapfig,graphicx,amssymb,textcomp,array,amsmath}
\usepackage{algpseudocode} 
\usepackage{enumitem}
\algtext*{EndWhile}\algtext*{EndIf}\usepackage{algorithm}

\setlength{\arraycolsep}{0in}


\newcommand{\MC}{M_\times}
\newcommand{\MNC}{M_=}
\newcommand{\MOPT}{M^*}
\newcommand{\bt}{\lambda}
\newcommand{\btopt}{\lambda^*}
\newcommand{\ConvexShape}{\bigtriangledown}
\newcommand{\kTD}[2]{\text{-}TD#2}
\newcommand{\kDT}[2]{\text{-}DG#2}
\newcommand{\kGG}[2]{\text{-}GG#2}
\newcommand{\kRNG}[2]{\text{-}RNG#2}
\newcommand{\Tm}{t_{min}}
\newcommand{\Tg}{T_{gst}}
\newcommand{\WS}[1]{\text{WS}}
\newcommand{\GC}{G_{\mathcal{C}}}

\newcommand{\require}{\textbf{Input: }}
\newcommand{\ensure}{\textbf{Output: }}


\title{Higher-Order Triangular-Distance Delaunay Graphs: Graph-Theoretical Properties}

\author{
Ahmad Biniaz\thanks{School of Computer Science, Carleton University, 
                    Ottawa, Canada. Research supported by NSERC.}
\and 
Anil Maheshwari\footnotemark[1]
\and 
Michiel Smid\footnotemark[1]
}
\date{\today}
\newtheorem{lemma}{Lemma}
\newtheorem{corollary}{Corollary}
\newtheorem{theorem}{Theorem}
\newtheorem{observation}{Observation}
\begin{document}

\maketitle

\begin{abstract}
We consider an extension of the triangular-distance Delaunay graphs (TD-Delaunay) on a set  of points in the plane. In TD-Delaunay, the convex distance is defined by a fixed-oriented equilateral triangle , and there is an edge between two points in  if and only if there is an empty homothet of  having the two points on its boundary. We consider higher-order triangular-distance Delaunay graphs, namely \kTD{k}{}, which contains an edge between two points if the interior of the homothet of  having the two points on its boundary contains at most  points of . We consider the connectivity, Hamiltonicity and perfect-matching admissibility of \kTD{k}{}. Finally we consider the problem of blocking the edges of \kTD{k}{}.
\end{abstract}

\section{Introduction}
The {\em triangular-distance Delaunay graph} of a point set  in the plane, TD-Delaunay for short, was introduced by Chew \cite{Chew1989}. A TD-Delaunay is a graph whose convex distance function is defined by a fixed-oriented equilateral triangle. Let  be a downward equilateral triangle whose barycenter
is the origin and one of its vertices is on negative -axis. A {\em homothet} of  is obtained by scaling  with respect to the origin by some factor , followed by a translation to a point  in the plane: .
In the TD-Delaunay graph of , there is a straight-line edge between two points  and  if and only if there exists a homothet of  having  and  on its boundary and whose interior does not contain any point of . In other words,  is an edge of TD-Delaunay graph if and only if there exists an empty downward equilateral triangle having  and  on its boundary. In this case, we say that the edge  has the {\em empty triangle property}. 
The TD-Delaunay graph is a planar graph, see \cite{Bose2010}.
We define  as the smallest homothet of  having  and  on its boundary. See Figure~\ref{TD}(a). Note that  has one of  and  at a vertex, and the other one on the opposite side. Thus,

\begin{observation}
\label{side-point-obs}
 Each side of  contains either  or .
\end{observation}

In \cite{Babu2013}, the authors proved a tight lower bound of  on the size of a maximum matching in a TD-Delaunay graph. In this paper we study higher-order TD-Delaunay graphs. An {\em order-k TD-Delaunay graph} of a point set , denoted by \kTD{k}{}, is a geometric graph which has an edge  iff the interior of  contains at most  points of ; see Figure~\ref{TD}(b). The standard TD-Delaunay graph corresponds to \kTD{0}{}. We consider graph-theoretic properties of higher-order TD-Delaunay graphs, such as connectivity, Hamiltonicity, and perfect-matching admissibility. We also consider the problem of blocking TD-Delaunay graphs.

\begin{figure}[htb]
  \centering
\setlength{\tabcolsep}{0in}
  
  \caption{(a) Triangular-distance Delaunay graph (\kTD{0}{}), (b) \kTD{1}{} graph, the light edges belong to \kTD{0}{} as well, and (c) Delaunay triangulation.}
\label{TD}
\end{figure}

\subsection{Previous Work}
\label{previous-work}
A {\em Delaunay triangulation} (DT) of  is a graph whose distance function is defined by a fixed circle {\footnotesize } centered at the origin. DT has an edge between two points  and  if there exists a homothet of {\footnotesize } having  and  on its boundary and whose interior does not contain any point of ; see Figure~\ref{TD}(c). In this case the edge  is said to have the {\em empty circle property}. An {\em order-k
Delaunay Graph} on , denoted by \kDT{k}{}, is defined to have an edge  iff there exists a homothet of {\footnotesize } having  and  on its boundary and whose interior contains at most  points of . The standard Delaunay triangulation corresponds to \kDT{0}{}.  

For each pair of points  let  be the closed disk having  as diameter. A {\em Gabriel Graph} on  is a geometric graph which has an edge between two points  and  iff  does not contain any point of . An {\em order- Gabriel Graph} on , denoted by \kGG{k}{}, is defined to have an edge  iff  contains at most  points of .

For each pair of points , let  be the intersection of the two open disks with radius  centered at  and . A {\em Relative Neighborhood Graph} on  is a geometric graph which has an edge between two points  and  iff  does not contain any point of . An {\em order- Relative Neighborhood Graph} on , denoted by \kRNG{k}{}, is defined to have an edge  iff  contains at most  points of . It is obvious that .

The problem of determining whether an order- geometric graph always has a (bottleneck) perfect matching or a (bottleneck) Hamiltonian cycle is quite of interest. We will define these notions in Section~\ref{graph-notions}. 
Chang et al. \cite{Chang1992b, Chang1992, Chang1991} proved that a Euclidean bottleneck biconnected spanning graph, bottleneck perfect matching, and bottleneck Hamiltonian cycle of  are contained in \kRNG{1}{}, \kRNG{16}{}, \kRNG{19}{}, respectively. This implies that \kRNG{16}{} has a perfect matching and \kRNG{19}{} is Hamiltonian. Since \kRNG{k}{} is a subgraph of \kGG{k}{}, the same results hold for \kGG{16}{} and \kGG{19}{}. It is known that \kGG{k}{} is -connected \cite{Bose2013} and 
\kGG{15}{} (and hence \kDT{15}{}) is Hamiltonian. Dillencourt showed that a Delaunay triangulation (\kDT{0}{}) admits a perfect matching \cite{Dillencourt1990} but it can fail to be Hamiltonian \cite{Dillencourt1987a}. 

Given a geometric graph  on a set  of  points, we say that a set  of points blocks  if in  there is no edge connecting two points in . Actually  is an independent set in .
Aichholzer et al.~\cite{Aichholzer2013} considered the problem of blocking the Delaunay triangulation (i.e. \kDT{0}{}) for  in general position. They show that  points are sufficient to block DT() and at least  points are necessary. To block a Gabriel graph,  points are sufficient \cite{Aronov2013}.

In a companion paper, we considered the matching and blocking problems in higher-order Gabriel graphs. We showed that \kGG{10}{} contains a Euclidean bottleneck matching and \kGG{8}{} may not have any. As for maximum matching, we proved a tight lower bound of  in \kGG{0}{}. We also showed that \kGG{1}{} has a matching of size at least  and \kGG{2}{} has a perfect matching (when  is even). In addition, we showed that  points are necessary to block \kTD{0}{} and this bound is tight.
\subsection{Our Results}
\label{our-results}
We show for which values of , \kTD{k}{} contains a bottleneck biconnected spanning graph, bottleneck Hamiltonian cycle, and (bottleneck) perfect-matching. We define these notions Section~\ref{graph-notions}. In Section~\ref{connectivity} we prove that every \kTD{k}{} graph is -connected. In addition we show that a bottleneck biconnected spanning graph of  is contained in \kTD{1}{}. Using a similar approach as in \cite{Abellanas2009, Chang1991}, in Section~\ref{Hamiltonicity} we show that a bottleneck Hamiltonian cycle of  is contained in \kTD{7}{}. We also show a configuration of a point set  such that \kTD{5}{} fails to have a bottleneck Hamiltonian cycle. In Section~\ref{matching} we prove that a bottleneck perfect matching of  is contained in \kTD{6}{}, and we show that for some point set , \kTD{5}{} does not have a bottleneck perfect matching. In Section~\ref{matching2} we prove that \kTD{2}{} has a perfect matching and \kTD{1}{} has a matching of size at least . In Section~\ref{blocking-section} we consider the problem of blocking \kTD{k}{}. We show that at least  points are necessary and  points are sufficient to block a \kTD{0}{}. The open problems and concluding remarks are presented in Section~\ref{conclusion}.

\section{Preliminaries}
\label{preliminaries}
\subsection{Some Geometric Notions}
\label{geometric-notions}
Bonichon et al. \cite{Bonichon2010} showed that a half- graph of a point set  in the plane is equal to a TD-Delaunay graph of . They also showed that every plane triangulation is TD-Delaunay realizable. 

\begin{wrapfigure}{R}{0.35\textwidth}
  \begin{center}
\includegraphics[width=.3\textwidth]{fig/cones.pdf}
  \end{center}
  \caption{The construction of the TD-Delaunay graph.}
\label{cones}
\end{wrapfigure}

A half- graph (or equivalently a TD-Delaunay graph) on a point set  can be constructed in the following way. For each point  in , let  be the horizontal line through . Define  as the line obtained by rotating  by -degrees in counter-clockwise direction around . Actually . Consider three lines , , and  which partition the plane into six disjoint cones with apex . Let  be the cones in counter-clockwise order around  as shown in Figure~\ref{cones}. We partition the cones into the set of {\em odd cones} , and the set of {\em even cones} . For each even cone  connect  to the ``nearest'' point  in . The {\em distance} between  and , , is defined as the Euclidean distance between  and the orthogonal projection of  onto the bisector of . See Figure~\ref{cones}. The resulting graph is the half- graph which is defined by even cones \cite{Bonichon2010}. Moreover, the resulting graph is the TD-Delaunay graph defined with respect to homothets of . By considering the odd cones, another half- graph is obtained. The well-known  graph is the union of half- graphs defined by odd and even cones. To construct \kTD{k}{}, for each point  we connect  to its  nearest neighbors in each even cone around . It is obvious that \kTD{k}{} has  edges. The \kTD{k}{} can be constructed in -time, using the algorithm introduced by Lukovszki~\cite{Lukovszki1999} for computing fault tolerant spanners.

Recall that  is the smallest homothet of  having  and  on its boundary. In other words,  is the smallest downward equilateral triangle through  and . Similarly we define  as the smallest upward equilateral triangle having  and  on its boundary. It is obvious that the even cones correspond to downward triangles and odd cones correspond to upward triangles.   
We define an order on the equilateral triangles: for each two equilateral triangles  and  we say that  if the area of  is less than the area of . Since the area of  is directly related to , 


\begin{figure}[htb]
  \centering
\setlength{\tabcolsep}{0in}
  
  \caption{Illustration of Observation~\ref{obs1}: the point  is contained in . The triangles  and  are inside .}
\label{smaller-triangle-fig}
\end{figure}

As shown in Figure~\ref{smaller-triangle-fig} we have the following observation:

\begin{observation}
\label{obs1}
 If  contains a point , then  and  are contained in .
\end{observation}
As a direct consequence of Observation \ref{obs1}, if a point  is contained in , then . It is obvious that,

\begin{observation}
 \label{equal-triangles}
 For each two points , .
\end{observation}
Thus, we define  as a regular hexagon centred at  which has  on its boundary, and its sides are parallel to , , and . 
\begin{observation}
\label{obs2}
 If  contains a point , then .
\end{observation}
For each edge  in \kTD{k}{} we define its {\em weight}, , to be equal to the area of .
\subsection{Some Graph-Theoretic Notions}
\label{graph-notions}
A graph  is {\em connected} if there is a path between any pair of vertices in . Moreover,  is -{\em connected} if there does not exist a set of at most  vertices whose removal disconnects . In case ,  is called {\em biconnected}. In other words a graph  is biconnected iff there is a simple cycle between any pair of its vertices. A {\em matching} in  is a set of edges in  without common vertices. A {\em perfect matching} is a matching which matches all the vertices of . A {\em Hamiltonian cycle} in  is a cycle (i.e., closed loop) through  that visits each vertex of  exactly once.
In case that  is an edge-weighted graph, a {\em bottleneck matching} (resp. {\em bottleneck Hamiltonian cycle}) is defined to be a perfect matching (resp. Hamiltonian cycle) in  with the weight of the maximum-weight edge is minimized. A {\em bottleneck biconnected spanning subgraph} of  is a spanning subgraph, , of  which is biconnected and the weight of the longest edge in  is minimized. For  we denote the bottleneck of , i.e., the length of the longest edge in , by .

 
For a graph  and , let  be the subgraph obtained from  by removing vertices in , and let  be the number of odd components in . The following theorem by Tutte~\cite{Tutte1947} gives a characterization of the graphs which have perfect matching: 

\begin{theorem}[Tutte~\cite{Tutte1947}] 
\label{Tutte} 
 has a perfect matching if and only if  for all .
\end{theorem}

Berge~\cite{Berge1958} extended Tutte’s theorem to a formula (known as Tutte-Berge formula) for the maximum size of a matching in a graph. In a graph , the {\em deficiency}, , is . Let .


\begin{theorem}[Tutte-Berge formula; Berge~\cite{Berge1958}] 
\label{Berge} 
The size of a maximum matching in  is 
\end{theorem}

For an edge-weighted graph  we define the {\em weight sequence} of , \WS{G}, as the sequence containing the weights of the edges of  in non-increasing order. A graph  is said to be less than a graph  if \WS{G_1} is lexicographically smaller than \WS{G_2}.

\section{Connectivity}
\label{connectivity}
In this section we consider the connectivity of higher-order triangular-distance Delaunay graphs.

\subsection{-connectivity}
\label{connectivity-k-plus-1}
For a set  of points in the plane, the TD-Delaunay graph, i.e., \kTD{0}{}, is not necessarily a triangulation \cite{Chew1989}, but it is connected and internally triangulated \cite{Babu2013}. As shown in Figure~\ref{TD}(a), the outer face may not be convex and hence \kTD{0}{} is not necessarily biconnected. As a warm up exercise we show that every \kTD{k}{} is -connected. 

\begin{theorem}
\label{k-connectivity-thr}
 For every point set , \kTD{k}{} is -connected. In addition, for every , there exists a point set  such that \kTD{k}{} is not -connected.
\end{theorem}
\begin{proof}
We prove the first part of this theorem by contradiction. Let  be the set of (at most)  vertices removed from \kTD{k}{}, and let , where , be the resulting maximal connected components. Let  be the set of all triangles defined by any pair of points belonging to different components, i.e., . Consider the smallest triangle . Assume that  is defined by two points  and , i.e., , where , , and .

{\em Claim 1}:  does not contain any point of  in its interior.
By contradiction, suppose that  contains a point  in its interior. Three cases arise: (i) , (ii) , (iii) , where  and . In case (i) the triangle  between  and  is contained in . In case (ii) the triangle  between  and  is contained in . In case (iii) both triangles  and   are contained in . All cases contradict the minimality of . Thus,  contains no point of  in its interior, proving Claim 1.

By Claim 1,  may only contain points of . Since , there must be an edge between  and  in \kTD{k}{}. This contradicts that  and  belong to different components  and  in . Therefore, \kTD{k}{} is -connected.


We present a constructive proof for the second part of theorem. Let , where  and . Place the points of  in the plane. Let . Place the points of  in . Let . Place the points of  in . Consider any pair  of points where  and . It is obvious that any path between  and  in \kTD{k}{} goes through the vertices in . Thus by removing the vertices in ,  and  become disconnected. Therefore, \kTD{k}{} of  is not -connected. 
\end{proof}
\subsection{Bottleneck Biconnected Spanning Graph}
As shown in Figure~\ref{TD}(a), \kTD{0}{} may not be biconnected. By Theorem~\ref{k-connectivity-thr}, \kTD{1}{} is biconnected. In this section we show that a bottleneck biconnected spanning graph of  is contained in \kTD{1}{}. 

\begin{theorem}
\label{biconnectivity-thr}
 For every point set , \kTD{1}{} contains a bottleneck biconnected spanning graph of .
\end{theorem}

\begin{proof}                                                                                       
Let  be the set of all biconnected spanning graphs with vertex set . We define a total order on the elements of  by their weight sequence. If two elements have the same weight sequence, we break the ties arbitrarily to get a total order.
Let  be a graph in  with minimal weight sequence. Clearly,  is a bottleneck biconnected spanning graph of . We will show that all edges of  are in \kTD{1}{}. By contradiction suppose that some edges in  do not belong to \kTD{1}{}, and let  be the longest one (by the area of the triangle ). If the graph  is biconnected, then by removing , we obtain a biconnected spanning graph  with ; contradicting the minimality of . Thus, there is a pair  of points such that any cycle between  and  in  goes through . Since ~\kTD{1}{},  contains at least two points of , say  and . Let  be the graph obtained from  by removing the edge  and adding the edges , , , . We show that in  there is a cycle between  and  which does not go through . Consider a cycle  in  between two points  and  (which goes through ). If none of  and  belong to , then  is a cycle in  between  and . If one of  or , say , belongs to , then  is a cycle in  between  and . If both  and  belong to , consider the partition of  into four parts: (a) edge , (b) path  between  and , (c) path  between  and , and (d) path  between  and . 
There are four cases:
\begin{enumerate}
 \item None of  and  are on . Then  is a cycle in  between  and .
 \item Both  and  are on . Then  is a cycle in  between  and .
 \item One of  and  is on  and the other one is on . Then  is a cycle in  between  and .
 \item One of  and  is on  and the other one is on . Then  is a cycle in  between  and .
\end{enumerate}

Thus, between any pair of points in  there exists a cycle, and hence  is biconnected. Since  and  are inside , by Observation~\ref{obs1}, . Therefore, ; contradicting the minimality of .   
\end{proof}


\section{Hamiltonicity}
\label{Hamiltonicity}

In this section we show that \kTD{7}{} contains a bottleneck Hamiltonian cycle. In addition, we will show that for some point sets, \kTD{5}{} does not contain any bottleneck Hamiltonian cycle.


\begin{theorem}
\label{hamiltonicity-thr}
 For every point set , \kTD{7}{} contains a bottleneck Hamiltonian cycle.
\end{theorem}

\begin{proof}
Let  be the set of all Hamiltonian cycles through the points of . Define a total
order on the elements of  by their weight sequence. If two elements have exactly the same weight sequence, break ties arbitrarily to get a total order. 
Let  be a cycle in  with minimal weight sequence. It is obvious that  is a bottleneck Hamiltonian cycle of . We will show that all the edges of  are in \kTD{7}{}. Consider any edge  in  and let  be the triangle corresponding to  (all index manipulations are modulo ).

{\em Claim 1}: None of the edges of  can be completely inside . Suppose there is an edge  inside . Let  be a cycle obtained from  by deleting  and , and adding  and . By Observation~\ref{obs1}, , and hence . This contradicts the minimality of .

Therefore, we may assume that no edge of  lies completely inside . Suppose there are  points of  inside . Let  represent these points indexed in the order we would encounter them on  starting from . Let  and  represent the vertices where  is the vertex preceding  on the cycle and  is the vertex succeeding  on the cycle.
Without loss of generality assume that , and  is anchored at , as shown in Figure~\ref{hamiltonicity-fig1}. 

{\em Claim 2}: For each , . Suppose there is a point  such that . Construct a new cycle  by removing the edges ,  and adding the edges  and . Since the two new edges have length strictly less than , ; which is a contradiction.

{\em Claim 3}: For each pair  and  of points in , . Suppose there is a pair  and  such that .  Construct a new cycle  from  by first deleting , ,  . This results in three paths. One of the paths must contain both  and either  or . W.l.o.g. suppose that  and  are on the same path. Add the edges , , . Since , ; which is a contradiction.

\begin{figure}[htb]
  \centering
  \includegraphics[width=.6\columnwidth]{fig/theta6.pdf}
 \caption{Illustration of Theorem~\ref{hamiltonicity-thr}.}
  \label{hamiltonicity-fig1}
\end{figure}

Now, we use Claim 2 and Claim 3 to show that the size of  (and consequently ) is at most seven, i.e., .
Consider the lines , , , and  as shown in Figure~\ref{hamiltonicity-fig1}. Let  and  be the rays starting at the corners of  opposite to  and parallel to  and  respectively, as shown in Figure~\ref{hamiltonicity-fig1}. These lines and rays, partition the plane into 12 regions. We will show that each of the regions , , , , , , and  contains at most one point of , and the other regions do not contain any point of . Consider the hexagon . By Claim 2 and Observation~\ref{obs2}, no point of  can be inside . Moreover, no point of  can be inside the cones , , and , because if , the (upward) triangle  contains . Then by Observation~\ref{obs2}, ; which contradicts Claim 2.

Now we show that each of the regions , ,  and  contains at most one point of . Consider the region ; by similar reasoning we can prove this claim for , , and . Using contradiction, let  and  be two points in , and w.l.o.g. assume that  is the farthest to . Then  can lie inside any of the cones , , and  (but not in ). If , then  is smaller than  which means that . If , then  contains , that is . If , then  contains , that is . All cases contradict Claim 3. 

Now consider the region  (or its symmetric region ) and by contradiction assume that it contains two points  and . Let  be the farthest from . It is obvious that the  contains , that is ; which contradicts Claim 3. 

Now consider the region . If both  and  belong to , then  is smaller that . If  and , then  contains , and hence . If both  and  belong to , let  be the farthest from . Clearly,  contains  and hence . All cases contradict Claim 3.

Therefore, any of the regions , , , , , , and  contains at most one point of . Thus, , and  contains at most 7 points of . Therefore,  is an edge of \kTD{7}{}.
\end{proof}

As a direct consequence of Theorem~\ref{hamiltonicity-thr} we have shown that:
\begin{corollary}
 \kTD{7}{} is Hamiltonian.
\end{corollary}

\begin{figure}[htb]
  \centering
  \includegraphics[width=.7\columnwidth]{fig/theta6-hamiltonicity2.pdf}
 \caption{ contains 7 points while the conditions in the proof of Theorem~\ref{hamiltonicity-thr} hold.}
  \label{hamiltonicity-fig2}
\end{figure}

An interesting question is to determine if \kTD{k}{} contains a bottleneck Hamiltonian cycle for .
Figure~\ref{hamiltonicity-fig2} shows a configuration where  contains 7 points while the conditions of Claim 1, Claim 2, and Claim 3 in the proof of Theorem~\ref{hamiltonicity-thr} hold. In Figure~\ref{hamiltonicity-fig2}, , , ,  for  and .  



\begin{figure}[htb]
  \centering
  \includegraphics[width=.8\columnwidth]{fig/theta6-hamiltonicity3.pdf}
 \caption{The points  are connected to their first and second closest point (the bold edges). The edge  should be in any bottleneck Hamiltonian cycle, while  contains 6 points.}
  \label{hamiltonicity-fig3}
\end{figure}

Figure~\ref{hamiltonicity-fig3} shows a configuration of  with 17 points such that \kTD{5}{} does not contain a bottleneck Hamiltonian cycle. In Figure~\ref{hamiltonicity-fig3},  and  contains 6 points . In addition , ,  for  and . Let . The dashed hexagons are centered at  and  and have diameter 1. The dotted hexagons are centered at vertices in  and have diameter . Each point in  is connected to its first and second closest points by edges of length  (the bold edges). Let  be the set of these edges. Let  be a cycle formed by , i.e., . It is obvious that  is a Hamiltonian cycle for  and . Thus, the bottleneck of any bottleneck Hamiltonian cycle for  is at most . We will show that any bottleneck Hamiltonian cycle for  contains the edge  which does not belong to \kTD{5}{}. By contradiction, let  be a bottleneck Hamiltonian cycle which does not contain . In ,  is connected to two vertices  and , where  and . Since the distance between  and any vertex in  is strictly bigger than  and ,  and . Thus  and  belong to . Let . Consider two cases:

\begin{itemize}
 \item  or . W.l.o.g. assume that  and . Since  is the first/second closest point of  and , in  one of  and  must be connected by an edge  to a point that is farther than its second closet point;  has length strictly greater than .
 \item  and . Thus, both  and  belong to . That is, in ,  should be connected to a point  where . If  then the edge  has length more than . If , w.l.o.g. assume ; by the same argument as in the previous case, one of  and  must be connected by an edge  to a point that is farther than its second closet point;  has length strictly greater than .
\end{itemize}

Since , both cases contradicts that . Therefore, every bottleneck Hamiltonian cycle contains edge . Since  is not an edge in \kTD{5}{}, a bottleneck Hamiltonian cycle of  is not contained in \kTD{5}{}.  

\section{Perfect Matching Admissibility}
\label{matching}
In this section we consider the matching problem in higher-order triangular-distance Delaunay graphs. In Subsection~\ref{bottleneck-matching-section} we show that \kTD{6}{} contains a bottleneck perfect matching. We also show that for some point sets , \kTD{5}{} does not contain any bottleneck perfect matching. In Subsection~\ref{matching2} we prove that every \kTD{2}{} has a perfect matching when  has an even number of points, and \kTD{1}{} contains a matching of size at least .

\subsection{Bottleneck Perfect Matching}
\label{bottleneck-matching-section}
\begin{theorem}
\label{matching-thr}
 For a set  of an even number of points, \kTD{6}{} contains a bottleneck perfect matching.
\end{theorem}

\begin{proof}
Let  be the set of all perfect matchings through the points of . Define a total order on the elements of  by their weight sequence. If two elements have exactly the same weight sequence, break ties arbitrarily to get a total order.
Let  be a perfect matching in  with minimal weight sequence. It is obvious that  is a bottleneck perfect matching for . We will show that all edges of  are in \kTD{6}{}. Consider any edge  in  and its corresponding triangle .


\begin{figure}[htb]
  \centering
  \includegraphics[width=.6\columnwidth]{fig/theta6-matching1.pdf}
 \caption{Proof of Theorem~\ref{matching-thr}.}
  \label{matching-fig1}
\end{figure}

{\em Claim 1}: None of the edges of  can be inside . Suppose there is an edge  inside . Let  be a perfect matching obtained from  by deleting , and adding . By Observation~\ref{obs1}, the two new edges are smaller than the old ones. Thus,  which contradicts the minimality of .

Therefore, we may assume that no edge of  lies completely inside . Suppose there are  points of  inside . Let  represent the points inside , and  represent the points where . W.l.o.g. assume that , and  is anchored at  as shown in Figure~\ref{matching-fig1}.  

{\em Claim 2}: For each , . By a similar argument as in the proof of Claim 2 in Theorem \ref{hamiltonicity-thr} we can either match  with  or  to obtain a smaller matching ; which is a contradiction.

{\em Claim 3}: For each pair  and  of points in , . The proof is similar to the proof of Claim 3 in Theorem \ref{hamiltonicity-thr}.

Consider Figure~\ref{matching-fig1} which partitions the plane into eleven regions. As a direct consequence of Claim 2, the hexagons  and  do not contain any point of . By a similar argument as in the proof of Theorem \ref{hamiltonicity-thr}, the regions , ,  do not contain any point of . In addition, the region  does not contain any point  of , because otherwise  contains , that is  which contradicts Claim 2. As shown in the proof of Theorem \ref{hamiltonicity-thr} each of the regions , , , , , and  contains at most one point of  (note that  and ). Thus, , and  contains at most 6 points of . Therefore,  is an edge of \kTD{6}{}.
\end{proof}

As a direct consequence of Theorem~\ref{matching-thr} we have shown that:
\begin{corollary}
 For a set  of even number of points, \kTD{6}{} has a perfect matching.
\end{corollary}

\begin{figure}[htb]
  \centering
  \includegraphics[width=.6\columnwidth]{fig/theta6-matching3.pdf}
 \caption{The points  are matched to their closest point. The edge  should be an edge in any bottleneck perfect matching, while  contains 6 points.}
  \label{matching-fig3}
\end{figure}

We show that the bound  proved in Theorem~\ref{matching-thr} is tight. We will show that there are point sets  such that \kTD{5}{} does not contain any bottleneck perfect matching.
Figure~\ref{matching-fig3} shows a configuration of  with 14 points such that  and  contains six points . In addition ,  where , for . Let . In Figure~\ref{matching-fig3}, the dashed hexagons are centered at  and , each of diameter 1, and the dotted hexagons centered at vertices in , each of diameter . Consider a perfect matching  where each point  is matched to its closest point . It is obvious that , and hence the bottleneck of any bottleneck perfect matching is at most . We will show that any bottleneck perfect matching for  contains the edge  which does not belong to \kTD{5}{}. By contradiction, let  be a bottleneck perfect matching which does not contain . In ,  is matched to a point . If , then . If , w.l.o.g. assume . Thus, in  the point  is matched to a point  where . Since  is the closest point to  and , . Both cases contradicts that . Therefore, every bottleneck perfect matching contains . Since  is not an edge in \kTD{5}{}, a bottleneck perfect matching of  is not contained in \kTD{5}{}.  

\subsection{Perfect Matching}
\label{matching2}
In \cite{Babu2013} the authors proved a tight lower bound of  on the size of a maximum matching in \kTD{0}{}. In this section we prove that \kTD{1}{} has a matching of size  and \kTD{2}{} has a perfect matching when  has an even number of points.

\begin{figure}[htb]
  \centering
\setlength{\tabcolsep}{0in}
  
  \caption{(a) Illustration of Lemma~\ref{triangle3}, and (b) proof of Lemma~\ref{triangle3}.}
\label{intersection-fig}
\end{figure}

For a triangle  through the points  and , let , , and  respectively denote the top, left, and right sides of . Refer to Figure~\ref{intersection-fig}(a) for the following lemma.
\begin{lemma}
\label{triangle3}
 Let  and  intersect a horizontal line , and  intersects  in such a way that  contains the lowest corner of . If  and  lie above , and  and  lie above , then, .
\end{lemma}
\begin{proof}
Recall that  is the smallest downward triangle through  and . By Observation~\ref{side-point-obs} each side of  contains either  or . 
In Figure~\ref{intersection-fig}(a) the set of potential positions for point  on the boundary of  is shown by the line segment ; and similarly by , ,  for , , , respectively. We will show that . By similar reasoning we can show that . Let  denote the intersection of  and . Consider a ray  initiated at  and parallel to  which divides  into (at most) two parts  and  as shown in Figure~\ref{intersection-fig}(b). Two cases may appear:

\begin{itemize}
 \item . Let  be a downward triangle anchored at  which has its  side on the line through  (the dashed triangle in Figure~\ref{intersection-fig}(b)). The top side of  and  lie on the same horizontal line. The bottommost corner of  is on  while the bottommost corner of  is below . Thus, . In addition,  contains  and , thus, for any two points  and , . Therefore, .
 \item . Let  be a downward triangle anchored at the intersection of  and  which has one side on the line through  (the dotted triangle in Figure~\ref{intersection-fig}(b)). This triangle is contained in , and has  on its right side. If we slide  upward while its top-left corner remains on , the segment  remains on the right side of . Thus, any triangle connecting a point  to a point  has the same size as . That is, . 
\end{itemize}

Therefore, we have . By similar argument we conclude that .  
\end{proof}
Let  be a partition of the points in .
Let  be a complete graph with vertex set . For each edge  in , let  be equal to the area of the smallest triangle between a point in  and a point in , i.e. . That is, the weight of an edge  corresponds to the size of the smallest triangle  defined by the endpoints of . Let  be a minimum spanning tree of . Let  be the set of triangles corresponding to the edges of , i.e. . 

\begin{lemma}
 \label{empty-triangle-lemma}
The interior of any triangle in  does not contain any point of .
\end{lemma}
\begin{proof}
  By contradiction, suppose there is a triangle  which contains a point . Let  be the edge in  which corresponds to . Let  and  respectively be the points in  and  which define , i.e.  and . Three cases arise: (i) , (ii) , (iii)  where  and . In case (i) the triangle  between  and  is smaller than ; contradicts that  in .  In case (ii) the triangle  between  and  is smaller than ; contradicts that  in . In case (iii) the triangle  (resp. ) between  and  (resp.  and ) is smaller than ; contradicts that  is an edge in . 
\end{proof}

\begin{lemma}
 \label{intersection-lemma}
Each point in the plane can be in the interior of at most three triangles in . 
\end{lemma}

\begin{proof}
For each , the sides , , and  contains at least one of  and . In addition, by Lemma~\ref{empty-triangle-lemma},  does not contain any point of  in its interior. Thus, none of , , and  is completely inside the other triangles. Therefore, the only possible way that two triangles  and  can share a point is that one triangle, say , contains a corner of  in such a way that  and  are outside . In other words  intersects  through one of the sides , , or . If  intersects  through a direction  we say that . 

By contradiction, suppose there is a point  in the plane which is inside four triangles . Out of these four, either (i) three of them are like  or (ii) there is a triangle  such that , where  and . Figure~\ref{configuration-fig} shows the two possible configurations (note that all other configurations obtained by changing the indices of triangles and/or the direction are symmetric to Figure~\ref{configuration-fig}(a) or Figure~\ref{configuration-fig}(b)).
\begin{figure}[htb]
  \centering
\setlength{\tabcolsep}{0in}
  
  \caption{Two possible configurations: (a) , (b) .}
\label{configuration-fig}
\end{figure}

 Recall that each of  corresponds to an edge in . In the configuration of Figure~\ref{configuration-fig}(a) consider , , and  which is shown in more detail in Figure~\ref{matching3-fig}(a). Suppose  (resp. ) is defined by points  and  (resp.  and ). By Lemma~\ref{empty-triangle-lemma},  and  are above ,  and  are above . By Lemma~\ref{triangle3}, . This contradicts the fact that both of the edges representing  and  are in , because by replacing  with  or , we obtain a tree  which is smaller than . In the configuration of Figure~\ref{configuration-fig}(b), consider all pairs of potential positions for two points defining  which is shown in more detail in Figure~\ref{matching3-fig}(b). The pairs of potential positions on the boundary of  are shown in red, green, and orange. Consider the red pair, and look at , , and . By Lemma~\ref{triangle3} and the same reasoning as for the previous configuration, we obtain a smaller tree ;  which contradicts the minimality of . By symmetry, the green and orange pairs lead to a contradiction.
Therefore, all configurations are invalid; which proves the lemma.



\begin{figure}[htb]
  \centering
\setlength{\tabcolsep}{0in}
  
  \caption{Illustration of Lemma~\ref{intersection-lemma}.}
\label{matching3-fig}
\end{figure}
\end{proof}

Our results in this section are based on Lemma~\ref{empty-triangle-lemma}, Lemma~\ref{intersection-lemma} and the two theorems by Tutte~\cite{Tutte1947} and Berge~\cite{Berge1958}. 

Now we prove that \kTD{2}{} has a perfect matching.

\begin{theorem}
 \label{mt-thr}
For a set  of an even number of points, \kTD{2}{} has a perfect matching.
\end{theorem}
\begin{proof}
First we show that by removing a set  of  points from \kTD{2}{}, at most  components are generated. Then we show that at least one of these components must be even. Finally by Theorem~\ref{Tutte} we conclude that \kTD{2}{} has a perfect matching.

Let  be a set of  vertices removed from \kTD{2}{}, and let  be the resulting  components, where  is a function depending on . Actually  and  is a partition of the vertices in . 

{\bf\em  Claim 1.} . Let  be a complete graph with vertex set  which is constructed as described above. Let  be a minimum spanning tree of  and let  be the set of triangles corresponding to the edges of . It is obvious that  contains  edges and hence . Let  be the set of all (point, triangle) pairs where , , and  is inside . By Lemma~\ref{intersection-lemma} each point in  can be inside at most three triangles in . Thus, .
Now we show that each triangle in  contains at least three points of .  
Consider any triangle . Let  be the edge of  which is corresponding to , and let  and  be the points defining . By Lemma~\ref{empty-triangle-lemma},  does not contain any point of  in its interior. Therefore,  contains at least three points of , because otherwise  is an edge in \kTD{2}{} which contradicts the fact that  and  belong to different components in . Thus, each triangle in  contains at least three points of  in its interior. That is, . Therefore, , and hence .

{\bf \em Claim 2}: . By Claim 1, . If , then . Assume that . Since , the total number of vertices of  can be defined as . Consider two cases where (i)  is odd, (ii)  is even. In both cases if all the components in  are odd, then  is odd; contradicts our assumption that  has an even number of vertices. Thus,  contains at least one even component, which implies that .

Finally, by Claim 2 and Theorem~\ref{Tutte}, we conclude that \kTD{2}{} has a perfect matching.
\end{proof}

\begin{theorem}
\label{matching-1TD}
 For every set  of points, \kTD{1}{} has a matching of size .
\end{theorem}

\begin{proof}
Let  be a set of  vertices removed from \kTD{1}{}, and let  be the resulting  components. Actually  and  is a partition of the vertices in . Note that .
Let  be a maximum matching in \kTD{1}{}. By Theorem~\ref{Berge}, 



where


Define , , , and  as in the proof of Theorem~\ref{mt-thr}. By Lemma~\ref{intersection-lemma}, .
By the same reasoning as in the proof of Theorem~\ref{mt-thr}, each triangle in  has at least two points of  in its interior. Thus, . Therefore, , and hence

 

In addition, , and hence



By Inequalities~(\ref{ineq1}) and ~(\ref{ineq2}), 



Thus, by (\ref{align1}) and (\ref{ineq3})



where the last equation is achieved by setting  equal to , which implies . Finally by substituting (\ref{align2}) in Equation (\ref{align0}) we have

\end{proof}
\section{Blocking TD-Delaunay graphs}
\label{blocking-section}
In this section we consider the problem of blocking TD-Delaunay graphs. Let  be a set of  points in the plane such that no pair of points of  is collinear in the , , and  directions. Recall that a point set  blocks \kTD{k}{} if in \kTD{k}{} there is no edge connecting two points in . That is,  is an independent set in \kTD{k}{}.

\begin{theorem}
\label{blocking-thr1}
  At least  points are necessary to block \kTD{k}{}.
\end{theorem}
\begin{proof}
Let  be a set of  points which blocks \kTD{k}{}. Let  be a complete graph with vertex set . Let  be a minimum spanning tree of  and let  be the set of triangles corresponding to the edges of . It is obvious that . By Lemma~\ref{empty-triangle-lemma} the triangles in  are empty, thus, the edges of  belong to any \kTD{k}{} where . To block each edge, corresponding to a triangle in , at least  points are necessary. By Lemma~\ref{intersection-lemma} each point in  can lie in at most three triangles of . Therefore, , which implies that at least  points are necessary to block all the edges of  and hence \kTD{k}{}.
\end{proof}
\begin{figure}[htb]
  \centering
\setlength{\tabcolsep}{0in}
  
  \caption{(a) a \kTD{0}{} graph which is shown in bold edges is blocked by  white points, (b)  blocks all the edges connecting  to the vertices above .}
\label{blocking-fig}
\end{figure}

Theorem~\ref{blocking-thr1} gives a lower bound on the number of points that are necessary to block a TD-Delaunay graph. By this theorem, at least , ,  points are necessary to block , , \kTD{2}{} respectively. Now we introduce another formula which gives a better lower bound for \kTD{0}{}. For a point set , let  and  respectively denote the size of a maximum matching and a maximum independent set in \kTD{k}{}. For every edge in the maximum matching, at most one of its endpoints can be in the maximum independent set. Thus,

Let  be a set of  points which blocks \kTD{k}{}. By definition there is no edge between points of  in \kTD{k}{}. That is,  is an independent set in \kTD{k}{}. Thus, 

By (\ref{matching-independent}) and (\ref{ineq4}) we have

\begin{theorem}
\label{blocking-0TD-thr}
  At least  points are necessary to block \kTD{0}{}.
\end{theorem}
\begin{proof}
Let  be a set of  points which blocks \kTD{k}{}. Consider \kTD{0}{}. It is known that the ; see~\cite{Babu2013}. By Inequality~(\ref{ineq5}),  and consequently  (note that  is an integer number).
\end{proof}

Figure~\ref{blocking-fig}(a) shows a \kTD{0}{} graph on a set of 12 points which is blocked by 6 points. By removing the topmost point we obtain a set with odd number of points which can be blocked by 5 points. Thus, the lower bound provided by Theorem~\ref{blocking-0TD-thr} is tight. 

Now let . By Theorem~\ref{matching-1TD} we have , and by Inequality~(\ref{ineq5})  and consequently ; the same lower bound as in Theorem~\ref{blocking-thr1}. 

Now let . By Theorem~\ref{mt-thr} we have  (note that  may be odd). By Inequality~(\ref{ineq5})  and consequently , where  is even, and , where  is odd.  

\begin{theorem}
\label{blocking-thr2}
 There exists a set  of  points that blocks \kTD{0}{}.
\end{theorem}
\begin{proof}
Let  be the Euclidean distance between  and . Let .
 For each point  let  and  respectively denote the  and  coordinates of  in the plane. Let  be the points of  in the increasing order of their -coordinate. Let . See Figure~\ref{blocking-fig}(b). For each point , let  (resp. ) denote the edges of \kTD{0}{} between  and the points above  (resp. below ). It is easy to see that the downward triangle between  and any point  above  (i.e. any point ) contains . Thus,  blocks all the edges in . In addition, the edges in  are blocked by . Therefore, all the edges of \kTD{0}{} are blocked by the  points in .
\end{proof}
Note that the bound of Theorem~\ref{blocking-thr2} is tight, because \kTD{0}{} can be a path representing  disjoint triangles and for each triangle we need at least one point to block its corresponding edge. 
We can extend the result of Theorem~\ref{blocking-thr2} to \kTD{k}{} where . For each point  we put  copies of  very close to . Thus, 

\begin{corollary}
 There exists a set  of  points that blocks \kTD{k}{}.
\end{corollary}
\section{Conclusion}
\label{conclusion}
In this paper, we considered some combinatorial properties of higher-order triangular-distance Delaunay graphs of a point set . We proved that
\begin{itemize}

  \item \kTD{k}{} is  connected.
  \item \kTD{2}{} contains a bottleneck biconnected spanning graph of .
  \item \kTD{7}{} contains a bottleneck Hamiltonian cycle and \kTD{5}{} may not have any.
  \item \kTD{6}{} contains a bottleneck perfect matching and \kTD{5}{} may not have any.
  \item \kTD{1}{} has a matching of size at least .
  \item \kTD{2}{} has a perfect matching when  has an even number of points.
  \item  points are necessary to block \kTD{0}{}.
  \item  points are necessary and  points are sufficient to block \kTD{k}{}.
\end{itemize}

We leave a number of open problems:
\begin{itemize}

  \item What is a tight lower bound for the size of maximum matching in \kTD{1}{}?
  \item Does \kTD{6}{} contain a bottleneck Hamiltonian cycle?
 \item As shown in Figure~\ref{TD}(a) \kTD{0}{} may not have a Hamiltonian cycle. For which values of , is the graph \kTD{k}{} Hamiltonian?
\end{itemize}

\bibliographystyle{abbrv}
\bibliography{Higher-Order-TDDEL.bib}
\end{document}
