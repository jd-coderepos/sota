\documentclass{article} 

\usepackage{amsmath}
\usepackage{amssymb}
\usepackage{amstext}
\usepackage{xspace}
\usepackage{url}
\usepackage{xy}
\xyoption{all}
\usepackage{proof}

\message{<Paul Taylor's Proof Trees, 2 August 1996>}


\def\introrule{{\cal I}}\def\elimrule{{\cal E}}\def\andintro{\using{\land}\introrule\justifies}\def\impelim{\using{\Rightarrow}\elimrule\justifies}\def\allintro{\using{\forall}\introrule\justifies}\def\allelim{\using{\forall}\elimrule\justifies}\def\falseelim{\using{\bot}\elimrule\justifies}\def\existsintro{\using{\exists}\introrule\justifies}

\def\andelim#1{\using{\land}#1\elimrule\justifies}\def\orintro#1{\using{\lor}#1\introrule\justifies}

\def\impintro#1{\using{\Rightarrow}\introrule_{#1}\justifies}\def\orelim#1{\using{\lor}\elimrule_{#1}\justifies}\def\existselim#1{\using{\exists}\elimrule_{#1}\justifies}



\newdimen\proofrulebreadth \proofrulebreadth=.05em
\newdimen\proofdotseparation \proofdotseparation=1.25ex
\newdimen\proofrulebaseline \proofrulebaseline=2ex
\newcount\proofdotnumber \proofdotnumber=3
\let\then\relax
\def\hfi{\hskip0pt plus.0001fil}
\mathchardef\squigto="3A3B
\newif\ifinsideprooftree\insideprooftreefalse
\newif\ifonleftofproofrule\onleftofproofrulefalse
\newif\ifproofdots\proofdotsfalse
\newif\ifdoubleproof\doubleprooffalse
\let\wereinproofbit\relax
\newdimen\shortenproofleft
\newdimen\shortenproofright
\newdimen\proofbelowshift
\newbox\proofabove
\newbox\proofbelow
\newbox\proofrulename
\def\shiftproofbelow{\let\next\relax\afterassignment\setshiftproofbelow\dimen0 }
\def\shiftproofbelowneg{\def\next{\multiply\dimen0 by-1 }\afterassignment\setshiftproofbelow\dimen0 }
\def\setshiftproofbelow{\next\proofbelowshift=\dimen0 }
\def\setproofrulebreadth{\proofrulebreadth}

\def\prooftree{\ifnum  \lastpenalty=1
\then   \unpenalty
\else   \onleftofproofrulefalse
\fi
\ifonleftofproofrule
\else   \ifinsideprooftree
        \then   \hskip.5em plus1fil
        \fi
\fi
\bgroup \setbox\proofbelow=\hbox{}\setbox\proofrulename=\hbox{}\let\justifies\proofover\let\leadsto\proofoverdots\let\Justifies\proofoverdbl
\let\using\proofusing\let\endprooftree\fi
\proofdotsfalse\doubleprooffalse
\let\thickness\setproofrulebreadth
\let\shiftright\shiftproofbelow \let\shift\shiftproofbelow
\let\shiftleft\shiftproofbelowneg
\let\ifwasinsideprooftree\ifinsideprooftree
\insideprooftreetrue
\setbox\proofabove=\hbox\bgroup\egroup  \shortenproofleft=\dimen0
\shortenproofright=\dimen1
\proofrulebreadth=\dimen2
\proofbelowshift=\dimen3
\proofdotseparation=\dimen4
\proofdotnumber=\count255
}

\def\proofover{\eproofbit \setbox\proofbelow=\hbox\bgroup \let\wereinproofbit\proofover
\displaystyle
}\def\proofoverdots{\eproofbit \proofdotstrue
\setbox\proofbelow=\hbox\bgroup \let\wereinproofbit\proofoverdots

}

\def\endprooftree{\eproofbit \dimen5 =0pt\dimen0=\wd\proofabove \advance\dimen0-\shortenproofleft
\advance\dimen0-\shortenproofright
\dimen1=.5\dimen0 \advance\dimen1-.5\wd\proofbelow
\dimen4=\dimen1
\advance\dimen1\proofbelowshift \advance\dimen4-\proofbelowshift
\ifdim  \dimen1<0pt
\then   \advance\shortenproofleft\dimen1
        \advance\dimen0-\dimen1
        \dimen1=0pt
\ifdim  \shortenproofleft<0pt
        \then   \setbox\proofabove=\hbox{\kern-\shortenproofleft\unhbox\proofabove}\shortenproofleft=0pt
        \fi
\fi
\ifdim  \dimen4<0pt
\then   \advance\shortenproofright\dimen4
        \advance\dimen0-\dimen4
        \dimen4=0pt
\fi
\ifdim  \shortenproofright<\wd\proofrulename
\then   \shortenproofright=\wd\proofrulename
\fi
\dimen2=\shortenproofleft \advance\dimen2 by\dimen1
\dimen3=\shortenproofright\advance\dimen3 by\dimen4
\ifproofdots
\then
        \dimen6=\shortenproofleft \advance\dimen6 .5\dimen0
        \setbox1=\vbox to\proofdotseparation{\vss\hbox{}\vss}\setbox0=\hbox{\advance\dimen6-.5\wd1
                \kern\dimen6
                \unhbox\proofrulename}\else   \dimen6=\fontdimen22\the\textfont2 \dimen7=\dimen6
        \advance\dimen6by.5\proofrulebreadth
        \advance\dimen7by-.5\proofrulebreadth
        \setbox0=\hbox{\kern\shortenproofleft
                \ifdoubleproof
                \then   \hbox to\dimen0{\mkern-2mu=\mkern-2mu}\else   \vrule height\dimen6 depth-\dimen7 width\dimen0
                \fi
                \unhbox\proofrulename}\ht0=\dimen6 \dp0=-\dimen7
\fi
\let\doll\relax
\ifwasinsideprooftree
\then   \let\VBOX\vbox
\else   \ifmmode\else\fi
        \let\VBOX\vcenter
\fi
\VBOX   {\baselineskip\proofrulebaseline \lineskip.2ex
        \expandafter\lineskiplimit\ifproofdots0ex\else-0.6ex\fi
        \hbox   spread\dimen5   {\hfi\unhbox\proofabove\hfi}\hbox{\box0}\hbox   {\kern\dimen2 \box\proofbelow}}\doll \global\dimen2=\dimen2
\global\dimen3=\dimen3
\egroup \ifonleftofproofrule
\then   \shortenproofleft=\dimen2
\fi
\shortenproofright=\dimen3
\onleftofproofrulefalse
\ifinsideprooftree
\then   \hskip.5em plus 1fil \penalty2
\fi
}

 
\newcommand{\pullback}[1][dr]{\save*!/#1-1.2pc/#1:(-1,1)@^{|-}\restore}
\makeatother
 \newdir{ >}{{}*!/-7.5pt/@{>}}
 \newdir{|>}{!/4.5pt/@{|}*:(1,-.2)@^{>}*:(1,+.2)@_{>}}
 \newdir{ |>}{{}*!/-3pt/@{|}*!/-7.5pt/:(1,-.2)@^{>}*!/-7.5pt/:(1,+.2)@_{>}}
\newcommand{\xyline}[2][]{\ensuremath{\smash{\xymatrix@1#1{#2}}}}
\newcommand{\xyinline}[2][]{\ensuremath{\smash{\xymatrix@1#1{#2}}}^{\rule[8.5pt]{0pt}{0pt}}}
\makeatletter

\newif\ifignore \ignorefalse
\newcommand{\auxproof}[1]{
\ifignore\mbox{}\newline
\textbf{PROOF:} \dotfill\newline
{\it #1}\mbox{}\newline
\textbf{ENDPROOF}\dotfill
\fi}

\newtheorem{theorem}{Theorem}
\newtheorem{lemma}[theorem]{Lemma}
\newtheorem{proposition}[theorem]{Proposition}
\newtheorem{corollary}[theorem]{Corollary}
\newtheorem{definition}[theorem]{Definition}
\newtheorem{example}[theorem]{Example}
\newtheorem{remark}[theorem]{Remark}
\newtheorem{question}[theorem]{Question}
\newtheorem{conjecture}[theorem]{Conjecture}
\newtheorem{assumption}[theorem]{Assumption}
\newenvironment{proof}[1][Proof]{ \begin{trivlist}\item[\hskip \labelsep {\bfseries #1}]}{ \end{trivlist}}
\newcommand{\QEDbox}{\square}
\newcommand{\QED}{\hspace*{\fill}}
\newcommand{\QEDhere}{\tag*{}}

\newcommand{\after}{\mathrel{\circ}}
\newcommand{\cat}[1]{\ensuremath{\mathbf{#1}}}
\newcommand{\Cat}[1]{\ensuremath{\mathbf{#1}}}
\newcommand{\field}[1]{\ensuremath{\mathbb{#1}}}
\newcommand{\op}{\ensuremath{^{\mathrm{op}}}}
\newcommand{\idmap}[1][]{\ensuremath{\mathrm{id}_{#1}}}
\newcommand{\id}[1][]{\idmap[#1]}
\renewcommand{\Im}{\ensuremath{\mathrm{Im}}}
\newcommand{\coIm}{\ensuremath{\mathrm{coIm}}}
\newcommand{\coker}{\ensuremath{\mathrm{coker}}}
\newcommand{\charac}{\ensuremath{\mathrm{char}}}
\newcommand{\PMod}[1][]{\ensuremath{\Cat{PMod}_{#1}}}
\newcommand{\Endo}{\ensuremath{\Cat{Endo}}}
\newcommand{\SA}{\ensuremath{\Cat{SA}}}
\newcommand{\Proj}{\ensuremath{\mathrm{Proj}}}
\newcommand{\BG}{\ensuremath{\Cat{BG}}}
\newcommand{\cQ}{\ensuremath{\mathcal{Q}}}
\newcommand{\kleisli}[1]{\ensuremath{\mathrm{Kl}(#1)}}
\newcommand{\Atom}{\ensuremath{\mathrm{At}}}
\newcommand{\Dom}{\ensuremath{\mathrm{Dom}}}
\newcommand{\codom}{\ensuremath{\mathrm{codom}}}
\newcommand{\inprod}[2]{\ensuremath{\langle #1\,|\,#2 \rangle}}
\newcommand{\pisom}{\ensuremath{_{\mathrm{pisom}}}}
\newcommand{\PIsom}{\ensuremath{\mathbf{PIsom}}}
\newcommand{\Sub}{\ensuremath{\mathrm{Sub}}}
\newcommand{\KSub}{\ensuremath{\mathrm{KSub}}}
\newcommand{\orthogonal}{\mathrel{\bot}}
\newcommand{\biorthogonal}{\mathop{\rlap{}\raisebox{.2ex}{}}}
\newcommand{\sasaki}{\mathrel{\supset}}
\newcommand{\andthen}{\mathrel{\&}}
\newcommand{\powerset}{\mathcal{P}}
\newcommand{\nul}{\ensuremath{\underline{0}}}

\newcommand{\Rel}{\Cat{Rel}\xspace}
\newcommand{\PInj}{\Cat{PInj}\xspace}
\newcommand{\Hilb}{\Cat{Hilb}\xspace}
\newcommand{\PHilb}{\Cat{PHilb}\xspace}
\newcommand{\Sets}{\Cat{Sets}\xspace}
\newcommand{\DCK}{\Cat{DCK}\xspace}
\newcommand{\cotuple}[2]{\ensuremath{[ #1,\,#2 ]}}
\newcommand{\Karoubi}[1]{{\cal K}(#1)}
\newcommand{\dagKaroubi}[1]{{\cal K}^{\dag}(#1)}

\newcommand{\set}[2]{\{#1\;|\;#2\}}
\newcommand{\setin}[3]{\{#1\in#2\;|\;#3\}}
\newcommand{\conjun}{\mathrel{\wedge}}
\newcommand{\disjun}{\mathrel{\vee}}
\newcommand{\all}[2]{\forall_{#1}.\,#2}
\newcommand{\allin}[3]{\forall_{#1\in#2}.\,#3}
\newcommand{\ex}[2]{\exists_{#1}.\,#2}
\newcommand{\exin}[3]{\exists_{#1\in#2}.\,#3}
\newcommand{\lamin}[3]{\lambda{#1\in#2}.\,#3}
\newcommand{\fibno}[2]{\raisebox{.00in}
           {\mbox{ \scriptstyle #1}}}
\newcommand{\downset}{\mathop{\downarrow}\!}
\newcommand{\Perp}{\mathop{\perp}}
\newcommand{\effect}[1]{\mathfrak{E}(#1)}
\newcommand{\sai}[1]{[\,#1\,]}
\newcommand{\EndoHom}[1]{{\cal E}{\kern-.5ex}\textit{n}{\kern-.2ex}\textit{d}{\kern-.2ex}\textit{o}(#1)}


\renewcommand{\arraystretch}{1.3}

\title{Orthomodular lattices, Foulis Semigroups \\
       and Dagger Kernel Categories}
\author{Bart Jacobs, \\
{\small Institute for Computing and Information Sciences (iCIS),} \-.5em]
{\small Webaddress: \url{www.cs.ru.nl/B.Jacobs}}}

\date{\small \today}

\newenvironment{bijectivecorrespondence}
  {\newif\ifbijnotfirst
   \global\bijnotfirstfalse
   \global\def\bijprev{}
   \renewcommand{\baselinestretch}{0}\normalsize
   \begin{tabular}{cl}}
  {\end{tabular}
   \renewcommand{\baselinestretch}{1}}
\newcommand{\correspondence}[2][]{\ifbijnotfirst \rule{0pt}{5.8pt}\smash{\ensuremath{\infer={\hphantom{#2}}{\hphantom{\bijprev}}}} \\\fi \global\bijnotfirsttrue \global\def\bijprev{#2}\ensuremath{#2} & #1 \\}

\begin{document}
\maketitle

\begin{abstract}
This paper is a sequel to~\cite{HeunenJ09a} and continues the study of
quantum logic via dagger kernel categories. It develops the relation
between these categories and both orthomodular lattices and Foulis
semigroups. The relation between the latter two notions has been
uncovered in the 1960s. The current categorical perspective gives a
broader context and reconstructs this relationship between
orthomodular lattices and Foulis semigroups as special instance.
\end{abstract}


\section{Introduction}\label{IntroSec}

Dagger kernel categories have been introduced in~\cite{HeunenJ09a} as
a relatively simple setting in which to study categorical quantum
logic. These categories turn out to have orthomodular logic built in,
via their posets  of kernel subobjects that can be used to
interprete predicates on .  The present paper continues the study
of dagger kernel categories, especially in relation to orthomodular
lattices and Foulis semigroups. The latter two notions have been
studied extensively in the context of quantum logic. The main results
of this paper are as follows.
\begin{enumerate}
\item[(1)] A special category \Cat{OMLatGal} is defined with
  orthomodular lattices as objects and Galois connections between them
  as morphisms; it is shown that:
\begin{itemize}
\item \Cat{OMLatGal} is itself a dagger kernel category---with some
additional structure such a dagger biproducts, and an opclassifier;

\item for each dagger kernel category \Cat{D} there is a functor
   preserving the dagger kernel
  structure; hence \Cat{OMLatGal} contains in a sense all dagger
  kernel categories.
\end{itemize}

\item[(2)] For each object  in a dagger kernel category, the homset
   of endo-maps is a Foulis semigroup.

\item[(3)] Every Foulis semigroup  yields a dagger kernel category
   via the ``dagger Karoubi'' construction
  .
\end{enumerate}

Translations between orthomodular lattices and Foulis semigroups have
been described in the 1960s, see
\textit{e.g.}~\cite{Foulis60,Foulis62,Foulis63,BlythJ72,Kalmbach83}. These
translations appear as special instances of the above results:
\begin{itemize}
\item given a Foulis semigroup , all the kernel posets 
  are orthomodular lattices, for each object  of
  the associated dagger kernel category (using point~(3) mentioned
  above). For the unit element  this yields the ``old''
  translation from Foulis semigroups to orthomodular lattices;

\item given an orthomodular lattice , the set of (Galois) endomaps
   on  in the dagger kernel
  category \Cat{OMLatGal} forms a Foulis semigroup---using points~(1)
  and~(2). Again this is the ``old'' translation, from orthomodular
  lattices to Foulis semigroups.
\end{itemize}

\noindent Since dagger kernel categories are essential in these
constructions we see (further) evidence of the relevance of categories
in general, and of dagger kernel categories in particular, in the
setting of quantum (logical) structures.

The paper is organised as follows. Section~\ref{DagKerSec} first
recalls the essentials about dagger kernel categories
from~\cite{HeunenJ09a} and also about the (dagger) Karoubi envelope.
It shows that dagger kernel categories are closed under this
construction. Section~\ref{OMLatDagKerSec} introduces the fundamental
category \Cat{OMLatGal} of orthomodular lattices with Galois
connections between them, investigates some of its properties, and
introduces the functor 
for any dagger kernel category \Cat{D}. Subsequently,
Section~\ref{FoulisDagKerSec} recalls the definition of Foulis semigroups,
shows how they arise as endo-homsets in dagger kernel categories,
and proves that their dagger Karoubi envelope yields a dagger
kernel category. The paper ends with some final remarks and further
questions in Section~\ref{ConclusionSec}.


\section{Dagger kernel categories}\label{DagKerSec}

Since the notion of dagger kernel category is fundamental in this
paper we recall the essentials from~\cite{HeunenJ09a}, where this type
of category is introduced. Further details can be found there.

A dagger kernel category consists of a category \Cat{D} with a dagger
functor , a zero object
, and dagger kernels. The functor  is the identity
on objects  and satisfies  on morphisms
. The zero object  yields a zero map, also written as ,
namely  between any two objects
. A dagger kernel of a map  is
a kernel map, written as , which is---or can be chosen as---a dagger mono,
meaning that . Often we write , and  for the cokernel of
. The definition  for a kernel 
yields an orthocomplement.

We write \Cat{DKC} for the category with dagger kernel categories
as objects and functors preserving .

The main examples of dagger kernel categories are: \Rel, the category
of sets and relations, its subcategory \Cat{pInj} of sets and partial
injections, \Hilb, the category of Hilbert spaces and
bounded/continuous linear maps between them, and \PHilb, the category
of projective Hilbert spaces. This paper adds another example, namely
\Cat{OMLatGal}.

The main results from~\cite{HeunenJ09a} about dagger kernel categories
are as follows.
\begin{enumerate}
\item Each poset  of kernel subobjects of an object  is
  an orthomodular lattice; this is the basis of the relevance of
  dagger kernel categories to quantum logic.

\item Pullbacks of kernels exist along arbitrary maps , yielding a pullback (or substitution) functor
  . Explicitly, like
  in~\cite{Freyd64}, .

\item This pullback functor  has a left adjoint , corresponding to image
  factorisation. These  and  only preserve part of
  the logical structure---meets are preserved by  and joins by
  , via the adjointness---but for instance negations and
  joins are not preserved by substitution , unlike what is
  standard in categorical logic, see \textit{e.g.}~\cite{Jacobs99a}.

  Substitution  and existential quantification 
  are inter-expressible, via .

\item The logical ``Sasaki'' hook  and ``and-then''
   connectives---together with the standard adjunction
  between them~\cite{Finch70,CoeckeS04}---arise via this adjunction
  , namely for  as:


\noindent where 
is the effect (see~\cite{DvurecenskijP00}) associated with the kernel
.
\end{enumerate}




\subsection{Karoubi envelope}\label{KaroubiSubsec}

Next we recall the essentials of the so-called Karoubi envelope
(see~\cite{Karoubi78} or~\cite[Chapter~2, Exercise~B]{Freyd64})
construction---and its ``dagger'' version---involving the free
addition of splittings of idempotents to a category. The construction
will be used in Section~\ref{FoulisDagKerSec} to construct a dagger
kernel category out of a Foulis semigroup. It is thus instrumental,
and not studied in its own right.

An idempotent in a category is an endomap 
satisfying . A splitting of such an idempotent  is a
pair of maps  and 
with  and . Clearly,  is then
a mono and  is an epi. Such a splitting, if it exists, is unique
up-to-isomorphism.

\auxproof{
Assume also 
with . Then ,
so that:


\noindent Hence the maps 
and  are each other's
inverses:

}

For an arbitrary category \Cat{C} the so-called Karoubi envelope
 has idempotents  in
\Cat{C} as objects. A morphism  in
 consists of a map  in
\Cat{C} with . The identity on an object
 is the map  itself. Composition in
 is as in \Cat{C}. The mapping  thus yields a full and faithful functor .

The Karoubi envelope  can be understood as the free
completion of \Cat{C} with splittings of idempotents. Indeed, an
idempotent  in  can
be split as . If  is a functor to a category \Cat{D} in
which endomorphisms split, then there is an up-to-isomorphism unique
functor  with
.

\auxproof{
For an object , define  via the splitting
in \Cat{D}:


\noindent For  we take
.  Then:


\noindent This  is unique up-to-isomorphism: each
idempotent  can be understood as a 
splitting in , namely of:


\noindent Since  preserves splittings and satisfies
, we get the definition as
described above.
}

Hayashi~\cite{Hayashi85} (see also~\cite{HoofmanM95}) has developed a
theory of semi-functors and semi-adjunctions that can be used to
capture non-extensional features, without uniques of mediating maps,
like for of exponents ~\cite{Scott80a,LambekS86},
products ~\cite{Jacobs91b}, or exponentials
~\cite{Hoofman92}. The Karoubi envelope can be used to turn such
``semi'' notions into proper (extensional) ones. This also happens in
Section~\ref{FoulisDagKerSec}.


\smallskip

Now assume \Cat{D} is a dagger category. An endomap  in \Cat{D} is called a self-adjoint idempotent if
. A splitting of such an  consists, as
before, of maps  with  and .
In that case  is also a splitting of , so that
we get an isomorphism  in a commuting
diagram:


\noindent Hence , as subobjects, and  as
quotients.

The dagger Karoubi envelope  of  is the
full subcategory of  with self-adjoint idempotents
as objects, see also~\cite{Selinger08}. This is again a dagger
category, since if  in
, then 
because:


\noindent Similarly .  The functor
 factors via
. One can
understand  as the free completion of \Cat{D}
with splittings of self-adjoint idempotents.

Selinger~\cite{Selinger08} shows that the dagger Karoubi envelope
construction  preserves dagger biproducts and dagger
compact closedness. Here we extend this with dagger kernels in the
next result. It will not be used in the sequel but is included to show
that the dagger Karoubi envelope is quite natural in the current
setting.


\begin{proposition}
\label{KaroubiKernelProp}
If \Cat{D} is a dagger kernel category, then so is
. Moreover, the embedding  is a map of dagger kernel
categories.
\end{proposition}


\begin{proof}
For each object , the zero map 
is a zero object in \Cat{D}, since there is precisely one map
 in , namely the zero
map . As canonical choice we take the zero
object  with zero map , which is in the range of .

For an arbitrary map  in
, let  be the kernel
of  in \Cat{D}. We obtain a map , as in:


\noindent since . We obtain that
 is a self-adjoint idempotent, using that  is a dagger mono
(\textit{i.e.}~satisfies ).


\noindent This yields a dagger kernel in ,


\noindent since:
\begin{itemize}
\item  is a morphism in :  and ;

\item  is a dagger mono:


\item ;

\item if  satisfies ,
  then there is a map  in \Cat{D} with
  .  Then , since . Similarly, , since . Hence  is a morphism  in
   with . It is the unique one with this property since  is a
  (dagger) mono. \QED
\end{itemize}
\end{proof}


\begin{example}
\label{KaroubiKernelEx}
In the category \Hilb self-adjoint idempotents  are also called projections. They can be written as  for a closed subspace , see any textbook on Hilbert spaces
(\textit{e.g.}~\cite{Dvurecenskij92}). Hence they split already in
\Hilb, and so the dagger Karoubi envelope  is
isomorphic to \Hilb: it does not add anything.

For the category \Rel of sets and relations the sitation is different.
A self-adjoint idempotent  is a relation
 that is both symmetric (since )
and transitive (since ), and thus a ``partial
equivalence relation'', commonly abbreviated as PER. The dagger
Karoubi envelope  has such PERs as objects. A
morphism  is a relation  with .  



\end{example}


Finally we note that the ``effect'' operation  can be described as a functor from the (total) category
 of kernels of a dagger kernel category
(see~\cite{HeunenJ09a}) to the dagger Karoubi envelope
 as in the diagram:


\noindent via:


\noindent We use that the necessarily unique map  with  satisfies
.
Hence:


\noindent so that  is a morphism  in the dagger Karoubi envelope . It
is not hard to see that this functor is full.

\auxproof{
We check functoriality:


\noindent As to fulness, assume  in . Taking  we get:


\noindent Hence  is a map  in .
It is mapped to itself: .
}



\section{Orthomodular lattices and Dagger kernel categories}\label{OMLatDagKerSec}

In~\cite{HeunenJ09a} it was shown how each dagger kernel category
gives rise to an indexed collection of orthomodular lattices, given by
the posets of the kernel subobjects  of each object
. Here we shall give a more systematic description of the situation
and see that a suitable category \Cat{OMLatGal} of orthomodular
lattices---with Galois connections between them---is itself a dagger
kernel category.  The mapping  turns out to be functor to
this category \Cat{OMLatGal}, providing a form of representation of
dagger kernel categories.

We start by recalling the basic notion of orthomodular lattices.  They
may be understood as a non-distributive generalisation of Boolean
algebras. The orthomodularity formulation is due to~\cite{Husimi37},
following~\cite{BirkhoffN36}.



\begin{definition}
\label{OMLatDef}
A meet semi-lattice  is called an ortholattice if it
comes equipped with a function  satisfying:
\begin{itemize}
   \item ;
   \item  implies ;
   \item .
\end{itemize}

\noindent One can then define a bottom element as  and join by , satisfying .

Such an ortholattice is called orthomodular if it satisfies (one of)
the three equivalent conditions:
\begin{itemize}
\item  implies ;

\item  implies ;

\item  and  implies .
\end{itemize}
\end{definition}


We shall consider two ways of organising orthomodular lattices
into a category.


\begin{definition}
The categories \Cat{OMLat} and \Cat{OMLatGal} both have orthomodular
lattices as objects.
\begin{enumerate}
\item A morphism  in \Cat{OMLat} is a function
   between the underlying sets that preserves
  ---and thus also ,  and ;

\item A morphism  in \Cat{OMLatGal} is a pair  of ``antitone'' functions  and  forming a Galois
  connection (or adjunction ):  iff
   for  and .

The identity morphism on  is the pair  given by the
self-adjoint map . Composition of  is given by:

\end{enumerate}
\end{definition}






The category \Cat{OMLat} is the more obvious one, capturing the
(universal) algebraic notion of morphism as structure preserving
mapping. However, the category \Cat{OMLatGal} has more interesting
structure, as we shall see. It arises by restriction from a familiar
construction to obtain a (large) dagger category with involutive
categories as objects and adjunctions between them,
see~\cite{Heunen09}. The components 
and  of a map 
in \Cat{OMLatGal} are not required to preserve any structure, but the
Galois connection yields that  preserves meets, as right
adjoint, and thus sends joins in  (meets in ) to meets in
, and dually,  preserves joins and sends joins in  to
meets in . 

The category  indeed has a dagger, namely by twisting:


\noindent A morphism  in \Cat{OMLatGal} is a
dagger mono precisely when it safisfies 
for all , because  and:



In a Galois connection like  one map determines
the other. This standard result can be useful in proving equalities,
which, for convenience, we make explicit.


\begin{lemma}
\label{MapEqLem}
Suppose we have parallel maps  in
\Cat{OMLatGal}. In order to prove  it suffices to prove either
 or .
\end{lemma}


\begin{proof}
We shall prove that  suffices to obtain also 
. For all  and ,


\noindent Given  this holds for all , and so in particular for
 and , which yields . \QED
\end{proof}


Despite this result we sometimes explicitly write out both equations
 and , in particular when there is a special
argument involved.

The following elementary lemma is fundamental.


\begin{lemma}
\label{DownsetLem}
Let  be an orthomodular lattice, with element . 
\begin{enumerate}
\item The (principal) downset  is
  again an orthomodular lattice, with order, conjunctions and
  disjunctions as in , but with its own orthocomplement  given
  by , where  is the
  orthocomplement from .

\item There is a dagger mono  in
  \Cat{OMLatGal}, for which we also write , with

\end{enumerate}
\end{lemma}


\begin{proof}
For the first point we check, for ,


\noindent by orthomodularity, since . We get a map in
\Cat{OMLatGal} because for arbitrary  and ,


\noindent This map  is a dagger mono
since:

\end{proof}



We should emphasise that the equation  only
holds for , and not for arbitrary elements .

Later, in Proposition~\ref{DownsetIsKerProp}, we shall see that these
maps  are precisely the kernels in the
category \Cat{OMLatGal}. But we first show that this category has
kernels in the first place.

To begin,  has a zero object , namely the
one-element orthomodular lattice . We can write its unique
element as . Let us show that the lattice  is indeed a
final object in . Let  be an arbitrary orthomodular
lattice. The only function  is . It
has an obvious left adjoint  defined by :


\noindent Likewise, the unique morphism  is given
by  and . Hence the zero morphism  is determined by  and .



\begin{theorem}
\label{OMLatGalDagKerCatThm}
  The category  is a dagger kernel category. The
  (dagger) kernel of a morphism  is , where , like in
  Lemma~\ref{DownsetLem}.
\end{theorem}


\begin{proof}
The composition  is the zero map .
First, for ,


\noindent because  in  and so  in
. And for ,


\noindent because  preserves joins as a
left adjoint.

Now suppose that  equals the zero morphism, for . Then  and . Hence for  we have , so .  Define  by , and define
 by .  Then  since for  and :


\noindent  whence  is a well-defined morphism of .
It satisfies:


\noindent by orthomodularity since  because  which follows from . Hence  is a mediating morphism satisfying
.  It is the unique such morphism, since  is a
(dagger) mono.  \QED

\auxproof{
In order to prove  we use
the following observation. Since  we have
, for each , and thus:


\noindent We use this property  twice below, as indicated:

}
\end{proof}


For convenience we explicitly describe some of the basic structure
that results from dagger kernels, see~\cite{HeunenJ09a}, namely
cokernels and factorisations, given by dagger kernels and zero-epis.
We start with cokernels and zero-epis.

\begin{lemma}
\label{CokerLem}
The cokernel of a map  in \Cat{OMLatGal} is:


\noindent Then:

\end{lemma}


\begin{proof}
Since:

\end{proof}


We recall from~\cite{HeunenJ09a} that each map  in a dagger kernel
category has a zero-epi/kernel factorisation . In combination with the factorisation of  it yields
a factorisation  as in:


\noindent where the map  is both zero-epic and zero-monic, and
where , the zero-epic part
of .



\begin{lemma}
\label{ImFacLem}
For a map  in \Cat{OMLatGal} one has:

\end{lemma}



\begin{proof}
This is just a matter of unravelling definitions. For instance, 


\noindent since . We check that , as
required.


\noindent by orthomodularity, using that , 
since . This map  is indeed zero-epic by the previous
lemma, since:


Next we first observe:


\noindent since there is a ``unit'' . We use
this twice, in the marked equations, in:


\noindent The map  is zero-epic since:


\noindent Similarly one shows that  is zero-monic. \QED

\auxproof{
Similarly, one has


\noindent since there is a ``counit'' ,
and thus:


\noindent Via this auxiliary result one obtains that  is zero-monic:

}
\end{proof}




For the record, inverse and direct images are described explicitly.


\begin{lemma}
\label{InvDirImLem}
For a map  in \Cat{OMLatGal} the associated
inverse and direct images are:

\end{lemma}



\begin{proof}
For  and , we have, using the
formulation for pullback of kernels from Section~\ref{DagKerSec}
(or~\cite[Lemma~2.4]{HeunenJ09a}) and Lemma~\ref{ImFacLem} above,


\noindent For  we also use Lemma~\ref{ImFacLem}  in:

\end{proof}



In the category \Cat{OMLatGal}, like in any dagger kernel category,
the kernel posets  are orthomodular lattices. They turn out
to be isomorphic to the underlying object .



\begin{proposition}
\label{DownsetIsKerProp}
Each dagger mono  from
Lemma~\ref{DownsetLem}, for , is actually a dagger
kernel. This yields an isomorphism of orthomodular lattices


\noindent It is natural in the sense that for 
in \Cat{OMLatGal} the following squares commute by Lemma~\ref{InvDirImLem}.

\end{proposition}


\begin{proof}
  We first check that  is indeed a
  kernel, namely of its cokernel , see Lemma~\ref{CokerLem}, where . Thus,
  .

  Theorem~\ref{OMLatGalDagKerCatThm} says that the mapping  is surjective. Here we shall show that it is
  an injective homomorphism of orthomodular lattices reflecting the
  order, so that it is an isomorphism in the category \Cat{OMLat}.

Assume that  in . We can define  by  and , for
 and . Then, clearly,  iff , so that  is a
morphism in \Cat{OMLatGal}. In order to show  in 
we prove . First, for ,


\auxproof{
Similarly, for an arbitrary ,


\noindent The marked equation holds because for arbitrary ,

}

\noindent The map  does not only preserve the
order, but also reflects it: if we have an arbitrary map  in \Cat{OMLatGal} with , then:


\noindent This  map also preserves ,
since


\noindent where, according to Theorem~\ref{OMLatGalDagKerCatThm},
.

It remains to show that the mapping  preserves
finite conjunctions. It is almost immediate that it sends the top
element  to the identity map (top) in . It also
preserves finite conjunctions, since the intersection of the kernels
 and  is given by
. Since  there
are appropriate maps 
and . Suppose that we
have maps  and ,
where  is a kernel of
. Since, as we have seen, the order is
reflected, we get , and thus , yielding the required map . \QED
\end{proof}



The adjunction  that exists in arbitrary
dagger kernel categories (see Section~\ref{DagKerSec}
or~\cite[Proposition~4.3]{HeunenJ09a}) boils down in our example
\Cat{OMLatGal} to the adjunction between  in the
definition of morphisms in \Cat{OMLatGal}, since:


\noindent Moreover, the Sasaki hook  and and-then operators
 defined categorically
in~\cite[Proposition~6.1]{HeunenJ09a}, see Section~\ref{DagKerSec},
amount in \Cat{OMLatGal} to their usual definitions in the theory of
orthomodular lattices, see \textit{e.g.}~\cite{Finch70,Kalmbach83}. This will
be illustrated next.  We use the ``effect''  associated
with a kernel in:


\noindent where, according to the description of inverse image 
 in the previous lemma,


\noindent Similarly for and-then :


\noindent where the description of direct image from
Lemma~\ref{InvDirImLem} yields:


\noindent These  and  are, by construction, related
via an adjunction (see also~\cite{Finch70,CoeckeS04}).

Also one can define a weakest precondition modality  from dynamic
logic in this setting: for  and , put:


\noindent for `` holds after ''. This operation 
preserves conjunctions, as usual. An element  yields a test
operation . Then one can recover the
Sasaki hook  via this modality as , and hence
complement  as , see also
\textit{e.g.}~\cite{BaltagS06}.

There is another isomorphism of interest in this setting.


\begin{lemma}
\label{PointLem}
Let  be the 2-element Boolean algebra, considered as an
orthomodular lattice . For each orthomodular lattice
, there is an isomorphism (of sets): 


\noindent which maps  to 
given by:


\noindent This isomorphism is natural: for  one has:

\end{lemma}


\begin{proof}
The thing to note is that for a map  in
\Cat{OMLatGal} we have  because  is a right adjoint. Hence we can only choose .  Once this is chosen, the left adjoint  is completely determined, namely as  iff .

As to naturality, it suffices to show:

\end{proof}


By combining the previous two results we obtain a way to classify
(kernel) subobjects, like in a topos~\cite{MacLaneM92}, but with
naturality working in the opposite direction. In~\cite{HeunenJ09a} a
similar structure was found in the category \Rel of sets and
relations, and also in the dagger kernel category associated with a
Boolean algebra.


\begin{corollary}
\label{OpClassCor}
The 2-element lattice  is an ``opclassifier'':
there is a ``characteristic'' isomorphism:


\noindent which is natural: . \QED
\end{corollary}


We conclude our investigation of the category \Cat{OMLatGal} with
the following observation.


\begin{proposition}
\label{BiprodProp}
The category  has (finite) dagger biproducts .
Explicitly,  is the Cartesian product of (underlying
sets of) orthomodular lattices, with coprojection  defined by  and
. The dual product structure is given by
.
\end{proposition}



\begin{proof}
  Let us first verify that  is a well-defined morphism of
  , \textit{i.e.} that :


\noindent Also,  is a dagger mono since:
 

\noindent Likewise, there is a dagger mono . For , one finds that  is the zero morphism.

\auxproof{

}

In order to show that  is indeed a coproduct, suppose
that morphisms  are given.  We then define the
cotuple  by

and . Clearly,
, and:

\auxproof{
  
}



\auxproof{

}

\noindent so that . Likewise, . Moreover,
if  also satisfies
, then:

\end{proof}




\subsection{From dagger kernel categories to orthomodular lattices}

The aim in this subsection is to show that for a dagger kernel \Cat{D}
the kernel subobject functor  is a functor . On a morphism  of
, define  by:


\noindent Then indeed : 


\auxproof{
As a result we get a functor , since:

}

\noindent The functor  preserves the relevant
structure. This requires the following auxiliary result.


\begin{lemma}
\label{lem:kernelschangeofbase}
In a dagger kernel category, for any kernel  in , there is an order isomorphism .
\end{lemma}


\begin{proof}
The direction  of the desired bijection is
given by . This is well-defined since kernels
are closed under composition.  The other direction  is , where . One easily checks that these maps are each other's
inverse, and preserve the order.  \QED
\end{proof}












\begin{theorem}
\label{KSubPreservationThm}
Let  be a dagger kernel category. The functor
,
\begin{enumerate}
\item[(a)] is a map of dagger kernel categories;

\item[(b)] preserves (finite) biproducts, in case they exist in .
\end{enumerate}
\end{theorem}



\begin{proof}
Preservation of daggers follows because  and  are
inter-expressible, see Section~\ref{DagKerSec}
and~\cite[Proposition~4.3]{HeunenJ09a}:


Preservation of the zero object is easy: .

Next, let  be the kernel of a morphism  in . We recall
from~\cite[Corollary~2.5~(ii)]{HeunenJ09a} that this kernel  can be
described as inverse image . Hence by Lemmas~\ref{lem:kernelschangeofbase}
and~\ref{DownsetLem}, we have the isomorphism on the left in:


\noindent It yields a commuting triangle since for ,


\noindent Similarly for , 


For~(b), it suffices to prove that  is a coproduct in .  Let
morphisms  in  be
given. Define a cotuple  by


\noindent Indeed : 


\noindent This morphism  of  satisfies:


\auxproof{
\noindent In the other direction, using that meets  are
preserved under pullback,

}

\noindent Towards uniqueness, assume  in \Cat{OMLatGal} also satisfies . Then:


\noindent This last step needs justification. By
Theorem~\ref{OMLatGalDagKerCatThm},  can
be written as  for certain . Then, by
Lemma~\ref{InvDirImLem},


\noindent since:

\end{proof}



At this stage we conclude that these  functors yield a
well-behaved translation of a dagger kernel category into a collection
of orthomodular lattices, indexed by the objects of the category. For
the special case , the functor  is the identity, up to
isomorphism, by Proposition~\ref{DownsetIsKerProp}. A translation in
the other direction, from orthomodular lattices to dagger kernel
categories will be postponed until after the next section, after we
have seen the translation from Foulis semigroups to orthomodular
lattices.

In the remainder of this section we shall briefly consider two
special subcategories of \Cat{OMLatGal}, namely with Boolean
and with complete orthomodular lattices.



\subsection{The Boolean case}\label{BooleanSubsec}

Let  be the full
subcategory of Boolean algebras with (antitone) Galois connections
between them. We recall that a Boolean algebra can be described as an
orthomodular lattice that is distributive.

The main (and only) result of this subsection is simple.


\begin{proposition}
\label{BoolGalStructProp}
The category \Cat{BoolGal} inherits dagger kernels and biproducts
from \Cat{OMLatGal}. Moreover, as a dagger kernel category it is
Boolean.
\end{proposition}


\begin{proof}
An arbitrary map  in \Cat{BoolGal} has a
kernel  as in
Theorem~\ref{OMLatGalDagKerCatThm} for orthomodular lattices because
the downset  is a Boolean algebra. Similarly, the
biproducts from Proposition~\ref{BiprodProp} also exist in
\Cat{BoolGal} because  is a Boolean algebra if
 and  are Boolean algebras.

For each  one has  so that
 is a Boolean algebra. Hence \Cat{BoolGal} is a Boolean
dagger kernel category by~\cite[Theorem~6.2]{HeunenJ09a}. \QED 

\auxproof{
Old, direct proof.

We check the property ,
for kernels , that defines Booleanness for dagger kernel
categories, see~\cite{HeunenJ09a}. Now suppose we have kernels
 and  in
\Cat{BoolGal} with


\noindent see Proposition~\ref{DownsetIsKerProp}. 
Then  in , and thus 


\noindent Hence , so that the corresponding kernels
satisfy, for ,

}
\end{proof}


Boolean algebras thus give rise to (Boolean) dagger kernel categories
on two different levels: the ``large'' category \Cat{BoolGal} of all
Boolean algebras is a dagger kernel category, but also each individual
Boolean algebra can be turned into a ``small'' dagger kernel category,
see~\cite[Proposition~3.5]{HeunenJ09a}.



\subsection{Complete orthomodular lattices}\label{CompleteSubsec}

We shall write  for the
full subcategory of orthomodular lattices that are complete,
\textit{i.e.}~that have joins  (and thus also meets
) of all subsets  (and not just the finite ones).
Notice that the functor  from Theorem~\ref{KSubPreservationThm} is actually a
functor  for \Cat{D} =
\Rel, \PInj, \Hilb.

A morphism  in \Cat{OMSupGal} is completely
determined by either  preserving all
meets, or by  preserving all
joins. This forms the basis for the next result.


\begin{proposition}
\label{FreeOMLatGalProp}
The forgetful functor  given by
 on objects and  on morphisms
has a left adjoint  given by , with
 and , for  in .
\end{proposition}


\begin{proof}
For  and  there is a bijective
correspondence:


\noindent given by  and
 with
. Then:


\noindent Further,


\auxproof{
We check naturality:

}
\end{proof}



The left adjoint  of this adjunction between \cat{OMSupGal} and
\Sets factors via the graph functor , as in:


It is not hard to see that the kernels from
Theorem~\ref{OMLatGalDagKerCatThm} and biproducts  from
Proposition~\ref{BiprodProp} also exist in \Cat{OMSupGal}.  For
instance, the join of a subset  is given as pair
of joins:


\noindent Hence \Cat{OMSupGal} is also a dagger kernel category
with dagger biproducts.





\section{Foulis semigroups and dagger kernel categories}\label{FoulisDagKerSec}

In this section we shall relate dagger kernel categories and Foulis
semigroups. Without a definition yet, we first illustrate that these
Foulis semigroups arise quite naturally in the context of kernel
dagger categories.

In every category \Cat{D} the homset  of
endomaps  is a monoid (or semigroup with
unit), with obvious composition operation  and identity map
 as unit element. If \Cat{D} is a dagger category, there is
automatically an involution  on this monoid. If it is
moreover a dagger kernel category, every endomap 
yields a self-adjoint idempotent, namely the effect of its kernel:


\noindent with the special property that for ,


\noindent Indeed, if , then:


\noindent Conversely, if , then there is a map
 in \Cat{D} with . Hence  satisfies:


This structure of an involutive monoid  with such an operation  has been introduced in the 1960s by
Foulis~\cite{Foulis60,Foulis62,Foulis63} and has since then been
studied under the name `Baer *-semigroup' or `Foulis semigroup',
see~\cite[Chapter~5, \S\S18]{Kalmbach83} for a brief overview.




\begin{definition}
\label{FoulisDef}
A Foulis semigroup consists of a monoid (semigroup with unit)  together with two endomaps 
and  satisfying:
\begin{enumerate}
\item  and 
  and , making  an involutive monoid;

\item  is a self-adjoint idempotent, \textit{i.e.}~satisfies
  ;

\item  is a zero
element: ;

\item  iff .

\end{enumerate}

\noindent Or, equivalently (see~\cite[Chapter~5, \S\S18,
Lemma~1]{Kalmbach83}),
\begin{enumerate}
\item[4.]  and  and 
.
\end{enumerate}

We form a category \Cat{Fsg} of such Foulis semigroups with monoid
homomorphisms that commute with  and  as morphisms.
\end{definition}


\auxproof{
We show the equivalence of  and .

Assume . Since  we have  for
some . Hence . 

Next, , so that 
and thus .

Finally, in order to prove  we first note:


\noindent Hence ,
for some , and thus:


Conversely, assume . If , then 


\noindent And if , then .
}



The constructions before this definition show that for each object
 of a (locally small) dagger kernel category \Cat{D}, the
homset  of endomaps on  is a Foulis
semigroup. Functoriality of this construction is problematic: for an
arbitrary map  in \Cat{D} there is a mapping
, namely , but it does not preserve the structure of Foulis semigroups,
and thus only gives rise to presheaf.


\begin{proposition}
\label{DagKer2FoulisProp}
For a dagger kernel category \Cat{D}, each endo homset ,
for , is a Foulis semigroup. The mapping  yields a presheaf . \QED
\end{proposition}




The lack of functoriality in this construction is problematic. One
possible way to address it is via another notion of morphism between
Foulis semigroups, like Galois connections between orthomodular
lattices in the category \Cat{OMLatGal}. We shall not go deeper into
this issue. Also the possible sheaf-theoretic aspects involved in this
situation (see also~\cite{GravesS73}) form a topic on its own that is
not pursued here.  We briefly consider some examples.

For the dagger kernel category \Hilb of Hilbert spaces, the set  of (bounded/continuous linear) endomaps on a Hilbert space
 forms a Foulis semigroup---but of course also a -algebra. The
associated (Foulis) map  maps  to  given by , where  is the kernel map
.



For the category \Rel of sets and relations the endomaps on a set 
are the relations  on . The associated
 is .

An interesting situation arises when we apply the previous proposition
to the dagger kernel category \Cat{OMLatGal} of orthomodular lattices
(with Galois connections between them). One gets that for each
orthomodular lattice  the endo-homset  forms a Foulis semigroup. This construction is
more than 40 years old, see~\cite{Foulis60} or
\textit{e.g.}~\cite[Chapter~II, Section~19]{BlythJ72}
or~\cite[Chapter~5, \S\S18]{Kalmbach83}, where it is described in
terms of Galois connections.  In the present setting it comes for
free, from the structure of the category \Cat{OMLatGal}. Hence we
present it as a corollary, in particular of
Proposition~\ref{DagKer2FoulisProp} and
Theorem~\ref{OMLatGalDagKerCatThm}.


\begin{corollary}
\label{OMLat2FoulisCor}
For each orthomodular lattice  the set of (Galois) endomaps
 is a Foulis semigroup with
composition as monoid, dagger  as involution, and
self-adjoint idempotent , for , defined as in~(\ref{DagKerSaiEqn}). Equivalently,
 can be described via the Sasaki hook  or and-then
operator :

\end{corollary}


\begin{proof}
We recall from~(\ref{DagKerSaiEqn}) that the operation  on
endomaps  is defined as . In \Cat{OMLatGal} one
has ---see Proposition~\ref{DownsetIsKerProp}---so
that:

\end{proof}


\auxproof{
Old, direct proof.

Obviously  and  and , for endomaps
 in \Cat{OMLatGal}. Also,  as defined
above is a morphism in \Cat{OMLatGal}, since for ,


\noindent These 's are indeed projections: by construction,
 and:


\noindent Further,


\noindent Hence  is the zero map , satisfying  and , as we
have seen before. Also, this zero map satisfies 
since: . From  we get,
via the Galois connection, . Hence . We immediately use this in:


\noindent If we put  then we
have just seen that . We have to prove . As first step, observe that for any ,


\noindent By the adjunction:  and
thus . Now we are done by
orthomodularity:

}




\subsection{From Foulis semigroups to dagger kernel categories}

Each involutive monoid  forms a dagger category
with one object, and morphisms given by elements of
. Requirement~(4) in Definition~\ref{FoulisDef} says that this
category has ``semi'' kernels, given by . Hence it is natural
to apply the Karoubi envelope to obtain proper kernels. It turns out
that this indeed yields a dagger kernel category.

For a Foulis semigroup as in Definition~\ref{FoulisDef}, we thus write
 for the dagger Karoubi envelope applied to  as
one-object dagger category. Thus  has self-adjoint
idempotents  as objects, and morphisms  given by elements  with .



\begin{theorem}
\label{Foulis2DagKerThm}
This  is a dagger kernel category. The mapping
 yields a functor .
\end{theorem}


\begin{proof}
The zero element  is obviously a self-adjoint
idempotent, and thus an object of . It is a zero
object because for each  there is precisely one
map , namely , because .

For an arbitrary map  in 
we claim that there is a dagger kernel of the form:


\noindent This will be checked in a number of small steps.
\begin{itemize}
\item ,
by~ in Defintion~\ref{FoulisDef};

\item By the previous point there is an element  with
. Hence:


\noindent This is equation is very useful. It yields first of all that
 is idempotent:


\noindent This element is also self-adjoint:


\noindent Hence  is a self-adjoint idempotent, and
thus an object of .

\item  is also a dagger mono:


\item Finally, if  in 
  satisfies , then there is a 
  with . Then:


\noindent Hence  is the mediating map ,
since . Uniqueness follows because  is a dagger mono.
\end{itemize}

As to functoriality, assume  is a morphism of
Foulis semigroups. It yields a functor  by  and .
This  preserves all the dagger kernel structure because it
preserves the Foulis semigroup structure. \QED
\end{proof}


By combining this result with Proposition~\ref{DagKer2FoulisProp}
we have a way of producing new Foulis semigroups from old.


\begin{corollary}
\label{FoulisEndoCor}
Each self-adjoint idempotent  in a Foulis semigroup  yields
a Foulis semigroup of endo-maps:


\noindent with composition , unit , involution  and
. The special case  yields the original semigroup:
.
\end{corollary}


\begin{proof}
We only check the formulation following~(\ref{DagKerSaiEqn}):

\end{proof}


The posets of kernel subobjects in a dagger kernel category are
orthomodular lattices. This applies in particular to the category
 and yields a way to construct orthomodular lattices
out of Foulis semigroups. We first investigate this lattice structure
in more detail, via (isomorphic) subsets of .


\begin{lemma}
\label{FoulisOMKerLem}
Let  be a Foulis semigroup with self-adjoint idempotent ,
considered as object . The subset


\noindent is an orthomodular lattice with the following structure.


\noindent In fact, .
\end{lemma}



\begin{proof}
In fact it suffices to prove the last isomorphism  and use it to translate the orthomodular structure from
 to . Instead we proceed in a direct manner and show
that each  is an orthomodular lattice in a number of small
consecutive steps, resembling the steps taken in~\cite[Chapter~5,
  \S\S18]{Kalmbach83}. One observation that is used a number of times
is:


\noindent for arbitrary , Indeed, if , then by
requirement~(4) in Definition~\ref{FoulisDef} there is a  with
. But then .

Let  now be a fixed self-adjoint idempotent. 
\begin{enumerate}
\renewcommand{\theenumi}{(\alph{enumi})}
\item Each  is a self-adjoint idempotent, a dagger kernel
, and also an idempotent 
in .

Indeed, if , then , so that  by . 
Hence:


\noindent Also,  is the kernel of , using the description of kernels in
 from the proof of Theorem~\ref{Foulis2DagKerThm}.

\item The set  carries a transitive order  iff
. This  is a partial order on .

Transitivity is obvious: if , then  and
 so that ,
showing that .

Reflexivity  holds for  since we have  as shown in~(a). For symmetry assume  and 
where . Then  and . Hence .

\item For an arbitrary  put . Hence from~(a) we get equations  and  that are useful in calculations.

We will show 
and  for .

Assume , \textit{i.e.}~. Then, applying the
equation  from
requirement~ in Definition~\ref{FoulisDef} for  and  yields:


\noindent This gives us what we need to show :


Next we notice that 


\noindent Requirement~ in Definition~\ref{FoulisDef}, applied to , 
says:


\noindent It says that . In particular,
this means  for . Since 
reverses the order we get:


\noindent If we finally apply this to , say for
 we get:


\item As motivation for the definition of meet, consider for  their meet as kernels:


\noindent We force this  into  via double negation and hence
define . Showing that it is the
meet of  requires a bit of work.
\begin{itemize}
\item We have , so that  and thus also .

\item We first observe that


\noindent Hence by applying  we get . Via  we obtain , and thus also


\noindent This says , from
which we get .

\item If also  satisfies  and ,
\textit{i.e.}~, then, by
Definition~\ref{FoulisDef}~,


\noindent Hence 
by~ and so . Thus .
\end{itemize}

\item We get , for , as follows.
  Since  one has 
  by~. Hence:


\item Finally, orthomodularity holds in . We assume  (\textit{i.e.}~) and , for , and have to show 
  (\textit{i.e.}~, and thus ). To start,
  , so that . Using~ yields , and also . Hence:


\noindent By~ we get  so that
, as required to get
.
\end{enumerate}

Finally we need to show . As we have seen
in~(a), each  yields (an equivalence class of) a kernel
. Conversely, each kernel  of a map  in
---see the proof of
Theorem~\ref{Foulis2DagKerThm}---is an element of . This yields
an order isomorphism: if  for , then  so that we get a commuting
triangle:


\noindent showing that  in . Conversely,
if there is an  with ,
then ,
showing that  in . \QED
\end{proof}



\subsection{Generators}\label{GeneratorSubsec}

Recall that a generator in a category is an object  such that for
each pair of maps , if  for all , then . Every singleton set
is a generator in \Sets, and also in \Rel. The complex numbers
 form a generator in the category \Hilb of Hilbert spaces
of . And the two-element orthomodular lattice is a
generator in \Cat{OMLatGal} by Lemma~\ref{PointLem}.

We shall write  for the subcategory
of dagger kernel categories with a given generator, and with morphisms
preserving the generator, up-to-isomorphism.


\begin{lemma}
The dagger kernel category  associated with a Foulis
semigroup has the unit  as generator. The functor  from Theorem~\ref{Foulis2DagKerThm} restricts
to .
\end{lemma}


\begin{proof}
Assume  in  with  for each map . Then, in
particular for  we get . Every
morphism  of Foulis semigroups satisfies
, so that the induced functor  preserves the generator. \QED
\end{proof}



\begin{lemma}
The mapping  yields a functor
.
\end{lemma}


\begin{proof}
If  is a functor in ,
then one obtains a mapping 
by:


\noindent Since all the orthomodular structure in kernel posets
 is defined in terms of kernels and daggers, it is preserved
by . \QED
\end{proof}


By composition we obtain the original (``old'') way to construct an
orthomodular lattice out of a Foulis semigroup, see~\cite{Foulis63}.


\begin{corollary}
\label{Foulis2OMLatCor}
The composite functor  maps a Foulis semigroup  to the orthomodular lattice
 from
Lemma~\ref{FoulisOMKerLem}, over the generator . \QED
\end{corollary}



\auxproof{
We follow the proof in~\cite[Chapter~5, \S\S18]{Kalmbach83} in a
number of small steps, involving elements , often
using requirement  from Definition~\ref{FoulisDef}.
\begin{enumerate}
\renewcommand{\theenumi}{\alph{enumi}}
\item  iff . For the (if)-part, assume
  ; then .  Conversely, if  then .

\item , and thus . This is obtained by substituting  for  and
   for  in the equation  from Defintion~\ref{FoulisDef}~. It
  yields:


\noindent Hence the result follows by applying the involution
 on both sides.

\item , so that  for 
. To start, 


\noindent Hence we can apply~(a), so that , where the last equation
follows by applying~(b) to .

\item  implies  (and thus ). We assume  and have to prove . Since  the following suffices.


\item , by~(d), (b) and
  transitivity of . Hence  is
  self-adjoint wrt.\ , with  as set of closed elements, with a
  Galois connection:


\noindent since for  and  one has 
iff .

\item Multiplication  is idempotent on , so that 
  is reflexive on .  This is easy since  by~(b) and~(c).

\item The order  is also symmetric on , since if  and , then:


\item Elements  have a meet , where .
\begin{itemize}
\item  holds since  by Definition~\ref{FoulisDef}~. Hence
  by~(a) we get:


\noindent But then  follows:


\noindent Hence .

\item Also  since:


\noindent so that  by~(a). Hence:


\noindent This means  and thus .

\item If also  satisfies  and ,
\textit{i.e.}~, then, by
Definition~\ref{FoulisDef}~,


\noindent Hence  by~(a). Then:


\noindent Hence .
\end{itemize}

\item , for , by a straightforward
calculation:


\item Finally, orthomodularity holds in . We assume 
  and  and have to show  (and thus
  ). One gets:


\noindent Hence , by~(a), and thus , so that . \QED
\end{enumerate}
}


In the reverse direction we have seen in
Corollary~\ref{OMLat2FoulisCor} that the set  of (Galois)
endomaps on an orthomodulair lattice  is a Foulis semigroup, but
functoriality is problematic. However, we can now solve a problem that
was left open in~\cite{HeunenJ09a}, namely the construction of a
dagger kernel category out of an orthomodular lattice
. Theorem~\ref{Foulis2DagKerThm} says that the dagger Karoubi
envolope  is a dagger kernel category. Its
objects are self-adjoint idempotents , and its
morphisms  are maps  in \Cat{OMLatGal} with .



\section{Conclusions}\label{ConclusionSec}

There is a relatively recent line of research applying categorical
methods in quantum theory, see for
instance~\cite{ButterfieldI98,AbramskyC04,Selinger07,DoeringI08,HeunenLS09,CoeckePV09}. This
paper fits into this line of work, with a focus on quantum logic
(following~\cite{HeunenJ09a}), and establishes a connection to early
work on quantum structures. It constructs new (dagger kernel)
categories of orthomodular lattices and of self-adjoint idempotents in
Foulis semigroups (also known as Baer *-semigroups). These categorical
constructions are shown to generalise translations between
orthomodular lattics and Foulis semigroups from the 1960s.  They
provide a framework for the systematic study of quantum (logical)
structures.

The current (categorical logic) framework may be used to address some
related research issues. We mention three of them.
\begin{itemize}
\item As shown, the category \Cat{OMLatGal} of orthomodular lattices
  and Galois connections has (dagger) kernels and biproducts
  . An open question is whether it also has tensors ,
  to be used for the construction of (logics of) compound systems. The
  existence of such tensors is a subtle matter, given the restrictions
  described in~\cite{RandallF79}.

\item A dagger kernel category gives rise to not just one orthomodular
  lattice (or Foulis semigroup), but to a collection, indexed by the
  objects of the category, see for instance the presheaf description
  in Proposition~\ref{DagKer2FoulisProp}. The precise, possibly
  sheaf-theoretic (see~\cite{GravesS73}), nature of this indexing is
  not fully understood yet.

\item So-called effect algebras have been introduced as more recent
  generalisations of orthomodular lattices, see~\cite{DvurecenskijP00}
  for an overview. An open question is how such quantum structures
  relate to the present approach.
\end{itemize}


\subsubsection*{Acknowledgements}

Many thanks to Chris Heunen for discussions and joint
work~\cite{HeunenJ09a}.





\begin{thebibliography}{10}

\bibitem{AbramskyC04}
S.~Abramsky and B.~Coecke.
\newblock A categorical semantics of quantum protocols.
\newblock In {\em Logic in Computer Science}, pages 415--425. IEEE, Computer
  Science Press, 2004.

\bibitem{BaltagS06}
A.~Baltag and S.~Smets.
\newblock {LQP}: the dynamic logic of quantum information.
\newblock {\em Math. Struct. in Comp. Sci.}, 16:491--525, 2006.

\bibitem{BirkhoffN36}
G.~Birkhoff and J.~von Neumann.
\newblock The logic of quantum mechanics.
\newblock {\em Ann. Math.}, 37:823--843, 1936.

\bibitem{BlythJ72}
T.S. Blyth and M.F. Janowitz.
\newblock {\em Residuation Theory}.
\newblock Pergamum Press, 1972.

\bibitem{ButterfieldI98}
J.~Butterfield and C.J. Isham.
\newblock A topos perspective on the {K}ochen-{S}pecker theorem: {I}. quantum
  states as generalized valuations.
\newblock {\em Int. Journ. Theor. Physics}, 37(11):2669–--2733, 1998.

\bibitem{CoeckePV09}
B.~Coecke, D.~Pavlovi{\'c}, and J.~Vicary.
\newblock A new description of orthogonal bases.
\newblock {\em Math. Struct. in Comp. Sci.}, 2009, to appear.
\newblock Available from \url{http://arxiv.org/abs/0810.0812}.

\bibitem{CoeckeS04}
B.~Coecke and S.~Smets.
\newblock The {Sasaki} hook is not a [static] implicative connective but
  induces a backward [in time] dynamic one that assigns causes.
\newblock {\em Int. Journ. of Theor. Physics}, 43(7/8):1705--1736, 2004.

\bibitem{DoeringI08}
A.~D{\"o}ring and C.J. Isham.
\newblock A topos foundation for theories of physics: {I - IV}.
\newblock {\em Journ. of Math. Physics}, 49:053515--053518, 2008.

\bibitem{Dvurecenskij92}
A.~Dvure\v{c}enskij.
\newblock {\em Gleason's Theorem and Its Applications}.
\newblock Number~60 in Mathematics and its Applications. Kluwer Acad. Publ.,
  Dordrecht, 1992.

\bibitem{DvurecenskijP00}
A.~Dvure\v{c}enskij and S.~Pulmannov{\'a}.
\newblock {\em New Trends in Quantum Structures}.
\newblock Kluwer Acad. Publ., Dordrecht, 2000.

\bibitem{Finch70}
P.~D. Finch.
\newblock Quantum logic as an implication algebra.
\newblock {\em Bull. Amer. Math. Soc.}, 2:101--106, 1970.

\bibitem{Foulis60}
D.~J. Foulis.
\newblock {Baer} *-semigroups.
\newblock {\em Proc. Amer. Math. Soc.}, 11:648--654, 1960.

\bibitem{Foulis62}
D.~J. Foulis.
\newblock A note on orthomodular lattices.
\newblock {\em Portugaliae Mathematica}, 21:65--72., 1962.

\bibitem{Foulis63}
D.~J. Foulis.
\newblock Relative inverses in {Baer} *-semigroups.
\newblock {\em Michigan Math. Journ.}, 10(1):65--84, 1963.

\bibitem{Freyd64}
P.J. Freyd.
\newblock {\em Abelian Categories: An Introduction to the Theory of Functors}.
\newblock Harper and Row, New York, 1964.
\newblock Available via \url{www.tac.mta.ca/tac/reprints/articles/3/tr3.pdf}.

\bibitem{GravesS73}
W.H. Graves and S.A. Selesnick.
\newblock An extension of the {Stone} representation for orthomodular lattices.
\newblock {\em Coll. Mathematicum}, XXVII:21--30, 1973.

\bibitem{Hayashi85}
S.~Hayashi.
\newblock Adjunction of semifunctors: categorical structures in nonextensional
  lambda calculus.
\newblock {\em Theor. Comp. Sci.}, 41:95--104, 1985.

\bibitem{Heunen09}
C.~Heunen.
\newblock {\em Categorical Quantum Models and Logics}.
\newblock PhD thesis, Univ. Nijmegen, 2009.

\bibitem{HeunenJ09a}
C.~Heunen and B.~Jacobs.
\newblock Quantum logic in dagger kernel categories.
\newblock In B.~Coecke, P.~Panangaden, and P.~Selinger, editors, {\em
  Proceedings of the 6th International Workshop on Quantum Programming
  Languages (QPL 2009)}, Elect. Notes in Theor. Comp. Sci. Elsevier, Amsterdam,
  2009, to appear.
\newblock Available from \url{http://arxiv.org/abs/0902.2355}.

\bibitem{HeunenLS09}
C.~Heunen, N.P. Landsman, and B.~Spitters.
\newblock A topos for algebraic quantum theory.
\newblock {\em Comm. in Math. Physics}, 2009, to appear.
\newblock Available from \url{http://arxiv.org/abs/0709.4364}.

\bibitem{Hoofman92}
R.~Hoofman.
\newblock {\em Non-Stable Models of Linear Logic}.
\newblock PhD thesis, Univ. Utrecht, 1992.

\bibitem{HoofmanM95}
R.~Hoofman and I.~Moerdijk.
\newblock A remark on the theory of semi-functors.
\newblock {\em Math. Struct. in Comp. Sci.}, 5(1):1--8, 1995.

\bibitem{Husimi37}
K.~Husimi.
\newblock Studies on the foundation of quantum mechanics {I}.
\newblock {\em Proc. physicomath. Soc. Japan}, 19:766--789, 1937.

\bibitem{Jacobs91b}
B.~Jacobs.
\newblock Semantics of the second order lambda calculus.
\newblock {\em Math. Struct. in Comp. Sci.}, 1(3):327--360, 1991.

\bibitem{Jacobs99a}
B.~Jacobs.
\newblock {\em Categorical Logic and Type Theory}.
\newblock North Holland, Amsterdam, 1999.

\bibitem{Kalmbach83}
G.~Kalmbach.
\newblock {\em Orthomodular Lattices}.
\newblock Academic Press, London, 1983.

\bibitem{Karoubi78}
M.~Karoubi.
\newblock {\em K-theory. An Introduction}.
\newblock Springer, 1978.

\bibitem{LambekS86}
J.~Lambek and P.J. Scott.
\newblock {\em Introduction to higher order Categorical Logic}.
\newblock Number~7 in Cambridge Studies in Advanced Mathematics. Cambridge
  Univ. Press, 1986.

\bibitem{MacLaneM92}
S.~Mac Lane and I.~Moerdijk.
\newblock {\em Sheaves in Geometry and Logic. A First Introduction to Topos
  Theory}.
\newblock Springer, New York, 1992.

\bibitem{RandallF79}
C.~Randall and D.J. Foulis.
\newblock Tensor products of quantum logics do not exist.
\newblock {\em Notices Amer. Math. Soc.}, 26(6):A--557, 1979.

\bibitem{Scott80a}
D.S. Scott.
\newblock Relating theories of the -calculus.
\newblock In J.R. Hindley and J.P. Seldin, editors, {\em To H.B. Curry: Essays
  on Combinatory Logic, Lambda Calculus and Formalism}, pages 403--450, New
  York and London, 1980. Academic Press.

\bibitem{Selinger07}
P.~Selinger.
\newblock Dagger compact closed categories and completely positive maps
  (extended abstract).
\newblock In P.~Selinger, editor, {\em Proceedings of the 3rd International
  Workshop on Quantum Programming Languages (QPL 2005)}, number 170 in Elect.
  Notes in Theor. Comp. Sci., pages 139--163. Elsevier, Amsterdam, 2007.
\newblock DOI \url{http://dx.doi.org/10.1016/j.entcs.2006.12.018}.

\bibitem{Selinger08}
P.~Selinger.
\newblock Idempotents in dagger categories (extended abstract).
\newblock In P.~Selinger, editor, {\em Proceedings of the 4th International
  Workshop on Quantum Programming Languages (QPL 2006)}, number 210 in Elect.
  Notes in Theor. Comp. Sci., pages 107--122. Elsevier, Amsterdam, 2008.
\newblock DOI \url{http://dx.doi.org/10.1016/j.entcs.2008.04.021}.

\end{thebibliography}


\end{document}
