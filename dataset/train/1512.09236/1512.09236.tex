\documentclass[a4, 11pt]{article}


\usepackage{fullpage}

\usepackage[utf8]{inputenc}
\usepackage[T1]{fontenc}

\usepackage{amssymb}
\usepackage{latexsym}
\usepackage{array}


\newcommand{\eps}{\epsilon}

\usepackage{graphicx}
\usepackage{algorithm}
\usepackage{algpseudocode}

\usepackage{tikz}
\usetikzlibrary{decorations.pathmorphing}
\usetikzlibrary{positioning,shapes,calc}
\usepackage{latawce}
\usepackage{subcaption}
\usepackage{authblk}


\title{A  - Approximation Algorithm  for the Maximum  Traveling Salesman Problem\thanks{Partly supported by Polish National Science Center grant UMO-2013/11/B/ST6/01748}}

\author{Szymon Dudycz\thanks{szymon.dudycz@gmail.com}, \enskip
		Jan Marcinkowski\thanks{jasiekmarc@stud.cs.uni.wroc.pl}, \enskip 
		Katarzyna Paluch\thanks{abraka@cs.uni.wroc.pl},\enskip and \enskip
		Bartosz Rybicki\thanks{rybicki.bartek@gmail.com}}

\affil{Institute of Computer Science,  University of Wroc{\l}aw}

\date{}

\newcommand{\Keywords}[1]{\par\vskip11pt\noindent
    {\small{\em Keywords\/}: #1}}

\newcommand{\dowod}{\noindent{\bf Proof.~}}
\newcommand{\koniec}{\hfill \.1ex]}

\newcommand{\<}{\langle}
\renewcommand{\>}{\rangle}
\newtheorem{fact}{Fact}




\newcommand{\Kt}{{\cal K}_3}
\newcommand{\Kd}{{\cal K}_2}
\newtheorem{abbreviation} {Abbreviation}
\newtheorem{lemma}{Lemma}
\newtheorem{theorem}{Theorem}
\newtheorem{corollary}{Corollary}
\newtheorem{definition}{Definition}
\newtheorem{observation}{Observation}
\newtheorem{claim}{Claim}
\newtheorem{invariant}{Invariant}
\def\cudzyslow{\raisebox{-1.4ex}[0pt][0pt]{''}}
\def\cudz{\cudzyslow}

\def\Sigg{\Sigma^*} \def\Sign{\Sigma^{[n]}}






\begin{document}


\maketitle
\thispagestyle{empty}
\begin{abstract}
In the maximum  traveling salesman problem (Max TSP)  we are given a complete undirected graph with nonnegative weights on the edges and we wish to compute a traveling salesman tour of maximum weight. We present a fast combinatorial   - approximation algorithm for Max TSP.
The previous best approximation for this problem was . The new algorithm is based on a novel technique of eliminating difficult subgraphs via {\em half-edges}, a new method of edge coloring and a technique of exchanging edges. A {\it half-edge} of edge  is informally speaking ``a half of  containing either  or ''.

\end{abstract}

\newpage

\section{Introduction}
The maximum  traveling salesman problem (Max TSP) is a classical variant of the
famous traveling salesman problem. In the problem we are given a complete undirected graph
 with nonnegative weights on the edges and we wish to compute a
traveling salesman tour of maximum weight. Max TSP, also informally known as the ``taxicab
ripoff problem'' is both of theoretical and practical interest.

Previous approximations of  Max
TSP have found applications in combinatorics and computational biology: the problem is useful in understanding RNA
interactions~\cite{RNA} and providing algorithms for compressing the results of
DNA sequencing~\cite{DNASEQ}. It has also been  applied to a problem of finding
a maximum weight triangle cover of the graph~\cite{HRTri} and to a combinatorial
problem called \emph{bandpass-2}~\cite{CW}, where we are supposed to find the
best permutation of rows in a boolean-valued matrix, so that the weighted sum of
structures called \emph{bandpasses} is maximised.



{\bf Previous results.}  The first approximation algorithms for Max TSP were devised by Fisher, Nemhauser and Wolsey \cite{Fish}. They showed several algorithms having approximation ratio 
and  one  having a guarantee of . In \cite{Kos} Kosaraju, Park and Stein  presented an improved algorithm having a  ratio   (\cite{BH}). This was in turn improved by Hassin and Rubinstein, who gave a - approximation (\cite{HR1}).
In the meantime  Serdyukov \cite{Ser} presented (in Russian) a simple and elegant -approximation algorithm. The algorithm
is deterministic and runs in , where  denotes the number of vertices in the graph.
Afterwards, Hassin and Rubinstein (\cite{HR}) gave a randomized algorithm having  expected approximation ratio at least   and running in , where  is an arbitrarily small constant.
The first deterministic approximation algorithm with the ratio
better than  was given in \cite{Chen} by Chen, Okamoto and  Wang. It is a -approximation and a nontrivial derandomization of the algorithm from \cite{HR} and runs in . The currently best known approximation has been given by Paluch, Mucha and Madry \cite{Paluch}  and achieves the ratio of . Its running time  is also .

{\bf Related Work.} It is known that Max-TSP  is max-SNP-hard \cite{bgww}, so there exists a constant , which is an upper bound on the approximation ratio of any algorithm for this problem. The geometric version of the problem, where all vertices are in  and the  weight of each edge is defined as the Euclidean distance of its endpoints, was considered in \cite{g_maxtsp}. The algorithm presented in this paper solves the problem exactly in polynomial time, assuming that the number  of dimensions is constant. Moreover, it is quite fast for real-life instances, in which  is small. 

Regarding the path version of Max TSP - Max-TSPP (the Maximum Traveling Salesman Path Problem), the approximation algorithms with ratios correspondingly  and  have been given in \cite{Monnot}. The first one is for the case when both endpoints of the path are specified and the other for the case when only one endpoint is given.

Another related problem is called the maximum scatter TSP (see \cite{arkin}). In it the goal is to find a TSP tour (or a path) which maximizes the weight of the minimum weight (lightest) edge selected in the solution. The problem is motivated by medical imaging and some manufacturing applications.  In general there is no constant approximation for this problem, but if  the weights of the edges obey the triangle inequality, then it is possible to give a -approximation  algorithm. The paper studies also the more general version of the maximum scatter TSP -- the max-min--neighbour TSP. The improved approximation results  for the max-min--neighbour problem have been given in \cite{Chiang}.

In the maximum latency TSP problem  we are given a complete undirected graph with vertices . Our goal is to find a Hamiltonian path starting at a fixed vertex , which maximizes the total latency of the vertices. If in a given path  the weight of the -th edge is , then the latency of the -th vertex is  and the total latency is defined as . A ratio   approximation algorithm for this problem is presented in \cite{motwani}.

{\bf Our approach and results.} We start with computing a maximum weight {\em cycle cover}   of . A cycle cover of a graph  is  a collection of cycles such  that each vertex belongs to exactly one of them. The weight of a maximum weight cycle cover  is an upper bound on , where by  we denote the weight of a maximum weight traveling salesman tour.  By computing a maximum weight perfect matching  we get another, even simpler than ,  upper bound -- on . From  and  we build a multigraph  which consists of two copies of  and one copy of , i.e.,  for each edge  of   the multigraph  contains between zero and three copies of . Thus the total weight of the edges of  is at least .  Next we would like to  {\em path--color} , that is  to color the edges of  with three colors, so that each color class
contains only vertex-disjoint paths. The paths from  the color class with maximum weight can then be patched in an arbitrary manner  into a tour of weight at least .



{\em Technique of eliminating difficult subgraphs via half-edges.} \  However, not every multigraph  can be path-3-colored. For example, a subgraph of  obtained  from a triangle    of   such that   contains one of the edges of  (such triangle is called a {\em -kite (of )}) cannot be path-3-colored as, clearly,  it is impossible to color such seven edges  with three colors and not create a monochromatic triangle.
Similarly, a subgraph of  obtained from a square   (i.e., a cycle of length four) of   such that   contains two edges connecting vertices of   (such square is called a {\em -kite (of )}) is not path-3-colorable. To find a way around this difficulty, we compute another cycle cover  {\em
improving  with respect to }, which is a cycle cover that does not contain any -kite or -kite of  and whose weight is also at least .  An important feature of  is that it may contain
{\em half-edges}. A half-edge of an edge  is, informally speaking, a half of the edge  that contains exactly one of its endpoints. Half-edges have already been introduced in \cite{PEZ}. Computing  is done via a novel reduction to a maximum weight perfect matching. It is, to some degree, similar to computing a directed cycle cover without  length two cycles in \cite{PEZ}, but for Max TSP we need much more complex gadgets. 

 From one copy of  and  we build another multigraph  with weight at least . It turns out that  can always be {\em path--colored}. The multigraph  may be non-path--colorable - if it contains at least one kite.  We notice, however, that if we remove one arbitrary edge from each kite, then  becomes path--colorable.  The edges removed from  are added to . As a result,
the modified  may stop being path--colorable. To remedy this, we in turn remove some edges from  and add them to .  In other words, we find  two disjoint sets of edges - a set  and a set , called {\em exchange sets} such that the multigraph  is path--colorable and the multigraph  is path--colorable. Since  and  have the total weight at least , by path--coloring  and path--coloring  we obtain a  - approximate solution to Max TSP.

{\em Edge coloring.} The presented algorithms for path--coloring and path--coloring are essentially based on a simple notion of a {\em safe edge}, i.e., 
an edge colored in such a way that it is guaranteed not to belong to any monochromatic cycle, used in an inductive way.  The adopted approach may appear simple and straightforward. For comparison, let us point out that the method of path--coloring the multigraph obtained from two directed cycle covers described in \cite{Svir} is rather convoluted.

Generally, the new techniques are somewhat similar to the ones used for the directed version of the problem - Max ATSP in \cite{Pal34}.
We are convinced that they will prove useful for other problems related with TSP, cycle covers or matchings.

The main result of the paper is
\begin{theorem}
There exists a -approximation algorithm for Max TSP. Its running time is .
\end{theorem}

\section{Path--coloring of }
We  compute a maximum weight cycle cover  of a given complete undirected graph  and a maximum weight perfect matching  of .
We are going to call cycles of length , i.e., consisting of  edges {\bf \em -cycles}. Also sometimes -cycles will be called {\bf \em triangles} and -cycles -- {\bf \em squares}.
The multigraph  consists of two copies of  and one copy of . We want to color each edge of  with one of three colors of   so that each color class consists of vertex-disjoint  paths. 
The {\em graph}  is a subgraph of the {\em multigraph}  that contains an edge   iff the multigraph  contains an edge between  and . The path--coloring of  can be equivalently defined as coloring each edge of (the graph)  with the number of colors equal to the number of copies contained in the multigraph . From this time on, unless stated otherwise,  denotes a graph
and not a multigraph.



We say that a colored edge  of    is {\bf \em safe}  if no matter how we color the so far uncolored edges of   is guaranteed not to belong to any monochromatic cycle of . An edge  of  is said to be {\bf \em external} if its two endpoints belong to two different cycles of . Otherwise,  is {\bf \em internal}.
We say that an edge  is incident to a cycle  if it is incident to at least one vertex of .

We prove the following useful lemma.
\begin{lemma} \label{col}
Consider a partial coloring of . Let  be any cycle of  such that for each color  there exists an edge of  incident to  that is colored . Then we can color  so that each edge of  and each edge incident to one of the edges of  is safe.
\end{lemma}

\dowod   \ \ The proposed procedure of coloring  is as follows.

If there exists an edge of  that also belongs to , we color it with all three colors of . For each uncolored edge of  incident to , we color it with an arbitrary color of .
Next, we orient the edges of  (in any of the two ways) so that  becomes a directed cycle . Let  be any  uncolored edge of  oriented from  to . Then, there exists an edge  of  incident to . If  is contained in , then we color  with any two colors of . Otherwise  is colored with some color  of .
Then we color  with the two colors belonging to .

First, no vertex of  has three incident edges colored with the same color, as for each vertex its outgoing edge is colored with different colors than an incident matching edge.  Second, as for each color  there is a matching edge incident to  colored with , there exists an edge of  that is not colored , thus  does not belong to any color class, i.e. there exists no color  such that each edge of  is colored with .  Let us consider now any edge   of  incident to some edge of  and not belonging to .  The edge  is colored with some color . 
Suppose also that  vertex   belongs to  ( may  belong to  or  may not belong to .) Let  be any other vertex of  such that some edge of   colored  is incident to it  ( may be equal to  if  is internal). 
To show that  is safe, it suffices to show that there exists no path consisting of edges of  that connects  and  and whose every edge is colored .  However, by the way we color edges of  we know that the outgoing edges of  and  are not colored with  because of the way we oriented the cycle, there is no path connecting  and  contained in  that starts and ends with incoming edge. \koniec

For each cycle  of  we define its {\bf \em degree of flexibility} denoted as  and its {\bf \em colorfulness}, denoted as . The degree of flexibility of a cycle   is the number of internal edges of  incident to  
and the colorfulness of  is the number of colors of  that are used for coloring the external edges of  incident to .

From Lemma \ref{col} we can easily derive 
\begin{lemma}\label{cola}
If a cycle  of  is such that , then we can color  so that each edge of  and each edge incident to one of the edges of  is safe.
\end{lemma}

Sometimes, even if a cycle  of  is such that  , we can color the edges of  so that each of them is safe.
For example, suppose that  is a square consisting of edges   and there are four external edges of  incident to , all colored  . Suppose also that each external edge incident to  is already safe. Then we can color  with  and ,
 with  and  and both  and  with  and . We can notice that  is guaranteed not to belong to a cycle colored  because external edges incident to  are colored  and are safe. Analogously, we can easily check that each other  edge of  is safe. However, for example, a triangle  of  that has three external edges of  incident to it, all colored with the same color of , cannot be colored in such a way that it does not contain a monochromatic cycle.

Consider a cycle  of  such that every external edge of  incident to  is colored. We say that   is {\bf \em  non-blocked} if and only if (1)     or   (2)    contains at least  vertex-disjoint edges, each of which has the property that it has exactly two incident external edges of  and the two external edges of   incident to it  are colored with the same color of  or (3)  is a square such that .

Otherwise we say that  is {\bf \em blocked}.
We can see  that a cycle  of  is  blocked  if
\begin{itemize}
\item  is a triangle and all external edges of  incident to  are colored with the same color of ,
\item  is a square with two internal edges of  incident to it ,
\item  is a cycle of even length,    and  there exist two colors  such that external edges of  incident to  are colored alternately with  and .
\end{itemize}

Among blocked cycles we distinguish kites. We say that a cycle  is a {\bf \em kite} if it is a triangle such that  and then we call it a {\bf \em -kite} or it is a square such that  - called a {\bf \em -kite}.
A cycle of  which is not a kite is called {\bf \em unproblematic}.


Now, we are ready to present the algorithm for path--coloring . \\



\begin{algorithm}
	\caption{Color }
	\label{alg:col_g1}
	\begin{algorithmic}
	  \While{ an uncolored external edge  of }
	    \State   an unproblematic uncolored cycle of  with the fewest 
	    uncolored external edges incident to 

	    \State color uncolored external edges incident to  so that no unproblematic cycle
	    of  becomes blocked and if possible, 
			
			\State so that 

	    \State color  using Lemma \ref{coluzup} and internal edges incident to it in such a way, that
	    each edge incident to  is safe
	  \EndWhile

	  \While{ -- an unproblematic, uncolored cycle of }
	    \State color  and internal edges incident to it in such a way, that
	    each edge incident to  is safe
	  \EndWhile
	\end{algorithmic}
\end{algorithm}


\begin{lemma} 
Let  be an unproblematic  cycle of  that at some step of Algorithm Color  has  the fewest uncolored external edges  incident to it. Then, it is always possible to color all uncolored external edges incident to  so that
no unproblematic cycle of  becomes blocked. Moreover, if    has at least two uncolored ext. edges incident to   then, additionally, it is always possible to do it in such a way that  .
If  has exactly one  uncolored external edge   of  incident to it, then we can color  so that   or so that  is safe.
\end{lemma}
\dowod If  has at least two uncolored external edges of  incident to it, then we can use at least two different colors for coloring the edges. Moreover if , then we can choose them in such a way that , i.e. so that for every color  at least one external edge of  incident to  is colored with . At this stage, every other uncolored cycle  of  has also at least two uncolored external edges of  incident to it. Therefore  is in danger of becoming blocked only if it has an even number of incident external edges of ,
 all of them are colored with the same two colors, say  and ,  in an alternate way and it has exactly two incident uncolored external edges ,   of .  However,  even if we would like to also use  and  for coloring the external edges of  incident to , we can do it in such a way that  does not become blocked, because, as one can easily see, one of the ways of coloring  and  with  and 
does not make  blocked.

If  has exactly one uncolored external edge  of  incident to it and  is in  danger of becoming blocked, then either  is a triangle whose two other incident external edges are colored with the same color of   or   has even length and all of its  incident external edges of   are colored with the same two colors in an alternate way. In each of these cases we have a choice  and  can color  with one of two colors so that  does not become blocked. If  is  incident to a cycle  that is also in danger of becoming blocked,
then with respect to  we can also color  with one of two colors of   so that it does not become blocked.  As the intersection of two two-element subsets of  is always nonempty, we can color , say with ,  so that no cycle of  becomes blocked. As all other external edges of  were safe, then  is also safe. \koniec

From the above lemma we get
\begin{corollary}\label{colsafe}
After all external edges are colored, each of them is incident to a cycle  of  such that   or is safe.
\end{corollary}

We say that a cycle  of  is {\bf \em  a subcycle}  of cycle  of  if it goes only through vertices that belong to .

\begin{lemma} \label{coluzup}
Let  be an unproblematic  and non-blocked cycle of  whose all incident external edges of  are already colored and safe.  Then it is always possible to color  and internal  edges incident to   in such a way that each edge incident to  is safe.
\end{lemma} 

\dowod
If  is such that   , then by Lemmas \ref{col}  and \ref{cola},  the claim holds.
Now let us first  prove that if  is not blocked and   , then it is always  possible to color the edges of  so that no color class contains all edges of  any subcycle of  .

{\bf \em Case 1:}  All  edges of  incident to  are colored with the same color, say . \\
We can  then assume that all edges of  incident to  are external. (Otherwise we would have colored internal edges  of   with a different color than .)
 must have length at least . (Otherwise it would be blocked.)  Let   denote the two colors of .
We choose two nonadjacent edges of , color one of them with  and  and the other with  and . The remaining edges of  are colored with  and .

{\bf \em Case 2:}  All  edges of  incident to  are colored with two colors, say  and . \\
We can assume that either (1)   has no incident internal edges of   or (2)  that it has exactly one incident internal edge of  and all external edges of  incident to  are colored in the same way.

Let  denote the color belonging to   and assume that  goes through vertices  in the given order.
Then let  denote a vertex of  such that edges of  incident to   and  are colored in the same way, say with , and  is colored with . Then for each   we color  edge   of  with colors belonging to 
, where  denotes the color used on an edge of  incident to . Edge  is going to be colored with  and . 

We colored the edges, so that there is no monochromatic cycle on edges of  and internal matching edges. Therefore, together with the safety of all external edges, it ensures the safety of all internal edges. \koniec





\section{A cycle cover improving  with respect to }

Since  may contain kites, we may not be able to path--color . Therefore, our next aim is to compute another cycle cover   of  such that it does not contain any cycle of  which is problematic and whose weight is an upper bound on . Since computing such  may be hard, we relax the notion of a cycle cover and allow  to contain {\bf \em half-edges}. A half-edge of the edge  is informally speaking
a half of the edge  that contains exactly one of the endpoints of . Let us also point out  here that  may contain kites which do not belong to .
To be able to  give a formal definition of such a relaxed cycle cover, we introduce a graph . We say that an edge   is {\bf \em problematic} if  and  belong to the same kite. An edge connecting vertices of a kite   is also said to be a problematic edge of . A -kite has no diagonals  and a -kite has two diagonals.    is the graph obtained from  by splitting each problematic  edge  with a vertex  into two edges 
 and , each with weight .  Each of the edges  and  of  is said to be {\bf \em a half-edge of the edge  of }.  In what follows, when we speak of an edge of a kite, we mean an edge of the original graph .

\begin{definition}\label{relst}

A {\bf \em relaxed cycle cover   improving  with respect to }  is a subset  such that
\begin{itemize}
\item[(i)]
each vertex in  has exactly two incident edges  in ;

\item[(ii)]
for each -kite   of   the number of  half-edges  of  the edges of  contained in    is even  and   not greater than four;
\item[(iii)]
for each -kite   of  the number of half-edges  of  the edges  or diagonals of   contained in  is even  and not greater than six.
\end{itemize}
\end{definition}

To compute a relaxed cycle cover    improving   with respect to   we construct the following  graph .
The set of vertices   is a superset of the set of verices  of .
For each problematic edge   of  we add two vertices  to  and edges  to . For each problematic edge  which is not a diagonal of a -kite we add also an edge . The edge  has weight  in 
and each of  the edges     has weight  .
Each of the vertices    is called {\bf \em a splitting vertex of the edge }.
For each edge  of  which is not problematic we add an edge  to  with weight . 

\begin{figure}[h!]
    
    \begin{subfigure}{.48\textwidth}
  \centering
  \begin{tikzpicture}[scale=2.5]
    \tikzset{every node/.style={draw,fill=black,circle,scale=1,inner sep=0pt,outer sep=2pt}}
    \tikzstyle{pom}=[draw=black!60,thin]
    \node[vert,label={right:}] (a) at (1,0) {};
    \node[vert,label={45:}] (b) at (2,1.73) {};
    \node[vert,label={135:}] (c) at (0,1.73) {};

    \node[label={right:\scriptsize{{}}}] (aab) at (1.33,.57) {};
    \node[label={right:\scriptsize{{}}}] (abb) at (1.67, 1.15) {};

    \node[label={85:\scriptsize{{}}}] (bbc) at (1.33,1.73) {};
    \node[label={85:\scriptsize{{}}}] (bcc) at (.67,1.73) {};

    \node[label={left:\scriptsize{{}}}] (acc) at (.33, 1.15) {};
    \node[label={left:\scriptsize{{}}}] (aac) at (.67, .57) {};


    \draw[sol] (a) -- (aab);
    \draw[sol] (abb) -- (b);
    \draw[sol] (b) -- (bbc);
    \draw[sol] (bcc) -- (c);
    \draw[sol] (c) -- (acc);
    \draw[sol] (aac) -- (a);

    \draw[pom] (aab) -- (abb);
    \draw[pom] (bbc) -- (bcc);
    \draw[pom] (acc) -- (aac);

    \node[label={110:\textcolor{black!60}{\small{}}}] (Pa) at (1.01, .8) {};
    \draw[pom] (aab) to [out=180, in=0] (Pa);
    \draw[pom] (aac) to [out=0, in=180] (Pa);
    \draw[pom] (bbc) to [out=220, in=30] (Pa);

    \node[label={right:\textcolor{black!60}{\small{}}}] (Pc) at (.8,1.3) {};
    \draw[pom] (bcc) to [out=240, in=60] (Pc);
    \draw[pom] (acc) to [out=60, in=240] (Pc);
    \draw[pom] (abb) to [out=120, in=300] (Pc);
  \end{tikzpicture}
  \caption{, ,
  }
\end{subfigure}
~
\begin{subfigure}{.48\textwidth}
  \centering
  \begin{tikzpicture}[scale=2.5]
    \tikzset{every node/.style={draw,fill=black,circle,scale=1,inner sep=0pt,outer sep=2pt}}
    \tikzstyle{pom}=[draw=black!60,thin]
    \node[vert,label={left:}]   (a) at (-1, -1) {};
    \node[vert,label={right:}]   (b) at (1, -1) {};
    \node[vert,label={right:}]   (c) at (1, 1) {};
    \node[vert,label={left:}]   (d) at (-1, 1) {};

    \node[label={275:\scriptsize{{}}}] (aab) at (-.33,-1) {};
    \node[label={275:\scriptsize{{}}}] (abb) at (.33, -1) {};

    \node[label={right:\scriptsize{{}}}] (bbc) at (1, -.33) {};
    \node[label={right:\scriptsize{{}}}] (bcc) at (1, .33) {};

    \node[label={85:\scriptsize{{}}}] (ccd) at (.33, 1) {};
    \node[label={85:\scriptsize{{}}}] (cdd) at (-.33, 1) {};

    \node[label={left:\scriptsize{{}}}] (add) at (-1, .33) {};
    \node[label={left:\scriptsize{{}}}] (aad) at (-1, -.33) {};

    \node[label={90:\scriptsize{{}}}] (aac) at (-.7, -.7) {};
    \node[label={below:\scriptsize{{}}}] (acc) at (.7, .7) {};
    \node[label={above:\scriptsize{{}}}] (bbd) at (.7, -.7) {};
    \node[label={below:\scriptsize{{}}}] (bdd) at (-.7, .7) {};

    \draw[sol] (a) -- (aab);
    \draw[sol] (abb) -- (b);
    \draw[sol] (b) -- (bbc);
    \draw[sol] (bcc) -- (c);
    \draw[sol] (c) -- (ccd);
    \draw[sol] (cdd) -- (d);
    \draw[sol] (d) -- (add);
    \draw[sol] (aad) -- (a);

    \draw[pom] (aab) -- (abb);
    \draw[pom] (bbc) -- (bcc);
    \draw[pom] (ccd) -- (cdd);
    \draw[pom] (add) -- (aad);

    \draw[sol] (a) -- (aac);
    \draw[sol] (b) -- (bbd);
    \draw[sol] (c) -- (acc);
    \draw[sol] (d) -- (bdd);

    \node[label={right:{\textcolor{black!60}{\small{}}}}] (pu) at (-.3, -.3) {};
    \draw[pom] (aab) to [out=135, in=300] (pu);
    \draw[pom] (aac) to [out=0, in=205] (pu);
    \draw[pom] (aad) to [out=345, in=135] (pu);

    \node[label={left:{\textcolor{black!60}{\small{}}}}] (pv) at (.3, -.3) {};
    \draw[pom] (abb) to [out=45, in=240] (pv);
    \draw[pom] (bbd) to [out=180, in=335] (pv);
    \draw[pom] (bbc) to [out=195, in=45] (pv);

    \node[label={left:{\textcolor{black!60}{\small{}}}}] (pw) at (.3, .3) {};
    \draw[pom] (bcc) to [out=165, in=330] (pw);
    \draw[pom] (acc) to [out=180, in=25] (pw);
    \draw[pom] (ccd) to [out=300, in=135] (pw);

    \node[label={right:{\textcolor{black!60}{\small{}}}}] (pz) at (-.3, .3) {};
    \draw[pom] (add) to [out=15, in=210] (pz);
    \draw[pom] (bdd) to [out=0, in=155] (pz);
    \draw[pom] (cdd) to [out=240, in=45] (pz);

    \node[label={right:{\textcolor{black!60}{\small{}}}}] (q) at(0,0) {};
    \draw[pom] (pu) to [out=75, in=255] (q);
    \draw[pom] (pv) to [out=105, in=285] (q);
    \draw[pom] (pw) to [out=255, in=75] (q);
    \draw[pom] (pz) to [out=285, in=105] (q);
  \end{tikzpicture}
  \caption{, , }
\end{subfigure}

    \caption{Gadgets for 3-kites~\textbf{(a)} and
    4-kites~\textbf{(b)} of~ in graph~. Half-edges corresponding to the
    original edges are thickened, the auxiliary edges are thin. Original
    vertices (thick dot) are connected with all the other original vertices of
    graph~. The auxiliary vertices have no connections outside of the gadget.
    The figures are subtitled with the specifications of  values for
    different vertices. For a vertex  with , the resulting
    b-matching will contain exactly  edges ending in .}
    \label{fig:maxtsp_gadgets}
\end{figure}

Next we build so-called gadgets.
For each -kite  on vertices  we add two vertices  to . Let's assume that  is incident to external edge of .  Vertex  is connected to the splitting vertices of edges of  that are neighbors of , i.e. to vertices  and to vertex .  Vertex  is connected to every other splitting vertex of , i.e. . All edges incident to verices 
have weight  in .

For each -kite of   on vertices     we add five vertices   to .  Vertex  is connected to the splitting vertices of edges of  that are neighbors of , i.e. to vertices .  Vertices  are connected analogously. Vertex  is connected to vertices . All edges incident to verices 
have weight .


We will reduce the problem of computing a relaxed cycle cover improving  with respect to  to the problem of computing a perfect -matching of the graph .
We define the function  in the following way. For each vertex  we set . For each splitting vertex  of some problematic edge we set .
For all  vertices  and , where  denotes a -kite of  we have .
For all  vertices  and , where  denotes a -kite of  and  one of its vertices we have .


\begin{theorem}
Any perfect -matching of  yields  a relaxed cycle cover    improving  with respect to .
A maximum weight perfect - matching of  yields a relaxed cycle cover  improving   with respect to  such that .
\end{theorem}
\dowod  First we  show that any perfect -matching of  yields  a relaxed cycle cover  improving  with respect to .
Let  be any perfect -matching of .  defines   as follows. A half-edge   is contained in  iff edge   of  is contained in . A non-problematic edge  is contained in  iff  is contained in .
Since  for any vertex  of , we can see that  the property  of Definition \ref{relst} is satisfied.

Consider now an arbitrary -kite  of . There are 3 problematic edges of  and thus six half-edges. Suppose that  is on vertices . We can notice that a half-edge   is not contained in  
iff a splitting vertex  is connected in  to one of the vertices  or to a splitting vertex . Since  and  are connected to one splitting vertex each, at most 4 half-edges of the problematic edges of  are contained in . If
 a splitting vertex  is connected in  to  , then both half-edges of the edge  are excluded from .
This shows that the number of half-edges of problematic edges of  contained in  is even.

Consider now an arbitrary problematic square  of . There are six problematic edges of  and thus twelve half-edges of these edges. Suppose that  is on vertices . We can notice that a half-edge   is not contained in  
iff a splitting vertex  is not matched to  in . Thus a half-edge  does not occur in  iff 
 a splitting vertex  is connected in  to one of the vertices  or to a splitting vertex .
Since  is connected to two of the vertices  and , exactly six splitting vertices of the problematic edges of  are connected in  to vertices . This means  that at least six half-edges of the problematic edges of  are not contained in .  If
 a splitting vertex  is connected in  to  , then both half-edges of the edge  are excluded from .
This shows that the number of half-edges of problematic edges of  contained in  is even.

In order to show that  it suffices to prove the following lemma.
\begin{lemma}
\label{lem:cycle_cover_optimality}
Every cycle cover not containing kites of  corresponds to some  perfect b-matching of .
\end{lemma} 
The proof is in Section \ref{sec:c2_opt_proof} \koniec


\section{Exchange sets  and path--coloring of }

We construct a multigraph  from one copy of  a relaxed cycle cover  and one copy of a maximum weight perfect matching .
Since  may contain half-edges and we want  to  contain only edges of , for each half-edge of edge  contained in , we will either include the whole edge  in  or not include it at all. While doing so we have to ensure that the total weight of the constructed multigraph  is at least . 

 The main idea behind deciding which half-edges are extended to full edges and included in  is that we compute two sets  and   such that for each kite in  half of the edges containing half-edges belongs to  and the other half to . (Note that by Lemma \ref{} each kite in  contains an even number of half-edges in .) Let  denote the set consisting of whole edges of  contained in . This way . Next, let  denote the one of the sets  and  with maximum  weight. Then  is defined as a multiset consisting of edges of , edges of  and edges of . We obtain

\begin{fact}
The total weight of the constructed multigraph  is at least .
\end{fact} 
\dowod
The weight of  is at least . The weight of  is at least . Since , we obtain
that . \koniec



Since  contains at least one kite,  is non-path--colorable. We can notice, however, that if we remove one edge from each kite from the multigraph , then the obtained multigraph is path--colorable. 



If we manage to construct a set  with one edge per each kite such that additionally the multigraph  is path--colorable, then we have a -approximation of Max TSP. Since computing such  may be difficult, we allow, in turn, certain edges of  to be removed from  and added to . Thus, roughly,  our goal is to compute such disjoint sets  that:

\begin{enumerate}

\item  contains at least  one edge of each kite;
\item for each kite ,  contains exactly one edge not contained in ; 
\item the multigraph  is path--colorable;
\item the multigraph  is path--colorable.
\end{enumerate} 


Let  and  be two sets of edges that satisfy properties 1. and 2. of the above. Then  the set of edges   can be partitioned into {\bf \em cycles and paths of }, where  denotes the resulting multigraph .  The partition of  into cycles and paths is carried out in such a way that two incident edges of  belonging to a common path or cycle of , belong also to a common path or cycle of  (and ). Also, the partition is maximal, i.e., we cannot
add any edge  of  to any path  of  so that  is also a path or cycle of .

We say that  is a {\bf \em double edge} of , or that  is {\bf \em double}, if the multigraph  contains two copies of . In any path--coloring of  every double edge must have both colors of  assigned to it. 

We observe that in order for  to be path--colorable,  we have to guarantee that  there does not exist   a cycle
  of  of odd length  that  has  incident double edges. Since every two consecutive edges of  are incident to some double edge, they must be assigned different colors of  and because the length of  is odd, this is clearly impossible. The way to avoid 
this is to choose one edge of each such potential cycle and add it to . 

We say that a path  of  beginning at  and ending at    is {\bf \em amenable} if (i) neither  nor  has degree  in 
or (ii)  has degree ,  has degree smaller than  and  ends with a double edge, the last-but-one edge of  is a double edge or the last-but-one and the last-but-three vertices in  are matched in .

It turns out that  that does not contain odd cycles described above and whose every path is amenable is path--colorable - we show it in
 Section \ref{path2}. To facilitate the construction of , whose every path is amenable and to ensure that  and  have certain other useful properties we create  two opposite orientations of :  and . In each of these orientations  contains directed cycles and paths and each kite has the same number of incoming and outgoing edges. (This can be achieved by pairing the endpoints of paths ending at the same kite and combining them. For example, if   contains half-edges  and  of a certain -kite  and edges , then in the orientation  in which  is directed from   to  the edge  must be directed from  to .)  Apart from whole edges  contains also half-edges. Let  denote the set of edges of  such that  contains exactly
one half-edge of each of these edges. We partition  into two sets  so that for each kite  half of the edges of 
is contained in  and the other half in .
With each of the orientations  we associate
one of the sets . Thus,  we assume that  contains , with the edges of  being oriented in a consistent way with the edges of  under orientation , and  contains ,  with its edges being oriented accordingly.
The exact details of the construction of  and  are given in the proof of Lemma \ref{F12}.



Depending on which of the sets  has bigger weight,  we either choose the orientation  or .  Hence, from now on, we assume that the edges of  are directed.

\begin{lemma}\label{F12}
It is possible to compute sets     such that they and  the resulting  satisfy:
\begin{enumerate}
\item ;
\item ;
\item for each kite , (i) the set  contains exactly one edge of  and the set  contains zero edges of  or (ii) (it can happen only for -kites) the set  contains exactly two edges of  and the set  contains one edge of ;
\item for each kite  the set  contains exactly one outgoing edge of ;
\item for each kite  and each vertex  of  the number of edges of  incident to  is at most one greater than the number of edges of  incident to ;
\item there exists no cycle of  of odd length  that has  double edges incident to it;
\item each path of  is amenable.

\end{enumerate}
\end{lemma}

The property 1. of this lemma guarantees that  does not contain more than two copies of any edge. We  show in Appendix \ref{path2} that properties 6. and 7. are essentially sufficient for the multigraph  to be path--colorable. Properties  4. and 5. will be helpful
in finding a path--coloring of . Property 5. ensures that no vertex  has six incident edges in . 

The proof of this lemma is given in Section \ref{dowodf12}.

The path--coloring of  is quite similar to the path--coloring of . It is described in Section \ref{path2}.

\section{Completing the path-coloring of }
After the construction and path--coloring of  we are presented with the task of extending the partial path--coloring of  to the complete path--coloring of . In particular, we have to color the  edges
of kites, edges of  that have been added during the construction of  and external edges of  incident to -kites, called {\bf \em tails}. A tail incident to a -kite  is said to be a {\bf \em tail of }.

Let us now describe the set of uncolored edges of  in more detail. Each one of them is incident to some kite and has either (1) two endpoints belonging to the same kite  (an internal edge of ), or (2) one of its endpoints belongs to some kite  and the other does not belong to any kite (an external edge of ) or (3) its endpoints belong to two different kites  and  (an external edge both of  and ). Let  denote a -kite. Then by Lemma \ref{F12} exactly one edge  of  belongs to , no edge of  belongs to  and there also exists  exactly one edge in   that is an outgoing edge of , i.e.,  is an external edge of  and is directed from an endpoint belonging to  in .
 may also contain up to three incoming edges of , each one incident to a different vertex of .  Any incoming edge of  is also an outgoing edge of some other kite. A tail of  is also uncolored in . 

Each uncolored edge  of  has a requirement  denoting the number of colors of  that must be assigned to it. 
 Then for any edge  contained in some -kite,  if  ,  if  and  otherwise. Thus, for each -kite  we have to color exactly six of its edges in the {\em multigraph} .

Let  denote a -kite. Then by Lemma \ref{F12} either (1) exactly one edge  of  belongs to  and no edge of  belongs to  or (2) exactly two edges of  belong to  and one edge of  belongs to . There also exists  exactly one edge   that is an outgoing edge of .  may also contain up to four incoming edges of , each one incident to a different vertex of . For any edge  belonging to some -kite,  if  or ,  if  and  otherwise.  Thus, for each -kite  we have to color exactly nine of its edges in the {\em multigraph} .


Each uncolored external edge  in  has requirement . 
Let  denote the subgraph of  comprising all edges  with positive requirement.




We need to assign colors of  to edges of  (or, in other words, color edges of  with colors of ) in such a way that each color class in the whole graph   forms a collection of disjoint paths. The coloring of edges of  is an extension of the already existing partial path--coloring of . Therefore, for some edges there exist restrictions on colors of  that can be assigned to them. Consider any vertex  that does not belong to any kite and that has one or two incident edges in . If  has an incident tail in , then it has exactly two incident edges in    that are colored with two different pairs of colors of  (while path--coloring  we can easily guarantee that two consecutive edges of  incident to  such that an edge of  incident to  is also incident to a -kite are colored with two different pairs of colors). Let these pairs of colors be  and . Hence any edge of  incident to  may be colored only with  or  - we associate with   a  two-element subset .  Moreover, if   has two incident edges in  and we color one of them with , then the other one {\em must} be colored with .  If  does not have an incident tail in , then it has at most one incident edge in  and exactly five edges in the {\em multigraph}  as well as in the multigraph  . In this case there exists exactly one color   of  that can be assigned to an edge of  incident to  and we associate a one-element subset  with .


Let  be a -kite. Then a vertex of  incident to its tail  is called a {\bf \em foot vertex (of )}. If  is incident to the foot vertex of , then  is said to be {\bf \em vertical}; otherwise it is {\bf \em horizontal}.
Two -kites  and  having a common tail are called {\bf \em twins}. Also, each one of them is called a twin and  is said to be a {\bf \em brother of }. A -kite that is not a twin is said to be {\bf \em non-twin}.

Some of the edges contained in  are directed.  The directions of edges of  satisfy:

\begin{enumerate}
\item each internal edge is undirected (i.e., each edge contained in a kite);
\item the direction of each edge of  is the same as in ; the properties of edges of  are described in Lemma \ref{F12} in properties (3), (4) and (5);

\item a tail of  two twins is undirected; otherwise, a tail of a -kite  is an incoming edge of . (It may happen that a tail  of some -kite belongs also to  and . Then  contains two copies of , each one with the requirement  and the copy corresponding to a tail is treated as a tail and the other copy is treated as an external directed edge.)
\end{enumerate} 


From graph  we build a graph  by shrinking each kite to a single vertex. Each vertex of  that corresponds to a kite in  is called, respectively a {\bf \em t-vertex} (if it is a -kite) or an {\bf \em s-vertex} (if it is a -kite); each remaining vertex is called an {\bf \em o-vertex}.  In any coloring of  or , we say that an o-vertex  is {\bf \em respected} if any edge incident to   is assigned a color belonging to  and if there are two edges incident to , then they have different colors assigned to them.

To {\bf \em pre-color} a  directed cycle  or path  of  means to
color each of its edges with a color of  so that each o-vertex of  is respected.  To {\bf \em color a kite } means to  color each edge  of  with  colors of . 

We are going to color the edges of  in portions - by considering directed cycles and paths in . For each such cycle or path we will color its edges as well as some of the kites corresponding to its vertices. To be able to talk more precisely about these operations we introduce below the notions of {\bf \em processing} a directed cycle or path  in  and {\bf \em step-processing} a vertex  on .
Processing a directed cycle or path  in  consists in step-processing each of its vertices on .


\begin{definition}
Let  be a directed cycle or path in  and
  a vertex on   that has an outgoing edge that belongs to .


To {\em step-process}  (or in case  corresponds to a kite , to step-process ) {\em on } means: 
\begin{itemize}
\item if an outgoing edge of  is uncolored - to color it,
\item if  has  an incoming edge contained in  - to color it,
\item if  corresponds to a kite  -  to color the kite  unless  is a horizontal twin, whose brother has not been step-processed (on any directed cycle or path in ),
\item if  corresponds to a non-twin -kite  -  to color the tail of ,
\item if   corresponds to a twin -kite , whose brother  has already been step-processed - to color the common tail of  and  and in case  has not already been colored, to color ,
\item to carry out the above so that each color class forms a collection of vertex-disjoint paths in  and so that each o-vertex in  is respected.
\end{itemize}
\end{definition}


To {\bf \em process} a directed  path  in  that goes through vertices  and directed from  to  means to step-process each of the vertices  in turn, starting from .
When we process such a path, then we start the step-processing  by coloring an outgoing edge of  incident also to .   We then continue step-processing   and afterwards, proceed to ste-processing , then  and so on. If  and  of  correspond to twins  and  such that  is horizontal and considered before  on , then while step-processing  we only color the edges incident to  and leave  and its tail uncolored. When we come to , we color the incoming edge of  incident to  and both twins  and  and their common tail.
In an analogous way we define the processing of a directed cycle  in  - we start from  step-processing any vertex on  and continue with step-processing subsequent vertices along .

Let us notice that if a vertex  corresponding to a kite  has not been step-processed, then  is uncolored and either (1) every external 
edge of  is also uncolored or (2) an outgoing edge of  is colored because we have just step-processed  on some directed path or cycle  such that  contains an edge ; apart from this every other external edge of  is uncolored. Also, a given vertex   has exactly one outgoing edge in  but may belong to more than one directed path in  or it may belong to a directed cycle and some directed path(s) in .  However, in Algorithm 2  the first time we encounter  while processing a directed cycle or path, we will step-process it, because each considered directed path is maximal under inclusion. If we encounter  again
while processing a different cycle or path, we will just color some of its incoming edges (and possibly a tail and so on) but will not step-process  again.


\begin{algorithm} \label{AH}
	\caption{Color }
	\label{alg:col_h}
	\begin{algorithmic}
	  \While{ a directed cycle in }
	    \State process it and remove its edges from 
	  \EndWhile
		\While{ a directed maximal path in }
	    \State process it and remove its edges from 
	  \EndWhile
		
	\end{algorithmic}
\end{algorithm}

\vspace{0.5cm}




In Section \ref{AHcor} we  prove that every  directed cycle or path can be processed.





\section{Summary}

\noindent \fbox{
\begin{minipage}[t]{\textwidth}
\vspace{0.5cm}
{\bf \em \hspace{0.5cm} Algorithm MaxTSP}
\vspace{0.5cm}
\begin{enumerate}
\item Compute a cycle cover  of  of maximum weight and a perfect matching  of  of maximum weight.
\item Let  denote a multigraph obtained from two copies of  and one copy of  - its weight is at least .
Path--color  with colors of  leaving kites and edges of  incident to kites uncolored.
\item Compute a maximum weight relaxed cycle cover  improving  with respect to .
\item  Let  denote a multigraph obtained from one copy of  and one copy of  - its weight is at least . Compute the sets of edges  such that the multigraph  is path--colorable
and  the multigraph  is path--colorable (i.e.  are as in Lemma \ref{F12}).
\item Path--color  with colors of .
\item Extend the partial path--coloring of  to the complete path--coloring of .
\item Choose the color class of maximum weight - its weight is at least  and complete the paths of this class into a traveling salesman tour in an arbitrary way.
\end{enumerate}
\vspace{0.5cm}
\end{minipage}
}


\vspace{1cm}

The presented algorithm works for graphs with an even number of vertices. If the number of vertices of a given graph is odd, then we can guess one edge, shrink it and compute the remaining part of the solution in the graph with even vertices. 

\section{Correctness of Algorithm 2} \label{AHcor}

We  are going to prove that every  directed cycle or path in  can be processed.
First we give several auxiliary lemmas.


\begin{lemma}\label{preparz}
Let  be a  directed cycle in  of even length,   whose every other vertex is an o-vertex. Then we are able to pre-color   in such a way that its every two consecutive edges get assigned different colors.
\end{lemma}
\dowod

First, let us notice that  an o-vertex  may have two incident edges in  only if one of them is a tail of some -kite. Thus, every vertex  of  that is not an o-vertex must correspond to a -kite and be a t-vertex.

If the length of  is two, then  contains exactly one o-vertex . We then assign one  color of  to one edge of  and the other color of  to the other edge of  and are done. 

Suppose now that  has length greater than two. Let  be any o-vertex of  and  the edges of  incident
to . We assign one of the colors  of  to  and the other  to . Assume that  is an incoming edge of ,  is an outgoing edge of   and  are the subsequent edges of . The edges  and  are incident
 to another o-vertex  of . We will show now that whatever the set , we are always able to asign colors to  and  in such a way that  {\em does not} get assigned  - the color already assigned to . If  contains  and some other color , then we assign  to  and  to . If  does not contain , then it contains  and  and we assign  to  and  to . This way (i)  edges  and   get assigned  different colors
and (ii)  gets assigned a color different from . 

If  has length , then we notice that the edges  and   of  get assigned different colors as well and we are done.

If  has length greater than , then we consider the next pairs of edges and continue in the manner described above.  More precisely, when we consider the pair of edges  and  incident to some o-vertex ,
we know that the invariant that  and  have different colors assigned is satisfied. Our goal is to color  and  in such a way that (i)  gets assigned a color different from the color assigned to  and (ii)  gets assigned a color different from . From the way we have analysed coloring  and , we know that it can always be done.
 
\koniec 

\begin{corollary} \label{cyklparz}
Let  be a  directed cycle in  of even length,   whose every other vertex is an o-vertex. Then we are able to process  .
\end{corollary}
\dowod First, let us notice that every -vertex of  corresponds to a non-twin -kite, because the tail of each such kite is contained in  and thus is directed. 

While pre-coloring  whenever two edges  of  adjacent to the same kite  get colored, we also color .
While coloring  we only have to see to it that no vertex of  gets three incident edges of the same color in  and to that  does not contain a monochromatic cycle i.e. a -cycle. We show how to color  in Figure \ref{fig:lemma_compl} and in Figure \ref{fighoriz}.
 Let us notice that after pre-coloring 
 and all -kites corresponding to t-vertices on , no color class contains a cycle - this is because every edge  of  is incident to a t-vertex corresponding to a -kite  and the only external edges incident to  in the whole graph  are  and some other edge  of . We know, however, that every two consecutive edges of  are colored differently. Hence  is colored differently from . Thus, neither  nor  can belong to a monochromatic cycle, which means that  in this way we process .
\koniec


Suppose that the tail  of  is uncolored. Then  is said to be {\bf \em flexible} if there exist such two colors  that  can be colored both with   and , by which we mean that if we color the tail of  with  (or correspondingly ), then the foot of  does not have more than three incident edges colored with  (resp. ). The flexibility of a -kite  is useful when  is a vertical twin that is step-processed before its twin . Then while step-processing  we color  but leave its tail uncolored and later later while step-processing  we have a greater 'flexibility' in coloring  and its tail.

\begin{lemma}\label{compl}
Let  be any uncolored vertical -kite and  two external edges incident to  colored with, respectively,  and . Let  be the foot vertex  and  the tail of . Additionally,  and  are not both incident to  and  . Then it is possible to color the edges of  so that  becomes flexible and so that  can be colored with . 
\end{lemma}
\dowod For all possible triangles we will show how to color the edges for . These colorings are presented in Figure \ref{fig:lemma_compl}.

\begin{figure}[h!]
\centering
	\begin{subfigure}{0.3\textwidth}
	\centering
		\begin{tikzpicture}
			\trikite{}{}
			\tlExta{}{}
			\trExta{}{}
			\tLabel{}
			\lLabel{}
			\rLabel{}
		\end{tikzpicture}
		\subcaption{ can be colored with all colors in }
	\end{subfigure}
	\quad
	\begin{subfigure}{0.3\textwidth}
	\centering
		\begin{tikzpicture}
			\trikite{}{}
			\tlExta{}{}
			\tlExtb{}{}
			\tLabel{}
			\lLabel{}
			\rLabel{}
		\end{tikzpicture}
		\subcaption{By Lemma \ref{F12} edge in  must be incident to  and .  can be colored with all colors in }
	\end{subfigure}
	\quad
	\begin{subfigure}{.3\textwidth}
		\centering
		\begin{tikzpicture}
			\trikite{}{}
			\tlExta{}{}
			\blExtb{}{}
			\tLabel{}
			\lLabel{}
			\rLabel{}
		\end{tikzpicture}
		\subcaption{Both edges of  incident to  can be colored with , depending on which one is not in .  can be colored with  and }
	\end{subfigure}
	\caption{Vertical triangles with  and  not incident to }
	\label{fig:lemma_compl}
\end{figure}

\koniec

\begin{lemma} \label{wlasciwosci}
The computed sets  satisfy:
\begin{enumerate}
\item No foot of a -kite has two incident edges of .
\item If a -kite  has four incident edges of , then it is vertical.
\end{enumerate}
\end{lemma}
The proof follows from the proof of Lemma \ref{F12}.


\begin{lemma} \label{ver1}
Let  be a vertical -kite, whose tail  is uncolored and that has been colored at some point as in Lemma \ref{compl}. Then, however, we color any further  external edges of  incident to  apart from its tail,  always stays flexible.
\end{lemma}
\dowod The lemma follows from the fact that the foot of  has not two incident edges of . \koniec
 
Let  be an uncolored -kite , whose tail  is also uncolored. Then we say that  is {\bf \em weakly flexible} if  there exist two colors  such that  can be colored in at least two ways and
in one of these colorings  can be colored with  and in the other with , i.e., after coloring  with  or , the foot of  has at most two incident edges colored with respectively  or .
We say that an uncolored twin  is {\bf \em versatile} if every two colored edges of  incident to  have different colors assigned to them. The weak flexibility of a -kite  is useful when  is a horizontal twin that is step-processed before its twin .  While step-processing  on some directed cycle or path  we do not color it or its tail but only the incident edges of  and later while step-processing  we color both  and  and their common tail.


\begin{lemma} \label{horiz}
Every uncolored versatile horizontal -kite is weakly flexible.
\end{lemma}
\dowod Let  be any triangle on vertices  as in Lemma \ref{horiz} and let  three external edges incident to  colored with, respectively, ,  and . Let  be the foot vertex and  the tail of . Let us assume that  and  are not incident to . Then we can color  with  and . For each of these colors we have to show how to color edges of . As these cases are symmetric, we assume that  is colored with . Let us assume that  is incident to . Then we color  and  with .

\begin{figure}[h!] \label{fighoriz}
	\centering
	\begin{tikzpicture}
		\trikite{}{}
		\tlExta{}{}

		\rLabel{}
		\tLabel{}
	\end{tikzpicture}

\end{figure}

As  is horizontal, we still have to color  with  colors, and the other edges with one color. If there is an edge, say , incident to  we color  and  with . If there is an  edge incident to  other than , say , we color  and  with . If there are both of these edges, than it is correct coloring. Otherwise there is an edge incident to , say , and we can color  and either  or  with , so we can always color . \koniec
 



\begin{lemma} Every  directed cycle or path can be processed in such a way that at all times every uncolored horizontal twin is versatile.
\end{lemma} 
\dowod
Let us consider a directed path  going through vertices  and directed from  to . 
We can notice that since cycles are processed before paths, each vertex of  is distinct.
We observe also that  is either an o-vertex or corresponds to a kite that has already been step-processed - otherwise we could extend , because then the outgoing edge of  would be uncolored.  Vertex , on the other hand, is either an o-vertex or corresponds to an uncolored (and not step-processed) kite. We begin by  coloring the arc  with any color of  that is available. 
Let us note that some color of  is always available because of the following.  If  is an o-vertex, then it has exactly six incident edges in the {\em multigraph}  - apart from five edges in the {\em multigraph} , it  has an additional incoming edge that is an outgoing edge of some kite. If  corresponds to a kite, then  Lemma \ref{F12} Property 5 guarantees that any vertex in  belonging to a kite has degree at most six.

 Also, if  corresponds to an uncolored -kite  that has already been step-processed, then we color  with such a color  of  that no external edge of  is colored with .  Such a color  always exists because only horizontal -kites can be left uncolored and they have at most three incident edges of . Thus we can guarantee that  remains versatile.

We step-process subsequent vertices on  according to the rules listed below. 
 

Let  be an outgoing edge of  colored with  and  an uncolored incoming edge of . Depending on whether  is an o-, t- or s-vertex and other conditions we proceed
as follows:
\begin{enumerate}
\item . Then  must be a t-vertex corresponding to a -kite  and  is an outgoing edge of an o-vertex . We color ,  and an incoming edge  of . If , then we  color  with  and  with . Otherwise . Then we color  with  and  with  or the other way around.

\item  is a t-vertex corresponding to a non-twin -kite  and  is the tail of  incident to an o-vertex . We color  and .  If , then we  color  with  and  with . Otherwise . Then we color  with  and  with  or the other way around.

\item  is a t-vertex corresponding to a twin -kite , whose brother  has already been colored and  is the tail of  . Then by Lemmas \ref{compl} and \ref{ver1}, there exist two colors of , such that if we look only at , then  can be colored with either of them. Let  denote the set consisting of these two colors.  Now we proceed almost identically as in the case above. If , then we  color  with  and  with . Otherwise . Then we color  with  and  with  or the other way around. We color . Note  that  each colored external edge of  is colored with a different color. This cannot be said about  - it may happen that the tail of  is colored with the same color as some other external edge  of , but we do not have to worry about edge  ending in a monochromatic cycle because then the tail of  and  would also have to belong to such cycle. 




\item  is a t-vertex corresponding to a twin -kite , whose brother  has not been step-processed.  We color  with  or .
If  is vertical, then we color . Otherwise we leave  uncolored.

\item  is a t-vertex corresponding to a twin -kite , whose brother  has  been step-processed but is uncolored. Since  has been step-processed, but is uncolored, it is horizontal.  By Lemma \ref{horiz}  is weakly flexible - therefore there exist two colors  that can be used for coloring the tail of . If , then we color the tail of  and  with the color belonging to   and  with the remaining color of  - note that this way each colored external edge of  is colored with a different color.  If , then we color the tail with  and  with .  We also color both  and .


\item  is an s-vertex. Then we color  with any color of  different from . We also color .

 
\end{enumerate}

We argue that by proceeding as above, we do not create a monochromatic cycle in  and thus process . This is so, because every external edge colored with  while processing  is contained in some  path  consisting of edges colored with  that ends at a vertex of  corresponding to a kite  such that  has only one incident external edge in  colored with .


Let us now turn our attention to directed cycles. Let  be a  directed cycle of . We can assume that  is not as in Lemma \ref{preparz}, because we have already dealt with such cycles. Thus  contains two subsequent vertices   such that neither  nor  is an o-vertex and  contains an edge  directed from  to . If possible we choose  that is an s-vertex or corresponds to a -kite , whose tail also belongs to . If such  does not exist then we choose  that corresponds to a -kite that is either non-twin or whose brother has already been step-processed. For now, we assume that this is the case.

 We start by coloring an incoming edge of . If  corresponds to a -kite , whose tail  does not belong to , then  can be colored with some two colors . In this case we color the incoming edge of  with .


We continue processing  according to the rules described above until we reach the vertex . If applying the rules also to  would result in the arc  being colored with a different color than an incoming edge of , we apply the rules to  and are done. Otherwise  must correspond to a -kite  which is either non-twin or that is a twin whose brother has already been colored. Also, assume  that the incoming edge of  is colored with . It follows  that the tail of  can be colored with  or , both different from , and that the outgoing edge of  is colored with  or . Otherwise we would be able to color  with a color different from . Suppose that the outgoing edge of  is colored with . In this case we color the tail of  with  and color  with  - if it is not incident to the tail of  and with  otherwise. By Lemma \ref{wlasciwosci} property 1, it cannot happen that both the outgoing and incoming edge of  is incident to the foot of .

We are left with the case when each vertex of  corresponds to a twin -kite whose brother also occurs on . We leave this case to the reader. \koniec


\section{Path--coloring} \label{path2}
The partition of  into cycles and paths is carried out in such a way that two edges of  belonging to a common path or cycle of , belong also to a common path or cycle of . Also, the partition is maximal, i.e., we cannot
add any edge  of  to any path  of  so that  is also a path or cycle of . We may assume that each path and cycle of  is directed - the orientations of edges are consistent with those in . 



A {\bf \em surrounding} of a cycle  of , denoted as , contains  every edge of  and  every edge of  incident to .
Let  be a path of  directed from  to . If  has degree  in , then an edge  of  incident to  is said to be a {\bf \em border} of .
The {\bf \em surrounding} of  , denoted as , contains  every edge of   and every edge of  incident to .  

We construct a directed graph  such that each path of  is represented by some vertex of  and  contains an edge  iff  has a border and the border of  is incident to some vertex of .
Thus each vertex of  has at most one outgoing edge. Below we describe the algorithm for path--coloring the graph . In it we first color the cycles of  and their surroundings.  The order of coloring the paths of 
is dictated by the structure of graph : we begin by coloring the paths of  that form cycles in ; next at each step we color an uncolored path, whose outdegree in  is zero.


The presence of borders complicates path--coloring in two aspects:
\begin{enumerate}
\item Suppose that edges  and  belong to some path of  and that  is incident to a double edge  different from  and .
Since  has to be colored with two colors of , edges  and  must be assigned different colors of . Therefore while path--coloring  we will preserve the following invariant:
\begin{invariant}\label{invdouble}
Every two edges  of  such that their common endpoint  is incident to a double edge  different from  and  are assigned different colors of . 
\end{invariant}

\item Each border  of a path  of  is colored while coloring the path  and not before. In particular, if  is double and is incident to a path or cycle  such that  is colored before , then while coloring  we assign only one color to . The second one is assigned while coloring .
If  is double we may also think of it as of two edges - one being a border and the other an edge of the matching .

Because of this we modify the meaning of a safe edge in this section as follows. We say that a colored edge  is safe if no matter how we color the so far uncolored edges except for any uncolored borders,  is guaranteed not to belong to any monochromatic cycle. In particular, it means that if we want to prove that a newly colored border  is safe we have to explicitly show that it does not belong to any monochromatic cycle - without taking use of the fact that previously colored 
edges are safe.

\end{enumerate}



\begin{algorithm}
  \caption{Color }
  \label{alg:col_gprim2}
  \begin{algorithmic}
	  \State During the whole execution ensure that Invariant \ref{invdouble} is satisfied. 
    \While{ -- an uncolored cycle of }
      \For{}
        \State color  in such a way, that it is safe
      \EndFor
    \EndWhile 
    
    	\While{ -- a directed cycle of }
      \For{ such that  is a vertex on }
        \State color each  in such a way, that it is safe
      \EndFor
      \State remove each vertex of  together with incident edges from 
    \EndWhile
    
    \While{ -- an uncolored path of  such that }
      \For{ such that }
        \State color   in such a way, that it is safe
      \EndFor
      \State remove  together with incident edges from 
    \EndWhile
		
		
    \end{algorithmic}
\end{algorithm}




\begin{lemma} \label{cykl45}
Let  be an uncolored cycle  of  considered at some step of Algorithm Color . Then it is possible to color each edge belonging to  in such a way that it is safe.
\end{lemma}
\dowod The procedure of coloring the edges of  is similar to that described in the proofs of Lemmas \ref{col} and \ref{coluzup}. 
We orient the edges of  so that  becomes directed. 

{\em Case 1:} (i) For each color  there exists an edge of  incident to one of the edges of  that is colored   or (ii) there exists an uncolored edge of  incident to one of the edges of . 
First we color every uncolored non-double edge  of  incident to  so that case (i) holds. Next we color each double edge incident to .
Let  be a double edge such that  belongs to . Then, necessarily   belongs to some path of  and since we color cycles of  before coloring paths of ,  is uncolored. We start with such a double edge  that the predecessor  of  on the cycle  has no incident double edge. The existence of such double edge is guaranteed by Lemma \ref{F12}. Let  and  be two edges of  incident to  and let  be an edge of  incident to  and  the color of  assigned to .  To preserve Invariant \ref{invdouble} we  have to color the edges  with different colors of . To make it possible we color  with a color  belonging to , i.e., for the time being we color  only with one color instead of two.  
We proceed with each subsequent double edge incident to  in the same way, i.e., we color such edges in order of their occurrence along . 

Further we color all edges of . Let  be an edge of  oriented from  to  and let  be an edge of  incident to 
 the color of  assigned to . Then we color  with a color  belonging to . We can notice that each so far colored edge is safe. Suppose that  is colored with . Then we additionally assign  to . 


{\em Case 2:} All edges of  incident to  are colored with the same color . \\
We color any chosen one edge of  with  and the remaining ones with .  \koniec


\begin{lemma}\label{path45}
Let  be an  uncolored path  of  considered at some step of Algorithm Color  such that . Then it is possible to color each edge belonging to  in such a way that it is safe.
\end{lemma}

\dowod 
Generally  we proceed in a very similar way as in Lemma \ref{cykl45}. The path  is already oriented.  First we color each edge  of  incident to  with one color of  in order of their occurrence along . If a given edge  of  incident to  is double, then  we color it with one color only and with the one different from that assigned to an edge of  incident to  which proceeds  on . Next we color each edge  of  directed from  to , which is not a border of  with a color different from that assigned to an edge  of  incident to . 

We must also color the border  of , if  has one. 

If  is double, then it must have got assigned one color of  before we started coloring  - that is because , which means that  got colored while coloring the path or cycle of  incident to . It may also happen that the border  of  is incident to some ''internal'' vertex of  but then we have also already assigned one color of  to it. If  is already colored with ,  then we additionally assign  to it. 
The safety of  follows from the following. The edge  proceeding  on  is colored with one color  of . From the way we color edges of , we notice that  is contained in a monochromatic path  colored with , whose one endpoint lies on . In other words we claim that  has a ''dead end''. We can observe  that a part of  starting with  is contained in  and does not leave . It follows from the fact that  each edge  of  is colored with a color different from the one assigned to the edge of  incident to . This means that  is safe, because we have already colored every edge of  and every edge of  incident to  (except possibly for some borders), hence  is safe.

If the border  of  is not double, then we still have to color it. Suppose that  is the endpoint of . Then three edges of the multigraph  incident to  have already been colored. This means that there is only one color of  that can be used for coloring . We must also ensure that after coloring , it does not belong to any monochromatic cycle. Since  is amenable,  is either proceeded by a double edge on  or an edge  of  incident to  is also incident to a last-but-three vertex of . In the first case, the safety of  follows from the fact that an edge of  proceeding a double edge proceeding  is safe. (The argument is the same as above.) In the second case 
we leave the edge  uncolored till this point. Once we know that we are forced to color , with say , we color  with the other color of  and we also color accordingly the two edges proceeding  and are done.
\koniec

\begin{lemma}
\label{lemma:coloring_cycle_of_paths}
Let  be a directed cycle of  considered at some step of Algorithm Color . Then it is possible to color each edge belonging to the surrounding of each path of  occurring on  in such a way that it is safe.
\end{lemma}
\dowod
Suppose that the cycle  goes through vertices  in this order. Let  denote the border of path  of  for each .  We start by coloring the path  and its surrounding in the manner described in the proof of Lemma \ref{path45}. If the border  is not double, then we leave it uncolored. Next we color each of the paths  and their surroundings together with their borders, also in the way described in the proof of Lemma \ref{path45}. Next we have to check two possibilities of dealing with the path . First we  color the path  together with  its border and surrounding in the same manner
as the remaining paths  and if the border  is uncolored, because it is not double, we color it with the only possible color of . It may happen, however, that by doing so we create a monochromatic cycle  that is formed by the part of  between 
 and , the part of  between  and  and so on until the part of  between  and . If this is the case, then we leave the part of  between  and  colored as it is and uncolor the remaining part of  . If the border 
is not double, then we change its color to the opposite one. If the border  is double, then we change the color of the edge proceeding it on  to the opposite one. Next we change the orientation of the uncolored part of  as follows. The endpoints of  are  and some vertex  and originally  is oriented from  to . Now we change the orientation of the part  of  between  and  so that it is directed from  to .  Let  denote the edge of  incident to .  Since  has degree  in the multigraph , there exists only one color  of  that can be used for coloring . The rest of  is colored in the standard way. We only have to show that the edge  is safe, as every other edge considered in this lemma is safe by reasoning analogous to that used in two previous lemmas. The edge  is safe because it is colored with same color  that every edge of  but one is colored with. Also,  is the only edge incident to  but not lying on  that is colored with . The example of this algorithm is presented on Figure \ref{fig:path_recoloring}.

\begin{figure}
  \centering
\begin{subfigure}[t]{0.47\textwidth}
	\centering
	\begin{tikzpicture}
		\node[vert,label={below right:}] (a) at (0,0) {};
		\node[vert] (b) at ([shift={(a)}] 15:1.25) {};
		\node[vert,label={below right:}] (d) at ([shift={(b)}] 45:1.25) {};
		\node[vert,label={above:}] (e) at ([shift={(d)}] 75:1.25) {};
		\node[vert] (f) at ([shift={(e)}] 195:1.25) {};
		\node[vert,label={right:}] (g) at ([shift={(f)}] 225:1.25) {};

		\node[vert,label={above:}] (i) at ([shift={(a)}] 180:1.25) {};
		\node[vert] (j) at ([shift={(a)}] 270:1.25) {};
		\node[vert] (k) at ([shift={(i)}] 270:1.25) {};
		\node[vert,label={above right:}] (l) at ([shift={(e)}] 0:1.25) {};

		\node[vert] (m) at ([shift={(e)}] 315:1.25) {};
		\node[vert] (n) at ([shift={(l)}] 325:1.25) {};

		\draw[sol,->, bend right=15] (a) to (b);
		\draw[sol,->, bend right=0] (b) to (d);
		\draw[sol,->, bend right=15] (d) to (e);
		\draw[sol,->, bend right=15] (e) to (f);
		\draw[sol,->, bend right=0] (f) to (g);
		\draw[sol,->, bend right=15] (g) to (a);
		\draw[sol,->] (i) to (a);
		\draw[sol,->] (l) to (e);

		\draw[snake] (j) to node[midway, label={left:\small{}}] {} (a);
		\draw[snake] (k) to node[midway, label={left:\small{}}] {} (i);
		\draw[snake] (m) to node[midway, label={right:\small{}}] {} (e);
		\draw[snake] (n) to node[midway, label={right:\small{}}] {} (l);

		\node[draw=none,inner sep=0] (bt) at ([shift={(b)}] 135:0.15) {};
		\node[draw=none,inner sep=0] (dt) at ([shift={(d)}] 135:0.15) {};
		\draw[snake] (bt) to (dt);


		\node[draw=none,inner sep=0] (ft) at ([shift={(f)}] 135:0.15) {};
		\node[draw=none,inner sep=0] (gt) at ([shift={(g)}] 135:0.15) {};
		\draw[snake] (gt) to (ft);


		\node[draw=none] (tmp) at (0,-2) {}; 

	\end{tikzpicture}
	\subcaption{Paths of  before coloring. There are two paths: the first one from  to  and the second one from  to .}
\end{subfigure}
\quad
\begin{subfigure}[t]{0.47\textwidth}
	\centering
	\begin{tikzpicture}
		\node[vert] (a) at (0,0) {};
		\node[vert] (b) at ([shift={(a)}] 15:1.25) {};
		\node[vert] (d) at ([shift={(b)}] 45:1.25) {};
		\node[vert] (e) at ([shift={(d)}] 75:1.25) {};
		\node[vert] (f) at ([shift={(e)}] 195:1.25) {};
		\node[vert] (g) at ([shift={(f)}] 225:1.25) {};

		\node[vert] (i) at ([shift={(a)}] 180:1.25) {};
		\node[vert] (j) at ([shift={(a)}] 270:1.25) {};
		\node[vert] (k) at ([shift={(i)}] 270:1.25) {};
		\node[vert] (l) at ([shift={(e)}] 0:1.25) {};

		\node[vert] (m) at ([shift={(e)}] 315:1.25) {};
		\node[vert] (n) at ([shift={(l)}] 325:1.25) {};

		\draw[sol,->, bend right=15] (a) to node[midway, label={below:\small{}}] {} (b);
		\draw[sol,->, bend right=0] (b) to node[midway, label={below right:\small{}}] {} (d);
		\draw[sol,->, bend right=15] (d) to node[midway, label={left:\small{}}] {} (e);
		\draw[sol,->, bend right=15] (e) to node[midway, label={above:\small{}}] {} (f);
		\draw[sol,->, bend right=0] (f) to node[midway, label={above left:\small{}}] {} (g);
		\draw[sol,->, bend right=15] (g) to node[midway, label={above left:\small{}}] {} (a);
		\draw[sol,->] (i) to node[midway, label={above:\small{}}] {} (a);
		\draw[sol,->] (l) to node[midway, label={above:\small{}}] {} (e);

		\draw[snake] (j) to node[midway, label={left:\small{}}] {} (a);
		\draw[snake] (k) to node[midway, label={left:\small{}}] {} (i);
		\draw[snake] (m) to node[midway, label={right:\small{}}] {} (e);
		\draw[snake] (n) to node[midway, label={right:\small{}}] {} (l);

		\node[draw=none,inner sep=0] (bt) at ([shift={(b)}] 135:0.15) {};
		\node[draw=none,inner sep=0] (dt) at ([shift={(d)}] 135:0.15) {};
		\draw[snake] (bt) to (dt);


		\node[draw=none,inner sep=0] (ft) at ([shift={(f)}] 135:0.15) {};
		\node[draw=none,inner sep=0] (gt) at ([shift={(g)}] 135:0.15) {};
		\draw[snake] (gt) to (ft);

		\node[draw=none] (tmp) at (0,-2) {};

	\end{tikzpicture}
	\subcaption{Preliminary coloring of paths of . There is a monochromatic cycle in color }
\end{subfigure}
\\
\begin{subfigure}[t]{0.47\textwidth}
	\centering
	\begin{tikzpicture}
		\node[vert] (a) at (0,0) {};
		\node[vert] (b) at ([shift={(a)}] 15:1.25) {};
		\node[vert] (d) at ([shift={(b)}] 45:1.25) {};
		\node[vert] (e) at ([shift={(d)}] 75:1.25) {};
		\node[vert] (f) at ([shift={(e)}] 195:1.25) {};
		\node[vert] (g) at ([shift={(f)}] 225:1.25) {};

		\node[vert] (i) at ([shift={(a)}] 180:1.25) {};
		\node[vert] (j) at ([shift={(a)}] 270:1.25) {};
		\node[vert] (k) at ([shift={(i)}] 270:1.25) {};
		\node[vert] (l) at ([shift={(e)}] 0:1.25) {};

		\node[vert] (m) at ([shift={(e)}] 315:1.25) {};
		\node[vert] (n) at ([shift={(l)}] 325:1.25) {};

		\draw[sol,->, bend right=15] (a) to node[midway, label={below:\small{}}] {} (b);
		\draw[sol,->, bend right=0] (b) to node[midway, label={below right:\small{}}] {} (d);
		\draw[sol,->, bend right=15] (d) to node[midway, label={left:\small{}}] {} (e);
		\draw[sol,->, bend right=15] (e) to node[midway, label={above:\small{}}] {} (f);
		\draw[sol,->, bend right=0] (f) to node[midway, label={above left:\small{}}] {} (g);
		\draw[sol,->, bend right=15] (g) to node[midway, label={above left:\small{}}] {} (a);
		\draw[sol,->] (i) to node[midway, label={above:\small{}}] {} (a);
		\draw[sol,->] (e) to node[midway, label={above:\small{}}] {} (l);

		\draw[snake] (j) to node[midway, label={left:\small{}}] {} (a);
		\draw[snake] (k) to node[midway, label={left:\small{}}] {} (i);
		\draw[snake] (m) to node[midway, label={right:\small{}}] {} (e);
		\draw[snake] (n) to node[midway, label={right:\small{}}] {} (l);

		\node[draw=none,inner sep=0] (bt) at ([shift={(b)}] 135:0.15) {};
		\node[draw=none,inner sep=0] (dt) at ([shift={(d)}] 135:0.15) {};
		\draw[snake] (bt) to (dt);


		\node[draw=none,inner sep=0] (ft) at ([shift={(f)}] 135:0.15) {};
		\node[draw=none,inner sep=0] (gt) at ([shift={(g)}] 135:0.15) {};
		\draw[snake] (gt) to (ft);

	\end{tikzpicture}
	\subcaption{We recolor edge  and path from  to . All edges are now safe}
\end{subfigure}
  \caption{Example of algorithm described in Lemma \ref{lemma:coloring_cycle_of_paths}}
  \label{fig:path_recoloring}
\end{figure}
\koniec

\section{The proof of Lemma \ref{F12}} \label{dowodf12}
First we want to guarantee that property 6. is satisfied. Let us say that an edge  is a {\bf \em d-edge} if it belongs to  and some kite. Let  contain every cycle of  of odd length  that has  different incident d-edges. Let  denote the set of all d-edges.
We build a bipartite graph  such that there exists an edge in  between a cycle  of  and  edge  of  iff  is incident to . Furthermore for each 4-kite  incident to at most three cycles in  we merge vertices corresponding to d-edges in s into one vertex. Let us notice that the degree of each d-edge of  in  is at most  and the degree of each cycle  of  is at least . We compute a matching  of size  in the graph . By Hall's Theorem such a matching always exists. Then for each cycle  and matched to it d-edge  we will either (i) add an outgoing edge of  incident to  to  or (ii) ensure that  is not a double edge. 

We begin with the proof for the case when there are only -kites. 

Let  be any -kite on vertices  such that  is a d-edge of  and  and . 


We begin with the case when  has three incoming and three outgoing edges of  incident to it. We add  to . To  we add any outgoing edge of  belonging to  (it can be an edge required by ).  If at some later point  gets three incident incoming edges of , then we remove  from  and replace it with that one of the edges  that is incident to an outgoing edge of  and orient it so that it is directed to . We do it so that Lemma \ref{wlasnosci} Property (2) is satisfied.

Whenever  does not belong to , we add it to  and an outgoing edge of  incident to  to .

 We consider now the cases when  has two incoming and two outgoing edges
of  incident to it. In  Figure \ref{fig:triangles_2out2in} we show how to assign edges of all possible -kites with two incoming and two outgoing edges to  and .
\begin{figure}[h!]
	\centering
\begin{subfigure}[t]{0.47\textwidth}
	\centering
	\begin{tikzpicture}
		\trikite{}{}
		\trExta{}{}
		\tlExta{}{}
		\tInternal
		\lLabel{}
		\rLabel{}
		\blExta{}{}
		\blExtb{}{}
		\tLabel{}
	\end{tikzpicture}
	\subcaption{We add  to  and  to . In  we orient  so that it is directed from a common endpoint with .}
\end{subfigure}
\quad
\begin{subfigure}[t]{.47\textwidth}
	\centering
	\begin{tikzpicture}
		\trikite{}{}
		\trExta{}{}
		\trExtb{}{}
		\tlExta{}{}
		\lLabel{}
		\rLabel{}
		\blExtb{}{}
		\ltHalfEdge
		\rbHalfEdge
		\tLabel{}
	\end{tikzpicture}
	\subcaption{We add  to  and  to . If  is in  we orient it from  to  and make  outgoing. If  is in  we orient it from  to  and make  outgoing}
\end{subfigure}
\\
\begin{subfigure}[t]{0.47\textwidth}
	\centering
	\begin{tikzpicture}
		\trikite{}{}
		\trExta{}{}
		\tlExta{}{}
		\lLabel{}
		\rLabel{}
		\blExta{}{}
		\blExtb{}{}
		\trHalfEdge
		\ltHalfEdge
		\tLabel{}
	\end{tikzpicture}
	\subcaption{If  is in  we make  outgoing. In this case we add  to  and  to  and orient  from  to . If  is in  we make  outgoing, add  to  and  to  and we orient  from  to }
\end{subfigure}
\quad
\begin{subfigure}[t]{0.47\textwidth}
	\centering
	\begin{tikzpicture}
		\trikite{}{}
		\trExta{}{}
		\trExtb{}{}
		\blExta{}{}
		\lLabel{}
		\rLabel{}
		\blExtb{}{}
		\tlHalfEdge
		\ltHalfEdge
		\tLabel{}
	\end{tikzpicture}
	\subcaption{If  is in  we add  to  and  to . If  is in  we add  to  and  to }
\end{subfigure}
\\
\begin{subfigure}[t]{0.47\textwidth}
	\centering
	\begin{tikzpicture}
		\trikite{}{}
		\trExta{}{}
		\tlExta{}{}
		\lLabel{}
		\rLabel{}
		\tlExtb{}{}
		\blExtb{}{}
		\trHalfEdge
		\lbHalfEdge
		\tLabel{}
	\end{tikzpicture}
	\subcaption{If  is in  we make  outgoing, add  to  and  to . If  is in  we make  outgoing, add  to  and  to }
\end{subfigure}
\quad
\begin{subfigure}[t]{0.47\textwidth}
	\centering
	\begin{tikzpicture}
		\trikite{}{}
		\trExta{}{}
		\tlExta{}{}
		\lLabel{}
		\rLabel{}
		\tlExtb{}{}
		\blExtb{}{}
		\lInternal
		\tLabel{}
	\end{tikzpicture}
	\subcaption{We add  to  and  to }
\end{subfigure}
	
	
	
	
	\caption{Assigning edges of -kites with two incoming and two outgoing edges}
	\label{fig:triangles_2out2in}
\end{figure}

In case of triangles with one incoming and one outgoing edge there are already two edges of  in . Therefore we add remaining edge to  and incident edge (we can guarantee that it is outgoing when constructing ) to .

To finish the proof we must consider all cases for -kites.
We say that an edge  is a {\bf \em side edge} if it belongs to 4-kite, but not to . Let  be any -kite on vertices  such that  and  are d-edges and  and  are side edges. Let  be the set of those edges in , whose both endpoints are in  (so  contains all incoming and outgoing edges incident to ). First we consider the cases when  has one incoming and one outgoing edge of  incident to it (by our construction of  these edges must be incident to different vertices of ). If these edges are not incident to the same side edge, we add outgoing edge to  and the side edge incident to it to . Otherwise let's assume that they are incident to , and that edge incident to  is outgoing in , whereas edge incident to  is outgoing in .
\begin{enumerate}
	\item  and  are incident to at most  half-edges or  edge of  and  half-edge - we divide half-edges into  and  so that degree of  in  is 3 and degree of  in  is 3. Then we add  to  and outgoing edge to .
	\item  is incident to  edge of  and  half-edges and  is not in  - then half-edge  is in  so  is not incident to any edge in . We divide half-edges into  and  so that  is not in  and degree of  in  is 3. If outgoing edge is incident to  we add  to , and otherwise we add  to . In both cases we add outgoing edge to .
	\item  is in  - we divide half-edges so that  is not in  and  is not in . We add d-edge incident to outgoing edge to  and outgoing edge to .
\end{enumerate}

Now let's consider the cases when two vertices incident to the same d-edge, say , are incident to one incoming and one outgoing edge each and the other two vertices aren't incident to any incoming or outgoing edges. If  and  are incident to two half-edges each then two half-edges incident to  go to , and the other two half-edges go to . If  and  are incident to one half-edge each then  must contain  and we divide half-edges into  and  arbitrarily. There are three cases depending on which edges incident to  are in  (cases when in  there are half-edges incident to  are symmetric):
\begin{enumerate}
	\item  and  are in  - we add outgoing edge incident to  and  to  and add  and  to .
	\item  and  are in  - we add outgoing edge incident to  to  and add  to .
	\item  and  are in  - we add outgoing edge incident to  to . If incoming edge incident to  is also in  we add  to . Otherwise we add  to 
\end{enumerate}

The next cases are similar to the previous one, but now vertices incident to one incoming and one outgoing edge each are incident to the same side edge, say . We divide half-edges same as before, so two half-edges incident to  go to , and the other two go to . Now the cases are:
\begin{enumerate}
	\item  and  are in  and  was matched in  with  - we add outgoing edge incident to  and  to . If incoming edge incident to  is in  we add  and  to . Otherwise we add  and  to .
	\item  and  are in  and  wasn't matched in  with  - we add outgoing edge incident to  and  to  and we add  and  to .
	\item  and  are in  - we add outgoing edge incident to  and  to . If incoming edge incident to  is in  we add  and  to . Otherwise we add  and  to .
	\item  and  are in  and  was matched in  with  - we add outgoing edge incident to  to  and  to .
	\item  and  are in  and  wasn't matched in  with  - we add outgoing edge incident to  to . If incoming edge incident to  is in  we add  to . Otherwise we add  to .
\end{enumerate}

Now there are three cases in which there is a vertex in , say , incident to two edges in , two vertices incident to one edge in  each and a vertex incident to no edge in . The first case is when  isn't incident to any edge in . Let's assume that edge incident to  is outgoing in . We divide half-edges into  and  so that (i) in  there is an edge incident to  and edge incident to  and, similarly, in  there is an edge incident to  and edge incident to  (ii) no two half-edges incident to  are in the same set (this condition can be satisfied because in  there are at most two half-edges incident to ). Now we consider all subcases of which edges are in :
\begin{enumerate}
	\item  and  are in  - we add outgoing edge incident to  to  and  to .
	\item  and  are in  - we add outgoing edge incident to  and  to  and add  and  to .
	\item  and  are in  - we add outgoing edge incident to  to . If incoming edge incident to  is also in  we add  to . Otherwise we add  to .
	\item  and  are in  - we add outgoing edge incident to  to  and  to .
	\item  and  are in  - there is no edge incident to , so edge incident to  is outgoing. We add outgoing edge incident to  to  and  to .
	\item  and  are in  - we add outgoing edge incident to  to  and  to .
	\item  and  are in  - we add outgoing edge incident to  and  to  and add  and  to .
	\item  and  are in  - we add outgoing edge incident to  and  to  and add  and  to .
	\item  and  are in  -  is incident to neither  nor , so edge incident to  is outgoing. We add outgoing edge incident to  to . If incoming edge incident to  is also in  we add  to . Otherwise we add  to .
	\item  and  are in  - we add outgoing edge incident to  to  and  to .
\end{enumerate}

In the second case  isn't incident to any edge in . We divide half-edges same as in the previous case, so we guarantee that (i) in  there is an edge incident to  and edge incident to  and, in  there is an edge incident to  and edge incident to . Condition (ii) remains the same. Now the subcases are as follows:
\begin{enumerate}
	\item  and  are in  - we add outgoing edge incident to  to  and  to .
	\item  and  are in  - there is no edge incident to , so edge incident to  is outgoing. We add outgoing edge incident to  to . If incoming edge incident to  is also in  we add  to . Otherwise we add  to .
	\item  and  are in  - we add outgoing edge incident to  and  to  and add  and  to .
	\item  and  are in  - we add outgoing edge incident to  to . If incoming edge incident to  is also in  we add  to , and otherwise we add  to . In this case at least one of  is not double edge, so cycle going through  and  satisfies condition 6 from lemma.
	\item  and  are in  - we add outgoing edge incident to  and  to  and add  and  to .
	\item  and  are in  - there is no edge incident to , so edge incident to  is outgoing. We add outgoing edge incident to  to . If incoming edge incident to  is also in  we add  to . Otherwise we add  to .
	\item  and  are in  - we add outgoing edge incident to  to . If incoming edge incident to  is also in  we add  to . Otherwise we add  to .
	\item  and  are in  - there is no edge incident to , so edge incident to  is outgoing. We add outgoing edge incident to  to  and  to .
	\item  and  are in  -  is incident to neither  nor , so edge incident to  is outgoing. We add outgoing edge incident to  to . If incoming edge incident to  is also in  we add  to . Otherwise we add  to .
	\item  and  are in  - there is no edge incident to , so edge incident to  is outgoing. We add outgoing edge incident to  to . If incoming edge incident to  is also in  we add  to . Otherwise we add  to .
\end{enumerate}

In the third case  isn't incident to any edge in . Similarly as before we divide half-edges to guarantee that (i) in  there is an edge incident to  and edge incident to  and, in  there is an edge incident to  and edge incident to . Once again condition (ii) remains the same. Now the subcases are as follows:
\begin{enumerate}
	\item  and  are in  - we add outgoing edge incident to  to  and  to . Condition 6 from lemma is satisfied, because  is not a double edge.
	\item  and  are in  - we add outgoing edge incident to  to . If incoming edge incident to  is also in  we add  to . Otherwise we add  to .
	\item  and  are in  - we add outgoing edge incident to  and  to . If incoming edge incident to  is also in  we add  and  to . Otherwise we add  and  to .
	\item  and  are in  -  is incident to neither  nor , so edge incident to  is outgoing. We add  to  and  to .
	\item  and either  or  are in  - we add outgoing edge incident to  to  and  to .
	\item  and  are in  - we add outgoing edge incident to  to . If incoming edge incident to  is also in  we add  to . Otherwise we add  to .
	\item  and  are in  - we add outgoing edge incident to  and  to . If incoming edge incident to  is also in  we add  and  to . Otherwise we add  and  to .
	\item  and  are in  -  is incident to neither  nor , so edge incident to  is outgoing. We add outgoing edge incident to  to  and  to .
	\item  and  are in  -  is incident to neither  nor , so edge incident to  is outgoing. We add outgoing edge incident to  and  to  and add  and  to .
\end{enumerate}

The final case when  is incident to two incoming and two outgoing edges of  is when each vertex of  is incident to one edge of . First suppose that  is in . Then we can assume that in  there is half-edge incident to  (or there is  in ), because other cases are symmetric:
\begin{enumerate}
	\item  is in  - we add outgoing edge incident to  or  to  and  to .
	\item  is in  - we add outgoing edge incident to  or  and  to  and  and  to .
	\item  is in  - we add outgoing edge incident to  or  to  and d-edge adjacent to added outgoing edge to .
\end{enumerate}
If  is in  we assume that in  there is an edge incident to  and that edge incident to  is outgoing. If edge incident to  is outgoing we add it to  and add  to . If edge incident to  is outgoing then there are three cases:
\begin{enumerate}
	\item  is in  - we add outgoing edge incident to  and  to  and add  and  to .
	\item  is in  - we add outgoing edge incident to  to  and add  to .
	\item  is in  - we add outgoing edge incident to  to  and add  to .
\end{enumerate} 
Now suppose that  is in . We assume that in  there is an edge incident to  and that edge incident to  is outgoing:
\begin{enumerate}
	\item  or  is in  - we add outgoing edge incident to  to  and  to .
	\item  is in  - we add outgoing edge incident  to  and  to .
\end{enumerate}

Finally suppose that there are no whole edges inside c, so there are four half-edges. Into  belong half-edges adjacent to outgoing edges in  and into  those adjacent to outgoing edges in . Now we have to consider all possible edges in :
\begin{enumerate}
	\item  and one other edge is in  - we act the same as in case with side edge.
	\item  and  are in  - if edge incident to  is outgoing we add it to . Otherwise edge incident to  is outgoing and we add it to . In both cases we add  to .
	\item  and  are in  - we add any outgoing edge to  and adjacent side edge to .
	\item  and  are in  - we add any outgoing edge to  and adjacent d-edge to .
\end{enumerate}

Now let's consider the case when  has three incoming and three outgoing edges of  incident to it and there is a vertex, say  which is not incident to any incoming or outgoing edge. Then  is incident to two half-edges, on of which is in . If in  there is half-edge , we add outgoing edge incident to  to  and add  to . If incoming edge incident to  is also in  we add  to , so that property 5 from lemma is satisfied. If in  there is either  or , we add outgoing edge incident to  to  and  to .

In all other cases when  has three incoming and three outgoing edges of  incident to it, there are at most two half-edges, each incident to different vertex. In these cases we divide half-edges into  and  in such way, that to  belongs half-edge incident to outgoing edge in . Now we have to consider all cases to which vertices incoming and outgoing vertices are incident:
\begin{enumerate}
	\item Vertices incident to two edges of  are incident to the same d-edge, say  - let's assume that edge incident to  is outgoing (and therefore either half-edge incident to  is in  or  is in ). Then we add outgoing edge incident to  to  and either add  to  if  is in  or add  to  otherwise.
	\item Vertices incident to two edges of  are incident to the same side edge say  - let's assume that edge incident to  is outgoing (and therefore either half-edge incident to  is in  or  is in ). Then if  is in  we add outgoing edge incident to  to  and  to . If  is in  we add outgoing edge incident to  to  and  to . Finally if  is in  we add outgoing edge incident to  to  and  to . If incoming edge incident to  is also in  we add  to , so that property 5 from lemma is satisfied
	\item None of the above cases - let's assume that  and  are incident to two edges in  and edge incident to  is outgoing (and therefore either half-edge incident to  is in  or  is in ). Then if  is in  we add outgoing edge incident to  to  and  to . If  is in  we add outgoing edge incident to  to  and  to . Finally if  is in  there are two subcases:
		\begin{enumerate}
			\item In matching  cycle incident to vertex  is matched to d-edge  - then we add outgoing edge incident to  to  and  to 
			\item Otherwise we add outgoing edge incident to  to . If also incoming edge incident to  is in  we add  to , and if it isn't in  we add  to .
		\end{enumerate}
\end{enumerate}
 
In the case when  is incident to four incoming and four outgoing edges of  we add outgoing edge incident to  or  to  (depending on which one of cycles incident to these vertices was matched to  in ; if none of them we choose arbitrarily) and add  to .

\koniec

\section{Proof of Lemma \ref{lem:cycle_cover_optimality}}
\label{sec:c2_opt_proof}

Let's now see, that the cycle cover we have found using our gadgets is indeed
  what had been promised --- the maximum weight cycle cover (in which we agree
  to having paths ending with half-edges) not containing kites from . To
  prove that we will show, that no such cycle cover of  has been blocked by
  our gadgets and demands, so for every proper cycle cover of , it can be
  translated into a b-matching in the modified graph. Let's start off with
  triangles.

  \begin{lemma}
    \label{lem:maxtsp_g2compute_trikite}
    Let  be a 3-kite in the graph . Let  be a cycle
    cover of  not containing  (as one of the cycles). There exists a selection
    of edges in the gadget  corresponding to , that is
    compliant with the cycle cover  and every node  in
     has exactly  adjacent edges in it. Its total weight
    will be equal to the weight of .
  \end{lemma}
  \dowod
    Since the cycle cover  doesn't contain  as one of its cycles, it will
    have at least two edges connecting the nodes of this triangle with other
    vertices in the graph (that are \emph{external} with regard to ). These
    edges are replicated in the gadget-modified graph, so there is no doubt,
    they can be selected into the b-matching. We will now present, how to handle
    the edges of the triangle  and the gadget . We will
    consider different interactions between  and .
\begin{itemize}
    \item If  (no edge of the kite is used
    in the cycle cover), then the demands of vertices ,  and  are
    fulfilled by the external edges. Additionally, we select the edges  (middle edge on the right side of the gadget),  (middle on the left side),  and
    .

    \item  If  (the cycle cover contains
    one side of the triangle), the b-matching obviously contains  and
     --- the halves of the edge . We also select the middle
    edges of two other sides of the triangle, namely  and
    . We satisfy the demands of  and  by connecting them with
     and  respectively.

    \item  Finally, if  (the cycle
    cover contains two sides of the triangle), we select the corresponding
    half-edges , ,  and . The nodes
     and  are connected with  and .
		
\end{itemize}		
  \koniec

  In turn, for the 4-kite it will turn out, that our gadgets not only block
  selecting a length-4 cycle into the b-matching, but also prevent it from
  containing a length-3 cycle built on three vertices of the 4-kite.
  \begin{lemma}
    \label{lem:maxtsp_g2compute_nobreak_sqkite}
    Let  be a 4-kite in . Let  be a cycle cover of  not containing any length-4 or length-3 cycle built on the vertices of  as one of its cycles. There exists a selection of edges, that is compliant with the cycle cover , such that every vertex  has exactly  adjacent edges in the selection (so the selection forms a b-matching). The weight of the b-matching is equal to that of .
  \end{lemma}
  \dowod
    Similarly to the proof of Lemma \ref{lem:maxtsp_g2compute_trikite}, we
    need to look into all the possible interactions of the cycle cover
     with the edges of  (together with its diagonals). For every
    such option, we will show, how to expand it into a compliant b-matching. The
    analysis is presented in the
    Figure \ref{fig:maxtsp_g2compute_nobreak_sqkite}
  \koniec

  \begin{figure}[h]
    \centering
    
		
		\begin{center}
\begin{subfigure}{.32\textwidth}
  \begin{tikzpicture}
    \tikzset{every node/.style={draw,fill=black,circle,scale=.5,inner sep=0pt,outer sep=2pt}}
    \tikzstyle{pom}=[draw=black!60,thin]
    \node[vert]   (a) at (-1, -1) {};
    \node[vert]   (b) at (1, -1) {};
    \node[vert]   (c) at (1, 1) {};
    \node[vert]   (d) at (-1, 1) {};

    \node (aab) at (-.33,-1) {};
    \node (abb) at (.33, -1) {};

    \node (bbc) at (1, -.33) {};
    \node (bcc) at (1, .33) {};

    \node (ccd) at (.33, 1) {};
    \node (cdd) at (-.33, 1) {};

    \node (add) at (-1, .33) {};
    \node (aad) at (-1, -.33) {};

    \node (aac) at (-.7, -.7) {};
    \node (acc) at (.7, .7) {};
    \node (bbd) at (.7, -.7) {};
    \node (bdd) at (-.7, .7) {};

    \draw[pom] (a) -- (aab);
    \draw[pom] (abb) -- (b);
    \draw[pom] (b) -- (bbc);
    \draw[pom] (bcc) -- (c);
    \draw[pom] (c) -- (ccd);
    \draw[pom] (cdd) -- (d);
    \draw[pom] (d) -- (add);
    \draw[pom] (aad) -- (a);

    \draw[pom] (aab) -- (abb);
    \draw[sol] (bbc) -- (bcc);
    \draw[sol] (ccd) -- (cdd);
    \draw[sol] (add) -- (aad);

    \draw[pom] (a) -- (aac);
    \draw[pom] (b) -- (bbd);
    \draw[pom] (c) -- (acc);
    \draw[pom] (d) -- (bdd);

    \node (pu) at (-.3, -.3) {};
    \draw[sol] (aab) to [out=135, in=300] (pu);
    \draw[sol] (aac) to [out=0, in=205] (pu);
    \draw[pom] (aad) to [out=345, in=135] (pu);

    \node (pv) at (.3, -.3) {};
    \draw[sol] (abb) to [out=45, in=240] (pv);
    \draw[sol] (bbd) to [out=180, in=335] (pv);
    \draw[pom] (bbc) to [out=195, in=45] (pv);

    \node (pw) at (.3, .3) {};
    \draw[pom] (bcc) to [out=165, in=330] (pw);
    \draw[sol] (acc) to [out=180, in=25] (pw);
    \draw[pom] (ccd) to [out=300, in=135] (pw);

    \node (pz) at (-.3, .3) {};
    \draw[pom] (add) to [out=15, in=210] (pz);
    \draw[sol] (bdd) to [out=0, in=155] (pz);
    \draw[pom] (cdd) to [out=240, in=45] (pz);

    \node (q) at(0,0) {};
    \draw[pom] (pu) to [out=75, in=255] (q);
    \draw[pom] (pv) to [out=105, in=285] (q);
    \draw[sol] (pw) to [out=255, in=75] (q);
    \draw[sol] (pz) to [out=285, in=105] (q);
  \end{tikzpicture}
  \caption{No side or diagonal of the square was taken into }
\end{subfigure}
~
\begin{subfigure}{.32\textwidth}
  \begin{tikzpicture}
    \tikzset{every node/.style={draw,fill=black,circle,scale=.5,inner sep=0pt,outer sep=2pt}}
    \tikzstyle{pom}=[draw=black!60,thin]
    \node[vert]   (a) at (-1, -1) {};
    \node[vert]   (b) at (1, -1) {};
    \node[vert]   (c) at (1, 1) {};
    \node[vert]   (d) at (-1, 1) {};

    \node (aab) at (-.33,-1) {};
    \node (abb) at (.33, -1) {};

    \node (bbc) at (1, -.33) {};
    \node (bcc) at (1, .33) {};

    \node (ccd) at (.33, 1) {};
    \node (cdd) at (-.33, 1) {};

    \node (add) at (-1, .33) {};
    \node (aad) at (-1, -.33) {};

    \node (aac) at (-.7, -.7) {};
    \node (acc) at (.7, .7) {};
    \node (bbd) at (.7, -.7) {};
    \node (bdd) at (-.7, .7) {};

    \draw[pom] (a) -- (aab);
    \draw[pom] (abb) -- (b);
    \draw[pom] (b) -- (bbc);
    \draw[pom] (bcc) -- (c);
    \draw[sol] (c) -- (ccd);
    \draw[sol] (cdd) -- (d);
    \draw[pom] (d) -- (add);
    \draw[pom] (aad) -- (a);

    \draw[pom] (aab) -- (abb);
    \draw[sol] (bbc) -- (bcc);
    \draw[pom] (ccd) -- (cdd);
    \draw[sol] (add) -- (aad);

    \draw[pom] (a) -- (aac);
    \draw[pom] (b) -- (bbd);
    \draw[pom] (c) -- (acc);
    \draw[pom] (d) -- (bdd);

    \node (pu) at (-.3, -.3) {};
    \draw[sol] (aab) to [out=135, in=300] (pu);
    \draw[sol] (aac) to [out=0, in=205] (pu);
    \draw[pom] (aad) to [out=345, in=135] (pu);

    \node (pv) at (.3, -.3) {};
    \draw[sol] (abb) to [out=45, in=240] (pv);
    \draw[sol] (bbd) to [out=180, in=335] (pv);
    \draw[pom] (bbc) to [out=195, in=45] (pv);

    \node (pw) at (.3, .3) {};
    \draw[pom] (bcc) to [out=165, in=330] (pw);
    \draw[sol] (acc) to [out=180, in=25] (pw);
    \draw[pom] (ccd) to [out=300, in=135] (pw);

    \node (pz) at (-.3, .3) {};
    \draw[pom] (add) to [out=15, in=210] (pz);
    \draw[sol] (bdd) to [out=0, in=155] (pz);
    \draw[pom] (cdd) to [out=240, in=45] (pz);

    \node (q) at(0,0) {};
    \draw[pom] (pu) to [out=75, in=255] (q);
    \draw[pom] (pv) to [out=105, in=285] (q);
    \draw[sol] (pw) to [out=255, in=75] (q);
    \draw[sol] (pz) to [out=285, in=105] (q);
  \end{tikzpicture}
  \caption{ contains one side of the graph.}
\end{subfigure}
~
\begin{subfigure}{.32\textwidth}
  \begin{tikzpicture}
    \tikzset{every node/.style={draw,fill=black,circle,scale=.5,inner sep=0pt,outer sep=2pt}}
    \tikzstyle{pom}=[draw=black!60,thin]
    \node[vert]   (a) at (-1, -1) {};
    \node[vert]   (b) at (1, -1) {};
    \node[vert]   (c) at (1, 1) {};
    \node[vert]   (d) at (-1, 1) {};

    \node (aab) at (-.33,-1) {};
    \node (abb) at (.33, -1) {};

    \node (bbc) at (1, -.33) {};
    \node (bcc) at (1, .33) {};

    \node (ccd) at (.33, 1) {};
    \node (cdd) at (-.33, 1) {};

    \node (add) at (-1, .33) {};
    \node (aad) at (-1, -.33) {};

    \node (aac) at (-.7, -.7) {};
    \node (acc) at (.7, .7) {};
    \node (bbd) at (.7, -.7) {};
    \node (bdd) at (-.7, .7) {};

    \draw[pom] (a) -- (aab);
    \draw[pom] (abb) -- (b);
    \draw[sol] (b) -- (bbc);
    \draw[sol] (bcc) -- (c);
    \draw[pom] (c) -- (ccd);
    \draw[pom] (cdd) -- (d);
    \draw[sol] (d) -- (add);
    \draw[sol] (aad) -- (a);

    \draw[pom] (aab) -- (abb);
    \draw[pom] (bbc) -- (bcc);
    \draw[sol] (ccd) -- (cdd);
    \draw[pom] (add) -- (aad);

    \draw[pom] (a) -- (aac);
    \draw[pom] (b) -- (bbd);
    \draw[pom] (c) -- (acc);
    \draw[pom] (d) -- (bdd);

    \node (pu) at (-.3, -.3) {};
    \draw[sol] (aab) to [out=135, in=300] (pu);
    \draw[sol] (aac) to [out=0, in=205] (pu);
    \draw[pom] (aad) to [out=345, in=135] (pu);

    \node (pv) at (.3, -.3) {};
    \draw[sol] (abb) to [out=45, in=240] (pv);
    \draw[sol] (bbd) to [out=180, in=335] (pv);
    \draw[pom] (bbc) to [out=195, in=45] (pv);

    \node (pw) at (.3, .3) {};
    \draw[pom] (bcc) to [out=165, in=330] (pw);
    \draw[sol] (acc) to [out=180, in=25] (pw);
    \draw[pom] (ccd) to [out=300, in=135] (pw);

    \node (pz) at (-.3, .3) {};
    \draw[pom] (add) to [out=15, in=210] (pz);
    \draw[sol] (bdd) to [out=0, in=155] (pz);
    \draw[pom] (cdd) to [out=240, in=45] (pz);

    \node (q) at(0,0) {};
    \draw[pom] (pu) to [out=75, in=255] (q);
    \draw[pom] (pv) to [out=105, in=285] (q);
    \draw[sol] (pw) to [out=255, in=75] (q);
    \draw[sol] (pz) to [out=285, in=105] (q);
  \end{tikzpicture}
  \caption{ contains two opposite sides of }
\end{subfigure}
~
\begin{subfigure}{.32\textwidth}
  \begin{tikzpicture}
    \tikzset{every node/.style={draw,fill=black,circle,scale=.5,inner sep=0pt,outer sep=2pt}}
    \tikzstyle{pom}=[draw=black!60,thin]
    \node[vert]   (a) at (-1, -1) {};
    \node[vert]   (b) at (1, -1) {};
    \node[vert]   (c) at (1, 1) {};
    \node[vert]   (d) at (-1, 1) {};

    \node (aab) at (-.33,-1) {};
    \node (abb) at (.33, -1) {};

    \node (bbc) at (1, -.33) {};
    \node (bcc) at (1, .33) {};

    \node (ccd) at (.33, 1) {};
    \node (cdd) at (-.33, 1) {};

    \node (add) at (-1, .33) {};
    \node (aad) at (-1, -.33) {};

    \node (aac) at (-.7, -.7) {};
    \node (acc) at (.7, .7) {};
    \node (bbd) at (.7, -.7) {};
    \node (bdd) at (-.7, .7) {};

    \draw[pom] (a) -- (aab);
    \draw[pom] (abb) -- (b);
    \draw[sol] (b) -- (bbc);
    \draw[sol] (bcc) -- (c);
    \draw[sol] (c) -- (ccd);
    \draw[sol] (cdd) -- (d);
    \draw[pom] (d) -- (add);
    \draw[pom] (aad) -- (a);

    \draw[pom] (aab) -- (abb);
    \draw[pom] (bbc) -- (bcc);
    \draw[pom] (ccd) -- (cdd);
    \draw[sol] (add) -- (aad);

    \draw[pom] (a) -- (aac);
    \draw[pom] (b) -- (bbd);
    \draw[pom] (c) -- (acc);
    \draw[pom] (d) -- (bdd);

    \node (pu) at (-.3, -.3) {};
    \draw[sol] (aab) to [out=135, in=300] (pu);
    \draw[sol] (aac) to [out=0, in=205] (pu);
    \draw[pom] (aad) to [out=345, in=135] (pu);

    \node (pv) at (.3, -.3) {};
    \draw[sol] (abb) to [out=45, in=240] (pv);
    \draw[sol] (bbd) to [out=180, in=335] (pv);
    \draw[pom] (bbc) to [out=195, in=45] (pv);

    \node (pw) at (.3, .3) {};
    \draw[pom] (bcc) to [out=165, in=330] (pw);
    \draw[sol] (acc) to [out=180, in=25] (pw);
    \draw[pom] (ccd) to [out=300, in=135] (pw);

    \node (pz) at (-.3, .3) {};
    \draw[pom] (add) to [out=15, in=210] (pz);
    \draw[sol] (bdd) to [out=0, in=155] (pz);
    \draw[pom] (cdd) to [out=240, in=45] (pz);

    \node (q) at(0,0) {};
    \draw[pom] (pu) to [out=75, in=255] (q);
    \draw[pom] (pv) to [out=105, in=285] (q);
    \draw[sol] (pw) to [out=255, in=75] (q);
    \draw[sol] (pz) to [out=285, in=105] (q);
  \end{tikzpicture}
  \caption{ contains two adjacent sides of the cycle }
\end{subfigure}
~
\begin{subfigure}{.32\textwidth}
  \begin{tikzpicture}
    \tikzset{every node/.style={draw,fill=black,circle,scale=.5,inner sep=0pt,outer sep=2pt}}
    \tikzstyle{pom}=[draw=black!60,thin]
    \node[vert]   (a) at (-1, -1) {};
    \node[vert]   (b) at (1, -1) {};
    \node[vert]   (c) at (1, 1) {};
    \node[vert]   (d) at (-1, 1) {};

    \node (aab) at (-.33,-1) {};
    \node (abb) at (.33, -1) {};

    \node (bbc) at (1, -.33) {};
    \node (bcc) at (1, .33) {};

    \node (ccd) at (.33, 1) {};
    \node (cdd) at (-.33, 1) {};

    \node (add) at (-1, .33) {};
    \node (aad) at (-1, -.33) {};

    \node (aac) at (-.7, -.7) {};
    \node (acc) at (.7, .7) {};
    \node (bbd) at (.7, -.7) {};
    \node (bdd) at (-.7, .7) {};

    \draw[pom] (a) -- (aab);
    \draw[pom] (abb) -- (b);
    \draw[sol] (b) -- (bbc);
    \draw[sol] (bcc) -- (c);
    \draw[sol] (c) -- (ccd);
    \draw[sol] (cdd) -- (d);
    \draw[sol] (d) -- (add);
    \draw[sol] (aad) -- (a);

    \draw[pom] (aab) -- (abb);
    \draw[pom] (bbc) -- (bcc);
    \draw[pom] (ccd) -- (cdd);
    \draw[pom] (add) -- (aad);

    \draw[pom] (a) -- (aac);
    \draw[pom] (b) -- (bbd);
    \draw[pom] (c) -- (acc);
    \draw[pom] (d) -- (bdd);

    \node (pu) at (-.3, -.3) {};
    \draw[sol] (aab) to [out=135, in=300] (pu);
    \draw[sol] (aac) to [out=0, in=205] (pu);
    \draw[pom] (aad) to [out=345, in=135] (pu);

    \node (pv) at (.3, -.3) {};
    \draw[sol] (abb) to [out=45, in=240] (pv);
    \draw[sol] (bbd) to [out=180, in=335] (pv);
    \draw[pom] (bbc) to [out=195, in=45] (pv);

    \node (pw) at (.3, .3) {};
    \draw[pom] (bcc) to [out=165, in=330] (pw);
    \draw[sol] (acc) to [out=180, in=25] (pw);
    \draw[pom] (ccd) to [out=300, in=135] (pw);

    \node (pz) at (-.3, .3) {};
    \draw[pom] (add) to [out=15, in=210] (pz);
    \draw[sol] (bdd) to [out=0, in=155] (pz);
    \draw[pom] (cdd) to [out=240, in=45] (pz);

    \node (q) at(0,0) {};
    \draw[pom] (pu) to [out=75, in=255] (q);
    \draw[pom] (pv) to [out=105, in=285] (q);
    \draw[sol] (pw) to [out=255, in=75] (q);
    \draw[sol] (pz) to [out=285, in=105] (q);
  \end{tikzpicture}
  \caption{Three sides of the square are taken into .}
\end{subfigure}
~
\begin{subfigure}{.32\textwidth}
  \begin{tikzpicture}
    \tikzset{every node/.style={draw,fill=black,circle,scale=.5,inner sep=0pt,outer sep=2pt}}
    \tikzstyle{pom}=[draw=black!60,thin]
    \node[vert]   (a) at (-1, -1) {};
    \node[vert]   (b) at (1, -1) {};
    \node[vert]   (c) at (1, 1) {};
    \node[vert]   (d) at (-1, 1) {};

    \node (aab) at (-.33,-1) {};
    \node (abb) at (.33, -1) {};

    \node (bbc) at (1, -.33) {};
    \node (bcc) at (1, .33) {};

    \node (ccd) at (.33, 1) {};
    \node (cdd) at (-.33, 1) {};

    \node (add) at (-1, .33) {};
    \node (aad) at (-1, -.33) {};

    \node (aac) at (-.7, -.7) {};
    \node (acc) at (.7, .7) {};
    \node (bbd) at (.7, -.7) {};
    \node (bdd) at (-.7, .7) {};

    \draw[pom] (a) -- (aab);
    \draw[pom] (abb) -- (b);
    \draw[pom] (b) -- (bbc);
    \draw[pom] (bcc) -- (c);
    \draw[pom] (c) -- (ccd);
    \draw[pom] (cdd) -- (d);
    \draw[pom] (d) -- (add);
    \draw[pom] (aad) -- (a);

    \draw[pom] (aab) -- (abb);
    \draw[sol] (bbc) -- (bcc);
    \draw[pom] (ccd) -- (cdd);
    \draw[sol] (add) -- (aad);

    \draw[sol] (a) -- (aac);
    \draw[pom] (b) -- (bbd);
    \draw[sol] (c) -- (acc);
    \draw[pom] (d) -- (bdd);

    \node (pu) at (-.3, -.3) {};
    \draw[sol] (aab) to [out=135, in=300] (pu);
    \draw[pom] (aac) to [out=0, in=205] (pu);
    \draw[pom] (aad) to [out=345, in=135] (pu);

    \node (pv) at (.3, -.3) {};
    \draw[sol] (abb) to [out=45, in=240] (pv);
    \draw[sol] (bbd) to [out=180, in=335] (pv);
    \draw[pom] (bbc) to [out=195, in=45] (pv);

    \node (pw) at (.3, .3) {};
    \draw[pom] (bcc) to [out=165, in=330] (pw);
    \draw[pom] (acc) to [out=180, in=25] (pw);
    \draw[sol] (ccd) to [out=300, in=135] (pw);

    \node (pz) at (-.3, .3) {};
    \draw[pom] (add) to [out=15, in=210] (pz);
    \draw[sol] (bdd) to [out=0, in=155] (pz);
    \draw[sol] (cdd) to [out=240, in=45] (pz);

    \node (q) at(0,0) {};
    \draw[sol] (pu) to [out=75, in=255] (q);
    \draw[pom] (pv) to [out=105, in=285] (q);
    \draw[sol] (pw) to [out=255, in=75] (q);
    \draw[pom] (pz) to [out=285, in=105] (q);
  \end{tikzpicture}
  \caption{ contains one diagonal of .}
\end{subfigure}
~
\begin{subfigure}{.32\textwidth}
  \begin{tikzpicture}
    \tikzset{every node/.style={draw,fill=black,circle,scale=.5,inner sep=0pt,outer sep=2pt}}
    \tikzstyle{pom}=[draw=black!60,thin]
    \node[vert]   (a) at (-1, -1) {};
    \node[vert]   (b) at (1, -1) {};
    \node[vert]   (c) at (1, 1) {};
    \node[vert]   (d) at (-1, 1) {};

    \node (aab) at (-.33,-1) {};
    \node (abb) at (.33, -1) {};

    \node (bbc) at (1, -.33) {};
    \node (bcc) at (1, .33) {};

    \node (ccd) at (.33, 1) {};
    \node (cdd) at (-.33, 1) {};

    \node (add) at (-1, .33) {};
    \node (aad) at (-1, -.33) {};

    \node (aac) at (-.7, -.7) {};
    \node (acc) at (.7, .7) {};
    \node (bbd) at (.7, -.7) {};
    \node (bdd) at (-.7, .7) {};

    \draw[pom] (a) -- (aab);
    \draw[pom] (abb) -- (b);
    \draw[pom] (b) -- (bbc);
    \draw[pom] (bcc) -- (c);
    \draw[pom] (c) -- (ccd);
    \draw[pom] (cdd) -- (d);
    \draw[sol] (d) -- (add);
    \draw[sol] (aad) -- (a);

    \draw[pom] (aab) -- (abb);
    \draw[sol] (bbc) -- (bcc);
    \draw[pom] (ccd) -- (cdd);
    \draw[pom] (add) -- (aad);

    \draw[sol] (a) -- (aac);
    \draw[pom] (b) -- (bbd);
    \draw[sol] (c) -- (acc);
    \draw[pom] (d) -- (bdd);

    \node (pu) at (-.3, -.3) {};
    \draw[sol] (aab) to [out=135, in=300] (pu);
    \draw[pom] (aac) to [out=0, in=205] (pu);
    \draw[pom] (aad) to [out=345, in=135] (pu);

    \node (pv) at (.3, -.3) {};
    \draw[sol] (abb) to [out=45, in=240] (pv);
    \draw[sol] (bbd) to [out=180, in=335] (pv);
    \draw[pom] (bbc) to [out=195, in=45] (pv);

    \node (pw) at (.3, .3) {};
    \draw[pom] (bcc) to [out=165, in=330] (pw);
    \draw[pom] (acc) to [out=180, in=25] (pw);
    \draw[sol] (ccd) to [out=300, in=135] (pw);

    \node (pz) at (-.3, .3) {};
    \draw[pom] (add) to [out=15, in=210] (pz);
    \draw[sol] (bdd) to [out=0, in=155] (pz);
    \draw[sol] (cdd) to [out=240, in=45] (pz);

    \node (q) at(0,0) {};
    \draw[sol] (pu) to [out=75, in=255] (q);
    \draw[pom] (pv) to [out=105, in=285] (q);
    \draw[sol] (pw) to [out=255, in=75] (q);
    \draw[pom] (pz) to [out=285, in=105] (q);
  \end{tikzpicture}
  \caption{A diagonal and a side edge of  are in .}
\end{subfigure}
~
\begin{subfigure}{.32\textwidth}
  \begin{tikzpicture}
    \tikzset{every node/.style={draw,fill=black,circle,scale=.5,inner sep=0pt,outer sep=2pt}}
    \tikzstyle{pom}=[draw=black!60,thin]
    \node[vert]   (a) at (-1, -1) {};
    \node[vert]   (b) at (1, -1) {};
    \node[vert]   (c) at (1, 1) {};
    \node[vert]   (d) at (-1, 1) {};

    \node (aab) at (-.33,-1) {};
    \node (abb) at (.33, -1) {};

    \node (bbc) at (1, -.33) {};
    \node (bcc) at (1, .33) {};

    \node (ccd) at (.33, 1) {};
    \node (cdd) at (-.33, 1) {};

    \node (add) at (-1, .33) {};
    \node (aad) at (-1, -.33) {};

    \node (aac) at (-.7, -.7) {};
    \node (acc) at (.7, .7) {};
    \node (bbd) at (.7, -.7) {};
    \node (bdd) at (-.7, .7) {};

    \draw[pom] (a) -- (aab);
    \draw[pom] (abb) -- (b);
    \draw[sol] (b) -- (bbc);
    \draw[sol] (bcc) -- (c);
    \draw[pom] (c) -- (ccd);
    \draw[pom] (cdd) -- (d);
    \draw[sol] (d) -- (add);
    \draw[sol] (aad) -- (a);

    \draw[pom] (aab) -- (abb);
    \draw[pom] (bbc) -- (bcc);
    \draw[pom] (ccd) -- (cdd);
    \draw[pom] (add) -- (aad);

    \draw[sol] (a) -- (aac);
    \draw[pom] (b) -- (bbd);
    \draw[sol] (c) -- (acc);
    \draw[pom] (d) -- (bdd);

    \node (pu) at (-.3, -.3) {};
    \draw[sol] (aab) to [out=135, in=300] (pu);
    \draw[pom] (aac) to [out=0, in=205] (pu);
    \draw[pom] (aad) to [out=345, in=135] (pu);

    \node (pv) at (.3, -.3) {};
    \draw[sol] (abb) to [out=45, in=240] (pv);
    \draw[sol] (bbd) to [out=180, in=335] (pv);
    \draw[pom] (bbc) to [out=195, in=45] (pv);

    \node (pw) at (.3, .3) {};
    \draw[pom] (bcc) to [out=165, in=330] (pw);
    \draw[pom] (acc) to [out=180, in=25] (pw);
    \draw[sol] (ccd) to [out=300, in=135] (pw);

    \node (pz) at (-.3, .3) {};
    \draw[pom] (add) to [out=15, in=210] (pz);
    \draw[sol] (bdd) to [out=0, in=155] (pz);
    \draw[sol] (cdd) to [out=240, in=45] (pz);

    \node (q) at(0,0) {};
    \draw[sol] (pu) to [out=75, in=255] (q);
    \draw[pom] (pv) to [out=105, in=285] (q);
    \draw[sol] (pw) to [out=255, in=75] (q);
    \draw[pom] (pz) to [out=285, in=105] (q);
  \end{tikzpicture}
  \caption{ contains one diagonal and two opposite side edges
    of~.}
\end{subfigure}
~
\begin{subfigure}{.32\textwidth}
  \begin{tikzpicture}
    \tikzset{every node/.style={draw,fill=black,circle,scale=.5,inner sep=0pt,outer sep=2pt}}
    \tikzstyle{pom}=[draw=black!60,thin]
    \node[vert]   (a) at (-1, -1) {};
    \node[vert]   (b) at (1, -1) {};
    \node[vert]   (c) at (1, 1) {};
    \node[vert]   (d) at (-1, 1) {};

    \node (aab) at (-.33,-1) {};
    \node (abb) at (.33, -1) {};

    \node (bbc) at (1, -.33) {};
    \node (bcc) at (1, .33) {};

    \node (ccd) at (.33, 1) {};
    \node (cdd) at (-.33, 1) {};

    \node (add) at (-1, .33) {};
    \node (aad) at (-1, -.33) {};

    \node (aac) at (-.7, -.7) {};
    \node (acc) at (.7, .7) {};
    \node (bbd) at (.7, -.7) {};
    \node (bdd) at (-.7, .7) {};

    \draw[pom] (a) -- (aab);
    \draw[pom] (abb) -- (b);
    \draw[pom] (b) -- (bbc);
    \draw[pom] (bcc) -- (c);
    \draw[pom] (c) -- (ccd);
    \draw[pom] (cdd) -- (d);
    \draw[pom] (d) -- (add);
    \draw[pom] (aad) -- (a);

    \draw[pom] (aab) -- (abb);
    \draw[pom] (bbc) -- (bcc);
    \draw[sol] (ccd) -- (cdd);
    \draw[pom] (add) -- (aad);

    \draw[sol] (a) -- (aac);
    \draw[sol] (b) -- (bbd);
    \draw[sol] (c) -- (acc);
    \draw[sol] (d) -- (bdd);

    \node (pu) at (-.3, -.3) {};
    \draw[sol] (aab) to [out=135, in=300] (pu);
    \draw[pom] (aac) to [out=0, in=205] (pu);
    \draw[sol] (aad) to [out=345, in=135] (pu);

    \node (pv) at (.3, -.3) {};
    \draw[sol] (abb) to [out=45, in=240] (pv);
    \draw[pom] (bbd) to [out=180, in=335] (pv);
    \draw[sol] (bbc) to [out=195, in=45] (pv);

    \node (pw) at (.3, .3) {};
    \draw[sol] (bcc) to [out=165, in=330] (pw);
    \draw[pom] (acc) to [out=180, in=25] (pw);
    \draw[pom] (ccd) to [out=300, in=135] (pw);

    \node (pz) at (-.3, .3) {};
    \draw[sol] (add) to [out=15, in=210] (pz);
    \draw[pom] (bdd) to [out=0, in=155] (pz);
    \draw[pom] (cdd) to [out=240, in=45] (pz);

    \node (q) at(0,0) {};
    \draw[pom] (pu) to [out=75, in=255] (q);
    \draw[pom] (pv) to [out=105, in=285] (q);
    \draw[sol] (pw) to [out=255, in=75] (q);
    \draw[sol] (pz) to [out=285, in=105] (q);
  \end{tikzpicture}
  \caption{Two diagonals of  are taken into .}
\end{subfigure}
~
\begin{subfigure}{.32\textwidth}
  \begin{tikzpicture}
    \tikzset{every node/.style={draw,fill=black,circle,scale=.5,inner sep=0pt,outer sep=2pt}}
    \tikzstyle{pom}=[draw=black!60,thin]
    \node[vert]   (a) at (-1, -1) {};
    \node[vert]   (b) at (1, -1) {};
    \node[vert]   (c) at (1, 1) {};
    \node[vert]   (d) at (-1, 1) {};

    \node (aab) at (-.33,-1) {};
    \node (abb) at (.33, -1) {};

    \node (bbc) at (1, -.33) {};
    \node (bcc) at (1, .33) {};

    \node (ccd) at (.33, 1) {};
    \node (cdd) at (-.33, 1) {};

    \node (add) at (-1, .33) {};
    \node (aad) at (-1, -.33) {};

    \node (aac) at (-.7, -.7) {};
    \node (acc) at (.7, .7) {};
    \node (bbd) at (.7, -.7) {};
    \node (bdd) at (-.7, .7) {};

    \draw[pom] (a) -- (aab);
    \draw[pom] (abb) -- (b);
    \draw[pom] (b) -- (bbc);
    \draw[pom] (bcc) -- (c);
    \draw[sol] (c) -- (ccd);
    \draw[sol] (cdd) -- (d);
    \draw[pom] (d) -- (add);
    \draw[pom] (aad) -- (a);

    \draw[pom] (aab) -- (abb);
    \draw[pom] (bbc) -- (bcc);
    \draw[pom] (ccd) -- (cdd);
    \draw[pom] (add) -- (aad);

    \draw[sol] (a) -- (aac);
    \draw[sol] (b) -- (bbd);
    \draw[sol] (c) -- (acc);
    \draw[sol] (d) -- (bdd);

    \node (pu) at (-.3, -.3) {};
    \draw[sol] (aab) to [out=135, in=300] (pu);
    \draw[pom] (aac) to [out=0, in=205] (pu);
    \draw[sol] (aad) to [out=345, in=135] (pu);

    \node (pv) at (.3, -.3) {};
    \draw[sol] (abb) to [out=45, in=240] (pv);
    \draw[pom] (bbd) to [out=180, in=335] (pv);
    \draw[sol] (bbc) to [out=195, in=45] (pv);

    \node (pw) at (.3, .3) {};
    \draw[sol] (bcc) to [out=165, in=330] (pw);
    \draw[pom] (acc) to [out=180, in=25] (pw);
    \draw[pom] (ccd) to [out=300, in=135] (pw);

    \node (pz) at (-.3, .3) {};
    \draw[sol] (add) to [out=15, in=210] (pz);
    \draw[pom] (bdd) to [out=0, in=155] (pz);
    \draw[pom] (cdd) to [out=240, in=45] (pz);

    \node (q) at(0,0) {};
    \draw[pom] (pu) to [out=75, in=255] (q);
    \draw[pom] (pv) to [out=105, in=285] (q);
    \draw[sol] (pw) to [out=255, in=75] (q);
    \draw[sol] (pz) to [out=285, in=105] (q);
  \end{tikzpicture}
  \caption{Two diagonals and a side of  are in .}
\end{subfigure}
\end{center}

		
		
    \caption{Proof of the Lemma \ref{lem:maxtsp_g2compute_nobreak_sqkite}. For
    every selection of the edges and diagonals of  in the cycle cover
     we are showing, how to select edges of the gadget
    , to realize the  cycle cover in the gadgets-modified graph.}
    \label{fig:maxtsp_g2compute_nobreak_sqkite}
  \end{figure}


\begin{thebibliography}{99}
\bibitem{Bar} A.Barvinok, E.Kh.Gimadi, A.I.Serdyukov: The maximun traveling salesman problem.
In: The Traveling Salesman Problem and its variations, 585-607, G.Gutin and A.Punnen, eds., Kluwer, 2002
\bibitem{BH} R.Bhatia: private communication

\bibitem{Chen} Zhi-Zhong Chen, Yuusuke Okamoto, Lusheng Wang: Improved deterministic approximation algorithms for Max TSP.
Information Processing Letters, 95, 2005, 333-342

\bibitem{Chiang} Yi-Jen Chiang: New Approximation Results for the Maximum Scatter TSP. Algorithmica 41(4): 309-341 (2005)


\bibitem{Fish} M.L.Fisher, G.L.Nemhauser, L.A.Wolsey: An analysis of approximation for finding a maximum weight Hamiltonian circuit. Oper.Res.27 (1979) 799-809
\bibitem{HR} R.Hassin, S.Rubinstein: Better Approximations for Max TSP. Information Processing Letters, 75, 2000,
181-186
\bibitem{HR1} R. Hassin, S. Rubinstein: An Approximation Algorithm for the Maximum Traveling Salesman Problem. Inf. Process. Lett. 67(3): 125-130 (1998)
\bibitem{HRm} R.Hassin, S.Rubinstein: A -approximation algorithm for metric Max TSP,  Information Processing Letters, 81(5): 247-251, 2002
\bibitem{Svir} H.Kaplan, M. Lewenstein, N. Shafrir, M. Sviridenko: Approximation Algorithms for Asymmetric TSP by
Decomposing Directed Regualar Multigraphs. J.ACM 52(4):602-626 (2005)

\bibitem{Kos} S.R.Kosaraju, J.K.Park, C.Stein: Long tours and short superstrings. In: Proc. 35th Annual Symposium on
Foundations of Computer Science (FOCS),166-177 (1994)

\bibitem{Ser1} A.V.Kostochka, A.I.Serdyukov: Polynomial algorithms with the estimates  and 
for the traveling salesman problem of he maximum (in Russian). Upravlyaemye Sistemy, 26:55-59, 1985

\bibitem{Monnot} J. Monnot: Approximation algorithms for the maximum Hamiltonian path problem with specified endpoint(s). European Journal of Operational Research 161(3): 721-735 (2005)
\bibitem{Paluch}
Katarzyna~E. Paluch, Marcin Mucha, and Aleksander Madry.
\newblock A 7/9 - approximation algorithm for the maximum traveling salesman
  problem.
\newblock {In {\em Proceedings of the 12th International Workshop on
  Approximation Algorithms for Combinatorial Optimization}, volume 5687 of {\em
  Lecture Notes in Computer Science}, pages 298--311. Springer, 2009.


\bibitem{PEZ}
Katarzyna E. Paluch and Khaled M. Elbassioni and Anke van Zuylen.
\newblock Simpler Approximation of the Maximum Asymmetric Traveling Salesman Problem.
\newblock In {\em Proceedings of the 29th Symposium on Theoretical Aspects of Computer Science, STACS'2012}, Leibniz International Proceedings of Informatics 14}, pages 501--506, 2012.

\bibitem{Pal34}
Katarzyna Paluch.
\newblock Better Approximation Algorithms for Maximum Asymmetric Traveling Salesman and Shortest Superstring.
\newblock  CoRR abs/1401.3670 (2014).

\bibitem{PY} C.H.Papadimitriou, M.Yannakakis:
The traveling salesman problem with distances one and two.
Mathematics of Operations Research 18 Issue 1 (1993), pp. 1-11

\bibitem{Sch} A.Schrijver: Nonbipartite Matching and Covering. In:
Combinatorial Optimization, Volume , 520-561, Springer 2003

\bibitem{Ser} A.I.Serdyukov:  An Algorithm with an Estimate for the Traveling Salesman Problem of Maximum (in Russian).
Upravlyaemye Sistemy, 25 (1984):80-86

\bibitem{HRTri} R. Hassin and S. Rubinstein: An approximation algorithm for maximum triangle packing.
Discrete Applied Mathematics, 154 (2006), 971--979

\bibitem{DNASEQ} Z. Sichen, L. Zhao, Y. Liang, M. Zamani, R. Patro, R. Chowdhury, E. M. Arkin, J. S. B. Mitchell and Steven Skiena: Optimizing Read Reversals for Sequence Compression - (Extended Abstract).
Algorithms in Bioinformatics - 15th International Workshop ({WABI}), 2015


\bibitem{RNA} W. Tong, R. Goebel, T. Liu and G. Lin: Approximation Algorithms for the Maximum Multiple {RNA} Interaction Problem.
Combinatorial Optimization and Applications - 7th International Conference, {COCOA} 2013, pp. 49--59

\bibitem{CW} Z. Chen and L. Wang: An Improved Approximation Algorithm for the Bandpass-2 Problem.
Combinatorial Optimization and Applications - 6th International Conference, {COCOA} 2012, pp. 188--199

\bibitem{bgww}
A.I. Barvinok, D.S. Johnson, G.J. Woeginger, R. Woodroofe.
\newblock Finding maximum length tours under polyhedral norms.
\newblock In {\em Proceedings of IPCO VI, Lecture Notes in Computer Science},
Vol. 1412, 1998, pp. 195–201.

\bibitem{g_maxtsp}
Alexander I. Barvinok, Sándor P. Fekete, David S. Johnson, Arie Tamir, Gerhard J. Woeginger, Russell Woodroofe.
\newblock The geometric maximum traveling salesman problem. 
\newblock In {\em J. ACM 50(5): 641-664 (2003)}

\bibitem{arkin}
Esther M. Arkin, Yi-Jen Chiang, Joseph S. B. Mitchell, Steven Skiena, Tae-Cheon Yang.
\newblock On the Maximum Scatter TSP (Extended Abstract). 
\newblock In {\em SODA 1997: 211-220}

\bibitem{motwani}
Prasad Chalasani, Rajeev Motwani.
\newblock Approximating Capacitated Routing and Delivery Problems. 
\newblock {\em SIAM J. Comput. 28(6): 2133-2149 (1999)}


\end{thebibliography}






\end{document}
