\documentclass[twocolumn]{IEEEtran} \usepackage{epsfig}
\usepackage{color}
   \usepackage{latexsym}
   \usepackage{pslatex}
  \usepackage{algorithm}
   \usepackage{algorithmic}



\newtheorem{definition}{Definition}
\newtheorem{lemma}{Lemma}
\newtheorem{theorem}{Theorem}

\markboth{IEEE Transactions On Information Theory, Vol. XX, No. Y, Month
XX}
{Dubrova: Synthesis of Binary -Stage Machines}


\begin{document}

\title{Synthesis of Binary -Stage Machines} 
\author{Elena~Dubrova, \IEEEmembership{Member, IEEE}\thanks{E. Dubrova is with the Royal Institute of Technology (KTH), Stockholm, Sweden.}}

\maketitle

\begin{abstract}
An algorithm for constructing a shortest binary -stage machine 
generating a given binary sequence is presented. This algorithm 
can be considered as an extension of Berlekamp-Massey algorithm to the non-linear case. 
\end{abstract}


\begin{keywords}
Berlekamp-Massey algorithm, feedback shift register, nonlinear complexity
\end{keywords}

\section{Introduction}

In his seminal book~\cite{Golomb_book} Golomb described an extended version of the traditional
feedback shift register, shown in Figure~\ref{bin_machine}. He called such a device {\em binary}
-{\em stage machine}. Each stage  has its own next state function .
Both feedback and feedforward connections are allowed.  

In this paper, we address the problem of constructing a binary
-stage machine with the minimum  generating a given
binary sequence. We present a synthesis algorithm and derive the exact lower bound on
. Our work can be considered as an extension of Berlekamp-Massey
algorithm~\cite{Ma69} to the non-linear case. 


For the traditional Non-Linear Feedback Shift Registers (NLFSRs), 
the problem of finding a shortest NLFSR generating a
given binary sequence has been considered in~\cite{Ja91,RiK05,RiKK05}
and~\cite{LiKK07}.  

\section{Preliminaries} 

A {\em binary sequence}  of length  is an -tuple 
where  for . The {\em Hamming weight} of a binary sequence ,
denoted by  ,  is the number of 1s in . A binary sequence  of length  is  {\em balanced}
if .

For a Boolean function , the {\em support} of  is defined by 


The {\em algebraic normal form} (ANF) of a Boolean function  is a polynomial in  of type

where  and  is the binary
expansion of  with  being the least significant bit. 

The {\em gate complexity}~\cite{massey_talk}  (or {\em circuit-size complexity}) of a Boolean function  is the smallest 
number of gates in any acyclic circuit computing , given that the 
gates are restricted to have at most two inputs.

A {\em state} of a binary -stage machine is a vector of values of its  stages.

\section{Synthesis Algorithm} \label{syn_alg}

The algorithm presented in this section exploits the property of binary
-stage machines that {\em  any} binary -tuple can be the next state of 
a given current state. Note that, in a traditional NLFSR in the Fibonacci configuration~\cite{Golomb_book}, the next state
overlaps with a current state in  positions.
The Galois configuration of NLFSRs, introduced in~\cite{Du09j}, is more flexible.
However, since feedforward connections are not allowed in NLFSRs,
the set of possible next states is still limited.

First, we show how to construct a sequence of integers 
whose least significant bits follow a given aperiodic binary sequence of length .


Let  be an infinite vector of all even non-negative integers starting from 0.  
Let  be an infinite vector of all odd positive integers starting from 1.  
We denote by  and  be the th elements of  and , respectively,
for .

Let  and .  Given an aperiodic binary sequence  of length , for every  from 0 to , we repeat the following: 

\begin{itemize}
\item[]
If , then assign  
and increment  by one. Otherwise, assign  
and increment  by one. 
\end{itemize}

\begin{figure}[t!]
\begin{center}
\resizebox{0.9\columnwidth}{!} {\begin{picture}(0,0)\includegraphics{binary_machine.xfig.pstex}\end{picture}\setlength{\unitlength}{3947sp}\begingroup\makeatletter\ifx\SetFigFont\undefined \gdef\SetFigFont#1#2#3#4#5{\reset@font\fontsize{#1}{#2pt}\fontfamily{#3}\fontseries{#4}\fontshape{#5}\selectfont}\fi\endgroup \begin{picture}(5754,2649)(2389,-5548)
\put(5493,-3577){\makebox(0,0)[lb]{\smash{{\SetFigFont{12}{14.4}{\rmdefault}{\mddefault}{\updefault}{\color[rgb]{0,0,0}}}}}}
\put(2776,-3586){\makebox(0,0)[lb]{\smash{{\SetFigFont{12}{14.4}{\rmdefault}{\mddefault}{\updefault}{\color[rgb]{0,0,0}}}}}}
\put(3976,-3586){\makebox(0,0)[lb]{\smash{{\SetFigFont{12}{14.4}{\rmdefault}{\mddefault}{\updefault}{\color[rgb]{0,0,0}}}}}}
\put(6451,-4120){\makebox(0,0)[lb]{\smash{{\SetFigFont{12}{14.4}{\rmdefault}{\mddefault}{\updefault}{\color[rgb]{0,0,0}}}}}}
\put(6376,-4786){\makebox(0,0)[lb]{\smash{{\SetFigFont{12}{14.4}{\rmdefault}{\mddefault}{\updefault}{\color[rgb]{0,0,0}}}}}}
\put(6376,-5386){\makebox(0,0)[lb]{\smash{{\SetFigFont{12}{14.4}{\rmdefault}{\mddefault}{\updefault}{\color[rgb]{0,0,0}}}}}}
\end{picture} }
\caption{ A binary - stage machine.}\label{bin_machine}
\end{center}
\end{figure}


The algorithm described above is summarized as Algorithm~\ref{alg1}.
Its worst-case time complexity is .

Let  be a sequence constructed by the Algorithm~\ref{alg1}.
Each integer  can be represented as a binary expansion
 where 
 is the number of bits needed to represent the largest integer of 
and  is the least significant bit of the expansion.
We interpret each -tuple  as a state
of a binary -stage machine.
By construction,  for all .

Next, we define a mapping , for all , where  is modulo . This mapping assigns 
 to be the next state of a current state  of a binary -stage machine. Each of  remaining
states of the binary -stage machine are mapped into the all-0 state. This
implies that they do not contribute any 1s to the supports  of
the next state functions.


The supports of the next state functions 
implementing the resulting mapping are derived as follows.
Initially , for all . For every  from 0 to ,
we repeat the following:

\begin{itemize}
\item[]
For every  from 0 to : If , where  is modulo , then 

\end{itemize}

The algorithm described above is summarized as Algorithm~\ref{alg2}.
Its worst-case time complexity is .

\begin{algorithm}[t]
\caption{Construct a sequence of non-negative integers whose least significant bits
follow an aperiodic binary sequence .}
\label{alg1}
\begin{algorithmic}[1]
\STATE ; /*even non-negative integers*/
\STATE ; /*odd positive integers*/
\STATE ;
\STATE ;
\FOR{every   from 0 to }
\IF{}
\STATE ; /* is the th element of  */
\STATE ;
\ELSE
\STATE ; /* is the th element of  */
\STATE ;
\ENDIF
\ENDFOR
\STATE Return ;
\end{algorithmic}
\end{algorithm}

\begin{theorem} \label{th1}
The algorithm presented in this section constructs a binary -stage machine 
generating a finite aperiodic binary sequence  where  is given by

where  is the length of .
\end{theorem}
{\bf Proof:}  
When the Algorithm~\ref{alg1} terminates, .
Since  is aperiodic, we have .
Therefore,  the largest odd integer used from  is .
The binary expansion of this odd integer has  bits.
Similarly, when the Algorithm~\ref{alg1} terminates, we have .
The largest even integer used from  is .
The binary expansion of this even integer has  bits.
\begin{flushright}

\end{flushright}

The following property trivially follows from the Theorem~\ref{th1}.

\begin{lemma} 
If  is balanced, then (\ref{bound}) reduces to

\end{lemma} 

\begin{algorithm}[t!]
\caption{Construct the next state functions for a binary -stage machine
which follows the sequence of states , .}
\label{alg2}
\begin{algorithmic}[1]
\FOR{every   from 0 to }
\STATE ;
\ENDFOR
\FOR{every   from 0 to }
\FOR{every   from 0 to }
\STATE /*Each  is of type */
\IF{} 
\STATE ;
\ENDIF
\ENDFOR
\ENDFOR
\STATE Return ;
\end{algorithmic}
\end{algorithm}

As an example, consider the following sequence of length  taken from the Example V.1 in~\cite{LiKK07}:

It was shown in~\cite{LiKK07} that the shortest NLFSR generating this sequence has 7
stages. Below we show that the same sequence can be
generated using a binary machine with 5 stages. This comes as no surprise, since a binary machine is more general than an NLFSR. Using the Algorithm~\ref{alg1}, we construct the following sequence 
of integers whose least significant bits
follow :

By applying the Algorithm~\ref{alg2} to , we get the following supports for the next state functions:
1mm]
\Omega_{f_3}  =   \{(00110),(00111),(01000),(01010),(01011),\\
~~~~~~~~  (01101),(10001),(10101)\}\1mm]
\Omega_{f_1}  =   \{(00000),(00001),(00101),01000),01001),\\
~~~~~~~~  (01011),(01100),(01101),(10101)\}\
These supports have the following ANF expressions:


As we can see, the resulting next state functions have a substantial gate complexity. We can potentially reduce the gate complexity as follows:
\begin{enumerate}
\item By using a different sequence of states to generate . In general, any permutation of even integers from the set  and any permutation of odd integers from the set  can be used in the Algorithm~\ref{alg1} instead of vectors  and , respectively, to construct a sequence of integers whose least significant bits follow .
\item By mapping the remaining  states of the binary -stage machine in a different way.
For example, rather than being mapped into the all-0 state, these states can form another cycle of states.
The resulting binary -stage machine will be branchless.
\end{enumerate}

In general, the problem of constructing a binary -stage machine with the minimum gate complexity 
of next state functions is very hard. It is unlikely that there exists an exact algorithm for solving this problem 
which is feasible for large .

\section{Bound on the Size}

The theorem below shows that the bound given by~(\ref{bound}) is exact.

\begin{theorem} \label{th2}
Given a finite aperiodic binary sequence  of length , any 
binary machine which can generate   has at least  stages, where
 is given by~(\ref{bound}).
\end{theorem}
{\bf Proof:} 
The existence of a binary machine with  stages which can generate 
follows from the Theorem~\ref{th1}.  
It remains to prove that no binary -stage machine with  can generate .

Assume that  is given by (\ref{bound}) and that there exists a binary machine with  
 stages, , which can generate the same sequence . 

Let . One one hand, from~(\ref{bound}), we have .
On the other hand, to be able to generate an aperiodic binary sequence , a binary -stage machine 
must have at least  distinct states with the least significant bit 1. Therefore, it must have at least 
 stages. This contradict the assumption .

In a similar way, we can come to a contradiction for the case  .
Therefore, no binary machine with less than  stages can generate .
\begin{flushright}

\end{flushright}



\section{Conclusion}

We presented an algorithm for constructing a shortest binary
-stage machine generating a given binary sequence. 
Since binary -stage machines are probably the most general extension of NLFSRs,
the lower bound given by the Theorem~\ref{th2} might be useful for estimating non-linear 
complexity of sequences. 

Future work includes finding a heuristic approach for choosing a sequence of states which 
minimizes the gate complexity of the next state functions.




\bibliographystyle{ieeetr}
\bibliography{bib}



\begin{biography} []{Elena Dubrova}
received the Diploma Engineer degree in Computer Science from the Technical University of Sofia, Bulgaria, in 1993, and  the Ph.D. degree in Computer Science from University of Victoria, B.C., Canada, in 1997. Currently she is a professor in Electronic System Design at the School of Information and Communication Technology at Royal Institute of Technology, Stockholm, Sweden. 

She held visiting appointments at the University of New South Wales, Sydney, in 2002, the University of California at Berkeley in 2003, and the University of Queensland in 2005. She has authored over 100 publications in the area of electronic system design. Major contributions include new algorithmic techniques for Boolean decomposition, FPGA technology mapping, and probabilistic verification. Her work has been awarded prestigious prices such as IBM faculty partnership award for outstanding contributions to IBM research and development. Her current research interests include logic synthesis, fault-tolerant computing, formal verification, cryptography, and systems biology. 
\end{biography}

\end{document}
