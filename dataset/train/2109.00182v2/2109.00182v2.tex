
\section{Experiments}
In the following, we evaluate two YOHO models for point cloud registration, which are the YOHO using coarse rotation verification (YOHO-C) and the YOHO with the one-shot transformation estimation (YOHO-O).

\subsection{Experimental protocol}

\textbf{Datasets}. We follow exactly the same experiment protocol as \cite{3dmatch} to prepare the training and testing data on the indoor 3DMatch dataset. In this setting, 5000 predefined keypoints are extracted on every scan. 
However, since the original testset on 3DMatch only contains scan pairs with $>$30\% overlap, we also evaluate the model on the 3DLoMatch dataset~\cite{predator} with overlap between 10\% and 30\% to demonstrate our robustness to low overlap. 
We also evaluate the proposed method on the outdoor ETH dataset~\cite{smooth} and the outdoor WHU dataset~\cite{dong2020registration}, which contain more point density variations than the 3DMatch. Note that our model is trained on the 3DMatch dataset and is solely evaluated on the ETH dataset, the WHU-TLS dataset and the 3DLoMatch dataset. 
For the FCGF backbone, the downsampled voxel size is 2.5cm, 15cm and 0.8m for the 3DMatch, ETH and WHU-TLS datasets according to their scales.

\textbf{Metrics}. Feature Matching Recall (FMR)~\cite{smooth,FCGF,predator}, Inlier Ratio (IR)~\cite{predator} and Registration Recall (RR)~\cite{FCGF,d3feat,predator} are used as metrics for evaluations. A correspondence is regarded as a correct one if the distance between its two matched points is $\le \tau_{c}$ under the ground truth transformation. FMR is the percentage of scan pairs with correct correspondence proportions more than 5\% found by the local descriptor, same as used in \cite{ao2020spinnet,predator}. IR is the average correct correspondence proportions. RR is the percentage of correctly aligned scan pairs, which means that the average distance between the points under the estimated transformation and these points under the ground truth transformation is less than $\tau_{r}$. 
For RR, we follow \cite{predator} and report the number of RANSAC iterations to achieve the performance. Due to the randomness of RANSAC, we run YOHO three times to compute the mean of RR on all datasets.

\textbf{Baselines}. We mainly compare YOHO with state-of-the-art learning based descriptors: PerfectMatch~\cite{smooth}, FCGF~\cite{FCGF},  D3Feat~\cite{d3feat}, LMVD \cite{li2020end}, EPN~\cite{EPN} SpinNet~\cite{ao2020spinnet}, and a learning-based matcher: Predator~\cite{predator}. On the 3DMatch dataset, we also include the results from RelativeNet~\cite{deng20193d} which also learns rotation equivariance for registration.  For all baseline methods, we report the results in their papers or evaluate with their official codes or models.
For a fair comparison, baseline methods uses the RANSAC implementation in Open3D~\cite{zhou2018open3d} with engineering optimization like concurrent computation, optimized hyperparameters and distance checks.


\begin{table}[]
\begin{center}
\resizebox{\linewidth}{!}{
\begin{tabular}{lcccccc}
\toprule[1.3pt]
\multirow{2}{*}{}       & \multicolumn{1}{l}{} & \multicolumn{2}{c}{3DMatch}   & \multicolumn{2}{c}{3DLoMatch} \\
                        & \multicolumn{1}{l}{} & Origin        & Rotated       & Origin        & Rotated       \\ \hline
                        & \multicolumn{1}{l}{} & \multicolumn{4}{c}{Feature Matching Recall (\%)}              \\ \hline
RelativeNet\cite{deng20193d}& \multicolumn{1}{l}{} & 74.6          & -             & -             & -             \\
PerfectMatch\cite{smooth}    & \multicolumn{1}{l}{} & 95.0          & 94.9          & 63.6          & 63.4          \\
FCGF\cite{FCGF}              & \multicolumn{1}{l}{} & 97.4          & 97.6          & 76.6          & 75.4          \\
D3Feat\cite{d3feat}          & \multicolumn{1}{l}{} & 95.6          & 95.5          & 67.3          & 67.6          \\
LMVD\cite{li2020end}         & \multicolumn{1}{l}{} & 97.5          & 96.9          & \underline{78.7}    & \underline{78.4}    \\
EPN\cite{EPN}                & \multicolumn{1}{l}{} & \underline{97.6}    & \underline{97.6}    & 76.3          & 76.6            \\
SpinNet\cite{ao2020spinnet}  & \multicolumn{1}{l}{} & \underline{97.6}    & 97.5          & 75.3          & 75.3          \\
Predator\cite{predator}      & \multicolumn{1}{l}{} & 96.6          & 96.7          & 78.6          & 75.7          \\
YOHO                         & \multicolumn{1}{l}{} & \textbf{98.2} & \textbf{98.1} & \textbf{79.4} & \textbf{79.2} \\ \hline

                        & \multicolumn{1}{l}{} & \multicolumn{4}{c}{Inlier Ratio (\%)}              \\ \hline
PerfectMatch\cite{smooth}    & \multicolumn{1}{l}{} & 36.0          & 35.8          & 11.4          & 11.7          \\
FCGF\cite{FCGF}              & \multicolumn{1}{l}{} & 56.8          & 56.2          & 21.4          & 21.6          \\
D3Feat\cite{d3feat}          & \multicolumn{1}{l}{} & 39.0          & 39.2          & 13.2          & 13.5          \\
LMVD\cite{li2020end}         & \multicolumn{1}{l}{} & 45.1          & 45.0          & 17.3          & 17.0          \\
EPN\cite{EPN}                & \multicolumn{1}{l}{} & 49.5          & 48.6          & 20.7          & 20.6          \\
SpinNet\cite{ao2020spinnet}  & \multicolumn{1}{l}{} & 47.5          & 47.2          & 20.5          & 20.1          \\
Predator\cite{predator}      & \multicolumn{1}{l}{} & \underline{58.0}    & \underline{58.2}    & \textbf{26.7} & \underline{26.2}    \\
YOHO                         & \multicolumn{1}{l}{} & \textbf{64.4} & \textbf{65.1} & \underline{25.9}    & \textbf{26.4} \\ \hline

                             & \#Iters              & \multicolumn{4}{c}{Registration Recall (\%)}                  \\ \hline
RelativeNet\cite{deng20193d} & $\sim$1k             & 77.7          & -             & -             & -             \\
PerfectMatch\cite{smooth}    & 50k                  & 78.4          & 78.4          & 33.0          & 34.9          \\
FCGF\cite{FCGF}              & 50k                  & 85.1          & 84.8          & 40.1          & 40.8          \\
D3Feat\cite{d3feat}          & 50k                  & 81.6          & 83.0          & 37.2          & 36.1          \\
LMVD\cite{li2020end}         & 50k                  & 82.5          & 82.2          & 41.3          & 40.1          \\
EPN\cite{EPN}                & 50k                  & 88.2          & 87.6          & 58.1          & 58.9           \\
SpinNet\cite{ao2020spinnet}  & 50k                  & 88.6          & 88.4          & 59.8          & 58.1          \\
Predator\cite{predator}      & 50k                  & 89.0          & 88.4          & 59.8          & 57.7          \\
\multirow{2}{*}{YOHO-C}      & 0.1k                 & 90.1          & 89.4          & 62.9          & 64.2          \\
                             & 1k                   & \underline{90.5}    & \underline{90.6}    & \underline{64.9}    & 64.5          \\
\multirow{2}{*}{YOHO-O}      & 0.1k                 & 90.3          & 90.2          & 64.8          & \underline{65.5}    \\
                             & 1k                   & \textbf{90.8}          & \textbf{90.6} & \textbf{65.2} & \textbf{65.9} \\
\bottomrule[1.3pt]
\end{tabular}}
\end{center}
\caption{Results on the 3DMatch and 3DLoMatch datasets. The rotated version means that we adding additional arbitrary rotations to all point clouds. Same as \cite{ao2020spinnet,predator}, we set $\tau_{c}$=0.1m and $\tau_{r}$=0.2m to compute metrics.}
\label{tab:main}
\vspace{-20pt}
\end{table}



\begin{figure*}
\begin{center}
\includegraphics[width=1\linewidth]{./figures/main.pdf}
\end{center}
\vspace{-10pt}
\caption{(Left) Qualitative comparison with baselines. (Right) Completed scenes by YOHO and some input partial scans.}
\label{fig:3dmatchresult}
\vspace{-5pt}
\end{figure*}

\subsection{Results on 3DMatch/3DLoMatch}
Results of YOHO and baseline models on the 3DMatch dataset and the 3DLoMatch dataset are shown in Table~\ref{tab:main}. Some qualitative results are shown in Fig.~\ref{fig:3dmatchresult}.
With 50 times fewer RANSAC iterations, YOHO still outperforms all baseline methods. 
The improvements on RR of YOHO mainly come from YOHO's utilization of rotation equivariance, which greatly reduces the searching space. 

\begin{table}[]
\begin{center}
\renewcommand{\arraystretch}{1.1}\resizebox{\linewidth}{!}{
\begin{tabular}{lccccccc}
\toprule[1.5pt]
             &           & \multicolumn{3}{c}{$\tau_{c}=0.1m, \tau_{r}=0.2m$}       & \multicolumn{3}{c}{$\tau_c=0.2m, \tau_r=0.5m$} \\
             & \multicolumn{1}{c}{T(min)} & \multicolumn{1}{c}{FMR} & \multicolumn{1}{c}{IR} & \multicolumn{1}{c}{RR} & \multicolumn{1}{c}{FMR} & \multicolumn{1}{c}{IR} & \multicolumn{1}{c}{RR} \\ \hline
PerfectMatch\cite{smooth}   & 53.7        & \underline{79.2}     & \underline{13.3}    & 83.5          & \underline{96.8}    & 19.7          & 86.4                   \\
FCGF\cite{FCGF}             & 5.2         & 47.3           & 5.5           & 48.2          & 59.0          & 16.7          & 52.2                   \\
D3Feat\cite{d3feat}         & \textbf{4.1}& 59.1           & 7.9           & 55.2          & 63.3          & 18.2          & 59.1                   \\
SpinNet\cite{ao2020spinnet} & 62.6        & \textbf{92.0}  & \textbf{14.3} & \textbf{93.4} & \textbf{99.4} & \underline{23.2}    & \underline{96.0}             \\
Perdator\cite{predator}     & 16.7        & 25.4            & 3.7          & 52.3           & 65.6           & 11.1           & 74.7                    \\
YOHO-C                      & \underline{4.6}   & 71.1           & 10.6          & \underline{87.1}    & \underline{96.8}    & \textbf{26.8} & \textbf{96.8}          \\
YOHO-O                      & 6.2         & 71.1           & 10.6          & 74.1          & \underline{96.8}    & \textbf{26.8} & 94.7                   \\
\bottomrule[1.5pt]
\end{tabular}
}
\end{center}
\caption{Results on the ETH dataset. RANSAC is executed 1k iterations for YOHO and 50k for other methods. T is the total time for the registration on the ETH dataset.}
\label{tab:eth}
\vspace{-20pt}
\end{table}

\begin{table}[]
\begin{center}
\resizebox{\linewidth}{!}{
\begin{tabular}{lclllllll}
\toprule[1.5pt]
            &           & \multicolumn{7}{c}{$\tau_r(m)$}     \\ \cline{3-9} 
            & T(min)    & \multicolumn{1}{c}{0.05} & \multicolumn{1}{c}{0.1} & \multicolumn{1}{c}{0.15} & \multicolumn{1}{c}{0.2} & \multicolumn{1}{c}{0.25} & \multicolumn{1}{c}{0.3} & \multicolumn{1}{c}{0.5} \\ \hline
SpinNet\cite{ao2020spinnet}+ICP     &  66.8         & 92.1          & 96.4          & 96.2          & 96.7          & 96.7          & 96.8          & 96.8              \\
YOHO-C+ICP                          & \textbf{9.1}  & \textbf{92.3} & \textbf{97.0} & \textbf{97.5} & \textbf{97.5} & \textbf{97.5} & \textbf{97.6} & \textbf{97.8}  \\
YOHO-O+ICP                            &  11.2         & 89.5          & 94.1          & 94.7          & 94.9          & 95.1          & 95.4          & 95.9          \\ 
\bottomrule[1.5pt]
\end{tabular}
}
\end{center}
\caption{RR on the ETH dataset with ICP. T is the total time for the registration, including the time used in ICP.}
\label{tab:ethicp}
\vspace{-20pt}
\end{table}


\subsection{Results on the ETH dataset}
\label{sec:maineth}

In Table~\ref{tab:eth}, we provide results in two different thresholds for correspondences and registration and some qualitative results are shown in Fig.~\ref{fig:3dmatchresult}. In the strict setting with $\tau_c=0.1m$ and $\tau_r=0.2m$, YOHO underperforms SpinNet. The reason is that the FCGF backbone of YOHO downsamples input point clouds with a voxel size of 0.15m, which limits the accuracy of correspondences found by YOHO. SpinNet extracts local descriptors on every point independently, which has better accuracy but at a noticeable cost of longer computation time (62.6 min) than YOHO (4.6 min). In a slightly loose setting with $\tau_c=0.2m$ and $\tau_r=0.5m$, YOHO outperforms SpinNet in terms of IR and RR. 

Though the downsampled voxel size in FCGF makes YOHO unable to produce very accurate matches, we show that this can be easily improved by a commonly-used ICP~\cite{icp} post-processing. In Table~\ref{tab:ethicp}, we show RR using ICP post-processing, in which YOHO-C+ICP outperforms SpinNet+ICP in all thresholds including the strictest one with $\tau_r=0.05m$. Additional results on the WHU-TLS dataset can be found in the appendices.


\subsection{Ablation studies}
\label{sec:ablation}
To show the effectiveness of each component in YOHO, we conduct ablation studies on the 3DMatch dataset. The results are shown in Table~\ref{tab:ablation}. We design 5 models sequentially and every model only differs from the previous model on one component. 


\textbf{Invariance via group features}. The model 0 in Table~\ref{tab:ablation} simply applies FCGF~\cite{FCGF} to construct descriptors. Based on the model 0, The model 1 achieves rotation invariance by constructing the icosahedral group features with the same FCGF and average-pooling on the group features. By comparing the model 1 with the model 0, we can see that achieving rotation invariance from group features is more robust. Further analysis on robustness against noise and density variations can be found in the appendices.

\textbf{Coarse rotation verification}. Based on the model 1, the model 2 estimates the coarse rotation from the group features $f_0$ and uses the coarse rotation verification (CRV) in the RANSAC. By comparing the model 2 with the model 1, CRV achieves higher RR than the vanilla RANSAC with 50 times fewer iterations.

\textbf{Group convolution layer}. Based on the model 2, the model 3 adds the proposed group convolutions before average-pooling, i.e. YOHO-C. The group convolution enables the network to exploit patterns defined on the icosahedral group, which brings about 4.9\% improvements on the FMR and 9.6\% improvements on the RR.

\textbf{One-shot transformation estimation}. Based on the model 3, the model 4 estimates the rotation residuals and uses the refined rotations to do the one-shot transformation estimation (OSE) in RANSAC, i.e. YOHO-O, which brings further improvements on RR.



\subsection{Analysis}
We provide more comparisons, analysis on iteration number and running time in the following. More analysis about robustness to noise, point density can be found in the appendices.


\begin{table}
\begin{center}
\resizebox{0.8\linewidth}{!}{
\begin{tabular}{ccccc|rr}
\toprule[1.3pt]
id & Inv. & GConv. & \#Iter & RANSAC & FMR & RR \\ 
\hline
 0 & None  &            & 50k & Vanilla  & 90.0 & 76.2 \\
 1 & Group &            & 50k & Vanilla  & 93.3 & 77.8 \\ 
 2 & Group &            & 1k & CRV       & 93.3 & 80.9 \\ 
 3 & Group & \checkmark & 1k & CRV       & 98.2 & 90.5 \\ 
 4 & Group & \checkmark & 1k & OSE       & 98.2 & 90.8 \\ 
\bottomrule[1.3pt]
\end{tabular}
}
\end{center}
\caption{Ablation studies on the 3DMatch dataset. The ``Inv." means how to get rotation invariance. ``None" means no rotation invariance while ``Group" means average-pooling on icosahedral group features. ``GConv" means the proposed group convolution. ``CRV" means coarse rotation verification while ``OSE" means one-shot rotation estimation. }
\label{tab:ablation}
\vspace{-25pt}
\end{table}


\textbf{Performance with different numbers of sampled points.}
Results in Table~\ref{tab:sample} show that YOHO consistently outperforms all other local descriptors in all cases but underperforms the learning-based matcher Predator~\cite{predator} when very few keypoints are used. The reason is that as a matcher, Predator is able to simultaneously utilize information from both source and target point clouds to find overlapped regions. However, YOHO only extracts descriptors on every point cloud separately, which is oblivious of the other point cloud to be aligned. Meanwhile, it is possible to incorporate YOHO within a learning-based matcher like Predator~\cite{predator} for better performance, which we leave for future works.

\begin{table}[]
\begin{center}
\renewcommand{\arraystretch}{1.3}\resizebox{\linewidth}{!}{
\begin{tabular}{lcccccccccc}
\toprule[1.5pt]
          & \multicolumn{5}{c}{3DMatch}                                                                        & \multicolumn{5}{c}{3DLoMatch}                                                 \\
\#Samples & 5000          & 2500          & 1000          & 500           & 250                                & 5000          & 2500          & 1000          & 500           & 250           \\ \hline
\multicolumn{11}{c}{Feature Matching Recall (\%)}                                                                                                                                              \\ \hline
FCGF\cite{FCGF}      & 97.4          & \underline{97.3}          & \underline{97.0}          & \underline{96.7}          & \multicolumn{1}{c|}{\textbf{96.6}}          & 76.6          & 75.4          & 74.2          & 71.7          & 67.3          \\
D3feat\cite{d3feat}      & 95.6          & 95.4          & 94.5          & 94.1          & \multicolumn{1}{c|}{93.1}          & 67.3          & 66.7          & 67.0            & 66.7          & 66.5          \\
SpinNet\cite{ao2020spinnet}     & \underline{97.6}    &  97.2    & 96.8    & 95.5          & \multicolumn{1}{c|}{94.3}          & 75.3          & 74.9          & 72.5          & 70.0          & 63.6          \\
Predator\cite{predator}    & 96.6          & 96.6          & 96.5          & 96.3    & \multicolumn{1}{c|}{\underline{96.5}} & \underline{78.6}    & \underline{77.4}    & \textbf{76.3} & \textbf{75.7} & \textbf{75.3} \\
YOHO      & \textbf{98.2} & \textbf{97.6} & \textbf{97.5} & \textbf{97.7} & \multicolumn{1}{c|}{96.0}    & \textbf{79.4} & \textbf{78.1} & \textbf{76.3} & \underline{73.8}    & \underline{69.1}    \\ \hline
\multicolumn{11}{c}{Inlier Ratio (\%)}                                                                                                                                                        \\ \hline
FCGF\cite{FCGF}        & 56.8          & 54.1          & 48.7          & 42.5          & \multicolumn{1}{c|}{34.1}          & 21.4          & 20.0          & 17.2          & 14.8          & 11.6          \\
D3feat\cite{d3feat}      & 39.0          & 38.8          & 40.4          & 41.5          & \multicolumn{1}{c|}{\underline{41.8}}    & 13.2          & 13.1          & 14.0          & 14.6          & 15.0          \\
SpinNet\cite{ao2020spinnet}     & 47.5          & 44.7          & 39.4          & 33.9          & \multicolumn{1}{c|}{27.6}          & 20.5          & 19.0          & 16.3          & 13.8          & 11.1          \\
Predator\cite{predator}    & \underline{58.0}    & \underline{58.4}    & \textbf{57.1} & \textbf{54.1} & \multicolumn{1}{c|}{\textbf{49.3}} & \textbf{26.7} & \textbf{28.1} & \textbf{28.3} & \textbf{27.5} & \textbf{25.8} \\
YOHO      & \textbf{64.4} & \textbf{60.7} & \underline{55.7}    & \underline{46.4}    & \multicolumn{1}{c|}{41.2}          & \underline{25.9}    & \underline{23.3}    & \underline{22.6}    & \underline{18.2}    & \underline{15.0}    \\ \hline
\multicolumn{11}{c}{Registration Recall (\%)}                                                                                                                                                  \\ \hline
FCGF\cite{FCGF}        & 85.1          & 84.7          & 83.3          & 81.6          & \multicolumn{1}{c|}{71.4}          & 40.1          & 41.7          & 38.2          & 35.4          & 26.8          \\
D3feat\cite{d3feat}      & 81.6          & 84.5          & 83.4          & 82.4          & \multicolumn{1}{c|}{77.9}          & 37.2          & 42.7          & 46.9          & 43.8          & 39.1          \\
SpinNet\cite{ao2020spinnet}     & 88.6          & 86.6          & 85.5          & 83.5          & \multicolumn{1}{c|}{70.2}          & 59.8          & 54.9          & 48.3          & 39.8          & 26.8          \\
Predator\cite{predator}    & 89.0            & \underline{89.9}          & \textbf{90.6} & \underline{88.5} & \multicolumn{1}{c|}{\textbf{86.6}} & 59.8          & 61.2          & \underline{62.4}          & \textbf{60.8} & \textbf{58.1} \\
YOHO-C    & \underline{90.5}    & 89.7    & 88.4    & 87.6    & \multicolumn{1}{c|}{82.8}          & \underline{64.9}    & \underline{65.1}    & 61.4    & 54.5          & 43.9          \\
YOHO-O    & \textbf{90.8} & \textbf{90.3} & \underline{89.1}          & \textbf{88.6}          & \multicolumn{1}{c|}{\underline{84.5}}    & \textbf{65.2} & \textbf{65.5} & \textbf{63.2} & \underline{56.5}    & \underline{48.0}      \\ 
\bottomrule[1.5pt]
\end{tabular}}
\end{center}
\caption{Quantitative results on the 3DMatch and the 3DLoMatch datasets using different numbers of sampled points. RANSAC is executed 1k iterations for YOHO and 50k for other methods.}
\label{tab:sample}
\vspace{-20pt}
\end{table}




\textbf{Necessary iteration number}. To further show how many iterations are necessary to find a correct transformation, we further conduct an experiment on the 3DMatch/3DLoMatch datasets. For every scan pair, we count the number of iterations required to find a correct transformation. As shown in Fig.~\ref{fig:TR}, a point (R,N) in the figure means R\% scan pairs use less than N iterations to find the true transformation. 
The iteration ends once a true transformation is found while RANSAC chooses the transformation with a max inlier number. Thus, the curve only reveals correspondence quality but is not affected by the termination criteria of RANSAC. 

The results show that both YOHO-C and YOHO-O find true transformations very fast with less than 400 iterations. In comparison, all baseline methods only find a small portion of true transformations within 500 iterations, even though they can also achieve good RRs after 50k iterations as shown in Table~\ref{tab:main}. 
Moreover, an even more significant performance gap is shown on the 3DLoMatch dataset. The reason is that the smaller overlap brings a lower inlier ratio on the putative correspondences, which is less than 0.1 in general. In this case, selecting a triplet of inliers in baseline methods will have a probability less than $0.1^3=0.001$ while the probability of finding true transformations in YOHO-O is the same as the inlier ratio because it only needs one matched pair to compute a transformation.


\begin{figure}
\begin{center}
\includegraphics[width=1\linewidth]{./figures/TRcrop.pdf}
\vspace{-15pt}
\end{center}
   \caption{Ratio of correct transformations versus iteration number on the 3DMatch dataset and the 3DLoMatch dataset.}
\label{fig:TR}
\vspace{-15pt}
\end{figure}


\textbf{Running time}.
On a desktop with an i7-10700 CPU and a 2080Ti GPU, the time consumption is listed in Table~\ref{tab:time_new}. We provide the time $t_1$ used in the feature extraction of one point cloud fragment and the time $t_2$ used in aligning a point cloud pair by RANSAC, and the total time cost $T$ on registration of the 3DMatch and the 3DLoMatch dataset. The feature extraction in YOHO costs longer time than baselines because it needs to compute the backbone network 60 times. 
However, since these 60 point clouds are all rotated versions of the original one, they share many common computations like neighborhood querying for speeding up. 
Aligning scan pairs with RANSAC in YOHO is much faster than baselines. Meanwhile, the baselines uses advanced RANSAC implementation with engineering optimization, so YOHO with 1k iterations is not strictly 50 times faster than baselines with 50k iterations. In total, YOHO still takes the shortest time because there are 433 partial scans, 1623 scan pairs in 3DMatch and 1781 pairs in 3DLoMatch and we only need to extract features once and use them in all subsequent pair alignments.

\begin{table}[]\footnotesize
\renewcommand{\arraystretch}{1.02}\begin{center}
\setlength{\tabcolsep}{3.0mm}{
\begin{tabular}{lcccc}
\toprule[1.0pt]
Method    & \multicolumn{1}{l}{\#Iters} & \multicolumn{1}{l}{$t_1$(s/pc)} & \multicolumn{1}{l}{$t_2$(s/pcp)}& \multicolumn{1}{l}{$T$(min)} \\
\hline
PerfectMatch\cite{smooth}    & 50k   & 22.125    & 0.368   & 180.547   \\
FCGF\cite{FCGF}              & 50k   & 0.381     & 0.384   & 24.535    \\
D3Feat\cite{d3feat}          & 50k   & 0.122     & 0.351   & 20.794    \\
EPN\cite{EPN}                & 50k   & 342.246    & 0.437   & 2494.668\\
SpinNet\cite{ao2020spinnet}  & 50k   & 26.556    & 0.413   & 215.077   \\
Predator\cite{predator}      & 50k   & -         & 1.221   & 69.271    \\
YOHO-C                       & 1k    & 1.812     & 0.056   & \textbf{16.253}    \\
YOHO-O                       & 1k    & 1.812     & 0.167   & 22.553    \\
\bottomrule[1.0pt]
\end{tabular}
}
\end{center}
\caption{The time consumption for the registration on the 3DMatch dataset and the 3DLoMatch dataset. We provide the time $t_1$ used in the feature extraction of one point cloud fragment and the time $t_2$ in aligning a point cloud pair, and the total time $T$ on the registration of the 3DMatch and the 3DLoMatch dataset.}
\label{tab:time_new}
\vspace{-25pt}
\end{table}


\textbf{Limitations}. 
The limitation of our methods mainly lies in two fronts: 1) When the overlap region of two scans mainly consists of planar points
, YOHO may fail to find the correct rotations since rotation estimation is ambiguous on planar points as in Fig.~\ref{fig:fail}.
2) Though YOHO is overall computation-efficient due to 
the improvement in RANSAC 
, the construction of YOHO-Desc is less efficient due to 60 times forward passes of backbone. 
This may possibly be optimized by group simplification~\cite{EMVN}, limiting rotations to SO(2), or advanced equivariance learning techniques~\cite{VN,tensorfield}, which we leave for future works.



\begin{figure}
\begin{center}
\includegraphics[width=0.9\linewidth]{./figures/Fail.pdf}
\vspace{-10pt}
\end{center}
   \caption{A failure case on the 3DLoMatch dataset. The overlap area is planar, leading to ambiguity in rotation estimation.}
\label{fig:fail}
\vspace{-10pt}
\end{figure}

