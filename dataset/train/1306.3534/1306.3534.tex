\documentclass{sigcomm-alternate}

\usepackage{flushend} 

\usepackage{graphicx}
\usepackage{subfigure}

\usepackage{amsmath}
\usepackage{amsfonts}
\usepackage{mdwlist}
\usepackage{enumitem}

\usepackage{url}
\newcommand{\parheading}[1]{\medskip{} \noindent \textbf{#1}}

\usepackage{xcolor}
\newcommand{\cut}[1]{}
\newcommand{\fixme}[1]{{\bf\textcolor{red}{[#1]}}}
\newcommand{\new}[1]{#1}


\title{A cost-benefit analysis of low latency via added utilization}





\numberofauthors{5}
\author{
\alignauthor
Ashish Vulimiri\\
\affaddr{UIUC}\\
\email{vulimir1@illinois.edu}\\
\alignauthor
P.\ Brighten Godfrey\\
\affaddr{UIUC}\\
\email{pbg@illinois.edu}\\
\alignauthor
Sri Varsha Gorge\\
\affaddr{Salesforce}\\
\email{gorge.srivarsha@gmail.com}\\
\and  \alignauthor
Zitian Liu\\
\affaddr{Microsoft}\\
\email{liu236@illinois.edu}\\
\alignauthor
Scott Shenker\\
\affaddr{UC Berkeley and ICSI}\\
\email{shenker@icsi.berkeley.edu}
}

\date{}

\begin{document}
\maketitle

\begin{abstract}
Several recently proposed techniques achieve latency reduction by trading it off for some amount of additional bandwidth usage.  But how would one quantify whether the tradeoff is actually beneficial in a given system?  We develop an economic cost vs.\ benefit analysis for answering this question.  We use the analysis to derive a benchmark for wide-area client-server applications, and demonstrate how it can be applied to reason about a particular latency saving technique --- redundant DNS requests.
\end{abstract}

\section{Introduction}

Many techniques for improving latency in the Internet trade off some amount of extra bandwidth consumption for reduced latency.  Examples include DNS prefetching~\cite{ChromiumPrefetching}, redundant~\cite{Vulimiri2012,Vulimiri2013} and hedged~\cite{Dean2013} queries, and speculative TCP loss recovery mechanisms~\cite{Flach2013}.  But what is the true cost of the added overhead, and when is it outweighed by the latency reduction achieved?  In this brief note, we use an economic cost vs.\ benefit analysis to study these questions.  We consider the tradeoff between cost and benefit in a specific class of systems: wide-area client-server applications (such as web browsing, DNS queries, etc.) involving clients using consumer-level connectivity and service providers in the cloud.  The framework we develop here serves as a baseline; it can be refined or extended for other systems.

Our framework allows for various combinations of incentives at the servers and the clients.  In the common scenario where both servers and clients care exclusively about their own benefit, we show that any technique that saves more than  ms of latency (in the mean or the tail, depending on the metric we are concerned with) for every kilobyte of extra traffic that it sends is useful, even with very pessimistic estimates for the additional cost induced at both clients and servers.  This is a conservative bound assuming the most expensive cost estimates we found; the threshold can be orders of magnitude lower in many realistic scenarios, such as when clients use DSL instead of cellular connectivity.

We develop a framework for comparing the cost of and benefit from latency-saving techniques (\S\ref{sec:framework}); use this framework to derive a benchmark for wide-area client-server applications (\S\ref{sec:analysis}); and demonstrate how the benchmark can be applied in practice via a case study (\S\ref{sec:case-study}).

\section{Framework}
\label{sec:framework}

Consider DNS prefetching, where web browsers pre-emptively initiate DNS lookups for links on a webpage to save latency if the user chooses to follow the link.  Prefetching adds overhead both to the client, which potentially sends DNS requests for more links than the user actually follows, and to the DNS infrastructure, which needs to service these additional requests.  The corresponding benefit is the latency reduction at the client when following a prefetched link, which also translates to an increase in expected ad revenue at the server~\cite{brutlag09}.  DNS prefetching affects several entities, including clients, servers, and network operators.

We account for the cost and benefit to all the stakeholders affected by any given latency-saving technique by comparing the following five quantitities:

\begin{itemize}[noitemsep]
\item  (ms/KB): the average latency savings achieved by the technique, normalized by the volume of extra traffic it adds
\item ,  (\v_sv_c/ms): the average value from latency improvement to the servers and the clients
\end{itemize}

We denote increased utilization in units of data transfer volume and measure added cost at the server and the client.   Note however that these calculated costs are a proxy for \emph{all} the costs (not just bandwidth) incurred by \emph{all} affected entities.  For instance, network operator costs are accounted for via the bandwidth costs ISPs charge servers and clients, and CDN costs are accounted for via the usage fees paid by servers.  One kilobyte of added client-side traffic in a web service has server-side costs including server utilization, energy, network operations staff, network usage fees, and so on.  In essence, we amortize all these diverse costs over units of client- and server-side traffic.

From the perspective of a selfish client, any latency-saving technique is useful as long as the benefit it adds outweights the cost to the client: that is, , or in other words


Similarly, a selfish server would need



\definecolor{darker-gray}{gray}{0.5}
\definecolor{lighter-gray}{gray}{0.7}
\newcommand{\deemphWeak}[1]{\textcolor{darker-gray}{#1}}
\newcommand{\deemphStrong}[1]{\textcolor{lighter-gray}{#1}}

\begin{table*}[ht!]
\begin{center}
{\footnotesize
\begin{tabular}{l|r@{\extracolsep{0pt}.}l|r@{\extracolsep{0pt}.}lr@{\extracolsep{0pt}.}l}
\hline
Service plan & \multicolumn{2}{c|}{Cost } & \multicolumn{4}{c}{Break-even benefit  (msec/KB), assuming...}\\
& \multicolumn{2}{c|}{(\v_s = \/hr} & \multicolumn{2}{c}{24.54p_sp_s/v_sp_s/v_cp_cp_c/v_sp_c/v_c_{2}\to\ell\ell \ge \max \left\{ p_s / (v_s + v_c), p_c / v_c \right\}p_s120300 effectively models the cost (per transferred GB) of an average operation in this system, including the cost of all utilized services.\footnote{This is likely pessimistic since it includes, for example, the cost of increased storage which would not scale linearly with an increase in service operations.}  The other services listed in the table model the cost of more limited operations, such as DNS or bandwidth alone.

On the client side, we limit this investigation to clients in which the dominant cost of incrementally added utilization is due to network bandwidth.  Table~\ref{table:cost-benefit} lists costs  based on several types of connectivity. For these calculations, we assume a user who has paid for basic connectivity already, and calculate the cost of bandwidth from overage charges. Client-side bandwidth costs can be substantially higher than server-side total costs in extreme (cellular) cases but are comparable or cheaper with DSL connectivity.

Of course, there are scenarios which the above range of application costs does not model. For example, a cellular client whose battery is nearly empty may value energy more than bandwidth.  But in a large class of situations, bandwidth is the most constrained resource on the client.




\parheading{Value estimates.} The value of time  is more difficult to calculate, at both the client and server.

For the server, direct value may come from obtaining revenue (ads, sales).  We consider the case of Google.  A study by Google indicated that users experiencing an artifical  ms added delay on each search performed \% fewer searches after 4-6 weeks~\cite{brutlag09}.  Google's revenue per search has been estimated\footnote{Based on forecasts at \url{http://www.trefis.com}.} at \1.545001.24.32\, we have a \1.5424.54 per hour in August 2014~\cite{avgHourlyWage}, which implies /\textrm{ms}\ell \geq p / v\ell3.12p_s / v_s\ellp_c / v_cp_s / v_cp_c / v_s610\approx1024 - 62\%2376110\le 10109k \in [1, 10]k303010101023410$ ms for each KB of extra traffic it adds is economically a net positive, even with very pessimistic estimates of the added costs at both servers and clients.  We showed how the analysis can be applied in practice by using it to identify the choices of parameters with which a particular latency-saving technique, DNS redundancy~\cite{Vulimiri2013,Vulimiri2012}, would prove beneficial.

\bibliographystyle{abbrv}
\bibliography{benchmark}

\end{document}
