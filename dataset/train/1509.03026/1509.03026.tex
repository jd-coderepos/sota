We hypothesized that recognizing and organizing social media data according to narrative roles could help workers overcome narrative gaps present in social media in order to produce evocative and automatic summaries of social media events.


\subsection{Method}
We tested our hypothesis through a controlled study comparing the output of two crowdsourcing workflows: the workflow used by Storia, described above (see Appendix A), and a control version of the workflow where we asked crowd workers to write four-sentence paragraphs for each moment of the event, but without prompts to guide the summary writing phase or the questions and answers generated about the event (see Appendix B). All other aspects of the control workflow remained the same as in the Storia workflow: paragraphs in the control condition went through a de-duplication process similar to that of the Storia condition in order to create a story with one paragraph for each moment.

We ran both the Storia and control workflows over social media posts about four different events (Table~\ref{tab:story_table}), and randomly sampled up to 12,000 posts from the entire corpus of content for each event. Stories ranged from sports events to television specials; in all cases, we chose widely viewed events based on topics that most crowd workers would be familiar with. Furthermore, we chose events that were well-structured (making it easy to compare crowd-created interpretations of the event with the actual sequence of events) and had some element of emotional arousal (and thus suitable for narration rather than just description). 
We may be able to apply our findings to events that are interpreted and processed over an indeterminate amount of time (such as natural disasters and other breaking news), but leave this as future work.

To measure how well a non-expert reader might understand the generated text stories for each condition, we asked 30 Mechanical Turk workers to complete a short evaluation task. As we were interested in emotional reactions of a general population rather than an objective sense of the quality of stories written, we chose not to evaluate stories with experts. Participants were shown the Storia and control stories for a random event in random order, then asked to choose the story they would be more likely to recommend to someone who had wanted to see the event but missed it. We also asked workers to briefly explain their choice. Participants were paid \$0.30 for this two-minute task. Participants' free-form responses to the task were analyzed to look for themes in how participants justified their choice.

A second within-subjects evaluation task asked 30 additional Mechanical Turk workers to evaluate stories according to more fine-grained dimensions (such as informativeness) using 7-point Likert scales. The stories from both conditions for a randomly chosen event were shown in random order. Participants were paid \$0.40 for this task. For all evaluation tasks, participants were restricted to Mechanical Turk workers who had not participated in any of the story creation tasks.



\begin{table}[!t]
  \centering
  \begin{tabularx}{\columnwidth}{X c c c}
    \toprule
    \multicolumn{1}{c}{\textbf{Event}} &
    \multicolumn{1}{c}{\textbf{\# posts}} & 
    \multicolumn{1}{c}{\textbf{\# moments}} & 
    \multicolumn{1}{c}{\textbf{Event Date}} \\
    \midrule
    Sochi Winter Olympics Opening Ceremony & 4691 & 27 & 7 Feb. 2014 \\
    \hline
    2014 FIFA World Cup Semi-finals & 1483 & 45 & 8 Jul. 2014 \\
    \hline
    State of the Union (SOTU) 2015 & 11921 & 48 & 20 Jan. 2015 \\
\hline
    Glee Series Finale & 5574 & 29 & 20 Mar. 2015 \\
    \bottomrule
  \end{tabularx}
  \caption{The events used to generate stories through Storia and the control system. Content was randomly sampled from the entire corpus of posts for each event.}
  \label{tab:story_table}
\end{table}

\begin{table}[!t]
  \centering
  \begin{tabularx}{\columnwidth}{X c c}
    \toprule
    \multicolumn{1}{c}{\textbf{Task}} &
    \textbf{\# of HITs} & 
    \textbf{\$ per HIT} \\
    \midrule
    \normalsize Ask questions & 2 per moment & \$0.10 \\
    \hline
    \normalsize Answer questions & 2 per question & \$0.20 \\
    \hline
    \normalsize Write summaries & 3 per moment & \$0.50 \\
    \hline
    \normalsize Voting for summaries & 5 per moment & \$0.15 \\
    \hline
    \normalsize De-duplication & $\geq$ 2 per moment & \$0.30 \\
    \bottomrule
  \end{tabularx}
  \caption{Summary of the tasks and costs for both the Storia workflow and the control workflow.}
  \label{tab:final_table}
\end{table}

\subsection{Results}
Event type seemed to have a strong effect on the differences we observed between the Storia and control stories for each event. For this reason, we divide this section into two parts --- the first section describes the results for the FIFA and Winter Olympics events, and the second section describes the results for the SOTU and Glee events.

\subsubsection{``Placing me back in the game''}
For the FIFA and Winter Olympics events, crowd workers followed the establisher-initial-peak-release pattern in the paragraphs written for Storia stories, as seen in this example paragraph from the FIFA semi-finals Storia story:

\begin{quote}
\emph{The fans are sitting in front of their TVs and smartphones getting excited about the match as it starts.}

\emph{Germany scores their first goal against Brazil, and the fans are going wild rooting for Germany.}

\emph{Germany then goes on to scored their second, then their third and finally their fourth goal against Brazil, who has zero goals.}

\emph{Fans cannot believe what they're seeing and they're wondering if this is a match or a bloodbath because Germany has completely demolished Brazil.}
\end{quote}

In contrast, paragraphs written for control stories conveyed less of a dramatic arc, and instead tended to dwell on the same idea for most of its sentences. For example, each sentence in the following paragraph from the Winter Olympics control story mentions that viewers are ready for the event: 

\begin{quote}
\emph{People watching the ceremony announced they were ready for it to begin.}

\emph{The people watching were ready to support their individual nations.}

\emph{People tweeted out picture of themselves wearing gear showing their commitment to their country.}

\emph{Some people even tweeted out pictures of babies getting ready for their first opening ceremony.}
\end{quote}











For these two events, participants significantly preferred Storia stories over control stories (FIFA: $\chi^2(1)=6.5333, p<0.05$; Winter Olympics: $\chi^2(1)=10.8, p<0.01$); participants preferred the Storia story over the control story 73\% of the time for the FIFA event and 80\% of the time for the Winter Olympics event.


Participants who picked Storia stories appreciated the large amount of detail included and felt that they were a more complete view of the event. Notably, most of the participants also justified their choice with some variant of ``I felt like I was getting a vivid recap'' or ``the story captured the emotion'':

\begin{quote}
\emph{While it would be simple to just say that [the Storia story] is longer etc...  It actually really expresses more emotion, more detail, and the ability to get a real feel for how the game went, the sentiments involved, everything to make it a better read!}

Participant, FIFA event
\end{quote}

Participants who voted for the control story stated they chose it because it was more concise, conveying major points about the event without including extraneous information:

\begin{quote}
\emph{[The Storia story] seems like a lot of non-quality information designed to entertain... while [the control story] is more informative.}


Participant, Winter Olympics event
\end{quote}

To these participants, the control story seemed more professional. However, the participants that chose Storia stories stated they \emph{did not} pick control stories for very similar reasons: the control story felt like a bland generalization or a brief report. The second set of participants corroborated this sentiment; Friedman tests indicated that participants thought Storia stories had more interesting introductions (FIFA: $\chi^2(1)=8.067, p<0.01$; Winter Olympics: $\chi^2(1)=5.762, p<0.05$), giving Storia stories mean scores of 4.679 ($SD=1.307$) for the FIFA event and 5.111 ($SD=1.22$) for the Winter Olympics event, and control stories mean scores of 3.571 ($SD=1.501$) for the FIFA event and 4.37 ($SD=1.363$) for the Winter Olympics event.

Participants also thought Storia stories for these events were more informative (FIFA: $\chi^2(1)=16.2, p<0.01$; Winter Olympics: $\chi^2(1)=7.1176, p<0.01$), giving Storia stories mean scores of 5.679 ($SD=0.612$) for the FIFA event and 5.519 ($SD=1.087$) for the Winter Olympics event, and control stories mean scores of 4.393 ($SD=1.343$) for the FIFA event and 4.778 ($SD=1.086$) for the Winter Olympics event.

Lastly, participants felt that the Storia stories for these events were more likely to make readers feel as if they were there (FIFA: $\chi^2(1)=15.385, p<0.01$; Winter Olympics: $\chi^2(1)=4.262, p<0.05$), giving Storia stories mean scores of 5.143 ($SD=1.079$) for the FIFA event and 5 ($SD=1.144$) for the Winter Olympics event, and control stories mean scores of 3.25 ($SD=1.404$) for the FIFA event and 4.111 ($SD=1.528$) for the Winter Olympics event.



\subsubsection{``I disliked both stories, but...''}
The stories for the Glee and SOTU events were characteristically different from the stories for the FIFA and Winter Olympics events. For example, in both SOTU stories, workers added their own opinions regarding certain politicians using words not present in the provided set of social media content:

\begin{quote}
\emph{President Obama walked into the SOTU with the normal pomp and circumstance.}

\emph{President Obama came out firm, expressing his overwhelming victories to the obstinate and denialist Congress.}

\emph{John Boehner had the same, half asleep, half drunk look on his face.}

\emph{With that, the gauntlet was thrown.}

\end{quote}

The control condition went so far to parody political relationships to the point of absurdity, which extended through several paragraphs:

\begin{quote}
\emph{Vladimir Putin was not happy, the endless enlargement of NATO could not stand!}

\emph{John McCain on the other hand couldn't help but peer over at his sore buddy Putin and laugh to himself.}

\emph{But the murmering in the crowd quieted as Barack Obama approached the podium...}

\emph{``Suck it Putin!'' he exclaimed as he ripped off his shirt and exposed the ``Superman S'' on his undershirt! ``I'm Barack Obama, you got that'' and he flew away.}

\end{quote}

For these stories, there was no significant effect of study condition on participants' preferences (Glee: $\chi^2(1)=0.133, n.s.$; SOTU: $\chi^2(1)=1.2, n.s.$). For the Glee story, most participants stated they based their choice on writing quality rather than on emotional aspects. The second set of participants reflected this, as there was no significant difference in which story they thought would be more likely to make readers feel like they were at the event. In hindsight, this makes sense, as these events were meant to be televised rather than attended.

Opinions on which story participants preferred for the SOTU event was divided on the control story's use of parody---some workers thought it was amusing, but others disapproved:

\begin{quote}
\emph{[The control story] was juvenile and unintelligent in too many places... [The Storia story] provides a better summary that is more sophisticated and intelligent (even though it's not great either).}

Participant, SOTU event
\end{quote}

This is reflected in the results from the second evaluation task; neither Glee story was seen as more informative than the other ($\chi^2(1)=7.118, n.s.$), but the Storia story for the SOTU event was seen as more informative ($\chi^2(1)=10.889, p<0.01$), receiving a mean score of 4.96 ($SD=1.695$) while the control story received a mean score of 3.28 ($SD=1.969$). Neither story, in both events, was seen as having a more interesting introduction (Glee: $\chi^2(1)=3.556, n.s.$; SOTU: $\chi^2(1)=2, n.s.$). 







Overall, the approach of framing social media summarization around narrative seemed to be more effective for the FIFA and Winter Olympics events. For these events, participants did perceive the Storia story as having higher emotional value, choosing to recommend Storia stories to someone who wanted to learn more about the event.