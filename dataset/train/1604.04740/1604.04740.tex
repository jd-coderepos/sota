\documentclass[twocolumn,english,onecolumn]{IEEEtran}
\usepackage[T1]{fontenc}
\usepackage{babel}
\usepackage{array}
\usepackage{prettyref}
\usepackage{multirow}
\usepackage{amsthm}
\usepackage{amsmath}
\usepackage{amssymb}
\usepackage{graphicx}
\usepackage[unicode=true,
 bookmarks=true,bookmarksnumbered=true,bookmarksopen=true,bookmarksopenlevel=1,
 breaklinks=false,pdfborder={0 0 0},backref=false,colorlinks=false]
 {hyperref}
\hypersetup{pdftitle={Your Title},
 pdfauthor={Your Name},
 pdfpagelayout=OneColumn, pdfnewwindow=true, pdfstartview=XYZ, plainpages=false}
\usepackage{breakurl}

\makeatletter

\providecommand{\tabularnewline}{\\}

\theoremstyle{plain}
\newtheorem{thm}{\protect\theoremname}
\theoremstyle{plain}
\newtheorem{prop}[thm]{\protect\propositionname}


\usepackage{babel}
\ifCLASSOPTIONcompsoc
\else
\fi
\usepackage{cite}\usepackage{MnSymbol}\usepackage{subfigure}

\usepackage{balance}



\newcommand{\E}{\mathop{\mathbb E}}
\let\labelindent\relax
\usepackage{enumitem}



\usepackage{babel}
\providecommand{\definitionname}{Definition}
\providecommand{\theoremname}{Theorem}

\makeatother

\providecommand{\propositionname}{Proposition}
\providecommand{\theoremname}{Theorem}

\begin{document}

\title{Reliable Linear, Sesquilinear and Bijective Operations On Integer
Data Streams Via Numerical Entanglement}


\author{Mohammad Ashraful Anam and~Yiannis Andreopoulos\emph{, Senior
Member, IEEE}\thanks{Corresponding author. This paper will appear in the IEEE Trans. on Signal Processing. This work was supported by EPSRC, project
EP/M00113X/1. The authors are with the Electronic and Electrical Engineering
Department, University College London, Roberts Building, Torrington
Place, London, WC1E 7JE, Tel. +44 20 7679 7303, Fax. +44 20 7388 9325
(both authors), Email: \{mohammad.anam.10, i.andreopoulos\}@ucl.ac.uk.
Parts of this paper have been presented at the 2015 IEEE International
On-Line Testing Symposium (IEEE IOLTS 2015). }}
\maketitle
\begin{abstract}
A new technique is proposed for fault-tolerant linear, sesquilinear
and bijective (LSB) operations on  integer data streams (),
such as: scaling, additions/subtractions, inner or outer vector products,
permutations and convolutions. In the proposed method, the  input
integer data streams are linearly superimposed to form  \emph{numerically-entangled}
integer data streams that are stored in-place of the original inputs.
A series of LSB operations can then be performed directly using these
entangled data streams. The results are extracted from the  entangled
output streams by additions and arithmetic shifts. Any soft errors
affecting any single disentangled output stream are guaranteed to
be detectable via a specific post-computation reliability check. In
addition, when utilizing a separate processor core for each of the
 streams, the proposed approach can recover all outputs after
any single fail-stop failure. Importantly, unlike algorithm-based
fault tolerance (ABFT) methods, the number of operations required
for the entanglement, extraction and validation of the results is
linearly related to the number of the inputs and does not depend on
the complexity of the performed LSB operations. We have validated
our proposal in an Intel processor (Haswell architecture with AVX2
support) via several types of operations: fast Fourier transforms,
circular convolutions, and matrix multiplication operations. Our analysis
and experiments reveal that the proposed approach incurs between 
to  reduction in processing throughput for a wide variety of
LSB operations. This overhead is 5 to 1000 times smaller than that
of the equivalent ABFT method that uses a checksum stream. Thus, our
proposal can be used in fault-generating processor hardware or safety-critical
applications, where high reliability is required without the cost
of ABFT or modular redundancy. \end{abstract}

\begin{IEEEkeywords}
linear operations, sum-of-products, algorithm-based fault tolerance,
silent data corruption, core failures, numerical entanglement
\end{IEEEkeywords}


\section{Introduction}

\IEEEPARstart{T}{he} current technology roadmap for high-performance
computing (HPC) indicates that digital signal processing (DSP) routines
running on such systems must become resilient to transient or permanent
faults occurring in arithmetic, memory or logic units. Such faults
can be caused by process variations, silent data corruptions (e.g.,
due to particle strikes, circuit overclocking or undervolting, or
other hardware non-idealities) \cite{nicolaidis2012design}, scheduling
and runtime-induced faults \cite{AndreopoulosTMM_PlentyOfRoom}, misconfigured
application programming interfaces (APIs) \cite{lu2013cloud} and
opportunistic resource reservation \cite{poola2014fault} in cloud
computing systems. For example, DSP systems such as: webpage or multimedia
retrieval \cite{carterette2009million}, object or face recognition
in images \cite{yang2004two}, machine learning and security applications
\cite{bradski2008learning}, transform decompositions and video encoding
\cite{andreopoulos2000hybrid,andreopoulos2002new,andreopoulos2001local,munteanu2003control,andreopoulos2003high,andreopoulos2007adaptive,kontorinis2009statistical,foo2008analytical,barbarien2004scalable},
and visual search and retrieval systems \cite{jegou2012aggregating},
are now deployed using Amazon Elastic Compute Cloud (EC2) spot instances
with substantially-reduced billing cost (e.g., in the order of 0.01\9\%34\%4844\%9\%18\%45\%MN_{\text{in}}\mathbf{A}a_{m,n}\mathbb{N}{}^{\star}\hat{d}d\left\lfloor a\right\rfloor a\left\lceil a\right\rceil a\left\Vert \mathbf{a}\right\Vert a\:\&\: baba\ll ba\gg baba\,\text{mod}\, b=a-\left\lfloor \frac{a}{b}\right\rfloor ba\leftarrow bbaM\geq3,\: N_{\text{in}}\in\mathfrak{\mathbb{N}{}^{\star}}MM\mathbf{g}\mathbf{d}_{m}m\text{op}\mathbf{\mathfrak{I}}\mathbf{\mathfrak{G}}\mathbf{g}\mathbf{g}\text{op}=\left\{ \times\right\} \mathbf{g}=1\mathbf{g}M\text{op}nM0\leq n<N_{\text{in}}\mathbf{c}_{0},\ldots,\mathbf{c}_{M-1}\mathbf{r}M+1\mathbf{r}M\frac{1}{M}\times100\%\left\lceil \log_{2}M\right\rceil MMM+1MM\geq3N_{\text{in}}\mathbf{c}_{m}0\leq m<MMN_{\text{in}}\boldsymbol{\epsilon}_{m}m\epsilon_{m,n}0\leq n<N_{\text{in}}c_{x,n}c_{y,n}xy0\leq x,y<Mx\neq y.\boldsymbol{\delta}_{m}N_{\text{out}}\mathbf{\hat{d}}_{m}M\text{op}\mathbf{g}\mathbf{g}M\mathbf{c}_{m}\mathbf{g}\boldsymbol{\epsilon}_{m}\mathbf{g}MMw\frac{1}{M}\times100\%\left\lceil \log_{2}M\right\rceil \left\lceil \frac{w}{M}\right\rceil MM\frac{1}{M}\times100\%0.03\%7\%100\%100\%M=3MM=3c_{0,n}c_{1,n}c_{2,n}0\leq n<N_{\text{in}}w\in\left\{ 32,64\right\} abc_{0,n}c_{1,n}c_{2,n}l+klkl+kk\leq l\boldsymbol{\epsilon}_{m}0\leq m<M\text{op}\in\left\{ +,-\right\} \mathbf{g}g_{n}\leftarrow\mathcal{S}_{l}\left\{ g_{n}\right\} +g_{n}\mathbf{g}\boldsymbol{\delta}_{m}N_{\text{out}}\delta_{0,n}\delta_{1,n}\delta_{2,n}kd_{0,n}d_{1,n}d_{2,n}ll+k2l+kw=32l=11k=10nd_{\text{temp}}\hat{d}_{0,n}\hat{d}_{1,n}\hat{d}_{2,n}0\leq n<N_{\text{out}}2wd_{\textrm{temp}}2w\hat{d}_{0,n}\hat{d}_{1,n}\hat{d}_{2,n}l+kd_{\text{temp}}\hat{d}_{0,n}l+k\hat{d}_{2,n}2ld_{\textrm{temp}}3l+kd_{\text{temp}}\hat{d}_{2,n}\left(2w-2l\right)d_{\text{temp}}\hat{d}_{2,n}\hat{d}_{0,n}d_{\text{temp}}\hat{d}_{0,n}\hat{d}_{1,n}\hat{d}_{0,n}\hat{d}_{1,n}\hat{d}_{2,n}\delta_{0,n}w2wwd_{\text{temp}}wl+k\hat{d}_{0,n}\hat{d}_{1,n}\hat{d}_{2,n}l2l+kM=3\mathbf{\hat{d}}_{0}\mathbf{\hat{d}}_{1}\mathbf{\hat{d}}_{2}M=3MM\geq3MMn0\leq n<N_{\text{out}}MMMM-1\forall m,nlk\left(M-2\right)l+kk\leq l1\times M\begin{bmatrix}1 & 0 & \cdots & 0 & \mathcal{S}_{l}\end{bmatrix}\boldsymbol{\mathcal{E}}M=3MlMn0\leq n<N_{\text{in}}MM=4MMMM=4w\in\left\{ 32,64\right\} k=l=\frac{w}{4}M\delta_{m,n}0\leq m<M0\leq n<N_{\text{out}}\boldsymbol{\delta}_{r}0\leq r<M2w2wwwd_{\textrm{temp}}M=4r=0\boldsymbol{\delta}_{r}\hat{d}_{r,n}\hat{d}_{\left(r+M-1\right)\text{mod}M,n}d_{\textrm{temp}}1\leq m<M-2d_{\textrm{temp}}M=4w\in\left\{ 32,64\right\} k=l\mathbf{\hat{d}}_{0},\ldots,\mathbf{\hat{d}}_{M-1}M\geq3MM\leq1019M\geq11M\begin{bmatrix}\epsilon_{0,n} & \cdots & \epsilon_{M-1,n}\end{bmatrix}^{\text{T}}KKMKMKMlkw=32P=1MMlk\left(M-2\right)l+kw-\left\lceil \log_{2}M\right\rceil MNM\text{op}\mathbf{g}\mathbf{g}NMN\times NN\times N\mathrm{C}_{\text{GEMM}}=MN^{3}MN\mathbf{g}O\left(MN^{2}\right)O\left(MN\log_{2}N\right)\mathrm{C}_{\text{conv,time}}=4MN^{2}\mathrm{C}_{\text{conv,freq}}=M\left[\left(45N+15\right)\log_{2}\left(3N+1\right)+3N+1\right]MNO\left(MN\right)C_{\text{ne,conv}}=2MNMN\times NC_{\text{ne,GEMM}}=2MN^{2}\frac{\mathrm{C}_{\text{ne,GEMM}}}{\mathrm{C}_{\text{GEMM}}}\times100\%\frac{\mathrm{C}_{\text{ne,conv}}}{\mathrm{C}_{\text{conv,time}}}\times100\%\frac{\mathrm{C}_{\text{ne,conv}}}{\mathrm{C}_{\text{conv,freq}}}\times100\%NM0.3\%0\%NM\frac{\mathrm{C}_{\text{ABFT,GEMM}}}{\mathrm{C}_{\text{GEMM}}}\times100\%\frac{\mathrm{C}_{\text{ABFT,conv,time}}}{\mathrm{C}_{\text{conv,time}}}\times100\%\frac{\mathrm{C}_{\text{ABFT,conv,freq}}}{\mathrm{C}_{\text{conv,freq}}}\times100\%\mathrm{C}_{\text{ABFT,GEMM}}=2MN^{2}+\frac{1}{M}\mathrm{C}_{\text{GEMM}}\mathrm{C}_{\text{ABFT,conv,time}}=2MN+\frac{1}{M}\mathrm{C}_{\text{conv,time}}\mathrm{C}_{\text{ABFT,conv,freq}}=2MN+\frac{1}{M}\mathrm{C}_{\text{conv,freq}}0.2\%\frac{1}{M}\times100\%10\%M\leq8M>84\%MNMNM=3MNM=32M=3M=8MM=3M=8M=3M=8MM=3M=8M=3M=8M2000\times NN\times1200M=3M=8MMN\in\left\{ 2^{10},\ldots,10\times2^{10}\right\} M0.4\%6.9\%13.8\%34.1\%N_{\text{in}}=10^{6}N_{\text{kernel}}\in\left[100,\:4500\right]M1.8\%2.8\%16.1\%37.8\%M2000\times NN\times1200N\in\left[200,\:2000\right]13.3\%35.1\%3.5\%5.5\%2000\times1200MM\in\left[3,\:8\right]MM\geq3M\delta_{0,n}\mathbf{\mathbf{\delta}}_{0}n0\leq n<N_{\text{out}}\delta_{1,n}\delta_{2,n}\hat{d}_{0,n}\hat{d}_{2,n}\boldsymbol{\delta}_{1}\boldsymbol{\delta}_{2}\hat{\boldsymbol{\mathbf{d}}}_{0}\hat{\mathbf{d}}_{1}\hat{\mathbf{d}}_{2}\boldsymbol{\delta}_{0}\boldsymbol{\delta}_{1}\boldsymbol{\delta}_{2}\boldsymbol{\hat{\delta}}_{0}\hat{\mathbf{d}}_{0}\hat{\mathbf{d}}_{2}\boldsymbol{\delta}_{1}\boldsymbol{\delta}_{2}\boldsymbol{\delta}_{0}n0\leq n<N_{\text{out}}\left\{ \delta_{0,n},\,\delta_{1,n},\,\delta_{2,n}\right\} \hat{d}_{0,n}\hat{d}_{2,n}3l+kd_{\text{temp}}2l+k\delta_{1,n}\delta_{2,n}\left\{ \delta_{0,n},\,\delta_{1,n},\,\delta_{2,n}\right\} n0\leq n<N_{\text{out}}\left\{ \delta_{0,n},\,\delta_{1,n},\,\delta_{2,n}\right\} \delta_{0,n}\delta_{1,n}\delta_{2,n}n0\leq n<N_{\text{out}}n0\leq n<N_{\text{out}}M\mathrm{\mathbf{\delta}}_{r,n}MM-1r\boldsymbol{\delta}_{r}1\leq r\leq M0\leq n<N_{\text{out}}Mn\hat{d}_{\left(r+m\right)\text{mod}M,n}\hat{d}_{\left(r+M-1\right)\text{mod}M,n}2\left(M-1\right)l+kd_{\text{temp}}\left(M-1\right)l+k\left\{ \delta_{0,n},\,\ldots,\,\delta_{r-1,n},\,\delta_{r+1,n},\ldots,\,\delta_{M-1,n}\right\} \left\{ \delta_{0,n},\,\ldots,\,\delta_{M-1,n}\right\} n0\leq n<N_{\text{out}}\left\{ \delta_{0,n},\,\ldots,\,\delta_{M-1,n}\right\} \left\{ \delta_{0,n},\,\ldots,\,\delta_{M-1,n}\right\} n0\leq n<N_{\text{out}}$.
\end{IEEEproof}
\bibliographystyle{IEEEbib}
\bibliography{literatur}





\end{document}
