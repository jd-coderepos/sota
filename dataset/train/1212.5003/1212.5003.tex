\documentclass{article}

\usepackage{amsmath}
\usepackage[T1]{fontenc}
\usepackage{amssymb}
\usepackage{amsthm}
\usepackage{float}
\usepackage{tikz}
\usetikzlibrary{automata}
\usetikzlibrary{shadows}
\usetikzlibrary{arrows}
\usetikzlibrary{shapes}
\usetikzlibrary{decorations.pathmorphing}
\usetikzlibrary{fit}
\usetikzlibrary{trees}
\usetikzlibrary{intersections}


\def\K{{\mathbb K}}

\tikzstyle{every picture}=[
  >=stealth', shorten >=1pt, node distance=1.44cm,auto,bend angle=45,initial text=,
  every state/.style={inner sep=0.75mm, minimum size=1mm},font=\scriptsize,
]


\newcommand{\Card}[1]{\##1}


\newcommand{\ocap}{\otimes}
\newcommand{\ocup}{\oplus}
\newcommand{\oneg}{\ominus}
\newcommand{\bigwedgee}{{\bigwedge_e}}



\newcommand{\oin}{\text{\textcircled{\begin{tiny}  \end{tiny}}} }
\newcommand{\bigocap}{\text{\textcircled{\begin{tiny}  \end{tiny}}} }
\newcommand{\Intervalle}[2]{\{#1,\ldots,#2\}}

\newtheorem{definition}{Definition}
\newtheorem{proposition}{Proposition}
\newtheorem{corollary}{Corollary}
\newtheorem{theorem}{Theorem}
\newtheorem{example}{Example}






\bibliographystyle{plain}
\pagestyle{plain}

\begin{document}

\title{A General Framework for the Derivation of Regular Expressions}

  \author{
    Pascal Caron, Jean-Marc Champarnaud and Ludovic Mignot\\
    \{pascal.caron,jean-marc.champarnaud,ludovic.mignot\}@univ-rouen.fr\\
    LITIS, Universit\'e de Rouen,\\
     76801 Saint-\'Etienne du Rouvray Cedex, France
  }

  \maketitle

  
   \begin{abstract}
The aim of this paper is to design a theoretical 
framework
that allows us to perform the computation of regular expression derivatives through a space of generic structures.
Thanks to this formalism, the main properties of regular expression derivation, such as 
the finiteness of the set of derivatives,
need only be stated and proved one time, at the top level. 
Moreover, it is shown how to
construct an alternating automaton
associated with the derivation of a regular expression in this general 
framework.
Finally, Brzozowski's derivation and Antimirov's derivation turn out to be a particular case of this general scheme
and it is shown how to construct a DFA, a NFA and an AFA for both of these derivations.
  \end{abstract}
 

\section{Introduction}


The (left) quotient of a language  over an alphabet  with respect to a word  in 
is the language obtained by stripping the leading  from the words in  that are prefixed by . 
The quotient operation plays a fundamental role in language theory and is especially involved in two main issues.
First, checking whether a word  belongs to a language 
turns out to be equivalent to checking whether the empty word belongs to the quotient of  w.r.t. . 
Secondly it was proved by Myhill~\cite{Myh57} and Nerode~\cite{Ner58} that a language is regular
if and only if the set of its quotients w.r.t. all the words in  is finite.
In the case of a regular language , 
the different quotients are the states of the so-called quotient automaton of 
that is isomorphic to its minimal deterministic automaton.

Since the equality of two languages amounts to 
the isomorphism of their minimal deterministic automata, 
the construction of the quotient automaton via the computation of the left quotients is intractable.
The seminal work of Brzozowski~\cite{Brz64}
that introduced the notion of a word derivative of a regular expression
and the construction of the associated DFA,
gave rise to a long series of studies~(see for example~\cite{BS86,Ant96,CZ01c,IY03,CCM11b}) 
that are all based on 
the simulation of the computation of a language 
quotient
by the one of an expression derivative.
In all these research works, the first rule is that if the expression  denotes the language ,
then the derivative of  w.r.t. , for any , denotes the quotient of  w.r.t. the word .
Thus, checking whether a word  belongs to the language denoted by the expression   is equivalent to 
checking if the empty word belongs to the language denoted by the derivative of  w.r.t. .
The second rule is that as far as the set  of all the derivatives of  is finite,
a finite automaton recognizing the language denoted by  can be constructed,
admitting  as a set of states.

Let us notice 
that Brzozowski derivatives~\cite{Brz64} handle 
unrestricted regular expressions and provide a deterministic automaton;
Antimirov derivatives~\cite{Ant96} only address simple regular expressions 
and provide both a deterministic automaton and a non-deterministic one;
Antimirov derivatives have been recently extended to 
regular expressions~\cite{CCM11b} and this extension
provides a deterministic automaton, a non-deterministic one
and, as shown in this paper, an alternating automaton.
Berry and Sethi continuations~\cite{BS86} are based on the linearization of the (simple) input expression
and allow the construction of its Glushkov (non-deterministic) automaton.
Champarnaud and Ziadi c-continuations~\cite{CZ01c} and Ilie and Yu derivatives~\cite{IY03}
allow both the construction of the Glushkov automaton and of the Antimirov non-deterministic automaton.
Let us mention that derivation has been extended to expressions with multiplicity~\cite{LS01,COZ09}. 
 
As mentioned by Antimirov~\cite{Ant96},
derivatives of regular expressions have proved to be a productive concept to investigate theoretical topics such as
the algebra of regular expressions~\cite{Con71} or
of -regular expressions~\cite{Kro92},
the systems of language equations~\cite{BL80},
the equivalence of simple regular expressions~\cite{Gin67} or of regular expressions~\cite{AM95}.
More recently, Brzozowski introduced a new approach for finding upper bounds
for the state complexity of regular languages,
based on the counting of their quotients (or of their derivatives)~\cite{Brz10}.

Moreover, derivatives provide a useful tool to implement regular matching algorithms:
Brzozowski's DFA and Antimirov's NFA turn out to be 
competitive matching automata~\cite{SL07},
compared for instance with Thompson's -automaton~\cite{Tho68}.
The derivative-based techniques are well-suited to functional languages,
that are characterized by a good support for symbolic term manipulation.
As an example, two derivative-based scanner generators have been recently developed,
one for PLT Scheme and one for Standard ML, as reported in~\cite{ORT09}.
Similarly, Brzozowski's derivatives are used in the implementation of the XML schema language RELAX NG~\cite{Cla02}.
Finally, let us notice that derivatives can be extended to context-free grammars, seen as recursive regular expressions,
yielding a system for parsing context-free grammars~\cite{MDS11}.


The aim of this paper is to design a general 
framework
where the computation of the set of derivatives of a regular expression, called derivation, is performed over a space of generic structures.
Of course Brzozowski's derivation and Antimirov's one appear as particular cases of this general scheme.
A first benefit of this formalism is that the properties inherent to the mechanism of derivation, such as 
the equality between the language denoted by a derivative and the corresponding quotient,
the finiteness of the set of derivatives
and the way for constructing the associated automata,
need only be stated and proved one time, at the top level. 
A second benefit is that the general 
framework
allows us to design the construction of an AFA from the set of derivatives.
As a consequence, we show how to construct a DFA, a NFA and an AFA for any finite derivation, including both Brzozowski's one and Antimirov's one.

The next section is a preliminary section; it gathers classical notions concerning 
regular languages, regular expressions and finite automata, as well as boolean formulas and alternating automata.
The notion of a regular expression derivation via a support is defined in Section~\ref{sec deriv via},
and the properties of the corresponding derivatives are investigated. 
Section~\ref{sec from der to aut} is devoted to the construction of the alternating automaton
associated with the derivation of a regular expression via a support.
This construction is illustrated in Section~\ref{sec deriv via set clausal form},
where the support is based on the set of clausal forms over the alphabet of the regular expressions.

\section{Preliminaries}\label{sec prelim}

  Let . A \emph{boolean formula}  over a set  is inductively defined by  where , or  , where  is the operator associated to the -ary boolean function  from  to  and  are boolean formulas over . The set of the boolean formulas over  is denoted by .
    Let  be a function from  to . The \emph{evaluation of}  \emph{with respect to}  is the boolean  inductively defined by: , . The set  is the subset of  inductively defined by  and .
    
    Let  be an alphabet,  be a word in  and  be a language over . Deciding whether  belongs to  is called the \emph{membership problem for the language} . We denote by  the boolean equal to  if ,  otherwise. Let  be  languages. We denote by ,  and  for any -ary boolean function  the operators defined as follows:
     ,
,     
      . The \emph{quotient of} a regular language  with respect to a word , that is defined as the set  can be inductively computed as follows: 
,  and for ,

	    
	      

 for .

  

  An \emph{Alternating Automaton} (\textbf{AA}) is a -tuple  where  is an \emph{alphabet},  is a set of states,  is a boolean formula over ,  is a function from  to  and  is a function from  to . The function  is extended from  to  as follows: , ,  where  is any symbol in ,  is any word in  and  are any  boolean formulas over .
  The \emph{accessible part of}  is the alternating automaton  defined by: 
;
  , ; .  
  The \emph{language recognized} by the alternating automaton  is the subset  of  defined by .
  Whenever  is a finite set,  is said to be an \emph{Alternating Finite state Automaton} (\textbf{AFA}).   
    
  A \emph{regular expression}  over an alphabet  is inductively defined by: , , , ,  or , where ,  is a regular expression,  and  is the operator associated to the -ary boolean function , \emph{e.g.}  is the operator associated to .
  In the following, we assume that the regular expression operators as well as the boolean formula operators have no specific algebraic properties, unlike the boolean functions. For instance, the operator  is not associative, not commutative, nor idempotent. 
 A regular expression is said to be \emph{simple} if the only boolean operator used is the sum.  
 
  The set of the regular expressions over an alphabet  is denoted by .
  The \emph{language denoted by}  is the subset  of  inductively defined as follows: , , , , , and  with  and  any two regular expressions,  any symbol in  and  the operator associated with  (\emph{e.g.}  is associated to ). Whenever two expressions  and  denote the same language,  and  are said to be \emph{equivalent} (denoted by ).
In the following, we denote by  the boolean .	  
Notice that the boolean  is straightforwardly computed as follows:
	  
	  \centerline{
	    , 
	  }
	  
	  \centerline{
	    ,
	  }
	  
	  \centerline{
	    
	  }
	  
	  \centerline{
	    .
	  }
	  
Given a regular expression  over an alphabet  and a symbol  in , the \emph{Brzozowski derivative} of  w.r.t.  is the expression  inductively defined by:

\centerline{ , ,}

\centerline{ , ,}

\centerline{
  }
  
  where  is any symbol in ,  are any  regular expressions over , and  is the operator associated with the -ary boolean function . Notice that Brzozowski defines the \emph{dissimilar derivative} of  as the expression  inductively computed by substituting the operator  to the  operator in the derivative formulas, where  is the associative, commutative and idempotent version of .
  
  Given a simple regular expression  over an alphabet  and a symbol  in , the \emph{Antimirov partial derivative} of  w.r.t.  is the set of expression  inductively defined by:

\centerline{ , ,}

\centerline{ , ,}

\centerline{
  }
  
  where  is any symbol in ,  are any  regular expressions over , and for any set  of regular expression, for any regular expression , .
  
	  
	  
\section{Derivation \emph{via} a Support}\label{sec deriv via}
We now introduce the notion of a derivation {\it via} a support.
Recall that a Brzozowski derivative is an expression, whereas an Antimirov derivative is a set of expressions.
In our 
framework,
a derivative is more generally an element of an arbitrary set, called the structure set. For example, a structure can be an expression, a set of expressions or a set of set of expressions.
A support is essentially made of a structure set equipped with operators,
and of a mapping that transforms a structure into an expression.

  
  \begin{definition}\label{def support}
    Let  be an alphabet. Let  be a set and  be a mapping from  to . Let  be a set containing:
    \begin{itemize}
      \item for any -ary boolean function , an operator  from  to ,
      \item an operator  from  to .
    \end{itemize}
    Let  and  be two elements in . The -tuple  is said to be a \emph{support} if the three following conditions are satisfied:
    \begin{enumerate}
      \item for any  elements  in :
    
      \centerline{
        ,
      }
    
      \item for any element  in , for any expression  in :
    
      \centerline{
        ,
      }
      
      \item  and .
    \end{enumerate}
  \end{definition}
  
  Notice that the expressions  and  need not to be identical. They are only required to define the same language.  
A support is based on a set of generic structures that can be used to handle regular expressions. We now define the notion of regular expression derivation \emph{via} a support.
  
  \begin{definition}
    Let  be a support. The \emph{derivation via}  is the mapping  from  to  inductively defined for any  , for any word  in  and for any expression  in  by:
    

    \centerline{
        
    }      
     \centerline{   \\
       }
    
    
  \end{definition}
   
     Furthermore, if for all expression  in , the set  is finite, the derivation  is said to be a \emph{finite derivation}.
     
     
  \subsection{Classical Derivations are Derivations \emph{via}  a Support} 
  
     This subsection illustrates the fact that both Antimirov's derivation and  Brzozowski's one are derivations \emph{via} a support.
     
     \begin{definition}\label{def support anti}
       We denote by  the -tuple defined by:
   \begin{itemize}
     \item for any , ,
     \item  where for any elements   in ,
     \begin{itemize}
        \item ,
        \item  if  and ,
        \item  otherwise.
      \end{itemize}
    \end{itemize}
     \end{definition}
   
   \begin{proposition}\label{prop supp anti est supp}
     The -tuple  is a support. Furthermore, for any simple regular expression  over , for any symbol , it holds , where   is the derivation \emph{via} .
   \end{proposition}
   \begin{proof}
     Let  and  be two sets of simple regular expressions and  be a simple regular expression. The condition \textbf{(1)} of Definition~\ref{def support} is satisfied since:
     
     \textbf{(a)} if  and ,
     
     \centerline{
       \begin{tabular}{l@{\ }l}
          & \\
         & \\
         & \\
         & \\
         & 
       \end{tabular}
       } 
     
     \textbf{(b)} and otherwise,
     
     \centerline{
       \begin{tabular}{l@{\ }l}
          & \\
         & \\
       \end{tabular}
       }  
       
     The condition \textbf{(2)} of Definition~\ref{def support} is satisfied since:
     
     \centerline{
       \begin{tabular}{l@{\ }l}
          & \\
         & \\
         & \\
         & \\
       \end{tabular}
       } 
     
       
     The condition \textbf{(3)} of Definition~\ref{def support} is satisfied since:
     
     \centerline{
       \begin{tabular}{l@{\ }l}
          & \\
       \end{tabular}
       } 
     
     \centerline{
       \begin{tabular}{l@{\ }l}
          & \\
       \end{tabular}
       } 
       
     Consequently,  is a support. Moreover, since the operators  and  in  are the operations used in partial derivation, it can be shown by induction that for any simple regular expression  over  and for any symbol , .
   \end{proof}
     
   \begin{definition}\label{def support brzo}
        We denote by  the -tuple defined by:
    \begin{itemize}
      \item for any , ,
      \item , where for any  elements in ,
      \begin{itemize}
        \item ,
        \item  if  and ,
        \item  otherwise.
      \end{itemize}
    \end{itemize}
   \end{definition}
   
   \begin{proposition}
     The -tuple  is a support. Furthermore, for any regular expression  over , for any symbol , it holds , where   is the derivation \emph{via} .
   \end{proposition}
   \begin{proof}
     Since  is the identity and , it is obvious that  is a support. Moreover, since the operators in  are the operators of regular expressions, it can be shown by induction that for any regular expression  over  and for any symbol , .
   \end{proof}
   
    Let us notice that, by definition, the derivation  more generally addresses  unrestricted expressions ; therefore it provides a natural extension for Antimirov derivation. See~\cite{CCM11b} and Section~\ref{sec deriv via set clausal form} for alternative extensions.
   
 
 
 \subsection{Main Properties of Supports}
  
We now show that the language denoted by the expression associated with any derivative  is equal to the corresponding quotient. 

 \begin{proposition}\label{preservation du langage}
   Let  be the derivation via a support  . Then for any word  in , for any expression  in , it holds:
   
   \centerline{.}
 \end{proposition}
 \begin{proof}
 
   By recurrence over the length of .
   
   \begin{enumerate}
     \item Let . By induction over the structure of .
     

    \centerline{
  \\
    }
\medskip
    
    

    \centerline{
     \begin{tabular}{l@{\ }l}
      &\\
      & \\
     \end{tabular}
    }

 
 \medskip 
 
    \centerline{
     \begin{tabular}{l@{\ }l}
     & \\
       &\\
     \end{tabular}
    }
  \medskip
     
  
       \centerline{
     \begin{tabular}{llll}
      &\\
       &\\
        &\\
       &\\\
     \end{tabular}
    }
\medskip
    
    Let us consider that :


\centerline{
       \begin{tabular}{lll}
      &\\
      &\\
      &\\
      &\\
      &\\
\end{tabular}}
\medskip
    
    Let us consider that :


\begin{center}
     \begin{tabular}{llll}
     &&&\\
     &&&\\
       &&&\\
       \end{tabular}
    \end{center}   

     \item Let  with  and . According to the recurrence hypothesis, 


     \begin{center}
       \begin{tabular}{llll}
         &&&\\
          &&&\\
       \end{tabular}
     \end{center}

   \end{enumerate}
   
\end{proof}
 
From Proposition \ref{preservation du langage} we deduce that . This property does not depend whether the derivation is finite or not
and since the boolean  can be inductively computed for any regular expression , any support defines a syntactical solution of the membership problem of the language .

  \begin{corollary}
    For a given regular expression , any derivation via a support can be used to solve the membership problem for .
  \end{corollary}
  
As an example, the support  of Definition~\ref{def support brzo} can be used to solve the membership test, even if the associated derivation is not finite. 
  
Given an expression, the finiteness of the set of its derivatives
is a necessary condition for the construction of an associated finite automaton.
It is well-known that the set of Brzozowski's derivatives is not necessarily finite whereas the set of
dissimilar derivatives
and the set of 
Antimirov's derived
terms are finite sets.
We now give two sufficient conditions of finiteness in the general case. The first one has already be stated in the case of Brzozowski derivatives~\cite{Brz64}. The second one is related to the mapping  of the support, that needs to satisfy specific properties. 

    \begin{proposition}\label{prop finitude}
    Let  be the derivation via a support . The following set of conditions is sufficient for the mapping  to be a finite derivation:
    
      \begin{enumerate}
        \item  is associative, commutative and idempotent (\textbf{H1});
        \item for any -ary boolean function , for any  elements  in ,
        
        \centerline{
           (\textbf{H2}).
        }
      \end{enumerate}
      
  \end{proposition}
  \begin{proof}  
    Let us show by induction over the structure of regular expressions that for any expression  in , the set  is finite.
    
    
     If  :      
        According to the definition of derivation, the proposition holds. \\
           
           
       If :
        Let  be a word in . 
        Let us show by recurrence over the length of  that  is a finite -combination of elements in the set .
          \begin{enumerate}
            \item Let . Since , the property holds.
            \item Let  with  and . By definition, . By the recurrence hypothesis,  is a finite -combination of elements in the set . According to hypothesis \textbf{H2},  is a finite -combination of elements in the set . So,  is a finite -combination of elements in the set .
          \end{enumerate}  
          
          
        As a consequence, since the set  is a finite set by induction hypothesis, since  is a suffix of   and since  is associative, commutative and idempotent, we get:
          
          \centerline{
            .
          } 
          
\ \\
        
        
        
        
    If :      
        Let  be a word in .
         Let us show by recurrence over the length of  that  .
          
          \begin{enumerate}
            \item Let . According to the definition of , the property holds.
            \item Let  with  and . By definition,   . By the recurrence hypothesis,  . According to hypothesis \textbf{H2},
            
            \centerline{   }
            
            \centerline{ .}
            
          \end{enumerate}  
          
         As a consequence, since for all integer  in  the set  is finite by induction hypothesis, we get:
       \begin{center}
        \\
        .
      \end{center}
    
    If :
        Let  be a word in .
          Let us show by recurrence over the length of  that either  or  where  is a finite -combination of elements in the set .
          
          \begin{enumerate}
            \item Let . According to the definition of , the property holds.
            \item Let  with  and .
            
            By definition, .
            
            Two cases have to be considered:
             
             \begin{enumerate}
               \item . Either  , or . According to the recurrence hypothesis, both of these cases satisfy the proposition.
               
               \item  where  is a finite -combination of elements in the set . Either  or . According to hypothesis \textbf{H2},  is a finite -combination of elements in the set , set that equals  is a suffix of .
             \end{enumerate} 
             
             Consequently,  where  is a finite -combination of elements in the set . 
          \end{enumerate} 
          
          As a consequence, since the sets  and  are finite by induction hypothesis and since  is associative, commutative and idempotent,  we get:
          
         \begin{center}
            \\
            .
         \end{center}
         
\end{proof}
  


 
  
The derivation  of Definition~\ref{def support anti} is an example of finite derivation since  is associative, commutative and idempotent, and since for any -ary boolean function  and for any  elements  in , we have:
        
        \centerline{
          .
        }
      
    On the opposite, since  is not an ACI law, Proposition~\ref{prop finitude} does not allow us to conclude for the derivation  of Definition~\ref{def support brzo}. Brzozowski showed in~\cite{Brz64} that it is possible to compute a finite set of dissimilar derivatives using a quotient of the expressions  w.r.t. an ACI sum. It can be achieved by considering the support  defined as follows:
    
    \begin{definition}\label{def support brzo aci}
      We denote by  the -tuple defined by:
    \begin{itemize}
      \item for any ,  (Definition~\ref{def support anti}),
      \item , where for any  elements in ,
      \begin{itemize}
        \item ,
        \item  if  and ,
        \item  otherwise.
      \end{itemize}
    \end{itemize}
    \end{definition}
    
    \begin{proposition}
      The -tuple  is a support. Furthermore, for any regular expression  and for any symbol , it holds that  is the dissimilar derivative of  w.r.t. , where  is the derivation \emph{via} the support .
    \end{proposition}
    \begin{proof}
    According to Definition~\ref{def support anti}, Definition~\ref{def support brzo aci} and Proposition~\ref{prop supp anti est supp}, the conditions \textbf{(1)} and \textbf{(3)} of Definition~\ref{def support} are satisfied by .    
    
     The condition \textbf{(2)} of Definition~\ref{def support} is satisfied since:
     
     \centerline{
       \begin{tabular}{l@{\ }l}
          & \\
         & \\
         & \\
         & \\
       \end{tabular}
       } 
       
       Consequently,  is a support. 
       
       Moreover, since the operator  is an ACI law and since the catenation product  returns a singleton, it can be shown by induction that for any regular expression  and for any symbol , it holds that  is the dissimilar derivative of  w.r.t. .
    \end{proof}
  
\section{From Derivation \emph{via} a Support to Automata}\label{sec from der to aut}

  Computing the set of derivatives of a regular expression  w.r.t. a derivation  is similar as computing the transition function  of an automaton, where . As far as alternating automata are concerned, the resulting expression  needs to be transformed into a boolean formula. This computation is performed through a \emph{base function} defined as follows.
  
  
  \begin{definition}\label{def base}
    Let  be an alphabet. A \emph{base function}  is a mapping from  to  such that for any expression  and for any word  in :
    
    \centerline{  .}
  \end{definition}
  
  \begin{definition}
    Let  be a base function and  be the derivation \emph{via} the support . Let  be an expression in . Let  be the automaton defined by:
    
    \begin{itemize}
      \item ,
      \item ,
      \item , 
        
      \item .
    \end{itemize}
    
    The accessible part of  is said to be the -\emph{alternating automaton of} . Notice that there may exist an infinite number of states.
  \end{definition}
  
  \begin{theorem}~\label{thm aa lang}
     The -alternating automaton of a regular expression  recognizes .
  \end{theorem}
  \begin{proof}
  Let  be the -alternating automaton of . 
    Let  be the derivation via the support . Let  be a word in . 
       Let us show by recurrence over the length of  that for any boolean formula  in , the following \textbf{(P)} proposition holds:
      
        \centerline{
          .
        }
        
        \begin{enumerate}
          \item If , by definition of the transition function ,          
             . By construction, for any integer  in , .
          
           Since for any state , , the following proposition holds:   .
          
          \item Let  with  and . Then it holds:
          
              \begin{tabular}{l@{\ }l}
                & \\
                &  \hfill\textbf{(Definition of )}\\
                &  \hfill\textbf{(Definition of )}\\
                &  \hfill\textbf{(Definition of )}\\
                & \\ & \hfill\textbf{(Construction of )}\\
                &  \\ & \hfill\textbf{(Definition of )}\\
                & \\
                &\ \ \ \ \ \ \ \ \ \ \ \ \ \ \ \ \ \ \ \ \\ &  \hfill\textbf{(Induction hypothesis)}\\
                & \\ &  \hfill\textbf{(Definition of )}\\
                &  \hfill\textbf{(Definition of )}\\
              \end{tabular}
        \end{enumerate}
        
        Finally, it holds:
        
          \begin{tabular}{l@{\ }l}
             &   \hfill\textbf{(Proposition~\ref{preservation du langage})}\\
            &   \hfill\textbf{(Definition~\ref{def base})}\\
            &   \hfill\textbf{(Proposition \textbf{(P)})}\\
            &  \ \ \ \ \ \hfill\textbf{(Definition of the language of an AA)}\\
          \end{tabular}
          
    
\end{proof}
  

  Previous definitions and properties address non necessarily finite automata. We now give sufficient conditions for the finiteness of the automaton.
The basic idea is that if it is equivalent to derive an expression  or to derive its atoms , then the set of atoms obtained by a finite derivation is finite. 
  
  \begin{definition}
    Let  be a support, let  be the derivation \emph{via}  and  be a base function. The couple  satisfies the \emph{atom-derivability property} if for any expression  in  and for any symbol  in :
    
    \centerline{
      .
    }
  \end{definition}
  
   
  \begin{theorem}\label{thm afa}
    Let  be the -alternating automaton of a regular expression . If  is finite and if  satisfies the atom-derivability property, then:
    
    \centerline{
       is an AFA.
    }
  \end{theorem}
  \begin{proof}
    Let  be the derivation via the support . Let  be a word in .
    
    \begin{enumerate}
    
    \item  Let us show by recurrence over the length of  that for any expression  in , .
  
  \begin{enumerate}
        
      \item If , . Then,  .
      
      \item If  with  and , by the recurrence hypothesis, .
      
      
         \begin{tabular}{l@{\ }l}
          & \\
          &  \hfill\textbf{(Definition of )}\\
          &  \\ & \hfill\textbf{(Definition of  with )}\\
          &  \hfill\textbf{(Definition of )}\\
          &  \hfill\textbf{(Definition of )}\\
&  \hfill\textbf{(Definition of )}\\
          & \\ &  \hfill\textbf{(Induction hypothesis and construction of )}\\
          &  \hfill\textbf{(Atom-derivability property)}\\
         \end{tabular}
         
        
    \end{enumerate}
    \item As a direct consequence of the previous point, since the set  is finite, so are the sets  and  . Finally, the set , that is equal to , is a finite set.
    \end{enumerate}
\end{proof}
  
   \begin{example}
    Let  and  be the derivations of Definition~\ref{def support anti} and Definition~\ref{def support brzo aci}.
    Let  be the base inductively defined for any expression by  ,  otherwise.
    Let  the base defined for any expression by .
    It can be shown that any couple in  satisfies the atom-derivability property.
    
    Furthermore, the -AFA of  can be straightforwardly transformed into the 
derived
    term NFA of  defined by Antimirov in~\cite{Ant96}; the -AFA of  can be transformed into the partial derivative DFA of  defined by Antimirov in~\cite{Ant96};  the -AFA of  can be transformed into the dissimilar derivative DFA of  defined by Brzozowski in~\cite{Brz64}. Notice that the -AFA of  can be transformed into a NFA that is different from the NFA of Antimirov.
    
    Finally, the base  inductively defined by   for any operator ,  otherwise, provides an AFA construction both from  and .
  \end{example}

  \section{Derivation \emph{via} the Set of Clausal Forms}\label{sec deriv via set clausal form}
    
    In this section, we show that the set of clausal forms over the set of regular expressions and equipped with the right operators is a derivation support and that the associated  derivation is finite. Furthermore we prove that the atom-derivability property is satisfied whenever the  derivation is associated with a base function in the set . Finally we illustrate these results by the construction of the -AFA of the expression .
    
    \medskip
    
    Let us first recall some definitions about clausal forms and their operators.  
  A \emph{clausal form} over a set  is an element in  where . Let  and  be the two mappings from  to  and  be the function from   to  defined for any  in  by:
  
  \begin{itemize}
    \item ,
    \item , 
    \item     
  \end{itemize}
  
  where 
  
  It can be shown that for any element , for any clause  in , the following conditions are satisfied:
  
  \begin{itemize}
    \item ,
    \item ,
    \item ,
    \item .
  \end{itemize}
  
  Furthermore, let us notice that there exist clauses  such that  or .
  
  \medskip
    
    From now on, we will consider the set  of the clausal forms over the set of regular expressions.    
    We now explain how a clausal form over the set of regular expressions is transformed into a regular expression. Let us consider the function  defined from   to  for any element  in  by:
    
    \centerline{
      \begin{tabular}{l@{\ }l}
         & \\
        where  & \\
      \end{tabular}
    }
    
    Let us now give the definition of the support operators. For any -ary boolean function , let  be the operator from  to  associated with   defined by:
    
    \begin{itemize}
      \item ,
      \item where 
    \end{itemize}
    
    The operator  from  to  is defined, for any clause  in  and for any expression  in , by:
    
    \centerline{
      .
    }
    
    Finally, we consider the operation  set  defined by 
    
    \centerline{
    }
      
     Let us notice that, by definition, the operator  (resp. ) is equal to  the operator  (resp. ) whereas the operator  is different from , since:
     
     \centerline{ .}
     
     There exist several expressions (combinations of ,  and ) for a given  operator. As an example, with the ternary  is associated an operator  that can be expressed as the combination of two  operators:

    \centerline{
      .
    }
     
     Reduced expressions can be found using Karnaugh maps for instance.
    
     We now consider the -tuple  and show that it is a support.
    
    \begin{proposition}\label{prop sc support}
      The -tuple  is a support.
    \end{proposition} 
    \begin{proof}
    
      Properties of support are trivially checked for the clauses ,  and . Let us consider that  are elements in . According to the definitions of  and :
      
      \centerline{
        \begin{tabular}{l@{\ }l} 
           & \\
          & \\
          & \\
          & \\
          & \\
          & \\
          & \\
        \end{tabular}
      }
      
      \centerline{
        \begin{tabular}{l@{\ }l} 
           & \\
          & \\
          & \\
          & \\
          & \\
          & \\
          & \\
          & \\
        \end{tabular}
      }     
      
      Moreover,
      
      \centerline{
        \begin{tabular}{l@{\ }l}
           & \\
          & \\
          & \\
& \\
& \\
& \\
        \end{tabular}
      } 
      
      
      Consequently :
      
      \centerline{
        \begin{tabular}{l@{\ }l} 
           & \\
          & \\
          & \\
          & \\
          & \\
          & \\
          & \\
          & \\
          & \\
        \end{tabular}
      }
      
      Hence, according to definition of :
      
      \centerline{
        \begin{tabular}{l@{\ }l} 
           & \\
& \\
& \\
& \\
& \\
          & \\
        \end{tabular}
      }
      
      Furthermore,
      
      \centerline{
        \begin{tabular}{l@{\ }l}
           & \\
          & \\
          & \\
          & \\
          & \\
          & \\
          & \\
        \end{tabular}
      }
      
      
      
\end{proof}
 
    
    
    
     We now study the properties of the derivation  associated with the support .
     
    
    \begin{theorem}\label{thm db afa le}
      Let  be an expression in . Let  be a couple in . Then:
      
      \centerline{
        The -automaton of  is an AFA recognizing .
      }
    \end{theorem}
    \begin{proof}
    
      \textbf{(I)} Let us show that the derivation  satisfies the sufficient conditions for finiteness of Proposition~\ref{prop finitude}. \textbf{(a)} Since , the function  is associative, commutative and idempotent (\textbf{H1}). \textbf{(b)} According to the definition of the operators ,  and :
      
        \centerline{
          \begin{tabular}{l@{\ }l}
             &  \\
            &  \\
          &  \\
          &  \\
          & \\
          &  \\
          & \\
& .\\
          \end{tabular}
        }
      
        \centerline{
          \begin{tabular}{l@{\ }l}
             & \\
            & \\
            & \\
            & \\ 
& .\\
          \end{tabular}
        }
        
        \centerline{
          \begin{tabular}{l@{\ }l}
             & \\
& \\
            & \\
            & \\
            & \\            
            & \\           
            & \\          
            & \\
            & \\
            & \\
            & \\
& .\\
          \end{tabular}
        }
        
        Finally, since any  is defined as combination of ,  and  operators, hypothesis \textbf{H2} holds. According to Proposition~\ref{prop finitude},  is finite.
      
      \textbf{(II)} Let us show that any couple in  satisfies the atom-derivability property. \textbf{(a)} By definition of , the atom-derivability property is satisfied by . \textbf{(b)} By induction over the structure of . Let  be a symbol in . \textbf{(i)} If  or if  or if , since , . \textbf{(ii)} Let  be a -ary boolean function.
       Let us first prove that the equation of the atom-derivanility property is satisfied for the operators ,  and . If  or  equal  or , equation is trivially satisfied. Let  and  be two clauses different from  and .
       
      \centerline{
        \begin{tabular}{l@{\ }l}
         & \\
        & \\
        & \\
        & \\
        & \\
& \\
        & \\
& \\
        & \\
& \\
        \end{tabular}
      }
      
      \centerline{
        \begin{tabular}{l@{\ }l}
         & \\
        & \\
        & \\
        & \\        
        & \\
        & \\
& \\
        & \\
& \\
        \end{tabular}
      }
      
      
      \centerline{
        \begin{tabular}{l@{\ }l}
         & \\
        & \\
        & \\
        & \\
        & \\
        \end{tabular}
      }
      
      Hence, since any operator  is a composition of ,  and :
      
      \centerline{
        .
      }
      
      
      
      Consequently:
      
      \centerline{
        \begin{tabular}{l@{\ }l}
          & \\ & \\
          & \\
          & \\
          & \\
        \end{tabular}
      }
      
      Furthermore, if , .
      
      \centerline{
        .
      }
      
      Finally,
      
      \centerline{
        \begin{tabular}{l@{\ }l}
         & \\
        & \\
        & \\
        & \\
        & \\
& \\
        & \\
& \\
        & \\
& \\
        \end{tabular}
      }
      
      Consequently, any couple in  satisfies the atom-derivability property.
      
      \textbf{(III)} According to Theorem~\ref{thm afa}, from \textbf{(I)} and \textbf{(II)}, the theorem holds.      
\end{proof}
    
    
    
    The following example illustrates the computation of an AFA from a regular expression. In order to improve readability, regular expressions are simplified according to the following rules:
    
    \centerline{
      \begin{tabular}{l}
        \\
        \\
        \\
      \end{tabular}
    }
    
    \begin{example}
      Let . We now construct the  -automaton of , that is the AFA . We first compute the derivatives  of  w.r.t. any symbol in , according to the derivation  and then compute the derivatives of the expressions that are atoms of the base of . This scheme is repeated until no more expression is produced.
      
      
      \centerline{\begin{tabular}{l@{\ }l}
         & \\
        & \\
        & \\
        & \\
        & \\
        & \\
        & \\
        & \\        
      \end{tabular}}
      
      \begin{minipage}{0.49\linewidth}
      \begin{tabular}{l@{}l}
         & \\
         & \\
         & \\
         & \\
        & \\
         & \\
         & \\
         & \\
         & \\
      \end{tabular}
      \end{minipage}
      \hfill      
      \begin{minipage}{0.49\linewidth}
      \begin{tabular}{l@{}l}
         & \\
         & \\
         & \\
         & \\
         & \\
         & \\
        &\\
         & \\
      \end{tabular}
      \end{minipage}
      
      From the computation of the derivatives, we deduce:
      
      \begin{itemize}
      \item the set  of states :
      
      \centerline{
        , , , ,
      }
      
      \centerline{
        , , ,
      }
      
      \item the function  from  to :
      
      \centerline{
        ,
      }
      
      \centerline{
        ,
      }
      
      \item and the function  from  to :
      
      \centerline{\begin{tabular}{|c||c|c|c|c|c|c|c|c|}
      \hline
        &  &  &  &  &  &  &  & \\
      \hline\hline
       & 
        \begin{tabular}{c}
          \\
          \\
          \\
          \\
          \\
          \\
          \\
        \end{tabular}
         &
         
         &
         
         &
         
         &
         
         &
         
         &
         
         &
         
         \\
         \hline
       &  &  &  &  &  &  &  & \\
      \hline
      \end{tabular}}      
      \end{itemize}      
      
      Let us notice that in this example, substituting  to  in  would  produce a smaller automaton.
    \end{example}
    
    

  
\section{Conclusion}
This paper provides two main results.
First, the theoretical scheme of derivations {\it via} a support 
allows us to formalize intrinsic properties of (unrestricted) regular expression derivations.
As a by-product we obtain a kind of unification of the classical derivations that
compute word derivatives, partial derivatives or extended partial derivatives. 
Secondly, the notion of base function that associates a boolean formula with a regular expression
allows us to show how to deduce an alternating automaton equivalent to a given regular expression
from the set of its derivatives {\it via} a given support.
We are now investigating new features: it is possible, for example,
to define morphisms from one support to another one
in order to study the relations between the associated automata.
An other perspective is to replace the derivation mapping by an other mapping (right derivation
or left-and-right derivation for example, or any transformation with good properties).
There also exist well-know algorithms to reduce boolean formulas (Karnaugh, Quine-McCluskey);
we intend to investigate reduction techniques based on derivation.
Finally we intend to extend the theoretical derivation scheme in order to handle the derivation of regular expression 
with multiplicities.
 
\bibliography{biblio}
  
    
\end{document}
