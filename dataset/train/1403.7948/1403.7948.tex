
\documentclass[final]{dmtcs-episciences}

\usepackage[utf8]{inputenc}
\usepackage{caption}
\usepackage{subcaption}
\usepackage[round]{natbib}
\usepackage{amsmath,url}
\usepackage{amsfonts}
\usepackage{amssymb}
\usepackage{verbatim}
\usepackage{graphicx} 
\usepackage{color,multirow,multicol}
\DeclareMathOperator{\union}{\mathop{\cup}}
\newcommand*{\Union}{\bigcup}
\newtheorem{theorem}{Theorem}
\newtheorem{lemma}[theorem]{Lemma}
\newtheorem{corollary}[theorem]{Corollary}
\newtheorem{remark}[theorem]{Remark}
\newtheorem{proposition}[theorem]{Proposition}
\newtheorem{fact}{Fact}



\newcommand\mar[1]{\textcolor{black}{#1}}
\newcommand\ces[1]{\textcolor{black}{#1}}
\newcommand\fer[1]{\textcolor{black}{#1}}
\newcommand\tur[1]{\textcolor{black}{#1}}







\author{Ferhat Alkan\affiliationmark{1}
\and T\"{u}rker B{\i}y{\i}ko\u{g}lu\affiliationmark{2}
\thanks{Author is also partially supported by TUBA GEBiP/2009 and ESF EUROCORES TUBITAK Grant 210T173.}
\and Marc Demange\affiliationmark{3}
\and Cesim Erten\affiliationmark{4}
\thanks{Corresponding author. Part of this work was done while the author was visiting CIPF, Valencia. The author is also partially supported by TUBITAK-BIDEB Grant 1059B191501053.}}


\title{Structure of conflict graphs in constrained alignment  problems and algorithms\thanks{This work was partially supported by TUBITAK grant 112E137.}}

\affiliation{Division of Oncogenomics, The Netherlands Cancer Institute, Amsterdam, The Netherlands\\
2. Cadde, 12/9, 06500, Ankara, Turkey\\
School of  Science, RMIT University,
Melbourne, Australia\\
Computer Engineering, Antalya Bilim University, Antalya, Turkey}

\keywords{Graph algorithms, graph alignment, constrained alignments, conflict graph, maximum independent set, protein-protein interaction networks, functional orthologs, -free graphs}
\received{2017-8-14}
\revised{2019-7-13}
\accepted{2019-8-6}
\begin{document}
\publicationdetails{21}{2019}{4}{10}{3857}




\maketitle

 \begin{abstract}
We consider the constrained graph alignment problem which has applications in biological network analysis. 
Given two input graphs , two vertices \mar{ of  paired respectively to two vertices  of }  induce an  \emph{ edge conservation} if \mar{ and  are adjacent in their respective graphs}. The goal is to provide a one-to-one mapping between \mar{some} vertices 
of the input graphs in order to maximize edge conservation. However the allowed mappings are restricted since each vertex from  (resp. ) is allowed to be mapped to at most  (resp. ) specified vertices in  (resp. ). Most of the results in this paper deal with the case  which attracted most attention in the related literature.
We formulate the problem as a maximum independent set problem in a related   {\em conflict graph} and investigate structural properties of this graph in terms of forbidden subgraphs. We are interested, in particular, in excluding certain wheels, fans, cliques or claws (all terms are defined in the paper), which in turn corresponds to excluding certain cycles, paths, cliques or independent sets in the neighborhood of each vertex. Then, we investigate algorithmic consequences  of some of these properties, which illustrates the potential of this approach and raises new horizons for further works. In particular this approach allows us to reinterpret a known polynomial case in terms of conflict graph and to improve known approximation and fixed-parameter tractability results through efficiently solving the maximum independent set problem in conflict graphs. Some of our new approximation results  involve  approximation ratios that are functions of the optimal value, in particular its square root; this  kind of results cannot be achieved for maximum independent set in general graphs.  
\end{abstract}

 
 
\section{Introduction}\label{sec:intro}
The  \emph{ graph alignment} problem has important applications in biological network alignment, in particular 
in the alignments of protein-protein interaction (PPI) 
networks~(\citet{AbakaBE13,AladagE13,sharan06,ZaslavskiyBV09,beams13}). 
Undirected graphs  (not necessarily connected) correspond to PPI networks 
for a pair of species, where  the vertex sets  represent the sets of proteins, and 
 represent the sets of known protein interactions pertaining to the   
networks of species under consideration. 
The informal goal is to find \mar{similar patterns between two PPI networks by identifying} a one-to-one mapping 
\mar{between some vertices of}  and  that maximizes the "similarity" of the mapped proteins, usually
scored with respect to the aminoacid sequence similarity and 
the conservation of interactions between mapped proteins. Functional orthology is an important application that
serves as the main motivation to study the alignment problems as part of a comparative analysis of PPI networks.  
\fer{A successful protein interaction network alignment across multiple species could provide a basis for deciding the proteins with similar functions, which may further 
be used in predicting functions of proteins with unknown functions or in verifying those with known functions, in detecting common orthologous pathways between species, or in reconstructing the evolutionary dynamics~(\citet{pmid28194172})}. 

A graph theory problem related to the biological network alignment problem is that 
of finding the  \emph{ maximum common edge subgraph} (MCES) of a pair of graphs, a problem 
commonly employed in 
the matchings of 2D/3D chemical structures~(\citet{Raymond02maximumcommon}). 
The MCES of two undirected graphs 
 is a common subgraph (not necessarily induced) that contains the largest number of edges common to both 
and . The NP-hardness of the MCES problem proposed in~\citet{GareyJ79} trivially implies 
that the biological network alignment problem is also NP-hard. 

A specific version of the problem
reduces its size by restricting the output alignment mappings to those 
chosen among certain subsets of protein mappings. The subsets of allowed mappings are assumed 
to be predetermined via some measure of similarity, usually that of sequence similarity~(\citet{AbakaBE13,ZaslavskiyBV09}). 
The  \emph{ constrained alignment} problem we consider herein can be considered as a graph theoretical generalization 
of this biological network alignment problem version. 
Formally, an instance  is defined by  a pair of undirected graphs  and a bipartite graph  with parts  and  \mar{representing possible matching between vertices of  and vertices of }. For , we denote by , the maximum degree in  of vertices from part .  
A  \emph{ legal alignment}  is a matching of , \mar{i.e., a set of independent edges (pairwise non adjacent). An edge   is said to be {\em conserved}, if there is an edge  such that  and  are in , or  and  are in . Then, the edge  is equivalently called conserved and, by definition of a matching, the number of conserved edges of  is equal to the number of conserved edges of }.
The constrained alignment problem is that of finding a legal alignment that 
maximizes \mar{the number of conserved edges in  (or equivalently in )}. 

Several related problems have been studied previously like, 
for instance, the {\em contact map overlap} problem introduced in~\citet{Goldman1999}. The goal  is to maximize the number of conserved edges;
however 
contrary to the constrained alignment problem, no constraint is given in terms of the bipartite graph . Furthermore their problem definition assumes a linear order of the vertices of both 
which should be preserved by the output mapping. 
The problem 
of -\emph{ matching with orthologies}, was introduced in~\citet{Fagnot2008}. Similar to the constrained alignment problem, it is to find a mapping \mar{respecting a set of constraints} represented by a bipartite graph  \mar{but all edges of  are requested to be conserved. 
Assuming  and denoting by  
  for an instance of the problem, where  denotes the maximum degree of graph , the problem of -{matching with orthologies} is shown NP-complete 
even when  and  and  are bipartite,  and , or if  and .} It is linear-time solvable if  and   (see also~\citet{Fertin200990}). \mar{Finally,  
 the problem  \emph{ MAX} considered in~\citet{Fertin200990} is the optimization version of -{matching with orthologies} with the objective to maximize the number of conserved edges. It is almost the same as the constrained alignment problem with  with the additional requirement that every vertex of  is  mapped to a vertex in . We discuss more precisely the relations between these problems in Section~\ref{sec:definitions}}.  
 In~\citet{Fertin200990},   
 only the case  is considered.  
It is shown APX-hard even if  and  (APX-complete if   has bounded degree) and both graphs are bipartite.  They also propose several approximability and 
fixed-parameter tractability results (see~\citet{ausiellobook} and \citet{ParameterizedComplexity} for definitions about approximation and parameterized complexity, respectively). In particular,
they show that the problem can be approximated within ratio 
for even  and ratio  for odd 
. They also show that the problem is fixed-parameter tractable 
on the size of the output assuming ,  is constant and  has a bounded degree.



In this paper, we consider the maximum constrained alignment problem as a maximum independent set  problem in a related  \emph{ conflict graph}, constructed from , and . Our aim is to investigate structural properties of this conflict graph in order to derive efficient  algorithms for the alignment problem. Although  
a conflict graph is also proposed in~\citet{Fertin200990} for , with in particular a fixed-parameter tractability result based on a degree argument, no further structural property is provided. Here, we deepen this approach and strengthen algorithmic results.
Our main results and comparison with known results are given in Tables~\ref{table-structure},  \ref{table-approx} and \ref{table-parameter} \mar{at the end of this section}. 

Table~\ref{table-structure} shows our main structural results: the basic metrics of the graph \-- size and maximum degree \-- in the most general case as well as forbidden subgraphs for the case . Some of these results have direct algorithmic consequences but even those without algorithmic applications are interesting, in particular since they motivate some graph classes for further studies. This is in particular the case for classes of graphs excluding some wheels or fans (related definitions are given in Section~\ref{sec:definitions}).

Table~\ref{table-approx} describes our approximation results that
extend the results in~\citet{Fertin200990} in several ways; it also illustrates  the potential of our approach. For instance,  an analysis of the degree of the conflict graph, generalizing the one in~\citet{Fertin200990},  immediately leads to an approximation ratio for the  general case  with a ratio  when  are constant; it is improved to  if  and  is constant. For the case  and  constant, we propose as well a -approximation as well as a -approximation, where  is the optimal value of instance . To our knowledge such kinds of ratios are totally new for this problem. Finally, one of our structural results gives a  approximation if , improving also the previous known ratios. 

Table~\ref{table-parameter} presents two fixed parameter tractability results with respect to the size of the output. Both extend the results of~\citet{Fertin200990} to more general cases and both are direct consequences of structural results and known maximum independent set results. 

Finally, a last illustration of the potential of the maximum independent set approach is the case where  and  is acyclic. This case was already shown polynomial in~\citet{AbakaBE13}, using a specific dynamic programming method. A structural analysis of the conflict graphs allows to prove the same result and to interpret it as a maximum stable set polynomial case. Moreover it allows us to derive an explicit expression of the related complexity. Table~\ref{table-complex} sums-up all known complexity results for the maximum constrained alignment problem. \mar{Despite being obtained for {MAX} the hardness results also apply to the constrained alignment problem as noticed at the end of Section~\ref{sec:definitions}}.


 
 The paper is organized as follows. Section~\ref{sec:definitions} gives the main definitions, introduces the conflict graph and investigates its first characteristics (size and degree), leading  to first approximation and fixed parameter tractability results.
 Section~\ref{sec:m2is1} is dedicated to the case  that raised the main attention in the literature.
 We first investigate in Subsection~\ref{subsec:struct-approx} some structural properties   of the conflict graph in terms of forbidden subgraphs (wheels and fans and cliques and claws) with their algorithmic consequences. This part constitutes our main contribution. Then, in Subsection~\ref{subsec:acyclic}, we revisit the case where  and  is acyclic. Finally Section~\ref{sec:conclude}  discusses further research directions.
 



\begin{center}
\begin{table}[ht]
\centering
\footnotesize
\begin{tabular}{|c|c|c|c|c|}

\hline 
 &\multicolumn{1}{|c|}{}& \multicolumn{3}{|c|}{}\\
\hline
 &\multicolumn{3}{|c|}{}& \\
\hline
 and & & acyclic&&\\
\hline
\hline
\multirow{4}{*}{Structural }&\multicolumn{1}{|c|}{}&&-free, &-free, \\
\multirow{4}{*}{property}&\multicolumn{1}{|c|}{(Lem.~\ref{vertexlemma})}&Weakly triangulated&\multicolumn{2}{|c|}{(Th.~\ref{lemW8})}\\
\cline{4-5}
&\multicolumn{1}{|c|}{}&(Th.~\ref{th:weaklytriangl})&-free&-free\\
\multirow{3}{*}{of }&\multicolumn{1}{|c|}{}&&\multicolumn{2}{|c|}{(Th.~\ref{Th:F8})}\\
\cline{4-5}
&\multicolumn{1}{|c|}{(Lem.~\ref{deglemma})}& &\multicolumn{2}{|c|}{-free}\\
&\multicolumn{1}{|c|}{} &&\multicolumn{2}{|c|}{(Th.~\ref{lem_clique})}\\
\cline{4-5}
&\multicolumn{1}{|c|}{Bound of  using the first Zagreb Index}&&\multicolumn{2}{|c|}{-free}\\
&\multicolumn{1}{|c|}{(Lem.~\ref{edgelemma})}&&\multicolumn{2}{|c|}{(Th.~\ref{claw})}\\
\hline
\end{tabular}
\caption{Main structural Properties of .}
\label{table-structure}
\end{table}
\end{center}


\begin{center}
\begin{table}[ht]
\centering
\footnotesize
\begin{tabular}{|c|c|c|c|}
\hline 
 &\multicolumn{1}{|c|}{}& \multicolumn{2}{|c|}{}\\
\hline
 &\multicolumn{2}{|c|}{}& \\
\hline
\hline
\multirow{12}{*}{Approximation ratio} &\multirow{6}{*}{}&\multicolumn{2}{|c|}{}\\
&&\multicolumn{2}{|c|}{(\citet{Fertin200990})}\\
\cline{3-4}
&&\multicolumn{2}{|c|}{}\\
&&\multicolumn{2}{|c|}{( constant \-- Prop.~\ref{cor:approxD1})}\\
\cline{3-4}
&( const.,  \-- Prop.~\ref{prop:generalapprox})&&\\
&&\multicolumn{2}{|c|}{(Prop.~\ref{pro:approxsqrt})}\\
\cline{3-4}
&&\multicolumn{2}{|c|}{}\\
&&\multicolumn{2}{|c|}{( constant \-- Th.~\ref{th:approxlog})}\\
\cline{3-4}
&&\multicolumn{2}{|c|}{}\\
&&\multicolumn{2}{|c|}{(Prop.~\ref{cor16})}\\
\hline
\end{tabular}
\caption{Main approximation results ( denotes the optimal value of instance ).}
\label{table-approx}
\end{table}
\end{center}

\begin{center}
\begin{table}[ht]
\centering
\footnotesize
\begin{tabular}{|c|c|c|c|}
\hline 
 &\multicolumn{1}{|c|}{Bounded }& \multicolumn{2}{|c|}{}\\
\hline
 &\multicolumn{3}{|c|}{Bounded }\\
\hline
 and & \multicolumn{2}{|c|}{Bounded degree}&Any degree\\
\hline
\hline
\multirow{2}{*}{Parameterized tractability} &FTP&FTP&FTP\\
&(Prop.~\ref{prop:parameter1})&(\citet{Fertin200990})&(Prop.~\ref{cor14})\\
\hline
\end{tabular}
\caption{FTP results parameterized by the size of the output}
\label{table-parameter}
\end{table}
\end{center}


\begin{center}
\begin{table}[ht]
\centering
\footnotesize
\begin{tabular}{|c|c|c|c|}

\hline 
 &\multicolumn{3}{|c|}{}\\
\hline
 &\multicolumn{3}{|c|}{}\\

\hline
\multirow{2}{*}{ and }&\multicolumn{2}{|c|}{ 
Even bipartite}&\multirow{2}{*}{ acyclic}\\
\cline{2-3}
&Any degree & Bounded degree&\\
\hline
\hline
\multirow{2}{*}{Complexity}  &APX-hard&APX-complete&Polynomial\\
&\multicolumn{2}{|c|}{(\citet{Fertin200990})}&(\citet{AbakaBE13} and~Subs.~\ref{subsec:acyclic})\\
\hline
\end{tabular}
\caption{Complexity of the constrained alignment problem}
\label{table-complex}
\end{table}
\end{center}





\newpage
 

 \section{Definitions and first remarks}\label{sec:definitions}

\subsection{Main definitions and the considered problem}\label{sub:def}

For all graph-theoretical definitions not given here, the reader is referred to~\citet{golumbicbook}. \mar{A  \emph{ matching} in a graph is a set of independent edges, i.e., pairwise non adjacent. The extremities of the edges in the matching are called  \emph{ saturated}.}
For any ,  denotes a path with  vertices ( \emph{ t-path}),  denotes a  cycle with  vertices ( \emph{ t-cycle}) and  denotes a clique with  vertices ( \emph{ t-clique}). \mar{A  or a  will be denoted as list of successive vertices like . In the case of a -path  and  are the extremities while, in the case of a -cycle,  is any vertex and the order correspond to one of the two possible orientations of the cycle. Sometimes, when a confusion is possible, the -cycle will be denoted  to distinguish it from a -path.}
Denote the complement of  with . An  \emph{ induced subgraph} of  is a subgraph of  induced by a subset of vertices, . It will be denoted by . Given a graph ,  will be called {\em -free} if it does not have any induced subgraph isomorphic to . A partial graph of  is a graph  with  and a partial induced subgraph is a partial graph of an induced subgraph. For a vertex  we will denote by  its (open) neighborhood and by  its close neighborhood. For any vertex  we will denote by  the subgraph induced by  and its neighborhood. For a vertex ,  is its degree in . When no ambiguity may occur, we simply denote  instead of .

A graph is called {\em weakly triangulated} if it is -free and -free, for .


For , a {\em wheel}  is a graph consisting of a -cycle  with an additional vertex, called {\em center}, adjacent to all the vertices of the  cycle . A fan graph  consists of a path  with  vertices and a new vertex  
that is adjacent to all the vertices of the path. As a consequence, a graph  is -free (resp. -free) if and only if, for every vertex ,  is -free (resp. -free).

An {\em independent set} is a set of  pairwise non adjacent vertices, i.e., they induced a graph without any edge. 
 Given a graph ,  denotes its independent number, i.e., the maximum size of an independent set in . Consider a graph class  and a polynomial algorithm 
 determining, for every graph  of a graph class, an independent set of size , is said to guarantee the approximation ratio of , for a function , on  if:
 
 
 Polynomial approximation algorithms are defined similarly for other graph maximization problems. \mar{If an algorithm guarantees a ratio that belongs to the class of functions  (resp. ), then we will simply say that the algorithm guarantees a ratio of  (resp. ) or constitutes a - (resp. -)approximation.} The reader is referred to \cite{ausiellobook} for all concepts in approximation not defined here. Throughout the paper we only use natural logarithms, so  stands for .   

\mar{Finally, in Subsection~\ref{sec:anym1m2}, we will use the {\em first Zagreb index} of a graph ; it is denoted .  is defined as the sum of  squares of  degrees of the vertices. It has been extensively studied, in particular for its interest in computational chemistry~(see, e.g. \citet{Zagreb-30years} for an introduction to this index).}\\

\noindent
\mar{The constrained alignment problem is formally defined as follows:}

\mar{
\begin{tabular}{ll}
\textbf{ Input:}& , where  are undirected graphs \\
& and    is a bipartite graph with parts ;\\
&  will be called an {\em instance}.\\
\textbf{ Output:}& A matching  of , called legal alignment;\\
\textbf{ Objective:}& Maximize the number of conserved edges in , or equivalently in , i.e., \\
& the number of pairs , where  or .\\
&
\end{tabular}}

\ces{For the ease of description, the edges of the bipartite graph  will be called  \emph{ similarity edges}.}
\mar{A legal alignment is called {\em minimal} if the removal of any similarity edge in the alignment creates an alignment that conserves less edges. Any legal alignment includes at least one minimal alignment and consequently,  an optimal minimal alignment is an optimal alignment. Therefore, we can restrict ourselves to minimal alignments.}

\mar{We conclude this subsection with few remarks comparing the constrained alignment problem and relaqted problems introduced in Section~\ref{sec:intro}. Note that the conserved edges of  and  as well as their extremities respectively induce  isomorphic partial subgraphs of  and  . So, if  is a complete bipartite graph, then the problem corresponds to finding two isomorphic partial subgraphs of  and  with a maximum number of edges, which is exactly the maximum common edge subgraph.} 
\mar{However, in our case, the bipartite graph  constraints the possible  isomorphisms since a vertex of  (resp. ) can only be mapped to one of its neighbors in . In an applied context, such constraints represent a priori knowledge about the system that makes only some matchings meaningful.}

\mar{The only difference with the problem  \emph{ MAX},  with  
 (\citet{Fertin200990}), is that in this latter problem, the matching  is required to saturate all vertices in , thus defining an injective (one-to-one) mapping from  to . Contrary to the problems considered in~\citet{Fagnot2008,Fertin200990}, our problem is symmetric in . All our results can be equivalently formulated by swapping indexes 1 and 2. When we will assume that one of  has a specific structure, in particular acyclic like in Subsection~\ref{subsec:acyclic}, we can assume without loss of generality that the condition holds for .  Roughly speaking, the problems considered in~\citet{Fagnot2008,Fertin200990} correspond to detecting, in  a specific structure as close as possible to the pattern represented by . Our version however,   aims to detect similar patterns in the two graphs. We believe that both versions make sense for the suggested  applications.}
 
 
 \mar{With the constraint for the solution to define an injective mapping from  to , some instances of {MAX} may have no feasible solution while every instance of the constrained alignment problem has at least one feasible solution. For this reason, \citet{Fertin200990} restrict their problem to the so called  \emph{ trim instances} for which  has a matching saturating , every vertex in  has a degree at least~1 in  and there is no  \emph{ bad edge} in , i.e., an edge that cannot be conserved for any matching of . The constrained alignment problem does not require  the first assumption. Removing bad edges as well as isolated vertices in  can be performed in polynomial time and leads to an equivalent instance. So, we can assume that there is neither bad edge nor isolated vertex in .} 
 
 \mar{Note finally that any -instance of the constrained alignment problem (with ) can be transformed into an  instance of MAX() with the same optimal value by adding to  a set  of  independent vertices and linking, in , every vertex in  to its copy in . This transformation does not modify . In addition, note that, with the restriction that  has no isolated vertex, the alignment problem with  is equivalent to {MAX} problem and if there is no bad edge, then all instances are trim instances for the latter problem. Indeed, if all vertices of   have a degree at least~1 in  and if vertices in  have the degree~1 , then all maximal matchings of the graph  saturate . As a consequence, all known results  for {MAX} also hold for the alignment problem with .} 


\subsection{Conflict graph}\label{subsec:conflict}
 
 \subsubsection{The notion of s and their conflicting configurations}
\ces{For the following,  we will call  some specific 4-cycles   , where 
,  and . These are partial induced 's of the graph , obtained as the union of  and , for the instance .} 
\mar{Throughout the paper, we adopt the following notations to avoid any confusion between the different graphs we will refer to. When referring to s, we will use simple letters from  to  (without indexes) to denote vertices of . A  is then denoted as a list of four vertices, where the two first ones are in  and the two last are in . Letters  (sometimes with indexes) will denote vertices of the conflict graph defined below.}

We say that
two s  \emph{ conflict}, if at least two of their similarity edges are adjacent \mar{but distinct} (then, they cannot coexist in any matching of ).
Let   be a  conflicting with the  , where , , and . 
In the case , we can identify five generic configurations corresponding to the relative position of  with respect to . These possible configurations are shown in Figure~\ref{twoc4}; note that if  or , then only the label in  is represented. In , we have , and the rest of the vertices are all distinct; in this case, we say that it is a  conflict. Analogously, in 
, we have , and the rest of the vertices are all distinct. In , we have
, and the rest of the vertices are all distinct. In , we have , and 
the rest of the vertices  are distinct. Analogously, in , we have , and the rest of the vertices 
 are distinct. So, the number in the name of the conflicting configuration represents the number of vertices the two s have in common. Similar to  conflict, we will refer to , ,  or  conflicts. 

\begin{figure}[t]	   
\begin{center}	   
\includegraphics[width=12cm]{twoc4inconflict_conf.pdf} 
\caption{\sf \ces{Given two conflicting s,  and , all possible conflicting configurations with respect to , when . 
For each configuration, the vertices at the top are  vertices and the vertices at the bottom are  vertices. }}
\label{twoc4}	   
\end{center}	   
\vspace*{-.4cm}	   
\end{figure} 





  

  


For larger , one can also observe all symmetric conflicting configurations obtained by exchanging  and  with similarity edges adjacent on  vertices plus one configuration with two similarity edges adjacent on a  vertex and two adjacent on a  vertex.

\subsubsection{The conflict graph and its independent sets}

\mar{With} a given instance , we \mar{associate} a  \emph{ conflict graph}, \mar{}, as follows. 
 For each , create a vertex in \mar{} and for each pair of conflicting s, create an edge between their respective vertices in \mar{}. 


\mar{We will denote by  the one-to-one correspondence mapping vertices of the conflict graph  to s in .
Thus, 
for any vertex  of the conflict graph,  is} the corresponding ; for instance, if the related  is  with  and , we will write . We 
call  ``the  \mar{associated with} ''. \mar{In Theorem~\ref{Th:F8}, we will need the notation  to denote the set of vertices in  and visited by the  .}  





With this construction of the conflict graph, the constrained alignment problem  reduces to the maximum independent set problem \mar{as stated in the following proposition. This will be illustrated in the example detailed in Paragraph~\ref{subsub:figure}.}




\begin{proposition}\label{prop:reduction_alpha}\mbox{}\\
{\bf{(i)}} There is a one-to-one correspondence (bijective mapping) between independent sets in the conflict graph and minimal alignments in the instance . An independent set of  vertices maps to an alignment that conserves  edges.\\
  {\bf{(ii)}} A maximum independent set of  maps to to an optimal alignment for .\\
   {\bf{(iii)}} The maximum possible number of conserved edges is .
\end{proposition}
\begin{proof}
\textbf{(i)}
 Let  be an independent  set in the conflict graph ; by definition of the conflict graph, the s  are pairwise not conflicting in the graph  and consequently their similarity edges constitute a legal alignment . An edge  is conserved for this alignment if and only if there are two edges  in  and ; in this case  for some .  Since two distinct non conflicting s cannot share an edge of  (neither of ), exactly  edges of  are conserved by this alignment. This also implies that the alignment  is minimal.
 
 Conversely, for any minimal legal alignment that conserves  edges of , the conserved edges are in one-to-one correspondence with non-conflicting s in the graph . Through , these s correspond to an independent set  in .
 
 \textbf{(ii)} Since the one-to-one correspondence transforms an independent of cardinality  set into  an alignment conserving  edges, a maximum independent set maps to an alignment maximising the number of conserved edges.
 
 
 \textbf{(iii)} It follows immediately that the maximum possible number of conserved edges is .
 
\end{proof}



\begin{corollary}\label{cor:reduction}
\mar{Any polynomial approximation algorithm for the maximum independent set in a graph   guaranteeing the ratio   can be turned into a  polynomial approximation algorithm for the constrained alignment problem guaranteeing the ratio , where  is the conflict graph associated with the instance .}
\end{corollary}

\begin{proof}
\mar{The conflict graph as well as the mapping   can be computed in polynomial time with respect to the size  of the instance   since it only requires identifying all s and testing the compatibility of every two s. The conflict graph is of polynomial size (details about its size are given in Subsection~\ref{sec:anym1m2}) and it follows immediately from the proof of Proposition~\ref{prop:reduction_alpha}-(i) that, given an independent set of size  in , computing the corresponding minimal alignment that conserves  edges can be done in polynomial. We conclude by using the fact that the maximum possible number of conserved edges is .}
\end{proof}

\mar{Approximation ratios for the maximum independent set problem are usually expressed as functions of the number of vertices and/or maximum degree of the graph instance.  To derive an approximation ratio for the constrained alignment expressed as a function of the instance  will require evaluating the main parameters of the conflict graph. This is the purpose of the Subsection~\ref{sec:anym1m2}.}

\mar{\begin{remark}\label{rem:noninjective}
Several minimal alignments (thus, several independent sets of the conflict graph) may correspond to the same set of conserved edges.
\end{remark}}

 \mar{Consider for instance as the graph  a path  of length 2 and  as the graph  a path . If similarity edges are   and , then, the two minimal alignments  and  conserve the same edges  and  of . We give in paragraph~\ref{subsub:figure} another possible situation, where two different alignments correspond to the same conserved edges in  but not in .}

\subsubsection{The underlying graph}

\mar{A direct consequence of Proposition~\ref{prop:reduction_alpha} is that removing from the instance   all -edges, -edges or similarity edges that do not belong to any  does not change the problem in the sense that minimal alignments remain the same. For this reason, we  
 consider the graph  
obtained from the union of , and  by excluding all the vertices and edges that are not part of 
any s. In particular, this includes removing all bad edges~(\citet{Fertin200990}) of  and . We call  the {\em underlying graph} associated with the instance .  It can be seen  as a simplified equivalent instance and consequently, we can always assume that we work on  instead of  or, equivalently, that each edge in  belongs to at least one . In particular, in all our results,  can be seen as the maximum number of similarity edges in  incident to vertices of .}   

\subsubsection{An example}\label{subsub:figure}

\mar{Figure~\ref{sample} gives an example that illustrates the notions of conflict graph, of underlying graph, the function  and the correspondence between minimal alignments in the original instance and independent sets in the conflict graph.  The left chart represents the instance  and the related underlying graph .  is represented on the top, with vertices  and dashed edges and  on the bottom  with vertices  and dotted edges.  Blue edges/vertices correspond to edges/vertices in  that are not part of the underlying graph. So, the underlying graph  appears in black color. In the original instance  but, in the equivalent simplified instance defined by , it becomes 2.}

\mar{The list of s and the related function 
are represented in the middle part of the figure.
Note that  or  are 4-cycles in  but not s. }

\mar{Finally, the related conflict graph is represented on the right hand side. 
This instance has four different optimal solutions corresponding to the minimal alignments  and . They correspond respectively to the independent sets  and  in the conflict graph. Each optimal solution corresponds to two conserved edges in : ,  and , respectively. In this example, these conserved edges correspond to an induced  in the graph  but, in the graph , the related conserved edges which are respectively   and , are not induced subgraphs of  but only partial induced subgraphs. Note finally that the two alignments  and  correspond to the same conserved edges in  but not in . This is another illustration of Remark~\ref{rem:noninjective}.} 


In what follows we provide several graph-theoretic properties of conflict graphs
arising from possible constrained alignment instances under various restrictions. 
Such properties are then employed in applying relevant independent set 
results. 

Throughout the paper we will assume  and  since, in the opposite case, the conflict graph is empty and the maximum alignment problem would be trivial \mar{(the only minimal alignment is empty)}. For a vertex  of , , we will denote by  its degree in .

\begin{figure}[t]	   
\begin{center}	   
\includegraphics[width=12cm]{samplefig.pdf} 
\caption{\sf \mar{An instance  with the underlying graph  and the conflict graph .  and . In the left graph, dashed lines correspond to edges in  while dotted lines correspond to edges in . Blue edges and vertices (left graph) are edges and vertices in  that are not part of the underlying graph and can be ignored. The list of s also defines the function . }
} 
\label{sample}	   
\end{center}	   
\vspace*{-.4cm}	   
\end{figure}


\subsection{General properties of the conflict graph and applications}\label{sec:anym1m2}
In this subsection we first investigate the first basic properties of the conflict graph and deduce first approximation results using some standard results on the maximum independent set problem. For an instance , we denote by  the related conflict graph.



\begin{lemma}
\label{vertexlemma}
Given an instance  with conflict graph , the number  of vertices of  satisfies:\\ 
{\centerline{.}}
\end{lemma}
\begin{proof}
Consider a similarity edge , .  
The edge  can belong to at most  different s. Consequently the number of possible  s satisfies: 

Since  has at most  incident edges in  and  we deduce:  
 

Similarly we have:
 
which concludes the proof.
\end{proof}


Given an independent set in , \mar{Proposition~\ref{prop:reduction_alpha} states that} all similarity edges involved in the related s constitute a matching. Consequently, 

the optimal value  can be bounded using Lemma~\ref{vertexlemma} with  and . This leads immediately to the following bound:


\begin{corollary}\label{cor:alpha}
Given an instance  with conflict graph , the independence number of   satisfies:\\ 
{\centerline{.}}
\end{corollary}

The following lemma generalises the 
bound for degrees provided in~\citet{Fertin200990} for the case where .


\begin{lemma}
\label{deglemma}
Given an instance  with conflict graph , let  be a  corresponding to a vertex  in  , then the degrees in  satisfy:

\noindent

\end{lemma}
\begin{proof} 
\textbf{(i)}
Denote the set of 
s in  
conflicting with  
with , where  is the set of s in conflict with  that include  or , and  consists of all other s conflicting with .  
It is clear that, if a  from  shares the edge  with , it must also include either  or  in order to create a conflict with . In any case, since the total number of valid similarity edges (edges that can create the conflict with ) incident to  and  ( and ) is bounded by , this implies that  is upper-bounded by . 
For the second set , we first note that a pair of similarity edges can create only one . 
This implies that any edge in 
different from 
can be part of at most  different s in  and any edge in  different from  can be part of at most 
different s in .
Since the number of  edges incident to  or , and different from  is 
 , and respectively the number of  edges incident to  or , and different from  is 
at most , the number of s in  that do not include  or 
is bounded by .
The edges  and  themselves can be part of at most  and  different s in  respectively, which concludes the proof of \textbf{(i)}. \textbf{(ii)} is immediately deduced since   and .
\end{proof}


\mar{When evaluating the number of edges of the conflict graph, the first Zagreb index of the graphs  and  appear naturally, as stated in the following lemma. Note that if  (resp., ), then the bound only depends on  (resp., ).} 


\begin{lemma}
\label{edgelemma}
Given an instance  with conflict graph , the number  of edges of  is bounded by: 

\end{lemma}
\begin{proof}
We have .
Using Lemma~\ref{deglemma} and the fact that  (resp., ) participates to at most  (resp., ) s we get:

We conclude by noting that    and similarly for  in the graph . \end{proof}

\mar{If we want a bound for  only dependent on the degree, number of vertices and edges of , then  several upper bounds exist for the first Zagreb index.  We mention here two of these bounds.}

\begin{theorem}\label{th:zagreb-bound} Given a connected graph  with maximum degree  and minimum degree~,\\
\textbf{ (i)}~(\citet{Zagreb-Liu}) ;\\
\textbf{ (ii)}~(\citet{Zagreb-Tabar})
.
\end{theorem}


Note that the bound  is trivial for all graph  and with maximum degree . This bound meets the two bounds in Theorem~\ref{th:zagreb-bound} for regular graphs. In Subsection~\ref{subsec:acyclic}, we will consider the class of acyclic graphs. For this class ( and ), the bound \textbf{ (i)} immediately gives , thus twice better than the trivial bound. Note also that in the case where one of these graphs has much less edges,   or , then a direct application of Lemma~\ref{vertexlemma}, using  , can give better bounds. 

\mar{Lemma~\ref{edgelemma}  and Theorem~\ref{th:zagreb-bound} will be used in Subsection~\ref{subsec:acyclic}. Below we provide direct consequences of Lemmas~\ref{vertexlemma} and~\ref{deglemma} leading to the design of polynomial-time approximation algorithms for the constrained alignment problem.}





The best known approximation ratios guaranteed by polynomial algorithms   for the maximum independent set problem are  ~(\citet{approx-stable}) and ~(\citet{Boppana92}), where  and  denote respectively the maximum degree and the number of vertices of the input graph. 
Combining it with
Lemmas~\ref{vertexlemma} and~\ref{deglemma}   leads to the following approximation for the general setting.

\begin{proposition}\label{prop:generalapprox}\mbox{}\\
\textbf{ (i)} For any positive constant , , 
the constrained alignment problem can be approximated in polynomial time with an approximation ratio of
;\\
\textbf{ (ii)} If only  (resp. ) is constant, then the constrained alignment problem can be approximated in polynomial time with an approximation ratio of
 (resp. ).
\end{proposition}

It is known that using bounded search techniques~(\citet{ParameterizedComplexity}),
one can find an independent set of size  in a graph  in  time, or return that no such subset exists.
In~\citet{Fertin200990}, this result  is used to show that the constrained alignment problem is fixed-parameter tractable for bounded degree graphs with . Lemma~\ref{deglemma} immediately provides a generalisation for the general setting.

\begin{proposition}\label{prop:parameter1}
Provided that  and  are bounded degree graphs, for any positive constants , , 
the constrained alignment problem is fixed-parameter tractable for parameter  and 
solvable in\\  time, where  is the number of final conserved edges and .
\end{proposition}

In what follows we consider the case  \-- which, \mar{to our knowledge}, is the most studied case \-- and investigate specific properties of the conflict graph. \mar{This case, by itself already very hard, simplifies the possible conflicts and then perfectly illustrates the use of the conflict graph. As explained in the conclusion, the following results motivate the further study of conflict graphs and their independent sets for a more general set-up. } 


\section{The case }\label{sec:m2is1}


The case with  is the main case considered in~\citet{Fertin200990}.  We remind that, in this case,  the possible conflicting configurations are listed in Figure~\ref{twoc4}.  Some improved results deal with the particular case . 
It is 
known that the problem is APX-hard even for the case where
 and both  are bipartite~(\citet{Fertin200990}). 

 


\subsection{Structure of  and approximation}\label{subsec:struct-approx}


In this subsection we present graph theoretic properties of conflict graphs in terms of forbidden subgraphs when . 
In addition to providing valuable information regarding 
structural properties of conflict graphs, it has also algorithmic applications, mainly approximation results. 

Note first that, if , Lemma~\ref{deglemma} states that the maximum degree of the conflict graph is at most  and consequently Proposition~\ref{prop:generalapprox} can be immediately replaced by:

\begin{proposition}\label{cor:approxD1}
For  and any positive constant , the constrained alignment problem can be approximated in polynomial time with an approximation ratio of
.
\end{proposition}
 
This approximation ratio in  improves the result of~\citet{Fertin200990} \-- 
for even  and  for odd 
 \-- also obtained for . We will give later another improvement in the case where  is less than this ratio. 

\mar{We first establish some  properties of conflict graphs when  \-- Facts~\ref{onenode} and~\ref{twonodes}, Lemmas~\ref{lem_H} and~\ref{lem:5nodes} and Corollary~\ref{lem_P4} \--  that will be useful for the main structural and algorithmic results. Then, in paragraphs~\ref{subsub:wf} and~\ref{subsub:cc}, we derive structural results and their algorithmic consequences. }


\begin{fact}
\label{onenode}
Any pair of conflicting s in  must share at least one vertex from .
\end{fact}

\begin{fact}
\label{twonodes}
Any pair of distinct s in  sharing two vertices from  has a conflict.
\end{fact} 


\begin{lemma}
\label{lem_H}
Given an instance  with conflict graph , suppose  and consider an induced subgraph   of  such that  is connected and  has an induced .
Then the s in  cannot all share a vertex from .
\end{lemma}

\begin{proof}
Let \mar{} be an induced  in   and let . 
Assume for the sake of contradiction that  is a vertex common to all the s \mar{associated with vertices of} . For every two vertices  in  not linked by an edge,  and  must share the similarity edge including  to avoid any conflict. As a consequence and since   is connected, all 
the s \mar{associated with vertices of}  must share the edge . 
This implies that  any conflict between any pair of these s can only be either a
 or a  conflict, which further implies that all the s  include  .
By Fact~\ref{twonodes} and since , this implies a conflict between  and , a contradiction.\end{proof}

For instance, a  or  \-- the independent union of a  and an isolated vertex \-- clearly both satisfy the conditions on : they both have an induced  and moreover,  is a  as well while  is a triangle with a pendent vertex, both connected. So, we immediately deduce:

\begin{corollary}
\label{lem_P4}
Given an instance  with conflict graph , if , the four s of an induced  or an induced  of  
cannot all share a vertex from .
\end{corollary}


The following lemma will be useful for studying the structure of .

\begin{lemma}\label{lem:5nodes}
Given an instance  with conflict graph , suppose  and that  we have in  an induced   as well as two vertices  not linked to  and an additional vertex   linked to the seven vertices , . Denote .
Then if  does not include , neither does . 
\end{lemma}

\begin{proof}
Since  are all linked to , Fact~\ref{onenode} ensures the related s include   or . 
Assume for the sake of contradiction that  does not include  while   \mar{does}. Since  conflicts \mar{with} , we have   with  and . Let  , . Since ,  are all paire wise distinct.

As mentioned above  must include  or . Since  do not conflict with  nor with , if it includes  it must include the edge 
and if it includes  it must include the edge . Moreover, none of them can include both  and . Indeed, in this case  
 for some  and since any ,  , can neither include an edge ,  nor , , it cannot conflict with , a contradiction.

On the other hand, since  has a conflict with both  and  and since   and  are not conflicting, there must be two similarity edges , , , , where  is an edge of  and  is an edge of both  and . Since , one of them includes the edge  and the other includes the edge .

We consider below the possible cases that all lead to a contradiction.

 \emph{Case-1:}
Suppose  and , thus   
 is either  or . 

  \emph{Case-1.1:} If , then since  conflicts with  but not with  it must include the edge  but in this case it would conflict with .
 
  \emph{Case-1.2:} Similarly if , then since  conflicts with  but not with  it must include the edge  but in this case it would conflict with .

  \emph{Case-2:}
Suppose now   and , thus   
 is either  or .
In both cases we get the same contradiction as in Case-1 exchanging the roles of  and . This concludes the proof.
\end{proof}

\subsubsection{Wheels and Fans}\label{subsub:wf}

\begin{theorem}
\label{lemW8} 
Given an instance  with conflict graph ,\\
\textbf{ (i)}
If ,  is -free, for ;\\
\textbf{ (ii)} If furthermore ,  is also  and -free. 
\end{theorem}

\begin{proof}
Assume for the sake of contradiction an induced  exists with   and 
let  be the center vertex with . Let 
be the induced  of the wheel  in the conflict graph.
By Fact~\ref{onenode}
every  must include at least one of 
 or . By Corollary~\ref{lem_P4} (the cycle  has an induced ), it is not possible for all of \mar{these s} to share 
, nor can they all share . This implies that there must exist 
a pair of conflicting s, , such that their corresponding vertices in  are neighbors in , one including  and the other including  and one of them does not contain both  and . 
Without loss of generality, let the former be   with  and the latter be . 



\textbf{(i)}
Assume first . Then apply Lemma~\ref{lem:5nodes} with  and  gives a contradiction.

\textbf{(ii)} Now we show directly that there is also a contradiction if  and . 

We consider two cases ,  and ,  ensuring the conflict between  and . In both cases  ensures the conflict between  and .

 \emph{Case-1:} . Since  have no conflict with  but with , they both include  and not . Moreover, since  does not conflict  , it cannot include  and thus .  Since    conflicts with both  and  we have ,  and . Then, since  conflicts with 
but not with , it must include the edge . To conflict with  it should include  or , a contradiction since .


 \emph{Case-2:} , .

 Since  conflicts with  and with  and since ,  cannot include  and thus includes  and  .

 conflicts with  but not with  and includes   or .
Since  the only possibility is 
. 
Then,  conflicts with  but not with  and includes   or ; the two only candidates are  and , both impossible since  (note that ). It concludes the proof.
\end{proof}

\begin{figure}[t]	   
\begin{center}	   
\includegraphics[width=12cm]{W7sample_new.pdf} 
\caption{\sf Sample configurations for s inducing  (left) and  (right) in their respective 
conflict graphs
for the case where .
The central vertices of the wheels in each case correspond to the s  
indicated with .
} 
\label{w7sample}	   
\end{center}	   
\end{figure}  

Note that for , it is still possible to have a  and 
 in a conflict graph  as illustrated in Figure~\ref{w7sample}. Note also that   and  can still exist in  even if . Figure~\ref{cor8fig} gives a sample construction with a  while Figure~\ref{cliquesample} gives an example with a . 
It means that, in terms of induced wheels, Theorem~\ref{lemW8}  leaves no gap. 


  
  \begin{figure}[t]     
\begin{center}	   
\includegraphics[width=12cm]{cor8fig_new.pdf} 
\caption{\sf \textbf{Left:} Sample construction for a  in . The central vertex of the wheel corresponds to the   . The upper partition corresponds to vertices of  and the lower partition to those of . Similarity edges are drawn between the partitions. \textbf{ Right:} Depiction of the construction defined in Lemma~\ref{lem:F8}. The vertex   is shown in the center and vertices in  are shown at the peripheral.  corresponds to all vertices in components of  of size~. The black vertices constitute a maximum independent set of . 
} 
\label{cor8fig}	   
\end{center}	   
\end{figure}  


\begin{figure}[t]	   
\begin{center}	   
\includegraphics[width=14cm]{f5andf7sample.pdf} 
\caption{\sf \textbf{ Left:} Sample configuration for  inducing
 for the case where . 
\textbf{Right:} Sample configuration for  inducing  
for the case where . In each case 
the central vertex corresponds to the   
indicated with . 
Each  edge is marked with the related   in  (left) or  (right). 
} 
\label{f5and7sample}	   
\end{center}	   
\end{figure}



The following lemma gives an example how considering the different kind of conflicts, for  and , (see Figure~\ref{twoc4}) helps understanding the structure of the conflict graph. 

\begin{lemma}\label{lem:F8}
Given an instance  with conflict graph , suppose  and  and consider a vertex  in  and the set  of s that conflict  with a  or  configuration. Then,  is an independent collection of s, s, s and isolated vertices.\end{lemma}

\begin{proof}
Since , in , at most two s   can conflict a fixed  and consequently the graph  has degree at most~2, which means it is an independent  collection of cycles and paths. For any , consider a connected component of  of size~. 

Assume we have  in  with edges  and . Since   and   cannot all include  and neither can they all include .  Suppose without loss of generality that two of them include  and one  and in this case, the structure of conflicts  and  imposes that  includes , say  with . Suppose then without loss of generality that  includes  and  includes : ,  and necessarily  to create a conflict with . Moreover, since , .

Note then that we cannot have any conflict between  and , which means that  is triangle-free. Moreover suppose a fourth ,  conflicting  in . It necessarily includes  and thus conflicts , which means that  is -free, which completes the proof. Figure~\ref{cor8fig} (Right) describes the structure of , where , , is the union of components of  of size~.
\end{proof}





\begin{corollary}\label{polalg1}
Given an instance  with conflict graph , if  and , then for every , removing at most two vertices to  makes it an independent collection of s, s, s and isolated vertices. 
\end{corollary}
\begin{proof} 
If , at most one  conflicts   with a    configuration, and 
at most one with   configuration. Let us remove these vertices.  There can be at most one  conflicting  with a   configuration and moreover such a  necessarily \mar{corresponds to} an isolated vertex in . Since all the other neighbors of  correspond to    or  configurations,  Lemma~\ref{lem:F8} immediately concludes the proof.   
\end{proof}

\begin{corollary}\label{rem:degree}
If  and  we are ensured to find in polynomial time a legal alignment with at least 
 conserved edges.
\end{corollary}

\mar{\begin{proof}
It is an immediate consequence of Corollary~\ref{polalg1} applied to a vertex  of maximum degree in . An exhaustive search or just the detection of   and 
 configurations involving  allows to identify the vertices to be removed to make  an independent collection of s, s, s and isolated vertices. Picking in this collection an independent set of two vertices in each   s and , one vertex in each  and all the isolated vertices gives a independent set of size at least~ in . Using Proposition~\ref{prop:reduction_alpha} and the fact that the function  can be computed in polynomial time (see Corollary~\ref{cor:reduction}) allows to conclude. 
\end{proof}}

The following result concerns the existence of induced fans  in the conflict graph. Note that for ,  is an induced subgraph of  and consequently an -free graph is also -free. 



\begin{theorem}
\label{Th:F8} 
 Given an instance  with conflict graph  such that , then:\\
\textbf{ (i)}
For ,  is -free;\\
\textbf{ (ii)} For ,  is -free.
\end{theorem}



\begin{proof}
Consider an induced  and let  be the center vertex. 
 
 \textbf{(i)} Assume for the sake of contradiction that  and
 denote by \mar{}
be the induced  in the neighborhood of .
By Fact~\ref{onenode}
every   must include at least one of 
 or  in . 

Suppose first . Without loss of generality we assume they both include  and either both include  as well or none of them. Consider then the subgraph induced by , inducing a .
By Corollary~\ref{lem_P4}, the s  and   cannot all include  and let  such that    does not include . Then, Lemma~\ref{lem:5nodes} with  and  and  corresponding to  leads to a contradiction. 

Suppose now , then one only includes  and we get a contradiction as well by applying Lemma~\ref{lem:5nodes} with  and  and  corresponding to , which concludes the proof of \textbf{ (i)}.

\textbf{(ii)}
Assume now . Corollary~\ref{polalg1} immediately shows that it is possible to remove at most two neighbors of  so that  cannot be the center of a . It excludes the possibility of a  in this case. 
 \end{proof}

Figure~\ref{f5and7sample}-Left shows an example of  in a conflict graph with  and  and Figure~\ref{f5and7sample}-Right shows an  in a conflict graph with  and .
 


Theorems~\ref{lemW8} and~\ref{Th:F8} as well as Corollary~\ref{polalg1}  give us information about the structure of the subgraphs , , induced by  : as already mentioned  a graph  is -free (resp. -free) if for all vertex ,  is -free (resp. -free), two classes of graphs that raised a lot of interest from researchers (see, eg., \citet{isgci,graphclassesbook}). 

We give now an example how to use the structure of neighborhoods to approximate the maximum independent set problem. It will give us algorithmic  applications of  Corollaries~\ref{polalg1} and~\ref{lem_P4}. 

A very classical approximation algorithm for maximum independent set in a graph  is the  algorithm {\tt 2-opt} determining an independent set  such that  is not and independent set (there is no 2-improvement). Let us revisit the very usual analysis of {\tt 2-opt} (see, e.g.,~\citet{ejorapprox})  which consists in considering the bipartite graph  induced by , where  is an optimum independent set. Denote by  the value of the solution provided by the algorithm on  and 
 the independent number of . Then the number of edges of  is at least  since 2-optimality ensures that, for every two edges  in  incident to the same vertex   , there is an additional edge incident to  or . On the other hand this number is at most , where  is
the minimum among all optimal independent sets  of the maximum number of vertices in  a vertex can be adjacent to:
  
This implies:

 This remark emphasises that the usual maximum degree can actually be replaced by .
 We propose below a strategy that can be used where large independent sets can be found in polynomial time in the neighborhood of each vertex. It leads to a new kind of approximation ratios depending on the independence number.
 \begin{theorem}\label{th:approx}
Consider a class of graphs  for which there is a polynomial time algorithm  approximating the maximum independent set problem within the ratio  for every graph , where  and  .\\
 Then  the maximum independent set problem can be approximated within .\end{theorem}
 \begin{proof}
 The strategy, for an input graph  in  is as follows: 
 \begin{quote}
  Apply  in all subgraphs ;\\ 
  Compute also a {\tt 2-opt}-solution;\\
 Take the best solution among the  different solutions obtained.
  \end{quote}
  Note first that, if , then {\tt 2-opt} finds an optimal solution, so we assume .
  
Suppose first that . Then, when applied to a graph  such that , the algorithm  computes a solution of value at least  leading to the approximation ratio  .

Suppose now , then Relation~(\ref{eq:2opt}) gives the ratio: 

where the inequality uses . 
In all cases, the ratio is at most ,
 which concludes the proof.
\end{proof}

Given an instance , we denote by  the optimal value of the constrained alignment problem on . 
\begin{proposition}\label{pro:approxsqrt}
Given an instance  with conflict graph  and ,\\
\textbf{(i)} The constrained alignment problem can be approximated within ;\\
\textbf{(ii)} If furthermore , this is improved to  .
\end{proposition}
\begin{proof}



This is a direct application of Theorem~\ref{th:approx}.\\
\textbf{(i)} Consider a vertex  in the conflict graph  and the graph . We denote . Using Fact~\ref{onenode},   
the s in the neighborhood of  in  can be partitioned into  and , where all s in  include  while the others include  but not . This partition can be determined in polynomial time.  
Corollary~\ref{lem_P4} ensures that  and  are -free. It is well known that the maximum independent set problem can be solved in linear time in -free graphs (also called {\em cographs}) (see, e.g., \citet{golumbicbook}). Determining a maximum independent set in  and  and choosing the best one clearly solves the maximum independent set problem in  \mar{} within an approximation ratio of~2. We apply Theorem~\ref{th:approx} with constant .\\
\textbf{(ii)} If , then Corollary~\ref{polalg1} ensures that a maximum independent set can be found in polynomial time in graph  and we apply Theorem~\ref{th:approx} with constant .
\end{proof}

Note that we obtain a ratio depending on the optimal value, which is not usual. Roughly speaking this result means that the logarithmic version of the problem \-- where the objective is to maximise the logarithm of the number of similarities in a legal alignment \-- is -approximable. For instance, such a ratio for the maximum independent set in conflict graphs cannot be achieved in general graphs: the usual -hardness result~ (\citet{hastad}) states that, under some complexity hypothesis, the logarithm of the independence number cannot be approximated within a constant ratio.

Combining Proposition~\ref{pro:approxsqrt} with Corollary~\ref{cor:alpha} leads to the following ratio:


\begin{proposition}\label{cor:approxalpha2}
Given an instance  with conflict graph  and ,\\
{\bf{(i)}} The constrained alignment problem can be approximated within the ratio:\\
{\centerline{;}}
{\bf{(ii)}} If furthermore , this ratio becomes:\\{\centerline{;}}
\end{proposition}
\begin{proof}
\mar{Using the definition of the approximation ratio guaranteed by an algorithm for a maximization problem, any upper bound of a guaranteed approximation ratio is still a guaranteed approximation ratio. Using Proposition~\ref{prop:reduction_alpha}-(iii), the optimal value  of the instance  of the constrained alignment problem  equals the independence number  of the related conflict graph. By Corollary~\ref{cor:alpha}, we deduce  
 
Since the function  is increasing, we conclude the proof using Proposition~\ref{pro:approxsqrt}.}
\end{proof}

\mar{Proposition~\ref{cor:approxD1} states the ratio  in the case  and  is constant.
When  or , the ratio obtained in Proposition~\ref{cor:approxalpha2}-(i) can be better than the ratios we achieved  as functions of the maximum degree. In addition, Proposition~\ref{cor:approxalpha2}-(i) does not require any assumption about . }

Given the known results for the maximum independent set, a natural question is whether the constrained alignment problem is -approximable or even whether any approximation in  can be guaranteed. We give a first answer to this question in Theorem~\ref{th:approxlog} below.
The ratio  gives also a first answer for some classes of graphs satisfying  (but  still large). In particular, if  is acyclic, we have  and consequently:

\begin{corollary}\label{cor: approxacyclic}
Instances of the constrained alignment problem satisfying  and  acyclic can be approximated within the ratio .
\end{corollary}



Let now  be an instance of the constrained alignment problem with conflict graph  and ; suppose we are given a subset  and a maximal matching  of , the subgraph of  corresponding to similarity edges incident to . We denote by  the set of s in   including at least one vertex of  and no similarity edge  with ; in other words, these s include vertices in  but only with similarity edges in . Then, considering the subgraph  of  induced by these s, we have: 

\begin{lemma}\label{lem:P_4-new}
For any induced , \mar{}, in ,  and  have the same neighborhood  in . In particular  is -free.
\end{lemma}

\begin{proof}
 Since  and by definition of , for every two conflicting s in , there must be   a vertex  and two disjoint vertices  such that  is an edge of the former and  an edge of the latter; moreover the other similarity edges of these s are in . Suppose we are given an induce , , in 
 . There are such vertices , where  and  both include the edge  while  includes . Moreover, every  in   that conflicts  \mar{with}  (resp. ) must include a similarity  edge  and thus it conflicts \mar{with}  (resp. ), which concludes the proof.\end{proof}

We deduce the following theorem that gives a first step towards non trivial  approximation ratios. It corresponds to a sequence of approximation algorithms parametrized by , called {\em approximation chain} in~\citet{demange-chain}.

\begin{theorem}\label{th:approxlog}
Consider instances of the constrained alignment problem satisfying  and  constant and let  be a positive constant. One can find in polynomial time a legal alignment guaranteeing the approximation ratio of .
\end{theorem}

\begin{proof}
Consider an instance  verifying the assumptions and denote by  the related conflict graph. We recall that . Denote by  the optimal value for the instance .
Let  be a maximum independent set of , . 
Our strategy is to subdivide the vertex set of the conflict graph, , into  subsets such that the maximum independent set can be solved in polynomial time on the subgraph induced by each part. This subdivision is not necessarily a partition.

Fix a constant  and partition vertices of 
 into  sets of vertices   with . For each of them we denote by   the set of all s in  including at least one vertex of   and by  the graph . Note that: 
 
 

We claim that there is a polynomial-time algorithm that computes, for every , a maximum independent set of .  
Note first that the similarity edges involved in s contributing to any independent set of  form a matching of the graph  and consequently, is part of a maximal matching of this graph. Denoting by  the set of maximal matchings of , we deduce:



Lemma~\ref{lem:P_4-new} ensures that, for any fixed maximal matching ,  is -free. In this case a maximum independent set can be computed in polynomial (linear) time~(\citet{golumbicbook}). The related complexity is  since s in  include at least one edge of  and .  But  is a fixed constant and . Thus, we can exhaustively list all maximal matchings of  in , a polynomial function.  

Our algorithm runs as follows: 
\begin{quote}
\begin{em}
For all  and all maximal matching  of , compute   and a maximum independent set  \-- keep the best such solution. 
\end{em}
\end{quote}
\noindent
Computing each  and a maximum independent set takes, for bounded , ;  the whole complexity is then
, a polynomial function.
 
To complete the proof we need to justify it guarantees the required ratio. Equation~(\ref{eq:union}) ensures that the value  of the computed solution satisfies:

which shows that the related approximation ratio is . 
\end{proof}





\subsubsection{Cliques and Claws}\label{subsub:cc}

Next we present results regarding the existence of cliques as subgraphs of conflict graphs
for any .
Assume that there is a clique , , in  and let a corresponding  associated with a vertex  from this  be .
We partition all the corresponding s in  into three disjoint reference sets with respect to .
Let  consist of all the s respectively conflicting  with a   and  configuration. 
Let  be the set of all s 
with other kinds of conflicts ( ,  or ) with  and  itself.  

\begin{lemma}
\label{lem_edgesharing}
Given an instance  with conflict graph  and the reference sets defined as above, then any pair of s from different reference sets do not share a similarity edge.
\end{lemma}
\begin{proof}
Note that since the pair of s correspond, in , to different vertices of the same clique , they should conflict
by sharing at least one vertex from . 
We consider two cases. For the first case assume one of the s is in  or , and the other is in .
Without loss of generality assume the former  is in  including vertices  and  from , where .
Since the latter  from  includes both  from , the pair of s can only share the vertex  from 
 giving rise to a  or a  conflict between them. 
For the second case assume one of the s is in  and 
the other is in . In this case the former must have a  conflict whereas
the latter must have a  conflict with the reference . Since  
the s from  and  can only share one vertex from , thus giving rise to a  or a  conflict between the pair. 
In both cases we show that both s are in  or  conflict with each other. 
The fact that any pair of s with a  or a  conflict do not share a similarity edge
completes the proof.
\end{proof}


 
\begin{theorem}
\label{lem_clique}
Given an instance  with conflict graph  and , the maximum size of any clique in  is 
, or equivalently  is -free.
\end{theorem}

\begin{proof} 
We consider two cases. 

 \emph{ Case-1:} We first handle the case where at least one of  is empty. 
Assume without loss of generality  is empty. Let  be the number of similarity edges
incident to  in the s of . Since each pair of similarity edges, one incident to  and one
incident to , gives rise to at most 
one , the number of s in  is at most . 
By Lemma~\ref{lem_edgesharing}, s in 
cannot share an edge from  with the s in . This implies that the 
number of similarity  edges incident to  in the s of  is at most . 
Let  be such an edge and let  denote the set of s in  sharing . Since any pair of s from  share a similarity edge, they 
must be in a  or   conflict with each other
and thus must share one more vertex from 
in addition to the vertex . This implies that  which further implies a total of at most  s in . 
The clique consisting of s from  has at most  vertices. 

 \emph{ Case-2:} Now we handle the case where  and  are both not empty. 
It must be the case that all s in  must share a vertex  from 
such that , . This is due to the fact that any pair of s, one from  the other from ,
can only have a  or  conflict and the shared node in \mar{this} conflict cannot be neither  nor . 
Let  be the number of edges from  incident respectively to  and  in the s of . 

The number of s in  is at most . 
By Lemma~\ref{lem_edgesharing}, the 
number of similarity edges edges incident to  in the s of  are at most  and 
the number of similarity edges incident to  in the s of  are at most .
Let  be the number of 
similarity edges incident to  in the s of . Again by Lemma~\ref{lem_edgesharing},
the number of similarity incident to  in the s of  are at most .
This implies that the maximum number of s in  and  are respectively 
and . The size of the clique consisting of 
s from all three reference sets 
is at most , where .
Without loss of generality let . Then we have
.
\end{proof}

\begin{figure}[t]	   
\begin{center}	   
\hspace*{-.3cm}
\includegraphics[width=12.5cm]{newK4K9_new.pdf} 
\caption{\sf Sample s giving rise to s in their
respective conflict graphs.
The reference  is . The first two  show sample constructions for  and the last for .
The employed reference sets as
described in the proof of Theorem~\ref{lem_clique} are as follows:  \emph{ (Left)} All s are in , 
 \emph{ (Middle)} s in  are those induced by black vertices and , s in  are those induced by 
white vertices and , 
 \emph{ (Right)} s in  are those induced by black vertices and , s in  are those induced by 
white vertices and . 
} 
\label{cliquesample}	   
\end{center}	   
\end{figure}  


We note that  is possible in a conflict graph  
for any positive integer . Indeed  \emph{ Case-1} of the above proof provides
an actual construction method; see Figure~\ref{cliquesample}.


Note that under the setting of , the size of  is bounded by  (Lemma~\ref{vertexlemma}).
It is known that 
the maximum independent set problem is fixed-parameter tractable, parameterized by the size of the 
output, in the class of -free 
graphs for constant integer ~(\citet{DBLP:conf/swat/RamanS06,DabrowskiLMR12}). Combining this result
with Theorem~\ref{lem_clique}, leads to the following result:

\begin{proposition} 
The constrained alignment problem is fixed-parameter tractable when  is any fixed 
positive integer constant and 
. 
\label{cor14}
\end{proposition} 

Note that the analogous result in~\citet{Fertin200990} is more restrictive since it applies only to the bounded degree
graphs.



We conclude this part by considering induced claws in conflict graphs. A  \emph{ d-claw} is an induced subgraph of an undirected graph, 
that consists of an independent set of  vertices, called , and 
the  vertex that is adjacent to all vertices in this set. 
Let .

\begin{theorem}
\label{claw}
Given an instance  with conflict graph  and , then\\  is -claw-free.\end{theorem}

\begin{proof}
Let  be the corresponding  associated with the center vertex of a claw. 
Let  be the  corresponding to a talon that has a ,  or   conflict with . Since 
any other  corresponding to a talon with a ,  or  conflict with  would also 
have to share the vertices , by Fact~\ref{twonodes}, 
it would conflict with , which is not possible. Thus,
the total number of talons the s of which create a ,  or  conflict with 
is at most 1. With regards to the number of talons corresponding, in , to  a   or  
conflict with , we first count the maximum number  of possible  conflicts. 
Let  be the  of a talon with a   conflict with . 
Any talon the  of which conflicts \mar{with}  \mar{through} a  conflicting configuration  must share the 
edge , since otherwise it would conflict with . 
Any  edge incident to vertex  can belong only
to a single  since otherwise by Fact~\ref{twonodes}
there would be a conflict between a pair of s 
corresponding to talons. In addition, since ,
every  edge can belong only to a single . Thus
the number of talons inducing in   conflicts is bounded by
. The same holds for  conflicts
giving rise to at most  talons that 
are independent.
\end{proof}

The above theorem in conjunction with the result of~\citet{Berman00} which states 
that a  approximation for maximum independent sets can be found 
in polynomial-time for -claw free graphs gives rise to 
a polynomial-time approximation for the constrained alignment problem. 

\begin{proposition}  
If , the constrained alignment problem can be -approximated in polynomial time.
\label{cor16}
\end{proposition}

\mar{This results improves (by at least a factor ) the approximation ratio of 
for even  and  for odd 
 proposed in~(\citet{Fertin200990}). As already mentioned, the -approximation stated in Proposition~\ref{cor:approxD1} already improved it. If , then the ratio in Proposition~\ref{cor:approxD1} is better but if , then the ratio established in  Proposition~\ref{cor16} can be better than the one in Proposition~\ref{cor:approxD1}.}

\mar{We conclude Subsection~\ref{subsec:struct-approx} by emphasizing that some of our structural results lead to a new hardness result for the maximum independent set problem.  Indeed, the combination of Lemma~\ref{deglemma}, Theorem~\ref{lemW8}, Theorem~\ref{Th:F8} and Theorem~\ref{Th:F8} states that, for any instance of the constrained alignment problem with  and  are of bounded degree, the related conflict graph is 
-free and of bounded degree.}

\mar{On the other hand,  the constrained alignment problem is shown to be APX-complete  even for the case where
, both  are bipartite and of  bounded degree (\citet{Fertin200990}).  As a consequence, we derive the following new hardness result for the maximum independent set problem:}

\begin{proposition}
The maximum independent set problem is APX-complete in the class of bounded degree, -free graphs. 
\end{proposition}



\subsection{Acyclic  and }\label{subsec:acyclic}

We conclude by investigating the case where   is acyclic and   for which the constrained alignment
problem is shown to be polynomial-time solvable in~\citet{AbakaBE13} without a precise complexity analysis. We refine this previous analysis by showing that in this case the conflict graph has a very particular structure. More precisely it is {\em weakly triangulated} (-free and -free, for ). Weakly triangulated graphs are known to be perfect~(\citet{hayward}) and moreover the maximum independent set problem can be solved in  in a graph ~(\citet{hayward2}). It allows us to deduce a new polynomial-time algorithm for this case with its complexity analysis. This illustrates again how the structure of the conflict graph can be used to achieve algorithmic results. 

We need two technical lemmas; remind that, given an instance  the graph   is defined in Subsection~\ref{subsec:conflict}.  
\begin{lemma}
\label{edgesharing}
Given an instance  with conflict graph  where  is acyclic.
Suppose a , denoted by , is an induced subgraph of the conflict graph . For , the 
s of  corresponding to the 
end vertices of  neither share a vertex nor an edge in . 
\end{lemma}

\begin{proof}
Suppose first that  is an induced  ,  in the conflict graph and consider the two s in  \mar{associated with} the extremities of . They can neither
share an edge from   nor a vertex from  without sharing a similarity edge incident to it ().
They also cannot share an edge from  nor a vertex from  without sharing a similarity edge incident to it since otherwise they would conflict.
Thus we simply need to show that they do not share 
a similarity edge.

The proof is by \mar{strong} induction on . For the base case  , suppose there is a  \mar{} in the conflict graph and that 
the s  and  share a similarity edge. Let   and 
let   with the edge  in common. There are two cases for 
. Since it does not conflict with , it  must either be of the form , where  ( conflict with ) or of the form  where  ( conflict with ). 
Now considering , to create a conflict with , 
one edge of  must be  where . Placing the other edge of  from \mar{} such that 
it creates a conflict with  is now impossible, since it either gives rise to a cycle in  (cycle \mar{} or \mar{}, )  or creates a conflict with .

For the inductive part, assume that the lemma holds for all  where . Consider the s of  corresponding to the vertices of a , \mar{} in the conflict graph. 
Let  and . By the inductive hypothesis, these two s are disjoint. Consider in 
 the subset  of edges from  that belong to the s  associated with vertices in the  , .  contains in particular  and . Edges in  form a connected subgraph  of  and without loss of generality we assume that the shortest path  between  and   contains neither  nor . This path  has at least one edge; let its last edge be  which is part of  for some , . Let   and . If at least one of  is on the path, say , and , , then  must be one of  or , since  must conflict with , which implies a cycle in .  If  then  to create a conflict with  without creating a cycle in .   
This implies a conflict between  and , which is impossible since  
 is an induced path. Finally, if , ,  and  do not share a similarity edge, which concludes the proof.
\end{proof}

The subgraph of  that corresponds to an induced , \mar{} in the conflict graph 
is said to be in  \emph{ chain configuration} if each , , , shares only a distinct -vertex with the next , , if  and one with the previous one, , if 
 and does not share any - or -vertices with any other of these s; see Figure~\ref{chain} for a sample chain configuration. 
Note that a chain configuration imposes a certain order of the involved s in . 

\begin{figure}[t]	   
\begin{center}	   
\includegraphics[width=8cm]{chainconf_new.pdf} 
\caption{\sf Chain configuration of a -path in . } 
\label{chain}	   
\end{center}	   
\end{figure}  


\begin{lemma}
\label{threeinchain}
\mar{Let  be an instance of the constrained alignment problem with conflict graph  and acyclic . Let  be three vertices of  such that  and  do not share a vertex nor an edge in  and  conflicts  with both  and . Then  and  must be in chain configuration where  is in the middle in any left to right order.}  
\end{lemma}
\begin{proof}
If 
the conflict configuration of  and  were of  or , then  could conflict with 
  only if it 
shared a vertex in  (more specifically a vertex from , since ) with , which is not possible. 
It follows that the only possible conflict configuration for 
  and  
is  or . Applying the same reasoning to the conflict between  and , it follows that 
all three must be in chain configuration, where  is in the middle of the chain in any left to right order. 
\end{proof}


We are now ready to prove the main result of this subsection.

\begin{theorem}
\label{th:weaklytriangl}
Given an instance  with conflict graph  such that  is acyclic and  then  is weakly triangulated. 
\end{theorem}
\begin{proof} 
Assume first for the sake of contradiction that   is an induced subgraph of a conflict graph for .
\mar{The cycle  can be divided into  s: , , , . 
We show that for each s, , ,  and  must be in chain configuration in . There exists indeed a -path starting at vertex  and ending 
at vertex  as an induced subgraph of , thus of  as well. Since , by Lemma~\ref{edgesharing}, 
the s  and , neither   
share a vertex nor an edge in . By definition of , they do not conflict. 
Since  conflicts with both  and , by Lemma~\ref{threeinchain}, all three must be in chain 
configuration, where  is in the middle of the configuration in any left to right order. 
Since each of the  triples , , ,  is in chain configuration similarly,
the s corresponding to the whole path ,  are in chain configuration in this order. This implies there cannot be a conflict between  and 
 , since in the opposite case it would correspond to a cycle in graph . This contradicts the fact vertices  and  are adjacent in }. 
 
To prove that  is not an induced subgraph in any conflict graph, we first note that
since  is isomorphic to ,  cannot be an induced subgraph 
of any conflict graph. For , we prove it by contradiction as well. Suppose , with  is an induced subgraph of .
\mar{Consider the path . This is an induced -path 
in , thus also in . By Lemma~\ref{edgesharing},  and  do not share any vertex and neither                                                                         an edge in . By definition of  they do not conflict.
Since   
conflicts with both  and  (vertex  is adjacent to  and  in ), by Lemma~\ref{threeinchain},
, and  must be in chain configuration in that order. By the same reasoning 
, and  must be in chain configuration again in the same order. However this is 
only possible if   and  are identical, which constitutes a contradiction.}   
\end{proof}



In~\citet{AbakaBE13}, the constrained alignment
problem is shown to be polynomial-time solvable  if  is acyclic and , using a dynamic programming approach. Theorem~\ref{th:weaklytriangl} gives an alternative proof using the  algorithm for maximum independent set in weakly triangulated graphs~(\citet{hayward2}). In this case, 
 Lemma~\ref{edgelemma} and Theorem~\ref{th:zagreb-bound}.\textbf{ (i)} give    while Lemma~\ref{vertexlemma} gives .
 The related complexity is  if  is a fixed constant. 



\section{Concluding remarks}\label{sec:conclude}

We consider the constrained alignment of a pair of input graphs. 
We heavily investigate the combinatorial properties of a conflict graph 
which was introduced in~\citet{Fertin200990} but not studied in detail as far 
as graph theoretical properties are concerned. \mar{The constrained alignment problem appears as being closely related to the maximum independent set problem in conflict graphs.}
\mar{Known results on the maximum independent set problem associated with} several structural properties of conflict graphs lead to algorithmic results \mar{for the constrained alignment problem}: a polynomial-time  case, polynomial-time
approximations, and fixed-parameter tractability results.  


Our contribution is twofold. First, we improve known approximation results \mar{for the constrained alignment problem} in several ways.  In terms of the maximum degrees of  and ,  we propose the first -approximation using basic properties of conflict graphs. This ratio is similar to the known approximation ratios, function of the maximum degree, for the maximum independent set problem in  and . This is due to the fact that the maximum degree of the conflict graph is of the same order as . On the contrary, the number of vertices of the conflict graph does not allow to derive interesting results from known maximum independent set approximation ratios \mar{expressed as functions of} the number of vertices. We design the first non trivial approximation result with a ratio function of  for the constrained alignment problem.  The related ratio, , is better than  directly obtained from ratios function of the degree but it is still large compared to the -approximation of the maximum independent set in . 
\begin{quote}
A first open question raised by these results is 
to strengthen  hardness approximation results for the constrained alignment problem and in particular to investigate whether a ratio of  or even a constant approximation can be achieved in polynomial time. \mar{It is indeed well-known that such ratios cannot be achieved for the maximum independent set problem}.
\end{quote}
We also derive a ratio of  \mar{for the constrained alignment problem with }, while a similar result is not possible for the maximum independent set in general graphs unless P=NP. This kind of unusual result (Theorem~\ref{th:approx} and Proposition~\ref{pro:approxsqrt}) seems  interesting to investigate.
\begin{quote}
Studying more in detail in which extend similar -approximation results, parametrised by the size of the optimal solution,  can be obtained for the maximum independent set problem or other problems is  an interesting line of research raised by this work.\end{quote}
 Our second contribution is about structural results on the conflict graph. \mar{After general considerations (Subsection~\ref{sec:anym1m2}) valid for any , we focus on the case  (any ) that has been considered in~\cite{Fagnot2008,Fertin200990}}. For this case, we investigate graph classes that all can be characterised by forbidden subgraphs  in the neighborhood of any vertex: the case where  is a large clique or a large independent set is pretty usual, it just corresponds, in the whole graph, to  exclude large cliques and/or large claws. The case where  is an induced path or cycle \-- thus excluding wheels or fans \-- is less current even thought the classes of -free graphs themselves have 
raised great interest in the recent years: for instance many researches deal with maximum independent set problem in graphs excluding  or  for some . In particular, it is known that the maximum independent set  problem is  polynomial for -free graphs~(\citet{ISP5}) and the case of larger  is still unknown. For instance, if the maximum independent set problem was polynomial in -free graphs, then combining Theorems~\ref{Th:F8} and~\ref{th:approx} would lead to a -approximation for the constrained alignment problem. Note also that -free subgraphs of  play a crucial role for several results in this work; it would be interesting to study whether this approach can be applied in a more general setting using -free subgraphs instead of -free ones. 
\begin{quote}
So far, this work motivates the study of maximum independent set in graphs excluding fans and/or wheels
and more generally in classes of graphs with forbidden subgraphs in the neighborhood of any vertex.
\end{quote}
Theorem~\ref{th:approx} gives a first step in this direction with a strategy to efficiently solve the maximum independent set in a graph  when a good solution can be found in all subgraphs . 

\mar{As a first attempt to investigate properties of the conflict graph to derive efficient algorithms, the case  revealed to be very rich and promising as it allows to derive interesting properties of the conflict graph, even for large values of . Even if, as outlined in~\cite{Fagnot2008}, the underlying biological application motivates the case where both  and  are small, it is worth to note that reduction of the constrained alignment problem to a maximum independent set problem in the conflict graph is valid for any values of .   As mentioned in Subsection~\ref{sub:def}, the largest possible values for  ()  leads to   another well studied problem, the maximum common edge subgraph problem that includes many well-known problems like the maximum clique problem. If  are large, the size of the conflict graph increases very fast and it becomes dense. As a consequence,  this approach is likely to lead to good computational results if at least one of  is small.} 

\begin{quote}
    \mar{The last research direction we want to outline is to investigate properties of conflict graphs for larger values of  for, at least, some classes of graphs  and .}
\end{quote}

\section*{Acknowledgements}
We are grateful to the anonymous reviewers for their helpful comments and suggestions.

\bibliographystyle{abbrvnat}
\bibliography{journal}	   



\end{document}
