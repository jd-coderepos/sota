\documentclass{article}

\PassOptionsToPackage{numbers, compress}{natbib}






\usepackage[final]{controllable_text_to_image_generation}



\usepackage[utf8]{inputenc} \usepackage[T1]{fontenc}    \usepackage{hyperref}       \usepackage{url}            \usepackage{booktabs}       \usepackage{amsfonts}       \usepackage{nicefrac}       \usepackage{microtype}      \usepackage{graphicx}
\graphicspath{ {./images/} }
\usepackage{subcaption}
\usepackage{float}
\usepackage{sidecap}
\usepackage{tikz}
\usepackage{amsmath}
\usepackage{mathtools}
\definecolor{colour}{RGB}{228, 24, 133}
\newcommand{\xjqi}[1]{\textcolor{blue}{{[\textbf{Xiaojuan}: #1]}}}
\newcommand{\bowen}[1]{\textcolor{purple}{{[\textbf{Bowen}: #1]}}}
\title{Controllable Text-to-Image Generation}



\author{Bowen Li,  Xiaojuan Qi, Thomas Lukasiewicz, Philip H. S. Torr\\
  University of Oxford\\
  \texttt{\{bowen.li, thomas.lukasiewicz\}@cs.ox.ac.uk} \\
  \texttt{\{xiaojuan.qi, philip.torr\}@eng.ox.ac.uk}
}

\begin{document}

\maketitle

\begin{abstract}

In this paper, we propose a novel controllable text-to-image generative adversarial network (ControlGAN), which can effectively synthesise high-quality images and {also control parts of the image generation} according to natural language descriptions. To achieve this, we introduce a word-level spatial and channel-wise attention-driven generator that can disentangle different visual attributes, and allow the model to focus on generating and manipulating subregions corresponding to the most relevant words. Also, a word-level discriminator is proposed to provide fine-grained supervisory feedback by correlating words with image regions, facilitating training an effective generator which is able to manipulate specific visual attributes without affecting the generation of other content. Furthermore, perceptual loss is adopted to reduce the randomness involved in the image generation, and to encourage the generator to manipulate specific attributes required in the modified text. Extensive experiments on benchmark datasets demonstrate that our method outperforms existing state of the art, and is able to effectively manipulate synthetic images using natural language descriptions. Code is available at \textcolor{colour}{https://github.com/mrlibw/ControlGAN}.
\end{abstract}

\section{Introduction}
\label{intro}
Generating realistic images that semantically match given text descriptions is a challenging problem and has tremendous potential applications, such as image editing, video games, and computer-aided design. Recently, thanks to the success of generative adversarial networks (GANs) \cite{denton2015deep, goodfellow2014generative, radford2015unsupervised} in generating realistic images, text-to-image generation has made remarkable progress \cite{reed2016generative, xu2018attngan, zhang2017stackgan} by implementing conditional GANs (cGANs) \cite{dong2017semantic, reed2016generative, reed2016learning}, which are able to generate realistic images conditioned on given text descriptions.

However, current generative networks are typically uncontrollable, which means that if users change some words of a sentence, the synthetic image would be significantly different from the one generated from the original text as shown in Fig.~\ref{fig:issue}. When the given text description (e.g., colour) is changed, corresponding visual attributes of the bird are modified, but other unrelated attributes (e.g., the pose and position) are changed as well. This is typically undesirable in real-world applications, when a user wants to further modify the synthetic image to satisfy her preferences.

The goal of this paper is to generate images from text, and also allow the user to manipulate synthetic images using natural language descriptions, in one framework. In particular, we focus on modifying visual attributes (e.g., category, texture, and colour) of objects in the generated images by changing given text descriptions.
To achieve this, we propose a novel controllable text-to-image generative adversarial network (ControlGAN),
which can synthesise high-quality images, and also allow the user to manipulate objects' attributes, without affecting the generation of other content.


Our ControlGAN contains three novel components. 
The first component is the word-level spatial and channel-wise attention-driven generator, where an attention mechanism is exploited to allow the generator to synthesise subregions corresponding to the most relevant words. Our generator follows a multi-stage architecture \cite{xu2018attngan, zhang2018stackgan++} that synthesises images from coarse to fine, and progressively improves the quality. 
The second component is a word-level discriminator, where the correlation between words and image subregions is explored to disentangle different visual attributes, {which} can provide the generator with fine-grained training signals related to visual attributes.
The third component is the adoption of the perceptual loss \cite{johnson2016perceptual} in text-to-image generation, which can reduce the randomness involved in the generation, 
and enforce the generator to preserve visual appearance related to the unmodified text. 

To this end, an extensive analysis is performed, which demonstrates that our method can effectively disentangle different attributes and accurately manipulate parts of the synthetic image without losing diversity. Also, experimental results on the CUB \cite{wah2011caltech} and COCO \cite{lin2014microsoft} datasets show that our method outperforms existing state of the art both qualitatively and quantitatively.

\begin{figure}[t]
\begin{minipage}{\textwidth}
\quad\begin{minipage}{0.4\textwidth}\raggedright
\small
This bird has a \textbf{yellow} back and rump, 
\textbf{gray} outer rectrices, and a light \textbf{gray} breast.\\
(original text)
\end{minipage}
\quad\begin{minipage}{0.12\textwidth}
\includegraphics[width=1\linewidth, height=1\linewidth]{issue_stack1.png}
\end{minipage}
\begin{minipage}{0.12\textwidth}
\includegraphics[width=1\linewidth, height=1\linewidth]{issue_stack++1.png}
\end{minipage}
\begin{minipage}{0.12\textwidth}
\includegraphics[width=1\linewidth, height=1\linewidth]{issue_our1.png}
\end{minipage}
\begin{minipage}{0.12\textwidth}
\includegraphics[width=1\linewidth, height=1\linewidth]{issue_orig.jpg}
\end{minipage}

\quad\begin{minipage}{0.4\textwidth}\raggedright
\small
This bird has a \textbf{red} back and rump, 
\textbf{yellow} outer rectrices, and a light \textbf{white} breast. (modified text)\\
\end{minipage}
\quad\begin{minipage}{0.12\textwidth}
\includegraphics[width=1\linewidth, height=1\linewidth]{issue_stack2.png}
\end{minipage}
\begin{minipage}{0.12\textwidth}
\includegraphics[width=1\linewidth, height=1\linewidth]{issue_stack++2.png}
\end{minipage}
\begin{minipage}{0.12\textwidth}
\includegraphics[width=1\linewidth, height=1\linewidth]{issue_our2.png}
\end{minipage}
\begin{minipage}{0.12\textwidth}
\includegraphics[width=1\linewidth, height=1\linewidth]{issue_orig.jpg}
\end{minipage}
\medskip

\quad\begin{minipage}{0.4\textwidth}\centering
\small{Text}
\end{minipage}
\quad\begin{minipage}{0.12\textwidth}
\centering
\small{\cite{zhang2017stackgan}}
\end{minipage}
\begin{minipage}{0.12\textwidth}
\centering
\small{\cite{xu2018attngan}}
\end{minipage}
\begin{minipage}{0.12\textwidth}
\centering
\small{Ours}
\end{minipage}
\begin{minipage}{0.12\textwidth}
\centering
\small{Original}
\end{minipage}
\end{minipage}

\centering
\caption{Examples of modifying synthetic images using a natural language description. The current state of the art methods generate realistic images, but fail to generate plausible images when we slightly change the text. In contrast, our method allows parts of the image to be manipulated in correspondence to the modified text description while preserving other unrelated content.}
\label{fig:issue}
\end{figure}

\section{Related Work}
\label{rel_work}
\paragraph{Text-to-image Generation.} Recently, there has been a lot of work and interest in text-to-image generation. Mansimov et al. \cite{mansimov2015generating} proposed the AlignDRAW model that used an attention mechanism over words of a caption to draw image patches in multiple stages. Nguyen et al. \cite{nguyen2017plug} introduced an approximate Langevin approach to synthesise images from text. Reed et al. \cite{reed2016generative} first applied the cGAN to generate plausible images conditioned on text descriptions. Zhang et al. \cite{zhang2017stackgan} decomposed text-to-image generation into several stages generating image from coarse to fine.
However, all above approaches mainly focus on generating a new high-quality image from a given text, and cannot allow the user to manipulate the generation of specific visual attributes using natural language descriptions.

\paragraph{Image-to-image translation.} Our work is also closely related to conditional image manipulation methods. Cheng et al. \cite{cheng2014imagespirit} produced high-quality image parsing results from verbal commands. Zhu et al. \cite{zhu2016generative} proposed to change the colour and shape of an object by manipulating latent vectors.
Brock et al. \cite{brock2016neural} introduced a hybrid model using VAEs \cite{kingma2014semi} and GANs, which achieved an accurate reconstruction without loss of {image quality}.
Recently, Nam et al. \cite{nam2018text} built a model for multi-modal learning on both text descriptions and input images, and proposed a text-adaptive discriminator which utilised word-level text-image matching scores as supervision.
However, they adopted a global pooling layer to extract image features, which may lose important fine-grained spatial information. Moreover, the above approaches focus only on image-to-image translation instead of text-to-image generation, which is {probably} more challenging.  
\paragraph{Attention.} The attention mechanism has shown its efficiency in various research fields including image captioning \cite{xu2015show, zhang2017mdnet}, machine translation \cite{bahdanau2014neural}, object detection \cite{oliva2003top, zhang2018progressive}, and visual question answering \cite{yang2016stacked}. 
It can effectively capture task-relevant information and reduce the interference from less important one.
Recently, Xu et al. \cite{xu2018attngan} built the AttnGAN model that designed a word-level spatial attention to guide the generator to focus on subregions corresponding to the most relevant word.
{However, spatial attention only correlates words with partial regions without taking channel information into account. Also, different channels of features in CNNs may have different purposes, and it is crucial to avoid treating all channels without distinction, such that the most relevant channels in the visual features can be fully exploited. }
\begin{figure}[t]
\centering
\begin{minipage}{1\textwidth}
\includegraphics[width=1\linewidth, height=0.4492\linewidth]{archi.pdf}
\end{minipage}
\caption{The architecture of our proposed ControlGAN. In (b),  is the correlation loss discussed in Sec.~\ref{sec:word}. In (c),  is the perceptual loss discussed in Sec.~\ref{sec:semantic}.}
\label{fig:archi}
\end{figure}

\section{Controllable Generative Adversarial Networks}
\label{controlGAN}

Given a sentence , we aim to synthesise a realistic image  that semantically aligns with  (see Fig.~\ref{fig:archi}), and also make this generation process controllable -- if  is modified to be , the synthetic result  should semantically match  while preserving irrelevant content existing in  (shown in Fig.~\ref{fig:qual_show}). To achieve this, we propose three novel components: 1) a channel-wise attention module, 2) a word-level discriminator, and 3) the adoption of the perceptual loss in the text-to-image generation. We elaborate our model as follows.  

\subsection{Architecture}

We adopt the multi-stage AttnGAN \cite{xu2018attngan} as our backbone architecture (see Fig.~\ref{fig:archi}).
Given a sentence , the text encoder -- a pre-trained bidirectional RNN~\cite{xu2018attngan} --
encodes the sentence  into a sentence feature  with dimension  describing the whole sentence, and word features  with length  (i.e., number of words) and dimension .  
Following \cite{zhang2017stackgan}, we also apply conditioning augmentation (CA) to .
The augmented sentence feature  is further concatenated with a random vector  to serve as the input to the first stage. 
The overall framework generates an image from coarse- to fine-scale in multiple stages, and, in each stage, the network produces a hidden visual feature , which is the input to the corresponding generator  to produce a synthetic image.
Spatial attention \cite{xu2018attngan} and our proposed channel-wise attention modules {take  and  as inputs}, and output attentive word-context features.
These attentive features are further concatenated with the hidden feature  and then serve as input for the next stage.

The generator exploits the attention mechanism via incorporating a spatial attention module \cite{xu2018attngan} and the proposed channel-wise attention module. The spatial attention module \cite{xu2018attngan} can only correlate words with individual spatial locations without taking channel information into account.
Thus, we introduce a channel-wise attention module (see Sec.~\ref{sec:channel-attention}) {to exploit the connection between words and channels}.
We experimentally find that the channel-wise attention module highly correlates semantically meaningful parts with corresponding words, while the spatial attention focuses on colour descriptions (see Fig.~\ref{fig:channel}).
{Therefore,} {our proposed channel-wise attention module, together with the spatial attention}, can help the generator disentangle different visual attributes, and allow it to focus {only} on the most relevant subregions and channels.

\begin{figure}[t]
\centering
\begin{minipage}{1\textwidth}
\includegraphics[width=1\linewidth, height=0.4375\linewidth]{archi_channel_discriminator.pdf}
\end{minipage}
\caption{The architecture of proposed channel-wise attention module and word-level discriminator.}
\label{fig:archi_channel_dis}
\end{figure}

\subsection{Channel-Wise Attention}
\label{sec:channel-attention}
At the  stage, the channel-wise attention module (see Fig.~\ref{fig:archi_channel_dis} (a)) takes two inputs: the word features  and hidden visual features , where  and  denote the height and width of the feature map at stage . The word features  are first mapped into the same semantic space as the visual features  via a perception layer , producing , where .

Then, we calculate the channel-wise attention matrix  by multiplying the converted word features  and visual features , denoted as . Thus,  aggregates correlation values between channels and words across all spatial locations. Next,  is normalised by the softmax function to generate the normalised channel-wise attention matrix  as

The attention weight  represents the correlation between the  channel in the visual features  and the  word in the sentence , and higher value means larger correlation.

Equipped with the channel-wise attention matrix , we obtain the final channel-wise attention features , denoted as . 
Each channel in  is a dynamic representation weighted by the correlation between words and corresponding channels in the visual features.
Thus, channels with high correlation values are enhanced resulting in a high response to corresponding words, which can facilitate disentangling word attributes into different channels, and also reduce the influence from irrelevant channels by assigning a lower correlation.

\subsection{Word-Level Discriminator}
\label{sec:word}
 
To encourage the generator to modify only parts of the image according to the text, the discriminator should provide the generator with fine-grained training feedback, which can guide the generation of subregions corresponding to the most relevant words.
Actually, the text-adaptive discriminator \cite{nam2018text} also explores the word-level information in the discriminator, but it adopts a global average pooling layer to output a 1D vector as image feature, and then calculates the correlation between image feature and each word. By doing this, the image feature may lose important spatial information, which provides crucial cues for disentangling different visual attributes. 
To address the issue, we propose a novel word-level discriminator inspired by \cite{nam2018text} to explore the correlation between image subregions and each word; see Fig.~\ref{fig:archi_channel_dis} (b).

Our word-level discriminator takes two inputs: 1) word features  encoded from the text encoder, which follows the same architecture as the one (see Fig.~\ref{fig:archi} (a)) used in the generator, where  and  denote word features {encoded from} the original text  and a randomly sampled mismatched text, {respectively}, and 2) visual features , both encoded by a GoogleNet-based \cite{szegedy2015going} image encoder from the real image  and generated images , respectively. 

For simplicity, in the following, we use  to represent visual features  and , and use  for both original and mismatched word features.
The word-level discriminator contains a perception layer  that is used to align the channel dimension of visual feature  and word feature , denoted as , where  is a weight matrix to learn. Then,
the word-context correlation matrix  can be derived via , and is further normalised by the softmax function to get a correlation matrix :

where  represents the correlation value between the  word and the  subregion of the image.
Then, the image subregion-aware word features  can be obtained by , which aggregates all spatial information weighted by the word-context correlation matrix .

Additionally, to further reduce the negative impact of less important words, we adopt the word-level self-attention \cite{nam2018text} to derive a 1D vector  with length  reflecting the relative importance of each word. 
Then, we repeat  by  times to produce , which {has} the same size as .
Next,  is further reweighted by  to get , denoted as , where  represents element-wise multiplication.
Finally, we derive the correlation between the  word and the whole image as Eq.~\eqref{eqn:exist}:

where  is the sigmoid function,  evaluates the correlation between the  word and the image, and  and  represent the  column of  and , respectively.

Therefore, the final correlation value  between image  and sentence  is calculated by summing all word-context correlations, denoted as .
By doing so, the generator can receive fine-grained feedback from the word-level discriminator for each visual attribute, which can further help supervise the generation and manipulation of each subregion independently.

\subsection{Perceptual Loss}
\label{sec:semantic}
Without {adding} any constraint on text-irrelevant regions (e.g., backgrounds), the generated results can be highly random, and may also fail to be semantically consistent with other content. 
To mitigate this randomness, we adopt the perceptual loss \cite{johnson2016perceptual} based on a 16-layer VGG network \cite{simonyan2014very} pre-trained on the ImageNet dataset \cite{russakovsky2015imagenet}. 
The network is used to extract semantic features from both the generated image  and the real image , and the  perceptual loss is defined as

where  is the activation of the  layer of the VGG network, and  and  are the height and width of the feature map, respectively.

To our knowledge, we are the first to apply the perceptual loss \cite{johnson2016perceptual} in controllable text-to-image generation, {which can} reduce the randomness involved in the image generation by matching feature space. 


\subsection{Objective Functions}
\label{obj}
The generator and discriminator are trained alternatively by minimising both the generator loss  and discriminator loss .

\paragraph{Generator objective.} 
The generator loss  as Eq.~\eqref{eq:loss} contains an adversarial loss , a text-image correlation loss , a perceptual loss , and a text-image matching loss ~\cite{xu2018attngan}.  

where  is the number of stages,  is the real image sampled from the true image distribution  at stage ,  is the generated image at the  stage sampled from the model distribution ,  are hyper-parameters controlling different losses,  is the perceptual loss described in Sec.~\ref{sec:semantic}, which puts constraint on the generation process to reduce the randomness, the  \cite{xu2018attngan} is used to measure text-image matching score based on the cosine similarity, and  reflects the correlation between the generated image and the given text description considering spatial information.

The adversarial loss  is composed of the unconditional and conditional adversarial losses shown in Eq.~\eqref{eq:adv}: the unconditional adversarial loss is applied to make the synthetic image be real, and the conditional adversarial loss is utilised to make the generated image match the given text .


\paragraph{Discriminator objective.} The final loss function for training the discriminator  is defined as:

where  is the correlation loss determining whether word-related visual attributes exist in the image (see Sec.~\ref{sec:word}),  is a mismatched text description that is randomly sampled from the dataset and is irrelevant to , and  is a hyper-parameter controlling the importance of additional losses.

The adversarial loss  contains two components: the unconditional adversarial loss determines whether the image is real, and the conditional adversarial loss determines whether the given image matches the text description :


\section{Experiments}
\label{experiments}

To evaluate the effectiveness of our approach, we conduct extensive experiments on the CUB bird \citep{wah2011caltech} and the MS COCO \citep{lin2014microsoft} datasets.
We compare with two state of the art GAN methods on text-to-image generation, StackGAN++ \cite{zhang2018stackgan++} and AttnGAN \cite{xu2018attngan}.
Results for the state of the art are reproduced based on the code released by the authors.

\subsection{Datasets}

Our method is evaluated on the CUB bird \citep{wah2011caltech} and the MS COCO \citep{lin2014microsoft} datasets. The CUB dataset contains 8,855 training images and 2,933 test images, and each image has 10 corresponding text descriptions. As for the COCO dataset, it contains 82,783 training images and 40,504 validation images, and each image has 5 corresponding text descriptions. We preprocess these two datasets based on the methods introduced in \cite{zhang2017stackgan}.

\subsection{Implementation}

There are three stages () in our ControlGAN generator following \cite{xu2018attngan}. The three scales are , , and , and spatial and channel-wise attentions {are applied at the stages 2 and 3}. The text encoder is a pre-trained bidirectional LSTM \cite{schuster1997bidirectional} to encode the given text description into a sentence feature with dimension 256 and word features with length 18 and dimension 256. In the perceptual loss, we compute the content loss at layer relu22 of VGG-16 \cite{simonyan2014very} pre-trained on the ImageNet \cite{russakovsky2015imagenet}. The whole network is trained using the Adam optimiser \cite{kingma2014adam} with the learning rate 0.0002. The hyper-parameters , , , and  are set to 0.5, 1, 1, and 5 for both datasets, respectively. 

\subsection{Comparison with State of the Art}

\paragraph{Quantitative results.}
We adopt the Inception Score \cite{salimans2016improved} to evaluate the quality and diversity of the generated images.
However, as the Inception Score cannot reflect the relevance between an image and a text description, we utilise R-precision \cite{xu2018attngan} to measure the correlation between a generated image and its corresponding text.
 We compare the top-1 text-to-image retrieval accuracy (Top-1 Acc) on the CUB and COCO datasets following \cite{nam2018text}.

Quantitative results are shown in Table~\ref{tabel:quant_cmp}, our method achieves better IS and R-precision values on the CUB dataset compared with the state of the art, and has a competitive performance on the COCO dataset.
This indicates that our method can generate higher-quality images with better diversity, which semantically align with the text descriptions. 

To further evaluate whether the model can generate controllable results, we compute the  reconstruction error \cite{nam2018text} between the image generated from the original text and the one from the modified text shown in Table~\ref{tabel:quant_cmp}.
Compared with other methods, ControlGAN achieves a significantly lower reconstruction error, which demonstrates that our method can better preserve content in the image generated from the original text.

\paragraph{Qualitative results.}
We show qualitative comparisons in Fig.~\ref{fig:qual_show}.
As we can see, according to modifying given text descriptions, our approach can successfully manipulate specific visual attributes accurately. 
Also, our method can even handle out-of-distribution queries, e.g., \emph{red zebra on a river} shown in the last two columns of Fig.~\ref{fig:qual_show}.
All the above indicates that our approach can manipulate different visual attributes independently, which demonstrates the effectiveness of our approach in disentangling visual attributes for text-to-image generation.


Fig.~\ref{fig:qual_cmp} shows the visual comparison between ControlGAN, AttnGAN~\cite{xu2018attngan}, and StackGAN++~\cite{zhang2018stackgan++}. It can be observed that when the text is modified, the two compared approaches are more likely to generate new content, or change some visual attributes that are not relevant to the modified text. {For instance, as shown in the first two columns, when we modify the colour attributes, StackGAN++ changes the pose of the bird, and AttnGAN generates new background.}
In contrast, our approach is able to accurately manipulate parts of the image generation corresponding to the modified text, while preserving the visual attributes related to unchanged text.

In the COCO dataset, our model again achieves much better results compared with others shown in Fig.~\ref{fig:qual_cmp}. {For example, as shown in the last four columns, the compared approaches cannot preserve the shape of objects and even fail to generate reasonable images.}
Generally speaking, the results on COCO are not as good as on the CUB dataset. We attribute this to the few text-image pairs and more abstract captions in the dataset. Although there are a lot of categories in COCO, each category only has a few number of examples, and captions focus mainly on the category of objects rather than detailed descriptions, which makes text-to-image generation more challenging. 


\begin{table}
\small
  \caption{Quantitative comparison: Inception Score, R-precision, and  reconstruction error of state of the art and ControlGAN on the CUB and COCO datasets.}
  \label{quantity}
  \centering
  \bigskip
  \begin{tabular}{lllllll}
    \toprule
    \multicolumn{5}{c}{CUB} &
    \multicolumn{1}{c}{COCO}     \\             
    \cmidrule(r){2-4}
    \cmidrule(r){5-7}
    Method     & IS     & Top-1 Acc(\%)  &  error     & IS & Top-1 Acc(\%) &  error\\
    \midrule
    StackGAN++ & 4.04  .05  & 45.28  3.72  & 0.29  & 8.30  .10 & 72.83  3.17 & 0.32 \\
    AttnGAN    & 4.36  .03  & 67.82  4.43 & 0.26     & \textbf{25.89  .47} & \textbf{85.47  3.69} & 0.40\\
    Ours     & \textbf{4.58  .09}   & \textbf{69.33  3.23}   & \textbf{0.18}     & 24.06  .60 &82.43  2.43 & \textbf{0.17}\\
    \bottomrule
  \end{tabular}
  \label{tabel:quant_cmp}
\end{table}

\begin{figure}[t]
\begin{minipage}{0.092\textwidth}
\centering
\tiny{This bird is \textbf{yellow} with \textbf{black} and has a very short beak.}
\end{minipage}
\begin{minipage}{0.092\textwidth}
\centering
\tiny{This bird is \textbf{orange} with \textbf{grey} and has a very short beak.}
\end{minipage}
\noindent\begin{minipage}{0.092\textwidth}
\centering
\tiny{The small bird has a dark \textbf{brown} head and light \textbf{brown} body.}
\end{minipage}
\noindent\begin{minipage}{0.092\textwidth}
\centering
\tiny{The small bird has a dark \textbf{tan} head and light \textbf{grey} body.}
\end{minipage}
\noindent\begin{minipage}{0.092\textwidth}
\centering
\tiny{A large group of \textbf{cows} on a farm.}
\end{minipage}
\noindent\begin{minipage}{0.092\textwidth}
\centering
\tiny{A large group of \textbf{white cows} on a farm.}
\end{minipage}
\noindent\begin{minipage}{0.092\textwidth}
\centering
\tiny{A crowd of people fly kites on a \textbf{hill}.}
\end{minipage}
\noindent\begin{minipage}{0.092\textwidth}
\centering
\tiny{A crowd of people fly kites on a \textbf{park}.}
\end{minipage}
\noindent\begin{minipage}{0.092\textwidth}
\centering
\tiny{A group of \textbf{zebras} on a \textbf{grassy field} with trees in background.}
\end{minipage}
\noindent\begin{minipage}{0.092\textwidth}
\centering
\tiny{A group of \textbf{red zebras} on a \textbf{river} with trees in background.}
\end{minipage}
\smallskip

\begin{minipage}{0.092\textwidth}
\includegraphics[width=1\linewidth, height=1\linewidth]{q1.png}
\end{minipage}
\begin{minipage}{0.092\textwidth}
\includegraphics[width=1\linewidth, height=1\linewidth]{q2.png}
\end{minipage}
\noindent\begin{minipage}{0.092\textwidth}
\includegraphics[width=1\linewidth, height=1\linewidth]{q3.png}
\end{minipage}
\noindent\begin{minipage}{0.092\textwidth}
\includegraphics[width=1\linewidth, height=1\linewidth]{q4.png}
\end{minipage}
\noindent\begin{minipage}{0.092\textwidth}
\includegraphics[width=1\linewidth, height=1\linewidth]{q5.png}
\end{minipage}
\noindent\begin{minipage}{0.092\textwidth}
\includegraphics[width=1\linewidth, height=1\linewidth]{q6.png}
\end{minipage}
\noindent\begin{minipage}{0.092\textwidth}
\includegraphics[width=1\linewidth, height=1\linewidth]{q7.png}
\end{minipage}
\noindent\begin{minipage}{0.092\textwidth}
\includegraphics[width=1\linewidth, height=1\linewidth]{q8.png}
\end{minipage}
\noindent\begin{minipage}{0.092\textwidth}
\includegraphics[width=1\linewidth, height=1\linewidth]{q9.png}
\end{minipage}
\noindent\begin{minipage}{0.092\textwidth}
\includegraphics[width=1\linewidth, height=1\linewidth]{q10.png}
\end{minipage}
\hfill 

\centering
\caption{Qualitative results on the CUB and COCO datasets. Odd-numbered columns show the original text and even-numbered ones the modified text. The last two are an out-of-distribution case.}
\label{fig:qual_show}
\bigskip

\begin{minipage}{0.12\textwidth}\raggedright
\scriptsize{Input}\\
\end{minipage}
\quad\begin{minipage}{0.10\textwidth}
\centering
\tiny{This bird has a \textbf{white} neck and breast with a \textbf{turquoise} crown and feathers a small short beak.}
\end{minipage}
\noindent\begin{minipage}{0.10\textwidth}
\centering
\tiny{This bird has a \textbf{grey} neck and breast with a \textbf{blue} crown and feathers a small short beak.}
\end{minipage}
\noindent\begin{minipage}{0.10\textwidth}
\centering
\tiny{This bird has wings that are \textbf{yellow} and has a \textbf{brown} body.}
\end{minipage}
\noindent\begin{minipage}{0.10\textwidth}
\centering
\tiny{This bird has wings that are \textbf{black} and has a \textbf{red} body.}
\end{minipage}
\noindent\begin{minipage}{0.10\textwidth}
\centering
\tiny{A giraffe is standing on the dirt.}
\end{minipage}
\noindent\begin{minipage}{0.10\textwidth}
\centering
\tiny{A giraffe is standing on the dirt \textbf{in an enclosure}.}
\end{minipage}
\noindent\begin{minipage}{0.10\textwidth}
\centering
\tiny{A \textbf{zebra} stands on a pathway near grass.}
\end{minipage}
\noindent\begin{minipage}{0.10\textwidth}
\centering
\tiny{A \textbf{sheep} stands on a pathway near grass.}
\end{minipage}
\smallskip

\begin{minipage}{0.12\textwidth}\raggedright
\scriptsize{StackGAN++ \cite{zhang2018stackgan++}}\\
\end{minipage}
\quad\begin{minipage}{0.10\textwidth}
\includegraphics[width=1\linewidth, height=1\linewidth]{stack1.png}
\end{minipage}
\noindent\begin{minipage}{0.10\textwidth}
\includegraphics[width=1\linewidth, height=1\linewidth]{stack11.png}
\end{minipage}
\noindent\begin{minipage}{0.10\textwidth}
\includegraphics[width=1\linewidth, height=1\linewidth]{stack2.png}
\end{minipage}
\noindent\begin{minipage}{0.10\textwidth}
\includegraphics[width=1\linewidth, height=1\linewidth]{stack22.png}
\end{minipage}
\noindent\begin{minipage}{0.10\textwidth}
\includegraphics[width=1\linewidth, height=1\linewidth]{stack3.png}
\end{minipage}
\noindent\begin{minipage}{0.10\textwidth}
\includegraphics[width=1\linewidth, height=1\linewidth]{stack33.png}
\end{minipage}
\noindent\begin{minipage}{0.10\textwidth}
\includegraphics[width=1\linewidth, height=1\linewidth]{stack4.png}
\end{minipage}
\noindent\begin{minipage}{0.10\textwidth}
\includegraphics[width=1\linewidth, height=1\linewidth]{stack44.png}
\end{minipage}
\hfill 

\begin{minipage}{0.12\textwidth}\raggedright
\scriptsize{AttnGAN \cite{xu2018attngan}}\\
\end{minipage}
\quad\begin{minipage}{0.10\textwidth}
\includegraphics[width=1\linewidth, height=1\linewidth]{attn1.png}
\end{minipage}
\noindent\begin{minipage}{0.10\textwidth}
\includegraphics[width=1\linewidth, height=1\linewidth]{attn11.png}
\end{minipage}
\noindent\begin{minipage}{0.10\textwidth}
\includegraphics[width=1\linewidth, height=1\linewidth]{attn2.png}
\end{minipage}
\noindent\begin{minipage}{0.10\textwidth}
\includegraphics[width=1\linewidth, height=1\linewidth]{attn22.png}
\end{minipage}
\noindent\begin{minipage}{0.10\textwidth}
\includegraphics[width=1\linewidth, height=1\linewidth]{attn3.png}
\end{minipage}
\noindent\begin{minipage}{0.10\textwidth}
\includegraphics[width=1\linewidth, height=1\linewidth]{attn33.png}
\end{minipage}
\noindent\begin{minipage}{0.10\textwidth}
\includegraphics[width=1\linewidth, height=1\linewidth]{attn4.png}
\end{minipage}
\noindent\begin{minipage}{0.10\textwidth}
\includegraphics[width=1\linewidth, height=1\linewidth]{attn44.png}
\end{minipage}
\hfill 

\begin{minipage}{0.12\textwidth}\raggedright
\scriptsize{Ours}\\
\end{minipage}
\quad\begin{minipage}{0.10\textwidth}
\includegraphics[width=1\linewidth, height=1\linewidth]{cmp_our1.png}
\end{minipage}
\noindent\begin{minipage}{0.10\textwidth}
\includegraphics[width=1\linewidth, height=1\linewidth]{cmp_our11.png}
\end{minipage}
\noindent\begin{minipage}{0.10\textwidth}
\includegraphics[width=1\linewidth, height=1\linewidth]{cmp_our2.png}
\end{minipage}
\noindent\begin{minipage}{0.10\textwidth}
\includegraphics[width=1\linewidth, height=1\linewidth]{cmp_our22.png}
\end{minipage}
\noindent\begin{minipage}{0.10\textwidth}
\includegraphics[width=1\linewidth, height=1\linewidth]{cmp_our3.png}
\end{minipage}
\noindent\begin{minipage}{0.10\textwidth}
\includegraphics[width=1\linewidth, height=1\linewidth]{cmp_our33.png}
\end{minipage}
\noindent\begin{minipage}{0.10\textwidth}
\includegraphics[width=1\linewidth, height=1\linewidth]{cmp_our4.png}
\end{minipage}
\noindent\begin{minipage}{0.10\textwidth}
\includegraphics[width=1\linewidth, height=1\linewidth]{cmp_our44.png}
\end{minipage}
\hfill \centering
\caption{Qualitative comparison of three methods on the CUB and COCO datasets. Odd-numbered columns show the original text and even-numbered ones the modified text.}
\label{fig:qual_cmp}
\end{figure}

\begin{figure}[ht]
\begin{minipage}{0.90\textwidth}
\includegraphics[width=1\linewidth, height=0.44\linewidth]{channel_spatial_attn.pdf}
\end{minipage}
\centering
\caption{Top: visualisation of feature channels at stage 3. The number at the top-right corner is the channel number, and the word that has the highest correlation  in Eq.~\ref{eqn:corr} with the channel is shown under the image. Bottom: spatial attention produced in stage 3.}
\label{fig:channel}
\end{figure}

\subsection{Component Analysis}

\paragraph{Effectiveness of channel-wise attention.} Our model implements channel-wise attention in the generator, together the spatial attention, to generate realistic images. To better understand the effectiveness of attention mechanisms, we visualise the intermediate results and corresponding attention maps at different stages.

We experimentally find that the channel-wise attention correlates closely with semantic parts of objects, while the spatial attention focuses mainly on colour descriptions. Fig.~\ref{fig:channel} shows several channels of feature maps that correlate with different semantics, and our channel-wise attention module assigns large correlation values to channels that are semantically related to the word describing parts of a bird. This phenomenon is further verified by the ablation study shown in Fig.~\ref{fig:abla_channel_word} (left side). Without channel-wise attention, our model fails to generate controllable results when we modify the text related to parts of a bird. In contrast, our model with channel-wise attention can generate better controllable results.

\paragraph{Effectiveness of word-level discriminator.}
To verify the effectiveness of the word-level discriminator, we first conduct an ablation study that our model is trained without word-level discriminator, shown in Fig.~\ref{fig:abla_channel_word} (right side), and then we construct a baseline model by replacing our discriminator with a text-adaptive discriminator \cite{nam2018text}, which also explores the correlation between image features and words. Visual comparisons are shown in Fig.~\ref{fig:abla_spm_word} (right side). We can easily observe that the compared baseline fails to manipulate the synthetic images. For example, as shown in the first two columns, the bird generated from the modified text has a totally different shape, and the background has been changed as well. This is due to the fact that the text-adaptive discriminator \cite{nam2018text} uses a global pooling layer to extract image features, which may lose important spatial information.

\paragraph{Effectiveness of perceptual loss.} Furthermore, we conduct an ablation study that our model is trained without the perceptual loss, shown in Fig.~\ref{fig:abla_spm_word} (left side).
Without perceptual loss, images generated from modified text are hard to preserve content that are related to unmodified text, which indicates that the perceptual loss can potentially introduce a stricter semantic constraint on the image generation and help reduce the involved randomness. 

\begin{figure}[t]
\begin{minipage}{0.48\textwidth}
\begin{minipage}{0.184\textwidth}
\centering
\tiny{Input}
\end{minipage}
\begin{minipage}{0.184\textwidth}
\centering
\tiny{This yellow bird has grey and white wings and a red \textbf{head}.}
\end{minipage}
\noindent\begin{minipage}{0.184\textwidth}
\centering
\tiny{This yellow bird has grey and white wings and a red \textbf{belly}.}
\end{minipage}
\noindent\begin{minipage}{0.184\textwidth}
\centering
\tiny{The bird is small and round with white \textbf{belly} and blue wings.}
\end{minipage}
\noindent\begin{minipage}{0.184\textwidth}
\centering
\tiny{The bird is small and round with white \textbf{head} and blue wings.}
\end{minipage}
\smallskip

\begin{minipage}{0.184\textwidth}
\centering
\tiny{Ours without channel-wise attention}
\end{minipage}
\begin{minipage}{0.184\textwidth}
\includegraphics[width=1\linewidth, height=1\linewidth]{no_channel_orig1.png}
\end{minipage}
\noindent\begin{minipage}{0.184\textwidth}
\includegraphics[width=1\linewidth, height=1\linewidth]{no_channel_mod1.png}
\end{minipage}
\noindent\begin{minipage}{0.184\textwidth}
\includegraphics[width=1\linewidth, height=1\linewidth]{no_channel_orig2.png}
\end{minipage}
\noindent\begin{minipage}{0.184\textwidth}
\includegraphics[width=1\linewidth, height=1\linewidth]{no_channel_mod2.png}
\end{minipage}

\begin{minipage}{0.184\textwidth}
\centering
\tiny{Ours}
\end{minipage}
\begin{minipage}{0.184\textwidth}
\includegraphics[width=1\linewidth, height=1\linewidth]{with_channel_orig1.png}
\end{minipage}
\noindent\begin{minipage}{0.184\textwidth}
\includegraphics[width=1\linewidth, height=1\linewidth]{with_channel_mod1.png}
\end{minipage}
\noindent\begin{minipage}{0.184\textwidth}
\includegraphics[width=1\linewidth, height=1\linewidth]{with_channel_orig2.png}
\end{minipage}
\noindent\begin{minipage}{0.184\textwidth}
\includegraphics[width=1\linewidth, height=1\linewidth]{with_channel_mod2.png}
\end{minipage}
\end{minipage}
\begin{minipage}{0.01\textwidth}
\centering





\end{minipage}
\begin{minipage}{0.48\textwidth}
\noindent\begin{minipage}{0.184\textwidth}
\centering
\tiny{Input}
\end{minipage}
\noindent\begin{minipage}{0.184\textwidth}
\centering
\tiny{This bird's wing bar is brown and yellow, and its belly is \textbf{yellow}.}
\end{minipage}
\noindent\begin{minipage}{0.184\textwidth}
\centering
\tiny{This bird's wing bar is brown and yellow, and its belly is \textbf{white}.}
\end{minipage}
\noindent\begin{minipage}{0.184\textwidth}
\centering
\tiny{A small bird with a brown colouring and white \textbf{belly}.}
\end{minipage}
\noindent\begin{minipage}{0.184\textwidth}
\centering
\tiny{A small bird with a brown colouring and white \textbf{head}.}
\end{minipage}
\smallskip

\noindent\begin{minipage}{0.184\textwidth}
\centering
\tiny{Ours without word-level discriminator}
\end{minipage}
\noindent\begin{minipage}{0.184\textwidth}
\includegraphics[width=1\linewidth, height=1\linewidth]{no_word_orig1.png}
\end{minipage}
\noindent\begin{minipage}{0.184\textwidth}
\includegraphics[width=1\linewidth, height=1\linewidth]{no_word_mod1.png}
\end{minipage}
\noindent\begin{minipage}{0.184\textwidth}
\includegraphics[width=1\linewidth, height=1\linewidth]{no_word_orig2.png}
\end{minipage}
\noindent\begin{minipage}{0.184\textwidth}
\includegraphics[width=1\linewidth, height=1\linewidth]{no_word_mod2.png}
\end{minipage}

\noindent\begin{minipage}{0.184\textwidth}
\centering
\tiny{Ours}
\end{minipage}
\noindent\begin{minipage}{0.184\textwidth}
\includegraphics[width=1\linewidth, height=1\linewidth]{word_orig1.png}
\end{minipage}
\noindent\begin{minipage}{0.184\textwidth}
\includegraphics[width=1\linewidth, height=1\linewidth]{word_mod1.png}
\end{minipage}
\noindent\begin{minipage}{0.184\textwidth}
\includegraphics[width=1\linewidth, height=1\linewidth]{word_orig2.png}
\end{minipage}
\noindent\begin{minipage}{0.184\textwidth}
\includegraphics[width=1\linewidth, height=1\linewidth]{word_mod2.png}
\end{minipage}

\end{minipage}
\centering
\caption{Left: ablation study of channel-wise attention; right: ablation study of the word-level discriminator.}
\label{fig:abla_channel_word}
\bigskip
\begin{minipage}{0.48\textwidth}
\begin{minipage}{0.184\textwidth}
\centering
\tiny{Input}
\end{minipage}
\begin{minipage}{0.184\textwidth}
\centering
\tiny{The bird is small with a pointed bill, has black eyes, and a \textbf{yellow} crown.}
\end{minipage}
\noindent\begin{minipage}{0.184\textwidth}
\centering
\tiny{The bird is small with a pointed bill, has black eyes, and an \textbf{orange} crown.}
\end{minipage}
\noindent\begin{minipage}{0.184\textwidth}
\centering
\tiny{A bird with a white \textbf{belly} and metallic blue wings with a small beak.}
\end{minipage}
\noindent\begin{minipage}{0.184\textwidth}
\centering
\tiny{A bird with a white \textbf{head} and metallic blue wings with a small beak.}
\end{minipage}
\smallskip

\begin{minipage}{0.184\textwidth}
\centering
\tiny{Ours without perceptual loss}
\end{minipage}
\begin{minipage}{0.184\textwidth}
\includegraphics[width=1\linewidth, height=1\linewidth]{no_spm1.png}
\end{minipage}
\noindent\begin{minipage}{0.184\textwidth}
\includegraphics[width=1\linewidth, height=1\linewidth]{no_spm11.png}
\end{minipage}
\noindent\begin{minipage}{0.184\textwidth}
\includegraphics[width=1\linewidth, height=1\linewidth]{no_spm2.png}
\end{minipage}
\noindent\begin{minipage}{0.184\textwidth}
\includegraphics[width=1\linewidth, height=1\linewidth]{no_spm22.png}
\end{minipage}

\begin{minipage}{0.184\textwidth}
\centering
\tiny{Ours}
\end{minipage}
\begin{minipage}{0.184\textwidth}
\includegraphics[width=1\linewidth, height=1\linewidth]{spm_our1.png}
\end{minipage}
\noindent\begin{minipage}{0.184\textwidth}
\includegraphics[width=1\linewidth, height=1\linewidth]{spm_our11.png}
\end{minipage}
\noindent\begin{minipage}{0.184\textwidth}
\includegraphics[width=1\linewidth, height=1\linewidth]{spm_our2.png}
\end{minipage}
\noindent\begin{minipage}{0.184\textwidth}
\includegraphics[width=1\linewidth, height=1\linewidth]{spm_our22.png}
\end{minipage}
\end{minipage}
\begin{minipage}{0.01\textwidth}
\centering





\end{minipage}
\begin{minipage}{0.48\textwidth}
\noindent\begin{minipage}{0.184\textwidth}
\centering
\tiny{Input}
\end{minipage}
\noindent\begin{minipage}{0.184\textwidth}
\centering
\tiny{A songbird is \textbf{white} with \textbf{blue} feathers and black eyes.}
\end{minipage}
\noindent\begin{minipage}{0.184\textwidth}
\centering
\tiny{A songbird is \textbf{yellow} with \textbf{blue and green} feathers and black eyes.}
\end{minipage}
\noindent\begin{minipage}{0.184\textwidth}
\centering
\tiny{A tiny bird, with green flank, \textbf{white} belly, \textbf{yellow} crown, and a pointy bill.}
\end{minipage}
\noindent\begin{minipage}{0.184\textwidth}
\centering
\tiny{A tiny bird, with green flank, \textbf{grey} belly, \textbf{blue} crown, and a pointy bill.}
\end{minipage}
\smallskip

\noindent\begin{minipage}{0.184\textwidth}
\centering
\tiny{Ours with text-adaptive discriminator}
\end{minipage}
\noindent\begin{minipage}{0.184\textwidth}
\includegraphics[width=1\linewidth, height=1\linewidth]{nips1.png}
\end{minipage}
\noindent\begin{minipage}{0.184\textwidth}
\includegraphics[width=1\linewidth, height=1\linewidth]{nips11.png}
\end{minipage}
\noindent\begin{minipage}{0.184\textwidth}
\includegraphics[width=1\linewidth, height=1\linewidth]{nips2.png}
\end{minipage}
\noindent\begin{minipage}{0.184\textwidth}
\includegraphics[width=1\linewidth, height=1\linewidth]{nips22.png}
\end{minipage}

\noindent\begin{minipage}{0.184\textwidth}
\centering
\tiny{Ours}
\end{minipage}
\noindent\begin{minipage}{0.184\textwidth}
\includegraphics[width=1\linewidth, height=1\linewidth]{nips_our1.png}
\end{minipage}
\noindent\begin{minipage}{0.184\textwidth}
\includegraphics[width=1\linewidth, height=1\linewidth]{nips_our11.png}
\end{minipage}
\noindent\begin{minipage}{0.184\textwidth}
\includegraphics[width=1\linewidth, height=1\linewidth]{nips_our2.png}
\end{minipage}
\noindent\begin{minipage}{0.184\textwidth}
\includegraphics[width=1\linewidth, height=1\linewidth]{nips_our22.png}
\end{minipage}

\end{minipage}
\centering
\caption{Left: ablation study of the perceptual loss \cite{johnson2016perceptual}; right: comparison between our word-level discriminator and text-adaptive discriminator \cite{nam2018text}.}
\label{fig:abla_spm_word}
\end{figure}

\section{Conclusion}
We have proposed a controllable generative adversarial network (ControlGAN), which can generate and manipulate the generation of images based on natural language descriptions. Our ControlGAN can successfully disentangle different visual attributes and allow parts of the synthetic image to be manipulated accurately, while preserving the generation of other content. Three novel components are introduced in our model: 1) the word-level spatial and channel-wise attention-driven generator can effectively disentangle different visual attributes, 2) the word-level discriminator provides the generator with fine-grained training signals related to each visual attribute, and 3) the adoption of perceptual loss reduces the randomness involved in the generation, and enforces the generator to reconstruct content related to unmodified text.
Extensive experimental results demonstrate the effectiveness and superiority of our method on two benchmark datasets.
\bibliographystyle{abbrv}
\bibliography{reference}

\end{document}
