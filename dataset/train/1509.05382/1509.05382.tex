

\subsection{Common Sense 1}
\label{subsec:cs1}
Common Sense 1 compares the price of entire single stop tickets with the prices 
of its separate legs when booked as independent direct flights. 
For each single stop ticket, we retrieve the equivalent direct fares for each leg 
of the single stop ticket. 
To be considered equivalent, such a direct fare must have the same flight number 
and booking code (T, Y, X, etc.) with the respective leg flight number of the 
single stop ticket.\footnote{However, in some cases, airlines do not offer the same booking codes for 
both the single stop fare and the separate leg fare. 
A possible explanation of this is the seat availability management done by the airlines.}

If any of the respective direct tickets  is more expensive than the relevant 
single stop ticket then we have a violation of CS1 and count it. 

In Fig.~\ref{fig:cs1Violations} we show the violations percentages for each route. 
The labels indicate the number of fares for each route, and the violations percentage
(y-axis) indicate the portion of these fares that violate CS1. Also, 
in the labels we show the number of airlines that have fares in each route.
All CS1 violations in our dataset are caused by different fare prices. This
is in contrast to CS2,3 where there are violations caused by different taxes 
(section 3.2).

\begin{figure}[H]
\centering
\includegraphics[width=\columnwidth]{cs3.eps}
\caption{Percentages of the tickets that violate common sense 1. 
The labels indicate the number of fares per route and the number of airlines that 
have fares in each route}
\label{fig:cs1Violations}
\end{figure}

As shown in Fig.~\ref{fig:cs1Violations} there are 6 routes out of the 64 that 
have a CS1 violations percentage above 5\%. Overall half of the routes (32) 
had at least one violation. 
Although the CS1 violations are not so common across the entire dataset (1.53\%), when 
they are observed on a particular route, their frequency can be high, as in 
BRU-STR \footnote{\url{www.world-airport-codes.com}} where it reaches 24\%. 
To investigate what happens in the BRU-STR route 
we plot the breakdown of CS1 violations per route and airline. 
We intend to see is whether a high percentage of CS1 violation on a route is 
attributed to several of the airlines operating on the route, or it is the 
result of the pricing policy of one or few airlines. 

We do this in Fig.~\ref{fig:cs1PerRoutePerAirline} from which we conclude that, usually, 
one or two airlines cause the majority of the violations on a route with high 
frequency of CS1 violations.  This means that the violations are a result of the
policies of particular airlines rather than a result of the route itself.  
From Fig.~\ref{fig:cs1PerRoutePerAirline} it is clear that for FRA-ZRH, BRU-STR,
HEL-WAW the majority of violations come from just two airlines (the Scandinavian airline, the Dutch airline).

\begin{figure*}[t]
\centering
\includegraphics[width=\textwidth]{cs3PerRoutePerAirline.eps}
\caption{Percentages of the violations per airline and route}
\label{fig:cs1PerRoutePerAirline}
\end{figure*}

We now turn into re-examining the data per airline instead of per route. 
In Fig.~\ref{fig:cs1PerAirline} we show the percentages of violations
for each airline. For each airline these violations might be spread across
one or many routes. We examine this distribution in the next section.

\begin{figure}[H]
\centering
\includegraphics[width=\columnwidth]{cs3PerAirline.eps}
\caption{Percentages of the tickets per airline that violate common sense 1}
\label{fig:cs1PerAirline}
\end{figure}

We see that the Dutch airline has an exceptionally high percentage of violations (53.5\%). 
The Greek airline and the Belgian airline are around 10\% and most others are below 10\%.
In absolute numbers, the Dutch airline has the most violations (565) with the Scandinavian airline coming second (209).
In the next section we focus in the airlines with the most violations, and we
research the distribution of the violations in different routes.

\subsubsection{The airlines with the highest number of CS1 violations}
As shown in Fig.~\ref{fig:cs1PerAirline}, the Dutch airline has the highest percentage of 
violations on the fares that it offers (53.5\%). The Greek airline comes second with 13.6\% of its 
single stop fares violating CS1.
The distribution of the violations in the routes that the Dutch airline offers is shown in Table~\ref{tab:KLM}.
The violations happen in five routes. As shown, most violations of the Dutch airline (37.9\%) happen on FRA-ZRH.
Furthermore, for the first four (FRA-ZRH, BRU-STR, SVO-BRU, HEL-WAW), 
the Dutch airline causes the vast majority of the violations for these routes, 
as shown in Fig.~\ref{fig:cs1PerRoutePerAirline}.

\begin{table}[H]
\caption{Distribution of the violations per route for the Dutch airline}
\centering
\begin{tabular}{l*{10}{c}}
Route & Violations (\%) \\
\hline
FRA-ZRH & 37.88 \\
BRU-STR & 23.54 \\
SVO-BRU & 21.6 \\
HEL-WAW & 15.04 \\
MUC-IST & 1.95 \\
\end{tabular}
\label{tab:KLM}
\end{table}
The Scandinavian airline is second in absolute number of violations (218 violations). 
The distribution of violations is shown in Fig.~\ref{fig:The Scandinavian airline}. 
Almost half of the violations (48.62\%) that the Scandinavian airline has in its fares take place in FRA-ZRH. 
In total, the Scandinavian airline has CS1 violations in ten routes.

\begin{figure}[H]
\centering
\includegraphics[width=\columnwidth]{scandinavian.eps}
\caption{Distribution of the violations per route for the Scandinavian airline}
\label{fig:The Scandinavian airline}
\end{figure}

Based on the above graphs, it is clear that both airlines seem to have a 
special policy for FRA-ZRH. The majority of the violations in this route are by the Dutch airline and the Scandinavian airline.

As shown in Fig.~\ref{fig:cs1PerAirline}, the Greek airline has the second largest
percentage of CS1 violations (13.6\%). 
In the dataset, the Greek airline has all its CS1 violations in the ARN-ATH route.

A similar situation exists for the Belgian airline: half of the violations take place in 
ATH-LHR. Three more routes follow (ZRH-LHR, GVA-ATH, ZRH-MUC) with much lower frequencies.

We conclude that the violating airlines have their violations 
distributed in a number of routes (from one to ten in the case of the Scandinavian airline).
However, in each airline there is a single route where the majority (40-50\%)
of the airline's violations take place.
