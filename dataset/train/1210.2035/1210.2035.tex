\documentclass{sig-alternate}

\RequirePackage{ifthen}
\usepackage{amsmath,amssymb,url,listings,mathrsfs,wasysym}
\usepackage{tikz}
\usetikzlibrary{shadows}
\usetikzlibrary{shapes}

\usepackage[figtopcap]{subfigure}

\usepackage[all]{xy}
\usepackage{enumerate}
\usepackage{mathtools}


\newenvironment{example}{\textbf{Example 1}}{\qed}
\newenvironment{excont}[1]{\textbf{Example 1 (#1)}}{\qed}

\usepackage{graphicx}
\usepackage{psfrag}

\usepackage{color}



\let\next\undefined
\DeclareMathOperator{\next}{\mathop\bigcirc}
\DeclareMathOperator{\eventually}{\mathop\lozenge}
\DeclareMathOperator{\always}{\mathop\square}
\DeclareMathOperator{\until}{\mathbin\mathcal{U}}
\renewcommand{\v}{\varphi}
\newcommand{\e}{\epsilon}
\newcommand{\condition}{\gamma}
\newcommand{\trans}[1]{\mathrm{pr}(#1)}
\renewcommand{\d}{\delta}
\newcommand{\dropprob}{\Delta}
\newcommand{\D}{\mathcal{D}}
\newcommand{\B}{\mathcal{B}}
\newcommand{\M}{\mathcal{M}}
\newcommand{\R}{\mathcal{R}}
\newcommand{\U}{\mathcal{U}}
\newcommand{\C}{\mathcal{C}}
\newcommand{\E}{\mathcal{E}}
\newcommand{\SE}{\mathcal{E}_S}
\newcommand{\EE}{\mathcal{E}_E}
\newcommand{\G}{\mathcal{E}_G}
\newcommand{\SAP}{\mathfrak{C}}
\newcommand{\CSA}{\mathfrak{M}}
\newcommand{\Synth}{\mathfrak{S}}
\newcommand{\Realiz}{\mathfrak{R}}
\newcommand{\V}{\mathcal{V}}
\newcommand{\MSG}{\mathrm{MSG}}
\newcommand{\timeout}{\mathrm{T.O.}}
\newcommand{\cond}{\mathrm{cond}}

\newcommand{\locev}[1]{\underline{#1}}

\newcommand{\success}{\locev{\mathrm{success}}}
\newcommand{\fail}{\locev{\mathrm{fail}}}

\newcommand{\vareq}[1]{\overset{#1}{\approx}}
\newcommand{\Mmodels}[2]{\models_{\M(\d)}^{#1} #2}
\newcommand{\generated}{\models {\M(\d)}}
\newcommand{\mmodels}[1]{\models_M{#1}}
\newcommand{\proj}[1]{[\![{#1}]\!]}
\newcommand{\len}[1]{|{#1}|}
\newcommand{\fsm}[1]{\langle\!\lvert{#1}\rvert\!\rangle}
\newcommand{\csa}[2]{\langle\!\lVert{#1}\rVert\!\rangle^{#2}}
\newcommand{\bbl}[1]{\mathrm{wp}({#1})}
\newcommand{\bst}[1]{\mathrm{trg}({#1})}

\newcommand{\snd}{\mathrm{snd}}
\newcommand{\rcv}{\mathrm{rcv}}
\newcommand{\ack}{\mathrm{ack}}
\newcommand{\rack}{\mathrm{rack}}
\newcommand{\nack}{\mathrm{nack}}
\newcommand{\rnack}{\mathrm{rnack}}
\newcommand{\downcall}{\downarrow}
\newcommand{\upcall}{\uparrow}


\newcommand{\glob}[4]{{#1}_{{#2} \rightarrow {#3}}(#4)}
\newcommand{\eglob}[3]{{#1}_{{#2} \rightarrow {#3}}}
\newcommand{\env}[4]{\locev{#1}_{{#2} \rightarrow {#3}}(#4)}
\newcommand{\eenv}[3]{\locev{#1}_{{#2} \rightarrow {#3}}}
\newcommand{\sys}[4]{\locev{#1}_{{#2} \leftarrow {#3}}(#4)}
\newcommand{\esys}[3]{\locev{#1}_{{#2} \leftarrow {#3}}}

\newcommand{\define}{\sl}

\newcommand{\fig}[1]{Fig.\ \ref{fig:#1}}
\renewcommand{\sec}[1]{Sec.\ \ref{sec:#1}}
\newcommand{\tab}[1]{Table~\ref{tab:#1}}


\newcommand{\figvspace}{\vspace{-0.2in}}
\newcommand{\figsmallvspace}{\vspace{-0.1in}}




\begin{document}


\conferenceinfo{Technical report for the paper with the same title prepared for submission to ICCPS'13,} {April 8--11, 2013, Philadelphia, PA, USA.} 
\CopyrightYear{2013} 
\clubpenalty=10000 
\widowpenalty = 10000

\title{Synthesis of Reactive Protocols for Vehicle-to-Vehicle Communication -- Technical Report\thanks{This work is supported partly by the Studienstiftung des deutschen Volkes, the Boeing Corporation and the AFOSR award number FA9550-12-1-0302.}}

\numberofauthors{3}

\author{
\alignauthor
Clemens Wiltsche\vspace{0.05in} \\
       \affaddr{University of Oxford, UK}\1\jot]
       \email{utopcu@seas.upenn.edu}
\alignauthor
Richard M.\ Murray\vspace{0.05in} \\
       \affaddr{California Institute of Technology, USA}\
    \xymatrix@R=0.1cm@C=0.5cm{ {\begin{array}{l}\text{\sl Transport}\\\text{\sl Layer and}\\\text{\sl higher}\end{array}} & {\mathrm{ASC}_A} \ar@<1ex>[ddd]^<<<<{\env{\snd}{A}{B}{d}} & & {\mathrm{ASC}_B}  \ar@<1ex>[ddd]^<<<<{\eenv{\ack}{B}{A}} \\ \\ \\ {\begin{array}{l}\text{\sl Network}\\\text{\sl Layer {\phantom{anda}}}\end{array}} & *+++[][F]{M_A} \ar@<1ex>[d]^>>{!!a_{A \rightarrow B}(d)} \ar@<1ex>[uuu]^<<<<{\esys{\ack}{A}{B}} & & *+++[][F]{M_B} \ar@<1ex>[d]^>>{!!b_{B \rightarrow A}} \ar@<1ex>[uuu]^<<<<{\sys{\snd}{B}{A}{d}} \\ {\begin{array}{l}\text{\sl Data-link}\\\text{\sl Layer and}\\\text{\sl below}\end{array}} & {} \ar@<1ex>[u]^<<{?b_{A \leftarrow B}} & {\begin{array}{c}\\\\\text{Medium}\end{array}} \ar@<2.5ex>@{==}[l] \ar@<-2.5ex>@{==}[r] & {}\ar@<1ex>[u]^<<{?a_{B \leftarrow A}(d)}
    }

	\v ::=  e^p | e \rightarrow \next\v | \v \vee \v,

	\v = \glob{\snd}{A}{B}{d} \rightarrow \next (\eglob{\ack}{B}{A}^{p_1} \vee \eglob{\nack}{B}{A}^{p_2}), \label{eq:protspec}

	\always(\dropprob \leq \d) \rightarrow \v.

\begin{array}{rlll}
	(e)^p &\models e^q &\Leftrightarrow & p \geq q \\
	(e, \sigma)^p &\models e \rightarrow\next\v &\Leftrightarrow & (\sigma)^p \models \v \\
	(\sigma)^p &\models \v_1 \vee \v_2 &\Leftrightarrow &(\sigma)^p \models \v_1 \mathrm{~or~} (\sigma)^p \models \v_2,
\end{array}

\xymatrix{ & *++[o][F-]{} \ar[d]^{\snd_{x \rightarrow y}(d)} \\ & *++[o][F-]{} \ar[dl]_(0.7){\ack_{y \rightarrow x}} \ar[dr]^(0.7){\nack_{y \rightarrow x}} \\ *+[o][F-]{p_1} & & *+[o][F-]{p_2}}

	M \triangleq \langle S, \V, s^{init}, S^f, T \rangle,

\frac{
\begin{array}{cccc}
	H_1 & H_2 & \ldots & H_n
\end{array}}{
C},

\hspace{-0.3cm}\begin{array}{ll}

\text{[env]} &
\frac{
\begin{array}{c}
T(s, \env{\e}{y}{z}{d}) \vareq{\V} s'
\end{array}}{
\begin{array}{c}
\langle \rho, s \rangle \smallstep{\mathrm{e}}{y} \langle \rho + \env{\e}{y}{z}{d}, s' \rangle
\end{array}} \\lsp]

\text{[to-sys]} &
\frac{
\begin{array}{c}
T(s, (\timeout, \sys{\e}{y}{z}{d})) \vareq{\V} s'
\end{array}}{
\begin{array}{c}
\langle \rho, s \rangle \smallstep{\mathrm{t}}{y} \langle \rho + \timeout + \sys{\e}{y}{z}{d}, s' \rangle
\end{array}} \\lsp]

\text{[b-c]} &
\frac{
\begin{array}{cc}
T(s, (!!m_{y \rightarrow z}(d), \condition)) \vareq{\V} s' & \condition(s)
\end{array}}{
\begin{array}{c}
\langle \rho, s \rangle \smallstep{\mathrm{e}}{y} \langle \rho + !!m_{y \rightarrow z}(d), s' \rangle
\end{array}} \\lsp]

\text{[r-upd]} &
\frac{
\begin{array}{cc}
T(s, (?m_{y \leftarrow z}(d), \nu\!+\!+)) \vareq{\V \backslash \{\nu\}} s' & s'(\nu) = s(\nu) + 1
\end{array}}{
\begin{array}{c}
\langle \rho + ?m_{y \leftarrow z}(d), s \rangle \smallstep{\mathrm{r}}{y} \langle \rho + ?m_{y \leftarrow z}(d), s' \rangle
\end{array}}

\end{array}

\begin{array}{ll}
\text{[env]} &
\frac{
\begin{array}{c}
T(s, \env{\e}{y}{z}{d}) \vareq{\V} s'
\end{array}}{
\begin{array}{c}
\langle \rho + \env{\e}{y}{z}{d}, s \rangle \smallstep{\mathrm{e}}{y} \langle \rho + \env{\e}{y}{z}{d}, s' \rangle
\end{array}}
\end{array}

\hspace{-0.5cm}\begin{array}{ll}


\text{[trans]} &
\frac{
\begin{array}{c}
\langle \rho + ?m_{z \leftarrow y}(d), s_z \rangle \smallstep{\mathrm{r}}{z} \langle \rho', s_z' \rangle \\ \varsigma = !!m_{y \rightarrow z}(d)
\end{array}}{
\begin{array}{c}
\langle (\rho + \varsigma)^p, s, y \rangle \bigstep{} \langle (\rho')^{(1-\d)p}, s[z \leftarrow s_z'], z \rangle
\end{array}} \\lsp]

\text{[nacc]} &
\frac{
\begin{array}{c}
\neg(\langle \rho + ?m_{z \leftarrow y}(d), s_z \rangle \smallstep{\mathrm{r}}{z} \langle \rho', s_z' \rangle) \\ \varsigma = !!m_{y \rightarrow z}(d) \hspace{1cm} \langle \rho, s_x \rangle \smallstep{}{x} \langle \rho', s_x' \rangle
\end{array}}{
\begin{array}{c}
\langle (\rho + \varsigma)^p, s, y \rangle \bigstep{} \langle (\rho)^p, s, z \rangle
\end{array}} \\lsp]

\text{[pr-t]} &
\frac{
\begin{array}{cl}
\varsigma \neq !!m_{y \rightarrow z}(d) & \neg(\langle \rho + \varsigma, s_y \rangle \smallstep{\mathrm{e}}{y} \langle \rho'', s_y'' \rangle) \\ & \langle \rho + \varsigma, s_y \rangle \smallstep{\mathrm{t}}{y} \langle \rho', s_y' \rangle
\end{array}}{
\begin{array}{c}
\langle (\rho + \varsigma)^p, s, y \rangle \bigstep{} \langle (\rho')^p, s[y \leftarrow s_y'], y \rangle
\end{array}} \
\caption{Global semantics for deducing behavior of several CSAs . In particular, the rules [trans], [drop] and [nacc] model the transmission medium.}
\label{tab:compositionrules}
\figvspace
\end{table}




In the global semantics, we are interested in ensuring that several CSAs together satisfy the global protocol specification by interacting with each other. The transmission medium therefore acts as an arbiter or scheduler of transitions. Hence, we can think of the global behavior of several CSAs  as an interleaving  of the locally generated sequences  of the respective CSAs. Messages are only transmitted with a certain probability. Hence, the sequence  is tagged with a probability , indicating how likely it occurs.


The relation   defines the global semantics according to the rules in \tab{compositionrules}. It means that  with drop probability  at state  transforms  into  by making a transition to state  while the priority changes from  to . In the statement of the rules, updating the  element  in state  with  is written as .

The [trans], [drop] and [nacc] rules define the transmission medium dynamics. If the last deduced element in the sequence is a broadcast message, i.e.\ , the medium tries to transmit. An application of the [trans] rule models a successful message transmission. This only occurs if the CSA for which the message was destined,  makes a transition labelled with the corresponding reception. That is,  is only satisfied if  can execute [r-sys] or [r-upd]. Since a message transmission occurs with probability , the probability with which the sequence  is tagged in the conclusion of [trans] is .

An application of the [drop] rule models a dropped message. It has exactly the same hypotheses as [trans], but its conclusion reflects that no progress has been made. The sequence  is tagged with  due to the message drop probability . Note that the priority is at the source CSA , which may now execute a timeout transition (if enabled).

The [nacc] rule is applied when a message should be transmitted, but the destination CSA has no transition enabled that is labelled by the corresponding reception. Similar to the [drop] rule, no progress is made. Also, the probability of the deduced sequence is not affected.

The [pr-e], [pr-t] and [npr] rules may be applied if the last element of the sequence is not a message transmission. Then the transmission medium is inactive, and and the CSA that is currently prioritized may execute: If a transition that is not a timeout or reception is enabled, then [pr-e] is applied. If a timeout transition is enabled, then [pr-t] is applied. The [npr] rule may only applied if the currently prioritized CSA has no such transitions enabled. In this case, any CSA  may execute.

The transitive closure  denotes that  is transformed into  in an arbitrary number of deduction steps. The CSAs  execute by starting in state  with an empty 1-sequence  and any CSA  prioritized. Valid deductions are the tuples  for which .

Note that an example of how the local and global semantic rules are used is included in \sec{Pdeduction}.












\subsubsection{Global Event Sequences}

By applying the deduction rules, sequences over both envi\-ronment-triggered and system-triggered events, broadcasts, receptions and timeouts can be obtained from a set of CSAs. Since protocol specifications are over global events, we need to extract the synchronizations of local events in the sequences generated by a set of CSAs. We therefore define the projection function  to find a sequence over global events from . It is defined by

We use the projection function  to express whether a set of CSAs satisfies a protocol specification  under the environment assumptions .






\subsection{Correctness} \label{sec:correctness}



In this section we define correctness of a protocol's implementation in form of a set of CSAs  with respect to a specification . If  satisfies this specification this is written as . Correctness depends on the probability of sequences  being synchronized correctly by the CSAs  if the transmission medium's drop probability  is bounded from above by , i.e.\ it satisfies . If this assumption on the transmission medium is not satisfied, the specification  is trivially satisfied by any set of CSAs. However, this case is useless in practice, as the protocol will not deliver the data with the required QoS.


\subsubsection{Definitions}

Given a protocol specification , correctness of an implementation depends on whether all CSAs involved in synchronizing a sequence of global events are in a final state. We therefore define the set of {\define globally final} states  to include all tuples of states  so that if there is some sequence involving CSAs , the states  are actually final states from .

We say that a -sequence  is {\define generated} by a set of CSAs  and drop probability , and write , if it can be deduced by the rules in \tab{singleCSArules} and \tab{compositionrules} and the deduction ends in a globally final state . Formally,

As noted above in \sec{protocol}, a -sequence  satisfies a specification  exactly if the probability  that all (global) events in  are correctly synchronized is high enough given that the corresponding environment-triggered events are all triggered through calls by the higher level ASCs. For a set of CSAs therefore to satisfy a specification, it is required that the synchronization of events in each sequence is performed with high enough probability.


\subsubsection{Correctness Condition}

The important criterion for correctness is not whether a sequence  is generated, but whether the QoS requirements are satisfied. This is because the decisions between environment-triggered events (which essentially generate the sequence) are made by the higher level ASCs, over which a CSA has no control. For example, in \fig{services}, the receiver CSA has no control over whether  or  is triggered by its ASC in state . In our case the only QoS requirement is the probability of all global events being correctly synchronized, so the question for correctness becomes: Given that the ASCs trigger the events necessary to generate , how likely is it that all synchronizations performed?

 might generate a given sequence  in many different ways, since several sequences  deducible by the rules in \tab{singleCSArules} might satisfy . For a sequence , we evaluate the sum of all probabilities  for distinct sequences  that satisfy

and get the probability

expressing the likelihood of the events in the sequence  being correctly synchronized when executing all CSAs in  in parallel (i.e. using the global semantics). {\define Correctness} then is expressed by

i.e.\ if  is a sequence allowed by the specification ,  synchronizes the events  at least as likely as it is required.\\
The algorithmically challenging part in establishing correctness is to evaluate . However, we only need to compute this for the CSAs that we are synthesizing.












\vspace{-.06in}



\section{Synthesis} \label{sec:method}

The protocol synthesis method  translates a specification into a set of CSAs that is guaranteed to satisfy the specification. The inputs to the synthesis are a protocol specification , a set of cars  and the specification on the transmission medium dynamics .  produces a CSA for each car  that interacts with the higher level ASCs as outlined in \sec{preliminaries}.



\subsection{Realizability and Well-Posedness}


Synthesis is preceded by a realizability check, i.e.\ checking whether a specification \emph{can} be implemented. That is, checking realizability amounts to deciding whether there exists a set of CSAs that satisfies the specification . If a protocol specification  is realizable for a set of cars  under a drop probability , this is written as .

Checking realizability consists of two parts: Firstly, the specification itself must be well-posed, i.e.\  must admit a ``reasonable'' implementation in the form of CSAs. Secondly, it must be possible to find retransmission bounds so that the QoS requirements are satisfied under the given drop probability .

Well-posedness is a purely syntactic requirement on the specification. We introduce this concept because it is easy to check and simplifies the presentation of the synthesis algorithm. A protocol specification  is {\define well posed} if on every -sequence satisfying , two ASCs take turns in triggering the events, and there are at least two events on each path through the tree induced by the specification.

These rather strict requirements on the specifications for well-posedness can be relaxed by generalizing the synthesis method presented in the next section appropriately. For example, a straightforward relaxation would be to allow protocol specifications in which for any disjunction , the system-triggered events corresponding to the immediately following global events are all triggered by the same ASC.

We do not develop a separate test for realizability but rather show how our method fails for well-posed but nonrealizable specifications.


\subsection{Synthesis Algorithm}


The synthesis method is implemented in two parts: First, the retransmission bounds are calculated. Then the CSAs are constructed using the retransmission bounds. The retransmission bounds are calculated with the structure of the resulting CSAs in mind, so we present the CSA construction first.

For any specification  and any set of cars , if the specification is realizable, the resulting set  of CSAs from the synthesis,  must satisfy . Formally,

The synthesis method  is implemented in two parts: First, the retransmission bounds are calculated. Then the CSAs are constructed using the retransmission bounds. The retransmission bounds are calculated with the structure of the resulting CSAs in mind, so we present the CSA construction first.


\subsubsection{CSA Construction}

\tab{synthesis} shows the algorithm {\sc{Synthesize}}. This algorithm constructs the CSA  for car  from the specification . The parameter  is used to uniquely index states in the CSA, and  is a set of global events that is used to construct appropriate criteria for retransmission (explained below).  is the list of retransmission bounds calculated in the first step (cf.\ \sec{retransmission}).

Each global event  that occurs in the protocol specification  is assigned an environment-triggered event , a system-triggered event , a message , a variable (as retransmission counter) , a retransmission bound  from , and system-triggered events  and .

The algorithm is invoked by {\sc{Synthesize}}, for each car :\footnote{Note that  does not need to occur in the protocol specification .} It synthesises a CSA for the well-posed protocol specification  for car , where states are indexed starting from , no previous events are stored () and the retransmission bounds  are used.

{\sc{Synthesize}} recursively decomposes  into its subparts. If , then two CSAs  and  are constructed from  and  first and joined together by forming the union of their state spaces, final states and transitions and substituting the initial state  by the initial state . For this purpose we define  to be the CSA  with all occurrences of  substituted by .

If , the set  of global events that has last been received on the path through the CSA is updated first. Then again the CSA  for  is constructed. Depending on which car  the CSA is constructed for, different transitions are now introduced. If  then the ASC on car  is responsible for triggering the event , and a retransmission loop is introduced:

\begin{minipage}{0.1\textwidth}
(I)
\end{minipage}
\begin{minipage}{0.2\textwidth}

\end{minipage}

\noindent If , then car  synchronizes  by the system-triggered event :

\begin{minipage}{0.1\textwidth}
(II)
\end{minipage}
\begin{minipage}{0.2\textwidth}

\end{minipage}

\noindent In any other case, simply the CSA for  is returned as then the car  is not directly involved in the transmission.

Finally, if  then no recursive call to {\sc{Synthesize}} is necessary, but a CSA is directly constructed. If  then a retransmission loop is constructed:

\begin{minipage}{0.1\textwidth}
(III)
\end{minipage}
\begin{minipage}{0.2\textwidth}

\end{minipage}

\noindent In this case, a retransmission is not triggered by a timeout, because  is the last global event in a sequence of required synchronizations and no feedback from the car  can be expected. Therefore, a retransmission is initiated by receiving the last message  from car  again, because this indicates that  has not received the message  correctly. The message  is taken from , the set of global events that has last been received on the path through the CSA. Only if no such message is received is a timeout transition made, which indicates success by an upcall to the ASC. The global semantics of CSAs were carefully constructed so that this timeout is only taken if no message  is received.

If , then car  synchronizes  by the system-triggered event :

\begin{minipage}{0.1\textwidth}
(IV)
\end{minipage}
\begin{minipage}{0.2\textwidth}

\end{minipage}

\noindent In any other case a trivial CSA with one state is returned.

\begin{excont}{Continued}
The resulting CSAs from synthesizing the specification in \eqref{eq:protspec} are shown in \fig{services}. The CSAs were generated by calling {\sc{Synthesize}} for . The retransmission bounds  are calculated as explained in the next section according to the QoS requirements and to the bound on the drop probability .
\end{excont}

\newcommand{\tb}{\hspace{0.7cm}}


\begin{table}[ht]
\begin{tabular}{l}
\sc{Synthesize}\\
\tb\textbf{If}  \textbf{Then}\\
\tb\tb~\sc{Synthesize}\\
\tb\tb~\sc{Synthesize}\\
\tb\tb\textbf{Return} \\
\tb\tb\tb\tb\tb\tb \\
\tb\textbf{Else If}  \textbf{Then}\\
\tb\tb\textbf{If}  \textbf{Then}\\
\tb\tb\tb\textbf{Replace}  \textbf{By}  \textbf{In} \\
\tb\tb\textbf{Else}\\
\tb\tb\tb\textbf{Insert}  \textbf{Into} \\
\tb\tb\textbf{If}  \textbf{Then}\\
\tb\tb\tb~\sc{Synthesize}\\
\tb\tb\tb\hspace{-1cm}(I)\\
\tb\tb\tb\textbf{Return} \\
\tb\tb\textbf{Else If}  \textbf{Then}\\
\tb\tb\tb~\sc{Synthesize}\\
\tb\tb\tb\hspace{-1cm}(II)\\
\tb\tb\tb\textbf{Return} \\
\tb\tb\textbf{Else}\\
\tb\tb\tb\textbf{Return} \sc{Synthesize}\\
\tb\textbf{Else If}  \textbf{Then}\\
\tb\tb\textbf{If}  \textbf{Then}\\
\tb\tb\tb\\
\tb\tb\tb\hspace{-1cm}(III)\\
\tb\tb\tb\textbf{Return} \\
\tb\tb\tb\tb\tb\tb \\
\tb\tb\textbf{Else If}  \textbf{Then}\\
\tb\tb\tb\hspace{-1cm}(IV)\\
\tb\tb\tb\textbf{Return} \\
\tb\tb\textbf{Else}\\
\tb\tb\tb\textbf{Return} \\
\end{tabular}
\caption{Pseudocode of synthesis algorithm. The CSA is constructed from the diagrams explained in the text and referred to by Roman numerals.}
\label{tab:synthesis}
\figvspace
\end{table}





\subsubsection{Retransmission Bounds} \label{sec:retransmission}

Each global event  gets assigned a unique message  and a unique retransmission bound . The retransmission bounds are evaluated according to the QoS requirements defined in the protocol specification .

Recall that the protocol specification  induces a tree, cf.\ \fig{protspec}. Each edge of this tree is translated by the synthesis into a retransmission loop in the CSA of exactly one car, with a retransmission bound associated with that loop. The retransmission bounds have to be selected so that correctness as defined in \sec{correctness} is guaranteed.

We use the semantics to find the conditions on the retransmission bounds that are sufficient for correctness. We can exploit the tree-like structure of the synthesized CSAs: Apart from the last two retransmission loops in each sequence, the message associated with a retransmission loop is never used at a later point in the same sequence.

Each sequence of global events  is associated with a sequence of retransmission bounds . Depending on the values of the retransmission bounds, the sequence  is generated correctly with a certain {\define synchronization probability}  that depends on all retransmission bounds associated with any event in . Note that this probability is the likelihood of  being generated correctly \emph{given} that the calls are made by the ASCs that generate .


\subsubsection{Deduction of } \label{sec:Pdeduction}

The synchronization probability  is evaluated as follows: The case of only one retransmission bound () never occurs. The case of exactly two retransmission bounds () means that the last retransmission loop uses the message transmitted in the retransmission loop one before last, cf.\ \fig{services}. The example in this figure can be used to deduce the general expression for . This is because the synthesis will always generate the same pattern for the last two global events in a specification.

We use the sequence  to deduce  by evaluating the probability of correct synchronization by applying the deduction rules in \tab{singleCSArules} and \tab{compositionrules}. The CSAs  start in the initial state . We omit writing the values of the retransmission counters within the states in our presentation. We let , where  are an auxiliary functions describing the probability of reaching a globally final state from  when the calls in  are made.

Initially only  can execute by applying the [env] rule locally. Hence, we start the deduction at . We find  by applying the global rules to deduce all sequences for which  ends in a globally final state and the calls in  are made. First, apply [pr-e] globally and [env] locally and deduce

globally from

locally. This deduction step yields , as the probability is not changed from state  to . From now on we omit the left hand side of the relation , since it is equivalent to the right hand side of the previous deduction. We further omit writing the local deductions.

At , globally only the [pr-e] rule can be applied. Locally, either [sys-c] or [b-c] can be applied, depending on the value of the retransmission counter . Applying [sys-c] corresponds to taking the transition labelled by . Since then no final state can ever be reached, we only apply the [b-c] rule locally. So we apply locally the [b-c] rule:

This transition leads to . At this point the transmission medium is invoked and globally both the [trans] and [drop] rules can be applied. If [trans] is applied,  receives the message and we apply the [r-sys] rule locally. If [drop] is applied, merely the probability and prioritization changes. Hence, we can deduce either

where . Since two transitions may be taken, we get . After the application of [drop],  is prioritized but cannot make a transition. In state , globally only [npr] can be applied, with  making a timeout transition using [to-upd] locally:

Since the application of [to-upd] increases the retransmission counter  by one, we get  and the base case . This indicates that in state , the deduction may be repeated with the bound  decreased by one, corresponding to a retransmission. If , i.e.\ in the base case, no more retransmissions are possible.

In state  after the transmission,  makes a transition in response to a call from its ASC. By applying [env] instead of [env], we model that a is made immediately. Applying [pr-e] globally and [env] locally yields:

where . Then only [pr-e] with [b-c] can be applied (because again, applying [sys-c] does not conform with wanting to reach a final state). Therefore we get

where . This generates the equalities  and .

In  we can apply either [trans] globally with [r-sys] locally on , modelling a successful transmission, or we apply [drop] globally, modelling a dropped message. Hence we can either deduce

where . We get . In state , the sequence has been synchronized successfully. Here only [npr] with [to-sys] on  can be applied to yield

where  . This deduction ends in a globally final state and hence , because the sequence  is correctly synchronized. In state  after the message has been dropped, only [pr-t] with [to-upd] locally on  can be applied:

where . This step yields  with base case . Now  retransmits (if its retransmission count is not yet exceeded) and we deduce with [pr-e] and [b-c] locally on :

where . This yields . Now [trans] can be applied with [r-upd] locally on . However, when applying [drop], no final state can be reached by any sequence of applications of deduction rules. Hence we only apply [trans] and [r-upd] and get

where . This yields  with base case .


\subsubsection{Optimization Problem}

For notational convenience, we drop the  superscript if the context is clear and we are not referring to a particular sequence. When the sequence of global events  has exactly two elements (), we get
\newcommand{\dd}{\varrho}

where  is the {\define reception probability} and . When  has more than two elements (), the synchronization probability can be similarly deduced:

where . Ideally, we want to find the smallest retransmission bounds that ensure correctness. Each -sequence  that satisfies the specification  induces a condition on the retransmission bounds associated with the elements of . For example, a sequence  induces the condition  . This inequality ensures that the sequence  is generated by the CSAs with high enough probability as required by the correctness criterion set out above.

We can find the retransmission bounds by solving an optimization problem:

where  is the set of -sequences  that satisfy the protocol specification .

\begin{excont}{Continued}
Checking realizability of a specification amounts to checking well-posedness of the specification and feasibility of the optimization problem. For our example specification \eqref{eq:protspec}, the optimization problem is

In the case that ,  and , we get , , and .
\end{excont}


\subsection{Correctness of Synthesis}

Take any protocol specification , drop probability bound , and any -sequence  for which  holds. Then, correctness of the synthesis method is established by showing that , where  is the result of synthesis.

The definition of the feasible region of the optimization problem (OPT) contains the inequality  for each such sequence . By the semantics of protocol specifications  for any sequence . It is therefore sufficient to show that  and , because then , as required to establish correctness.

First, if the retransmission bounds  are part of a feasible solution to (OPT), then we necessarily have , and so  follows from .

Second, we have  by construction of  (note that the superscript  has been dropped from ):  is the sum of all probabilities  for which , i.e.\ the environment-triggered events and system-triggered events in  synchronize to the sequence of global events  and the -sequence  is generated by  and drop probability . By definition, . It is therefore sufficient to show that in the deduction of the expression for  exactly those -sequences  are taken into account that end in a globally final state  (the prioritization of  and  can safely be ignored) and for which .

The deduction of  in \sec{Pdeduction} is essentially done by constructing a product automaton of all CSAs in  using the global semantics, and adding the probabilities along all paths that end in a globally final state corresponding to the sequence  having been executed.

Note that it would have been enough to show . A synthesis method that generates CSAs with  would be perfectly correct, but not very useful: The larger  gets, the greater the feasible region of (OPT) gets and the more specifications can be synthesized. So by having , we have maximized the capabilities of the synthesis method.


\subsection{Computational Considerations} \label{sec:computation}

The time required to generate a CSA from a protocol specification  by the {\sc{Synthesize}} algorithm is proportional to the number of global events and disjunctions () in  (ignoring the set operations on ), which can easily be seen from \tab{synthesis}, where the implementation of {\sc{Synthesize}} is shown as a simple structural recursion on . When also the set operations on  are taken into account, the algorithm is quadratic in the number of global events in .

The main computational complexity arises from the optimization problem OPT, which is an integer program and in general is NP-hard. There are however a few points to be noted that may simplify finding a solution. First, both the objective function and the function  are monotonous in their arguments along any dimension. Hence, if OPT is feasible for some , it is also feasible for any .

Second, since correctness depends on , it is sufficient to solve an optimization problem with a strictly smaller feasible set than that of (OPT). This is helpful if a function  can be found s.t.\ for all ,  while still maintaining that there exist retransmission bounds  s.t.\  for all . The solution to the resulting optimization problem might not be optimal, but the resulting CSAs are still correct.

Lastly, since any suboptimal solution to OPT still gives rise to correct CSAs,  the retransmission bounds may be chosen to be arbitrarily high as long as they are feasible. Note however that there might not be a solution to OPT at all, in which case the specification was unrealizable in the first place.


\subsection{Discussion} \label{sec:validity}

\begin{figure}
\centering
	\psfrag{data length}[cc][cc]{data length }
	\psfrag{number of cars}[rb][rt]{N}
	\psfrag{minimum delay}[cc][cc]{minimum delay }
	\includegraphics[width=0.45\textwidth]{feasibility9.eps}
\figsmallvspace
\caption{Feasible region of OPT for the protocol specification in \eqref{eq:protspec} with . The drop probability bound  is calculated as a function of ,  and . All points on and under the surface are feasible.}
\label{fig:feasibility6}
\figvspace
\end{figure}

The implementations of a communication protocol specification provide the ASCs with sufficient information on what messages are received so that accidents can effectively be prevented. In this section we develop the continuing example of the cars at an intersection further by explaining how our protocol can be embedded in an active safety application.

\begin{excont}{Continued}
When transmitting data  from car  to car , six cases can occur. We distinguish the cases by the final system-triggered events that generate upcalls to the ASCs on either car. The case we call ``correct'' is when  receives ,  knows about it and  assumes correctly that  knows.  then correctly receives a ``'' upcall, which is consistent with 's last upcall. The ASCs can then correctly react in a consistent way, e.g.\ by one car gracefully decelerating.

In other cases the ASCs can still react in a safe way even if  and  have inconsistent information about each other: If  receives  correctly,  never receives an acknowledgement and  assumes  never did, then both ASCs receive ``'' upcalls and can react accordingly. If  receives  correctly and  receives the acknowledgement, but  assumes  did not receive it, then  receives a ``'' upcall and can react conservatively. If  does not receive  and  holds that it did not, then the ASC on  can react conservative on its ``'' upcall. If  receives  correctly,  misses the acknowledgement but  holds that  received it, then  incorrectly assumes the worst case but yet reacts conservatively.

The only problematic case is when  does not receive  but  holds that it did. Then the ASC on neither  nor   takes conservative action, potentially resulting in an accident. However, the synthesis method constructs the CSAs so that this case never occurs under the given assumptions.
\end{excont}

We now conclude the example by presenting numerical results that illustrate in which hypothetical scenarios protocols that we are considering are realizable.

\begin{excont}{Continued}
As introduced above, the drop probability bound  on the transmission medium may be calculated from other more readily available parameters. The realizability of a given protocol specification  depends on the drop probability bound . For demonstrative purposes, we calculate  from the number of cars  at the intersection that may use the transmission medium simultaneously, the minimum time  it may take for a message to be sent between two cars and the maximum amount of data  that may be carried in a message. Given an empirically obtained function  that maps a data-rate  to a drop-probability of the transmission medium, we calculate  with  (we take  as we consider the environment to be all cars except the two that are communicating).

We illustrate the effectiveness of our synthesis method by asserting the sigmoid  with  and . Using the protocol specification in \eqref{eq:protspec}, we illustrate how realizability changes with different values for the number of cars , minimum time to deliver a message  and maximum amount of data in a message . \fig{feasibility6} shows the feasible region of (OPT) for  with , i.e.\ for which values of  calculated as a function of ,  and  the synthesis problem is realizable.

It is clearly visible from \fig{feasibility6} that the more cars are sharing the transmission medium, the smaller the delay, and the larger the packets, the higher the worst-case data rate could be on the network, and the specification becomes harder to realize. If moreover the requirements  and  are made more stringent, the feasible region decreases even further.
\end{excont}



\vspace{-.06in}


\section{Conclusion} \label{sec:conclusion}


This work demonstrates a framework for reliable communication protocols for intervehicular communication in active safety applications. The framework, consisting of a precisely defined specification language and execution model (in the form of CSAs), allows for correct-by-construction synthesis of protocol implementations that satisfy the specifications even in the presence of several other cars sharing the transmission medium.

In our synthesis method we only take into account the drop probability of the transmission medium and assume that this is sufficient to synthesise reliable protocols. This also only enables to guarantee QoS requirements on the reception probability. Furthermore, in the current formulation, only two cars can participate in a dialogue, but some active safety applications might require to extend this. Also, note that if a communication is under way, the arrival of another message cannot directly be handled even if it is required to satisfy the QoS requirements.

Our approach permits several extensions: (i) Allowing the higher level to specify the QoS requirements and the destination address at runtime (i.e.\ for each transmission), (ii) Guaranteeing QoS requirements on the end-to-end delay of the communication and more general assumptions on the transmission medium dynamics in order to widen the range of applicability, and (iii) including the capability to relay messages over several cars to create a routed network. The latter would also require a rigorously developed synthesis method for protocols to discover the network topology, which we are currently working on.\\

\vspace{-.06in}

\section*{Acknowledgements} The authors would like to extend thanks to Rohit Pandita and Vladimeros Vladimerou from Toyota as well as Scott Livingston, Pavithra Prabhakar and Eric Wolff at the California Institute of Technology for fruitful discussions.



\bibliographystyle{abbrv}
\small{
\bibliography{iccps2013-wiltsche}
}


\end{document}
