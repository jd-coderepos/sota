\iffalse
FOCS critical:
    submit
    format?
    refutation of BDM


    complete bibliography
FOCS todo:
    JL statement sloppy
    prop:outline-correct : proof needs polishing.
    are we really using m>=n somewhere?
    after thm 9,
    most important alg is the one with random projection.
    define eps-approximation...
    polytime vs arithmetic ops
    discuss running times with Amit: kmn^2 or kmn^w?
    explain how this is "feature selection"
    similarity with det processes and rand spanning trees.
    from abstract: \item \comment{improving DRVW to  passes? anything else related to spanning trees or computing permanents of low-rank matrices?}
    fix lemma 11 and 16?
TODO:
    S=empty: \pi, det
    notation .
    1+eps approximate volume sampling
    time means arithmetic operations. small comment about polytime.
    define \omega
    T disjoint from S, i, etc.
    template -> outline?
    if rationals, size of rationals?
    use number of non-zeros instead of mn when possible.
    probabilities mostly coeffs. of suitable characteristic polynomial
    don't like running time of GvL svd, not rigorous enough at first reading, there is an iterative procedure...

tuples vs subsets of rows
connection with det. sampling, random spanning trees
css gu eisenstat BDM, frobenius and l_2
explain exactly how this is feature selection.
connection with svd, motivation for "actual rows" vs linear combinations. dna
tropp
High level intuition, two choices: either baur strassen or rank-1 updates.
number of passes, 1 ,2? pass efficient, streaming.
improves bdm substantially? running time, approx. factor, number of passes?
derandomization of exact
practical algorithms

compute characteristic polynomial:
    from eigenvalues (numerically practical)
    krilov (what to do in the singular case, also unstable)

issues with Jeannerod and Villard: genericity and powers of 2
\fi

\documentclass[11pt]{article}

\usepackage{mathtools}
\newcommand{\RR}{\mathbb{R}}
\newcommand{\norms}[1]{{\lVert#1\rVert}^2}
\newcommand{\snorms}[1]{{\lVert#1\rVert}_2^2}
\newcommand{\rowmatrix}[4]{
\begin{pmatrix}
#1 & #2 \\
#3 & #4
\end{pmatrix}
}
\newcommand{\keywords}[1]{\emph{Keywords:} #1}
\newcommand{\trace}{\operatorname{trace}}
\newcommand{\rank}{\operatorname{rank}}
\newcommand{\card}[1]{\lvert{#1}\rvert}
\newcommand{\giventhat}{\: / \:}
\newcommand{\proj}{\pi}
\newcommand{\NN}{\mathbb{N}}

\usepackage{fullpage,amssymb,amsmath,amsthm,framed,color}
\usepackage{hyperref}

\newtheorem{theorem}{Theorem}
\newtheorem{prop}[theorem]{Proposition}
\newtheorem{lemma}[theorem]{Lemma}
\newtheorem{definition}[theorem]{Definition}
\newtheorem{conj}[theorem]{Conjecture}
\newtheorem{coro}[theorem]{Corollary}
\newtheorem{alg}{Algorithm}

\def\reals{\mathbb{R}}
\def\eps{\epsilon}
\def\suchthat{\;:\;}
\def\given{\;|\;}

\newcommand{\tr}[1]{\operatorname{trace}\left(#1\right)}
\newcommand{\deter}[1]{\operatorname{det}\left(#1\right)}
\newcommand{\linspan}[1]{\operatorname{span}\left(#1\right)}
\newcommand{\vol}[1]{\operatorname{vol}\left(#1\right)}
\newcommand{\conv}[1]{\operatorname{conv}\left(#1\right)}
\newcommand{\norm}[1]{\left\|#1\right\|}
\newcommand{\frob}[1]{\left\|#1\right\|_{F}}
\newcommand{\prob}[1]{{\sf Pr} \left(#1\right)}
\newcommand{\expec}[1]{{\sf E} \left[#1\right]}
\newcommand{\abs}[1]{\left|#1\right|}
\newcommand{\size}[1]{\left|#1\right|}
\newcommand{\inner}[2]{\left\langle#1, #2\right\rangle}

\newcommand{\comment}[1]{{\sf\textcolor{blue}{#1}}}

\title{Efficient volume sampling for row/column subset selection}
\author{Amit Deshpande \\
Microsoft Research India \\
\url{amitdesh@microsoft.com}
\and
Luis Rademacher \\
Computer Science and Engineering \\
Ohio State University \\
\url{lrademac@cse.ohio-state.edu}}

\date{}

\begin{document}
\maketitle

\begin{abstract}
We give efficient algorithms for volume sampling, i.e., for picking -subsets of the rows of any given matrix with probabilities proportional to the squared volumes of the simplices defined by them and the origin (or the squared volumes of the parallelepipeds defined by these subsets of rows).
This solves an open problem from the monograph on spectral algorithms by Kannan and Vempala (see Section  of \cite{KV}, also implicit in \cite{BDM, DRVW}).

Our first algorithm for volume sampling -subsets of rows from an -by- matrix runs in  arithmetic operations and a second variant of it for -approximate volume sampling runs in  arithmetic operations, which is almost linear in the size of the input (i.e., the number of entries) for small .

Our efficient volume sampling algorithms imply the following results for low-rank matrix approximation:
\begin{enumerate}
\item Given , in  arithmetic operations we can find  of its rows such that projecting onto their span gives a -approximation to the matrix of rank  closest to  under the Frobenius norm. This improves the -approximation of Boutsidis, Drineas and Mahoney \cite{BDM} and matches the lower bound shown in \cite{DRVW}. The method of conditional expectations gives a \emph{deterministic} algorithm with the same complexity. The running time can be improved to  at the cost of losing an extra  in the approximation factor.
\item The same rows and projection as in the previous point give a -approximation to the matrix of rank  closest to  under the spectral norm. In this paper, we show an almost matching lower bound of , even for .
\end{enumerate}
\end{abstract}

\keywords{volume sampling, low-rank matrix approximation, row/column subset selection}

\section{Introduction}

Volume sampling, i.e., picking -subsets of the rows of any given matrix with probabilities proportional to the squared volumes of the simplicies defined by them, was introduced in \cite{DRVW} in the context of low-rank approximation of matrices. It is equivalent to sampling -subsets of  with probabilities proportional to the corresponding  by  principal minors of any given  by  positive semidefinite matrix.

In the context of low-rank approximation, volume sampling is related to a problem called \emph{row/column-subset selection} \cite{BDM}. Most large data sets that arise in search, microarray experiments, computer vision, data mining etc. can be thought of as matrices where rows and columns are indexed by objects and features, respectively (or vice versa), and we need to pick a small subset of features that are dominant. For example, while studying gene expression data biologists want a small subset of genes that are responsible for a particular disease. Usual dimension reduction techniques such as principal component analysis (PCA) or random projection fail to do this as they typically output singular vectors or random vectors which are linear combinations of a large number of feature vectors. A recent article by Mahoney and Drineas \cite{MD} highlights the limitations of PCA and gives experimental data on practical applications of low-rank approximation based on row/column-subset selection.

\section{Row/column-subset selection and volume sampling}
While dealing with large matrices in practice, we seek smaller or low-dimensional representations of them which are close to them but can be computed and stored efficiently. A popular notion for low-dimensional representation of matrices is low-rank matrices, and the most popular metrics used to measure the closeness of two matrices are the Frobenius or Hilbert-Schmidt norm (i.e., the square root of the sum of squares of entries of their difference) and the spectral norm (i.e., the largest singular value of their difference). The singular value decomposition (SVD) tells us that any matrix  can be written as

where ,  are orthonormal and  are orthonormal. Moreover, the nearest rank- matrix to , let us call it , under both the Frobenius and the spectral norm, is given by

In other words, the rows of  are projections of the rows of  onto . Because of this, most dimension reduction techniques based on the singular value decomposition, e.g., principal component analysis (PCA), are interpreted as giving 's as the \emph{dominant} vectors, which happen to be linear combinations of a large number of the rows or feature vectors of .

The \emph{row-subset selection} problem we consider in this paper is: Can we pick a -subset of the rows (or feature vectors) of  so that projecting onto their span is almost as good as projecting onto ?

Several row-sampling techniques have been considered in the past as an approximate but faster alternative to the singular value decomposition, in the context of streaming algorithms and large data sets that cannot be stored in random access memory \cite{FKV,DMM,DRVW}. The first among these is the squared-length sampling of rows introduced by Frieze, Kannan and Vempala \cite{FKV}. Another sampling scheme due to Drineas, Mahoney and Muthukrishnan \cite{DMM} uses the singular values and singular vectors to decide the sampling probabilities. Later Deshpande, Rademacher, Vempala and Wang \cite{DRVW} introduced volume sampling as a generalization of squared-length sampling.
\begin{definition} \label{def:vol-sampling}
Given , volume sampling is defined as picking a -subset  of  with probability proportional to

where  denotes the -th row of ,  denotes the row-submatrix of  given by rows with indices , and  denotes the convex hull.
\end{definition}

The application of volume sampling to low-rank approximation and, more importantly, to the \emph{row-subset selection} problem, is given by the following theorem shown in \cite{DRVW}. It says that picking a subset of  rows according to volume sampling and projecting all the rows of  onto their span gives a -approximation to the nearest rank- matrix to .
\begin{theorem} \cite{DRVW} \label{thm:vol-sampling}
Given any ,

when  is picked according to volume sampling,  denotes the matrix obtained by projecting all the rows of  onto , and  is the matrix of rank  closest to  under the Frobenius norm.
\end{theorem}
As we will see later, this easily implies

Theorem \ref{thm:vol-sampling} gives only an existence result for row-subset selection and we also know a matching lower bound that says this is the best we can possibly do.
\begin{theorem} \cite{DRVW} \label{thm:DRVW-lower}
For any , there exists a matrix  such that picking any -subset  of its rows gives

\end{theorem}

However, no efficient algorithm was known for volume sampling prior to this work. An algorithm mentioned in Deshpande and Vempala \cite{DV} does -approximate volume sampling in time , which means that plugging it in Theorem \ref{thm:vol-sampling} can only guarantee -approximation instead of . Finding an efficient algorithm for volume sampling is mentioned as an open problem in the recent monograph on spectral algorithms by Kannan and Vempala (see Section  of \cite{KV}).

Boutsidis, Drineas and Mahoney \cite{BDM} gave an alternative approach to row-subset selection (without going through volume sampling) and here is a re-statement of the main theorem from their paper which uses columns instead of rows.


\begin{theorem} \cite{BDM} \label{thm:bdm-soda}
For any , a -subset  of its rows can be found in time  such that

\end{theorem}

Row/column-subset selection problem is related to rank-revealing decompositions considered in linear algebra \cite{GE,P}, and the previous best algorithmic result for row-subset selection in the spectral norm case was given by a result of Gu and Eisenstat \cite{GE} on strong rank-revealing QR decompositions. The following theorem is a direct consequence of \cite{GE} as pointed out in \cite{BDM}.
\begin{theorem}
Given , an integer  and , there exists a -subset  of the columns of  such that

Moreover, this subset  can be found in time .
\end{theorem}
In the context of volume sampling, it is interesting to note that Pan \cite{P} has used an idea of picking submatrices of \emph{locally maximum volume} (or determinants) for rank-revealing matrix decompositions. We refer the reader to \cite{P} for details.


The results of Goreinov, Tyrtyshnikov and Zamarashkin \cite{GT,GTZ} on pseudo-skeleton approximations of matrices look at submatrices of maximum determinants as good candidates for row/column-subset selection.
\begin{theorem} \cite{GT}
If  can be written as

where  is the  by  submatrix of  of maximum determinant. Then,

\end{theorem}

Because of this relation between row/column-subset selection and the related ideas about picking submatrices of maximum volume, \c{C}ivril and Magdon-Ismail \cite{CM1,CM2} looked at the problem of picking a -subset  of rows of a given matrix  such that  is maximized. They show that this problem is NP-hard \cite{CM1} and moreover, it is NP-hard to even approximate it within a factor of , for some constant  \cite{CM2}. This is interesting in the light of our results because we show that even though finding the row-submatrix of maximum volume is NP-hard, we can still sample them with probabilities proportional to their volumes in polynomial time.



\subsection{Our results}
Our main result is a polynomial time algorithm for exact volume sampling. In Section \ref{sec:algo-outline}, we give an outline of our Algorithm \ref{alg:outline}, followed by two possible subroutines given by Algorithms  \ref{alg:omegaSub} and \ref{alg:svdSub} that could be plugged into it.

\begin{theorem}[polynomial-time volume sampling]\label{thm:polytimeVS}
The randomized algorithm given by the combination of the algorithm outlined in Algorithm \ref{alg:outline} with Algorithm \ref{alg:omegaSub} as its subroutine, when given a matrix  and an integer , outputs a random -subset of the rows of  according to volume sampling, using  arithmetic operations.
\end{theorem}
The basic idea of the algorithm is as follows: instead of picking a -subset, pick an ordered -tuple of rows according to volume sampling (i.e., volume sampling suitably extended to all -tuples such that for any fixed -subset, all its  permutations are all equally likely). We observe that the marginal distribution of the first coordinate of such a random tuple can be expressed in terms of coefficients of the characteristic polynomials of  and , where  is the matrix obtained by projecting each row of  orthogonal to the -th row . Using this interpretation, it is easy to sample the first index of the -tuple with the right marginal probability. Now we project the rows of  orthogonal to the chosen row and repeat to pick the next row, until we have picked  of them.

The algorithm just described informally, if implemented as stated, would have a polynomial dependence in ,  and , for some low-degree polynomial. We can do better and get a linear dependence in  by working with  in place of  and computing the projected matrices using rank- updates (Theorem \ref{thm:polytimeVS}), while still having a polynomial time guarantee and sampling exactly. It would be even faster to perform rank-1 updates to the characteristic polynomial itself, but that requires the computation of the inverse of a polynomial matrix (Proposition \ref{prop:svdVS}), and it is not clear to us at this time that there is a fast enough exact algorithm that works for arbitrary matrices. Jeannerod and Villard \cite{JV} give an algorithm to invert a \emph{generic} -by- matrix with entries of degree , with  a power of two, in time . This would lead to the computation of all marginal probabilities for one row in time  (a variation of Algorithm \ref{alg:svdSub} and its analysis). 
Instead, if we are willing to be more practical, while sacrificing our guarantees, then we can perform rank- updates to the characteristic polynomial by using the singular value decomposition (SVD). In \cite{GVL}, an algorithm with cost  arithmetic operations is given for the singular value decomposition but the SVD cannot be computed \emph{exactly} and we do not know how its error propagates in our algorithm which uses many such computations. If the SVD of an -by- matrix can be computed in time , this leads to a nearly-exact algorithm for volume sampling in time . See Proposition \ref{prop:svdVS} for details.

Volume sampling was originally defined in \cite{DRVW} to prove Theorem \ref{thm:vol-sampling}, in particular, to show that any matrix  contains  rows in whose span lie the rows of a rank- approximation to  that is no worse than the best in the Frobenius norm. Efficient volume sampling leads to an efficient selection of  rows that satisfy this guarantee, \emph{in expectation}. In Section \ref{sec:derand}, we use the method of conditional expectations to derandomize this selection. This gives an efficient deterministic algorithm (Algorithm \ref{alg:derand}) for row-subset selection with the following guarantee in the Frobenius norm. This guarantee immediately implies a guarantee in the spectral norm, as follows:

\begin{theorem}[deterministic row subset selection] \label{thm:det-sub-select}
Deterministic Algorithm \ref{alg:derand}, when given a matrix  and an integer , outputs a -subset  of the rows of , using  arithmetic operations, such that

\end{theorem}
This improves the -approximation of Boutsidis, Drineas and Mahoney \cite{BDM} for the Frobenius norm case and matches the lower bound shown in Theorem \ref{thm:DRVW-lower} due to \cite{DRVW}.

The superlinear dependence on  might be too slow for some applications, while it might be acceptable to perform volume sampling or row/column-subset selection approximately. Our volume sampling algorithm (Algorithm \ref{alg:outline}) can be made faster, while losing on the exactness, by using the idea of random projection that preserves volumes of subsets. Magen and Zouzias \cite{MZ} have the following generalization of the Johnson-Lindenstrauss lemma: for  points in  there exists a random projection of them into , where , that preserves the volumes of simplices formed by subsets of  or fewer points within . Therefore, we get a -approximate volume sampling algorithm that requires -time to do the random projection (by matrix multiplication) and then  time for volume sampling on the new -by- matrix (according to Theorem \ref{thm:polytimeVS}).

\begin{theorem}[fast volume sampling] \label{thm:fast-VS}
Using random projection for dimensionality reduction, the polynomial time algorithm for volume sampling mentioned in Theorem \ref{thm:polytimeVS} (i.e., Algorithm \ref{alg:outline} with Algorithm \ref{alg:omegaSub} as its subroutine), gives -approximate volume sampling, using

arithmetic operations.
\end{theorem}

Finally, we show a lower bound for row/column-subset selection in the spectral norm that almost matches our upper bound in terms of the dependence on .
\begin{theorem}[lower bound] \label{thm:lower-bound}
There exists a matrix  such that

where  is the matrix obtained by projecting each row of  onto the span of its -th row .
\end{theorem}


\section{Preliminaries and notation}\label{sec:prelim}
For , let  denote the set . For any matrix , we denote its rows by . For , let  be the row-submatrix of  given by the rows with indices in . By  we denote the linear span of  and let  be the matrix obtained by projecting each row of  onto . Hence,  is the matrix obtained by projecting each row of  orthogonal to .

Throughout the paper we assume . This assumption is not needed most of the time, but justifies sometimes working with  instead of  and, more generally, some choices in the design of our algorithms. It is also partially justified by our use of a random projection as a preprocessing step that makes  small.

The singular values of  are defined as the positive square-roots of the eigenvalues of  (or , up to some extra singular values equal to zero), and we denote them by . Well-known identities of the singular values like

can be generalized into the following lemma.


\begin{lemma} (Proposition 3.2 in \cite{DRVW}) \label{lemma:char-poly}
For any ,

where  are the singular values of , i.e., eigenvalues of , and

is the characteristic polynomial of . Using , we can alternatively use  in the above formula, for .
\end{lemma}

Let  be the exponent of the arithmetic complexity of matrix multiplication. We use that there is an algorithm for computing the characteristic polynomial of an -by- matrix using  arithmetic operations \cite[Section 16.6]{ACT}.

Here is another lemma that we will need about dividing determinants into products of two determinants.
\begin{lemma} \label{lemma:det-division}
Let , ,  and . Then

\end{lemma}
\begin{proof}
Without loss of generality, we can reduce ourselves to the case where  is all rows of the given matrix: Let ,  . We have . Then what we want to prove can be rewritten as:

To show this, we consider two cases. If  is singular, then both sides of the equality are zero. If  is invertible, then we can perform block Gaussian elimination and write

applied to . Writing the determinants of the block-triangular matrices gives

Now, the projection of the rows of a matrix  onto the row-space of a matrix  can be written as

so that ,  and

This completes the proof.
\end{proof}

Finally, a well-known lemma about how the determinant of a matrix changes under a rank- update.
\begin{lemma}[matrix determinant lemma] \label{lemma:matrix-det}
For any invertible  and ,

\end{lemma}



\section{Efficient volume sampling algorithms} \label{sec:algo-outline}
We first outline our volume sampling algorithm to convince the reader that volume sampling can be done in polynomial time. In the subsequent subsections, we give improved subroutines to get faster implementations of the same idea.

The main idea behind our algorithm is based on Lemma \ref{lemma:marginals} about the marginal probabilities encountered in volume sampling. To explain this, it is more convenient to look at volume sampling defined as a distribution on -tuples  instead of -subsets, where each of the  permutations of a -subset is equally likely, i.e., for any ,

Then the marginal probabilities  have the following interpretation in terms of the coefficients of certain characteristic polynomials.
\begin{lemma} \label{lemma:marginals}
Let  such that , for a random -tuple  from the extended volume sampling over -tuples. Let ,  and . Then, 
\end{lemma}
\begin{proof}

\end{proof}


With this lemma in hand, let us consider the following outline of our algorithm. We will later give two more efficient implementations of this outline, depending on how the 's are computed.

\begin{framed}
\begin{alg}\label{alg:outline}
{\bf Outline of our volume sampling algorithm}
\end{alg}
\noindent Input: a matrix  and .

\noindent Output: a subset  of  rows of  picked with probability proportional to .

\begin{enumerate}
\item Initialize  and . For  to  do:
\begin{enumerate}
\item For  to  compute:

where  is a matrix obtained by projecting each row of  orthogonal to .
\item Pick  with probability proportional to . Let  and .
\end{enumerate}
\item Output .
\end{enumerate}
\end{framed}

Now we show the correctness of the algorithm:
\begin{prop} \label{prop:outline-correct}
The probability that our volume sampling algorithm outlined above picks a -subset  is proportional to . This algorithm can be implemented with a cost of  arithmetic operations.
\end{prop}
\begin{proof}
By Lemma \ref{lemma:marginals}, for any  such that , the probability that our algorithm picks a sequence of rows indexed  in that order is equal to

Otherwise, the probability is zero because in
the execution of the algorithm,
 for some step . This proves the correctness of our algorithm.

Given that one can compute the characteristic polynomial of an -by- matrix in  (see Section \ref{sec:prelim}), our outline can be implemented with the following count of arithmetic operations: for every  and ,  to compute ,  in total for . Thus, volume sampling in .
\end{proof}

\subsection{Efficient volume sampling without SVD} \label{subsec:no-svd}
Here we present the first (faster) subroutine for computing the marginal probabilities 's within the volume sampling algorithm outlined in Section \ref{sec:algo-outline}. The two main ideas behind this subroutine are: (1) We can work with  instead of . Assuming , this saves on running time. (2) Each  is a rank- update of  and therefore, once we have , it can be used to compute all  efficiently.

\begin{framed}
\begin{alg}\label{alg:omegaSub}
{\bf First subroutine for marginal probabilities}
\end{alg}
\noindent Input: .

\noindent Output: . \\

For  to  do:
\begin{enumerate}
\item Compute the matrix  by the following formula

\item Compute the characteristic polynomial of  and output

\end{enumerate}
\end{framed}

\begin{prop} \label{prop:omegaVS}
For any given , the Algorithm \ref{alg:omegaSub} above computes  in  arithmetic operations.
\end{prop}
\begin{proof}
 can be computed in time . Observe that since  is obtained by projecting each row of  orthogonal to ,

and therefore,

So once we have , for each ,  can be computed in time  and the characteristic polynomial of  can be computed in time  \cite[Section 16.6]{ACT}. By Lemma \ref{lemma:char-poly},  and hence, the above subroutine results into an  time algorithm for volume sampling.
\end{proof}

\begin{theorem}[same as Theorem \ref{thm:polytimeVS}]
The randomized algorithm given by the combination of the algorithm outlined in Algorithm \ref{alg:outline} with Algorithm \ref{alg:omegaSub} as its subroutine, when given a matrix  and an integer , outputs a random -subset of the rows of  according to volume sampling, using  arithmetic operations.
\end{theorem}
\begin{proof}
The proof follows by combining Proposition \ref{prop:outline-correct} and Proposition \ref{prop:omegaVS}, and since we compute all the 's simultaneously in each round in  arithmetic operations, the total number of arithmetic operations is .
\end{proof}

\subsection{Efficient volume sampling using SVD} \label{subsec:svd}
Taking further the idea that each  is a rank- update of , we can give a faster algorithm based on the singular value decomposition of . Given the singular value decomposition of a matrix and using the matrix determinant lemma (Lemma \ref{lemma:matrix-det}), one can give a precise formula for how the characteristic polynomial changes under a rank- update. Using this subroutine in the volume sampling algorithm outlined in Section \ref{sec:algo-outline} we get an algorithm for nearly-exact volume sampling (depending on the precision of the computed SVD) in time , where  is the running time of SVD on an -by- matrix.

\begin{framed}
\begin{alg}\label{alg:svdSub}
{\bf Second subroutine for marginal probabilities.}
\end{alg}
\noindent Input: . \\
\noindent Output: .

\begin{enumerate}
\item Compute the (thin) singular value decomposition , say  and , and keep the singular values  and define . Also keep the columns of , i.e., the left singular vectors .
\item Compute the polynomial products

\item For  to  output:

\end{enumerate}
\end{framed}
\begin{prop} \label{prop:svdVS}
In the real arithmetic model and given exact  and , using the Algorithm \ref{alg:svdSub} as a subroutine inside Algorithm \ref{alg:outline} outlined for volume sampling, we get an algorithm for volume sampling. If  is the running time for computing the singular value decomposition of -by- matrices, the algorithm runs in time .
\end{prop}

\begin{proof}
Using the matrix determinant lemma (Lemma \ref{lemma:matrix-det}), the characteristic polynomial of  can be written as

Thus,

Once we have the singular value decomposition of ,  and  can all be computed in time  using polynomial products. This is because there are at most  non-zero 's. Thus,  and all the  for  can be computed in time  and then using the above formula we get .
\end{proof}

\subsection{Approximate volume sampling in nearly linear time}\label{subsec:random-proj}
Magen and Zouzias \cite{MZ} showed that the random projection lemma of Johnson and Lindenstrauss can be generalized to preserve volumes of subsets after embedding. Here is a restatement of Theorem 1 of \cite{MZ} using  instead of  in their original statement.
\begin{theorem} \label{thm:random-projection} \cite{MZ}
For any ,  and , there is

and there is a mapping  such that

for all  such that , where  has its -th row as . Moreover,  is a linear mapping given by multiplication with a random  by  matrix with i.i.d. Gaussian entries, so computing  takes time .
\end{theorem}

\begin{theorem}[same as Theorem \ref{thm:fast-VS}]
Using random projection for dimensionality reduction, the polynomial time algorithm for volume sampling mentioned in Theorem \ref{thm:polytimeVS} (i.e., Algorithm \ref{alg:outline} with Algorithm \ref{alg:omegaSub} as its subroutine), gives -approximate volume sampling, using

arithmetic operations.
\end{theorem}
\begin{proof}
Using Theorem \ref{thm:random-projection} and doing volume sampling of -subsets of rows from  gives -approximation to the volume sampling of -subsets of rows from . This can be done in two steps: first, we compute  using matrix multiplication in time  and second, we do volume sampling on  using the algorithm from Subsection \ref{subsec:no-svd}. Overall, it takes time , which is equal to

Moreover, this can be implemented using only one pass over the matrix  with extra space .
\end{proof}


\section{Derandomized row/column-subset selection} \label{sec:derand}
Our derandomized row-subset selection algorithm is based on a derandomization of the volume sampling algorithm in Section \ref{sec:algo-outline}, using the method of conditional expectations. Again, it may be easier to consider volume sampling extended to random -tuples  where

From Theorem \ref{thm:vol-sampling} we know that

where the expectation is over .

Let us consider  for which . Let  and look at the conditional expectation. The following lemma shows that these conditional expectations have an easy interpretation in terms of the coefficients of certain characteristic polynomials, and hence can be computed efficiently.

\begin{lemma} \label{lemma:conditional}
Let  be such that  for a random -tuple  from extended volume sampling. Let  and . Then

\end{lemma}
\begin{proof}

\end{proof}

Knowing the above lemma, it is easy to derandomize our algorithm outlined for volume sampling. In each step, we just compute the new conditional expectations for each additional , and finally pick the  that minimizes the conditional expectation.

\begin{framed}
\begin{alg} \label{alg:derand}
{\bf Derandomized row/column-subset selection}
\end{alg}
\noindent Input: a matrix  and .

\noindent Output: a subset  of  rows of  with the guarantee


\begin{enumerate}
\item Initialize  and . For  to  do:
\begin{enumerate}
\item For  to  do: compute  and , where  is the matrix obtained by projecting each row of  orthogonal to .
\item Pick  that minimizes . Let  and .
\end{enumerate}
\item Output .
\end{enumerate}
\end{framed}

\begin{theorem}[same as Theorem \ref{thm:det-sub-select}]
Deterministic Algorithm \ref{alg:derand}, when given a matrix  and an integer , outputs a -subset  of the rows of , using  arithmetic operations, such that

\end{theorem}
\begin{proof}
By applying Lemma \ref{lemma:conditional} to  instead of , as , we see that the step  of our algorithm picks  that minimizes

The correctness of our algorithm follows immediately from observing that in each step ,

and that we started with


The guarantee for spectral norm follows immediately from our guarantee in the Frobenius norm, just using properties of norms and the fact that :


Moreover, this algorithm runs in time  if we use the subroutine in Subsection \ref{subsec:no-svd} to compute the characteristic polynomial of  using that of .
\end{proof}



\section{Lower bound for rank- spectral approximation using one row}
Here we show a lower bound for row/column-subset selection. We prove that there is a matrix  such that using the span of any single row of it, we can get only -approximation in the spectral norm for the nearest rank- matrix to . This can be generalized to a similar  lower bound for general  by using a matrix with  block-diagonal copies of .

\begin{theorem}[same as Theorem \ref{thm:lower-bound}]
There exists a matrix  such that

where  is the matrix obtained by projecting each row of  onto the span of its -th row .
\end{theorem}
\begin{proof}
Consider  with entries as follows:

Let  be the best rank- approximation to  whose rows lie in the span of  (or for that matter, any fixed row of ). Then, we want to show that

We first compute the singular values of , i.e., the positive square roots of the eigenvalue of .

 is an eigenvector of  with eigenvalue . Thus, . Observe that, by symmetry, all other singular values of  must be equal, i.e., . However,

Therefore, .

Now denote the -th row of  by . By definition, the -th row of  is the projection of  onto . We are interested in the singular values of . For :

Thus,  can be written as

Again,  is the top eigenvector of  and using this we get,

Therefore,

\end{proof}

\section{Discussion}


We analyzed efficient algorithms for volume sampling that can be used for row/column subset selection. Here are some ideas for future investigation suggested by this work:
\begin{itemize}
\item It would be interesting to explore how these algorithmic ideas are related to determinantal sampling \cite{lyons2003determinantal,hough2006determinantal} and, in particular, the generation of random spanning trees.

\item Find practical counterparts of the algorithms discussed here. In particular, we do not analyze the numerical stability of our algorithms.

\item Is there an efficient algorithm for volume sampling based on random walks? This question is inspired by MCMC as well as random walk algorithms for the generation of random spanning trees.
\end{itemize}



\bibliographystyle{abbrv}    
\bibliography{subset-ref}

\end{document} 