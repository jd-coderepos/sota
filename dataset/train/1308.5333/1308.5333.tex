A timed automaton \TA is generated from a finite partition \patitS as follows.

\begin{definition}[Generation of Timed Automaton]\label{def:generation_of_TA}
Given  a finite collection of slice-families $\mathcal{S}=\{\mathcal{S}^{i}|~i\in\bm{k}\}$, and pairs of times $\mathcal T = \{(\underline{t}^{i}_{g_{i}}, \overline{t}^{i}_{g_{i}}) \in \mathds R_{\geq0}^2|~i\in\bm{k},\,g_{i}\in\{1,\dots,|\mathcal{S}^{i}|\} \}$.
We define the timed automaton $\mathcal{A}(\mathcal S, \mathcal T)=(E, E_{0}, C, \Sigma, I, \Delta)$ by
\begin{itemize}
\item \textbf{Locations:} The locations of $\mathcal{A}$ are given by
\begin{align}
E=E(\mathcal{S}).\label{eqn:TA_locations}
\end{align}
This means that a location $e_{(g,h)}$ is identified with the cell $e_{(g,h)}=\alpha_{E(\mathcal S)}^{-1}(\{e_{(g,h)}\})$ of the partition $E(\mathcal{S})$, see Definition~\ref{def:abstraction_function}.

\item \textbf{Clocks:} The set of clocks is
$C = \{c^{i}|~i\in\bm{k}\}$.

\item \textbf{Alphabet:} The alphabet is
$\Sigma = \mathcal \{\sigma^{i}|~i\in\bm{k}\}$.


\item \textbf{Invariants:} In each location $e_{(g,h)}$, we impose an invariant
\begin{align}
I(e_{(g,h)})&= \bigwedge_{i=1}^{k}c^{i}\bm{\leq} \overline{t}^{i}_{g_{i}}.\label{eqn:TA_invariants}
\end{align}

\item \textbf{Transition relations:} If a pair of locations $e_{(g,h)}$ and $e_{(g',h')}$ satisfy the following two conditions
\begin{enumerate}
\item $e_{(g,h)}$ and $e_{(g',h')}$ are adjacent; that is $e_{(g,h)}\cap e_{(g',h')}\neq\emptyset$, and
\item $g'_{i}\leq g_{i}$ for all $i\in\bm{k}$.
\end{enumerate}
Then there is a transition relation
\begin{subequations}
\begin{align}
\notag\delta_{(g,h)\rightarrow (g',h')}& = (e_{(g,h)}, G_{(g,h)\rightarrow(g',h')},\sigma^i, R_{(g,h)\rightarrow(g',h')}, e_{(g',h')}),
\end{align}
where
\begin{align}
&G_{(g,h)\rightarrow(g',h')}=\bigwedge_{i=1}^{k}\begin{cases}c^{i}\geq \underline{t}^{i}_{g_{i}}&\tn{if }g_{i}-g'_{i}=1\\c^{i}\geq0&\tn{otherwise.}\end{cases}\label{eqn:TA_guard}
\end{align}
Note that $g_{i}-g'_{i}=1$ whenever a transition labeled $\sigma^{i}$ is taken.

Let $i\in\bm{k}$. We define $R_{(g,h)\rightarrow(g',h')}$ by
\begin{align}
&c^{i}\in R_{(g,h)\rightarrow(g',h')}\label{eqn:TA_reset}
\end{align}
iff $g_{i}-g'_{i}=1$.
\end{subequations}
\end{itemize}
\end{definition}

To ensure that the properties of an abstraction is not only valid for a particular choice of level sets, we impose the derived condition for any choice of level set in the partition.


From Definition~\ref{def:generation_of_TA}, it is seen that to generate a timed automaton, it is required to devise a partition of the state space, and find a set of invariant and guard conditions. Therefore, we provide a condition under which an abstraction is complete, recall the definition of a complete abstraction in Section~\ref{sec:abstractions_of_dynamical_systems}.
\begin{proposition}[\cite{Complete_Abstractions_of_Dynamical_Systems_by_Timed_Automata}]\label{prop:completeness_of_partition}
Given a dynamical system \dynSysAll, a collection of partitioning functions $\{\varphi^{i}|~i\in\bm{k}\}$, a collection of values $\{a^{i} _{g_{i}}\in\mathds{R} |~i\in\bm{k},\,g_{i}\in\{1,\dots,|\mathcal{S}^{i}|\} \}$ generating $\mathcal{S}$, and pairs of times $\mathcal T = \{(\underline{t}^{i}_{g_{i}}, \overline{t}^{i}_{g_{i}}) |~i\in\bm{k},\,g_{i}\in\{1,\dots,|\mathcal{S}^{i}|\} \}$. The timed automaton $\TA(\mathcal{S}, \mathcal{T})$ is a complete abstraction of $\Gamma$ if and only if for any $i \in \bm{k}$
\begin{enumerate}
\item for any pair of regular values $(a^{i}_{g_{i}-1},a^{i}_{g_{i}})$ with $g\in\mathcal{G}(\mathcal{S})$ (see the definition of $\mathcal{G}(\mathcal{S})$ in Definition~\ref{def:extended_cell}) there exists a time $t^{i}_{g_{i}}$ such that for all $x_{0}\in (\varphi^{i})^{-1}(a^{i}_{g_{i}})$
\begin{align}
\phi_{\Gamma}(t^{i}_{g_{i}},x_{0})\in (\varphi^{i})^{-1}(a^{i}_{g_{i}-1})\label{eqn:suf_cond_sound2}
\end{align}
and
\item $\overline{t}_{S^{i}_{g_{i}}}=\underline{t}_{S^{i}_{g_{i}}}=t^{i}_{g_{i}}$.
\end{enumerate}
\end{proposition}
We say that $\{\varphi^i|~i\in\bm{k}\}$ generates a complete abstraction if there exist times $\mathcal{T}$ such that Proposition~\ref{prop:completeness_of_partition} is satisfied.

In the introduction of the paper, we claim that it is impossible to generate a complete abstraction of a system with a saddle point using transversal partitioning functions. In the following example a complete abstraction is given for such a system by nonincreasing partitioning functions.

\begin{example}\label{ex:saddle_lin}
Consider the following two-dimensional linear vector field
\begin{align}
\begin{bmatrix}
\dot{x}_1\\
\dot{x}_2
\end{bmatrix}&=
\begin{bmatrix}
-x_1\\
x_2
\end{bmatrix}\label{eqn:vectFieldEx1}.
\end{align}
The system has a saddle point and its phase plot is shown in Figure~\ref{fig:example1_saddle_phase_plot}.
\begin{figure}[!htb]
    \centering
       \includegraphics[scale=1]{example1_saddle_phase_plot.pdf}
    \caption{Phase plot of a linear system with a saddle point.\label{fig:example1_saddle_phase_plot}}
\end{figure}

We choose the following partitioning functions
\begin{subequations}
\begin{align}
\varphi_{1}(x_1,x_2) &= x_1^2,\\
\varphi_{2}(x_1,x_2) &= -x_2^2,
\end{align}
\end{subequations}
Recall from \eqref{eqn:Lyap_der} that we denote Lie derivatives by $\psi$. The Lie derivatives of the partitioning functions along the vector field in \eqref{eqn:vectFieldEx1} become
\begin{subequations}
\begin{align}
\psi_{1}(x_1,x_2) &= -2x_1^2,\\
\psi_{2}(x_1,x_2) &= -2x_2^2.
\end{align}
\end{subequations}
This implies that the abstraction generated by $\{\varphi_{1},\varphi_{2}\}$ is complete. Completeness is concluded from Proposition~\ref{prop:completeness_of_partition}.
\end{example}

Proposition~\ref{prop:completeness_of_partition} does not provide a straightforward method for computing a complete abstraction, as the conditions are not numerically tractable. Therefore, we rephrase \eqref{eqn:suf_cond_sound2} as a relation between the level sets of the partitioning function and its derivative along the vector field.
It is difficult to determine if the partitioning functions in Example~\ref{ex:saddle_lin} generate a complete abstraction from Proposition~\ref{prop:completeness_of_partition}; however, this is clear from the following proposition, as the level sets of $\phi$ and $\psi$ coincide.






\begin{proposition}\label{prop:prop2}
Let \dynSys be a dynamical system and $\varphi$ be a smooth function, in particular a partitioning function. Then for any regular value $a\in\mathds{R}$, the following statements are equivalent
\begin{enumerate}
\item there exists $b\in\mathds{R}$ such that \begin{align}
\{x\in\mathds{R}^{n}|~\varphi(x)-a=0\}\subseteq\{x\in\mathds{R}^{n}|~\psi(x)-b=0\},\label{eqn:subset_lev_set11}
\end{align}
where
\begin{align}
\notag\psi(x) \equiv d \varphi (f) (x) = \sum_{i}\frac{\partial\varphi}{\partial x_{i}}(x)f_{i}(x).
\end{align}
\item there exists $\epsilon>0$ such that for any $t\in(-\epsilon,\epsilon)$
\begin{align}
\varphi(\phi_{\Gamma}(t,x_{1}))=\varphi(\phi_{\Gamma}(t,x_{2}))~~~~\forall x_{1},x_{2}\in \varphi^{-1}(a).\label{eqn:new2}
\end{align}
\end{enumerate}
\end{proposition}
\begin{proof}
We show that 2) implies 1).
We differentiate both sides of
\begin{align*}
\varphi(\phi_{\Gamma}(t,x_{1}))=\varphi(\phi_{\Gamma}(t,x_{2}))
\end{align*}
with respect to $t$
\begin{subequations}
\begin{align}
\sum_{i}\frac{\partial\varphi}{\partial x_{i}}(\phi_{\Gamma}(t,x_{1}))f_{i}(\phi_{\Gamma}(t,x_{1}))&=
\sum_{i}\frac{\partial\varphi}{\partial x_{i}}(\phi_{\Gamma}(t,x_{2}))f_{i}(\phi_{\Gamma}(t,x_{2})),\\
\psi(\phi_{\Gamma}(t,x_{1}))&=\psi(\phi_{\Gamma}(t,x_{2})).
\end{align}
\end{subequations}
At $t=0$, $\psi(x_{1})=\psi(x_{2})$. Hence, \eqref{eqn:subset_lev_set11} is satisfied.

To show that 1) implies 2), we define a convenient state transformation inspired by \cite[p.~13]{Morse_Theory_book}. In the new coordinates, the vector field has only one nonzero component.

Let $M=\varphi^{-1}(a)$ be a smooth manifold. By Sard's Theorem, there exists an open neighborhood $U$ of $a$ such that any point in $U$ is a regular value of $\varphi$ \cite[p.~132]{Introduction_to_Smooth_Manifolds}. Define the smooth function $\eta:\varphi^{-1}(U) \rightarrow \mathds{R}$ by
\begin{align}
\eta \equiv \frac{1}{||\nabla\varphi||^2},
\end{align}
where $\nabla\varphi$ is the gradient of $\varphi$ (with respect to a Riemannian metrics $\left<\cdot, \cdot \right>$ on M). The function $\eta$ is well defined, since $\varphi^{-1}(U) \subset M$ is an (open) set of regular points of $\varphi$. Define the vector field $\xi$ on $\varphi^{-1}(U)$ by
\begin{align}
\xi \equiv \eta \nabla \varphi.
\end{align}
The derivative of $\varphi$ in the direction of $\xi$ is
\begin{align}
d \varphi(\xi)=\langle\nabla\varphi,\xi\rangle=\eta\langle\nabla\varphi,\nabla\varphi\rangle=1,
\label{unit_increment}
\end{align}
Choose $a', a'' \in \mathds R$ such that $a'<a<a''$ and $[a',a'']\subset U$. We define the map $F(\cdot,\cdot):M\times[a',a'']\rightarrow\varphi^{-1}([a',a'']) \subset X$, where $F(x_{0},t)$ is the solution of $\xi$ from initial state $x_{0}$. From \eqref{unit_increment}, we have
\begin{align}
t\mapsto\varphi\circ F(x_{0},t)=a+t.\label{eqn:good_map}
\end{align}

We represent the vector field $f$ and function $\varphi$ in new coordinates $q = F(x,t)$
\begin{subequations}
\begin{align}
\tilde f (q) & = (D F(q))^{-1}f \circ F(q) \\
\tilde \varphi (q) & = \varphi\circ F(q) \label{eqn:varphi_tilde}
\end{align}
\end{subequations}

Notice that the vector fields $f$ and $\tilde f$ are $F$-related. The differential of $\tilde \varphi$ is
\begin{align}
d\tilde{\varphi}(q)=d\varphi(F(q))DF(q).
\end{align}
Hence, the derivative of $\tilde \varphi$ if the direction of $\tilde f$ is
\begin{align}
d\tilde{\varphi}(\tilde{f})&=d \varphi \circ F ~ DF (DF)^{-1} f \circ F\\
&=d\varphi(f) \circ F.
\end{align}

Let $x_{1},x_{2}\in M$, then $F(x_1,t)$ and $F(x_2,t)$ are regular points for $t \in (-\epsilon, \epsilon)$. By \eqref{eqn:subset_lev_set11}
\begin{align}
d \varphi (f)(F(x_{1},t)) = d\varphi (f)(F(x_{2},t)).
\end{align}
We define $\tilde{\psi}=d\tilde{\varphi}(\tilde{f})$, and conclude that
\begin{align}
\tilde{\psi}(x_1,t) = \tilde{\psi}(x_2,t)
\end{align}

From~\eqref{eqn:good_map}, $\tilde{\varphi}$ only depends on the last coordinate. Therefore, denoting $f = (f_1, \hdots, f_n)$, we have
\begin{align}
\tilde{\psi} = d\tilde{\varphi}\tilde{f}=\tilde{f}_{n}.
\end{align}

Since $\tilde \psi (x_1,t) = \tilde \psi(x_2,t)$ for any pair $x_1, x_2 \in M$ and $t \in [a',a'']$, we have $\tilde f_n(x_1,t) = \tilde f_n(x_2,t)$. In other words, the $n$th component of the vector field $\tilde f$ depends only on its last $n$ coordinate $t$. As a consequence, denoting the flow map of the vector field $\tilde f$ by $\phi_{\tilde \Gamma}$, we have
\begin{align}
\phi_{\tilde \Gamma} ((x_1,0),t ) = \phi_{\tilde \Gamma} ((x_2,0),t ) \in M \times (-\epsilon, \epsilon).
\end{align}
The vector field $f$ and $\tilde f$ are $F$-related, hence also
\begin{align}
\phi_{\Gamma} (x_1,t ) = \phi_{\Gamma} (x_2,t ).
\end{align}
Thus,  the inclusion \eqref{eqn:suf_cond_sound2} holds.
\end{proof}

Unfortunately, we cannot relax proposition to include critical values as well, i.e., it does not hold for any $a\in\mathds{R}$. This is clarified in the following, by presenting a dynamical system with state space $\sts\subseteq\mathds{R}^2$ for which there exists no complete abstraction generated by transversal partitioning functions.

According to Proposition~\ref{prop:completeness_of_partition}, it takes the same time for two trajectories to propagate between level sets of complete partitioning functions. Consider a dynamical system with flow lines shown in Figure~\ref{fig:vectorFieldmm}.
\begin{figure}[!htb]
    \centering
       \includegraphics[scale=1.2]{vectorFieldmm.pdf}
    \caption{Flow lines of a dynamical system, with a saddle point, a stable equilibrium point, and an unstable equilibrium point.\label{fig:vectorFieldmm}}
\end{figure}
Level sets of two partitioning functions are illustrated at the saddle point inside the gray circle. From the red level set, one trajectory goes to the saddle point; however, the remaining trajectories diverge from the saddle point. Therefore, it cannot take the same time to propagate between any two level sets, if the partitioning function is decreasing along the vector field.








In the following, we denote the stable (unstable) manifold by $W^{\tn{s}}(p)$ ($W^{\tn{u}}(p)$) \cite{hirsch}.
\begin{lemma}\label{lem:crit}
Let $\varphi$ be a partitioning function generating a complete abstraction of \dynSysAll and let $p$ be a singular point of \vectField. Then $p$ is a critical point of $\varphi$.
\end{lemma}
\begin{proof}
Suppose that $p$ is a regular point of $\varphi$ with regular value $c$. Then $\varphi^{-1}(c)$ is an $n-1$ dimensional manifold. Without loss of generality, we assume that $p$ is not stable. Let $x\in\varphi^{-1}(c)\backslash W^{\tn{s}}(p)$. Then there exists $\tau>0$ such that  $\phi(\tau,x)\notin\varphi^{-1}(c)$, but $p=\phi(\tau,p)\in\varphi^{-1}(c)$, since $p$ is a singular point of $f$. This contradicts completeness.
\end{proof}

The following theorem is the main contribution of the paper, showing that complete abstractions identify stable and unstable manifolds.\begin{theorem}\label{thm:res}
Let $\tn{Reg}(\varphi)$ denote the set of regular values of $\varphi$, generating a complete abstraction of \dynSysAll, let \sts be a connected compact manifold, and let $p$ be a singular point of \vectField. Suppose that $\varphi^{-1}(\tn{Reg}(\varphi))\cap W^{\tn{s}}(p)\neq\emptyset$. Then \[W^{u}(p)\subseteq\varphi^{-1}(\varphi(p))  .\]
\end{theorem}
\begin{sketchProof}
Let $Z\equiv\varphi^{-1}(\tn{Reg}(\varphi))\cap W^{\tn{s}}(p)$ and let $R(x)\equiv\varphi^{-1}(\varphi(x))$. Suppose that there exists a point $x\in Z$ such that $Y(x)\equiv R(x)\backslash W^{\tn{s}}(p)\neq\emptyset$.

Per definition of $W^{\tn{s}}(p)$, $\lim_{t\rightarrow\infty}\phi(x,t)= p$. Let $y\in Y(x)$ then there exists a singular point $p'\neq p$ such that $y\in W^{\tn{s}}(p')$. Note that any solution goes to a singular point, as the state space \sts is compact.

Since $\varphi(x)=\varphi(y)$, the abstraction is complete, and the flow map $\phi$ is continuous, we conclude from Proposition~\ref{prop:prop2} that $\varphi(p)=\varphi(p')$. Thus, for any $z\in W(p,p')\equiv W^{\tn{u}}(p)\cap W^{\tn{s}}(p')$ the value of $\varphi$ is constant, i.e., $\varphi(p)=\varphi(z)$. Therefore, $W(p,p')\subseteq R(p)$.

To show that $W^{\tn{u}}(p)\subseteq R(p)$, we exploit that any $z\in R(x)$, $z\in W^{\tn{s}}(p'')$ for some singular point $p''$. Therefore, $\varphi(p)=\varphi(p'')$ for any such singular point.
We use notation $p\preceq p'$ iff there is a flow line from $p$ to $p'$. Whenever there is a sequence $\{p_1,\dots,p_n\}$ of singular points of the vector field $f$ such that
\[p=p_1\preceq p_2\preceq\dots\preceq p_n,\]
then $\varphi(p_i)=\varphi(p)$ for all $i\in\{1,\dots,n\}$.
Such a scenario is illustrated in Figure~\ref{fig:multi_sing_proof}.
\begingroup \makeatletter \providecommand\color[2][]{\errmessage{(Inkscape) Color is used for the text in Inkscape, but the package 'color.sty' is not loaded}\renewcommand\color[2][]{}}\providecommand\transparent[1]{\errmessage{(Inkscape) Transparency is used (non-zero) for the text in Inkscape, but the package 'transparent.sty' is not loaded}\renewcommand\transparent[1]{}}\providecommand\rotatebox[2]{#2}\ifx\svgwidth\undefined \setlength{\unitlength}{396bp}\ifx\svgscale\undefined \relax \else \setlength{\unitlength}{\unitlength * \real{\svgscale}}\fi \else \setlength{\unitlength}{\svgwidth}\fi \global\let\svgwidth\undefined \global\let\svgscale\undefined \makeatother \begin{figure}$ $\put(0,0){\includegraphics[width=\unitlength]{multi_sing_proof.pdf}}\put(0.45894143,0.35794543){\color[rgb]{0,0,0}\makebox(0,0)[lb]{\smash{$W^{\tn{s}}(p_1)$}}}\put(0.35988045,0.16176776){\color[rgb]{0,0,0}\makebox(0,0)[lb]{\smash{$p_1$}}}\put(0.56116808,0.12436601){\color[rgb]{0,0,0}\makebox(0,0)[lb]{\smash{$p_2$}}}\put(0.88746961,0.18442785){\color[rgb]{0,0,0}\makebox(0,0)[lb]{\smash{$p_3$}}}\put(0.47248211,0.31349297){\color[rgb]{0,0,0}\makebox(0,0)[lb]{\smash{$\varphi^{-1}(\varphi(x))$}}}

 \caption{State space of a system with singular points $p$, $q_1$, and $q_2$. No solution initialized on the red level set reaches $q_1$, but solutions get arbitrarily close to $q_1$.}
\label{fig:multi_sing_proof}
  \end{figure}\endgroup
Thus, we have
\[W^{\tn{u}}(p)=\bigcup_{p\preceq\dots\preceq p'}W(p,p')\subseteq R(p).\]


If $Y(x)=\emptyset$ for all $x\in Z$, then
\[\bigcup_{x\in Z} R(x)\subseteq W^{\tn{s}}(p).\]
The dimension of $R(x)$ is $n-1$. Furthermore, since $R(x)$ is a closed compact manifold and the vector field is transversal to $R(x)$, there exists $\tau>0$ such that
\[\phi(R(x),(-\tau,\tau))\]
that is an embedded manifold of dimension $n$. Thereby $\tn{dim}(W^{\tn{s}}(p))=n$, and $W^{\tn{u}}(p)=\{p\}\subseteq R(p)$.
\qed
\end{sketchProof}
One could think that the inclusion $W^{u}(p)\subseteq \varphi^{-1}(\varphi(p))$ in Theorem~\ref{thm:res} could be replaced by an equality; however, the following counterexample demonstrates that this cannot be done.
\begin{example}
Consider the following two dimensional linear vector field from Example~\ref{ex:saddle_lin}
\begin{align}
\begin{bmatrix}
\dot{x}_1\\
\dot{x}_2
\end{bmatrix}&=
\begin{bmatrix}
-x_1\\
x_2
\end{bmatrix}.
\end{align}
The system has a saddle point $p=(0,0)$ and its phase plot is shown in Figure~\ref{fig:example1_saddle_phase_plot}. We modify the partitioning function $\varphi_{1}(x_1,x_2)=x_1^2$ from Example~\ref{ex:saddle_lin}, to obtain a case where
\[W^{u}(p)\subsetneq\tilde{\varphi}_1^{-1}(\tilde{\varphi}_1(p)).\]
For this purpose, we construct a bump function, according to \cite{An_Introduction_to_Manifolds}. Let
\begin{align}
\notag f(t)&=\begin{cases}e^{-1/t}&\text{for }t>0\\0&\text{for }t\leq0\end{cases}\\
\notag g(t)&=\frac{f(t)}{f(t)+f(1-t)}.
\end{align}
From $f$ and $g$, we choose the partitioning function
\begin{align}
\tilde{\varphi}_{1}(x_1,x_2) &= g\left(\frac{x_1^2-a^2}{b^2-a^2}\right)\cdot x_1^2\label{eqn:bump_varphi1},
\end{align}
with $a=0.5$ and $b=2$. The graphs of $g\left(\frac{x_1^2-a^2}{b^2-a^2}\right)$, $\tilde{\varphi}_{1}(x_1,x_2)$, and $\varphi_{1}(x_1,x_2)$ are shown in Figure~\ref{fig:bump_example}.
\begin{figure}[!htb]
    \centering
       \includegraphics[scale=1]{bump_example1.pdf}
    \caption{The left subplot shows the graph of $g\left(\frac{x_1^2-a^2}{b^2-a^2}\right)$ and the right subplot shows the graphs of $\varphi_{1}(x_1,x_2)$  (blue) and $\tilde{\varphi}_{1}(x_1,x_2)$ (dashed red).\label{fig:bump_example}}
\end{figure}

The partitioning function shown in \eqref{eqn:bump_varphi1} generates a complete abstraction and for this particular function $W^{u}(p)$ is a proper subset of $\tilde{\varphi}_1^{-1}(\tilde{\varphi}_1(p))$.
\end{example}

Theorem~\ref{thm:res} is vital in the field of abstracting generic dynamical systems, as it shows that a partitioning function generating a complete abstraction has to be constant on the unstable manifolds; hence, finding such a function is as difficult as finding the stable and unstable manifolds. This is practically impossible for systems of dimension greater than three. As an example, on all black lines in Figure~\ref{fig:morse_complex} a complete partitioning function must be constant.
\begin{figure}[!htb]
    \centering
       \includegraphics[scale=0.8]{morse_complex.pdf}
    \caption{Morse complex, where a "+" indicates a saddle, a "$\cdot$" indicates a maximum, and a "$\circ$" indicates a minimum. The lines connecting the singular points are stable and unstable manifolds.\label{fig:morse_complex}}
\end{figure}
In conclusion, research in the field should be focused on method development for generating sound rather than complete abstractions.
