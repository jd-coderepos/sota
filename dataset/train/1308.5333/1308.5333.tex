A timed automaton \TA is generated from a finite partition \patitS as follows.

\begin{definition}[Generation of Timed Automaton]\label{def:generation_of_TA}
Given  a finite collection of slice-families , and pairs of times .
We define the timed automaton  by
\begin{itemize}
\item \textbf{Locations:} The locations of  are given by

This means that a location  is identified with the cell  of the partition , see Definition~\ref{def:abstraction_function}.

\item \textbf{Clocks:} The set of clocks is
.

\item \textbf{Alphabet:} The alphabet is
.


\item \textbf{Invariants:} In each location , we impose an invariant


\item \textbf{Transition relations:} If a pair of locations  and  satisfy the following two conditions
\begin{enumerate}
\item  and  are adjacent; that is , and
\item  for all .
\end{enumerate}
Then there is a transition relation

\notag\delta_{(g,h)\rightarrow (g',h')}& = (e_{(g,h)}, G_{(g,h)\rightarrow(g',h')},\sigma^i, R_{(g,h)\rightarrow(g',h')}, e_{(g',h')}),

&G_{(g,h)\rightarrow(g',h')}=\bigwedge_{i=1}^{k}\begin{cases}c^{i}\geq \underline{t}^{i}_{g_{i}}&\tn{if }g_{i}-g'_{i}=1\\c^{i}\geq0&\tn{otherwise.}\end{cases}\label{eqn:TA_guard}

&c^{i}\in R_{(g,h)\rightarrow(g',h')}\label{eqn:TA_reset}

\end{itemize}
\end{definition}

To ensure that the properties of an abstraction is not only valid for a particular choice of level sets, we impose the derived condition for any choice of level set in the partition.


From Definition~\ref{def:generation_of_TA}, it is seen that to generate a timed automaton, it is required to devise a partition of the state space, and find a set of invariant and guard conditions. Therefore, we provide a condition under which an abstraction is complete, recall the definition of a complete abstraction in Section~\ref{sec:abstractions_of_dynamical_systems}.
\begin{proposition}[\cite{Complete_Abstractions_of_Dynamical_Systems_by_Timed_Automata}]\label{prop:completeness_of_partition}
Given a dynamical system \dynSysAll, a collection of partitioning functions , a collection of values  generating , and pairs of times . The timed automaton  is a complete abstraction of  if and only if for any 
\begin{enumerate}
\item for any pair of regular values  with  (see the definition of  in Definition~\ref{def:extended_cell}) there exists a time  such that for all 

and
\item .
\end{enumerate}
\end{proposition}
We say that  generates a complete abstraction if there exist times  such that Proposition~\ref{prop:completeness_of_partition} is satisfied.

In the introduction of the paper, we claim that it is impossible to generate a complete abstraction of a system with a saddle point using transversal partitioning functions. In the following example a complete abstraction is given for such a system by nonincreasing partitioning functions.

\begin{example}\label{ex:saddle_lin}
Consider the following two-dimensional linear vector field

The system has a saddle point and its phase plot is shown in Figure~\ref{fig:example1_saddle_phase_plot}.
\begin{figure}[!htb]
    \centering
       \includegraphics[scale=1]{example1_saddle_phase_plot.pdf}
    \caption{Phase plot of a linear system with a saddle point.\label{fig:example1_saddle_phase_plot}}
\end{figure}

We choose the following partitioning functions

\varphi_{1}(x_1,x_2) &= x_1^2,\\
\varphi_{2}(x_1,x_2) &= -x_2^2,

Recall from \eqref{eqn:Lyap_der} that we denote Lie derivatives by . The Lie derivatives of the partitioning functions along the vector field in \eqref{eqn:vectFieldEx1} become

\psi_{1}(x_1,x_2) &= -2x_1^2,\\
\psi_{2}(x_1,x_2) &= -2x_2^2.

This implies that the abstraction generated by  is complete. Completeness is concluded from Proposition~\ref{prop:completeness_of_partition}.
\end{example}

Proposition~\ref{prop:completeness_of_partition} does not provide a straightforward method for computing a complete abstraction, as the conditions are not numerically tractable. Therefore, we rephrase \eqref{eqn:suf_cond_sound2} as a relation between the level sets of the partitioning function and its derivative along the vector field.
It is difficult to determine if the partitioning functions in Example~\ref{ex:saddle_lin} generate a complete abstraction from Proposition~\ref{prop:completeness_of_partition}; however, this is clear from the following proposition, as the level sets of  and  coincide.






\begin{proposition}\label{prop:prop2}
Let \dynSys be a dynamical system and  be a smooth function, in particular a partitioning function. Then for any regular value , the following statements are equivalent
\begin{enumerate}
\item there exists  such that 
where

\item there exists  such that for any 

\end{enumerate}
\end{proposition}
\begin{proof}
We show that 2) implies 1).
We differentiate both sides of

with respect to 

\sum_{i}\frac{\partial\varphi}{\partial x_{i}}(\phi_{\Gamma}(t,x_{1}))f_{i}(\phi_{\Gamma}(t,x_{1}))&=
\sum_{i}\frac{\partial\varphi}{\partial x_{i}}(\phi_{\Gamma}(t,x_{2}))f_{i}(\phi_{\Gamma}(t,x_{2})),\\
\psi(\phi_{\Gamma}(t,x_{1}))&=\psi(\phi_{\Gamma}(t,x_{2})).

At , . Hence, \eqref{eqn:subset_lev_set11} is satisfied.

To show that 1) implies 2), we define a convenient state transformation inspired by \cite[p.~13]{Morse_Theory_book}. In the new coordinates, the vector field has only one nonzero component.

Let  be a smooth manifold. By Sard's Theorem, there exists an open neighborhood  of  such that any point in  is a regular value of  \cite[p.~132]{Introduction_to_Smooth_Manifolds}. Define the smooth function  by

where  is the gradient of  (with respect to a Riemannian metrics  on M). The function  is well defined, since  is an (open) set of regular points of . Define the vector field  on  by

The derivative of  in the direction of  is

Choose  such that  and . We define the map , where  is the solution of  from initial state . From \eqref{unit_increment}, we have


We represent the vector field  and function  in new coordinates 

\tilde f (q) & = (D F(q))^{-1}f \circ F(q) \\
\tilde \varphi (q) & = \varphi\circ F(q) \label{eqn:varphi_tilde}


Notice that the vector fields  and  are -related. The differential of  is

Hence, the derivative of  if the direction of  is


Let , then  and  are regular points for . By \eqref{eqn:subset_lev_set11}

We define , and conclude that


From~\eqref{eqn:good_map},  only depends on the last coordinate. Therefore, denoting , we have


Since  for any pair  and , we have . In other words, the th component of the vector field  depends only on its last  coordinate . As a consequence, denoting the flow map of the vector field  by , we have

The vector field  and  are -related, hence also

Thus,  the inclusion \eqref{eqn:suf_cond_sound2} holds.
\end{proof}

Unfortunately, we cannot relax proposition to include critical values as well, i.e., it does not hold for any . This is clarified in the following, by presenting a dynamical system with state space  for which there exists no complete abstraction generated by transversal partitioning functions.

According to Proposition~\ref{prop:completeness_of_partition}, it takes the same time for two trajectories to propagate between level sets of complete partitioning functions. Consider a dynamical system with flow lines shown in Figure~\ref{fig:vectorFieldmm}.
\begin{figure}[!htb]
    \centering
       \includegraphics[scale=1.2]{vectorFieldmm.pdf}
    \caption{Flow lines of a dynamical system, with a saddle point, a stable equilibrium point, and an unstable equilibrium point.\label{fig:vectorFieldmm}}
\end{figure}
Level sets of two partitioning functions are illustrated at the saddle point inside the gray circle. From the red level set, one trajectory goes to the saddle point; however, the remaining trajectories diverge from the saddle point. Therefore, it cannot take the same time to propagate between any two level sets, if the partitioning function is decreasing along the vector field.








In the following, we denote the stable (unstable) manifold by  () \cite{hirsch}.
\begin{lemma}\label{lem:crit}
Let  be a partitioning function generating a complete abstraction of \dynSysAll and let  be a singular point of \vectField. Then  is a critical point of .
\end{lemma}
\begin{proof}
Suppose that  is a regular point of  with regular value . Then  is an  dimensional manifold. Without loss of generality, we assume that  is not stable. Let . Then there exists  such that  , but , since  is a singular point of . This contradicts completeness.
\end{proof}

The following theorem is the main contribution of the paper, showing that complete abstractions identify stable and unstable manifolds.\begin{theorem}\label{thm:res}
Let  denote the set of regular values of , generating a complete abstraction of \dynSysAll, let \sts be a connected compact manifold, and let  be a singular point of \vectField. Suppose that . Then 
\end{theorem}
\begin{sketchProof}
Let  and let . Suppose that there exists a point  such that .

Per definition of , . Let  then there exists a singular point  such that . Note that any solution goes to a singular point, as the state space \sts is compact.

Since , the abstraction is complete, and the flow map  is continuous, we conclude from Proposition~\ref{prop:prop2} that . Thus, for any  the value of  is constant, i.e., . Therefore, .

To show that , we exploit that any ,  for some singular point . Therefore,  for any such singular point.
We use notation  iff there is a flow line from  to . Whenever there is a sequence  of singular points of the vector field  such that

then  for all .
Such a scenario is illustrated in Figure~\ref{fig:multi_sing_proof}.
\begingroup \makeatletter \providecommand\color[2][]{\errmessage{(Inkscape) Color is used for the text in Inkscape, but the package 'color.sty' is not loaded}\renewcommand\color[2][]{}}\providecommand\transparent[1]{\errmessage{(Inkscape) Transparency is used (non-zero) for the text in Inkscape, but the package 'transparent.sty' is not loaded}\renewcommand\transparent[1]{}}\providecommand\rotatebox[2]{#2}\ifx\svgwidth\undefined \setlength{\unitlength}{396bp}\ifx\svgscale\undefined \relax \else \setlength{\unitlength}{\unitlength * \real{\svgscale}}\fi \else \setlength{\unitlength}{\svgwidth}\fi \global\let\svgwidth\undefined \global\let\svgscale\undefined \makeatother \begin{figure}\put(0,0){\includegraphics[width=\unitlength]{multi_sing_proof.pdf}}\put(0.45894143,0.35794543){\color[rgb]{0,0,0}\makebox(0,0)[lb]{\smash{}}}\put(0.35988045,0.16176776){\color[rgb]{0,0,0}\makebox(0,0)[lb]{\smash{}}}\put(0.56116808,0.12436601){\color[rgb]{0,0,0}\makebox(0,0)[lb]{\smash{}}}\put(0.88746961,0.18442785){\color[rgb]{0,0,0}\makebox(0,0)[lb]{\smash{}}}\put(0.47248211,0.31349297){\color[rgb]{0,0,0}\makebox(0,0)[lb]{\smash{}}}

 \caption{State space of a system with singular points , , and . No solution initialized on the red level set reaches , but solutions get arbitrarily close to .}
\label{fig:multi_sing_proof}
  \end{figure}\endgroup
Thus, we have



If  for all , then

The dimension of  is . Furthermore, since  is a closed compact manifold and the vector field is transversal to , there exists  such that

that is an embedded manifold of dimension . Thereby , and .
\qed
\end{sketchProof}
One could think that the inclusion  in Theorem~\ref{thm:res} could be replaced by an equality; however, the following counterexample demonstrates that this cannot be done.
\begin{example}
Consider the following two dimensional linear vector field from Example~\ref{ex:saddle_lin}

The system has a saddle point  and its phase plot is shown in Figure~\ref{fig:example1_saddle_phase_plot}. We modify the partitioning function  from Example~\ref{ex:saddle_lin}, to obtain a case where

For this purpose, we construct a bump function, according to \cite{An_Introduction_to_Manifolds}. Let

From  and , we choose the partitioning function

with  and . The graphs of , , and  are shown in Figure~\ref{fig:bump_example}.
\begin{figure}[!htb]
    \centering
       \includegraphics[scale=1]{bump_example1.pdf}
    \caption{The left subplot shows the graph of  and the right subplot shows the graphs of   (blue) and  (dashed red).\label{fig:bump_example}}
\end{figure}

The partitioning function shown in \eqref{eqn:bump_varphi1} generates a complete abstraction and for this particular function  is a proper subset of .
\end{example}

Theorem~\ref{thm:res} is vital in the field of abstracting generic dynamical systems, as it shows that a partitioning function generating a complete abstraction has to be constant on the unstable manifolds; hence, finding such a function is as difficult as finding the stable and unstable manifolds. This is practically impossible for systems of dimension greater than three. As an example, on all black lines in Figure~\ref{fig:morse_complex} a complete partitioning function must be constant.
\begin{figure}[!htb]
    \centering
       \includegraphics[scale=0.8]{morse_complex.pdf}
    \caption{Morse complex, where a "+" indicates a saddle, a "" indicates a maximum, and a "" indicates a minimum. The lines connecting the singular points are stable and unstable manifolds.\label{fig:morse_complex}}
\end{figure}
In conclusion, research in the field should be focused on method development for generating sound rather than complete abstractions.
