We lift define a function $\M\seqpar{\cdot}$ over sequences $\NonT\ComN^*(\NComN\ComN^*)^*$ in the following way:
\begin{align*}
\M\seqpar{C_1\cdots C_n}         &= \left\{ \bigoplus_{i=1}^n \M(w_i) \mid C_i \toSF^* w_i, w_i \in \CommWords\right\}\\
\M\seqpar{C_1\cdots C_n Z\alpha} &= \M\seqpar{C_1\cdots C_n} \cdot Z \cdot \M\seqpar{\alpha}\\
\M\seqpar{X\alpha} &= X \cdot \M\seqpar{\alpha}\\
\M\seqpar{a\alpha} &= a \cdot \M\seqpar{\alpha}\\
\end{align*}
Let $U,V \subseteq \Control^\M$, we define $U \toSM V$ just if for all $\gamma' \in V$ there exists a $\gamma \in U$ such that $\gamma \toSM \gamma'$.
\begin{lemma}\label{apx:lemma:mseqpar_sim_SF}
If $\alpha \toSF \beta$ such that $\alpha \in \NonT^*$ then $\M\seqpar{\alpha} \toSM \M\seqpar{\beta}$.
\end{lemma}
\begin{proof}
Since $\alpha \toSF \beta$ we have $\alpha = X\alpha_0$ and $\beta = \alpha_1\alpha_0$ such that 
$X \rightarrow \alpha_1$. And so $\M\seqpar{\alpha} = X\M\seqpar{\alpha_0}$. We will proceed by case analysis on $X \rightarrow \alpha_1$.
\begin{itemize}
 	\item $X \rightarrow a$, $a \in \Sigma \union \{\epsilon\}$. \newline 
 		  Take $a\delta \in a\cdot\M\seqpar{\alpha_0} = \M\seqpar{a\alpha_0} = \M\seqpar{\beta}$.
 		  Then $X\delta \in \M\seqpar{\alpha}$ and $X\delta \toSM a\delta$. Hence $\M\seqpar{\alpha} \toSM \M\seqpar{\beta}$.
 	\item $X \rightarrow aA$, $a \in \Sigma$. \newline 
 		  Take $aA\delta \in a\cdot A\cdot\M\seqpar{\alpha_0} = \M\seqpar{aA\alpha_0} = \M\seqpar{\beta}$, then since
 		  $X\delta \in \M\seqpar{\alpha}$ and $X\delta \toSM aA\delta$. Hence $\M\seqpar{\alpha} \toSM \M\seqpar{\beta}$.	 
 	\item $X \rightarrow AB$, $B \in \NComN$. \newline 
 		  Suppose $AB\delta \in A \cdot B \cdot \M\seqpar{\alpha_0} = \M\seqpar{\alpha_1\alpha_0} = \M\seqpar{\beta}$, then
 		  $X\delta \in \M\seqpar{\alpha}$ and $X\delta \toSM AB\delta$. Hence $\M\seqpar{\alpha} \toSM \M\seqpar{\beta}$.	
 	\item $X \rightarrow AB$, $B \in \ComN$. \newline 
 		  Suppose ${A M'\delta \in \M\seqpar{AB\alpha_0}} = \M\seqpar{\alpha_1\alpha_0} = \M\seqpar{\beta}$. Then clearly ${M' = \M(w) \oplus M}$ such that $B \toSF^* w$ and $M\delta \in \M\seqpar{\alpha_0}$.
 		  Now then $X M\delta \in \M\seqpar{\alpha}$ and $X M\delta \toSM A(\M(w) \oplus M)\delta$.
 		  Thus $\M\seqpar{\alpha} \toSM \M\seqpar{\beta}$.	 	   
 \end{itemize} 

\end{proof}

For $U, V \subseteq \M[\Control^\M]$ define
\begin{align*}
U \parallel V &:= \{\Pi_0 \parallel \Pi_1 \mid \Pi_0 \in U, \Pi_1 \in V\}\\
\intertext{Further we define}
\M\seqpar{\Pi \,\parallel\, \Pi'} &:= \M\seqpar{\Pi} \parallel \M\seqpar{\Pi'}
\end{align*}

We say that for $U, V \subseteq \M[\Control^\M]$ $U \ChanPar \Gamma \toCM V \ChanPar \Gamma'$, just if for all $\Pi' \in V$ there exists $\Pi \in U$ such that $\Pi \ChanPar \Gamma \toCM \Pi' \ChanPar \Gamma'$. Note this means that if $\M\seqpar{\alpha} \toSM \M\seqpar{\beta}$ then clearly $\M\seqpar{\cproc{\alpha} \parallel \Pi} \ChanPar \Gamma \toCM \M\seqpar{\cproc{\beta} \parallel \Pi} \ChanPar \Gamma$ for all $\Pi$, $\Gamma$.

\begin{lemma}\label{apx:lemma:leftmost_term_simulation}
\begin{enumerate}
	\item If ${a \in \ComSigma}$ and
			  $\cproc{a\alpha\alpha'} \parallel \Pi \ChanPar \Gamma \toCF 
			   \cproc{\alpha\alpha'}  \parallel \Pi \oplus \Pi(a) \ChanPar \Gamma \oplus \Gamma(a)$
			where 
				  $\alpha \in {(\NonT \union \{\epsilon\})\ComN^*}$, 
				  $\alpha' \in (\NComN \ComN)^*$, 
			then 
			$$\M\seqpar{\cproc{a\alpha\alpha'} \parallel \Pi} \ChanPar \Gamma \toCM 
			  \M\seqpar{\cproc{\alpha\alpha'}  \parallel \Pi \oplus \Pi(a)} \ChanPar \Gamma \oplus \Gamma(a).$$
	\item If $\cproc{(\rec{c}{m})\alpha\alpha'} \parallel \Pi \ChanPar \Gamma \oplus \Gamma(\snd{c}{m}) 
			  \toCF 
			  \cproc{\alpha\alpha'}           \parallel \Pi \ChanPar \Gamma $
			where 
				  $\alpha \in {(\NonT \union \{\epsilon\})\ComN^*}$, 
				  $\alpha' \in (\NComN \ComN)^*$, 
			then 
			$$\M\seqpar{\cproc{(\rec{c}{m})\alpha\alpha'} \parallel \Pi} \ChanPar \Gamma \oplus \Gamma(\snd{c}{m})\toCM 
			  \M\seqpar{\cproc{\alpha\alpha'}             \parallel \Pi} \ChanPar \Gamma.$$
\end{enumerate}
\end{lemma}
\begin{proof}[Claim 1]
We show that 
$\M\seqpar{\cproc{a\alpha\alpha'} \parallel \Pi} \ChanPar \Gamma \toCM 
  \M\seqpar{\cproc{\alpha\alpha'} \parallel \Pi'} \ChanPar \Gamma' $
by case analysis on $a$:
\begin{itemize}
\item $a = \snd{c}{m}$ \newline
	Then $\Pi \oplus \Pi(a) = \Pi$, 

	Take $\cproc{\gamma} \parallel \pi \in \M\seqpar{\cproc{\alpha\alpha'} \parallel \Pi}$ then 
	$\cproc{(\snd{c}{m})\gamma} \parallel \pi \in \M\seqpar{\cproc{(\snd{c}{m})\alpha\alpha'} \parallel \Pi}$. 
	Using rule \ref{def:con_mult_sem_send} we see that 
	$${\cproc{(\snd{c}{m})\gamma} \parallel \pi \ChanPar \Gamma \toCM \cproc{\gamma} \parallel \pi \ChanPar \Gamma \oplus \Gamma(\snd{c}{m})}.$$
	
	Hence we conclude 
	${\M\seqpar{\cproc{(\snd{c}{m})\alpha\alpha'} \parallel \Pi} \ChanPar \Gamma \toCM 
	     \M\seqpar{\cproc{\alpha\alpha'} \parallel \Pi} \ChanPar \oplus \Gamma(\snd{c}{m})}$.
	\item $a = \nu X$ \newline
	Then $\Pi \oplus \Pi(\nu X) = \Pi \parallel \cproc{X}$, 
	     $\Gamma = \Gamma$

	Take $\cproc{\gamma} \parallel \cproc{X} \parallel \pi \in \M\seqpar{\cproc{\alpha\alpha'} \parallel \cproc{X} \parallel\Pi} = \M\seqpar{\cproc{\alpha\alpha'} \parallel \Pi \oplus \Pi(\nu X)}$ then
	$\cproc{(\nu X)\gamma} \parallel \pi \in \M\seqpar{\cproc{(\nu X)\alpha\alpha'} \parallel \Pi}$.
	Using rule \ref{def:con_mult_sem_spawn} we see that 
	$${\cproc{(\nu X)\gamma} \parallel \pi \ChanPar \Gamma \toCM \cproc{\gamma} \parallel \cproc{X} \parallel \pi \ChanPar \Gamma}.$$ 
	
	Hence we conclude 
	${\M\seqpar{\cproc{(\nu X)\alpha\alpha'} \parallel \Pi} \ChanPar \Gamma \toCM 
	     \M\seqpar{\cproc{\alpha\alpha'} \parallel \Pi \oplus \Pi(\nu X)} \ChanPar \Gamma}$.
	\item $a = l$ \newline
	Then $\Pi    \oplus \Pi(l)    = \Pi$, 
	     $\Gamma \oplus \Gamma(l) = \Gamma$.
	Take $\cproc{\gamma} \parallel \pi \in \M\seqpar{\cproc{\alpha\alpha'} \parallel\Pi}$ then
	$\cproc{l\gamma} \parallel \pi \in \M\seqpar{\cproc{l\alpha\alpha'} \parallel \Pi}$. 
	Using rule \ref{def:con_mult_sem_label} we see that 
	$${\cproc{l\gamma} \parallel \pi \ChanPar \Gamma \toCM \cproc{\gamma} \parallel \pi \ChanPar \Gamma}.$$ 
	Hence we conclude 

	${\M\seqpar{\cproc{l\alpha\alpha'} \parallel \Pi} \ChanPar \Gamma \toCM 
	     \M\seqpar{\cproc{\alpha\alpha'} \parallel \Pi} \ChanPar \Gamma}$.
\end{itemize}
\end{proof}
\begin{proof}[Claim 2]
	Take $\cproc{\gamma} \parallel \pi \in \M\seqpar{\cproc{\alpha\alpha'} \parallel \Pi} = \M\seqpar{\cproc{\alpha\alpha'} \parallel \Pi}$
	then $\cproc{(\rec{c}{m})\gamma} \parallel \pi \in \M\seqpar{\cproc{(\rec{c}{m})\alpha\alpha'} \parallel \Pi}$

	Then using rule \ref{def:con_mult_sem_rec} we see that 
	$${\cproc{(\rec{c}{m})\gamma} \parallel \pi \ChanPar \Gamma \oplus \Gamma(\snd{c}{m}) \toCM 
	\cproc{\gamma} \parallel \pi \ChanPar \Gamma}.$$ 
	
	Hence we conclude 
	${\M\seqpar{\cproc{(\rec{c}{m})\alpha\alpha'} \parallel \Pi} \ChanPar \Gamma \oplus \Gamma(\snd{c}{m}) \toCM 
	     \M\seqpar{\cproc{\alpha\alpha'} \parallel \Pi} \ChanPar \Gamma}$.
\end{proof}
\pagebreak
\begin{lemma}\label{apx:lemma:leftmost_term_simulation_with_comsideeffects}
	\item If $a_1 \cdots a_n \in \ComSigma^*$, $\alpha_i \in \ComN^*$ and $\alpha' \in (\NComN \ComN)^*$, 
			\begin{align*}
			 \cproc{\alpha_1\alpha'} \parallel \Pi(\epsilon) \ChanPar \Gamma(\epsilon) &\toCF^*  
			 \cproc{a_1\alpha_2\alpha'} \parallel \Pi(\epsilon) \ChanPar \Gamma(\epsilon) \\
			 & \toCF \cproc{\alpha_2\alpha'} \parallel \Pi(a_1) \ChanPar \Gamma(a_1) \\
			 & \toCF^* \cdots \toCF^* \\
			 & \cproc{a_n\alpha_{n+1}\alpha'} \parallel \Pi(a_1\cdots a_{n-1}) \ChanPar \Gamma(a_1\cdots a_{n-1}) \\
			 & \toCF^* \cproc{\alpha_{n+1}\alpha'} \parallel \Pi(a_1\cdots a_{n}) \ChanPar \Gamma(a_1\cdots a_{n})
			\end{align*}  
			then 
			$$\M\seqpar{\cproc{\alpha_1\alpha'} \parallel \Pi(\epsilon)} \ChanPar \Gamma(\epsilon) \toCM^* 
			\M\seqpar{\cproc{\alpha_{n+1}\alpha'}  \parallel \Pi(a_1\cdots a_{n})} \ChanPar \Gamma(a_1\cdots a_{n}).$$
\end{lemma}
\begin{proof}
We prove the claim by induction on $n$.
For $n = 0$ the claim is vacuously true.

For $n = k + 1$, assuming the claim holds for $k$ it is enough to show that if
\begin{align*}
\cproc{\alpha_{k+1}\alpha'} \parallel \Pi(a_1\cdots a_{k}) \ChanPar \Gamma(a_1\cdots a_{k}) &\toCF^*
\cproc{a_{k+1}\alpha_{k+2}\alpha'} \parallel \Pi(a_1\cdots a_{k}) \ChanPar \Gamma(a_1\cdots a_{k})\\ 
&\toCF^* \cproc{\alpha_{k+2}\alpha'} \parallel \Pi(a_1\cdots a_{k+1}) \ChanPar \Gamma(a_1\cdots a_{k+1})
\end{align*}
then 
$$\M\seqpar{\cproc{\alpha_{k+1}\alpha'} \parallel \Pi(a_1\cdots a_{k})} \ChanPar \Gamma(a_1\cdots a_{k}) 
\toCM^*
\M\seqpar{\cproc{\alpha_{k+2}\alpha'}  \parallel \Pi(a_1\cdots a_{k+1})} \ChanPar \Gamma(a_1\cdots a_{k+1})$$
which we obtain by repeatedly applying Lemma \ref{apx:lemma:mseqpar_sim_SF} and then Lemma \ref{apx:lemma:leftmost_term_simulation}.
\end{proof}

\begin{lemma}\label{apx:lemma:term_sideeffect_pop}
If 
$\cproc{a\alpha\alpha'} \parallel \Pi \ChanPar \Gamma \toCF 
	\cproc{\alpha\alpha'} \parallel \Pi' \ChanPar \Gamma' \toCF^* 
	\cproc{\alpha'} \parallel \Pi'' \ChanPar \Gamma''$ 
where ${a \in \Sigma}$, 
	  $\alpha \in \ComN^*$, 
	  $\alpha' \in (\NComN \ComN)^*$, 
	  ${\alpha \toSF^* w \in \CommTermWords}$, 
	  $\Pi'' = \Pi' \oplus \Pi(w)$, 
	  $\Gamma'' = \Gamma' \oplus \Gamma(w)$
then 
$$\M\seqpar{\cproc{a\alpha\alpha'} \parallel \Pi} \ChanPar \Gamma \toCM \M\seqpar{\cproc{\alpha\alpha'} \parallel \Pi'} \ChanPar \Gamma' \toCM \M\seqpar{\cproc{\alpha'} \parallel \Pi''} \ChanPar \Gamma''.$$
\end{lemma}
\begin{proof}
For
 $\M\seqpar{\cproc{a\alpha\alpha'} \parallel \Pi} \ChanPar \Gamma \toCM 
  \M\seqpar{\cproc{\alpha\alpha'} \parallel \Pi'} \ChanPar \Gamma' $
we appeal to Lemma \ref{apx:lemma:leftmost_term_simulation}.
Now since $\alpha \toSF^* w$ we have 
$\cproc{\alpha\alpha'} \parallel \Pi' \ChanPar \Gamma' \toCF^* 
 \cproc{\alpha'} \parallel \Pi' \oplus \Pi(w) \ChanPar \Gamma' \oplus \Gamma(w)$.

Let $M = \M(w)$ and
take $\cproc{\gamma} \parallel \pi' \oplus \Pi(M) \ChanPar \Gamma' \oplus \Gamma(M) \in 
{\M\seqpar{\cproc{\alpha'} \parallel \Pi' \oplus \Pi(w)} \ChanPar \Gamma' \oplus \Gamma(w)}$
Then note
$\cproc{M\gamma} \parallel \pi' \in \M\seqpar{\cproc{\alpha\alpha'} \parallel \Pi'}$ 
and using rule \ref{def:con_mult_sem_disp}
$$\cproc{M\gamma} \parallel \pi' \ChanPar \Gamma' \toCM \cproc{\gamma} \parallel \pi' \oplus \Pi(M) \ChanPar \Gamma' \oplus \Gamma(M).$$
\end{proof}

\begin{lemma}\label{apx:lemma:nonterm_sideeffect_partialpop}
If 
$\cproc{a\alpha\alpha'}   \parallel \Pi \ChanPar \Gamma \toCF 
	\cproc{\alpha\alpha'} \parallel \Pi' \ChanPar \Gamma' \toCF^* 
	\cproc{w\alpha'}      \parallel \Pi' \ChanPar \Gamma' \toCF^* 
	\cproc{w'\alpha'}     \parallel \Pi'' \ChanPar \Gamma''$ 
where ${a \in \Sigma}$, 
	  $\alpha \in \ComN^*$, 
	  $\alpha' \in (\NComN \ComN)^*$, 
	  $\alpha'' \in \NComN^*$ and
	  ${\alpha \toSF^* w \in \CommWords}$, 
	  $w' \in \CommNonTermWords$,
	  $\Pi'' = \Pi' \oplus \Pi(w)$, 
	  $\Gamma'' = \Gamma' \oplus \Gamma(w)$
then 
$$\M\seqpar{\cproc{a\alpha\alpha'} \parallel \Pi} \ChanPar \Gamma \toCM \M\seqpar{\cproc{\alpha\alpha'} \parallel \Pi'} \ChanPar \Gamma' \toCM  \cproc{\M(w') \cdot \M\seqpar{\alpha'}} \parallel \M\seqpar{\Pi''} \ChanPar \Gamma''.$$
\end{lemma}
\begin{proof}
For
 $\M\seqpar{\cproc{a\alpha\alpha'} \parallel \Pi} \ChanPar \Gamma \toCM 
  \M\seqpar{\cproc{\alpha\alpha'} \parallel \Pi'} \ChanPar \Gamma' $
we appeal to Lemma \ref{apx:lemma:leftmost_term_simulation}.
Now since $\alpha \toSF^* w$ we have 
$\cproc{\alpha\alpha'} \parallel \Pi' \ChanPar \Gamma' \toCF^* 
\cproc{w'\alpha'} \parallel \Pi' \oplus \Pi(w) \ChanPar \Gamma' \oplus \Gamma(w)$.

Let $M' = M(w')$ then

$\cproc{M'\gamma} \parallel \pi' \oplus \Pi(M) \ChanPar \Gamma' \oplus \Gamma(M) \in \cproc{\M(w') \cdot \M\seqpar{\alpha'}} \parallel \M\seqpar{\Pi' \oplus \Pi(w)} \ChanPar \Gamma' \oplus \Gamma(w)$
and
$\cproc{M\gamma} \parallel \pi' \in \M\seqpar{\cproc{\alpha\alpha'} \parallel \Pi'}$ such that $M = \M(w)$.
Using rule \ref{def:con_mult_sem_non-term}
$$\cproc{M\gamma} \parallel \pi' \ChanPar \Gamma' \toCM \cproc{M'\gamma} \parallel \pi' \oplus \Pi(M) \ChanPar \Gamma' \oplus \Gamma(M).$$
\end{proof}

\begin{lemma}\label{apx:lemma:sim_fromcom_pop_or_nonterm}
Let $X \in \NonT$, $\alpha \in (\NComN\ComN^*)^*$, $w \in \ComSigma$ and $\beta,\alpha' \in \ComN^*$
\begin{enumerate}
	\item If $\cproc{X\beta\alpha} \ChanPar \Gamma(\epsilon) \Gamma(\epsilon) \toCF^* \cproc{\alpha} \parallel \Pi(w) \ChanPar \Gamma(w)$
	      then 
	      ${\M\seqpar{\cproc{X\beta\alpha}} \ChanPar \Gamma(\epsilon) \toCM^* \M\seqpar{\cproc{\alpha} \parallel \Pi(w)} \ChanPar \Gamma(w)}$
	\item If $\cproc{X\beta\alpha} \ChanPar \Gamma(\epsilon) \toCM^* \cproc{\alpha'\alpha} \parallel \Pi(w) \ChanPar \Gamma(w) $, then $\M\seqpar{\cproc{X\beta\alpha}} \ChanPar \Gamma(\epsilon) \toCF^* {\cproc{\M(\alpha')\cdot\M\seqpar{\alpha}}} \parallel \M\seqpar{\Pi(w)} \ChanPar \Gamma(w)$
\end{enumerate}
\end{lemma}
\begin{proof}[Claim 1]
Then $X\beta\alpha \toSF^* a\alpha_0\alpha$ where 
$a \in \ComSigma \union \{\epsilon\}$ such that $a\alpha_0 \toSF^* w$ 
so by Lemma \ref{apx:lemma:mseqpar_sim_SF} $\M\seqpar{X\beta\alpha} \toSM^* \M\seqpar{a\alpha_0\alpha}$.
By Lemma \ref{apx:lemma:term_sideeffect_pop}
$\M\seqpar{\cproc{a\alpha_0\alpha}} \ChanPar \Gamma(\epsilon) \toCM^* \M\seqpar{\cproc{\alpha} \parallel \Pi(w)} \ChanPar \Gamma(w)$ and clearly also $\M\seqpar{\cproc{X\beta\alpha}} \ChanPar \Gamma(\epsilon) \toCM^* \M\seqpar{\cproc{a\alpha_0\alpha}} \parallel \Pi(w) \ChanPar \Gamma(w)$.
\end{proof}
\begin{proof}[Claim 2]
Then $X\beta\alpha \toSF^* a\alpha_0\alpha$ where 
$a \in \ComSigma \union \{\epsilon\}$ such that $a\alpha_0 \toSF^* w\alpha'$ 
so by Lemma \ref{apx:lemma:mseqpar_sim_SF} $\M\seqpar{X\beta\alpha} \toSM^* \M\seqpar{a\alpha_0\alpha}$.
By Lemma \ref{apx:lemma:nonterm_sideeffect_partialpop}
$\M\seqpar{\cproc{a\alpha_0\alpha}} \ChanPar \Gamma(\epsilon) \toCM^* \M\seqpar{\cproc{\alpha'\alpha} \parallel \Pi(w)} \ChanPar \Gamma(w)$ and clearly also $\M\seqpar{\cproc{X\alpha}} \ChanPar \Gamma(\epsilon) \toCM^* \M\seqpar{\cproc{a\alpha_0\alpha}} \parallel \Pi(w) \ChanPar \Gamma(w)$.
\end{proof}

\begin{lemma}\label{apx:lemma:sim_fromncom_push_or_nonterm}
Let $X \in \NComN$, $\beta,\beta' \in \ComN^*$ and $\alpha,\alpha' \in (\NComN\ComN^*)^*$.
\begin{enumerate}
	\item If $\cproc{X\beta\alpha} \ChanPar \Gamma(\epsilon) \toCF^*  
	          \cproc{\alpha'\alpha} \parallel \Pi(w) \ChanPar \Gamma(w)$
	      then 
	      $\M\seqpar{\cproc{X\beta\alpha}} \ChanPar \Gamma(\epsilon) \toCM^* 
	      \M\seqpar{\cproc{\alpha'\alpha} \parallel \Pi(w)} \ChanPar \Gamma(w)$
	\item If $\cproc{X\beta\alpha} \ChanPar \Gamma(\epsilon) \toCF^* \cproc{(\rec{c}{m})\beta'\alpha'\alpha} \parallel \Pi(w) \ChanPar \Gamma(w)$, then $\M\seqpar{\cproc{X\alpha} \parallel \Pi(w)} \ChanPar \Gamma(\epsilon) \toCM^* \M\seqpar{\cproc{(\rec{c}{m})\beta'\alpha'\alpha}} \ChanPar \Gamma(\epsilon)$.
\end{enumerate}
\end{lemma}
\begin{proof}[Claim 1]
Then 
$$\cproc{X\beta\alpha} \ChanPar \Gamma(\epsilon) \toCF^* 
\cproc{X'\alpha'\alpha} \parallel \Pi(w_0) \ChanPar \Gamma(w_0)$$
such that $X' \toSF^* a\alpha_0$ where 
$a \in \ComSigma \union \{\epsilon\}$, $\alpha_0 \in \ComN^*$ such that $a\alpha_0 \toSF^* w_1$ and $w = w_0w_1$. 
By Lemma \ref{apx:lemma:leftmost_term_simulation_with_comsideeffects} 
$$\M\seqpar{\cproc{X\beta\alpha}} \ChanPar \Gamma(\epsilon) \toCM^* \M\seqpar{\cproc{X'\alpha'\alpha} \parallel \Pi(w_0)} \ChanPar \Gamma(w_0)$$
Then the proof of Lemma \ref{apx:lemma:sim_fromcom_pop_or_nonterm} Claim 1 applies to give the result.
\end{proof}
\begin{proof}[Claim 2]
Then 
$$\cproc{X\beta\alpha} \ChanPar \Gamma(\epsilon) \toCM^* \cproc{X'X''\beta'\alpha'\alpha} \parallel \Pi(w_0) \ChanPar \Gamma(w_0)$$
such that $X'' \toSF^* \rec{c}{m}$, and $X' \toSF^* w_1$ where $w=w_0w_1$. Hence
$$\cproc{X'X''\beta'\alpha'\alpha} \parallel \Pi(w_0) \ChanPar \Gamma(w_0) \toCM^* \cproc{(\rec{c}{m})\beta'\alpha'\alpha} \parallel \Pi(w) \ChanPar \Gamma(w).$$

so by Lemma \ref{apx:lemma:leftmost_term_simulation_with_comsideeffects} and Lemma \ref{apx:lemma:mseqpar_sim_SF}.
$$\M\seqpar{\cproc{X\beta\alpha}} \ChanPar \Gamma(\epsilon) \toCM^* 
  \M\seqpar{\cproc{(\rec{c}{m})\beta'\alpha'\alpha} \parallel \Pi(w)} \ChanPar \Gamma(w).$$
\end{proof}

For $\alpha_1,\ldots,\alpha_m \in \ComN^*$ and $Z_1,\ldots, Z_{m-1} \in \NComN$ define
\begin{align*}
\seqM(\alpha_1 Z_1\cdots \alpha_{m-1}Z_{m-1}\alpha_m) &:= \M(\alpha_1)Z_1 \cdots \M(\alpha_{m-1})Z_{m-1}\M(\alpha_m)\\
\seqM(\Pi \parallel \Pi') &:= \seqM(\Pi) \parallel \seqM(\Pi')
\end{align*}

\begin{proposition}\label{apx:prop:conc_reduction_simulation}
If $\cproc{S} \ChanPar \Gamma(\epsilon) \toCF^* \Pi' \ChanPar \Gamma'$
then $\seqM(\cproc{S}) \ChanPar \Gamma(\epsilon) \toCM^* \seqM(\Pi') \ChanPar \Gamma'$
\end{proposition}
\begin{proof}
Let $\Pi^f \in \M[\Control]$ and define the set
$P_{\Pi^f} = \{ \alpha \mid \exists \Pi. \cproc{\alpha} \parallel \Pi = \Pi^f \}$.
further define the set of configurations
$P := \{ \Pi \ChanPar \Gamma \mid \forall \cproc{\alpha} \in \Pi, \alpha \in \NonT(\NComN\ComN^*)^* \union \NComSigma\ComN^*(\NComN\ComN^*)^* \union \NonT \union P_{\Pi^f}\}$.

Now suppose that for some $\Pi, \Pi'$ and $\Gamma, \Gamma'$
$$\Pi \ChanPar \Gamma := \Pi_0 \ChanPar \Gamma_0 \toCF^* 
  \Pi_1 \ChanPar \Gamma_1 \toCF^* \cdots \toCF^* \Pi_n \ChanPar \Gamma_n =: \Pi' \ChanPar \Gamma'$$ 
such that $\Pi_i \ChanPar \Gamma_i \in P$ for $i = 0,\ldots n$. Without loss of generality we can assume that for all $i = 0,...,n$, $\Pi_i = \Pi_i^a \parallel \Pi_i^f$ such that for all 
$\cproc{\alpha} \in \Pi_i^f$ we have $\cproc{\alpha} \in P_{\Pi^f}$ and 
$\cproc{\alpha}$ is not involved in any transitions in 
$\Pi_i \ChanPar \Gamma_i \toCF^* \Pi_n \ChanPar \Gamma_n$. 
Note that we are not loosing generality, since a reduction 
$\cproc{\alpha} \parallel \Pi \toCF^* \cproc{\alpha} \parallel \Pi'$ can either be pre-empted or goes through a process state in $\NComSigma\ComN^*(\NComN\ComN^*)^*$. 
Note this also means that $\Pi_{i+1}^f = \Pi_{i}^f \parallel {\Pi'}_{i}^f$. We further assume w.l.o.g that for each $i$ it is the case that 
$\Pi^a_i = \cproc{\alpha} \parallel \Pi'_i$ and $\Pi^a_{i+1} = \cproc{\alpha'} \parallel \Pi'_i \oplus \Pi(w)$ 
and $\Gamma_{k+1} \oplus \Gamma(w') = \Gamma_k \oplus \Gamma(w)$ 
for some $w \in \ComSigma^*$ and $w' \in \{\epsilon\} \union \ComSigma$,
i.e. during each $\Pi^a_i \ChanPar \Gamma_i \toCF^* \Pi^a_{i+1} \ChanPar \Gamma_{i+1}$ only one process makes progress (note this can be achieved by delaying receptions and performing sends and spawns as early as possible) and none of the intermediate steps are configurations of $P$.

We will prove by induction on $n$: 
$$\M\seqpar{\Pi^a_0} \parallel \tilde{\Pi}^f_0 \ChanPar \Gamma_0 \toCM^* \M\seqpar{\Pi^a_1} \parallel \tilde{\Pi}^f_1 \ChanPar \Gamma_1 \toCM^* \cdots \toCM^* \M\seqpar{\Pi_n} \parallel \tilde{\Pi}^f_n \ChanPar \Gamma_n$$
where for all $i = 0,\ldots n$ and $\cproc{\alpha} \in \Pi_i^f$ we have either $\Pi_i^f(\cproc{\alpha}) = \tilde{\Pi}^f_i(\M\seqpar{\cproc{\alpha}})$ or $\Pi_i^f(\cproc{\alpha}) = \tilde{\Pi}^f_i(\cproc{\M(\alpha_0) \cdot \M\seqpar{\alpha_1}})$, $\alpha = \alpha_0\alpha_1$.
\begin{itemize}
	\item $n = 0$. \newline
	The claim holds trivially.
	\item $n = k+1$, assuming the claim holds for $k$. \newline
	To prove the inductive claim we need to show that from
	$\Pi_k = \cproc{\alpha} \parallel \Pi'_k$, $\Pi_{k+1} =\cproc{\alpha'} \parallel \Pi'_k \oplus \Pi(w)$
	and
	$\Gamma_{k+1} \oplus \Gamma(w') = \Gamma_k \oplus \Gamma{w}$
	where $\Pi_k \ChanPar \Gamma_k \toCF^* \Pi_{k+1} \ChanPar \Gamma_{k+1}$,
	we can infer $\M\seqpar{\Pi_k} \ChanPar \Gamma_k \toCF^* \M\seqpar{\Pi_{k+1}} \ChanPar \Gamma_{k+1}$.
	We will do so by a case analysis on the shape of $\alpha$ and $\alpha'$.
	\begin{itemize}
		\item $\alpha, \alpha' \in (\NComN\ComN^*)^*$ \newline
		Then $\alpha = X\alpha_0\alpha_1$, $X \in \NComN$, $\alpha_0 \in \ComN^*$, 
		$\alpha_1 \in (\NComN\ComN^*)^*$ and $\alpha' = \alpha'_0\alpha_1$ where $\alpha'_0 \in \epsilon \union (\NComN\ComN^*)^*$, i.e. either we increase the call-stack or we pop one non-commutative non-terminal off the call-stack. Otherwise we would end up either in an intermediate configuration in $P$ or in a different case.
		\begin{itemize}
			\item Case $\alpha'_0 = \epsilon$.\newline
			Then $\cproc{X\alpha_0\alpha_1} \parallel \Pi'_k \ChanPar \Gamma_k \toCF^* 
				\cproc{X'\alpha_2\alpha_0\alpha_1} \parallel \Pi'_k \ChanPar \Gamma_k$ such that 
				${X' \in \ComN}$, $\alpha_2 \in \ComN^*$ and 
				$\cproc{X'\alpha_2\alpha_0\alpha_1} \parallel \Pi'_k \ChanPar \Gamma_k \toCF^*
					\cproc{w\alpha_1} \parallel \Pi'_k \ChanPar \Gamma_k \toCF^* 
					\cproc{\alpha_1} \parallel \Pi'_k \oplus \Pi(w) \ChanPar \Gamma_k \oplus \Gamma(w)$, where $w \in \ComSigma^*$ such that 
					$\Pi_{k+1} =\cproc{\alpha'} \parallel \Pi'_k \oplus \Pi(w)$
					and
					$\Gamma_{k+1} = \Gamma_k \oplus \Gamma{w}$.
					Lemma \ref{apx:lemma:leftmost_term_simulation} then allows us to conclude that
					$\M\seqpar{\cproc{X\alpha_0\alpha_1} \parallel \Pi'_k} \ChanPar \Gamma_k 
					\toCM^* 
				    \M\seqpar{\cproc{X'\alpha_2\alpha_0\alpha_1} \parallel \Pi'_k} \ChanPar \Gamma_k$ and
				    Lemma \ref{apx:lemma:sim_fromcom_pop_or_nonterm}.1 gives us
				    $\M\seqpar{\cproc{X'\alpha_2\alpha_0\alpha_1} \parallel \Pi'_k} \ChanPar \Gamma_k \toCM^*
				    \M\seqpar{\cproc{\alpha_1} \parallel \Pi'_k \oplus \Pi(w)} \ChanPar \Gamma_k \oplus \Gamma(w) = \Pi_{k+1} \ChanPar \Gamma_{k+1}$.
			\item Case $\alpha'_0 \neq \epsilon$.\newline
				Follows directly from Lemma \ref{apx:lemma:sim_fromncom_push_or_nonterm}.1
		\end{itemize}
		\item $\alpha \in \NComSigma\ComN^*(\NComN\ComN^*)^*$ and $\alpha' \in (\NComN\ComN^*)^*$ \newline
		Follows from Lemma \ref{apx:lemma:term_sideeffect_pop}.
		\item $\alpha \in (\NComN\ComN^*)^*$ and $\alpha' \in \NComSigma\ComN^*(\NComN\ComN^*)^*$ \newline
		Follows from Lemma \ref{apx:lemma:sim_fromncom_push_or_nonterm}.2
		\item $\alpha \in \NonT$ and $\alpha' \in (\NComN\ComN^*)^*$\newline
		We can assume that $\alpha \in \ComN$ since otherwise a case above already applies.
		By the definition of $\ComN$ we can thus infer that $\alpha' = \epsilon$ since otherwise 
		$\alpha$ would not be commutative.
		Thus Lemma \ref{apx:lemma:sim_fromcom_pop_or_nonterm}.1 applies.
		\item $\alpha \in \NonT$ and $\alpha' \in \NComSigma\ComN^*(\NComN\ComN^*)^*$\newline
		There is nothing to prove for this case as, similarly to the case above, either $\alpha \in \NComN$ and so a case above applies or $\alpha \in \ComN$ but then $\alpha' \notin \NComSigma\ComN^*(\NComN\ComN^*)^*$ which is impossible; so the former must be the case.
		\item $\alpha \in (\NComN\ComN^*)^*$ and $\alpha' \in P_{\Pi_f}$\newline
			If $\alpha' \in (\NComN\ComN^*)^* \union \NComSigma\ComN^*(\NComN\ComN^*)^*$ the above cases apply. Otherwise it must be the case that $\alpha' \in \ComN^*(\NComN\ComN^*)^* \union \ComSigma\ComN^*(\NComN\ComN^*)^*$.
			\begin{itemize}
				\item $\alpha' \in \ComN^*(\NComN\ComN^*)^*$ \newline
				So it must be the case that
				$\alpha = X\alpha_0\alpha_1$, $X \in \NComN$, $\alpha_0 \in \ComN^*$ $\alpha_1 \in (\NComN\ComN^*)^*$ and $\alpha' = \alpha'_0\alpha'_1\alpha_1$ where 
				$\alpha'_1 \in (\NComN\ComN^*)^*$, $\alpha'_0 \in \ComN^*$
				Lemma \ref{apx:lemma:sim_fromcom_pop_or_nonterm}.2 applies to give
				$$\M\seqpar{\cproc{X\alpha_0\alpha_1} \parallel \Pi_k} \ChanPar \Gamma_k \toCM^* \cproc{\M(\alpha'_0) \cdot \M\seqpar{\alpha'_1\alpha_1}}\parallel \M\seqpar{\Pi_k} \parallel \M\seqpar{\Pi(w)} \ChanPar \Gamma_k \oplus \Gamma(w)$$

				\item $\alpha' \in \ComSigma\ComN^*(\NComN\ComN^*)^*$\newline
				Follows from Lemma \ref{apx:lemma:leftmost_term_simulation_with_comsideeffects}
			\end{itemize}
		\item $\alpha \in \NComSigma\ComN^*(\NComN\ComN^*)^*$ and $\alpha' \in P_{\Pi_f}$\newline
			Unless $\alpha' \in \ComN^*(\NComN\ComN^*)^* \union \ComSigma\ComN^*(\NComN\ComN^*)^*$ this case is covered by a case above.
			The remaining follows from Lemma \ref{apx:lemma:nonterm_sideeffect_partialpop}.
		\item $\alpha \in \NonT$ and $\alpha' \in P_{\Pi_f}$\newline
			Unless $\alpha' \in \ComN^*(\NComN\ComN^*)^* \union \ComSigma\ComN^*(\NComN\ComN^*)^*$ this case is covered by a case above.
			The remaining follows from Lemma \ref{apx:lemma:mseqpar_sim_SF} and Lemma \ref{apx:lemma:leftmost_term_simulation_with_comsideeffects}
	\end{itemize}
	This concludes the proof of the inductive step.
\end{itemize}
Now we apply the above for the case that
$\Pi_0 \ChanPar \Gamma_0 = \cproc{S} \ChanPar \Gamma(\epsilon)$ and $\Pi^f := \Pi'$.
We can then see that $\M\seqpar{\cproc{S}} \ChanPar \Gamma(\epsilon) \toCM^* \tilde{\Pi}^f \ChanPar \Gamma'$.

Then since for all $\alpha \in \ComN^*(\NComN\ComN^*)^*$ it is the case that 
$\seqM(\alpha) \in \M\seqpar{\alpha}$ and further for 
$\alpha_0 \in \ComN^*$, $\alpha_1 (\NComN\ComN^*)^*$
$\seqM(\alpha_0\alpha_1) \in \M(\alpha_0) \cdot \M\seqpar{\alpha_1}$ we can deduce from the definition of \toCM on sets of configurations that
$$\seqM(\cproc{S}) \toCM^* \seqM(\Pi') \ChanPar \Gamma'$$ which concludes the proof.
\end{proof}