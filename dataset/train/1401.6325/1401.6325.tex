We lift define a function  over sequences  in the following way:

Let , we define  just if for all  there exists a  such that .
\begin{lemma}\label{apx:lemma:mseqpar_sim_SF}
If  such that  then .
\end{lemma}
\begin{proof}
Since  we have  and  such that 
. And so . We will proceed by case analysis on .
\begin{itemize}
 	\item , . \newline 
 		  Take .
 		  Then  and . Hence .
 	\item , . \newline 
 		  Take , then since
 		   and . Hence .	 
 	\item , . \newline 
 		  Suppose , then
 		   and . Hence .	
 	\item , . \newline 
 		  Suppose . Then clearly  such that  and .
 		  Now then  and .
 		  Thus .	 	   
 \end{itemize} 

\end{proof}

For  define


We say that for  , just if for all  there exists  such that . Note this means that if  then clearly  for all , .

\begin{lemma}\label{apx:lemma:leftmost_term_simulation}
\begin{enumerate}
	\item If  and
			  
			where 
				  , 
				  , 
			then 
			
	\item If 
			where 
				  , 
				  , 
			then 
			
\end{enumerate}
\end{lemma}
\begin{proof}[Claim 1]
We show that 

by case analysis on :
\begin{itemize}
\item  \newline
	Then , 

	Take  then 
	. 
	Using rule \ref{def:con_mult_sem_send} we see that 
	
	
	Hence we conclude 
	.
	\item  \newline
	Then , 
	     

	Take  then
	.
	Using rule \ref{def:con_mult_sem_spawn} we see that 
	 
	
	Hence we conclude 
	.
	\item  \newline
	Then , 
	     .
	Take  then
	. 
	Using rule \ref{def:con_mult_sem_label} we see that 
	 
	Hence we conclude 

	.
\end{itemize}
\end{proof}
\begin{proof}[Claim 2]
	Take 
	then 

	Then using rule \ref{def:con_mult_sem_rec} we see that 
	 
	
	Hence we conclude 
	.
\end{proof}
\pagebreak
\begin{lemma}\label{apx:lemma:leftmost_term_simulation_with_comsideeffects}
	\item If ,  and , 
			  
			then 
			
\end{lemma}
\begin{proof}
We prove the claim by induction on .
For  the claim is vacuously true.

For , assuming the claim holds for  it is enough to show that if

then 

which we obtain by repeatedly applying Lemma \ref{apx:lemma:mseqpar_sim_SF} and then Lemma \ref{apx:lemma:leftmost_term_simulation}.
\end{proof}

\begin{lemma}\label{apx:lemma:term_sideeffect_pop}
If 
 
where , 
	  , 
	  , 
	  , 
	  , 
	  
then 

\end{lemma}
\begin{proof}
For
 
we appeal to Lemma \ref{apx:lemma:leftmost_term_simulation}.
Now since  we have 
.

Let  and
take 
Then note
 
and using rule \ref{def:con_mult_sem_disp}

\end{proof}

\begin{lemma}\label{apx:lemma:nonterm_sideeffect_partialpop}
If 
 
where , 
	  , 
	  , 
	   and
	  , 
	  ,
	  , 
	  
then 

\end{lemma}
\begin{proof}
For
 
we appeal to Lemma \ref{apx:lemma:leftmost_term_simulation}.
Now since  we have 
.

Let  then


and
 such that .
Using rule \ref{def:con_mult_sem_non-term}

\end{proof}

\begin{lemma}\label{apx:lemma:sim_fromcom_pop_or_nonterm}
Let , ,  and 
\begin{enumerate}
	\item If 
	      then 
	      
	\item If , then 
\end{enumerate}
\end{lemma}
\begin{proof}[Claim 1]
Then  where 
 such that  
so by Lemma \ref{apx:lemma:mseqpar_sim_SF} .
By Lemma \ref{apx:lemma:term_sideeffect_pop}
 and clearly also .
\end{proof}
\begin{proof}[Claim 2]
Then  where 
 such that  
so by Lemma \ref{apx:lemma:mseqpar_sim_SF} .
By Lemma \ref{apx:lemma:nonterm_sideeffect_partialpop}
 and clearly also .
\end{proof}

\begin{lemma}\label{apx:lemma:sim_fromncom_push_or_nonterm}
Let ,  and .
\begin{enumerate}
	\item If 
	      then 
	      
	\item If , then .
\end{enumerate}
\end{lemma}
\begin{proof}[Claim 1]
Then 

such that  where 
,  such that  and . 
By Lemma \ref{apx:lemma:leftmost_term_simulation_with_comsideeffects} 

Then the proof of Lemma \ref{apx:lemma:sim_fromcom_pop_or_nonterm} Claim 1 applies to give the result.
\end{proof}
\begin{proof}[Claim 2]
Then 

such that , and  where . Hence


so by Lemma \ref{apx:lemma:leftmost_term_simulation_with_comsideeffects} and Lemma \ref{apx:lemma:mseqpar_sim_SF}.

\end{proof}

For  and  define


\begin{proposition}\label{apx:prop:conc_reduction_simulation}
If 
then 
\end{proposition}
\begin{proof}
Let  and define the set
.
further define the set of configurations
.

Now suppose that for some  and 
 
such that  for . Without loss of generality we can assume that for all ,  such that for all 
 we have  and 
 is not involved in any transitions in 
. 
Note that we are not loosing generality, since a reduction 
 can either be pre-empted or goes through a process state in . 
Note this also means that . We further assume w.l.o.g that for each  it is the case that 
 and  
and  
for some  and ,
i.e. during each  only one process makes progress (note this can be achieved by delaying receptions and performing sends and spawns as early as possible) and none of the intermediate steps are configurations of .

We will prove by induction on : 

where for all  and  we have either  or , .
\begin{itemize}
	\item . \newline
	The claim holds trivially.
	\item , assuming the claim holds for . \newline
	To prove the inductive claim we need to show that from
	, 
	and
	
	where ,
	we can infer .
	We will do so by a case analysis on the shape of  and .
	\begin{itemize}
		\item  \newline
		Then , , , 
		 and  where , i.e. either we increase the call-stack or we pop one non-commutative non-terminal off the call-stack. Otherwise we would end up either in an intermediate configuration in  or in a different case.
		\begin{itemize}
			\item Case .\newline
			Then  such that 
				,  and 
				, where  such that 
					
					and
					.
					Lemma \ref{apx:lemma:leftmost_term_simulation} then allows us to conclude that
					 and
				    Lemma \ref{apx:lemma:sim_fromcom_pop_or_nonterm}.1 gives us
				    .
			\item Case .\newline
				Follows directly from Lemma \ref{apx:lemma:sim_fromncom_push_or_nonterm}.1
		\end{itemize}
		\item  and  \newline
		Follows from Lemma \ref{apx:lemma:term_sideeffect_pop}.
		\item  and  \newline
		Follows from Lemma \ref{apx:lemma:sim_fromncom_push_or_nonterm}.2
		\item  and \newline
		We can assume that  since otherwise a case above already applies.
		By the definition of  we can thus infer that  since otherwise 
		 would not be commutative.
		Thus Lemma \ref{apx:lemma:sim_fromcom_pop_or_nonterm}.1 applies.
		\item  and \newline
		There is nothing to prove for this case as, similarly to the case above, either  and so a case above applies or  but then  which is impossible; so the former must be the case.
		\item  and \newline
			If  the above cases apply. Otherwise it must be the case that .
			\begin{itemize}
				\item  \newline
				So it must be the case that
				, ,   and  where 
				, 
				Lemma \ref{apx:lemma:sim_fromcom_pop_or_nonterm}.2 applies to give
				

				\item \newline
				Follows from Lemma \ref{apx:lemma:leftmost_term_simulation_with_comsideeffects}
			\end{itemize}
		\item  and \newline
			Unless  this case is covered by a case above.
			The remaining follows from Lemma \ref{apx:lemma:nonterm_sideeffect_partialpop}.
		\item  and \newline
			Unless  this case is covered by a case above.
			The remaining follows from Lemma \ref{apx:lemma:mseqpar_sim_SF} and Lemma \ref{apx:lemma:leftmost_term_simulation_with_comsideeffects}
	\end{itemize}
	This concludes the proof of the inductive step.
\end{itemize}
Now we apply the above for the case that
 and .
We can then see that .

Then since for all  it is the case that 
 and further for 
, 
 we can deduce from the definition of \toCM on sets of configurations that
 which concludes the proof.
\end{proof}