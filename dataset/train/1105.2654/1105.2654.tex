\documentclass[twoside]{article}
\usepackage{flushend}

\usepackage{graphicx}
\usepackage{subfigure}
\usepackage{rotating}

\usepackage{amsmath}
\usepackage{amssymb}

\usepackage{indentfirst}
\usepackage{url}

\usepackage{subfigure}
\usepackage{multirow}
\usepackage{wrapfig}

\usepackage[ruled,vlined,linesnumbered,commentsnumbered]{algorithm2e}


\newcommand{\so}{}
\newcommand{\GET}{}
\newcommand{\COMMENTS}[1]{\STATE \emph{\##1}}
\newcommand{\FUNCTION}[1]{\textbf{\textsc{#1}}}

\newcommand{\intf}{\mathcal{I}}
\newcommand{\sched}{\mathcal{S}}

\usepackage[utf8x]{inputenc}
\newcommand\textessai{\ensuremath{\essai}}


\usepackage{footmisc}
\renewcommand{\thefootnote}{\fnsymbol{footnote}}

\usepackage[usenames]{color}
\newcommand{\TODO}[1]{\vspace{0.3cm}\textcolor{red}{\textsc{#1}}}\newcommand{\Fabrice}[1]{\textcolor{blue}{FAB: #1}}
\newcommand{\Carina}[1]{\textcolor{red}{CAR: #1}}


\newcommand{\ieee}{{\sc ieee} 802.11\xspace}





\begin{document}


\title{\vspace{-2cm}Broadcast Strategies with Probabilistic Delivery Guarantee in Multi-Channel Multi-Interface Wireless Mesh Networks}


 
  

\author{
Carina Teixeira de Oliveira 
\vspace{0.4cm}\\
{\small Grenoble Informatics Laboratory UMR 5217}
\\
{\small  UJF-Grenoble 1-CNRS}
\\
{\small 681 Rue de la Passerelle, BP72}
\\
{\small 38402 Saint Martin d'Heres, France}
\\
{\small Email: \url{oliveira@imag.fr}}
\and Fabrice Theoleyre
\vspace{0.4cm}\\
{\small CNRS -- LSIIT  -- UMR 7005}
\\
{\small University of Strasbourg}
\\
{\small Boulevard Sebastien Brant}
\\
{\small 67412 Illkirch, France}
\\
{\small Email: \url{theoleyre@unistra.fr}}
\
	\exists i \in Intf(u), \exists j \in Intf(v) \ such\ that\ 
	\sched(i) \cap \sched(j) \neq \emptyset

 \forall v\in V, \; \intf_{v}^D=0\rightarrow \intf_{v}= \intf_{v}^S.

 \forall v\in V, \; \intf_{v}^S=0\rightarrow \intf_{v}= \intf_{v}^D.

 \forall v\in V, \; \intf_{v}^S\geq 1, \; \intf_{v}^D\geq 1.

p_{p} = 1 - (1-p_e)^{size}

	\forall  v \in N(u), p_{cover}(u \rightarrow v) \geq p_{cover_{min}}.
	\label{eq:broadcast_successful}

	p_{cover}(u \rightarrow v) = 1 -  \left(1 - p_{deliv}(u,v)\right)^k
	\label{eq:pcover_kcopies}

	K = \left\lceil \frac{log (1 - p_{cover_{min}})}{log(1 - p_{deliv}(u, v) ) } \right\rceil
	\label{eq:per_common_channel}

	p_{cover} (u,v) = 1 - \left(1 - p_{cover} (u,v)\right) \left(1- p_{deliv}(u,v) \right)
	\label{eq:recursive_pcover}

		JainIndex = \frac{\left(\sum_{c= 1}^{C} B_c\right)^2}{C \cdot \sum_{c = 1}^{C} B_c^2}
	
\end{enumerate}

We denote each strategy as introduced in Section~\ref{section:strategies} and apply the broadcast algorithms defined in the previous section.
In particular, we have implemented the Dynamic/Adaptive strategy in a way
that each interface equally shares its time among all the channels following a pseudo-random sequence \cite{bahl04}. 
Two nodes are able to exchange packets if at least one pair of interfaces uses the same channel at the same instant.


\subsection{Packet Error Rate}

Simulation takes into account packet error probability through the Packet Error
Rate (PER) model represented in Figure~\ref{fig:per}  
\cite{camp06} that shows the relation between the PER and the distance.
As explained above, the neighbors with a PER superior to  have not
to be covered. 
For the numerical results, we have chosen the value of  although different values would lead to consistently the same results.





\begin{figure}
\centering
	\includegraphics[width=0.7\linewidth]{graphes/per/per}
 	\caption{Impact of the distance between the receiver and the transmitter on the Packet Error Rate (PER).}
	\label{fig:per}
\end{figure}





\subsection{Network cardinality}

\begin{figure}[t!]
\begin{center}
	\subfigure[Overhead]{
		\includegraphics[width=0.7\linewidth]{graphes/nb_nodes/overhead}
		\label{fig:nbnodes_overhead}
	}
	\subfigure[Jain Index]{
		\includegraphics[width=0.7\linewidth]{graphes/nb_nodes/jain}
		\label{fig:nbnodes_jain}
	}
	\caption{Impact of the number of nodes, 10 avg. neighbors, 3 interfaces, 12 channels}
\end{center}
\end{figure}



Figure~\ref{fig:nbnodes_overhead} presents the overhead in function of the number
of nodes when maintaining constant density. 
The Static/Common and Mixed/Common \& Adaptive strategies have the same minimal
overhead: no deafness arises and all neighbors receive a broadcast transmission. 
Since some neighbors may present a non-null Packet Error Rate, several non acknowledged broadcasts are required before considering they are \emph{covered}. 



The Dynamic/Adaptive strategy shows a limited overhead by greedily choosing the most suitable channels.
The Static/Pseudo-Random strategy requires a little less broadcast packets (7 transmissions). 
Indeed, this strategy results in lower connectivity: two nodes may be in the radio range with each other, but may not share a common static channel. 
In this case, this kind of a \emph{virtual neighbor} is not anymore a
\emph{neighbor} in the multi-channel topology and has not to be covered. 
This mechanically reduces the overhead.
The probability of such configuration is smaller with dynamic interfaces since
we increase the probability that a pair of nodes has at least one channel in common at a given instant. 


Finally, the Mixed/Pseudo-Random \& Adaptive strategy presents the worst
overhead, because it uses only one static interface, which reduces the possibilities to re-use one single transmission to cover several neighbors. 


We have also evaluated fairness between different channels with the Jain Index (Figure~\ref{fig:nbnodes_jain}).
Mixed Interfaces with Common/Adaptive Channel assignment result in the Jain index of about 0.08.
Indeed, only the control channel (1 of the 12 channels) is used for broadcast leading to high unfairness.

Other strategies lead to almost perfect fairness: they efficiently spread the broadcast traffic through orthogonal channels reducing the risk of congestion.
In particular, the Static/Common strategy does not use a single control channel, leading to a good fairness.







\subsection{Density}

\begin{figure}[t]
\begin{center}
	\includegraphics[width=0.7\linewidth]{graphes/density/overhead}
	\caption{Impact of the density on the overhead, 200 nodes, 3 interfaces, 12 channels}
	\label{fig:density_overhead}
\end{center}
\end{figure}


We have also evaluated the impact of the density on the overhead while
maintaining the number of nodes constant (cf. Figure~\ref{fig:density_overhead}). 
Only Static/Common and Mixed/Common \& Adaptive strategies have the same overhead, which is perfectly
scalable, because they have a common channel set. 

The overhead created by Algorithm~\ref{algo:greedyStatic} applied to the
Static/Pseudo-Random slightly increases with the density: the greedy approach
succeeds to better take advantage of transmissions.
This growth is more important when we use dynamic interfaces as more timeslots are necessary to cover the interface schedule of new neighbors.


The Mixed/Pseudo-Random \& Adaptive strategy keeps on presenting the worst overhead since only one static interface is used for reception limiting the possibilities to use one single packet to cover several neighbors.



In conclusion, our greedy strategies are particularly efficient in minimizing the overhead when the density is large.




\subsection{Number of interfaces}
\begin{figure}
\begin{center}
	\includegraphics[width=0.7\linewidth]{graphes/nb_intf/overhead}
	\caption{Impact of the number of interfaces on the overhead, 200 nodes, 10 avg. neighbors, 12 channels}
	\label{fig:nbint_overhead}
\end{center}
\end{figure}

We have also considered the influence of the number of interfaces (cf. Figure~\ref{fig:nbint_overhead}) on the overhead. 

The Dynamic/Adaptive and the Static/Pseudo-Random  strategies tend to have initially a growing overhead: the number of neighbors to be covered increases since they have more chance to have a common timeslot.
Then, the overhead decreases when it exceeds a threshold since the probability of having different neighbors that use the same channel increases with the number of interfaces.
The Dynamic/Adaptive begins to be more attractive when the number of interfaces is large compared to the number of channels ( interfaces).
Finally, for a very large number of interfaces (), these strategies tend to be similar to the common channel strategies. 

The strategies using a common channel for broadcast are not impacted by the number of interfaces. 
Besides, the Mixed/Pseudo-Random \& Adaptive strategy presents also a constant overhead since the unique receiving interface keeps on being the bottleneck. 




\subsection{Impact of threshold }
\begin{figure}
\begin{center}
	\includegraphics[width=0.7\linewidth]{graphes/reliability/overhead}
	\caption{Impact of threshold   on the overhead, 200 nodes, 10 avg. neighbors, 3 interfaces, 12 channels}
	\label{fig:reliability_overhead}
\end{center}
\end{figure}


Finally, we have measured the impact of threshold  on the overhead in Figure~\ref{fig:reliability_overhead}.
When , each neighbor is covered when the transmitter sends one single copy: we discard radio links with a larger PER, considering them unreliable. 
Thus, strategies with a common channel set required only one broadcast transmissions to cover all the neighbors.

When  increases, the overhead becomes larger: neighbors with a
large packet error probability may require the transmission of several copies.
However, we can note that all the strategies follow the same trend.  
The overhead becomes prohibitive when we require very large  (e.g. ).
Thus, the network protocols have to cope with broadcast unreliability.
In particular, they should work in a self-stabilizing manner: even if some neighbors do not receive a particular broadcast packet, the protocol must work properly. 







\section{Conclusion and Future Work}
\label{section:conclusion}


We have proposed algorithms for local broadcast in multi-channel
multi-interface wireless mesh networks.
In particular, they can cope with dynamic interfaces without a common control channel. 
To the best of our knowledge, these algorithms are the first ones to cope with deafness in this situation. 
Simulations show that all the strategies have an acceptable overhead and the load is fairly distributed among channels when the Common Channel Set strategy is not used.
A greedy approach is particularly efficient to take advantage of the broadcast
nature of transmissions.  

We plan to study how we can deal with multiple rates: different bit rates may
cover a different set of neighbors with different PER.  
We also plan to adapt the proposed strategies to dynamic conditions adopting an opportunistic approach.
Besides, we aim at optimizing the delay, e.g. consider the question of which
timeslot would present the best trade-off between the delay and the overhead
when we use dynamic interfaces. 



\section*{Acknowledgments}

This work was supported in part by the French Government (MESR) and the Competitive Clusters Minalogic and System@tic under contracts FUI SensCity as well as the French National Research Agency (ANR) project ARESA2 ANR-09-VERS-017. 



\bibliographystyle{unsrt}

\bibliography{biblio}


\end{document}
