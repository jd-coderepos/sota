\documentclass[11pt]{article}


\usepackage{verbatim}
\usepackage{amsmath}
\usepackage{amssymb}
\usepackage{amsthm}
\usepackage{yhmath}
\usepackage{graphicx}
\usepackage{subfigure}
\usepackage{epstopdf}
\usepackage[usenames,dvipsnames]{color}
\usepackage[margin=1in]{geometry}

    \newtheorem{theorem}{Theorem}[section]
    \newtheorem{lemma}[theorem]{Lemma}
    \newtheorem{definition}[theorem]{Definition}
    \newtheorem{proposition}[theorem]{Proposition}
    \newtheorem{corollary}[theorem]{Corollary}
    


\newcommand{\diag}{\mathop{\mathrm{diag}}}


\title{Duals of Orphan-Free Anisotropic Voronoi Diagrams are Triangulations}



\author{Guillermo D. Canas,   Steven J. Gortler \\
\small   - CBCL, McGovern Institute,  
Massachusetts Institute of Technology\\
\small   - School of Engineering and Applied Sciences, 
Harvard University\\
\small Email:  \tt guilledc@mit.edu, sjg@seas.harvard.edu}


\date{}


\begin{document}

\maketitle

\begin{abstract}
Given an anisotropic Voronoi diagram, we address the fundamental question of when its dual is embedded. 
We show that, by requiring only that the primal be orphan-free (have connected Voronoi regions), 
its dual is always guaranteed to be an embedded triangulation. 
Further, the primal  diagram and its dual have properties that parallel those of ordinary Voronoi diagrams:
the primal's vertices, edges, and faces are connected, and the dual triangulation has a simple, closed boundary. 
Additionally, if the underlying metric has bounded anisotropy (ratio of eigenvalues), the dual is guaranteed to triangulate the convex hull of the sites.   
These results apply to the duals of anisotropic Voronoi diagrams of \emph{any}
set of sites, so long as their Voronoi diagram is orphan-free. 
By combining this general result with existing conditions for obtaining orphan-free anisotropic Voronoi diagrams, 
a simple and natural condition for a set of sites to form an embedded anisotropic Delaunay triangulation follows. 
\end{abstract}

\newpage

\section{Introduction}

Voronoi diagrams and their dual Delaunay triangulations are fundamental constructions with
numerous associated guarantees, and extensive application in
practice~\cite{Delaunay1934,Edelsbrunner}. 
At their heart is the use of a distance between points, which in the original
version is taken to be Euclidean. 
This suggests that, by considering generalizations of the Euclidean
distance, it may be possible to obtain variants which can be well-suited to
a wider range of applications.  

Attempts in this direction have been met with some success. 
Power diagrams~\cite{power} generalize Euclidean distance by associating a {bias-term} to each site. The duals of these diagrams
are guaranteed to be embedded triangulations, in any number of
dimensions. 
Although this is a strict generalization of Euclidean distance, it is a somewhat 
limited one. The effect of the bias term is to locally enlarge or shrink the
region associated to each site, loosely-speaking ``equally in every
direction". It allows some freedom in choosing local scale, with no
preference for specific directions. 


Another way to generalize Voronoi diagrams is to endow Euclidean space with a
continuously varying Riemannian metric, and use its associated (geodesic) distance 
in the definition of the Voronoi diagram. 
This generalization carries a significant
amount of freedom, allowing for local scale and ``directionality" to be freely
specified at each point. 


Despite being of potentially great interest, this latter approach has faced several
obstacles. 
The dual of such an anisotropic Voronoi diagram is, in general, 
\emph{not} an embedded triangulation, even if the primal is orphan-free (the Voronoi regions are connected), 
and may produce element inversions and edge crossings. 
An additional obstacle is that the geodesic distance is extremely expensive to compute in practice, rendering traditional methods
for constructing Voronoi diagrams impractical. 

Two notable attempts to resolve these problems have been proposed. 
Independently, Labelle and Shewchuk~\cite{LS}, and Du and Wang~\cite{DW} propose
two efficient approximations of geodesic distance from a site to any point in the
domain. 
Although their associated Voronoi diagrams are, in general, no longer orphan-free, 
Labelle and Shewchuk show that a set of sites exists with an orphan-free diagram, whose dual is embedded, in two dimensions. They accomplish this by proposing an iterative site-insertion algorithm
that, for any given metric, constructs one such set of sites. 
Some recent work making use of the above definitions are~\cite{Boissonnat:2008:LUA:1377676.1377724,BOISSONNAT-2007-488446}.


In this paper, we show that if a set of sites produces an orphan-free 
anisotropic Voronoi diagram, using the definition of~\cite{DW}, 
then its dual is always an embedded triangulation (or a polygonal mesh with convex faces in general), in two dimensions (Thm.~\ref{th:final}). 
This effectively states that, regardless of the sites' positions, if the primal is well-behaved, then the
dual is as well. 
Further, in a way that parallels the ordinary Delaunay case, the dual has no
degenerate elements (Lem.~\ref{lem:degen}), its elements (vertices, edges, faces) are unique (Cor.~\ref{uniqueVD}), and, under mild assumptions on the metric, is guaranteed to triangulate the convex hull of the sites (Thm.~\ref{thm:boundary}). 
Note that, while~\cite{LS} prove a property of the output of an algorithm, the results in this paper are fundamental properties of the diagrams themselves 
(independent of how the orphan-freedom of the primal was obtained).

In practice, we may combine our results with those of~\cite{avd}  
to conclude that duals of anisotropic Voronoi diagrams of appropriate -nets 
are always embedded triangulations (Cor.~\ref{cor:enet}). 
This may be particularly useful in (asymptotically-optimal) function approximation applications, 
where we are often interested in 
constructing the anisotropic Delaunay triangulations of appropriate -nets~\cite{enets,GruberOQ}, and was the initial motivation for the current work. 
Finally, we note that, as discussed in Sec.~\ref{sec:implementation}, in practice, 
algorithms for constructing anisotropic diagrams of the type discussed here may have a more numerical flavor  
(i.e.\  front-propagation, fast marching methods), in contrast with the more combinatorial nature of other Voronoi diagrams. 




\section{Setup}\label{sec:setup}


Given a finite set  of sites on the plane, 
a Voronoi diagram decomposes Euclidean space into regions, each distinguished by the site
its points are closest to. 
The notion of closeness is defined as follows:
consider a continuous metric (in coordinates: , 
symmetric, positive definite) 
over two-dimensional Euclidean space and,
following~\cite{DW}, define the (asymmetric) ``distance" between a site
() and point () as



The Voronoi region of site  is the set of points no further from  than
from any other site:

Note that, because  is a set, and not a multiset, there are no coincident sites. 


This definition of anisotropic Voronoi diagram results in diagrams that take the usual shape, but
whose regions may have curved boundaries, and tend to be elongated along
certain directions, depending on the metric . 



Following the notation of~\cite{LS}, we say that a diagram is
\emph{orphan-free} if its regions  are connected, for all 
(equivalently: each \emph{cell}, or connected component of a Voronoi region, 
{contains} its generating site).



An important distinction in this construction is that  the ``interfaces" 
in-between Voronoi regions do not, in general, have null measure 
(and points equidistant to three or more sites are not always isolated), 
unlike geodesic-distance diagrams~\cite{LL2000} 
(see~\cite{enets} Lemma 5.2), and those of~\cite{LS}. 

Note that, given one such degenerate diagram, any interior point  of a Voronoi edge

of non-zero measure 
has an open neighborhood . 
By adding to  an arbitrarily
small perturbation supported on , an orphan is created.
That is, with respect to the metric, any degenerate diagram is
arbitrarily close to a diagram that has orphans, and therefore the
non-degeneracy requirement is only slightly more restrictive than the
orphan-freedom one. 



We begin with three simple lemmas, proved in Appendix A, which provide an introduction to some of the techniques used in the sequel. 
The first is a simple extension of Lemma 2.1 of~\cite{DW}, and the second follows directly from the continuity of  and the fact that every site is {strictly} closer to itself than to all other sites. 

\begin{lemma}\label{lem:midpoint}
	Given two sites , , the only point equidistant to  in their supporting
line is the midpoint . 
\end{lemma}


\begin{lemma}\label{lem:interior}
    Every site is an interior point of its corresponding Voronoi region. 
\end{lemma}


\begin{lemma}\label{lem:sc}
Every Voronoi region of an orphan-free anisotropic Voronoi diagram in 
 is simply connected. 
\end{lemma}



\noindent{\bf Primal diagram}. 
The Voronoi diagram is a collection of Voronoi regions , one for every site in , as well as a structure 
induced by the sets of points closest and equidistant to two or more sites. 
Let  to be the primal Voronoi diagram embedded on the sphere
 by stereographically projecting the plane onto the 
punctured sphere and
completing it with a ``point at infinity" .
 includes not just regions, but also embedded edges ,
which are \emph{connected} sets of points closest and equidistant to two sites , as well
as vertices (points closest and equidistant to three or more sites). 
Since we are assuming that sets of  equidistant points to two sites have
null measure, and that
points equidistant to three or more sites are isolated, 
we can consider the abstract mesh   with structure
derived from 
, where  are the vertices, edges, and faces of . 



A simple induction argument reveals that every edge in  connects two
vertices in . 
To see this, start with an orphan-free diagram with two sites. The only edge
of  is the unbounded set of points equidistant to the sites, passing through
their midpoint (Lem.~\ref{lem:midpoint}), and with endpoints at . 
The insertion of an additional site can only split existing edges of
 at points
where two edges cross (a vertex by definition), and thus the
original edge is split into pieces whose endpoints are now either  or 
an edge crossing (a vertex). 



Clearly,  is a planar embedded diagram since two edges cannot meet except at vertices:  
any point  where edge  crosses 
edge  is equidistant to three sites, , and thus a vertex. 
Therefore  is planar. \\









\noindent{\bf Dual of an orphan-free Voronoi diagram}.
The dual of the primal  mesh structure  is another abstract 
mesh , with one vertex per site, 
and edges connecting adjacent regions of . Its face structure is derived from that of  by duality   and, in particular,  is planar, since  is planar. 
Using the embedded primal , we can define an embedding  of  
on the plane, with vertices coinciding with sites (and curved edges). 
Although  was defined on the sphere, we embed  on
the plane instead, and therefore we exclude (from both  and ) the unbounded face dual to the point at infinity . 

Since  is planar, 
 derives the following properties from the primal:
\begin{itemize}
\item[(i)]Every edge in  is incident to two dual faces, 
except for \emph{boundary edges} (whose corresponding primal edge in
 connects unbounded regions), which are only incident to one dual face. 
\item[(ii)]To every dual face in , connecting vertices , corresponds a
primal vertex  of , 
which is equidistant to the corresponding sites . 
\end{itemize}
The property (ii) requires that the Voronoi diagram be orphan-free, as noted in the remark after Thm.~\ref{th:ece}. 




Notice that either  or  could be multigraphs (e.g.\  would be a multigraph if an edge  of  has multiple connected components). 
However, the following lemma ensures that both are simple graphs. 


\begin{lemma}\label{lem:simple}
 and  are simple (have no multi-edges or self-loops). \end{lemma}
\begin{proof}
See Appendix D. 
\end{proof}

By its definition, an edge of  is a connected set of points equidistant
and closest to two sites . Since  has no multi-edges, this implies the following:


\begin{corollary}\label{cor:Pij}
The set of points closest and equidistant to two sites of an orphan-free diagram is connected. 
\end{corollary}





Property (ii) above can be restated in a more useful way that parallels an equivalent property of duals of ordinary Voronoi diagrams. 


\begin{definition}[{\bf Empty circum-ellipse property}]\label{def:ece}
	A face of  incident to vertices  satisfies the empty circum-ellipse (ECE) property if there is
\emph{some} ellipse that circumscribes the sites corresponding to ,
and contains no site  in its interior. 
\end{definition}

Note that \emph{any} empty circum-ellipse serves as witness to the ECE property. 
We show that the ellipse centered at a Voronoi vertex , with axis given by , is always an empty circum-ellipse of its dual polygon.

\begin{theorem}\label{th:ece}
	Every face of  satisfies the empty circum-ellipse property. 
\end{theorem}
\begin{proof}
The proof is constructive. 
To every face in  connecting vertices with corresponding sites , 
by property (ii), corresponds a Voronoi vertex
 of  that is equidistant to . 
If ,
then the open ellipse 

circumscribes the  and does not contain any site (otherwise,  would be closer to that site than to , 
and therefore not in , a contradiction). 
\end{proof}



\begin{figure}[ht]
\centering
\subfigure[]{
\includegraphics[height=3.5cm]{ECE.eps}
\label{fig:f4a}
}
\subfigure[]{
\includegraphics[height=3.5cm]{ECEb.eps}
\label{fig:f4b}
}
\label{fig:f4}
\caption{In an orphan-free diagram (a), every face  satisfies the
empty-circum ellipse (ECE) property. If the diagram is not orphan-free, then
some faces may not satisfy the ECE property.}
\end{figure}



Although this may not at first seem apparent,
the above property uses the fact that the primal diagram is orphan free. 
The critical fact in the proof of Thm.~\ref{th:ece} is not that
there is a Voronoi vertex  equidistant to , but rather that 
it is . 
Consider Figs.~\ref{fig:f4a} and~\ref{fig:f4b}, in which sites  form a face of . 
In~\ref{fig:f4a}, the diagram is orphan-free, and the fact that  implies Thm.~\ref{th:ece}. 
Figure~\ref{fig:f4b} shows part of a Voronoi diagram that has an orphan cell (corresponding
to site ) ``covering" . 
If the dual is constructed such that the face  is not in the dual  
then a ``hole" in the triangulation may occur. 
If the face  is in the dual, 
then, 
despite the fact that  is equidistant
to , this  is no longer in , but in the Voronoi region
of . 
Since  is closer to  than to , 
any open ellipse centered at , and having
 on its boundary, contains .
Therefore, in Fig.~\ref{fig:f4b}, the ECE property does not hold for the dual
face of .  
Ensuring that Thm.~\ref{th:ece} holds is one of the main reasons for requiring orphan-freedom. 






\section{Summary of Results and Outline}

The two main results in this paper assume that we are given an orphan-free anisotropic Voronoi diagram (in two dimensions) and 
prove that: 1) \emph{the dual is an embedded triangulation} (Thm.~\ref{th:final}), 
and 2) \emph{if the metric has bounded anisotropy (ratio of eigenvalues), the dual triangulates the convex hull of the sites} (Thm.~\ref{gamma}).

We prove a number of additional results. Possibly the most important shows that the elements of an orphan-free diagram (vertices, edges, faces), 
are connected sets, and therefore, in some sense, unique (Cor.~\ref{uniqueVD}). 
This is important, since this is a natural well-behave-ness condition on the primal, which, together 
with the above results, indicates that orphan-freedom is sufficient to guarantee well-behave-ness of both primal and dual. 

A smaller result of interest is: the dual faces satisfy an \emph{empty circum-ellipse} property (Thm.~\ref{th:ece}), which parallels the empty-circumcircle property of ordinary Voronoi diagrams, 
and could have further practical implications from the ones described here. 

The order of the proofs is, however, different from the one just stated. In effect, we begin by proving the most restrictive case: that the dual of an orphan-free diagram with metric of bounded anisotropy is embedded (Sec.~\ref{sec:interior}), 
and that its boundary is the boundary of the convex hull of the sites (Sec.~\ref{sec:boundary}). We then relax the bounded anisotropy condition (Sec.~\ref{sec:gen}), and show that, in the general case, we loose the convex hull property, but the dual remains embedded. 
Finally, once these results are established, we prove that vertices of the primal are unique (uniqueness of edges was proven in Sec.~\ref{sec:setup}, and uniqueness of primal faces is the same as orphan-freedom). 

We assume in the remainder of the paper that there are more than two sites, not all of which are colinear, and relegate the (considerably simpler) colinear case to Appendix B. 











\section{Dual of Orphan-free Diagram (Part I: boundary)}\label{sec:boundary}

In this section, we assume that the metric has bounded anisotropy, and conclude that the boundary of the dual of an orphan-free diagram is the same as the boundary of the convex hull of the sites (and in particular is simple and closed). 
If  is the eigendecomposition of  at , with , , and  orthonormal, then we assume that there is some bound  on the anisotropy of , such that:  for . 
Note that this condition may commonly hold in practice, if the metric is sampled on a compact domain (and possibly extended to the plane by reusing sampled values only). 



We begin by defining  to be the straight-edge drawing of  with vertices at the sites. For the moment, we assume that every Voronoi vertex is equidistant to no more than three sites, and therefore that all faces in  are triangles 
(we extend the results to the general case in Appendix G). We associate to  a mapping (from the vertices, edges, and faces
of  to ) for which, because all faces of  are triangles, it is well defined whether a point in
 belongs to any given face of , a fact that will be
used in the proofs of Sec.~\ref{sec:interior}. 
This mapping will be shown in Sec.~\ref{sec:interior} to be an embedding. 
In the sequel, it is assumed that  encompasses both the mesh
structure, and the mapping. 










The boundary vertices of  are those whose corresponding primal regions in  are unbounded, 
while boundary edges of  connect boundary vertices. 
Note that this is a \emph{topological} property of , rather than a geometric one (boundary elements of  may, in principle, not lie in the boundary of the convex hull of the sites). 
For convenience, we call  the set of boundary edges of . 




The convex hull  of  is the minimal (w.r.t.\ set
containment) convex set that contains . 
For convenience, we name  the sites that are part of the boundary of
the convex hull , and order them in clock-wise order around . 
The boundary  of the convex hull is a simple circular chain 
. We prove that it is  
(loosely speaking: the topological boundary of , and the geometric boundary of its straight-edge embedding , are the same), 
which implies that  covers
the convex hull of the sites, and its boundary edges form 
a simple, closed polygonal chain. All the proofs of this section are in Appendix E. 



















\begin{lemma}\label{boundary_easy}
To every boundary edge  of  corresponds a segment in the boundary of . \emph{}
\end{lemma}





We now turn to the converse claim: that to every segment 
 corresponds a boundary edge
 in . Since  is the set of boundary edges of , whose primal edges in
 are
unbounded, the claim is equivalent to proving that, to every segment
 in the boundary of the convex hull corresponds an edge of the
dual , whose primal edge  is unbounded. 


The proof proceeds as follows. First, assume w.l.o.g.\  that the origin is in the interior of . 
Let   be a sufficiently large origin-centered circle. 
We define two  functions: 

 simply projects every point in  to its closest in  (with respect to the distance ), 
and  projects a point back to . 


\begin{lemma}\label{lem:piC0}
 is a continuous function in . 
\end{lemma}
\begin{proof}
We first prove by contradiction that  is unique. 
Let  be the unique~\cite{avd}, symmetric positive definite square-root matrix such that . 
Consider distinct
 closest to . 
Since , by the convexity of the Euclidean norm, and the positive-definiteness of , it is

Thus  is closer to  than to
, a contradiction. Therefore the closest point  to  is
unique. 

If  were not continuous at , then there is  such that 
for all  there is a point  such that 
and . 
Consider the sequence  of points in
. Because  is compact,  has a
subsequence that converges to some . 
By continuity of  (which follows from the continuity of ), it is

and therefore the closest point in  to  is not unique, a
contradiction. 
\end{proof}




Clearly,  is continuous in . 
Note that, because  contains the origin, then, as shown in Fig.~\ref{fig:pinu},  projects 
every point  on a segment of , 
\emph{outwards} from the convex hull (and on the empty side of
); that is, so that 
( is in the empty half-space of ). 



\begin{figure}[htbp]
   \centering
   	\includegraphics[width=2.7in]{pinu.eps} \caption{The construction for the proof of Thm.~\ref{thm:boundary}.}
\label{fig:pinu}
\end{figure}



Conveniently, Lem.~\ref{lem:mij} shows that, if , then
, and so the claim now reduces to showing that for each
segment  of , and for \emph{every} sufficiently large circle , 
there is  with . Since this implies that  is
unbounded, it means that the corresponding edge  is in . 

The proof is by contradiction. 
Lem.~\ref{lem:Sn} uses Brouwer's fixed point theorem to show that, for every segment  of , 
if there were no  with , then 
the function  must have a point  such that
, or equivalently such that 
 is ``behind" the segment  to which it is
closest (). 
On the other hand, Lem.~\ref{lem:contrad} shows that, for all sufficiently large circles , no
point  can be closest to some segment
 that it is \emph{behind} of, creating a contradiction. 






As mentioned above, we are interested in identifying points in 
that are also in some primal edge . The following lemma will be used
to show that, for sufficiently large , such a point  can be
characterized by satisfying . 



\begin{lemma}\label{lem:mij}
	If , with ,  , then
. 
\end{lemma}












The following technical lemma is the key in constructing a contradiction by showing that, 
for sufficiently large circles , no
point  can be closest to some segment
 that it is behind of (); 
where, as before, the open half space  is chosen to be 
the only of the two half spaces on either side of the supporting line  of  
such that . 





\begin{lemma}\label{lem:contrad}
	There is  such that, for any segment , with supporting line , every  with 
    whose closest point in  is  is 
	closer to a site in  than to . 
\end{lemma}







\begin{lemma}\label{lem:Sn}
	Every continuous function  that is not onto has a fixed point. 
\end{lemma}




\begin{lemma}\label{lem:hard}
 To every segment  in the boundary of
 corresponds a boundary edge of . \emph{}
\end{lemma}

Therefore,

\begin{theorem}\label{thm:boundary}
If the metric  has bounded ratio of eigenvalues, then the boundary edges of  are the same as the boundary edges of the convex hull of .  \emph{} 
\end{theorem}

\begin{corollary}\label{col:simple-boundary}
If the metric  has bounded ratio of eigenvalues, then the boundary edges of  form a simple, closed, convex polygonal chain. \end{corollary}


The ECE property is sufficient to show that no face in  is degenerate, a fact that will be used in the following section. 

\begin{lemma}\label{lem:degen}
	 has no degenerate (null area) elements. 
\end{lemma}




In Thm.~\ref{th:main}, we show that, even if Voronoi vertices are incident to more than three Voronoi regions (equidistant to more than three sites), every face in  is a (strictly) convex polygon, and therefore can be trivially triangulated (say in a fan arrangement). 
The resulting triangulation has no degenerate elements, and all its triangles satisfy the ECE condition (using the same witness ellipsoid as the convex polygon from which they are triangulated). 




\section{Dual of Orphan-free Diagram (Part II: interior)}\label{sec:interior}


In this section, we assume that the (topological) boundary of  is simple and closed, and prove that  must be embedded.
The main argument in the proof 
uses  Theorems~\ref{th:ece} and~\ref{thm:boundary}, Cor.~\ref{col:simple-boundary}, as well as 
the theory of discrete one-forms on graphs, 
to show that there are no
``edge foldovers" in  (edges whose two incident faces are on the same side of its supporting line), 
and use this to conclude that  is embedded (Thm.~\ref{th:main}). 
As in Sec.~\ref{sec:boundary}, we assume that not all sites are colinear 
(the simpler colinear case was addressed in Appendix B). We distinguish between the sites  that lie on the boundary of the convex hull, and the remaining, or
\emph{interior sites} (). 



The following definition, from~\cite{1form}, 
assumes that, for each edge
 in , we distinguish the two opposing half-edges
 and . 


\begin{definition}[Gortler et al.\ \cite{1form}]\label{def:1form}
A non-vanishing (discrete) one-form   is an assignment of a real value
 to each half edge  in , such that 
. 
\end{definition}



Since  has the same structure as , we can construct a non-vanishing one-form
over  as follows. 
Given some unit direction vector 
(in coordinates ), 
we assign a real
value  to each vertex  in , and define 
, which clearly satisfies 
. The one-form, denoted by , 
is non-vanishing if, for all edges , 
it is . 
That is, if the direction  is not orthogonal to any edge. 
The set of edges has cardinality , and in particular
it is finite. Therefore \emph{almost all} directions  generate a non-vanishing one-form . 






Since  is an planar graph with a well-defined face structure,
there is, for each face , a cyclically ordered set
 of half-edges round the face. 
Likewise, for each vertex , the set  of cyclically ordered
(oriented) half-edges emanating from each vertex is well-defined. 


\begin{definition}[Gortler et al.\ \cite{1form}]
Given non-vanishing one-form  corresponding to , 
the index of vertex  with respect to  is , where  is the
number of sign changes of  as one visits the half-edges of  in order. \\
The index of face  is  where  is the number
of sign changes of  as one visits the half-edges of  in
order. 
\end{definition}

Note that, by definition, it is always . 
A discrete analog of the Poincar\'e-Hopf index theorem relates 
the two indices above:

\begin{theorem}[Gortler et al.\ \cite{1form}]\label{lem:ph}
For any non-vanishing one-form , it is 

\end{theorem}

Note that this follows from Theorem 3.5 of~\cite{1form} because the unbounded, 
outside face, which is not in , is assumed in this section to be closed and simple, and therefore would have null index. Note that the machinery from~\cite{1form} to deal with degenerate cases
isn't needed here because vertices, by definition, cannot coincide ( is not a multiset). 
All  proofs in this section, except for that of Thm.~\ref{th:main}, are  in Appendix F. 





The one-forms defined above satisfy the following property. 

\begin{lemma}\label{lem:non-negative}
	Given a non-vanishing , the sum of indices of interior vertices () of  is non-negative. 
\end{lemma}




The next two lemmas relate the presence of edge foldovers and 
the ECE property of Definition~\ref{def:ece}  to the indices of vertices in . 





\begin{lemma}\label{lem:index-1}
If  has an edge foldover, then there is a non-vanishing one-form  such
that  for some interior vertex . \end{lemma}


\begin{lemma}\label{lem:index1}
Given  and non-vanishing one-form , if  has an interior vertex  with index
, then there is a face  of
 that does not satisfy the empty circum-ellipse property. 
\end{lemma}

The above provides the necessary tools to prove the following key lemma. 


\begin{lemma}\label{lem:ef}
 has no edge foldovers. 
\end{lemma}



Finally, the absence of edge foldovers, together with a simple and closed boundary, 
is sufficient to show that  is 
embedded.





\begin{lemma}\label{lem:main-weak}
If its (topological) boundary is simple and closed, then the straight-line dual  of an orphan-free diagram, with vertices incident to at most three sites, is an embedded triangulation. 
\end{lemma}

As shown in Appendix G, even if there are non-generic Voronoi vertices that are incident to more than three sites, the dual is composed of faces each of which is convex and satisfies the ECE condition (Def.~\ref{def:ece}). Each convex face can be triangulated (e.g.\ in a fan arrangement) in such a way that individual triangles satisfy the ECE condition with the same witness ellipse as the face from which they are derived. This leads to the following:


\begin{theorem}\label{th:main}
If its (topological) boundary is simple and closed, then the straight-line dual  of an orphan-free diagram is an  embedded polygonal mesh with convex faces. 
\end{theorem}



\section{Final Results}\label{sec:gen}


We can combine Corollary~\ref{col:simple-boundary} and Theorem~\ref{th:main} into
\begin{theorem}\label{gamma}
If the metric  has bounded ratio of eigenvalues, then the dual of an orphan-free Voronoi diagram with respect to  is an embedded polygonal mesh with convex faces, and covers the convex hull of the sites. 
\end{theorem}

This result can be generalized, dropping the bounded anisotropy condition on , but at the cost of losing the convex hull property, as shown in Appendix H:
\begin{theorem}\label{th:final}
The dual of an orphan-free Voronoi diagram is an embedded polygonal mesh with convex faces. 
\end{theorem}

From the above results, Corollary~\ref{cor:Pij}, and the definition of the dual, it follows (see Appendix H) that


\begin{corollary}\label{uniqueVD}
An orphan-free anisotropic Voronoi diagram is composed of unique (connected) vertices, edges, and faces.
\end{corollary}





Finally, we note that Thm.~\ref{th:main} can be combined with existing conditions for orphan-freedom~\cite{avd}, resulting in a simple and natural condition for a set of sites to induce an embedded polygonal mesh as the dual of their anisotropic Voronoi diagram:
\begin{corollary}\label{cor:enet}
	If  is an asymmetric -net w.r.t.\ ,  a continuous metric with metric variation , 
		and , then the dual of the anisotropic Voronoi diagram of 
		is an embedded polygonal mesh with convex faces.\end{corollary}
where an asymmetric -net is simply a weaker form of -net defined on non-symmetric functions , which can be computed with the iterative algorithm of~\cite{Gonz}, and the metric variation  is a Lipschitz-type condition  on ~\cite{avd}. 
The above condition is known to be conservative, and there may be simpler conditions to achieve orphan-freedom. As a practical observation, Du and Wang~\cite{DW} report orphans to be a rare occurrence in their experiments. 







\section{Proof-of-concept Implementation}\label{sec:implementation}

\begin{figure}[ht]
\centering
\subfigure[]{
\includegraphics[height=2.6cm]{c07.png}
\label{fig:img_a}
}
\subfigure[]{
\includegraphics[height=2.6cm]{c08.png}
\label{fig:img_b}
}\subfigure[]{
\includegraphics[height=2.6cm]{c03.png}
\label{fig:img_c}
}
\subfigure[]{
\includegraphics[height=2.6cm]{c04.png}
\label{fig:img_d}
}\quad
\subfigure[]{
\includegraphics[height=2.6cm]{c00.png}
\label{fig:img_e}
}
\subfigure[]{
\includegraphics[height=2.6cm]{c01.png}
\label{fig:img_f}
}\subfigure[]{
\includegraphics[height=2.6cm]{c05.png}
\label{fig:img_g}
}
\subfigure[]{
\includegraphics[height=2.6cm]{c06.png}
\label{fig:img_h}
}
\caption{
Anisotropic Voronoi diagrams, and their duals generated by our
proof-of-concept implementation. 
Voronoi vertices are marked in red, while dual vertices (sites) and edges are drawn
in black.}
\end{figure}



Though not aiming for an efficient implementation, 
we implemented a simple proof-of-concept that constructs anisotropic Voronoi diagrams
like the ones considered in this paper, and their duals
(Fig. 3). 
A closed-form metric, which has bounded ratio of eigenvalues, is discretized on a
fine regular grid, and linearly interpolated inside grid elements, resulting in a
continuous metric. The sites are generated randomly (Figs.~\ref{fig:img_a}
and~\ref{fig:img_b}), or using a combination of random, and equispaced
points forming an asymmetric -net (remaining figures). 

The primal diagram was obtained using front propagation from the sites
outwards, until fronts meet at Voronoi edges. 
The runtime is proportional to the grid size, since every grid-vertex is visited exactly six times (equal to their valence). 
The implementation does not guarantee the correctness of the diagram unless it \emph{is} orphan-free, and serves to verify the claims of the paper since well-behave-ness of the dual is predicated on that of the primal. 


The two main claims of the paper are clearly illustrated in these examples. 
In all examples, the dual covers the convex hull of the vertices
(Thm.~\ref{thm:boundary}), is a
single cover, embedded with straight edges without edge crossings
(Thm.~\ref{th:main}), 
and has no degenerate faces (Lem.~\ref{lem:degen}). 
By focusing on the primal diagrams (second and fourth column), further claims in
the paper become apparent, namely that Voronoi regions are simply connected (Lem.~\ref{lem:sc}), 
Voronoi vertices are unique (Cor.~\ref{uniqueVD}), 
Voronoi edges are connected (Cor.~\ref{cor:Pij}), 
unbounded regions correspond to boundary dual vertices, and unbounded
edges of the Voronoi diagram correspond to boundary dual edges.



\section{Conclusion and Open Questions}

We studied the properties of duals of orphan-free anisotropic Voronoi diagrams, for the
purposes of constructing triangulations on the plane. 
The main result (Theorems~\ref{th:main} and~\ref{thm:boundary}, Cor.~\ref{cor:enet}) is that
the dual, with straight edges and having the sites as vertices, is embedded
and covers the convex hull of the sites, mirroring similar results for
ordinary Voronoi diagrams and their duals. 

A few, somewhat less important  properties are proven, including the fact
that every primal region is simply connected, that elements of the primal are unique (Cor.~\ref{uniqueVD}, 
but perhaps most significantly
that every face in the dual satisfies an \emph{empty circum-ellipse}
property that has a direct parallel in the empty circum-circle property of
ordinary diagrams, and is the basis for proving that it is embedded with
straight edges. 

Perhaps the most important outstanding question may be whether these ideas
extend to higher dimensions. The results in Secs.~\ref{sec:boundary} and~\ref{sec:interior},
except for Lem.~\ref{lem:sc}, can be trivially extended to n dimensions. 
Sec.~\ref{sec:boundary} has been written only for the two-dimensional case,
but a similar construction, and the same argument would work in higher
dimensions (Lem.~\ref{lem:Sn} being a hint of this). 
It is the argument in Sec.~\ref{sec:interior} that becomes problematic. 
While the ECE property is shown to be sufficient to prevent foldovers in the
triangulation, it is not sufficient in 
higher dimensions. In particular, fixing the boundary to be simple and convex, 
there are simple arrangements of tetrahedra in  that contain 
face foldovers but do not break the ECE property. 
We plan to study these question next. 













\newpage
\bibliographystyle{plain}
\bibliography{vddw3}










\newpage
\section*{Appendix A}\label{app:prelim}

\begin{figure}[ht]
\centering
\subfigure[]{
\includegraphics[height=2.5cm]{f2a.png}
\label{fig:sca}
}
\subfigure[]{
\includegraphics[height=2.5cm]{f2b.png}
\label{fig:scb}
}
\label{fig:sc}
\caption{Diagram for the proof of Lem.~\ref{lem:sc}.}
\end{figure}


\noindent{\bf Lemma~\ref{lem:midpoint}}
\emph{Given two sites , , the only point equidistant to  in their supporting line is the midpoint . }
\begin{proof}
If  is in the supporting line of , then , . 
	 is equidistant to  (i.e.\ ) if and only if
		.

	Since  is positive definite and , it is , 
	and so the above equality holds iff  ( is the midpoint).
\end{proof}





\noindent {\bf Lemma~\ref{lem:sc}}
\emph{Every Voronoi region of an orphan-free anisotropic Voronoi diagram in 
 is simply connected. 
}
\begin{proof}	
A multiply connected region  is path connected, and is
such that there is a map
 that cannot be continuously contracted to a point. We
can assume  injective (simple), since a non-injective  can always be broken up
into injective pieces, at least one of which must be such that it cannot be
continuously contracted to a point (or else  would be). 
By the Jordan curve theorem,  encloses a bounded set . Since 
cannot be continuously contracted to a point, there must be
 with . Since  is a subset of , 
and  is bounded,  is bounded. 





 is part of the Voronoi diagram, so it must be composed of one or more
Voronoi cells. Because the diagram is orphan-free, each cell in the Voronoi
diagram contains its generating site: 
(Fig~\ref{fig:sca}), for some number . 


We can now consider what the diagram would look like if we remove, one by
one, all of the sites  in , until only
one () is left.
The new set of sites is . 
	From the definition of Voronoi diagram it is clear that 
	the region corresponding to site  in the new diagram is 
(since no point is \emph{strictly} closer to  than to ), and

(since, by Lem.~\ref{lem:interior}, it is ). 


In particular, since , and  is connected, 
 is \emph{still} multiply-connected,
with  playing the role of  in the new diagram (Fig.~\ref{fig:scb}). 
We show that this cannot be the case, resulting in a contradiction. 

Since  is bounded, then  must be bounded as well. 
And since  is an interior point of , and  is 
an interior point of  (and therefore an interior point of the complement of ), 
then the line passing through  and  must intersect  at
least at two distinct points  and . 


Since the boundary between  and  is composed of
equidistant points to , then both  must be equidistant to . 
By Lem.~\ref{lem:midpoint}, only one point on the line connecting  can be equidistant to them, a contradiction. 




\end{proof}





\section*{Appendix B: Colinear sites}\label{app:colinear}


\begin{figure}[ht]
\centering
\subfigure {
\includegraphics[height=3.5cm]{c09.png}
}
\subfigure {
\includegraphics[height=3.5cm]{c10.png}
}
\caption{If all sites are colinear, the dual is always a polygonal chain.}
\end{figure}


If all the sites are colinear then the structure of the Voronoi diagram
 is
greatly simplified, and always has the form shown in
Fig.\  5.In particular,  has no vertices aside from  since 
vertices are equidistant to three or more sites, 
and no point  can be equidistant to three colinear
sites (since points equidistant to  form an ellipse, and a line
intersects an ellipse at most twice). 

Consider the set of sites  ordered linearly along
their supporting line. 
We shown that the graph dual to , having  as vertices,
has edges . 

For every pair of sites , because they are consecutive, and by Lem.~\ref{lem:midpoint}, 
the midpoint  is closest and equidistant to ,
and therefore the edge  is in the dual. 

We finally show that every dual edge  is of the form
. Assume otherwise, and therefore that, since  are
not consecutive, there is some  between them. 
Because there is a dual edge , the
corresponding primal edge  in  is not empty, and therefore there is some 
point  that is closest and equidistant to . 
By the convexity of , and the fact that , it is , a
contradiction. Therefore  are consecutive sites. 

Since we have characterized the set of edges and vertices, and
there are no (interior) faces in the dual, this completely determines the
dual when all sites are colinear. 

Perhaps interestingly, this means that, in the colinear case, the structure of the dual does not depend on the metric.









\setcounter{section}{6}
\section*{Appendix C: Technical Lemmas}\label{app:technical}

We include in this appendix all claims that, 
although needed to prove the results in this paper, 
are used only as intermediate steps. 
Some of them reveal useful aspects of the structure of the problem. 
In particular, Lem.~\ref{lem:halfspace} is used as a basic step throughout the
paper. 
The proofs are quite technical, and can be skipped on a first read without
affecting the comprehension of the remainder of the paper. 


The following result 
states that, even though the regions and edges of an 
anisotropic Voronoi diagram may be unbounded, there is a sufficiently
large ball outside which there are no vertices. 

\begin{lemma}\label{lem:bounded-vertices}
The set of vertices in the primal diagram  is bounded. 
\end{lemma}
\begin{proof}
Let  be the unique symmetric, positive definite square root of . 
Define  
(which is continuous by virtue of the continuity of  and ) 
and its symmetric positive definite square root . 

Pick three sites . If they are colinear then no point is 
equidistant to them, and so the lemma holds vacuously.  
If they are not colinear, 
consider some  symmetric,
positive definite with ratio of eigenvalues in the range . 
Define   to be the {unique} center
of a circle passing through three (non-colinear) points .  is
continuous wherever  are not colinear. 

If  is equidistant to , and has , then clearly 

or equivalently . 
This means that , 
and so  is the unique point
that can be equidistant to  while having .

Since the space of  is compact and  is continuous w.r.t.\ , then set of possible equidistant points to
 is compact, and therefore bounded. 
Since this is true for every triple , and there is a
finite number of triples, the set of possible equidistant points to three or
more sites is compact, and therefore bounded. 
Since, by definition, vertices are equidistant to three or more sites, the lemma follows. 
\end{proof}



For the sake of conciseness, we define: \begin{definition}
	The open ellipse centered at  with  in its boundary is 
\end{definition}




Given some , we make frequent use of a map that transforms points
 into , where 
 defined in the
proof of Lem.~\ref{lem:bounded-vertices}. We follow the convention of denoting by  the transformed version of point
, where it will be clear from the context which matrix  is used in
the transformation. 
Since the eigenvalues of  are in the range , it's easy to
see that: \begin{itemize}
\item \emph{Given , it is }
\item \emph{If  is the minimum Euclidean distance between a point
 and a line , then .}\\ \\
\emph{Proof:} Given , , it is  where the numerator is , and the denominator satisfies 
. Therefore . 
\end{itemize}
The following technical lemma is 
the basis for the proof of Lem.~\ref{lem:halfspace}. 




\begin{figure}[ht]
\centering
\subfigure[]{
\includegraphics[height=3cm]{thetaa.png}
\label{fig:thetaa}
}
\subfigure[]{
\includegraphics[height=3cm]{thetab.png}
\label{fig:thetab}
}
\label{fig:theta}
\caption{Diagram for the proof of Lem.~\ref{lem:tech}.}
\end{figure}


\begin{lemma}\label{lem:tech}
	Given , with supporting line , and 
an open half space  on one side of , 
as well as a set  of points whose closest point in 
belongs to , 
then 
for all , if  has

then  is strictly closer to  than to . 
\end{lemma}
\begin{proof}
Given a point  that satisfies the above constraint, 
by the definition of , 
it is . 
Also, since it is  and since  doesn't include , 
 and therefore .
Because , 
it can be easily verified that the ellipse  must be 
tangent to  at  (Fig.~\ref{fig:thetaa}). 
When transforming it by ,  becomes the circle

(Fig.~\ref{fig:thetab}), which is tangent to  at . 
The lemma then reduces to showing that  since this
implies  and this in turn means that , and therefore  is closer to  than to . 




A simple calculation shows that  iff 
. Since  implies , it is 

and so , as claimed. 	
\end{proof}



Given  whose Voronoi regions are neighbors in , define  and  to be the two open half spaces 
on either side of the supporting line
 of . Clearly,  form a partition of
. 
If the edge  of  corresponding to sites  is unbounded
then, it must be either , , or
 unbounded.
By Lem.~\ref{lem:midpoint}, it is  
and thus  is always bounded. 
Therefore it must be that either  or
 are unbounded (or both). 
The following lemma proves that to each unbounded Voronoi edge corresponds
at least one open half-space that does not contain any site. 






\begin{lemma}\label{lem:halfspace}
    Given an edge  of the Voronoi diagram , corresponding to neighboring sites , if  () is unbounded, then it is 
 (), 
where  are open half spaces on either side of the
supporting line of . 
\end{lemma}
\begin{proof}
Assume . 
We make use of Lem.~\ref{lem:tech}, with 
taking the role of , respectively. Since  is unbounded, 
we can choose  of sufficiently large norm, according to
Lem.~\ref{lem:tech}, 
such that  which, 
by the convexity of , means that . 
Since  is closer to  than to , it is , a
contradiction. 
\end{proof}



Lastly, we show that every point of sufficiently large norm is strictly closer to
the sites  in the boundary of the convex
hull of , than to sites () in its interior. This means that, outside of a sufficiently large ball, the Voronoi 
diagram of  is the same as that of , a fact critical in proving the
results of Sec.~\ref{sec:boundary}. The proof of this property makes use of the
bound in the ratio of eigenvalues of , and it can easily be shown that
counter-examples exist in cases where no such bound exists. 
The proof assumes that the origin of  is chosen to be in the interior of . 


\begin{figure}
\centering
\subfigure[]{
\includegraphics[height=2.5cm]{wijcap.png}
\label{fig:wijcap}
}
\subfigure[]{
\includegraphics[height=2.5cm]{vHm.png}
\label{fig:vH-}
}
\quad\quad \subfigure[]{
\includegraphics[height=2.5cm]{vHp.png}
\label{fig:vH+}
}
\label{fig:vH}
\caption{Diagram for the proof of Lem.~\ref{lem:VW}.}
\end{figure}




\begin{lemma}\label{lem:VW}
	There is  such that all  with  
are closer to  than to . \end{lemma}
\begin{proof}
Begin by choosing  so that, since the
origin is in , every point of norm greater than  is outside .
Pick  such that it is also:


The proof is by contradiction. 

If  is closest to
 then, since , the ellipse 
 must intersect . 
Because  is not closer to any site in
 than to , then  cannot contain any
, and therefore  must intersect some segment 
as in Fig.~\ref{fig:wijcap} (to see this note that the intersection between an ellipse
and a line is an open line segment, and that the only way in which this
segment can overlap  but not contain either  is for
it to be ). 



Consider the two open half spaces  and  on either side of
the supporting line of a boundary segment , where
 is chosen to be on the ``empty" side
(, by Lem.~\ref{lem:halfspace}). 
Clearly,  partition . 
Since  must intersect some boundary segment, assume w.l.o.g.\   that
it intersects . We show that  cannot be in either
or the three sets in the partition, a contradiction. 


Clearly, if , then the fact that 
implies , which is precluded by
the fact that  is outside . 

Assume , as in Fig.~\ref{fig:vH-}. 
Since not all sites are colinear, then there is  that is not colinear
with , and therefore . 
Eq.~\ref{eqi} has been constructed so that, for any choice of
, Lem.~\ref{lem:tech} implies that,
since , if  is the
closest point to  in , then it is . 
The convexity of , and the fact that
, 
implies , 
contradicting the fact that the site closest to  is . 


Lastly, if we assume that  (Fig.~\ref{fig:vH+}), 
we obtain a similar contradiction using Eq.~\ref{eqii}. 
The ellipse  is transformed by  into the circle
. 
Since it is , 
then it is  . 
A simple calculation reveals that this can only be true if the radius of
 satisfies

However, from Eq.~\ref{eqii} and the fact that , it is 

a contradiction. 

Since we have shown that , 
where  partition , 
this creates the contradiction that we were after. 
Therefore, given the above choice of , 
every  with  is closer to the boundary sites  than to any
site  in the interior of . 
\end{proof}





\section*{Appendix D}\label{app:simple}

\begin{figure}[ht]
\centering
\subfigure[]{
\includegraphics[height=3.5cm]{f3a.png}
\label{fig:3a}
}
\subfigure[]{
\includegraphics[height=3.5cm]{f3b.png}
\label{fig:3b}
}
\label{fig:subfigureExample}
\caption{Diagram for the proof of Lem.~\ref{lem:simple}.}
\end{figure}


\noindent{\bf Lemma~\ref{lem:simple}}
\emph{
 and  are simple (have no multi-edges or self-loops).
}
\begin{proof}



By definition,  doesn't have self-loops since primal edges in  are connected sets of equidistant points to distinct sites , and
therefore they always connect distinct vertices. 
The only possible exception are self-loops in  connecting
 to itself. 
Because an edge of  is a connected set  of points
equidistant to two sites , if the edge is a self-loop of 
then  is unbounded on the two open half-spaces on either side of the
supporting line  of  and therefore, by Lem.~\ref{lem:halfspace},
neither half-space can contain any site. Therefore all sites are contained
in , and are thus all colinear, a contradiction. Therefore there are no
self-loops. 

We prove by contradiction that  has no multi-edges, 
and therefore, by duality, that  doesn't either. 

Consider two edges in the embedding  that connect the same vertices 
(Fig.~\ref{fig:3a}).
One such edge is shown in blue in Fig.~\ref{fig:3a} and consists of a point , 
a path  from  to , 
and a path  from  to .
The second edge, shown in red, has analogous structure. 

From the construction of , it is clear that the two edges must not intersect (other
than meeting at the endpoints), and therefore their union is a simple, closed
curve whose complement has, by the Jordan curve theorem, an interior . 
 must contain some regions corresponding to sites other than , say , 
or else the two edges would be connected and thus not counted as separate edges. 

Since the diagram is orphan-free, the regions inside  contain their
generating sites and thus there are sites  in the
interior of . 
We show that this is not possible, resulting in a contradiction. 


Consider any site  in , as in Fig.~\ref{fig:3b}. 
We first split , as in the figure, into two regions  and
, and show that  cannot belong to either. 
Although  and  may not always be disjoint, it is easy to show that
they cover  (i.e.\  ).

 (shaded in Fig.~\ref{fig:3b}) is the set of points in  that are \emph{inside} some segment
 with , or inside
some segment  with . 
If a site  belongs to  then, without loss of generality, there is  
such that  
(the argument would be the same for ) 
and so , . 
Since , it is . However, , , 
	implies that , a contradiction. 


 is the interior of the triangles  and
. We will only show that , but the 
exact same argument proves that .
Because ,  is closest, and equidistant 
to , and therefore the open ellipse  cannot contain any site (or else  would
belong to its corresponding Voronoi region instead). 
If some site  is  then, since
, it is 
, a
contradiction. 

Since , and , then it is
, which is the contradiction that we were after in the first
place, and therefore no two edges in  connect the same vertices. 
\end{proof}



\noindent{\bf Lemma~\ref{lem:degen}}
\emph{
	 has no degenerate (null area) elements. }
\begin{proof}
Because  is dual to , to every degenerate face of  with 
vertices  corresponds a primal vertex in : a point , and therefore
all of  are in the boundary of the ellipse 
 (with ). 

Since a line can intersect an ellipse at most at two points, 
three or more points cannot be both colinear and in . 
If  are in a degenerate face, they are colinear, and in , a contradiction. 
Therefore no faces of  are degenerate. 
\end{proof}












\section*{Appendix E}\label{app:boundary}




\noindent{\bf Lemma~\ref{boundary_easy}}
\emph{	To every boundary edge  of  corresponds a segment in the boundary of .  \emph{}
}
\begin{proof}
By the definition of , 
to every boundary edge  corresponds a primal edge in 
that is incident to the point at infinity . 
In turn, to this edge corresponds an edge in : 
an unbounded set  of points closest to .


Consider the two open half-planes  and  on either
side of the supporting line  of . 
Since, by Lem.~\ref{lem:midpoint}, it is , it must be either that  or
 are unbounded. 
By Lem.~\ref{lem:halfspace}, they cannot both be, or else 
 and , and therefore all sites
would be in  (all colinear). 
Assume w.l.o.g.\  that  is unbounded. 

By Lem.~\ref{lem:halfspace},  unbounded implies 
, and therefore
 and   
( are vertices in the boundary of ). 

It only remains to show that  are consecutive in the sequence . 
We prove this by contradiction. 
If they are not consecutive, since , 
there must be a site , . 
Pick some
, by definition closest to . 
By the convexity of , it is , a contradiction. 

Since  are consecutive vertices in , then 
. 
\end{proof}






\noindent{\bf Lemma~\ref{lem:mij}}
\emph{
	If , with , , then
. }
\begin{proof}
	Let . If , 
then the fact that  means that  is closest to 
in the support line of . Therefore, , or equivalently:

and thus . 
\end{proof}




\noindent{\bf Lemma~\ref{lem:contrad}}
\emph{
	There is  such that, for any segment , every  with 
    whose closest point in  is  is 
	closer to  than to . }
\begin{proof}





For each choice of , define  to be the set of
points  whose closest point in  is
. 
Pick some , which always exists
since not all sites are colinear. 
We can use Lem.~\ref{lem:tech}, 
where  take the role of , 
to conclude that 
there is a sufficiently large  such that all  with
 are closer to  than to . 
By defining  as the maximum of  over all
, the lemma follows. 
\end{proof}




\noindent{\bf Lemma~\ref{lem:Sn}}
\emph{
	Every continuous function  that is not onto has a fixed point. 
}
\begin{proof}
	Assume  misses , and let
 be a diffeomorphism
between the punctured sphere and the open unit disk. 
Since  is continuous and  is compact,
then the set  is compact.

The function  with  
is continuous and therefore, by Brouwer's fixed point theorem~\cite{Milnor}, has a fixed point . 
The fact that  implies  
and thus  is a fixed point of . 
\end{proof}




\noindent{\bf Lemma~\ref{lem:hard}}
\emph{
To every segment  in the boundary of
 corresponds a boundary edge of .  \emph{}
}
\begin{proof}  
Pick a sufficiently large  such that every  with
 is outside  and such that Lemmas~\ref{lem:VW} and~\ref{lem:contrad} hold. 
For any , if  is the antipodal map , 
then, by continuity of  and by the continuity of , the function  is
continuous. 
By Lemmas~\ref{lem:mij} and~\ref{lem:VW}, if for some  it is 
 with
, then  is (strictly) closest to
, and therefore belongs to the primal edge , which implies
that . 

Showing that  now reduces to showing that for all 
, for all , there is  such that
. 

Assume otherwise: that there is  such that no  satisfies . Then the function 
 is not onto and therefore, 
by Lem.~\ref{lem:Sn} (and using the fact that  is isomorphic to ), 
it must have a fixed point . 

Since  then . 
Since  is the closest point to  in , 
	there is a segment  such that
. Consider two open half spaces
 and  on either side of the supporting line of
. Since not all sites are colinear, we can choose these half spaces so that 
 and . 
By the definition of , and recalling that the chosen origin of
 is in the interior  of the convex hull, 
it is , and . 
To see this note that the outward-facing normal  is defined so that
 and so .
On the other hand, since the origin is in
, the fact that  
implies . 

Since  was chosen sufficiently large for Lem.~\ref{lem:contrad} to
hold, and ,  is closer to some site  than to . 
Since , this contradicts the fact
that  is the closest point to  in
, and thus .
\end{proof}









\section*{Appendix F}\label{app:interior}



\noindent{\bf Lemma~\ref{lem:non-negative}}
\emph{
	Given a non-vanishing one-form , the sum of indices of interior vertices of  is non-negative. 
}
\begin{proof}
Given non-vanishing , 
the index of a face  is . 
To see this, assume otherwise: a face with vertices  around it, and index one satisfies, by the
definition of index and of , 
 
(or ), a
contradiction. 

Because, by Lem.~\ref{col:simple-boundary}, the boundary edges of  form a (not necessarily
strictly) convex, simple polygonal chain then, given any non-vanishing , all the boundary vertices have
index zero, except for the ``topmost" () 
and ``bottommost" () vertices, which have
index one. 

Since face indices are non-positive, and the sum of indices of
boundary vertices is two then, by Lem.~\ref{lem:ph},
the sum of indices of {interior} vertices must be non-negative. 
\end{proof}






\noindent{\bf Lemma~\ref{lem:index-1}}
\emph{
If  has an edge foldover then there is a non-vanishing one-form  such
that  for some interior vertex . }
\begin{proof}
If edge foldover  is a non-boundary edge, then at least one
of its incident vertices, say  is an interior vertex . 

Consider the two faces  incident to , which, by definition of
edge foldover, are on the same side of its
supporting line, and the two edges  in  respectively,
incident to . 
Taking the half-line  from  towards  as reference, consider 
the (open) set  of directions ranging from  to
. 
The set  is not empty since, by Lem.~\ref{lem:degen},
 are not degenerate, and therefore neither  are parallel
to .  is also uncountable, since it is a range of the form

where  are the direction vectors of , and
 is one of the two orthogonal directions to , 
chosen to fit the definition. 

Because  is not empty, and it is uncountable, and because the set of edges  is finite,
then there is always some direction  that is not orthogonal to any edge in
. We prove that the non-vanishing one-form  is such that
. 

The (cyclic) sequence of oriented half-edges {around}  is, without loss of generality,
, and therefore the values of the
one-form around  are , ,
, . 
By the definition of , it is , ,
and , and therefore 
the number of sign changes in 
the subsequence
 is four. 
Since the number of sign changes in the full sequence  cannot
be less
than that of its subsequence , 
it is  and therefore . 
\end{proof}




\noindent{\bf Lemma~\ref{lem:index1}}
\emph{
Given  and non-vanishing one-form , if 
has an interior vertex  with index
, then there is a face  of
 that does not satisfy the empty circum-ellipse property.  
}
\begin{proof}
We must prove that there is a face  all of whose circumscribing ellipses
contain some vertex in its interior. 

Consider the vertex  with
, and thus
with . If  is the cyclic sequence of vertices neighboring
, then 
 implies either , , or , . 
Assume the former w.l.o.g. The line , passing through
, strictly separates  from the convex hull of its neighbors. 

Consider the mesh , with the same structure as  but in which all the incident faces to 
are eliminated. 
We show that, in , the face count of  (the number of faces in which  lies) is at
least one. 
Since  separates  from its neighbors, it also separates all the faces
incident to  from  (except for  itself, which lies on ). 
Pick any direction  with . 
The half-line  starting at  with direction  does not intersect any face in  that is incident to . 
Since there is only a finite number of edges and vertices, it is always
possible to choose  not to contain any vertex other than , and not to be parallel to any edge. 
Since  is bounded and  isn't, there is
some point  outside , whose face count must be zero. 
Moving from  toward ,  crosses  only once 
(since  is convex), incrementing the face count to one. 
Because every interior edge is incident to exactly two faces, 
every subsequent edge cross (which is transversal because  is not
parallel to any edge) modifies the face count by either zero, two, or
minus two. Since the face count cannot be negative, and it is one at 
, then it must be at least one at . 
Since  does not contain any face incident to , 
this implies that there is
some face  not incident to  such that . 

We prove that the face  above cannot satisfy the ECE property. 
Since  is in  but is not incident to it, and  is convex
then, by Carath\'eodory's theorem,  can be written as a 
convex combination ,
,  of vertices 
incident to . 
Given an ellipse  circumscribing the vertices incident to , 
because  is convex, and  lie in its boundary, 
then any convex combination of them with  must be in the
interior of , and therefore  does not satisfy the ECE property. 
\end{proof}







\noindent{\bf Lemma~\ref{lem:ef}}
\emph{
 has no edge foldovers. }
\begin{proof}
Assume  has an edge foldover. 
By Lem.~\ref{lem:index-1}, there is a non-vanishing one-form
 such that some interior vertex  has . 
Since, by Lem.~\ref{lem:non-negative}, the sum of indices of interior vertices is
non-negative,
then there must be
at least one interior vertex  with positive index
. In that case, by Lem.~\ref{lem:index1}, there is a face of  that does not satisfy the
ECE property, raising a contradiction. 
Therefore  has no edge foldovers. 
\end{proof}







\noindent{\bf Lemma~\ref{lem:main-weak}}
\emph{
If its (topological) boundary is simple and closed, then the straight-line dual  of an orphan-free diagram, with vertices incident to at most three sites, is an embedded triangulation. 
}
\begin{proof}


Given a point  in
the interior of the convex hull of , we show that its \emph{face count}
(the number of faces with straight edges that contain it) is one. 
Consider a line  passing through  that does not pass through any vertex of
, and is not parallel to any (straight) edge. 
It is always possible to find such a line since the set of vertices and edges is finite. 
Because the line is unbounded and  is bounded, there is a
point  that is outside . At this point clearly the
face count is zero. 
Moving from  toward ,  crosses the boundary of 
(and therefore, by Thm.~\ref{thm:boundary}, the boundary of ) 
only once, since it is a simple convex polygonal chain, incrementing the face
count by one. At every edge crossing (which is transversal by the choice of
line), the face count remains one since, by Lem.~\ref{lem:ef} there are no
edge foldovers, and thus every non-boundary edge is incident to two faces
that lie on either side of its supporting plane. Therefore the face count at
 must be one. 


Since every point inside  is covered once by faces in
,
and the boundaries of  and  coincide, then
 is a single-cover
of . 
Because two straight edges that cross at a non-vertex always generate points with
face count higher than one, then the edges of  can only meet at vertices, and
therefore  is embedded. 
\end{proof}







\section*{Appendix G}\label{app:generic}



	So far we have assumed that Voronoi vertices are incident (equidistant) to no more than three sites. 
	In general, however, Voronoi vertices may be incident to three or more sites, and the straight-edge dual 
	 will be a polygonization, instead of a triangulation (simplicial complex). 
	We show here that, even in this, more general case, the polygonization  is embedded, and can 
	be easily triangulated into  an embedded simplicial complex. 
	The argument is quite simple and is given in summary. \\



\noindent{\bf Theorem~\ref{th:main}}
\emph{
If its (topological) boundary is simple and closed, then the straight-line dual  of an orphan-free diagram is an  embedded polygonal mesh with convex faces. 
}
\begin{proof}
	First assume that only Voronoi vertex  is incident to more than three sites , , ordered in, say, clockwise order around  (which is possible since they are equidistant to  and therefore 
	lie at the boundary of the ellipse ). 
	Note that, since the  are equidistant to , then any polygon (or triangle) connecting any of the  
		satisfies the empty circum-ellipse property (with empty circum-ellipse ).  
	The  Voronoi regions , corresponding to those sites, are the only ones incident to . 
	We can order these regions in clockwise order around :  
	(for some index sequence  with  satisfying ), 
	and therefore the polygon  dual to  will connect vertices , in that order. 
	Now choose any  in  and triangulate  in a fan arrangement centered at . 
	As pointed out above, both  and every triangle in the triangulation of  satisfies the empty circum-ellipse property. 
	Therefore the new, triangulated  (which replaces  by its triangulation), 
	satisfies all the conditions needed for Theorems~\ref{th:main} and~\ref{thm:boundary} (as well as Lem.~\ref{lem:degen}) to hold, 
	and in particular, it is embedded. 


	The claim now easily follows: any closed polygon with vertices , 
		such that all of its fan-arranged triangulations centered at every one of its vertices is embedded, must be convex 
		(given a polygon with non-convex vertex , triangulate it in a fan centered at a vertex incident to : this triangulation is not embedded). 
	Lastly, since vertices  are equidistant from , the only polygon connecting them 
		that is also convex is the one where vertices are arranged in clockwise (counter-clockwise) order around . 

The argument has been made for a single Voronoi vertex incident to more than three sites, but the same argument applies to diagrams in which there are more such vertices 
	(the key being that the triangulations of each dual polygon are independent of one another). 
\end{proof}


\section*{Appendix H}\label{app:noanisotropy}

We show here that we may drop the bounded anisotropy condition from Thm.~\ref{gamma}, at the expense of loosing the property of covering the convex hull of the sites; 
thus concluding that the dual of an orphan-free anisotropic diagram is an embedded polygonal mesh with convex faces. The proof proceeds by reducing this case to that of Thm.~\ref{gamma}. Because of its simplicity, it is given here in summary. 

Assume given a continuous metric  defined over , and a set  of sites forming an orphan-free anisotropic Voronoi diagram .
For now, assume that the set of Voronoi vertices is finite, and therefore bounded. 
Since there is  such that all Voronoi vertices are inside the origin-centered ball  of radius , we can construct a new metric 
, where 


Clearly,  is continuous, since both  and  are. Since it is  outside of , and  is compact, then 
 has bounded anisotropy (ratio of eigenvalues). 
We may then apply Thm.~\ref{gamma} to conclude that the dual  of  is an embedded polygonal mesh with convex faces. 
Since  includes all Voronoi vertices of , and it is  inside , then all Voronoi vertices in  are also in . 
Therefore, by duality, all faces of  are also in . Finally, since  is embedded with convex faces, 
and removing faces from a polygonal mesh preserves both properties, then  is also embedded with convex faces. 

The only assumption we have made is that the set of Voronoi vertices of  is finite,  which can be justified as follows. 



First, we show that Voronoi vertices cannot be isolated, and are always incident to some Voronoi edge. Consider a Voronoi vertex  closest and equidistant to sites  (the reasoning for vertices equidistant to more sites is analogous). 
Since , and therefore , is continuous, there is a (possibly small) closed ball  centered at  where all points are \emph{strictly} closer to  than to any other site, and therefore in , the Voronoi diagram of  is the same as that of . 
We consider the Voronoi diagrams  and  of  and  (restricted to  ), respectively. 
By Corollary~\ref{cor:Pij}, the only Voronoi edge of  is connected, and likewise for . Since  is in their intersection, then it must be incident to both, and therefore it is not isolated. 


By Corollary~\ref{cor:Pij}, the edges (connected sets of points equidistant to two given sites) of  are unique (thus finite since  is finite), and by the induction argument of Sec.~\ref{sec:setup}, each edge connects two Voronoi vertices. If there are  Voronoi edges then, since Voronoi vertices are always incident to some Voronoi edge, and each edge connects two vertices, there cannot be more than  Voronoi vertices. In particular, the number of Voronoi vertices is finite. This concludes the proof. 

Finally, note that, from the way we have defined the dual , and the fact that it is embedded, we can conclude that Voronoi vertices are unique (otherwise, multiple Voronoi vertices would result in coincident dual polygons, which would contradict the fact that  is embedded). 
This, together with Corollary~\ref{cor:Pij}, and the  orphan-freedom assumption means that, as stated in Corollary~\ref{uniqueVD}, orphan-freedom is sufficient to ensure that all the elements (vertices, edges, faces) of the primal diagram are unique. 






\end{document}
