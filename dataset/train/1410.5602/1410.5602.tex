\subsection{An $O(n\log n)$ Verification Algorithm}
\label{subsec:eff_ver_alg:det}

Let \calTSi denote the trapezoidal map obtained after inserting the
first $i$ curves.
We also use this notation in order to identify
the set of trapezoids of this map.
We denote by \Tall the collection of {\it all} trapezoids created during the
construction of the DAG, including intermediate trapezoids that are
killed by the insertion of later segments.
More formally:
$$\Tall = \bigcup\limits_{i=1}^n \calTSi .$$
Let \Arr{\Tall}
denote the arrangement of all trapezoids in \Tall .
Notice that a face of the arrangement
may be covered by overlapping trapezoids.
The {\it ply} of a point
$p$ in \Arr{\Tall} is defined as the number of
trapezoids in \Tall that cover $p$.
The key to the improved algorithm is the following observation by
Har-Peled~\cite{Har-personal-2012}.
\\

\noindent
\begin{observation} \label{obs:path_len_and_ply_relation}
The length of a path in the DAG for a query point
$q$ is at most three times the ply of $q$ in \Arr{\Tall}.
\end{observation}



It follows that we need to verify that the maximum ply of a point
in \Arr{\Tall} is $c_1 \log n$ for
some constant $c_1 > 0$.
We remark that this ply is established in an
interior of a face of \Arr{\Tall},
since the longest path will always end in a leaf of the DAG,
which, under the general position assumption, represents a trapezoid.
Moreover, for any query point that falls on either a curve or an endpoint
of the initial subdivision
the search path will end in an internal node of the DAG.
If, on the other hand, the query point $q$
falls on a vertical edge of a trapezoid,
the search path for $q$ will be identical to a path
for a query point in a neighboring trapezoid.
Therefore, we consider the boundaries of the trapezoids as open.

Since the input curves are interior
pairwise disjoint,
according to the separation property deduced from \cite{GY-TSR-80},
one can define a total order on the curves;
see more details in Subsection~\ref{subsubsec:eff_ver_alg:reduction} below.
This order allows us to apply a modified version
of an algorithm by Alt and Scharf~\cite{as-cdaaa-13},
which originally detects the
maximum ply in an arrangement of $n$
axis-parallel rectangles in $O(n \log n)$ time.
Recall that we only apply this verification algorithm on DAGs of
linear size.

We would like to describe a linear space
algorithm with $O(n\log n)$ running time
for computing the ply of an arrangement of
open trapezoids with the following properties:
their bases are $y$-axis parallel (vertical walls) and if
the top or bottom curves of two different trapezoids intersect
not only in a joint endpoint
then the two curves overlap completely in their joint $x$-range.
The ply of such an arrangement is the maximum
number of trapezoids containing a common point,
that is, we are only interested in points located
in faces of this arrangement.
In Appendix~\ref{appendix_alt_scharf} we restate the algorithm by Alt \& Scharf~\cite{as-cdaaa-13}
such that the general position assumption can be dropped.
\arxivdcg{The restated algorithm
is more general than what we essentially need
as it can handle rectangles
with independently open or closed boundaries,
while we are only interested in open rectangles.}{}
The algorithm constructs a balanced binary tree representing the possible $x$-intervals.
It then performs a vertical sweep, recording the events of creation and destruction of a rectangle.
The data is kept in the tree nodes, and a final traversal pushes the collected information to the leaves.
The maximal ply will appear in one of the leaves.

Next, we define a reduction from
the collection of open trapezoids \Tall to a collection \Rall of
open axis-parallel rectangles such that the maximum ply in \Arr{\Rall}
is the same as the maximum ply in \Arr{\Tall}.
Using this reduction we can finally
describe a modification for the
restated algorithm such that it can compute the ply of
the arrangement of all trapezoids
created during the
construction of the DAG.




\ignore{
\noindent{\bf An Algorithm for Computing the Ply of an Arrangement
 of Axis-aligned Rectangles}
\label{subsubsubsec:eff_ver_alg:depth_rect}
The algorithm of Alt \& Scharf~\cite{as-cdaaa-13}
is an $O(n\log n)$ algorithm that computes
the ply of an arrangement of axis-aligned rectangles
in general position, using $O(n)$ space.
{
We present here a minor modification,
which does not assume general position,
i.e., rectangles may share boundaries.
Moreover, it can consider each of the four boundaries
of a rectangle as either belonging to the rectangle or not; we call these closed or open boundaries, respectively.

Given a finite set of rectangles,
the set of all $x$-coordinates of the
vertical sides of the input rectangles is first sorted.
Let $x_1,x_2,...,x_m$, $m \leq 2n$ be the sorted set of $x$-coordinates.
The ordered set of intervals $\cal I$, is defined as follows;
for $i\in{1,2,...,m-1}$,
the $2(i-1)$th and $2(i-1)+1$st intervals in the set $\cal I$
are $[x_{i},x_{i}]$ and $(x_{i},x_{i+1})$, respectively.
The last interval is $[x_m,x_m]$.
A balanced binary tree $T$ is then constructed,
holding all intervals in $\cal I$ in its leaves,
according to their order in $\cal I$.
An internal node represents the union of
the intervals of its two children,
which is a contiguous interval.
In addition, each internal node $v$ stores in a variable $v.x$
the $x$-value of the merge point between the intervals
of its two children. Since we extended the algorithm
to support both open or closed boundaries, internal nodes
also maintain a flag indicating whether the merge point
is to the left or to the right of the $x$-value.


According to the description of the algorithm in~\cite{as-cdaaa-13},
a sweep is performed using a horizontal line from
$y=\infty$ to $y=-\infty$.
The sweep-line events occur when a rectangle starts or ends,
i.e., when top or bottom boundary of a rectangle is reached.
Since the rectangles are not in general position,
several events may share the same $y$-coordinate.
In such a case, the order of event processing in
each $y$-coordinate is as follows:
\begin{enumerate}
  \item Closing rectangle with open bottom boundary events.
  \item Opening rectangle with closed top boundary events.
  \item Closing rectangle with closed bottom boundary events.
  \item Opening rectangle with open top boundary events.
\end{enumerate}
The order of event processing within each of these
four groups in a specific $y$-coordinate is not important.

The basic idea of the algorithm is that
each sweep event updates the leaves of the tree $T$
that span the intervals that are covered by the event.
Therefore, each leaf holds a counter $c$ for the
number of covering rectangles in the current position
of the horizontal sweep line.
In addition, each leaf maintains in
a variable $c_m$ the maximal number of
covering rectangles for this leaf seen so far.
Clearly, the maximal coverage of an interval
is the maximal $c_m$ of all leaves.
The problem with this na\"{i}ve approach
is that one such update can already
take $O(n)$ time.
Therefore, the key idea of~\cite{as-cdaaa-13}
is that when updating an event of a rectangle
whose $x$-range is $(a,b)$,
one should follow only two paths;
the path to $a$ and the path to $b$.
The nodes on the path should hold the
information of how to update the
interval spanned by their children.
In the end of the update the union of
intervals spanned by the updated nodes
(internal nodes and only 2 leaves) is $(a,b)$.

In order to hold the information in the internal nodes
each internal node should maintain the following variables:
\begin{itemize}
\item[$l$]
A counter storing the difference between the number of
rectangles that were opened and that were closed
since the last traversal of the left child of $v$
and that cover the interval spanned by that child.
\item[$r$]
A counter storing the difference between the number of
rectangles that were opened and that were closed
since the last traversal of the right child of $v$
and that cover the interval spanned by that child.
\item[$l_m$]
A counter storing the maximum value of~$l$
since the last traversal of that child.
\item[$r_m$]
A counter storing the maximum value of~$r$
since the last traversal of that child.
\end{itemize}

\noindent
A leaf, on the other hand, holds two variables:
\begin{itemize}
\item[$c$]
The coverage of the associated interval during the sweep
at the point the leaf was traversed for the last time. \item[$c_m$]
The maximum coverage of the associated
interval during the sweep from the start until the leaf was traversed for the last time. \end{itemize}

\noindent
In relation to these values we define the following functions:
\[
 t(v) =  \left\{ \begin{array}{ll}
        u.l + t(u) & \mbox{if $v$ is the left child of $u$} \\
        u.r + t(u) & \mbox{if $v$ is the right child of $u$} \\
        0 & \mbox{if $v$ is the root} \\
        \end{array} \right.
,
\]

\[
 t_m(v) =  \left\{ \begin{array}{ll}
        max(u.l_m, u.l + t_m(u)) & \mbox{if $v$ is the left child of $u$} \\
        max(u.r_m, u.r + t_m(u)) & \mbox{if $v$ is the right child of $u$} \\
        0 & \mbox{if $v$ is the root} \\
        \end{array} \right.
 .
\]

\noindent
At any point of the sweep the following two invariants hold for every leaf $\ell$
and its associated interval $I$:
\begin{itemize}
    \item
    The current coverage of $I$ is: $\ell.c + t(\ell)$.
    \item
    The maximum coverage of $I$ that was seen so far is: $\max(\ell.c_m, \ell.c + t_m(\ell))$.
\end{itemize}

\noindent
Updating the structure with an event is done as follows:
Let $I$ be the $x$-interval spanned by the processed rectangle creating the event.
Depending on whether the rectangle starts or ends,
we set a variable $d=1$ or $d=-1$, respectively.
We follow the two search paths to the leftmost leaf and the rightmost leaf that are covered by $I$.
In the beginning the two paths are joined until they split, for every node $w$ on this path
(including the split node) we can ignore $d$ and simply
update the tuple $(w.l,w.r,w.l_m,w.r_m)$ using $t(w)$ and $t_m(w)$
according to the invariants stated above. Note that this process needs to clear the
corresponding values in the parent node as otherwise the invariants would be violated.\footnote{Notice that using $t(w)$ and $t_m(w)$ here
takes constant time since we only need to access the parent
node as all previous nodes on the path towards the root are already processed.}
After the split the paths are processed separately.
We discuss here the left path, the behavior for the right path is symmetric.
Let $v$ be a node on the left path.
As long as $v$ is not a leaf we update $(v.l,v.r,v.l_m,v.r_m)$ as usual.
However, if the path continues to the left we also have to incorporate $d$ into
$v.r$ and $v.r_m$ as the subtree to the right is covered by $I$.
If $v$ is a leaf we simply update $v.c$ and $v.c_m$ using $t(v), t_m(v)$ and $d$.
A more detailed description (including pseudo code) can be found in~\cite{as-cdaaa-13}.
In total, this process takes $O(\log n)$ time.

Finally, in order to find the maximal number of rectangles covering
an interval one last propagation from root to leaves is needed, such that
all $l, r, l_m, r_m$ values of internal nodes are cleared. This is done using one
traversal on $T$.
Now, the maximal number of rectangles covering an interval
is the maximal $c_m$ of all leaves of $T$.

Clearly, the running time of the algorithm is $O(n\log n)$,
since constructing the tree and sorting the $y$-events takes
$O(n\log n)$ time. Updating each of the $2n$ $y$-events takes $O(\log n)$
time, and the final propagation of values to the leaves takes $O(n)$ time.
The algorithm uses $O(n)$ space.

We remark that the above algorithm is not optimal in
memory usage in practice. A more efficient variant which stores less variables
in the nodes of the tree can be easily implemented.
}
{
In~\cite{} \textbf{(MICHAL: ref_to_arxiv,ref_to_thesis)}
we describe a minor modification to the algorithm,
which does not assume general position,
i.e., rectangles may share boundaries.
Moreover, it can consider each of the four boundaries
of a rectangle as either belonging to the rectangle or not; we call these closed or open boundaries, respectively.
}
}

\subsubsection {A Ply Preserving Reduction}
\label{subsubsec:eff_ver_alg:reduction}

Let \TcollReduc be a collection of open trapezoids with $y$-axis parallel bases
with the following property:
if the top or bottom curves of two different trapezoids intersect
not only in joint endpoints
then the two curves overlap completely in their joint $x$-range.
Let \Arr{\TcollReduc} denote the arrangement
of the trapezoids in \TcollReduc . Notice that each arrangement face can be covered by overlapping trapezoids.
We describe a reduction from \TcollReduc to
\RcollReduc, where \RcollReduc is a collection of
axis-parallel rectangles, such that the maximum ply
in \Arr{\RcollReduc} equals to the
maximum ply in \Arr{\TcollReduc}.

In order to define the reduction
we need to have a total order~$<$
on the non-vertical curves of the
trapezoids in~\TcollReduc,
such that one can translate the curves
one by one
according to this order
to $y=-\infty$
without hitting other curves that
have not been moved yet.
Guibas~\&~Yao~\cite{GY-TSR-80} defined
an acyclic relation $\prec$ on a set $C$ of $n$ interior disjoint
$x$-monotone curves as follows:

\begin{definition}
For two such curves $cv_i, cv_j \in C$,
let the open interval $(a,b)$ be the $x$-range of $cv_i$
and the open interval $(c,d)$ be the $x$-range of $cv_j$.\\
If $x\text{-range}(cv_i) \bigcap x\text{-range}(cv_j) \neq\emptyset$ then:\\
\mbox{\ \ } $cv_i \prec cv_j \Leftrightarrow
 cv_i(x) < cv_j(x) \text{ for some } x \in x\text{-range}(cv_i) \bigcap x\text{-range}(cv_j)$.
\end{definition}

As a matter of fact, their definition is more specific,
in a way that the relation $cv_i \prec cv_j$ exists only
if $cv_i$ is the first curve encountered by $cv_j$
in their joint $x$-range
while translating $cv_j$ to $y=-\infty$.
In~\cite{GY-TSR-80} it is also mentioned that $\prec^+$, which is the transitive closure
of $\prec$, is a partial order (as it allows transitivity).
This partial order $\prec^+$ can be extended to a total order $<$ in many ways.
One possible extension is defined as follows:

\begin{definition}
\label{def:ord}
Let $C$ be a set of interior disjoint $x$-monotone curves.
For two curves $cv_i, cv_j \in C$,
let the open interval $(a,b)$ be the $x$-range of $cv_i$
and the open interval $(c,d)$ be the $x$-range of $cv_j$.\\
The total order $<$ on $C$ is defined as follows:\\
\mbox{\ \ } $cv_i < cv_j \Leftrightarrow (cv_i \prec^+ cv_j)$ or $(\neg(cv_j \prec^+ cv_i)$ and $(cv_i$ \emph{left} $cv_j))$ \\
where
$(cv_i$ \emph{left} $cv_j)$ is true if the $x$-value of the left endpoint of $cv_i$ is
less than the $x$-value of the left endpoint of $cv_j$.
\end{definition}

Clearly, if $cv_i \prec^+ cv_j$ is true then $cv_i < cv_j$ is true as well.
If for two different curves $cv_i, cv_j$ the expression $cv_j \prec^+ cv_i$
is true then obviously $cv_i \prec^+ cv_j$ is false
and also the right-hand side expression in the ``or" phrase is false, since~$\neg(cv_j \prec^+ cv_i)$ is false. Therefore, $cv_i < cv_j$ is also false.
If the partial order $\prec^+$ does not say anything about $cv_i$ and $cv_j$
then both $(cv_i \prec^+ cv_j)$ and~$(cv_j \prec^+ cv_i)$ are false.
Thus, $cv_i < cv_j$ will be true only if $(cv_i$ \emph{left} $cv_j)$ is true.

Ottmann~\&~Widmayer~\cite{OW-TSLS-83} presented a one-pass
$O(n\log n)$ time algorithm for computing
$<$, as in Definition~\ref{def:ord}, using linear space.
Their algorithm performs a sweep using a horizontal line
from bottom to top which stops at each endpoint of a curve.
The data structure maintained by the algorithm represents
the curves encountered so far in reverse order.
When a bottom endpoint of a curve is met the curve is inserted into an auxiliary structure
holding the active curves only. A curve is removed from the auxiliary structure when its
top endpoint is met by the sweep line.
Since we would like to translate the curves to $y=-\infty$,
then we should only require the curves to be $x$-monotone.
In addition, we can require the curves to be interior disjoint,
rather than completely disjoint.



\begin{definition}
Let Rank$: C\rightarrow\{1,...,n\}$ denote a function
returning the rank of a given $x$-monotone curve $cv \in C$
when sorting $C$ according to the total order~$<$ as defined above. \end{definition}

\begin{definition}
\label{def:reduction}
We define a reduction from \TcollReduc to
\RcollReduc as follows;
Every trapezoid~$\tred~\in~\TcollReduc$ is
reduced to a rectangle $\rred \in \RcollReduc$,
such that:
\begin{itemize}
  \item \tred and \rred have the same $x$-range,\\
  i.e., $($left$(\tred) =$ left$(\rred))$ and
  $($right$(\tred) =$ right$(\rred))$,
  where left and right denote the left $x$-value
  and the right $x$-value of \tred (or \rred), respectively.
  \item top$(\rred)$ and bottom$(\rred)$ lie on $y=$Rank$($top$(\tred))$ and
  $y=$Rank$($bottom$(\tred))$, respectively.
\end{itemize}
\end{definition}

Definition~\ref{def:reduction} provides a mapping from \TcollReduc to \RcollReduc,
such that $r$ is the rectangular region corresponding to $t$.
\arxivdcg{
We will now show that this mapping is bijective.
One can partition the plane into
vertical slabs by passing a vertical line through every endpoint
of the subdivision, and then partition each slab into regions by
intersecting it with all possible curves in the subdivision.
This defines a decomposition of the plane into at most~$2(n+1)^2$
regions (see~\cite{CG-alg-app}, for example).


\begin{lemma}
\label{lemma:regions_bijection}
Let Regions(arr) denote the collection of regions
of an arrangement arr, as defined above.
For any region $a_t \in$ Regions(\Arr{\TcollReduc})
let $a_r \in$ Regions(\Arr{\RcollReduc}) be the corresponding
rectangular region to $a_t$.
The collection Regions(\Arr{\RcollReduc})
of all such rectangular regions spans the plane.
\end{lemma}

\begin{proof}
Trivial. The slabs remain the same and
within each slab the rectangular regions remain adjacent.
\myqed
\end{proof}

\begin{lemma}
\label{lemma:reduction_from_trpz_to_rect}
Let $a_t \in$ Regions(\Arr{\TcollReduc}) be a region
and let $a_r \in$ Regions(\Arr{\RcollReduc}) be the
corresponding rectangular region to $a_t$.
The number of rectangles in \RcollReduc that cover $a_r$
is at least the number of trapezoids in \TcollReduc that cover $a_t$.
In other words,
for every $\tred \in \TcollReduc$ that covers $a_t$
its corresponding rectangle $\rred \in \RcollReduc$ covers $a_r$.
\end{lemma}

\begin{proof}
Let $\{\tred_1,\tred_2,...,\tred_m\}\subseteq \TcollReduc$
be the set of trapezoids, ordered by creation time,
such that for every $i\in\{1,...,m\}$, $\tred_i$ covers $a_t$.
Let $\{\rred_1,\rred_2,...,\rred_m\}\subseteq \RcollReduc$
be the set of corresponding
rectangles, such that $\rred_i$ corresponds to $\tred_i$ for
$i \in \{1,...,m\}$.
For any $\tred_i$, since $\tred_i$ covers $a_t$
we get that $x\text{-range}(a_t)\subseteq x\text{-range}(\tred_i)$.
By Definition~\ref{def:reduction} the
$x$-ranges remain the same after the reduction,
and therefore $x\text{-range}(a_r) \subseteq x\text{-range}(\rred_i)$.
Since $\tred_i$ covers $a_t$ then we also get that
in the shared $x$-range top$(\tred_i)$ is above or on top$(a_t)$
and bottom$(\tred_i)$ is below or on bottom$(a_t)$.
According to Definition~\ref{def:reduction},
it immediately follows that
Rank$($top$(\tred_i)) \geq$ Rank$($top$(a_t))$.
In other words, top$(\rred_i)$ is above or on top$(a_r)$.
Similarly, bottom$(\rred_i)$ is below or on bottom$(a_r)$.
We conclude that $\rred_i$ covers $a_r$.
\myqed
\end{proof}

\begin{lemma}
\label{lemma:reduction_from_rect_to_trpz}
Let $a_r\in$ Regions$(\Arr{\RcollReduc})$ be a rectangular region,
whose corresponding region is $a_t \in$ Regions$(\Arr{\TcollReduc})$.
The number of trapezoids in \TcollReduc that cover $a_t$
is at least the number of rectangles in \RcollReduc that cover $a_r$.
In other words,
for every $\rred \in \RcollReduc$ that covers $a_r$
its corresponding trapezoid $\tred \in \TcollReduc$ covers $a_t$.
\end{lemma}

\begin{proof}
Let $\{\rred_1,\rred_2,...,\rred_m\}\subseteq \RcollReduc$
be the set of rectangles,
such that for every $i\in\{1,...,m\}$, $\rred_i$ covers $a_r$.
Let $\{\tred_1,\tred_2,...,\tred_m\}\subseteq \TcollReduc$
be the set of corresponding
trapezoids, such that $\tred_i$ corresponds to $\rred_i$ for
$i \in \{1,...,m\}$.
Proving that for any $i \in \{1,...,m\}$, $\tred_i$ covers $a_t$,
is done symmetrically to the proof of Lemma~\ref{lemma:reduction_from_trpz_to_rect}.
\myqed
\end{proof}

Combining Lemma~\ref{lemma:reduction_from_trpz_to_rect} and
Lemma~\ref{lemma:reduction_from_rect_to_trpz}
we conclude that the number of trapezoids in \TcollReduc
that cover a region $a_t$ equals to the number
of rectangles in \RcollReduc that cover $a_r$, which is
the corresponding region to $a_t$.
The covering rectangles are the reduced trapezoids
in the set of trapezoids covering $a_t$.
Since both Regions$(\Arr{\TcollReduc})$
and Regions$(\Arr{\RcollReduc})$ span the plane
(Lemma~\ref{lemma:regions_bijection}),
In summary, we obtain the following theorem.
}
{
In Appendix~\ref{appendix_reduction_bijection} we show that this mapping is bijective.
We show there that the number of trapezoids in \TcollReduc
that cover a region $a_t$ equals to the number
of rectangles in \RcollReduc that cover $a_r$, which is
the region corresponding to $a_t$.
In summary, we obtain the following theorem.
}

\begin{theorem}
\label{theorem:reduction_correctness}
Let \TcollReduc be a collection of
open trapezoids with the following properties:
their bases are $y$-axis parallel (vertical walls) and if
the top or bottom curves of two different trapezoids intersect
not only in joint endpoints
then the two curves overlap completely in their joint $x$-range.
Let \Arr{\TcollReduc} denote the arrangement
of the trapezoids in~\TcollReduc. Notice
that each arrangement face can be covered by
overlapping trapezoids.
\TcollReduc can be reduced to a collection of open
axis-parallel rectangles~\RcollReduc,
such that the maximum ply in \Arr{\RcollReduc} equals
to the maximum ply in \Arr{\TcollReduc}.
\end{theorem}


\subsubsection{ Modification of Alt \& Scharf}
\label{subsubsubsec:eff_ver_alg:depth_general}

Based on the correctness of the reduction
described above we can extend the basic algorithm by Alt \& Scharf~\cite{as-cdaaa-13}
to support not only collections of axis-aligned rectangles but
also collections of open trapezoids with $y$-axis parallel bases and
non-intersecting top and bottom boundaries, if they intersect
not only in joint endpoints
then they overlap completely in their joint $x$-range.
The only part of the basic algorithm that should change
is the top-to-bottom sweep.
More precisely, the simple predicate that is used for sorting the $y$-events
should be replaced with a new predicate that
compares according to the reverse order of~$<$, as given in Definition~\ref{def:ord}.
The total order~$<$ can be computed in a preprocessing phase
using the algorithm in~\cite{OW-TSLS-83}.


Notice that for simplicity
we assumed that no two distinct endpoints
in the original subdivision have
the same $x$-value.
However, if this is not the case,
lexicographical comparison can be used on the endpoints
of the curves in order to define the order
of the induced vertical walls.


\subsection{Summary}\label{subsec:eff_ver_alg:static_bounds}

The two algorithms described in Subsection~\ref{subsec:eff_ver_alg:rec}
and Subsection~\ref{subsec:eff_ver_alg:det} can be used
for defining
efficient construction algorithms for static settings,
according to the scheme presented in Theorem~\ref{thrm:gnrl_preproc_time}.

Using either verification algorithm, a construction algorithm with expected $O(n\log {n})$ running time is obtained,
implying the following main contribution of the paper:

\begin{theorem}
\label{thrm:new}
Let $S$ be a set of $n$
pairwise interior disjoint $x$-monotone curves
inducing a planar subdivision.
A \PL data
structure for $S$, which has~$O(n)$ size and $O(\log{n})$ query time
in the worst case, can be built in expected~$O(n\log{n})$ time.
\end{theorem}









