\documentclass[journal]{IEEEtran}
\newcommand{\ee}{\epsilon}			
\usepackage{epsf,psfrag,amssymb,amsfonts,color,cite,fancybox}
\usepackage[mathscr]{eucal}
\usepackage[dvips]{graphicx}
\usepackage{ifpdf}
\usepackage{longtable}	
\usepackage{subfigure}
\usepackage{multicol}


























\ifCLASSINFOpdf
\else
\fi





\usepackage[cmex10]{amsmath}
\interdisplaylinepenalty=2500













































\hyphenation{op-tical net-works semi-conduc-tor}


\begin{document}
\title{Application of Pseudo-Transient Continuation Method in Dynamic Stability Analysis}

\author{Xiaozhe Wang,  ~\IEEEmembership{Student Member,~IEEE,}
        Hsiao-Dong Chiang,~\IEEEmembership{Fellow,~IEEE.}
\thanks{Xiaozhe Wang is with the Department
of Electrical and Computer Engineering, Cornell University, Ithaca,
NY, 14853 USA e-mail: xw264@cornell.edu}\thanks{Hsiao-Dong Chiang is with the Department of Electrical and Computer Engineering, Cornell University, Ithaca, NY 14853 USA email:hc63@cornell.edu}}

















\maketitle


\begin{abstract}
\boldmath
In this paper, pseudo-transient continuation method has been modified and implemented in power system long-term stability analysis. This method is a middle ground between integration and steady state calculation, thus is a good compromise between accuracy and efficiency. Pseudo-transient continuation method can be applied in the long-term stability model directly to accelerate simulation speed and can also be implemented in the QSS model to overcome numerical difficulties. Numerical examples show that pseudo-transient continuation method can provide correct approximations for the long-term stability model in terms of trajectories and stability assessment.
\end{abstract}

\begin{IEEEkeywords}
pseudo-transient continuation, long-term stability model, quasi steady-state model, long-term stability.
\end{IEEEkeywords}






\IEEEpeerreviewmaketitle



\section{introduction}

\IEEEPARstart{T}IME-domain simulation is an important approach for power system dynamic analysis. However, the complete system model, or interchangeably the long-term stability model, typically includes different components where each component requires several differential and algebraic equations (DAE) to represent, at the same time, these dynamics involve different time scales from millisecond to minute. As a result, the total number of DAE of a real power system can be formidably large and complex such that time domain simulation over long time intervals is expensive\cite{Chiang:book}. These constraints are even more stringent in the context of on-line stability assessment.

Intense efforts have been made to accelerate the simulation of long-term stability model. One approach is to use a larger time step size to filter out the fast dynamics or use automatic adjustment of step size according to system behavior in time-domain simulation \cite{Kundur:book}\cite{Kurita:article}\cite{Cutsem:book} from the aspect of numerical method. Another approach is to implement the Quasi Steady-State (QSS) model in long-term stability analysis \cite{Cutsem:book}\cite{Cutsem:artical} from the aspect of model approximation. Nevertheless, the QSS model suffers from numerical difficulties when the model gets close to singularities which were addressed in \cite{Cutsem:artical2}-\cite{Wang:artical}. Moreover, the QSS model can not provide correct approximations of the long-term stability model consistently as numerical examples shown in \cite{Wangxz:article}\cite{Wangxz:wiley}. In addition, sufficient conditions of the QSS model were developed in \cite{Wangxz:journal} which pointed to a direction to improve the QSS model. As a result, the QSS model requires improvements in both model development and numerical implementation. This paper contributes to the latter one.


In this paper, we apply pseudo-transient continuation () which is a theoretical-based numerical method in power system long-term stability analysis. Pseudo-transient continuation method can be implemented directly in the long-term stability model to accelerate simulation speed compared with conventional implicit integration method. On the other hand, the method can also be applied in the QSS model to overcome possible numerical difficulties due to good stability property.

This paper is organized as follows. Section \ref{sectiondyPTC} briefly reviews general pseudo-transient continuation method in DAE system. Section \ref{sectionPTCinpowersystem} includes a introduction about power system models followed by implementation of pseudo-transient continuation method in the long-term stability model and the QSS model respectively. Section \ref{sectionnumerical} presents three numerical examples to show the feasibility of the method. And conclusions are stated in Section \ref{sectionconclusion}. 


\section{pseudo-transient continuation method}\label{sectiondyPTC}
Pseudo-transient continuation is a physically-motivated method and can be used in temporal integration. The method follows the solution of dynamical system accurately in early stages until the steady state is approaching. The time step is thereafter increased by sacrificing temporal accuracy to gain rapid convergence to steady state \cite{Kelley:article}. If only the steady state of a dynamical system instead of intermediate trajectories is of interest, pseudo-transient continuation method is a better choice than accurate step-by-step integration. On the other hand, compared with methods that solve nonlinear equations for steady state such as line-search and trust region methods, pseudo-transient continuation method can avoid converging to nonphysical solutions or stagnating when the Jacobian matrix is singular. This is particularly the case when the system has complex features such as discontinuities which exist in power system models.

Therefore,  method can be regarded as a middle ground between integrating accurately and calculating the steady state directly.  method can help reach the steady state quickly while maintain good accuracy for the intermittent trajectories. For ODE dynamics, sufficient conditions for convergence of  were given in \cite{Kelley:article}. The results were further extended the semi-explicit index-one DAE system in \cite{Kelley:article2}. We recall the basic algorithm here.


We consider the following semi-explicit index-one DAE system:

with initial value . Here , , , and

where  is a nonsingular scaling matrix. We assume the initial condition for (\ref{DAE}) is consistent, i.e.  and seek to find the equilibrium point  such that  and satisfies
.

As stated before, the step-by-step integration is too time consuming if the intermediate states are not of interest. On the other hand, Newton's method for  alone usually fails as the initial condition is not sufficiently near the equilibrium point. 
The  procedure is defined by the iteration:

where  is adjusted to efficiently find  rather than to enforce temporal accuracy. The convergence results in \cite{Kelley:article}\cite{Kelley:article2} assume that the time step is updated with "switched evolution relaxation"(SER):


The algorithm is shown as below:
\begin{IEEEdescription}[\IEEEusemathlabelsep\IEEEsetlabelwidth{A.}]
\item[\textbf{Algorithm 1 ( for general DAE)}]
\item[\textbf{\textit{1.}}] Set  and . Evaluate .
\item[\textbf{\textit{2.}}] While  is too large.
\begin{LaTeXdescription}
\item[\textbf{\textit{a}}] Solve .
\item[\textbf{\textit{b}}] Set .
\item[\textbf{\textit{c}}] Evaluate .
\item[\textbf{\textit{d}}] Update  according to (\ref{delta}).
\end{LaTeXdescription}
\end{IEEEdescription}

Step 2.a is a Newton step which is typically solved by an iterative method which terminates on small linear residuals while it may also be solved by inexact Newton iteration. Note that the failure of  usually can be well signaled by reaching the bound on the total number of iterations \cite{Kelley:article}.

The convergence of  for smooth  was proved in \cite{Kelley:article2} under the assumptions that the DAE has index one in a certain uniform sense, that it has a global solution in time, and that the solution converges to a steady state. The result were further extended to nonsmooth  in \cite{Kelley:article3} with  in (\ref{PTC}) replaced by a generalized derivative. 
Next we explain why  has a better stability property. Firstly, conventional integration methods insist on a small norm of the linear residual at each step and will either converge, diverge to infinity, or stagnate at a point where iteration matrix is singular. However,  method will accept an increase in the residual, responding to that increase by decreasing \cite{Kelley:article2}. In addition,  stems from backward Euler method which is an attractive choice when stability is the desired property instead of accuracy.



One may think of  method as a predictor-corrector method where the simple predictor is from previous time step and the corrector is backward Euler with Newton iteration \cite{Kelley:article}. To see this, consider the implicit Euler step from  with ,

Thus,  is the root of the following equation:

The Newton's method to find the root of the above equation is:

If we take , then by Binomial Inverse Theorem, the first Newton iterate is:

which is exactly  step. As  method has a better stability property, it can be applied in the QSS model when conventional integration method fails to converge.

\section{ applied in power system models}\label{sectionPTCinpowersystem}
In this section, we firstly introduce power system models in long-term stability analysis. Then we apply  method to the long-term stability model with modifications. Finally, we present an algorithm of  method in the QSS model to overcome possible numerical difficulties.
\subsection{Power System Models}
The long-term stability model for calculating system dynamic response relative to a disturbance can be described as:


Equation (\ref{algebraic eqn}) describes the transmission system and the internal static behaviors of passive devices, and (\ref{fast ode}) describes the internal dynamics of devices such as generators, their associated control systems, certain loads, and other dynamically modeled components.  and  are continuous functions, and vector  and  are the corresponding short-term state variables and algebraic variables. Besides, Equations (\ref{slow ode}) and (\ref{slow dde}) describe long-term dynamics including exponential recovery load, turbine governor, load tap changer (LTC), over excitation limiter (OXL), etc.  and  are the continuous and discrete long-term state variables respectively, and  is the maximum time constant among devices. Note that shunt compensation switching and LTC operation are typical discrete events captured by (\ref{slow dde}) and  is shunt susceptance and the transformer ratio correspondingly. Transitions of  depend on system variables, thus  change values from  to  at distinct times  where , otherwise, these variables remain constants. Since short-term dynamics have much smaller time constants compared with those of long-term dynamics,  and  are also termed as slow state variables, and  are termed as fast state variables.

If we represent the long-term stability model and the QSS model in  time scale, where , and we denote  as , then the long-term stability model of power system can be represented as:

where the study region , and , , , .

At the same time, the QSS model can be represented as:



Moreover, the long-term stability model (\ref{complete}) can be regarded as two decoupled systems (\ref{couple1}) and (\ref{couple2}) showed as below when  jump from  to :

and

discrete variables  are updated first and then system (\ref{couple2}) works with fixed parameters .

Similarly, when  jump from  to , the QSS model (\ref{QSS}) can be decoupled as:

and


\subsection{ for the long-term stability model}
Assuming  is nonsingular, then the long-term stability model (\ref{couple2}) with  fixed as parameters is a semi-explicit index-1 DAE system.  method requires initial condition to satisfy the algebraic equations, however, the discrete equation (\ref{couple1}) will violate this condition whenever it works. As a result, we need to modify the original  method for its implementation in long-term stability model.

In power system long-term stability model, , , , where  is the identiy matrix of size . In order to make the initial condition of  consistent, we switch back to implicit integration method whenever discrete variables jump and set the step length to be . Moreover,  method is implemented for the post-fault system starting from ---several seconds after the contingency. In examples of this paper,  was set to be . The proposed algorithm is shown as below.
\begin{IEEEdescription}[\IEEEusemathlabelsep\IEEEsetlabelwidth{A.}]
\item[\textbf{Algorithm 2 ( in long-term stability model)}]
\item[\textbf{\textit{1.}}] Run the long-term stability model up to  by implicit integration method. Set the value  at  as the initial condition  of , and set .
\item[\textbf{\textit{2.}}] While  is too large.
\begin{LaTeXdescription}
\item[\textbf{\textit{a}}] If discrete variables jump at 
\begin{LaTeXdescription}
\item[\textbf{\textit{ }}]              Update  according to (\ref{couple1}).
\item[\textbf{\textit{ }}]              Set , .
\item[\textbf{\textit{ }}]              Solve the Newton step .
\item[\textbf{\textit{ }}]              Set .
\item[\textbf{\textit{ }}]              Evaluate .
\end{LaTeXdescription}
\item[\textbf{\textit{b}}] Otherwise
\begin{LaTeXdescription}
\item[\textbf{\textit{ }}] Set .
\item[\textbf{\textit{ }}] Solve the Newton step .
\item[\textbf{\textit{ }}] Set .
\item[\textbf{\textit{ }}] Evaluate .
\item[\textbf{\textit{ }}] Update  according to (\ref{delta}).
\end{LaTeXdescription}
\end{LaTeXdescription}
\end{IEEEdescription}

Note that  and  depend on the specific integration method used. For instance, if implicit trapezoidal method is used, then



\subsection{ for the QSS model}
Assuming  is nonsingular, then the QSS model (\ref{coupleqss2}) with  fixed as parameters is a semi-explicit index-1 DAE system.  method requires initial condition to satisfy the algebraic equations, however, the discrete equation (\ref{coupleqss1}) will violate this condition whenever it works. As a result, we need to modify the original  method for its implementation in the QSS model.

In the QSS model, , , , where  is the identity matrix of size . In order to make the initial condition of  consistent, we switch back to implicit integration method whenever discrete variables jump and set the step length to be . Besides, the QSS model is implemented at ---when short-term dynamics settle down after the contingency. Usually,  can be set as 30s. The proposed algorithm is shown as below.
\begin{IEEEdescription}[\IEEEusemathlabelsep\IEEEsetlabelwidth{A.}]
\item[\textbf{Algorithm 3 ( in QSS model)}]
\item[\textbf{\textit{1.}}] Run the long-term stability model up to  by implicit integration method. Set the value  at  as the initial condition  of the QSS model, and set . Start to run the QSS model.
\item[\textbf{\textit{2.}}] If the QSS model has a numerical difficulty by using implicit integration method, then go to step 3, otherwise, continue with the QSS model.
\item[\textbf{\textit{3.}}] While  is too large.
\begin{LaTeXdescription}
\item[\textbf{\textit{a}}] If discrete variables jump at 
\begin{LaTeXdescription}
\item[\textbf{\textit{ }}]              Update  according to (\ref{coupleqss1}).
\item[\textbf{\textit{ }}]              Set , .
\item[\textbf{\textit{ }}]              Solve the Newton step .
\item[\textbf{\textit{ }}]              Set .
\item[\textbf{\textit{ }}]              Evaluate .
\end{LaTeXdescription}
\item[\textbf{\textit{b}}] Otherwise
\begin{LaTeXdescription}
\item[\textbf{\textit{ }}] Set .
\item[\textbf{\textit{ }}] Solve the Newton step .
\item[\textbf{\textit{ }}] Set .
\item[\textbf{\textit{ }}] Evaluate .
\item[\textbf{\textit{ }}] Update  according to (\ref{delta}).
\end{LaTeXdescription}
\end{LaTeXdescription}
\end{IEEEdescription}

Similarly,  and  depend on the specific integration method used. If implicit trapezoidal method is used, then




\section{Numerical Illustration}\label{sectionnumerical}
In this section, three examples are to be presented. The first two examples were the same 145-bus system in which the QSS model met numerical difficulties during simulation while the long-term stability model was stable in long-term time scale. Firstly,  method was implemented in the long-term stability model and the speed was more than 7 times faster than the trapezoidal method. Secondly, when the QSS model by trapezoidal method met difficulty,  method was implemented in the QSS model and provided correct approximations and the speed was still more than 5 times faster than the long-term stability model by trapezoidal method. And in the last example which was a 14-bus system, the long-term stability model was unstable and  method successfully captured the instability which was signaled by reaching the bound of maximum iteration in the Newton step. All simulations were done in psat-2.1.6\cite{Milano:article}.

\subsection{Numerical Example I}\label{qss_numerical}
The system was a 145-bus system \cite{Vittal:article}. There were exciters and power system stabilizers for each of Generator 1-20. And there were turbine governors for each of Generator 30-40. Besides, there were three load tap changers at lines between Bus 73-74, Bus 73-81 and Bus 90-92 respectively. The contingency was a line loss between Bus 1-6.
In this case, the post-fault system was stable after the contingency in the long-term time scale and it took 122.39s for the time domain simulation of the long-term stability model by implicit trapezoidal method. However,  method only took 16.12s for the simulation of the long-term stability model which was about  of the time consumed by trapezoidal method. Fig. \ref{TDSPTC} shows that the trajectories by  method followed closely to the trajectories by trapezoidal method, and finally both converged to the same long-term stable equilibrium point. Thus  method provided correct approximations for the long-term stability model in terms of trajectories and stability assessment.


\begin{figure}[!ht]
\centering
\begin{minipage}[t]{0.5\linewidth}
\includegraphics[width=1.8in ,keepaspectratio=true,angle=0]{v2ofPss13_ptc.eps}
\end{minipage}\begin{minipage}[t]{0.5\linewidth}
\includegraphics[width=1.8in ,keepaspectratio=true,angle=0]{vr1ofExc15_ptc.eps}
\end{minipage}
\begin{minipage}[t]{0.5\linewidth}
\includegraphics[width=1.8in ,keepaspectratio=true,angle=0]{tgofTg5_ptc.eps}
\end{minipage}\begin{minipage}[t]{0.5\linewidth}
\includegraphics[width=1.8in ,keepaspectratio=true,angle=0]{voltageBus94_ptc.eps}
\end{minipage}
\caption{The trajectory comparisons of the long-term stability model by implicit trapezoidal method and  method.  method provided correct approximations.}\label{TDSPTC}
\end{figure}

\subsection{Numerical Example II}
In this example, the system was the same as the last case. However, the QSS model met numerical difficulties at 40s when implicit trapezoidal method was used, thus  method was implemented in the QSS model starting from 40s. Fig. \ref{TDSPTCQSS} shows that the trajectories by  method converged to the long-term stable equilibrium point which the long-term stability model converged to, and also provided good accuracy for the intermittent trajectories. It took 21.75s for  method which  was about  of the time consumed by the long-term stability model using implicit trapezoidal method.


\begin{figure}[!ht]
\centering
\begin{minipage}[t]{0.5\linewidth}
\includegraphics[width=1.8in ,keepaspectratio=true,angle=0]{vr2ofExc13_ptcQSS.eps}
\end{minipage}\begin{minipage}[t]{0.5\linewidth}
\includegraphics[width=1.8in ,keepaspectratio=true,angle=0]{vr1ofExc9_ptcQSS.eps}
\end{minipage}
\begin{minipage}[t]{0.5\linewidth}
\includegraphics[width=1.8in ,keepaspectratio=true,angle=0]{tg1oftg5_ptcQSS.eps}
\end{minipage}\begin{minipage}[t]{0.5\linewidth}
\includegraphics[width=1.8in ,keepaspectratio=true,angle=0]{voltageBus94_ptcQSS.eps}
\end{minipage}
\caption{The trajectory comparisons of the long-term stability model by implicit trapezoidal method and the QSS model by  method.  method overcame numerical difficulties and provided correct approximations.}\label{TDSPTCQSS}
\end{figure}

\subsection{Numerical Example III}
In this case, the 14-bus system was long-term unstable due to wild oscillations of fast variables. The system was modified based on the 14-bus test system in psat-2.1.6\cite{Milano:article}. There were three exponential recovery loads at Bus 9, 10 and 14 respectively and two turbine governors at Generator 1 and 3. Besides, there were over excitation limiters at all generators and three load tap changers at lines between Bus 4-9, Bus 12-13 and Bus 2-4. 
The system suffered from long-term instabilities and simulation by implicit trapezoidal method could not continue after 101.22s.  method also stopped at 103.34s when the bound on the total number of iterations for the Newton step was reached. Thus  method was able to capture instabilities of the long-term stability model.
\begin{figure}[!ht]
\centering
\begin{minipage}[t]{0.5\linewidth}
\includegraphics[width=1.8in ,keepaspectratio=true,angle=0]{deltaofSyn2_14tryptc.eps}
\end{minipage}\begin{minipage}[t]{0.5\linewidth}
\includegraphics[width=1.8in ,keepaspectratio=true,angle=0]{vr1ofExc1_14tryptc.eps}
\end{minipage}
\begin{minipage}[t]{0.5\linewidth}
\includegraphics[width=1.8in ,keepaspectratio=true,angle=0]{vr1ofExc2_14tryptc.eps}
\end{minipage}\begin{minipage}[t]{0.5\linewidth}
\includegraphics[width=1.8in ,keepaspectratio=true,angle=0]{voxlofOxl2_14tryptc.eps}
\end{minipage}
\caption{The trajectory comparisons of the long-term stability model using implicit trapezoidal method and  method.  method was able to capture instabilities of the long-term stability model.}\label{my14completeqss_try}
\end{figure}

\section{Conclusion}\label{sectionconclusion}
In this paper, modified  methods for the long-term stability model and the QSS model are given for power system long-term stability analysis with illustrative numerical examples. We make use of the fast asymptotic convergence of  method in the long-term stability model to achieve fast simulation speed. On the other hand, we take advantage of good stability property of  method in the QSS model to overcome numerical difficulties. Numerical examples show that   can successfully provide correct stability assessment for the long-term stability model and overcome numerical difficulties in the QSS model, as well as offer good accuracy for the intermediate trajectories.  can be regarded as a good in-between method with respect to integration and steady state calculation, thus serves as an alternative method in power system long-term stability analysis.

\section{Acknowledgment}
This work was supported by the Consortium for Electric Reliability Technology Solutions provided by U.S. Department No. DE-FC26-09NT43321.


















\appendices























\ifCLASSOPTIONcaptionsoff
  \newpage
\fi






\begin{thebibliography}{1}
\bibitem{Chiang:book}
H. D. Chiang, {\em Direct Methods for Stability Analysis of Electric Power Systems-Theoretical Foundation, BCU Methodologies, and Applications}.
\newblock New Jersey: John Wiley \& Sons, Inc, 2011.

\bibitem{Kundur:book}
P. Kundur, {\em Power System Stability and Control}.
\newblock New York: McGraw-Hill, Inc. 1994.

\bibitem{Kurita:article}
Kurita, A., H. Okubo, K. Oki, S. Agematsu, D. B. Klapper, N. W. Miller, W. W. Price Jr, J. J. Sanchez-Gasca, K. A. Wirgau, and T. D. Younkins,
{\em Multiple time-scale power system dynamic simulation},
\newblock IEEE Transactions on Power Systems, Vol. 8, No. 1, pp. 216-223, 1993.


\bibitem{Cutsem:book}
T. V. Cutsem, {\em Voltage Stability of Electric Power Systems}.
\newblock Boston/London/Dordrecht: Kluwer Academic Publishers, 1998.


\bibitem{Cutsem:artical}
T. V. Cutsem, Y. Jacquemart, J. N. Marquet, {\em A Comprehensive Analysis of Mid-term Voltage Stability}.
\newblock IEEE Transactions on Power Systems, Vol. 10, No. 3, pp. 1173-1182, August 1995.

\bibitem{Cutsem:artical2}
T. V. Cutsem, M. E. Grenier, D. Lefebvre, {\em Combined Detailed and Quasi Steady-State Time Simulations for Large-disturbance Analysis}.
\newblock International Journal of Electrical Power and Energy Systems, Vol. 28, Issue 9, pp. 634-642, November 2006.

\bibitem{Cutsem:artical3}
P. Rousseaux, T. V. Cutsem {\em Quasi Steady-State Simulation Diagnosis Using Newton Method with Optimal Multiplier}.
\newblock Power Engineering Society General Meeting, 2006.

\bibitem{Wang:artical}
Q. Wang, H. Song, V. Ajjarapu {\em Continuation-Based Quasi-Steady-State Analysis}.
\newblock IEEE Transactions on Power Systems, Vol. 21, No. 1, February 2006.


\bibitem{Wangxz:article}
X. Z. Wang, H. D. Chiang, {\em Some issues with Quasi-Steady State Model in Long-term Stability}
\newblock IEEE PES general meeting 2013.

\bibitem{Wangxz:wiley}
X. Z. Wang, H. D. Chiang, {\em Numerical Investigations on Quasi Steady-State Model for Voltage Stability: Limitations and Nonlinear Analysis}.
\newblock submitted to International Transactions on Electrical Energy Systems.

\bibitem{Wangxz:journal}
X. Z. Wang, H. D. Chiang,
{\em Analytical Studies of Quasi Steady-State Model in Power System Long-term Stability Analysis}.
\newblock to appear in IEEE Transactions on Circuits and Systems I: Regular Papers. DOI 10.1109/TCSI.2013.2284171.



\bibitem{Kelley:article}
C. T. Kelley, D. E. Keyes {\em Convergence Analysis of Pseudo-Transient Continuation}.
\newblock SIAM Journal on Numerical Analysis, Vol. 35, No. 2, pp. 508-523, April 1998.

\bibitem{Kelley:article2}
T. S. Coffey, C. T. Kelley, D. E. Keyes {\em Pseudo-Transient Continuation and Differential-Algebraic Equations}.
\newblock SIAM Journal of Scientific Computing, Vol. 25, No. 2, pp. 553-569, 2003.

\bibitem{Kelley:article3}
K. R. Fowler, C. T. Kelley {\em Pseudo-transient continuation for nonsmooth nonlinear equations}.
\newblock SIAM journal on numerical analysis, Vol. 43, No. 4, pp 1385-1406, 2005.


\bibitem{Milano:article}
F. Milano {\em Power System Analysis Toolbox Documentation for PSAT version 2.1.5, November 1, 2009}.

\bibitem{Vittal:article}
V. Vittal, D. Martin, R. Chu, J. Fish, J. C. Giri, C. K. Tang, F. Eugenio Villaseca, and R. G. Farmer.
{\em Transient stability test systems for direct stability methods}.
\newblock IEEE Transactions on Power Systems, Vol. 7, No. 1, pp. 37-43, Feb. 1992.



\end{thebibliography}












\end{document}
