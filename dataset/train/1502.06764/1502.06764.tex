\documentclass[letterpaper,12pt]{article}
\usepackage{amssymb, amsmath}
\usepackage{amsthm}
\usepackage{algorithm}
\usepackage{algorithmic}
\usepackage{enumerate}
\usepackage[numbers,square,comma,sort&compress]{natbib}
\usepackage[pdftex,colorlinks]{hyperref}
\usepackage{a4}

\newtheorem{theorem}{Theorem}
\newtheorem{corollary}[theorem]{Corollary}
\newtheorem{fact}[theorem]{Fact}
\newtheorem{lemma}[theorem]{Lemma}
\newtheorem{definition}[theorem]{Definition}
\newcommand{\comment}[1]{}
\newcommand\bm[1]{\mbox{\boldmath\unboldmath}} 
\newcommand\bs[1]{\boldsymbol{#1}}
\DeclareMathOperator*{\argmin}{argmin}


\begin{document}
\title{A deterministic sublinear-time
nonadaptive
algorithm for metric
-median selection}

\author{
Ching-Lueh Chang \footnote{Department of Computer Science and
Engineering,
Yuan Ze University, Taoyuan, Taiwan. Email:
clchang@saturn.yzu.edu.tw}
\footnote{Innovation Center for Big Data and Digital Convergence,
Yuan Ze University, Taoyuan, Taiwan.}
\footnote{Supported in part by the Ministry of Science and Technology
of Taiwan under
grant
103-2221-E-155-026-MY2.}
}


\maketitle

\begin{abstract}
We
give
a deterministic -time -approximation
nonadaptive
algorithm for -median selection in
-point metric
spaces,
where
 is arbitrary.
Our proof generalizes that of Chang~\cite{Cha13}.
\end{abstract}


\section{Introduction}

A metric space  is a nonempty set 
endowed with a function  such that
for all , , ,
\begin{itemize}
\item  if and only if ,
\item ,
and
\item  (triangle inequality).
\end{itemize}
The
{\sc metric -median}
problem asks for a point in
an -point
metric space

with the minimum average distance to other points.
For ,
a point

is said to be -approximate for {\sc metric -median} if

An algorithm for {\sc metric -median} is nonadaptive
if the sequence of distances that it inspects
depends only on  but not on .
Because there are  nonzero distances,
``sublinear-time'' will mean ``-time.''



Indyk~\cite{Ind99, Ind00} shows that {\sc metric -median} has a
Monte-Carlo -time -approximation algorithm
for each .
In ,
where ,
{\sc metric -median} has a
Monte-Carlo -time
-approximation algorithm for all ~\cite{KSS10}.
Many other algorithms are known for -median
selection~\cite{GMMMO03, KSS10, ABS10}.
For example,
Guha et al.~\cite{GMMMO03}
give
a deterministic, -time, -space,
-approximation
and one-pass
algorithm as well as a Monte-Carlo algorithm for -median
selection in metric spaces, where .




We show
that {\sc metric -median} has a deterministic
-time -approximation nonadaptive algorithm for all
, generalizing the following theorems:

\begin{theorem}[\cite{Cha13}]\label{mypreviousupperboundresult}
{\sc Metric -median} has a deterministic -time
-approximation nonadaptive algorithm.
\end{theorem}

\begin{theorem}[\cite{Wu14}]
For each
,
{\sc metric -median} has a deterministic -time
-approximation (adaptive) algorithm.\footnote{The time complexity of
 is originally presented as  because  is
independent of .
We include the  factor, which is implicit in the original proof,
for ease of comparison.}
\end{theorem}

When  is a perfect square and , our proof is equivalent to that of
Theorem~\ref{mypreviousupperboundresult}~\cite{Cha13}.
Chang~\cite{Cha14arXiv} shows that {\sc metric -median} has no
deterministic -query -approximation algorithms
(where
an algorithm's
query complexity
is the number of distances that it inspects).
So the approximation ratio of  in Theorem~\ref{mypreviousupperboundresult}
cannot be
improved
to
a
smaller
constant.


\section{Our algorithm}

Let  be a metric space,
 and .
For all ,
denote
the
(unique)
-ary representation of 
by

i.e.,


For all , ,

By convention,
empty sums vanish, e.g.,
.

\begin{lemma}\label{pseudodistanceislarger}
For all , ,

\end{lemma}
\begin{proof}
By equation~(\ref{pseudodistance}) and the triangle inequality for ,

This and
equation~(\ref{multiaryrepresentation})
complete the proof.
\end{proof}



\begin{lemma}
\label{approximationratiolemma}
For all

with

we have

\end{lemma}
\begin{proof}
Let 
be
a
uniformly random
element
of  and

breaking ties arbitrarily.
It is easy to see that

Furthermore,

By equation~(\ref{pseudodistance}),


Finally,
{\small 
}where the
inequality follows from the triangle inequality for , and the
second-to-last equality is true because

distributes uniformly at random over 
for {\em any}  and .
Inequalities~(\ref{startofequations})--(\ref{endofequations})
imply
inequality~(\ref{approximationinequality}).
\end{proof}

For a predicate ,
let  if  is true and  otherwise.
Define

to be
the -ary representation of .
So .
For  and ,
{\small 
}Clearly,



\begin{lemma}\label{theDPresultisthesumofpseudodistanceslemma}
For all ,

\end{lemma}
\begin{proof}
As ,

By the existence and uniqueness of a -ary representation
of each
,

Equations~(\ref{pseudodistance})~and~(\ref{thethingwewantintheformofwhatwecancompute})--(\ref{justanequation})
complete the proof.
\end{proof}


\begin{lemma}
\label{recurrencelemma1}
For all  and ,

\end{lemma}
\begin{proof}
By equation~(\ref{subsumlessthan}),
{\footnotesize 
}for .
Furthermore,

\end{proof}

\begin{lemma}
\label{recurrencelemma2}
For all  and ,

\end{lemma}
\begin{proof}
Observe
the following
for all
, , , :
\begin{enumerate}[(i)]
\item\label{item1}
If , then

if and only if
;
\item\label{item2}
If , then
;
\item\label{item3}
If , then
.
\end{enumerate}

\comment{ For convenience,

}
We have
{\footnotesize 
}
By equation~(\ref{subsumlessthanorequalto}),
{\small 
}By equation~(\ref{subsumlessthan}),
{\small 
}
Because each number in 
can be written
uniquely
as
,
where , , , ,

Finally,

Equations~(\ref{thethingwewanttorecurseon})--(\ref{lastoftediousequations})
complete the proof.
\end{proof}



\begin{figure}
\begin{algorithmic}[1]
\STATE ;
\STATE Find
the -ary representation
of , denoted
;
\FOR{}
  \STATE ;
  \STATE ;
\ENDFOR
\FOR{ up to }
  \FOR{}
    \STATE ;
    \STATE ;
\STATE ;
    \STATE ;
    \STATE ;
    \STATE ;
  \ENDFOR
\ENDFOR
\STATE Output ,
breaking ties arbitrarily;
\end{algorithmic}
\caption{Algorithm {\sf find-median} with input
a
metric space  and .}
\label{mainalgorithm}
\end{figure}

\comment{ Lemmas~\ref{recurrencelemma1}--\ref{recurrencelemma2}
allow us to compute

by dynamic programming.
}
\begin{lemma}
\label{algorithmapproximationratiolemma}
Algorithm {\sf find-median} in Fig.~\ref{mainalgorithm}
is -approximate for {\sc metric -median}.
\end{lemma}
\begin{proof}
By
equations~(\ref{thebasecaseofthefirstfunction})--(\ref{thebasecaseofthesecondfunction}),
lines~4--5 of {\sf find-median}
compute 
and .
Then, by Lemmas~\ref{recurrencelemma1}--\ref{recurrencelemma2},

and 
can be found by dynamic programming as in lines~9--14.
So line~17 outputs
,
which equals

by Lemma~\ref{theDPresultisthesumofpseudodistanceslemma}.
Now Lemma~\ref{approximationratiolemma}
gives the approximation ratio of .
\end{proof}

We now state our main theorem.

\begin{theorem}
{\sc Metric -median} has a
deterministic
-time
-approximation
nonadaptive
algorithm for each
.
\end{theorem}
\begin{proof}
Clearly, {\sf find-median} is deterministic and nonadaptive.
Furthermore, it is -approximate for {\sc metric -median}
by Lemma~\ref{algorithmapproximationratiolemma}.
As  for all ,
the loop in lines~3--6 of {\sf find-median} takes  time.
By precomputing  and  for all
, each iteration of the loop in lines~8--15
takes  time.
\end{proof}


\comment{ By equation~(\ref{pseudodistance}),

for all .








Define

to be
the -ary representation of .
For ,
{\small 
}Clearly,  is the set of -ary representations of the numbers in
.





}

\bibliographystyle{plain}
\bibliography{median_refinement}

\noindent

\end{document}
