\documentclass[adraft,copyright,creativecommons]{eptcs}
\providecommand{\event}{EXPRESS/SOS 2014} \usepackage{breakurl}             

\usepackage[all]{xy}
\usepackage{amsmath}
\usepackage{amssymb}
\usepackage{eurosym}
\usepackage{wasysym}

\newtheorem{theorem}{Theorem}
\newtheorem{axiom}[theorem]{Axiom}
\newtheorem{claim}{Claim}
\newtheorem{question}{Q}
\newtheorem{answer}{A}
\newtheorem{conjecture}[theorem]{Conjecture}
\newtheorem{corollary}[theorem]{Corollary}
\newtheorem{definition}[theorem]{Definition}
\newtheorem{example}[theorem]{Example}
\newtheorem{exercise}[theorem]{Exercise}
\newtheorem{lemma}[theorem]{Lemma}
\newtheorem{notation}[theorem]{Notation}
\newtheorem{problem}[theorem]{Problem}
\newtheorem{proposition}[theorem]{Proposition}
\newtheorem{remark}[theorem]{Remark}
\newtheorem{solution}[theorem]{Solution}
\newtheorem{summary}[theorem]{Summary}
\newenvironment{proof}[1][Proof]{\noindent\textbf{#1.} }{{\hfill  \\}}

\DeclareSymbolFont{cmlargesymbols}{OMX}{cmex}{m}{n}
\DeclareMathSymbol{\mycoprod}{\mathop}{cmlargesymbols}{"60}

\newcommand{\TK}{\fbox{\textbf{TK}}}

\newcommand{\pto}{\rightharpoonup}
\newcommand{\To}{\Longrightarrow}
\newcommand{\Bid}{\tilde{B}}
\newcommand{\Bb}{B^{\infty}}
\newcommand{\Sid}{\bar{\S}}
\newcommand{\Ss}{\Sigma^*}
\renewcommand{\S}{\Sigma}
\newcommand{\id}{\textrm{id}}
\newcommand{\Id}{\textrm{Id}}
\newcommand{\goes}[1]{\stackrel{#1}{\longrightarrow}}
\newcommand{\true}{{\tt tt}}
\newcommand{\false}{{\tt ff}}
\newcommand{\labA}{\{\scriptstyle#2}{\scriptstyle#1}A\lambda:\S B\To B\S\SB\Sigma X=X\times XBX = \Pf(A\times X)\textbf{Set}\rtimes\lambda:\S B\To B\S\Sigma X=X\times XBX=A\times X\S\SBB\Sigma\SigmaB\rho:\S(B\times\Id)\To B\Ss\Ss\S\rho:\S\Bb\To B(\Id+\S)\BbB\lambda:\Ss\Bb\To\Bb\SsBBX=\Pf(A\times X)\Pfx_i{\tt t}x_iy_{i,j}B{\tt t}B\lambda:\Ss\Bb\to\Bb\Ss\Ss\S\BbBX = A\times XBX = \Pf(A\times X)\rho:\S \Bb\To B\Ss\lambda\rho\lambda\rho\SXg:\S X\to Xg:\S X\to Xh:\S Y \to Yf:X\to Yf\circ g = h\circ \S f\S\SX\S(-)+X\Ss X\Ss\psi:\S\Ss\To\Ss\eta:\Id\To\Ss\Ss\mu:\Ss\Ss\To\Ss\Ss\S\iota:\S\To\Ss\iota = \psi \circ \S\eta\psi = \mu\circ\iota\Ss({\tt q}_i)_{i\in I}{\tt q}_in_i\in\mathbb{N}\Sigma X = \mycoprod_{i\in I}X^{n_i}\S\S\Ss XX\eta\psi\mu\iotaBXg:X\to BXg:X\to BXh:Y \to BYf:X\to Yh\circ f = Bf\circ gBXB(-)\times X\Bb X\Bb\theta:\Bb\To B\Bb\epsilon:\Bb\To\Id\Bb\delta:\Bb\To\Bb\Bb\BbB\pi:\Bb\To B\pi = B\epsilon \circ \theta\theta = \pi\Bb\circ\deltaBX = A\times XABXAXB\Bb X = (X\times A)^{\omega}\Bb Xx_i\in Xa_i\in An\in\mathbb{N}\sigma^{(n)}\in\Bb Xn\sigma\sigmax_n\Bb XX\Bb\Bb X\delta_X(\sigma)\sigma\sigma\PfBX ={\cal P_\omega}(A\times X)ABB\BbXAX{\tt T}\in\Bb X\epsilon_X({\tt T})\in X{\tt T}\delta_X({\tt T})\in \Bb\Bb X{\tt T}\theta_X({\tt T})\in B\Bb X\pi_X({\tt T})\in BX{\tt T}_1{\tt T}_2\delta_X({\tt T}_1)\delta_X({\tt T}_2)B\Bid=\Id\times B\SB\Ss\S\rho:\S\Bid\To B\SsBX=A\times X\S X=X\times X{\tt zip}x,x',y,y'\in Xa,b\in A\S\Sid=\Id+\S\SB\BbB\rho:\S\Bb\To B\SidBX=A\times X\S X=X{\tt q}x,x',x'',\hdots \in Xa_1,a_2,a_3,\hdots \in ABX={\cal P_\omega}(A\times X)\S X = X\rho:\S\Bb\To B\Sid\rho_X({\tt q}({\tt T}))(a,{\tt q}(x)){\tt T}xaax(\Ss,\eta,\mu)(\Bb,\epsilon,\delta)\lambda:\Ss\Bb\To\Bb\SsBX=A\times X\S X = \mycoprod_{q\in Q}X \cong Q\times XQ = \{{\tt q}_1,\ldots,{\tt q}_k\}\Bb X = (X\times A)^{\omega}\Ss X = Q^*\times X{\tt t}\in Q^*{\tt t}(x)({\tt t},x)\in \Ss Xx\epsilon(x)\epsilon\in Q^*\lambda:\Ss\Bb\To\Bb\Ssi\in\mathbb{N}b_i=c_i\gamma_i\in\Ss Y\tau_i\in\Ss Xx_jy_j\lambda_X{\tt t}(\sigma){\tt t}\sigmax_j\sigma\tau_kX\lambda_X({\tt t}(\sigma))\sigma\lambda_X(x_0 \goes{a_0} x_1 \goes{a_1} x_2 \goes{a_2} \cdots) \,= \, x_0 \goes{a_0} x_1 \goes{a_1} x_2 \goes{a_2} \cdots\lambda_X({\tt t}(x_0 \goes{a_0} x_1 \goes{a_1} x_2 \goes{a_2} \cdots)) \, = \, \tau_0\goes{b_0}\tau_1\goes{b_1}\tau_2\goes{b_2}\cdots\quad\tau_0={\tt t}(x_0)\lambda_X({\tt s}(x_0 \goes{a_0} x_1 \goes{a_1} x_2 \goes{a_2} \cdots)) \quad\!\! = \, \tau_0\goes{b_0}\tau_1\goes{b_1}\tau_2\goes{b_2}\cdots\quad\lambda_{\Ss X}({\tt t}(\tau_0 \goes{b_0} \tau_1 \goes{b_1} \tau_2 \goes{b_2} \cdots)) \, = \, \gamma_0\goes{c_0}\gamma_1\goes{c_1}\gamma_2\goes{c_2}\cdots\quad\lambda_{X}({\tt ts}(x_0 \goes{a_0} x_1 \goes{a_1} x_2 \goes{a_2} \cdots)) \, = \, \gamma_0\goes{c_0}\gamma_1\goes{c_1}\gamma_2\goes{c_2}\cdots\quad\lambda\S\lambda_X({\tt t}(x_0 \goes{a_0} x_1 \goes{a_1} x_2 \goes{a_2} \cdots)) \quad\!\! = \, \tau_0\goes{b_0}\tau_1\goes{b_1}\tau_2\goes{b_2}\cdots\quadi\in\mathbb{N}\lambda_X(\overline{\tau_i}) = \tau_i\goes{b_i}\tau_{i+1}\goes{b_{i+1}} \cdots\overline{\tau_i}\in\Ss\Bb X\tau_i\in\Ss Xx_j\lambda_X\SB\Ss\S\BbB\rho:\S\Bb\To B\Ss\rho'\rho''\langle\epsilon,\pi\rangle:\Bb\To\Bid[\eta,\iota]:\Sid\To\Ss{\tt f}\Sigma=\Sigma_{\rm GSOS}+\Sigma_{\rm coGSOS}\rho_{\rm GSOS}\rho_{\rm coGSOS}\lambda:\Ss\Bb\To\Bb\Ss\rho:\S\Bb\To B\Ss\lambda\rho\rho\Sigma1B1BX=A\times X\, consider syntax with one constant  and one unary operation , so that  and . Consider  defined by rules:

Consider any distributive law , and present  as:

with each  and . 

If  extends  then, by~\eqref{eq:ibvneverv} applied to , we have  and .
Since  is a distributive law, by axioms (ii) and (iv) of Definition~\ref{def:dist-law} as explained in Example~\ref{ex:dist-laws}, from~\eqref{eq:oeirnvwqv} we get

Now, by~\eqref{eq:ibvneverv} applied to , we have:
 
(only the first step of the stream on the right is determined this way). By axiom (iii) of Definition~\ref{def:dist-law} as explained in Example~\ref{ex:dist-laws}, the stream~\eqref{eq:oesvnaefv} is equal to~\eqref{eq:vinvewv} (or, more precisely, it is mapped to it by pointwise application of ); as a result, . However, there is no such term  and, as a consequence, a distributive law  that extends  does not exist.
\end{example}

\begin{example}\label{ex:onvawrar}\rm
Consider the previous example with the rightmost rule slightly modified to:

If, say, , then the corresponding  can be extended e.g. to distributive laws  such that:

\end{example}

This example shows that distinct distributive laws  can sometimes be equalized by composing with both  and  (see Definition~\ref{def:extn}). However, distinct distributive laws cannot be equalized by composing with only one of these transformations:

\begin{lemma}\label{lem:wonvaev}\rm
For any distributive laws :
\begin{center}
 (a)\ \ if  then , \qquad and \qquad (b)\ \  if  then .
\end{center}
\end{lemma}

It makes sense to say that a biGSOS law  induces a distributive law if there is a unique distributive law that extends . This is consistent with known results about GSOS and coGSOS laws, which, as has been understood since~\cite{turiplotkin}, induce distributive laws:

\begin{theorem}\label{thm:GSOSextends}\rm
For every GSOS law , and for every coGSOS law   there is a unique distributive law  that extends the associated biGSOS law.
\end{theorem}

\begin{proof}[Proof sketch]
For the existence of , constructions of distributive laws from GSOS and coGSOS laws were given already in~\cite{turiplotkin}, and later explained more elegantly in~\cite{lenisapowerwatanabe2}. It is not difficult to prove that those constructions extend the respective GSOS and coGSOS laws in the sense of Definition~\ref{def:extn}.

For the uniqueness of , Lemma~\ref{lem:wonvaev} is used.
\end{proof}

\section{Queue machines}\label{sec:queue}

We shall prove that it is undecidable whether a given biGSOS law uniquely extends to a distributive law. To this end, we use the undecidability of the halting problem of queue machines.

A queue machine (QM) is a deterministic finite automaton additionally equipped with a first-in-first-out queue to store letters. A machine can read letters off the queue, and depending on their contents change their state while adding new letters to the queue.
Under the classical definition~\cite{kozen}, a QM in each transition (a) removes exactly one letter from the queue and (b) adds some (possibly zero) letters to it. 
For our purposes, it will be convenient to consider instead a variant of QMs that, in each step:
(a) remove zero, one or two letters from the queue, and
(b) add exactly one letter to it. 
Formally:
\begin{definition}\label{MQM_def}\rm
A {\em queue machine} (QM) , q_1, \delta_0, \delta_1, \delta_2)QA\, a starting state , and three partial transition functions:

that are disjointly defined and jointly total, i.e., such that for each  and , exactly one of ,  or  is defined.
A {\em configuration} of  is a pair ; the machine induces a transition function  on the set of configurations by:

\end{definition}
Note that an MQM never makes a queue empty, and it terminates if and only if it reaches a configuration  with a single letter  in the queue, such that  and  are undefined.

\begin{theorem}\label{thm:qmundecid}\rm
It is undecidable whether a given QM terminates from the configuration ){\cal M}{\cal M}\overline{\cal M}{\cal M}{\cal M}=(Q,A, \, consider a signature with a single constant  and a family of unary operation symbols , and a family of rules:

for all  and  subject to the following conditions:
\begin{itemize}
\item {\bf R0} is included when ,
\item {\bf R1} is included when  is undefined and , and
\item {\bf R2} is included when  and  are undefined and .
\end{itemize}
These rules are mixed GSOS, so they define a biGSOS law , where  and .
We shall now prove, in a sequence of lemmas, that  uniquely extends to a distributive law if and only if  does {\em not} terminate from the initial configuration. Our argument relies on the following correspondence between partial runs of  and prefixes of streams produced by distributive laws that extend :

\begin{lemma}\label{lem:owefnw}\rm
For every , if a QM  makes  steps from the initial configuration:

(where \lambda\rho_{\cal M}{\tt C}\in \Ss\Bb0\lambda_0({\tt C})\in\Bb\Ss0}\tau_1\goes{a_1}\tau_2\goes{a_3}\tau_3\goes{a_3}\cdots\goes{a_{n-1}}\tau_n,

\lambda_0({\tt C})={\tt C}\goes{\{\cal M}n\lambda\rho_{\cal M}\lambda_0({\tt C})}\tau_1\goes{a_1}\tau_2\goes{a_3}\tau_3\goes{a_3}\cdots\goes{a_{n-1}}\tau_n,

	\sigma \quad =\quad  \lambda_X(t)\quad =\quad \lambda'_X(t) \quad = \quad \tau_0\goes{a_0}\tau_1\goes{a_1}\tau_2\goes{a_2}\cdots \qquad (\tau_i\in\Ss X).

	\lambda_X({\tt C})  \quad = \quad \tau_0 \goes{a_0} \tau_1\goes{a_1}\tau_2\goes{a_2}\cdots \quad \in \quad \Bb \Ss X
, 
\item for any , , where the -th configuration reached by  is  and ; moreover,  is the first letter of .
\end{itemize} 
To define  on other terms in , note that apart from rule {\bf C}, the entire specification  is a coGSOS specification, therefore, by Theorem~\ref{thm:GSOSextends}, there exists a distributive law  that extends all rules of  apart from {\bf C}. For any term  where  does not appear, define  to be . If  appears in , replace it with the stream  and use  followed by  on the term obtained.

It is easy to see that  defined in this manner is natural and satisfies axioms (i)-(iii) of Definition~\ref{def:dist-law} (see also Example~\ref{ex:dist-laws}). 

The only remaining axiom is (iv), which in principle could fail if the above procedure, on one of the terms  present in , returned a stream that differs from the substream of  starting at . This is, however, not the case, as can be proved by induction on , using case analysis similar to that used in the proof of Lemma~\ref{lem:owefnw}.
\end{proof}

\begin{lemma}\label{lem:wbvvw}\rm
If a QM  terminates from the initial configuration, then there is no distributive law that extends .
\end{lemma}
\begin{proof}
Assume to the contrary, that  terminates after  steps in a configuration  and there is a distributive law  that extends . By Lemma \ref{lem:owefnw}, the stream  begins with:
 
where . Note that  can terminate in  only if  has length 1, hence  and ; moreover,  and  must be undefined.

The remaining argument follows the line of Example~\ref{ex:ivbuqwvw}. Suppose that the next step in  is , for some  and . Since  and  are undefined,  must be defined for some  and . As a result,  contains an {\bf R2} rule:

and, since  extends , instantiating  to  we obtain  and , a contradiction.
\end{proof}

Note that all rules in  are either GSOS or coGSOS rules; we call specifications with this property {\em mixed-GSOS specifications}. We arrive at a proof of our Claim from the Introduction:

\begin{theorem}\label{thm:ssundec}\rm
For the case of stream systems,
it is undecidable whether a given mixed-GSOS specification extends to a unique distributive law.
\end{theorem}
\begin{proof}
Combine Lemmas~\ref{lem:ibvwavv}-\ref{lem:wbvvw} with Theorem~\ref{thm:qmundecid}
\end{proof}

\section{Labelled transition systems}\label{sec:qm2lts}

We shall now show how to encode Queue Machines into mixed-GSOS specifications for LTSs, to prove that distributive laws admit no format for  either. Since the general idea and most technical details are the same as in the case of stream systems (Section~\ref{sec:qm2ss}), we only sketch the differences between the two cases.

To begin with, note that the set of rules~\eqref{eq:ssrules} from Section~\ref{sec:qm2ss} can be read as rules in the mixed-GSOS format for . However, taking the same rules for a QM  would give rise to a biGSOS law that always extends to some distributive law (a counterpart of Lemma~\ref{lem:wbvvw} would fail). Intuitively, unlike in the case of , a distributive law for  is allowed to produce an empty set of successors for a term that corresponds to a terminating configuration of .

Our solution is to extend the specification~\eqref{eq:ssrules}, now understood as a mixed-GSOS specification for the LTS behaviour, with additional rules:

for  and . These new rules are included whenever  and  are undefined.
We denote the biGSOS law defined by the extended specification by , where  and .

For any QM , the biGSOS law  uniquely extends to a distributive law if and only if  does not terminate from the initial configuration. The proof of this follows the line of Section~\ref{sec:qm2ss}, and we shall only explain the main differences here.

The main technical step in Section~\ref{sec:qm2ss}, Lemma~\ref{lem:owefnw}, holds in a very similar form:

\begin{lemma}\label{lem:osnvvsv}\rm
For every , if a QM  makes  steps from the initial configuration as in Lemma~\ref{lem:owefnw}, 
then every distributive law  that extends  maps the constant symbol  to a tree  that begins with
a degenerate tree, i.e., a sequence:

where  and  are as in Lemma~\ref{lem:owefnw}.
\end{lemma}
\begin{proof}
By induction on  entirely analogous to the proof of Lemma~\ref{lem:owefnw}. Intuitively, the initial part of  is degenerate because the specification  is deterministic, i.e., it only infers one transition from , and infers at most one transition for  if  can make at most one transition.
\end{proof}

The next two lemmas are proved entirely analogously to Section~\ref{sec:qm2ss}:

\begin{lemma}\label{lem:wainbvwavaw}\rm
For an QM  that does not terminate from the initial configuration, the transformation  is extended by at most one distributive law.
\end{lemma}

\begin{lemma}\label{lem:tyjdrtvx}\rm
If a QM  does not terminate from the initial configuration, then there exists a distributive law that extends .
\end{lemma}

In particular, the distributive law defined in Lemma~\ref{lem:tyjdrtvx} is exactly as in the proof of Lemma~\ref{lem:awonbaerb}, with the streams produced in the latter considered as (degenerate) trees.

The only step that requires some care is Lemma~\ref{lem:wbvvw}, which now takes the form:

\begin{lemma}\label{lem:bnobbes}\rm
If a QM  terminates from the initial configuration, then there is no distributive law that extends .
\end{lemma}
\begin{proof}
Assume to the contrary, that  terminates after  steps in a configuration and there is a distributive law  that extends . By Lemma~\ref{lem:osnvvsv}, the tree  begins with a sequence:
 
where, as in the proof of Lemma~\ref{lem:wbvvw}, , and  and  are undefined.

What successors can  have in the tree ? Assume first that is has no successors. Since  extends , by applying a corresponding rule {\bf R2'} instantiated to ,  and  we infer that  indeed does have at least one successor, which is a contradiction.

Now assume that  has some successors. All these successors are terms in . Some of these successors are minimal, i.e., have the smallest depth of nesting of operations . Pick one of these minimal successors and call it . Since  extends , the transition  must be derivable from rules in . The only rule that can be used to this end is a corresponding rule {\bf R2}, instantiated to , ,  and . But this means that  must have a successor  such that , which contradicts the minimality of .
\end{proof}

Thus we prove our Claim from the Introduction for the case of LTSs:

\begin{theorem}\label{thm:ltsundec}\rm
For the case of labeled transition systems (),
it is undecidable whether a given mixed-GSOS specification extends to a unique distributive law.
\end{theorem}
\begin{proof}
Combine Lemmas~\ref{lem:wainbvwavaw}-\ref{lem:bnobbes} and Theorem~\ref{thm:qmundecid}.
\end{proof}

\section{Related work}\label{sec:related}

We have proved, for the case of stream systems and LTSs, that there is no format for distributive laws of monads over comonads that would be complete for mixed-GSOS specification, i.e., that would cover exactly those mixed-GSOS specification that extend to a distributive law. The specifications used in our proofs are actually coGSOS specifications extended with only one GSOS rule that has no premises. Moreover, the coGSOS rules only uses lookahead of depth 2, and the GSOS rule uses a rule conclusion of height 2. As a result, there is no complete format even for such restricted specifications.

On the other hand, our results do not contradict the existence of formats complete for classes of specifications that do not cover the mixed-GSOS format. Indeed as shown in~\cite{statontyft}, in the context of LTSs one can combine GSOS and coGSOS but restrict to specifications with positive premises only, and guarantee the existence of a corresponding distributive law. (Note that specifications used in Section~\ref{sec:qm2lts} rely on negative rule premises.)

Our proofs can be easily modified to show undecidability of other problems related to operational specifications, some of them phrased without reference to distributive laws. For example, in the case of LTSs, it is undecidable whether a transition system specification (or even a mixed-GSOS specification) has a supported model, a unique supported model, or a unique stable model~\cite{vGnegative}; the constructions needed for these are minor variations of the one used in Section~\ref{sec:qm2lts}.

In the case of stream systems, our results are related to studies of the productivity of stream definitions~\cite{endrullis-pebble}. Specifications used in Section~\ref{sec:qm2ss} can be seen as definitions in the ``pure stream specification format'' of~\cite{endrullis-pebble}. Indeed, that format is closely related to stream coGSOS extended with premise-less GSOS rules for constants. In~\cite{endrullis-pebble} it was proved that productivity of pure stream specifications is decidable for specifications that are data-oblivious, i.e., natural with respect to transition labels. Our specifications are not data-oblivious in that sense. It is easy to use the constructions of Section~\ref{sec:qm2ss} to prove that productivity of pure stream specifications becomes undecidable without data-obliviousness.

\noindent
{\bf Acknowledgment.} We are grateful to Jurriaan Rot for several helpful discussions, and to anonymous referees for spotting embarrassing mistakes both in the content and the presentation of our results.
\bibliographystyle{eptcs}
\bibliography{noformat}

\end{document}
