In this section we define a translation
from stream constants in flat
or friendly nesting specifications to production terms.
In particular, the root  of a specification 
is mapped by the translation to a production term 
with the property that if\/  is flat (friendly nesting),
then the \daob{} lower bound on the production of  in 
equals (is bounded from below by)
the production of~.

\subsection{Translation of Flat and Friendly Nesting Symbols}\label{sec:translation:subsec:functions}





As a first step of the translation,
we describe how for a flat (or friendly nesting)
stream function symbol  in a stream specification 
a periodically increasing function 
can be calculated
that is (that bounds from below)
the \daob{} lower bound on the production of 
in . 

Let us again consider the rules
(i)~,
and
(ii)~
  from Fig.~\ref{fig:pascal}.
We model the \daob{} lower bound on the production of 
by a function from  to  defined as the unique solution for  of the following system of equations.\label{transl:Pascal:f}
We disregard what the concrete stream elements are,
and therefore we take the infimum over all possible traces:

where the solutions for  and 
are the \daob{} lower bounds of  assuming
that the first rule applied in the rewrite sequence is (i) or (ii), respectively.
The rule~(i) consumes two elements, produces one element
and feeds one element back to the recursive call.
For rule~(ii) these numbers are 1, 2, 0 respectively.
Therefore we get:

The unique solution for  is ,
represented by the \ioterm{} . \label{trans:pascal:f}

In general, functions may have multiple arguments,
which during rewriting may get permuted, deleted or duplicated.
The idea is to trace single arguments, and to take the infimum
over traces in case an argument is duplicated.

In the definition of the translation of stream functions,
we need to distinguish the cases according to whether a symbol is weakly guarded or not:
On  we define

the \emph{dependency relation} between symbols in .
We say that a symbol  is \emph{weakly guarded}
if  is strongly normalising with respect to 
and \emph{unguarded}, otherwise.
Note that since  is finite it is (easily) decidable whether
a symbol  is weakly guarded or unguarded.


The translation of a stream function symbol
is defined as the unique solution of a (usually infinite)
system of defining equations where the unknowns are functions.
More precisely, for each symbol  of a flat
or friendly nesting stream specification,
this system has a \pein{} function
 as the solution for ,
which is unique among the continuous functions.
Later we will see 
(Prop.~\ref{prop:corr:pein:gate:transl} and Lem.~\ref{lem:transl:soundness})
that the translation  of a flat (friendly nesting)
stream function symbol  coincides with (is a lower bound for)
the \daob~lower bound  of . 
\begin{definition}\normalfont
  \label{def:transl:flat:fn:symbols:pifuncspec}
Let  be a stream specification.
For each flat or friendly nesting symbol
   with
  arities  and 
  we define ,
  called the \emph{(\pein~function) translation of\/  in }, 
  as the unique solution  for  
  of the following system of defining equations,
  where the solution of an equation of the form
   is a function 
  
  (it is to be understood that  if ):
For all , 
  , and :
-0.4ex]
        0 & \text{if  is unguarded,}
      \end{cases}
    \\
    \specvarap{\astrfun,i}{n}
    = \begin{cases}
        \inf\,
          \big\{
            \specvarap{\astrfun,i,\rho}{n}
            \where \text{ a defining rule of }
          \big\}
        & \text{if  is weakly guarded,} \
We write  for
  ,
  and  for .
For specifying  we distinguish
  the possible forms the rule  can have.
If  is nesting, then
  ,
  and 
   for all .
  Otherwise,  is non-nesting and of the form:
  
  where either (a)~, or
  (b)~
  with , , and
  .
Let:
  
where we agree .
\end{definition}


We mention a useful intuition for understanding the fabric of 
the formal definition above of the translation  
of a stream function ,
the solution  for 
of the system of equations in Def.~\ref{def:transl:flat:fn:symbols:pifuncspec}. 
For each  where 
the solution  for 
(a unary \pein~function) in this system
describes to what extent consumption from the -th component of 
`delays' the overal production.
Since in a \daob~rewrite sequence from

it may happen that all stream variables , \ldots, 
are erased at some point, and that the sequence subsequently continues rewriting stream constants,
monitoring the delays caused by individual arguments is not sufficient
alone to define the production function of .
This is the reason for the use, in the definition of ,
of the solution  for the variable
, which defines a `glass ceiling' for
the not adequate `overall delay' function
,
taking account of situations in which in \daob~rewrite sequences
from terms

all input components have been erased.

Concerning non-nesting rules on which defining rules for
friendly nesting symbols depend via ,
this translation uses the fact that
their production is bounded below by `'.
These bounds are not necessarily optimal,
but can be used to show productivity of examples like
 with
.

Def.~\ref{def:transl:flat:fn:symbols:pifuncspec} can be used to define, 
for every flat or friendly nesting symbol  
in a stream specification~, a `gate translation' of :
by defining this translation by choosing
a gate that represents the \pein~function~. 

However, it is desirable to define this translation also in an
equivalent way that lends itself better for computation,
and where the result directly is a production term (gate) representation 
of a \pein~function:
by specifying the \ioterm{s} in a gate translation
as denotations of rational \ioseq{s} that are
the solutions of `\ioseq\ specifications'. 
In particular, the gate translation of a symbol 
will be defined in terms of the solutions, 
for each argument place of a stream function symbol , 
of a finite \ioseq\ specification
that is extracted from an infinite one which precisely
determines the \daob{} lower bound of  in that argument place.






\begin{definition}\normalfont\label{def:infioseqspecs}
  Let  be a set of variables.
  The set of \emph{\infioseqspecexp{s} over }
  is defined by the following grammar:
  
where .
  Guardedness is defined by induction:
  An \infioseqspecexp\ is called \emph{guarded}
  if it is of one of the forms 
  , ,
  or , or if it is of the form
   for guarded 
  and .

  Suppose that 
  for some (countable) set .
  Then by an \emph{\infioseqspec\ over (the set of recursion variables) }
  we mean
  a family 
  of \emph{recursion equations}, where, for all ,
   is an \infioseqspecexp\ over .
  Let  be an \infioseqspec.
  If   consists of finitely many recursion equations,
  then it is called \emph{finite}.
   is called \emph{guarded} (\emph{weakly guarded})
  if the right-hand side of every recursion equation in 
  is guarded (or respectively, can be rewritten, using equational logic
  and the equations of , to a guarded \infioseqspecexp).

  Let 
  an \infioseqspec. Furthermore let  and .
  We say that  is \emph{a solution of  for }
  if there exist \ioseq{s}  
  such that ,
  and all of the recursion equations in  
  are true statements 
  (under the interpretation of  as defined in
   Def.~\ref{def:iosq:inf}),
  when, for all ,
   is substituted for , respectively.
\end{definition}


It turns out that weakly guarded \infioseqspec{s} have unique solutions,
and that the solutions of finite, weakly guarded \infioseqspec{s} 
are rational \ioseq{s}.


\begin{lemma}\label{lem:wgnf}\label{lem:infioseqspec:ratsol}
For every weakly guarded \infioseqspec~
  and recursion variable  of ,
  there exists a unique solution  of  for .
  Moreover, if  is finite then the solution 
  of  for  is a rational \ioseq{}, 
  and an \ioterm\ that denotes  can
  be computed on the input of  and . 
\end{lemma}








  Let  be a stream specification, and .
  By exhaustiveness of  for , there is at least one
  defining rule for  in .
  Since  is a constructor stream TRS, it follows that
  every defining rule  for  is of the form:
 \label{page:simple:non-nest:form}
  
  with ,
  ,
  
  and  where \sstrcns.
If  is non-nesting then either  or
  
  where  is shorthand for
  ,
  and 
  is a function that describes how the stream arguments are permuted and
  replicated.


We now define, for every given stream specification , 
an infinite \infioseqspec~ that will be instrumental
for defining gate translations of the stream function symbols in .


\begin{definition}\normalfont
  \label{def:infioseqspec}
Let  be a stream specification.
Based on the set:

of tuples
  we define the infinite \infioseqspec~
  by listing the equations of .
  We start with the equations

  Then we let, for all friendly nesting (or flat) 
  with arities  and :
  -0.4ex]
        \Xm & \text{if  is unguarded,}
      \end{cases}
  
\Xfiq{\astrfun}{i}{q}
    = \begin{cases}
        \siosqInfm
          \big\{
            \Xfiqr{\astrfun}{i}{q}{\rho}
            \where \text{ a defining rule of }
          \big\}
        & \text{if  is weakly guarded,} \
  For specifying  and 
  we distinguish the possible forms the rule  can have.
In doing so, we abbreviate terms 
  
  with  a variable of sort stream by 
  and let .
If  is nesting, then we let 

Otherwise,  is non-nesting and of the form:
  
  where either (a)~, or
  (b)~
  with , , and
  .
Let:

where we agree .
\end{definition}


We formally state an easy observation about the system 
defined in Def.~\ref{def:transl:flat:fn:symbols:infioseqspec},
and an immediate consequence due to Lem.~\ref{lem:infioseqspec:ratsol}.


\begin{proposition}\label{prop:infioseqspec:wg}
Let  be a stream specification. 
  The \infioseqspec~ (defined in
  Def.~\ref{def:transl:flat:fn:symbols:infioseqspec}) is weakly guarded.
  As a consequence, 
   has a unique solution in 
  for every variable  in .
\end{proposition}


Another easy observation is that the solutions for the 
variables  in a system 
correspond very directly to numbers in .


\begin{proposition}\label{prop:infioseqspec:argnone}
Let  be a stream specification and .
  The unique solution of  (defined in
  Def.~\ref{def:transl:flat:fn:symbols:infioseqspec})
  for  is of the form
   or .
\end{proposition}

Furthermore observe that  is infinite in case that
, and hence typically is infinite.
Whereas Lem.~\ref{lem:infioseqspec:ratsol} implies unique solvability
of  for individual variables
in view of (the first statement in) Prop.~\ref{prop:infioseqspec:wg}, 
it will usually not
guarantee that these unique solutions are rational \ioseq{s}.
Nevertheless it can be shown that all solutions
of  are rational \ioseq{s}.
For our purposes it will suffice to show this only for the solutions 
of  for certain of its variables.

We will only be interested in the solutions of 
for variables  and .
It turns out that from 
a finite weakly guarded specification  can be extracted
that, 
for each of the variables  and ,
has the same solution as , respectively.
Lem.~\ref{lem:infioseqspec:build:finite} below states that,
for a stream specification , the finite 
\infioseqspec~ can always be obtained algorithmically,
which together with Lem.~\ref{lem:infioseqspec:ratsol} implies that
the unique solutions of  for the
variables  and 
are rational \ioseq{s}\
for which representing \ioterm{s} can be computed. 
As a consequence these solutions, for a stream specification ,
of the recursion system , can be viewed to represent \pein~functions: 
the \pein~functions that are represented by the \ioterm\
denoting the respective solution, a rational \ioseq. 



As an example let us consider an \infioseqspec\ that corresponds to 
 the defining rules for  in Fig.~\ref{fig:pascal}:

The unique solution for  of this system is the rational \ioseq\
(and respectively, \ioterm)
,
which is the translation of  (as mentioned earlier).



\begin{lemma}\label{lem:infioseqspec:build:finite}
Let  be a stream specification.
There exists an \infioseqspec\
  
such that:
\begin{enumerate}
    \renewcommand{\labelenumi}{(\roman{enumi})}
\item  is finite and weakly guarded;    
\item ;\\
for each  
      and ,
       has the same solution 
      for ,
      and respectively for ,
      as the \infioseqspec{s}~ 
      (see Def.~\ref{def:infioseqspec});
\item on the input of ,  
      can be computed. 
\end{enumerate}
\end{lemma}

\begin{proof}[Sketch]
An algorithm for obtaining 
  from  can be obtained as follows.
  On the input of ,
  set  and 
  repeat the following step on  as long as it is applicable:
\begin{description}
\item[{\sf (RPC)}]
      Detect and remove a reachable \emph{non-consuming pseudo-cycle} from the 
      \infioseqspec~:
      Suppose that, for a function symbol , for ,
      and for a recursion variable  that is reachable
      from , we have
      
      (from the recursion variable 
       the variable  is reachable
       via a path in the specification on which the finite~\ioseq~
       is encountered as the word formed by consecutive labels),
      where  and  only contains symbols `'.
      Then modify  by setting
      .
\end{description}
It is not difficult to show that a step {\sf (RPC)} 
  preserves weakly guardedness and the unique solution 
  of , 
  and that, on the input of , 
  the algorithm terminates in finitely many steps, 
  having produced an \infioseqspec~ 
  with  as the solution for 
  and the property
  that only finitely many recursion variables are reachable
  in  from
  .
  \qed
\end{proof}


As an immediate consequence of 
Lem.~\ref{lem:infioseqspec:build:finite},
 
and of Lem.~\ref{lem:infioseqspec:ratsol} we obtain the following lemma.


\begin{lemma}\label{lem:wd:def:transl:flat:fn:symbols:infioseqspec}
Let  be a stream specification,
  and let .
\begin{enumerate}
\item For each 
       there is precisely one \ioseq~ that solves
        for ;
       furthermore,  is rational.
       Moreover there exists an algorithm that,
       on the input of , , and ,
       computes the shortest \ioterm\ that denotes the 
       solution of  for .
\item There is precisely one \ioseq~ that solves
        for ;
       this \ioseq\ is rational.
       And there is an algorithm that, 
       on the input of  and ,
       computes the shortest \ioterm\ that denotes the 
       solution of  
       for .
\end{enumerate}
\end{lemma}


Lem.~\ref{lem:wd:def:transl:flat:fn:symbols:infioseqspec} guarantees 
the well-definedness in the definition below
of the `gate translation' for flat or friendly nesting stream functions 
in a stream specification. 

\begin{definition}\normalfont
  \label{def:transl:flat:fn:symbols:infioseqspec}
Let  be a stream definition. 
For each flat or friendly nesting symbol
   
  with stream arity  we define the \emph{gate translation  of\/~} by:

where  is defined by:

and, for ,
   is the shortest \ioterm\ that denotes
  the unique solution for 
  of the weakly guarded \infioseqspec~
  in Def.~\ref{def:infioseqspec}.
\end{definition}


From Lem.~\ref{lem:wd:def:transl:flat:fn:symbols:infioseqspec}
we also obtain that, for every stream specification,
the function which maps stream function symbols
to their gate translations is computable. 


\begin{lemma}\label{lem:transl:termination}
There is an algorithm that, on the input
  of a stream specification~, and
  a flat or friendly nesting symbol~,
  computes the gate translation  of .
\end{lemma}


\begin{example}
Consider a flat stream specification
consisting of the rules:

The translation of  is ,
  where  is the unique solution for 
  of the \infioseqspec~:


  By the algorithm referred to in Lem.~\ref{lem:transl:termination}
  this infinite specification can be turned into a finite one. The `non-consuming pseudocycle'
  
  justifies the modification of  by setting
  ;
  likewise we set .
  Furthermore all equations not reachable from  are removed
  (garbage collection), 
  and we obtain a finite specification , which,
  by Lem.~\ref{lem:infioseqspec:ratsol}, has a periodically increasing
  solution, with \ioterm-denotation
  .
The reader may try to calculate the gate corresponding to ,
  it is:
  .

  Second, consider the flat stream specification
  with data constructor symbols  and :

denoted , ,
  and , respectively.
  Then,  is the solution
  for  of 
\end{example}

\begin{example}\label{ex:pure}
  Consider a pure stream specification with the function layer: 
The translation of  is ,
  the unique solution for  of the system:

An algorithm for solving such systems of equations is described
  in (the proof of) Lemma~\ref{lem:infioseqspec:build:finite}; here we solve the system directly.
  Note that
  ,
  hence .
  Likewise we obtain  if  and  for ,
  and  if  and  for .
  Then we get , ,
  and  for all ,
  represented by the gate .
  The gate corresponding to  is
  .
  \end{example}
\begin{example}\label{ex:flat}
  Consider a flat stream function specification
  with the following rules which use pattern matching on the data constructors  and :
  
denoted , ,
  and , respectively.
  Then,  is the solution for  of:

  As solution we obtain an overlapping of both traces
   and
  ,
  that is,
  
  represented by the gate .
\end{example}


The following proposition explains the correspondence between
Def.~\ref{def:transl:flat:fn:symbols:pifuncspec}
and Def.~\ref{def:transl:flat:fn:symbols:infioseqspec}.


\begin{proposition}\label{prop:corr:pein:gate:transl}
Let  be a stream specification.
  For each  it holds:

where .
  That is,
  the \pein~function translation  of\/ 
  coincides with the \pein~function that is represented by the
  gate translation .
\end{proposition}


The following lemma states that the translation 
of a flat stream function symbol  
(as defined in Def.~\ref{def:transl:flat:fn:symbols:pifuncspec})
is the \daob{} lower bound on the production function of .
For friendly nesting stream symbols  it states
that  pointwisely bounds from below
the \daob{} lower bound on the production function of .


\begin{lemma}\label{lem:transl:soundness}
  Let  be a stream specification,
  and let\/ .
\begin{enumerate}
    \item If\/  is flat, then:\/
      .
      Hence,  is periodically increasing.
    \item If\/  is friendly nesting, then it holds:\/
      
      (pointwise inequality).
  \end{enumerate}
\end{lemma}




\subsection{Translation of Stream Constants}
  \label{sec:translation:subsec:constants}



In the second step, we now define a translation of
stream constants
in a flat or friendly nesting stream specification into production terms
under the assumption that gate translations for the stream functions are given.
Here the idea is that the recursive definition of a stream constant
 is unfolded step by step;
the terms thus arising are translated according to their
structure using gate translations of the stream function symbols
from a given family of gates;
whenever a stream constant is met that has been unfolded before,
the translation stops after establishing a binding to a -binder
created earlier.


\begin{definition}\label{def:trnsl:nets}\normalfont
Let  be a stream specification,
  and  a family of gates that
  are associated with the symbols in .
  
  Let  be the set of terms in 
  of sort stream that do not contain variables of sort stream. 
The \emph{translation function}
  
  is defined by 
  based on the following definition of expressions
  , where ,
  by induction 
  on the structure of ,
  using the clauses:

where , 
  
  a vector of terms in , 
  ,\linebreak
  ,
  and 
  .
Note that the definition
  of 
  does not depend on the vector  of terms
  in .
  Therefore we define, for all ,
  the \emph{translation of  with respect to } by 
  
  where 
  is a vector of data variables. 
\end{definition}





The following lemma is the basis of our result in Sec.~\ref{sec:results},
Thm.~\ref{thm:decide:prod:flat:pure}, concerning the decidability of 
\daob{} productivity for flat stream specifications.
In particular the lemma states that if we use gates that represent \daobly{} optimal lower bounds
on the production of the stream functions,
then the translation of a stream constant 
yields a production term that rewrites to the \daob{} lower bound of the production of .




\begin{lemma}
  \label{lem1:outsourcing}
Let  be a stream specification, and
   be a family of gates
  such that, for all\/ , the arity of\/ 
  equals the stream arity of\/ .
  Then the following statements hold:
\begin{enumerate}
    \renewcommand{\labelenumi}{(\roman{enumi})}
\item\label{lem1:outsourcing:item:1}
      Suppose that\/ 
      holds for all\/ .
Then for all\/  
      and vectors  
      of data terms

      holds.                   
And consequently,  is \daobly{} productive
      if and only if\/ .
\vspace*{0.75ex}
\item\label{lem1:outsourcing:item:2}
      Suppose that\/ 
      holds for all\/ .
Then for all\/  
      and vectors  
      of data terms 

      holds.          
And consequently,  is \daobly{} non-productive
      if and only if\/ .
\end{enumerate}
\end{lemma}
Lem.~\ref{lem1:outsourcing} is an immediate consequence of the following lemma,
which also is the basis of our result in Sec.~\ref{sec:results} 
concerning the recognizability of productivity for flat and friendly nesting
stream specifications. 
In particular the lemma below asserts that if we use gates that
represent \pein{} functions which are lower bounds 
on the production of the stream functions,
then the translation of a stream constant 
yields a production term that rewrites to a number in 
smaller or equal to the \daob{} lower bound of the production of .


\begin{lemma}
  \label{lem2:outsourcing}
Let  be a stream specification, and let\/
   be a family of gates
  such that, for all\/ , the arity of\/ 
  equals the stream arity of\/ .
Suppose that one of the following statements holds:
\begin{enumerate}[(a)]
\item  for all\/ ;\label{lem2:outsourcing:item:1}\vspace{0.5ex}
\item  for all\/ ;\label{lem2:outsourcing:item:2}\vspace{0.5ex}
\item  for all\/ ;\label{lem2:outsourcing:item:3}\vspace{0.5ex} 
\item  for all\/ .\label{lem2:outsourcing:item:4}\vspace{0.5ex}
\end{enumerate}
Then, for all 
  and vectors 
   
  of data terms in ,
  the corresponding one of the following statements holds:
\begin{enumerate}[(a)]
\item ;\vspace{0.5ex}
\item ;\vspace{0.5ex}
\item ;\vspace{0.5ex}
\item .
\end{enumerate}
\end{lemma}



