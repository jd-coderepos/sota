\documentclass[times,10pt,onecolumn]{article}


\usepackage{epsf}\epsfverbosetrue
\usepackage{graphics,epsfig,color}
\usepackage{graphicx,epsfig} \usepackage{multirow}
\usepackage{alltt}
\usepackage{subfigure}
\usepackage{float}
\usepackage{url}
\usepackage{graphicx}
\usepackage{verbatim} \usepackage{algorithm}
\usepackage{algpseudocode}
\usepackage{footnote} \usepackage{latexsym} \usepackage{amsmath}
\usepackage{sidecap} \usepackage{color}



\newcommand{\ali}[1]{\textcolor{green}{#1}}
\newcommand{\mub}[1]{\textcolor{red}{#1}}



\begin{document}

\title{A Tutorial on the Implementation of Ad-hoc On Demand Distance Vector (AODV) Protocol in Network Simulator (NS-2)\vspace{50pt}}

\author{Mubashir Husain Rehmani, \\ Sidney Doria, and Mustapha Reda Senouci \vspace{170pt} \thanks{M. H. Rehmani is with INRIA, France, e-mail: mubashir.rehmani@inria.fr; S. Doria is with UFCG, Brazil, e-mail: sidney@dsc.ufcg.edu.br; M. R. Senouci is with Laboratory of Research in Artificial Intelligence, Algeria, email: mrsenouci@gmail.com; Special thanks to Hajer Ferjani, she has a Masters Degree in Networking from the National School of Computer Science of Tunisia, CRISTAL Laboratory in 2006, Tunisia, e-mail: 
f.hajer@gmail.com. The author would like to thanks Aline Carneiro Viana, who is with INRIA, France, e-mail: aline.viana@inria.fr;} }

\date{Version 1 \vspace{55pt} \\  June 2009}

\maketitle
\pagebreak
\thispagestyle{plain}

\tableofcontents
\pagebreak

\begin{abstract}
The Network Simulator (NS-2) is a most widely used network simulator. It has the capabilities to simulate a range of networks including wired and wireless networks. In this tutorial, we present the implementation of Ad Hoc On-Demand Distance Vector (AODV) Protocol in NS-2. This tutorial is targeted to the novice user who wants to understand the implementation of AODV Protocol in NS-2.
\end{abstract}



\section{Introduction}
\label{sec:introduction}

The Network Simulator (NS-2) \cite{IEEEhowto:ns} is a most widely used network simulator. This tutorial presents the implementation of Ad Hoc On-Demand Distance Vector (AODV) Protocol \cite{IEEEhowto:aodv} in NS-2. The expected audience are students who want to understand the code of AODV and researchers who want to extend the AODV protocol or create new routing protocols in NS-2. The version considered is NS-2.32 and 2.33, but it might be useful to other versions as well. Throughout the rest of this tutorial, the under considered files are aodv.cc, aodv.h, aodv\_logs.cc, aodv\_packet.h, aodv\_rqueue.cc, aodv\_rqueue.h, aodv\_rtable.cc, aodv\_rtable.h which can be found in AODV folder in the NS-2 base directory.



\section{File Dependency of AODV Protocol}
\label{dep}
Fig.~\ref{fig0} and~\ref{fig1} shows the file dependency of AODV Protocol~\cite{ref}. As AODV is a routing protocol, so it is derived from the class {\it Agent}, see agent.h.

\begin{figure}[htbp]
    \begin{center}
    \includegraphics[width=9cm]{./Figures/aodv.eps}
\vspace{-0.4cm} \caption{File Reference of `AODV.CC'. \vspace{-1.5cm}} 
    \label{fig0}
\end{center}
\end{figure}

\begin{figure}[htbp]
    \begin{center}
    \includegraphics[width=9cm]{./Figures/aodv1.eps}
\vspace{-0.4cm} \caption{File Reference of `AODV.H'.} \vspace{-0.7cm} 
    \label{fig1}
\end{center}
\end{figure}

\section{Flow of AODV}
\label{sec:flow}

In this section, we describes the general flow of AODV protocol through a simple example:

\begin{enumerate}

\item In the TCL script, when the user configures AODV as a routing protocol by using the command,\\
\ >>\&>ns trace-all \ns namtrace-all-wireless \val(x) \topo load\_flatgrid \val(y)

create-god \val(nn)] and attach them to the channel. 

set chan\_1\_ [new \ns node-config -adhocRouting \val(ll) \textbackslash \\
			 -macType \chan\_1\_ \textbackslash \\
			 -ifqType \val(ifqlen) \textbackslash \\
			 -antType \val(prop) \textbackslash \\ 
			 -phyType \topo \textbackslash \\
			 -agentTrace ON \textbackslash \\
			 -routerTrace ON \textbackslash \\
			 -macTrace OFF \textbackslash \\
			 -movementTrace ON \textbackslash \\
			 
			 
	for \{set i 0\} \{\val(nn) \} \{ incr i \} \{ \\
		set node\_(\ns node]	\\
	\} \\


\# Provide initial location of mobilenodes\\
\node\_(0) set Y\_ 5.0 \\
\node\_(1) set X\_ 490.0 \\
\node\_(1) set Z\_ 0.0 \\
\node\_(2) set Y\_ 240.0 \\
\ns at 10.0 ``\ns at 15.0 ``\ns at 110.0 ``\tcp set class\_ 2 \\
set sink [new Agent/TCPSink] \\
\node\_(0) \ns attach-agent \sink \\
\tcp \ftp attach-agent \ns at 10.0 ``\i \textless  \ns initial\_node\_pos \i) 30\\
\}\\

\# Telling nodes when the simulation ends\\
for \{set i 0\} \{\val(nn) \} \{ incr i \} \{\\
    \val(stop) ``\i) reset";\\
\}\\

\# ending nam and the simulation \\
\val(stop) ``\val(stop)"\\
\val(stop) ``stop"\\
\ns halt"\\
proc stop \{\} \{\\
    global ns tracefd namtrace\\
    \tracefd\\
    close \ns run\\



\end{document}