\begin{figure}
     \centering
     \begin{tabular}{ccc}
     \rotatebox[origin=c]{90}{\footnotesize Goog. scan} &
	     \includegraphics[align=c,width=0.4\columnwidth]{figures/distributions/google_azimuth_distribution_crop.png} &
	     \includegraphics[align=c,width=0.4\columnwidth]{figures/distributions/google_elevation_crop.png}
     \end{tabular}
     \centering
     \begin{tabular}{ccc}
     \rotatebox[origin=c]{90}{\footnotesize ABC} &
	     \includegraphics[align=c,width=0.4\columnwidth]{figures/distributions/abc_azimuth_distribution_crop.png} &
	     \includegraphics[align=c,width=0.4\columnwidth]{figures/distributions/abc_elevation_crop.png}
     \end{tabular}
     \centering
     \begin{tabular}{ccc}
     \rotatebox[origin=c]{90}{\footnotesize Bricks} &
	     \includegraphics[align=c,width=0.4\columnwidth]{figures/distributions/lego_azimuth_distribution_crop.png} &
	     \includegraphics[align=c,width=0.4\columnwidth]{figures/distributions/lego_elevation_crop.png}
     \end{tabular}
     \centering
     \begin{tabular}{ccc}
         \rotatebox[origin=c]{90}{\footnotesize Amz. Ber.} &
	     \includegraphics[align=c,clip,width=0.4\columnwidth]{figures/distributions/amazon_azimuth_distribution_crop.png} &
	     \includegraphics[align=c,clip,width=0.4\columnwidth]{figures/distributions/amazon_elevation_crop.png}
     \end{tabular}

\vspace{-0.2cm}
\caption{Camera azimuth and elevation distributions for our different environments}
\label{fig:camera_dist}
\vspace{-0.4cm}
\end{figure}
\section{Appendix}
\subsection{Data generation}

\paragraph{Camera pose distribution.}
Our RTMV dataset is designed to be more challenging than existing novel view synthesis datasets.
In addition to its diversity in terms of scene lighting and objects materials and textures (see Figure~\ref{fig:dataset}),
it contains two types of camera placements: hemisphere and free.
The `hemisphere' camera placement is an easy setup where cameras are placed at an equal distance from the scene (\ie, on a hemisphere).
Therefore, trained multi view synthesis models do not need to reason about multi-scale.
In the `free' camera placement, the camera is allowed to roam freely within a cube surrounding the scene,
thus providing a variety of multi-scale images of the scene.
Figure~\ref{fig:camera_dist} shows the distribution of camera azimuth and elevation for each of our four environments.

\paragraph{Python-based ray tracer.}
Our Python-based ray-tracing renderer is built on NVIDIA OptiX~\cite{optix} with a C++/CUDA backend.
It uses path tracing to render high-quality and physically-based images.
In addition, it appeals to non-graphics experts through its simple Python API.
The scripting capability makes it easy to install, use and share.
Figure~\ref{fig:nvisii} shows a simple code example highlighting the simplicity of our Python API.

\subsection{Additional qualitative results}
We show more qualitative results on each of our four environments in Figures~\ref{fig:more_qual_results_1},~\ref{fig:more_qual_results_2},~\ref{fig:more_qual_results_3},~\ref{fig:more_qual_results_4},
and the corresponding depth maps in Figures~\ref{fig:more_qual_results_1_depth},~\ref{fig:more_qual_results_2_depth},~\ref{fig:more_qual_results_3_depth},~\ref{fig:more_qual_results_4_depth}.
SVLF trains with depth supervision, and thus constructs better depth maps compared to NeRF and mip-NeRF.
Instant-NGP gives the best results but suffers from floating artifacts which can also be seen in the recovered depth maps.
All baselines perform reasonably well on the Google Scanned Objects and the Bricks environments.
However, the performance of both NeRF~\cite{mildenhall2020nerf} and SVLF drops on the ABC environment due its challenging camera poses and lighting conditions.
And eventually, all baselines struggle on our most challenging environment (Amazon Berkeley).
The multi-scale training of mip-NeRF~\cite{barron2021mip} makes it more robust to random camera poses compared to the other baselines, especially on the Amazon Berkeley environment.
Instant-NGP suffers from floating artifacts especially with random camera poses (\eg, ABC and Amazon Berkeley environments).
But overall, Instant-NGP recovers fine details, renders sharp results, and achieves the best performance on our dataset.

The voxel-based formulation of SVLF allows for a better representation of the scene that accommodates free-moving cameras.
This is because SVLF focuses on rays intersecting voxels, rather than where rays originate or end.
In contrast, the near and far render planes parametrization of NeRF and mip-NeRF poses a limitation.
On one hand, setting near and far planes to tightly bound the scene leads to better point sampling along the ray during training,
but allows for less flexibility for moving the camera at render time.
On the other hand, setting the near and far plane to loosely enclose the scene allows for more flexible camera movement at inference time,
but leads to poor training quality as NeRF and mip-NeRF are sensitive to how densely points are sampled along the ray.
Figure~\ref{fig:render_planes} highlights this limitation
by showing example results for when the camera is placed at very close (first two rows) or far (third row) distance from the scene.

\begin{figure}
\begin{lstlisting}[language=Python]
  import raytracer 
  raytracer.initialize()
  # Create camera
  my_camera = raytracer.entity.create(
      name      = 'cam',
      transform = raytracer.transform.create('c_tfm'),
      camera    = raytracer.camera.create('c_cam')
  )
  my_camera.get_transform().look_at(
      eye = [3, 3, 3], at = [0, 0, 0], up = [0, 0, 1]
  )
  raytracer.set_camera_entity(my_camera)
  # Create object
  my_object = raytracer.entity.create(
      name      = 'obj', 
      transform = raytracer.transform.create('o_tfm'),
      mesh      = raytracer.mesh.create_sphere('o_mesh'),
      material  = raytracer.material.create('o_mat')
  )
  raytracer.material.get('o_mat').set_base_color([1, 0, 0])
  # Render image
  raytracer.render_to_file(
      width = 512, height = 512, 
      samples_per_pixel = 1024, 
      file_path = 'image.png'
  )
  raytracer.deinitialize()
\end{lstlisting}
\vspace*{-5.5em}
\raggedleft\includegraphics[width=1.5cm]{figures/sample.png}\hspace{.1em}
\caption{A minimal Python script that renders the inset image.}
\label{fig:nvisii}
\end{figure}


\begin{figure*}
     \centering
     \begin{tabular}{cccc}


       \rotatebox[origin=lt]{90}{\Large \ \ \ \ \ \ \ \ \ \ \ \ \ \ \ \ \ \ \ SVLF} &
	     \includegraphics[width=0.31\textwidth]{figures/render_planes/svlf-000.png} &
	     \includegraphics[width=0.31\textwidth]{figures/render_planes/svlf-060.png} &
	     \includegraphics[width=0.31\textwidth]{figures/render_planes/svlf-far.png}
       \\
       \rotatebox[origin=lt]{90}{\Large \ \ \ \ \ \ \ \ \ \ \ \ \ \ mip-NeRF} &
	     \includegraphics[width=0.31\textwidth]{figures/render_planes/mip-000.png} &
	     \includegraphics[width=0.31\textwidth]{figures/render_planes/mip-060.png} &
	     \includegraphics[width=0.31\textwidth]{figures/render_planes/mip-far.png}
       \\
       \rotatebox[origin=lt]{90}{\Large \ \ \ \ \ \ \ \ \ \ \ \ \ \ \ \ \ \ \ NeRF} &
	     \includegraphics[width=0.31\textwidth]{figures/render_planes/nerf-000.png} &
	     \includegraphics[width=0.31\textwidth]{figures/render_planes/nerf-060.png} &
	     \includegraphics[width=0.31\textwidth]{figures/render_planes/nerf-far.png}
     \end{tabular}
\vspace{-0.2cm}
\caption{The render planes parameterization of the baselines can lead to sub-optimal results when the camera moves freely in the space.}
\label{fig:render_planes}
\vspace{-0.4cm}
\end{figure*}



\subsection{SVLF limitations}

In this section we discuss some of the limitations of the proposed SVLF approach.

\paragraph{Voxel resolution.}
SVLF uses a sparse octree to represent a voxel grid of size .
While this voxel resolution allows for high quality results over the test set of our scenes,
it can produce voxel boundary artifacts if the camera zooms in too close to the scene.
Specifically, the seams between voxels become more visible as the camera zooms in (see Figure~\ref{fig:limitations}).

\paragraph{Sub-optimal training of optical thickness.}
SVLF learns voxel-based functions.
Although using a shared decoder as well as the tri-linear interpolation between voxel features provide a degree of continuity between adjacent voxels,
voxel features could still suffer from a degree of independence.
This especially happens to the optical thickness features, , where the tri-linear interpolations happens at the intersection points, , between the ray and the voxel, rather than at the estimated surface point .
This could lead to a lack of global context between voxels along the ray, thus resulting in errors in the learned optical thickness .
Figure~\ref{fig:limitations} show example failures where incorrect predictions of the optical thickness  lead to either floaters or holes in the rendered image.
Note the holes inside and beside the red shoe in Figure~\ref{fig:limitations}-(a),
and random floaters in Figure~\ref{fig:limitations}-(a), (b).

\paragraph{Translucent surfaces.}
SVLF evaluates the ray color within a voxel at a single point location (the estimated surface hit ).
Therefore, we assume solid surfaces, and bootstrap the training using surface rendering (Sec.~{\color{red} 4.4} of the main paper).
While SVLF utilizes a volumetric training stage to learn non-binary optical thicknesses/densities, the support for translucent surfaces remains limited 
due to sampling a single point within each voxel.

\paragraph{View-dependent effects.}
We observe that SVLF does not capture view dependent effects very well (\eg, specular highlight on the vase in Figure~\ref{fig:limitations}-column 3.)
We hypothesize this is due to the absence of an explicit view direction input to our network.
Adding the view direction as an extra input besides the estimated surface hit  could help mitigate the problem. 



\begin{figure*}
  \centering
  \begin{tabular}{ccc|cc|cc}
    \rotatebox{90}{\small Ground truth} &
    \includegraphics[width=0.14\linewidth]{figures/svlf_limitations/google-2-00112-gt-rect.png} &
    \includegraphics[width=0.14\linewidth]{figures/svlf_limitations/google-2-00112-gt-zoom.png} &
    \includegraphics[width=0.14\linewidth]{figures/svlf_limitations/lego-four-00120-gt-rect.png} &
    \includegraphics[width=0.14\linewidth]{figures/svlf_limitations/lego-four-00120-gt-zoom.png} &
    \includegraphics[width=0.14\linewidth]{figures/svlf_limitations/amazon-5-00128-gt-rect.png} &
    \includegraphics[width=0.14\linewidth]{figures/svlf_limitations/amazon-5-00128-gt-zoom.png}
    \\
    \rotatebox{90}{\small  \ \ \ \ \ \ SVLF} &
    \includegraphics[width=0.14\linewidth]{figures/svlf_limitations/google-2-00112-svlf-rect.png} &
    \includegraphics[width=0.14\linewidth]{figures/svlf_limitations/google-2-00112-svlf-zoom.png} &
    \includegraphics[width=0.14\linewidth]{figures/svlf_limitations/lego-four-00120-svlf-rect.png} &
    \includegraphics[width=0.14\linewidth]{figures/svlf_limitations/lego-four-00120-svlf-zoom.png} &
    \includegraphics[width=0.14\linewidth]{figures/svlf_limitations/amazon-5-00128-svlf-rect.png} &
    \includegraphics[width=0.14\linewidth]{figures/svlf_limitations/amazon-5-00128-svlf-zoom.png}
    \\

    &
    \multicolumn{2}{c}{(a)} &
    \multicolumn{2}{c}{(b)} &
    \multicolumn{2}{c}{(c)}
    \\

  \end{tabular}
  \caption{Example limitations of SVLF.
           (a) and (b) show occasional floater/hole artifacts due to sub-optimal training of the optical thickness .
  				 (c) shows seams between voxel boundaries when the camera is too close to the scene, and a lack of specular highlights (figure best seen in zoom).
          }
  \label{fig:limitations}
\end{figure*}


\begin{figure*}
\centering
\begin{tabular}{cc|c|c|c}
\rotatebox[origin=lt]{90}{\small \ \ \ \ \ \ \ \ \ Ground truth} &
\includegraphics[trim=0 0 0 -5, width=0.40\columnwidth]{figures/more_qual_results/google-7-00143-gt-rgb.png}
&

\includegraphics[trim=0 0 0 -5, width=0.40\columnwidth]{figures/more_qual_results/google-0-00121-gt-rgb.png}
&

\includegraphics[trim=0 0 0 -5, width=0.40\columnwidth]{figures/more_qual_results/google-1-00108-gt-rgb.png}
&

\includegraphics[trim=0 0 0 -5, width=0.40\columnwidth]{figures/more_qual_results/google-8-00122-gt-rgb.png}
\\

\rotatebox[origin=lt]{90}{\small \ \ \ \ \ \ \ \ \ \ \ \ \ \ SVLF} &
\includegraphics[trim=0 0 0 -5, width=0.4\columnwidth]{figures/more_qual_results/google-7-00143-svlf-rgb.png}
&

\includegraphics[trim=0 0 0 -5, width=0.4\columnwidth]{figures/more_qual_results/google-0-00121-svlf-rgb.png}
&

\includegraphics[trim=0 0 0 -5, width=0.4\columnwidth]{figures/more_qual_results/google-1-00108-svlf-rgb.png}
&

\includegraphics[trim=0 0 0 -5, width=0.4\columnwidth]{figures/more_qual_results/google-8-00122-svlf-rgb.png}
\\

\rotatebox[origin=lt]{90}{\small \ \ \ \ \ \ \ \ \ \ Ins.-NGP} &
\includegraphics[trim=0 0 0 -5, width=0.4\columnwidth]{figures/more_qual_results/google-7-00143-hash-rgb.png}
&

\includegraphics[trim=0 0 0 -5, width=0.4\columnwidth]{figures/more_qual_results/google-0-00121-hash-rgb.png}
&

\includegraphics[trim=0 0 0 -5, width=0.4\columnwidth]{figures/more_qual_results/google-1-00108-hash-rgb.png}
&

\includegraphics[trim=0 0 0 -5, width=0.4\columnwidth]{figures/more_qual_results/google-8-00122-hash-rgb.png}
\\

\rotatebox[origin=lt]{90}{\small \ \ \ \ \ \ \ \ \ mip-NeRF} &
\includegraphics[trim=0 0 0 -5, width=0.4\columnwidth]{figures/more_qual_results/google-7-00143-mip-rgb.png}
&

\includegraphics[trim=0 0 0 -5, width=0.4\columnwidth]{figures/more_qual_results/google-0-00121-mip-rgb.png}
&

\includegraphics[trim=0 0 0 -5, width=0.4\columnwidth]{figures/more_qual_results/google-1-00108-mip-rgb.png}
&

\includegraphics[trim=0 0 0 -5, width=0.4\columnwidth]{figures/more_qual_results/google-8-00122-mip-rgb.png}
\\

\rotatebox[origin=lt]{90}{\small \ \ \ \ \ \ \ \ \ \ \ \ \ \ NeRF} &
\includegraphics[trim=0 0 0 -5, width=0.4\columnwidth]{figures/more_qual_results/google-7-00143-nerf-rgb.png}
&

\includegraphics[trim=0 0 0 -5, width=0.4\columnwidth]{figures/more_qual_results/google-0-00121-nerf-rgb.png}
&

\includegraphics[trim=0 0 0 -5, width=0.4\columnwidth]{figures/more_qual_results/google-1-00108-nerf-rgb.png}
&

\includegraphics[trim=0 0 0 -5, width=0.4\columnwidth]{figures/more_qual_results/google-8-00122-nerf-rgb.png}
\\

\end{tabular}
\caption{More qualitative results on the Google Scanned Objects environment (figure best seen in zoom).}
\label{fig:more_qual_results_1}
\end{figure*}


\begin{figure*}
\centering
\begin{tabular}{cc|c|c|c}
\rotatebox[origin=lt]{90}{\small \ \ \ \ \ \ \ \ \ Ground truth} &
\includegraphics[trim=0 0 0 -5, width=0.4\columnwidth]{figures/more_qual_results/depth/google-7-00143-gt-depth.png}
&

\includegraphics[trim=0 0 0 -5, width=0.4\columnwidth]{figures/more_qual_results/depth/google-0-00121-gt-depth.png}
&

\includegraphics[trim=0 0 0 -5, width=0.4\columnwidth]{figures/more_qual_results/depth/google-1-00108-gt-depth.png}
&

\includegraphics[trim=0 0 0 -5, width=0.4\columnwidth]{figures/more_qual_results/depth/google-8-00122-gt-depth.png}
\\

\rotatebox[origin=lt]{90}{\small \ \ \ \ \ \ \ \ \ \ \ \ \ \ SVLF} &
\includegraphics[trim=0 0 0 -5, width=0.4\columnwidth]{figures/more_qual_results/depth/google-7-00143-svlf-depth.png}
&

\includegraphics[trim=0 0 0 -5, width=0.4\columnwidth]{figures/more_qual_results/depth/google-0-00121-svlf-depth.png}
&

\includegraphics[trim=0 0 0 -5, width=0.4\columnwidth]{figures/more_qual_results/depth/google-1-00108-svlf-depth.png}
&

\includegraphics[trim=0 0 0 -5, width=0.4\columnwidth]{figures/more_qual_results/depth/google-8-00122-svlf-depth.png}
\\

\rotatebox[origin=lt]{90}{\small \ \ \ \ \ \ \ \ \ \ Ins.-NGP} &
\includegraphics[trim=0 0 0 -5, width=0.4\columnwidth]{figures/more_qual_results/depth/google-7-00143-hash-depth.png}
&

\includegraphics[trim=0 0 0 -5, width=0.4\columnwidth]{figures/more_qual_results/depth/google-0-00121-hash-depth.png}
&

\includegraphics[trim=0 0 0 -5, width=0.4\columnwidth]{figures/more_qual_results/depth/google-1-00108-hash-depth.png}
&

\includegraphics[trim=0 0 0 -5, width=0.4\columnwidth]{figures/more_qual_results/depth/google-8-00122-hash-depth.png}
\\

\rotatebox[origin=lt]{90}{\small \ \ \ \ \ \ \ \ \ mip-NeRF} &
\includegraphics[trim=0 0 0 -5, width=0.4\columnwidth]{figures/more_qual_results/depth/google-7-00143-mip-depth.png}
&

\includegraphics[trim=0 0 0 -5, width=0.4\columnwidth]{figures/more_qual_results/depth/google-0-00121-mip-depth.png}
&

\includegraphics[trim=0 0 0 -5, width=0.4\columnwidth]{figures/more_qual_results/depth/google-1-00108-mip-depth.png}
&

\includegraphics[trim=0 0 0 -5, width=0.4\columnwidth]{figures/more_qual_results/depth/google-8-00122-mip-depth.png}
\\

\rotatebox[origin=lt]{90}{\small \ \ \ \ \ \ \ \ \ \ \ \ \ \ NeRF} &
\includegraphics[trim=0 0 0 -5, width=0.4\columnwidth]{figures/more_qual_results/depth/google-7-00143-nerf-depth.png}
&

\includegraphics[trim=0 0 0 -5, width=0.4\columnwidth]{figures/more_qual_results/depth/google-0-00121-nerf-depth.png}
&

\includegraphics[trim=0 0 0 -5, width=0.4\columnwidth]{figures/more_qual_results/depth/google-1-00108-nerf-depth.png}
&

\includegraphics[trim=0 0 0 -5, width=0.4\columnwidth]{figures/more_qual_results/depth/google-8-00122-nerf-depth.png}
\\

\end{tabular}
\caption{Depth qualitative results on the Google Scanned Objects environment (figure best seen in zoom).}
\label{fig:more_qual_results_1_depth}
\end{figure*}


\begin{figure*}
\centering
\begin{tabular}{cc|c|c|c}
\rotatebox[origin=lt]{90}{\small \ \ \ \ \ \ \ \ \ Ground truth} &
\includegraphics[trim=0 0 0 -5, width=0.4\columnwidth]{figures/more_qual_results/abc-6-00143-gt-rgb.png}
&

\includegraphics[trim=0 0 0 -5, width=0.4\columnwidth]{figures/more_qual_results/abc-7-00109-gt-rgb.png}
&

\includegraphics[trim=0 0 0 -5, width=0.4\columnwidth]{figures/more_qual_results/abc-8-00142-gt-rgb.png}
&

\includegraphics[trim=0 0 0 -5, width=0.4\columnwidth]{figures/more_qual_results/abc-9-00109-gt-rgb.png}
\\

\rotatebox[origin=lt]{90}{\small \ \ \ \ \ \ \ \ \ \ \ \ \ \ SVLF} &
\includegraphics[trim=0 0 0 -5, width=0.4\columnwidth]{figures/more_qual_results/abc-6-00143-svlf-rgb.png}
&

\includegraphics[trim=0 0 0 -5, width=0.4\columnwidth]{figures/more_qual_results/abc-7-00109-svlf-rgb.png}
&

\includegraphics[trim=0 0 0 -5, width=0.4\columnwidth]{figures/more_qual_results/abc-8-00142-svlf-rgb.png}
&

\includegraphics[trim=0 0 0 -5, width=0.4\columnwidth]{figures/more_qual_results/abc-9-00109-svlf-rgb.png}
\\

\rotatebox[origin=lt]{90}{\small \ \ \ \ \ \ \ \ \ \ Ins.-NGP} &
\includegraphics[trim=0 0 0 -5, width=0.4\columnwidth]{figures/more_qual_results/abc-6-00143-hash-rgb.png}
&

\includegraphics[trim=0 0 0 -5, width=0.4\columnwidth]{figures/more_qual_results/abc-7-00109-hash-rgb.png}
&

\includegraphics[trim=0 0 0 -5, width=0.4\columnwidth]{figures/more_qual_results/abc-8-00142-hash-rgb.png}
&

\includegraphics[trim=0 0 0 -5, width=0.4\columnwidth]{figures/more_qual_results/abc-9-00109-hash-rgb.png}
\\

\rotatebox[origin=lt]{90}{\small \ \ \ \ \ \ \ \ \ mip-NeRF} &
\includegraphics[trim=0 0 0 -5, width=0.4\columnwidth]{figures/more_qual_results/abc-6-00143-mip-rgb.png}
&

\includegraphics[trim=0 0 0 -5, width=0.4\columnwidth]{figures/more_qual_results/abc-7-00109-mip-rgb.png}
&

\includegraphics[trim=0 0 0 -5, width=0.4\columnwidth]{figures/more_qual_results/abc-8-00142-mip-rgb.png}
&

\includegraphics[trim=0 0 0 -5, width=0.4\columnwidth]{figures/more_qual_results/abc-9-00109-mip-rgb.png}
\\

\rotatebox[origin=lt]{90}{\small \ \ \ \ \ \ \ \ \ \ \ \ \ \ NeRF} &
\includegraphics[trim=0 0 0 -5, width=0.4\columnwidth]{figures/more_qual_results/abc-6-00143-nerf-rgb.png}
&

\includegraphics[trim=0 0 0 -5, width=0.4\columnwidth]{figures/more_qual_results/abc-7-00109-nerf-rgb.png}
&

\includegraphics[trim=0 0 0 -5, width=0.4\columnwidth]{figures/more_qual_results/abc-8-00142-nerf-rgb.png}
&

\includegraphics[trim=0 0 0 -5, width=0.4\columnwidth]{figures/more_qual_results/abc-9-00109-nerf-rgb.png}
\\

\end{tabular}
\caption{More qualitative results on the ABC environment (figure best seen in zoom).}
\label{fig:more_qual_results_2}
\end{figure*}


\begin{figure*}
\centering
\begin{tabular}{cc|c|c|c}
\rotatebox[origin=lt]{90}{\small \ \ \ \ \ \ \ \ \ Ground truth} &
\includegraphics[trim=0 0 0 -5, width=0.4\columnwidth]{figures/more_qual_results/depth/abc-6-00143-gt-depth.png}
&

\includegraphics[trim=0 0 0 -5, width=0.4\columnwidth]{figures/more_qual_results/depth/abc-7-00109-gt-depth.png}
&

\includegraphics[trim=0 0 0 -5, width=0.4\columnwidth]{figures/more_qual_results/depth/abc-8-00142-gt-depth.png}
&

\includegraphics[trim=0 0 0 -5, width=0.4\columnwidth]{figures/more_qual_results/depth/abc-9-00109-gt-depth.png}
\\

\rotatebox[origin=lt]{90}{\small \ \ \ \ \ \ \ \ \ \ \ \ \ \ SVLF} &
\includegraphics[trim=0 0 0 -5, width=0.4\columnwidth]{figures/more_qual_results/depth/abc-6-00143-svlf-depth.png}
&

\includegraphics[trim=0 0 0 -5, width=0.4\columnwidth]{figures/more_qual_results/depth/abc-7-00109-svlf-depth.png}
&

\includegraphics[trim=0 0 0 -5, width=0.4\columnwidth]{figures/more_qual_results/depth/abc-8-00142-svlf-depth.png}
&

\includegraphics[trim=0 0 0 -5, width=0.4\columnwidth]{figures/more_qual_results/depth/abc-9-00109-svlf-depth.png}
\\

\rotatebox[origin=lt]{90}{\small \ \ \ \ \ \ \ \ \ \ Ins.-NGP} &
\includegraphics[trim=0 0 0 -5, width=0.4\columnwidth]{figures/more_qual_results/depth/abc-6-00143-hash-depth.png}
&

\includegraphics[trim=0 0 0 -5, width=0.4\columnwidth]{figures/more_qual_results/depth/abc-7-00109-hash-depth.png}
&

\includegraphics[trim=0 0 0 -5, width=0.4\columnwidth]{figures/more_qual_results/depth/abc-8-00142-hash-depth.png}
&

\includegraphics[trim=0 0 0 -5, width=0.4\columnwidth]{figures/more_qual_results/depth/abc-9-00109-hash-depth.png}
\\

\rotatebox[origin=lt]{90}{\small \ \ \ \ \ \ \ \ \ mip-NeRF} &
\includegraphics[trim=0 0 0 -5, width=0.4\columnwidth]{figures/more_qual_results/depth/abc-6-00143-mip-depth.png}
&

\includegraphics[trim=0 0 0 -5, width=0.4\columnwidth]{figures/more_qual_results/depth/abc-7-00109-mip-depth.png}
&

\includegraphics[trim=0 0 0 -5, width=0.4\columnwidth]{figures/more_qual_results/depth/abc-8-00142-mip-depth.png}
&

\includegraphics[trim=0 0 0 -5, width=0.4\columnwidth]{figures/more_qual_results/depth/abc-9-00109-mip-depth.png}
\\

\rotatebox[origin=lt]{90}{\small \ \ \ \ \ \ \ \ \ \ \ \ \ \ NeRF} &
\includegraphics[trim=0 0 0 -5, width=0.4\columnwidth]{figures/more_qual_results/depth/abc-6-00143-nerf-depth.png}
&

\includegraphics[trim=0 0 0 -5, width=0.4\columnwidth]{figures/more_qual_results/depth/abc-7-00109-nerf-depth.png}
&

\includegraphics[trim=0 0 0 -5, width=0.4\columnwidth]{figures/more_qual_results/depth/abc-8-00142-nerf-depth.png}
&

\includegraphics[trim=0 0 0 -5, width=0.4\columnwidth]{figures/more_qual_results/depth/abc-9-00109-nerf-depth.png}
\\

\end{tabular}
\caption{Depth qualitative results on the ABC environment (figure best seen in zoom).}
\label{fig:more_qual_results_2_depth}
\end{figure*}


\begin{figure*}
\centering
\begin{tabular}{cc|c|c|c}
\rotatebox[origin=lt]{90}{\small \ \ \ \ \ \ \ \ \ Ground truth} &
\includegraphics[trim=0 0 0 -5, width=0.4\columnwidth]{figures/more_qual_results/lego-Action-00117-gt-rgb.png}
&

\includegraphics[trim=0 0 0 -5, width=0.4\columnwidth]{figures/more_qual_results/lego-Four-00125-gt-rgb.png}
&

\includegraphics[trim=0 0 0 -5, width=0.4\columnwidth]{figures/more_qual_results/lego-NASA-00107-gt-rgb.png}
&

\includegraphics[trim=0 0 0 -5, width=0.4\columnwidth]{figures/more_qual_results/lego-Night-00128-gt-rgb.png}
\\

\rotatebox[origin=lt]{90}{\small \ \ \ \ \ \ \ \ \ \ \ \ \ \ SVLF} &
\includegraphics[trim=0 0 0 -5, width=0.4\columnwidth]{figures/more_qual_results/lego-Action-00117-svlf-rgb.png}
&

\includegraphics[trim=0 0 0 -5, width=0.4\columnwidth]{figures/more_qual_results/lego-Four-00125-svlf-rgb.png}
&

\includegraphics[trim=0 0 0 -5, width=0.4\columnwidth]{figures/more_qual_results/lego-NASA-00107-svlf-rgb.png}
&

\includegraphics[trim=0 0 0 -5, width=0.4\columnwidth]{figures/more_qual_results/lego-Night-00128-svlf-rgb.png}
\\

\rotatebox[origin=lt]{90}{\small \ \ \ \ \ \ \ \ \ \ Ins.-NGP} &
\includegraphics[trim=0 0 0 -5, width=0.4\columnwidth]{figures/more_qual_results/lego-Action-00117-hash-rgb.png}
&

\includegraphics[trim=0 0 0 -5, width=0.4\columnwidth]{figures/more_qual_results/lego-Four-00125-hash-rgb.png}
&

\includegraphics[trim=0 0 0 -5, width=0.4\columnwidth]{figures/more_qual_results/lego-NASA-00107-hash-rgb.png}
&

\includegraphics[trim=0 0 0 -5, width=0.4\columnwidth]{figures/more_qual_results/lego-Night-00128-hash-rgb.png}
\\

\rotatebox[origin=lt]{90}{\small \ \ \ \ \ \ \ \ \ mip-NeRF} &
\includegraphics[trim=0 0 0 -5, width=0.4\columnwidth]{figures/more_qual_results/lego-Action-00117-mip-rgb.png}
&

\includegraphics[trim=0 0 0 -5, width=0.4\columnwidth]{figures/more_qual_results/lego-Four-00125-mip-rgb.png}
&

\includegraphics[trim=0 0 0 -5, width=0.4\columnwidth]{figures/more_qual_results/lego-NASA-00107-mip-rgb.png}
&

\includegraphics[trim=0 0 0 -5, width=0.4\columnwidth]{figures/more_qual_results/lego-Night-00128-mip-rgb.png}
\\

\rotatebox[origin=lt]{90}{\small \ \ \ \ \ \ \ \ \ \ \ \ \ \ NeRF} &
\includegraphics[trim=0 0 0 -5, width=0.4\columnwidth]{figures/more_qual_results/lego-Action-00117-nerf-rgb.png}
&

\includegraphics[trim=0 0 0 -5, width=0.4\columnwidth]{figures/more_qual_results/lego-Four-00125-nerf-rgb.png}
&

\includegraphics[trim=0 0 0 -5, width=0.4\columnwidth]{figures/more_qual_results/lego-NASA-00107-nerf-rgb.png}
&

\includegraphics[trim=0 0 0 -5, width=0.4\columnwidth]{figures/more_qual_results/lego-Night-00128-nerf-rgb.png}
\\

\end{tabular}
\caption{More qualitative results on the Bricks environment (figure best seen in zoom).}
\label{fig:more_qual_results_3}
\end{figure*}


\begin{figure*}
\centering
\begin{tabular}{cc|c|c|c}
\rotatebox[origin=lt]{90}{\small \ \ \ \ \ \ \ \ \ Ground truth} &
\includegraphics[trim=0 0 0 -5, width=0.4\columnwidth]{figures/more_qual_results/depth/lego-Action-00117-gt-depth.png}
&

\includegraphics[trim=0 0 0 -5, width=0.4\columnwidth]{figures/more_qual_results/depth/lego-Four-00125-gt-depth.png}
&

\includegraphics[trim=0 0 0 -5, width=0.4\columnwidth]{figures/more_qual_results/depth/lego-NASA-00107-gt-depth.png}
&

\includegraphics[trim=0 0 0 -5, width=0.4\columnwidth]{figures/more_qual_results/depth/lego-Night-00128-gt-depth.png}
\\

\rotatebox[origin=lt]{90}{\small \ \ \ \ \ \ \ \ \ \ \ \ \ \ SVLF} &
\includegraphics[trim=0 0 0 -5, width=0.4\columnwidth]{figures/more_qual_results/depth/lego-Action-00117-svlf-depth.png}
&

\includegraphics[trim=0 0 0 -5, width=0.4\columnwidth]{figures/more_qual_results/depth/lego-Four-00125-svlf-depth.png}
&

\includegraphics[trim=0 0 0 -5, width=0.4\columnwidth]{figures/more_qual_results/depth/lego-NASA-00107-svlf-depth.png}
&

\includegraphics[trim=0 0 0 -5, width=0.4\columnwidth]{figures/more_qual_results/depth/lego-Night-00128-svlf-depth.png}
\\

\rotatebox[origin=lt]{90}{\small \ \ \ \ \ \ \ \ \ \ Ins.-NGP} &
\includegraphics[trim=0 0 0 -5, width=0.4\columnwidth]{figures/more_qual_results/depth/lego-Action-00117-hash-depth.png}
&

\includegraphics[trim=0 0 0 -5, width=0.4\columnwidth]{figures/more_qual_results/depth/lego-Four-00125-hash-depth.png}
&

\includegraphics[trim=0 0 0 -5, width=0.4\columnwidth]{figures/more_qual_results/depth/lego-NASA-00107-hash-depth.png}
&

\includegraphics[trim=0 0 0 -5, width=0.4\columnwidth]{figures/more_qual_results/depth/lego-Night-00128-hash-depth.png}
\\

\rotatebox[origin=lt]{90}{\small \ \ \ \ \ \ \ \ \ mip-NeRF} &
\includegraphics[trim=0 0 0 -5, width=0.4\columnwidth]{figures/more_qual_results/depth/lego-Action-00117-mip-depth.png}
&

\includegraphics[trim=0 0 0 -5, width=0.4\columnwidth]{figures/more_qual_results/depth/lego-Four-00125-mip-depth.png}
&

\includegraphics[trim=0 0 0 -5, width=0.4\columnwidth]{figures/more_qual_results/depth/lego-NASA-00107-mip-depth.png}
&

\includegraphics[trim=0 0 0 -5, width=0.4\columnwidth]{figures/more_qual_results/depth/lego-Night-00128-mip-depth.png}
\\

\rotatebox[origin=lt]{90}{\small \ \ \ \ \ \ \ \ \ \ \ \ \ \ NeRF} &
\includegraphics[trim=0 0 0 -5, width=0.4\columnwidth]{figures/more_qual_results/depth/lego-Action-00117-nerf-depth.png}
&

\includegraphics[trim=0 0 0 -5, width=0.4\columnwidth]{figures/more_qual_results/depth/lego-Four-00125-nerf-depth.png}
&

\includegraphics[trim=0 0 0 -5, width=0.4\columnwidth]{figures/more_qual_results/depth/lego-NASA-00107-nerf-depth.png}
&

\includegraphics[trim=0 0 0 -5, width=0.4\columnwidth]{figures/more_qual_results/depth/lego-Night-00128-nerf-depth.png}
\\

\end{tabular}
\caption{Depth qualitative results on the Bricks environment (figure best seen in zoom).}
\label{fig:more_qual_results_3_depth}
\end{figure*}


\begin{figure*}
\centering
\begin{tabular}{cc|c|c|c}
\rotatebox[origin=lt]{90}{\small \ \ \ \ \ \ \ \ \ Ground truth} &
\includegraphics[trim=0 0 0 -5, width=0.4\columnwidth]{figures/more_qual_results/amazon-5-00129-gt-rgb.png}
&

\includegraphics[trim=0 0 0 -5, width=0.4\columnwidth]{figures/more_qual_results/amazon-7-00124-gt-rgb.png}
&

\includegraphics[trim=0 0 0 -5, width=0.4\columnwidth]{figures/more_qual_results/amazon-8-00106-gt-rgb.png}
&

\includegraphics[trim=0 0 0 -5, width=0.4\columnwidth]{figures/more_qual_results/amazon-9-00111-gt-rgb.png}
\\

\rotatebox[origin=lt]{90}{\small \ \ \ \ \ \ \ \ \ \ \ \ \ \ SVLF} &
\includegraphics[trim=0 0 0 -5, width=0.4\columnwidth]{figures/more_qual_results/amazon-5-00129-svlf-rgb.png}
&

\includegraphics[trim=0 0 0 -5, width=0.4\columnwidth]{figures/more_qual_results/amazon-7-00124-svlf-rgb.png}
&

\includegraphics[trim=0 0 0 -5, width=0.4\columnwidth]{figures/more_qual_results/amazon-8-00106-svlf-rgb.png}
&

\includegraphics[trim=0 0 0 -5, width=0.4\columnwidth]{figures/more_qual_results/amazon-9-00111-svlf-rgb.png}
\\

\rotatebox[origin=lt]{90}{\small \ \ \ \ \ \ \ \ \ \ Ins.-NGP} &
\includegraphics[trim=0 0 0 -5, width=0.4\columnwidth]{figures/more_qual_results/amazon-5-00129-hash-rgb.png}
&

\includegraphics[trim=0 0 0 -5, width=0.4\columnwidth]{figures/more_qual_results/amazon-7-00124-hash-rgb.png}
&

\includegraphics[trim=0 0 0 -5, width=0.4\columnwidth]{figures/more_qual_results/amazon-8-00106-hash-rgb.png}
&

\includegraphics[trim=0 0 0 -5, width=0.4\columnwidth]{figures/more_qual_results/amazon-9-00111-hash-rgb.png}
\\

\rotatebox[origin=lt]{90}{\small \ \ \ \ \ \ \ \ \ mip-NeRF} &
\includegraphics[trim=0 0 0 -5, width=0.4\columnwidth]{figures/more_qual_results/amazon-5-00129-mip-rgb.png}
&

\includegraphics[trim=0 0 0 -5, width=0.4\columnwidth]{figures/more_qual_results/amazon-7-00124-mip-rgb.png}
&

\includegraphics[trim=0 0 0 -5, width=0.4\columnwidth]{figures/more_qual_results/amazon-8-00106-mip-rgb.png}
&

\includegraphics[trim=0 0 0 -5, width=0.4\columnwidth]{figures/more_qual_results/amazon-9-00111-mip-rgb.png}
\\

\rotatebox[origin=lt]{90}{\small \ \ \ \ \ \ \ \ \ \ \ \ \ \ NeRF} &
\includegraphics[trim=0 0 0 -5, width=0.4\columnwidth]{figures/more_qual_results/amazon-5-00129-nerf-rgb.png}
&

\includegraphics[trim=0 0 0 -5, width=0.4\columnwidth]{figures/more_qual_results/amazon-7-00124-nerf-rgb.png}
&

\includegraphics[trim=0 0 0 -5, width=0.4\columnwidth]{figures/more_qual_results/amazon-8-00106-nerf-rgb.png}
&

\includegraphics[trim=0 0 0 -5, width=0.4\columnwidth]{figures/more_qual_results/amazon-9-00111-nerf-rgb.png}
\\

\end{tabular}
\caption{More qualitative results on the Amazon Berkeley environment (figure best seen in zoom).}
\label{fig:more_qual_results_4}
\end{figure*}


\begin{figure*}
\centering
\begin{tabular}{cc|c|c|c}
\rotatebox[origin=lt]{90}{\small \ \ \ \ \ \ \ \ \ Ground truth} &
\includegraphics[trim=0 0 0 -5, width=0.4\columnwidth]{figures/more_qual_results/depth/amazon-5-00129-gt-depth.png}
&

\includegraphics[trim=0 0 0 -5, width=0.4\columnwidth]{figures/more_qual_results/depth/amazon-7-00124-gt-depth.png}
&

\includegraphics[trim=0 0 0 -5, width=0.4\columnwidth]{figures/more_qual_results/depth/amazon-8-00106-gt-depth.png}
&

\includegraphics[trim=0 0 0 -5, width=0.4\columnwidth]{figures/more_qual_results/depth/amazon-9-00111-gt-depth.png}
\\

\rotatebox[origin=lt]{90}{\small \ \ \ \ \ \ \ \ \ \ \ \ \ \ SVLF} &
\includegraphics[trim=0 0 0 -5, width=0.4\columnwidth]{figures/more_qual_results/depth/amazon-5-00129-svlf-depth.png}
&

\includegraphics[trim=0 0 0 -5, width=0.4\columnwidth]{figures/more_qual_results/depth/amazon-7-00124-svlf-depth.png}
&

\includegraphics[trim=0 0 0 -5, width=0.4\columnwidth]{figures/more_qual_results/depth/amazon-8-00106-svlf-depth.png}
&

\includegraphics[trim=0 0 0 -5, width=0.4\columnwidth]{figures/more_qual_results/depth/amazon-9-00111-svlf-depth.png}
\\

\rotatebox[origin=lt]{90}{\small \ \ \ \ \ \ \ \ \ \ Ins.-NGP} &
\includegraphics[trim=0 0 0 -5, width=0.4\columnwidth]{figures/more_qual_results/depth/amazon-5-00129-hash-depth.png}
&

\includegraphics[trim=0 0 0 -5, width=0.4\columnwidth]{figures/more_qual_results/depth/amazon-7-00124-hash-depth.png}
&

\includegraphics[trim=0 0 0 -5, width=0.4\columnwidth]{figures/more_qual_results/depth/amazon-8-00106-hash-depth.png}
&

\includegraphics[trim=0 0 0 -5, width=0.4\columnwidth]{figures/more_qual_results/depth/amazon-9-00111-hash-depth.png}
\\

\rotatebox[origin=lt]{90}{\small \ \ \ \ \ \ \ \ \ mip-NeRF} &
\includegraphics[trim=0 0 0 -5, width=0.4\columnwidth]{figures/more_qual_results/depth/amazon-5-00129-mip-depth.png}
&

\includegraphics[trim=0 0 0 -5, width=0.4\columnwidth]{figures/more_qual_results/depth/amazon-7-00124-mip-depth.png}
&

\includegraphics[trim=0 0 0 -5, width=0.4\columnwidth]{figures/more_qual_results/depth/amazon-8-00106-mip-depth.png}
&

\includegraphics[trim=0 0 0 -5, width=0.4\columnwidth]{figures/more_qual_results/depth/amazon-9-00111-mip-depth.png}
\\

\rotatebox[origin=lt]{90}{\small \ \ \ \ \ \ \ \ \ \ \ \ \ \ NeRF} &
\includegraphics[trim=0 0 0 -5, width=0.4\columnwidth]{figures/more_qual_results/depth/amazon-5-00129-nerf-depth.png}
&

\includegraphics[trim=0 0 0 -5, width=0.4\columnwidth]{figures/more_qual_results/depth/amazon-7-00124-nerf-depth.png}
&

\includegraphics[trim=0 0 0 -5, width=0.4\columnwidth]{figures/more_qual_results/depth/amazon-8-00106-nerf-depth.png}
&

\includegraphics[trim=0 0 0 -5, width=0.4\columnwidth]{figures/more_qual_results/depth/amazon-9-00111-nerf-depth.png}
\\

\end{tabular}
\caption{Depth qualitative results on the Amazon Berkeley environment (figure best seen in zoom).}
\label{fig:more_qual_results_4_depth}
\end{figure*}



