Lemmas~\ref{lem:dimreduct} and~\ref{lem:representatives} give
rise to a simple approximation algorithm.
Let  be  color classes
that each embrace the origin, and set . Then, the following algorithm 
recursively computes a -embracing -colorful choice.
First, we prune  with Lemma~\ref{lem:findminembr} and 
partition it into two sets
 of size at most . Using Lemma~\ref{lem:representatives}, we compute two
representative points  for this partition of .
Then, we project the remaining  color classes onto the
-dimensional space that is orthogonal to 
, and we recursively compute a 
-embracing -colorful choice  with respect to
the projections of . By Lemmas~\ref{lem:dimreduct}
and~\ref{lem:representatives}, one of the two sets , , 
say , is -embracing
equivalent to  with respect to . Since  is a
-colorful choice that does not contain points from  and since
, the set  is a -embracing -colorful
choice. The recursion stops once only one color class is left. Then, 
we are in dimension . Since , pruning the 
single remaining color class with
Lemma~\ref{lem:findminembr} results already in a -embracing 
-colorful choice.  For details, see Algorithm~\ref{alg:simapx}.

\begin{alg}
  \SetKwFunction{Recurse}{recurse}
  \KwIn{ sets  that each embrace 
  the origin, and for each , , the coefficients 
  of the convex combination of  with the points in }
  \KwOut{minimally -embracing 
  -colorful choice}
   prune  with Lemma~\ref{lem:findminembr}\;
  \lIf{}{\Return{}}
   partition of  into two sets, each of size at most
  \;
  Compute representative points  for \;
   orthogonal projection of
   onto \;
   \Recurse{}\;
   replace projected points in  by original points from
  \;
  Determine which point  is -embracing
  equivalent to  with Lemma~\ref{lem:dimreduct} 
  and let  be the corresponding subset of \;
  \Return{ \textup{pruned with
      Lemma~\ref{lem:findminembr}}}\;
  \caption[Computing -embracing -colorful
  choices.]{Simple Approximation}
  \label{alg:simapx}
\end{alg}

\begin{theorem}\label{thm:simpledimreduct}
  Let  be
   color classes such that  is a -embracing set
  of size , for . On input  and given the
  coefficients of the convex combination of the origin for each set ,
  Algorithm~\ref{alg:simapx} computes a -embracing -colorful choice in  time.
  In particular, for , the algorithm 
  computes a -colorful choice.
\end{theorem}
\begin{prf}
  The correctness of Algorithm~\ref{alg:simapx} is a direct consequence of
  Lemmas~\ref{lem:dimreduct} and~\ref{lem:representatives}. It remains to
  analyze the running time. In each step of the recursion except for the 
  last one, we prune two times a set of size  with
  Lemma~\ref{lem:findminembr}. This needs  time. Furthermore, by
  Lemma~\ref{lem:representatives}, computing two representative points 
  also takes  time. Finally, given the set , determining which
  representative point is -embracing equivalent to  takes also 
   by Lemma~\ref{lem:dimreduct} and using the fact that
  the recursively computed solution is minimally embracing. Thus, we 
  need  time per step of the recursion and there are 
   recursion steps in total. The total running time is 
  .
\end{prf}

Although nontrivial, the fact that we can take in polynomial time 
half of the points from each color class to construct a -embracing
-colorful choice may not be too 
surprising.  In the remainder of this section, we present a 
generalization of Algorithm~\ref{alg:simapx} that computes 
-embracing -colorful choices in 
polynomial time for any fixed . The improved approximation
guarantee is achieved by repeatedly replacing subsets of  with
Lemmas~\ref{lem:dimreduct} and~\ref{lem:representatives} in each 
step of the recursion. To still ensure polynomial running time, we 
reduce the dimensionality by a constant fraction in each step of 
the recursion. Additionally, we slightly worsen the desired 
approximation guarantee in each level of the recursion, i.e., if the
current recursion level is  and the dimensionality is , then 
we do not compute an -colorful
choice, but a -colorful
choice. As we will see, this additional ``slack'' in the approximation 
guarantee limits the recursion depth to a constant depending only on 
. 

In more detail, let  be  sets that each
embrace the origin, and let  be a parameter. We want to compute 
an -colorful choice that embraces the origin. 
Set

for . The sequence  controls the dimension reduction
argument with Lemmas~\ref{lem:dimreduct} and~\ref{lem:representatives}, 
i.e., in the th recursion level, the dimensionality of the input 
will be . The sequence  defines the approximation guarantee 
in the th recursion level.  Note that  and 
. Assume now we are in recursion 
level . That is, the input consists of 
color classes  that each embrace the
origin together with the coefficients of their convex combinations 
of the origin.
We want to compute a -embracing -colorful choice.
As in the previous algorithm, we begin by computing a minimal 
-embracing subset  of  with Lemma~\ref{lem:findminembr}. 
If , then  is already a valid approximation. 
Otherwise, we iteratively transform  into a -colorful choice. 
For this, we repeatedly replace subsets
of  with points from 
until it contains at most  points from each color. This 
is done as follows. Set . 
In the general situation,  contains points from several
color classes, and we partition  into sets  by
distributing the points from each color in  equally among 
these  sets. Then,
we compute representative points  for this partition. 
Let
 be
 color classes, where we discuss shortly how they are chosen.
We recursively compute a -colorful choice  for
 that embraces the
origin when projected on . 
Note that 
and hence the dimensionality of the input in recursion level 
 is , as desired. Then, by Lemmas~\ref{lem:dimreduct}
and~\ref{lem:representatives}, at least one representative 
point  and
hence at least one of the sets  is
-embracing equivalent to . We set  to  and prune it with Lemma~\ref{lem:findminembr}. 
We repeat these steps until  is a -colorful choice. 

To ensure progress,  should be smaller than  so that 
 is guaranteed to contain a point from each 
color that appears more than  times
in . Furthermore,  should not contain points with colors that appear
``often'' in . We call a color class  \emph{light} with 
respect to  if , and 
\emph{heavy}, otherwise. For the recursion, we use only light 
color classes. A -colorful choice with light colors can be 
added safely to  without increasing any color over the
threshold . In particular, since we start with  and 
use only light color classes, no other color class can ever occur 
more than  times in  and hence we are finished once the 
number of points from  is at most .
Please refer to Algorithm~\ref{alg:mainapx} for details.

\begin{alg}
  \SetKwFunction{Recurse}{recurse}
  \KwIn{recursion depth  (initially ), original 
  dimension , approximation parameter ,  sets
   that each embrace the origin, 
  and for each  the coefficients of the convex
  combination of  with the points in 
  }
  \KwOut{minimally -embracing -colorful choice}
  \;
  \;
  \;
   prune  with
  Lemma~\ref{lem:findminembr}\;\label{alg:mainapx:initprune}
  \While{}{\label{alg:mainapx:while}
     partition of  s.t.\ the
    points from each color class are evenly 
    distributed\;\label{alg:mainapx:part}
    Compute representative points  for
         with
        Lemma~\ref{lem:representatives}\;\label{alg:mainapx:representatives}
    Find  light color classes 
    \;
     orthogonal projection of
     onto
        \;\label{alg:mainapx:proj}
    \Recurse{, , , }\;\label{alg:mainapx:rec}
     replace projected points in  by original 
    points from
    \;
    Determine which point  
    is -embracing equivalent to  with Lemma~\ref{lem:dimreduct}\;
     pruned with
        Lemma~\ref{lem:findminembr}\;
  }
  \Return{}\;\label{alg:mainapx:return}
  \caption[Computing -embracing -colorful
  choices.]{-Approximation}
  \label{alg:mainapx}
\end{alg}

The next lemma states that for  fixed, the number of necessary 
recursions before a trivial approximation with 
Lemma~\ref{lem:findminembr} suffices is constant.

\begin{lemma}\label{lem:recdepth}
  For any  there exists a 
   such that .
\end{lemma}
\begin{prf}
Replacing  with its definition, we obtain

Using  if , we 
have for
,

Furthermore, using that , we have for 

Combining~\eq{lem:recdepth:geq1} and~\eq{lem:recdepth:eps}
with~\eq{lem:recdepth:dj}, we get

For , there is a  with
. 
The claim follows.
\end{prf}

Next, we show that if the recursion depth is not too large, then we can
always find enough light color classes.

\begin{lemma}\label{lem:light}
  Let  and let  be 
   color classes.
  Furthermore, let  be a set 
  of size at most . For all , 
  there exist  light color classes with respect to .
\end{lemma}
\begin{prf}
We recall that a color class , , is light with 
respect to  if . Then, the 
number of heavy color classes  is bounded by

since  for all . We can bound the denominator 
as follows

where we apply  in 
the last
inequality.  Using that  if 
, we have for 

and hence~\eq{lem:light:denom} can be simplified to

Plugging (\ref{lem:light:denomfinal}) into (\ref{lem:light:heavy}) and
using~\eq{lem:light:denomsim},
we obtain

Then, the number  of light color classes is at least

For , using
 if , we have

and thus~\eq{lem:light:light1} implies

For , using
 if , we can bound

Combining~\eq{lem:light:epshalf} with (\ref{lem:light:lightinterm}), we get

Thus, for , there are at least  light color classes with respect to .
\end{prf}

Before we finally prove correctness, we show if the recursion depth 
 is not too large, then each set of the partition of  contains 
at least one point from  until  is a -colorful choice. 
This implies that each iteration of
the while-loop decreases the amount of points from  in .

\begin{lemma}\label{lem:while}
  For all , we have 
  .
\end{lemma}
\begin{prf}
  First, we upper bound  as follows:

For , 
with  if 
,
we obtain . 
Using this in (\ref{lem:while:k}), we get

as desired.
\end{prf}

\begin{theorem}
\label{thm:bapx}
  Let  be
   sets such that  is a -embracing set
  of size , for , and let  be a
  parameter. On input , , , , and given the
  coefficients of the convex combination of the origin with the points 
  in , for ,
  Algorithm~\ref{alg:mainapx} computes a -embracing 
  -colorful
  choice in  time.
\end{theorem}

\begin{prf}
We begin by showing that if the algorithm enters the while 
loop in recursion level , it is always possible to find  light color classes and that the projections
 of these color
classes are -embracing subsets of 
~(Line~\ref{alg:mainapx:proj}). In other words, we show
that recursion is possible if  is not a -colorful choice.
Assume now the algorithm enters the while loop in recursion level .
Then,  is a minimally -embracing subset of 
 and has size at least . In
Line~\ref{alg:mainapx:part}, we partition  into  sets 
 by distributing the points from each color class equally. 
By Lemma~\ref{lem:while}, we have , 
for ,
and hence each set  is nonempty. Thus,
the algorithm from Lemma~\ref{lem:representatives} can be applied in
Line~\ref{alg:mainapx:representatives}
to compute the representative points . Moreover  by Lemma~\ref{lem:representatives} and
Lemma~\ref{lem:minembr}.
Thus, .
Now, Lemma~\ref{lem:light} guarantees that we can always find  
light color classes , 
if . Because each color 
class , , is
-embracing, so are their orthogonal projections onto
. Thus, recursion is possible 
if . By Lemma~\ref{lem:recdepth}, 
the recursion depth is limited to
, since then pruning  with
Lemma~\ref{lem:findminembr} in Line~\ref{alg:mainapx:initprune} is already a
-embracing -colorful choice. In this case, the while loop is never
executed. We conclude that for , recursion is always
possible as long as  is not a -colorful choice.


Next, we prove that the algorithm computes in recursion level  a
-embracing -colorful choice. As discussed above, the
recursion terminates after  steps when
the set  from Line~\ref{alg:mainapx:initprune} is already a 
-embracing -colorful choice. If  is not already a 
valid approximation, the while loop is executed. In each 
iteration of the while loop,  is partitioned into  sets
 by distributing the points from each color 
equally among the . By Lemma~\ref{lem:while}, 
 for 
and hence each set , , contains at least one 
point from . Applying Lemmas~\ref{lem:dimreduct} 
and~\ref{lem:representatives}, one of these
sets, say , is replaced in   by a recursively computed
-colorful choice 
that is -embracing when projected onto .
Since we use in the recursion only light color classes with
respect to , and since  is not a light color class, 
each iteration of the while loop strictly decreases the number of 
points from  in .  Moreover, because  contains 
only points from light color classes and since it
is a -colorful choice, 
contains at most  points from the color classes .
Thus, after  iterations,  is a -embracing 
-colorful choice.

It remains to analyze the running time. The initial computation of  in
Line~\ref{alg:mainapx:initprune} and each iteration of the 
while loop except for the recursive call takes  time. 
Since the while loop is executed 
times and since the recursion depth is bounded by , the total running time of Algorithm~\ref{alg:mainapx} is 
.
\end{prf}

\subsection{Applications}
\label{sec:kcol:applications}

As discussed in the introduction, the main motivation for -colorful
choices is their application in polynomial-time reductions to \CCP. We
begin by presenting the proofs whose interpretation as algorithms 
results in the polynomial reductions. Then, we give precise 
bounds on the quality of the
obtained approximation algorithms for \Centerpoint, \Tverberg, and
\ColKirchberger when having access to an algorithm that on input 
color classes , each -embracing and of size at 
most , computes a -embracing -colorful choice in time .


\begin{theorem}[{Centerpoint 
theorem~\cite[Theorem~1]{Rado1946}}]\label{thm:centerpoint}
Let  be a point set. Then, there exists a point  such that for any halfspace  with , we 
have . \qed
\end{theorem}

Teng~\cite[Theorem~8.4]{Teng1991} showed that given a point set 
 and a candidate centerpoint , it is 
\coNP-complete to decide whether  is
a centerpoint of , if  is part of the input. 
For , a centerpoint is equivalent to a median of a set of 
numbers and hence can be computed in  
time~\cite{BlumFlPrRiTa1973}. Jadhav and Mukhopadhyay~\cite{JadhavMu1994} 
showed that linear time is sufficient even in two dimensions. For 
 fixed, the best known algorithm is by Chan~\cite{Chan2004} 
who showed how to compute a point with maximum Tukey depth,
a stronger notion than being a centerpoint, in expected time 
.

Although it is in general \coNP-complete to verify centerpoints, Tverberg
partitions serve as
polynomial-time checkable certificates for a subset of centerpoints.
In recent years, this property has been exploited algorithmically to derive
efficient approximation algorithms for
centerpoints~\cite{MulzerWe2013,MillerSh2010}.
The existence of Tverberg points is guaranteed by Tverberg's
theorem~\cite{Tverberg1966}.

\begin{theorem}[Tverberg's theorem~\cite{Tverberg1966}]\label{thm:tverberg}
  Let  be a point set of size . Then, there 
  always exists a
  Tverberg -partition for .
  Equivalently, let  be of size , with . 
  Then, there exists a Tverberg -partition for .
\end{theorem}

While Tverberg's first proof is quite involved, several
simplified subsequent 
proofs~\cite{Tverberg1981,TverbergVr1993,Sarkaria1992,Roudneff2001} 
have been published.
Here, we present Sarkaria's proof~\cite{Sarkaria1992} with further
simplifications by \Barany and Onn~\cite{BaranyOn1997} and Arocha
\etal~\cite{ArochaBaBrFaMo2009}.
The main tool is the next lemma that establishes a correspondence 
between the intersection of convex hulls of low-dimensional point sets
and the embrace of the origin of certain high-dimensional
point sets. It was extracted from Sarkaria's proof by Arocha
\etal~\cite{ArochaBaBrFaMo2009}. In the following, we denote with 
 the \emph{tensor product} that maps two points 
,  to the point

where  denotes the vector  scaled by the th component
of , for .
Then,  is bilinear, i.e., for all , , and , we have

and similarly, for all , , and
, we have


\begin{lemma}[Sarkaria's 
lemma~\cite{Sarkaria1992},~{\cite[Lemma~2]{ArochaBaBrFaMo2009}}]
\label{lem:sarkaria}
Let  be  point sets and 
let  be  vectors with  for  and
. For , we define

Then, the intersection of the convex hulls  
is nonempty if and only if  embraces the origin.
\end{lemma}
\begin{prf}
  Assume there is a point .
  There exist coefficients
   that sum to  such that
  .
  Consider the points , , 
  that we obtain by using the same convex coefficients for the points 
  in , i.e., set

We claim that  and thus
. Indeed, we have

using the bilinearity of  .

Assume now that  embraces the origin. 
We want to show that  is nonempty. 
Then, we can express the origin as a convex combination  with
 for  and , and . Hence, we have

again using the bilinearity of .
By the choice of , there is (up to multiplication 
with a scalar) exactly one linear dependency: .
Thus,

where  and . In particular, the last equality
implies that

Now, as for all  and , the coefficient
 is nonnegative and as the sum  is , we must have 
. Hence,
the point  is common to all convex
hulls , , .
\end{prf}

Please refer to Figure~\ref{fig:sarkaria} for an example of Sarkaria's 
lifting argument.  Little work is now left to obtain Tverberg's theorem 
from Lemma~\ref{lem:sarkaria} and the colorful \Caratheodory theorem.
\begin{figure}[htbp]
  \begin{center}
    \includegraphics{sarkaria.pdf}
  \end{center}
  \caption[Example of Sarkaria's Lemma.]{An example of Sarkaria's lemma for
   and . The set 
  consists of the red points and the set  consists of the blue points.
  Since the convex hulls of  and  intersect, the lifted points 
  embrace the origin.}
  \label{fig:sarkaria}
\end{figure}

\begin{prf}[Proof of Theorem~\ref{thm:tverberg}]\label{thm:tverberg:proof}
Let  be a point set of size
 and let  denote  copies of .
For each set , , we construct a -dimensional set  as in Lemma~\ref{lem:sarkaria}, i.e.,

For , we denote with 

the set of points  that 
correspond to ,
and we color these points with color . For , note that
Lemma~\ref{lem:sarkaria} applied to  copies of the singleton set 
 guarantees that the color class  embraces the origin. Hence, we have  color classes
 that embrace the origin in . Now,
by Theorem~\ref{thm:colcara}, there is a
colorful choice  with  that embraces the
origin, too. Because  embraces the origin,
Lemma~\ref{lem:sarkaria} guarantees that the convex hulls of the sets 
, , 
have a point in common.
Moreover, since all points in 
that correspond to the same point in  have the same
color, each point  appears in exactly one set 
, .
Thus,  is a Tverberg -partition of .
\end{prf}

Even less effort is required to obtain the colorful Kirchberger theorem 
from Lemma~\ref{lem:sarkaria}. Let  be two point sets.
Kirchberger's theorem~\cite{Kirchberger1903} states that if for all 
subsets  of size at most ,
the sets  and  have an empty 
intersection, then  and  have an empty intersection.
Arocha~\etal~\cite{ArochaBaBrFaMo2009} presented a generalization based 
on the colorful \Caratheodory theorem.\footnote{Actually, Arocha~\etal 
present an even stronger result (the ``very colorful Kirchberger
theorem''~\cite[Theorem~3]{ArochaBaBrFaMo2009}) using a generalization 
of the colorful \Caratheodory theorem. Here, we consider the weaker 
version that can be obtained from Theorem~\ref{thm:colcara}.}

\begin{theorem}[Colorful Kirchberger theorem~{\cite[special case of
Theorem~3]{ArochaBaBrFaMo2009}}]\label{thm:colkirchberger}
Let  be  pairwise disjoint 
color classes and let  
denote a Tverberg
  -partition for , where . Then, there exists a colorful
  choice , , such that the family of sets

  is a Tverberg -partition for .
\end{theorem}
\begin{prf}
  We lift each Tverberg partition to  as in 
  Lemma~\ref{lem:sarkaria}: for  and , we 
  denote with  the set

By Lemma~\ref{lem:sarkaria} and since each set 
, , is a Tverberg
partition, the sets , ,
embrace the origin. We color the points in  with color . Since
there are  color classes that embrace the origin in  dimensions,
Theorem~\ref{thm:colcara} guarantees the existence of a colorful 
choice  that
embraces the origin. For , let  denote all points from a th 
element in a Tverberg partition in . Since 
 embraces
the origin, Lemma~\ref{lem:sarkaria} implies that the convex hulls of
the sets 
 have a nonempty intersection. 
Further, since for , the set
 is a subset of , we have . Moreover, since all points
that correspond to the Tverberg partition , , have
color , exactly one of the sets  contains a point 
from . The colorful choice  can be obtained by projecting 
 down to .
\end{prf}

We now give precise bounds on the quality and the running time of 
approximation algorithms obtained by
combining algorithms for -colorful choices with the presented
reductions to \CCP. Unfortunately, the approximation
guarantee of Algorithm~\ref{alg:mainapx} is too weak to obtain a nontrivial
approximation algorithm for \Tverberg and therefore also for \Centerpoint. 
On the positive side, it leads to a nontrivial approximation algorithm for
\ColKirchberger.

In the following, let  be an algorithm that, when given  
color classes
, each embracing the origin and of size
, and for each  the coefficients of the convex combination 
of the origin, outputs a -embracing -colorful choice in 
 time, where  are arbitrary but fixed functions.

\begin{corollary}\label{cor:app:tverberg}
  Let  be a point set of size  and let  
  be as above.  Set
  
  Then, a Tverberg -partition  of
   and a point  can be computed in total time .
\end{corollary}
\begin{prf}
\newcommand{\kch}[1]{\widetilde{#1}}
Set .
In the proof of Theorem~\ref{thm:tverberg},
we lift  copies of  with Lemma~\ref{lem:sarkaria} to . 
Lifting one
point needs  time and hence lifting all  copies takes
 time. Then, each point  from 
corresponds to a color class 
of size  and a -embracing colorful choice of 
corresponds to the Tverberg partition  that 
we obtain
by assigning  to  if .
By construction of the color classes in the proof of
Theorem~\ref{thm:tverberg}, the barycenter of  is the origin, for 
.
Since we know then for each color class the coefficients of the convex
combination of
the origin, we can apply  to
obtain a -embracing -colorful choice  together with the coefficients of the convex 
combination of
the origin with the points in .
Let
 be a family of
subsets of  that we construct as before by assigning  to 
 if
. Here,  is a multiset, i.e., we allow
 for . Since  embraces the origin, Lemma~\ref{lem:sarkaria}
guarantees that the intersection  is
nonempty. Moreover, because we know the coefficients of the convex 
combination of the origin with the points in , we can 
compute in  time a
point  together with the
coefficients of the convex combination of  with the points in
 for , as described in the proof of 
Lemma~\ref{lem:sarkaria}.

Now, we construct a Tverberg partition for  out of
 by a greedy strategy that iteratively
removes sets from . Let  be some set and remove it from . 
Since we know the coefficients of the convex
combination of  with the points in ,
Lemma~\ref{lem:constr_cara} can be applied to prune 
to a -embracing set of size at most  in 
 time. Then, for each point , we remove the at most  other sets from
 that contain . We continue with the next set
in  that has not yet been removed until . Let  be the family of sets that we obtain by this process.
Clearly,  is a Tverberg partition and 
because , we have .
Moreover, for each set
, we remove at most 
other sets from . Thus, the size of the Tverberg partition
 is at least


Constructing the \CCP instance takes  time.
Using , we need  time to compute a -colorful choice
. Pruning every set of  with 
Lemma~\ref{lem:constr_cara} to
at most  points needs  time.
Finally, constructing
 out of  takes  time with the naive
algorithm. This results in the claimed running time of
.
\end{prf}

Furthermore, we can use  to approximate \ColKirchberger.

\begin{corollary}\label{cor:app:colkirchberger}
Let  be as above and let  be
 pairwise disjoint color
classes that are each of size . Furthermore, for , let
 denote a Tverberg
-partition for .
Then, given for each Tverberg partition , , a 
point , and for all  and 
, the coefficients of the
convex combination of  with the points in ,
we can compute in  time a
-colorful choice  such that

is a Tverberg -partition for .
\end{corollary}
\begin{prf}
  In the proof of Theorem~\ref{thm:colkirchberger},
  we lift the points  to  such that the
  set of points  that corresponds to the color class 
  still
  embraces the origin, where . Moreover, if  is a -embracing colorful choice of the 
  lifted points, then there is a corresponding colorful choice  
  with respect to  such that

is a Tverberg -partition for .
Similarly, a -embracing -colorful choice  of the
lifted color classes corresponds to a -colorful choice  with 
respect to  such that

is a Tverberg -partition for .

Computing the tensor product , where 
 and , needs  time and hence
lifting the point sets  to  with
Lemma~\ref{lem:sarkaria} needs  time in total.
Since we know for each Tverberg partition , , a point
 together with the
coefficients of the convex combination of  with the points in 
 for , we can compute in  time the 
coefficients of the convex combination of the origin with the points 
in  as described in the proof of Lemma~\ref{lem:sarkaria}.
Then,  can be applied to compute a -embracing -colorful
choice  with respect to the lifted point sets in
 time. Finally, constructing  and  out of
 needs
 time. Hence, the total time needed is .
\end{prf}

Now, given  color classes  that 
embrace the origin, we can compute with Algorithm~\ref{alg:mainapx} an 
-colorful choice that embraces
the origin in polynomial time. Combining this
with Corollary~\ref{cor:app:tverberg},
we obtain an algorithm that computes Tverberg partitions of size 
 in polynomial time, a trivial result.
However, combining Algorithm~\ref{alg:mainapx} with
Corollary~\ref{cor:app:colkirchberger}, we do obtain a nontrivial 
approximation algorithm for \ColKirchberger: given  color classes , each of size , and for each
color class a Tverberg -partition 
 together with a
point  and the coefficients 
of the convex combination of  with the points in , for 
all , we can compute
in  time
an -colorful choice  such that

is a Tverberg -partition for , where  is arbitrary but 
fixed.
