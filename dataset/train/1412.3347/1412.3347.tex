Lemmas~\ref{lem:dimreduct} and~\ref{lem:representatives} give
rise to a simple approximation algorithm.
Let $C_1,\dots,C_m \subset \R^d$ be $m$ color classes
that each embrace the origin, and set $k=\max\lt(d-m+2,
\lt\lceil\frac{d+1}{2}\rt\rceil\rt)$. Then, the following algorithm 
recursively computes a $\0$-embracing $k$-colorful choice.
First, we prune $C_1$ with Lemma~\ref{lem:findminembr} and 
partition it into two sets
$C',\,C''$ of size at most $\lt\lceil
(d+1) / 2\rt\rceil$. Using Lemma~\ref{lem:representatives}, we compute two
representative points $\rr',\, \rr''$ for this partition of $C_1$.
Then, we project the remaining $m-1$ color classes onto the
$(d-1)$-dimensional space that is orthogonal to 
$\lsp(\rr',\rr'')^\perp$, and we recursively compute a 
$\0$-embracing $k$-colorful choice $Q$ with respect to
the projections of $C_2,\dots,C_m$. By Lemmas~\ref{lem:dimreduct}
and~\ref{lem:representatives}, one of the two sets $C'$, $C''$, 
say $C'$, is $\0$-embracing
equivalent to $Q$ with respect to $C_1$. Since $Q$ is a
$k$-colorful choice that does not contain points from $C_1$ and since
$|C'|,|C''|\leq k$, the set $C'' \cup Q$ is a $\0$-embracing $k$-colorful
choice. The recursion stops once only one color class is left. Then, 
we are in dimension $d-m+1$. Since $d-m+2 \leq k$, pruning the 
single remaining color class with
Lemma~\ref{lem:findminembr} results already in a $\0$-embracing 
$k$-colorful choice.  For details, see Algorithm~\ref{alg:simapx}.

\begin{alg}
  \SetKwFunction{Recurse}{recurse}
  \KwIn{$m$ sets $C_1,\ldots,C_{m} \subset \R^{d}$ that each embrace 
  the origin, and for each $C_i$, $i \in [m]$, the coefficients 
  of the convex combination of $\0$ with the points in $C_i$}
  \KwOut{minimally $\0$-embracing 
  $\max\lt(d-m+2, \lt\lceil\frac{d+1}{2}\rt\rceil\rt)$-colorful choice}
  $C \gets $ prune $C_1$ with Lemma~\ref{lem:findminembr}\;
  \lIf{$m = 1$}{\Return{$C$}}
  $C',\, C'' \gets $ partition of $C$ into two sets, each of size at most
  $\lt\lceil\frac{d+1}{2}\rt\rceil$\;
  Compute representative points $\rr',\, \rr''$ for $C',\, C''$\;
  $\DOWN{C}_2, \dots, \DOWN{C}_m \gets$ orthogonal projection of
  $C_2,\dots,C_m$ onto $\lsp(\rr',\rr'')^\perp$\;
  $\DOWN{Q} \gets $ \Recurse{$\DOWN{C}_2, \dots, \DOWN{C}_m$}\;
  $Q \gets $ replace projected points in $\DOWN{Q}$ by original points from
  $\bigcup_{i=2}^m C_i$\;
  Determine which point $\rr^{\times} \in \{\rr',\rr''\}$ is $\0$-embracing
  equivalent to $Q$ with Lemma~\ref{lem:dimreduct} 
  and let $C^\times$ be the corresponding subset of $C$\;
  \Return{$\lt(C \setminus C^{\times}\rt) \cup Q$ \textup{pruned with
      Lemma~\ref{lem:findminembr}}}\;
  \caption[Computing $\0$-embracing $(\lceil d/2\rceil +1)$-colorful
  choices.]{Simple Approximation}
  \label{alg:simapx}
\end{alg}

\begin{theorem}\label{thm:simpledimreduct}
  Let $C_1, \ldots, C_m\subset \R^d$ be
  $m\leq d$ color classes such that $C_i$ is a $\0$-embracing set
  of size $\Oh{d}$, for $i \in [m]$. On input $C_1,\dots,C_m$ and given the
  coefficients of the convex combination of the origin for each set $C_i$,
  Algorithm~\ref{alg:simapx} computes a $\0$-embracing $\max\lt(d-m+2,
  \lt\lceil\frac{d+1}{2}\rt\rceil\rt)$-colorful choice in $\Oh{d^5}$ time.
  In particular, for $m=\lt\lfloor d / 2 \rt\rfloor + 1$, the algorithm 
  computes a $(\lceil d/2\rceil +1)$-colorful choice.
\end{theorem}
\begin{prf}
  The correctness of Algorithm~\ref{alg:simapx} is a direct consequence of
  Lemmas~\ref{lem:dimreduct} and~\ref{lem:representatives}. It remains to
  analyze the running time. In each step of the recursion except for the 
  last one, we prune two times a set of size $\Oh{d}$ with
  Lemma~\ref{lem:findminembr}. This needs $\Oh{d^4}$ time. Furthermore, by
  Lemma~\ref{lem:representatives}, computing two representative points 
  also takes $\Oh{d^4}$ time. Finally, given the set $Q$, determining which
  representative point is $\0$-embracing equivalent to $Q$ takes also 
  $\Oh{d^4}$ by Lemma~\ref{lem:dimreduct} and using the fact that
  the recursively computed solution is minimally embracing. Thus, we 
  need $\Oh{d^4}$ time per step of the recursion and there are 
  $\Oh{d}$ recursion steps in total. The total running time is 
  $\Oh{d^5}$.
\end{prf}

Although nontrivial, the fact that we can take in polynomial time 
half of the points from each color class to construct a $\0$-embracing
$\lt(\lt\lceil d/2\rt\rceil+1\rt)$-colorful choice may not be too 
surprising.  In the remainder of this section, we present a 
generalization of Algorithm~\ref{alg:simapx} that computes 
$\0$-embracing $\lceil \eps d\rceil$-colorful choices in 
polynomial time for any fixed $\eps > 0$. The improved approximation
guarantee is achieved by repeatedly replacing subsets of $C$ with
Lemmas~\ref{lem:dimreduct} and~\ref{lem:representatives} in each 
step of the recursion. To still ensure polynomial running time, we 
reduce the dimensionality by a constant fraction in each step of 
the recursion. Additionally, we slightly worsen the desired 
approximation guarantee in each level of the recursion, i.e., if the
current recursion level is $j$ and the dimensionality is $d'$, then 
we do not compute an $\lt\lceil \eps d'\rt\rceil$-colorful
choice, but a $\lt\lceil (1-\eps/2)^{-j/2} \eps d'\rt\rceil$-colorful
choice. As we will see, this additional ``slack'' in the approximation 
guarantee limits the recursion depth to a constant depending only on 
$\eps$. 

In more detail, let $C_1,\dots,C_{d+1}\subset\R^d$ be $d+1$ sets that each
embrace the origin, and let $\eps > 0$ be a parameter. We want to compute 
an $\lt\lceil \eps d\rt\rceil$-colorful choice that embraces the origin. 
Set
\[
d_j = \lt\lceil \lt(1-\frac{\eps}{2}\rt)^j d \rt\rceil \text{~and }
k_j = \lt\lceil \eps \lt(1-\frac{\eps}{2}\rt)^{j/2}
d\rt\rceil,
\]
for $j \in \N$. The sequence $d_j$ controls the dimension reduction
argument with Lemmas~\ref{lem:dimreduct} and~\ref{lem:representatives}, 
i.e., in the $j$th recursion level, the dimensionality of the input 
will be $d_j$. The sequence $k_j$ defines the approximation guarantee 
in the $j$th recursion level.  Note that $d_0 = d$ and 
$k_0 = \lt\lceil \eps d \rt\rceil$. Assume now we are in recursion 
level $j$. That is, the input consists of $d_j+1$
color classes $C_1,\dots,C_{d_j+1} \subset \R^{d_j}$ that each embrace the
origin together with the coefficients of their convex combinations 
of the origin.
We want to compute a $\0$-embracing $k_j$-colorful choice.
As in the previous algorithm, we begin by computing a minimal 
$\0$-embracing subset $C$ of $C_1$ with Lemma~\ref{lem:findminembr}. 
If $k_j \geq d_j + 1$, then $C$ is already a valid approximation. 
Otherwise, we iteratively transform $C$ into a $k_j$-colorful choice. 
For this, we repeatedly replace subsets
of $C$ with points from $C_2 \cup \dots \cup C_{d_j + 1}$
until it contains at most $k_j$ points from each color. This 
is done as follows. Set $m = d_j - d_{j+1} + 1$. 
In the general situation, $C$ contains points from several
color classes, and we partition $C$ into sets $D_1,\dots,D_m$ by
distributing the points from each color in $C$ equally among 
these $m$ sets. Then,
we compute representative points $\rr_1,\dots,\rr_m$ for this partition. 
Let
$C^\star_1,\dots,C^\star_{d_{j+1}+1} \in \lt\{C_2,\dots,C_{d_j+1}\rt\}$ be
$d_{j+1}+1$ color classes, where we discuss shortly how they are chosen.
We recursively compute a $k_{j+1}$-colorful choice $Q$ for
$C^\star_1,\dots,C^\star_{d_{j+1}+1}$ that embraces the
origin when projected on $U = \lsp(\rr_1,\dots,\rr_m)^\perp$. 
Note that $\dim U = d_j-(m-1) = d_{j+1}$
and hence the dimensionality of the input in recursion level 
$j+1$ is $d_{j+1}$, as desired. Then, by Lemmas~\ref{lem:dimreduct}
and~\ref{lem:representatives}, at least one representative 
point $\rr_{i^\times}$ and
hence at least one of the sets $D_{i^\times}$ is
$\0$-embracing equivalent to $Q$. We set $C$ to $\lt(C \setminus
D_{i^\times}\rt) \cup Q$ and prune it with Lemma~\ref{lem:findminembr}. 
We repeat these steps until $C$ is a $k_j$-colorful choice. 

To ensure progress, $m$ should be smaller than $k_j$ so that 
$D_{i^\times}$ is guaranteed to contain a point from each 
color that appears more than $k_j$ times
in $C$. Furthermore, $Q$ should not contain points with colors that appear
``often'' in $C$. We call a color class $C_i$ \emph{light} with 
respect to $C$ if $|C \cap C_i| \leq k_j - k_{j+1}$, and 
\emph{heavy}, otherwise. For the recursion, we use only light 
color classes. A $k_{j+1}$-colorful choice with light colors can be 
added safely to $C$ without increasing any color over the
threshold $k_j$. In particular, since we start with $C=C_1$ and 
use only light color classes, no other color class can ever occur 
more than $k_j$ times in $C$ and hence we are finished once the 
number of points from $C_1$ is at most $k_j$.
Please refer to Algorithm~\ref{alg:mainapx} for details.

\begin{alg}
  \SetKwFunction{Recurse}{recurse}
  \KwIn{recursion depth $j\in\N_0$ (initially $0$), original 
  dimension $d \in \N$, approximation parameter $\eps > 0$, $d_j+1$ sets
  $C_1,\ldots,C_{d_j+1} \subset \R^{d_j}$ that each embrace the origin, 
  and for each $C_i$ the coefficients of the convex
  combination of $\0$ with the points in $C_i$
  }
  \KwOut{minimally $\0$-embracing $k_j$-colorful choice}
  $k_j \gets \lt\lceil \eps \lt(1-\frac{\eps}{2}\rt)^{j/2} d \rt\rceil$\;
  $d_{j+1} \gets \lt\lceil \lt(1-\frac{\eps}{2}\rt)^{j+1} d \rt\rceil$\;
  $m \gets d_j - d_{j+1} + 1$\;
  $C \gets $ prune $C_1$ with
  Lemma~\ref{lem:findminembr}\;\label{alg:mainapx:initprune}
  \While{$|C \cap C_1| > k_j$}{\label{alg:mainapx:while}
    $D_1,\dots, D_m \gets $ partition of $C$ s.t.\ the
    points from each color class are evenly 
    distributed\;\label{alg:mainapx:part}
    Compute representative points $\rr_1,\dots,\rr_m$ for
        $D_1,\dots,D_m$ with
        Lemma~\ref{lem:representatives}\;\label{alg:mainapx:representatives}
    Find $d_{j+1} + 1$ light color classes 
    $C^\star_1,\dots,C^\star_{d_{j+1} + 1} \in 
    \lt\{C_2,\dots,C_{d_j+1}\rt\}$\;
    $\DOWN{C}_1, \dots, \DOWN{C}_{d_{j+1}+1} \gets$ orthogonal projection of
    $C^\star_1,\dots,C^\star_{d_{j+1}+1}$ onto
        $\lsp(\rr_1,\dots,\rr_m)^\perp$\;\label{alg:mainapx:proj}
    $\DOWN{Q} \gets $\Recurse{$j+1$, $d$, $\eps$, $\DOWN{C}_1, \dots,
        \DOWN{C}_{d_{j+1}+1}$}\;\label{alg:mainapx:rec}
    $Q \gets $ replace projected points in $\DOWN{Q}$ by original 
    points from
    $\bigcup_{i=1}^{d_{j+1}+1} C^\star_i$\;
    Determine which point $\rr_{i^\times} \in \{\rr_1,\dots,\rr_m\}$ 
    is $\0$-embracing equivalent to $Q$ with Lemma~\ref{lem:dimreduct}\;
    $C \gets \lt(C \setminus D_{i^\times}\rt) \cup Q $ pruned with
        Lemma~\ref{lem:findminembr}\;
  }
  \Return{$C$}\;\label{alg:mainapx:return}
  \caption[Computing $\0$-embracing $\lt\lceil \eps d\rt\rceil$-colorful
  choices.]{$\lt\lceil \eps d\rt\rceil$-Approximation}
  \label{alg:mainapx}
\end{alg}

The next lemma states that for $\eps$ fixed, the number of necessary 
recursions before a trivial approximation with 
Lemma~\ref{lem:findminembr} suffices is constant.

\begin{lemma}\label{lem:recdepth}
  For any $\eps = \Om{d^{-1/4}}$ there exists a 
  $j = \Th{ \eps^{-1} \ln \eps^{-1}}$ such that $k_j \geq d_j+1$.
\end{lemma}
\begin{prf}
Replacing $d_j$ with its definition, we obtain
\begin{equation}\label{lem:recdepth:dj}
  d_j + 1 = \left\lceil \lt(1-\frac{\eps}{2}\rt)^j d \right\rceil + 1
  \leq \lt(1-\frac{\eps}{2}\rt)^j d + 2.
\end{equation}
Using $\ln\lt(1- \frac{\eps}{2}\rt) \geq  -\eps$ if $\eps\leq 1$, we 
have for
$j \leq \frac{1}{\eps} \ln d$,
\begin{equation}\label{lem:recdepth:geq1}
\lt(1-\frac{\eps}{2}\rt)^j d \geq e^{-\eps j}d \geq 1.
\end{equation}
Furthermore, using that $\ln\lt(1-\frac{\eps}{2}\rt) \leq
-\frac{\eps}{2}$, we have for $j \geq \frac{4}{\eps} \ln \frac{3}{\eps}$
\begin{equation}\label{lem:recdepth:eps}
  3 \lt(1 - \frac{\eps}{2}\rt)^{j/2} \leq 
  3 e^{-\eps j/4} \leq \eps.
\end{equation}
Combining~\eq{lem:recdepth:geq1} and~\eq{lem:recdepth:eps}
with~\eq{lem:recdepth:dj}, we get
\[
  d_j + 1
  \leq 3 \lt(1 - \frac{\eps}{2}\rt)^j d
  \leq \eps \lt(1 - \frac{\eps}{2}\rt)^{j/2} d
  \leq \lt\lceil\eps \lt(1 - \frac{\eps}{2}\rt)^{j/2} d\rt\rceil
  = k_j.
\]
For $d = \Om{\eps^{-1/4}}$, there is a $j$ with
$\frac{4}{\eps}\ln \frac{3}{\eps} \leq j \leq \frac{1}{\eps} \ln d$. 
The claim follows.
\end{prf}

Next, we show that if the recursion depth is not too large, then we can
always find enough light color classes.

\begin{lemma}\label{lem:light}
  Let $j \in \N$ and let $C_1,\dots,C_{d_j+1}\subset \R^{d_j}$ be 
  $d_j+1$ color classes.
  Furthermore, let $C \subseteq \bigcup_{i=1}^{d_j+1} C_i$ be a set 
  of size at most $d_j+1$. For all $j=\Oh{\eps^{-1} \ln (\eps^3 d)}$, 
  there exist $d_{j+1}+1$ light color classes with respect to $C$.
\end{lemma}
\begin{prf}
We recall that a color class $C_i$, $i \in [d_j+1]$, is light with 
respect to $C$ if $|C \cap C_i| \leq k_j - k_{j+1}$. Then, the 
number of heavy color classes $h$ is bounded by
\begin{equation}\label{lem:light:heavy}
  h \leq \lt\lceil\frac{d_j + 1}{k_j - k_{j+1}}\rt\rceil \leq
  \frac{2 d_j}{k_j - k_{j+1}} + 1,
\end{equation}
since $d_j \geq 1$ for all $j \in \N$. We can bound the denominator 
as follows
\begin{multline}\label{lem:light:denom}
    k_j - k_{j+1}
    = \lt\lceil \eps \lt(1-\frac{\eps}{2}\rt)^{j/2} d\rt\rceil
        - \lt\lceil \eps \lt(1-\frac{\eps}{2}\rt)^{(j+1)/2} d\rt\rceil
    \geq \eps \lt(1-\frac{\eps}{2}\rt)^{j/2} d
        - \eps \lt(1-\frac{\eps}{2}\rt)^{(j+1)/2} d - 1
    \\
    = \eps \lt(1-\frac{\eps}{2}\rt)^{j/2} d 
    \lt( 1 - \sqrt{1-\frac{\eps}{2}}\rt) - 1
    \geq \frac{\eps^2}{4} \lt(1-\frac{\eps}{2}\rt)^{j/2} d - 1,
\end{multline}
where we apply $1 - \sqrt{1 - \frac{\eps}{2}} \geq \frac{\eps}{4}$ in 
the last
inequality.  Using that $\ln\lt(1-\frac{\eps}{2}\rt) \geq -\eps$ if 
$\eps \leq 1$, we have for $j \leq \frac{2}{\eps} \ln \frac{\eps^2 d}{8}$
\begin{equation}\label{lem:light:denomsim}
1 \leq 
\frac{\eps^{2}}{8} e^{-\eps j/2} d \leq
\frac{\eps^2}{8} \lt(1-\frac{\eps}{2}\rt)^{j/2} d
\end{equation}
and hence~\eq{lem:light:denom} can be simplified to
\begin{equation}\label{lem:light:denomfinal}
    k_j - k_{j+1} \geq \frac{\eps^2}{8} \lt(1-\frac{\eps}{2}\rt)^{j/2} d.
\end{equation}
Plugging (\ref{lem:light:denomfinal}) into (\ref{lem:light:heavy}) and
using~\eq{lem:light:denomsim},
we obtain
\begin{equation*}\label{lem:light:heavyfinal}
  h \leq \frac{2 \lt\lceil \lt(1-\frac{\eps}{2}\rt)^j d\rt\rceil}
  {\frac{\eps^2}{8}
      \lt(1-\frac{\eps}{2}\rt)^{j/2} d} + 1
  \leq \frac{2 \lt(1-\frac{\eps}{2}\rt)^j d}{\frac{\eps^2}{8}
      \lt(1-\frac{\eps}{2}\rt)^{j/2} d} + 3
  = \frac{16}{\eps^2} \lt(1-\frac{\eps}{2}\rt)^{j/2} + 3.
\end{equation*}
Then, the number $\ell$ of light color classes is at least
\begin{multline}\label{lem:light:light1}
    \ell = d_j + 1 - h
    \geq \lt\lceil \lt(1-\frac{\eps}{2}\rt)^j d \rt\rceil - 
    \frac{16}{\eps^2}
        \lt(1-\frac{\eps}{2}\rt)^{j/2} - 2
    \\
    \geq \lt(1-\frac{\eps}{2}\rt)^j d \lt(1 - \frac{16}{\eps^2
          \lt(1-\frac{\eps}{2}\rt)^{j/2} d} -
    \frac{2}{\lt(1-\frac{\eps}{2}\rt)^j d} \rt).
\end{multline}
For $j \leq \frac{2}{\eps}\ln \frac{\eps^3 d}{128}$, using
$\ln \lt(1-\frac{\eps}{2}\rt) \geq  -\eps$ if $\eps \leq 1$, we have
\begin{equation*}
    \frac{16}{\eps^2 \lt(1-\frac{\eps}{2}\rt)^{j/2} d} +
    \frac{2}{\lt(1-\frac{\eps}{2}\rt)^j d} 
 \leq    \frac{16}{\eps^2 e^{-\eps j/2} d} +
    \frac{2}{e^{-\eps j/2} d}
    \leq \frac{\eps}{8} + \frac{\eps}{8}  
    \leq \frac{\eps}{4}
\end{equation*}
and thus~\eq{lem:light:light1} implies
\begin{equation}\label{lem:light:lightinterm}
  \ell \geq \lt(1-\frac{\eps}{4}\rt)\lt(1-\frac{\eps}{2}\rt)^j d.
\end{equation}
For $j \leq \frac{2}{\eps}\ln \frac{\eps d}{2}$, using
$\ln \lt(1-\frac{\eps}{2}\rt) \geq  -\eps$ if $\eps \leq 1$, we can bound
\begin{equation}\label{lem:light:epshalf}
  \frac{\eps}{4} \lt(1-\frac{\eps}{2}\rt)^j d 
  \geq \frac{\eps}{4} e^{-\eps j/2} d \geq 2.
\end{equation}
Combining~\eq{lem:light:epshalf} with (\ref{lem:light:lightinterm}), we get
\[
  \ell
  \geq \lt(1-\frac{\eps}{2}\rt)^{j+1} d
      + \frac{\eps}{4} \lt(1 - \frac{\eps}{2}\rt)^j d
  \geq \lt(1-\frac{\eps}{2}\rt)^{j+1} d + 2
  \geq \lt\lceil \lt(1-\frac{\eps}{2}\rt)^{j+1} d \rt\rceil + 1
  = d_{j+1} + 1.
\]
Thus, for $j = \Oh{\eps^{-1} \ln (\eps^3 d)}$, there are at least $d_{j+1}
+1$ light color classes with respect to $C$.
\end{prf}

Before we finally prove correctness, we show if the recursion depth 
$j$ is not too large, then each set of the partition of $C$ contains 
at least one point from $C_1$ until $C$ is a $k_j$-colorful choice. 
This implies that each iteration of
the while-loop decreases the amount of points from $C_1$ in $C$.

\begin{lemma}\label{lem:while}
  For all $j = \Oh{\eps^{-1} \ln (\eps d)}$, we have 
  $m = d_j - d_{j+1} + 1 \leq k_j + 1$.
\end{lemma}
\begin{prf}
  First, we upper bound $m$ as follows:
\begin{equation}\label{lem:while:k}
  \begin{split}
    m = d_j - d_{j+1} + 1
    & = \lt\lceil \lt(1-\frac{\eps}{2}\rt)^j d \rt\rceil
        - \lt\lceil \lt(1-\frac{\eps}{2}\rt)^{j+1} d \rt\rceil + 1
    \\
    & \leq \lt(1-\frac{\eps}{2}\rt)^j d - 
    \lt(1-\frac{\eps}{2}\rt)^{j+1} d + 2
    = \frac{\eps}{2} \lt(1-\frac{\eps}{2}\rt)^j d + 2.
  \end{split}
\end{equation}
For $j \leq  \frac{2}{\eps} \ln \frac{\eps d}{2}$, 
with $\ln\lt(1-\frac{\eps}{2}\rt) \geq -\eps$ if 
$\eps \leq 1$,
we obtain $\frac{\eps}{2}
\lt(1-\frac{\eps}{2}\rt)^j d \geq \frac{\eps}{2} e^{-\eps j/2}d \geq 1$. 
Using this in (\ref{lem:while:k}), we get
\begin{equation*}
    m
  \leq \eps \lt(1-\frac{\eps}{2}\rt)^j d + 1
  \leq \lt\lceil \eps \lt(1-\frac{\eps}{2}\rt)^j d \rt\rceil + 1 = k_j + 1,
\end{equation*}
as desired.
\end{prf}

\begin{theorem}
\label{thm:bapx}
  Let $C_1, \ldots, C_{d+1}\subset \R^d$ be
  $d+1$ sets such that $C_i$ is a $\0$-embracing set
  of size $\Oh{d}$, for $i \in [d+1]$, and let $\eps = \Om{d^{-1/4}}$ be a
  parameter. On input $0$, $d$, $\eps$, $C_1,\dots,C_{d+1}$, and given the
  coefficients of the convex combination of the origin with the points 
  in $C_i$, for $i \in [d+1]$,
  Algorithm~\ref{alg:mainapx} computes a $\0$-embracing 
  $\lceil \eps d\rceil$-colorful
  choice in $d^{\Oh{\eps^{-1} \ln \eps^{-1}}}$ time.
\end{theorem}

\begin{prf}
We begin by showing that if the algorithm enters the while 
loop in recursion level $j$, it is always possible to find $d_{j+1}
+ 1$ light color classes and that the projections
$\DOWN{C}_1,\dots,\DOWN{C}_{d_{j+1}+1}$ of these color
classes are $\0$-embracing subsets of 
$\R^{d_{j+1}}$~(Line~\ref{alg:mainapx:proj}). In other words, we show
that recursion is possible if $C$ is not a $k_j$-colorful choice.
Assume now the algorithm enters the while loop in recursion level $j$.
Then, $C$ is a minimally $\0$-embracing subset of 
$C_1 \subset \R^{d_j}$ and has size at least $k_j+1$. In
Line~\ref{alg:mainapx:part}, we partition $C$ into $m$ sets 
$D_1,\dots,D_m$ by distributing the points from each color class equally. 
By Lemma~\ref{lem:while}, we have $m\leq k_j + 1$, 
for $j = \Oh{\eps^{-1} \ln (\eps d)}$,
and hence each set $D_i$ is nonempty. Thus,
the algorithm from Lemma~\ref{lem:representatives} can be applied in
Line~\ref{alg:mainapx:representatives}
to compute the representative points $\rr_1,\dots,\rr_m$. Moreover $\dim
\lsp\lt(\rr_1,\dots,\rr_m\rt) = m-1$ by Lemma~\ref{lem:representatives} and
Lemma~\ref{lem:minembr}.
Thus, $\dim \lsp\lt(\rr_1,\dots,\rr_m\rt)^\perp = d - m + 1 = d_{j+1}$.
Now, Lemma~\ref{lem:light} guarantees that we can always find $d_{j+1}+1$ 
light color classes $C^\star_1,\dots,C^\star_{d_{j+1}+1}$, 
if $j = \Oh{\eps^{-1} \ln \eps^3 d}$. Because each color 
class $C^\star_i$, $i \in [d_{j+1}+1]$, is
$\0$-embracing, so are their orthogonal projections onto
$\lsp(\rr_1,\dots,\rr_k)^T$. Thus, recursion is possible 
if $j = \Oh{\eps^{-1} \ln \eps^3 d}$. By Lemma~\ref{lem:recdepth}, 
the recursion depth is limited to
$\Th{\eps^{-1} \ln \eps^{-1}}$, since then pruning $C_1$ with
Lemma~\ref{lem:findminembr} in Line~\ref{alg:mainapx:initprune} is already a
$\0$-embracing $k_j$-colorful choice. In this case, the while loop is never
executed. We conclude that for $\eps = \Om{d^{-1/4}}$, recursion is always
possible as long as $C$ is not a $k_j$-colorful choice.


Next, we prove that the algorithm computes in recursion level $j$ a
$\0$-embracing $k_j$-colorful choice. As discussed above, the
recursion terminates after $\Oh{\eps^{-1} \ln \eps^{-1}}$ steps when
the set $C$ from Line~\ref{alg:mainapx:initprune} is already a 
$\0$-embracing $k_j$-colorful choice. If $C$ is not already a 
valid approximation, the while loop is executed. In each 
iteration of the while loop, $C$ is partitioned into $m$ sets
$D_1,\dots,D_m$ by distributing the points from each color 
equally among the $D_i$. By Lemma~\ref{lem:while}, 
$m\leq k_j + 1$ for $j = \Oh{\eps^{-1} \ln \eps d}$
and hence each set $D_i$, $i \in [m]$, contains at least one 
point from $C_1$. Applying Lemmas~\ref{lem:dimreduct} 
and~\ref{lem:representatives}, one of these
sets, say $D_{i^\times}$, is replaced in $C$  by a recursively computed
$k_{j+1}$-colorful choice $Q$
that is $\0$-embracing when projected onto $\lsp(\rr_1,\dots,\rr_m)^\perp$.
Since we use in the recursion only light color classes with
respect to $C$, and since $C_1$ is not a light color class, 
each iteration of the while loop strictly decreases the number of 
points from $C_1$ in $C$.  Moreover, because $Q$ contains 
only points from light color classes and since it
is a $k_{j+1}$-colorful choice, $\lt(C \setminus D_{i^\times}\rt) \cup Q$
contains at most $k_j$ points from the color classes $C_2,\dots,C_{d_j+1}$.
Thus, after $\Oh{d}$ iterations, $C$ is a $\0$-embracing 
$k_j$-colorful choice.

It remains to analyze the running time. The initial computation of $C$ in
Line~\ref{alg:mainapx:initprune} and each iteration of the 
while loop except for the recursive call takes $\Oh{d^4}$ time. 
Since the while loop is executed $\Oh{d}$
times and since the recursion depth is bounded by $\Oh{\eps^{-1} \ln
\eps^{-1}}$, the total running time of Algorithm~\ref{alg:mainapx} is 
$d^{\Oh{\eps^{-1} \ln \eps^{-1}}}$.
\end{prf}

\subsection{Applications}
\label{sec:kcol:applications}

As discussed in the introduction, the main motivation for $k$-colorful
choices is their application in polynomial-time reductions to \CCP. We
begin by presenting the proofs whose interpretation as algorithms 
results in the polynomial reductions. Then, we give precise 
bounds on the quality of the
obtained approximation algorithms for \Centerpoint, \Tverberg, and
\ColKirchberger when having access to an algorithm that on input $d+1$
color classes $C_1,\dots,C_{d+1}$, each $\0$-embracing and of size at 
most $d+1$, computes a $\0$-embracing $k(d)$-colorful choice in time $W(d)$.


\begin{theorem}[{Centerpoint 
theorem~\cite[Theorem~1]{Rado1946}}]\label{thm:centerpoint}
Let $P \subset \R^d$ be a point set. Then, there exists a point $\qq \in
\R^d$ such that for any halfspace $h^-$ with $\qq \in h^-$, we 
have $|P \cap h^-| \geq \left\lceil\frac{|P|}{d+1}\right\rceil$. \qed
\end{theorem}

Teng~\cite[Theorem~8.4]{Teng1991} showed that given a point set 
$P \in \R^d$ and a candidate centerpoint $\qq \in \R^d$, it is 
\coNP-complete to decide whether $\qq$ is
a centerpoint of $P$, if $d$ is part of the input. 
For $d=1$, a centerpoint is equivalent to a median of a set of 
numbers and hence can be computed in $\Oh{|P|}$ 
time~\cite{BlumFlPrRiTa1973}. Jadhav and Mukhopadhyay~\cite{JadhavMu1994} 
showed that linear time is sufficient even in two dimensions. For 
$d\geq 3$ fixed, the best known algorithm is by Chan~\cite{Chan2004} 
who showed how to compute a point with maximum Tukey depth,
a stronger notion than being a centerpoint, in expected time 
$\Oh{n^{d-1}}$.

Although it is in general \coNP-complete to verify centerpoints, Tverberg
partitions serve as
polynomial-time checkable certificates for a subset of centerpoints.
In recent years, this property has been exploited algorithmically to derive
efficient approximation algorithms for
centerpoints~\cite{MulzerWe2013,MillerSh2010}.
The existence of Tverberg points is guaranteed by Tverberg's
theorem~\cite{Tverberg1966}.

\begin{theorem}[Tverberg's theorem~\cite{Tverberg1966}]\label{thm:tverberg}
  Let $P \subset \R^d$ be a point set of size $n$. Then, there 
  always exists a
  Tverberg $\left\lceil\frac{|P|}{d+1}\right\rceil$-partition for $P$.
  Equivalently, let $P$ be of size $(m-1)(d+1)+1$, with $m \in \N$. 
  Then, there exists a Tverberg $m$-partition for $P$.
\end{theorem}

While Tverberg's first proof is quite involved, several
simplified subsequent 
proofs~\cite{Tverberg1981,TverbergVr1993,Sarkaria1992,Roudneff2001} 
have been published.
Here, we present Sarkaria's proof~\cite{Sarkaria1992} with further
simplifications by \Barany and Onn~\cite{BaranyOn1997} and Arocha
\etal~\cite{ArochaBaBrFaMo2009}.
The main tool is the next lemma that establishes a correspondence 
between the intersection of convex hulls of low-dimensional point sets
and the embrace of the origin of certain high-dimensional
point sets. It was extracted from Sarkaria's proof by Arocha
\etal~\cite{ArochaBaBrFaMo2009}. In the following, we denote with 
$\otimes$ the \emph{tensor product} that maps two points 
$\pp \in \R^d$, $\qq \in \R^m$ to the point
\[
\pp \otimes \qq =
\begin{pmatrix}
  (\qq)_1 \pp \\
  (\qq)_2 \pp \\
  \vdots \\
  (\qq)_m
  \pp
\end{pmatrix} \in \R^{dm},
\]
where $(\qq)_i \pp$ denotes the vector $\pp$ scaled by the $i$th component
of $\qq$, for $i \in [m]$.
Then, $\otimes$ is bilinear, i.e., for all $\pp_1,\pp_2 \in
\R^d$, $\qq \in \R^m$, and $\alpha_1,\alpha_2 \in \R$, we have
\[
\lt(\alpha_1 \pp_1 + \alpha_2 \pp_2\rt) \otimes \qq
= \alpha_1 \lt(\pp_1 \otimes \qq\rt) + \alpha_2 \lt(\pp_2 \otimes \qq\rt)
\]
and similarly, for all $\pp \in \R^d$, $\qq_1,\qq_2 \in \R^m$, and
$\alpha_1,\alpha_2 \in \R$, we have
\[
\pp \otimes \lt(\alpha_1 \qq_1 + \alpha_2 \qq_2\rt)
= \alpha_1 \lt(\pp \otimes \qq_1\rt) + \alpha_2 \lt(\pp \otimes \qq_2\rt).
\]

\begin{lemma}[Sarkaria's 
lemma~\cite{Sarkaria1992},~{\cite[Lemma~2]{ArochaBaBrFaMo2009}}]
\label{lem:sarkaria}
Let $P_1,\dots,P_m \subset \R^d$ be $m$ point sets and 
let $\qq_1,\dots,\qq_{m}
\subset \R^{m-1}$ be $m$ vectors with $\qq_i = \e_i$ for $i \in [m-1]$ and
$\qq_m = -\1$. For $i \in [m]$, we define
\[
  \UP{P}_i = \set{\TwoRowVec{\pp}{1} \otimes \qq_i
  \midd \pp \in P_i}\subset \R^{(d+1) (m-1)}.
\]
Then, the intersection of the convex hulls $\bigcap_{i=1}^m \convv{P_i}$ 
is nonempty if and only if $\;\bigcup_{i=1}^m \UP{P}_i$ embraces the origin.
\end{lemma}
\begin{prf}
  Assume there is a point $\pp^\star \in \bigcap_{i=1}^m \convv{P_i}$.
  There exist coefficients
  $\lambda_{i,\pp} \in \Rp$ that sum to $1$ such that
  $\pp^\star = \sum_{\pp \in P_i} \lambda_{i,\pp}\pp$.
  Consider the points $\up{\pp}_i \in \convv{\UP{P}_i}$, $i \in [m]$, 
  that we obtain by using the same convex coefficients for the points 
  in $\UP{P}_i$, i.e., set
\[
  \up{\pp}_{i}
  = \sum_{\pp \in P_i} \lambda_{i,\pp}
          \lt(\TwoRowVec{\pp}{1} \otimes \qq_i\rt) \in \convv{\UP{P}_i}.
\]
We claim that $\sum_{i=1}^{m} \up{\pp}_i = \0$ and thus
$\0 \in \convv{\bigcup_{i=1}^{m} \UP{P}_i}$. Indeed, we have
\begin{multline*}
  \sum_{i=1}^{m} \up{\pp}_i = 
  \sum_{i=1}^{m} \sum_{\pp \in P_i} \lambda_{i,\pp}
  \left( \TwoRowVec{\pp}{1} \otimes \qq_i \right)
  \\
  = \sum_{i=1}^{m} \left( \sum_{\pp \in P_i} \lambda_{i,\pp}
  \TwoRowVec{\pp}{1} \right) \otimes \qq_i
  = \sum_{i=1}^{m} \TwoRowVec{\pp^\star}{1} \otimes \qq_i
  \\
  = \TwoRowVec{\pp^\star}{1} \otimes \left( \sum_{i=1}^{m} \qq_i \right)
  = \TwoRowVec{\pp^\star}{1} \otimes \0 = \0,
\end{multline*}
using the bilinearity of  $\otimes$.

Assume now that $\bigcup_{i=1}^m \UP{P}_i$ embraces the origin. 
We want to show that $\bigcap_{i=1}^m \convv{P_i}$ is nonempty. 
Then, we can express the origin as a convex combination $\sum_{i=1}^{m}
\sum_{\up{\pp} \in \UP{P}_i} \lambda_{i,\up{\pp}} \up{\pp}$ with
$\lambda_{i,\up{\pp}} \in\Rp$ for $i \in
[m]$ and $\up{\pp} \in \UP{P}_i$, and $\sum_{i=1}^{m} \sum_{\up{\pp} \in
\UP{P}_i} \lambda_{i,\up{\pp}} = 1$. Hence, we have
\[
\0 = \sum_{i=1}^{m} \sum_{\up{\pp} \in \UP{P}_i} \lambda_{i,\up{\pp}}
\left( \TwoRowVec{\pp}{1} \otimes \qq_i \right)
 = \sum_{i=1}^{m} \left( \sum_{\up{\pp} \in \UP{P}_i} \lambda_{i,\up{\pp}}
 \TwoRowVec{\pp}{1} \right) \otimes \qq_i,
\]
again using the bilinearity of $\otimes$.
By the choice of $\qq_1,\dots,\qq_m$, there is (up to multiplication 
with a scalar) exactly one linear dependency: $\0 = \sum_{i=1}^{m} \qq_i$.
Thus,
\[
\sum_{\up{\pp} \in \UP{P}_1} \lambda_{1,\up{\pp}} \TwoRowVec{\pp}{1}
= \dots =
\sum_{\up{\pp} \in \UP{P}_m} \lambda_{m,\up{\pp}} \TwoRowVec{\pp}{1} =
\TwoRowVec{\pp^\star}{c},
\]
where $\pp^\star \in \R^d$ and $c \in \R$. In particular, the last equality
implies that
\[
\sum_{\up{\pp} \in \UP{P}_1} \lambda_{1,\up{\pp}}
= \dots =
\sum_{\up{\pp} \in \UP{P}_m} \lambda_{m,\up{\pp}}
= c.
\]
Now, as for all $i \in [m]$ and $\up{\pp} \in \UP{P}_i$, the coefficient
$\lambda_{i,\up{\pp}}$ is nonnegative and as the sum $\sum_{i \in [m]}
\sum_{\up{\pp} \in \UP{P}_i} \lambda_{i,\up{\pp}}$ is $1$, we must have 
$c = 1/m \in (0,1]$. Hence,
the point $m \pp^\star$ is common to all convex
hulls $\convv{P_1}$, $\ldots$, $\convv{P_m}$.
\end{prf}

Please refer to Figure~\ref{fig:sarkaria} for an example of Sarkaria's 
lifting argument.  Little work is now left to obtain Tverberg's theorem 
from Lemma~\ref{lem:sarkaria} and the colorful \Caratheodory theorem.
\begin{figure}[htbp]
  \begin{center}
    \includegraphics{sarkaria.pdf}
  \end{center}
  \caption[Example of Sarkaria's Lemma.]{An example of Sarkaria's lemma for
  $d=1$ and $m=2$. The set $P_1$
  consists of the red points and the set $P_2$ consists of the blue points.
  Since the convex hulls of $P_1$ and $P_2$ intersect, the lifted points 
  embrace the origin.}
  \label{fig:sarkaria}
\end{figure}

\begin{prf}[Proof of Theorem~\ref{thm:tverberg}]\label{thm:tverberg:proof}
Let $P = \set{\pp_1,\dots,\pp_n} \subset \R^d$ be a point set of size
$n=(d+1)(m-1)+1$ and let $P_1,\dots,P_m$ denote $m$ copies of $P$.
For each set $P_j \subset \R^d$, $j \in [m]$, we construct a $((d+1)
(m-1))$-dimensional set $\UP{P}_j$ as in Lemma~\ref{lem:sarkaria}, i.e.,
\[
  \UP{P}_j = \set{ \up{\pp}_{i,j} = 
  \TwoRowVec{\pp_i}{1} \otimes \qq_j \midd \pp_i \in P}
  \subset \R^{(d+1) (m-1)} = \R^{n-1}.
\]
For $i \in [n]$, we denote with 
$\UP{C}_i \subseteq \bigcup_{j=1}^m \UP{P}_j$
the set of points $\set{\up{\pp}_{i,j}  \midd j \in [m]}$ that 
correspond to $\pp_i \in P$,
and we color these points with color $i$. For $i \in [n]$, note that
Lemma~\ref{lem:sarkaria} applied to $m$ copies of the singleton set 
$\set{\pp_i}
\subseteq P$ guarantees that the color class $\UP{C}_i \in
\R^{n-1}$ embraces the origin. Hence, we have $n$ color classes
$\UP{C}_1,\dots,\UP{C}_n$ that embrace the origin in $\R^{n-1}$. Now,
by Theorem~\ref{thm:colcara}, there is a
colorful choice $\UP{C} = \set{\up{\cc}_1,\dots,\up{\cc}_n} \subseteq
\bigcup_{i=1}^n \UP{C}_i$ with $\up{\cc}_i \in \UP{C}_i$ that embraces the
origin, too. Because $\UP{C}$ embraces the origin,
Lemma~\ref{lem:sarkaria} guarantees that the convex hulls of the sets 
$T_j = \set{ \pp_i \in P \midd \up{\pp}_{i,j} \in \UP{C}}$, $j \in [m]$, 
have a point in common.
Moreover, since all points in $\bigcup_{j=1}^m \UP{P}_j$
that correspond to the same point in $P$ have the same
color, each point $\pp_i \in P$ appears in exactly one set 
$T_j$, $j \in [m]$.
Thus, $\mc{T} =\set{T_1,\dots, T_m}$ is a Tverberg $m$-partition of $P$.
\end{prf}

Even less effort is required to obtain the colorful Kirchberger theorem 
from Lemma~\ref{lem:sarkaria}. Let $A, B \subset\R^d$ be two point sets.
Kirchberger's theorem~\cite{Kirchberger1903} states that if for all 
subsets $C \subset A \cup B$ of size at most $d+2$,
the sets $\convv{A\cap C}$ and $\convv{B \cap C}$ have an empty 
intersection, then $\convv{A}$ and $\convv{B}$ have an empty intersection.
Arocha~\etal~\cite{ArochaBaBrFaMo2009} presented a generalization based 
on the colorful \Caratheodory theorem.\footnote{Actually, Arocha~\etal 
present an even stronger result (the ``very colorful Kirchberger
theorem''~\cite[Theorem~3]{ArochaBaBrFaMo2009}) using a generalization 
of the colorful \Caratheodory theorem. Here, we consider the weaker 
version that can be obtained from Theorem~\ref{thm:colcara}.}

\begin{theorem}[Colorful Kirchberger theorem~{\cite[special case of
Theorem~3]{ArochaBaBrFaMo2009}}]\label{thm:colkirchberger}
Let $C_1,\dots, C_n \subset \R^d$ be $n=(m-1)(d+1)+1$ pairwise disjoint 
color classes and let $\mc{T}_i = \set{T_{i,1},\dots,T_{i,m}}$ 
denote a Tverberg
  $m$-partition for $C_i$, where $i \in [n]$. Then, there exists a colorful
  choice $C$, $|C|=n$, such that the family of sets
\[
  \mc{T}_{C} = \left\{ C \cap \left( \bigcup_{i=1}^n T_{i,j} \right) 
  \midd j \in [m] \right\}
\]
  is a Tverberg $m$-partition for $C$.
\end{theorem}
\begin{prf}
  We lift each Tverberg partition to $\R^{n-1}$ as in 
  Lemma~\ref{lem:sarkaria}: for $i \in [n]$ and $j \in [m]$, we 
  denote with $\UP{T}_{i,j}$ the set
\[
  \UP{T}_{i,j} = \set{ \TwoRowVec{\pp}{1} \otimes \qq_j \midd 
  \pp \in T_{i,j}} \subset
  \R^{n-1}.
\]
By Lemma~\ref{lem:sarkaria} and since each set 
$\mc{T}_i$, $i \in [n]$, is a Tverberg
partition, the sets $\UP{C}_i = \bigcup_{j=1}^m \UP{T}_{i,j}$, $i \in [n]$,
embrace the origin. We color the points in $\UP{C}_i$ with color $i$. Since
there are $n$ color classes that embrace the origin in $n-1$ dimensions,
Theorem~\ref{thm:colcara} guarantees the existence of a colorful 
choice $\UP{C}$ that
embraces the origin. For $j \in[m]$, let $\UP{T}_{j}=\UP{C} \cap \left(
\bigcup_{i=1}^n \UP{T}_{i,j} \right)$ denote all points from a $j$th 
element in a Tverberg partition in $\UP{C}C$. Since 
$\UP{C}= \bigcup_{j=1}^m \UP{T}_{j}$ embraces
the origin, Lemma~\ref{lem:sarkaria} implies that the convex hulls of
the sets 
$T_j = \set{ \pp \in \bigcup_{i=1}^n P_i \midd \TwoRowVec{\pp}{1} 
\otimes \qq_j \in \UP{T}_j}$ have a nonempty intersection. 
Further, since for $j \in [m]$, the set
$\UP{T}_{j}$ is a subset of $\bigcup_{i=1}^n \UP{T}_{i,j}$, we have $T_j
\subset \left( \bigcup_{i=1}^n T_{i,j} \right)$. Moreover, since all points
that correspond to the Tverberg partition $\mc{T}_i$, $i \in [n]$, have
color $i$, exactly one of the sets $T_1,\dots,T_m$ contains a point 
from $C_i$. The colorful choice $C$ can be obtained by projecting 
$\UP{C}$ down to $\R^d$.
\end{prf}

We now give precise bounds on the quality and the running time of 
approximation algorithms obtained by
combining algorithms for $k$-colorful choices with the presented
reductions to \CCP. Unfortunately, the approximation
guarantee of Algorithm~\ref{alg:mainapx} is too weak to obtain a nontrivial
approximation algorithm for \Tverberg and therefore also for \Centerpoint. 
On the positive side, it leads to a nontrivial approximation algorithm for
\ColKirchberger.

In the following, let $\mc{A}$ be an algorithm that, when given $d+1$ 
color classes
$C_1,\dots,C_{d+1} \subset \R^d$, each embracing the origin and of size
$\Oh{d}$, and for each $C_i$ the coefficients of the convex combination 
of the origin, outputs a $\0$-embracing $k(d)$-colorful choice in 
$W(d)$ time, where $k, W: \N \rightarrow \N$ are arbitrary but fixed functions.

\begin{corollary}\label{cor:app:tverberg}
  Let $P \subset \R^d$ be a point set of size $n$ and let $\mc{A}$ 
  be as above.  Set
  \[
  \widetilde{m} = 
  \lt\lceil \frac{n}{(d+1)^2\big(k(n-1)-1\big)+d+1}\rt\rceil =
  \Om{\frac{n}{d^2k(n-1)}}.
  \]
  Then, a Tverberg $\widetilde{m}$-partition $\mc{T}$ of
  $P$ and a point $\pp \in \bigcap_{T \in \mc{T}}
  \conv(T)$ can be computed in total time $\Oh{(d^2 + m) n^2 + W(n-1)}$.
\end{corollary}
\begin{prf}
\newcommand{\kch}[1]{\widetilde{#1}}
Set $m = \lt\lceil n / (d+1)\rt\rceil$.
In the proof of Theorem~\ref{thm:tverberg},
we lift $m$ copies of $P$ with Lemma~\ref{lem:sarkaria} to $\R^{n-1}$. 
Lifting one
point needs $\Oh{dm} = \Oh{n}$ time and hence lifting all $m$ copies takes
$\Oh{mn^2}$ time. Then, each point $\pp_i \in \R^d$ from $P$
corresponds to a color class $C_i =
\lt\{\up{\pp}_{i,j} \midd j \in [m]\rt\} \subset \R^{n-1}$
of size $m$ and a $\0$-embracing colorful choice of $C_1,\dots,C_n$
corresponds to the Tverberg partition $\mc{T}=\{T_1,\dots,T_m\}$ that 
we obtain
by assigning $\pp_i \in P$ to $T_j$ if $\up{\pp}_{i,j} \in C$.
By construction of the color classes in the proof of
Theorem~\ref{thm:tverberg}, the barycenter of $C_i$ is the origin, for 
$i \in [n]$.
Since we know then for each color class the coefficients of the convex
combination of
the origin, we can apply $\mc{A}$ to
obtain a $\0$-embracing $k(n-1)$-colorful choice $\kch{C} \subseteq
\bigcup_{i=1}^n C_i$ together with the coefficients of the convex 
combination of
the origin with the points in $\kch{C}$.
Let
$\kch{T}=\lt\{\kch{T}_1,\dots,\kch{T}_m\rt\}$ be a family of
subsets of $P$ that we construct as before by assigning $\pp_i$ to 
$\kch{T}_j$ if
$\up{\pp}_{i,j} \in \kch{C}$. Here, $\kch{T}$ is a multiset, i.e., we allow
$\kch{T}_i = \kch{T}_j$ for $i\neq
j$. Since $\kch{C}$ embraces the origin, Lemma~\ref{lem:sarkaria}
guarantees that the intersection $\bigcap_{i=1}^m \convv{\kch{T}_i}$ is
nonempty. Moreover, because we know the coefficients of the convex 
combination of the origin with the points in $\kch{C}$, we can 
compute in $\Oh{dn}$ time a
point $\pp^\star \in \bigcap_{i=1}^m \convv{\kch{T}_i}$ together with the
coefficients of the convex combination of $\pp^\star$ with the points in
$\kch{T}_i$ for $i \in [m]$, as described in the proof of 
Lemma~\ref{lem:sarkaria}.

Now, we construct a Tverberg partition for $P$ out of
$\kch{\mc{T}}$ by a greedy strategy that iteratively
removes sets from $\kch{\mc{T}}$. Let $\kch{T} \in
\kch{\mc{T}}$ be some set and remove it from $\kch{\mc{T}}$. 
Since we know the coefficients of the convex
combination of $\pp^\star$ with the points in $\kch{T}$,
Lemma~\ref{lem:constr_cara} can be applied to prune $\kch{T}$
to a $\pp^\star$-embracing set of size at most $d+1$ in 
$\Oh{d^3n + n^2}$ time. Then, for each point $\pp \in
\kch{T}$, we remove the at most $k(n-1)-1$ other sets from
$\kch{\mc{T}}$ that contain $\pp$. We continue with the next set
in $\kch{\mc{T}}$ that has not yet been removed until $\kch{\mc{T}} =
\emptyset$. Let $\mc{T}^\star \subseteq
\kch{\mc{T}}$ be the family of sets that we obtain by this process.
Clearly, $\mc{T}^\star$ is a Tverberg partition and 
because $\mc{T}^\star \subseteq \kch{\mc{T}}$, we have $\pp^\star \in
\bigcap_{\kch{T} \in \mc{T}^\star} \convv{\kch{T}}$.
Moreover, for each set
$\kch{T}_i \in \mc{T}^\star$, we remove at most $(d+1)(k(n-1)-1)$
other sets from $\kch{\mc{T}}$. Thus, the size of the Tverberg partition
$\mc{T}^\star$ is at least
\[
  \lt|\mc{T}^\star\rt|
  \geq \lt\lceil \frac{m}{(d+1)(k(n-1)-1)+1}\rt\rceil
  \geq \lt\lceil \frac{n}{(d+1)^2(k(n-1)-1)+d+1}\rt\rceil.
\]

Constructing the \CCP instance takes $\Oh{m n^2}$ time.
Using $\mc{A}$, we need $W(n-1)$ time to compute a $k(n-1)$-colorful choice
$\kch{C}$. Pruning every set of $\kch{\mc{T}}$ with 
Lemma~\ref{lem:constr_cara} to
at most $d+1$ points needs $\Oh{m (d^3 n + n^2)}=\Oh{(d^2 + m) n^2}$ time.
Finally, constructing
$\mc{T}^\star$ out of $\kch{\mc{T}}$ takes $\Oh{n^2}$ time with the naive
algorithm. This results in the claimed running time of
$\Oh{(d^2 + m) n^2 + W(n-1)}$.
\end{prf}

Furthermore, we can use $\mc{A}$ to approximate \ColKirchberger.

\begin{corollary}\label{cor:app:colkirchberger}
Let $\mc{A}$ be as above and let $C_1,\dots, C_n \subset \R^d$ be
$n=(m-1)(d+1)+1$ pairwise disjoint color
classes that are each of size $n$. Furthermore, for $i \in [n]$, let
$\mc{T}_i = \left\{T_{i,1},\dots,T_{i,m} \right\}$ denote a Tverberg
$m$-partition for $C_i$.
Then, given for each Tverberg partition $\mc{T}_i$, $i \in [n]$, a 
point $\pp_i
\in \bigcap_{j=1}^m \convv{T_{i,j}}$, and for all $i \in [n]$ and 
$j \in [m]$, the coefficients of the
convex combination of $\pp_i$ with the points in $T_{i,j}$,
we can compute in $\Oh{n^3 + W(n-1)}$ time a
$k(n-1)$-colorful choice $C\subseteq \bigcup_{i=1}^n C_i$ such that
\[
  \mc{T}_{C} = \left\{ C \cap \left( \bigcup_{i=1}^n T_{i,j} \right) \midd j \in
  [m] \right\}
\]
is a Tverberg $m$-partition for $C$.
\end{corollary}
\begin{prf}
  In the proof of Theorem~\ref{thm:colkirchberger},
  we lift the points $\bigcup_{i=1}^n C_i$ to $\R^{n-1}$ such that the
  set of points $\UP{C}_i$ that corresponds to the color class $C_i$
  still
  embraces the origin, where $i \in [n]$. Moreover, if $\UP{C}' \subseteq
  \bigcup_{i=1}^n \UP{C}_i$ is a $\0$-embracing colorful choice of the 
  lifted points, then there is a corresponding colorful choice $C'$ 
  with respect to $C_1,\dots,C_n$ such that
\[
  \mc{T}_{C'} = \left\{ C' \cap \left( \bigcup_{i=1}^n T_{i,j} \right)
  \midd j \in [m] \right\}
\]
is a Tverberg $m$-partition for $C'$.
Similarly, a $\0$-embracing $k(n-1)$-colorful choice $\UP{C}$ of the
lifted color classes corresponds to a $k(n-1)$-colorful choice $C$ with 
respect to $C_1,\dots,C_n$ such that
\[
  \mc{T}_{C} = \left\{ C \cap \left( \bigcup_{i=1}^n T_{i,j} \right)
  \midd j \in [m] \right\}
\]
is a Tverberg $m$-partition for $C$.

Computing the tensor product $\TwoRowVec{\pp}{1} \otimes \qq$, where 
$\pp \in \R^{d}$ and $\qq \in
\R^{m-1}$, needs $\Oh{dm} = \Oh{n}$ time and hence
lifting the point sets $C_1,\dots,C_n \subset \R^d$ to $\R^{n-1}$ with
Lemma~\ref{lem:sarkaria} needs $\Oh{n^3}$ time in total.
Since we know for each Tverberg partition $\mc{T}_i$, $i \in [n]$, a point
$\pp_i \in \bigcap_{j=1}^m \convv{T_{i,j}}$ together with the
coefficients of the convex combination of $\pp_i$ with the points in 
$T_{i,j}$ for $j \in [m]$, we can compute in $\Oh{n}$ time the 
coefficients of the convex combination of the origin with the points 
in $\UP{C}_i$ as described in the proof of Lemma~\ref{lem:sarkaria}.
Then, $\mc{A}$ can be applied to compute a $\0$-embracing $k(n-1)$-colorful
choice $\UP{C}$ with respect to the lifted point sets in
$W(n-1)$ time. Finally, constructing $C$ and $\mc{T}_C$ out of
$\UP{C}$ needs
$\Oh{n}$ time. Hence, the total time needed is $\Oh{n^3 + W(n-1)}$.
\end{prf}

Now, given $d+1$ color classes $C_1,\dots,C_{d+1} \subset \R^d$ that 
embrace the origin, we can compute with Algorithm~\ref{alg:mainapx} an 
$\lt\lceil \eps d\rt\rceil$-colorful choice that embraces
the origin in polynomial time. Combining this
with Corollary~\ref{cor:app:tverberg},
we obtain an algorithm that computes Tverberg partitions of size 
$\Oh{1}$ in polynomial time, a trivial result.
However, combining Algorithm~\ref{alg:mainapx} with
Corollary~\ref{cor:app:colkirchberger}, we do obtain a nontrivial 
approximation algorithm for \ColKirchberger: given $n =
(m-1)(d+1)+1$ color classes $C_1,\dots,C_n$, each of size $n$, and for each
color class a Tverberg $m$-partition 
$\mc{T}_i = \left\{T_{i,1},\dots,T_{i,m} \right\}$ together with a
point $\pp_i \in \bigcap_{j=1}^m \convv{T_{i,j}}$ and the coefficients 
of the convex combination of $\pp_i$ with the points in $T_{i,j}$, for 
all $j \in [m]$, we can compute
in $n^{\Oh{\eps^{-1} \ln \eps^{-1}}}$ time
an $\lceil \eps n \rceil$-colorful choice $C$ such that
\[
  \mc{T}_{C} = \left\{ C \cap \left( \bigcup_{i=1}^n T_{i,j} \right)
  \midd j \in [m] \right\}
\]
is a Tverberg $m$-partition for $C$, where $\eps > 0$ is arbitrary but 
fixed.
