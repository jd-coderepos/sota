\documentclass[letterpaper,11pt]{article} 
\usepackage[hmargin=1in,vmargin=1in]{geometry}
\usepackage{amsmath, amssymb}
\usepackage{bbm}
\usepackage{graphicx}
\usepackage{algorithm}
\usepackage{algorithmic}
\usepackage{xspace}
\title{\vspace{-1cm}Approximating Graphic TSP by Matchings\thanks{This research
was supported by ERC Advanced investigator grant 226203.}}
\author{Tobias M\"{o}mke and Ola Svensson \\
 Royal Institute of Technology - KTH, Stockholm, Sweden \\
  {\tt \{moemke,osven\}@kth.se }}

\newtheorem{theorem}{Theorem}[section]
\newtheorem{proposition}[theorem]{Proposition}
\newtheorem{observation}[theorem]{Observation}
\newtheorem{lemma}[theorem]{Lemma}
\newtheorem{claim}[theorem]{Claim}
\newtheorem{corollary}[theorem]{Corollary}
\newtheorem{definition}[theorem]{Definition}
\newtheorem{convention}[theorem]{Convention}
\newtheorem{conjecture}[theorem]{Conjecture}
\newtheorem{notation}[theorem]{Notation}
\newtheorem{remarkn}[theorem]{Remark}
\newtheorem{example}[theorem]{Example}
\newtheorem{property}[theorem]{Property}
\newenvironment{proofclaim}{\begin{trivlist}
\item[\hskip\labelsep {\it Proof of Claim}.]}{\QED \end{trivlist}}
\newenvironment{proof}{\begin{trivlist}
\item[\hskip\labelsep {\bf Proof}.]}{\QED \end{trivlist}}
\newenvironment{proofoverview}{\begin{trivlist}
\item[\hskip\labelsep {\bf Proof overview}.]}{\QED \end{trivlist}}
\newenvironment{approof}{\begin{trivlist}
\item[\hskip\labelsep {\bf Proof}]}{\QED \end{trivlist}}
\newenvironment{proofof}[1]{\begin{trivlist}
    \item[\hskip\labelsep {\bf Proof of #1.}]}{\QED \end{trivlist}}


\newenvironment{sketch}{\begin{trivlist}
    \item[\hskip\labelsep {\bf Proof Sketch}.]}{\QED \end{trivlist}}
\newenvironment{remark}{\begin{trivlist}
        \item[\hskip\labelsep {\bf Remark}.]}{\QED \end{trivlist}}
\newenvironment{remarks}{\begin{trivlist}
        \item[\hskip\labelsep {\bf Remarks}.]}{\QED \end{trivlist}}

\newenvironment{applemma}{\begin{trivlist}
        \item[\hskip\labelsep {\bf Lemma}]}{ \end{trivlist}}
\newenvironment{apptheorem}{\begin{trivlist}
        \item[\hskip\labelsep {\bf Theorem}]}{ \end{trivlist}}
\newenvironment{appclaim}{\begin{trivlist}
        \item[\hskip\labelsep {\bf Claim}]}{ \end{trivlist}}


\newcommand{\QED}{\hfill }

\newcommand{\nota}[1]{
\marginpar{\footnotesize #1 } 
}

\newcommand{\hide}[1]{
}

\newcommand{\sug}[2]{#2}
\newcommand{\TR}[1]{\mathcal T\left(#1\right)}
\def\P{\mathbf{P}}
\def\GP{\mathbf{G}_\mathbf{P}}
\def\HP{\mathbf{H}_\mathbf{P}}
\def\HPd{\mathbf{H}_\mathbf{P^d}}

\newcommand{\inst}{ }
\newcommand{\instUnit}{ }
\newcommand{\J}{J||C_{\max} }
\newcommand{\Flow}{F||C_{\max} }
\newcommand{\Jacyclic}{J|acyclic|C_{\max} }
\newcommand{\Jpmtn}{J|pmtn|C_{\max} }
\newcommand{\Jtwo}{J2||C_{\max} }

\newcommand{\ssc}[0]{\ensuremath{1|prec|\sum w_jC_j}}
\newcommand{\sscinterval}[0]{}
\newcommand{\sscsemiorder}[0]{}
\newcommand{\sscP}[1]{1|#1|\sum w_jC_j}
\newcommand{\prob}[0]{1|prec|\sum w_jC_j}
\newcommand{\ktcol}[0]{\text{\it k:t}}

\newcommand{\graph}[0]{\ensuremath{G_\mathbf{P}^S}}
\newcommand{\graphi}[0]{\ensuremath{G_\mathbf{I}^S}}


\newcommand{\pmax}[1]{p_{\it max}^{(#1)}}
\def\eps{\varepsilon}
\def\reals{\bf R}
\newcommand{\sV}[1]{\mathcal V^{(#1)}}
\newcommand{\sE}[2]{\mathcal B_{#1}^{(#2)}}
\def\tri{\raisebox{-0.3ex}{\includegraphics[width=0.8em]{triarrow.eps}}}
\def\l{l^{(k)}}


\newcommand{\f}[3]{f_{#1#2}^{(#3)}}
\newcommand{\ed}[2]{#1#2}
\newcommand{\cy}[3]{\langle#1#2#3\rangle}
\newcommand{\deltas}[3]{\delta_{\ed#1#2}+\delta_{\ed#2#3}+\delta_{\ed#3#1}}
\newcommand{\vv}[1]{(#1_1,#1_2)}
\newcommand{\frdim}[1]{{\fdim}(\mathbf{#1})}
\newcommand{\intdim}[1]{{\dim_{I}}(\mathbf{#1})}

\def\gran{\ensuremath{\Delta}}
\def\R{\mathcal{R}}
\def\M{\mathcal{M}}
\def\sD{\mathcal D}
\def\F{\mathcal{F}}
\def\prob{1|prec\,|\sum w_jC_j}
\def\cs{G_{CS}(P)}
\def\am{G_{AM}(P)}
\def\uniformrealizer{uniform realizer\ }
\def\uniform{uniform\ }

\def\GP{G_\mathbf{P}}

\def\NP{\textsc{NP}}
\def\P{\textsc{P}}

\newcounter{myclaim}
\setcounter{myclaim}{0}
\newenvironment{myclaim}{\refstepcounter{myclaim}\begin{sloppypar}\noindent {\it Claim }\arabic{myclaim}.\ }{\end{sloppypar}}


\newcommand{\HK}{\ensuremath{\mbox{[HK]}}}
\newcommand{\MS}{\ensuremath{\mbox{removable pairing}}\xspace}
\newcommand{\LP}[1]{\ensuremath{LP(#1)}}
\newcommand{\OLP}[1]{\ensuremath{OPT_{LP}(#1)}}
\newcommand{\LPST}[3]{\ensuremath{OPT_{LP}(#1,#2,#3)}}
\newcommand{\oa}[1]{\ensuremath{\overrightarrow{#1}}}
\newcommand{\opt}{\ensuremath{OPT}}
\newcommand{\TSP}{graph-TSP\xspace}
\newcommand{\HPP}{graph-TSPP\xspace}

\newcommand{\dist}[1]{\ensuremath{dist(#1)}}
\renewcommand{\algorithmicrequire}{\textbf{Input:}}
\renewcommand{\algorithmicensure}{\textbf{Output:}}

\begin{document}
\maketitle

\begin{abstract}
  We present a framework for approximating the metric TSP based on a
  novel use of matchings. Traditionally, matchings have been used to
  add edges in order to make a given graph Eulerian, whereas our
  approach also allows for the removal of certain edges leading to a
  decreased cost.

  For the TSP on graphic metrics (\TSP), the approach yields a
  -approximation algorithm with respect to the
  Held-Karp lower bound. For \TSP restricted to a class of graphs that
  contains degree three bounded and claw-free graphs, we show that the
  integrality gap of the Held-Karp relaxation matches the conjectured
  ratio . The framework allows for generalizations in a natural way and
  also leads to a -approximation algorithm for the traveling
  salesman path problem on graphic metrics where the start and end vertices are
  prespecified.
\end{abstract}


\section{Introduction}\label{sec:intro}
    The traveling salesman problem in metric graphs is one of most
    fundamental NP-hard optimization problems. In spite of a vast amount
    of research several important questions remain open. While the problem
    is known to be APX-hard and NP-hard to approximate with a ratio better than
     \cite{PV06}, the best upper bound is still the
    1.5-approximation algorithm obtained by Christofides~\cite{Chr76}
    more than three decades ago.
    A promising direction to improve this approximation guarantee, has
    long been to understand the power of a linear program known as the
    Held-Karp relaxation~\cite{HK70}. On the one hand, the best lower
    bound on its integrality gap (for the symmetric case) is  and
    indeed conjectured to be tight \cite{Goe95}. On the other hand, the
    best known analysis~\cite{SW90, Wol80} is based on Christofides'
    algorithm and gives an upper bound on the integrality gap of .

    In the light of this difficulty of even determining the integrality gap of
    the Held-Karp relaxation, a reasonable way to approach the metric TSP
    is to restrict the set of feasible inputs. One promising candidate is
    the \emph{\TSP,} that is, the traveling salesman problem where
    distances between cities are given by any graphic metric, i.\,e., the
    distance between two cities is the length of the shortest path in a
    given (unweighted) graph. Equivalently, \TSP can be formulated as the
    problem of finding an Eulerian multigraph within an unweighted input
    graph so as to minimize the number of edges. In contrast to TSP on
    Euclidean metrics that admits a PTAS~\cite{Arora98,Mitch99}, the \TSP
    seems to capture the difficulty of the metric TSP in the sense that,
    as stated in \cite{GKP95}, it is APX-hard and the lower bound 4/3 on
    the integrality gap of the Held-Karp relaxation is established using a
    \TSP instance.

    The TSP on graphic metrics has recently drawn considerable attention.
    In 2005, Gamarnik et al.~\cite{GLS05} showed that for cubic
    3-edge-connected graphs, there is an approximation algorithm achieving
    an approximation ratio of . This result was generalized to
    cubic graphs by Boyd et al.~\cite{BSSS11}, who obtained an improved
    performance guarantee of 4/3.  For subcubic graphs, i.\,e., graphs of
    degree at most , they also gave an 7/5-approximation algorithm with
    respect to the Held-Karp lower bound.  In a major achievement, Gharan
    et al.~\cite{GSS11} recently presented an approximation algorithm for
    \TSP with performance guarantee strictly better than 1.5. The approach
    in~\cite{GSS11} is similar to that of Christofides in the sense that
    they start with a spanning tree and then add a perfect matching of
    those vertices of odd-degree to make the graph Eulerian. The main
    difference is that instead of starting with a minimum spanning tree,
    their approach uses the solution of the Held-Karp relaxation to sample
    a spanning tree. Although the proposed algorithm in~\cite{GSS11} is
    surprisingly simple, the analysis is technically involved and several novel
    ideas are needed to obtain the improved performance guarantee
     for an  of the order .

\paragraph{Our Results and Overview of Techniques.}
    We propose an alternative framework for approximating the metric
    TSP and use it to obtain  an improved approximation
    algorithm for \TSP.
    \begin{theorem}\label{thm:approximationratio}
      There is a polynomial time approximation algorithm for \TSP with
      performance guarantee .
    \end{theorem}
    The result implies an upper bound on the integrality gap of the Held-Karp
    relaxation for \TSP that matches the approximation ratio.
    For the restricted class of graphs, where each block (i.\,e., each
    maximally 2-vertex-connected subgraph) is either claw-free or of
    degree at most , we use the framework to construct a polynomial
    time -approximation algorithm showing that the conjectured
    integrality gap of the Held-Karp relaxation is tight for those
    graphs. In fact, the techniques allow us to prove the tight result
    that any -vertex-connected graph of degree at most  has a
    spanning Eulerian multigraph with at most  edges, which
    settles a conjecture of Boyd et al.~\cite{BSSS11} affirmatively.

    Our framework is based on earlier works by Frederickson \&
    Ja'ja'~\cite{FJJ89} and Monma et al.~\cite{MMP90}, who related the
    cost of an optimal tour to the size of a minimum -vertex-connected
    subgraph.  More specifically, Monma et al. showed that a
    -vertex-connected graph  always has a spanning Eulerian
    multigraph with at most  edges, generalizing a
    previous result of Frederickson \& Ja'ja' who obtained
    the same result for the special case of planar -vertex-connected
    graphs.
    One interpretation of their approaches is the following.  Given a
    -vertex-connected graph , they show how to pick a random
    subset  of edges satisfying: (i) an edge is in  with probability
     and (ii) the multigraph  with vertex set  and edge set  is spanning and Eulerian. From property  of
    , the expected number of edges in  is  yielding
    their result.

    Although the factor  is asymptotically tight for some classes of
    graphs (one example is the family of integrality gap instances for the
    Held-Karp relaxation described in Section~\ref{sec:prelim}), the bound
    rapidly gets worse for -vertex-connected graphs with significantly
    more than  edges. The novel idea to overcome this issue is the
    following.  Instead of adding all the edges in  to , some of the
    edges in  might instead be removed from  to form . As long as
    the removal of the edges does not disconnect the graph, this will
    again result in a spanning Eulerian multigraph . To specify a
    subset  of edges that safely may be removed we introduce, in
    Section~\ref{sec:tspframework}, the notion of a ``\MS''. The framework
    is then completed by Theorem~\ref{thm:main}, where we show that a
    -vertex-connected graph  with a set  of removable edges
    has a spanning Eulerian multigraph with at most 
    edges.

    In order to use the framework, one of the main challenges is to find a
    sufficiently large set of removable edges. In
    Section~\ref{sec:circulation}, we show that this problem can be
    reduced to that of finding a min-cost circulation in a certain
    circulation network.  To analyze the circulation network we then (in
    Section~\ref{sec:algorithms}) use several properties of an extreme point solution to
    the Held-Karp relaxation to obtain our main algorithmic result. The
    better approximation guarantees for special graph classes follows from
    that the circulation network has an easier structure in these cases,
    which in turn allows for a better analysis.

    Finally, we note that the techniques generalize in a natural way. Our
    results can be adapted to the more general traveling salesman path
    problem (\HPP) with prespecified start and end vertices to improve on the
    approximation ratio of  by Hoogeveen \cite{Hoo91} when
    considering graphic metrics. More specifically, we obtain the
    following.
    \begin{theorem}\label{thm:approximationratiohpp}
        For any , there is a polynomial time approximation algorithm for
        \HPP with performance guarantee 
        .

        If furthermore each block of the given graph is degree three bounded, there is a
        polynomial time approximation algorithm for \HPP with performance
        guarantee , for any .
    \end{theorem}
    The generalization to the traveling salesman problem is presented in
    Section~\ref{sec:tspp}.

\section{Preliminaries}\label{sec:prelim}
    \vspace{-0.1cm}
    \paragraph{Held-Karp Relaxation.} The linear program known as the
    Held-Karp (or subtour elimination) relaxation is a well studied lower
    bound on the value of an optimal tour. It has a variable 
    for each pair of vertices with the intuitive meaning that
     should take value  if the edge  is used in
    the tour and  otherwise. Letting  be the complete graph on
    the set of vertices and  be the distance between vertices
     and , the Held-Karp relaxation can then be formulated as the
    linear program where we wish to  minimize  subject to

    where  denotes the set of edges crossing the cut  and
     for any .

    Goemans \& Bertsimas~\cite{GB90} proved that for metric distances the
    above linear program has the same optimal value as the linear program
    obtained by dropping the equality constraints. Moreover, when
    considering a \TSP instance  we only need to consider the
    variables . Indeed, any solution  to the
    Held-Karp relaxation without equality constraints such that 
    for a pair of vertices  can be transformed into a
    solution  with no worse cost and  by setting
     for each edge on the shortest path
    between  and , and  for the 
    other edges.  The Held-Karp relaxation for \TSP on a graph 
    can thus be formulated as follows:
    
    We shall refer to this linear program as \LP{G} and denote the value
    of an optimal solution by \OLP{G}.  Its integrality gap was previously known to
    be at most  and at least  for graphic instances.
The lower bound
    is obtained by a claw-free graphic instance of degree at most  that
    consists of three paths of equal length with endpoints  and  that are connected so as  and  form two triangles (see Figure~\ref{fig:intgap}).

    We end our discussion of \LP{G} with a useful observation. When
    considering \TSP{}, it is intuitively clear that we can restrict
    ourselves to \emph{-vertex-connected} graphs, i.\,e., graphs that
    stay connected after deleting a single vertex. Indeed, if we consider
    a graph with a vertex  whose removal results in components  with  then we can recursively solve the
    \TSP{} problem on the  subgraphs 
    induced by . The union of these solutions will then provide a solution to
    the original graph that preserves the approximation guarantee with
    respect to the linear programming relaxation since one can see that
    . We summarize this
    observation in the following lemma (see Appendix~\ref{app:2connTSP}
    for a fullproof).
    \begin{lemma}
    \label{lemma:2connTSP}
      Let  be a connected graph. If there is an -approximation
      algorithm for \TSP{} on each -vertex-connected subgraph  of  (with
      respect to \OLP{H}) then there is an -approximation algorithm for \TSP on 
       (with respect to \OLP{G}).
    \end{lemma}

    \vspace{-0.4cm}
\paragraph{Matchings of Cubic -Edge-Connected Graphs.}
    Edmonds~\cite{Edmonds1965b} showed that the following set of
    equalities and inequalities on the variables 
    determines the perfect matching polytope (i.\,e.,  all extreme points of the polytope are integral and correspond to perfect matchings)
    of a given graph :
    
    The linear description is useful for understanding the structure of
    the perfect matchings. For example, Naddef and Pulleyblank~\cite{NP81}
    proved that  defines a feasible solution when  is
    \emph{cubic} and \emph{-edge connected}, i.\,e., every vertex has
    degree  and the graph stays connected after the removal of an
    edge. They used that result to deduce that such graphs always have a
    perfect matching of weight at least  of the total weight of the
    edges.

    Standard algorithmic versions of Carath\'{e}odory's theorem (see
    e.\,g. Theorem~ in~\cite{GLS1988}) say that, in polynomial time, we can
    decompose a feasible solution to the perfect matching polytope into a
    convex combination of polynomially many perfect matchings (see
    also~\cite{Barahona04} for a combinatorial approach for the matching
    polytope).
    Combining these results leads to the following lemma (see~\cite{BSSS11,GLS05,MMP90} for
    closely related variants that also have  been useful for the
    \TSP problem).
    \begin{lemma}
    \label{lem:matching}
      Given a cubic -edge-connected graph , we can in polynomial time
      find a distribution over polynomially many perfect matchings so that
      with probability  an edge is in a  perfect matching picked from this distribution.
    \end{lemma}
    Note that all -vertex-connected graphs except the trivial graph on
     vertices are -edge connected. We can therefore apply the above
    lemma to cubic -vertex-connected graphs.

    \vspace{-0.2cm}
\section{Approximation  Framework}
\label{sec:tspframework}
Lemma~\ref{lemma:2connTSP} says that the technical difficulty in
approximating the \TSP problem lies in approximating those instances
that are -vertex connected.
As alluded to in the introduction, we shall generalize previous
results~\cite{FJJ89,MMP90} that relate the cost of an optimal tour to
the size of a minimum -vertex-connected subgraph. The main
difference is the use of matchings. Traditionally, matchings have been
used to add edges to make a given graph Eulerian whereas our framework
offers a structured way to specify a set of edges that safely may be
removed leading to a lower cost. To identify the set of edges that may
be removed we use the following definition.

\vspace{-0.1cm}
\begin{definition}[Removable pairing of edges] \label{def:pairing}
  Given a -vertex-connected graph  we call a tuple  consisting of a  subset  of removable edges and  a subset    of pairs of edges a  \emph{\MS{}} if
\vspace{-0.2cm}
  \begin{itemize}\itemsep-1mm
    \item an edge is in at most one pair;
    \item the edges in a pair are incident to a common vertex of degree at least ;
    \item any graph obtained by deleting removable edges so that at most one edge in each pair is deleted stays connected.
  \end{itemize}
\end{definition}

The following theorem generalizes the corresponding result of~\cite{MMP90} (their
result follows from the the special case of an empty removable pairing).
\begin{theorem}
\label{thm:main}
Given a -vertex-connected graph  with a \MS{} ,
there is a polynomial time algorithm that returns a
 spanning Eulerian
multigraph in  with at most
   edges.
\end{theorem}
The proof of the  theorem is presented after the following lemma on which it is based.
\begin{lemma}
\label{lemma:sample}
Given a -vertex-connected graph  with a \MS{} , we
can in polynomial time find a distribution over polynomially many
subsets of edges such that a random subset  from this distribution
satisfies:
\vspace{-0.2cm}
\begin{itemize}\itemsep-1mm
\item[(a)] each edge is in  with probability ;
\item[(b)] at most one edge in each pair is in ; and
\item[(c)] each vertex has an even degree in the multigraph with edge set .
\end{itemize}
\end{lemma}
\begin{proof}
  We shall use Lemma~\ref{lem:matching} and will therefore need a
  cubic -edge-connected graph. In the spirit of~~\cite{FJJ89}, we replace
  all vertices of  that are not of degree three by gadgets to
  obtain a cubic graph  as follows (see also
  Figure~\ref{fig:degreplace}):
\begin{itemize}
\item A vertex  of degree 2 with neighbors  and  is replaced
  by a cycle consisting of four vertices , , , 
  with the chord . The gadget is then connected to the
  neighbours of  by the the edges  and .
    
\item A vertex  with  is replaced by a tree  that
  has  leaves, a binary root if  is odd,
  and otherwise only degree  internal vertices. Each leaf is
  connected to two neighbours of  such that the edges incident to 
  that form a pair in  are incident to the same leaf.  If  is
  odd, one of the neighbors is left and connected to the binary root.
\end{itemize}

\begin{figure}[bt]
\begin{center}
\includegraphics[width=14cm]{degreereplace}
\end{center}
\caption{Examples of the used gadgets to obtain a cubic graph.}
\label{fig:degreplace}
\end{figure}
The above gadgets guarantee that the graph  is cubic and it is
-vertex connected since  was assumed to be -vertex connected.
We can therefore apply Lemma~\ref{lem:matching} in order to obtain
a random perfect matching . Each edge of  is in  with a
probability of exactly 1/3. Let  be the set of edges obtained by
restricting  to the edges of  in the obvious way. Now 
contains each edge of  with probability
1/3.
We complete the proof by showing that  also satisfies properties
 and .  As each pair of edges in  is incident to a vertex
of degree at least , we have, by the construction of the gadgets,
that they are incident to a common vertex in  and hence at most
one edge of each pair is in . 
Finally, property  follows from that  is clearly a spanning Eulerian
multigraph of  and compressing a set of even-degree vertices results in one vertex of even degree.
\end{proof}

Equipped with the above lemma we are now ready to prove the main result
of this section.
\begin{proofof}{Theorem~\ref{thm:main}}
  Pick a random subset  of edges that satisfies the
  properties of Lemma~\ref{lemma:sample}. Let  be the set of those edges of
   that are removable and let  be the set of the remaining edges of
  .

Consider the multigraph  on vertex set  and edge set . Observe that both adding an edge and
removing an edge swaps the parity of the degree of an incident
vertex. We have thus from property  of Lemma~\ref{lemma:sample}
that the degree of each vertex in  is even. Moreover, as  is a
removable pairing, property  of Lemma~\ref{lemma:sample}
gives that  is connected. Alltogether we have that  is an
Eulerian graph, i.\,e., a \TSP{} solution. We continue to calculate its
expected number of edges, which is
 
Using that each edge is in  with probability ,  we have, by linearity of
expectation, that~\eqref{eq:nredges} equals
 

To conclude the proof, we note that the selection of  can be
derandomized since there are, by Lemma~\ref{lemma:sample},
polynomially many edge subsets to choose from; taking the one that
minimizes the number of edges of   is sufficient.
\end{proofof}

\section{Finding a Removable Pairing by Minimum Cost Circulation}\label{sec:circulation}
In order to use our framework, one of the main challenges is to find a
\MS{} that is sufficiently large. In the following, we show how to obtain a
useful \MS{} based on circulations.

Consider a -vertex connected graph  and let  be a spanning
tree of  obtained by depth-first search (starting from some
arbitrary root ). Then each edge in  connects a vertex to either
one of its predecessors or one of its successors. We call the edges in
 \emph{tree-edges} and those in  but not in 
\emph{back-edges}.

We shall now define a circulation network .
We start
by introducing an orientation of : all tree-edges
become tree-arcs directed from the root to the leaves and all 
back-edges become back-arcs directed towards the root. To distinguish the circulation
network and the original graphs, we use the names  and  for the
network versions of  and .  In order to ensure connectivity properties of
subnetworks obtained from feasible circulations, we replace some of the vertices
by gadgets. 

For each vertex  except the root that has  children  in the tree, we introduce  new vertices ,
, ,  and replace the tree-arc  by the tree-arcs  and  for . 
Then we redirect all incoming back-arcs of  from the subtree rooted
by  to . For an illustration of the gadget see
Figure~\ref{fig:circreplace} and for an example of a complete network
see Figure~\ref{fig:circreplace_appendix}.  This way,
all back-arcs start in old vertices and lead to new vertices or the
root.  In the following, we call the new vertices and the root
\emph{in-vertices} and the remaining old ones \emph{out-vertices}. We
also let  be the set of all in-vertices.

\begin{figure}[bt]
\begin{center}
  \includegraphics[width=10cm]{circreplace2}
\end{center}
\caption{The gadget that, for each child of , introduces a new vertex (depicted in
  white) and redirects back-arcs.}
\label{fig:circreplace}
\end{figure}


We now specify
a lower bound (demand) and an upper bound (capacity) on the circulation.
For each arc  in , we set the demand of  to 1 and for
all other arcs to
0.
The capacity is  for any arc.  Finally, the cost of a circulation 
in  is the piecewise linear function , where  is the set of incoming back-arcs of
. One can think of the cost as the total circulation on the
back-arcs except that each in-vertex accepts a circulation of  for
free.  Note that algorithmically there is no considerable difference
whether we use our cost function or define a linear cost function on
the arcs: for any in-vertex  we can redirect all back-arcs of 
to a new vertex  and introduce two arcs , one of cost 
and capacity  and the other  of cost  and capacity . All
remaining arcs then have a cost of .

The following lemma shows how to use a circulation in  to approximate
\TSP.
\begin{lemma}
\label{lemma:costcirc}
Given a -vertex connected graph  and a depth first search tree
 of  let  be the minimum cost circulation to  of
cost . Then there is a spanning Eulerian multigraph  in
 with at most  edges.
\end{lemma}
\begin{proof}
  \newcommand{\flowT}{\ensuremath{\oa{T}}}
  \newcommand{\graphC}{\ensuremath{{C^*(G,T)}}} We first note that,
  for any arc of , the demand and the capacity is integral.
  Therefore, applying Hoffman's circulation theorem (see \cite{Sch03},
  Corollary 12.2a), we can assume the circulation  to be
  integral.  Let  be the support of  in ,
  i.\,e., the induced subgraph of the arcs with non-zero circulation
  in , and let  be the subgraph of  obtained from
   by compressing the gadges of the circulation network in
  the obvious way. 

  To prove the lemma, we shall first prove that graph  is
  -vertex connected and then define a removable pairing  on
   in order to apply Theorem~\ref{thm:main}. That  is
  -vertex connected follows from flow conservation, that each arc
   in  has demand , and the design of the
  gadgets. Indeed, if  would have a cut vertex  with children
   in  then one of the subtrees, say the
  one rooted by , has no back-edges to the ancestors of  which
  in turn, by flow conservation, would contradict that the tree-arc
   in  carries a flow of at least . (Recall that
  the edge  in  is replaced by tree-arcs  and
   in .)

  We now determine a \MS  on . For ease of argumentation we
  shall first slightly abuse notation and define a \MS  on
  .
  The set  consists of all  such that  is a
  back-arc of cost zero in ,  has at least two incoming
  arcs, and  is a tree-arc.  Note that each such  is an
  in-vertex, the number of incoming back-arcs of cost zero is at most
  one,  is the unique outgoing tree-arc of , and the only
  possible vertex  with only one incoming back-arc and no other
  incoming arc is the root.
  The set  contains all edges from  and additionally all
  remaining back-arcs of .  In other words, each edge of
   that is neither in  nor in  is a back-arc with
  integer non-zero cost in the circulation or a back-arc to the
  root. Hence,  if the root has more than one
  incoming back-arc and  otherwise.

  The \MS  on  is now obtained from , by
  mereley compressing the gadgets used to form  and by
  dropping the orientations of the arcs. As all edges in  are
  either back-arcs or they are tree-arcs starting from an in-vertex,
  no arc in  is removed by the compression and thus 
  and . Moreover,  has 
  edges and, assuming  is a valid \MS, Theorem~\ref{thm:main} 
  yields that  (and thus  has a spanning Eulerian multigraph
  with at most  edges. The
  last inequality followed  from that   is at most
  .
 
  Therefore, we can conclude the proof by showing that  is a
  valid \MS.  It is easy to verify that  satisfies the first
  two conditions of Definition~\ref{def:pairing}, that is, each edge
  is contained in at most one pair and the edges in each pair are
  incident to one common vertex of degree at least three. The third
  condition follows from that, for any vertex  of , the
  vertices in the subtree  of  rooted by  form a connected
  subgraph of  even after removing edges according to . To
  see this we do a simple induction on the depth of . In the base
  case,  is a leaf and the statement is clearly true. For the
  inductive step, consider a vertex  with  children  in . By the inductive hypothesis, the
  vertices in  for  stay connected after
  the removal of edges according to . To complete the inductive
  step it is thus sufficient to verify that  is connected to each
   after the removal of edges. If  is not in 
  this clearly holds. Otherwise if  then by
  the definition of  there is an edge  such that  and  is incident to  and a vertex in . Since
  at most one edge in each pair is removed we have that  also stays
  connected to  in this case, which completes the inductive
  step. We have thus proved that  satisfies the properties of a
  \MS which completes the proof of the statement.

\end{proof}

\section{Improved Approximation Algorithms}
\label{sec:algorithms}
We first show how to apply our framework to restricted graph classes
for which we obtain a tight bound on the integrality gap of the
Held-Karp relaxation. We then show how to use our framework to obtain
an improved approximation algorithm for general graphs.

\subsection{Bounded Degree and Claw-Free Graphs}
We consider the class of graphs that have a degree bounded by three.
\begin{lemma}\label{lem:boundeddeg}
  Given a -vertex-connected graph  with  vertices, there is a
  polynomial time algorithm that computes a spanning Eulerian
  multigraph  in  with at most  edges.
\end{lemma}
\begin{proof}
  If  has one or two vertices, we obtain an Eulerian multigraph of
  zero or two edges. Otherwise, we compute a depth-first search tree
   in  and determine the circulation network . We now
  show that this network has a feasible circulation  of cost at
  most one. Let us assign a circulation of one to each back-arc  in
   and push it through the path in  that is incident
  to both the start and end vertex of . By the construction of
   and from the assumption that  is -vertex connected,
  each tree-arc is in a directed cycle that contains exactly one
  back-arc. Therefore, all demand constraints are satisfied. Due to
  the degree-bounds, no vertex but the root has more than one incoming
  back-arc. The cost  of
  the circulation is therefore at most one and zero if the root has
  only one back-arc.  If the circulation cost is zero, by
  Lemma~\ref{lemma:costcirc} we obtain a spanning Eulerian multigraph
   in  with at most  edges. For those circulations
  where the cost is one, the proof of Lemma~\ref{lemma:costcirc}
  allows to save an additional constant of  (since then the root
  has more than one incoming back-arc) and we obtain the same bound on
  the number of edges.

\end{proof}
Note that it is sufficient to find a 2-vertex-connected
degree three bounded spanning subgraph (a 3-trestle) and thus, using a
result from \cite{KKN01}, we can apply Lemma~\ref{lem:boundeddeg} also
to claw-free graphs.  Applying Lemma~\ref{lemma:2connTSP}, we obtain
an upper bound of 4/3 on the integrality gap for the Held-Karp
relaxation for the considered class of graphs. In addition, along the
lines of the proof of Lemma~\ref{lemma:2connTSP}, one can see that the
above arguments imply that any connected graph  decomposed into 
blocks, i.\,e., maximal -connected subgraphs, such that each block is
either degree three bounded or claw-free, has a spanning Eulerian
multigraph with at most  edges.

\subsection{General Graphs}
We now apply our framework to graphs without degree constraints.  We
start with an algorithm that achieves an approximation ratio better
than  for graphs for which the linear programming relaxation has
a value close to .  Let  be an -vertex graph.  The
support  of an extreme point  of \LP{G} is
known to contain at most  edges~(see Theorem~
in~\cite{CFN85}). Moreover, if we let  be an optimal solution,
then any -approximate solution to graph  with respect to
\OLP{G'} is an -approximate solution to  with respect to
\OLP{G}, because  and . We can
thus restrict ourselves to -vertex graphs with at most  edges
and, by Lemma~\ref{lemma:2connTSP}, we can further assume the graph to
be -vertex connected.
\begin{algorithm}[h]
\begin{algorithmic}[1]
\REQUIRE A 2-vertex-connected graph  with  vertices and at most  edges.
\STATE Obtain an optimal solution  to \LP{G}.
\STATE Obtain a depth-first-search tree  of  by starting at some
  root and in each iteration pick, among the possible edges, the edge 
  with maximum .
\STATE Solve the min cost circulation problem  to obtain a circulation  with cost .

\STATE Apply Lemma~\ref{lemma:costcirc} to find a spanning Eulerian
  multigraph with less than  edges.
\end{algorithmic}
\caption{}
\label{alg:allgraphs}
\end{algorithm}

To analyze the approximation ratio achieved by
Algorithm~\ref{alg:allgraphs}, we bound the cost of the circulation.
\begin{lemma}
\label{lemma:circcost}
  We have .
\end{lemma}
\begin{proof}
  For notational convenience, when considering an arc  in the flow
  network we shall slightly abuse notation and use  to denote
  the value of the corresponding edge in  according to the optimal
  LP-solution .
  We prove the statement by defining a fractional circulation  of
  cost at most . The
  circulation  will in turn be the sum of two circulations  and
  . We obtain the circulation  as follows: for each back-arc
   we push a flow of size  along the cycle formed by
   and the tree-arcs in .  We shall now define the circulation
   so as to guarantee that  forms a feasible circulation,
  i.\,e., one that satisfies the demands  for each . As out- and in-vertices are alternating in  and
  in-vertices have only one child in  and no outgoing
  back-edges, a sufficient condition for  to be feasible can be
  seen to be  for each  that is from an
  out-vertex to an in-vertex. To ensure this, we now define  as
  follows. For each vertex  of  that is replaced by a gadget
  consisting of an out-vertex  and a set  of
  in-vertices, we push for each  a flow of size
   along a cycle that includes the arc 
  (and one back-arc).
  Note that such a cycle is guaranteed to exist since  was assumed to be
  -vertex connected.
  From the definition of , we have thus
  that  defines a feasible circulation.

  We proceed by analyzing the cost of , i.\,e., , where  is the set
  of all in-vertices and  is the set of incoming back-arcs of
  . Note that the cost is upper bounded by
   and we can thus analyze these two terms
  separately. We start by bounding the second summation and then
  continue with the first one. If  then one can see that . Moreover,
  \begin{claim}
    \label{claim:firstcost}
    We have .
  \end{claim}
  \begin{proofclaim}
    When considering a vertex  as done above in the definition of
    , the flow pushed on back-arcs is  which equals , where . Letting  be the set of vertices of  in the
    subtree of the undirected tree  rooted by the child of  ,
    we have, by the definition of ,

The second equality follows from that if  for some  then  and hence . We have thus 

As we are considering a depth-first-search tree (see
Figure~\ref{fig:circcostOLA}),

\begin{figure}[bt]
\begin{center}
\includegraphics[width=7cm]{circcost}
\end{center}
\caption{An illustration of Equality~\eqref{eq:equalsums} with : both the
  left-hand-side and the right-hand-side of the equality express two
  times the value of the fat edges.}
\label{fig:circcostOLA}
\end{figure}
Since by the feasibility of  each of the sets corresponds to a cut of fractional value at least  we use  as a lower bound on~\eqref{eq:equalsums}.

\noindent Summarizing the above calculations yields


Repeating this argument for each  we have ,
which equals  since .
\end{proofclaim}
We proceed by bounding  from above.  
\begin{claim}
\label{claim:secondcost}
We have 
\end{claim}
\begin{proofclaim}
To analyze this expression we shall use two facts. First
 has at most  edges, and therefore the number of back-arcs
is at most .  Second, as the depth-first-search
chooses (among the available edges) the edge  with maximum  in
each iteration, we have that  for each  where  is the outgoing tree-arc of . Moreover, as  for each
back-arc, the number of back-arcs in  is at least .
Combining these two facts gives us that
 
For , we partition  into  and . Furthermore, let . With
this notation we can upper bound  by

and relax Inequality~\eqref{eq:cap2} to


The cost~\eqref{eq:fcost} (where we ignore ) subject
to~\eqref{eq:cap} can now be interpreted as a knapsack problem of
capacity  that is packed with an item of profit  and size  for each . Consequently, we can upper bound~\eqref{eq:fcost} by
considering the fractional knapsack problem with capacity  and
infinitely many items of a maximized profit to size ratio. Associating
a variable  with  and  with  this ratio is  For
any  the ratio is maximized by letting  and we can thus
restrict our attention to items with profit to size ratio
. A simple analysis (see Appendix~\ref{sec:maxsizratio})
shows that the maximum is achieved when . Therefore, the profit~\eqref{eq:fcost} is upper bounded by

As the fractional degree of a vertex  that is replaced by a gadget
with a set  of in-vertices is at least , we have . Hence,

which equals .
\end{proofclaim}
Finally, by summing up the bounds given by Claim~\ref{claim:firstcost}
and Claim~\ref{claim:secondcost} we bound the cost of  and hence  from above by

which equals .
\end{proof}

Having analyzed Algorithm~\ref{alg:allgraphs}, we are ready to prove our main algorithmic result.
\begin{apptheorem}{\textbf{\ref{thm:approximationratio}}} (Restated)\emph{ 
  There is a polynomial time approximation algorithm for \TSP with
  performance guarantee .}
\end{apptheorem}

\begin{proof}
  By Lemma~\ref{lemma:2connTSP} and the discussion before
  Algorithm~\ref{alg:allgraphs}, we can restrict ourselves to
  -vertex graphs that are -vertex connected and have at most
   edges.  The statement now follows by using
  Algorithm~\ref{alg:allgraphs} if  is close to  and
  otherwise by using Christofides' algorithm. 

  On the one hand, since Christofides' algorithm returns a solution
  with at most  edges (see~\cite{SW90} for an
analysis of Christofides' algorithm in terms of \OLP{G}), it has an approximation guarantee of at
  most

On the other hand, by Lemma~\ref{lemma:circcost}, the approximation
guarantee of Algorithm~\ref{alg:allgraphs} is at most

In particular, the approximation guarantee of
Algorithm~\ref{alg:allgraphs} for a graph  with  is  but deteriorates as \OLP{G}
increases. The approximation guarantee of Christofides' algorithm on
the other hand is getting better and better as \OLP{G} increases.
\begin{figure}[tb]
    \begin{center}
        \includegraphics[width=10cm]{approx.mps}
    \end{center}
    \caption{The approximation ratios of
    Algorithm~\ref{alg:allgraphs} and Christofides' algorithm depending on
    the ratio .}
    \label{fig:ratios}
\end{figure}
Comparing these two ratios, one gets that the worst case happens when
 (see Figure~\ref{fig:ratios}) and, by using simple
arithmetics, the approximation guarantee can be seen to be
.
\end{proof}

\section{The Traveling Salesman Path Problem}\label{sec:tspp}
\label{sec:tspp}
In this section, we describe a sequence of generalizations and modifications of
the techniques that we previously presented for \TSP and conclude with improved
approximation algorithms for the traveling salesman path problem on graphic
metrics, \HPP.
\subsection{Using Held-Karp for Graph-TSPP}\label{sec:HKpath}
We can obtain a natural generalization of  to \HPP by distinguishing
whether the end vertices  and  are in the same set of
vertices. To this end, let .  Then the
relaxation can be written as
2mm]
  x(\delta(S)) & \geq  2, & \emptyset \neq S \subset V, S \in \Phi\2mm]
  x & \geq 0.
\end{aligned} 

\frac{4}{3} (|E|+1) - \frac{2}{3} |R| + \frac{\dist{s,t}-2}{3} - 2/3 =
\frac{4}{3} |E| - \frac{2}{3} |R| + \frac{\dist{s,t}}{3}.

    &&\frac{4}{3}n + \frac{2}{3}(6(1-\sqrt{2})n + (4\sqrt{2} -3)\OLP{G'}) -
        \frac{2}{3} + \frac{\dist{s,t}}{3} \nonumber\\
    &=&(16/3 - 4\sqrt{2})n + \dist{s,t}/3 + (8 \sqrt{2}/3 - 2)(\OLP{G'}) - 2/3
\label{eqn:hppapprox}
\frac{16/3-4\sqrt{2}+d/3}{\zeta} + 8\sqrt{2}/3 - 2 + \epsilon_1,
\label{eqn:hppapproxtwo}
    (2-d)/\zeta.
\label{eqn:equalhpp}
    \zeta = \frac{12\sqrt{2}-4d-10}{8\sqrt{2}-6}.

\frac{8\sqrt{2}-6-4d\sqrt{2}+3d}{6\sqrt{2}-2d-5}.

\frac{8\sqrt{2}-6-4(\sqrt{2}-1)\sqrt{2}+3(\sqrt{2}-1)}{6\sqrt{2}-2(\sqrt{2}-1)-5}
= \frac{15\sqrt{2}-17}{4\sqrt{2}-3} = 3 - \sqrt{2}.

3-\sqrt{2} + \varepsilon
s_i = \begin{cases} 
        s & \mbox{if }s\in C_i \\ 
        v & \mbox{otherwise}
      \end{cases}\qquad  \mbox{and} \qquad t_i = \begin{cases} 
        t & \mbox{if }t\in C_i \\ 
        v & \mbox{otherwise}
      \end{cases}.
\frac{d}{dT}f(T) = \frac{1}{2-T} + \frac{T}{(2-T)^2}  - \left(\frac{2T}{2-T} + \frac{T^2}{(2-T)^2}\right) = \frac{1-2T}{2-T} + \frac{T-T^2}{(2-T)^2}.

(1-2T)(2-T) + T-T^2 = 0 \Leftrightarrow T^2 - 4T + 2 = 0 \Leftrightarrow T = 2 \pm \sqrt{2}.

It is now easy to verify that the unique maximum of  for  is obtained when .
\end{document}
