We first focus on the problem of language inclusion \upto~an independence relation (\defref{langinc}).  This question corresponds to preemption-safety (\thmref{correctness}, \propref{inclusion}) and its solution drives our synchronization synthesis (\secref{algo}).


\begin{theorem}\label{thm:langinc.undecidable}
For \nfas~ over alphabet  and an independence relation ,  is undecidable\cite{bertoni1982equivalence}.  
\end{theorem} 

Fortunately, a bounded version of the problem is decidable. 
Recall the relation  over  from \subsecref{langincdef}. 
We define a symmetric binary relation  over :
 iff : 
,  and 
.
Thus  consists of all words that can be optained from each other by commuting 
the symbols at positions  and .
We next define a symmetric binary relation  over : 
 iff : 
 
and .
The relation  intuitively consists of words obtained from each other
by making a single forward pass commuting multiple pairs of adjacent symbols.
Let  denote the -composition of  with itself. 
Given a language  over , 
we use  to denote the set \scriptstyle k.
In other words,  consists of all words which can be generated from 
 using a finite-state transducer that remembers at most  symbols 
of its input words in its states.  

\begin{definition}[Bounded language inclusion \upto~an independence relation]
Given \nfas~ over ,  and a constant , the -bounded language inclusion problem \upto~ is: ? 
\end{definition}

\begin{theorem}\label{thm:boundlanginc}
For \nfas~ over ,  and a constant ,  
 is decidable. 
\end{theorem}

We present an algorithm to check -bounded language inclusion
\upto~, based on the antichain algorithm for standard language
inclusion \cite{de2006antichains}. 

\noindent{\bf Antichain algorithm for language inclusion}. 
Given a partial order , an antichain over  is
a set of elements of  that are incomparable w.r.t. . In order
to check  for \nfas~ and ,
the antichain algorithm proceeds by exploring  and  in lockstep.
While  is explored nondeterministically,  is determinized on the fly 
for exploration. The algorithm maintains
an antichain, consisting of tuples of the form , where  and . 
The ordering relation  is given by 
 iff  and . The algorithm also maintains a {\em frontier} set of 
tuples {\em yet} to be explored. 

Given state  and a symbol ,
let  denote .  Given set of states , let  denote .  
Given tuple  in the frontier set, let 
 denote . 

In each step, the antichain algorithm explores  and  by computing
-successors of all tuples in its current frontier set for all possible
symbols . Whenever a tuple  is found 
with  and ,
the algorithm reports a counterexample to language inclusion. Otherwise, the
algorithm updates its frontier set and antichain to include 
the newly computed successors using the two rules enumerated below. 
Given a newly computed successor tuple :
\begin{compactitem}
\item Rule 1: if there exists a tuple  in the antichain 
with , then  is not added to the frontier set or antichain,  
\item Rule 2: else, if there exist tuples  in the  antichain 
with , then  are removed from the antichain.
\end{compactitem}
The algorithm terminates by either reporting a counterexample, or by declaring
success when the frontier becomes empty.  

\noindent{\bf Antichain algorithm for -bounded language inclusion \upto~}. 
This algorithm is essentially the same as the standard antichain algorithm, with 
the automaton  above replaced by an automaton 
 accepting . 
The set  of states of  consists of triples , 
where  and  are -length words over . 
Intuitively, the words  and  store symbols that are expected to be matched later along a run. 
The set of initial states of  is . 
The set of final states of  is  . 
The transition relation  is constructed by repeatedly applying the following rules, in order,  
for each state  and each symbol . In what follows,  
denotes the word obtained from  by removing its  symbol. 
\begin{compactenum}
\item Pick {\em new}  and  such that 
\item 
\begin{inparaenum}[(a)]
	\item If :  and 
         is independent of all symbols in , \\
         and ,
        \item else, if :   and 
	 is independent of all symbols in  prior to , 
         and 
        \item else, go to 1
   \end{inparaenum}
\item 
\begin{inparaenum}[(a)]
        \item If :  and
	 is independent of all symbols in ,\\
        , 
        \item else, if :  and 
	 is independent of all symbols in  prior to , 
	
	\item else, go to 1
\end{inparaenum}
\item Add  to  and go to 1.
\end{compactenum}


\begin{example} 
In \figref{langincex}, we have an \nfa~ with , 
 and . The states of  are triples , where 
 and . 
We explain the derivation of a couple of transitions of . The transition shown in bold 
from  on symbol  is obtained by applying the following rules once: 
1. Pick  since . 2(a). , . 
3(a). . 4. Add  to . 
The transition shown in bold from  on symbol  
is obtained as follows:
1. Pick  since . 2(b). , .  
3(b). . 4. Add  to .
It can be seen that  accepts the language .
\end{example}

\begin{figure}
\begin{center}
\footnotesize{
\begin{tikzpicture}[->,node distance=1.25cm,semithick,scale=0.9, every node/.style={transform shape}]
\tikzstyle{every state}=[ellipse,text=black]

 \node[state] (q0)              {};
 \node[right of=q0] (start) {start};
 \node[state,below of=q0,xshift=-1.2cm,yshift=4mm] (q1)   {};
 \node[state,accepting,below of=q0, node distance=2.5cm,yshift=4mm]  (q2)   {};
 
 \node[state,draw=none] (B) [left of=q0,node distance=1.2cm,yshift=0.2cm]  {\small{:}};

 \path (q0) edge [above left] node {} (q1)
 (start) edge (q0)
         (q0)   edge [right] node {} (q2)
  (q1) edge node [below left] {} (q2);


 \node[state,right of=q0,node distance=5.5cm] (r0)              {};
 \node[right of=r0,xshift=0.5cm] (start2) {start};
 \node[state,below of=r0,xshift=-2.7cm,yshift=2mm] (r1)   {};
 \node[state,below of=r0,xshift=-0.9cm,yshift=2mm] (r1a)     {};
 \node[state,accepting,below of=r0,xshift=0.9cm,yshift=2mm] (r1b)   {};
 \node[state,below of=r0,xshift=2.7cm,yshift=2mm] (r1c)   {};

 \node[state,below of=r1,xshift=-1cm,yshift=2mm]  (r2)   {};
 \node[state,accepting,below of=r1,xshift=1cm,yshift=2mm]  (r2a)   {};
 \node[state,accepting,below of=r1c,xshift=-1cm,yshift=2mm]  (r2b)   {};
 \node[state,below of=r1c,xshift=1cm,yshift=2mm]  (r2c)   {};

 \node[state,draw=none] (autCB) [left of=r0,node distance=1.9cm,yshift=0.2cm]  {\small{:}};
 
 \path (r0) edge [left] node[yshift=1mm] {} (r1)
 (start2) edge (r0)
           (r0) edge [left] node {} (r1a)
            edge [right] node {} (r1b)
            edge [right,very thick] node[yshift=1mm] {} (r1c)
       (r1) edge node [left,yshift=0.3mm] {} (r2)
            edge node [right,yshift=0.3mm] {} (r2a)
       (r1c) edge [very thick,yshift=0.3mm] node [left] {} (r2b)
             edge node [right,yshift=0.3mm] {} (r2c);

\end{tikzpicture}
}
\end{center}
\vspace{-5mm}
\caption{Example for illustrating construction of  for  and .}
\label{fig:langincex}
\end{figure}

\begin{proposition}
Given , \nfa~ described above accepts .
\end{proposition}


We develop a procedure to check 
language inclusion \upto~ by iteratively increasing the bound 
\ifappendix
(see Appendix~\ref{app:algo}
\else
(see the full version~\cite{fullversion}
\fi
for the complete algorithm). 
The procedure is {\em incremental}: the check for -bounded language
inclusion \upto~ only explores paths along which the bound 
was exceeded in the previous iteration. 






