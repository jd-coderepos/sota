We first focus on the problem of language inclusion \upto~an independence relation (\defref{langinc}).  This question corresponds to preemption-safety (\thmref{correctness}, \propref{inclusion}) and its solution drives our synchronization synthesis (\secref{algo}).


\begin{theorem}\label{thm:langinc.undecidable}
For \nfas~$A, B$ over alphabet $\Sigma$ and an independence relation $I\subseteq \Sigma\times\Sigma$, $\Lang \A \subseteq \CI(\Lang \B)$ is undecidable\cite{bertoni1982equivalence}.  
\end{theorem} 

Fortunately, a bounded version of the problem is decidable. 
Recall the relation $\approx$ over $\Sigma^*$ from \subsecref{langincdef}. 
We define a symmetric binary relation $\approx_i$ over $\Sigma^*$:
$(\sigma, \sigma') \in \, \approx_i$ iff $\exists (\alpha,\beta) \in I$: 
$(\sigma, \sigma') \in \, \approx$, $\sigma[i] = \sigma'[i+1] = \alpha$ and 
$\sigma[i+1] = \sigma'[i] = \beta$.
Thus $\approx^i$ consists of all words that can be optained from each other by commuting 
the symbols at positions $i$ and $i+1$.
We next define a symmetric binary relation $\asymp$ over $\Sigma^*$: 
$(\sigma,\sigma') \in \, \asymp$ iff $\exists \sigma_1,\ldots,\sigma_t$: 
$(\sigma,\sigma_1) \in \, \approx_{i_1},\ldots, (\sigma_{t},\sigma') \in \, \approx_{i_{t+1}}$ 
and $i_1 < \ldots < i_{t+1}$.
The relation $\asymp$ intuitively consists of words obtained from each other
by making a single forward pass commuting multiple pairs of adjacent symbols.
Let $\asymp^k$ denote the $k$-composition of $\asymp$ with itself. 
Given a language ${\cal L}$ over $\Sigma$, 
we use $\CIk({\cal L})$ to denote the set $\{\sigma \in \Sigma^*: \exists \sigma' \in \cal L \text{ with } (\sigma,\sigma') \in \, \asymp^{\mbox{$\scriptstyle k$}}  \}$.
In other words, $\CIk({\cal L})$ consists of all words which can be generated from 
${\cal L}$ using a finite-state transducer that remembers at most $k$ symbols 
of its input words in its states.  

\begin{definition}[Bounded language inclusion \upto~an independence relation]
Given \nfas~$\A, \B$ over $\Sigma$, $I\subseteq \Sigma\times\Sigma$ and a constant $k>0$, the $k$-bounded language inclusion problem \upto~$I$ is: $\lang \A \subseteq \CIk(\lang \B)$? 
\end{definition}

\begin{theorem}\label{thm:boundlanginc}
For \nfas~$\A, \B$ over $\Sigma$, $I\subseteq \Sigma\times\Sigma$ and a constant $k>0$,  
$\lang A \subseteq \CIk(\lang B)$ is decidable. 
\end{theorem}

We present an algorithm to check $k$-bounded language inclusion
\upto~$I$, based on the antichain algorithm for standard language
inclusion \cite{de2006antichains}. 

\noindent{\bf Antichain algorithm for language inclusion}. 
Given a partial order $(X, \sqsubseteq)$, an antichain over $X$ is
a set of elements of $X$ that are incomparable w.r.t. $\sqsubseteq$. In order
to check $\Lang \A \subseteq \CI(\Lang \B)$ for \nfas~$A =
(Q_\A,\Sigma,\Delta_\A,Q_{\iota,\A},F_\A)$ and $B = (Q_\B,\Sigma,\Delta_\B,Q_{\iota,\B},F_\B)$,
the antichain algorithm proceeds by exploring $\A$ and $\B$ in lockstep.
While $\A$ is explored nondeterministically, $\B$ is determinized on the fly 
for exploration. The algorithm maintains
an antichain, consisting of tuples of the form $(s_\A, S_\B)$, where $s_\A \in Q_\A$ and $S_\B \subseteq Q_\B$. 
The ordering relation $\sqsubseteq$ is given by 
$(s_\A, S_\B) \sqsubseteq (s'_\A, S'_\B)$ iff $s_\A = s'_\A$ and $S_\B
\subseteq S'_\B$. The algorithm also maintains a {\em frontier} set of 
tuples {\em yet} to be explored. 

Given state $s_\A \in Q_\A$ and a symbol $\alpha \in \Sigma$,
let $\suc_\alpha(s_\A)$ denote $\{s_\A'
\in Q_\A : (s_\A,\alpha,s_\A') \in \Delta_\A\}$.  Given set of states $S_\B
\subseteq Q_\B$, let $\suc_\alpha(S_\B)$ denote $\{s_\B'\in Q_\B:
\exists s_\B \in S_\B:\ (s_\B,\alpha,s_\B')\in\Delta_\B\}$.  
Given tuple $(s_\A, S_\B)$ in the frontier set, let 
$\suc_\alpha(s_\A, S_\B)$ denote $\{(s'_\A,S'_\B): s'_\A \in
\suc_\alpha(s_\A), S'_\B = \suc_\alpha(s_\B)\}$. 

In each step, the antichain algorithm explores $\A$ and $\B$ by computing
$\alpha$-successors of all tuples in its current frontier set for all possible
symbols $\alpha \in \Sigma$. Whenever a tuple $(s_\A, S_\B)$ is found 
with $s_\A \in F_\A$ and $S_\B \cap F_\B=\emptyset$,
the algorithm reports a counterexample to language inclusion. Otherwise, the
algorithm updates its frontier set and antichain to include 
the newly computed successors using the two rules enumerated below. 
Given a newly computed successor tuple $p'$:
\begin{compactitem}
\item Rule 1: if there exists a tuple $p$ in the antichain 
with $p \sqsubseteq p'$, then $p'$ is not added to the frontier set or antichain,  
\item Rule 2: else, if there exist tuples $p_1, \ldots, p_n$ in the  antichain 
with $p' \sqsubseteq p_1, \ldots, p_n$, then $p_1, \ldots, p_n$ are removed from the antichain.
\end{compactitem}
The algorithm terminates by either reporting a counterexample, or by declaring
success when the frontier becomes empty.  

\noindent{\bf Antichain algorithm for $k$-bounded language inclusion \upto~$I$}. 
This algorithm is essentially the same as the standard antichain algorithm, with 
the automaton $\B$ above replaced by an automaton 
$\autCB$ accepting $\CIk(\lang \B)$. 
The set $Q_{\autCB}$ of states of $\autCB$ consists of triples $(s_\B, \early, \late)$, 
where $s_\B \in Q_\B$ and $\early, \late$ are $k$-length words over $\Sigma$. 
Intuitively, the words $\early$ and $\late$ store symbols that are expected to be matched later along a run. 
The set of initial states of $\autCB$ is $\{(s_\B,\emptyset,\emptyset): s_\B \in I_\B\}$. 
The set of final states of $\autCB$ is  $\{(s_\B,\emptyset,\emptyset): s_\B \in F_\B\}$. 
The transition relation $\Delta_{\autCB}$ is constructed by repeatedly applying the following rules, in order,  
for each state $(s_\B, \early, \late)$ and each symbol $\alpha$. In what follows, $\eta[\setminus i]$ 
denotes the word obtained from $\eta$ by removing its $i^{th}$ symbol. 
\begin{compactenum}
\item Pick {\em new} $s'_\B$ and $\beta\in\Sigma$ such that $(s_\B, \beta, s_\B') \in \Delta_\B$
\item 
\begin{inparaenum}[(a)]
	\item If $\forall i$: $\early[i] \neq \alpha$ and 
        $\alpha$ is independent of all symbols in $\early$, \\
        $\late' \, \assign \, \late \cdot \alpha$ and $\early' \, \assign \, \early$,
        \item else, if $\exists i$:  $\early[i] = \alpha$ and 
	$\alpha$ is independent of all symbols in $\early$ prior to $i$, 
        $\early' \, \assign \, \early[\setminus i]$ and $\late'\, \assign\, \late$
        \item else, go to 1
   \end{inparaenum}
\item 
\begin{inparaenum}[(a)]
        \item If $\forall i$: $\late'[i] \neq \beta$ and
	$\beta$ is independent of all symbols in $\late'$,\\
        $\early' \, \assign \early' \,\cdot \beta$, 
        \item else, if $\exists i$: $\late'[i] = \beta$ and 
	$\beta$ is independent of all symbols in $\late'$ prior to $i$, 
	$\late' \, \assign \, \late'[\setminus i]$
	\item else, go to 1
\end{inparaenum}
\item Add $((s_\B, \early, \late),\alpha,(s'_\B, \early', \late'))$ to $\Delta_{\autCB}$ and go to 1.
\end{compactenum}


\begin{example} 
In \figref{langincex}, we have an \nfa~$\B$ with $\lang \B =  \{\alpha\beta, \beta\}$, 
$I = \{(\alpha,\beta)\}$ and $k = 1$. The states of $\autCB$ are triples $(q, \early, \late)$, where 
$q \in Q_\B$ and $\early, \late \in \{\emptyset,\alpha,\beta\}$. 
We explain the derivation of a couple of transitions of $\autCB$. The transition shown in bold 
from $(q_0, \emptyset,\emptyset)$ on symbol $\beta$ is obtained by applying the following rules once: 
1. Pick $q_1$ since $(q_0, \alpha, q_1) \in \Delta_\B$. 2(a). $\late'\ \assign\ \beta$, $\early'\ \assign\ \emptyset$. 
3(a). $\early'\ \assign\ \alpha$. 4. Add $((q_0, \emptyset, \emptyset),\beta,(q_1, \alpha, \beta))$ to $\Delta_{\autCB}$. 
The transition shown in bold from $(q_1, \alpha,\beta)$ on symbol $\alpha$ 
is obtained as follows:
1. Pick $q_2$ since $(q_1, \beta, q_2) \in \Delta_\B$. 2(b). $\early'\ \assign\ \emptyset$, $\late'\ \assign\ \beta$.  
3(b). $\late'\ \assign\ \emptyset$. 4. Add $((q_1, \alpha, \beta),\beta,(q_2, \emptyset, \emptyset))$ to $\Delta_{\autCB}$.
It can be seen that $\autCB$ accepts the language $\{\alpha\beta,\beta\alpha,\beta\} = \CIk(\B)$.
\end{example}

\begin{figure}
\begin{center}
\footnotesize{
\begin{tikzpicture}[->,node distance=1.25cm,semithick,scale=0.9, every node/.style={transform shape}]
\tikzstyle{every state}=[ellipse,text=black]

 \node[state] (q0)              {$q_0$};
 \node[right of=q0] (start) {start};
 \node[state,below of=q0,xshift=-1.2cm,yshift=4mm] (q1)   {$q_1$};
 \node[state,accepting,below of=q0, node distance=2.5cm,yshift=4mm]  (q2)   {$q_2$};
 
 \node[state,draw=none] (B) [left of=q0,node distance=1.2cm,yshift=0.2cm]  {\small{$\B$:}};

 \path (q0) edge [above left] node {$\alpha$} (q1)
 (start) edge (q0)
         (q0)   edge [right] node {$\beta$} (q2)
  (q1) edge node [below left] {$\beta$} (q2);


 \node[state,right of=q0,node distance=5.5cm] (r0)              {$q_0,\emptyset,\emptyset$};
 \node[right of=r0,xshift=0.5cm] (start2) {start};
 \node[state,below of=r0,xshift=-2.7cm,yshift=2mm] (r1)   {$q_1,\emptyset,\emptyset$};
 \node[state,below of=r0,xshift=-0.9cm,yshift=2mm] (r1a)     {$q_2,\beta,\alpha$};
 \node[state,accepting,below of=r0,xshift=0.9cm,yshift=2mm] (r1b)   {$q_2,\emptyset,\emptyset$};
 \node[state,below of=r0,xshift=2.7cm,yshift=2mm] (r1c)   {$q_1,\alpha,\beta$};

 \node[state,below of=r1,xshift=-1cm,yshift=2mm]  (r2)   {$q_2,\beta,\alpha$};
 \node[state,accepting,below of=r1,xshift=1cm,yshift=2mm]  (r2a)   {$q_2,\emptyset,\emptyset$};
 \node[state,accepting,below of=r1c,xshift=-1cm,yshift=2mm]  (r2b)   {$q_2,\emptyset,\emptyset$};
 \node[state,below of=r1c,xshift=1cm,yshift=2mm]  (r2c)   {$q_2,\alpha,\beta$};

 \node[state,draw=none] (autCB) [left of=r0,node distance=1.9cm,yshift=0.2cm]  {\small{$\B_{1,\{(\alpha,\beta)\}}$:}};
 
 \path (r0) edge [left] node[yshift=1mm] {$\alpha$} (r1)
 (start2) edge (r0)
           (r0) edge [left] node {$\alpha$} (r1a)
            edge [right] node {$\beta$} (r1b)
            edge [right,very thick] node[yshift=1mm] {$\beta$} (r1c)
       (r1) edge node [left,yshift=0.3mm] {$\alpha$} (r2)
            edge node [right,yshift=0.3mm] {$\beta$} (r2a)
       (r1c) edge [very thick,yshift=0.3mm] node [left] {$\alpha$} (r2b)
             edge node [right,yshift=0.3mm] {$\beta$} (r2c);

\end{tikzpicture}
}
\end{center}
\vspace{-5mm}
\caption{Example for illustrating construction of $\autCB$ for $k = 1$ and $I = \{(\alpha, \beta)\}$.}
\label{fig:langincex}
\end{figure}

\begin{proposition}
Given $k>0$, \nfa~$\autCB$ described above accepts $\CIk(\lang \B)$.
\end{proposition}


We develop a procedure to check 
language inclusion \upto~$I$ by iteratively increasing the bound $k$
\ifappendix
(see Appendix~\ref{app:algo}
\else
(see the full version~\cite{fullversion}
\fi
for the complete algorithm). 
The procedure is {\em incremental}: the check for $k+1$-bounded language
inclusion \upto~$I$ only explores paths along which the bound $k$
was exceeded in the previous iteration. 






