\documentclass{article}



\PassOptionsToPackage{numbers, sort, compress}{natbib}

\usepackage[preprint]{neurips_2022}








\usepackage{subcaption,booktabs}
\usepackage{caption}

\usepackage[utf8]{inputenc} \usepackage[T1]{fontenc}    \usepackage{hyperref}       \usepackage{url}            \usepackage{booktabs}       \usepackage{amsfonts}       \usepackage{nicefrac}       \usepackage{microtype}      

\usepackage[table]{xcolor}
\usepackage{amsmath}
\usepackage{wrapfig}
\usepackage{multicol}        \usepackage{multirow}
\usepackage{changepage}
\usepackage{makecell}
\usepackage{bbding}
\newcommand{\tabincell}[2]{\begin{tabular}{@{}#1@{}}#2\end{tabular}}
\usepackage{amssymb}\usepackage{pifont}\newcommand{\cmark}{\ding{51}}\newcommand{\xmark}{\ding{55}}\usepackage{array}
\newcolumntype{P}[1]{>{\centering\arraybackslash}p{#1}}
\usepackage{siunitx}
\usepackage{bbm}
\usepackage{dsfont}
\usepackage{caption} 
\usepackage{setspace}
\usepackage{balance}
\captionsetup[table]{skip=3pt}
\usepackage{bm} \usepackage{color}
\usepackage{stfloats}
\usepackage{graphicx}
\usepackage{mathrsfs}
\usepackage{booktabs} \newcommand*\samethanks[1][\value{footnote}]{\footnotemark[#1]}
\usepackage{tikz}
\usetikzlibrary{decorations.pathreplacing,calc}
\newcommand{\tikzmark}[1]{\tikz[overlay,remember picture] \node (#1) {};}

\newcommand*{\AddNote}[4]{\begin{tikzpicture}[overlay, remember picture]
        \draw [decoration={brace,amplitude=0.5em},decorate,ultra thick,black]
            ((#3)-(0,1)) --  
            ((#3)-(0,1))
                node [align=center, text width=3cm, pos=0.5, anchor=west] {#4};
    \end{tikzpicture}
}\usepackage{floatrow}
\newfloatcommand{capbtabbox}{table}[][\FBwidth]

\usepackage[ruled,lined]{algorithm2e}
\usepackage{enumitem} 


\usepackage{ulem}



\title{NUWA-Infinity: Autoregressive over Autoregressive\\ Generation for Infinite Visual Synthesis}





\author{Chenfei Wu\thanks{Both authors contributed equally to this research.} \quad Jian Liang\samethanks[1] \quad Xiaowei Hu \quad  Zhe Gan \quad Jianfeng Wang\\ \textbf{Lijuan Wang}\quad  \textbf{Zicheng Liu}\quad  \textbf{Yuejian Fang}\quad \textbf{Nan Duan}\thanks{Corresponding author.} \\
 {Microsoft Research Asia \quad Peking University
 \quad Microsoft Azure AI} \\
{\tt\small\{chewu,xiaowei.hu,zhe.gan,jianfw,lijuanw,zliu,nanduan\}@microsoft.com}\\ {\tt\small\{j.liang@stu,fangyj@ss\}.pku.edu.cn}}



\begin{document}


\maketitle
\vspace{-9mm}

\begin{abstract}
\vspace{-4mm}
In this paper, we present NUWA-Infinity, a generative model for infinite visual synthesis, which is defined as the task of generating arbitrarily-sized high-resolution images or long-duration videos. An autoregressive over autoregressive generation mechanism is proposed to deal with this variable-size generation task, where a global patch-level autoregressive model considers the dependencies between patches, and a local token-level autoregressive model considers dependencies between visual tokens within each patch. A Nearby Context Pool (NCP) is introduced to cache-related patches already generated as the context for the current patch being generated, which can significantly save computation costs without sacrificing patch-level dependency modeling. An Arbitrary Direction Controller (ADC) is used to decide suitable generation orders for different visual synthesis tasks and learn order-aware positional embeddings. Compared to DALL·E, Imagen and Parti, NUWA-Infinity can generate high-resolution images with arbitrary sizes and support long-duration video generation additionally. Compared to NUWA, which also covers images and videos, NUWA-Infinity has superior visual synthesis capabilities in terms of resolution and variable-size generation. The GitHub link is \url{https://github.com/microsoft/NUWA}. The homepage link is \url{https://nuwa-infinity.microsoft.com}.

 






\begin{figure}[htbp]
\vspace{-3mm}
    \centering
    \includegraphics[width=1\textwidth]{Figure/rqf_homeV1.pdf}
    \caption{The painting at the top () is created as Unconditional Image Generation\textsuperscript{\textbf{HD}} by NUWA-Infinity, which is trained on the famous painting \textit{Along the River During the Qingming Festival}. The patches in the middle or at the bottom highlight more details of this AI-created painting.}
    \label{fig:qm}
\end{figure}





\begin{figure}[p]
 \vspace*{-2.3cm}
    \makebox[\linewidth]{
        \includegraphics[width=1.5\linewidth]{Figure/IOHD.pdf}
    }
    \caption{Image Outpainting\textsuperscript{\textbf{HD}} examples. Four examples in the upper part () show the outpainting of four different directions. The example in the lower part () shows outpainting in all directions of the famous painting of Monet, \textit{Garden at Sainte-Adresse}.}
\end{figure}

\begin{figure}[p]
 \vspace*{-2.3cm}
    \makebox[\linewidth]{
        \includegraphics[width=1.5\linewidth]{Figure/T2IHD.pdf}
    }
    \caption{Text-to-Image\textsuperscript{\textbf{HD}} examples in the resolution of .  }
    \label{fig:tasks}
\end{figure}

\begin{figure}[p]
 \vspace*{-2.3cm}
    \makebox[\linewidth]{
        \includegraphics[width=1.5\linewidth]{Figure/i2v_homeV2.pdf}
    }
    \caption{Image Animation\textsuperscript{\textbf{HD}} examples in the resolution of  with 20 frames.}
    \label{fig:ia_home}
\end{figure}



\end{abstract}


\section{Introduction}
\vspace{-2mm}
\newcommand{\chapquote}[3]{\begin{quotation} \textit{#1} \end{quotation} \begin{flushright} - #2, \textit{#3}\end{flushright} }



The ongoing convergence of vision-language representation and modeling techniques brings new opportunities to visual synthesis research. By learning visual knowledge and patterns from a large-scale visual and multimodal corpus, recent visual synthesis models can generate images or videos based on given text, visual, or multimodal inputs and support various visual content creation tasks, such as text-to-image or video generation, image inpainting or outpainting, video prediction, etc. We also witness a notable trend in this area, where more and more work begin to explore how to generate images with higher resolutions~\cite{dingCogViewMasteringTexttoImage2021,rameshHierarchicalTextConditionalImage2022,skorokhodovAligningLatentImage2021,linInfinityGANInfinitePixelImage2022,changMaskGITMaskedGenerative2022}, or generate videos with longer durations~\cite{villegasHighFidelityVideo2019,clarkAdversarialVideoGeneration2019,liuInfiniteNaturePerpetual2021}. This is because high-resolution images and long-duration videos can provide 
better visual effects to practical applications, such as design, advertisement, presentation, entertainment, etc.

However, generating arbitrarily-sized high-quality images (in resolution) or videos (in both resolution and duration) is not a trivial task. First, compared to text generation in the NLP field, the research of generating variable-size visual content is still in its early stage, and therefore is not well studied. Some existing works~\cite{struskiLocoGANLocallyConvolutional2022,skorokhodovAligningLatentImage2021,esserTamingTransformersHighResolution2021,changMaskGITMaskedGenerative2022} try to solve this problem with a divide-and-conquer strategy, which divides images or videos into patches, trains the model to generate patches separately without considering their dependencies, and composes the generated patches to form the final image or video. As such methods do not model the dependencies between the generated patches explicitly, they struggle to guarantee the consistency of generated contents, especially when generating high-resolution images or long-duration videos. 
Second, different from text generation that usually follows a fixed order (e.g., left-to-right), images have two dimensions (i.e., width and height), and videos have three (i.e., width, height, and duration). These suggest that visual synthesis models should consider and model different generation orders and directions for different types of tasks.

In this paper, we use \textbf{Infinite Visual Synthesis} to denote the task of generating arbitrarily-sized high-quality images or videos, and propose \textbf{NUWA-Infinity} as a general visual synthesis model that can solve the two challenges of this task mentioned before. First, NUWA-Infinity is based on an autoregressive over autoregressive generation mechanism, where a global patch-level autoregressive model considers the dependencies between patches, and a local token-level autoregressive model considers the dependencies between visual tokens within each patch. Compared to diffusion-based approaches~\cite{dhariwalDiffusionModelsBeat2021,hoDenoisingDiffusionProbabilistic2020,guVectorQuantizedDiffusion2022} that are only able to generate images with a fixed size, the autoregressive formulation naturally considers different levels of dependencies and perfectly deals with the variable-size generation task. We also introduce a Nearby Context Pool (NCP) to cache-related patches already generated as the context for the current patch being generated, which can significantly save the computation cost without sacrificing the patch-level dependency modeling. Second, we propose an Arbitrary Direction Controller (ADC) to decide suitable generation orders and learn order-aware position embeddings, which is extremely useful for image outpainting.

We evaluate NUWA-Infinity on five high-resolution visual synthesis tasks, including Unconditional Image Generation\textsuperscript{\textbf{HD}}, Text-to-Image\textsuperscript{\textbf{HD}}, Text-to-Video\textsuperscript{\textbf{HD}}, Image Animation\textsuperscript{\textbf{HD}} and Image Outpainting\textsuperscript{\textbf{HD}}. Compared to DALL·E~\cite{rameshZeroShotTexttoImageGeneration2021}, Imagen~\cite{sahariaPhotorealisticTexttoImageDiffusion2022} and Parti~\cite{yuScalingAutoregressiveModels2022}, which generate images with a fixed resolution (i.e., ), NUWA-Infinity can generate high-resolution images with arbitrary sizes and support long-duration video generation additionally. Compared to NUWA~\cite{wuUWAVisualSynthesis2022}, which also supports image and video synthesis at the same time, the generation quality of NUWA-Infinity has been improved significantly. We also show the huge application potential of NUWA-Infinity on creative visual synthesis tasks, such as image outpainting and cartoon creation from natural language descriptions. We hope this technique can help visual content creators to save time, cut costs and improve their productivity and creativity.








































\section{Related Work}\label{sec:relatedwork}
\vspace{-2mm}



\paragraph{Autoregressive Methods}
DALL·E \cite{rameshZeroShotTexttoImageGeneration2021} tokenizes each image into discrete visual tokens and trains an autoregressive model to generate visual tokens from the corresponding text. The output image is reconstructed by the VQVAE decoder, which takes the visual tokens generated by the autoregressive model as inputs.     
Parti \cite{yuScalingAutoregressiveModels2022} follows the same architecture of DALL·E, but uses ViT-VQGAN \cite{yuVectorquantizedImageModeling2022} to discretize and reconstruct images, which is an improved version of VQGAN from both architecture and codebook learning aspects.
NUWA \cite{wuUWAVisualSynthesis2022} is the first autoregressive visual synthesis pre-trained model to support both image and video generation tasks. 
Compared to these previous works, NUWA-Infinity introduces the autoregressive over autoregressive mechanism into the generation procedure, which enables the capability of generating variable-size images and videos. 

\paragraph{Diffusion Methods}
DALL·E 2 \cite{rameshHierarchicalTextConditionalImage2022} generates image embedding from an input text based on either an autoregressive or a diffusion model, and uses a diffusion model to produce the output image.
Imagen \cite{sahariaPhotorealisticTexttoImageDiffusion2022} uses a frozen large-scale pre-trained language model T5-XXL \cite{raffelExploringLimitsTransfer2020} to encode each input text, and uses two diffusion models to generate high-resolution images based on the text embeddings.
Both of these two diffusion-based text-to-image generation methods cannot support arbitrarily-sized image generation, as the size of the output images is pre-defined before training and inference.




\paragraph{Infinite Visual Synthesis}
To support infinite visual synthesis, most existing works follow the divide-and-conquer strategy to first divide a large image into several patches, and then train them in an independent way. GAN-based models~\cite{struskiLocoGANLocallyConvolutional2022,skorokhodovAligningLatentImage2021} attempt to divide large images into patches and optimize each of them from global or coordinated latent space independently. Since different patches have no explicit dependency, these models struggle to merge different patches during inference, and can easily lead to inconsistent results. To address this issue, autoregressive models~\cite{esserTamingTransformersHighResolution2021,changMaskGITMaskedGenerative2022} incorporate a sliding window to enforce dependencies between different patches during inference. Recently, Mask-Predict~\cite{choXLXMERTPaintCaption2020, zhangM6ufcUnifyingMultimodal2021, changMaskGITMaskedGenerative2022} also uses the sliding window approach, but incorporates a progressively mask-and-predict strategy to model dependencies between patches during the window sliding in inference. However, both autoregressive models and Mask-Predict models bring a huge gap between training and inference, since different patches are still trained independently but inferred in a dependent way. By introducing the autoregressive over autoregressive mechanism with Nearby Context Pool and Arbitrary Direction Controller, NUWA-Infinity can enable variable-size image and video generation, and save the computation cost without losing global dependency and consistency modeling.






















\section{Model}\label{sec:method}
 \begin{figure}[tbp]
    \centering
    \includegraphics[width=\textwidth]{Figure/Overview.pdf}
    \caption{An overview of the proposed NUWA-Infinity model during the training process.}
    \label{fig:architecture}
\end{figure}




Given an input , which can be a text or an image, the infinite visual synthesis task aims to generate an image or a video  with a user-specified resolution and duration, i.e., , where ,  and  denote the width, height and channel of each image or video frame, respectively,  denotes the frame number.  denotes an image if  and denotes a video if . 

In general, NUWA-Infinity follows an autoregressive over autoregressive model to solve this task:

 denotes the global autoregressive generation procedure, where  is the  patch being generated,  is the previous  patches already generated,  is the total number of patches.  denotes the local autoregressive generation procedure, where  is the  visual token being generated in ,  is the previous  visual tokens already generated,  is the total number of visual tokens in each patch. Each  is reconstructed by a pre-trained VQGAN decoder~\cite{esserTamingTransformersHighResolution2021}, which takes as input the visual token sequence . The final image or video is formed by composing all generated patches  based on the specified resolution (i.e., ) and duration (i.e., ).

NUWA-Infinity uses an encoder-decoder architecture to model the above generation procedure. In this paper, we mainly focus on five types of high-definition (\textbf{HD}) visual synthesis tasks, including Unconditional Image Generation\textsuperscript{\textbf{HD}}, Image Outpainting\textsuperscript{\textbf{HD}}, Image Animation\textsuperscript{\textbf{HD}}, Text-to-Image\textsuperscript{\textbf{HD}} and Text-to-Video\textsuperscript{\textbf{HD}}. In the  task, the output image is generated by the vision decoder without any input. In the  and  tasks, the input image is fed into the vision decoder directly as the prefix to generate the output image or video. In the  and  tasks, the input text is encoded by the text encoder, and the output image or video is generated by the vision decoder.

One problem is that, different from text generation that follows the left-to-right order, images have two dimensions (i.e., width and height), and videos have three (i.e., width, height, and duration). These suggest the model should consider and handle different patch generation orders for different visual synthesis tasks. Motivated by this, we propose an Arbitrary Direction Controller (ADC) (Section~\ref{sec:adc}) that can plan proper patch generation orders and learn order-aware positional embeddings.

Another problem is that, the length (i.e., ) of the visual tokens to be generated could be extremely long, which is challenging for most existing sequence generation models. To alleviate this issue, we propose a Nearby Context Pool (NCP) (Section~\ref{sec:ncp}) to cache-related patches already generated as the context for the current patch being generated, which can significantly save the computation cost without sacrificing the patch-level dependency modeling.

We train NUWA-Infinity (Section~\ref{sec:train}) using high-quality image-text pairs crawled from the web, and image-video pairs extracted from high-quality videos.














































 














\subsection{Arbitrary Direction Controller (ADC)}\label{sec:adc}


In this subsection, we introduce Arbitrary Direction Controller (ADC), which provides two functions: , which splits images/videos and decides the patch generation order for training and inference procedures; , which assigns order-aware positional embeddings based on the current context.

\begin{figure}[t!]
    \centering
    \includegraphics[width=1.0\textwidth]{Figure/ADM.pdf}
    \caption{Illustration of patch order control in NUWA-Infinity. The left part shows four basic patch generation orders () during training. The right part shows how NUWA-Infinity performs the image outpainting task by composing these four orders. Arabic numerals indicate the order of global autoregression, arrows indicate the order of local autoregression.}
    \label{fig:adc}
    \vspace{-3mm}
\end{figure}
\begin{itemize}[leftmargin=*]
\item \textbf{Split}. This function takes the shape of an existing or to-be-generated image or video  as input and returns an ordered patch sequence:

 denotes the ordered patch sequence. For simplicity, we use Fig.~\ref{fig:adc} to explain how this function splits an image into ordered patches in the training and inference stages. It is straightforward to extend from images to videos by considering the temporal dimension.

The left part of Fig.~\ref{fig:adc} shows how  works in the training stage. We define four basic generation orders and represent them using four Greek letters, respectively, according to their writing orders: -order (↓→), -order (↓←), -order (→↓), -order(→↑), where  denotes the reversed writing order. When choosing the -order in training,  will return patches in order of the red numerical sequence (top-left example in Fig.~\ref{fig:adc}). Similarly, the other three options will let NUWA-Infinity learn how to generate patches based on the corresponding orders. Note that there are more orders such as (↑←) or a snake-like order, but the above four basic orders and their compositions are enough to generate images in arbitrary resolutions or shapes.

The right part of Fig.~\ref{fig:adc} shows how  works in the inference stage for Image Outpainting\textsuperscript{\textbf{HD}}. Given a small image of a volcanic vent as input and its relative position in the targeted image with a specified larger resolution, the goal is to synthesize this targeted image by generating all the surrounding patches of the input. In order to leverage as many contextual patches as possible when generating new patches, 
 selects a patch generation order illustrated as a numerical sequence from  to . 



\begin{figure}[t!]
    \centering
    \includegraphics[width=1.0\textwidth]{Figure/ADC_V3.pdf}
    \caption{Illustration of dynamic position control in NUWA-Infinity. 
}
    \label{fig:pos}
\end{figure}

\item \textbf{Emb}. This function assigns positional embeddings to the patch  being generated and the patches in  that are already generated and selected as the context of :


It is crucial to design a proper positional embedding for the order-aware patch generation procedure, since the absolute positional embedding \cite{vaswaniAttentionAllYou2017} is unable to consider and model all relative positions in image and video generation tasks. Motivated by the recent work on relative positional embedding~\cite{huangImproveTransformerModels2020,liuVideoSwinTransformer2022}, we propose a dynamic positional embedding in ADC, where dynamic means the positional embeddings could change according to different situations. Fig.~\ref{fig:pos} shows 18 dynamic relative embeddings from ``'' to ``''. The center patch being generated (in {\color{green}green} color) is always labeled as ``'' and the relative positions of previously generated patches based on ``'' are labeled by other symbols (in {\color{orange}orange} color). As a result, an embedding matrix of the size of  is formed. The right part shows how the embeddings are dynamically assigned to different patches when generating a specific patch. For example, when generating the last (i.e., ) patch, the positional embedding ``'' is assigned to the current patch and the positional embeddings of ``'',``'',``'',``'',``'',``'',``'' are assigned to patch 14, 15, 17, 18, 23, 24 and 26 in , respectively.








\end{itemize}































\subsection{Nearby Context Pool (NCP)}\label{sec:ncp}

\begin{figure}[htbp]
    \centering
    \includegraphics[width=1.0\textwidth]{Figure/NCP_V3.pdf}
    \caption{Illustration of NCP in -order with a context extent of (1,1,1). 
    }
\label{fig:ncp_demo}
\end{figure}

An image or a video could be extremely large, thus the previous patches  in Eq.~\ref{eq:pxy} could have a large size. A natural idea is to only consider nearby patches as contexts. However, simply ignoring distant patches will lose long-term memory and thus harm the global consistency of the generated image or video. 
To address this issue, we propose a Nearby Context Pool (NCP) with three functions illustrated in Fig.~\ref{fig:ncp_demo}: 
\textbf{Select}, which dynamically selects nearby patches as the context to promote infinite generation; 
\textbf{Add}, which saves multi-layer hidden states of previously generated patches to help long-term memory; 
\textbf{Remove}, which removes expired caches for self-cleaning.

\begin{itemize}[leftmargin=*]

\item \textbf{Add}. This function adds the cache  of the patch  already generated into NCP:

In NCP, the cache  of the patch  is defined as all the resulting multi-layer hidden states from the generation of . Since NUWA-Infinity will not retain all generation history, this cache mechanism ensures the transmission of the necessary information in the whole generation procedure.
\item \textbf{Select}. This function selects the context  for the patch  to be generated:
In NCP, the context  of  is defined as the caches of nearby patches already generated within a pre-defined 3D extent , denoting the width extent, height extent, and frame extent. For example, when NUWA-Infinity generates the  patch (as ) in Fig. \ref{fig:ncp_demo} and the maximum extent is set as , the context will include the patches from 1 to 13.




\item \textbf{Remove}. This function removes the caches of those patches in NCP that no longer have any effect on the generation of future patches:

In NCP, a cache is only cleaned when it cannot serve as the context for any patch to be generated. The  function will be invoked after each  function.

\end{itemize}








































\begin{wrapfigure}[12]{R}{0.5\textwidth}
\vspace{-0.17in}
\begin{minipage}{1\linewidth}
\begin{algorithm}[H]\label{alg:training}
\SetAlgoNoLine
\caption{Training Strategy}
\KwIn{images or videos , optional text }
\KwOut{optimized NUWA-Infinity model}

Initial Arbitrary Direction Controller \textbf{ADC}\;
Initial Nearby Context Pool  \;
\;
\For{all  from  to }{
    \;
    \;
    \;
    \;
    \;
    \;
    optimize \;
}
\end{algorithm}
\end{minipage}
\end{wrapfigure}

\subsection{Training and Inference Strategy}\label{sec:train}
This section will introduce the training and inference strategies of NUWA-Infinity in Algorithm~\ref{alg:training} and Algorithm~\ref{alg:inference}, respectively.


\subsubsection{Training Strategy}\label{sec:ts}

Given each input-output pair  in the pre-training corpus, we first split the visual data  into patches and then randomly select one patch generation order  from the four orders  described in Section  \ref{sec:adc}. A pre-trained VQGAN encoder \cite{esserTamingTransformersHighResolution2021} transforms all images in  into visual tokens , and each patch  is represented by its corresponding visual tokens .  is encoded by a text encoder as , which denotes a sequence of token embeddings.



We train NUWA-Infinity based on each ordered patch sequence . For the  patch , we first select its context  based on the NCP described in Section~\ref{sec:ncp}:

where  denotes the number of patches in ,  denotes the layer number of the vision decoder,  denotes the visual token number in each context patch, and  denotes the dimension of each visual token embedding. Note that  can be changed during training, as different patches may have different numbers of context patches in . The positional embeddings  of  and its context  are dynamically assigned by the ADC operation described in Section~\ref{sec:adc}:


Then, an -layer vision decoder takes as input  and . In the  layer,  and the  layer hidden states  of all patches in  are fed into a self-attention module, enhanced by the positional embeddings :
 
 are queries, keys and values, respectively,  are parameters to be learned,  denotes the attended results.

For the Text-to-Image\textsuperscript{\textbf{HD}} task,  and  are further fed into a cross-attention module as shown in Eq.~(\ref{eq:qkv2}).

where  are queries, keys and values, respectively,  are parameters to be learned,  is the number of token embeddings in ,  denotes the attended results.

By feeding  (for tasks without text input) or  (for tasks with text input) into a feed forward network, the output of the  layer  is obtained:






By iteratively stacking Eq.~(\ref{eq:qkv})(\ref{eq:pn1}) into  layers, we obtain .  and the previous  layer outputs are concatenated to obtain a -layer cache of the  patch :

where . 
For simplicity, the procedure from Eq.~(\ref{eq:qkv}) to Eq.~(\ref{eq:an}) is defined as :

where  denotes the output embeddings. Then, NCP will collect the cache of  to help the prediction of the next patches and conduct a self-cleaning to remove useless patches, as shown in Eq.~(\ref{eq:addremove}).


Finally, the cross-entropy loss is used to optimize model parameters based on  and the ground-truth .


\noindent\makebox[\textwidth][c]{
\begin{minipage}{0.75\textwidth}
\begin{algorithm}[H]\label{alg:inference}
\SetAlgoNoLine
\caption{Inference Strategy}

\KwIn{a text  or an image }
\KwOut{generated image/video }

Initial Arbitrary Direction Controller \textbf{ADC}\; 
Initial Nearby Context Pool  \tikzmark{right1}  \ ;
\; 
\For{all  from  to  \tikzmark{top1}}
{
    \;
    \;
    \;
    \;
    \;
    \tikzmark{bottom1}
}





\; 
\For{all  from 1 to }{
    \;
    \;
    \;
    \;
    \;
    \;
}
\eIf{target  is an image \tikzmark{top2}}
{return }
{return  \tikzmark{bottom2}}

\AddNote{top1}{bottom1}{right1}{Image Condition Pre-caching}
\AddNote{top2}{bottom2}{right2}{Pixel-Guided VQGAN}
\end{algorithm}
\vspace{-3mm}
\end{minipage}
}














































\subsubsection{Inference Strategy}\label{sec:is}
\vspace{-3mm}
NUWA-Infinity can support various visual synthesis scenarios, and we focus on five tasks in this paper: Unconditional Image Generation\textsuperscript{\textbf{HD}}, Image Outpainting\textsuperscript{\textbf{HD}}, Image Animation\textsuperscript{\textbf{HD}}, Text-to-Image\textsuperscript{\textbf{HD}} and Text-to-Video\textsuperscript{\textbf{HD}}. There are two specific designs for the last four tasks: Image Condition Pre-caching and Pixel-Guided VQGAN.


\paragraph{Image Condition Pre-caching}
\vspace{-3mm}
For Image Outpainting\textsuperscript{\textbf{HD}} and Image Animation\textsuperscript{\textbf{HD}}, the input is an image condition 
and the output is a spatial extended image or a temporal extended video. 
A VQGAN encoder is used to encode  into a list of patches with corresponding visual tokens, where  denotes the number of conditional patches. Then, these patches and visual tokens are fed into the vision decoder (Eq.~\ref{eq:en}\ref{eq:addremove}) as a prefix to initialize NCP, which will be used next to generate the extended image or following video frames.


\paragraph{Pixel-Guided VQGAN}
\vspace{-3mm}
\begin{wrapfigure}{r}{0.4\textwidth}
\vspace{-7mm}\includegraphics[width=0.8\textwidth]{Figure/pixel-guidedVQGAN.pdf}
    \caption{Pixel-Guided VQGAN.}
    \label{fig:pgvqgan}
\end{wrapfigure}

For Image Animation\textsuperscript{\textbf{HD}} and Text-to-Video\textsuperscript{\textbf{HD}}, the output are videos. Since traditional VQVAE is trained only on images, simply decoding a video frame-by-frame will lead to inconsistency between frames. To solve this issue, we propose a Pixel-Guided VQGAN (PG-VQGAN), as shown in Fig.~\ref{fig:pgvqgan}. Different from traditional VQGAN which is trained on images independently, we sample 2 consecutive frames  and  as a training pair and use the pixel-level information of the  frame to enhance the decoder of frame . In detail, the frame  is encoded with the same number of layers as the traditional VQGAN decoder, and the output of each encoder layer is fused with the corresponding output layer of the VQGAN decoder. We simply use an element-sum operation as the fusion strategy and observe promising results (see Fig.~\ref{fig:pgvqgan}). 
When decoding the first frame during inference, Image Animation\textsuperscript{\textbf{HD}} has a ground-truth  frame. However, 
For Text-to-Video\textsuperscript{\textbf{HD}}, since there are no ground-truth frames, we instead use traditional VQGAN to decode the first frame and Pixel-Guided VQGAN to decode the following frames.













 \begin{figure}[tbp]
    \centering
    \includegraphics[width=\textwidth]{Figure/Inference.pdf}
    \caption{Inference pipeline of NUWA-Infinity for downstream tasks.}
    \label{fig:inference}
    \vspace{-4mm}
\end{figure}






















































\section{Experiments}
\vspace{-2mm}
\subsection{Datasets}
\vspace{-2mm}
Different from most visual synthesis works, NUWA-Infinity focuses on generating images and videos with high resolutions and long durations. As a result, most existing datasets cannot be used in training or evaluation. To evaluate the ability of NUWA-Infinity on the five tasks mentioned before, we build four datasets with high resolutions () in the following:
\begin{itemize}[leftmargin=*]

\item \textbf{RQF}. We build a new dataset, Riverside of Qingming Festival (RQF), based on the version of \emph{\textit{Along the River During the Qingming Festival}} drawn by Qiu Ying\footnote{https://www.comuseum.com/painting/masters/qiu-ying/}. This version is 30.5 centimeters high and 987 centimeters wide in reality, and we download a digital print with  pixels. We resize the whole image to  resolution and split it into several overlapped  patches instead of non-overlapping ones. We end with a dataset of 128 images.  We train NUWA-Infinity with the  pairs on the dataset with  resolution but qualitatively evaluate Unconditional Image Generation\textsuperscript{\textbf{HD}} at higher resolution .


\item \textbf{LHQC}. We build a new dataset, Landscape High Quality with Captions (LHQC), based on the publicly available dataset LHQ~\cite{skorokhodovAligningLatentImage2021}. The original LHQ dataset consists of K high-resolution () nature landscapes. To support the text prompt, we use the image captioning model from~\cite{wangGITGenerativeImagetotext2022} to generate the captions for the dataset first and manually fix some errors in the generated results. We finally obtain a dataset with K text-image pairs. We spilt the dataset into two splits: train (K) and test (K). We train NUWA-Infinity with the  pairs on the train split and evaluate Text-to-Image\textsuperscript{\textbf{HD}} and Image Outpainting\textsuperscript{\textbf{HD}} on the test split.

\item \textbf{LHQ-V}. We build a new dataset, Landscape High Quality for Videos (LHQ-V), based on the videos scrapped from the www.pexels.com website. We first build a query set with 100 landscape related keywords (e.g., ``sky'', ``forest'', ``cloud''). Then, we query the website using these keywords and obtain K high-resolution videos. To further make the dataset cleaner, we use Mask R-CNN \cite{heMaskRCNN2017} to detect objects from these videos and remove videos containing objects that are not related to landscape (i.e., ``table'', ``human'', ``computer''). Finally, we obtain a dataset with K videos. We split the dataset into two splits: train (K) and test (K). We train NUWA-Infinity with the  pairs on the train split and evaluate Image Animation\textsuperscript{\textbf{HD}} on the test split.

\item \textbf{PeppaPig}. We build a new dataset, PeppaPig, based on the famous cartoon PeppaPig. We collect Season 1-4 videos of PeppaPig and split the videos into multiple clips based on the timeline of subtitles. We then ask 20 trained annotators to annotate the captions for these videos. If the clip is not smooth, the annotators are asked to provide a ``N/A'' caption. We also ask a meta annotator to check these captions. Finally, we remove videos with the ``N/A'' caption and obtain K text-video pairs. We split the dataset into two splits: train (K) and test (K). We train NUWA-Infinity with the  pairs on the train split and evaluate Text-to-Video\textsuperscript{\textbf{HD}} on the test split.
\end{itemize}

\begin{table}[t]
\small
\centering
\setlength{\tabcolsep}{5pt}
\begin{tabular}{lcccc}
\toprule
Setting                      & UIG\textsuperscript{\textbf{HD}} & T2I\textsuperscript{\textbf{HD}} \& IO\textsuperscript{\textbf{HD}} & IA\textsuperscript{\textbf{HD}} & T2V\textsuperscript{\textbf{HD}} \\
\midrule
\multicolumn{5}{l}{\textit{VQVAE}}
\\ Backbone & VQGAN & VQGAN & PG-VQGAN & PG-VQGAN
\\ 
Codebook               & 16384      &16384         & 16384     & 16384           \\
Dimension              & 256         & 256         & 256        & 256           \\
Compression Ratio      & 16          & 16        & 16           & 16        \\
\midrule
\multicolumn{5}{l}{\textit{Transformer}}\\
Layer Number     & 24                  & 24                   & 24                  & 24\\
Hidden Dimension  & 1280                & 1280                 & 1280                & 1280                 \\
Head Number     & 20                  & 20                   & 20                  & 20                   \\
Self-attention   & \checkmark                 & \checkmark                   & \checkmark                 & \checkmark                   \\
Cross-attention  & \XSolidBrush                & \checkmark               & \XSolidBrush                    & \checkmark               \\
\midrule
\multicolumn{5}{l}{\textit{NCP \& ADC}}\\
Patch Number  & 64 & 16 & 80 & 80\\
Patch Tokens  & 256 & 256 & 256 & 256\\
Context Extent       & (2, 2, 0)       & (2, 2, 0)               & (1, 1, 3)       & (2, 2, 3)              \\
\midrule
\multicolumn{5}{l}{\textit{Dataset}}\\
Name                      & RQF                & LHQC               & LHQ-V                & PeppaPig                \\
Train Scale              & 128                 & 85K                  & 38K                 & 10K                  \\
Test Scale                  & N/A                  & 5K                    & 2K                  & 1K                   \\
\midrule
\multicolumn{5}{l}{\textit{Training \& Inference}}\\
Epoch               & 6000                  & 50                   & 50                  & 150                   \\
Visual Size             &          &           &          &           \\
Frame Length  & N/A & N/A & 5 & 5 \\
Text Length              & N/A              & 77                 & N/A               & 77                 \\
Batch Size                   & 128                 & 512                  & 256                 & 256                 \\
Learning Rate                &                &                  &                &                  \\
Warmup Ratio                 & 2\%                 & 5\%                 & 5\%                 & 5\%                 \\
\bottomrule
\end{tabular}
\caption{Implementation details of model training for different tasks. }
\label{tab:implementation_details}
\end{table}




\subsection{Implementation Details}\label{sub:imp}


The training of NUWA-Infinity can be split into two stages. In the first stage, all images are cropped into  and videos are cut into  with  fps. Then, they are encoded into discrete visual tokens using the VQGAN model
with a compression ratio of 16. 
In the second stage, we train the model using Adam optimizer~\cite{kingmaAdamMethodStochastic2014} with a learning rate of , a batch size of , and warm-up \% of total  epochs. We train four models on four datasets and inference on five tasks. More settings for different tasks can be found in Tab.~\ref{tab:implementation_details}. 




\subsection{Metrics}

\begin{itemize}[leftmargin=*]

\item \textbf{FID/Block-FID}. Fréchet Inception Distance (FID)~\cite{heuselGansTrainedTwo2017} is used to calculate the quality of the generated image. In this work, we also propose Block-FID, which splits a large image into several blocks and calculates the average Fréchet Inception Distance of all blocks. 


\item \textbf{IS}. Inception Score (IS)~\cite{salimansImprovedTechniquesTraining2016} is a common metric to calculate the diversity of the generated results. A higher IS indicates the model generates more diversified results.
\item \textbf{FVD}. Fréchet Video Distance (FVD) \cite{unterthinerAccurateGenerativeModels2018} is widely used to calculate the quality of generated videos. It measures the distance between the ground-truth video and the generated video. A lower FVD denotes a higher similarity.
\item \textbf{CLIP-SIM}. Recently, the CLIP Similarity Score (CLIP-SIM)~\cite{radfordLearningTransferableVisual2021} is used to measure the  semantic consistency between images and text. We use the CLIP model to calculate the similarity score between the generated image and given text to judge its semantic consistency.
\end{itemize}

\subsection{Main Results}
\subsubsection{Unconditional Image Generation\textsuperscript{\textbf{HD}}}
\vspace{-2mm}

Unconditional Image Generation\textsuperscript{\textbf{HD}} aims to generate images without conditions. Fig.~\ref{fig:uig1} and \ref{fig:uig2} show a single long image generated by the model trained on the RQF dataset for Unconditional Image Generation\textsuperscript{\textbf{HD}}. The generated image has an extremely high resolution of . To better fit the page, we split the complete image into 6 splits, each having a resolution of . The generated results demonstrate the following abilities of NUWA-Infinity: 
\begin{itemize}[leftmargin=*]

\item \textbf{Infinity ability}.  NUWA-Infinity can generate visual content with large size. Although trained on  patches as illustrated in Tab.~\ref{tab:implementation_details}, NUWA-Infinity can generate images 19 times longer than each training instance. This is attributed to our proposed NCP module, which makes the computation grow linearly with the output size, as only a small number of previously generated patches in the context are used during inference. 

\item \textbf{Creation ability}. By comparing the generated results with the original painting, we find that NUWA-Infinity has a strong creative ability. For example, for the gate wall of the second split in Fig.~\ref{fig:uig1}, it is a composition of multiple walls in different directions. Also, for the first split in Fig.~\ref{fig:uig2}, NUWA-Infinity generates a lot of pedestrians and houses. Many different people walk together, which leads to a crowded situation in this split. Note that these scenes do not appear in the original painting, and they are fully created by NUWA-Infinity.

\begin{figure}[h]
\makebox[\linewidth]{
        \includegraphics[width=1.5\linewidth]{Figure/QM60.jpg}
    }
    \makebox[\linewidth]{
        \includegraphics[width=1.5\linewidth]{Figure/QM61.jpg}
    }
    \caption{Part I of a huge image () synthesized in Unconditional Image Generation\textsuperscript{\textbf{HD}} task on RQF dataset. Splits are connectable by row, each has a resolution of .}
    \label{fig:uig1}
\end{figure}


\begin{figure}[p]
 \vspace*{-2.3cm}
    \makebox[\linewidth]{
        \includegraphics[width=1.5\linewidth]{Figure/QM62.jpg}
    }
    \makebox[\linewidth]{
        \includegraphics[width=1.5\linewidth]{Figure/QM63.jpg}
    }
        \makebox[\linewidth]{
        \includegraphics[width=1.5\linewidth]{Figure/QM64.jpg}
    }
        \makebox[\linewidth]{
        \includegraphics[width=1.5\linewidth]{Figure/QM65.jpg}
    }
\caption{Part II of the huge image () illustrated in Fig.~\ref{fig:uig1}.}
    \label{fig:uig2}
\end{figure}



\item \textbf{Local details and global consistency}. The generated results also show decent local details and global consistency. In NUWA-Infinity, the local autoregression generates visual tokens one by one, thus the human faces, human gestures, leaves of trees, roof tiles, and many other details are clearly painted. The global autoregression generates patches one-by-one. As shown in Fig.~\ref{fig:uig2}, the human figures gradually dwindle, and then the picture transitions to the mountains. The smooth transitions make the picture look natural globally even though it is extremely long.

\end{itemize}


























\subsubsection{Text-to-Image\textsuperscript{\textbf{HD}}}
Text-to-Image\textsuperscript{\textbf{HD}} aims to generate an image based on the input text. For a fair comparison, all models are trained from scratch on the LHQC dataset. As shown in Tab.~\ref{tab:t2i_task}, when generating images with  resolution, NUWA-Infinity outperforms AR-based models Taming Transformer~\cite{esserTamingTransformersHighResolution2021} and Mask-Predict model MaskGIT~\cite{changMaskGITMaskedGenerative2022}, in visual quality (Block-FID), semantic consistency (CLIP-SIM), and diversity (IS). When generating images of size   which is 4 times as long as the training images, the performance of MaskGIT \cite{changMaskGITMaskedGenerative2022} decreases rapidly but NUWA-Infinity still maintains excellent visual quality with Block-FID of . As also shown in Fig.~\ref{fig:vis_t2i}, NUWA-Infinity generates significantly better results, and the reflection of the hill can be clearly seen. Note that we did not compare with DALL·E 2~\cite{rameshHierarchicalTextConditionalImage2022}, Imagen \cite{sahariaPhotorealisticTexttoImageDiffusion2022}, or Parti~\cite{yuScalingAutoregressiveModels2022} directly because of three reasons: () all of them do not support arbitrary-large visual synthesis (i.e, generating images with  resolution);  () they did not make their pre-trained models public; and () NUWA-Infinity focuses on enabling infinite visual synthesis and is not pre-trained on large-scale datasets, thus it is hard to make a direct comparison.



\begin{table}[htbp]
\small
\centering
\setlength{\tabcolsep}{14pt}
\begin{tabular}{lcccc}
 \toprule
Method    & Block-FID↓    & Block-FID()↓ & IS↑           & CLIP-SIM↑       \\    
\midrule
Taming \cite{esserTamingTransformersHighResolution2021}    & 38.89         & 46.37          & 4.58          & 0.2662          \\
MaskGIT \cite{changMaskGITMaskedGenerative2022}   & 24.33         & 45.76          & 4.61          & 0.2754          \\
NUWA-Infinity & \textbf{9.71} & \textbf{15.65} & \textbf{4.98} & \textbf{0.2807} \\
 \bottomrule
\end{tabular}
\vspace{-2mm}
\caption{Comparisons on LHQC dataset for Text-to-Image\textsuperscript{\textbf{HD}} task.}
\label{tab:t2i_task}
\end{table}



\begin{figure}[h]\makebox[\linewidth]{
    \includegraphics[width=1.1\linewidth]{Figure/t2i_compare.pdf}
    }
    \caption{Samples on LHQC dataset for Text-to-Image\textsuperscript{\textbf{HD}}. Left: , Right: . For a fair comparison, we did not provide input sketch for Taming Transformer.}
    \label{fig:vis_t2i}
    	\vspace{-3mm}
\end{figure}


\subsubsection{Image Animation\textsuperscript{\textbf{HD}}}
\begin{table}
\centering
\begin{tabular}{lc}
\toprule
Method        & FVD↓           \\
\midrule
NUWA-Infinity (NAR)   &368.16          \\
NUWA-Infinity (P-NAR) & 165.39          \\
NUWA-Infinity (AR) & \textbf{62.57} \\
\bottomrule
\end{tabular}
\caption{Comparisons on LHQ-V dataset for Image Animation\textsuperscript{\textbf{HD}} task.}
\label{tab:i2v_task}
\end{table}
Image Animation\textsuperscript{\textbf{HD}} aims to generate a video based on an input image. We build two baseline models as a comparison shown in Tab.~\ref{tab:i2v_task}. All three models are globally autoregressive, but they use different local generative methods (shown in round brackets). NUWA-Infinity (AR) is our default autoregressive over autoregressive model with a local autoregressive mechanism to generate a patch. NUWA-Infinity (NAR) can be viewed as an autoregressive over non-autoregressive model, as the tokens inside a patch are generated locally in parallel instead of autoregressively. NUWA-Infinity (P-NAR) can be viewed as an autoregressive over progressive non-autoregressive model, as locally, the tokens inside a patch are generated by a Mask-Predict method~\cite{changMaskGITMaskedGenerative2022} introduced in Sec.~\ref{sec:relatedwork}. NUWA-Infinity achieves significantly better performance than baselines, with an FVD score of 62.57. Fig.~\ref{fig:vis_i2v} provides a qualitative comparison between the generated -frame videos from the same input image of  resolution. We find that NUWA-Infinity with a default local AR mechanism can generate more realistic images. However, we do find a speed-performance trade-off as AR generation takes more time compared with NAR and P-NAR. We will discuss it in Sec.~\ref{sec:ablvd}.




\begin{figure}[h]
    \centering
    \includegraphics[width=1\textwidth]{Figure/vis_i2v.pdf}
    \caption{Samples on LHQ-V dataset for Image Animation\textsuperscript{\textbf{HD}} task. We compare local NAR, P-NAR, and AR by generating  frames with a high resolution of .}
    \label{fig:vis_i2v}
	\vspace{-3mm}
\end{figure}

\subsubsection{Image Outpainting\textsuperscript{\textbf{HD}}}
Image Outpainting\textsuperscript{\textbf{HD}} aims to generate an out-painted image based on the input image. We do not train NUWA-Infinity for the Image Outpainting\textsuperscript{\textbf{HD}} task, but used the model trained for Text-to-Image\textsuperscript{\textbf{HD}} directly. To evaluate the model's ability of outpainting in all directions, we set up four settings: Right Extend , Left Extend , Down Extend  and Up Extend . For example, in the Right Extend  setting, we input the image with the left half in the LHQC dataset and ask the model to predict the right half. Since for the output image in this task, half is ground-truth and half is extended, we calculate Block-FID between the extended area and the same area in the ground-truth test set of LHQC. As shown in Tab.~\ref{tab:ie}, NUWA-Infinity significantly outperforms Taming~\cite{esserTamingTransformersHighResolution2021} and MaskGIT~\cite{changMaskGITMaskedGenerative2022} by a large margin in all four directions. Since the model trained by Text-to-Image\textsuperscript{\textbf{HD}} also supports a text prompt, we also try to outpaint the image by adding text controlling, and we find better performance on Down Extend  and Up Extend , while similar performance on Right Extend  and Left Extend  compared with one without text. We hypothesize that it is because the upper or lower half of the image contains less information, while the left or right half has more visual semantic hints. Note that for a fair comparison, Taming and MaskGIT use text prompt by default.

In Fig.~\ref{fig:vis_ie}, we provide four input images illustrating four directions. Taming Transformer only successfully generates the input image in the third column, as it predicts the lower half of the hill based on the upper half. This is because Taming Transformer only trains token-by-token in , and this order only fits the down extension. MaskGIT successfully predicts another half in all directions benefited by its bidirectional masked language model. Compared with MaskGIT, NUWA-Infinity can generate more realistic results. For example, in the fourth column, when inputting the reflection of a tree in the lake, NUWA-Infinity successfully generates the most consistent tree on the shore.
\begin{figure}[t!]
    \centering
    \includegraphics[width=1\textwidth]{Figure/ie_compareV2.pdf}
    \caption{Samples for Image Outpainting\textsuperscript{\textbf{HD}} task on LHQC dataset. The input image has a resolution of half of .}
    \vspace{-4mm}
    \label{fig:vis_ie}
\end{figure}
\begin{table}[htbp]
\small
\begin{tabular}{lcccc}
\toprule
\multirow{2}{*}{Method} & \multicolumn{4}{c}{Block-FID↓}                                \\ \cmidrule{2-5} 
                        & Right Extend    & Left Extend    & Down Extend   & Up Extend     \\ 
                        \midrule
Taming \cite{esserTamingTransformersHighResolution2021}                  & 22.53         & N/A           & 26.38         & N/A           \\
MaskGIT \cite{changMaskGITMaskedGenerative2022}                 & 14.68         & 14.81         & 25.57         & 25.38         \\
NUWA-Infinity w/o text      & \textbf{6.43} & \textbf{6.71} & 11.47         & 8.03          \\
NUWA-Infinity       & 6.45          & 6.72          & \textbf{9.84} & \textbf{7.43} \\
\bottomrule
\end{tabular}
\caption{Comparisons on LHQC dataset for Image Outpainting\textsuperscript{\textbf{HD}} task.}
\label{tab:ie}
\end{table}











\subsection{Ablation Studies}
\vspace{-2mm}
We conduct detailed ablations on three components of our model, ADC, NCP, and Vision Decoder. 

\subsubsection{Ablation Study on ADC}
\vspace{-2mm}
\paragraph{Patch size of ADC.Split}
\begin{wrapfigure}[11]{r}{0.5\textwidth}
\vspace{-0.17in}
    \centering
    \includegraphics[scale=0.45]{Figure/ab_patchsizeV3.pdf}
    \vspace{-1.5em}
    \caption{Impact of patch size.}
    \label{fig:adcsplit}
\end{wrapfigure}
As introduced in Sec.~\ref{sec:train}, each patch representation has a size of . The horizontal axis of Fig.~\ref{fig:adcsplit} shows different patch sizes , and the vertical axis shows FID scores for Text-to-Image\textsuperscript{\textbf{HD}}. The blue line shows the FID score of generating image resolution of . The red line shows the Block-FID score of generating image resolution of . A smaller patch size harms the FID score of the generated image, but a larger patch size requires larger GPU memory. When patch size , a good balance between performance and memory can be achieved.

\paragraph{Feed position of ADC.Emb} 
\begin{wraptable}[6]{r}{0.42\textwidth}
\vspace{-0.25in}
\centering
\vskip 0.1in
\setlength{\tabcolsep}{2.5pt}
\begin{tabular}{lcc}
    \toprule
    RPE  & Block-FID↓ & Block-FID(×4) ↓ \\ \hline
Pre  & \textbf{10.05}      & \textbf{17.78}         \\
    Post & 10.47      & 18.89  \\  
    \bottomrule
    \end{tabular}
\caption{Impact of feed position.}
 \label{tab:adcemb}
\end{wraptable}
As introduced in Sec.~\ref{sec:adc}, ADC dynamically provides relative positional embeddings during training and inference stages. In Eq.~\ref{eq:qkv}, the relative positional embedding  is added to the key of self-attention. We call it pre-feeding, as the positional information is fed before the computation of attention. This is different from traditional post-feeding in Transformers, where the positional information is fed after the computation of attention. Tab.~\ref{tab:adcemb} shows that pre-feeding brings better performance than post-feeding. This is because pre-feeding controls which contexts to attend to with positional information while post-feeding only adjusts the attention distributions after the attention. 

















\subsubsection{Ablation Study on NCP}

\paragraph{Caches in NCP}
\begin{wraptable}[7]{r}{0.46\textwidth}
\vspace{-0.16in}
\centering
\setlength{\tabcolsep}{2pt}
\begin{tabular}{lcc}
\toprule
Context      & Block-FID↓ & Block-FID(×4) ↓ \\ 
\midrule
w/o caches & 15.80       & 38.32           \\
w/ caches  & \textbf{10.21}      & \textbf{23.62} \\
\bottomrule
\end{tabular}
\caption{Ablation results in caches in NCP.}
\label{tab:transfer}
\end{wraptable}

As introduced in Sec.~\ref{sec:ncp}, the Add operation saves multi-layer hidden states of previously generated patches as ``caches''. 
This allows information transmission between patches during training and inference. In other words, even though distant patches are removed from NCP, their information can be still captured in the hidden states of the nearby patches in NCP. 
To verify the effectiveness of this design, we train another model without using the caches in NCP and show its result of Text-to-Image\textsuperscript{\textbf{HD}} in Tab.~\ref{tab:transfer}.
The NCP design with information transmission significantly outperforms the one without information transmission.




\begin{figure}
\centering
\begin{subfigure}[t]{0.44\textwidth}
    \centering
    \includegraphics[width=\textwidth]{Figure/ab_spatial.pdf}
\caption{Spatial extension.}
    \label{fig:spatialextend}
\end{subfigure}
\begin{subfigure}[t]{0.44\textwidth}
    \centering
    \includegraphics[width=\textwidth]{Figure/ab_temporal.pdf}
\caption{Temporal extension.}
    \label{fig:temporalextend}
\end{subfigure}
\vspace{1.5em}
\caption{Ablation results on extents in NCP.}

\vspace{-2em}
\end{figure}








\paragraph{Extents in NCP}
As introduced in Sec.~\ref{sec:ncp}, NCP saves caches in a 3D extent . Fig.~\ref{fig:spatialextend} shows the comparisons between different extent size of  for Text-to-Image\textsuperscript{\textbf{HD}}. Although larger extent size brings better Block-FID scores, the performance becomes limited when  while computation improves significantly. For videos, Fig.~\ref{fig:temporalextend} shows the impact of temporal extent on Image Animation\textsuperscript{\textbf{HD}}. We find the FVD improvements become limited if the  increases to 3. As a result, we choose  as our default setting. 






\subsubsection{Ablation Study on Vision Decoder}\label{sec:ablvd}
Tab.~\ref{tab:rendermodel} shows ablations on different vision decoders. NUWA-Infinity is an autoregressive over autoregressive model and it follows the autoregressive (AR) formulation in the vision decoder. We also try another two vision decoders based on a non-autoregressive (NAR) formulation and a progressive autoregressive (P-NAR) formulation. For these two models, the global autoregression is still maintained, while the local autoregression is changed into NAR and P-NAR, respectively. We find AR-based vision decoder achieves the best performance and NAR-based vision decoder achieves the fastest speed.

Tab.~\ref{tab:modelsize} shows two settings of NUWA-Infinity: Base and Large. We find that the base model can also achieve acceptable performance compared with the large model. This is due to the limited training samples in the dataset. In this paper, we focus on the effectiveness of NUWA-Infinity architecture, instead of large-scale pre-training. We will pre-train NUWA-Infinity with more data in the future.


Tab.~\ref{tab:lossregion} shows when to optimize the loss between the predicted patch of vision decoder and the ground-truth patches. During training, as long as a patch is predicted, NUWA-Infinity calculates the patch loss and optimizes it immediately instead of accumulating the gradients of all patches. We simply call this mechanism ``patch loss'' and the accumulated one ``accumulated loss''. We find that patch loss accelerates convergence from  epochs to  epochs and improves the Block-FID score compared with the accumulated loss. This is because patch loss will share optimized parameters between each patch, which helps the model learn large images or videos.

\begin{table}[t!] 
\small
\begin{subtable}{1.\textwidth}
\centering 
\begin{tabular}{lcccc} 
\toprule
Vision Decoder & Block-FID↓ & Block-FID( 4)↓ & CLIP-SIM↑ & Inference Speed↑ \\ \hline
NUWA-Infinity (NAR)   & 92.34 & 98.67 & 0.2451  & \textbf{95} \\
NUWA-Infinity (P-NAR) & 19.86 & 38.59 & 0.2726  & 15          \\
NUWA-Infinity (AR)    & \textbf{10.05} & \textbf{17.78} & \textbf{0.2753}    & 1  \\
\bottomrule
\end{tabular}
\caption{Decoder model.}
\label{tab:rendermodel}
\end{subtable}
\begin{subtable}{.48\linewidth}
\setlength{\tabcolsep}{4.5pt}
\begin{tabular}{lccc}
\toprule
Parameters  & Depth & Dim  & Block-FID↓ \\ \hline
202M (Base)  & 16    & 768  & 10.05      \\
809M (Large) & 24    & 1280 & \textbf{9.71}  \\
\bottomrule
\end{tabular}
\caption{Decoder size.}
\label{tab:modelsize}
\end{subtable}
\hspace{0.6em}
\begin{subtable}{.45\linewidth}
\setlength{\tabcolsep}{1.5pt}
\centering
\begin{tabular}{lcc}
\toprule
Loss  & Block-FID↓ & Convergence epoch↓ \\ \hline
Patch & \textbf{10.05}         & \textbf{50}                 \\
Accumulated  & 11.62         & 70   \\ 
\bottomrule
\end{tabular}
\caption{Decoder loss.}
\label{tab:lossregion}
\end{subtable}
\caption{Ablation experiments with text-to-image generation on LHQC. We use a base model with  layers and dim of  except (b). } \label{tab:ablation_five}
\end{table}

\section{Discussions}

\paragraph{Training Data}
The model for infinite visual synthesis requires high-resolution images and videos as the training data. Such high-quality visual data are harder to collect duo to quality and license issues. In the future, we will collect large-scale datasets satisfying the quality and license criteria for the development of this research direction.

\paragraph{Evaluation Metric}
Compared to text generation tasks, such as machine translation and text summarization, visual synthesis is more difficult to evaluate as the number of ground-truths of a model's output could be unlimited. Currently, we follow the traditions (i.e., using FID, FVD, IS, and CLIP-SIM scores) to measure the quality of generated images and videos. In the future, we will explore better evaluation metrics for visual synthesis tasks.

\paragraph{Inference Speed} Autoregressive models can deal with dependencies between generated contents well. However, the training and inference efficiency of this generation mechanism is still a blocking issue for deploying such models for practical usage. In the future, we will explore ways to combine the advantages of autoregressive models and non-autoregressive models (such as the diffusion model) to achieve both generation quality and inference (or training) efficiency.

\paragraph{Pre-trained Version} In this paper, we train NUWA-Infinity for different downstream tasks directly, due to the lack of large-scale high-quality visual data. In the future, we will pre-train the next version of NUWA-Infinity with more collected visual data and report its generalization  capabilities on open-domain inputs. 










\section{Conclusion}
NUWA-Infinity is a visual synthesis framework that can be trained to generate high-quality images and videos from the given text or image input. Different from DALL·E, DALL·E 2, Imagen and Parti, an autoregressive over autoregressive mechanism is proposed to support variable-size visual content generation tasks, such as image outpainting, image animation, text-to-image generation, and text-to-video generation. We hope such models help visual content creators save time, cut costs, and increase productivity and creativity.


\section*{Acknowledgements}

We'd like to thank Minheng Ni, Xiaodong Wang, and Bei Li for the figure and table formats of this paper. We'd also like to thank Yu Liu, Jieyu Xiao, Scarlett Li, and Jane Ma for the discussion of potential application scenarios. We'd also like to thank Yang Ou and Bella Guo for the design of the homepage, and Tiantian Xue and Daisy Hou for the implementation of the homepage. We'd also like to thank Ting Song, Yan Xia, and Shiyou Ren for the help with the dataset construction. We'd also like to thank Yan Fan and Quanlu Zhang for their system support.
































\bibliographystyle{plain}
\bibliography{neurips_2022}



\end{document}