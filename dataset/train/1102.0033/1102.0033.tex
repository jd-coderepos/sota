

\documentclass[times]{rncauth}

\usepackage{moreverb}

\usepackage[dvips,colorlinks,bookmarksopen,bookmarksnumbered,citecolor=red,urlcolor=red]{hyperref}

\newcommand\BibTeX{{\rmfamily B\kern-.05em \textsc{i\kern-.025em b}\kern-.08em
T\kern-.1667em\lower.7ex\hbox{E}\kern-.125emX}}

\def\volumeyear{2010}

\usepackage{subfigure}
\usepackage{multirow}
\graphicspath{{fig/}}







\newcommand{\be}{\ensuremath \mathbf{e}}
\newtheorem{thm}{Theorem}[section]
\newtheorem{cor}{Corllary}[section]
\newtheorem{property}{Property}[section]
\newtheorem{lem}{Lemma}[section]
\newtheorem{exam}{Example}[section]
\newtheorem{prop}{Proposition}[section]
\newtheorem{defn}{Definition}[section]
\newtheorem{rem}{Remark}[section]
\newtheorem{assum}{Assumption}[section]
\newtheorem{prob}{Problem}[section]
\newtheorem{alg}{Algorithm}[section]





\begin{document}

\runningheads{H.~ HUANG, C. YU and Q. Wu }{Control of Multi-Agent Formations with Only
Shape Constraints}

\title{Control of Multi-Agent Formations with Only Shape Constraints
\footnotemark[2]}

\author{H.~Huang\footnotemark[3], C. Yu\corrauth and Q. Wu\footnotemark[3]}

\corraddr{Research School of Information Sciences and Engineering, The Australian National University, Canberra ACT 2600, Australia (Brad.yu@anu.edu.au)}

\begin{abstract}This paper considers a novel problem of how to
choose an appropriate geometry for a group of agents with only shape
constraints but with a flexible scale. Instead of assigning the
formation system with a specific geometry, here the only requirement
on the desired geometry is a shape without any location, rotation
and, most importantly, scale constraints. Optimal rigid
transformation between two different geometries is discussed with
especial focus on the scaling operation, and the cooperative
performance of the system is evaluated by what we call the geometries
 \emph{degrees of similarity} (DOS) with respect to the desired shape during the entire convergence process. The design  of the
scale when measuring the DOS is discussed from constant value and time-varying function
perspectives respectively. Fixed structured nonlinear  control laws
that are functions on the scale are developed to guarantee the
exponential convergence of the system to the assigned shape. Our
research is originated from a three-agent formation system and is
further extended to multiple () agents by defining a
\emph{triangular complement graph}. Simulations demonstrate that
formation system with the time-varying scale function outperforms
the one with an arbitrary constant scale, and the relationship
between underlying topology and the system performance is further
discussed based on the simulation observations. Moveover, the
control scheme is applied to bearing-only sensor-target localization
to show its application potentials.
\end{abstract}

\keywords{Multi-agent systems; Nonlinear formation control; Shape
constraints.}

\maketitle

\footnotetext[2]{This work was completed during H. Huang's visit at
The Australian National University which was supported by the
University Fund of Beijing Institute of Technology. This work is
also supported by the National Natural Science Foundation of China
under grant 61074031. C. Yu is supported by the Australian Research
Council through Queen Elizabeth II Fellowship under
DP-110100538.}
\footnotetext[3]{School of Automation, Beijing Institute of Technology, Beijing 100081, China (huanghuang,qinghew@bit.edu.cn)}


\vspace{-6pt}







\section{Introduction}
\vspace{-2pt}
 A group of agents working together in formation is
seen in various field applications including, for example,
spacecrafts exploring the deep space, underwater vehicles mapping
out oceanbed and unmanned aerial vehicles detecting an unknown
territory. Studies concerning this subject focus primarily on the
stability of the formation systems where there are three  well-known
methodologies, namely distance-based formation control
laws\cite{Anderson07threeForm,Dimarogonas08distanceFormation},
position-based control
laws\cite{Yu09minimalPersistent,Krick09infinitesimallyRigid,Ren08distributed},
and very recently, angle-based
algorithms\cite{Moshtagh09visionFormationBearing,Basiri10angleonly}.
Meanwhile, it is well known that the stability and the performance
of the formation systems depend highly on the underlying
communication topologies. This fact leads to the exploration of
information flow laws among
agents\cite{Fax04,Olfati04consensusSwitching} and  the communication
topology properties of the formation
systems\cite{Anderson08rigidMagazine,Olfati02rigidity}. Consequently
there are control algorithms that aim to deal with various network
dynamics\cite{Olfati04consensusSwitching,Luke09formationDelay,Secchi08formationDelay}.


Current research on formation control relies on the assumption that
the desired geometry is specified, fixed, and known a
priori either in a global coordinate or a local one. Inspired by the
fact that V-shaped formation provides birds with more aerodynamical and visual
advantages than other types of flight in
flocks\cite{Nature01Vformation}, it is reasonable to conjecture that
the geometry formed by the group of agents is closely related to the
behaviors of the formation system. Thus with all those abundant
results under the determined assumptions, we are now ready to move
on to the next stage, that is, to consider situations where the only
concern of the final state is the shape of the system rather than some specific geometry.
In other words, this research considers the scenario where the
requirement on the desired formation is its geometrical information
without any constraints on its location, rotation and scale.

This particular problem arises in angle-based emitter-target
localization  that emerged in
1950s\cite{Bishop09bearingOnlyLocalization}. It were proved in
\cite{Bishop10sensor}, \cite{Dogancay08AOA} and
\cite{Zhang05contour} that there were a set of geometries such as
equilateral triangles in which sensors could provide  the most
accurate estimations of the target position. When the distances
between sensors to the
 target are  sufficiently larger than the ones between sensors, the shape formed by
 the group of agents is then the major concern instead of a specific geometry. It can be further
verified that the geometry with a higher degree of similarity (DOS) (please refer to Section 3 for its definition) to the optimal shape can provide better target location estimation. Thus \emph{in order to improve the accuracy of measurements on the target before sensors reaching the desired shape so as to allow more time for strategy making, a formation  system that maintains a high DOS to the optimal shape during convergence is more preferable than simply focusing on the final shape.}


However, despite of the potential advantages brought by the
geometrical resemblance, optimal formation control  with only shape
constraints has not yet received enough attentions. To our best
knowledge, there are only a few literatures that are \emph{related}
to this topic. One was reported in
\cite{Spletzer05optimalPositioning} where the authors focused on
finding the optimal scale of the prescribed shape for a set of
robots such that the total distance they traveled was minimized.
However, robot control during the process was ignored and thus the
group of robots did not exhibit any  cooperative behaviors. In
another work\cite{Bhatt09geometric} on optimal relative formation
control of three wheeled robots, the cost function was the kinematic
energy sum of the three robots. By choosing a robot as the rotation
center, the optimal relative positions of robots were designed.
Shape control irrespective of any objectives is seen in a small
number of literatures. By modeling each agent as a double
integrator, reference \cite{Pais09formationShape} considered shape control on
the projected orthogonal reference frame, and very recently,
reference \cite{Basiri10angleonly} proposed a distributed algorithm that
stabilized three agents in the assigned shape. However, those research ignored the performance during the convergence process and thus are not  favorable to tasks such as sensor-target localization.






To this end, this paper aims to explore the novel topic of formation
control under only shape constraints, and exploit the design of
scales that provide the optimal rigid transformation(rotation, translation and scaling operations) between two different geometries. The objective is to minimize  a cost function that integrates the difference between the geometry and the desired shape over the entitle convergence process.
Although there are quite a few discussions on the optimal formation
control, the objectives are normally  fuel
consumption\cite{Kim09pathplanning}, traveling length and
time\cite{Tillerson02}. Fixed structured nonlinear control laws on the
edge state and the pending scale are proposed. When choosing the
appropriate scale of the final geometry so as to minimize the cost
value, two situations are considered one of which is to determine an
optimal value a priori and set it constant during the entire
process. Another more intelligent method is to design a function
named the scale function that adjusts the scale online.

Here is an outline of the paper: Section II contains the notations
and definitions that are used in the paper. A nonlinear control law,
which is the starting point of this research, is introduced in this
section as well. In Section III, the difference between two geometries is defined and
further the cost function is proposed which represents the geometrical performance during formation attainment.
Design of a constant  scale is discussed in Section IV, and in Section V a scale function is  proposed that  further reduces the minimum from the
previous section. By defining a \emph{triangular complement graph}
of the minimally rigid graph, the algorithm applies to multiple
 agents case as well. In order to broaden the application of
the nonlinear algorithm to situations with determined final
geometries, controllability of the system is discussed by adding a
positive weight factor to the nonlinear control law. Moreover,
applications of the proposed control schema in sensor-target
localization are presented in Section VI. Section VII presents
extensive experimental simulations that verify the effectiveness of
the algorithm and show its application potentials. Conclusions are
made in Section VIII.

\vspace{-6pt}

\section{Preliminary}
\vspace{-2pt}
 Positive real numbers are denoted by 
and a matrix of size  is denoted by  where when , it is always abbreviated by .
The (block) diagonal matrix of vector  is denoted by
 with (vector)  being the th (block)
diagonal entry. Sometimes we may also use
 instead.


  For an undirected graph , the numbers of edges and vertices are
  denoted by  and  respectively.
Vertex  is the neighbor of vertex , denoted by , if
they are connected by an edge in . Edge  is the neighbor of
edge  if they share a vertex. The neighbor sets of vertex  and
edge  are denoted by  and 
respectively. The minimal distance between vertex  and vertex 
is the minimal number of  edges in  that are needed to have them
connected.


Throughout the paper, we focus on 2D formations on a plane, and we
do not consider a system where agents are collinear or coincided,
that is
\begin{assum}\label{assumption_generic}
The initial positions and the desired positions of agents in a
formation system are non-collinear.
\end{assum}


For graph , if we label the nodes in  from 1 to  and the
edges in  from 1 to , the \emph{representation} of  is a
vector  whose th entry  is the
position of the th node. Given , if we assign an arbitrary
direction on each edge, then the  representation 
uniquely determines an edge vector  where  corresponds to the th directed edge in , and
 is called a \emph{relative representation} or a
\emph{geometry} of . Note that in this paper, a geometry is
associated with a graph, which is a more strict definition than that
of a polygon. Given a graph , the distance between two
representations  and  is
. The
representation  is further called a \emph{realization}
of  if all the distances between neighboring vertexes in  are compatible, and consequentially  is the \emph{relative
realization} of .
 All possible realizations of  form a vector space denoted by
, and consequently a link/edge space .  A
\emph{formation} system is a group of agents that communicate over
 and whose deployment is depicted by a realization of .

A realization  of  in two dimensions is rigid if
there exists  such that for all realizations
 that satisfy  and , there
holds . A rigid graph is
further minimally rigid if no single edge can be removed without
losing rigidity. Finally, a globally rigid graph  is a graph
where  holds for an
arbitrary . More rigorous definitions could be found,
for example, in
\cite{Yu09minimalPersistent,Anderson08rigidMagazine}.

Given a graph  and two realizations   and
,  is \emph{similar} to 
if

and consequently,   and  are \emph{similar}
geometries. Without lose of generality, assume node 1 in both
realizations is deployed at the origin. Then it is necessary that

where  and  is a rotation transformation
through angle . Fig. \ref{fig:sim_geometry} and Fig.
\ref{fig:nonsim_geometry} give examples showing similar and
dissimilar geometries. Although the two triangles in Fig.
\ref{fig:nonsim_geometry} are similar ones in the usual sense of
plane geometry, in our case this is no longer true due to the mismatching of indexed nodes.

Given a relative realization  of , a shape vector
 generated from  is determined by
 where . In most of cases, we will
use  for abbreviation. The set  consists of all
similar geometries to  over the same graph. Fig.
\ref{fig:expla_shape_geome} gives an explanation on the
relationships between those three notations. Elements are uniquely
determined along the arrows and the reverse is not true. In fact,
given a shape vector , it uniquely determines a shape or a
geometry  if and only if the underlying graph is a
globally rigid one\cite{Anderson08rigidMagazine}.

\begin{figure}
  \subfigure[Similar geometries]{\label{fig:sim_geometry}
\begin{minipage}[b]{0.31\linewidth}
\centering
\includegraphics[scale=0.6]{sim_geometry.eps}
\end{minipage}}
  \subfigure[Dissimilar geometries]{\label{fig:nonsim_geometry}
\begin{minipage}[b]{0.31\linewidth}
\centering
\includegraphics[scale=0.6]{nonsim_geometry.eps}
\end{minipage}}
  \subfigure[Dissimilar geometries]{\label{fig:nonsim_geometry2}
\begin{minipage}[b]{0.31\linewidth}
\centering
\includegraphics[scale=0.6]{nosim_geometry.eps}
\end{minipage}}
\caption{Geometries with ordered vertexes}
\end{figure}

\begin{figure}
\centering
  \includegraphics[width=6.7cm,height=1cm]{expla_shape_geometry.eps}\\
  \caption{The relationship between the shape and the geometry}\label{fig:expla_shape_geome}
\end{figure}



Due to the nature of the algorithms introduced in this paper and for
convenience, the scale of a geometry in  with
respect to  is

which is  identical to all edges.

The incidence matrix is frequently used to describe the structure of
a directed graph. For an  undirected graph, by assigning each edge
an arbitrary direction, we have the oriented incidence matrix as
seen, for example, in \cite{Diestel10}.
\begin{defn}[Oriented Incidence Matrix]
For an undirected graph , each edge is assigned with a
random direction to generate a directed graph .
The oriented incidence matrix of , denoted by , is a matrix
of dimension  whose th row corresponds to the
th edge in  with entries

\end{defn}
\begin{lem}\label{lemma:rankofH}
  The oriented incidence matrix  has rank  when  is connected.
\end{lem}


  By introducing the expended incidence matrix , where  is the Kronecker product, the edge space 
   can then be considered as a
  codomain of  from the vector space :
   where  and . Thus we have .

Further we define a rigidity function\cite{Yu09minimalPersistent}
(also known as the edge function\cite{Dorfler09cooperativeBehavior})

and the rigidity matrix is further defined as
 with
, . For a
rigid graph, .




\begin{lem}[\cite{Dorfler09cooperativeBehavior}]\label{lemma:dorfler}
For a desired relative realization  over  and
the vector  with
, by choosing the potential function

the control law

is an \emph{inverse} optimal solution to the optimization problem

and the formation system converges to the largest  invariant set

\end{lem}
The proof of the lemma is based on the Hamilton-Jacobi-Bellman (HJB)
equation and Lyapunov Theory. For details please refer to
\cite{Dorfler09cooperativeBehavior}.






Algorithm \eqref{eq:rigidoptU} requires that each agent has the
knowledge of the global coordinates of its neighbors. This
requirement is also assumed to be true in our research.

\section{Cost function and cooperative performance}
With the wide range of applications of sensor networks, the geometric characters of  a group of sensors are sometimes crucial to the achievement of tasks, such as bearing only sensor-target localization, shape-constrained formation reconfiguration and obstacle avoidance. Thus it is exigent to propose an  efficient way to distinguish two geometries.

In image processing, especially in pattern matching, the Hausdorff distance is commonly used to evaluate how much do two geometries resemble each other under rigid transformation\cite{Hutten93hausdorff}. The problem of shape matching is to find the optimal transformation that minimizes the distance. Sometimes points on the edges are assigned with different weights indicating their importance to the shape. However, the Hausdorff distance metric is on the basis a minimization function, which is  not favorable as an objective function. Meanwhile, at each  time step, the time complexity to compute the Hausdorff distance  for two points sets of size  and  is . Thus another metric function is expected that produces intuitively reasonable results and is beneficial to optimization.

Before we dive into the mathematical formalizations, here are two observations for determining the resemblance of  geometries with respect to the reference/desired shape:
\begin{itemize}
  \item [i] A geometry is less sensitive to perturbations on \emph{long} edges
  \item [ii] A geometry is less sensitive to perturbations on edges with large included angles 
\end{itemize}
Fig. \ref{fig:DOS} further explains the above two observations.
It is intuitive that  under the same perturbation, the long edge bring small deformation to the geometry
than that of a short edge, as the two geometries in Fig. \ref{fig:DOS_lengths}.
Meanwhile, apart from the lengths of edges, another profile of a shape is the vertex,
especially those sharp vertexes, i.e., the included angle of the two  edges adjacent at  the vertex is small.


 When two geometries are similar to each other, the scale between them is the ratio of two corresponding edges in the geometries and is identical to all  the pairs of edges. However, for two dissimilar geometries, there is no common ratio for any pairs of edges, thus choosing an appropriate scale is premise to the analysis of geometries resemblance.

\begin{figure}
 \subfigure[When the short edge suffer a small perturbation, the perturbed geometry various a lot w.r.t. the nominal one; When the same perturbation is added to the long edge, the new geometry is quite close to the nominal one]{\label{fig:DOS_lengths}
\begin{minipage}[b]{0.45\linewidth}
\centering
\includegraphics[scale=0.4]{DOS_length.eps}
\end{minipage}}
  \subfigure[When the angle included in the  two edges is large(), perturbations on the two edges will result in small change to shape; When the angle is small(), the shape is greatly affected by the same perturbations]{\label{fig:DOS_angles}
\begin{minipage}[b]{0.45\linewidth}
\centering
\includegraphics[scale=0.4]{DOS_angles.eps}
\end{minipage}}
\caption{The sensitivity  of a geometry}\label{fig:DOS}
\end{figure}


According to the above two observations, it is obviously inappropriate to simply use the differences between edges lengths of the two geometries to measure how two geometries are related:

because  dose not consider observation  or observation .

We are interested in the resemblance of two geometries, thus under observations  and , the difference between geometry  and  is

As one can tell, an appropriate scale  is key to the evaluation. In our research, the geometry is depicted by  directed edges
rather than the coordinates of the vertexes, thus the translation
and rotation operations are not our concerns.
 The element  indicates that
 each edge in one geometry is scaled by the edge difference between the two geometries.
 This is designed to take care of observation .b
Inspired by the weighted Hausdorff distance, we introduce the transformation  which allows two neighboring edges communicate and  consequently, as shown in Fig. \ref{fig:DOS_angles}, evolve into the edges in dot lines, the lengths of which are positive monotony to the sharpness of the vertexes.


Given a geometry    and a shape  , the degree of similarity (DOS) of  with respect to  is measured by

A geometry with a higher DOS to  is said to be more resemble to . When the geometry is similar to , .

As mentioned before, by taking care of the geometries DOS during the entire convergence, sensors can provide reliable measurement of the target in an early stage.  Thus in this research, the \emph{geometrical performance of a formation system is the integral of the DOS with respect to the desired shape during convergence}:

with  and . We use 
instead of  due to the composition of  in
\eqref{eq:rigidityFun}. Apart from the initial state 
and the control law , the value of  is also a function on the scale .


For example  we consider three agents in a formation as shown in Fig.
\ref{fig:three_topo}. If we set  and denote
, it yields

The value of  is the sum of  lengths of the three new vectors in Fig. \ref{fig:three_topo2}.

\begin{rem}
A formation system with a smaller cost value \eqref{eq:myJ0} is considered to
exhibit better geometrical/cooperative performance during convergence.
\end{rem}


By minimizing the cost function \eqref{eq:myJ0}, the two situations as, for example, the ones in Fig.
\ref{fig:cooperative_good_bad}, are expected to be distinguished. The upper case has better geometric
performance as the geometries during convergence have consistently high DOS
 with respect to the desired shape. For the bottom
situation, although the three agents attain the desired geometry, it
differs a lot from the desired shape during the process. In sensor
networks for localization, formation system in the upper case may
allow sensors to provide some rough estimate of the target before
attaining the formation. We will show this later through
experimental results in Section \ref{sec:exp}.
\begin{figure}
 \subfigure[Performance]{\label{fig:cooperative_good_bad}
\begin{minipage}[b]{0.43\linewidth}
\centering
\includegraphics[scale=0.6]{cooperative_good_bad.eps}
\end{minipage}}
  \subfigure[Triangle]{\label{fig:three_topo}
\begin{minipage}[b]{0.24\linewidth}
\centering
\includegraphics[scale=0.5]{three_topo.eps}
\end{minipage}}
  \subfigure[New space]{\label{fig:three_topo2}
\begin{minipage}[b]{0.23\linewidth}
\centering
\includegraphics[scale=0.5]{meaningofM_three2.eps}
\end{minipage}}
\caption{Triangular geometry of three agents. For the definitions of DOS, please refer to \eqref{eq:distance}}\label{fig:unknown}
\end{figure}

 When the
underlying graph of the formation system
 is a minimally rigid graph, the convergence of  is equivalent
to . Two forms of 
are considered in this paper, namely  being constant:
, and  being a time-varying
function on : , where  is the set of continuous mappings. The cost
functions are then written, respectively, as:

  and


The main concern in this research is the determination of  and
  such that the cost functions  and 
are minimized. Note that  and  are not just quadratic
expressions but a function on ,  and  or
, thus minimizing the values of  and
 is not trivial. Inspired by the control law in Lemma
\ref{lemma:dorfler}, in this research, we will fix the structure of
the control law and focus on the design of the optimal scale
function.
\begin{prob}\label{prob:general}
  Consider a formation system where each agent is modeled by

and the underlying graph  is a minimally rigid graph.
\begin{itemize}
\item Let , and consider the following control law on    and :

Find the optimal value of  such that  where
 is the stable realization and

\item Let  where , and consider the following control
law on   and :

Find the control gain  and the scale function  such that
 and

\end{itemize}
\end{prob}

When , it is actually a matching of two fixed geometries without considering the scaling operation. When , it turns to a matching of a geometry and a  shape, which gives us more flexibility to optimize  the geometry's DOS.


In the remaining context of the paper, we will discuss the
optimization problem \eqref{eq:jc0} and \eqref{eq:jv0} respectively.

\section{Optimal formations under time-invariant shape
scale}\label{sec:inva} System with a constant scale  during the
entire converging process is simple but typical. Instead of
considering a determined final geometry at a randomly selected scale
as discussed in Lemma \ref{lemma:dorfler}, here given the initial
relative realization , we focus on finding an optimal 
 such that the formation system has the best cooperative performance.


When  and under the control law \eqref{eq:myde_eqS},

The partial derivative of  with respect to  is

  By letting \eqref{eq:partialOfStaticS} equal to zero, it immediately yields
  that the optimal  solution  to problem \eqref{eq:jc0} is
  
  Under Assumption \ref{assumption_generic}, the analytical expression of the optimal scale is
  
  and it is obvious that .


Rigidity corresponds to a unique realization of the graph only if
the domain is small enough. For a globally unique realization, we
have the idea of global rigidity\cite{Anderson08rigidMagazine} which
tells that once the distances between any of the two neighboring
agents are known, all the other distances for non-neighbors are
uniquely determined. However, globally rigid graph is rigorous and
there is no polynomial time algorithm to determine weather a graph
is a globally rigid one or not. Meanwhile, its rigidity matrix is
not a full rank matrix, which may brought some unexpected
equilibriums in the rigidity matrix-based control laws. Thus we
would like to focus ourselves on minimally rigid graph whose
rigidity matrix is exactly of rank .


\begin{thm}
The formation system \eqref{eq:basic_sys} over a minimally rigid
graph converges to the largest invariant set

where

for a sufficiently small  if the control law  is the one in
\eqref{eq:myde_eqS}. Furthermore,  is obtained when
 with  given in \eqref{eq:optStaticS}.
\end{thm}
\begin{proof}
  Similar to the proof of Lemma \ref{lemma:dorfler},
  the control law \eqref{eq:myde_eqS} for an arbitrary  
   is a stabilization control law that guarantees the convergence to
   .
The derivation of  indicates that \eqref{eq:optStaticS} is
the optimal solution to \eqref{eq:jc0}, which finishes the proof.
\end{proof}
\begin{rem}\label{rem:initial}
 When the underlying graph  of the  shape is a minimally rigid
one, a shape  dose not determines a geometry uniquely, and
consequently  dose not ensures
. However, when the initial
geometry  of the formation system is restrained within
a sufficiently small neighborhood  of a candidate
geometry,  guarantees the convergence to the
desired shape.
\end{rem}

The optimal scale  is, not surprisingly, uniquely determined
by the initial geometry  and the desired shape . It
is the global optimum and is always positive with an upper bound
when the formation is non-collinear.

In a formation system, when a set of agents equipped with sensors is
deployed within a neighborhood of the desired shape specified by the
shape vector , \eqref{eq:optStaticS} helps us to decide the
exact final geometry that requires a small range of transformations
from , and the control law drives the agents approach this
optimal geometry.

\section{Formation control under time-varying scale
function}\label{sec:varying}
As mentioned previously, there are two kinds of strategies to find
an appropriate shape scale one of which  is to determine an optimal
value a priori and set it constant during the entire process, as
discussed in the previous section. Another more intelligent strategy
is to consider the time-varying function  that adjusts
the final scale online. This allows us to evaluate the resemblance of the geometry with respect to the desired shape more objective.



\subsection{Formations with three agents}
 Triangular
formations with three agents interconnected with one another are the
most fundamental pattern in formation systems, and thus are
embraced as the starting point in many literatures, as for example
in
\cite{Yu09minimalPersistent,Huang10optimalFormation,Anderson08UAV}.
Here we will also start with this simple and typical case and then
further extend the results into multiple  agents systems.

Under the fixed-structured control law  \eqref{eq:u_varying}, the
dynamics of the formation system in the edge space is

The control law  is a function of both 
and , and  in  is
also determined by , which means  and 
are highly coupled. This makes the solution to \eqref{eq:jv0}
becomes complex and the problem may not be solvable  exactly.

Alternatively, we would like to seek for a function
 for the following problem:
\begin{prob}\label{prob:threeOptimal}
Find a trajectory for , which is an
explicit function of edge , such that

for all .
\end{prob}
Instead of concentrating on the complex optimization problem
\eqref{eq:jv0}, we suggest to design an  adjusting rule for the scale such that the formation system under the control law always exhibits better
cooperative performance than the case with a constant scale.

Consider the equation

which has the equivalent form of

  where  is the loss function of .
By using the fact   and
the cosine law

where ,  and  form an triangle, we obtain

where
 If we let
,  a trajectory of
 is then

and next we prove that the cost function  with the scale
 \eqref{eq:mydynS} is consistently smaller than .
\begin{thm}\label{thm:JvJc}
  Given the same initial geometry  and the same desired shape , the inequality
  
  with  given in \eqref{eq:mydynS}
  always holds true for arbitrary . The
  equality is satisfied if and only if .
\end{thm}
\begin{proof}
  Substituting  into , the difference of  and
   is
  
Recall the characters of ,

Based on the expression of  in \eqref{eq:mydynS},

According to \eqref{temp} and \eqref{temp2}, equation
\eqref{eq:JvJc} has the simplified form of
  
  The positiveness of  is guaranteed according to the equality of
   which further
  yields
  
  where equality holds if and only if ,
i.e., .
\end{proof}

 Even when , the conclusion of  having a smaller
value still holds true, which indicates that better
cooperative performance is always observed on .


\begin{thm}\label{thm:stable}
The triangular formation system \eqref{eq:mydes_M} is exponentially
stable and converges to the invariant set

if
 with parameters  and 
given in \eqref{eq:D_three} and \eqref{eq:mydynS} respectively, and
 is the stable value of the scale function
.
\end{thm}
Before giving the proof to Theorem \ref{thm:stable}, we first
propose the following lemma
\begin{lem}\label{lemma:M_nonsigular}
  Matrix  in \eqref{eq:myM} is a singular matrix
  if and only if .
\end{lem}
\begin{proof}
Let  be the transformation matrix such that

Left multiplying matrix  by  yields

where
 and 
is the elements of matrix . Matrix  is of
full rank if and only if the diagonalize matrix
 with
 consists of only
nonzero diagonal entries, that is  or equivalently,

Recall \eqref{eq:lemma_temp}, if we consider  as
the inner product of the two vectors  and , we obtain

Substituting  and  given in \eqref{eq:mydynS}  into
\eqref{eq:dp_sd2d1} and by some trivial calculations, we conclude
that the necessary and sufficient condition for 
being a singular matrix is
 which is
satisfied if and only if .
\end{proof}

\renewcommand{\proofname}{Proof of Theorem \ref{thm:stable}}
\begin{proof}
Consider the positive semidefinite function

and its partial derivative with respect to 

The partial derivative  is calculated by first solving

  and 
which then yield


Hence the derivative of  with respect to  is



 When

the derivative of  in \eqref{eq:mydvdt_temp} is
negatively semi-definite with the expression

Indeed,  is a valid Lyapunov function candidate.


For the autonomous system when the derivative of the candidate
Lyapunov function is negative semi-definite, the asymptotic
stability is concluded based on the powerful invariant set theory.

For the negative semi-definite function , there is an
invariant set

The set of points in  where 
satisfies both conditions

and

where condition \eqref{eq:invariantsetCondition_1} is always true
under Assumption \ref{assumption_generic}.

When the three agents are connected over the graph shown in Fig.
\ref{fig:three_topo}, matrix  has rank  for all
.

In order to find the largest invariant set, the singularity of
matrix  is crucial to the asymptotic stability. As
pointed out by Lemma \ref{lemma:M_nonsigular},  is a
full rank matrix when . Thus equation
\eqref{eq:temp3} is satisfied if and only if there exist
 such that
. Thus

where  is the largest
invariant set for dynamic system \eqref{eq:mydes_M}.





If we let  being the smallest eigenvalue of 
during the entire convergence, the derivative of  is
bounded by 
which further yields . According to some trivial calculations, we conclude
that  exponentially
converges to zero.
\end{proof}
\renewcommand{\proofname}{PROOF}

According to , once the formation system forms a
geometry in , it stays there from then on.

Equation \eqref{eq:mydynS} is a nonlinear map  where given an
assigned shape , the
 initial geometry   in the nonlinear control law
\eqref{eq:mydynS} uniquely determines the stable scale
.  However, due to the nonlinearity,  it is difficult
to predict the final stable value .


On the other hand, although the exact value of  is
what we are seeking for, we would like to discuss the
controllability of the algorithm when the scale is expected to
converge to some arbitrary fixed value. In such a case, the problem
is recast into a general formation control problem with a specified
desired geometry, which is explored in quite a few works, e.g.,
\cite{Belta04optimal}. However, we believe it is still necessary to
make this extension so as to broaden the applications of the
algorithm. This problem is covered in Subsection \ref{sec:multiple}
after we extend the nonlinear control law to multiple() agents
case.

\subsection{Formation with multiple agents}
Some of related literatures concerning formation control were
restricted to three agents\cite{Basiri10angleonly,Bhatt09geometric}
and left the multiagents case as the future work. Extending
algorithms into multiple agents case would have to deal with the
complexity issue where the primary one is the selection of the
underlying graph. A graph with three nodes connected with each other
is a quite special case: it is a complete graph, a minimally rigid
graph and a ring graph. During the exposition of Theorem
\ref{thm:stable}, based on the cosine law, this particular property
of the three nodes ensures the compact form of the partial
derivative of the value function . However, when the number of
agents exceeds three, the cosine law dose not always applies.


In order to obtain the nonlinear control law as in three agents
formations, an intuitive idea is to  adopt triangle as the basic
unit of the underlying graph. We define the \emph{triangular
complement} of a graph which would lead to some interesting and
convenient results.
\begin{defn}
  The \emph{triangular complement} of a graph  is a graph 
  with the same
  vertex set as  and node  are connected to node  in  if and only if
  the minimal distance between  in  are two.
   The graph sum 
  has vertex set  and edges .
\end{defn}
\begin{rm}
  When the triangular complement graph is added to the minimally rigid
graph, its edges are labeled from  to .
\end{rm}
\begin{cor}
  The graph sum  equals to
  .
\end{cor}
For the definition of , please refer to, for example,
\cite{Diestel10,Anderson09easilyLocalizable}.

 The
triangular complement graph  contains the third edges that
constitute triangles with the neighboring edges in , as shown in
 Fig. \ref{fig:minimalandComplement_a} and Fig.
\ref{fig:minimalandComplement_b} where the solid lines are edges in
 and the dashed lines belong to . However, the neighboring
edges from  and  respectively do not
  necessarily have a third edge in  that form a
  triangle, as shown in Fig. \ref{fig:minimalandComplement_c}.




It is not necessarily true that , or equivalently
, being a complete graph even when  is minimally rigid or
globally rigid, as for instance the situation in Fig.
\ref{fig:minimalandComplement_d} where the length between node 4 and
node 1 is greater than 2. That is the reason we name it triangular
complement graph rather than the general complement graph as in
\cite{Diestel10}.


\begin{figure}
 \subfigure[]{\label{fig:minimalandComplement_a}
\begin{minipage}[b]{0.23\linewidth}
\centering
\includegraphics[scale=0.5]{minimalandComplement_a.eps}
\end{minipage}}
\subfigure[]{\label{fig:minimalandComplement_b}
\begin{minipage}[b]{0.23\linewidth}
\centering
\includegraphics[scale=0.5]{minimalandComplement_b.eps}
\end{minipage}}
\subfigure[]{\label{fig:minimalandComplement_c}
\begin{minipage}[b]{0.23\linewidth}
\centering
\includegraphics[scale=0.5]{connectedandComplement_c.eps}
\end{minipage}}
\subfigure[]{\label{fig:minimalandComplement_d}
\begin{minipage}[b]{0.23\linewidth}
\centering
\includegraphics[scale=0.5]{rigid_partialcomplete.eps}
\end{minipage}}
\caption{Graphs and their triangular complement graphs shown with
dashed lines.  and  are minimally rigid graphs,  is
only a connected graph and  is a globally rigid
graph.}\label{fig:example_triangular}
\end{figure}


The three edges ,  and  in  that
form a triangle is denoted by . We have made
the assumption in the preliminary  that each agent has the knowledge
of the coordinates of its neighbors, thus the agent that is adjacent
to  and  in  is able to calculate the coordinates of
the edges  in , i.e.,  or
 for .


For multiple agents case, the parallel form of matrix  in
\eqref{eq:D_three} is denoted by   with entries

If we define an expended shape vector  and similarly
,
 where ,
the norm of edges in , is calculated by the corresponding agent
using local coordinates of the other two edges in , and
consequently, similar to \eqref{eq:mydynS} the trajectory of the
scale function in terms of  is

and


where .
This further yields

where  is still the oriented incidence matrix of .

The derivative of the Lyapunov function \eqref{eq:laypu_three} is
then


\begin{thm}\label{thm:stable_multi}
The formation system

over a minimally rigid graph is exponentially stable and converges
to the largest invariant set  in
\eqref{eq:invariantSet_2} if  and 
are given in \eqref{eq:dynamicS_multiple} and \eqref{eq:M_multiple}
respectively.
\end{thm}
\begin{proof}
The first part of the stability proof is identical to that of three
agents situation. When , for a minimally rigid graph we
have the similar results that 
if and only if , and it is also trivial that
 is of rank .
Following the proof of Theorem \ref{thm:stable} we conclude that the
largest invariant set is  in
\eqref{eq:invariantSet_2}, which indicates that the formation system
converges to the desired shape at some scale  for a
sufficiently small .

The proof of exponentially convergence is consistent to that of
three agents case and is omitted here.
\end{proof}
\begin{rem}
When we considers only three agents in a formation system, the
underlying graph is a globally rigid one and a minimally rigid one
simultaneously, thus the requirements on the initial realization is
relaxed. However for system with up to four agents, this privilege
dose not holds true. Thus it is required that the initial geometry
, and Remark \ref{rem:initial} applies.
\end{rem}
\begin{rem}
  If we look at  in the control law,
  we require only a local
  coordinate for each node rather than a global one.
  The reason is that in the control law, we only concern with the
  relative distance between two agents, as represented by vector
  . On the other hand, in order to calculate ,
  local coordinates of two neighboring edges would be enough as well.
  However, for the rigidity matrix ,
  it requires the relative coordinates in a common reference frame. So for convenience,
  we assign a global coordinate for the formation system.
\end{rem}

With the time-varying scale function , the desired
geometry  is adjusted according to the current
geometry. Equation \eqref{eq:dynamicS_multiple} is a
nonlinear mapping  where the
 initial condition   uniquely determines the stable scale
 for the prescribed shape . The scale function
 would lead to a bounded scale and ensure a relatively
small cost value. This performance is observed in the experiments in
Section \ref{sec:exp}.

\subsection{Controllability of the multi-agent formation
system}\label{sec:multiple} The aforementioned nonlinear functions
 and  put us in a blind position about the
exact side lengths when agents get stabilized, and thus restricts
the applications of the algorithms. For instance, in the task of
payload transport, the desired geometry is constrained within a
certain area determined by the  cargo. Formation that shrinks to
some geometry with a relatively small scale  around the center of
gravity may result in a sensitive balance system and a sparse
formation where agents locate at the edge of the cargo may easily
failed.

In order to drive the formation system to converge to an assigned
geometry that belongs to , we add an additional
control gain  to the nonlinear control law

where ,  and . The edge
and the state of each agent that evolve along \eqref{eq:mydzs_2} are
denoted by  and  respectively.

The weight factor  adjusts the convergence rate of each edge
and thus tunes the stable value  and
, and we will prove that  the
formation system under control law \eqref{eq:mydzs_2} is
controllable.
\begin{thm}
  Given a desired geometry  in , there exists  satisfying
  in \eqref{eq:mydzs_2} such that a formation
  system over a minimally rigid underlying graph exponentially converges to  given that
  .
  \end{thm}

\begin{proof}
    We prove the theorem by finding the appropriate  with respect to 
  such that .
The procedure could be carried out by taking the original system
\eqref{eq:mydes_M} as a reference system.

Assume  and
 where , and
. Let
.

  Apparently, . Integrating on each
  side of the equation yields
  
  The condition that  is equivalent
  to, for each edge vector ,
  
  with vector .
Substituting \eqref{eq:epieandz} into \eqref{eq:mytemp1} we
  obtain
  
  where  is a block diagonal matrix with  identical diagonals .
In order to analysis the solution of , \eqref{eq:myfindL-1}
  could be rewritten into a standard linear matrix equation  with respect to
  

where .

Noting that for a minimally rigid graph, , i.e. . On the other
hand, . We can always find a set of positive
numbers  in \eqref{eq:myfindPi-final} by choosing appropriate
 such that the number of the nonzero entries of the
right hand side vector is less than .




With the positive control gain , the stability of the formation
system under the advanced control law \eqref{eq:mydzs_2} could be
proved along the same line as Theorem \ref{thm:stable_multi}, which
finishes the proof.
\end{proof}
\begin{rem}
  In order to ensure the convergence to the desired shape,
  restrictions on the initial conditions, i.e., Remark
  \ref{rem:initial} remain applies.
\end{rem}


Recall Theorem \ref{thm:JvJc}, under the endpoint constraints
  , the control law  still inherits
the suboptimality of  with respect to
.



The existence of the positive gain  could also be validated by
the geometric representation of \eqref{eq:myfindPi-final}. Condition
\eqref{eq:myfindPi-final} is satisfied if and only if for all
,

where . Equation
\eqref{eq:controllability_geometry} determines the adjacency of
vectors in Fig. \ref{fig:controllability}. According to the triangle
inequality,
 where
, ,
,  and
. By positioning vector
 on the upper right side of , it is
always guaranteed that .

\begin{figure}
\begin{center}
\includegraphics[scale=0.8]{controllability.eps}
\caption{Geometric analysis of the control gain }
\label{fig:controllability}
\end{center}
\end{figure}

The discussion on the controllability allows the algorithms to be
applied to situation that requires the switching of the geometry's
scale during the mission. For example when underwater vehicles are
mapping out oceanbed\cite{Kalantar08scale}, the scale of the
formation is adjusted online to decrease the interpolation error.

As stated before, this special problem we considered above falls
into to the general
 formation control category where the optimization is carried out under the constraint of a fixed
 geometry. However those existing results mainly focus purely on
 minimal energy expenditure\cite{Belta04optimal}
  rather than the geometrical performance during the process. In our research the cost
  function is carefully designed so as to have the group of agents
  exhibits prominent geometrical performance during convergence.

  In reality, the final scale  should be restrained within an interval. Generally, the
  upper bound could be the communication
  range of agents, and for lower bound, the collisions of agents should be taken into
  account.

In this section, motivated by the fact that a scale being adjusted
online may result in better cooperative performance, we found a
time-varying scale function in terms of the edge vector 
that further reduces the global minimum of system with a constant
scale. By introducing the triangular complement graph, we derived a
compact form of the nonlinear control law for multiple agents. With
the additional control gain, the nonlinear control law is also
applicative to general formation control problems with fixed final
geometry.

\section{Geometrical performance in sensor-target localizations}\label{sec:localization}
 The localization of target in sensor networks is always carried out by a fleet of
 three
UAVs in triangle. Each of the UAVs is equipped with sensors that
gather information of the target based on bearing
measurements\cite{Dovganccay05bearing}, range
measurements\cite{Martinez06rangelocalization} or scan
measurements\cite{Dogancay07scan}. A lot of literatures are devoted
to the exploit of the sensors positioning strategy as it is strongly
related to the localization performance. Based on Fisher information
matrix, the pioneering work demonstrated in \cite{Bishop10sensor}
proved that sensors in a equilateral triangle provide the minimal
variance estimation of the target.

Consider the special scenario where sensors (or UAVs that are
equipped with sensors) are deployed on a circle of radius  around
the target. When  is sufficiently large such that the distances
from the target to the three sensors could be considered as being
constant during the entire formation attainment process, one of the
main results in \cite{Bishop10sensor} could be restated specifically
as
\begin{lem}
  Consider angle-based bearing-only localization with three sensors. Assume the distance between sensor  and the
  target, denoted by , satisfies .
  Then the variance estimation of the target position is minimized
  if the triangle formed by the three sensors satisfies
  
  where   is the angle subtended at the target by two sensors  and
  . Moreover, reflecting a sensor about the target position does not
affects the optimality of the geometry.
\end{lem}
When  the distances between sensors and the target are congruent,
the optimal geometries generated from \eqref{eq:localization_shape}
have only angles constrains and are divided into two subsets where
geometries in each of them are similar ones. Examples of the optimal
geometries in the two subsets are shown in Fig.
\ref{fig:optimal_triangles_equa} and Fig.
\ref{fig:optimal_triangles_isos} respectively. For situations in
Fig. \ref{fig:optimal_triangles_equa}, any equilateral triangles
with the target locate at the geometry center are considered to be
the optimal deployment of sensors while irrespective of their exact
sizes. Thus here we focus on this optimal case to exploit the
relationship between the variance estimation of the target and the
resemblance of a geometry with respect to the equilateral triangle.

Consider a target that lies close to an equilateral triangle whose
angles are denoted by ,  and
 respectively, as shown in Fig.
\ref{fig:optimal_triangles_similar}. Without lose of generality,
assume . Then for simplicity, the
difference between a shape with respect to the
equilateral triangle could be measured by

A geometry with a larger  has a higher degree of similarity
to a equilateral triangle.

\begin{figure}
 \subfigure[Equilateral]{\label{fig:optimal_triangles_equa}
\begin{minipage}[b]{0.31\linewidth}
\centering
\includegraphics[scale=0.3]{geometry_triangle1.eps}
\end{minipage}}
\subfigure[Isosceles]{\label{fig:optimal_triangles_isos}
\begin{minipage}[b]{0.31\linewidth}
\centering
\includegraphics[scale=0.3]{geometry_triangle2.eps}
\end{minipage}}
\subfigure[Scalene]{\label{fig:optimal_triangles_similar}
\begin{minipage}[b]{0.31\linewidth}
\centering
\includegraphics[scale=0.3]{geometry_off.eps}
\end{minipage}}
\caption{Optimal triangles for bearing-only sensor-target
localization and a normal triangle as shown in color
blue}\label{fig:optimal_triangles}
\end{figure}



The variance estimation of the  target position measured by the
sensors are evaluated by the determinant of the Fisher information
matrix . Under constraints \eqref{eq:localization_shape},
the determinant of  achieves its maximum
of\cite{Bishop10sensor}

where  is the congruent error variance of sensors.

For a scalene triangle with angles , the difference
between the determinant values of  and
 is


The following theorem shows how  in \eqref{eq:similarity}
affects the localization accuracy.
\begin{thm}\label{thm:estimation_opt}
The following two conditions are true if
.
\begin{itemize}
\item[i).] when  is constant, 
increases as  increases;
\item[ii).] when  is constant,  increases as
 increases.
\end{itemize}

\end{thm}
\begin{proof}

Similarly,

The sufficient conditions  for  and  are
 when
,
 when
, and  when
. Thus irrespective of the relationship
between  and ,  and  are
sufficiently true if

and

or equivalently,

Thus under the assumption that
, inequality

is the sufficient condition to  and .
\end{proof}
It can be proved with some trivial calculation that the DOS
defined in \eqref{eq:distance} is positively monotonic to ,
that is
\begin{cor}\label{cor:similarity}
  In bearing-only sensor-target localization with three sensors, assume the smallest angle that is
  subtended by two sensors
  at the target is greater than .  Then the three
  sensors provides better estimation of the target location if the DOS of the triangle to the
   equilateral triangle is higher.
\end{cor}

Generally in UAV localizations, the three UAVs are set off from
certain locations and are commanded to fly towards the destination
separately. Localization of the target is carried out after they
attain the desired geometry at the destination. However, instead of
relying completely on the final attainment of the formation,
according to Theorem \ref{thm:estimation_opt}, we suggest that if we
can take the formation performance  into
account \emph{during} attainment, the geometries that are resemble to
the optimal one can provide pre-measurements of the target location
before arriving at the destination. Strategies could be made based
on those forecasted data and the executing time could be cut down if
the localization is not expected to be highly accurate.

Simulations are given at the end of Section \ref{sec:exp} to show
the improvement of the localization capability when the control law
\eqref{eq:u_varying} with  is
applied.

\section{Examples and simulations}\label{sec:exp}
In order to validate the effectiveness and the system performance of
the proposed nonlinear control laws and the superiority of the
time-varying scale function, we compare system performance under the
control laws introduced in Section \ref{sec:inva} and Section
\ref{sec:varying}.

For multiple agents, the underlying  minimally rigid graph is not
unique, which inspires us to explore how the topology affects the
system performance. Recall that a triangular complement graph is
defined when applying the control law to multiple agents, we suggest
to categorize the minimally rigid graphs by the number of edges in
their
 triangular complement graphs, and then analysis their corresponding performance.

Apart from comparing the cost values, we also observe the maximal
and total distance agents travel. These might be important in
certain scenarios although they are not our primary concerns in this
research. However, this observation could be a starting point for
the future work.

The cooperative performance and its relationship to the cost
function is illustrated in an intuitive way  to further demonstrate
the feature of the cost function.

Conclusions and conjectures are drawn from the results on how the
underlying topology and the initial realizations are related to the
cooperative performance.





\subsection{The scales of geometries and the performance}
We show the energy saving of a formation system under the optimal
invariant scale  and  compare it with other invariant scales.
Consider a formation system consists of four agents that are
initialized at  over a minimally
rigid graph  in Fig. \ref{fig:four_topo_a}. The desired shape
corresponds to  is measured by . The
time-invariant scales are scanned from 0.1 to 0.9 at step length 0.1
and the cost value  at different  are shown in Fig.
\ref{fig:time_invariant_s}. The data when  and 0.9 are
recorded in Table \ref{table:invar}. According to Fig.
\ref{fig:time_invariant_s}, the optimal time-invariant scale is
 which is exactly the value calculated from
\eqref{eq:optStaticS}, and the optimal cost value is ,
as shown in Table \ref{table:invar}. The formations when  and
 are shown in Fig. \ref{fig:invariant_s_05} and Fig.
\ref{fig:invariant_s_09} respectively. This verifies our conclusion
that \eqref{eq:optStaticS} helps to pick up the optimal geometry in
 and the control law \eqref{eq:myde_eqS} drives
the formation system attain this geometry when initialized within a
qualified neighborhood.

Moreover, we compare the differences between the cost functions
 and . The data in Table \ref{table:invar} infers that
when converge to the same geometry, formation system with
time-varying scale  has less cost value than system
with constant scale, which is measured by 1.7402 and 2.1249
respectively. This result verifies the advantages that are brought
by  in the nonlinear formation system.

We further observe the distance each agent travels during the
process and record the data in the table. According to the data, the
total distance the four agents travel has a relatively smaller value
of  when the scale is a time-varying function. On the other
hand, if we look at the maximal distance of the four agents travel
at different scales (as distinguished by underlines), agent 4 has
the smallest traveling distance of 0.6003 when .
\begin{figure}
 \subfigure[Graph ]{\label{fig:four_topo_a}
\begin{minipage}[b]{0.45\linewidth}
\centering
\includegraphics[scale=0.4]{four_topo_a.eps}
\end{minipage}}
\subfigure[Graph ]{\label{fig:four_topo_b}
\begin{minipage}[b]{0.45\linewidth}
\centering
\includegraphics[scale=0.4]{four_topo_b.eps}
\end{minipage}}
\caption{Two different graphs with the same number of triangular
complement edges}\label{fig:four_topo}
\end{figure}


\begin{figure}
 \subfigure[The cost value ]{\label{fig:time_invariant_s}
\begin{minipage}[b]{0.35\linewidth}
\centering
\includegraphics[width=2.8cm,height=2cm]{time_invariant_s.eps}
\end{minipage}}
\subfigure[]{\label{fig:invariant_s_05}
\begin{minipage}[b]{0.28\linewidth}
\centering
\includegraphics[width=2.8cm,height=2cm]{s05.eps}
\end{minipage}}
\subfigure[]{\label{fig:invariant_s_09}
\begin{minipage}[b]{0.28\linewidth}
\centering
\includegraphics[width=2.8cm,height=2cm]{s09.eps}
\end{minipage}}
\caption{Formation system under time-invariant
scale}\label{fig:time_invariant}
\end{figure}
\begin{table}
  \caption{Time-varying VS. Time-invariant scale under }\label{table:invar}
  \begin{center}
\newsavebox{\tablebox}
\begin{lrbox}{\tablebox}
\begin{tabular}{|l|l|l|l|l|}
\hline
\multicolumn{5}{|c|}{Initial Conditions   } \\
\hline
 &  &   &  : Length of each route & \\
\hline
\multirow{4}{*}{Invariant scale} &0.1 &5.5227 & [ 0.5167    0.7831    0.4686    ]&  2.6002\\
 &0.3 & 2.8944 & [0.4306  0.5534  0.3297  ] &   1.9837\\
 &0.9 & 5.1248 &[0.9559    0.6111        0.6331]& 3.1347\\
 &\textbf{\textbf{0.5 }} &  \textbf{2.1249 } &[0.5775  0.4608  0.4933  ] &2.1319\\
 \hline
Varying scale
 &\textbf{0.3}&\textbf{1.7402} & [0.3898 0.5190 0.2814 ] & 1.8422\\
\hline
\end{tabular}
\end{lrbox}
\scalebox{0.7}{\usebox{\tablebox}}
\end{center}
\end{table}





\subsection{The Topology, the initial geometry and the cooperative performance}
In the category of minimally rigid graphs, according to their
triangular complement graphs, they could be divided into different
subsets where graphs  that belong to the same subset have the
same number of edges in , i.e.,  of  are identical.
In this subsection, parallel experiments are carried out over graphs
that are within one subset and between different subsets
respectively.

For comparison, the four agents in the previous experiment are
assigned with another minimally rigid graph shown in Fig.
\ref{fig:four_topo_b}, which also has one edge in its triangular
complement graph. The initial condition and the desired shape are
supposed to be the same as that of . Note that for different
underlying graphs, the shape vector  differs as well. When
applying the nonlinear control law \eqref{eq:myopt_multi} on systems
interconnected over , the scale converges to 0.3563. The cost
value of the system over  is 4.2120, compared to that of 1.7402
for . The data is recorded in Table \ref{table:four}. Fig.
\ref{fig:four_formation} demonstrates the geometry of the agents
system and the trajectory of the scale . The circles
represent their initial positions
 and the bold blue lines are their final realization.

Apart from the cost value, Table \ref{table:four} also shows that
the total distance all agents travel is  for graph ,
which is larger than that of .

Topology affects the system performance in terms of cost value and
the sum of lengths each node travel. According to the experimental
results,  should be considered as being better than  both
in cooperative performance and the total traveling distance.
Although the sub-optimization problem we discussed throughout the
paper depends highly on the initial conditions, the fact that 
outperforms  stays true for all cases we tested with a large
number of different initial realizations.

We further investigate another situation where  two underlying
graphs have different numbers of triangular complement edges. We
consider six agents over four different minimally rigid graphs, as
shown in Fig. \ref{fig:six_topo}.  The trajectories of the formation
system are shown in Fig. \ref{fig:six_formation} with experimental
results recorded in Table \ref{table:six}. The curves for the
time-varying scale  over the four graphs are
demonstrated in Fig. \ref{fig:valueofS}.




A significant difference between  and the other three graphs is
that  only requires 5 additional edges for its triangular
complement graph while the number is 6 for the other three. This
interesting property explains the differences of he cost values and
total traveling distances in Table \ref{table:six}. Even with
different initial realizations  and 
that differs on the position of
 agent 4, the cost  value under  is consistently smaller than that of the other three graphs, as
shown in the upper part and the bottom part of the table
respectively.

The data corresponding to the four graphs infer the possibility that
the system whose underlying graph having a smaller number of
triangular complement edges, such as , may exhibit a better
cooperative performance than the one with more triangular complement
edges, such as ,  and .

On the other hand, the maximal distance the six agents travel, as
highlighted with underlines in the table, also proves the
superiority of graph  over the others. The smallest maximal
distance agents travel is observed in , which helps to prevent
the overwhelming of a single agent. This is again only a conjecture
from the current experiment without theoretical proofs, which is an
ongoing work to this research.



\begin{table}
  \caption{Systems over different underlying graphs with identical }
  \label{table:four}
\begin{center}
\begin{lrbox}{\tablebox}
\begin{tabular}{|l|l|l|l|l|}
\hline \multicolumn{5}{|c|}{
} \\
\hline
&   &   &  : Length of each route & \\
\hline
 &0.3159 & 1.7402&  [0.3898  0.5190  0.2814  ]  &  1.8422\\
& 0.3563  &4.2120 & [0.3711  0.5013  0.3705  ]  &  1.8766\\
\hline
\end{tabular}
\end{lrbox}
\scalebox{0.7}{\usebox{\tablebox}}
\end{center}
\end{table}

\begin{figure}
 \subfigure[Trajectories]{\label{fig:four_trajec}
\begin{minipage}[b]{0.45\linewidth}
\centering
\includegraphics[scale=.3]{four_Ga.eps}
\end{minipage}}
 \subfigure[Time-varying scale ]{\label{fig:valueofS_four}
\begin{minipage}[b]{0.45\linewidth}
\centering
\includegraphics[scale=.3]{valueofS_four.eps}
\end{minipage}}
\caption{Four agents formations and the scale  over
}\label{fig:four_formation}
\end{figure}




 \begin{figure}
 \subfigure[, ]{\label{fig:six_topo_a}
\begin{minipage}[b]{0.23\linewidth}
\centering
\includegraphics[scale=0.18]{six_topo_a.eps}
\end{minipage}}
\subfigure[, ]{\label{fig:six_topo_b}
\begin{minipage}[b]{0.23\linewidth}
\centering
\includegraphics[scale=0.18]{six_topo_b.eps}
\end{minipage}}
\subfigure[, ]{\label{fig:six_topo_c}
\begin{minipage}[b]{0.23\linewidth}
\centering
\includegraphics[scale=0.18]{six_topo_c.eps}
\end{minipage}}
\subfigure[, ]{\label{fig:six_topo_d}
\begin{minipage}[b]{0.23\linewidth}
\centering
\includegraphics[scale=0.18]{six_topo_d.eps}
\end{minipage}}
\caption{Four different graphs with different number of triangular
complement edges. The bold lines indicate the  minimally rigid
graphs, and dashed lines are their triangular complement
edges}\label{fig:six_topo}
\end{figure}


\begin{figure}
 \subfigure[]{\label{fig:six_formation_a}
\begin{minipage}[b]{0.23\linewidth}
\centering
\includegraphics[scale=0.2]{Ga.eps}
\end{minipage}}
\subfigure[]{\label{fig:six_formation_b}
\begin{minipage}[b]{0.23\linewidth}
\centering
\includegraphics[scale=0.2]{Gb.eps}
\end{minipage}}
\subfigure[]{\label{fig:six_formation_c}
\begin{minipage}[b]{0.23\linewidth}
\centering
\includegraphics[scale=0.2]{Gc.eps}
\end{minipage}}
\subfigure[]{\label{fig:six_formation_d}
\begin{minipage}[b]{0.23\linewidth}
\centering
\includegraphics[scale=0.2]{Gd.eps}
\end{minipage}}
\caption{Trajectories of six agents interconnected over different
graphs}\label{fig:six_formation}
\end{figure}

\begin{figure}
\begin{center}
\includegraphics[scale=0.35]{valueofS.eps}
\caption{Scale  over the four
graphs}\label{fig:valueofS}
\end{center}
\end{figure}

\begin{table}
  \caption{Systems over different underlying graphs where  differs}
  \label{table:six}
 \begin{center}
\begin{lrbox}{\tablebox}
\begin{tabular}{|l|l|l|l|l|}
\hline \multicolumn{5}{|c|}{}\\
\multicolumn{5}{|c|}{} \\
\multicolumn{5}{|c|}{
}\\
\multicolumn{5}{|c|}{} \\
\hline
\multicolumn{5}{|c|}{;}\\
\hline
&   &   &  : Length of each route & \\
\hline
 &0.5873 & \textbf{0.6918} & [0.3143  0.2128  0.2214    0.4290  0.1411]& 1.8388\\
& 0.5652 & 0.7724  &[0.2805  0.1532  0.1737    0.4500  0.1488] &1.7351\\
 & 0.5369& 0.9152&  [0.2900  0.2767  0.1799    0.4139  0.2713]  &  1.9946\\
& 0.5589  &0.9169 & [0.2407  0.1185  0.2152    0.3916  0.1189]  &  1.6154\\
\hline \multicolumn{5}{|c|}{;}\\
\hline
&   &   &  : Length of each route & \\
\hline
 &0.6105 & \textbf{1.0284}  &[0.5100  0.2428  0.1906    0.3985 0.1467]& 2.2507\\
& 0.5838  &1.5533  &[0.6338  0.4744  0.5584    0.3977 0.3489] &3.2086\\
& 0.5421  &1.7603 & [0.4980  0.5902  0.4058    0.4816 0.4097] &3.1918\\
& 0.5570  &2.4951&  [0.5947  0.2069  0.4649    0.3718 0.4263] &2.8788\\
\hline
\end{tabular}
\end{lrbox}
\scalebox{0.7}{\usebox{\tablebox}}
\end{center}
\end{table}


The DOSs of the four graphs , ,  and
 to the desired shape  are shown in Fig. \ref{fig:for_bias_six}. The solid line,
which corresponds to graph , has both a fast convergence speed
and stay in a high DOS during the process. Although system over
 achieved the desired shape as fast as the one over  did,
its geometric distance to the desired shape during convergence was much larger.
\begin{figure}
\begin{center}
\includegraphics[scale=.35]{distance_6.eps}
\caption{The inverse of the geometry's DOS  at time }
\label{fig:for_bias_six}
\end{center}
\end{figure}

\subsection{Geometries in sensor-target localization}
A triangle that has a higher DOS to the equilateral
one could provide better estimation of the target location, as
discussed in Section \ref{sec:localization}.

We consider three agents initialized at
. The desired shape is a equilateral
triangle, which is one of the optimal shapes in bearing-only
sensor-target localization. The specialty of equilateral triangle  For systems with the time-varying scale
function and with a constant scale respectively, the geometries'
DOS during the process are shown in Fig.
\ref{fig:localization_fe}.

The determinant of the Fisher information matrix during the process
is calculated in Fig. \ref{fig:localization_fish}. The system with a
constant scale has consistently smaller degree of
similarity to the equilateral triangle, and also smaller determinant
value, which indicates the variance estimation is relatively worse.
This validates the conclusion of Corollary \ref{cor:similarity}.
\begin{figure}
 \subfigure[The inverse of  the triangles DOS  to the equilateral one]{\label{fig:localization_fe}
\begin{minipage}[b]{0.45\linewidth}
\centering
\includegraphics[scale=.3]{angledifference_fish.eps}
\end{minipage}}
 \subfigure[Determinant of the Fisher information matrix]{\label{fig:localization_fish}
\begin{minipage}[b]{0.45\linewidth}
\centering
\includegraphics[scale=.3]{localization_fish.eps}
\end{minipage}}
\caption{Formation control in bearing-only sensor-target
localization}
\end{figure}



\section{Conclusions and Future Works}
For a formation system with a flexible scale but under shape
constraints, we discussed the strategies of choosing the desired
geometries' scale so as to achieve better cooperative performance.
In order to ensure the exponential stability of the system over a minimally rigid graph,
fixed-structured nonlinear control laws on the edges and the scale were considered, where two
types of scale design methods were proposed, namely  the
time-invariant scale and the time-varying scale function.  It was
proved that a system with the time-varying scale function could
further reduce the minimum of the cost value when the scale is
constant. By defining a triangular complement graph, the algorithm
also applied to the multiple agents case. The controllability of the
formation system was also discussed by adjusting the convergence
rate of each edge with respect to the reference system.

These results were validated on various simulations where we
compared the cost values of  systems with constant scale and the
time-varying scale function respectively.  The experimental results
also inferred the possibility that the underlying topology with a
smaller number of triangular complement edges may always cost less
during convergence and thus may have better cooperative performance.
Theoretical proof to this conjecture is still undergoing. Moreover,
we also applied the control laws in sensor-target localization to
show the prominent features of the nonlinear formation system we
designed.

The nonlinear dynamics \eqref{eq:myopt_multi} is a centralized
control law as  contains states information
of every agent in the system rather than local information.
Decentralized/distributed algorithms should be considered in the
future so as to meet the scalability requirements. Moreover, further
research should focus on eliminating the restriction on the
minimally rigid underlying graph.

\ack The authors would like to thank Professor Brian Anderson for the insightful comments and discussions, which have significantly improved the presentation of this work.

\bibliographystyle{wileyj}
\bibliography{E:/papers/odds/odds}



\end{document}
