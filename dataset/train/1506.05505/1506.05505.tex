\documentclass{article}
\usepackage{graphicx,amssymb,amsmath,amsthm}
\usepackage[usenames,dvipsnames]{xcolor}
\newcommand{\todo}[2][ToDo]{\textcolor{red}{*** \textsc{#1:} #2 ***}} \newcommand{\ruy}[2][says]{\textcolor{blue}{\textsc{Ruy #1:} #2}}

\newtheorem{theorem}{Theorem}[section]
\newtheorem{cor}[theorem]{Corollary}
\newtheorem{lemma}[theorem]{Lemma}
\newtheorem{prop}[theorem]{Proposition}
\newtheorem{conj}[theorem]{Conjecture}
\newtheorem{rem}[theorem]{Remark}
\newtheorem{problem}{Problem}

\newtheorem{defn}{Definition}

\newcommand{\MyQuote}[1]{\vspace{0.5cm}\parbox{10cm}{\em #1}\hspace*{2cm}()\
f(i)= & \begin{cases}
0 & \,\mbox{if}\, i=1.\\
2^{\frac{i(i-1)}{2}-1} & \,\mbox{if}\, i\geq2.
\end{cases}\\
g(i)= & \begin{cases}
0 & \,\mbox{if}\, i=1.\\
f(i)-f(i-1) & \,\mbox{if}\, i\geq2.
\end{cases}

 &  & \hspace{-2em}\left|\begin{array}{ccc}
n-6 & f(k-1) & 1\\
n-4 & 0 & 1\\
1 & g(k) & 1
\end{array}\right|\\
 & = & 2g(k)-(n-5)f(k-1)\\
 & = & 2f(k)-(n-3)f(k-1)\\
 & = & 2f(k)-2^kf(k-1)+3f(k-1)\\
 & = & 3f(k-1)\\
 & > & 0.

 &  & \hspace{-2em}\left|\begin{array}{ccc}
0 & 0 & 1\\
2 & f(k-1) & 1\\
n-3 & g(k) & 1
\end{array}\right|\\
 & = & 2g(k)-(n-3)f(k-1)\\
 & = & 2f(k)-(n-1)f(k-1)\\
 & = & 2f(k)-2^kf(k-1)+f(k-1)\\
 & = & f(k-1)\\
 & > & 0.
 \sum_{i=1}^{k} g(i)=f(k)=2^{\frac{k(k-1)}{2}-1}=\frac{1}{2} n^{\frac{1}{2} \log (n/2)},-\Delta_y/\Delta_x=-\left ( \frac{d_1}{d_1+d_2}\operatorname{girth}_{4}(Q) \right )/\Delta_x=\frac{d_1}{(d_1+d_2)d_2}2^{l-t-1}\operatorname{girth}_{4}(Q).\Delta_{2^{t-1}-1}'\ge  2^{k-t-1}\cdot 2^{2^{t-1}-2}\ge 2^{\frac{1}{2}k^2+k-t-3} \ge 2^{\frac{1}{2}k^2+k-2\log(k)-4} \ge n^{\frac{1}{2}\log n}. D< n^{\frac{1}{8}\log n}; \label{eq:w>=1}
 \max\{\operatorname{width}_1(Q),\operatorname{width}_4(Q)\} \ge 1/D.
\label{eq:s(r)}
\max\{\operatorname{width}_1(S(R)),\operatorname{width}_4(S(R))\} \ge \frac{2^{(l-t-6)(l-t-7)/2}}{D}.
\max\{\operatorname{width}_1(S(Q)),\operatorname{width}_4(S(Q))\} \ge 1/D=\frac{2^{((t+6)-t-6)((t+6)-t-7)/2}}{D}.
  \operatorname{girth}_{1}(Q_m') & \ge \frac{2^{(m-t-6)(m-t-7)/2}}{D} \, \mbox{if} \, m \equiv l \mod 2, \label{eq:g1}\\ 
  \operatorname{girth}_{4}(Q_m') & \ge \frac{2^{(m-t-6)(m-t-7)/2}}{D} \, \mbox{if} \, m \not \equiv l \mod 2 \label{eq:g4}

 \operatorname{girth}_{1}(Q_m') &\ge  \left( \frac{(d_1)^2}{(d_1+d_2)d_2} \right ) 2^{m-t-2} 
 \operatorname{girth}_4(Q_{m-1}')-\operatorname{width}_1(S(Q_{m-1}'))\\
&\ge 2^{m-t-5} \operatorname{girth}_4(Q_{m-1}')-\frac{2^{(m-t-7)(m-t-8)/2}}{D}\\
&\ge 2^{m-t-5}\frac{2^{(m-t-7)(m-t-8)/2}}{D}-\frac{2^{(m-t-7)(m-t-8)/2}}{D}\\
&\ge 2^{m-t-6}\frac{2^{(m-t-7)(m-t-8)/2}}{D} \\
&=\frac{2^{(m-t-6)(m-t-7)/2}}{D}


Therefore  has size at least . This at least
, for a sufficiently large value of . Since ,
the result follows.
The proof when  has different parity as  is similar, but uses inequality (2) of  Lemma~\ref{lem:lower_bound}
instead.
\end{proof}


To prove the general lower bound we do the following. Take a drawing of the Horton set;
find a subset of half of its points, for which we know that there exists
a linear transformation that maps it into an isothetic drawing; afterwards, apply
Lemma~\ref{thm:lower_isothetic} to the image and use the obtained lower bound to lower bound
the size of original drawing.

\begin{theorem}\label{thm:lower_gen}
Every drawing of the Horton set of  points has size at least ,
for a sufficiently large value of  and some positive constant .
\end{theorem}
\begin{proof}
Let  be a (not necessarily isothetic) drawing of the Horton set of  points. 
  As  and  have the same order type we can label  with the
 same labels as , such that corresponding triples of points in  and 
 have the same orientation. Let  be 
 with these labels.
 
 Note that the clockwise order by angle of  around  is , 
and that   lies in an unbounded cell of the line arrangement of the lines defined
by every pair of points of ; thus, point   can be moved towards infinity without changing
 this radial order around . Therefore, there is a direction 
 in which if  is projected orthogonally the order of the projection
 is precisely . We may rotate  as long as it does not coincide
 with a direction defined by a pair of points of  and the order of 
in this projection does not change. Let  and  be the first vectors,
defined by pairs of points of , encountered when rotating  to the left
and to the right, respectively; let .

We may assume that ; otherwise
one of  and  has length at least ,
and therefore a coordinate of value at least .
Let . Consider a change of basis from the standard
basis to . Note that under this transformation  is mapped
to . We multiply the image
of  under this mapping by , to obtain an isothetic drawing of the Horton
set on  points. By Theorem~\ref{thm:lower_isothetic}, this drawing has size at least
. Therefore,  has size at least 
.
\end{proof}

We point out that the constants in the exponent of the lower bounds of Theorems
\ref{thm:lower_isothetic} and \ref{thm:lower_gen} can be improved. We 
simplified the exposition at the expense of these worse bounds.

\textbf{Acknowledgments.}
We thank Dolores Lara, Gustavo Sandoval and  Andr\'es Tellez for various
helpful discussions.


\small
\bibliographystyle{abbrv} \bibliography{horton}








\end{document}
