\documentclass[11pt]{article}
\usepackage{fullpage}
\usepackage{graphicx}
\usepackage{amsmath}
\usepackage{amssymb}
\usepackage{amsthm}
\usepackage{enumerate}
\usepackage[pdfpagelabels, pagebackref, naturalnames]{hyperref}
\usepackage[font=small, format=hang, labelfont=bf]{caption}
\usepackage[labelfont=default]{subcaption}
\usepackage{xspace}
\usepackage{todonotes}


\newcommand{\conv}{\ensuremath{\mathrm{conv}}}
\newcommand{\Real}{\ensuremath{\mathbb{R}}}
\newcommand{\Plane}{\ensuremath{\mathbb{R}^2}}
\newcommand{\intC}{\ensuremath{\mathcal{C}}}
\newcommand{\bd}{\ensuremath{\partial}}
\newcommand{\cl}{\ensuremath{\mathrm{cl}}}
\newcommand{\intr}{\ensuremath{\mathrm{int}}}
\newcommand{\rint}{\ensuremath{\mathrm{ri}}}
\newcommand{\seg}{\overline}
\newcommand{\cut}{\ensuremath{\mathrm{cut}}}
\newcommand{\poc}{\ensuremath{\mathrm{poc}}}
\newcommand{\A}{\mathcal{A}}
\newcommand{\Kernel}{\mathcal{K}}
\newcommand{\IKernel}{\mathcal{I}}
\newcommand{\RKernel}{\mathcal{R}}
\newcommand{\HG}{\mathcal{H}}
\newcommand{\VG}{\mathcal{V}}


\newtheorem{lemma}{Lemma}
\newtheorem{theorem}{Theorem}
\newtheorem{proposition}{Proposition}
\newtheorem{corollary}{Corollary}
\newtheorem{observation}{Observation}
\newtheorem{claim}{Claim}
\theoremstyle{definition}
\newtheorem{definition}{Definition}

\graphicspath{{./}}



\newcommand{\complain}[1]{\marginpar{\framebox{~~~}}{\color{red}\textit{#1}}}
\newcommand{\changed}[1]{\marginpar{\framebox{~~~}}{\color{blue}{#1}}}

\newcommand\Tstrut{\rule{0pt}{2.6ex}}       \newcommand\Bstrut{\rule[-1.1ex]{0pt}{0pt}} 

\let\geq\geqslant
\let\leq\leqslant


\def\denseitems{
    \itemsep1pt plus1pt minus1pt
    \parsep0pt plus0pt
    \parskip0pt\topsep0pt}


\title{Improved Bounds for Beacon-Based Coverage and Routing in Simple Rectilinear Polygons
\thanks{Work by S.W. Bae was supported by Basic Science Research Program through the National
Research Foundation of Korea (NRF) funded by the Ministry of Science, ICT \& Future Planning
(2013R1A1A1A05006927).
Work by C.-S. Shin was supported by Research Grant of Hankuk University of Foreign Studies.
}
}

\author{Sang Won Bae\footnote{Department of Computer Science, Kyonggi University, Suwon, Korea.
Email: \texttt{swbae@kgu.ac.kr}
}
\and Chan-Su Shin\footnote{Division of Electronic Systems \& Computer Engineering,
Hankuk University of Foreign
Studies, Yongin, Korea.
Email: \texttt{cssin@hufs.ac.kr, chansu@gmail.com}.
}
\and Antoine Vigneron\footnote{Visual Computing Center,
King Abdullah University of Science and	Technology (KAUST), Thuwal 23955-6900, Saudi Arabia.
Email: \texttt{antoine.vigneron@kaust.edu.sa}
}
}



\date{\today
}

\begin{document}
\maketitle



\begin{abstract}
We establish tight bounds for beacon-based coverage problems, and improve the bounds for beacon-based routing problems in simple rectilinear polygons. 
Specifically, we show that  beacons are always sufficient and sometimes necessary to cover a simple rectilinear polygon  with  vertices. 
 We also prove tight bounds for the case where  is monotone,
and we present an optimal linear-time algorithm that computes the beacon-based kernel
of .
For the routing problem, we show that  beacons
are always sufficient, and  beacons are sometimes necessary to route between all pairs of points in .
\end{abstract}


\section{Introduction} \label{sec:intro}


A \emph{beacon} is a facility or a device that attracts objects within a given domain.
We assume that objects in the domain, such as mobile agents or robots, know the exact location
or the direction towards an activated beacon in the domain, even if it is not directly visible.
More precisely, given a polygonal domain , a beacon is placed at a fixed point in .
When a beacon  is activated, an object  moves along the ray
starting at   and towards the beacon  until it either  hits the boundary  of ,
or it reaches . (See Figure~\ref{fig:intro}a.)
If  hits an edge  of , then it continues to move along  in the direction such
that the Euclidean distance to  decreases.
When  reaches an endpoint of , it may move along the ray from the current
position of  towards , if possible, until it again hits the boundary  of .
So,  is pulled by  in a greedy way, so that the Euclidean distance to 
is monotonically decreasing, as an iron particle is pulled by a magnet.
There are two possible outcomes: Either  finally reaches , or it stops at a local minimum,
called a \emph{dead point}, where there is no direction in which, locally, the distance to  
strictly decreases. In the former case,  is said to be \emph{attracted} by the beacon . 


This model of beacon attraction was recently suggested by 
Biro~\cite{b-bbrg-13, bgikm-cccg-13, bikm-wads-13},
and extends the classical notion of visibility.
We consider two problems based on this model:
the \emph{coverage} and the \emph{routing} problem,
introduced by Biro~\cite{b-bbrg-13} and Biro et al.~\cite{bgikm-cccg-13,bikm-wads-13}.
In the beacon-based coverage problem, we need to place beacons in 
so that any point  is attracted by at least one of the beacons.
In this case, we say the set of beacons \emph{covers} or \emph{guards} .
In the beacon-based routing problem, we want to place beacons in 
so that every pair  of points can be routed:
We say that  is  \emph{routed} to  if there is a sequence of beacons in 
that can be activated and deactivated one at a time,
such that the source  is successively attracted by each of beacon of this sequence,
and finally reaches the target , which is regarded as a beacon.

In this paper, we are interested in combinatorial bounds
on the number of beacons required for coverage and routing,
in particular when the given domain  is a simple rectilinear polygon.
Our bounds are variations on visibility-based guarding results, such
as the well-known \emph{art gallery theorem}~\cite{c-ctpg-75} and
its relatives~\cite{o-apragt-83,ghks-ggprp-96, kkk-tgrfw-83, g-spragt-86, mp-agtgg-03}.
The beacon-based coverage problem is analogous to the art gallery problem,
while the beacon-based routing problem is analogous to the \emph{guarded guards} 
problem~\cite{mp-agtgg-03}, 
which asks for a set of point guards in  such that every point is visible from
at least one guard and every guard is visible from another guard.
For the art gallery problem, it is known that
 point guards are sufficient, and sometimes necessary,
to guard a simple polygon  with  vertices~\cite{c-ctpg-75}.
If  is rectilinear, then  are necessary and 
sufficient~\cite{kkk-tgrfw-83,o-agta-87, g-spragt-86}.
In the guarded guards problem, this number
becomes  for simple polygons
and  for simple rectilinear polygons~\cite{mp-agtgg-03}.
Other related results are mentioned in the book~\cite{o-agta-87} by O'Rourke
or the surveys by Shermer~\cite{s-rrag-92} and Urrutia~\cite{u-agip-00}.

Biro et al.~\cite{bgikm-cccg-13} initiated research on combinatorial bounds
for beacon-based coverage and routing problems,
with several nontrivial bounds for different types of domains such as
rectilinear or non rectilinear polygons, with or without holes.
When the domain  is a simple rectilinear polygon with  vertices, they showed that  beacons are sufficient to cover
any rectilinear polygon with  vertices,
while  beacons are necessary to cover
the same example in \figurename~\ref{fig:intro},
and conjectured that  would be the tight bound.
They also proved that  beacons are always sufficient for routing, and some domains, such as the domain depicted in \figurename~\ref{fig:intro}a, require  beacons.

\begin{table}[tbh]
\centering
\begin{tabular}{c|c|c|c|c}
 \hline
 & \multicolumn{2}{c|}{Lower bound} & \multicolumn{2}{c}{Upper bound}  \Tstrut\Bstrut\\
 \cline{2-5}
 & Best results & Our results & Best results & Our results
 \Tstrut\Bstrut\\ 
  \hline\hline
 Coverage &  \quad\; \cite{bgikm-cccg-13}&
  \qquad\, [Theorem~\ref{thm:covering}] &  \quad\enspace\;\; \cite{bgikm-cccg-13} &  \qquad\quad\enspace [Theorem~\ref{thm:covering}]
  \Tstrut\Bstrut\\ \hline
 Routing &   \enspace \cite{bgikm-cccg-13} &  \enspace [Theorem~\ref{thm:routing}] &   \enspace \cite{bgikm-cccg-13} &   \enspace [Theorem~\ref{thm:routing}]
  \Tstrut\Bstrut\\
 \hline
\end{tabular}
\caption{Best known results and our results on the number of beacons required for beacon-based coverage and routing in a simple rectilinear polygon  with  vertices. Our lower bound on the routing problem holds for any .}
\label{tbl:summary}
\end{table}

\paragraph{Our results.}

In this paper, we first prove tight bounds on beacon-based coverage problems for simple rectilinear polygons. (See \tablename~\ref{tbl:summary}.) 
In Section~\ref{sec:coverage},
we give a lower bound construction that requires  beacons, and then we present a method of placing the same number of beacons to cover , which matches the lower bound we have constructed.
These results settle the open questions on the beacon-based coverage problems for simple rectilinear polygons
posed by Biro et al.~\cite{bgikm-cccg-13}.
We also consider the case of monotone polygons:
For routing in a monotone rectilinear polygon, the same bound 
 holds,
while  beacons are always sufficient to cover a monotone rectilinear polygon.

We next improve the upper bound from  to  for
the routing problem in Section~\ref{sec:routing}. We also slightly improve the lower bound from  to  for any . 

We also present an optimal linear-time algorithm that computes the \emph{beacon kernel}
 of a simple rectilinear polygon  in Section~\ref{sec:kernel}.
The {beacon kernel}  of  is defined to be
the set of points  such that placing a single beacon at  is sufficient to completely cover .
Biro first presented an -time algorithm that computes the kernel 
of a simple polygon  in his thesis~\cite{b-bbrg-13},
and  Kouhestani et al.~\cite{krs-cccg-14} soon improved it to  time
with the observation that  has a linear complexity.
Our algorithm is based on a new, yet simple, characterization of the kernel .


\begin{figure}[tb]
\centering
\includegraphics[width=\textwidth]{fig_intro}
\caption{(a) A lower bound construction  by Biro et al.~\cite{bgikm-cccg-13}.
 A point  is attracted by a beacon  through the beacon attraction path
 depicted by the thick gray path, while  is not since it stops at the dead point .
 (b) Another rectilinear polygon . If one partitions  by the horizontal cut  (dashed segment) 
 at  into two subpolygons  and  and handle each separately,
 then it does not guarantee that  is guarded.
 In this case,  is attracted by  inside the subpolygon 
 while it is not the case in the whole domain .}
\label{fig:intro}
\end{figure}


\section{Preliminaries} \label{sec:pre}


A \emph{simple rectilinear polygon} is a simple polygon whose edges are either horizontal or vertical.
The internal angle at each vertex of a rectilinear polygon is always  or .
We call a vertex with internal angle  a \emph{convex vertex}, and a vertex with internal 
angle  is called a \emph{reflex vertex}.
For any simple rectilinear polygon , we let  be the number of its reflex vertices.
If  has  vertices in total, then , because the sum of the signed turning angles
along  is .
An edge of  between two  convex vertices is called a \emph{convex edge},
and an edge between two reflex vertices is called a \emph{reflex edge}.
Each convex or reflex edge  shall be called \emph{top}, \emph{bottom}, \emph{left} or \emph{right}
according to its orientation:
If  is horizontal and the two adjacent edges of  are downwards from ,
then  is a top convex or reflex edge.
(The edge  in \figurename~\ref{fig:intro}b is a top reflex edge.)
For each edge  of , we are often interested in the half-plane 
whose boundary supports  and whose interior includes the interior of  locally at .
We shall call  the \emph{half-plane supporting }.

A rectilinear polygon  is called \emph{-monotone} (or \emph{-monotone})
if any vertical (resp., horizontal) line intersects  in at most one connected component.
If  is both -monotone and -monotone, then  is said to be \emph{-monotone}.
From the definition of the monotonicity,
we observe the following.
\begin{observation} \label{obs:monotone}
A rectilinear polygon  is -monotone if and only if  has no vertical reflex edge.
Hence,  is -monotone if and only if  has no reflex edge.
\end{observation}

Our approach to attain  tight upper bounds relies on
partitioning a given rectilinear polygon  into subpolygons by cuts.
More precisely, a \emph{cut} in  is a chord\footnote{A \emph{chord}  of a polygon is a line segment between two points on the boundary
such that all points on  except the two endpoints lie in the interior of the polygon.}
of  that is horizontal or vertical.
There is a unique cut at a point  on the boundary  of 
unless  is a vertex of .
If  is a reflex vertex, then there are two cuts at ,
one of which is \emph{horizontal} and the other is \emph{vertical},
while there is no cut at  if  is a convex vertex.
Any horizontal cut  in  partitions  into two subpolygons:
one below , denoted by , and the other above  denoted by .
Analogously, for any vertical cut ,
let  and  denote the subpolygons to the left and to the right of , respectively.

For a beacon  and a point , the \emph{beacon attraction path} of  with respect to ,
or simply the \emph{-attraction path} of ,
is the piecewise linear path from  created by the attraction of 
as described in Section~\ref{sec:intro}.
(See \figurename~\ref{fig:intro}a.)
If the -attraction path of  reaches , then we say that  is \emph{attracted} to .
As was done for the classical visibility notion~\cite{o-apragt-83,ghks-ggprp-96},
a natural approach would find a partition of  into smaller subpolygons of similar size,
and handle them recursively.
However, we must be careful when choosing a partition of , because
an attraction path within a subpolygon may not be an attraction path
within . (See \figurename~\ref{fig:intro}b.)
So  is not necessarily guarded by the union of the guarding sets of the subpolygons.

Thus, when applying a cut in , we want to make sure that
beacon attraction paths in a subpolygon  of  do not \emph{hit} the new edge
of  produced by .
To be more precise, we say that an edge  of  is \emph{hit} by  with respect to 
if the -attraction path of  makes a bend along .
\begin{observation} \label{obs:convex_edge}
 Let  be a beacon in  and  be any point
 such that  is attracted by .
 If the -attraction path of  hits an edge  of ,
 then  and , where  denotes the half-plane supporting .
 Therefore, no beacon attraction path hits a convex edge of .
\end{observation}
Thus, if we choose a cut that becomes a convex edge on both sides,
then we will be able to handle each subpolygon separately.

In this paper, we make the  general position assumption 
that \textit{no cut in  connects two reflex vertices.}
This general position can be obtained by perturbing the reflex vertices of  locally,
and such a perturbation does not harm the upper bounds on our problems in general.
It will be discussed in the full version of the paper.





\section{The Beacon Kernel} \label{sec:kernel}


Before continuing to the beacon-based coverage problem,
we consider simple rectilinear polygons that can be covered by a single beacon.
This is related to the  \emph{beacon kernel}  of a simple polygon ,
defined to be the set of all points 
such that a beacon placed at  attracts all points in .
Specifically, we give a characterization of rectilinear polygons  such that .
Our characterization is  simple and constructive, resulting in a linear-time algorithm
that computes the beacon kernel  of any simple rectilinear polygon .

Let  be the set of reflex vertices of . Let  be any reflex vertex with
two incident edges  and . 
For ,
define  to be the closed half-plane whose boundary is the line orthogonal to  through 
and whose interior excludes .
Let .
Observe that  is a closed cone with apex .
Biro~\cite[Theorem 5.2.8]{b-bbrg-13} showed that the kernel  of 
is the set of points in  that lie in  for all reflex vertices :
\begin{lemma}[Biro~\cite{b-bbrg-13}] \label{lem:kernel_biro}
 For any simple polygon  with set  of reflex vertices, it holds that
 
 where  denotes the complement of . 
\end{lemma}

Note that Lemma~\ref{lem:kernel_biro} holds for any simple polygon .
We now assume that  is a simple rectilinear polygon.
Then, for any reflex vertex , the set  forms a closed cone with aperture angle 
whose boundary consists of two rays following the two edges incident to .
Let  be the set of reflex vertices incident to a reflex edge,
and let .
So a vertex in  is adjacent to at least one reflex vertex that also belongs to ,
and a vertex in  is always adjacent to two convex vertices.
We then observe the following.
\begin{lemma}
\label{lem:kernel_lem}
For any simple rectilinear polygon ,

\end{lemma}
\begin{proof}
For a contradiction, suppose that there exists a point  that is included in
 but avoids .
Then, there must exist a reflex vertex  such that ,
or equivalently, .
That is,  and  have a nonempty intersection.
Let  and  be the two vertices adjacent to  such that , , and  appear
on  in counterclockwise order.
Note that both  and  are convex since .


\begin{figure}[tb]
\centering
\includegraphics[width=0.7\textwidth]{fig_kernel}
\caption{Proof of Lemma~\ref{lem:kernel_lem}.}
\label{fig:kernel}
\end{figure}

Since  and  is an open set,
the boundary  of  crosses  at some points other than the two edges  and .
Let  be the first point in 
that we encounter when traveling along  counterclockwise, starting at .
Analogously, let  be the first point in 
that we encounter when traveling along  clockwise, starting at .
Let  and  be the paths described above
from  to  and from  to , respectively.
As   and  are subpaths of , they do not intersect,
and we have  and .

The boundary  of  consists of two rays  and ,
starting from  towards  and , respectively.
We claim that either  lies on  or  lies on .
(See \figurename~\ref{fig:kernel}a.) 
Indeed, suppose that .
Then  should be contained in the region bounded by the simple closed curve
, since  does not intersect .
This implies that  must lie on .
Hence, our claim is true.

Without any loss of generality, we assume that ,
the edge  is horizontal, and the interior of  lies locally above ,
as shown in \figurename~\ref{fig:kernel}b.
Then, the path  must contain at least one top reflex edge  lying above
the line through  and ,
since  and  have the same -coordinate and  avoids .
Let  and  be the two endpoints of ,
so .
Then  is the half-plane  supporting .
Since  is a top reflex edge, .
This is a contradiction to our assumption that  intersects .
\end{proof}

Let  be the intersection of the half-planes  supporting 
over all reflex edges  of . We conclude the following.
\begin{theorem} \label{thm:kernel}
 Let  be a simple rectilinear polygon.
 A point  lies in its beacon kernel  if and only if
  for any reflex edge  of .
 Therefore, it always holds that , and
 the kernel  can be computed in linear time.
\end{theorem}
\begin{proof}
Recall that  for any  forms a cone with apex  and aperture angle .
Since any  is adjacent to another reflex vertex 
the intersection  forms exactly the half-plane  supporting the reflex edge 
with endpoints  and .
It implies that .
So by Lemma~\ref{lem:kernel_lem}, we have


The set  is an intersection of axis-parallel halfplanes,
so it is a (possibly unbounded) axis-parallel rectangle.
In order to compute the kernel ,
we identify the extreme reflex edge in each of the four directions to compute ,
and then intersect it with .
This can be done in linear time.
\end{proof}

\section{Beacon-Based Coverage} \label{sec:coverage}


In this section, we study the beacon-based coverage problem for rectilinear polygons.
A set of beacons in  is said to \emph{cover} or \emph{guard} 
if and only if every point  can be attracted by at least one of them.

Our main result is the following. 
\begin{theorem} \label{thm:covering}
 Let  be a simple rectilinear polygon  with  vertices and
  reflex vertices.
 Then  beacons are sufficient
 to guard , and sometimes necessary.
 Moreover, all these beacons can be placed at reflex vertices of .
\end{theorem}

We now sketch the proof of Theorem~\ref{thm:covering}. The lower bound construction is 
a rectangular spiral  consisting of a sequence of  thin rectangles, as depicted in 
\figurename~\ref{fig:spirals}. The sequence of vertices , where 
are the reflex vertices of ,  form
a polyline called the {\it spine} of the spiral. The key idea is the following. Consider
the case  (\figurename~\ref{fig:spirals}a). At first glance, it looks like the spiral
can be covered by two beacons  and  placed near  and , respectively.
However, at closer look, it appears that the small shaded triangular region on the bottom left 
corner is not covered. Hence,  requires  beacons, as announced.
More generally, we can prove that  for a suitable choice of the edge lengths of the
spine of , an optimal coverings for  consists in placing a
beacon at every third rectangle of , which yields the bound .
The spine of  is depicted in \figurename~\ref{fig:spirals}b and c, where the aspect
ratio of the rectangles is roughly .


\begin{figure}[tb]
\centering
\includegraphics[width=\textwidth]{fig_spirals}
\caption{Lower bound construction:
(a) Placing two beacons  and  in  near  and 
 is not enough to cover the shaded region near the reflex vertex .
 (b)(c) The spine of our construction  for  and for . }
\label{fig:spirals}
\end{figure}

The construction for the upper bound in Theorem~\ref{thm:covering} is more involved. 
We first prove that for any polygon with at most 3 reflex vertices, one beacon placed
at a suitable reflex vertex is sufficient. 
For a larger number  of reflex vertices, we proceed by induction.
So we partition  using a cut, and we handle each side recursively.
As mentioned in Section~\ref{sec:pre}, the difficulty is that in some cases, the
union of the two guarding sets of the subpolygons do not cover .
So we will first try to perform a {\it safe cut} , that is, a cut  which 
is not incident to any reflex vertex,  such that there is at least one reflex
vertex on each side, and such that 
.
(See \figurename~\ref{fig:covering_intro}a.)
If such a cut exists, then we can recurse on both side. By Observation~\ref{obs:convex_edge},
the union of the guarding sets of the two subpolygons guards .
Unfortunately, some polygons do not admit any safe cut. In this case, we show
by a careful case analysis that we can always find a suitable cut. 
(See the example in \figurename~\ref{fig:covering_intro}b.)

\begin{figure}[tb]
\centering
\includegraphics[width=.7\textwidth]{fig_covering_intro}
\caption{Upper bound construction. 
(a) A safe cut  of a polygon with  reflex vertices.
(b) This polygon admits no safe cut. We cut along  ,
and the polygon is guarded by any two beacons  and  placed at
reflex vertices of  and , respectively.
\label{fig:covering_intro}}
\end{figure}

\subsection{Proof of the lower bound for coverage}\label{subsec:lower_bound_coverage}


In this section, we prove the lower bound in Theorem~\ref{thm:covering}. Our construction is a 
spiral-like rectilinear polygon  that cannot be guarded by less than  
beacons. (See \figurename~\ref{fig:spirals}.) More precisely, a rectilinear polygon is called a \emph{spiral} if all its reflex vertices are consecutive along its boundary.

The \emph{spine} of a spiral  with  reflex vertices is
the portion of its boundary  connecting  consecutive vertices 
such that  are the reflex vertices of . (See \figurename~\ref{fig:spirals3}a.)
Note that the two end vertices  and  of the spine of a spiral
are the only convex vertices that are adjacent to a reflex vertex.
The spine can also be specified by the sequence of edge lengths 
such that  is the length of the edge  for .

\begin{figure}[tb]
\centering
\includegraphics[width=\textwidth]{fig_spirals3}
\caption{(a) The spine (bold) of a spiral. The point  appears before
 along the spine, so . (b) The partition of  into rectangles .
\label{fig:spirals3}}
\end{figure}

We define an order  among points in any spiral  as follows. Let  be two points in .
Let  and  denote the closest point to  and  on the spine, according to the geodesic
distance within .  (See \figurename~\ref{fig:spirals3}a.) Then we say that  precedes , 
which we denote by , if  precedes  along the spine, that is,  
is on the portion of the spine between  and .

We will use the following partition of a spiral  with  reflex vertices 
into  rectangular subpolygons. It is obtained
by applying the vertical and horizontal cuts at  for each 
and the cut at the midpoint of edge  for each .
We call these rectangles , ordered along
the spine. (See \figurename~\ref{fig:spirals3}b.)

For any integer , let  be the spiral with  reflex vertices
whose spine is determined by the following edge length sequence :
for any nonnegative integer ,

where  is a sufficiently small positive number
and  is a constant.
(See \figurename~\ref{fig:spirals}.)

Therefore, the rectangles  corresponding to  are as follows, for any .
Rectangle   and  have side lengths  and .
Rectangle  has side lengths  and , both of which are strictly less than .

Let  denote the smallest possible number of beacons that can guard . We will
say that a sequence of beacons  is a {\it greedy placement} if 
, and the
sequence  is maximum in lexicographical order. 
So intuitively, we obtain the greedy placement
by pushing the beacons as far as possible from the origin  of the spiral, and giving
priority to the earliest beacons in the sequence.
Clearly,  must be placed in .
We then observe the following for .
\begin{lemma} \label{lem:spiral_sub}
 For , the -th beacon  in a greedy placement for 
 is  in .
\end{lemma}

\begin{figure}[tb]
\centering
\includegraphics[width=\textwidth]{fig_spirals2}
\caption{Proof of Lemma~\ref{lem:spiral_sub}.}
\label{fig:spirals2}
\end{figure}

\begin{proof}
We prove the lemma by induction on .
We first verify the lemma for .
Without loss of generality, we assume that the edge  is a top reflex edge.
Let  be the line through  and .
Observe that  attracts all points in , but not all of those in .
More precisely,  attracts those in  below  but miss those above .
Hence,  must be placed on  to cover the points in  above .
For our purpose, we compare the slopes of  and any line through  and a point in .
(See \figurename~\ref{fig:spirals2}a.)
Recall that .
The slope of  is at least

since , , , and
the width  of  is at most .
On the other hand, the slope of any line through  and a point in  is at most

since , , and the height  of 
is at most .
This implies that  cannot intersect .
Thus, if , then  fails to attract some points near  and above ,
similarly as in \figurename~\ref{fig:spirals}a, so  must lie in .

For the inductive step, assume that  and   lies in .
If , then  attracts all points in ,
that is,  must be the last beacon in the greedy placement.
But this is not the case for our assumption that .
We thus have .
Then,  cannot be completely covered by , so the next beacon 
must cover  partially.
More precisely,  must lie on the line  through  and .
Without loss of generality, we assume that the edge  is a right reflex edge.
Also, note that , since otherwise placing  completes the greedy placement
and thus , which is not the case.

Let , and
let  be the positive number such that .
Recall that  is the length of edge .
We then have , , ,
and .
See \figurename~\ref{fig:spirals2}(b).
Similarly to the above argument,
the slope of  is at least

since  lies in  and
.
On the other hand, the slope of any line through  and any point in  is at most

since ,
This implies that the next beacon  also must be placed in . 
\end{proof}



The main result of this section follows:
\begin{lemma} \label{lem:coverage_lower_bound}
 The spiral  defined above cannot be guarded by less than
  beacons,
 where  denotes the number of vertices of .
\end{lemma}





\subsection{Proof of the upper bound for coverage}\label{subsec:upper_bound_coverage}

In this section, we prove the matching upper bound .
Our proof is by induction on .
The following lemma handles the base case.
\begin{lemma} \label{lem:coverage_base}
 Any rectilinear polygon  with at most three reflex vertices can be guarded by
 a single beacon.  Moreover, the beacon  can be placed at a reflex vertex of ,
 provided that  has at least one.
\end{lemma}
\begin{proof}
Let  be a rectilinear polygon with .
If  has no reflex edge, then  is -monotone by Observation~\ref{obs:monotone},
and thus a beacon placed at any point in  guards 
by Theorem~\ref{thm:kernel}.

Observe that  has at most two reflex edges,
and this is possible only when its three reflex vertices are consecutive.
Assume that this is the case.
Then, the two reflex edges  and  of  must be adjacent and share a reflex vertex .
Hence, the region  forms a cone with the right angle at apex .
Since  is obviously contained in ,  Theorem~\ref{thm:kernel}
implies that , so placing a beacon at  is sufficient to guard .

Now, suppose that  has exactly one reflex edge .
Then,  forms a half-plane, and  is contained in .
We place a beacon  at a reflex vertex incident to .
Since  by Theorem~\ref{thm:kernel},
 guards .
\end{proof}

If , then we will partition  using cuts.
A {\it normal} cut is a cut that is not incident to any vertex of .
We will try to use normal cuts as often as possible,
as the new edges in the subpolygons created by a normal cut are convex,
and thus by Observation~\ref{obs:convex_edge},
these subpolygons can be handled separately.
We are more interested in normal cuts with an additional property:
A normal cut  in  is called \emph{safe} if
, , and
.


A normal cut  in  is called an \emph{-cut} if .
We will abuse notation and write  instead of .
\begin{lemma} \label{lem:safe_cut}
 Let  be any normal cut in  such that  and .
 Then,  is safe if and only if either , or
  is a -cut or an -cut.
\end{lemma}
\begin{proof}
Let  and .
Note that .
Since we assume that  and ,
the cut  is safe if and only if
.


First, suppose that .
If  is  a -cut or a -cut, 
then  or , which implies 
.
If on the other  hand  is a -cut, then we have , and
thus


Now we assume that , that is,  or .
Then  is a -cut or an -cut if and only if   or , which 
is equivalent to 
.
\end{proof}

If there is a safe cut  in  when ,
then one can partition  into two subpolygons  and  by 
and attain the target bound  on the number of beacons
by handling each subpolygon by our induction hypothesis.
But, this is not always the case:
there exist rectilinear polygons  that do not admit a safe cut
when .

\begin{lemma} \label{lem:1-cut}
 Suppose that  admits no safe cut and .
 Then, there exists a horizontal normal cut  in  such that
 either  is a -cut with , or  is a -cut with .
\end{lemma}
\begin{proof}
Since  admits no safe cut, we know that  or 
by Lemma~\ref{lem:safe_cut}.
We separately handle the cases where  or .

\begin{figure}[tb]
\centering
\includegraphics[width=\textwidth]{fig_1-cut}
\caption{Proof of Lemma~\ref{lem:1-cut}}
\label{fig:1-cut}
\end{figure}


First, assume that .
In this case, we show a stronger claim (\figurename~\ref{fig:1-cut}a):
\begin{quote}
\textit{If , then for any top convex edge  of ,
the first reflex vertex  below  is not an endpoint of any horizontal reflex edge of .}
\end{quote}
This automatically proves the lemma: A normal cut  just below  is a -cut and
.
Suppose to the contrary that it is not the case,
so there is a top convex edge  of  such that
the first reflex vertex  below  is an endpoint of a reflex edge  of .
Let  be the other endpoint of .
There are two cases: Either  is a top reflex edge (\figurename~\ref{fig:1-cut}b), 
or a bottom reflex edge (\figurename~\ref{fig:1-cut}c).

Consider the first case, where  is a top reflex edge.
Let  and  be normal cuts just below  and , respectively.
We cannot have  as it would imply that  , an hence
 would be a safe cut. Similarly, we have .
So by Lemma~\ref{lem:safe_cut}, as there is no safe cut in , 
both  and  must be -cuts, and thus .
It implies , a contradiction.

Now, consider the latter case where  is a bottom reflex edge.
See \figurename~\ref{fig:1-cut}c.
Let  be a normal cut just below  and let  be a normal cut just above .
Since  and since
there is no safe cut in , both  and  must be -cuts,
so , a contradiction.
Thus, our claim for the case  is true.

Assume now that , and thus . By our assumption that  has not safe 
cut, there is no -cut  with  and .
Now suppose that the lemma is false.
Then, we have  for any -cut  in , and
 for any -cut  in .
Pick any -cut  in .
If there is no -cut in , then we rotate  by 
so that every -cut is transformed into a -cut.
Note that  and .
Let  be the first reflex vertex below .
If  is not incident to a horizontal reflex edge,
then a normal cut  just below  is a -cut with ,
a contradiction.
Thus,  is incident to a horizontal reflex edge .

There are two cases: either  is a top reflex edge or a bottom reflex edge.
Assume that  is a top reflex edge.
Let  and  be normal cuts just below  and , respectively,
where  is the other vertex incident to .
(See \figurename~\ref{fig:1-cut}c.)
For any , if  is a -cut, then 
and , a contradiction.
Thus, neither  nor  can be a -cut.
Moreover, since ,
we must have .
Since  admits no safe cut, it implies that 
, and hence , a contradiction
to the assumption that .

Assume that  is a bottom reflex edge.
Let  be a normal cut just below 
and  be a normal cut just above .
(See \figurename~\ref{fig:1-cut}c.)
In this case,  cannot be a -cut since ,
while  cannot be a -cut since .
On the other hand, we have ,
and thus .
So, we must have .
Since  has no safe cut, it implies that  ,
and thus , a contradiction.
\end{proof}

Now, we are ready to prove the main result of this section.
\begin{lemma} \label{lem:covering}
 Let  be a simple rectilinear polygon  with  vertices and
  reflex vertices.
 Then,  beacons are sufficient
 to guard .  Moreover, all these beacons can be placed at reflex vertices of .
\end{lemma}
\begin{proof}
Our proof is by induction on .
The base case where  is already handled by Lemma~\ref{lem:coverage_base},
so  we assume that .
If  admits a safe cut , then we partition it into two subpolygons  and ,
and  we handle each subpolygon recursively. 
Our guarding set for  is the union of the guarding sets
for  and . As  is safe, the total number of beacons we place is at most
 
These beacons indeed guard , because
 is a convex edge of each subpolygon,
and thus by Observation~\ref{obs:convex_edge}, no beacon attraction path hits .

Now we assume that  admits no safe cut.
Then, by Lemma~\ref{lem:safe_cut}, we have , so ,
and there is no -cut or -cut with at least one reflex vertex on each side.
Consider the set  of all -cuts  in  with .
By Lemma~\ref{lem:1-cut}, we may assume that  is nonempty:
If , then it immediately follows that , 
and  if  and ,
then we rotate  by .

Pick a -cut  such that  is minimum.
Let  be the first reflex vertex of  below .
If  is not an endpoint of a horizontal reflex edge, then a normal cut  just below 
is a -cut with  and ,
so it is a safe cut, a contradiction to the assumption that  admits no safe cut.
Hence,  must be an endpoint of a horizontal reflex edge .
We have two cases: Either  is a top reflex edge, or a bottom reflex edge.

\begin{figure}[tb]
\centering
\includegraphics[width=\textwidth]{fig_covering_top}
\caption{Proof of Theorem~\ref{thm:covering} when  is a top reflex edge.
(a) When  is a -cut and  is a -cut.
(b1)--(b3) When  and  are -cuts.}
\label{fig:covering_top}
\end{figure}


\paragraph{When  is a top reflex edge.}
Assume the former case where  is a top reflex edge.
Let  and  be the left and right endpoint of , respectively, and
 and  be normal cuts just below  and ,
respectively.
Also, let  be such that
 is an -cut and  is an -cut.
We then observe that , that is, .
We treat separately the two possible cases: 
Case (a), where  or , and Case (b), where .
(See \figurename~\ref{fig:covering_top}.)
\begin{enumerate}[(a)] \item Assume that , so  is a -cut and  is a -cut.
 (The case where  is symmetric, and can be handled in the same way.)
 As  is not a safe cut, we have .
 Consider the horizontal cut  at .
 (See \figurename~\ref{fig:covering_top}a.)
We have , and .
 We then place beacons in  and  separately and recursively.
 The total number of beacons placed is at most
 

 We still need to make sure that these beacons indeed guard ,
 Our induction hypothesis implies that all the beacons are placed at reflex vertices of .
 So there is no beacon placed below the horizontal cut at ,
 and thus, by Observation~\ref{obs:convex_edge}, no beacon attraction path in  hits .
 Again by Observation~\ref{obs:convex_edge}, the beacons placed in  indeed guard
 the region  in  since  is a convex edge of .
 This ensures that the beacons we placed separately in  and 
 indeed guard .

\item We now consider the case where .
 Then, we have  by our choice of .
 Let  and  be the edges other than   incident to  and , respectively.
 If  is not a reflex edge, then a beacon placed at  guards 
 since its kernel  is nonempty.
 The other part  can be guarded by at most 
 guards placed on reflex vertices of .
 Since  is a normal cut, these  beacons
 together guard .
 As , the number of beacons is bounded by
 
 as desired.
 The case where  is not a reflex edge   can be handled symmetrically
 by placing a beacon at .

 Thus, we now assume that both  and  are reflex edges.
 Let  and  be the vertical cuts at  and , respectively.
We handle three subcases separately:
 (i) either  or  for some ,
 (ii)  and  for each ,
 or (iii)  and  for each .
 If we are not in Case (i), we have either Case (ii) or (iii).
 So these three cases cover all possible situations.
 \begin{enumerate}[(i)]
 \item Without loss of generality, we assume that  or .
 In this case,  we handle  and  recursively.
 The union of the two guarding sets of  and  guards ,
 as all these beacons are placed at reflex vertices.
 (See \figurename~\ref{fig:covering_top}(b1).)
 The number of  beacons is at most  ,
 since  and .


 \item Assume that  and  for each .
 We partition  into  and  and handle them recursively.
 The total number of beacons is at most
 
 These beacons guard , as they are all placed at reflex vertices of .
 \item
 The remaining case is when  and 
 for each .
 Then a vertical cut just to the left of  is a -cut,
 and a vertical cut just to the right of  is a -cut.
 Since  admits no safe cut, this implies that .

 Let  be the first reflex vertex above , and let  be the vertical cut at .
 We consider two cases: Either  is incident to a bottom reflex edge, or not.
 In the former case, let  be the other endpoint of the bottom reflex edge.
 (See \figurename~\ref{fig:covering_top}(b2).)
 Since  is a -cut and , one of the two horizontal cuts just above  and 
 must be either a -cut or a -cut.
 Without loss of generality, assume that the cut above  is  either a -cut or a -cut.
 As , it means that the number of reflex vertices above this cut is  or 
 modulo 3, and hence  or .
 

 In the latter case, where  is not incident to a bottom reflex edge,
 we may assume without loss of generality that the horizontal edge incident to 
 is to the left of . (See \figurename~\ref{fig:covering_top}(b3).)
 So we still have  or .

 In both cases, we partition  by the vertical cut  at  into  and .
 The endpoint of  other than  always lies on the reflex edge ,
 since .
 Since  or , and , and
 ,
 we have either  or .
 We handle  and , separately, and recursively.
 Then the total number of beacons placed in  is at most
 
 We still need  to verify that these beacons guard .
 As , by our induction hypothesis, there must be a beacon at one of the 
 endpoints of .
 Such a beacon attracts all points to the left of , and thus
 the region  is covered by the beacons in .
 Since  is a convex edge of , no attraction path hits  inside 
 either.
 \end{enumerate}

\end{enumerate}
This completes the proof for the case where  is a top reflex edge.

\begin{figure}[tb]
\centering
\includegraphics[width=\textwidth]{fig_covering_bottom}
\caption{Proof of Theorem~\ref{thm:covering} when  is a bottom reflex edge.
(a) When .
(b) When  and  is a -cut.
(c) When ,  is a -cut, and  is a -cut.
(d) When ,  is a -cut, and  is a -cut. }
\label{fig:covering_bottom}
\end{figure}


\paragraph{When  is a bottom reflex edge.}
We now assume that  is a bottom reflex edge.
Let  be the other endpoint of . (See \figurename~\ref{fig:covering_bottom}.)
Without loss of generality, we assume that  is to the left of .
Let  be a normal horizontal cut just below  and
 be a normal cut just above .
Also, let  be such that  is an -cut
and  is an -cut.
So we have ,
that is, .
Recall that  has been chosen to be a -cut
with the smallest value of  among all -cuts  of  with .

We first assume that , and thus
 .
 If , then  is a safe cut.
 Similarly, if , then  is a safe cut.
 Since  admits no safe cut, we have that .
 So by our choice of , we have .
 Let  be the unique reflex vertex in .
 We pick any normal vertical cut  at any point on .
 (See \figurename~\ref{fig:covering_bottom}a.)
Since  is not a safe cut,  must be a -cut, that is, .
On the other hand, we have  if ,
and  if .
As  is a -cut, it implies that either  or , a contradiction, so .

We hence have  and . We first rule out .
So we assume, for sake of contradiction, that . We must have , 
as otherwise  would be safe cut.
We make a cut  to the left of .  (See \figurename~\ref{fig:covering_bottom}b.)
As  is a 1-cut, we have , and . We recursively
construct guarding sets of beacons for  and . We claim that these
beacons together guard . Indeed, the only way a point  may not be covered would
be that  lies in  and its attraction path crosses  from below.
But then  would be attracted, within , by the same beacon as . This
is impossible because there is no reflex vertex above the cut , and by our induction
hypothesis, beacons are placed at reflex vertices. To complete the proof for
this case, we only need to bound the number of beacons we placed. It is at most
	

We now rule out the case where . 
If it were the case, then we would have  since  has no safe cut.
We pick any normal vertical cut  at any point on .
Since  is a -cut and both sides of  contain at least one reflex vertex,
it is a safe cut, which contradicts our assumptions.
(See \figurename~\ref{fig:covering_bottom}c.)

We thus have .
Note that  by our choice of .
Let  be the unique reflex vertex in .
(See \figurename~\ref{fig:covering_bottom}d.)
Then  must be to the left of , because
otherwise, 
there would be a normal vertical cut  at a point on 
such that  lies to the right of , and
 would be a safe cut.

Let  be the horizontal cut at  and  be the vertical cut at .
We then handle  the two subpolygons  and , separately, in a recursive way,
though they partially overlap.
Due to the overlap, the union of the set of beacons placed separately in each subpolygon
guard the whole polygon .
Since  and
,
the number of beacons we placed is at most

This completes the proof.
\end{proof}


Our last result is to show that in the worst case, monotone rectilinear polygons require fewer
beacons than simple rectilinear polygons. It matches the lower bound by Biro~\cite{b-bbrg-13}.
\begin{theorem}\label{thm:coverage_monotone}
 For any rectilinear monotone polygon  with  vertices,  of which are reflex,
  beacons are sufficient to guard ,
 and sometimes necessary.
\end{theorem}
\begin{proof}
Without loss of generality, we assume that  is -monotone.
Thus,  has no vertical reflex edge by Observation~\ref{obs:monotone}.
Our proof is by induction on the number  of reflex vertices.
If  has at most one reflex edge ,
then we observe that any point on  is contained in the kernel 
by Theorem~\ref{thm:kernel}.
Thus, one beacon is sufficient to guard .

Now, assume that  has at least two reflex edges.
This implies that  since  is -monotone.
Let  be the right endpoints of the reflex edges
sorted from left to right.
Let  and  be the reflex edges that are incident to  and , respectively.
Let  be the vertical cut at .
We partition  into  and  by .
Then the left side subpolygon  has at most one reflex edge ,
and thus can be guarded by a single beacon placed at any point on .
The right side subpolygon  has  reflex vertices.
Thus, by induction, at most  beacons
can guard .
The total number of beacons placed in  is at most
 
as desired.

Finally, observe that cutting by  always makes a new convex edge in  and 
since there is no vertical reflex edge in .
This implies that separately guarding  and  is sufficient to guard the whole 
by Observation~\ref{obs:convex_edge}.
\end{proof}

\section{Beacon-Based Routing} \label{sec:routing}


In this section, we give an improved upper bound for the beacon-based routing problem in a simple rectilinear polygon  with  reflex vertices. 

Our result in this section is as follows.
\begin{theorem}
\label{thm:routing}
 For any simple rectilinear polygon  with  vertices,  of which are reflex,
  beacons are always sufficient
 to route between all pairs of points in . There are simple rectilinear polygons in which  beacons are necessary to route all point pairs for any .
\end{theorem}



\subsection{Proof of the lower bound for routing}
The polygon constructed by Biro et al.~\cite{bgikm-cccg-13} to show the lower bound  is -monotone. (See \figurename~\ref{fig:intro}a.) But we can construct non-monotone polygons for which  beacons are necessary for any , i.e., . When ,  is -monotone, so no beacon is needed.

We construct a spiral polygon  with  reflex vertices. The construction is similar to the one for the lower bound on the coverage problem, so we explain only the idea of the construction. For , no beacon is needed, and for , one beacon is necessary, so it suffices to construct  for . 

\begin{figure}[tb]
\centering
\includegraphics[width=\textwidth]{lb_for_routing}
\caption{Examples on the construction of  for the lower bound on the routing problem. (a) . (b) . (c) .}
\label{fig:lb_for_routing}
\end{figure}

As a base case for the construction, we consider . See \figurename~\ref{fig:lb_for_routing}a. Let  and  be the end convex vertices of the spine of . For a point  to be attracted to ,  must be in a polygonal region above the line , otherwise its beacon-based path to  is blocked by the reflex edge connecting  and . If another point  would be attracted to , then  must be in a polygonal region below the line . Since such two regions are disjoint except at , one beacon is not sufficient. Note here that a beacon placed at  is attracted neither to  nor to  because two paths toward  and  are both valid. So  needs (at least)  beacons. From , we construct  incrementally. 

Let us look at . Two beacons are sufficient and necessary for  as in \figurename~\ref{fig:lb_for_routing}b. We know that the second beacon  should be placed in the region bounded by the line  and the line , otherwise  cannot be attracted to  due to the obstruction of the edge from  to . Then we can make  with the endpoint , as in \figurename~\ref{fig:lb_for_routing}c, such that the line  passes above the region where  is placed.  This implies that  cannot be attracted to  because  lies below the line , so the third beacon is necessary. It is not hard to check that this argument can be applied to construct  from  for any . This shows the lower bound for the routing problem as follows.

\begin{lemma} \label{lem:coverage_lower_bound}
 The spiral  defined above cannot be guarded by less than
  beacons,
 where  denotes the number of vertices of , i.e., , for any  or .
\end{lemma}


\subsection{Proof of the upper bound for routing}

We now explain how to place  beacons to route all pairs of points in .

When  is -monotone, we do not need to place any beacon in order to route between
a pair  of points in , since the target  is regarded as a beacon.
We thus focus on the case where  ifs not -monotone. 
Our approach is to cut  by extending some of its edges, and
handle the resulting parts separately. More precisely,
for any reflex vertex  of   incident to an edge ,
we denote by  the cut obtained by extending  through .
So when   is horizontal,  is the horizontal cut at , and
if  is vertical, then  denotes the vertical cut at .
The cut  splits  into two subpolygons  and ,
one of which does not contain .
We call this subpolygon the \emph{pocket} of  at , denoted by .
For instance, if  is a top reflex edge, then we have .
So for any reflex edge  with endpoints  and ,
there are two cuts  and  extending , 
and two pockets  and  of .
(See \figurename~\ref{fig:monotone_pocket}a.)

Our solution to the routing problem is based on the following key lemma.

\begin{figure}[tb]
\centering
\includegraphics[width=\textwidth]{fig_monotone_pocket}
\caption{(a) The two cuts  and  extending a reflex edge 
and the pockets of , when  is a top reflex edge.
(b) Proof of Lemma~\ref{lem:monotone_pocket}.
\label{fig:monotone_pocket}}
\end{figure}


\begin{lemma} \label{lem:monotone_pocket}
 Suppose that  is not -monotone.
 Then there exist a reflex edge  of  and an endpoint  of  such that
 the pocket  of  at  is -monotone. 
\end{lemma}
\begin{proof} 
Let  be a reflex edge of , and  be an endpoint of , such that
the number of vertices of the pocket  is minimum.
If the pocket  has no reflex edge, then
 is -monotone by Observation~\ref{obs:monotone}.

Suppose that  contains at least one reflex edge .
We claim that one of the two pockets of  is always contained in .
If our claim is true, then such a pocket of  has fewer vertices
than  does, a contradiction.

Let  and  be the two endpoints of . (See \figurename~\ref{fig:monotone_pocket}b.)
Suppose that our claim is false,
that is, neither  nor   is contained in .
It means that each pocket contains points on both sides of  .
Pockets are simple polygons, and hence they are connected. Thus,  contains
a point  of  and a point  of . So the boundaries
of   and  must cross  between  and ,
which implies that  and  cross . This
is impossible, as   and  are
collinear.
\end{proof}

We are now ready to prove our upper bound 
on the beacon-based routing problem.
Our proof is by induction on the number  of reflex vertices of 
based on partitioning  into -monotone subpolygons in a recursive manner. 



\begin{lemma}
\label{lem:routing}
 For any simple rectilinear polygon  with  vertices,  of which are reflex,
  beacons are always sufficient
 to route between all pairs of points in . 
\end{lemma}
\begin{proof}
Our proof is by induction on , the number of reflex vertices of .
First consider the base case where .
In this case,  is -monotone by Observation~\ref{obs:monotone},
so no beacon is required for  as discussed above.
Hence, the upper bound holds.

Now, suppose that  is not -monotone, and thus .
Then, Lemma~\ref{lem:monotone_pocket} implies the existence of
a pocket  of  that is -monotone, where  is an endpoint of a reflex edge  of .
Without loss of generality, we assume that  is a top reflex edge and  is the left endpoint of .
Let  be the other endpoint of , so
 and .
We consider three subpolygons, as in Figure~\ref{fig:routing}a,
, , and . Let  be the endpoint other than  of  and let  be the end point than  of . Note that .
We split into two cases: either , or .

\begin{figure}[tb]
\centering
\includegraphics[width=0.8\textwidth]{fig_routing}
\caption{Proof of Theorem~\ref{thm:routing}. (a) When ,
two beacons  and  are placed at  and  (marked by ).
(b) When , a beacon  is first placed just above  (marked by ), and other beacons will be placed more according to the shape of .}
\label{fig:routing}
\end{figure}

We first consider the case that . We place two beacons  and  at  and , respectively, then place  beacons in  and  beacons in  recursively. Since , . Using this fact, we can bound the total number of beacons we have placed below by


We now check if any pair of  and  can be routed via these beacons. 
Segments , , and  used for the partition are all the convex edges of the corresponding subpolygons, so by Observation~\ref{obs:convex_edge}, none of the segments are hit by any beacon-based routing path between two points in a subpolygon. This implies that once  is routed to some point in (or on the boundary of) a subpolygon, it can be routed to any target  in the subpolygon by the induction hypothesis.  contains  and  contains . Moreover,  and  can attract each other along , so any pair of  for  can be routed via  or via  or both of them. This completes the case where .

Now, we consider the other case where , which means  is a rectangle. We place a beacon  infinitesimally above  as in Fig.~\ref{fig:routing}b. Note that  is placed inside , but it can attract any point in , and it can be attracted to any point in  because it is located above the line connecting  and the lower right corner of . We place  beacons in  recursively. For , we need a more careful placement method as follows. 

We first suppose that . Then no beacons inside  are required because  can attract any point in  and it can be attracted to any point in . Using this fact, we can easily verify that any two points in  can be routed to each other via . The route between  and a point in  is always possible by the induction hypothesis, which implies that any pair  can be routed wherever  and  belong to. The number of beacons we have placed is at most

since . Thus, from now on, we assume that . 

For , we partition  into at most three smaller subpolygons as follows. We sweep the interior of  with an initial sweeping line segment  downward as long as the swept region remains -monotone. See Fig.~\ref{fig:routing_C}. If the -monotonicity is violated, then there must be a reflex edge  that  intersects. Then  would be either horizontal or vertical. 

\begin{figure}[tb]
\centering
\includegraphics[width=\textwidth]{routing_C}
\caption{The case where the reflex edge  is horizontal. (a)  is a bottom reflex edge. The beacon  is placed because . (b)  is a top reflex edge, which is connected from . The beacon  is not placed because . The symmetric case where a top reflex edge  is connected from  is omitted in this figure.}
\label{fig:routing_C}
\end{figure}

Let us suppose that  is horizontal. For this case,  could be either a bottom reflex edge as in Fig.~\ref{fig:routing_C}a or a top reflex edge as in Fig.~\ref{fig:routing_C}b. Then we split  into three subpolygons by cutting  along the sweeping segment  containing ;  is the -monotone piece,  and  are the other two pieces as in Fig.~\ref{fig:routing_C}a-b. We place the beacons in  and  (not in ) recursively, and place additional beacons as follows. We place two beacons  and  at two end points of , where  is assumed to be in the left of . We place one more beacon  at the point  only when . It holds that . When , the total number of beacons we have placed is

When , the beacon  is not placed, so the number of the beacons is


Let us check if  can be routed to  in this placement. As in the previous case where , all the segments used for the partition are convex edges in their associated subpolygons. So, it is sufficient to show that any pair of beacons from  can be routed to each other. First,  and  are attracted to each other along the cut , and so are  and  because they are visible each other. Second,  or  is on the boundary of , and  is -monotone, so if , i.e.,  exists, then  can be routed to  or . Otherwise, if  as in Fig.~\ref{fig:routing_C}b, then  is either a rectangle or a union of two rectangles with the unique reflex vertex . We here claim that  can be attracted to  or , and  can also attract  or . If  can see directly the one of them, then it is done. Suppose that they are not visible from . For this to happen,  and an end vertex of  must obstruct the sight from  to  and to . Then there are four situations as in Fig.~\ref{fig:routing_C1}. In the first three situations (Fig.~\ref{fig:routing_C1}a-c),  can be clearly attracted to  or . For the last situation as in Fig.~\ref{fig:routing_C1}d, it cannot be attracted to , but it can be attracted to  via the end vertex of  and along the cut containing . Thus  can always reach  or . The reverse attraction is also possible. If  can see  or , say , then  first goes to  then reaches . Otherwise, only the last situation in Fig.~\ref{fig:routing_C1}d would be in trouble because  cannot be attracted to . But  can be attracted to , thus  can reach  via . As a result, the beacons in  can be routed to each other, which means that any pair  can be routed in this partition.

\begin{figure}[tb]
\centering
\includegraphics[width=\textwidth]{routing_C1}
\caption{Four different situations that  sees neither  nor .}
\label{fig:routing_C1}
\end{figure}

We now consider the last case where  is a vertical reflex edge. See Fig.~\ref{fig:routing_last}. Let  be the lower end vertex of , and let  be the horizontal edge incident to  with the other end vertex . If  is a reflex edge, that is,  is reflex, then it should be a top reflex edge, thus we cut  along  into three pieces , , and , where  is a pocket , and . We place two beacons  and  at two end points of . We place  at  only when . Finally, we place the beacons in  and  recursively. If  is not a reflex edge, i.e.,  is convex, then we cut  along  into two pieces  and , where . We place a beacon  at the endpoint of  (not at ), and one more beacon  at  regardless of the size of . Note that  and  are located both at the convex vertices of .

\begin{figure}[tb]
\centering
\includegraphics[width=0.8\textwidth]{routing_last}
\caption{The case where  is a vertical reflex edge. The symmetric cases are omitted in this figure.}
\label{fig:routing_last}
\end{figure}

The partition  actually corresponds to the previous case where  is horizontal, so we can apply the same arguments to bound the number of beacons and to ensure that any pair  can be routed. We now focus only on the partition . We have placed three beacons . Note here that  because  always contains the upper reflex vertex of . Using this with the fact that , we can bound the number of beacons by
 Let us check if  can be routed to . The cuts used for the partition are again served as convex edges in associated subpolygons. Thus it suffices to prove that three beacons  are routed to each other. The two beacons  and  are visible, so they can attract each other. Because  and  are both in the -monotone subpolygon , they also attract each other. Thus the three beacons can attract each other. This completes the proof of the theorem.
\end{proof}




\section{Concluding Remarks} \label{sec:conclusion}


In this paper, we attempt to reduce the gaps between the lower and upper bounds on the number of beacons required in beacon-based coverage and routing problems for a simple rectilinear polygon . For the coverage problem, we raised its lower bound, and presented an algorithm to place the same number of beacons to cover . These results settle the open questions on the coverage problem. For the routing problem, we improved the lower and upper bounds, but there is still a gap between them, which is an immediate open question. Furthermore, we presented an optimal linear time algorithm of computing the beacon-based kernel of . But it remains open to compute the \emph{inverse kernel} of , which is defined as a set of points in  that are attracted to all the points in , in a subquadratic time.



\begin{thebibliography}{10}

\bibitem{b-bbrg-13}
M.~Biro.
\newblock {\em Beacon-based routing and guarding}.
\newblock Dissertation, Stony Brook University, 2013.

\bibitem{bgikm-cccg-13}
M.~Biro, J.~Gao, J.~Iwerks, I.~Kostitsyna, and J.~S.~B. Mitchell.
\newblock Combinatorics of beacon-based routing and coverage.
\newblock {\em Proc. the 25th Canadian Conf. Comput. Geom. (CCCG 2013)}, 2013.

\bibitem{bikm-wads-13}
M.~Biro, J.~Iwerks, I.~Kostitsyna, and J.~S.~B. Mitchell.
\newblock Beacon-based algorithms for geometric routing.
\newblock In {\em Proc. the 13th WADS (WADS 2013)}, volume 8037 of {\em LNCS},
  pages 158--169, 2013.

\bibitem{c-ctpg-75}
V.~Chav\'{a}tal.
\newblock A combinatorial theorem in plane geometry.
\newblock {\em J. Combinat. Theory Series B}, 18:39--41, 1975.

\bibitem{g-spragt-86}
E.~Gy\"{o}ri.
\newblock A short proof of the rectilinear art gallery theorem.
\newblock {\em SIAM J. on Algebraic and Discrete Methods}, 7(3), 1986.

\bibitem{ghks-ggprp-96}
E.~Gy\"{o}ri, F.~Hoffmann, K.~Kriegel, and T.~Shermer.
\newblock Generalized guarding and partitioning for rectilinear polygons.
\newblock {\em Comput. Geom.: Theory Appl.}, 6(1):21--44, 1996.

\bibitem{kkk-tgrfw-83}
J.~Kahn, M.~Klawe, and D.~Kleitman.
\newblock Traditional galleries require fewer watchmen.
\newblock {\em SIAM J. on Algebraic and Discrete Methods}, 4(2), 1983.

\bibitem{krs-cccg-14}
B.~Kouhestani, D.~Rappaport, and K.~Salmoaa.
\newblock Routing in a polygonal terrain with the shortest beacon watchtower.
\newblock {\em Proc. the 26th Canadian Conf. Comput. Geom. (CCCG 2014)}, 2014.

\bibitem{mp-agtgg-03}
T.~S. Michael and V.~Pinciu.
\newblock Art gallery theorems for guarded guards.
\newblock {\em Comput. Geom.: Theory Appl.}, 26:247--258, 2003.

\bibitem{o-apragt-83}
J.~O'Rourke.
\newblock An alternative proof of the rectilinear art gallery theorem.
\newblock {\em J. Geometry}, 21:118--130, 1983.

\bibitem{o-agta-87}
J.~O'Rourke.
\newblock {\em Art Gallery Theorems and Algorithms}.
\newblock International Series of Monographs on Computer Sciences. Oxford
  University Press, 1987.

\bibitem{s-rrag-92}
T.~Shermer.
\newblock Recent results in art galleries.
\newblock {\em IEEE Proceedings}, 90(9), 1992.

\bibitem{u-agip-00}
J.~Urrutia.
\newblock Art gallery and illumination problems.
\newblock In J.-R. Sack and J.~Urrutia, editors, {\em Handbook of Computational
  Geometry}, chapter~22, pages 973--1027. North-Holland, 2000.

\end{thebibliography}




\end{document}
