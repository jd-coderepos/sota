\documentclass{acm_proc_article-sp}

\usepackage{amsmath}
\usepackage{amssymb}
\usepackage{algorithm}
\usepackage{algorithmic}
\usepackage{array}
\usepackage{complexity}
\usepackage{url}

\providecommand*{\Nset}{\mathbb{N}} \providecommand*{\Zset}{\mathbb{Z}} \providecommand*{\Qset}{\mathbb{Q}} \providecommand*{\Rset}{\mathbb{R}} 

\newcommand*{\bigland}{\mathop{\bigwedge}}

\newtheorem{theorem}{Theorem}[section]
\newtheorem{proposition}[theorem]{Proposition}
\newtheorem{lemma}[theorem]{Lemma}
\newtheorem{corollary}[theorem]{Corollary}
\newtheorem{definition}[theorem]{Definition}
\newtheorem{example}[theorem]{Example}
\newtheorem{remark}[theorem]{Remark}

\newcommand*{\coanote}[1]{{\footnote{\textbf{For co-authors:} {#1}}}}

\newcommand{\st}{\mathrel{.}}
\newcommand{\itc}{\mathrel{:}}

\newcommand{\dec}{\mathrm{DEC}}
\newcommand{\pos}{\mathrm{POS}}
\newcommand{\inc}{\mathrm{INC}}

\begin{document}

\title{Eventual Linear Ranking Functions}

\numberofauthors{2}
\author{
\alignauthor
Roberto Bagnara \\
    \affaddr{BUGSENG (\url{http://bugseng.com})} \\
    \affaddr{Dipartimento di Matematica e Informatica} \\
    \affaddr{Universit\`a  di Parma, Italy} \\
    \email{bagnara@cs.unipr.it}
\alignauthor
Fred Mesnard \\
    \affaddr{LIM/ERIMIA}\\
    \affaddr{Universit\'e de la R\'eunion, France} \\
    \email{frederic.mesnard@univ-reunion.fr}
}

\maketitle

\begin{abstract}
Program termination is a hot research topic in program analysis.
The last few years have witnessed the development of termination
analyzers for programming languages such as C and Java with remarkable
precision and performance.
These systems are largely based on techniques and tools coming
from the field of declarative constraint programming.
In this paper,\footnote{A preliminary version of this work, in French, has been
    presented to the \emph{Journ\'ees Francophones de Programmation par
    Contraintes}.
}
we first recall an algorithm based on Farkas' Lemma for
discovering linear ranking functions proving termination of a certain
class of loops.  Then we propose an extension of this method for
showing the existence of \emph{eventual linear ranking functions},
i.e., linear functions that become ranking functions after a
finite unrolling of the loop.
We show correctness and completeness of this algorithm.
\end{abstract}

\begin{keywords}
termination analysis, ranking function, eventual linear ranking function.
\end{keywords}


\section{Introduction}
\label{sec:introduction}

Program termination is a hot research topic in program analysis.
The last few years have witnessed the development of termination
analyzers for mainstream programming languages such as C~\cite{CookPR06}
and Java~\cite{AlbertAGGPRRZ09,OttoBEG10,SpotoMP10} with remarkable
precision and performance.
These systems are largely based on techniques and tools coming
from the field of declarative constraint programming.

Beyond the specificities of the targeted programming languages and after
several abstractions (see, e.g., \cite{SpotoMP10}), termination analysis
of entire programs boils down to termination analysis of individual loops.
Various categories of loops have been identified: for the purposes of
this paper we focus on \emph{single-path linear constraint} (SLC)
loops~\cite{Ben-AmramG13TR}.
An SLC loop over  variables , \dots,~ has the form

where 
and 
are column vectors of variables,
 is an integer matrix, ,
 and .
Such a loop can be conveniently written as a constraint logic
programming rule:

When variables take their values in  (resp., ),
we call such loops \emph{integer} (resp., \emph{rational}) loops.
They model a computation that starts from a point ;
if  is false, the loop terminates;
otherwise, a new point  is chosen that satisfies

and iteration continues replacing the values of 
by those of .

Loop termination can always be ensured by a \emph{ranking function}
, a function from  or  to a well-founded set.
As the domain of  is well-founded, the computation terminates.
To the best of our knowledge, decidability of universal termination
of SLC loops (i.e., from any starting point and for any choice of
the next point at each iteration) is an open question.
Some sub-classes have been shown to be
decidable~\cite{BozgaIK12,Braverman06,Tiwari04}.
For instance, Braverman proves that termination of
loops where the body is a \emph{deterministic} assignment
 is decidable when the variables range over .
The problem is open for the non-deterministic case, as stated in his paper.
On the other hand, various generalizations have been shown
to be undecidable~\cite{Ben-AmramGM12}.

A way to investigate loop termination is to restrict the class of
considered ranking functions.
In the following section, we recall a well-known technique
for computing linear ranking functions for rational SLC loops.

In Section~\ref{sec:eventual-linear-ranking-functions}
we present the main contribution of the paper,
namely the definition of \emph{eventual linear ranking functions}:
these are linear functions that become
ranking functions after a finite unrolling of the loop.
We shall see that the number of unrolling is not pre-defined, but
depends on the data processed by the loop.
Section~\ref{sec:eventual-linear-ranking-functions} presents
complete decision procedures for the existence of
eventual linear ranking functions of SLC loops.  The presentation is
gradual and illustrates the algorithms by means
of constraint logic programming (CLP) technology and dialogs with real
CLP tools.
Section~\ref{sec:related-work-and-experiments} discusses related
work and a preliminary experimentation conducted on the benchmarks
proposed in two very recent papers.
Section~\ref{sec:conclusion-and-future-work} concludes the paper.


\section{Linear Ranking Functions}
\label{sec:linear-ranking-functions}

We first define the notion of linear (resp., affine) ranking function
for an SLC loop.

\begin{definition}
\label{def-fn-rng-lin}
Let  be the SLC loop
 
where  is an n-ary relation symbol.
A \emph{linear} (resp., \emph{affine}) \emph{ranking function}  for 
is a linear (resp., affine) map from  to  such that

\end{definition}

In words, continuation of the iteration, i.e., ,
entails that  stays positive and strictly decreases by at least 
for each iteration.
We point out that if  is not satisfiable,
the loop ends immediately and any linear function is a ranking
function.
In the paper, we assume that  is
satisfiable.

\begin{remark}
\textup{Definition~\ref{def-fn-rng-lin}} might seem too restrictive
when working with rational numbers as one might prefer to replace the
decrease by  by a decrease by , a fixed positive
quantity.  Actually, by multiplying such an -decrease
ranking function by , we see that the two definitions
are equivalent with respect to the existence of a ranking function.
\end{remark}

\begin{remark}
Although the class of affine ranking functions subsumes the class of linear
ranking functions, any decision procedure for the existence of
linear ranking functions can be extended to a decision procedure
for the existence of affine ranking functions.
To see this, note that an affine ranking function for

where  is distinct from the variables in .
\end{remark}

In this section, we focus on linear ranking functions for SLC loops.
After the presentation of a formulation of Farkas' Lemma
we consider the problem of verifying linear ranking
functions, and then the detection of such ranking functions.


\subsection{Farkas' Lemma}

A linear inequation  over rational numbers is a logical consequence
of a finite satisfiable conjunction  of linear inequations when
 is a linear positive combination of the inequations of .
More formally, let  be

and suppose that  has at least one solution.
Farkas' Lemma states the equivalence of

and



\subsection{Verification}
\label{sec:lrf-verification}

Given an SLC loop  and a linear function ,
we can easily check whether  is a ranking function for 
by testing the unsatisfiability of
 and
.
This test has polynomial complexity and can be done with a complete
rational solver such as , e.g., CLP() \cite{Holzbaur95}.

\begin{example}
\label{ex:linear-ranking-function}
For the SLC loop :

the linear function  is a ranking function, as proved by
the following \emph{SICStus Prolog} session.
\begin{verbatim}
?- use_module(library(clpq)).
true.
?- {X >= 0, Y1 =< Y - 1, X1 =< X + Y, Y =< -1,
    X < 1 + X1}.
false.
?- {X >= 0, Y1 =< Y - 1, X1 =< X + Y, Y =< -1,
    X < 0}.
false.
?-
\end{verbatim}
\end{example}

\subsection{Detection}
\label{sec:lrf-detection}

Given an SLC loop, we would like to know whether it admits a linear ranking
function .
This problem, which has been studied in depth
\cite{BagnaraMPZ12IC,PodelskiR04,SohnVG91},
is decidable in polynomial time.

Let us consider Example~\ref{ex:linear-ranking-function}
and formally ask whether there exists a ranking function of the form
:

This formulation of the problem is executable by quantifier
elimination on a symbolic computation system like
Reduce~\cite{Hearn05}:
\begin{verbatim}
1: load_package redlog;
2: rlset r;
3: F:=ex({a,b},all({x,y,x1,y1},
  (x>=0 and y1<=y-1 and x1<=x+y and y<= -1)
  impl
  (a*x+b*y>=1+a*x1+b*y1 and a*x+b*y>=0)));
4: rlqe F;
\end{verbatim}
Statement~\texttt{1} loads the quantifier elimination module.
Statement~\texttt{2} defines  as the domain of discourse.
Statement~\texttt{3} initializes formula .
Statement~\texttt{4} runs quantifier elimination over  and returns
an equivalent formula, \texttt{true} in this case.
Hence, formula  is true and there exists at least one linear
ranking function.
We can now determine the coefficients of function  as follows:
\begin{verbatim}
5: G:=all({x,y,x1,y1},
  (x>=0 and y1<=y-1 and x1<=x+y and y<= -1)
  impl
  (a*x+b*y>=1+a*x1+b*y1 and a*x+b*y>=0));
6: rlqe G;
\end{verbatim}
We obtain

and all values for  and  satisfying the above formula,
such as  and , are equally good.
Unfortunately, the complexity of the algorithms involved
will prevent us from systematically obtaining such a result
within acceptable time and memory bounds.

We now recall the most famous algorithm for this
problem~\cite{PodelskiR04}.\footnote{See also \cite{BagnaraMPZ12IC}.}
Considering  and  as \emph{parameters}
of the problem, we can apply Farkas' Lemma.
For the strict decrease of the ranking function we have

Application of Farkas' Lemma to this problem can be depicted
as follows:

We know that formula~(\ref{decrease-of-the-ranking-function})
is equivalent to the existence of
four non-negative rational numbers
, \dots,~ such that:


The positivity of the ranking function, that is,

can be written as

By Farkas' Lemma, formula~\eqref{positivity-of-the-ranking-function}
is equivalent to the existence of four other non-negative rational numbers
, \dots,~ such that:


Summarizing, by Farkas Lemma,
formula~\eqref{formule-fnrglin} is equivalent to the conjunction
of formulas~\eqref{fnlindec} and \eqref{fnlinpos}:


In theory, the problem of the existence of a linear ranking function
is polynomial. Since computing one solution (that is, values for
 and ) is not harder than determining its existence,
a ``witness'' function, which would constitute a
\emph{termination certificate}, can also be computed in polynomial time.

The space of all linear ranking functions as defined in Definition \ref{def-fn-rng-lin},
described by parameters  and ,
can be obtained by elimination of  and 
from~\eqref{fnlinrf} using, e.g., the algorithm of Fourier-Motzkin.
For example the SICStus Prolog program
\begin{verbatim}
fm(A, B) :-
   {L1 >= 0, L2 >= 0, L3 >= 0, L4 >= 0,
    LP1 >= 0, LP2 >= 0, LP3 >= 0, LP4 >= 0,
    A = L1 + L2, B = L2 + L3 - L4,
    A = L2, B = L3, 1 =< L3 + L4,
    A = LP1 + LP2, B = LP2 + LP3 - LP4,
    0 = LP2, 0 = LP3, 0 =< LP3 + LP4}.
\end{verbatim}
can be queried as follows:
\begin{verbatim}
| ?- fm(A, B).
B = 0, {A >= 1}.
| ?-
\end{verbatim}
It can be shown that the computed answer is equivalent to the
(significantly more involved) condition generated by Reduce.


\section{Eventual Linear Ranking Functions}
\label{sec:eventual-linear-ranking-functions}

In the previous section we have illustrated a method to decide
the existence of a linear ranking function for a rational SLC loop,
something that implies termination of the loop.
Of course, the method cannot decide termination in all cases.

\begin{example}
\label{ex-fn-rng-lin-evt-p}
The loop

does not admit a linear ranking function.
\end{example}

Can we conclude that such loop does not always terminate?
No, because it may admit a non-linear ranking function.

In this section we will extend the previous method so as to detect
\emph{eventual linear ranking functions}, that is, linear functions
that behave as ranking functions
\emph{after a finite number of executions of the loop body}.
Suppose that the considered SLC loop is always given with
a linear function  that increases at each
iteration of the loop in the following sense:

\begin{definition}
\label{def-inc-lin-fn}
Let  be the SLC loop
 .
A function  is \emph{increasing for }
if it is linear and satisfies:

\end{definition}

\begin{example}
\label{ex-fn-rng-lin-evt-f}
The function  is increasing for the loop of
\textup{Example~\ref{ex-fn-rng-lin-evt-p}}, since  decreases by at
least  at each iteration.
\end{example}

\begin{remark}
The generalization to affine functions  is useless.
Moreover, as we are merely interested in the existence of an
increasing function, the value of the increase ( or )
is irrelevant.
\end{remark}

We can now give the definition which is central to our paper.

\begin{definition}
\label{def-fn-rng-lin-evt}
Let  be the rational SLC loop in clausal form
,
where  is an -ary relation;
let also  be a linear increasing function for .
An \emph{eventual linear ranking function}  for 
is a linear map of  to  such that

\end{definition}
For comparison with Definition~\ref{def-fn-rng-lin},
remark that the threshold  is existentially quantified
and that  is imposed in the implication antecedent.
It should also be noted that, if such a rational  exists,
then each  satisfies the condition
of Definition~\ref{def-fn-rng-lin-evt}.
On the other hand, since, by hypothesis,  strictly increases
at each iteration, there are two cases:
either  is bounded from above by a constant, and thus the loop will
terminate;
or, after a finite number of iterations,  will cross the threshold 
and  becomes a linear ranking function in the sense of
Section~\ref{sec:linear-ranking-functions} so that, again, the loop terminates.

Eventual linear ranking functions are a generalization
of linear ranking functions.

\begin{proposition}
\label{lrf-implies-elrf}
Let  be an SLC loop. If  is  a linear ranking function for ,
then there exists an increasing function  such that 
has an eventual linear ranking function.
\end{proposition}
\begin{proof}
By hypothesis, there exists a linear ranking function
 for .
The linear function 
is non-positive and strictly increasing for .
Considering  it can be seen that the function
 is an eventual linear
ranking function for .
\end{proof}

The generalization is strict as the loop of
Example~\ref{ex-fn-rng-lin-evt-p} has no linear ranking function, but
does have an eventual linear ranking function, as will be shown in the
next section.


\subsection{Detection given a Linear Increasing Function}
\label{sec:elrf-semi-detection}

As a first step towards full automation of the synthesis of eventual
linear ranking functions, we assume that an SLC loop is given with a
particular linear increasing function.
Let us consider, e.g., the SLC loop of
Example~\ref{ex-fn-rng-lin-evt-p} and the increasing
function of Example~\ref{ex-fn-rng-lin-evt-f}.
Defining ,
~is an eventual linear ranking function when

This definition of the problem, that we will denote for brevity with
, is also solvable via quantifier
elimination, hence the problem is decidable.
Considering ,  and  as parameters,
we can apply Farkas' Lemma as follows:

Hence, formula  is equivalent to the conjunction
of formulas , i.e.,

ensuring the positivity of the ranking function.

Let us focus on .
We observe that the product 
leads to a non-linearity that we can circumvent
by noting that, as ,
either  (hence ) or .
In the latter case, we introduce a new variable .
We have the property:
\begin{lemma}
\label{lem1-fn-rng-lin-evt}
Formula  is equivalent to the disjunction
.
\end{lemma}
In our case,  is equivalent to


\begin{proof}
() Let   be a rational number and
's for   four non-negative rational numbers
such that  holds.
If  then   simplifies to 
which is true.
If , we take  and
we can see that  is true.

\noindent
() Assume first that  is true.
Then, taking  and  (any rational number
would be fine for ), we see that  is true.
Assume then that   is true.
Taking  (this is always possible as
), we observe that there exists 
such that   is true.
\end{proof}

For the positivity condition, we can prove in a similar way
\begin{lemma}
\label{lem2-fn-rng-lin-evt}
Formula  is equivalent to the disjunction
.
\end{lemma}
In our case,  is equivalent to


Combining the previous results gives
\begin{proposition}
Formula  is equivalent to
.
\end{proposition}

\begin{proof}
Thanks to the previous lemmata, it only remains to justify the equivalence
between the formulas 
and .

() Let  be a rational such that .
We have  and  because


() Assume the existence of 
such that   and the existence of 
such that .
Then the rational 
verifies 
and shows that .
\end{proof}

Back to our initial problem, the existence of an eventual linear
ranking function is equivalent to the satisfiability of at least one
of the following four linear systems:

which we can decide in polynomial time.
For our running example,
 is satisfiable
as proved by the following SICStus Prolog query:
\begin{verbatim}
?- dec2pos1.
true.
?-
\end{verbatim}
after compilation of the program:
\begin{verbatim}
dec2pos1 :-
   {L1 >= 0, L2 >= 0, L3 >= 0, L4 > 0,
    A = L1 + L2, B = L2 + L3 - L4,
    A = L2, B = L3, 1 =< L3 + P,
    LP1 >= 0, LP2 >= 0, LP3 >= 0,
    A = LP1 + LP2, B = LP2 + LP3,
    0 = LP2, 0 = LP3, 0 =< LP3}.
\end{verbatim}

The procedure we have informally outlined by means of examples is
actually completely general.
It is embodied in Algorithm~\ref{algo1}, which is a (correct and
complete) decision procedure for the existence of an eventual
linear ranking function given a linear increasing function.

\begin{algorithm}
\caption{Existence of an eventual linear ranking function, given a linear increasing function}
\label{algo1}
\begin{algorithmic}[1]
\REQUIRE
, an SLC loop
,
and , a linear increasing function for 
\ENSURE
Returns \TRUE\ if and only if, for some vector ,

is an eventual linear ranking function for .
\STATE
\rho
\STATE
\dec(\mathbf{a}, k)
\STATE
\rho
\STATE
\pos(\mathbf{a}, k)
\IF { is satisfiable}
  \RETURN \TRUE
\ELSE
  \RETURN \FALSE
\ENDIF
\end{algorithmic}
\end{algorithm}

\begin{theorem}
\label{algo-is-a-decision-procedure-for-elrf}
Let  be an SLC loop and  an increasing function for .
\textup{Algorithm~\ref{algo1}} decides in polynomial time the existence
of an eventual linear ranking function for .
\end{theorem}

Computing an eventual linear ranking function 
and its associated threshold  can be done as follows:
\begin{itemize}
\item
if  is satisfiable,
we compute a solution ,
 is a standard linear ranking function
and Proposition~\ref{lrf-implies-elrf} applies;
\item
if  is satisfiable,
we compute a solution , , 
and we take ;
\item
if  is satisfiable,
we compute a solution , , 
and we take ;
\item
if  is satisfiable,
we compute a solution
, , , , 
and we take .
\end{itemize}

\begin{example}
Continuing with \textup{Example~\ref{ex-fn-rng-lin-evt-p}},
here is the most general solution of :
\begin{verbatim}
?-  {L1 >= 0, L2>= 0, L3 >= 0, L4 > 0,
     A = L1 + L2, B = L2 + L3 - L4,
     A = L2, B = L3, 1 =< L3 + P,
     LP1 >= 0, LP2 >= 0, LP3 >= 0,
     A = LP1 + LP2, B = LP2 + LP3,
     0 = LP2, 0 = LP3, 0 =< LP3}.
B = 0, L1 = 0, L3 = 0, LP2 = 0, LP3 = 0,
{LP1 = L4, L2 = L4, A = L4, L4 > 0, P >= 1}.
?-
\end{verbatim}
One particular solution is
,
,
.
Hence  is an eventual linear ranking function
from the threshold .
\end{example}

We also provide a decision procedure for the existence
of an eventual \emph{affine} ranking function.

\begin{corollary}
The existence of an eventual affine ranking function for
an SLC loop and associated increasing function, ,
can be decided in polynomial time.
\end{corollary}
\begin{proof}
From ,
,
we construct ,
 where  does not occur in .
Note that  is an SLC loop and that 
is an increasing function for .
Algorithm~\ref{algo1} applied to 
gives an answer in polynomial time.

If Algorithm~\ref{algo1} returns \textbf{true} then, by correctness,
there exists a threshold  and an eventual linear function
 for .
We readily check that 
is an eventual affine ranking function for  from .

If Algorithm~\ref{algo1} returns \textbf{false} then, by completeness,
there is no eventual linear ranking function for .
Assuming there exists an eventual affine ranking function
 from  for , then
 should be an
eventual linear ranking function from  for ,
which is a contradiction.
Hence there is no eventual affine ranking function for .
\end{proof}

\begin{example}
\label{ex-fn-rng-aff-evt-p}
The SLC loop

associated to the linear increasing function 
does not admit an eventual linear ranking function, but does admit
 as an eventual affine ranking function from .
\end{example}



\subsection{Fully Automated Detection}
\label{sec:elrf-full-detection}

We now consider the problem in its full generality:
given an SLC loop , does there exist an increasing function for 
such that  admits an eventual linear ranking function?

Note that the space of increasing functions can be obtained
as a convex set over their coefficients
via the Farkas' Lemma and existentially quantified variables
elimination.\footnote{See also \cite[Section 4.4]{BagnaraMPZ12IC}.}

\begin{definition}
\label{def-inc}
Let

be an SLC loop.
We denote by  the set of vectors 
such that 
is increasing for .
\end{definition}

\begin{example}
\label{ex-full-detection}
A linear ranking function does not exist for the SLC loop 


induces the space of functions of the form ,
which are increasing for .
\end{example}

Let us consider the SLC loop of Example~\ref{ex-full-detection}
associated to an increasing function 
induced by .
Defining  and
considering  and  as parameters,
 is an eventual linear ranking function when

This definition of the problem is denoted .
We can apply Farkas' Lemma as follows:

Formula  is equivalent to the conjunction
of formulas , i.e.,

ensuring the positivity of the ranking function.

Let us focus on .
We observe that the products with 
lead to a non-linearity that we can circumvent
by noting that, as ,
either  or .
In the latter case, we introduce a vector  of
two new variables where
 and  together
with, as previously, the new variable .
Formula  is equivalent to the disjunction
 where
in our case,  is equivalent to


For the positivity condition,
formula  is equivalent to the disjunction

where we introduce a vector  of
two new variables where
,  together
with, as previously, the new variable .
In our case,  is equivalent to


Back to our initial problem, the existence of an eventual linear
ranking function is equivalent to the satisfiability of at least one
of the following four systems:
\begin{enumerate}
\item
:
this case means that the increasing function
and  are irrelevant. In other words, for each solution ,
 is a standard linear ranking function
and Proposition~\ref{lrf-implies-elrf} applies.

\item
:
note that satisfiability of

is not sufficient, as its solution might lead to the coefficients
 and 
( is strictly positive by definition),
which could correspond to a non-increasing linear function.
The third conjunct, ,
ensures that we stay within the space of increasing functions.

\item
:
this case is symmetric to previous one.

\item
:
this case combines the two previous ones.
Note that the condition ensures that we consider the same linear
ranking function and the same increasing function both in  and
in .
\end{enumerate}

For our running example, the following SICStus Prolog query
proves that

is satisfiable
\begin{verbatim}
?- dec2incpos1.
true.
?-
\end{verbatim}
after compilation of the program
\begin{verbatim}
dec2incpos1 :-
   {L1 >= 0, L2 >= 0, L3 >= 0,
    A1 = L1 + L2 + P1, A1 = L2, L > 0,
    A2 = L2 - L3 + P2, A2 = L3, -1 >= -L3 - P,
P1 =< -2*L, P1 - 2*P2 = 0,
LP1 >= 0, LP2 >= 0, LP3 >= 0,
    A1 = LP1 + LP2, 0 = LP2,
    A2 = LP2 - LP3, 0 = LP3, 0 >= -LP3}.
\end{verbatim}

The procedure we have informally outlined by means of examples is
actually completely general and is embodied in Algorithm~\ref{algo2}.

\begin{algorithm}
\caption{Existence of an eventual linear ranking function}
\label{algo2}
\begin{algorithmic}[1]
\REQUIRE
, an SLC loop

\ENSURE
Returns \TRUE\ if and only if there exists an increasing function 
for  and

such that  is an eventual linear ranking function for .
\STATE
C
\STATE
\rho
\STATE
\dec(\mathbf{a}, k)
\STATE
\rho
\STATE
\pos(\mathbf{a}, k)
\STATE

\STATE
 
 \STATE
 
 \STATE

\IF { is satisfiable}
\RETURN \TRUE
\ELSE  \RETURN \FALSE
\ENDIF
\end{algorithmic}
\end{algorithm}

\begin{theorem}
\label{algo2-is-a-decision-procedure-for-elrf}
Let  be an SLC loop.
\textup{Algorithm~\ref{algo2}} decides the existence of an increasing
function  and a linear function  such that  is an
eventual linear ranking function for .
\end{theorem}

Exactly as in the previous section, if Algorithm~\ref{algo2}
returns \textbf{true} then we can extract an increasing function ,
a threshold , and a linear function .
We can also generalize the approach to the fully automated
detection of eventual affine ranking functions.

With respect to complexity,
Algorithm~\ref{algo2}  is not polynomial for two reasons.
In step 1, computing the set  of linear increasing functions for 
requires elimination of existentially quantified variables.
In step 2, formula   leads to a non-linear system
and we may have to check its satisfiability in step 10.
Although decidable, we are not aware of the existence of
polynomial algorithms for these problems.

\subsection{Verification}
\label{sec:verification}

Given  an SLC loop, an associated increasing
function , and a linear function , we want to know whether
 is a ranking function.
We can run Algorithm~\ref{algo1}, with the coefficients 
fully instantiated. If needed, we can compute the threshold 
as explained in Section~\ref{sec:elrf-semi-detection}.
It follows that the verification problem is polynomial.

\subsection{Implementation}
\label{sec:implementation}

We  have implemented both algorithms in SICStus Prolog.
However, as  of Algorithm~\ref{algo2}
leads to a non-linear system, we relaxed this formula to

which is now linear. As shown in the following proposition, the existence
of an eventual linear ranking function (hence termination) is preserved,
but the associated increasing function is not linear.

\begin{proposition}
Let  be an SLC loop and assume that

is true.
Then there exists a non-linear increasing function  such that

is an eventual linear ranking function for (,).
\end{proposition}

\begin{proof}
As 
is true, there exists an increasing function  and a rational
 such that when the value of  is beyond , 
decreases.
Similarly, as

is true, there exists an increasing function  and a rational
 such that when the value of  is beyond ,  is
non-negative.
Let  and
.
One readily checks that  is a non-linear increasing function for 
and  is an eventual linear ranking function for .
\end{proof}


\section{Related Work and Experiments }
\label{sec:related-work-and-experiments}

As eventual linear ranking functions generalize linear ranking
functions, we focus on related work that goes beyond linear ranking
functions for SLC loops.
In order to appreciate the relative power of the different methods,
we report on the results obtained with  our algorithms on the loops
discussed in the papers where the other approaches were introduced.

The method proposed in \cite{ChenFM12} repeatedly divides the state
space to find a linear ranking function on each subspace, and then
checks that the transitive closure of the transition relation is
included in the union of the ranking relations.
As the process may not terminate, one needs to bound the search.
\cite{ChenFM12} also proposes a test suite, upon which we tested
our approach.
As expected, every loop \cite[Table~1]{ChenFM12} which terminates with
a linear ranking also has an eventual linear ranking.
Moreover, loops 6, 12, 13, 18, 21, 23, 24, 26, 27, 28, 31, 32, 35, and~36
admit an eventual linear ranking function (which is discovered without
using neither  nor its relaxation).
These are all shown terminating with the tool of \cite{ChenFM12}.
On the other hand, loops 14, 34, and 38 do have a
\emph{disjunctive ranking function}
(following the terminology of \cite{ChenFM12}),
but do not admit an eventual linear ranking function.

\cite{GantyG13} shows how to partition the loop relation into
behaviors that terminate and behaviors to be analyzed in a subsequent
termination proof after refinement.
This work addresses both termination  and  conditional termination problems
in the same framework.
Concerning the benchmarks proposed in \cite[Table~1]{GantyG13},
loops 6--41 all have an eventually linear ranking function except for
loops 11, 14, 30, 34, and~38.

A method based on abstract interpretation for synthesizing
ranking functions is described in \cite{Urban13}.
Although the work contains no completeness result, the approach
is able to discover piecewise-defined ranking functions.

Finally, let us point out that the concept of \emph{eventual termination}
appeared first in~\cite{BradleyMS05ICALP,BradleyMS05VMCAI}.
The class loops studied in these works is wider but,
as the technique of~\cite{BradleyMS05VMCAI} relies on finite differences,
this approach is incomplete.
On the other hand, while \cite{BradleyMS05ICALP} is also based
on Farkas' Lemma, it seems
[A.~R.~Bradley, Personal communication, May 2013]
that the \emph{polyranking} approach cannot
prove, e.g., termination of the SLC loop
, which admits
an eventual linear ranking function.


\section{Conclusion and Future Work}
\label{sec:conclusion-and-future-work}

We have proposed a definition of eventual linear ranking
function for SLC loops that strictly generalizes the concept
of linear ranking function.
We also defined two correct and complete algorithms for detecting such
ranking functions under different hypotheses.
The first algorithm shows that the mere knowledge
of the right increasing function allows checking the existence
or even synthesizing an eventual linear ranking function in polynomial time.
The second algorithm decides the existence of an
eventual linear ranking function in its full generality but
is not polynomial. We have also explained how to extend the algorithms
for deciding eventual affine ranking functions.
The algorithms admit a simple formulation as a constraint logic
program and have been fully implemented in SICStus Prolog inside the
BinTerm termination prover~\cite{SpotoMP10}.

It has to be noted that a nice property of the notion of eventual
(not necessarily linear) ranking function is its simplicity.
This is important when functions that witness termination have to be
provided (and/or understood) by humans.  This is the case when
annotating a C/ACSL program with loop variants \cite{BaudinCFMM+13}:
for the cases when a ranking function to be specified in a
\verb+loop variant+ clause is not obvious, one could extend ACSL with a
\verb+loop prevariant+ clause that allows the annotator
to indicate a candidate increasing function.
In the linear case, our first algorithm can efficiently decide
whether the two clauses constitute a termination witness.

On the other hand, there obviously are, as indicated
in Section~\ref{sec:related-work-and-experiments},
more complex classes of ranking functions and algorithms that allow
to establish the termination of SLC loops that do not admit an
eventual linear ranking functions.
A proper assessment of the relative merits of these approaches,
all extremely recent, requires an extensive experimental evaluation
that is one of our objectives for future work.

The verification of linear ranking functions for integer SLC loops, i.e.,
checking the satisfiability of

and ,
is an -complete problem.
Concerning the existence of linear ranking functions,
as the Farkas' Lemma is not true for the integers,
the method presented in Section~\ref{sec:linear-ranking-functions} is not
valid.
The problem, which has been solved very recently in \cite{Ben-AmramG13},
is -complete, and the paper proposes an exponential-time algorithm.
Extending the present approach to integer SLC loops is another
interesting idea to consider for future work.

\paragraph{Acknowledgments}
We are grateful to Anthony Alezan, Aaron R. Bradley, \'Etienne Payet,
and some anonymous referees for their helpful comments.



\begin{thebibliography}{10}

\bibitem{AlbertAGGPRRZ09}
E.~Albert, P.~Arenas, S.~Genaim, M.~G{\'o}mez-Zamalloa, G.~Puebla, D.~V.
  Ram\'{\i}rez, G.~Rom{\'a}n, and D.~Zanardini.
\newblock Termination and cost analysis with {COSTA} and its user interfaces.
\newblock {\em Electronic Notes in Theoretical Computer Science},
  258(1):109--121, 2009.

\bibitem{BagnaraMPZ12IC}
R.~Bagnara, F.~Mesnard, A.~Pescetti, and E.~Zaffanella.
\newblock A new look at the automatic synthesis of linear ranking functions.
\newblock {\em Information and Computation}, 215:47--67, 2012.

\bibitem{BaudinCFMM+13}
P.~Baudin, P.~Cuoq, J.-C. Filli\^{a}tre, C.~March\'{e}, B.~Monate, Y.~Moy, and
  V.~Prevosto.
\newblock {\em {ACSL}: {ANSI/ISO} {C} Specification Language}.
\newblock CEA LIST and INRIA, 1.7 edition, 2013.

\bibitem{Ben-AmramG13TR}
A.~M. Ben-Amram and S.~Genaim.
\newblock On the linear ranking problem for integer linear-constraint loops.
\newblock Technical Report arXiv:1208.4041v2 [cs.PL], 2013.
\newblock Available from \url{http://arxiv.org/}.

\bibitem{Ben-AmramG13}
A.~M. Ben-Amram and S.~Genaim.
\newblock On the linear ranking problem for integer linear-constraint loops.
\newblock In R.~Giacobazzi and R.~Cousot, editors, {\em Proceedings of the 40th
  Annual ACM SIGPLAN-SIGACT Symposium on Principles of Programming Languages
  (POPL '13)}, pages 51--62, Rome, Italy, 2013. Association for Computing
  Machinery.

\bibitem{Ben-AmramGM12}
A.~M. Ben-Amram, S.~Genaim, and A.~N. Masud.
\newblock On the termination of integer loops.
\newblock {\em ACM Transactions on Programming Languages and Systems},
  34(4):16:1--16:24, 2012.

\bibitem{BozgaIK12}
M.~Bozga, R.~Iosif, and F.~Konecn{\'y}.
\newblock Deciding conditional termination.
\newblock In C.~Flanagan and B.~K{\"o}nig, editors, {\em Tools and Algorithms
  for the Construction and Analysis of Systems: Proceedings of the 18th
  International Conference (TACAS 2012)}, volume 7214 of {\em Lecture Notes in
  Computer Science}, pages 252--266, Tallinn, Estonia, 2012. Springer.

\bibitem{BradleyMS05ICALP}
A.~R. Bradley, Z.~Manna, and H.~B. Sipma.
\newblock The polyranking principle.
\newblock In L.~Caires, G.~F. Italiano, L.~Monteiro, C.~Palamidessi, and
  M.~Yung, editors, {\em Automata, Languages and Programming: Proceedings of
  the 32nd International Colloquium (ICALP 2005)}, volume 3580 of {\em Lecture
  Notes in Computer Science}, pages 1349--1361, Lisbon, Portugal, 2005.
  Springer.

\bibitem{BradleyMS05VMCAI}
A.~R. Bradley, Z.~Manna, and H.~B. Sipma.
\newblock Termination of polynomial programs.
\newblock In R.~Cousot, editor, {\em Verification, Model Checking and Abstract
  Interpretation: Proceedings of the 6th International Conference (VMCAI
  2005)}, volume 3385 of {\em Lecture Notes in Computer Science}, pages
  113--129, Paris, France, 2005. Springer-Verlag, Berlin.

\bibitem{Braverman06}
M.~Braverman.
\newblock Termination of integer linear programs.
\newblock In T.~Ball and R.~B. Jones, editors, {\em Computer Aided
  Verification: Proceedings of the 18th International Conference (CAV 2006)},
  volume 4144 of {\em Lecture Notes in Computer Science}, pages 372--385,
  Seattle, WA, USA, 2006. Springer.

\bibitem{CookPR06}
B.~Cook, A.~Podelski, and A.~Rybalchenko.
\newblock Termination proofs for systems code.
\newblock In M.~I. Schwartzbach and T.~Ball, editors, {\em Proceedings of the
  ACM SIGPLAN 2006 Conference on Programming Language Design and
  Implementation}, pages 415--426, Ottawa, Ontario, Canada, 2006. Association
  for Computing Machinery.

\bibitem{GantyG13}
P.~Ganty and S.~Genaim.
\newblock Proving termination starting from the end.
\newblock Technical Report~{\tt arXiv:abs/1302.4539}, 2013.
\newblock A version of this paper is due to appear in the proceedings of CAV
  2013.

\bibitem{Hearn05}
A.~C. Hearn.
\newblock {REDUCE:} the first forty years.
\newblock In A.~Dolzmann, A.~Seidl, and T.~Sturm, editors, {\em Algorithmic
  Algebra and Logic: Proceedings of the A3L 2005 Conference in Honor of the
  60th Birthday of Volker Weispfenning}, pages 19--24, Passau, Germany, 2005.

\bibitem{Holzbaur95}
C.~Holzbaur.
\newblock {\em {OFAI} clp({Q},{R})}.
\newblock Austrian Research Institute for Artificial Intelligence, Vienna,
  1.3.3 edition, 1995.
\newblock Published as TR-95-09.

\bibitem{OttoBEG10}
C.~Otto, M.~Brockschmidt, C.~{von Essen}, and J.~Giesl.
\newblock Automated termination analysis of {Java} bytecode by term rewriting.
\newblock In C.~Lynch, editor, {\em Proceedings of the 21st International
  Conference on Rewriting Techniques and Applications (RTA 2010)}, volume~6 of
  {\em Leibniz International Proceedings in Informatics (LIPIcs)}, pages
  259--276, Edinburgh, Scotland, UK, 2010. Schloss Dagstuhl--Leibniz-Zentrum
  fuer Informatik.

\bibitem{PodelskiR04}
A.~Podelski and A.~Rybalchenko.
\newblock A complete method for the synthesis of linear ranking functions.
\newblock In B.~Steffen and G.~Levi, editors, {\em Verification, Model Checking
  and Abstract Interpretation: Proceedings of the 5th International Conference
  (VMCAI 2004)}, volume 2937 of {\em Lecture Notes in Computer Science}, pages
  239--251, Venice, Italy, 2004. Springer.

\bibitem{SohnVG91}
K.~Sohn and A.~{Van Gelder}.
\newblock Termination detection in logic programs using argument sizes
  (extended abstract).
\newblock In D.~J. Rosenkrantz, editor, {\em Proceedings of the Tenth {ACM}
  {SIGACT-SIGMOD-SIGART} Symposium on Principles of Database Systems}, pages
  216--226, Denver, CO, USA, 1991. Association for Computing Machinery.

\bibitem{SpotoMP10}
F.~Spoto, F.~Mesnard, and {\'E}.~Payet.
\newblock A termination analyzer for {Java} bytecode based on path-length.
\newblock {\em ACM Transactions on Programming Languages and Systems}, 32(3),
  2010.

\bibitem{Tiwari04}
A.~Tiwari.
\newblock Termination of linear programs.
\newblock In R.~Alur and D.~Peled, editors, {\em Computer Aided Verification:
  Proceedings of the 16th International Conference (CAV 2004)}, volume 3114 of
  {\em Lecture Notes in Computer Science}, pages 70--82, Boston, MA, USA, 2004.
  Springer.

\bibitem{Urban13}
C.~Urban.
\newblock The abstract domain of segmented ranking functions.
\newblock In F.~Logozzo and M.~Fahndrich, editors, {\em Proceedings of the 20th
  International Symposium on Static Analysis (SAS 2013)}, Lecture Notes in
  Computer Science, Seattle, WA, USA, 2013. Springer.
\newblock To appear.

\bibitem{ChenFM12}
H.~{Yi Chen}, S.~Flur, and S.~Mukhopadhyay.
\newblock Termination proofs for linear simple loops.
\newblock In A.~Min{\'e} and D.~Schmidt, editors, {\em Proceedings of the 19th
  International Symposium on Static Analysis (SAS 2012)}, volume 7460 of {\em
  Lecture Notes in Computer Science}, pages 422--438, Deauville, France, 2012.
  Springer.

\end{thebibliography}

\end{document}
