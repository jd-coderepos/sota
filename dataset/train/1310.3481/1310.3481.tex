



\appendix

 \section{Proof of Theorem~\ref{thm:sound_abstraction}}
 \begin{proof}
   We begin with two general lemmas concerning interprocedural
   interpretations.  The main result is then a simple application of these two
   lemmas.
   \begin{lemma} \label{lem:pathexp_rel}
     Let  and
      be \QPKA{}
     interpretations, let  and  be summary
     assignments, and let  be
     any relation that is coherent with respect to the \QPKA{} operators and
     such that for all , 
     
     Then for any path expression ,
     

     Moreover, this holds even if  and  do not
     satisfy any of the \QPKA{} axioms.
   \end{lemma}
   \begin{proof}
     By induction on .
   \end{proof}
   \begin{lemma} \label{lem:analysis_rel}
     Let  and
      be \QPKA{}
     interpretations, and let  be any relation that is coherent with respect to the
     \QPKA{} operators and such that for all ,
     
     where  and  are
     inductive summary assignments (as defined in
     Section~\ref{sec:interproc}). Then for any vertex ,
     

     Moreover, this holds even if  and  do not
     satisfy any of the \QPKA{} axioms.
   \end{lemma}
   \begin{proof}
     This is essentially a corollary of Lemma~\ref{lem:pathexp_rel}.  Recall
     that
     
     
   So we need only to prove that
   
   and 
   
   The latter follows directly from Lemma~\ref{lem:pathexp_rel}.  It remains
   to show the former.

   Recall that the interpretation of a call graph edge 
   is defined in terms of the interpretations of the path expression
    describing paths to a vertex  that calls .
   It follows from Lemma~\ref{lem:pathexp_rel} that we must have
   
   for all edges  in the call graph.  We may then prove by
   induction on  that
   
   \end{proof}
   We can prove that for all ,
    in the
   standard way, by induction on the transfinite iteration sequence defining
    (using Lemma~\ref{lem:pathexp_rel} for the
   induction step).  We then have the main result by applying
   Lemma~\ref{lem:analysis_rel}.
 \end{proof}










   






































\section{Coincidence}

 We now consider the question of when the algorithm presented in
 Section~\ref{sec:inter_analysis} computes the ideal (merge-over-paths)
 solution to an analysis problem.  This is analogous to the development of
 \emph{coincidence theorems} \cite{Kam1977,Sharir1981,Knoop1992b,Lal2005} in
 the field of dataflow analysis, which state conditions under which the
 iterative algorithm for solving dataflow analyses computes the ideal
 solutions (i.e., when the join-over-paths solution to a dataflow analysis
 problem coincides with the minimum fixpoint solution\footnote{In dataflow
   analysis literature, coincidence theorems typically relate the
   meet-over-paths and the maximum fixpoint solutions.  We follow in the
   tradition of abstract interpretation, which uses the dual order.}).  In this
 section, we formulate and prove a coincidence theorem for our interprocedural
 framework.

 The first step towards formulating our coincidence theorem is to define the
 join-over-paths solution to an analysis.  This requires us to define what a
 path is, and how to interpret it.  Our treatment of interprocedural paths in
 Section~\ref{sec:inter_analysis} is not adequate for this purpose, because it
 relies on interpreting call edges using procedure summaries (which represent
 a set of paths, rather than a single path).  In this section, we will take
 interprocedural paths to be words over an alphabet of control flow graph
 edges such that each return is properly matched with a call.  For a formal
 definition of such a path, the reader may consult \cite{Sharir1981}.  We use
 \textsf{IPaths(v)} to denote the set of all interprocedural paths to .

 Now we must define the interpretation of an interprocedural path.  Let
  be an
 interpretation.  We define the interpretation of a path within 
 as a function that operates on \emph{stacks} of abstract values (as in
 \cite{Knoop1992b,Lal2005}).  We use the notation  to denote a stack of 
 \emph{activation records}, where an activation record is a tuple  consisting of a set of variables  that are local to
 that activation record, and an abstract value .  The first entry in this
 list corresponds to the top of the stack.  We define our semantic function on
 edges so that each call edge corresponds to a push on this stack, each return
 edge corresponds to a pop, and each intraprocedural edge modifies the top of
 the stack (i.e., the ``current'' activation record).  Formally, we have

  \noindent

   \hfill

   We may then associate with every path  a stack of activation records
    in the following way:
   

   Next, we define a  function that lowers a stack of
   activation records into a single abstract value, and existentially
   quantifies variables that are not in the scope of the current activation
   record.
   

   Finally, we take the interpretation of an interprocedural path within
    to be .


   Recalling the definition of  from
   Section~\ref{sec:inter_analysis}, we may state our coincidence theorem:
   \begin{theorem}[Coincidence] \label{thm:coincidence}
     Let  be an
     interpretation, where  is a quantale.  Then for any ,
     
   \end{theorem}
   \begin{proof}
     The proof proceeds in two steps.  First, we define a \emph{quantified
       word} interpretation , for which we can show a coincidence
     result.  Then, we show that any interpretation  over a quantale
     ``factors'' through the quantified word interpretation, in the sense that
     there is a quantale homomorphism  such that
     

     We first define the set of quantified words  to be the empty word or
     any word recognized by the following context free grammar:
     
     We then define a (weak) quantale
     
     where
     
     Note that  does not satisfy axioms (Q2)-(Q4).  It is for
     this reason that we call  a weak quantale.  However, the
     axioms (Q2)-(Q4) are irrelevant for this proof.

     We define the quantified word interpretation as , where
      is defined as the word consisting of a single
     letter .

     Next, we define a function  by
     
     where
     

     We will use the subscripts  and  on
      and  to distinguish between which
     interpretation it is being applied in.

     \begin{lemma} \label{lem:flatten}
       For any interprocedural path ,
       
     \end{lemma}
     \begin{proof}
       By induction on .
     \end{proof}

     \begin{lemma} \label{lem:factor}
       For any ,
       
     \end{lemma}
     \begin{proof}
       Define a relation  as
       the graph of the function .  That is,
       
       It is easy to show that  is a congruence w.r.t. the \QPKA{}
       operations.

       Next, we must show that for any ,
       .
       We may prove this by induction on the transfinite iteration
       sequence defining  and
        (using Lemma~\ref{lem:pathexp_rel} for the
       induction step).

       Finally, we may apply Lemma~\ref{lem:analysis_rel} to get
       
       whence .
     \end{proof}

     It is easy to see that  is exactly the set of
     interprocedural paths to , where calls and returns have been replaced
     by existential quantification of the appropriate variables.  This can be
     expressed formally as
     
     Or equivalently, as
     

     We may then conclude
     
   \end{proof}











\section{Experimental results}

We present a detailed comparison of {\sc LRA}, {\sc UFO}, and {\sc InvGen} on
our benchmark suite.  This includes result only for those benchmarks where one
of the answers among the three tools differed from the other two.
\begin{figure}
\begin{tabular}{|l|c|c|c|} \hline

 Benchmark Name           &  {\sc LRA}    & {\sc UFO}     & {\sc InvGen}                \\ \hline \hline

 apache-escape-absolute   & unsafe  & unsafe  & safe                  \\
 apache-get-tag           & safe    & unsafe  & safe                  \\
 gulwani\_cegar2           & safe    & safe    & unsafe                \\
 gulwani\_fig1a            & safe    & safe    & unsafe                \\
 half                     & safe    & unsafe  & unsafe                \\
 heapsort                 & safe    & unsafe  & safe                  \\
 heapsort2                & safe    & unsafe  & safe                  \\
 heapsort3                & safe    & unsafe  & safe                  \\
 id\_trans                 & unsafe  & unsafe  & safe (false negative) \\
 large\_const              & safe    & timeout & safe                  \\
 MADWiFi-encode\_ie\_ok     & safe    & unsafe  & safe                  \\
 nest-if6                 & safe    & safe    & unsafe                \\
 nest-if8                 & safe    & unsafe  & safe                  \\
 nested6                  & safe    & unsafe  & safe                  \\
 nested7                  & safe    & crash   & unsafe                \\
 nested8                  & safe    & timeout & safe                  \\
 nested9                  & safe    & unsafe  & unsafe                \\
 rajamani\_1               & safe    & timeout & safe                  \\
 sendmail-close-angle     & safe    & unsafe  & safe                  \\
 seq-len                  & safe    & safe    & unsafe                \\
 seq2                     & safe    & crash   & safe                  \\
 split                    & unsafe  & timeout & safe                  \\
 svd1                     & safe    & crash   & safe                  \\
 svd2                     & safe    & crash   & safe                  \\
 svd3                     & safe    & crash   & safe                  \\
 svd4                     & unsafe  & crash   & safe                  \\
 up-nd                    & safe    & unsafe  & safe                  \\
 up2                      & safe    & unsafe  & safe                  \\
 up3                      & safe    & unsafe  & safe                  \\
 up4                      & safe    & unsafe  & safe                  \\
 up5                      & safe    & unsafe  & safe                  \\ \hline \hline

 for\_infinite\_loop\_1\_safe & safe    & safe    & crash                 \\
 for\_infinite\_loop\_2\_safe & safe    & safe    & crash                 \\
 sum01\_safe               & safe    & unsafe  & unsafe                \\
 sum02\_safe               & unsafe  & unsafe  & unsafe                \\
 terminator\_02\_safe       & safe    & safe    & crash                 \\
 token\_ring01\_safe        & unsafe  & safe    & --                    \\
 toy\_safe                 & timeout & timeout & --                    \\ \hline \hline 

 Wrong Results (total)     & 6       & 29      & 14                    \\ \hline

\end{tabular}
\caption{Experimental Results of comparing LRA analysis with UFO and InvGen tools on 
the set of InvGen benchmarks, and the set of SVComp benchmarks. ``--'' indicates that the
benchmark has function calls, which not currently supported by {\sc InvGen}. The last row indicates
how many wrong results each tool had (incorrect result, timeout, or crash) over the entire set of benchmarks. }
\end{figure}
