



\appendix

 \section{Proof of Theorem~\ref{thm:sound_abstraction}}
 \begin{proof}
   We begin with two general lemmas concerning interprocedural
   interpretations.  The main result is then a simple application of these two
   lemmas.
   \begin{lemma} \label{lem:pathexp_rel}
     Let $\mathscr{I} = \tuple{\mathcal{I}, \sem{\cdot}_{\mathcal{I}}}$ and
     $\mathscr{J} = \tuple{\mathcal{J},\sem{\cdot}_{\mathcal{J}}}$ be \QPKA{}
     interpretations, let $S_{\mathcal{I}}$ and $S_{\mathcal{J}}$ be summary
     assignments, and let $\soundrel \subseteq \mathcal{I} \times \mathcal{J}$ be
     any relation that is coherent with respect to the \QPKA{} operators and
     such that for all $e \in E$, 
     \[ \sound{\mathscr{I}(S_{\mathcal{I}})[e]}{\mathscr{J}(S_\mathcal{J})[e]} \]
     Then for any path expression $p$,
     \[ \sound{\mathscr{I}(S_{\mathcal{I}})[p]}{\mathscr{J}(S_\mathcal{J})[p]} \]

     Moreover, this holds even if $\mathcal{I}$ and $\mathcal{J}$ do not
     satisfy any of the \QPKA{} axioms.
   \end{lemma}
   \begin{proof}
     By induction on $p$.
   \end{proof}
   \begin{lemma} \label{lem:analysis_rel}
     Let $\mathscr{I} = \tuple{\mathcal{I}, \sem{\cdot}_{\mathcal{I}}}$ and
     $\mathscr{J} = \tuple{\mathcal{J},\sem{\cdot}_{\mathcal{J}}}$ be \QPKA{}
     interpretations, and let $\soundrel \subseteq \mathcal{I}
     \times \mathcal{J}$ be any relation that is coherent with respect to the
     \QPKA{} operators and such that for all $e \in E$,
     \[ \sound{\mathscr{I}(\overline{S}_{\mathcal{I}})[e]}{\mathscr{J}(\overline{S}_\mathcal{J})[e]} \]
     where $\overline{S}_{\mathcal{I}}$ and $\overline{S}_{\mathcal{J}}$ are
     inductive summary assignments (as defined in
     Section~\ref{sec:interproc}). Then for any vertex $v$,
     \[ \sound{\mathscr{I}\tuple{v}}{\mathscr{J}\tuple{v}} \]

     Moreover, this holds even if $\mathcal{I}$ and $\mathcal{J}$ do not
     satisfy any of the \QPKA{} axioms.
   \end{lemma}
   \begin{proof}
     This is essentially a corollary of Lemma~\ref{lem:pathexp_rel}.  Recall
     that
     \[ \mathscr{I}\tuple{v} = \mathscr{I}(\overline{S}_{\mathcal{I}})[\pathexp{P_1}{P_k}] \mul_\mathcal{I} \mathscr{I}(\overline{S}_{\mathcal{I}})[\pathexp{\entry_k}{v}] \]
     \[ \mathscr{J}\tuple{v} = \mathscr{J}(\overline{S}_{\mathcal{J}})[\pathexp{P_1}{P_k}] \mul_\mathcal{J} \mathscr{J}(\overline{S}_{\mathcal{J}})[\pathexp{\entry_k}{v}] \]
   So we need only to prove that
   \[ \sound{\mathscr{I}(\overline{S}_{\mathcal{I}})[\pathexp{P_1}{P_k}]}{\mathscr{J}(\overline{S}_{\mathcal{J}})[\pathexp{P_1}{P_k}]} \]
   and 
   \[ \sound{\mathscr{I}(\overline{S}_{\mathcal{I}})[\pathexp{\entry_k}{v}]}{\mathscr{J}(\overline{S}_{\mathcal{J}})[\pathexp{\entry_k}{v}]} \]
   The latter follows directly from Lemma~\ref{lem:pathexp_rel}.  It remains
   to show the former.

   Recall that the interpretation of a call graph edge $P_i \rightarrow P_j$
   is defined in terms of the interpretations of the path expression
   $\pathexp{\entry_i}{c}$ describing paths to a vertex $c$ that calls $j$.
   It follows from Lemma~\ref{lem:pathexp_rel} that we must have
   \[ \sound{\mathscr{I}(\overline{S}_{\mathcal{I}})[P_i \rightarrow P_j]}{\mathscr{J}(\overline{S}_{\mathcal{J}})[P_i \rightarrow P_j]} \]
   for all edges $P_i \rightarrow P_j$ in the call graph.  We may then prove by
   induction on $\pathexp{P_1}{P_k}$ that
   \[ \sound{\mathscr{I}(\overline{S}_{\mathcal{I}})[\pathexp{P_1}{P_k}]}{\mathscr{J}(\overline{S}_{\mathcal{J}})[\pathexp{P_1}{P_k}]} \]
   \end{proof}
   We can prove that for all $i$,
   $\sound{\overline{S}_{\mathcal{C}}(i)}{\overline{S}_\mathcal{A}(i)}$ in the
   standard way, by induction on the transfinite iteration sequence defining
   $\overline{S}_{\mathcal{C}}$ (using Lemma~\ref{lem:pathexp_rel} for the
   induction step).  We then have the main result by applying
   Lemma~\ref{lem:analysis_rel}.
 \end{proof}










   






































\section{Coincidence}

 We now consider the question of when the algorithm presented in
 Section~\ref{sec:inter_analysis} computes the ideal (merge-over-paths)
 solution to an analysis problem.  This is analogous to the development of
 \emph{coincidence theorems} \cite{Kam1977,Sharir1981,Knoop1992b,Lal2005} in
 the field of dataflow analysis, which state conditions under which the
 iterative algorithm for solving dataflow analyses computes the ideal
 solutions (i.e., when the join-over-paths solution to a dataflow analysis
 problem coincides with the minimum fixpoint solution\footnote{In dataflow
   analysis literature, coincidence theorems typically relate the
   meet-over-paths and the maximum fixpoint solutions.  We follow in the
   tradition of abstract interpretation, which uses the dual order.}).  In this
 section, we formulate and prove a coincidence theorem for our interprocedural
 framework.

 The first step towards formulating our coincidence theorem is to define the
 join-over-paths solution to an analysis.  This requires us to define what a
 path is, and how to interpret it.  Our treatment of interprocedural paths in
 Section~\ref{sec:inter_analysis} is not adequate for this purpose, because it
 relies on interpreting call edges using procedure summaries (which represent
 a set of paths, rather than a single path).  In this section, we will take
 interprocedural paths to be words over an alphabet of control flow graph
 edges such that each return is properly matched with a call.  For a formal
 definition of such a path, the reader may consult \cite{Sharir1981}.  We use
 \textsf{IPaths(v)} to denote the set of all interprocedural paths to $v$.

 Now we must define the interpretation of an interprocedural path.  Let
 $\mathscr{I} = \langle \mathcal{D}, \sem{\cdot} \rangle$ be an
 interpretation.  We define the interpretation of a path within $\mathscr{I}$
 as a function that operates on \emph{stacks} of abstract values (as in
 \cite{Knoop1992b,Lal2005}).  We use the notation $[\langle V_1, a_1 \rangle,
   \ldots, \langle V_n, a_n \rangle]$ to denote a stack of $n$
 \emph{activation records}, where an activation record is a tuple $\langle
 V_i, a_i \rangle$ consisting of a set of variables $V_i$ that are local to
 that activation record, and an abstract value $a_i$.  The first entry in this
 list corresponds to the top of the stack.  We define our semantic function on
 edges so that each call edge corresponds to a push on this stack, each return
 edge corresponds to a pop, and each intraprocedural edge modifies the top of
 the stack (i.e., the ``current'' activation record).  Formally, we have

  \noindent$\widehat{\sem{e}}([\langle V_1, a_1 \rangle, \ldots, \langle V_n, a_n \rangle]) =$

   \hfill$\begin{cases}
     [\langle LV_i, 1 \rangle,\langle V_1, a_1 \rangle,\ldots,\langle V_n, a_n \rangle] & \text{if } \act(e) = \texttt{call } i\\
     [\langle V_2, a_2 \mul (\exists V_1.a_1) \rangle, \ldots, \langle V_n, a_n \rangle] & \text{if } \act(e) = \texttt{return}\\
     [\langle V_1, a_1 \mul \sem{e} \rangle, \ldots, \langle V_n, a_n \rangle] & \text{otherwise}
   \end{cases}$

   We may then associate with every path $\pi$ a stack of activation records
   $S(\pi)$ in the following way:
   \begin{eqnarray*}
     \textit{stack}(\epsilon) &=& [\langle LV_1, 1 \rangle]\\
     \textit{stack}(\pi a) &=& \widehat{\sem{a}}(\textit{stack}(\pi))
   \end{eqnarray*}

   Next, we define a $\textit{flatten}$ function that lowers a stack of
   activation records into a single abstract value, and existentially
   quantifies variables that are not in the scope of the current activation
   record.
   \[\textit{flatten}([\langle V_1, a_1 \rangle, \ldots, \langle V_n,
     a_n \rangle]) = (\exists V_n. a_n)\mul \dotsi \mul (\exists V_2. a_2)
   \mul a_1\]

   Finally, we take the interpretation of an interprocedural path within
   $\mathscr{I}$ to be $\textit{flatten}(\textit{stack}(\pi))$.


   Recalling the definition of $\mathscr{I}\langle v \rangle$ from
   Section~\ref{sec:inter_analysis}, we may state our coincidence theorem:
   \begin{theorem}[Coincidence] \label{thm:coincidence}
     Let $\mathscr{I} = \langle \mathcal{Q}, \sem{\cdot} \rangle$ be an
     interpretation, where $\mathcal{Q}$ is a quantale.  Then for any $v$,
     \begin{equation*}
       \mathscr{I}\langle v \rangle = \bigoplus_{\pi \in \mathsf{IPaths}(v)} \textit{flatten}(\textit{stack}(\pi))
     \end{equation*}
   \end{theorem}
   \begin{proof}
     The proof proceeds in two steps.  First, we define a \emph{quantified
       word} interpretation $\mathscr{W} = \tuple{\mathcal{W},
       \sem{\cdot}_{\mathcal{W}}}$, for which we can show a coincidence
     result.  Then, we show that any interpretation $\mathscr{I} =
     \tuple{\mathcal{Q}, \sem{\cdot}_{\mathcal{Q}}}$ over a quantale
     ``factors'' through the quantified word interpretation, in the sense that
     there is a quantale homomorphism $h$ such that
     \[\mathscr{I}\tuple{v} =
     h(\mathscr{W}\tuple{v})\]

     We first define the set of quantified words $W$ to be the empty word or
     any word recognized by the following context free grammar:
     \begin{align*}
       W ::= & e & e \in E\\
       | & (\exists x. W) & x \in \Var\\
       | & W W
     \end{align*}
     We then define a (weak) quantale
     \[ \mathcal{W} = \tuple{\powerset{W}, \cup, \mul_{\mathcal{W}}, \star_{\mathcal{W}}, \exists_{\mathcal{W}}, \emptyset, \{\epsilon\}} \]
     where
     \begin{align*}
       P\mul_{\mathcal{W}} Q &= \{ ab : a \in P, b \in Q \}\\
       P^{\star_{\mathcal{W}}} &= \bigcup_{n \in \mathbb{N}} P^n\\
         \exists x. P &= \{ (\exists x. a) : a \in P \}
     \end{align*}
     Note that $\mathcal{W}$ does not satisfy axioms (Q2)-(Q4).  It is for
     this reason that we call $\mathcal{W}$ a weak quantale.  However, the
     axioms (Q2)-(Q4) are irrelevant for this proof.

     We define the quantified word interpretation as $\mathscr{W} =
     \tuple{\mathcal{W}, \sem{\cdot}_{\mathcal{W}}}$, where
     $\sem{e}_{\mathcal{W}}$ is defined as the word consisting of a single
     letter $e$.

     Next, we define a function $h : \mathcal{W} \rightarrow \mathcal{Q}$ by
     \[ h(P) = \bigoplus_{\mathcal{Q}} \{ \tilde{h}(a) : a \in P \} \]
     where
     \begin{align*}
       \tilde{h}(e) &= \sem{e}_{\mathcal{Q}}\\
       \tilde{h}(ab) &= \tilde{h}(a) \mul_{\mathcal{Q}} \tilde{h}(b)\\
       \tilde{h}(\exists x.a) &= \exists_{\mathcal{Q}} x. \tilde{h}(a)
     \end{align*}

     We will use the subscripts $\mathcal{Q}$ and $\mathcal{W}$ on
     $\textit{stack}$ and $\textit{flatten}$ to distinguish between which
     interpretation it is being applied in.

     \begin{lemma} \label{lem:flatten}
       For any interprocedural path $\pi$,
       \[h(\textit{flatten}_{\mathcal{W}}(\textit{stack}_{\mathcal{W}}(\pi))) = 
       \textit{flatten}_{\mathcal{Q}}(\textit{stack}_{\mathcal{Q}}(\pi)) \]
     \end{lemma}
     \begin{proof}
       By induction on $\pi$.
     \end{proof}

     \begin{lemma} \label{lem:factor}
       For any $v$,
       \[ \mathscr{I}\tuple{v} = h(\mathscr{W}\tuple{v}) \]
     \end{lemma}
     \begin{proof}
       Define a relation $\sigma \subseteq \mathcal{W} \times \mathcal{Q}$ as
       the graph of the function $h$.  That is,
       \[ \sigma = \{ \tuple{P,h(P)} : P \in \mathcal{W} \} \]
       It is easy to show that $\sigma$ is a congruence w.r.t. the \QPKA{}
       operations.

       Next, we must show that for any $i$,
       $\sound{\overline{S}_{\mathcal{W}}(i)}{\overline{S}_{\mathcal{Q}}(i)}$.
       We may prove this by induction on the transfinite iteration
       sequence defining $\overline{S}_{\mathcal{W}}$ and
       $\overline{S}_{\mathcal{Q}}$ (using Lemma~\ref{lem:pathexp_rel} for the
       induction step).

       Finally, we may apply Lemma~\ref{lem:analysis_rel} to get
       \[ \sound{\mathscr{W}\tuple{v}}{\mathscr{I}\tuple{v}} \]
       whence $\mathscr{I}\tuple{v} = h(\mathscr{W}\tuple{v})$.
     \end{proof}

     It is easy to see that $\mathscr{W}\tuple{v}$ is exactly the set of
     interprocedural paths to $v$, where calls and returns have been replaced
     by existential quantification of the appropriate variables.  This can be
     expressed formally as
     \[w \in \mathscr{W}\tuple{v} \iff \exists \pi \in \mathsf{IPaths}(v). \{w\} = \textit{flatten}_{\mathcal{W}}(\textit{stack}_{\mathcal{W}}(\pi))\]
     Or equivalently, as
     \[ \mathscr{W}\tuple{v} = \bigcup_{\pi \in \mathsf{IPaths}(v)} \textit{flatten}_{\mathcal{W}}(\textit{stack}_{\mathcal{W}}(\pi)) \]

     We may then conclude
     \begin{align*}
      \mathscr{W}\tuple{v} &= \bigcup_{\pi \in \mathsf{IPaths}(v)} \textit{flatten}_{\mathcal{W}}(\textit{stack}_{\mathcal{W}}(\pi))\\
      h(\mathscr{W}\tuple{v}) &= h\Big(\bigcup_{\pi \in \mathsf{IPaths}(v)} \textit{flatten}_{\mathcal{W}}(\textit{stack}_{\mathcal{W}}(\pi))\Big)\\
      \mathscr{I}\tuple{v} &= h\Big(\bigcup_{\pi \in \mathsf{IPaths}(v)} \textit{flatten}_{\mathcal{W}}(\textit{stack}_{\mathcal{W}}(\pi))\Big) & \text{Lemma~\ref{lem:factor}}\\
      \mathscr{I}\tuple{v} &= \bigoplus_{\pi \in \mathsf{IPaths}(v)} h(\textit{flatten}_{\mathcal{W}}(\textit{stack}_{\mathcal{W}}(\pi))) & \text{Def'n of } h\\
      \mathscr{I}\tuple{v} &= \bigoplus_{\pi \in \mathsf{IPaths}(v)} \textit{flatten}_{\mathcal{Q}}(\textit{stack}_{\mathcal{Q}}(\pi)) & \text{Lemma~\ref{lem:flatten}}
     \end{align*}
   \end{proof}











\section{Experimental results}

We present a detailed comparison of {\sc LRA}, {\sc UFO}, and {\sc InvGen} on
our benchmark suite.  This includes result only for those benchmarks where one
of the answers among the three tools differed from the other two.
\begin{figure}
\begin{tabular}{|l|c|c|c|} \hline

 Benchmark Name           &  {\sc LRA}    & {\sc UFO}     & {\sc InvGen}                \\ \hline \hline

 apache-escape-absolute   & unsafe  & unsafe  & safe                  \\
 apache-get-tag           & safe    & unsafe  & safe                  \\
 gulwani\_cegar2           & safe    & safe    & unsafe                \\
 gulwani\_fig1a            & safe    & safe    & unsafe                \\
 half                     & safe    & unsafe  & unsafe                \\
 heapsort                 & safe    & unsafe  & safe                  \\
 heapsort2                & safe    & unsafe  & safe                  \\
 heapsort3                & safe    & unsafe  & safe                  \\
 id\_trans                 & unsafe  & unsafe  & safe (false negative) \\
 large\_const              & safe    & timeout & safe                  \\
 MADWiFi-encode\_ie\_ok     & safe    & unsafe  & safe                  \\
 nest-if6                 & safe    & safe    & unsafe                \\
 nest-if8                 & safe    & unsafe  & safe                  \\
 nested6                  & safe    & unsafe  & safe                  \\
 nested7                  & safe    & crash   & unsafe                \\
 nested8                  & safe    & timeout & safe                  \\
 nested9                  & safe    & unsafe  & unsafe                \\
 rajamani\_1               & safe    & timeout & safe                  \\
 sendmail-close-angle     & safe    & unsafe  & safe                  \\
 seq-len                  & safe    & safe    & unsafe                \\
 seq2                     & safe    & crash   & safe                  \\
 split                    & unsafe  & timeout & safe                  \\
 svd1                     & safe    & crash   & safe                  \\
 svd2                     & safe    & crash   & safe                  \\
 svd3                     & safe    & crash   & safe                  \\
 svd4                     & unsafe  & crash   & safe                  \\
 up-nd                    & safe    & unsafe  & safe                  \\
 up2                      & safe    & unsafe  & safe                  \\
 up3                      & safe    & unsafe  & safe                  \\
 up4                      & safe    & unsafe  & safe                  \\
 up5                      & safe    & unsafe  & safe                  \\ \hline \hline

 for\_infinite\_loop\_1\_safe & safe    & safe    & crash                 \\
 for\_infinite\_loop\_2\_safe & safe    & safe    & crash                 \\
 sum01\_safe               & safe    & unsafe  & unsafe                \\
 sum02\_safe               & unsafe  & unsafe  & unsafe                \\
 terminator\_02\_safe       & safe    & safe    & crash                 \\
 token\_ring01\_safe        & unsafe  & safe    & --                    \\
 toy\_safe                 & timeout & timeout & --                    \\ \hline \hline 

 Wrong Results (total)     & 6       & 29      & 14                    \\ \hline

\end{tabular}
\caption{Experimental Results of comparing LRA analysis with UFO and InvGen tools on 
the set of InvGen benchmarks, and the set of SVComp benchmarks. ``--'' indicates that the
benchmark has function calls, which not currently supported by {\sc InvGen}. The last row indicates
how many wrong results each tool had (incorrect result, timeout, or crash) over the entire set of benchmarks. }
\end{figure}
