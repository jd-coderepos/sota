\documentclass[runningheads,a4paper]{llncs}
\usepackage[margin=1.33in]{geometry}
\usepackage{amssymb}
\setcounter{tocdepth}{3}
\usepackage{graphicx}
\usepackage{verbatim}
\usepackage{mathrsfs}
\usepackage[ruled,vlined,lined,commentsnumbered,linesnumbered]{algorithm2e}
\usepackage{amsfonts}
\let\proof\relax
\let\endproof\relax
\usepackage{amsthm}
\usepackage{ifpdf}
\usepackage{amsmath}
\newtheorem{mydef}{Definition}
\newtheorem{mythm}{Theorem}
\newtheorem{mylmm}{Lemma}
\newtheorem{mycor}{Corollary}
\newtheorem{myclm}{Claim}
\usepackage[noend]{algpseudocode}
\usepackage{enumerate}
\usepackage{url}
\usepackage{color,soul}
\usepackage{hyperref}
\usepackage{cite}
\renewcommand{\citeleft}{\textcolor{cyan}{[}}
\renewcommand{\citeright}{\textcolor{cyan}{]}}
\DeclareMathOperator*{\argmin}{arg\,min}


\urldef{\mailsa}\path|{ibanerje, richards}@cs.gmu.edu|    
\newcommand{\keywords}[1]{\par\addvspace\baselineskip
\noindent\keywordname\enspace\ignorespaces#1}
\renewcommand\thefootnote{\textcolor{red}{\arabic{footnote}}}



\begin{document}

\mainmatter  

\title{Routing and Sorting Via Matchings On Graphs}

\titlerunning{Routing and Sorting Via Matchings On Graphs}

\author{Indranil Banerjee, Dana Richards}
\authorrunning{I  Banerjee, D Richards}


\institute{George Mason University\\ Department Of Computer Science\\ Fairfax Virginia 22030, USA\\
\mailsa}



\toctitle{Lecture Notes in Computer Science}
\tocauthor{Authors' Instructions}
\maketitle


\begin{abstract}
The paper is divided in to two parts. In the first part we present some new results for the \textit{routing via matching} model introduced by Alon et al\cite{5}. This model can be viewed as a communication scheme on a distributed network. The nodes in the network can communicate via matchings (a step), where a node exchanges data with its partner. Formally, given a connected graph  with vertices labeled from  and a permutation  giving the destination of pebbles on the vertices the problem is to find a minimum step routing scheme. This is denoted as the routing time  of  given .  In this paper we present the following new results, which answer one of the open problems posed in \cite{5}: 1) Determining whether  is  can be done in  deterministic time for any arbitrary connected graph . 2) Determining whether  is  for any  is NP-Complete. In the second part we study a related property of graphs, which measures how easy it is to design sorting networks using only the edges of a given graph. Informally, \textit{sorting number} of a graph is the minimum depth sorting network that only uses edges of the graph. Many of the classical results on sorting networks can be represented in this framework. We show that a tree with maximum degree  can accommodate a  depth sorting network. Additionally, we give two instance of trees for which this bound is tight.  

\keywords{Routing, NP Completeness, Sorting Networks}
\end{abstract}


\section{Permutation Routing via Matchings}
Originally introduced by Alon and others \cite{5} this problems explores permutation routing on graphs where routing is achieve through a  series of matchings called steps. Let  be an undirected labeled graph with vertex labeled  having a pebble labeled  initially. A permutation  gives the destinations of each pebble. The task is to route each pebble to their destination via a sequence of matchings. Given a matching we swap the pebbles on pairs of matched vertices. The \textit{routing time}  is defined as the minimum number of steps necessary to route all the pebbles for a given permutation. For given graph , the maximum routing time over all permutations is called  the routing number  of . Since their inception, permutation routing via matching have generated continual interest. However, prevailing literature focuses on determining the routing numbers of special graphs. We shall give a very brief survey about them in the next section. In this paper we shall focus on the computational aspect of the problem. In particular  we show that for a general graph determining whether  is  is  complete for . However, we show that it is possible to determine if  in polynomial time by determining whether a certain graph has a perfect matching. It remains open whether computing  is constant factor -.

\subsection{Introduction And Prior Results}
The routing via matching model has several variants and generalizations \cite{5,6,8}. For example a popular network routing model is the direct path routing model. In this model a packet move towards its destination directly and no two packets uses the same links (edges). In one version of the problem a path may be specified for each vertex. Costas and others \cite{8} show that this problem and some variants of it to be  complete.  In this paper we only consider the classical model as described in the previous section.  In the introductory paper Alon and others \cite{5}  show that for any connected graph , . This was shown by considering a spanning tree of  and using only the edges of the spanner to route the permutation in . Note that one can always route a permutation on a tree, by iteratively moving a pebble that belong to some leaf node and ignoring the node afterwards. The routing scheme is recursive and uses a well known property of trees: a tree has a vertex whose removal results in a forest of trees with size at most .
Later in \cite{7} Zhang improve this upper bound . This was done using a new decomposition called the caterpillar decomposition. This bound is essentially tight as it takes  steps to route a permutation on a star . There are  few known results for routing numbers of graphs besides trees. We know that for the complete graph and the complete bipartite graph the routing number is 2 and 4 respectively\cite{5}. Later Li and others \cite{6} extends these results to show  (). For the -cube  we know that . The lower bound is quite straightforward. The upper bound was discovered using the results for determinig the routing number of the Cartesian product of two graphs \cite{5}. If  be the Cartesian product of  and  then: 
Since  the result follow\footnote{The base case, which computes  was determined to be 4 via a computer search\cite{6}}. 

Here we take a detour to discuss a related problem of determining the \textit{acquaintance time} of a connected graph. Given a connected graph  whose vertices contains pebbles, its acquaintance time  is defined to be the minimum number of matching necessary for each pebble to be acquainted with  each other. We say two pebbles are acquainted if they happen to be on adjacent vertices. Hence the acquaintance time of a complete graph is 0. This notion  of acquaintance  was introduce by Benjamini and others in a recent paper\cite{9}. They show that routing number and acquaintance time of a graph are distinct parameters by giving a separation result for the complete bipartite graph. They show , which stands in  contrast to the routing number of 4 for . We believe that further investigation is necessary to study graphs which have large separation between the two parameters.

\subsection{Determining whether  Is Easy}
In this section we present a polynomial time deterministic algorithm to determine given a graph if a permutation can be routed in less than two steps. Determining whether  is trivial hence we consider only the case when .  The basic idea centers around whether we can route the individual cycles of the permutation within 2 steps. Let  be a permutation with  cycles and . We say a cycle  is  \textit{individually routable } if it can be routed using only edges of the induced subgraph . A pair of cycles  are \textit{mutually routable} if all the pebbles withing them can be routed using only the edges between the two subsets  and . The next lemma shows that we cannot route  a pair of cycles using edges between the components as well as within the components.

\begin{lemma}
	If a pair of cycles are mutually routable in 2-steps then they must be of same length.
\end{lemma}

\begin{figure}[h]
	\includegraphics[width=4cm]{fig_c_1.pdf}
	\centering
	\caption{The two permutations are shown as concentric circles. The direction of rotation for the outer circle is clockwise and the inner circle is anti-clockwise. Once, we choose  as the first matched pair, the rest of the matching is forced for both the stages. The crossed vertices in the figure will not be routed.} 
\end{figure}

\begin{proof}
	We prove this assuming  is a complete graph. Since for any other case the induced subgraph  would have fewer of edges, hence this is a stronger claim. Let . Consider the cycle . At the first step we have only three choices for matching some vertex with . 
	\begin{description}
		\item[case 1] \textit{ is not matched}. In this case, in the next(last) round  must be matched with . This implies  must be at  after the first round. This would force  to be matched with  in the first round, otherwise  will not be able to reach  in two rounds. Proceeding in this way we see that the matching for all the vertices are fixed once we decide not to match . This implies  and  cannot be routed mutually if we choose to omit any vertex from the matching in  in the first round.
		\item[case 2] \textit{ is matched with }. In  this case also we can show that the entire matching is forced. 
		\item[case 3] \textit{ is matched with }. From Figure 1 we see that unless ,  the pair  and  are not mutually routable in 2 steps.\qed
	\end{description}
\end{proof}

\begin{corollary}
	Three or more cycles are not mutually routable in 2 steps. 
\end{corollary}

\noindent Naively verifying whether a cycle  are individually or a pair  is mutually routable takes  and  respectively. However, we can employ a more sophisticated approach that takes time proportional to the number of edges in the induced subgraphs corresponding to the cycles. First consider the case of verifying whether a cycle  is individually routable. If  is clique then there are  different routing schemes, one for each choices of how we match the  vertex in the first round. For example, if we match  () in the first then the routing scheme is the following sequence   of matchings (slightly modified from \cite{5}): 

 

\noindent Where . Note for every , . Also, it follows from Lemma 1 that no edge can be in two different routings. Let us label these  different routing schemes from 1 to . Next we scan through the edges of  maintaining a array of counters of size  whose  element counts the number edges we have seen so far that belongs to the  routing scheme. After iterating over all the edges if some counter  has a value  or  depending on whether  is even or odd, respectively, then we know  can be routed using the  routing scheme otherwise  is not individually routable. Clearly this takes time linear in the number of edges  in . So the total time to verify all cycles is . We can extend this argument to show that we can also verify all pairs of cycles for mutual routability in  steps also. The main observation is that the pairs of cycles partition the routable edges into disjoints sets. 

Next define a graph  whose vertices are the cycles () and two cycle are adjacent iff they are mutually routable in 2-steps. Additionally,  has loops corresponding to vertices which are individually routable cycles. We can modify any existing maximum matching algorithm to check whether  has a perfect matching (accounting for the self loops) with only a linear overhead. We omit the details. Then the next lemma follows immediately:
\begin{lemma}
	 iff there is a perfect matching in .
\end{lemma}
\noindent It is apparent from the previous discussion that  can be constructed in  time. Since we have at most  cycles,   will have t most  vertices and at most  edges. 
Hence we can determine a maximum matching in  in   time \cite{10}. This gives a total runtime of  for our algorithm which finds a 2-step routing scheme of a connected graph if one exists.
\begin{corollary}
	 iff  is a clique.
\end{corollary}
\begin{proof}
	\begin{description}
		\item[ ] A two step routing scheme for  was given in \cite{5}.
		\item[] If  is not a clique then there is at least a pair of non-adjacent vertices. Let  be a non-edge. Then by Lemma 1 the permutation  cannot be routed in two steps.\qed
	\end{description}
\end{proof}

\subsection{Determining  is hard for any }
\begin{theorem}
	For  computing  is -complete.
\end{theorem}

\begin{proof}
	\begin{figure}[h]
		\includegraphics[width=10cm]{fig_c_2.pdf}
		\centering
		\caption{Atomic Gadgets, pairs  need to swap their pebbles. The unmarked circles have pebbles that is fixed.} 
	\end{figure}
	Proving it is in  is trivial, we can use a set of matchings as a witness. For the NP hardness proof we first define three \textit{atomic} gadgets (see Figure 6) which will be use to construct the variable and clause gadgets. Vertices whose pebbles are fixed (1 cycles) are represented as circles.  Otherwise they are represented as black discs. So in the first three sub-figures ((a)-(c)) the input permutation is \footnote{We do not write the 1 cycles explicitly for notational clarity.}. In all our construction we shall use permutation consisting of only 1 or 2 cycles. Each cycle labeled  will be represented as a pair . If the correspondence between a pair is clear from the figure then we shall omit the subscript.  It is an easy observation that . In the case of the hexagon  we see that in order to route the pebbles within 3 steps we have to use the left or the right path, but we cannot use both paths simultaneously (i.e.,  goes through left but  goes through the right). Figure 6(e) shows a chain of squares connecting  to . If vertex  is used during routing any pebble other than the two pebbles to its right then the chain construction forces  to be used in routing the two pebbles to its left. This chain is called a \textit{f-chain}. In our construction we  use chains of constant length to simplify the presentation of our construction.  
	\begin{figure}[h]
		\includegraphics[width=12cm]{fig_c_3.pdf}
		\centering
		\caption{Variable graph of . (a) is a special case for , (b) is the general case.} 
	\end{figure}
	\subsubsection{Clause Gadget:} Say we have clause . In Figure 6(d) we show how to create a clause gadget. This is referred to as the \textit{clause graph}  of the clause . The graph in Figure 6(d) can route  in three steps by using one of the three paths between  and . Say,  is routed to  via . Then it must be the case that vertex  is not used to route any other pebbles. We say the vertex  is \textit{owned} by the clause. Otherwise, it would be not possible to route  to  in three steps via . We can interpret this as follows. A clause has a satisfying assignment iff its clause graph has a owned vertex.
	
	\subsubsection{Variable Gadget:} Construction of the variable gadgets are little more involved than the clause gadgets. For some , let the variable  is in  clauses. The variable gadget corresponding to  is shown in Figure 7(b). Vertically aligned hexagons are all in one level. Number of levels is . The left most hexagon  and the rightmost hexagon  share a common edge as indicate in the figure making it circularly wrapped. The permutation we will route on  (the variable graph of ) is . For each variable we shall have a separate graph and a corresponding permutation on its vertices. In the graph  there are only two possible ways to route  in two steps. 1) If we route  using the right path in  this forces  and  to be routed using the right paths in their respective hexagons  and . Continuing in this way we see that  must be routed using the right path of . In during this routing the vertices  are not used and hence are free and can be owned by some clause. 2) If we route  using the left path of , the opposite happens and  will be the free vertices in this case. This forces variable assignment. The former and latter case corresponds to true (right) and false (left) assignment of  respectively.
	
	\subsubsection{Reduction:} For each clause , if the literal   then we connect   (for some ) to the vertex labeled  via an \textit{f-chain}. If  then we connect it with  via an \textit{f-chain}. This is our final graph  corresponding to an instance of a 3- formula. The input permutation is , which is the concatenation of all the individual permutation on the variable, clause  graphs and -\textit{chains}.  This completes our construction. We need to show,  iff  is satisfiable. Suppose  is satisfiable. Then for each variable , if the literal  is true then we use left routing in , otherwise we use right routing. This ensures in each clause graph there will be at least one owned vertex.  Now suppose . Then each clause graph has at least one owned vertex. If  is a free vertex in some clause graph then  is not a free vertex in any of the other clause graphs, otherwise variable graph  will not be able rout its own permutation in 3 steps. Hence the set of free vertices will be a satisfying assignment for . It is an easy observation that the number of vertices in  is polynomially bounded in ; the number of variables and clauses in  respectively and that  can be explicitly constructed in polynomial time.
	
	
\end{proof}



\section{Optimal Sorting Network For A Given Graph}
In this section we introduce {\it sorting numbers} for graphs in the context of  sorting networks. Majority of existing literature on sorting networks focus on optimizing the depth (number of concurrent stages) and size (total number of comparators used) of such networks. Here, we study a slightly different problem. Let the \textit{sorted order} of the vertices of a labeled graph  be the permutation  that assigns a rank  to the vertex labeled .
Given  the task is to design an optimal sorting network (in terms of depth) on . Before we make this notion formal we need to define a sorting network.


\begin{definition}[Sorting Network]
	A sorting network is a triple  such that:
	\begin{enumerate}
		\item  is a connected labeled graph having  vertices and a sorted ordering  on its vertices. Initially each vertices of  contains a pebble having some value, that is they act as input terminals of the network.
		\item The ordered set  consists of directed matchings in . In a directed matching some edges in the matching have been assigned a direction. Sorting occurs in stages.
		At stage  we use the matching  to exchange pebbles between matched vertices according to their orientation. For an edge , when swapped the smaller of the two pebble goes to . If an edge is undirected then both pebbles swap regardless of their order.
		\item After  stages the vertex labeled  contains the pebble whose rank is  in the sorted order of the pebbles.   is called the depth of the network. Additionally, this must hold for all () initial arrangement of the pebbles.
		\item Each edge in  is in some matching, that is  is minimal. 
	\end{enumerate}
	
\end{definition}

\noindent We say   is a sorting network on  if  is a spanning subgraph of . Let  be the set of all such sorting networks on  over all possible spanning subgraphs and sorted orderings (non-isomorphic).


\begin{definition}[Sorting Number]
	Sorting number  of a graph  is defined to be minimum depth of any sorting network on . Additionally,  is the sorting number of  over all possible sorting network on  with a fixed sorted order .
\end{definition}

\begin{lemma}
For any  we have .
\end{lemma}

\noindent The above lemma implies that if we construct a sorting network for some arbitrary sorted order on the vertices then we suffer a penalty of  on the depth of our network as compared to the optimal one. 
\begin{proof}
	Let  be an optimal sorting network on . Using this network we can create another sorting network  whose depth is  more than the optimal one. This can be done in two rounds. First we use the sorting network  to determine after  stages the ranks of each pebble in the sorted order. After this step we will know that the pebble at vertex  has a rank . But  sends a pebble with rank  to the vertex labeled . Hence we can route the fixed permutation  on  during the second round to arrive at the desired sorted order . Since we can route a permutation on any graph in  steps the lemma follows.\qed

\end{proof}

\noindent Note that if  is not connected than . Otherwise, there always exists a spanning tree  of  and . The main result of this section will be to obtain both a lower and an upper bounds for . We start by restating some previous results for sorting networks with restricted topology under this new framework. The path graph  is one of the simplest case. We know that . This follows from the fact that the classical odd-even transposition sort takes  matching steps and that is optimal. Next we discuss some known bounds for the sorting numbers of some common graphs starting with the complete graph. These results are summarized in Table 1. For the the complete graph . Ajtai-Komlos-Szemerdi (AKS) sorting network directly gives an upper bound of  for the sorting number of . In this case also the bound is tight. For the  -cube  we can use the Batcher's Bitonic sorting network, which has a depth of  \cite{2}. This was later improved to  by Plaxton and Suel\cite{11}.  We also have a lower bound of  due to Leighton and Plaxton. For the square mesh  it is known that , which is tight with respect to the constant factor of the largest term. This follows from results of Schnorr \& Shamir \cite{4}, where they introduced the -sorter for the square mesh.



\begin{table}[ht]
	\caption{Known Bounds On The Sorting Numbers Of Various Graphs} \centering \begin{tabular}{c c c c} \hline\hline Graph & Lower Bound & Upper Bound & Remark \\ [0.5ex] \hline Complete Graph () &  &  & AKS Network \cite{1}\\ Hypercube () & \cite{11} &   & Leighton and Plaxton \cite{13} \\
		Path () &  &  & Odd-Even Transposition Sort \\
		Mesh () &  &  & Schnorr \& Shamir \cite{4}\\ 
		Tree &  & & this paper\st(G_1 \times G_2) \le st(G_1)st(G_2) + st(G_1)+ st(G_2)
C(T) = \max(C(T'_1),\max_i{C(T_i)}) + S(n,\Delta; \alpha_1,\ldots,\alpha_d)

	\sum_{i=1}^{l-k+j-1}{d_i} + \sum_{i=1}^{j-1}{d_{1+i}} \le 2\sum_{i=1}^{l-1}{d_i}
	
	\sum_{i=1}^{l'}d_i + 2 + \sum_{i=1}^{s_{t}+1}d_{p-i} \le 2\max(\alpha_i,\alpha_j) p_{ij} + O(1)
	
c_j = \sum_{i=1}^{d-j}{(2n_i+\beta_ip^j_{i,i+1})}

S(n,d; \alpha_1,\ldots,\alpha_d) &= \sum_{i=1}^{d-1}{c_j} = \sum_{j=1}^{d-1}\sum_{i=1}^{d-j}{(2n_i+\beta_ip^j_{i,i+1})} \le 2dn + \sum_{i=1}^{d-1}{\beta_i\sum_{j=1}^{d-i}{p^j_{i,i+1}}} 

\nonumber \sum_{i=1}^{d-1}{\beta_i\sum_{j=1}^{d-i}{p^j_{i,i+1}}}  &\le \sum_{i=1}^{d-1}{\beta_i\sum_{j=i+1}^{d}{n_j}} \le \left(\sum_{i=1}^{d-1}{\beta_i}\right)\left(\sum_{i=2}^{d}{n_i}\right) \\&\le \left(\sum_{i=1}^{d-1}{\beta_i}\right)(n/2) \le cn\min(\Delta^2,n)

C(n) \le C(n/2) + O(\min(\Delta^2n,n^2))


\noindent  This shows that  requires  stages to correctly sort any input with respect to the MP ordering. We note that for the two extreme cases, 1) when  and 2)  the number stages needed by  is optimal up to a constant factor. It remains to be seen if there exists an  round sorting network for trees

\begin{thebibliography}{4}
	
	\bibitem{1} Ajtai, M., Komlós, J., \& Szemerédi, E. (1983, December). An 0 (n log n) sorting network. In Proceedings of the fifteenth annual ACM symposium on Theory of computing (pp. 1-9). ACM.
	\bibitem{2}Batcher, K. E. (1968, April). Sorting networks and their applications. In Proceedings of the April 30--May 2, 1968, spring joint computer conference (pp. 307-314). ACM.
	\bibitem{3}Komlós, J., Ma, Y., \& Szemerédi, E. (1998). Matching nuts and bolts in O (n log n) time. SIAM Journal on Discrete Mathematics, 11(3), 347-372.
	\bibitem{4}Schnorr, C. P., \& Shamir, A. (1986, November). An optimal sorting algorithm for mesh connected computers. In Proceedings of the eighteenth annual ACM symposium on Theory of computing (pp. 255-263). ACM.
	\bibitem{5} Alon, N., Chung, F. R., \& Graham, R. L. (1994). Routing permutations on graphs via matchings. SIAM journal on discrete mathematics, 7(3), 513-530.
	\bibitem{6}Li, W. T., Lu, L., \& Yang, Y. (2010). Routing numbers of cycles, complete bipartite graphs, and hypercubes. SIAM Journal on Discrete Mathematics, 24(4), 1482-1494.
	\bibitem{7}Zhang, L. (1999). Optimal bounds for matching routing on trees. SIAM Journal on Discrete Mathematics, 12(1), 64-77.
	\bibitem{8}Busch, C., Magdon-Ismail, M., Mavronicolas, M., \& Spirakis, P. (2004). Direct routing: Algorithms and complexity. In Algorithms–ESA 2004 (pp. 134-145). Springer Berlin Heidelberg.
	\bibitem{9}Benjamini, I., Shinkar, I., \& Tsur, G. (2014). Acquaintance time of a graph. SIAM Journal on Discrete Mathematics, 28(2), 767-785.
	\bibitem{10} Micali, S., \& Vazirani, V. V. (1980, October). An  algoithm for finding maximum matching in general graphs. In Foundations of Computer Science, 1980., 21st Annual Symposium on (pp. 17-27). IEEE.
	\bibitem{11}Plaxton, C. G., \& Suel, T. (1994). A super-logarithmic lower bound for hypercubic sorting networks (pp. 618-629). Springer Berlin Heidelberg.
	\bibitem{12}Knuth, D. E. (1998). The art of computer programming: sorting and searching (Vol. 3). Pearson Education.
	\bibitem{13} Leighton, T., \& Plaxton, C. G. (1998). Hypercubic sorting networks. SIAM Journal on Computing, 27(1), 1-47.
\end{thebibliography}




\end{document}