\documentclass[11pt]{article}
\usepackage{amsmath}
\usepackage{amsfonts}
\usepackage{graphicx}
\usepackage{mathrsfs}
\usepackage{geometry}
\usepackage{doublespace}
\usepackage{pstcol,pst-fill,pst-grad}

\addtolength{\textheight}{3cm}

\newtheorem{question}{Question}
\newtheorem{definition}{Definition}
\newtheorem{lemma}{Lemma}
\newtheorem{theorem}{Theorem}
\newtheorem{remark}{Remark}
\newtheorem{corollary}{Corollary}
\newtheorem{example}{Example}


\def\proof{\noindent{\bf Proof --~}}
\def\cqfd{\hfill}
\def\sgn{\mathrm{sign}}
\def\G{\mathscr{G}}
\def\GG{\mathcal{G}}
\def\N{\mathbb{N}}
\def\1n{1,\dots,n} 
\def\F{{\tilde F}} 
\def\f{\tilde f} 
\def\phi{\varphi}

\begin{document}


~\\~\\~\\
\begin{center}
\begin{LARGE}
Negative circuits and sustained oscillations\\ in asynchronous
automata networks
\end{LARGE}
~\\~\\
\begin{large}
Adrien Richard
\end{large}
~\\~\\
Laboratoire I3S, UMR 6070 CNRS \& Universit\'e de
Nice-Sophia Antipolis,\-4mm]
telephone: +33 4 92 94 27 51\
x=(x_1,\dots,x_n)\in X~\mapsto~ F(x)=(f_1(x),\dots,f_n(x))\in X,

F_i(x)=(x_1,\dots,x_{i-1},f_i(x),x_{i+1},\dots,x_n)\qquad (i=\1n).
\label{it}
x^{t+1}=F_{\phi(t)}(x^t)\qquad (t=0,1,2\dots). 

I_F(x)=\{i\in\{\1n\}~|~f_i(x)\neq x_i\}.

\{(x,F_i(x))\,|\, x\in X,~i\in I_F(x)\}.

f_i(x_1,\dots,x_j+1,\dots,x_n)-f_i(x_1,\dots,x_j,\dots,x_n)

\begin{array}{c|ccccccccc}
x    &(0,0)&(0,1)&(0,2)&(1,0)&(1,1)&(1,2)&(2,0)&(2,1)&(2,2)\\\hline
F(x) &(2,0)&(1,0)&(0,2)&(2,0)&(0,0)&(0,1)&(2,1)&(0,1)&(0,1)
\end{array}

\begin{array}{c}
\Gamma(F)\5mm]
\begin{picture}(0,0)\includegraphics{GraphDef.pstex}\end{picture}\setlength{\unitlength}{3947sp}\begingroup\makeatletter\ifx\SetFigFont\undefined \gdef\SetFigFont#1#2#3#4#5{\reset@font\fontsize{#1}{#2pt}\fontfamily{#3}\fontseries{#4}\fontshape{#5}\selectfont}\fi\endgroup \begin{picture}(2383,940)(2135,-3879)
\put(2850,-3571){\makebox(0,0)[b]{\smash{{\SetFigFont{11}{13.2}{\rmdefault}{\mddefault}{\updefault}{\color[rgb]{0,0,0}1}}}}}
\put(3900,-3571){\makebox(0,0)[b]{\smash{{\SetFigFont{11}{13.2}{\rmdefault}{\mddefault}{\updefault}{\color[rgb]{0,0,0}2}}}}}
\put(4426,-3548){\makebox(0,0)[b]{\smash{{\SetFigFont{8}{9.6}{\rmdefault}{\mddefault}{\updefault}{\color[rgb]{0,0,0}}}}}}
\put(3376,-3848){\makebox(0,0)[b]{\smash{{\SetFigFont{8}{9.6}{\rmdefault}{\mddefault}{\updefault}{\color[rgb]{0,0,0}}}}}}
\put(3376,-3248){\makebox(0,0)[b]{\smash{{\SetFigFont{8}{9.6}{\rmdefault}{\mddefault}{\updefault}{\color[rgb]{0,0,0}}}}}}
\put(3376,-3023){\makebox(0,0)[b]{\smash{{\SetFigFont{8}{9.6}{\rmdefault}{\mddefault}{\updefault}{\color[rgb]{0,0,0}}}}}}
\put(2251,-3548){\makebox(0,0)[b]{\smash{{\SetFigFont{8}{9.6}{\rmdefault}{\mddefault}{\updefault}{\color[rgb]{0,0,0}}}}}}
\end{picture} \end{array}

f'_i(x)=\sgn(f_i(x)-x_i)\qquad (i=\1n),

f'_i(x)\neq f'_i(F_j(x))\quad\textrm{and}\quad s=f'_j(x)f'_i(F_j(x))

x^p=(x_1,\dots,x_{j-1},x_j+p,x_{j+1},\dots,x_n).

s_{ki}=f'_k(x^{r-1})f'_i(x^r).

s_{jk}=f'_j(x^0)f'_k(x^p).

s_{ji}=s_{jk}s_{ki}=f'_j(x^0)f'_k(x^p)f'_k(x^{r-1})f'_i(x^r), 

\forall x\in X,\qquad 
H(x)=(h_1(x),h_2(x),\dots,h_n(x))=
(x_1,f_2(x),\dots,f_n(x)).

\F(x)=(\f_1(x),\dots,\f_n(x)),\qquad \f_i(x)=x_i+f'_i(x)\qquad (i=\1n).

f_i(x_1,\dots,x_j,\dots,x_n)\leq x_i <f_i(x_1,\dots,x_j+1,\dots,x_n),

f_i(x_1,\dots,x_j,\dots,x_n)> x_i \geq f_i(x_1,\dots,x_j+1,\dots,x_n).

y=(x_1,\dots,x_j+\f'_j(x),\dots,x_n)

f_i(x)\leq \f_i(x)\leq x_i=y_i < \f_i(y)\leq f_i(y).

f_i(x)\leq x_i<f_i(y)=f_i(x_1,\dots,x_j+1,\dots,x_n)

f_i(y_1,\dots,y_j+1,\dots,y_n)=f_i(x)\leq y_i<f_i(y)

\f_i(x_1,\dots,x_i+1,\dots,x_n)\leq x_i<\f_i(x)

f_i(x_1,\dots,x_i+1,\dots,x_n)\leq 
\f_i(x_1,\dots,x_i+1,\dots,x_n)\leq x_i<\f_i(x)\leq f_i(x).

\begin{array}{ccc}
\Gamma(F)&&\Gamma[F]\
We see that  has a cyclic attractor and that 
has no cyclic attractor. The interaction graph  is the
interaction graph with one vertex and a negative arc from this vertex
to itself: it has thus a negative circuit. The interaction graph
 is the interaction graph with one vertex and no arc (it is a
strict subgraph of ). This shows that the presence of a cyclic
attractor in  does not imply the presence of a negative
circuit~in~.
\end{example}


\begin{example}
,  and  defined by ,  and
. The state transitions graphs  and  are
the following:
5mm]
\input{Ex_11.pstex_t}&~~~~~~~~~~&\begin{picture}(0,0)\includegraphics{Ex_12.pstex}\end{picture}\setlength{\unitlength}{3947sp}\begingroup\makeatletter\ifx\SetFigFont\undefined \gdef\SetFigFont#1#2#3#4#5{\reset@font\fontsize{#1}{#2pt}\fontfamily{#3}\fontseries{#4}\fontshape{#5}\selectfont}\fi\endgroup \begin{picture}(2104,602)(1800,-1862)
\put(1951,-1606){\makebox(0,0)[b]{\smash{{\SetFigFont{11}{13.2}{\rmdefault}{\mddefault}{\updefault}{\color[rgb]{0,0,0}}}}}}
\put(2866,-1606){\makebox(0,0)[b]{\smash{{\SetFigFont{11}{13.2}{\rmdefault}{\mddefault}{\updefault}{\color[rgb]{0,0,0}}}}}}
\put(3766,-1606){\makebox(0,0)[b]{\smash{{\SetFigFont{11}{13.2}{\rmdefault}{\mddefault}{\updefault}{\color[rgb]{0,0,0}}}}}}
\end{picture} \end{array}

\begin{array}{l}
f_1(x)=x_3\\
f_2(x)=x_1\\
f_3(x)=x_2.
\end{array}

\begin{array}{c}
\Gamma(F)\5mm]
\begin{picture}(0,0)\includegraphics{Graph1.pstex}\end{picture}\setlength{\unitlength}{3947sp}\begingroup\makeatletter\ifx\SetFigFont\undefined \gdef\SetFigFont#1#2#3#4#5{\reset@font\fontsize{#1}{#2pt}\fontfamily{#3}\fontseries{#4}\fontshape{#5}\selectfont}\fi\endgroup \begin{picture}(1128,1006)(1987,-2454)
\put(2550,-1621){\makebox(0,0)[b]{\smash{{\SetFigFont{11}{13.2}{\rmdefault}{\mddefault}{\updefault}{\color[rgb]{0,0,0}1}}}}}
\put(3000,-2371){\makebox(0,0)[b]{\smash{{\SetFigFont{11}{13.2}{\rmdefault}{\mddefault}{\updefault}{\color[rgb]{0,0,0}2}}}}}
\put(2100,-2371){\makebox(0,0)[b]{\smash{{\SetFigFont{11}{13.2}{\rmdefault}{\mddefault}{\updefault}{\color[rgb]{0,0,0}3}}}}}
\put(2551,-2423){\makebox(0,0)[b]{\smash{{\SetFigFont{8}{9.6}{\rmdefault}{\mddefault}{\updefault}{\color[rgb]{0,0,0}}}}}}
\put(2851,-1898){\makebox(0,0)[b]{\smash{{\SetFigFont{8}{9.6}{\rmdefault}{\mddefault}{\updefault}{\color[rgb]{0,0,0}}}}}}
\put(2251,-1898){\makebox(0,0)[b]{\smash{{\SetFigFont{8}{9.6}{\rmdefault}{\mddefault}{\updefault}{\color[rgb]{0,0,0}}}}}}
\end{picture} \end{array}

\begin{array}{l}
f_1(x)=x_2\\
f_2(x)=x_1.
\end{array}

\begin{array}{ccc}
\Lambda(F)&&G(F)\
We see that  has a cyclic attractor and that  has no
negative circuit.
\end{example}


Finally, we can ask if, under the condition that  has a
cyclic attractor, a conclusion stronger than `` has a negative
circuit'' could be obtained. Following Example 2, the presence of a
cyclic attractor in  does not imply the presence of a
negative circuit in the subgraph  of . So, another
direction has to be taken. As showed below, previous results on
the links between the interaction graph and the dynamical properties of
automata networks suggest to improve the conclusion of Theorem 1 by
studying if the presence of a cyclic attractor in  implies
the presence of a negative circuit in a {\emph{local interaction
graph}} associated with .

\begin{definition}
For all , the {\emph{local interaction graph of  evaluated
at state }} is the interaction graph  that contains a
positive (negative) arc from  to  if ~and

is positive (negative), or if  and 

is positive (negative).
\end{definition}


\begin{remark}
 is a subgraph of . More precisely, .
\end{remark}
With this material, Richard and Comet {\cite{RC07}} prove the
following local version of first Thomas' conjecture:

\begin{theorem}{\bf{\cite{RC07}}}~
If  has several attractors, and in particular if  has
several fixed points, then there exists  such that 
has a positive circuit.
\end{theorem}
Let us also mention the following fixed point theorem proved by
Richard {\cite{R08}} (and previously proved by Shih and Dong
{\cite{SD05}} in the Boolean case):

\begin{theorem}{\bf{\cite{R08}}}~
If  has no circuit for all , then  has a unique
fixed point.
\end{theorem}
The proof of Theorem 4 done in {\cite{R08}} reveals that if 
has no circuit for all , then  has a unique fixed point
, and, in addition, for all ,  has a path from
 to . It is then clear that the presence of a cyclic attractor
in  implies the presence of a circuit in  for at
least one . We then arrive to the following natural question:

\begin{question}
Does the presence of a cyclic attractor in  or 
implies the presence of a negative circuit in  for at least
one ?
\end{question}
Clearly, a positive answer would improve significantly Theorem 1 or 2
by providing a local version of second Thomas' conjecture. However,
the following example shows that the answer is negative. This
highlights the fact that it is necessary to take a union of local
interaction graphs in order to obtain, from a cyclic attractor, a
negative circuit.

\begin{example} ,  and  is defined by:
13mm]
f_2(x)=
\left\{
\begin{array}{l}
3\textrm{ if  or if  and }\\
0\textrm{ otherwise} 
\end{array}
\right.
\end{array}

\begin{picture}(0,0)\includegraphics{Ex3.pstex}\end{picture}\setlength{\unitlength}{3947sp}\begingroup\makeatletter\ifx\SetFigFont\undefined \gdef\SetFigFont#1#2#3#4#5{\reset@font\fontsize{#1}{#2pt}\fontfamily{#3}\fontseries{#4}\fontshape{#5}\selectfont}\fi\endgroup \begin{picture}(4221,4219)(2090,-5170)
\put(2850,-4456){\makebox(0,0)[b]{\smash{{\SetFigFont{11}{13.2}{\rmdefault}{\mddefault}{\updefault}{\color[rgb]{0,0,0}(0,0)}}}}}
\put(2850,-3556){\makebox(0,0)[b]{\smash{{\SetFigFont{11}{13.2}{\rmdefault}{\mddefault}{\updefault}{\color[rgb]{0,0,0}(0,1)}}}}}
\put(3750,-3556){\makebox(0,0)[b]{\smash{{\SetFigFont{11}{13.2}{\rmdefault}{\mddefault}{\updefault}{\color[rgb]{0,0,0}(1,1)}}}}}
\put(3750,-4456){\makebox(0,0)[b]{\smash{{\SetFigFont{11}{13.2}{\rmdefault}{\mddefault}{\updefault}{\color[rgb]{0,0,0}(1,0)}}}}}
\put(4650,-4456){\makebox(0,0)[b]{\smash{{\SetFigFont{11}{13.2}{\rmdefault}{\mddefault}{\updefault}{\color[rgb]{0,0,0}(2,0)}}}}}
\put(5550,-4456){\makebox(0,0)[b]{\smash{{\SetFigFont{11}{13.2}{\rmdefault}{\mddefault}{\updefault}{\color[rgb]{0,0,0}(3,0)}}}}}
\put(5550,-3571){\makebox(0,0)[b]{\smash{{\SetFigFont{11}{13.2}{\rmdefault}{\mddefault}{\updefault}{\color[rgb]{0,0,0}(3,1)}}}}}
\put(5550,-2671){\makebox(0,0)[b]{\smash{{\SetFigFont{11}{13.2}{\rmdefault}{\mddefault}{\updefault}{\color[rgb]{0,0,0}(3,2)}}}}}
\put(5550,-1771){\makebox(0,0)[b]{\smash{{\SetFigFont{11}{13.2}{\rmdefault}{\mddefault}{\updefault}{\color[rgb]{0,0,0}(3,3)}}}}}
\put(4650,-1771){\makebox(0,0)[b]{\smash{{\SetFigFont{11}{13.2}{\rmdefault}{\mddefault}{\updefault}{\color[rgb]{0,0,0}(2,3)}}}}}
\put(3750,-1771){\makebox(0,0)[b]{\smash{{\SetFigFont{11}{13.2}{\rmdefault}{\mddefault}{\updefault}{\color[rgb]{0,0,0}(1,3)}}}}}
\put(2850,-1771){\makebox(0,0)[b]{\smash{{\SetFigFont{11}{13.2}{\rmdefault}{\mddefault}{\updefault}{\color[rgb]{0,0,0}(0,3)}}}}}
\put(2850,-2671){\makebox(0,0)[b]{\smash{{\SetFigFont{11}{13.2}{\rmdefault}{\mddefault}{\updefault}{\color[rgb]{0,0,0}(0,2)}}}}}
\put(3750,-2671){\makebox(0,0)[b]{\smash{{\SetFigFont{11}{13.2}{\rmdefault}{\mddefault}{\updefault}{\color[rgb]{0,0,0}(1,2)}}}}}
\put(4650,-2671){\makebox(0,0)[b]{\smash{{\SetFigFont{11}{13.2}{\rmdefault}{\mddefault}{\updefault}{\color[rgb]{0,0,0}(2,2)}}}}}
\put(4650,-3571){\makebox(0,0)[b]{\smash{{\SetFigFont{11}{13.2}{\rmdefault}{\mddefault}{\updefault}{\color[rgb]{0,0,0}(2,1)}}}}}
\end{picture} 
\begin{picture}(0,0)\includegraphics{Ex3Bis.pstex}\end{picture}\setlength{\unitlength}{3947sp}\begingroup\makeatletter\ifx\SetFigFont\undefined \gdef\SetFigFont#1#2#3#4#5{\reset@font\fontsize{#1}{#2pt}\fontfamily{#3}\fontseries{#4}\fontshape{#5}\selectfont}\fi\endgroup \begin{picture}(3053,2866)(2673,-4505)
\put(2850,-4456){\makebox(0,0)[b]{\smash{{\SetFigFont{11}{13.2}{\rmdefault}{\mddefault}{\updefault}{\color[rgb]{0,0,0}(0,0)}}}}}
\put(2850,-3556){\makebox(0,0)[b]{\smash{{\SetFigFont{11}{13.2}{\rmdefault}{\mddefault}{\updefault}{\color[rgb]{0,0,0}(0,1)}}}}}
\put(3750,-3556){\makebox(0,0)[b]{\smash{{\SetFigFont{11}{13.2}{\rmdefault}{\mddefault}{\updefault}{\color[rgb]{0,0,0}(1,1)}}}}}
\put(3750,-4456){\makebox(0,0)[b]{\smash{{\SetFigFont{11}{13.2}{\rmdefault}{\mddefault}{\updefault}{\color[rgb]{0,0,0}(1,0)}}}}}
\put(4650,-4456){\makebox(0,0)[b]{\smash{{\SetFigFont{11}{13.2}{\rmdefault}{\mddefault}{\updefault}{\color[rgb]{0,0,0}(2,0)}}}}}
\put(5550,-4456){\makebox(0,0)[b]{\smash{{\SetFigFont{11}{13.2}{\rmdefault}{\mddefault}{\updefault}{\color[rgb]{0,0,0}(3,0)}}}}}
\put(5550,-3571){\makebox(0,0)[b]{\smash{{\SetFigFont{11}{13.2}{\rmdefault}{\mddefault}{\updefault}{\color[rgb]{0,0,0}(3,1)}}}}}
\put(5550,-2671){\makebox(0,0)[b]{\smash{{\SetFigFont{11}{13.2}{\rmdefault}{\mddefault}{\updefault}{\color[rgb]{0,0,0}(3,2)}}}}}
\put(5550,-1771){\makebox(0,0)[b]{\smash{{\SetFigFont{11}{13.2}{\rmdefault}{\mddefault}{\updefault}{\color[rgb]{0,0,0}(3,3)}}}}}
\put(4650,-1771){\makebox(0,0)[b]{\smash{{\SetFigFont{11}{13.2}{\rmdefault}{\mddefault}{\updefault}{\color[rgb]{0,0,0}(2,3)}}}}}
\put(3750,-1771){\makebox(0,0)[b]{\smash{{\SetFigFont{11}{13.2}{\rmdefault}{\mddefault}{\updefault}{\color[rgb]{0,0,0}(1,3)}}}}}
\put(2850,-1771){\makebox(0,0)[b]{\smash{{\SetFigFont{11}{13.2}{\rmdefault}{\mddefault}{\updefault}{\color[rgb]{0,0,0}(0,3)}}}}}
\put(2850,-2671){\makebox(0,0)[b]{\smash{{\SetFigFont{11}{13.2}{\rmdefault}{\mddefault}{\updefault}{\color[rgb]{0,0,0}(0,2)}}}}}
\put(3750,-2671){\makebox(0,0)[b]{\smash{{\SetFigFont{11}{13.2}{\rmdefault}{\mddefault}{\updefault}{\color[rgb]{0,0,0}(1,2)}}}}}
\put(4650,-2671){\makebox(0,0)[b]{\smash{{\SetFigFont{11}{13.2}{\rmdefault}{\mddefault}{\updefault}{\color[rgb]{0,0,0}(2,2)}}}}}
\put(4650,-3571){\makebox(0,0)[b]{\smash{{\SetFigFont{11}{13.2}{\rmdefault}{\mddefault}{\updefault}{\color[rgb]{0,0,0}(2,1)}}}}}
\end{picture} 
\begin{picture}(0,0)\includegraphics{Graph3.pstex}\end{picture}\setlength{\unitlength}{3947sp}\begingroup\makeatletter\ifx\SetFigFont\undefined \gdef\SetFigFont#1#2#3#4#5{\reset@font\fontsize{#1}{#2pt}\fontfamily{#3}\fontseries{#4}\fontshape{#5}\selectfont}\fi\endgroup \begin{picture}(2285,715)(2233,-3879)
\put(2850,-3571){\makebox(0,0)[b]{\smash{{\SetFigFont{11}{13.2}{\rmdefault}{\mddefault}{\updefault}{\color[rgb]{0,0,0}1}}}}}
\put(3900,-3571){\makebox(0,0)[b]{\smash{{\SetFigFont{11}{13.2}{\rmdefault}{\mddefault}{\updefault}{\color[rgb]{0,0,0}2}}}}}
\put(2326,-3548){\makebox(0,0)[b]{\smash{{\SetFigFont{8}{9.6}{\rmdefault}{\mddefault}{\updefault}{\color[rgb]{0,0,0}}}}}}
\put(4426,-3548){\makebox(0,0)[b]{\smash{{\SetFigFont{8}{9.6}{\rmdefault}{\mddefault}{\updefault}{\color[rgb]{0,0,0}}}}}}
\put(3376,-3848){\makebox(0,0)[b]{\smash{{\SetFigFont{8}{9.6}{\rmdefault}{\mddefault}{\updefault}{\color[rgb]{0,0,0}}}}}}
\put(3376,-3248){\makebox(0,0)[b]{\smash{{\SetFigFont{8}{9.6}{\rmdefault}{\mddefault}{\updefault}{\color[rgb]{0,0,0}}}}}}
\end{picture} 
\begin{picture}(0,0)\includegraphics{Graph3_00.pstex}\end{picture}\setlength{\unitlength}{3947sp}\begingroup\makeatletter\ifx\SetFigFont\undefined \gdef\SetFigFont#1#2#3#4#5{\reset@font\fontsize{#1}{#2pt}\fontfamily{#3}\fontseries{#4}\fontshape{#5}\selectfont}\fi\endgroup \begin{picture}(2285,715)(2233,-3879)
\put(2850,-3571){\makebox(0,0)[b]{\smash{{\SetFigFont{11}{13.2}{\rmdefault}{\mddefault}{\updefault}{\color[rgb]{0,0,0}1}}}}}
\put(3900,-3571){\makebox(0,0)[b]{\smash{{\SetFigFont{11}{13.2}{\rmdefault}{\mddefault}{\updefault}{\color[rgb]{0,0,0}2}}}}}
\put(3376,-3248){\makebox(0,0)[b]{\smash{{\SetFigFont{8}{9.6}{\rmdefault}{\mddefault}{\updefault}{\color[rgb]{0,0,0}}}}}}
\put(2326,-3548){\makebox(0,0)[b]{\smash{{\SetFigFont{8}{9.6}{\rmdefault}{\mddefault}{\updefault}{\color[rgb]{1,1,1}}}}}}
\put(3376,-3848){\makebox(0,0)[b]{\smash{{\SetFigFont{8}{9.6}{\rmdefault}{\mddefault}{\updefault}{\color[rgb]{1,1,1}}}}}}
\put(4426,-3548){\makebox(0,0)[b]{\smash{{\SetFigFont{8}{9.6}{\rmdefault}{\mddefault}{\updefault}{\color[rgb]{1,1,1}}}}}}
\end{picture} 
\begin{picture}(0,0)\includegraphics{Graph3_30.pstex}\end{picture}\setlength{\unitlength}{3947sp}\begingroup\makeatletter\ifx\SetFigFont\undefined \gdef\SetFigFont#1#2#3#4#5{\reset@font\fontsize{#1}{#2pt}\fontfamily{#3}\fontseries{#4}\fontshape{#5}\selectfont}\fi\endgroup \begin{picture}(2285,715)(2233,-3879)
\put(2850,-3571){\makebox(0,0)[b]{\smash{{\SetFigFont{11}{13.2}{\rmdefault}{\mddefault}{\updefault}{\color[rgb]{0,0,0}1}}}}}
\put(3900,-3571){\makebox(0,0)[b]{\smash{{\SetFigFont{11}{13.2}{\rmdefault}{\mddefault}{\updefault}{\color[rgb]{0,0,0}2}}}}}
\put(2326,-3548){\makebox(0,0)[b]{\smash{{\SetFigFont{8}{9.6}{\rmdefault}{\mddefault}{\updefault}{\color[rgb]{1,1,1}}}}}}
\put(3376,-3848){\makebox(0,0)[b]{\smash{{\SetFigFont{8}{9.6}{\rmdefault}{\mddefault}{\updefault}{\color[rgb]{0,0,0}}}}}}
\put(4426,-3548){\makebox(0,0)[b]{\smash{{\SetFigFont{8}{9.6}{\rmdefault}{\mddefault}{\updefault}{\color[rgb]{1,1,1}}}}}}
\put(3376,-3248){\makebox(0,0)[b]{\smash{{\SetFigFont{8}{9.6}{\rmdefault}{\mddefault}{\updefault}{\color[rgb]{1,1,1}}}}}}
\end{picture} 
\begin{picture}(0,0)\includegraphics{Graph3_11.pstex}\end{picture}\setlength{\unitlength}{3947sp}\begingroup\makeatletter\ifx\SetFigFont\undefined \gdef\SetFigFont#1#2#3#4#5{\reset@font\fontsize{#1}{#2pt}\fontfamily{#3}\fontseries{#4}\fontshape{#5}\selectfont}\fi\endgroup \begin{picture}(2285,715)(2233,-3879)
\put(2850,-3571){\makebox(0,0)[b]{\smash{{\SetFigFont{11}{13.2}{\rmdefault}{\mddefault}{\updefault}{\color[rgb]{0,0,0}1}}}}}
\put(3900,-3571){\makebox(0,0)[b]{\smash{{\SetFigFont{11}{13.2}{\rmdefault}{\mddefault}{\updefault}{\color[rgb]{0,0,0}2}}}}}
\put(3376,-3248){\makebox(0,0)[b]{\smash{{\SetFigFont{8}{9.6}{\rmdefault}{\mddefault}{\updefault}{\color[rgb]{0,0,0}}}}}}
\put(3376,-3848){\makebox(0,0)[b]{\smash{{\SetFigFont{8}{9.6}{\rmdefault}{\mddefault}{\updefault}{\color[rgb]{1,1,1}}}}}}
\put(4426,-3548){\makebox(0,0)[b]{\smash{{\SetFigFont{8}{9.6}{\rmdefault}{\mddefault}{\updefault}{\color[rgb]{0,0,0}}}}}}
\put(2326,-3548){\makebox(0,0)[b]{\smash{{\SetFigFont{8}{9.6}{\rmdefault}{\mddefault}{\updefault}{\color[rgb]{0,0,0}}}}}}
\end{picture} 
\begin{picture}(0,0)\includegraphics{Graph3_12.pstex}\end{picture}\setlength{\unitlength}{3947sp}\begingroup\makeatletter\ifx\SetFigFont\undefined \gdef\SetFigFont#1#2#3#4#5{\reset@font\fontsize{#1}{#2pt}\fontfamily{#3}\fontseries{#4}\fontshape{#5}\selectfont}\fi\endgroup \begin{picture}(2285,624)(2233,-3879)
\put(2850,-3571){\makebox(0,0)[b]{\smash{{\SetFigFont{11}{13.2}{\rmdefault}{\mddefault}{\updefault}{\color[rgb]{0,0,0}1}}}}}
\put(3900,-3571){\makebox(0,0)[b]{\smash{{\SetFigFont{11}{13.2}{\rmdefault}{\mddefault}{\updefault}{\color[rgb]{0,0,0}2}}}}}
\put(4426,-3548){\makebox(0,0)[b]{\smash{{\SetFigFont{8}{9.6}{\rmdefault}{\mddefault}{\updefault}{\color[rgb]{0,0,0}}}}}}
\put(2326,-3548){\makebox(0,0)[b]{\smash{{\SetFigFont{8}{9.6}{\rmdefault}{\mddefault}{\updefault}{\color[rgb]{0,0,0}}}}}}
\put(3376,-3848){\makebox(0,0)[b]{\smash{{\SetFigFont{8}{9.6}{\rmdefault}{\mddefault}{\updefault}{\color[rgb]{0,0,0}}}}}}
\end{picture} 
\textrm{{\emph{Questions 1 and 2 remain open in
the Boolean case}}}.


{\subsection*{Acknowledgments}}


\noindent
Example 6 has been obtained with Jean-Paul Comet that I gratefully
thank. I also thank Bruno Soubeyran for stimulating discussions.

\begin{thebibliography}{7}

\bibitem{ADG04}J. Aracena, J. Demongeot, E. Goles, Positive and negative circuits in
discrete neural networks, {\emph{IEEE Trans. Neural
Networks}}, 15 (2004) 77-83.

\bibitem{A08}J. Aracena, Maximum number of fixed points in regulatory boolean
networks, {\emph{Bull. Math. Biol.}}, 70 (2008)
1398-1409.

\bibitem{BM00}
J. Bahi, C. Michel, Convergence of discrete asynchronous iterations,
{\emph{Int. J. Comput. Math.}}, 74 (2000) 113-125.

\bibitem{CD02}O. Cinquin, J. Demongeot, Positive and negative feedback: a striking
balance between necessary antagonists, {\emph{J. Theor. Biol.}}, 216
(2002) 229-241.

\bibitem{GK73}L. Glass, S.A. Kauffman, The logical analysis of continuous non linear
biochemical control networks, {\emph{J. Theor. Biol.}},
39 (1973) 103-129.

\bibitem{G98}J. L. Gouz\'e, Positive and negative circuits in dynamical systems,
{\emph{J. Biol. Syst.}}, 6 (1998) 11-15.

\bibitem{dJ02}H. de Jong, Modeling and simulation of genetic regulatory systems: a
literature review, {\emph{J. Comput. Biol.}}, 9 (2002)
67-103.

\bibitem{dJ04}H. de Jong, J.-L. Gouz\'e, C. Hernandez, M. Page, S. Tewfik,
J. Geiselmann, Qualitative simulation of genetic regulatory networks
using piecewise-linear models, {\emph{Bull. Math. Biol.}}, 66 (2004) 301-340.

\bibitem{KST07}M. Kaufman, C. Soul\'e, R. Thomas, A new necessary condition on
interaction graphs for multistationarity, {\emph{J.
Theor. Biol.}}, 248 (2007) 675-685.

\bibitem{PM95}E.~Plathe,~T.~Mestl,~S.W.~Omholt,~Feedback~loops,~stability~and
multistationarity in dynamical systems, {\emph{J. Biol. Syst.}}, 3
(1995) 569-577.

\bibitem{RR08}E. Remy, P. Ruet, D. Thieffry, Graphics requirement for multistability
and attractive cycles in a boolean dynamical framework,
{\emph{Adv. Appl. Math.}}, 41 (2008) 335-350.

\bibitem{R06}A. Richard, {\emph{Mod\`ele formel pour les r\'eseaux de r\'egulation
g\'en\'etique et influence des circuits de r\'etroaction}},
Ph.D. Thesis, University of Evry Val d'Essonne, France, 2006.

\bibitem{RC07}A. Richard, J.-P. Comet, Necessary conditions for multistationarity in
discrete dynamical systems, {\emph{Discrete Appl. Math.}}, 155 (2007)
2403-2413.

\bibitem{R08}
A. Richard, An extension of a combinatorial fixed point theorem of
Shih and Dong, {\emph{Adv. Appl. Math.}}, 41 (2008) 620-627.  

\bibitem{R09}A. Richard, Positive circuits and maximal number of fixed points in
discrete dynamical systems, {\emph{Discrete Appl. Math.}},
(2009) in press.

\bibitem{R86}F. Robert, Discrete iterations: a metric study, in: Series in
Computational Mathematics, Vol. 6, Springer-Verlag,
Berlin-Heidelber-New York, 1986.

\bibitem{R95}F. Robert, Les syst\`emes dynamiques discrets, in: Math\'ematiques et
Applications, Vol.~19, Springer-Verlag, Berlin-Heidelber-New York,
1995.

\bibitem{SD05}M.-H. Shih and J.-L. Dong, A combinatorial analogue of the Jacobian
problem in automata networks, {\emph{Adv. Appl. Math.}}, 34 (2005)
30-46.

\bibitem{S89}E. H. Snoussi, Qualitative dynamics of a piecewise-linear differential
equations : a discrete mapping approach,
{\emph{Dynam. Stabil. Syst.}}, 4 (1989) 189-207.

\bibitem{ST93}E.H. Snoussi, R. Thomas, Logical identification of all steady states :
the concept of feedback loop caracteristic states,
{\emph{Bull. Math. Biol.}}, 55 (1993) 973-991.

\bibitem{S98}E.H. Snoussi, Necessary conditions for multistationarity and stable
periodicity, {\emph{J. Biol. Syst.}}, 6 (1998) 3-9.

\bibitem{S03}C. Soul\'e, Graphical requirements for multistationarity,
\emph{ComPlexUs}, 1 (2003) 123-133.

\bibitem{S06}C. Soul\'e, Mathematical approaches to differentiation and gene
regulation, \emph{C.R. Paris Biologies}, 329 (2006) 13-20.

\bibitem{T73}R. Thomas, Boolean formalization of genetic control
circuits, {\emph{J. Theor. Biol.}}, 42 (1973) 563-585.

\bibitem{T81}R. Thomas, On the relation between the logical structure of systems
and their ability to generate multiple steady states and sustained
oscillations, in: \emph{Series in Synergetics}, volume 9, pages
180-193, Springer, 1981.

\bibitem{TA90}R. Thomas, R. d'Ari, \emph{Biological Feedback}, CRC Press, 1990.

\bibitem{T91}R. Thomas, Regulatory Networks Seen as Asynchronous Automata : A
logical Description, {\emph{J. Theor. Biol.}}, 153
(1991) 1-23.

\bibitem{TK01}R. Thomas, M. Kaufman, Multistationarity, the basis of cell
differentiation and memory. I. \& II., \emph{Chaos}, 11 (2001)
170-195.

\end{thebibliography}


\end{document}
