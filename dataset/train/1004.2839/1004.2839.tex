\documentclass[a4paper,11pt]{article}

\usepackage{authblk}

\usepackage{amsmath}
\usepackage{amsthm}
\usepackage{amssymb}

\usepackage{algorithmic}
\usepackage{algorithm}

\usepackage{graphicx}
\usepackage{epsfig}

\usepackage[left=1.0in,top=1.0in,right=1.0in,bottom=1.0in,nohead]{geometry}

\theoremstyle{definition}
\newtheorem{definition}{Definition}
\newtheorem{observation}{Observation}

\newtheorem{theorem}{Theorem}
\newtheorem{corollary}[theorem]{Corollary}
\newtheorem{lemma}[theorem]{Lemma}
\newtheorem{claim}[theorem]{Claim}

\newcommand{\G}{\mathcal G}
\newcommand{\V}{\mathcal V}
\newcommand{\E}{\mathcal E}

\long\def\longdelete#1{}



\title{\bf Approximation Algorithms for the Capacitated Domination Problem\thanks{
This work was supported in part by the National Science Council, Taipei 10622, Taiwan, under
the Grants NSC98-2221-E-001-007-MY3 and NSC98-2221-E-001-008-MY3.}}

\author[ ]{Mong-Jen Kao}
\author[ ]{Han-Lin Chen}

\affil[ ]{Department of Computer Science and Information Engineering, \newline
National Taiwan University, Taiwan.}

\affil[ ]{\newline Emails: d97021@csie.ntu.edu.tw, kalent37@ms89.url.com.tw}

\date{\empty}

\begin{document}

\maketitle



\begin{abstract}
We consider the {\em Capacitated Domination} problem, which models a
service-requirement assignment scenario and is also a
generalization of the well-known {\em Dominating Set} problem. In
this problem, given a graph with three parameters defined on each vertex, namely cost,
capacity, and demand, we want to find
an assignment of demands to vertices of least cost such that the demand of each vertex is
satisfied subject to the capacity constraint of each vertex providing the service.

In terms of polynomial time approximations, we present logarithmic approximation algorithms with respect to different demand assignment models for this problem on general
graphs, which also establishes the corresponding approximation
results to the well-known approximations of the
traditional {\em Dominating Set} problem. Together with our
previous work, this closes the problem of generally approximating
the optimal solution. On the other hand, from the perspective of
parameterization, we prove that this problem is {\it W[1]}-hard when parameterized by a structure of the graph called treewidth. Based on this hardness result, we present exact fixed-parameter tractable algorithms when parameterized by treewidth and maximum capacity of the vertices.
This algorithm is further extended to obtain pseudo-polynomial time approximation schemes for planar graphs.
\end{abstract}



\section{Introduction}

For decades, {\em Dominating Set} problem has been one of the most
fundamental and well-known problems in both 
graph theory and combinatorial optimization. Given a graph  and an integer ,
{\em Dominating Set} asks for a subset  whose cardinality
does not exceed  such that every vertex in the graph either
belongs to this set or has a neighbor which does. As this problem is
known to be NP-hard, approximation algorithms have been proposed in
the literature. On one hand, a simple greedy algorithm is shown to
achieve a guaranteed ratio of  \cite{VC79,804034,LL75}, where  is the
number of vertices, which is later proven to be the approximation
threshold by Feige \cite{285059}. On the other hand, algorithms
based on dual-fitting provide a guaranteed ratio of 
\cite{DSH82}, where  is the maximum degree of the vertices of the graph.

\smallskip

In addition to polynomial time approximations, {\em Dominating
Set} has its special place from the perspective of parameterized complexity as well \cite{RGDMRF99,JFMG06,RN06}. In
contrast to {\em Vertex Cover}, which is fixed-parameter tractable (FPT),
{\em Dominating Set} has been proven to be {\it W[2]}-complete when
parameterized by solution size, in the sense that no fixed-parameter
algorithm exists (with respect to solution size) unless FPT={\it W[2]}. Though {\em Dominating Set} is a fundamentally hard problem in the parameterized -hierarchy, it has been used as a benchmark problem for developing {\it sub-exponential time} parameterized algorithms \cite{JAHLBHFTKRN02,1101823,1139978} and 
{\it linear size kernels} have been obtained in planar graphs \cite{990309,JFMG06,DBLP:conf/icalp/GuoN07,RN06}, and more generally, in graphs that exclude a fixed graph  as a minor.

\smallskip

Besides {\em Dominating Set} problem itself, a vast body of
work has been proposed in the literature, considering possible variations from purely theoretical aspects to practical applications. See \cite{TWHSHPS98,FSR78} for a detailed survey. In particular, variations of {\em Dominating Set} problem occur in numerous practical settings, ranging from strategic decisions, such as locating radar stations or emergency services, to computational biology and to voting systems. For example, Haynes et al. \cite{587927} considered {\em Power Domination Problem} in electric networks \cite{587927,DBLP:conf/cocoon/LiaoL05} while Wan et al. \cite{PJWKMAOF03} considered {\em Connected Domination Problem} in wireless ad hoc networks.

\smallskip

Motivated by a general service-requirement assignment model, Kao et al., \cite{MJKCSLDTL09} considered a generalized domination problem called {\em Capacitated Domination}. In this problem, the input graph is given with tri-weighted vertices, referred to as {\em cost}, {\em capacity}, and {\em demand}, respectively. The demand of a vertex stands for the amount of service it requires from its adjacent vertices (including itself) while the capacity of a vertex represents the amount of service it can provide when it's selected as a server. The goal of this problem is to find a dominating multi-set as well as a demand assignment function such that the overall cost of the multi-set is minimized. For different underlying applications, there are two different demand assignment models, namely {\em splittable} demand model and {\em unsplittable} demand model, depending on whether or not the demand of a vertex is allowed to be served by different vertices. Moreover, there has been work studying the variation when the number of copies, or {\it multiplicity} of each vertex in the dominating multi-set, is limited, referred to as {\em hard} capacity, and as {\em soft} capacity when no such limit is specified. Kao et al., \cite{MJKCSLDTL09} considered the soft capacitated domination problem with splittable demand and provided a -approximation for general graphs, where  is the degree of the graph. For special graph classes, they proved that even when the input graph is restricted to a tree, the soft capacitated domination problem with splittable demand remains NP-hard, for which they also presented a polynomial time approximation scheme. Dom et al., \cite{DBLP:conf/iwpec/DomLSV08} considered the hard capacitated domination problem with uniform demand and showed that this problem is {\it W[1]}-hard even when parameterized by treewidth and solution size.

\smallskip



In this paper, we consider the (soft) {\em Capacitated Domination} problem and present logarithmic approximation algorithms with respect to different demand assignment models on general graphs. Specifically, we provide a -approximation
for weighted unsplittable demand model, a -approximation for weighted splittable demand model, and a -approximation for unweighted splittable demand model, where  is the number of vertices. Together with the -approximation result given by Kao et al., \cite{MJKCSLDTL09}, this establishes a corresponding near-optimal approximation result to the original {\em Dominating Set} problem. Although the result may look natural, the greedy choice we make is not obvious when non-uniform capacity as well as non-uniform demand is taken into consideration. On the other hand, from the perspective of parameterization, we prove that this problem is {\it W[1]}-hard when parameterized by a structure of the graph, called the treewidth, and present exact FPT algorithms when parameterized by the treewidth and the maximum capacity of the vertices. 
This algorithm is further extended to obtain pseudo-polynomial time approximation schemes for planar graphs, based on a framework due to Baker \cite{174650}.

\smallskip

The rest of this paper is organized as follows. In Section~\ref{preliminary}, we give formal definitions and notation adopted in the paper. In Section~\ref{logarithmic_approximations}, we present our ideas and algorithms that achieve the aforementioned approximation guarantees. We present the parameterized results in Section~\ref{parameterized_results} and conclude by listing some future work in Section~\ref{conclusion}.



\section{Preliminary} \label{preliminary}

We assume that all the graphs considered in this paper are simple and undirected. Let  be a graph with vertex set  and edge set . A vertex  is said to be adjacent to a vertex  if . The set of neighbors of a vertex  is denoted by . The closed neighborhood of  is denoted by .  The subscript  in  will be omitted when there is no confusion.

\smallskip

Consider a graph  with tri-weighted vertices, referred to as the cost, the capacity, and the demand of each vertex , denoted by , , and , respectively. Let  denote a multi-set of vertices of  and for any vertex , let  denote the {\it multiplicity} of  or the number of times of  in .
The cost of , denoted , is defined to be .

\begin{definition}[Capacitated Dominating Set]
A vertex multi-subset  is said to be a feasible capacitated dominating set with respect to a demand assignment function  if the following conditions hold.
\begin{itemize}
    \item {\bf Demand constraint:} , for each .
    \item {\bf Capacity constraint:} , for each .
\end{itemize}
\end{definition}

Given a problem instance, the capacitated domination problem asks for a capacitated dominating multi-set  and demand assignment function  such that  is minimized. For unsplittable demand model we require that  is either  or  for each edge . Note that since it is already NP-hard\footnote{This can be verified by making a reduction from {\sc Subset Sum}.} to compute a feasible demand assignment function from a given feasible capacitated dominating multi-set when the demand cannot be split, it is natural to require the demand assignment function be specified, in addition to the optimal vertex multi-set itself.

\smallskip

Parameterized complexity is a well-developed framework for studying the computationally hard problem \cite{RGDMRF99,JFMG06,RN06}. A problem is called {\it fixed-parameter tractable} (FPT) with respect to a parameter  if it can be solved in time , where  is a computable function depending only on . Problems (along with its defining parameters) belonging to {\it W[t]}-hard for any  are believed not to admit any FPT algorithms (with respect to the specified parameters). Now we define the notion of parameterized reduction.

\begin{definition}
Let  and  be two parameterized problems. We say that  reduces to  by a standard parameterized reduction if there exists an algorithm  that transforms  into  in time , where  are arbitrary functions and  is a constant independent of  and , such that  if and only if .
\end{definition}

Next we define the concept of {\em tree decomposition} \cite{HLBAMCAK08,DBLP:books/sp/Kloks94}.

\begin{definition}[Tree Decomposition of a Graph]
A tree decomposition of a graph  is a pair  where each node  has associated with it a subset of vertices , called the bag of , such that
\begin{enumerate}
    \item{Each vertex belongs to at least one bag: .}
    \item{For all edges, there is a bag containing both its end-points.}
    \item{For all vertices , the set of nodes  induces a subtree of .}
\end{enumerate}
\end{definition}

The width of a tree decomposition is . The treewidth of a graph  is the minimum width over all tree decompositions of .

\begin{definition}[Nice Tree Decomposition \cite{DBLP:books/sp/Kloks94}]
A tree decomposition  is a nice tree decomposition if one can root  in such a way that each node  is of one of the four following types.
\begin{enumerate}
    \item{{\em Leaf}: node  is a leaf of , and .}
    \item{{\em Join}: node  has exactly two children, say  and , and .}
    \item{{\em Introduce}: node  has exactly one child, say , and there is a vertex  such that .}
    \item{{\em Forget}: node  has exactly one child, say , and there is a vertex  such that .}
\end{enumerate}
\end{definition}

Given a tree decomposition of width , a nice tree decomposition of the same width can be found in linear time \cite{DBLP:books/sp/Kloks94}.



\section{Logarithmic Approximation} \label{logarithmic_approximations}

In this section, we present logarithmic approximation algorithms for
capacitated domination problems with respect to different cost and
demand models. Specifically, we provide a -approximation
for weighted unsplittable demand model, a -approximation for weighted splittable demand model, and a
-approximation for unweighted splittable demand model,
where  is the number of vertices.

The main idea is based on greedy approach in the sense that we keep choosing
a vertex with the best efficiency in each iteration until the whole
graph is dominated.
By best efficiency we mean the maximum cost-efficiency ratio defined for 
each vertex in the remaining graph.
We describe the results in more detail in the following
subsections.

\subsection{Weighted Unsplittable Demand}
\label{section_weighted_unsplittable}

In this section, we consider the {\em weighted capacitated
domination problem with unsplittable demand} and provide a simple
greedy algorithm that achieves the approximation guarantee of .

\smallskip

Let  be the set of vertices which are not dominated yet.
Initially, we have . For each vertex , let
 be the set of undominated vertices in the closed neighborhood of .
Without loss of generality, we shall assume that
the elements of , denoted by , are
sorted in non-decreasing order of their demands in the remaining
section.

\smallskip

In each iteration, the algorithm chooses a vertex of the most efficiency
from , where the efficiency of a vertex, say , is defined by
the largest effective-cost ratio of the number of vertices dominated by  over the total cost. That is, 
 where  is
the number of copies of  selected in order to dominate
 and . A high-level description of this algorithm
is presented in Figure~\ref{Algorithm for Weighted Unsplittable Demand}.

\begin{figure*}[t]
\rule{\linewidth}{0.2mm}
\medskip
{{\sc Algorithm} {\em Unsplit-Log-Approx
}}

\begin{algorithmic}[1]
\STATE 
\WHILE{}
    \STATE Pick a vertex in  with the most efficiency, say .
    \STATE let .
    \STATE Assign the demand of each vertex in  to  and remove them from .
\ENDWHILE
\STATE compute from the assignment the weight of the dominating set, and return the result.

\end{algorithmic}
\rule{\linewidth}{0.2mm} 
\caption{The pseudo-code for the 
weighted unsplittable demand model.} \label{Algorithm for
Weighted Unsplittable Demand}
\end{figure*}

In iteration , let  be the cost of the optimal solution
for the remaining problem instance, which is clearly upper bounded
by the cost, , of the optimal solution for the input instance.
Let the number of undominated vertices at the beginning of iteration  be
, and the number of vertices that are newly dominated in iteration  be .

\smallskip

Denote by  the cost in iteration . Note that , where  is the most efficient vertex chosen
in iteration . Assume that the algorithm repeats for 
iterations. 
We have the following lemma.

\begin{lemma} \label{lemma_greedy_weighted_unsplittable}
For each , , we have .
\end{lemma}

\begin{proof}
Since we always choose the vertex with the maximum 
efficiency, the efficiency is no less than that of each vertex
chosen in , which is no less than the average of . Therefore we have
 and the lemma follows.
\end{proof}

\smallskip

\begin{theorem} \label{theorem_weighted_unsplittable_logn}
Algorithm {\em Unsplit-Log-Approx} computes a -approximation for weighted capacitated domination problem with unsplittable demands in  time, where  is the number of vertices.
\end{theorem}

\begin{proof}
To see that the algorithm produces a logarithmic approximation, take
the sum over each ,  and observe that , we have

To see the time complexity, notice that it requires  time to compute a most efficient move for each vertex, which leads to an  computation for the most efficient choice in each iteration. The number of iterations is upper bounded by  since at least one vertex is satisfied in each iteration.
\end{proof}



\subsection{Weighted Splittable Demand} \label{section_weighted_splittable}

In this section, we
present an algorithm that produces a -approximation for
the {\em weighted capacitated domination problem with splittable
demand}. The difference between this algorithm and the previous one lies in the way we handle the demand assignment.  In each iteration the demand of a vertex may be partially served. 
The unsatisfied portion of the demand is called {\it residue demand}. For each vertex , let  be the residue demand of .  is set equal to  initially, and will be updated accordingly when a portion of the residue demand is assigned.  is said to be completely satisfied when .

\smallskip

We will inherit the notation used in the previous
section. We assume that the elements of ,
written as , are sorted according to
their demands in non-decreasing order.

\smallskip

In each iteration, the algorithm performs two greedy choices. First, the
algorithm chooses the vertex of the most efficiency from , where
the efficiency is defined similarly as in the previous section with some modification since the demand is splittable.

\begin{figure*}[t]
\rule{\linewidth}{0.2mm}
\medskip
{{\sc Algorithm} {\em Split-Log-Approx}}

\begin{algorithmic}[1]
\STATE , and  for each .
\WHILE{there exist vertices with non-zero residue demand}
    \STATE // {\bf  greedy choice}
    \STATE Pick a vertex in  with the most efficiency, say .
    \IF{ equals }
        \STATE Assign this amount  of residue demand of  to .
        \STATE 
    \ELSE
        \STATE Assign the residue demands of the vertices in  to .
        \IF{}
            \STATE Assign this amount  of residue demand of  to .
            \STATE 
        \ENDIF
    \ENDIF
    \STATE
    \STATE // {\bf  greedy choice}
    \IF{there is a vertex  with }
        \STATE Satisfy  by doubling the demand assignment of  to vertices in .
    \ENDIF
\ENDWHILE
\STATE compute from the assignment the cost of the dominating set, and return the result.

\end{algorithmic}
\rule{\linewidth}{0.2mm} \caption{The pseudo-code for the 
weighted splittable demand model.} \label{Algorithm for
Weighted Splittable Demand}
\end{figure*}

\smallskip

For each vertex , let  with  be the maximum index such that
.
Let  be the sum of the effectiveness over the vertices whose residue demand could be completely served by a single copy of . In addition, we let  if  and  otherwise. The efficiency of  is defined as .

Second, the algorithm maintains for each vertex  a set of vertices, denoted by , which consists of vertices that have partially served the demand of  before  is completely satisfied. 
That is, for each  we have a non-zero demand assignment of  to . Whenever there exists a vertex  whose residue demand is below half of its original demand, i.e., , after the first greedy choice, the algorithm immediately doubles the demand assignment of  to the vertices in . Note that in this way, we can completely satisfy the demand of  since  . A high-level description of this algorithm is presented in Figure~\ref{Algorithm for
Weighted Splittable Demand}.

\begin{observation} \label{observation_splittable_half}
After each iteration, the residue demand of each unsatisfied vertex is at least half of its original demand.
\end{observation}

Clearly, the observation holds in the beginning when the demand of each vertex is not yet assigned. For later stages, we argue that the algorithm properly maintains  so that in our second greedy choice, whenever there exists a vertex  for which , it's always sufficient to double the demand assignment  of  to  for each . If  is only modified under the condition , (line 12 in Figure~\ref{Algorithm for Weighted Splittable Demand}), then  contains exactly the set of vertices that have partially served . As mentioned above, since , it's sufficient to double the demand assignment in this case so  is completely satisfied. If  is reassigned through the condition  for some stage, then we have . Since we assign this amount  of residue demand of  to , this leaves at most half of the original residue demand and  will be satisfied by doubling this assignment.

\smallskip

By the description given above, we conclude that the algorithm produces a feasible demand assignment as well as a feasible capacitated dominating set. Let the cost incurred by the first greedy choice be  and the cost by the second choice be . To see that the solution achieves the desired approximation guarantee, first notice that  is bounded above by , for what we do in the second choice is merely to satisfy the residue demand of a vertex, if there exists one, by doubling its previous demand assignment.

\smallskip

In the following, we will bound the cost .
For each iteration , let  be the vertex of the maximum efficiency and  be the cost of the optimal solution for the remaining problem instance. Let  denote the sum of effectiveness of each vertex in the remaining problem instance at the beginning of this iteration.
Let  be the cost incurred by the first greedy choice in iteration .
Assume that the algorithm repeats for  iterations. We have the following lemma.

\begin{lemma} \label{lemma_splittable_greedy_choice}
For each , , we have , where  is the effectiveness covered by  in iteration .
\end{lemma}

\begin{proof}
The optimality of our choice in each iteration is obvious since we assume that the elements of  are sorted according to their original demands. Note that only in the case , the algorithm could possibly take more than one copy. 
In this case the efficiency of our choice remains unchanged since the cost and the effectiveness covered by  grows by the same factor. Therefore the efficiency of our choice, , is always no less than that of the optimal solution, which is , and the lemma follows.
\end{proof}

\begin{observation} \label{observation_splittable_n_decrease}
We have  for each .
\end{observation}

\begin{proof}
For iteration , , let  be the vertex of the maximum efficiency. Observe that  will be satisfied after this iteration. By Observation~\ref{observation_splittable_half}, we have . The lemma follows.
\end{proof}

\smallskip

By Lemma~\ref{lemma_splittable_greedy_choice} we have 

since  for any real number  and  for each .

\begin{lemma}

\end{lemma}

\begin{proof}
Note that by Observation \ref{observation_splittable_half} and Observation \ref{observation_splittable_n_decrease}, we have  for all . We will argue that this series together constitutes at most two harmonic series. By expanding the summand we have 

Since , the repetitions only occur at the first term and the last term if we expand the summation. By Observation \ref{observation_splittable_n_decrease}, the decrease of  to  is at least half. Therefore, the term  will never occur more than twice in the expansion. We conclude that

\end{proof}

\begin{theorem}
Algorithm {\em Split-Log-Approx}
computes a -approximation in  time,
where  is the number of vertices, for
weighted capacitated domination problem with splittable demands.
\end{theorem}



\subsection{Unweighted Splittable Demand} \label{section_unweighted_splittable}

In this section, we consider the {\em unweighted capacitated domination problem with splittable demand} and present a -approximation. In this case the weight  of each vertex  is considered to be 1 and the cost of the capacitated domination multiset  corresponds to the total multiplicity of the vertices in .   To this end, we first make a greedy reduction on the problem instance by spending at most  cost such that it takes at most one copy to satisfy each remaining unsatisfied vertex. Then we show that a -approximation can be computed for the remaining problem instance, based on the same framework of Section~\ref{section_weighted_splittable}.

\smallskip

For each , let  be the vertex in  with the maximum capacity. First, for each , we assign this amount  of the demand of  to . Let the cost of this assignment be , then we have the following lemma.

\begin{lemma} \label{lemma_last_copy}
We have , where  is the cost of the optimal solution.
\end{lemma}

\begin{proof}
Notice that an optimal solution  for the relaxation of this problem, where fractional copies are allowed, can be obtained by assigning the demand  of  to . Since  and , the lemma follows.
\end{proof}

In the following, we will assume that , for each . The algorithm of Section~\ref{section_weighted_splittable} is slightly modified. In particular, for the second greedy choice, whenever  for some vertex , we immediately assign the residue demand of  to . A high-level description of this algorithm 
is presented in Figure \ref{Algorithm for Unweighted Splittable Demand}.

\begin{figure*}[t]
\rule{\linewidth}{0.2mm}
\medskip
{{\sc Algorithm} {\em Unweighted-Split-Log-Approx}}

\begin{algorithmic}[1]
\STATE For each , assign  demands of  to , where  has the maximum capacity.
\STATE Reset the demands of the instance by setting  for each .
\WHILE{there exist vertices with non-zero residue demand}
    \STATE // {\bf  greedy choice}
    \STATE Pick a vertex in  with the most efficiency, say .
        \STATE Assign the demands of the vertices in  to .
        \IF{}
            \STATE Assign this amount  of the residue demand of  to .
        \ENDIF
    \STATE
    \STATE // {\bf  greedy choice}
    \IF{there is a vertex  with }
        \STATE Satisfy  by assigning the residue demand of u to .
    \ENDIF
\ENDWHILE
\STATE compute from the assignment the cost of the dominating set, and return the result.

\end{algorithmic}
\rule{\linewidth}{0.2mm} \caption{The pseudo-code for the 
unweighted splittable demand model.} \label{Algorithm for Unweighted Splittable Demand}
\end{figure*}

\begin{observation} \label{observation_unweighted_splittable_n_decrese}
We have  for each .
\end{observation}

\begin{proof}
Observe that in each iteration, at least one vertex is satisfied and the residue demand of each unsatisfied vertex is equal to its original demand.
\end{proof}

Clearly,  is bounded above by , as we always take one copy for the first greedy choice and at most one copy for the second greedy choice in each iteration.
By Observation~\ref{observation_unweighted_splittable_n_decrese} and the fact that  is integral for each , we have 

and 

We conclude the result as the following theorem.

\begin{theorem}
Algorithm {\em Unweight-Split-Log-Approx} computes
a -approximation in  time for weighted capacitated domination
problem with unsplittable demands, where  is the number of
vertices.
\end{theorem}



\section{Parameterized Results} \label{parameterized_results}

\subsection{Hardness Results} \label{section_w_1_hardness}

In this section we show that {\em Capacitated Domination Problem} is {\it W[1]}-hard when parameterized by
treewidth by making a reduction from {\em k-Multicolor Clique}, a
restriction of {\em k-Clique} problem.

\begin{definition}[\sc Multicolor Clique]
Given an integer  and a connected undirected graph  such that  induces an independent set for each , the {\sc Multicolor Clique} problem asks whether or not there exists a clique of size  in .
\end{definition}

Given an instance  of {\sc Multicolor Clique}, we will show how an instance  of {\em Capacitated Domination} with treewidth  can be built such that  has a clique of size  if and only if  has a capacitated dominating set of cost at most . For convenience, we shall distinguish the vertices of  by referring to them as {\em nodes}.

\smallskip

Let  be the number of vertices. Without loss of generality, we label the vertices of  by numbers, denoted , , between  and . For each , let  denote the set of edges between  and . The graph  is defined as follows. For each , , we create a node  with , , and . For each , we have a node  with , , and . We also connect  to . For convenience, we refer to the star rooted at  as vertex star . 

\smallskip

Similarly, for each , we create a node  with , , and . For each  we have a node  with , , and . We connect  to . We refer to the star rooted at  as edge star . The selection of nodes in  and  in the capacitated dominating set will correspond to the choices made in selecting the vertices that form a clique in .

\smallskip

In addition, for each , , we create two bridge nodes ,  with  and . The capacities of the bridge nodes are to be defined later. Now we describe the way how stars  and  are connected to bridge nodes such that the reduction claimed above holds. For each ,  and for each , 
we create two propagation nodes ,  and connect them to . Besides, we connect  to  and  to . We set  and . The demands of  and  are set to be  and . For each  and for each , we create four propagation nodes , , , and  with zero capacity and  cost. Without loss of generality, we assume that  and . The demands of the four nodes are set as the following: , , , and . 
Finally, for each bridge node , we set .

\begin{figure}[h]
\centering
\caption{The connections between stars and bridge nodes.}
\includegraphics{fig_w_hardness}
\end{figure}

\begin{lemma} \label{lemma_w_1_hardness}
The treewidth of  is . Furthermore,  admits a clique of size  if and only if  admits a capacitated dominating set of cost at most .
\end{lemma}

\begin{proof}[Proof of Lemma~\ref{lemma_w_1_hardness}.]
Consider the set of bridge nodes, . Since  is a forest and the removal of a vertex from a graph decreases the treewidth of the graph by at most one, the treewidth of  is upper bounded by the number of bridge nodes plus some constant, which is .

Let  be a clique of size  in . By choosing the bridge nodes,  and  for each ,  for each , and  for each  exactly once, we have a vertex subset of cost exactly . One can easily verify that this is also a feasible capacitated dominating multi-set for .

On the other hand, let  be a capacitated donimating multi-set of cost at most  in . First observe that none of the propagation nodes are chosen in , otherwise the cost would exceed . This implies  and  for each . Note that this already contributes cost at least  to  and the rest of the nodes in  together contributes at most . 

Similarly, we have  and  for each . Therefore, for each ,  such that , and for each ,  such that . Since we have  such stars, exactly one node from each star is chosen to be included in  and each node of  is chosen exactly once. Next we argue that the nodes chosen in each star will correspond to a clique of size  in .

For each , let  and  be the nodes chosen in . Let  be the node chosen in . In the following we shall prove that . Since the capacity of  equals the sum of the demands over , the closed neighborhood of , without loss of generality we can assume that the demands of nodes in  are served by . Consider the bridge vertex  and the set . The demands of vertices in  can only be served by either  or , as they are the only two vertices in  chosen to be included in . In particular, vertices in  apart from  can only be served by . Notice that the sum of the demands in  is  above the capacity of . Therefore we have , which implies  as well since we have  by our setting.

By a symmetric argument on  we obtain . Hence . By another symmetric argument on  and , we have . Therefore  by our construction.
\end{proof}

Note that this proof holds for both splittable and unsplittable demand models. We have the following theorem.

\begin{theorem} \label{theorem_w_1_hardness}
The {\em Capacitated Domination problem} is {\it W[1]}-hard when parameterized by treewidth.
\end{theorem}

\begin{proof}[Proof of Theorem~\ref{theorem_w_1_hardness}.]
This theorem follows directly from Lemma~\ref{lemma_w_1_hardness} and the fact that this reduction can be computed in time polynomial in both  and .
\end{proof}




\subsection{FPT Algorithms on Graphs of Bounded Treewidth}

In this section we show that {\em Capacitated Domination Problem} with unsplittable demand is FTP when parameterized by both treewidth and maximum capacity by giving a  exact algorithm.

\smallskip

To this end, we give a dynamic programming algorithm on a so-called {\em nice tree decomposition} \cite{DBLP:books/sp/Kloks94} of the input graph G. In the following, without loss of generality, we shall assume that the bag associated with the root of  is empty. For each node  in the tree , let  be the subtree rooted at  and . Starting from the leaf nodes of , our algorithm proceeds in a bottom-up manner and maintains for each node  of  a table  whose columns consist of the following information.

\begin{itemize}
\item{ with  indicating the set of vertices in  that have been served, and}
\item{ with  indicating the residue capacity of , for each .}
\end{itemize}

Clearly, each row of  corresponds to a possible configuration consisting of the unsatisfied vertices and the residue capacity of each vertex in  that can be used. The algorithm computes for each row of  the cost of the optimal solution to the subgraph induced by  under the constraint that the configuration of vertices in  agrees with that specified by the values of the row.

\smallskip

In the following, we describe the computation of the table  for each node  in the tree  in more detail. In order to keep the content clean, we use the terms "insert a new row" and "replace an old row by the new one" interchangeably. Whenever the algorithm attempts to insert a new row into a table while another row with identical configuration already exists, the one with the smaller cost will be kept.  According to different types of vertices we encounter during processing, we have the following situations.

\begin{itemize}
\item{\bf  is a leaf node.} Let . We add two rows to the table  which correspond to cases whether or not  is served.

\smallskip

\begin{algorithmic}[1]
\STATE let  be a new row with 
	\STATE let  mod  be a new row with 
	\STATE add  and  to 
	\newline \rule{\linewidth}{0.2mm}
\end{algorithmic}

\item{\bf  is an introduce node.} Let  be the child of , and let . The data in  is basically inherited by .We extend  by considering, for each existing row  in , all  possible ways of choosing vertices in  to be assigned to . In addition,  can be either unassigned or assigned to any vertex in . In either case, the cost and the residue capacity are modified accordingly. 

\smallskip

\begin{algorithmic}[1]
\FORALL{row }
    \FORALL{possible  such that }
        \STATE let  mod , and
        \newline let  be a new row with
        \newline 
        \STATE add  to 
        \FORALL{}
            \STATE let  mod  be a new row with
            \newline  the cost required by this assignment
            \STATE add  to 
        \ENDFOR
    \ENDFOR
\ENDFOR
\newline \rule{\linewidth}{0.2mm}
\end{algorithmic}

\item{\bf  is a forget node.} Let  be the child of , and let . In this case, for each row  such that , we insert a row  to  identical to  except for the absence of  in . The remaining rows in , which correspond to situations where  is not served, are ignored without being considered.

\smallskip

\begin{algorithmic}[1]
\FORALL{row  such that }
	\STATE let  be a new row with 
	\STATE add  to 
\ENDFOR
\newline \rule{\linewidth}{0.2mm}
\end{algorithmic}

\item{\bf  is a join node.} Let  and  be the two children of  in . We consider every pair of rows ,  where  and . We say that two rows  and  are {\em compatible} if . For each compatible pair of rows , we insert a new row  to  with ,  mod , for each , and .

\smallskip

\begin{algorithmic}[1]
\FORALL{compatible pairs  and }
	\STATE let  be a new row.
	\STATE , and
	\STATE  mod 
	\STATE add  to 
\ENDFOR
\newline \rule{\linewidth}{0.2mm}
\end{algorithmic}

\end{itemize}

\begin{theorem}
Capacitated Domination problem with unsplittable demand on graphs of bounded treewidth can be solved in time , where  is the treewidth and  is the largest capacity.
\end{theorem}

\begin{proof}
The correctness of the algorithm follows from the description above. The running time for computing the table  associated with each tree node  is bounded above by the time taken on the join nodes, which is clearly . The theorem follows.
\end{proof}

\smallskip

We state without going into details that by suitably replacing the set  with the residue demand  for each vertex  in the column of the table we maintained, the algorithm can be modified to handle the splittable demand model. We have the following corollary.

\begin{corollary}
Capacitated Domination problem with splittable demand on graphs of bounded treewidth can be solved in time , where  is the treewidth,  is the largest capacity, and  is the largest demand.
\end{corollary}



\subsection{Extension to Planar Graphs}

In this section we extend the above FPT algorithms based on a framework due to Baker \cite{174650} to obtain a pseudo-polynomial time approximation scheme for planar graphs. In particular, for unsplittable demand model, given a planar graph  with maximum capacity  and an integer , the algorithm computes an -approximation in time , where  is the number of vertices. Taking , where  is some constant, we get a pseudo-polynomial time approximation algorithm which converges toward optimal as  increases. On the other hand, for splittable demand model, we have a pseudo-polynomial time approximation scheme in  time, where  is the maximum demand. To get rid of the factor , we could apply the transformation used in Section~\ref{section_unweighted_splittable} and Lemma~\ref{lemma_last_copy} in advance and obtain a -approximation in  time.

\smallskip

This is done as follows. Given a planar graph , we generate a planar embedding and retrieve the vertices of each level using the linear-time algorithm of Hopcroft and Tarjan \cite{321852}. Let  be the number of levels of this embedding. Let  be the cost of the optimal capacitated dominating set of , and  be the cost contributed by vertices at level . Since , there exists one  with  such that 

For each , let  be the graph induced by vertices between level  and . In addition, we set the demands of vertices at level  and level  to be zero for each . Clearly, the treewidth of each  is upper bounded by  and the sum of the optimal cost for each  is no more than . Take  and we're done.



\section{Concluding Remarks} \label{conclusion}

In this paper we considered the {\em Capacitated Domination} problem, which is a generalization of the well-known {\em Dominating Set} problem and which models a service-requirement assignment scenario. In terms of polynomial time approximations, we have presented logarithmic approximation algorithms with respect to different demand assignment models for this problem on general graphs. Together with our previous
work on generally approximating this problem, this establishes the corresponding approximation results to the well-known approximations of the traditional {\em Dominating Set} problem and closes the problem of generally approximating the optimal solution. On the other hand, from the perspective of parameterization, we have proved that this problem is {\it W[1]}-hard when parameterized by treewidth of the graph. Based on this hardness result, we presented exact FPT algorithms when parameterized by treewidth and maximum capacity of the vertices. This algorithm is further extended under a framework of Baker \cite{174650} to obtain approximations for planar graphs.

We conclude with a few open problems and future research goals. First, although exact FPT algorithms are provided, the problem of approximating the optimal solution when parameterized by treewidth remains open. It would be nice to obtain faster approximation algorithms for graphs of bounded treewidth as this would provide faster approximations for planar graphs as well. Second, it would be nice to know how the problem behaves on special graph classes. As this problem has been shown to be difficult and admit a PTAS on trees when the demand can be split, approximations for other classes such as interval graphs remain unknown. Third, from the perspective of parameterization, it may be possible to find other parameters that are more closely related to the problem and obtain better results.



\bibliographystyle{plain}

\small
\bibliography{approx_capacitated_domination}

\end{document}
