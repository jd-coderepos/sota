\subsection{Programs and their Declarative Semantics}
\label{sec:sqclp:programs}



A $\sqclp{\simrel}{\qdom}{\cdom}$-program is a set $\Prog$ of  \emph{qualified program rules} (also called \emph{qualified clauses}) of the form $C : A \qgets{\alpha} \qat{B_1}{w_1}, \ldots, \qat{B_m}{w_m}$, where $A$ is a defined atom, $\alpha \in \aqdomd{\qdom}$ is called the {\em attenuation factor} of the clause and each $\qat{B_j}{w_j} ~ (1 \le j \le m)$ is an atom $B_j$ annotated with a so-called {\em threshold value} $w_j \in \bqdomd{\qdom}$.
The intended meaning of $C$ is as follows:  
if for all $1 \leq j \leq m$ one has $\qat{B_j}{e_j}$ (meaning that $B_j$ holds with qualification value $e_j$)
for some $e_j \dgeq^? w_j$,
then $\qat{A}{d}$ (meaning that $A$ holds with qualification value $d$)
can be inferred for any $d \in \aqdom$ such that $d \dleq \alpha \circ \infi_{j = 1}^m e_j$. 
By convention, $e_j \dgeq^? w_j$ means $e_j \dgeq w_j$ if $w_j ~{\neq}~?$ and is identically true otherwise.
In practice threshold values equal to `?' and attenuation values equal to $\tp$ can be omitted. 




As motivating example, consider a $\sqclp{\simrel}{\U\!\!\otimes\!\!\W}{\rdom}$-program $\Prog$ 
including the clauses and equations for $\simrel$ displayed in Figure \ref{fig:sample}. 
From Subsection \ref{sec:domains:qdoms} recall that qualification values in $\U\!\otimes\!\W$ are pairs
$(d,e)$ (where $d$ represents a certainty degree and $e$ represents a proof cost),
as well as the behavior of $\dleq$ and $\circ$ in $\U\!\otimes\!\W$. 
Consider the problem of proving $\qat{\texttt{goodWork(king\_liar)}}{(d,e)}$ from $\Prog$.
This can be achieved for $d = 0.75 \times \mbox{min}\{d_1,d_2\}$, $e = 3 + \mbox{max}\{e_1,e_2\}$
by using $R_1$ instantiated by $\{{\tt X} \mapsto {\tt king\_liar}, {\tt Y} \mapsto {\tt shakespeare}\}$,
and going on to prove  $\qat{\texttt{famousAuthor(shakespeare)}}{(d_1,e_1)}$
for some  $d_1 \geq 0.5$,  $e_1 \leq 100$ 
and $\qat{\texttt{wrote(shakespeare,king\_liar)}}{(d_2,e_2)}$ for some $d_2$, $e_2$.
Thanks to $R_2$, $R_3$ and $\simrel$, these proofs succeed with $(d_1,e_1) = (0.9,1)$ and $(d_2,e_2) = (0.8,2)$.
Therefore, the  desired proof succeeds with certainty degree  $d = 0.75 \times \mbox{min}\{0.9,0.8\} = 0.6$, 
and proof cost $e = 3 + \mbox{max}\{1,2\} = 5$.

\begin{figure}[h]
\figrule
\small
\flushleft
\hspace{3mm} $R_1$ : \verb+goodWork(X) <-(0.75,3)- famousAuthor(Y)#(0.5,100), wrote(Y,X)#?+\\
\hspace{3mm} $R_2$ : \verb+famousAuthor(shakespeare) <-(0.9,1)-+\\
\hspace{3mm} $R_3$ : \verb+wrote(shakespeare,king_lear) <-(1,1)-+\\[2mm]
\hspace{1cm} $\simrel$\verb+(king_lear,king_liar) = (0.8,2)+
\caption{$\sqclp{\simrel}{\,\U\!\otimes\!\W}{\rdom}$ Program Fragment}
\label{fig:sample}
\figrule
\vspace{-3mm}
\end{figure}



It is useful to define some special types of program clauses and programs, as follows:

\begin{itemize}
\item
A clause is called {\em attenuation-free} iff $\alpha = \tp$. The name is justified because $\tp$
is an identity element for the attenuation operator $\circ$, as explained in Subsection
\ref{sec:domains:qdoms}. By convention, attenuation-free clauses may be written with the simplified
syntax $A \gets \qat{B_1}{w_1}, \ldots, \qat{B_m}{w_m}$.
\item
A clause is called {\em threshold-free} iff $w_j =\,\, ?$ for all $j = 1 \ldots m$.
The name is justified because the threshold value $w_j =\,\, ?$ occurring as annotation of a body atom $B_j$
does not impose any particular requirement to the qualification value of $B_j$.
Threshold-free clauses may be written with the simplified syntax $A \qgets{\alpha}  B_1, \ldots, B_m$.
\item
A clause is called {\em qualification-free} iff it is both attenuation-free and threshold-free.
These clauses may be written with the simplified syntax $A \!\gets\! B_1, \ldots, B_m$.
They behave just like those used in the classical CLP scheme.
\item
A clause is called {\em constraint-free} iff all its body atoms are defined.
\item
A program is called  attenuation-free iff all its clauses are of this type.
Thresh\-old-free, qualification-free and constraint-free programs are defined similarly.
\end{itemize}




The more technical $\sqclp{\simrel}{\U}{\rdom}$-program $\Prog$ presented below will serve as a
\emph{running example} to illustrate various points in the rest of the report.


\begin{exmp}[Running example]
\label{exmp:running}
Assume unary constructors $c, c' \in DC^1$, binary predicate symbols $p, p', q \in DP^2$ and a ternary predicate symbol $r \in DP^3$.
Consider the admissible triple $\langle \simrel,\U,\rdom \rangle$,
where $\simrel$ is  an $\U$-valued proximity relation such that
$\simrel(c,c') = 0.9$ and $\simrel(p,p') = 0.8$.
Let $\Prog$ be the $\sqclp{\simrel}{\U}{\rdom}$-program consisting of the qualified clauses $R_1$, $R_2$ and $R_3$ listed below:
\begin{itemize}
  \item[] \small $R_1 : q(X,c(X)) \qgets{1.0}$
  \item[] \small $R_2 : p(c(X),Y) \qgets{0.9} \qat{q(X,Y)}{0.8}$
  \item[] \small $R_3 : r(c(X),Y,Z) \qgets{0.9} \qat{q(X,Y)}{0.8}, \qat{cp_{\geq}(X,0.0)}{?}$ \mathproofbox
 \end{itemize}
\end{exmp}



As we will see in the Conclusions, the classical $\mbox{CLP}$ scheme for Constraint Logic Programming originally introduced in \cite{JL87} can be seen as a particular case of the $\mbox{SQCLP}$ scheme.
In the rest of this subsection we present a declarative semantics for $\sqclp{\simrel}{\qdom}{\cdom}$-programs inspired by \cite{GL91,GDL95}. These papers provided three different program semantics $\mathcal{S}_{i}$ ($i = 1, 2, 3$)
characterizing {\em valid ground solutions for goals}, {\em valid open solutions for goals} and {\em computed answers for goals} in $\mbox{CLP}$, respectively.
In fact, the $\mathcal{S}_{i}$ semantics in \cite{GL91,GDL95} were conceived as the $\mbox{CLP}$ counterpart
of previously known semantics for logic programming,
namely the least ground Herbrand model semantics \cite{Apt90,Llo87},
the open Herbrand model semantics, also known as $\mathcal{C}$-semantics \cite{Cla79,FLMP93},
and the $\mathcal{S}$-semantics \cite{FLMP89,BGLM94};
see \cite{AG94} for a very  concise and readable overview.

In this report we restrict ourselves to develop a $\mathcal{S}_{2}$-like semantics which can be used
to characterize valid open solutions for SQCLP goals as we will see in Subsection \ref{sec:sqclp:goals}.
As a basis for our semantics we use so-called {\em qc-atoms} of the form $\cqat{A}{d}{\Pi}$,
intended to assert that the atom $A$ is entailed by the constraint set $\Pi$ with qualification degree $d$.
We also use a special entailment relation $\entail{\qdom,\cdom}$ intended to capture some implications between qc-atoms
whose validity depends neither on the proximity relation $\simrel$ nor on the semantics of defined predicates.
A formal definition of these notions is as follows:



\begin{defn}[qc-atoms, observables and $(\qdom, \cdom)$-entailment]
\label{defn:atoms-entail}
\begin{enumerate}
  \item \label{defn:atoms-entail:atoms}
  \emph{Qualified constrained atoms} (or simply \emph{qc-atoms}) are statements of the form $\cqat{A}{d}{\Pi}$,
  where $A \in \At$ is an atom, $d \in D$ is a qualification value, and $\Pi \subseteq \Con{\cdom}$ is a finite set of  constraints.
  \item \label{defn:atoms-entail:kinds}
  A qc-atom $\cqat{A}{d}{\Pi}$ is called {\em defined}, {\em primitive} or {\em equational} according to the syntactic form of $A$.
  \item \label{defn:atoms-entail:observable}
  A qc-atom $\cqat{A}{d}{\Pi}$ is called {\em observable} iff $d \in \aqdom$ and $\Pi$ is satisfiable.
  \item \label{defn:atoms-entail:entail}
  Given two qc-atoms  $\varphi : \cqat{A}{d}{\Pi}$ and $\varphi' : \cqat{A'}{d'}{\Pi'}$,
  we say that $\varphi$ $(\qdom, \cdom)$-\emph{entails} $\varphi'$ (in symbols, $\varphi \entail{\qdom,\cdom} \varphi'$)
  iff there is some $\cdom$-substitution $\theta$ satisfying $A' = A\theta$, $d' \dleq d$ and $\Pi' \model{\cdom} \Pi\theta$.
  \mathproofbox
\end{enumerate}
\end{defn}



We will focus our attention on observable qc-atoms
because they can be interpreted as observations of valid open solutions for atomic goals in $\sqclp{\simrel}{\qdom}{\cdom}$
as we will see in Subsection \ref{sec:sqclp:goals}.
The example below illustrates the main technical ideas from Definition \ref{defn:atoms-entail}.

\begin{exmp}[Observable qc-atoms and $(\qdom, \cdom)$-entailment]
\label{exmp:qc-atoms}
Consider the admissible triple underlying Example \ref{exmp:running} and the sets of $\rdom$-constraints:
$$
\begin{array}{c@{\hspace{1mm}}c@{\hspace{1mm}}l}
\Pi & = & \{cp_{>}(X,1.0),\, op_{+}(A,A,X),\, op_{\times}(2.0,A,Y)\} \\
\Pi' & = & \{cp_{\geq}(A,3.0),\, op_{\times}(2.0,A,X),\, op_{+}(A,A,Y)\}\\
\end{array}
$$
Then, the following are observable qc-atoms:
$$
\begin{array}{l@{\hspace{1cm}}l}
\varphi_1 = \cqat{q(X,c'(Y))}{0.9}{\Pi} & \varphi_3 = \cqat{r(c'(Y),c(X),Z)}{0.8}{\Pi} \\
\varphi_2 = \cqat{p'(c'(Y),c(X))}{0.8}{\Pi} & \varphi'_3 = \cqat{r(c'(Y),c(X),c(Z'))}{0.7}{\Pi'} \\
\end{array}
$$
and the $(\U, \rdom)$-entailment $\varphi_3 \entail{\U, \rdom} \varphi'_3$ is valid thanks to $\theta = \{Z \mapsto c(Z')\}$,
which satisfies $r(c'(Y),c(X),c(Z')) = r(c'(Y),c(X),Z)\theta$, $0.7 \leq 0.8$ and $\Pi' \model{\rdom} \Pi\theta$. \mathproofbox
\end{exmp}



The intended meaning of $\entail{\qdom,\cdom}$ as an entailment relation not depending on
the meanings of  defined predicates motivates the first item in the next definition.

\begin{defn}[Interpretations]
\label{defn:interpretations}
Let $\langle \simrel,\qdom,\cdom \rangle$ be any given admissible triple. Then:
\begin{enumerate}
\item
A {\em qualified constrained interpretation} (or {\em qc-interpretation}) is a set $\I$ of
observable defined qc-atoms closed under $(\qdom, \cdom)$-entailment.
In other words, a set $\I$ of qc-atoms which satisfies the following two conditions:
  \begin{enumerate}
\item
  Each $\varphi \in \I$ is an observable defined qc-atom.
\item
  If $\varphi \in \I$ and $\varphi'$ is another defined observable qc-atom such that $\varphi \entail{\qdom,\cdom} \varphi'$,
  then also $\varphi' \in \I$.
  \end{enumerate}
\item
Assume any given qc-interpretation $\I$.
For any observable qc-atom $\varphi$, we say that  $\varphi$ is valid in $\I$ modulo $\simrel$
(in symbols, $\I \isqchlrdc \varphi$) iff some of the three cases below holds:
\begin{enumerate}
  \item $\varphi$ is defined and $\varphi \in \I$.
  \item $\varphi : \cqat{(t == s)}{d}{\Pi}$ is equational and $t \approx_{d, \Pi} s$.
  \item $\varphi : \cqat{\kappa}{d}{\Pi}$ is primitive and $\Pi \model{\cdom} \kappa$.
  \enspace  \mathproofbox
\end{enumerate}
\end{enumerate}
\end{defn}

Note that a given interpretation $\I$ can include several observables $\cqat{A}{d_i}{\Pi}$ for the same (possibly not  ground) atom $A$ but is not required to include on ``optimal'' observable $\cqat{A}{d}{\Pi}$ with $d$ computed as the {\em lub} of all $d_i$.
By contrast, the other related works discussed in the Introduction view program interpretations as mappings $\I$ from the ground Herbrand base into some set of lattice elements (the real interval $[0,1]$ in many cases). In such interpretations, each ground atom $A$ has attached one single lattice element $d = \I(A)$ intended as ``the optimal qualification'' for $A$.
Our view of interpretations is closer to the expected operational behavior of goal solving systems and can be used to characterize the validity of  solutions computed by such systems, as we will see in Subsection \ref{sec:sqclp:goals}.


Note also that the notation $\I \isqchlrdc \varphi$ is defined only for the case that $\varphi$ is observable.
In the sequel, we will implicitly assume that $\varphi$ is observable in any context where the notation $\I \isqchlrdc \varphi$ is used. The next technical result shows that validity in any given interpretation is closed under entailment.

\begin{prop}[Entailment Property for Interpretations]
\label{prop:ep-i}
Assume that $\I \isqchlrdc \varphi$ and $\varphi \entail{\qdom,\cdom} \varphi'$. Then $\I \isqchlrdc \varphi'$.
\end{prop}
\begin{proof*}
Due to the hypothesis $\varphi \entail{\qdom,\cdom} \varphi'$
we can assume $\varphi = (\cqat{A}{d}{\Pi})$, $\varphi' = (\cqat{A'}{d'}{\Pi'})$
and some $\cdom$-substitution $\theta$ such that  $A' = A\theta$, $d' \dleq d$ and $\Pi' \model{\cdom} \Pi\theta$.
We now distinguish cases according to the syntactic form of $\varphi$:
\begin{enumerate}
\item $\varphi$ is defined.
In this case, $\varphi'$ is also defined.
Moreover, $\I \isqchlrdc \varphi$ is equivalent to $\varphi \in \I$ because of Definition \ref{defn:interpretations},
which implies $\varphi' \in \I$ because qc-interpretations are closed under $\entail{\qdom,\cdom}$,
which is equivalent to $\I \isqchlrdc \varphi'$  because of Definition \ref{defn:interpretations}.
\item $\varphi$ is equational.
In this case $A$ and $A'$ have the form $t == s$ and $t\theta == s\theta$, respectively.
Moreover, $\I \isqchlrdc \varphi$ is equivalent to
$t \approx_{d, \Pi} s$ because of Definition \ref{defn:interpretations},
which implies $t\theta \approx_{d, \Pi\theta} s\theta$ because of Lemma \ref{lem:slpeq},
which trivially  implies $t\theta \approx_{d', \Pi'} s\theta$
because of $\Pi' \model{\cdom} \Pi\theta$ and $d' \dleq d$,
which is equivalent to $\I \isqchlrdc \varphi'$  because of Definition \ref{defn:interpretations}.
\item $\varphi$ is primitive.
In this case $A$ and $A'$ have the form $\kappa$ and $\kappa\theta$, respectively.
Moreover, $\I \isqchlrdc \varphi$ is equivalent to $\Pi \model{\cdom} \kappa$ because of Definition \ref{defn:interpretations},
which implies $\Pi\theta \model{\cdom} \kappa\theta$ because of Lemma \ref{lem:sl},
which implies $\Pi' \model{\cdom} \kappa\theta$ because of $\Pi' \model{\cdom} \Pi\theta$,
which is equivalent to $\I \isqchlrdc \varphi'$  because of Definition \ref{defn:interpretations}.
\enspace \mathproofbox
\end{enumerate}
\end{proof*}



The definition below explains when a given  interpretation is regarded as a model of a given program,
as well as the related notion of semantic consequence.

\begin{defn}[Models and semantic consequence]
\label{defn:models}
Let a $\sqclp{\simrel}{\qdom}{\cdom}$-program $\Prog$ and an observable qc-atom 
$\varphi : \cqat{p'(\ntup{t'}{n})}{d}{\Pi}$ be given.
$\varphi$ is an {\em immediate consequence} of a qc-interpretation $\I$ via a program rule
$(R_l :  p(\ntup{t}{n}) \qgets{\alpha} \qat{B_1}{w_1}, \ldots, \qat{B_m}{w_m}) \in \Prog$ iff
there exist a $\cdom$-substitution $\theta$ and a choice of qualification values 
$d_0, d_1, \ldots, d_n, e_1, \ldots, e_m \in \aqdom$ such that:
\begin{enumerate}
\item[(a)]
$\simrel(p',p) = d_0$
\item[(b)]
$\I \isqchlrdc \cqat{(t'_i == t_i\theta)}{d_i}{\Pi}$ (i.e. $t'_i \approx_{d_i,\Pi} t_i\theta$) for $i = 1 \ldots n$
\item[(c)]
$\I \isqchlrdc \cqat{B_j\theta}{e_j}{\Pi}$ with $e_j \dgeq^? w_j$ for $j = 1 \ldots m$
\item[(d)]
$d \dleq \bigsqcap_{i = 0}^{n}d_i \sqcap \alpha \circ \bigsqcap_{j = 1}^m e_j$
[i.e., $d \dleq d_i ~ (0 \le i \le n)$ and $d \dleq \alpha \circ e_j ~ (1 \le j \le m)]$
\end{enumerate}
Note that  the qualification value $d$ attached to $\varphi$ is limited by two kinds of upper bounds:
$d_i ~ (0 \le i \le n)$, i.e. the $\simrel$-proximity between $p'(\ntup{t'}{n})$ and the head of $R_l\theta$;
and $\alpha \circ e_j ~ (1 \le j \le m)$, i.e. the qualification values of the atoms in the body of $R_l\theta$
attenuated w.r.t. $R_l$'s attenuation factor $\alpha$.
Moreover, the inequalities $e_j \!\dgeq^? \!\!w_j ~ (1 \le j \le m)$
are required in order to impose the threshold conditions within $R_l$'s body. 
As already explained at the beginning of this subsection, 
$e_j \dgeq^? \!\!w_j$ means that either $w_j =\ ?$ or else $w_j \in \aqdom$ and $e_j \dgeq w_j$.
Now we can define:
\begin{enumerate}
\item
$\I$ is a \emph{model} of a program rule $R_l \in \Prog$
(in symbols, $\I \model{\simrel,\qdom,\cdom} R_l$)
iff every defined observable qc-atom $\varphi$ which is an immediate consequence of $\I$ via $R_l$ verifies $\varphi \in \I$; and
$\I$ is a \emph{model} of $\Prog$ (in symbols, $\I \model{\simrel,\qdom,\cdom} \Prog$)
iff $\I$ is a model of every program rule $R_l \in \Prog$.
\item
$\varphi$ is a \emph{semantic consequence} of $\Prog$ (in symbols, $\Prog \model{\simrel,\qdom,\cdom} \varphi$)
iff $\I \isqchlrdc \varphi$ for every qc-interpretation $\I$ such that $\I \model{\simrel,\qdom,\cdom} \Prog$. \mathproofbox
\end{enumerate}
\end{defn}

The next example may serve as a concrete illustration:

\begin{exmp}[Models and semantic consequence]
\label{exmp:semantic-consequence}
Recall the $\sqclp{\simrel}{\U}{\rdom}$-program $\Prog$ from Example \ref{exmp:running}.
Let us show that the three qc-atoms $\varphi_1$, $\varphi_2$ and $\varphi_3$ from Example \ref{exmp:qc-atoms}
are semantic consequences of $\Prog$:
\begin{enumerate}
  \item
  Assume an arbitrary model $\I \model{\simrel,\U,\rdom} \Prog$.
  Note that the atom underlying $\varphi_1$ and the head atom of $R_1$ are $q(X,c'(Y))$ and $q(X,c(X))$, respectively.
  Since $\simrel(c,c') = 0.9$ and $\Pi \model{\cdom} X == Y$, $\varphi_1$ can be obtained as an immediate consequence of $\I$ via $R_1$ using $\theta = \varepsilon$.
  Therefore $\varphi_1 \in \I$ and we can conclude that $\Prog \model{\simrel,\U,\rdom} \varphi_1$.
   \item
   Assume an arbitrary model $\I \model{\simrel,\U,\rdom} \Prog$.
   Consider the substitution $\theta = \{Y \mapsto c'(Y)\}$.
   Note that the atom underlying $\varphi_2$ and the head atom of $R_2\theta$ are $p'(c'(Y),c(X))$ and $p(c(X),c'(Y))$, respectively.
   Moreover, $\varphi_1 \in \I$ (due to the previous item) and the atom $q(X,c'(Y))$ underlying $\varphi_1$
  is the same as the atom in the body of $R_2\theta$.
  These facts together with $\simrel(p,p') = 0.8$, $\simrel(c,c') = 0.9$  and $\Pi \model{\cdom} X == Y$
  allow to obtain $\varphi_2$ as an immediate consequence of $\I$ via $R_2$.
  Therefore $\varphi_2 \in \I$ and we can conclude that $\Prog \model{\simrel,\U,\rdom} \varphi_2$.
  \item
  Assume an arbitrary model $\I \model{\simrel,\U,\rdom} \Prog$.
   Consider again the substitution $\theta = \{Y \mapsto c'(Y)\}$.
   Note that the atom underlying $\varphi_3$ and the head atom of $R_3\theta$ are $r(c'(Y),c(X),Z)$ and $r(c(X),c'(Y),Z)$, respectively.
   Moreover, the two annotated atoms $\qat{B_j\theta}{w_j} ~ (1 \le j \le 2)$ occurring in the body of $R_3\theta$ are such that
   $\I \isqchlrdc \cqat{B_j\theta}{e_j}{\Pi}$ for suitable values $e_j \geq^? w_j$, namely $e_1 = 0.9$ and $e_2 = 1.0$.
   Note that $e_1 = 0.9$ works because $B_1\theta$ is the atom $q(X,c'(Y))$ underlying  $\varphi_1$
   and $\varphi_1 \in \I$, as proved in the first item of this example. On the  other hand, $e_2 = 1.0$ works because
   $B_2\theta$ is the primitive atom $cp_{\geq}(X,0.0)$ which is trivially entailed by $\Pi$.
   All these facts, together with $\simrel(c,c') = 0.9$, $0.8 \leq 0.9 \times 0.9$  and $\Pi \model{\cdom} X == Y$
   allow to obtain $\varphi_3$ as an immediate consequence of $\I$ via $R_3$.
  Therefore $\varphi_3 \in \I$ and we can conclude that $\Prog \model{\simrel,\U,\rdom} \varphi_3$.  \enspace  \mathproofbox
\end{enumerate}
\end{exmp}

Now we are ready to obtain results on the declarative semantics of programs in the $\mbox{SQCLP}$ scheme.
We will characterize the observable consequences of a given program $\Prog$ in two different, but equivalent, ways:
either using the interpretation transformer presented in Subsection \ref{sec:sqclp:fixpointsem},
or using the extension of Horn Logic presented in Subsection \ref{sec:sqclp:sqchl}.
In both approaches, we will prove the existence of a least model $\Mp$ for each given program $\Prog$.

\subsubsection{A Fixpoint Semantics}
\label{sec:sqclp:fixpointsem}

A well-known way of characterizing models and least models of programs in declarative languages
proceeds by considering a lattice structure for the family of all program interpretations,
and using an interpretation transformer to compute the immediate consequences obtained from program rules.
This kind of approach is well known for logic programming \cite{VEK76,AVE82,Llo87,Apt90}
and constraint logic programming \cite{GL91,GDL95,JMM+98}.
It has been used also in various extensions of logic programming designed to support uncertain reasoning,
such as quantitative logic programming \cite{VE86},
its extension to qualified logic programming \cite{RR08}
quantitative constraint logic programming \cite{Rie96,Rie98phd},
similarity-based logic programming \cite{Ses02}
and proximity-based logic programming in the sense of  \textsf{Bousi}$\sim$\textsf{Prolog}  \cite{JR09}.


The $\mbox{SQCLP}$ scheme  is intended to unify all these logic programming extensions in a common framework.
This subsection is based on the declarative semantics given in \cite{RR08,RR08TR},
extended to deal with constraints and proximity relations.
Our first result provides a lattice of program interpretations.



\begin{prop}[Lattice of Interpretations]
\label{prop:intdc-lattice}
$\Intdc$, defined as the set of all qc-interpretations over the qualification domain $\qdom$ and the constraint domain $\cdom$, is a complete lattice w.r.t. the set inclusion ordering $\subseteq$. Moreover, the bottom element $\ibot$ and the top element $\itop$ of this lattice are characterized as $\ibot = \emptyset$ and $\itop = \{\varphi \mid \varphi \mbox{ is a defined observable qc-atom}\}$ and for any subset $I \subseteq \Intdc$ its greatest lower bound (glb) and least upper bound (lub) are characterized as follows:
\begin{enumerate}
\item \label{prop:intdc-lattice:infimum}
The glb of $I$ (written as $\infi I$) is $\inter_{\I \in I} \I$, understood as $\itop$ if $I = \emptyset$; and
\item \label{prop:intdc-lattice:supremum}
The lub of $I$ (written as $\supr I$) is $\union_{\I \in I} \I$, understood as $\ibot$ if $I = \emptyset$.
\end{enumerate}
\end{prop}

\begin{proof*}
Both $\ibot$ and $\itop$ are qc-interpretations because they are sets of defined observable qc-atoms
and they are closed under $(\qdom, \cdom)$-entailment for trivial reasons, namely:
$\ibot$ is empty and $\itop$ includes all the defined observables.
Moreover, they are the minimum and the maximum of $\Intdc$ w.r.t. $\subseteq$
because $\ibot \subseteq \I \subseteq \itop$ is trivially true for each $\I \in \Intdc$.
Thus, we have only left to prove \ref{prop:intdc-lattice:infimum}. and \ref{prop:intdc-lattice:supremum}.:
\begin{enumerate}
\item
$\inter_{\I \in I} \I$ is obviously a set of defined observable qc-atoms because this is the case for each $\I \in I$.
Given any $\varphi \in \inter_{\I \in I}$ and any observable defined qc-atom $\varphi'$ such that $\varphi \entail{\qdom,\cdom} \varphi'$,
we get $\varphi' \in \inter_{\I \in I} \I$ as an obvious consequence of the fact that each $\I \in I$  is closed under $(\qdom, \cdom)$-entailment.
Therefore, $\inter_{\I \in I} \I \in \Intdc$.
Obviously,  $\inter_{\I \in I} \I$ is trivially a lower bound of $I$ w.r.t. $\subseteq$.
Moreover,  $\inter_{\I \in I} \I$ is the glb of $I$, because any given lower bound $\J$ of $I$
verifies $\J \subseteq \I$ for every $\I \in I$ and thus $\J \subseteq \inter_{\I \in I} \I$.
Therefore, $\inter_{\I \in I} \I = \infi I$.
\item
Using the properties of the union of a family of sets it is easy to prove that $\union_{\I \in I} \I \in \Intdc$
and also that $\union_{\I \in I} \I$ is the lub of $I$ w.r.t. $\subseteq$.
A more detailed reasoning would be similar to the previous item.
Therefore, $\union_{\I \in I} \I = \supr I$. \mathproofbox
\end{enumerate}
\end{proof*}

Next we define an \emph{interpretation transformer} $\Tp$, intended to compute the immediate consequences obtained from a given qc-interpretation via the program rules belonging to $\Prog$.



\begin{defn}[Interpretations Transformer]
\label{defn:tp}
Let $\Prog$ be a fixed $\sqclp{\simrel}{\qdom}{\cdom}$-program.
The interpretations transformer $\Tp : \Intdc \to \Intdc$ is defined by the condition:
$$\Tp(\I) \eqdef \{ \varphi \mid \varphi \mbox{ is an immediate consequence of } \I \mbox{ via some } R_l \in \Prog \}
\enspace . \mathproofbox$$
\end{defn}

The computation of immediate consequences of a given qc-interpretation $\I$ via a given program rule $R_l$
has been already explained in  Definition \ref{defn:models}. The following example illustrates the workings of $\Tp$.

\begin{exmp}[Interpretation transformer in action]
\label{exmp:tp} Recall again the $\sqclp{\simrel}{\U}{\rdom}$-program $\Prog$ from Example
\ref{exmp:running} and the observable defined qc-atoms $\varphi_1$, $\varphi_2$ and $\varphi_3$
from Example \ref{exmp:qc-atoms}. Then:
\begin{enumerate}
\item
The arguments given in Example \ref{exmp:semantic-consequence}(1)
can be easily reused to show that $\varphi_1$ is an immediate consequence of the empty interpretation $\ibot$ via the program rule $R_1$.
Therefore, $\varphi_1 \in \Tp(\ibot)$.
\item
The arguments given in Example \ref{exmp:semantic-consequence}(2)
can be easily reused to show that $\varphi_1$ is an immediate consequence of $\I$ via the program rule $R_2$,
provided that $\varphi_1 \in \I$. Therefore, $\varphi_2 \in \Tp(\Tp(\ibot))$.
\item
The arguments given in Example \ref{exmp:semantic-consequence}(3)
can be easily reused to show that $\varphi_3$ is an immediate consequence of $\I$ via the program rule $R_3$,
provided that $\varphi_1 \in \I$. Therefore, $\varphi_3 \in \Tp(\Tp(\ibot))$. \mathproofbox
\end{enumerate}
\end{exmp}


The next proposition states the main properties of interpretation transformers.


\begin{prop}[Properties of interpretation transformers]
\label{prop:tp-properties}
Let $\Prog$ be any fixed $\sqclp{\simrel}{\qdom}{\cdom}$-program. Then:
\begin{enumerate}
\item\label{prop:tp-properties:1}
$\Tp$ is a well defined mapping, 
i.e. for all $\I \in \Intdc$ one has  $\Tp(\I) \in \Intdc$.
\item\label{prop:tp-properties:2}
$\Tp$ is monotonic and continuous.
\item\label{prop:tp-properties:3}
For all $\I \in \Intdc$ one has: $\I \model{\simrel,\qdom,\cdom} \Prog \Longleftrightarrow \Tp(\I) \subseteq \I$,
That is, the models of $\Prog$ are precisely the pre-fixpoints of $\Tp$.
\end{enumerate}
\end{prop}

\begin{proof*}
\begin{enumerate}
\item
By definition, $\Tp(\I)$ is a set of observable defined qc-atoms. 
It is sufficient to prove that it is closed under  $(\qdom, \cdom)$-entailment.
Let us assume two observable defined qc-atoms $\varphi$ and $\varphi'$ such that $\varphi \in \Tp(\I)$ 
and $\varphi \entail{\qdom,\cdom} \varphi'$.
Because of $\varphi \entail{\qdom,\cdom} \varphi'$ we can assume $\varphi : \cqat{p(\ntup{t}{n})}{d}{\Pi}$, 
$\varphi' : \cqat{p(\ntup{t'}{n})}{d'}{\Pi'}$
and some substitution $\theta $ such that $p(\ntup{t'}{n}) = p(\ntup{t}{n})\theta$, $d' \dleq d$ and $\Pi' \model{\cdom} \Pi\theta$.
Because of $\varphi \in \Tp(\I)$, we can assume that $\varphi$ is an immediate consequence of $\I$ via some $R_l \in \Prog$.
More precisely, we can assume $(R_l : q(\ntup{s}{n}) \qgets{\alpha} \qat{B_1}{w_1}, \ldots, \qat{B_m}{w_m}) \in \Prog$, some substitution $\sigma$
and some qualification values $d_0, d_1, \ldots,$ $d_n, e_1, \ldots, e_m \in \aqdom$ such that
\begin{enumerate}
\item
$\simrel(p,q) = d_0$,
\item
$\I \isqchlrdc \cqat{(t_i == s_i\sigma)}{d_i}{\Pi}$ for $i = 1 \ldots n$,
\item
$\I \isqchlrdc \cqat{B_j\sigma}{e_j}{\Pi}$ with $e_j \dgeq^? w_j$ for $j = 1 \ldots m$,
\item
$d \dleq \bigsqcap_{i = 0}^{n}d_i \sqcap \alpha \circ \bigsqcap_{j = 1}^m e_j$ 
[i.e., $d \dleq d_i ~ (0 \le i \le n)$ and $d \dleq \alpha \circ e_j ~ (1 \le j \le m)$].
\end{enumerate}
In order to show that $\varphi' \in \Tp(\I)$, we claim that $\varphi'$ can be computed as an immediate consequence of $\I$
via the same program rule $R_l$, using the substitution $\sigma\theta$ and the qualification values
$d_0, d_1, \ldots,$ $d_n, e_1, \ldots, e_m \in \aqdom$. To justify this claim it is enough to check the following items:
\begin{enumerate}
\item[(a')]
$\simrel(p,q) = d_0$,
\item[(b')]
$\I \isqchlrdc \cqat{(t'_i == s_i\sigma\theta)}{d_i}{\Pi'}$ for $i = 1 \ldots n$,
\item[(c')]
$\I \isqchlrdc \cqat{B_j\sigma\theta}{e_j}{\Pi'}$ with $e_j \dgeq^? w_j$ for $j = 1 \ldots m$,
\item[(d')]
$d \dleq \bigsqcap_{i = 0}^{n}d_i \sqcap \alpha \circ \bigsqcap_{j = 1}^m e_j$ 
[i.e., $d \dleq d_i ~ (0 \le i \le n)$ and $d \dleq \alpha \circ e_j ~ (1 \le j \le m)$].
\end{enumerate}
These four items closely correspond to items (a)-(d) above. More specifically: \\
--- Items (a') and (d') are identical to items (a) and (d), respectively. \\
--- Regarding item (b'):
For $i = 1 \ldots n$,  $\I \isqchlrdc \cqat{(t'_i == s_i\sigma\theta)}{d_i}{\Pi}$ is the same as
$t_i\theta \approx_{d_i, \Pi'} s_i\sigma\theta$.
Because of Lemma \ref{lem:slpeq}, this is a consequence of $\Pi' \model{\cdom} \Pi\theta$
and $t_i \approx_{d_i, \Pi} s_i\sigma$,
which is ensured by item (b). \\
--- Regarding  item (c'):
For $j = 1 \ldots m$, $e_j \dgeq^? w_j$ is ensured by item (c),
and $\I \isqchlrdc \cqat{B_j\sigma\theta}{e_j}{\Pi'}$ follows from $\I \isqchlrdc \cqat{B_j\sigma}{e_j}{\Pi}$
--also ensured by item (c)--
and the entailment property for interpretations (Proposition \ref{prop:ep-i}),
which can be applied because $\cqat{B_j\sigma}{e_j}{\Pi} \entail{\qdom,\cdom} \cqat{B_j\sigma\theta}{e_j}{\Pi'}$.

\item
Monotonicity means that the inclusion  $\Tp (\I) \subseteq \Tp(\J)$ holds whenever $\I \subseteq \J$.
This follows very easily from
$$(\spadesuit) \quad \I \isqchlrdc \varphi \mbox{ and } \I \subseteq \J \Longrightarrow \J \isqchlrdc \varphi$$
which is a trivial consequence of Definition \ref{defn:interpretations}.

Continuity means that the equation $\Tp(\supr I) = \supr \{\Tp(\I) \mid \I \in I \}$ holds for any directed set $I \subseteq \Intdc$ of qc-interpretations.
Recall that $I \subseteq \Intdc$ is called directed iff every finite subset $I_0 \subseteq I$ has some upper bound $\I \in I$.
We show that $\Tp(\supr I) = \supr \{Tp(\I) \mid \I \in I \}$ holds by proving the  two inclusions separately:

\begin{enumerate}
\item
For each fixed $\I_0 \in I$, $\Tp(\I_0) \subseteq  \Tp(\supr I)$ follows from $\I_0 \subseteq \supr I$ and monotonicity of $\Tp$.
Then, the inclusion $\supr \{\Tp(\I) \mid \I \in I \} \subseteq \Tp(\supr I)$ holds by definition of supremum.
\item
In order to prove the opposite inclusion $\Tp(\supr I) \subseteq \supr \{\Tp(\I) \mid \I \in I \}$,
consider an arbitrary $\varphi \in \Tp(\supr I)$. Due to Definition \ref{defn:tp}, $\varphi$ is an
immediate consequence of $\supr I$ via some program rule $R_l \in \Prog$. Because of the first item
of Definition \ref{defn:models}, $\varphi$ is an immediate consequence of $\supr I$ via $R_l$ due
to finitely many qc-facts of the form $\cqat{B_j\theta}{e_j}{\Pi}$ (coming from the body of a
suitable instance of $R_l$) that are valid in $\supr I$. Because of $(\spadesuit)$ and the
assumption that  $I$ is a directed set,  it is possible to choose some $\I_0 \in I$ such that all
the qc-facts $\cqat{B_j\theta}{e_j}{\Pi}$ are valid in $\I_0$. Then, $\varphi$ is an immediate
consequence of this particular $\I_0 \in I$ via $R_l$. Therefore, $\varphi \in \Tp(\I_0) \subseteq
\supr \{\Tp(\I) \mid \I \in I \}$.
\end{enumerate}

\item
According to Definition \ref{defn:models}, $\I \model{\simrel,\qdom,\cdom} \Prog$ holds iff
every observable defined qc-atom $\varphi$ which is an immediate consequence of $\I$ via the program rules $R_l \in \Prog$ verifies $\varphi \in \I$.
According to Definition \ref{defn:tp}, $\Tp(\I)$ is just the set of all the defined observable qc-atoms $\varphi$
that can be obtained as immediate consequences of $\I$ via the program rules $R_l \in \Prog$.
Consequently, $\I \model{\simrel,\qdom,\cdom} \Prog$ holds iff $\Tp(\I) \subseteq \I$.  \enspace \mathproofbox
\end{enumerate}
\end{proof*}

The theorem below is the main result in this subsection.

\begin{thm}[Fixpoint characterization of least program models]
\label{thm:tp-leastmodel}
Every $\sqclp{\simrel}{\qdom}{\cdom}$-program $\Prog$ has a \emph{least model} $\Mp$,
smaller than any other model of $\Prog$ w.r.t. the set inclusion ordering of the interpretation lattice $\Intdc$.
Moreover, $\Mp$ can be characterized as the {\em least fixpoint} of $\Tp$ as follows:
$$\Mp = l\!f\!p(\Tp) = \union_{k\in\NAT} \Tp{\uparrow^k}(\ibot) \enspace . \mathproofbox$$
\end{thm}

\begin{proof}
As usual, a given $\I \in \Intdc$ is called a fixpoint of $\Tp$ iff $\Tp(\I) = \I$,
and $\I$ is called a pre-fixpoint of $\Tp$ iff $\Tp(\I) \subseteq \I$.
Due to a well-known theorem by Knaster and Tarski, see \cite{Tar55}, a monotonic mapping from a complete lattice into itself always has a least fixpoint which is also its least pre-fixpoint. In the case that the mapping is continuous, its least fixpoint can be characterized as the lub of the sequence of lattice elements obtained by reiterated application of the mapping to the bottom element. Combining these results with Proposition \ref{prop:tp-properties} trivially proves the theorem.
\end{proof}

\subsubsection{An equivalent Proof-theoretic Semantics}
\label{sec:sqclp:sqchl}



In order to give a logical view of program semantics and an alternative characterization of least program models,
we define the \emph{Proximity-based Qualified Constrained Horn Logic} $\SQCHL(\simrel,\qdom,\cdom)$
as a formal inference system consisting of the three inference rules displayed in Figure \ref{fig:sqchl}.

\begin{figure}[h]
  \figrule
  \centering
  \begin{tabular}{ll}&\\
\textbf{SQDA} & $\displaystyle\frac
    {~ (~ \cqat{(t'_i == t_i\theta)}{d_i}{\Pi} ~)_{i=1 \ldots n} \quad (~ \cqat{B_j\theta}{e_j}{\Pi} ~)_{j=1 \ldots m} ~}
    {\cqat{p'(\ntup{t'}{n})}{d}{\Pi}}$ \\ \\
  &if $(p(\ntup{t}{n}) \qgets{\alpha} \qat{B_1}{w_1}, \ldots, \qat{B_m}{w_m}) \in \Prog$, $\theta$ subst.,
  $\simrel(p',p) = d_0 \neq \bt$,\\ 
  &$e_j \dgeq^? w_j ~ (1 \le j \le m)$ and $d \dleq \bigsqcap_{i = 0}^{n}d_i \sqcap \alpha \circ \bigsqcap_{j = 1}^m e_j$.\\
&\\
\textbf{SQEA} & $\displaystyle\frac
    {}
    {\quad \cqat{(t == s)}{d}{\Pi} \quad}$ ~
  if $t \approx_{d, \Pi} s$. \\
  &\\
\textbf{SQPA} & $\displaystyle\frac
    {}
    {\quad \cqat{\kappa}{d}{\Pi} \quad}$ ~
  if $\Pi \model{\cdom} \kappa$. \\
  &\\
  \end{tabular}
  \caption{Proximity-based Qualified Constrained Horn Logic}
  \label{fig:sqchl}
  \figrule
\end{figure}




The three inference rules are intended to work with observable qc-atoms.
Rule \textbf{SQDA} is used to infer defined qc-atoms.
It formalizes an extension of  the classical \emph{Modus Ponens}  inference,
allowing  to infer a defined qc-atom $\cqat{p'(\ntup{t'}{n})}{d}{\Pi}$ by means of an instance of a program clause
with head $p(\ntup{t}{n})\theta$ and body atoms $\qat{B_j\theta}{w_j}$.
The $n$ premises $\cqat{(t'_i == t_i\theta)}{d_i}{\Pi}$ combined with the
side condition $\simrel(p',p) = d_0 \neq \bt$  ensure the ``equality'' between
$p'(\ntup{t'}{n})$ and $p(\ntup{t}{n})\theta$ modulo $\simrel$;
the $m$ premises $\cqat{B_j\theta}{e_j}{\Pi}$ require to prove the body atoms;
and the side conditions $e_j \dgeq^? w_j$ and $d \dleq \bigsqcap_{i =
0}^{n}d_i \sqcap \alpha \circ \bigsqcap_{j = 1}^m e_j$ check the threshold conditions of the body
atoms and impose the proper relationships between the qualification
value attached to  the conclusion and  the qualification values attached to the premises.
In particular, the inequality $d \dleq \alpha \circ \bigsqcap_{j = 1}^m e_j$ is imposed,
meaning that the qualification value attached to a clause's head cannot exceed the glb
of the qualification values attached to the body atoms attenuated by the clause's attenuation factor.
Rules \textbf{SQEA} and \textbf{SQPA} are used to infer equational and primitive qc-atoms, respectively.
Rule \textbf{SQEA} is designed to work with term proximity w.r.t. $\Pi$ in the sense of 
Definition \ref{defn:Pi-prox}, inferring  $\cqat{(t  == s)}{d}{\Pi}$ just in the case that $t \approx_{d, \Pi} s$ holds.
Rule  \textbf{SQPA} infers $\cqat{\kappa}{d}{\Pi}$ for an arbitrary $d \in \aqdom$, provided that $\Pi \model{\cdom} \kappa$ holds. This makes sense because the requirements for admissible triples in Definition \ref{defn:simrel:admissible} include the assumption that 
$\simrel(p,p') \neq \bt$ cannot happen if $p, p'  \in PP$ are syntactically different primitive predicate symbols. 

As usual in formal inference systems, $\SQCHL(\simrel,\qdom,\cdom)$ proofs can be represented as {\em proof trees} $T$ whose nodes correspond to qc-atoms, each node being  inferred from its children by means of some  $\SQCHL(\simrel,\qdom,\cdom)$ inference step. In the rest of the report we will use the following notations:

\begin{itemize}
\item
$\Vert T \Vert$ will denote the {\em size} of  the proof tree $T$,  measured as its number of nodes,
which equals   the number of inference steps in the $\SQCHL(\simrel,\qdom,\cdom)$ proof represented by $T$.
\item
$\Vert T \Vert_d$ will denote the number of nodes of  the proof tree $T$ that represent conclusions of
\textbf{SQDA} inference steps. Obviously, $\Vert T \Vert_d \leq \Vert T \Vert$.
\item
$\Prog \sqchlrdc \varphi$ will indicate that $\varphi$ can be inferred
from $\Prog$ in $\SQCHL(\simrel,\qdom,\cdom)$.
\item
$\Prog \sqchlrdcn{k} \varphi$ will indicate that $\varphi$ can be inferred from $\Prog$ in $\SQCHL(\simrel,\qdom,\cdom)$ using some proof tree $T$ such that $\Vert T \Vert_d = k$.
\end{itemize}

The next example  shows a $\SQCHL(\simrel,\U,\rdom)$ proof tree.



\begin{exmp}[$\SQCHL(\simrel,\qdom,\cdom)$ proof tree]
\label{exmp:sqchl-inference}
Recall the proximity relation $\simrel$ and the program  $\Prog$ from our running Example \ref{exmp:running},
as well as the observable qc-statement $\varphi_2 = \cqat{p'(c'(Y),c(X))}{0.8}{\Pi}$ already known from Example \ref{exmp:qc-atoms}.
A $\SQCHL(\simrel,\U,\rdom)$ proof tree witnessing $\Prog \sqchl{\simrel}{\U}{\rdom} \varphi_2$ can be displayed as follows:

$$
\spadesuit = \displaystyle\frac
{~
  \displaystyle\frac{}{\cqat{(Y == Y)}{1.0}{\Pi}} ~ (5) \qquad
  \displaystyle\frac{}{\cqat{(c(X) == c(Y))}{1.0}{\Pi}} ~ (6)
~}
{\cqat{q(Y,c(X))}{1.0}{\Pi}} ~ (4)
$$

$$
\displaystyle\frac
{~
  \displaystyle\frac{}{\cqat{(c'(Y) == c(Y))}{0.8}{\Pi}} ~ (2) \qquad
  \displaystyle\frac{}{\cqat{(c(X) == c(X))}{1.0}{\Pi}} ~ (3) \qquad
  \spadesuit ~ (4)
~}
{\cqat{p'(c'(Y),c(X))}{0.8}{\Pi}} ~ (1)
$$

\smallskip
The inference steps in this proof are commented below.
For the sake of clarity, we have used a different variant of the corresponding program clause
for each each application of the inference rule  \textbf{SQDA}.
\begin{enumerate}
  \item[(1)]
  \textbf{SQDA} step with clause $R_1 = (~ p(c(X_1),Y_1) \qgets{0.9} q(X_1,Y_1) ~)$
  instantiated by  substitution $\theta_1 = \{ X_1 \mapsto Y, Y_1 \mapsto c(X) \}$.
  Note that $0.8$ satisfies $0.8 \le \simrel(p,p') = 0.8$, $0.8 \le 0.8$, $0.8 \le 1.0$, $0.8 \le 0.9 \times 1.0$.
  \item[(2)]
  \textbf{SQEA} step. $c'(Y) \approx_{0.8, \Pi} c(Y)$ holds due to
  $c'(Y) \approx_{\Pi} c'(Y)$, $c(Y) \approx_{\Pi} c(Y)$ and $c'(Y) \approx_{0.8} c(Y)$.
   \item[(3)]
   \textbf{SQEA} step. $c(X) \approx_{1.0, \Pi} c(X)$ holds for trivial reasons.
  \item[(4)]
  \textbf{SQDA} step with clause $R_2 = (~ q(X_2,c(X_2)) \qgets{1.0} ~)$
  instantiated by substitution $\theta_2 = \{ X_2 \mapsto Y \}$.
  Note that $1.0$ satisfies $1.0 \le \simrel(q,q) = 1.0$ and $1.0 \le 1.0$.
  \item[(5)]
  \textbf{SQEA} step. $Y \approx_{1.0, \Pi} Y$ holds for trivial reasons.
  \item[(6)]
  \textbf{SQEA} step. $c(X) \approx_{1.0, \Pi} c(Y)$ holds due to
  $c(X) \approx_{\Pi} c(Y)$ (which follows from $\Pi \model{\rdom} X == Y$)
  and $c(X) \approx_{1.0} c(X)$.
  \enspace \mathproofbox
\end{enumerate}

\end{exmp}

The next technical lemma establishes two basic properties of formal inference in the $\SQCHL(\simrel,\qdom,\cdom)$ logic.

\begin{lem}[Properties of $\SQCHL(\simrel,\qdom,\cdom)$ derivability]
\label{lem:ep}
Let $\Prog$ be any $\sqclp{\simrel}{\qdom}{\cdom}$-program. Then:
\begin{enumerate}
\item \label{lem:ep:1}
\emph{$\Prog$-independent Inferences}: \\�
Given any $\cdom$-based qc-atom $\varphi$ and any qc-interpretation $\I$, one has:
$$\Prog  \sqchlrdcn{0} \varphi \Longleftrightarrow \Prog \sqchlrdc \varphi \Longleftrightarrow \I \isqchlrdc \varphi \enspace .$$
\item \label{lem:ep:2}
\emph{Entailment Property  for Programs}: \\�
Given any pair of qc-atoms $\varphi$ and $\varphi'$ such that $\Prog \sqchlrdc \varphi$ with inference proof tree $T$ and $\varphi \entail{\qdom,\cdom} \varphi'$, then $\Prog \sqchlrdc \varphi'$ with an inference proof tree $T'$ of the same size and structure as $T$.
\end{enumerate}
\end{lem}

\begin{proof*}[Proof of $\Prog$-independent Inferences]
Since $\varphi$ is $\cdom$-based, we can assume $\varphi = \cqat{A}{d}{\Pi}$ where $A$ is either an equation or a primitive atom.
In both cases the equivalence $\Prog  \sqchlrdcn{0} \varphi \Longleftrightarrow \Prog \sqchlrdc \varphi$ is obvious.
In order to prove the equivalence $\Prog \sqchlrdc \varphi \Longleftrightarrow \I \isqchlrdc \varphi$ we distinguish the two cases:
\begin{enumerate}
  \item
  $\varphi$ is equational.
  Then $A$ has the form $t == s$.
  Considering  the $\SQCHL(\simrel,\qdom,\cdom)$-inference rule \textbf{SQEA}
  and the second item of Definition \ref{defn:interpretations}, we get
  $$\Prog \sqchlrdc \varphi \Longleftrightarrow s \approx_{d, \Pi} t \Longleftrightarrow \I \isqchlrdc \varphi \enspace .$$
  \item
  $\varphi$ is primitive.
  Then $A$ is a primitive atom $\kappa$.
  Considering the $\SQCHL(\simrel,\qdom,\cdom)$-inference rule \textbf{SQPA}
  and the second item of Definition \ref{defn:interpretations}, we get
  $$\Prog \sqchlrdc \varphi \Longleftrightarrow \Pi \model{\cdom} \kappa \Longleftrightarrow \I \isqchlrdc \varphi \enspace . \mathproofbox$$
\end{enumerate}
\end{proof*}

\begin{proof*}[Proof of Entailment Property  for Programs]
Due to the hypothesis   $\varphi \entail{\qdom,\cdom} \varphi'$ and Definition \ref{defn:atoms-entail},
we can assume $\varphi = \cqat{A}{d}{\Pi}$ and $\varphi' = \cqat{A'}{d'}{\Pi'}$
with $A' = A\theta$, $d' \dleq d$ and $\Pi' \model{\cdom} \Pi\theta$ for some substitution $\theta$.
We reason by complete induction on $\Vert T \Vert$.
There are three possible cases,  according to the the syntactic form of the atom $A$.
In each case we argue how to build the desired proof tree $T'$.
\begin{enumerate}
\item $A$ is a defined atom:
In this case, $A$ is $p(\ntup{t}{n})$ with $p \in DP^n$, and $A'$: is  $p(\ntup{t'}{n})$ with $p(\ntup{t'}{n}) = p(\ntup{t}{n})\theta$.
Moreover, $T$ must be a proof tree of the following  form:
$$
T :
\displaystyle\frac
  {~
    \left( \displaystyle\frac{}{\cqat{(t_i == s_i\sigma)}{d_i}{\Pi}} \right)_{i=1 \ldots n} \quad
    \left( \displaystyle\frac{\cdots}{\cqat{B_j\sigma}{e_j}{\Pi}} \right)_{j=1 \ldots m}
  ~}
  {\cqat{p(\ntup{t}{n})}{d}{\Pi}} ~ \mathbf{SQDA}
$$
where:
  \begin{itemize}
\item
  The \textbf{SQDA} root inference uses
  some $R_l : (q(\ntup{s}{n}) \qgets{\alpha} \qat{B_1}{w_1}, \ldots, \qat{B_m}{w_m}) \in \Prog$,
  some substitution $\sigma$
  and some qualification values $d_0, d_1, \ldots, d_n, e_1, \ldots e_m \in \aqdom$ such that
  $\simrel(p,q) = d_0 \neq \bt$, $d \dleq d_i ~ (0 \le i \le n)$ and $d \dleq \alpha \circ e_j ~ (1 \le j \le m)$.
\item
  For $i = 1 \ldots n$, $\cqat{(t_i == s_i\sigma)}{d_i}{\Pi}$ has a proof tree $T_{i}^h$
  with $\Vert T_{i}^h \Vert < \Vert T \Vert$.
\item
  For $j = 1 \ldots m$, $\cqat{B_j\sigma}{e_j}{\Pi}$ has a proof tree $T_{j}^b$
  with $\Vert T_{j}^b \Vert < \Vert T \Vert$.
  \end{itemize}
Then, $T'$ can be built as a proof tree of the form:
$$
T' :
\displaystyle\frac
  {~
    \left( \displaystyle\frac{}{\cqat{(t'_i == s_i\sigma\theta)}{d_i}{\Pi'}} \right)_{i=1 \ldots n} \quad
    \left( \displaystyle\frac{\cdots}{\cqat{B_j\sigma\theta}{e_j}{\Pi'}} \right)_{j=1 \ldots m}
  ~}
  {\cqat{p(\ntup{t'}{n})}{d'}{\Pi'}} ~ \mathbf{SQDA}
$$
where:
  \begin{itemize}
\item
  The \textbf{SQDA} root inference uses the same program clause $R_l \in \Prog$,
  the substitution $\sigma\theta$
  and the same qualification values $d_i ~ (0 \le i \le n)$ and $e_j ~ (1 \le j \le m)$,
  satisfying $\simrel(p,q) = d_0 \neq \bt$, $d' \dleq d \dleq d_i ~ (0 \le i \le n)$ and $d' \dleq d \dleq \alpha \circ e_j ~ (1 \le j \le m)$.
\item
  For $i = 1 \ldots n$, $\cqat{(t'_i == s_i\sigma\theta)}{d_i}{\Pi'}$ has a proof tree $T_{i}^{'h}$
  of the same size and structure as $T_{i}^h$.
  In fact, $T_{i}^{'h}$ can be obtained by induction hypothesis applied to $T_{i}^h$,
  which is allowed because $\Vert T_{i}^h \Vert < \Vert T \Vert$
  and $\cqat{(t_i == s_i\sigma)}{d_i}{\Pi} \entail{\qdom,\cdom} \cqat{(t'_i == s_i\sigma\theta)}{d_i}{\Pi'}$.
  Note that this entailment holds thanks to substitution $\theta$, since $t'_i = t_i\theta$ and $\Pi' \model{\cdom} \Pi\theta$.
\item
  For $j = 1 \ldots m$, $\cqat{B_j\sigma\theta}{e_j}{\Pi'}$ has a proof tree $T_{j}^{'b}$
  of the same size and structure as $T_{j}^b$.
  In fact, $T_{j}^{'b}$ can be obtained by induction hypothesis applied to $T_{j}^b$,
  which is allowed because $\Vert T_{j}^b \Vert < \Vert T \Vert$
  and $\cqat{B_j\sigma}{e_j}{\Pi} \entail{\qdom,\cdom} \cqat{B_j\sigma\theta}{e_j}{\Pi'}$.
  Note that this entailment holds thanks to substitution $\theta$, since $\Pi' \model{\cdom} \Pi\theta$.
  \end{itemize}
By construction, $T'$ has the same size and structure as $T$, as desired.
\item $A$ is an equation:
In this case, $A : t == s$ and $A' : t' == s'$ with $t' = t\theta$, $s' = s\theta$.
Moreover, $T$ must consist of one single node $\cqat{(t == s)}{d}{\Pi}$ inferred by means of \textbf{SQEA}.
Therefore, $t \approx_{d, \Pi} s$ holds.
This implies $t\theta \approx_{d, \Pi\theta} s\theta$ (i.e. $t' \approx_{d, \Pi\theta} s'$)
due to the Substitution Lemma \ref{lem:slpeq}.
From this we conclude $t' \approx_{\Pi'} s'$ due to $d' \dleq d$ and $\Pi' \model{\cdom} \Pi\theta$.
Therefore, $T'$ can be built as a proof tree consisting of one single node $\cqat{(t' == s')}{d'}{\Pi'}$ inferred by means of \textbf{SQEA}.
\item $A$ is a primitive atom:
In this case, $A : \kappa$ and $A' : \kappa' = \kappa\theta$.
Moreover, $T$ must consist of one single node $\cqat{\kappa}{d}{\Pi}$ inferred by means of \textbf{SQPA}.
Therefore, $\Pi \model{\cdom} \kappa$ holds.
This implies $\Pi\theta \model{\cdom} \kappa\theta$ due to the Substitution Lemma \ref{lem:sl}.
From this we conclude $\Pi' \model{\cdom} \kappa'$ due to $\kappa' = \kappa\theta$ and $\Pi' \model{\cdom} \Pi\theta$.
Therefore, $T'$ can be built as a proof tree consisting of one single node $\cqat{\kappa'}{d'}{\Pi'}$ inferred by means of \textbf{SQPA}.
\mathproofbox
\end{enumerate}
\end{proof*}

The following theorem is the main result in this subsection.
It  characterizes the least model of a $\sqclp{\simrel}{\qdom}{\cdom}$-program $\Prog$ w.r.t. the logic $\SQCHL(\simrel,\qdom,\cdom)$:

\begin{thm}[Logical characterization of least program models]
\label{thm:SQCHL-leastmodel}
For any $\sqclp{\simrel}{\qdom}{\cdom}$-program $\Prog$, its least model can be characterized as: $$\Mp = \{\varphi \mid \varphi \mbox{ is a defined observable qc-atom and }\Prog \sqchlrdc \varphi\} \enspace .$$
\end{thm}
\begin{proof*}
By Theorem \ref{thm:tp-leastmodel}, we already know that $\Mp = \union_{k\in\NAT} \Tp{\uparrow}^k(\ibot)$.
Therefore, it is sufficient to prove that the two implications
\begin{enumerate}
\item \label{thm:SQCHL-leastmodel:1} 
$\Prog \sqchlrdcn{k} \varphi \Longrightarrow \exists k' : \varphi \in \Tp{\uparrow}^{k'}(\ibot)$
\item \label{thm:SQCHL-leastmodel:2} 
$\varphi \in \Tp{\uparrow}^k(\ibot) \Longrightarrow \exists k' : \Prog \sqchlrdcn{k'} \varphi$
\end{enumerate}
hold for any defined observable qc-atom $\varphi = \cqat{p(\ntup{t}{n})}{d}{\Pi}$ and for any integer value $k \geq 1$.
We prove both implications within one single inductive reasoning on $k$.

\begin{description}
\item[Basis ($k = 1$).] \hfill \\
--- {\it Implication 1.}
Assume $\Prog \sqchlrdcn{1} \varphi$.
Then, due to the single \textbf{SQDA} inference,
there must exist  some $R_l = (q(\ntup{s}{n}) \qgets{\alpha}) \in \Prog$ with empty body,
some substitution $\theta$ and some $d_0, d_1, \ldots, d_n \in \aqdom$
such that $\Prog \sqchlrdcn{0} \cqat{(t_i == s_i\theta)}{d_i}{\Pi}$ for $i = 1 \ldots n$,
$\simrel(p,q) = d_0 \neq \bt$, $d \dleq d_i ~ (0 \le i \le n)$ and $d \dleq \alpha$.
Then $\ibot \sqchlrdcn{0} \cqat{(t_i == s_i\theta)}{d_i}{\Pi}$ holds for $i = 1 \ldots n$,
because of Lemma \ref{lem:ep}(\ref{lem:ep:1}).
Therefore $\varphi$ is an immediate consequence of $\ibot$ via $R_l$,
which guarantees $\varphi \in \Tp{\uparrow}^1(\ibot)$. \\
--- {\it Implication 2.}
Assume now $\varphi \in \Tp{\uparrow}^1(\ibot)$.
Then $\varphi$ must be an immediate consequence of $\ibot$
via some $R_l = (q(\ntup{s}{n}) \qgets{\alpha}) \in \Prog$ with empty body.
Then there are some substitution $\theta$ and some $d_0, d_1, \ldots, d_n \in \aqdom$
such that $\ibot \isqchlrdc \cqat{(t_i == s_i\theta)}{d_i}{\Pi}$ for $i = 1 \ldots n$,
$\simrel(p,q) = d_0 \neq \bt$, $d \dleq d_i ~ (0 \le i \le n)$ and $d \dleq \alpha$.
Again because of Lemma \ref{lem:ep}(\ref{lem:ep:1}),
we get $\Prog \sqchlrdcn{0} \cqat{(t_i == s_i\theta)}{d_i}{\Pi}$ for $i = 1 \ldots n$,
which guarantees $\Prog \sqchlrdcn{1} \varphi$
with one single \textbf{SQDA} inference using $R_l$ instantiated by $\theta$.

\smallskip
\smallskip
\item[Inductive step ($k > 1$).] \hfill \\
--- {\it Implication 1.} Assume $\Prog \sqchlrdcn{k} \varphi$. Since the root inference must be \textbf{SQDA},  there must exist  some program rule $(R_l : q(\ntup{s}{n}) \qgets{\alpha} \qat{B_1}{w_1}, \ldots, \qat{B_m}{w_m}) \in \Prog$, some substitution $\theta$ and some qualification values $d_0, d_1$, \ldots, $d_n, e_1, \ldots, e_m \in \aqdom$ such that
      \begin{itemize}
        \item $\Prog \sqchlrdcn{0} \phi_i = (\cqat{(t_i == s_i\theta)}{d_i}{\Pi})$ for $i = 1 \ldots n$,
        \item $\Prog \sqchlrdcn{k_j} \psi_j = (\cqat{B_j\theta}{e_j}{\Pi})$ with $e_j \dgeq^? w_j$ for $j = 1 \ldots m$, and
        \item $\simrel(p,q) = d_0 \neq \bt$, $d \dleq d_i ~ (0 \le i \le n)$ and $d \dleq \alpha \circ e_j ~ (1 \le j \le m)$
      \end{itemize}
      where $\Sigma_{j = 1}^{m} k_j = k - 1$. For each $j = 1 \ldots m$,
      either $\psi_j$ is defined, and then induction hypothesis yields some $k'_j$ such that
       $\psi_j \in \Tp{\uparrow}^{k'_j}(\ibot)$ and therefore also $\Tp{\uparrow}^{k'_j}(\ibot) \isqchlrdc \psi_j$;
      or else $\psi_j$ is not defined and then $\Tp{\uparrow}^{k'_j}(\ibot) \isqchlrdc \psi_j$
      for any arbitrarily chosen $k'_j$, by Lemma \ref{lem:ep}(\ref{lem:ep:1}).
      Then $l = max \{k'_j \mid 1 \leq j \leq m\}$ verifies that $\varphi$ is an immediate consequence of
      $\Tp{\uparrow}^{l}(\ibot)$ via $R_l$, which implies $\varphi \in \Tp{\uparrow}^{k'}(\ibot)$ for $k' = l+1$. \\
--- {\it Implication 2.} Assume $\varphi \in \Tp{\uparrow}^k(\ibot) = \Tp(\Tp{\uparrow}^{k-1}(\ibot))$.
Then $\varphi$ is an immediate consequence of $Tp{\uparrow}^{k-1}(\ibot)$
via some clause $(R_l : q(\ntup{s}{n})\qgets{\alpha} \qat{B_1}{w_1}, \ldots,$ $\qat{B_m}{w_m}) \in \Prog$.
Therefore, there exist some substitution $\theta$ and some qualification values $d_0, d_1$, \ldots, $d_n, e_1, \ldots, e_m$ $\in$ $\aqdom$ such that:
      \begin{itemize}
        \item $\Tp{\uparrow}^{k{-}1}(\ibot) \isqchlrdc \phi_i = (\cqat{(t_i == s_i\theta)}{d_i}{\Pi})$ for $i = 1 \ldots n$,
        \item $\Tp{\uparrow}^{k{-}1}(\ibot) \isqchlrdc \psi_j = (\cqat{B_j\theta}{e_j}{\Pi})$ with $e_j \dgeq^? w_j$ for $j = 1 \ldots m$, and
        \item $\simrel(p,q) = d_0 \neq \bt$, $d \dleq d_i ~ (0 \le i \le n)$ and $d \dleq \alpha \circ e_j ~ (1 \le j \le m)$.
      \end{itemize}
For each $i = 1 \ldots n$, Lemma \ref{lem:ep}(\ref{lem:ep:1}) yields $\Prog \sqchlrdcn{0} \phi_i$.
For each $j = 1 \ldots m$,
either $\psi_j$ is defined,  in which case $\psi_j \in \Tp{\uparrow}^{k{-}1}(\ibot)$, $k-1 \geq 1$,
and induction hypothesis yields some $k'_j$ such that $\Prog \sqchlrdcn{k'_j} \psi_j$;
or else $\psi_j$ is not defined, in which case $\Prog \sqchlrdcn{k'_j} \psi_j$ for $k'_j = 0$, by Lemma \ref{lem:ep}(\ref{lem:ep:1}).
In these conditions, $\Prog \sqchlrdcn{k'} \varphi$ holds for $k' = 1 + \Sigma_{j=1}^{m} k'_j$,
with a proof tree using a \textbf{SQDA} root inference based on $R_l$ instantiated by $\theta$. \mathproofbox
\end{description}
\end{proof*}

As an easy consequence of the previous theorem we get:

\begin{cor}[$\SQCHL(\simrel,\qdom,\cdom)$ is sound and complete]
\label{cor:correctness}
For any $\sqclp{\simrel}{\qdom}{\cdom}$-program $\Prog$ and any observable qc-atom $\varphi$, the following three statements are equivalent:
$$
    (a) ~ \Prog \sqchlrdc \varphi \hspace*{1cm}
    (b) ~ \Prog \model{\simrel,\qdom,\cdom} \varphi \hspace*{1cm}
    (c) ~ \Mp \isqchlrdc \varphi
$$
Moreover, we also have:
\begin{enumerate}
  \item {\em Soundness}: $\Prog \sqchlrdc \varphi \Longrightarrow \Prog \model{\simrel,\qdom,\cdom} \varphi$.
  \item {\em Completeness}: $\Prog \model{\simrel,\qdom,\cdom} \varphi \Longrightarrow \Prog \sqchlrdc \varphi$.
\end{enumerate}
\end{cor}
\begin{proof}
Soundness and completeness are just a trivial consequence of $(a) \Leftrightarrow (b)$.
To finish the proof it suffices to  prove the two equivalences $(a) \Leftrightarrow (c)$ and $(b) \Leftrightarrow (c)$.
This is done as follows:

\smallskip
\noindent [$(a) \Leftrightarrow (c)$] In the case that $\varphi$ is a defined qc-atom, $\Mp \isqchlrdc \varphi$ reduces to $\varphi \in \Mp$ which is equivalent to $\Prog \sqchlrdc \varphi$ by Theorem \ref{thm:SQCHL-leastmodel}.
 Otherwise, $\Prog \sqchlrdc \varphi \Longleftrightarrow \Mp \isqchlrdc \varphi$ holds because of Lemma \ref{lem:ep}(\ref{lem:ep:1}).

\smallskip
\noindent [$(b) \Rightarrow (c)$] Assume $\Prog \model{\simrel,\qdom,\cdom} \varphi$ and recall Definition \ref{defn:models}.
Then $\I \isqchlrdc \varphi$ for every qc-interpretation $\I$ such that $\I \model{\simrel,\qdom,\cdom} \Prog$.
In particular, $\Mp \isqchlrdc \varphi$, since $\Mp \model{\simrel,\qdom,\cdom} \Prog$ was proved in Theorem \ref{thm:tp-leastmodel}.

\smallskip
\noindent [$(c) \Rightarrow (b)$] Assume $\Mp \isqchlrdc \varphi$.
In order to obtain $\Prog \model{\simrel,\qdom,\cdom} \varphi$ we must prove:
$$(\star) \quad \I \isqchlrdc \varphi \mbox{ holds for any qc-interpretation } \I \mbox{ such that } \I \model{\simrel,\qdom,\cdom} \Prog \enspace .$$
In the case that $\varphi$ is a defined qc-atom, $\Mp \isqchlrdc \varphi$ reduces to $\varphi \in \Mp$,
which implies $(\star)$ because $\Mp$ is the least model of $\Prog$, as proved in Theorem \ref{thm:tp-leastmodel}.
In the case that $\varphi$ is not defined but  $\cdom$-based, $(\star)$ follows form the fact
that $\I \isqchlrdc \varphi$ holds for any arbitrary qc-interpretation $\I$, as proved in Lemma \ref{lem:ep}(\ref{lem:ep:1}).
\end{proof}

We close this subsection with a brief discussion on the relationship between the entailment
relation $\!\!\entail{\qdom,\cdom}\!\!$ used in this report and a different one that was proposed
in \cite{CRR08} and noted $\!\!\entail{\simrel,\qdom}\!\!$.
In contrast to $\!\!\entail{\qdom,\cdom}\!\!$, the entailment $\!\!\entail{\simrel,\qdom}\!\!$
depended on a given {\em similarity} relation $\simrel$.
In the context of the SQCLP scheme, one could think of an entailment 
$\!\!\entail{\simrel,\qdom,\cdom}\!\!$ depending on $\simrel$ and defined in the following way:
given two qc-atoms $\varphi$ and $\varphi'$, we could say that $\varphi$ $(\simrel,\qdom,\cdom)$-entails
$\varphi'$ (in symbols, $\varphi \entail{\simrel,\qdom,\cdom} \varphi'$) iff $\varphi : \cqat{A}{d}{\Pi}$ and
$\varphi' : \cqat{A'}{d'}{\Pi'}$ such that there is some substitution $\theta$ satisfying 
$\simrel(A',A\theta) = \lambda \neq \bt$, $d' \dleq \lambda$, $d' \dleq d$ and $\Pi' \model{\cdom} \Pi\theta$.

However, $\!\!\!\!\entail{\simrel,\qdom,\cdom}\!\!\!\!$ would not work properly in the case that $\simrel$ is not transitive, as shown by the following simple example: think of a $\sqclp{\simrel}{\U}{\rdom}$-program $\Prog$ including just a clause
\begin{itemize}
  \item[] $R_1 : p_1 \qgets{1.0}$
\end{itemize}
and assume that $\simrel$ verifies $\simrel(p_1,p_2) = 0.9$, $\simrel(p_2,p_3) = 0.9$ and $\simrel(p_1,p_3) = 0.4$ where $p_1,p_2,p_3 \in DP^0$. Then, $\Prog \sqchl{\simrel}{\U}{\rdom} \cqat{p_2}{0.9}{\emptyset}$ can be easily proved with the SQCHL rule {\bf SQDA} and $\cqat{p_2}{0.9}{\emptyset} \entail{\simrel,\U,\rdom} \cqat{p_3}{0.9}{\emptyset}$ holds because of $\simrel(p_2,p_3) = 0.9$, but $\Prog \sqchl{\simrel}{\U}{\rdom} \cqat{p_3}{0.9}{\emptyset}$ does not hold. Therefore, the Entailment Property for Programs (Lemma \ref{lem:ep}(\ref{lem:ep:2})) would fail if the entailment $\!\!\entail{\simrel,\qdom,\cdom}\!\!$ were adopted in place of $\!\!\entail{\qdom,\cdom}\!\!$.

Since the Entailment Property for Programs is a very natural condition that must be preserved, we conclude that
the entailment relation $\!\!\entail{\qdom,\cdom}\!\!$ used in this report is the right choice in a framework where the
underlaying proximity relation is not guaranteed to be a similarity.