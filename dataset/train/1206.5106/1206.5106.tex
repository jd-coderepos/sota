\documentclass[12pt]{llncs}
\usepackage{graphicx,  amsmath, amssymb, algorithm, algpseudocode,caption} 







\title{On List Colouring and List Homomorphism of Permutation and Interval Graphs}

\author{Jessica Enright  \inst{1} \and Lorna Stewart \inst{1} \and G\'{a}bor Tardos \inst{2}}

\institute{University of Alberta\\ \and Simon Fraser University and R\'enyi Institute }

\date{}

\bibliographystyle{plain}

\begin{document}
\newenvironment{my_enumerate}{
\begin{enumerate}
  \setlength{\itemsep}{1pt}
  \setlength{\parskip}{0pt}
  \setlength{\parsep}{0pt}}
  {\end{enumerate}
}
\newenvironment{my_itemize}{
\begin{itemize}
  \setlength{\itemsep}{1pt}
  \setlength{\parskip}{0pt}
  \setlength{\parsep}{0pt}}
  {\end{itemize}
}
\newenvironment{my_proof}{   
\begin{proof} 
     }
     {  \end{proof}
}

\maketitle

\begin{abstract}
List colouring is an NP-complete decision problem even if the total number of
colours is three.  It is hard even on planar bipartite graphs.  We give a
polynomial-time algorithm for solving list colouring of permutation graphs
with a bounded total number of colours. More generally we give a
polynomial-time algorithm that solves the list-homomorphism problem to any
fixed target graph for a large class of input graphs including all permutation
and interval graphs.
\end{abstract}

\section{Introduction}
A {\em proper colouring} of a graph assigns colours to the vertices such that adjacent vertices receive distinct colours. (In this paper we deal only with vertex colourings.) The {-colouring problem} asks if a given graph has a proper colouring with at most  colours. For  this is NP-complete. 

In the {\em list colouring problem} each vertex of the input graph comes with a list of allowed colours and we ask if a proper colouring exists where each vertex receives a colour from its list. As a generalization of ordinary colouring, it is NP-complete \cite{toft}.  List colouring remains hard even on interval graphs \cite{biro1992}, as well as split graphs, cographs, and bipartite graphs \cite{jansenScheffler}.  It is solvable in polynomial time on trees \cite{jansenScheffler}.

Kratochv\'il and Tuza \cite{kratochvilTuza} showed that list colouring is NP-complete even if the size of each list assigned to a vertex is at most three, each colour appears in at most three lists, each vertex in the graph has degree at most three, and the graph is planar.  However, they gave polynomial-time algorithms to solve list colouring on a graph if the maximum list size is at most two, or each colour appears in at most two lists, or each vertex has degree at most two.

Let {\em -list colouring} stand for the list colouring problem where the total number of colours is bounded by the constant . This is a generalization of -colouring, thus for  it is NP-complete.  It remains NP-complete on planar bipartite graphs \cite{kratochvilFixedColourBound}, but is solvable in polynomial time on graphs of fixed treewidth \cite{bigPaper}.    

Note that -list colouring is solvable in polynomial time. Indeed,
2-colouring is solvable in polynomial time and has at most two
(complementary) solutions on each connected component. Thus, for the
-list colouring problem it is enough to check that one of these is
compatible with the lists on each component.

A {\em graph homomorphism} from a graph  to another graph  is a
function  satisfying that  and  are adjacent in
 whenever  and  are adjacent in . Note that here we allow the
graphs to have loops.

Let  be fixed graph. The -colouring problem takes a graph  as
input and asks if there is  to  homomorphism. In the list -colouring
problem each vertex of the input graph comes with a list of vertices of 
and we ask if a  to  homomorphism exists that maps each vertex to a
member of its list. Clearly, -colouring is a
graph-homomorphism to the complete graph , so list -colouring is a
generalization of -list colouring.

Permutation graphs are comparability cocomparability graphs (see definitions
in the next section). List colouring is NP-complete
on permutation graphs since cographs are
permutation graphs \cite{jansenScheffler}. The -list colouring problem is
NP-complete for comparability graphs for , since bipartite graphs are
comparability graphs. The complexity of -list colouring of cocomparability
graphs remains open.

In this paper we give a polynomial-time algorithm for the -list colouring
of permutation graphs for any fixed . More generally we give a
polynomial-time algorithm that solves the list-homomorphism problem to any
fixed target graph for permuatation graphs. The same algorithm also works for
interval graphs and more.

Our algorithm is based on what we call a {\em multi-chain
ordering} (see definition in the next section), a notion related to 
chain graphs \cite{Yann} and to a characterization of 
bipartite permutation graphs given in \cite{BL}.
The algorithm applies to every graph with all connected induced subgraphs
having a multi-chain ordering, among them all permutation graphs and
all interval graphs. We also remark that since adding loops to a graph does not have
any effect on the multi-chain ordering, our algorithm also applies to interval
and permutation graphs with loops added to some vertices.
The running time for -list colouring, or more
generally, for list -colouring for a graph  on  vertices is
, where  stands for the number of vertices of the input
graph.

 Ho{\`{a}}ng \textit{et al.} \cite{hoangP5} give an algorithm for -list colouring 
 -free graphs in polynomial time. 
 Their algorithm, like ours, is based on how a colouring of one side of a bipartition of a chain graph
 can restrict the other side to a polynomial number of possible colourings.  
 




We mention here that a polynomial-time -list colouring algorithm for
interval graphs cannot be considered
new. Indeed, another polynomial-time algorithm already exists for list
-colouring graphs
with bounded treewidth. The treewidth of an interval graph is
one less than the size of its largest clique. Thus, unless it is
bounded one does not have a proper colouring with a bounded number of
colours. The same cannot be said about permutation graphs though. Even
bipartite permutation graphs have unbounded treewidth.


Multi-chain orderings are based on distance from a
starting vertex. They give insight into the structure of permutation
or interval graphs, and may lead to algorithms for other problems on
these or similar graphs.

\section{Definitions and Preliminaries}
We consider finite graphs only with no multiple edges. We allow for loop
edges connecting a vertex to itself and call a graph {\em simple} if it has
no such edge. (Loop edges in the input graph only make
sense for the list
-colouring problem if  has at least one loop, so in particular, not for
-list colouring.) We represent graphs as a pair , where
 is the vertex set and  is the edge set. We denote the edge
connecting  to  by , so . In a {\em
directed graph} we have ordered pairs of vertices as edges and
denote such an edge as  saying it leaves the
vertex  and is oriented toward the vertex . A {\em sink} is a
vertex that no edge leaves. A directed graph is {\em transitive} if
the presence of the edges  and 
implies the presence of . An {\em orientation} of
the simple graph  is a directed graph ,
where  is obtained by replacing each edge
 by one of its orientations:  or
 but not both. A {\em comparability graph} is a
simple graph that admits a transitive orientation. Equivalently, a graph is a
comparability graph if there is a partial order on the vertices with
exactly the adjacent (distinct) vertices being comparable. The {\em
complement} of the simple graph  is ,
where  contains all possible non-loop edges on  not in . We
sometimes call the edges of  the {\em nonedges} of . A {\em
cocomparability graph} is a graph whose complement is a
comparability graph.
Graphs that are simultaneously comparability and cocomparability graphs are called {\em permutation graphs}. Permutation graphs are exactly the graphs  that are obtained from a permutation  by setting  and . A
simple graph is an {\em interval
graph} if one can identify its vertices with real intervals
such that two vertices are adjacent if and only if the corresponding
intervals intersect.
Such intervals can always be chosen to have distinct endpoints.
{\em Weakly chordal graphs} are simple graphs with no induced  or , for .

Let  be a graph.  A \emph{list mapping} of  is a
mapping that assigns a set (list) of colours to each vertex in .
A colouring of  \emph{obeys} a list mapping if it assigns every
vertex a colour from its list. More generally, if the graph  is fixed a
{\em list mapping} of  assigns a subset of  (a list) to every vertex
of . A homomorphism from  to  {\em obeys} the list mapping if each
vertex is mapped to member of its list.

A {\em chain graph} is a bipartite graph that contains no induced .
This name was introduced by Yannakakis \cite{Yann}. The following
characterization is easily seen to be equivalent to the definition. A
bipartite graph with sides (partite sets)  and  is a chain graph if and
only if for any two vertices in  the neighborhood of one of them contains
the neighborhood of the other. As a consequence we see that if we order the
vertices of  according to decreasing degree (breaking ties arbitrarily),
then the neighborhood of any vertex in  consists of 
a consecutive (in the ordering) set of vertices in , including the first vertex of .

Let  be a connected graph. The {\em distance layers of
} from a vertex  are , where 
consists of the vertices at distance  from  and  is the
largest integer for which this set is not empty. These layers form
a \emph{multi-chain ordering} of  if for every two consecutive layers 
and  the edges connecting these two layers form a chain graph.

Our algorithm processes each connected component of the input graph
separately. It is based on multi-chain orderings of the components and uses the
following simple properties of such orderings: (a) -colouring of one
layer in a multi-chain ordering has limited effect on the colouring
of the next layer and no direct effect on subsequent layers and (b)
each layer has a vertex that is adjacent to all vertices in the next
layer, thus if this vertex is mapped to  then all non-neighbours of  will be
missing from the -colouring of the next layer, practically reducing the
size of . Note that (b) does not apply if  has a vertex  that is
adjacent to every vertex of  including itself. Fortunately this easy
special case can be handled by alternate methods.

\begin{lemma}\label{lem:TransReg}
Let  be a transitive
orientation of a connected comparability graph . Let  be a
sink 
in  and let  be the distance layers
of  from .
Then for  all edges of  between the
vertices of two consecutive layers  and  are oriented
toward  if  is even and all these edges are oriented toward
 if  is odd.
\end{lemma}
\begin{my_proof}
We proceed by induction on . For  the statement of the lemma
holds since  is a sink. Each  for  has a neighbour
, and an edge between  and  oriented
``the wrong way'' would imply the presence of an edge between  and
 by transitivity, a contradiction.
\end{my_proof}

\begin{lemma}\label{lem:TransComp}
Let 
be a transitive orientation of the complement of a connected comparability graph
. Let   be a sink in  and let  be the distance layers of
 from the vertex . Then for
every pair of layers  where  all edges
of  between  and  are directed
toward .
\end{lemma}
\begin{my_proof}
We proceed by induction on .  For  the statement follows from
 being a sink. Let us consider  and assume for contradiction
that  is an edge of 
with  and , . Now  is not adjacent in 
to any vertex , so by the induction hypothesis we have
 and by transitivity
. But this
contradicts the fact that  has a neighbour  in
, so no orientation of an edge between  and this neighbour
should be present in .
\end{my_proof}

\begin{theorem}\label{perm}
Every connected permutation graph has a multi-chain ordering.  
\end{theorem}
\begin{my_proof}
Let  be a permutation graph and let 
be a transitive orientation of  and 
a transitive orientation of the complement of .

Let  be a vertex that is a sink in both
of the graphs  and ,
the existence of which is shown in \cite{pnueli}.
We claim that the distance layers  of  from
 form a multi-chain ordering.  
To see this assume for a contradiction that 
and  are vertices of two neighbouring layers and
 is adjacent with  but not with 
in  and  is adjacent with  but not with .  We
distinguish two cases according to whether  and  are adjacent in
.

Assume first that  and  are adjacent and assume without loss of
generality that this edge is oriented
toward  in .
By Lemma \ref{lem:TransReg}, either both the edge between  and 
and between  and  are oriented toward , or both are
oriented toward .  In the former case we should have
 by transitivity,
contradicting our assumption that  is not adjacent with  in
. In the latter case we similarly have , again a contradiction.

Now assume that  and  are not adjacent in  and assume
without loss of generality that this nonedge
is oriented toward  in . By
Lemma \ref{lem:TransComp}, the nonedge between  and  is
oriented toward  in . By
transitivity we have  contradicting our assumption that  and  are adjacent in .
\end{my_proof}

Not every graph with a multi-chain ordering is a
permutation graph. Further examples are given by interval graphs as shown by
the next theorem. 
In addition,  and  where , and the graph  defined as  with each edge subdivided once, do not have multi-chain orderings.
Therefore, there are cocomparability graphs and even trees that do not have multi-chain orderings. Moreover, any graph such that all induced subgraphs have multi-chain orderings must be a weakly chordal graph, but not all weakly chordal graphs have multi-chain orderings. We also note that the complement of the graph  is a cocomparability graph that is neither a permutation graph nor an interval graph, in which every connected subgraph has a multi-chain ordering. For further information about these graph classes, the reader is referred to \cite{BLS}.


\begin{theorem}\label{inter}
All connected interval graphs admit multi-chain orderings.
\end{theorem}

\begin{my_proof}
Consider an interval representation in which all interval endpoints are distinct.
We can choose  to be the vertex with the leftmost left endpoint.
One can find reals  such that the layer  of  at
distance  from  consists of the vertices with left endpoint
in . To see that these layers form  a multi-chain
ordering of  take two vertices (intervals) in  and let  be the one with
its left end point more to the left, and  the other. Clearly, all
intervals in  intersecting  must also intersect .
\end{my_proof}

Our list -colouring algorithm works for every graph whose
connected induced subgraphs all have multi-chain orderings, and it runs in
polynomial time as long as  is fixed. The last two theorems show that
this class includes all permutation and interval graphs (and the graphs
obtained from them by adding some loops). Restricting attention to complete
graphs  we get polynomial-time -list colouring algorithms. 


Given a connected graph  and vertex  of ,
we can check whether the distance layers from starting vertex 
form a multi-chain ordering in  time where  is the number of edges
of .
The algorithm for doing so uses breadth-first search to
generate the distance layers from  and to compute the degree
of each vertex in the next layer.
It then uses bucket sort to order the vertices of each layer
by decreasing size of their neighbourhood in the previous layer.  
Finally it checks that for each vertex it holds that its neighbours in the
next layer appear in the beginning of that layer before the
non-neighbours. Each step can be accomplished in  time. 

 As a naive algorithm to check if a connected graph has a multi-chain ordering,
 and generate it if it does, we can start a breadth-first search from
 each vertex, and check to see if that search has given us a
 multi-chain ordering in  time overall.  In some classes,
 for example permutation graphs, this can be done more quickly.  In
 the case of permutation graphs, we can use the output of the linear-time
 recognition algorithm provided by McConnell and Spinrad
 \cite{mcconnell} to  identify a vertex that is a sink in some
 transitive orientation of both the graph and its complement. We can
 then generate the distance layers from this vertex in  time which is
 a multi-chain ordering as the proof of Theorem~\ref{perm} shows. Similarly,
several linear time algorithms exist to find a ``leftmost'' vertex in a interval
 graph, the earliest one being \cite{BoL}. The distance layers can be constructed from there
 in linear time. As the proof of
 Theorem~\ref{inter} shows this is a multi-chain ordering.

\section{The algorithm}
In this section we present our algorithm to list -colour any graph with
the property that all connected induced subgraphs have multi-chain
orderings. The algorithm runs in polynomial time if  is fixed.
Since the algorithm handles connected components separately,
we consider only connected graphs in the following description.

Let  be a connected graph and let  form a multi-chain
ordering of . For  we introduce  for the number
of neighbours of  in  (or  if ) and  for
the number of neighbours of  in  (or  if ). We fix an
ordering of the vertices within each layer according to decreasing 
values breaking ties arbitrarily. 
As observed in the definition of chain graphs, this ordering ensures that the
neighbours of a vertex  among the vertices of the next layer
 must be the first  vertices in that layer.

Let us fix the target graph  with vertex set .
Let  be a list mapping of , so  for
every vertex .

A \emph{configuration} is a pair , where  and
 satisfying that  takes both  and
 as values. We introduce two more special configurations:
 and , where  is the constant
zero function.

These configurations form the vertices of the {\em configuration
graph}. This is a directed graph that contains the edge from
 to  if  and there
is a homomorphism  from the subgraph  of  induced by the
layer  to  {\em providing} for this edge, i.e.,
satisfying the following three conditions:

\begin{itemize}
\item { obeys , i.e., for  we have
  .}
  
\item{ does not assign  to the first 
  vertices in  (recall that  is ordered).}
  
\item {For each  and  with  not adjacent
  to  in  we have .}
\end{itemize}


We call a vertex of the graph  \emph{universal} if it is connected to every
vertex of . In particular, a universal vertex must be connected to itself
too. The importance of the configuration graph is shown by the following
theorem.

\begin{theorem}\label{path}
Assume  has no universal vertex. Then  has a homomorphism to  obeying
 if and only if there exists a directed path from  to 
in the configuration graph.
\end{theorem}

\begin{my_proof}
Assume  is a homomorphism from  to  obeying . For
 define the function  on  by setting  to be the largest
integer  satisfying that  does not map any of the
first  vertices of  to . Clearly,  takes the value  on
 for
the first vertex  of . We know that the vertices in the layer 
have a common neighbour  in . As  is not universal in 
there must exist  not adjacent to  and thus  cannot
take the value  on any neighbour of  making . Thus
 is a configuration. We
claim that  is a directed path in the
configuration graph. Indeed, for  the restriction of
 to  provides for the edge
. Conditions (1) and (2) are satisfied
trivially; to see (3) one has to use our observation that the
neighbours in  of any vertex  are the first
 vertices of that layer.

Conversely, let us assume that there is a directed path from  to
 in the configuration graph. By the layered structure of the
configuration graph this path must be of the form
 with  and appropriate
functions . For  let  be a homomorphism
providing for the  edge and let
 be the union of these maps. We claim that  is a  to
 homomorphism obeying .

The function  obeys  since all its parts  do so by
condition (1).

To see that  is a homomorphism we have to show that the image of every edge
 is an edge in . Clearly,  and  have to come from the same
or neighbouring layers. If they are in the same layer , then
 because  is a
homomorphism. Now assume that for some  we have
vertices  and  such that their images
 and  are not adjacent in . By
condition (2)  does not map the first  vertices
of  to . Thus  is not among the first 
vertices of . By condition (3) on  we have
, so  is not among the first  vertices
of , so it is not adjacent to  as needed.
\end{my_proof}

Our next theorem tells us how to construct the configuration graph,
more precisely, how to decide whether an edge is present. Let us fix
two configurations  and . Let  be
the subgraph of  induced on the layer  and let us define a
list mapping  on  as follows. For 
let  stand for the 'th vertex in the layer  and let us
set .

\begin{theorem}\label{const}
With , ,  and  as above there is an edge from
 to  in the configuration graph if and only if  has a homomorphism
to  obeying .
\end{theorem}

\begin{my_proof}
Any homomorphism providing for  obeys  by
the conditions (1--3). Conversely any homomorphism from  to  that
obeys  provides for this edge.
\end{my_proof}

We now present our algorithm for the list -colouring problem for graphs with
all connected induced subgraphs having a multi-chain ordering. 

\begin{algorithm}
\caption{LH(, , )}
\begin{algorithmic}

\smallskip

\State {\bf Input}: Graphs , , list mapping  where every connected
induced subgraph of  must have a multi-chain ordering
\State {\bf Output}: TRUE if there is a homomorphism from  to  obeying
; FALSE otherwise
\smallskip
\State Let  be the subgraph of  induced by vertices that appear in at
least one list of .

\If {}
  \Return LH(, , )
\EndIf

\If { has a universal vertex }
  \State Let  be the subgraph of  induced by the vertices  with
  
  \State \hspace{\algorithmicindent}  and let  be the restriction of 
  to this subgraph.
  \Return LH(,,)
\EndIf

\If { has a single vertex}
  \If { has a loop or  has no loop and  has at least one vertex}
    \Return TRUE
  \Else
    \Return FALSE
  \EndIf
\EndIf

\For {each connected component  of }

  \If { has at most two vertices}
      \State Find all (the at most two) homomorphisms from  to .
       \If {at least one of the homomorphisms obeys }
            \State  TRUE
        \Else
                 \State  FALSE
        \EndIf

    \Else
        \State Find a multi-chain ordering  of 
        and order the vertices of 
        \State \hspace{\algorithmicindent} each layer by decreasing size of
        neighbourhood in the next layer.
        
        \State Initialize the directed configuration graph to have a vertex
        for each 
        \State \hspace{\algorithmicindent} configuration of this multi-chain ordering including 
        and .

        \For { {\bf to} }
        \State Let  be the subgraph of  induced by .
             \For {each pair of configurations  and }
                \State Construct a list mapping  for  as follows.
                \For { {\bf to} }
                  \State Let  stand for the 'th
                vertex in the layer .
                  \State 
                                \State \hspace{\algorithmicindent}  
                \EndFor
                \If {LH(, , ) = TRUE}
                      \State Add edge  to the
                      configuration graph
                \EndIf
             \EndFor
        \EndFor
                 \algstore{myalg}
  \end{algorithmic}
\end{algorithm}
\clearpage

\begin{algorithm}
  \ContinuedFloat
  \caption{LH(, , ) (continued)}
  \begin{algorithmic}
      \algrestore{myalg}
        \If {there is a directed path from  to  in the configuration graph}
         \State  TRUE
        \Else 
                 \State  FALSE
        \EndIf
\EndIf
\EndFor

\If { TRUE for all components  of }
\Return TRUE
\Else
\Return FALSE
\EndIf
\end{algorithmic}
\end{algorithm}

We are given a fixed graph , an input graph  and the list mapping
. We start with very simple reductions.

If  has a universal vertex , then consider the subgraph  of 
induced by the vertices whose lists do not contain . Clearly,  has a
homomorphism to  obeying  if and only if  has such a
homomorphism as the vertices outside  can ``freely'' be mapped to .

Let  stand for the subgraph of  induced by all the vertices that appear
in the lists in . Clearly,  has a homomorphism to  obeying
 if and only if  has a homomorphism to  obeying .

 has homomorphism to  obeying  if and only if all connected
components of  have homomorphisms to  obeying .

We use the these reductions (repeatedly, if necessary) until we arrive at a
problem in which  is connected,  has no universal
vertex and each vertex of  appears on a list of .

Start by constructing the layers  of a multi-chain ordering of
 with the corresponding ordering of the vertices within the layers
according to decreasing  degrees. Construct the
configurations for this multi-chain ordering including  and .
Construct the edges of the configuration graph using a recursive
  call to check for the presence of each possible edge using the equivalent
  condition as given in Theorem~\ref{const}.
Return TRUE if there is directed path from  to
   in the configuration graph and
  return FALSE otherwise.

Note that the recursive calls to determine the presence of an edge
from the configuration  to  is simpler than the original
problem instance. Indeed, it is a list -colouring problem for  and
 has a single vertex for , while for  we have a vertex  of
 with  and this vertex does not show up in any of the lists ---
basically decreasing the number of vertices in the target graph . To give
base to this recursion we solve the trivial instances directly:
If either  or  has a single vertex, deciding the list -colouring
problem for  becomes trivial. We can also handle the case where  has
two vertices directly. If the two vertices are not adjacent in , we must
map each connected component of  to one or the other vertex. If the two
vertices of  are connected and there is no loop in  we face a -list
colouring problem already discussed in the introduction. Finally if the two
vertices of  are connected and there is also a loop in , then  has a
universal vertex and list -colouring reduces to list -colouring with
 having a single vertex.

Using Theorems~\ref{path} and \ref{const} it is
straightforward to see that the above algorithm correctly answers the
question whether  has a homomorphism to  obeying .

It is a bit more involved to estimate the running time. Let  and  stand
for the number of vertices in  and . We claim the the running time of the
algorithm is  (the constant of proportionality depends on
). We prove
this statement by induction on . For  the algorithm clearly
finishes in time . Let us assume . If  has a universal
vertex our reduction reduces list -colouring to a single list -colouring
instance with  having fewer vertices. If  has no universal vertex we
split  into connected components, find the multi-chain ordering of each
component and build the configuration graphs corresponding to them. The
number of configurations for a fixed layer  of a single component is
 because the value of the function  in a configuration
 is arbitrary for  vertices of , but it has to be either  or
 for two. So the number of configurations for all connected components
together can be bounded by  and the number of potential edges (the
number of recursive calls on the top level) is . In a
recursive call to test the presence of an edge in the configuration graph one
uses a list mapping that avoids at least one vertex of  completely, so the
inductive hypothesis can be used for . The only exception to this
rule is the test for an edge leaving the configuration  of one of the
components,  but there the recursive call is for a
trivial graph on  vertices. These trivial recursive calls
take constant time, the other recursive calls take
 time, so all recursive calls finish in
 time. This huge time bound
clearly dominates the time of the non-recursive part of the algorithm.

\section{Conclusion}
We have given a polynomial-time algorithm to solve the list -colouring
problem for fixed  if every connected induced subgraph of the input graph
has a multi-chain ordering. Every connected permutation or interval graph has
a multi-chain ordering, so this algorithm works for permutation and interval
graphs.

\bibliography{listColouring}

\end{document} 
