    \documentclass{article}



\usepackage[round]{natbib}




\usepackage[final]{neurips_2019}



\usepackage[utf8]{inputenc} \usepackage[T1]{fontenc}    \usepackage{url}            \usepackage{amsfonts}       \usepackage{nicefrac}       \usepackage{microtype}      \usepackage{graphicx}
\usepackage{amsmath}
\usepackage{subcaption}
\usepackage{array}
\usepackage{multirow}
\usepackage{booktabs}       \usepackage{bbding}

\usepackage{color} 


\DeclareMathOperator*{\argmax}{arg\,max}
\DeclareMathOperator*{\argmin}{arg\,min}

\title{
Learning Robust Global Representations\\
by Penalizing Local Predictive Power
}



\author{Haohan Wang, Songwei Ge, Eric P. Xing,
  Zachary C. Lipton \\
  School of Computer Science\\
  Carnegie Mellon University\\
  Pittsburgh, PA 15213 \\
  \texttt{\{haohanw,songweig,epxing,zlipton\}@cs.cmu.edu} \\
}

\begin{document}

\maketitle

\begin{abstract}
Despite their well-documented predictive power on i.i.d. data,
convolutional neural networks have been demonstrated
to rely more on high-frequency (textural) patterns 
that humans deem superficial 
than on low-frequency patterns 
that agree better with intuitions 
about what constitutes category membership.
This paper proposes a method for training robust convolutional networks
by penalizing the predictive power 
of the local representations learned by earlier layers.
Intuitively, our networks are forced to discard predictive signals 
such as color and texture that can be gleaned from local receptive fields
and to rely instead on the global structure of the image.
Across a battery of synthetic and benchmark domain adaptation tasks,
our method confers improved generalization. To evaluate cross-domain transfer,
we introduce ImageNet-Sketch, 
a new dataset consisting of sketch-like images
and matching the ImageNet classification validation set 
in categories and scale.





%
 \end{abstract}

\section{Introduction}
\label{sec:intro}
Consider the task of determining whether a photograph 
depicts a \emph{tortoise} or a \emph{sea turtle}.
A human might check to see whether the shell
is dome-shaped (indicating tortoise) or flat (indicating turtle).
She might also check to see whether the feet 
are short and bent (indicating tortoise)
or fin-like and webbed (indicating turtle).
However, the pixels corresponding to the turtle (or tortoise) itself 
are not alone in offering predictive value.
As easily confirmed through a Google Image search,
sea turtles tend to be photographed in the sea
while tortoises tend to be photographed on land.

Although an image's background may indeed be predictive
of the category of the depicted object,
it nevertheless seems unsettling that our classifiers
should depend so precariously on a signal 
that is in some sense \emph{irrelevant}.
After all, a tortoise appearing in the sea is still a tortoise
and a turtle on land is still a turtle. 
One reason why we might seek to avoid such a reliance 
on correlated but semantically unrelated artifacts 
is that they might be liable to change out-of-sample.
Even if all cats in a training set appear indoors,
we might require a classifier 
capable of recognizing an \emph{outdoors cat} at test time.
Indeed, recent papers have attested to the tendency 
of neural networks to rely on \emph{surface statistical regularities}
rather than learning global concepts \citep{jo2017measuring, geirhos2018imagenettrained}.
A number of papers have demonstrated unsettling drops in performance
when convolutional neural networks are applied to out-of-domain testing data,
even in the absence of adversarial manipulation.


The problem of developing robust classifiers 
capable of performing well on out-of-domain data 
is broadly known as Domain Adaptation (DA).
While the problem is known to be impossible 
absent any restrictions on the relationship 
between training and test distributions \citep{ben2010impossibility},
progress is often possible under reasonable assumptions.
Theoretically-principled algorithms have been proposed
under a variety of assumptions, including covariate shift 
\citep{shimodaira2000improving, gretton2009covariate}
and label shift 
\citep{storkey2009training, scholkopf2012causal, zhang2013domain, lipton2018detecting}. 
Despite some known impossibility results 
for general DA problems \citep{ben2010impossibility}, 
in practice, humans exhibit remarkable robustness 
to a wide variety of distribution shifts,
exploiting a variety of invariances,
and knowledge about what a label actually means.

\begin{figure}
    \centering
    \includegraphics[width=0.85\textwidth]{figs/model.pdf}
    \caption{
In addition to the primary classifier, our model consists of a number of side classifiers, applied at each  location in a designated early layer. 
    The side classifiers result in one prediction per spatial location. 
    The goal of \textbf{patch-wise adversarial regularization} 
    is to fool all of them (via reverse gradient) 
    while nevertheless outputting the correct class from the topmost layer.}
    \label{fig:model}
\end{figure}

Our work is motivated by the intuition 
that for the classes typically of interest in many image classification tasks,
the larger-scale structure of the image is what \emph{makes} the class apply
and while small local patches might be predictive of the label, 
Such local features, considered independently,
\emph{should not} (vis-a-vis robustness desiderata)
comprise the basis for outputting a given classification. 
Instead, we posit that classifiers that are required to 
(in some sense) discard this local signal (i.e., patches of an images correlated to the label within a data collection), 
basing predictions instead on global concepts 
(i.e., concepts that can only be derived by combining information intelligently across regions),
may better mimic the robustness that humans demonstrate in visual recognition.





In this paper, in order to coerce a convolutional neural network 
to focus on the global concept of an object, 
we introduce \textbf{Patch-wise Adversarial Regularization (PAR)},
a learning scheme that penalizes the predictive power
of local representations in earlier layers. 
The method consists of a patch-wise classifier
applied at each 
spatial location in low-level representation.
Via the reverse gradient technique popularized by \citet{ganin2016domain},
our network is optimized to \emph{fool} the side classifiers
and simultaneously optimized to output 
correct predictions at the final layer.
Design choices of PAR include the layer on which the penalty is applied,
the regularization strength, 
and the number of layers in the patch-wise network-in-network classifier.








In extensive experiments 
across a wide spectrum 
of synthetic and real data sets, 
our method outperforms the competing ones, especially when the domain information is not available.
We also take measures to evaluate our model's ability 
to learn concepts at real-world scale
despite the small scale of popular domain adaptation benchmarks.
Thus we introduce a new benchmark dataset 
that resembles ImageNet in the choice of categories and size,
but consists only of images
with the aesthetic of hand-drawn sketches.
Performances on this new benchmark also 
endorse our regularization. 




















 
\section{Related Work}
\label{sec:related}


A broad set of papers have addressed various formulations of DA \citep{bridle1991recnorm,ben2010theory}
dating in the ML and statistics literature 
to early works on covariate shift \citet{shimodaira2000improving}
with antecedents classic econometrics work
on sample selection bias \citep{heckman1977sample,manski1977estimation}. 
Several modern
works address 
principled learning techniques under covariate shift
(when  does not change) \citep{gretton2009covariate}
and under label shift (when  doesn't change) \citep{storkey2009training, zhang2013domain, lipton2018detecting},
and various other assumptions (e.g. bounded divergences between source and target distributions) \citep{mansour2009domain, hu2016robust}.

With the recent success of deep learning methods, 
a number of heuristic domain adaptation methods 
have been proposed that despite lacking theoretical backing
nevertheless confer improvements on a number of benchmarks,
even when traditional assumptions break down (e.g., no shared support).
At a high level these methods comprise two subtypes:
fine-tuning over target domain
\citep{long2016unsupervised,Hoffman_2017,Motiian_2017,Gebru_2017,NIPS2018_7779} 
and coercing domain invariance via adversarial learning (or further extensions) \citep{ganin2016domain,Bousmalis_2017,Tzeng_2017,pmlr-v80-xie18c,pmlr-v80-hoffman18a,NIPS2018_7436,NIPS2018_8075,NIPS2018_8146,NIPS2018_7913,NIPS2018_7380,schoenauer-sebag2018multidomain}. 
While some methods have justified domain-adversarial learning
by appealing to theoretical bounds due to \cite{ben2010theory},
the theory does not in fact guarantee generalization 
(recently shown by \citet{johansson2019support} and  \citet{wu2019domain})
and sometimes guarantees failure.
For a general primer, we refer to several literature reviews \citep{weiss2016survey,csurka2017domain,Wang_2018}. 



In contrast to the typical unsupervised DA setup, 
which requires access to both labeled source data and unlabeled target data,
several recent papers propose deep learning methods 
that confer robustness to a variety of natural-seeming distribution shifts (in practice)
without requiring any data (even unlabeled data) from the target distribution. 
In domain generalization (DG) methods \citep{muandet2013domain} (or sometimes known as ``zero shot domain adaptation'' \citep{kumagai2018zero, niu2015multi,erfani2016robust,li2017domain})  
one possesses domain identifiers for a number of known in-sample domains,
and the goal is to generalize to a new domain.
More recent DG approaches incorporate adversarial (or similar) techniques \citep{ghifary2015domain,wang2016select,motiian2017unified,li2018domain,carlucci2018agnostic}, 
or build ensembles of per-domain models 
that are then fused representations together \citep{bousmalis2016domain,ding2018deep,mancini2018best}. 
Meta-learning techniques have also been explored \citep{li2017learning,NIPS2018_7378}.

More recently, \citet{wang2018learning} demonstrated promising results on a number of benchmarks
without using domain identifiers.
Their method achieves addresses distribution shift
by incorporating a new component 
intended to be especially sensitive to domain-specific signals. 
Our paper extends the setup of \citep{wang2018learning} 
and empirically studies the problem of developing image classifiers robust to a variety of natural shifts
without leveraging any domain information at training or deployment time.  
\section{Method}
\label{sec:method}
We use  
to denote the samples and  
to denote a convolutional neural network, 
where  denotes the output 
of the bottom convolutional layers 
(e.g., the first layer),
and  and  are parameters to be learned. 
The traditional training process addresses the optimization problem

where  denotes the loss function, commonly cross-entropy loss in classification problems. 

Following the standard set-up of a convolutional layer, 
 is a tensor of  parameters, 
where  denotes the number of convolutional channels, 
and  is the size of the convolutional kernel. 
Therefore, for the \textsuperscript{th} sample, 
 is a representation of 
 of the dimension , 
where  (or ) is a function of the image dimension and  (or ). \footnote{The exact function depends on padding size and stride size, and is irrelevant to the discussion of this paper.}

\subsection{Patch-wise Adversarial Regularization}
We first introduce a new classifier,  
that takes the input of a -length vector and predicts the label. 
Thus,  can be applied onto 
the representation  and yield  predictions. 
Therefore, each of the  predictions 
can be seen as a prediction made only by considering a small image patch 
corresponding to each of the receptive fields in . 
If any of the image patches are predictive and 
summarizes the predictive representation well, 
 can be trained to achieve a high prediction accuracy. 

On the other hand, if  summarizes 
the patch-wise predictive representation well, 
higher layers () can  
directly utilize these representation for prediction 
and thus may not be required to learn a global concept. 
Our intuition is that by regularizing  
such that each fiber 
(i.e., representation at the same location from every channel)
in the activation tensor
should not be individually predictive of the label,
we can prevent our model from relying on local patterns
and instead force it to learn a pattern that can only be revealed
by aggregating information across multiple receptive fields. 


As a result, in addition to the standard optimization problem (Eq.~\ref{eq:train}), 
we also optimize the following term:

where the minimization consists of training  
to predict the label based on the local features (at each spatial location)
while the maximization consists of training  
to shift focus away from local predictive representations. 


We hypothesize that by jointly solving 
these two optimization problems 
(Eq.~\ref{eq:train} and Eq.~\ref{eq:minimax}), 
we can train a model that can predict the label well 
without relying too strongly on local patterns. 
The optimization can be reformulated 
into the following two problems:

where  is a tuning hyperparameter. 
We divide the loss by  to keep the two terms at a same scale. 

Our method can be implemented efficiently as follows: 
In practice, we consider  as a fully-connected layer. 
 consists of a  weight matrix and a -length bias vector, 
where  is the number of classes. 
The  forward operations as fully-connected networks
can be efficiently implemented as a  convolutional operation 
with  input channels and  output channels operating on the  representation. 

Note that although the input has  vectors,
 only has one set of parameters
that is used for all these vectors, 
in contrast to building a set of parameter 
for every receptive field of the  dimension. 
Using only one set of parameters can not only help 
to reduce the computational load and parameter space, 
but also help to identify the predictive local patterns well 
because the predictive local pattern does not necessarily appear
at the same position across the images. 
Our idea of our method is illustrated in Figure~\ref{fig:model}. 





\subsection{Other Extensions and Training Heuristics}
There can be many simple extensions to the basic PAR setting we discussed above. 
Here we introduce three extensions that we will experiment with 
later in the experiment section.  

\textbf{More Powerful Pattern Classifier:} 
We explore the space of discriminator architectures,
replacing the single-layer network  
with a more powerful network architecture, e.g. a multilayer perceptron (MLP). 
In this paper, we consider three-layer MLPs with ReLU activation functions. 
We name this variant as PAR\textsubscript{M}.

\textbf{Broader Local Pattern:} 
We can also extend the  convolution operation 
to enlarge the concept of ``local''. 
In this paper, we experiment with a  convolution operation, 
thus the number of parameters in  is increased. 
We refer to this variant as PAR\textsubscript{B}.

\textbf{Higher Level of Local Concept:} 
Further, we can also build the regularization upon higher convolutional layers. 
Building the regularization on higher layers 
is related to enlarging the patch of image,
but also considering higher level of abstractions. 
In this paper, we experiment the regularization on the second layer.
We refer this method as  PAR\textsubscript{H}. 

\textbf{Training Heuristics:}
Finally, we introduce the training heuristic 
that plays an important role in our regularization technique, 
especially in modern architectures such as AlexNet or ResNet. 
The training heuristic is simple: 
we first train the model conventionally 
until convergence (or after a certain number of epochs), 
then train the model with our regularization. 
In other words, we can also directly work on pretrained models 
and continue to fine-tune the parameters with our regularization.  
\section{Experiments}
\label{sec:exp}
In this section, we test PAR over a variety of settings, we first test with perturbed MNIST under the domain generalization setting, and then test with perturbed CIFAR10 under domain adaptation setting. 
Further, we test on more challenging data sets, with PACS data under domain generalization setting and our newly proposed ImageNet-Sketch data set. 
We compare with previous state-of-the-art when available, or with the most popular benchmarks such as DANN~\citep{ganin2016domain}, InfoDrop~\citep{achille2018information}, and HEX~\citep{wang2018learning} on synthetic experiments.\footnote{Clean demonstration of the implementation can be found at: \href{https://github.com/HaohanWang/PAR}{https://github.com/HaohanWang/PAR}}\textsuperscript{,}\footnote{Source code for replication can be found at : \href{https://github.com/HaohanWang/PAR_experiments}{https://github.com/HaohanWang/PAR\_experiments}}

\subsection{MNIST with Perturbation}
We follow the set-up of 
\citet{wang2018learning} 
in experimenting with MNIST data set 
with different superficial patterns.
There are three 
different 
superficial patterns 
(radial kernel, random kernel, and original image). 
The training/validation samples 
are attached with two of these patterns, 
while the testing samples are attached with the remaining one. 
As in \citet{wang2018learning}, training/validation samples 
are attached with patterns following two strategies: 
1) \textit{independently}: the pattern is independent of the digit,
and 2) \textit{dependently}:
images of digit 0-4 have one pattern 
while images of digit 5-9 have the other pattern. 

\begin{figure}
    \centering
    \includegraphics[width=1.0\textwidth]{figs/mnist.pdf}
    \caption{Prediction accuracy with standard deviation for MNIST with patterns. Notations: V: vanilla baseline, E: HEX, D: DANN, I: InfoDrop, P: PAR, B: PAR\textsubscript{B}, M: PAR\textsubscript{M}, H: PAR\textsubscript{H}}
    \label{fig:mnist}
\end{figure}

We use the same model architecture and learning rate as in \citet{wang2018learning}. 
The extra hyperparameter  is set as 1 as the most straightforward choice. Methods in \citet{wang2018learning} are trained for 100 epochs, 
so we train the model for 50 epochs as pretraining 
and 50 epochs with our regularization. 
The results are shown in Figure~\ref{fig:mnist}. 
In addition to the direct message 
that our proposed method outperforms competing ones in most cases, 
it is worth mentioning that the proposed methods 
behave differently in the ``dependent'' settings. 
For example,
PAR\textsubscript{M} performs the best in the ``original'' and ``radial'' settings, 
but almost the worst among proposed methods in the ``random'' setting,
which may indicate that the pattern attached by ``random'' kernel 
can be more easily detected and removed by PAR\textsubscript{M} 
during training (Notice that the name of the setting (``original'', ``radial'' or ``random'') 
indicates the pattern attached to testing images, 
and the training samples are attached with the other two patterns).
More information about hyperparameter choice is in Appendix~\ref{sec:mnist:hyper}. 

\subsection{CIFAR with Perturbation}

We continue to experiment on CIFAR10 data set by modifying the color and texture of test dataset with four different schemas:
1) greyscale; 2) negative color; 3) random kernel; 4) radial kernel. Some examples of the perturbed data are shown in Appendix~\ref{sec:cifar10data}. In this experiment, we use ResNet-50 as our base classifier, which has a rough 92\% prediction accuracy on original CIFAR10 test data set. 


As for PAR, 
we first train the base classifier 
for 250 epochs 
and then train with the adversarial loss for another 150 epochs. As for the competing models, we also train for 400 epochs with carefully selected hyperparameters.
The overall performances 
are shown in Table~\ref{tab:cifar10}. 
In general, PAR and its variants achieve the best performances 
on all four test data sets, 
even when DANN has an unfair advantage over others by seeing unlabelled testing data during training. 
To be specific, PAR achieves 
the best performances on the greyscale and radial kernel settings;
PAR\textsubscript{M} is the best on the negative color and random kernel settings.
One may argue that the numeric improvements are not significant and PAR may only affect the model marginally, 
but a closer look at the training process of the methods indicates that our regularization of local patterns benefits the robustness significantly while minimally impacting the original performance. 
More detailed discussions are in Appendix~\ref{sec:cifar10data}.


\begin{table}[t]
\caption{Test accuracy of PAR and variants on Cifar10 datasets with perturbed color and texture.\break \tiny}
\label{tab:cifar10}
\centering
\begin{tabular}{ccccccccc}
\hline
 & ResNet & DANN & InfoDrop & HEX & PAR & PAR\textsubscript{B} & PAR\textsubscript{M} & PAR\textsubscript{H} \\ \hline
Greyscale & 87.7 & 87.3 & 86.4 & 87.6 & \textbf{88.1} & 87.9 & 87.8 & 86.9 \\
NegColor & 62.8 & 64.3 & 57.6 & 62.4 & 66.2 & 65.3 & \textbf{67.6} & 62.7 \\
RandKernel & 43.0 & 33.4 & 41.3 & 42.5 & 47.0 & 40.5 & \textbf{47.5} & 40.8 \\
RadialKernel & 62.4 & 63.3 & 60.3 & 61.9 & \textbf{63.8} & 63.2 & 63.2 & 61.4 \\ \hline
Average & 63.9 & 62.0 & 61.4 & 63.6 & 66.3 & 64.2 & \textbf{66.5} & 62.9 \\ \hline
\end{tabular}
\end{table}



\subsection{PACS}
We test on the PACS data set \citep{li2017deeper}, which consists of collections of images 
over four domains, including photo, art painting, cartoon, and sketch. Many recent methods have been tested on this data set, which offers a convenient way for PAR to be compared with the previous state-of-the-art. 
Following \citet{li2017deeper}, we use AlexNet as baseline and build PAR upon it. We compare with recently reported state-of-the-art on this data set, including DSN \citep{bousmalis2016domain}, LCNN \citep{li2017deeper}, MLDG \citep{li2017learning}, Fusion \citep{mancini2018best}, MetaReg \citep{NIPS2018_7378}, Jigen~\citep{Carlucci_2019_CVPR}, and HEX \citep{wang2018learning}, in addition to the baseline reported in \citep{li2017deeper}. 
We are also aware that methods that explicitly use domain knowledge \citep[\textit{e.g.},][]{lee2018simple} may be helpful, but we do not directly compare with them numerically, as the methods deviate from the central theme of this paper. 

\begin{table}[h]
\caption{Prediction accuracy of PAR and variants on PACS data set in comparison with the previously reported state-of-the-art results. Bold numbers indicate the best performance (three sets, one for each scenario). 
We use  to denote the methods that use the training setting in \citep{Carlucci_2019_CVPR} (\textit{e.g.}, extra data augmentation, different train-test split, and different learning rate scheduling).  
Notably, PAR\textsubscript{H} achieves the best performance in sketch testing case even in comparison to all other methods without data augmentation.}
\label{tab:pacs}
\centering 
\begin{tabular}{cccccccc}
\hline
 & Art & Cartoon & Photo & Sketch & Average & Forgoing Domaim ID & Data Aug. \\ \hline
AlexNet & 63.3 & 63.1 & 87.7 & 54 & 67.03 & \Checkmark & \\ \hline
DSN & 61.1 & 66.5 & 83.2 & 58.5 & 67.33 &  &\\
L-CNN & 62.8 & 66.9 & 89.5 & 57.5 & 69.18 &  &\\
MLDG & 63.6 & 63.4 & 87.8 & 54.9 & 67.43 &  &\\
Fusion & 64.1 & 66.8 & 90.2 & \textbf{60.1} & 70.30 &  &\\
MetaReg & \textbf{69.8} & \textbf{70.4} & \textbf{91.1} & 59.2 & \textbf{72.63} & & \\ \hline
HEX & 66.8 & 69.7 & 87.9 & 56.3 & 70.18 & \Checkmark &\\
PAR & \textbf{66.9} & 67.1 & 88.6 & 62.6 & 71.30 & \Checkmark &\\
PAR\textsubscript{B} & 66.3 & 67.8 & 87.2 & 61.8 & 70.78 & \Checkmark &\\
PAR\textsubscript{M} & 65.7 & 68.1 & 88.9 & 61.7 & 71.10 & \Checkmark &\\
PAR\textsubscript{H} & 66.3 & \textbf{68.3} & \textbf{89.6} & \textbf{64.1} & \textbf{72.08} & \Checkmark &\\ \hline
Jigen & 67.6 & \textbf{71.7} & 89.0 & \textbf{65.1} & 73.38 & \Checkmark &\Checkmark\\
PAR & 68.0 & 71.6 & \textbf{90.8} & 61.8 & 73.05 & \Checkmark &\Checkmark\\
PAR\textsubscript{B} & 67.6 & 70.7 & 90.1 & 62.0 & 72.59 & \Checkmark &\Checkmark\\
PAR\textsubscript{M} & \textbf{68.7} & 71.5 & 90.5 & 62.6 & 73.33 & \Checkmark & \Checkmark \\
PAR\textsubscript{H} & \textbf{68.7} & 70.5 & 90.4 & 64.6 & \textit{\textbf{73.54}} & \Checkmark & \Checkmark \\ \hline
\end{tabular}
\end{table}

Following the training heuristics we introduced, 
we continue with trained AlexNet weights\footnote{\href{https://www.cs.toronto.edu/~guerzhoy/tf_alexnet/}{https://www.cs.toronto.edu/\textasciitilde guerzhoy/tf\_alexnet/}} 
and fine-tune on training domain data of PACS for 100 epochs. 
We notice that once our regularization is plugged in, 
we can outperform the baseline AlexNet with a 2\% improvement. 
The results are reported in Table~\ref{tab:pacs}, 
where we separate the results of techniques 
relying on domain identifications and techniques free of domain identifications. 

We also report the results based on the training schedule used by~\citep{Carlucci_2019_CVPR} as shown in the bottom part of Table~\ref{tab:pacs}. Note that \citep{Carlucci_2019_CVPR} used the random training-test split that are different from the official split used by the other baselines. In addition, they used another data augmentation technique to convert image patch to grayscale which could benefit the adaptation to Sketch domain.


While our methods are in general competitive, 
it is worth mentioning that our methods improve upon previous methods 
with a relatively large margin when Sketch is the testing domain. 
The improvement on Sketch is notable 
because Sketch is the only colorless domain 
out of the four domains in PACS. 
Therefore, when tested with the other three domains, 
a model may learn to exploit the color information, 
which is usually local, to predict, but when tested with Sketch domain, 
the model has to learn colorless concepts to make good predictions. 

\begin{figure}
\centering 
\begin{subfigure}{.1\textwidth}
  \centering
  \includegraphics[width=0.95\linewidth]{sketches/sketch_1.JPEG}
  \includegraphics[width=0.95\linewidth]{sketches/sketch_2.JPEG}
  \includegraphics[width=0.95\linewidth]{sketches/sketch_3.JPEG}
  \includegraphics[width=0.95\linewidth]{sketches/sketch_4.JPEG}
  \caption{}
\end{subfigure}
\begin{subfigure}{.1\textwidth}
  \centering
  \includegraphics[width=0.95\linewidth]{sketches/sketch_5.JPEG}
  \includegraphics[width=0.95\linewidth]{sketches/sketch_6.JPEG}
  \includegraphics[width=0.95\linewidth]{sketches/sketch_7.JPEG}
  \includegraphics[width=0.95\linewidth]{sketches/sketch_8.JPEG}
  \caption{}
\end{subfigure}
\begin{subfigure}{.1\textwidth}
  \centering
  \includegraphics[width=0.95\linewidth]{sketches/sketch_9.JPEG}
  \includegraphics[width=0.95\linewidth]{sketches/sketch_10.JPEG}
  \includegraphics[width=0.95\linewidth]{sketches/sketch_11.JPEG}
  \includegraphics[width=0.95\linewidth]{sketches/sketch_12.JPEG}
  \caption{}
\end{subfigure}
\begin{subfigure}{.1\textwidth}
  \centering
  \includegraphics[width=0.95\linewidth]{sketches/sketch_13.JPEG}
  \includegraphics[width=0.95\linewidth]{sketches/sketch_14.JPEG}
  \includegraphics[width=0.95\linewidth]{sketches/sketch_15.JPEG}
  \includegraphics[width=0.95\linewidth]{sketches/sketch_16.JPEG}
  \caption{}
\end{subfigure}
\begin{subfigure}{.1\textwidth}
  \centering
  \includegraphics[width=0.95\linewidth]{sketches/sketch_17.JPEG}
  \includegraphics[width=0.95\linewidth]{sketches/sketch_18.JPEG}
  \includegraphics[width=0.95\linewidth]{sketches/sketch_19.JPEG}
  \includegraphics[width=0.95\linewidth]{sketches/sketch_20.JPEG}
  \caption{}
\end{subfigure}
\begin{subfigure}{.1\textwidth}
  \centering
  \includegraphics[width=0.95\linewidth]{sketches/sketch_21.JPEG}
  \includegraphics[width=0.95\linewidth]{sketches/sketch_22.JPEG}
  \includegraphics[width=0.95\linewidth]{sketches/sketch_23.JPEG}
  \includegraphics[width=0.95\linewidth]{sketches/sketch_24.JPEG}
  \caption{}
\end{subfigure}
\begin{subfigure}{.1\textwidth}
  \centering
  \includegraphics[width=0.95\linewidth]{sketches/sketch_25.JPEG}
  \includegraphics[width=0.95\linewidth]{sketches/sketch_26.JPEG}
  \includegraphics[width=0.95\linewidth]{sketches/sketch_27.JPEG}
  \includegraphics[width=0.95\linewidth]{sketches/sketch_28.JPEG}
  \caption{}
\end{subfigure}
\begin{subfigure}{.1\textwidth}
  \centering
  \includegraphics[width=0.95\linewidth]{sketches/sketch_29.JPEG}
  \includegraphics[width=0.95\linewidth]{sketches/sketch_30.JPEG}
  \includegraphics[width=0.95\linewidth]{sketches/sketch_31.JPEG}
  \includegraphics[width=0.95\linewidth]{sketches/sketch_32.JPEG}
  \caption{}
\end{subfigure}
\begin{subfigure}{.1\textwidth}
  \centering
  \includegraphics[width=0.95\linewidth]{sketches/sketch_33.JPEG}
  \includegraphics[width=0.95\linewidth]{sketches/sketch_34.JPEG}
  \includegraphics[width=0.95\linewidth]{sketches/sketch_35.JPEG}
  \includegraphics[width=0.95\linewidth]{sketches/sketch_36.JPEG}
  \caption{}
\end{subfigure}

\caption{Sample Images from ImageNet-Sketch. Corresponding classes: (a) magpie (b) box turtle (c) goldfish (d) golden retriever (e) parachute (f) bookshop (g) acoustic guitar (h) racer (i) giant panda}
\label{fig:sketch}
\end{figure}

\subsection{ImageNet-Sketch}
\subsubsection{The ImageNet-Sketch Data}
Inspired by the Sketch data of \citep{li2017deeper} with seven classes, 
and several other Sketch datasets, 
such as the \emph{Sketchy} dataset \citep{sketchy2016} 
with  classes and the \emph{Quick Draw!}
dataset \citep{quickdraw2018} with  classes, 
and motivated by absence of a large-scale sketch dataset 
fitting the shape and size of popular image classification benchmarks,
we construct the ImageNet-Sketch data set for evaluating 
the out-of-domain classification performance of vision models trained on ImageNet. 

Compatible with standard ImageNet validation data set
for the classification task \citep{Deng09imagenet},
our ImageNet-Sketch data set consists of 50000 images, 
50 images for each of the 1000 ImageNet classes. 
We construct the data set with Google Image queries 
``sketch of \underline{\hspace{0.8cm}}'',
where \underline{\hspace{0.8cm}} is the standard class name. 
We only search within the ``black and white'' color scheme.
We initially query  images for every class, 
and then manually clean the pulled images 
by deleting the irrelevant images and images 
that are for similar but different classes. 
For some classes, there are less than  images after manually cleaning, 
and then we augment the data set by flipping and rotating the images. 

We expect ImageNet-Sketch to serve as a unique ImageNet-scale 
out-of-domain evaluation dataset for image classification. 
Also, notably, different from perturbed ImageNet validation sets \citep{geirhos2018imagenettrained,hendrycks2018benchmarking}, 
the images of ImageNet-Sketch are collected independently 
from the original validation images. 
The independent collection procedure is more similar to \citep{recht2019imagenet}, 
who collected a new set of standard colorful ImageNet validation images. 
However, while their goal was to assess overfitting to the benchmark validation sets,
and thus they tried replicate the ImageNet collection procedure exactly,
our goald is to collect out-of-domain black-and-white sketch images 
with the goal of testing a model's ability 
to extrapolate out of domain.\footnote{The ImageNet-Sketch data can be found at: \href{https://github.com/HaohanWang/ImageNet-Sketch}{https://github.com/HaohanWang/ImageNet-Sketch}} 
Sample images are shown in Figure~\ref{fig:sketch}.


\subsubsection{Experiment Results}
We use AlexNet as the baseline and test whether our method 
can help improve the out-of-domain prediction. 
We start with 
ImageNet pretrained AlexNet 
and continue to use PAR to tune AlexNet for another five epochs
on the original ImageNet training dataset. 
The results are reported in Table~\ref{tab:sketch}.

\begin{table}[]
\caption{Testing accuracy of competing methods on the ImageNet-Sketch data. 
The bottom half denotes the method that has extra advantages: 
 denotes the method that has access to unlabelled target domain data, 
and  denotes the method that use extra data augmentation. 
\break \tiny}
\label{tab:sketch}
\centering
\begin{tabular}{cccccccc}
\hline
 & AlexNet & InfoDrop & HEX & PAR & PAR\textsubscript{B} & PAR\textsubscript{M} & PAR\textsubscript{H} \\ \hline
Top 1 & 0.1204 & 0.1224 & 0.1292 & \textbf{0.1306} & 0.1273 & 0.1287 & 0.1266 \\
Top 5 & 0.2480 & 0.2560 & \textbf{0.2654} & 0.2627 & 0.2575 & 0.2603 & 0.2544 \\ \hline
 &  & DANN & JigGen & PAR & PAR\textsubscript{B} & PAR\textsubscript{M} & PAR\textsubscript{H} \\ \hline
Top 1 &  & 0.1360 & 0.1469 & 0.1494 & 0.1494 & \textbf{0.1501} & 0.1499 \\
Top 5 &  & 0.2712 & 0.2898 & 0.2949 & 0.2945 & \textbf{0.2957} & 0.2954 \\ \hline
\end{tabular}
\end{table}

We are particularly interested in how PAR improves upon AlexNet, so we further investigate the top-1 prediction results. 
Although the numeric results in Table~\ref{tab:sketch} seem to show that PAR only improves the upon AlexNet by predicting a few more examples correctly, we notice that these models share 5025 correct predictions, while AlexNet predicts another 1098 images correctly and PAR predicts a different set of 1617 images correctly. 



We first investigate the examples that are correctly predicted by the original AlexNet, but wrongly predicted by PAR. 
We notice some examples that help verify the performance of PAR. 
For examples, PAR incorrectly predicts three instances of ``keyboard'' as ``crossword puzzle,'' while AlexNet predicts these samples correctly.  
It is notable that two of these samples are ``keyboards with missing keys'' and hence look similar to a ``crossword puzzle.'' 

\begin{table}[]
\caption{Some examples that are predicted correctly with our method but wrongly with the original AlexNet because the original model seems to focus on the local patterns.\break \tiny}
\label{tab:sketch:results}
\centering
\begin{tabular}{ccccc}
\hline
 & \multicolumn{2}{c}{AlexNet-PAR} & \multicolumn{2}{c}{AlexNet} \\
 & prediction & confidence & prediction & confidence \\ \hline
\raisebox{-.5\height}{\includegraphics[width=0.09\textwidth]{resultSketches/resultSketch_6.JPEG}} & stethoscope & 0.6608 & hook & 0.3903 \\
\raisebox{-.5\height}{\includegraphics[width=0.09\textwidth]{resultSketches/resultSketch_12.JPEG}} & tricycle & 0.9260 & safety pin & 0.5143 \\
\raisebox{-.5\height}{\includegraphics[width=0.09\textwidth]{resultSketches/resultSketch_19.JPEG}} & Afghan hound & 0.8945 & swab (mop) & 0.7379 \\
\raisebox{-.5\height}{\includegraphics[width=0.09\textwidth]{resultSketches/resultSketch_23.JPEG}} & red wine & 0.5999 & goblet & 0.7427 \\ \hline
\end{tabular}
\end{table}

We also investigate the examples that are correctly predicted by PAR, but wrongly predicted by the original AlexNet. 
Interestingly, we notice several samples that are wrongly predicted by AlexNet because the model may only focus on the local patterns. 
Some of the most interesting examples are reported in Table~\ref{tab:sketch:results}: 
The first example is a stethoscope, PAR predicts it correctly with 0.66 confidence, while AlexNet predicts it to be a hook. 
We conjecture the reason to be that AlexNet tends to only focus on the curvature which resembles a hook. 
The second example tells a similar story, PAR predicts tricycle correctly with 0.92 confidence, but AlexNet predicts it as a safety pin with 0.51 confidence. 
We believe this is because part of the image (likely the seat-supporting frame) resembles the structure of a safety pin. 
For the third example, PAR correctly predicts it to be an Afghan hound with 0.89 confidence, but AlexNet predicts it as a mop with 0.73 confidence. 
This is likely because the fur of the hound is similar to the head of a mop. 
For the last example, PAR correctly predicts the object to be red wine with 0.59 confidence, but AlexNet predicts it to be a goblet with 0.74 confidence. 
This is likely because part of the image is indeed part of a goblet, but PAR may learn to make predictions based on the global concept considering the bottle, the liquid, and part of the goblet together. 
Table~\ref{tab:sketch:results} only highlights a few examples, and more examples are shown in Appendix~\ref{sec:appendix:sketch}.  
\section{Conclusion}

\label{sec:con}
In this paper, we introduced  \emph{patch-wise adversarial regularization},
a technique that regularizes models,
encouraging them to learn \emph{global concepts} for classifying objects 
by penalizing the model's ability to make predictions based on representations of  \emph{local} patches. 
We extended our basic set-up with several different variants and conducted extensive experiments, evaluating these methods with several datasets for domain adaptation and domain generalization tasks. 
The experimental results favored our methods, especially when domain information is unknown to the methods.
In addition to the superior performances we achieved through these experiments, we expected to further challenge our method at real-world scale. 
Therefore, we also constructed a dataset that matches the ImageNet classification validation set in classes and scales but contains only sketch-alike images. 
Our new ImageNet-Sketch data set can serve as new territory for evaluating models' ability to generalize to out-of-domain images at an unprecedented scale.  

\label{sec:diss}
While our method often confers benefits on out-of-domain data, 
we note that it may not help (or can even hurt)
in-domain accuracy
when local patterns are truly predictive of the labels. 
However, we argue that the local patterns,
while predictive in-sample,
may be less reliable out-of-domain as compared to larger-scale patterns, which motivates this paper.
For the three variations we introduced, 
our experiments indicate 
that different variants are applicable to different scenarios. 
We recommend that users decide which variant to use 
given their understanding of the problem
and hope in future work, 
to develop clear principles for guiding these choices.
While we did not give a clear choice of which PAR to use, 
we note that none of the variants of PAR
outperform the vanilla PAR consistently. 
However, the vanilla PAR outperforms most comparable baselines in the vast majority of our experiments.
%
 


\subsubsection*{Acknowledgments}
Haohan Wang is supported by NIH R01GM114311, NIH P30DA035778, and NSF IIS1617583. Any opinions, findings and conclusions or recommendations expressed in this material are those of the author(s) and do not necessarily reflect the views of the National Institutes of Health or the National Science Foundation.
Zachary Lipton thanks the Center for Machine Learning and Health, a joint venture of Carnegie Mellon University, UPMC, and the University of Pittsburgh for supporting our collaboration with Abridge AI to develop robust models for machine learning in healthcare. He is also grateful to Salesforce Research, Facebook Research, and Amazon AI for faculty awards supporting his lab's research on robust deep learning under distribution shift.  

\bibliographystyle{abbrvnat}
\bibliography{ref}  

\clearpage
\appendix

\section{Other Hyperparameter Choices for MNIST experiment}
\label{sec:mnist:hyper}
We also experimented with the parameter choices of the method in the MNIST experiment. 
We varied the  in  in PAR and reported the performance to guide further usage of the method. 

\begin{figure}
    \centering
    \includegraphics[width=1.0\textwidth]{figs/mnistParameter.pdf}
    \caption{Prediction accuracy with standard deviation for MNIST with superficial statistics perturbation data set. Notations: V: vanilla baseline, A: PAR with , B: PAR with , C: PAR with , D: PAR with , E: PAR with }
    \label{fig:mnistParameter}
\end{figure}

As we can see from Figure~\ref{fig:mnistParameter}, PAR seems to prefer the cases when  is relatively smaller, although what we reported in the main manuscript for the MNIST experiment is  as the most straightforward choice, to demonstrate the method's strength. 

Later in other experiments, especially the ImageNet-Sketch experiment, we notice that  is too strong (unless the learning rate is set to be much smaller) for the method to work. We observe that a too-strong  usually immediate deteriorates the performance during first epoches of training. Therefore, in practice, we recommend the users to set the  (or learning rate) to be smaller if the users observe that our method deteriorates the training performance. 


\newpage 
\section{Cifar10 discussion}
\label{sec:cifar10data}
\begin{figure}[h!]
  \centering
    \begin{subfigure}{.28\textwidth}
      \centering
      \includegraphics[width=0.95\linewidth]{figs/cifar/RadialKernel.pdf}
      \caption{Radial mask}
    \end{subfigure}
    \begin{subfigure}{.28\textwidth}
      \centering
      \includegraphics[width=0.95\linewidth]{figs/cifar/RandomMask.pdf}
      \caption{Random mask}
      \label{process}
    \end{subfigure}
\caption{Fourier filtering kernel}
\label{fig:kernel}
\end{figure}
\vspace{-0.5cm}

\begin{figure}[h!]
  \centering
    \begin{subfigure}{.19\textwidth}
      \centering
      \includegraphics[trim=150 0 150 0, clip, width=0.95\linewidth]{figs/cifar/cifardata0.pdf}
      \caption{Original}
    \end{subfigure}
    \begin{subfigure}{.19\textwidth}
      \centering
      \includegraphics[trim=150 0 150 0, clip, width=0.95\linewidth]{figs/cifar/cifardata1.pdf}
      \caption{Negative}
    \end{subfigure}
    \begin{subfigure}{.19\textwidth}
      \centering
      \includegraphics[trim=150 0 150 0, clip, width=0.95\linewidth]{figs/cifar/cifardata2.pdf}
      \caption{Greyscale}
    \end{subfigure}
    \begin{subfigure}{.19\textwidth}
      \centering
      \includegraphics[trim=150 0 150 0, clip, width=0.95\linewidth]{figs/cifar/cifardata3.pdf}
      \caption{RandomKernel}
    \end{subfigure}
    \begin{subfigure}{.19\textwidth}
      \centering
      \includegraphics[trim=150 0 150 0, clip, width=0.95\linewidth]{figs/cifar/cifardata4.pdf}
      \caption{RadialKernel}
    \end{subfigure}
\caption{Examples of Cifar10 images with perturbed color and texture.}
\label{fig:cifardata}
\end{figure}

\clearpage

\begin{figure}[h!]
  \centering
    \begin{subfigure}{.49\textwidth}
      \centering
      \includegraphics[width=0.95\linewidth]{requestedByReviewer/Adv_lv0_side.pdf}
      \caption{Layer 1}
      \label{process}
    \end{subfigure}
    \begin{subfigure}{.49\textwidth}
      \centering
      \includegraphics[width=0.95\linewidth]{requestedByReviewer/Adv_lv1_side.pdf}
      \caption{Layer 2}
    \end{subfigure}
    \begin{subfigure}{.49\textwidth}
      \centering
      \includegraphics[width=0.95\linewidth]{requestedByReviewer/Adv_lv2_side.pdf}
      \caption{Layer 3}
    \end{subfigure}
    \begin{subfigure}{.49\textwidth}
      \centering
      \includegraphics[width=0.95\linewidth]{requestedByReviewer/Adv_lv3_side.pdf}
      \caption{Layer 4}
    \end{subfigure}
\caption{Prediction accuracy of patch-wise classifier. The regularization is introduced at Epoch 250. we ran experiments
to validate the patch-wise classifier.
Without (PAR) regularization, the patch-wise classifier can achieve roughly 20\% accuracy on in-domain test data 
((a), orange, before epoch 250).
It achieves 12\% accuracy on texture-altered out-of-domain data ((a), magenta and green, before epoch 250)
and 5\% accuracy color-altered out-of-domain data ((a), maroon, before epoch 250). 
With PAR, the patch-wise classifier achieves 15\% in-domain prediction accuracy (5\% drop) 
((a), orange, after epoch 250), 
and 10\% on texture-altered out-of-domain data 
((a), magenta and green, after epoch 250) 
and 8\% on color-altered out-of-domain data ((a), maroon, after epoch 250). }
\label{fig:reviewer}
\end{figure}

\clearpage

\begin{figure}[h!]
  \centering
    \begin{subfigure}{.49\textwidth}
      \centering
      \includegraphics[width=0.95\linewidth]{figs/cifar/Adv_lv0.pdf}
      \caption{level 1}
      \label{process}
    \end{subfigure}
    \begin{subfigure}{.49\textwidth}
      \centering
      \includegraphics[width=0.95\linewidth]{figs/cifar/Adv_lv1.pdf}
      \caption{level 2}
    \end{subfigure}
    \begin{subfigure}{.49\textwidth}
      \centering
      \includegraphics[width=0.95\linewidth]{figs/cifar/Adv_lv2.pdf}
      \caption{level 3}
    \end{subfigure}
    \begin{subfigure}{.49\textwidth}
      \centering
      \includegraphics[width=0.95\linewidth]{figs/cifar/Adv_lv3.pdf}
      \caption{level 4}
    \end{subfigure}
\caption{The solid lines represent the test accuracy of PAR with different levels of local patterns during the training process. For a better comparison, we use the dashed lines to represent the test accuracy at 250 epoch when the adversarial training is firstly added. The default PAR is shown in (a), we can see a small jitter after 250 epoch when the model is coerced to forget the information of local patterns. Then the performances on the perturbed dataset start to increase while the performance on the original dataset is not greatly impacted. In addition, when higher level of local patterns are used, little improvement can be observed, except for using level 2 on the negative color.}
\label{fig:higher}
\end{figure}


\begin{figure}[!h]
    \centering
    \includegraphics[width=0.9\linewidth]{figs/cifar/breadth.pdf}
    \caption{Evaluation with different sizes of convolutional filters. Note that all the local pattern classifiers contain one layer and 10 channels but different filter sizes. In general, the performances with different filter sizes on the test datasets are very similar except for .}
    \label{fig:breath}
\end{figure}

\clearpage

\begin{figure}[h!]
  \centering
    \begin{subfigure}{.49\textwidth}
      \centering
      \includegraphics[width=0.95\linewidth]{figs/cifar/alf_m1_dk0_1.pdf}
      \caption{}
      \label{process}
    \end{subfigure}
    \begin{subfigure}{.49\textwidth}
      \centering
      \includegraphics[width=0.95\linewidth]{figs/cifar/alf_m1_dk0_2.pdf}
      \caption{}
    \end{subfigure}
    \begin{subfigure}{.49\textwidth}
      \centering
      \includegraphics[width=0.95\linewidth]{figs/cifar/alf_m1_dk0_5.pdf}
      \caption{}
    \end{subfigure}
    \begin{subfigure}{.49\textwidth}
      \centering
      \includegraphics[width=0.95\linewidth]{figs/cifar/alf_m1_dk1_0.pdf}
      \caption{}
    \end{subfigure}
\caption{Evaluation on adversarial training with multiple levels and the different decays. Note that all the layers are used for extracting local concepts with a decay, i.e. adding weights , ,  and  to the adversarial losses of four layers. Smaller decay (larger ) leads to unstable performances.}
\label{fig:multilayer}
\end{figure}

\newpage 

\section{More results of ImageNet-Sketch}
\label{sec:appendix:sketch}

We conducted a detailed analysis of ImageNet-Sketch results with the following rules: 
\begin{itemize}
    \item The samples are correctly predicted by one model, but wrongly predicted by the other. 
    \item When a model makes wrong predictions, the samples in the class tend to be predicted into a same another class. Therefore, we can exclude some random prediction errors. 
    \item For class A and B, if one model tends to predict samples in class A into class B, and the other model has the reverse tendency, we investigate neither of these classes, because the different prediction results may only be due to the similarity of these two classes. 
\end{itemize}

Table~\ref{tab:appendix:sketch} shows more samples that are correctly predicted by PAR but wrongly predicted by the original AlexNet, because (as we conjecture) the original AlexNet focuses on local patterns. 

\begin{table}[]
\small
\caption{More prediction comparisions between AlexNet-PAR and AlexNet}
\label{tab:appendix:sketch}
\begin{tabular}{ccccc|ccccc}
\hline
 & \multicolumn{2}{c}{AlexNet-PAR} & \multicolumn{2}{c}{AlexNet} &  & \multicolumn{2}{c}{AlexNet-PAR} & \multicolumn{2}{c}{AlexNet} \\
Image & Prediction & Conf. & Prediction & Conf. & Image & Prediction & Conf. & Prediction & Conf. \\ \hline
\raisebox{-.5\height}{\includegraphics[width=0.05\textwidth]{appendixImages/appendix_1.JPEG}} & Afghan hound & 0.89 & swab (mop) & 0.74 & \raisebox{-.5\height}{\includegraphics[width=0.05\textwidth]{appendixImages/appendix_21.JPEG}} & sunglass & 0.42 & strainer & 0.27 \\
\raisebox{-.5\height}{\includegraphics[width=0.05\textwidth]{appendixImages/appendix_2.JPEG}} & Afghan hound & 0.92 & swab (mop) & 0.82 & \raisebox{-.5\height}{\includegraphics[width=0.05\textwidth]{appendixImages/appendix_22.JPEG}} & sunglass & 0.31 & strainer & 0.19 \\
\raisebox{-.5\height}{\includegraphics[width=0.05\textwidth]{appendixImages/appendix_3.JPEG}} & Afghan hound & 0.80 & swab (mop) & 0.20 & \raisebox{-.5\height}{\includegraphics[width=0.05\textwidth]{appendixImages/appendix_23.JPEG}} & sunglass & 0.38 & strainer & 0.32 \\
\raisebox{-.5\height}{\includegraphics[width=0.05\textwidth]{appendixImages/appendix_4.JPEG}} & bull mastiff & 0.42 & shower cap & 0.23 & \raisebox{-.5\height}{\includegraphics[width=0.05\textwidth]{appendixImages/appendix_24.JPEG}} & totem pole & 0.30 & envelope & 0.39 \\
\raisebox{-.5\height}{\includegraphics[width=0.05\textwidth]{appendixImages/appendix_5.JPEG}} & bull mastiff & 0.33 & shower cap & 0.37 & \raisebox{-.5\height}{\includegraphics[width=0.05\textwidth]{appendixImages/appendix_25.JPEG}} & totem pole & 0.43 & envelope & 0.27 \\
\raisebox{-.5\height}{\includegraphics[width=0.05\textwidth]{appendixImages/appendix_6.JPEG}} & bull mastiff & 0.57 & shower cap & 0.77 & \raisebox{-.5\height}{\includegraphics[width=0.05\textwidth]{appendixImages/appendix_26.JPEG}} & totem pole & 0.50 & envelope & 0.40 \\
\raisebox{-.5\height}{\includegraphics[width=0.05\textwidth]{appendixImages/appendix_7.JPEG}} & ashcan & 0.17 & safety pin & 0.41 & \raisebox{-.5\height}{\includegraphics[width=0.05\textwidth]{appendixImages/appendix_27.JPEG}} & totem pole & 0.45 & envelope & 0.39 \\
\raisebox{-.5\height}{\includegraphics[width=0.05\textwidth]{appendixImages/appendix_8.JPEG}} & ashcan & 0.38 & safety pin & 0.26 & \raisebox{-.5\height}{\includegraphics[width=0.05\textwidth]{appendixImages/appendix_28.JPEG}} & tricycle & 0.17 & safety pin & 0.42 \\
\raisebox{-.5\height}{\includegraphics[width=0.05\textwidth]{appendixImages/appendix_9.JPEG}} & ashcan & 0.16 & safety pin & 0.53 & \raisebox{-.5\height}{\includegraphics[width=0.05\textwidth]{appendixImages/appendix_29.JPEG}} & tricycle & 0.66 & safety pin & 0.49 \\
\raisebox{-.5\height}{\includegraphics[width=0.05\textwidth]{appendixImages/appendix_10.JPEG}} & car mirror & 0.42 & buckle & 0.89 & \raisebox{-.5\height}{\includegraphics[width=0.05\textwidth]{appendixImages/appendix_30.JPEG}} & tricycle & 0.07 & safety pin & 0.12 \\
\raisebox{-.5\height}{\includegraphics[width=0.05\textwidth]{appendixImages/appendix_11.JPEG}} & car mirror & 0.57 & buckle & 0.43 & \raisebox{-.5\height}{\includegraphics[width=0.05\textwidth]{appendixImages/appendix_31.JPEG}} & tricycle & 0.93 & safety pin & 0.51 \\
\raisebox{-.5\height}{\includegraphics[width=0.05\textwidth]{appendixImages/appendix_12.JPEG}} & car mirror & 0.88 & buckle & 0.63 & \raisebox{-.5\height}{\includegraphics[width=0.05\textwidth]{appendixImages/appendix_32.JPEG}} & whiskey jug & 0.20 & perfume & 0.14 \\
\raisebox{-.5\height}{\includegraphics[width=0.05\textwidth]{appendixImages/appendix_13.JPEG}} & stethoscope & 0.46 & hook & 0.37 & \raisebox{-.5\height}{\includegraphics[width=0.05\textwidth]{appendixImages/appendix_33.JPEG}} & whiskey jug & 0.65 & perfume & 0.37 \\
\raisebox{-.5\height}{\includegraphics[width=0.05\textwidth]{appendixImages/appendix_14.JPEG}} & stethoscope & 0.58 & hook & 0.55 & \raisebox{-.5\height}{\includegraphics[width=0.05\textwidth]{appendixImages/appendix_34.JPEG}} & whiskey jug & 0.49 & perfume & 0.51 \\
\raisebox{-.5\height}{\includegraphics[width=0.05\textwidth]{appendixImages/appendix_15.JPEG}} & stethoscope & 0.77 & hook & 0.58 & \raisebox{-.5\height}{\includegraphics[width=0.05\textwidth]{appendixImages/appendix_35.JPEG}} & head cabbage & 0.44 & shower cap & 0.31 \\
\raisebox{-.5\height}{\includegraphics[width=0.05\textwidth]{appendixImages/appendix_16.JPEG}} & stethoscope & 0.75 & hook & 0.42 & \raisebox{-.5\height}{\includegraphics[width=0.05\textwidth]{appendixImages/appendix_36.JPEG}} & head cabbage & 0.78 & shower cap & 0.79 \\
\raisebox{-.5\height}{\includegraphics[width=0.05\textwidth]{appendixImages/appendix_17.JPEG}} & stethoscope & 0.46 & hook & 0.59 & \raisebox{-.5\height}{\includegraphics[width=0.05\textwidth]{appendixImages/appendix_37.JPEG}} & head cabbage & 0.34 & shower cap & 0.39 \\
\raisebox{-.5\height}{\includegraphics[width=0.05\textwidth]{appendixImages/appendix_18.JPEG}} & stethoscope & 0.66 & hook & 0.39 & \raisebox{-.5\height}{\includegraphics[width=0.05\textwidth]{appendixImages/appendix_38.JPEG}} & red wine & 0.32 & goblet & 0.84 \\
\raisebox{-.5\height}{\includegraphics[width=0.05\textwidth]{appendixImages/appendix_19.JPEG}} & stethoscope & 0.56 & hook & 0.46 & \raisebox{-.5\height}{\includegraphics[width=0.05\textwidth]{appendixImages/appendix_39.JPEG}} & red wine & 0.60 & goblet & 0.74 \\
\raisebox{-.5\height}{\includegraphics[width=0.05\textwidth]{appendixImages/appendix_20.JPEG}} & stethoscope & 0.56 & hook & 0.49 & \raisebox{-.5\height}{\includegraphics[width=0.05\textwidth]{appendixImages/appendix_40.JPEG}} & red wine & 0.82 & goblet & 0.67 \\ \hline
\end{tabular}
\end{table} 
\end{document}
