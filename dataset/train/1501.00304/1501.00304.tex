\documentclass{article}
\usepackage[margin=1.2in]{geometry}









\usepackage{graphicx}
\usepackage{enumerate}
\usepackage{verbatim}
\usepackage{amsmath}
\usepackage{amsfonts}
\usepackage{amsthm}
\usepackage{url}
\usepackage{xspace}
\usepackage{color}
\usepackage[labelfont=bf,size=small]{caption}
\usepackage[labelfont=normalfont,justification=centering]{subcaption}
\usepackage{algo}
\usepackage{hyperref}
\usepackage{authblk}


\newcounter{dummycount}


\newtheorem{theorem}{Theorem}
\newtheorem{lemma}[theorem]{Lemma}
\newtheorem{corollary}[theorem]{Corollary}
\newtheorem{proposition}[theorem]{Proposition}
\newtheorem{observation}[theorem]{Observation}

\newtheorem{claim}{Claim}

\newenvironment{claimproof}[1]{\noindent \textit{Proof of Claim #1.}}
{
\hfill \medskip
}




\newcommand{\wormhole}[1]
{
\newcounter{#1}
\setcounter{#1}{\value{theorem}}
}


\newenvironment{backInTime}[1]
{
\setcounter{dummycount}{\value{theorem}}
\setcounter{theorem}{\value{#1}}
}
{
\setcounter{theorem}{\value{dummycount}}
}













\title{Contact Representations of Graphs in 3D}






\begin{comment}
\author{Muhammad Jawaherul Alam\inst{1} \and
William Evans\inst{2} \and
Stephen G. Kobourov\inst{1} \and
Sergey~Pupyrev\inst{1} \and
Jackson~Toeniskoetter\inst{1} \and
Torsten~Ueckerdt\inst{3}}

\institute{
Department of Computer Science, University of Arizona, USA
\and
Department of Computer Science, University of British Columbia, Canada
\and
Department of Mathematics, Karlsruhe Institute of Technology, Germany
}


\end{comment}







\author[1]{Md. Jawaherul Alam}
\author[2]{William Evans}
\author[1]{Stephen G. Kobourov}
\author[1]{Sergey Pupyrev}
\author[1]{Jackson Toeniskoetter}
\author[3]{Torsten Ueckerdt}
\affil[1]{Department of Computer Science, University of Arizona, USA}
\affil[2]{Department of Computer Science, University of British Columbia, Canada}
\affil[3]{Department of Mathematics, Karlsruhe Institute of Technology, Germany}













\begin{comment}
\serieslogo{}\volumeinfo {Billy Editor and Bill Editors}{2}{Conference title on which this volume is based on}{1}{1}{1}\EventShortName{}
\DOI{10.4230/LIPIcs.xxx.yyy.p}\end{comment}








\newcommand{\TT}{{\mathcal{T}}}

\newcommand{\R}{\mathbb R}

\newcommand{\df}{\textit}
\newcommand{\WLOG}{w.l.o.g.\xspace}
\newcommand{\LL}{\xspace}
\newcommand{\LLs}{'s\xspace}

\newcommand{\hif}[1]{\hat{#1}}
\newcommand{\lof}[1]{\check{#1}}
\newcommand{\hiv}[1]{\hat{#1}}
\newcommand{\lov}[1]{\check{#1}}

\newcommand{\pperp}{\ensuremath{(\bot{-}\bot)}}
\newcommand{\ppar}{\ensuremath{(\bot{-}||)}}

\newcommand{\Oh}{{\ensuremath{\mathcal{O}}}}

\newcommand{\TODO}[1]{\textcolor{red}{\textbf{#1}}}

\begin{document}

\date{}

\maketitle




\begin{abstract}
We study contact representations of graphs in which vertices are represented by axis-aligned polyhedra in 3D and edges are realized by non-zero area common boundaries between corresponding polyhedra. We show that for every 3-connected planar graph, there exists a simultaneous representation of the graph and its dual with 3D boxes. We give a linear-time algorithm for constructing such a representation. This result extends the existing primal-dual contact representations of planar graphs in 2D using circles and triangles.
While contact graphs in 2D directly correspond to planar graphs, we next study representations of non-planar graphs in 3D. In particular we consider representations of optimal 1-planar graphs. A graph is 1-planar if there exists a drawing in the plane where each edge is crossed at most once, and an optimal -vertex 1-planar graph has the maximum () number of edges.  We describe a linear-time algorithm for representing optimal 1-planar graphs without separating 4-cycles with 3D boxes. However, not every optimal 1-planar graph admits a representation with boxes. Hence, we consider contact representations with the next simplest axis-aligned 3D object, L-shaped polyhedra. We provide a quadratic-time algorithm for representing optimal 1-planar graph with L-shaped polyhedra.
\end{abstract}

\section{Introduction}

Graphs are often used to capture relationships between objects, and graph embedding
techniques allow us to visualize such relationships via traditional node-links diagrams.
There are compelling theoretical and practical reasons to study \df{contact representations}
of graphs, where vertices are geometric objects and edges correspond to pairs of objects
touching in some specified fashion. In practice, 2D contact
representations with rectangles, circles, and polygons of low
complexity are intuitive, as they provide the viewer with the familiar metaphor
of geographical maps. Such representations are preferred in some contexts over the standard
node-link representations for displaying relational information~\cite{BGPV08,GHK10}.
Contact representations of graphs have practical applications in
data visualization~\cite{Shn92a}, cartography~\cite{Rai34}, geography~\cite{Tob04a},
sociology~\cite{HK98},
very-large-scale integration circuit design~\cite{Ull84}, and floor-planning~\cite{MCP02a}.

A large body of work considers representing graphs as contacts of
simple curves or polygons in 2D.
Graphs that can be represented in this way are planar.
In fact, Koebe's 1936 theorem established that \emph{all} planar
graphs can be represented by touching disks~\cite{Koe36}.
Any planar graph also has a contact representation with
triangles~\cite{FMR94,GLP12}. Partial planar -trees and some
series-parallel graphs admit a representation with homothetic
triangles~\cite{BBGDF+07}. Curve contacts~\cite{Hli98}, line-segment
contacts~\cite{FM07a,KM94}, and -shape contacts~\cite{CKU13,KUV13}
have also been used.
In particular, it is known that all planar bipartite graphs can be represented by axis-aligned
segment contacts~\cite{CKU98,FMP91,RT86}.
Furthermore, any planar graph has a representation with -shapes~\cite{FMR94}.
Despite our best efforts, however, many graphs remain stubbornly
non-planar and, for these, such contact representations in 2D are impossible.
Hence, a natural generalization is representing vertices with 3D-polyhedra, such
as cubes and tetrahedra, and edges with shared boundaries. While such contact
representations allow us to visualize non-planar graphs, there is much less known
about contact representations in 3D than in 2D. Contact graphs using 3D objects
have been studied for complete graphs and complete bipartite graphs
using spheres~\cite{BR13,HK01} and cylinders~\cite{Bez05}.


As a first step towards representing non-planar graphs, we consider
\df{primal-dual} contact representations, in which a plane graph (a planar graph with a fixed planar embedding)
and its dual are represented simultaneously. More formally, in such a representation
vertices and faces are represented by some geometric objects so that:

\begin{enumerate}[(i)]
	\item the objects for the vertices are interior-disjoint and induce a contact representation
		for the primal graph;
	\item the objects for the faces are interior-disjoint except for the object for the outer
		face, which contains all the objects for the internal faces, and together they induce a
		contact representation of the dual graph;
	\item the objects for a vertex  and a face  of the primal graph intersect if and only
		if  and  are incident.
\end{enumerate}

Primal-dual representations of planar graphs have been studied in 2D. Specifically,
every -connected plane graph has a primal-dual representation with circles~\cite{And70} and
triangles~\cite{GLP12}; see Fig.~\ref{fig:intro}(a)--(c). Our first result in this paper
is an analogous primal-dual representation using axis-aligned 3D boxes. While it is known that every planar graph has a contact representation with
3D boxes~\cite{Tho88,FF11,BEFHK+12},
 Theorem~\ref{thm:box} strengthens these results; see Fig.~\ref{fig:intro4}.

\wormhole{thm-box}
\begin{theorem}
\label{thm:box}
Every -connected planar graph  admits a proper primal-dual box-contact
representation in 3D and it can be computed in  time.
\end{theorem}

\begin{figure}[t]
\begin{subfigure}[t]{.2\textwidth}
    \centering
    \includegraphics[height=2.5cm]{intro-graph}
    \caption{}
  \end{subfigure}
  \hfill
  \begin{subfigure}[t]{.22\textwidth}
    \centering
    \includegraphics[height=2.75cm]{intro-circle}
    \caption{}
    \label{fig:intro2}
  \end{subfigure}
  \hfill
  \begin{subfigure}[t]{.22\textwidth}
    \centering
    \includegraphics[height=2.75cm]{intro-triangle}
    \caption{}
  \end{subfigure}
  \hfill
  \begin{subfigure}[t]{.3\textwidth}
    \centering
    \includegraphics[height=4cm]{intro-pri-du}
    \caption{}
    \label{fig:intro4}
  \end{subfigure}
    \caption{(a)~A planar graph  and its dual; primal-dual contact representations
     of the graph with (b)~circles and (c)~triangles. (d)~The primal-dual box-contact representation of 
     with dual vertices shown dashed. The outer box (shell) contains all other boxes.}
    \label{fig:intro}
\end{figure}

We would like to mention two differences with the existing primal-dual
 representations~\cite{And70,GLP12}. First, both of these existing constructions induce \df{non-proper} (point)
 contacts, while our contacts are always \df{proper}, that is, have non-zero areas.
Second, for a given -connected plane graph, it is not always possible to find a primal-dual
 representation with circles by a polynomial-time algorithm~\cite{BDEG14}, although it can be
 constructed numerically by polynomial-time iterative schemes~\cite{CS03,Moh97}.
There is also no known polynomial-time algorithm that computes a primal-dual representation
 with triangles for a given -connected plane graph.
Our representation, in contrast, can be found in linear time and can be
realized on the  grid, where  is the number of vertices of the graph.


We prove Theorem~\ref{thm:box} with two different constructive algorithms. The first one
uses the notions of Schnyder woods and orthogonal surfaces, as defined in~\cite{FZ08}.
Although it is known that every -connected planar graph induces an orthogonal surface, we show how to
construct a new contact representation with interior-disjoint boxes from the orthogonal surface.
Since the orthogonal surfaces for a -connected planar graph and its dual coincide
topologically,
the primal and the dual box-contact representations can be fit together to realize the desired contacts.
The alternative algorithm
builds a box-contact representation for a maximal planar graph using the notion of a canonical order
 of planar graph~\cite{FPP90}. Both construction ideas are inspired by recent box-contact representation
 algorithms for maximal planar graphs~\cite{BEFHK+12}; however, we generalize the algorithms to
 accommodate -connected planar graphs and show that the primal and dual representations can
 be combined together.
Our methods rely on a correspondence between Schnyder woods and generalized canonical orders for
 -connected planar graphs. Although the correspondence has been claimed earlier in~\cite{BBC11},
 the earlier proof appears to be incomplete. As another contribution of the paper, we provide a complete
 proof of the claim in Section~\ref{subsect:corr}. 


Theorem~\ref{thm:box} immediately gives box-contact representations for a special class of
non-planar graphs that are formed by the union of a planar graph and its dual. The graphs are
called \df{prime} and are defined as follows. A simple graph  is said to be \df{-planar}
if it can be drawn on the plane so that each of its edges crosses at most one other edge.
This class of graphs was first considered by Ringel~\cite{Rin65} in the context of simultaneously coloring a planar graph and its dual. A -planar graph has at most  edges~\cite{FM07,PT97}
and it is \df{optimal} if it has exactly  edges, that is, it is the
densest -planar graph on the vertex set~\cite{BSW84,BEGGH+13}. An optimal
-planar graph is called \df{prime} if it has no
separating -cycles, that is, cycles of length  whose removal disconnects the graph. These optimal -planar
graphs are exactly the ones that are -connected and the ones that can be obtained as the union of a -connected
simple plane graph, its dual and its vertex-face-incidence graph~\cite{Sch86}.
Note that in our primal-dual representation not all boxes are interior-disjoint, as one of the boxes
contains all other boxes. We call this special box the \df{shell} and such a
representation a \df{shelled} box-contact representation. Here all the vertices are represented
by 3D boxes, except for one vertex, which is represented by a shell, and the interiors of all the
boxes and the exterior of the shell are disjoint. Note that a similar shell
is also required in the circle-contact and triangle-contact
representations; see~Fig.~\ref{fig:intro}. The following is a direct
corollary of Theorem~\ref{thm:box}.

\wormhole{th-prime}
\begin{corollary}
\label{th:prime}
Every prime -planar graph  admits a shelled box-contact representation
in 3D and it can be computed in  time.
\end{corollary}


\begin{figure}
\centering
    \includegraphics[height=3cm]{L2}
  \caption{An -shaped polyhedron.}
  \label{fig:L}
\end{figure}
One may wonder whether every -planar graph admits a box-contact representation in 3D, but it is easy to see that there are -planar graphs, even as simple as , that do not admit a
box-contact representation. Furthermore, there exist optimal -planar graphs (which contain separating -cycles) that have neither a box-contact representation nor a shelled box-contact representation.


Therefore, we consider representations with the next simplest
axis-aligned object in 3D, an -shaped polyhedron
or simply an \LL, which is an axis-aligned box minus the intersection of
two axis-aligned half-spaces; see Fig.~\ref{fig:L}. Such an object can also be
considered as the union of two boxes in 3D. In the paper, we provide a quadratic-time algorithm
for representing every optimal 1-planar graph with~\LLs.

\wormhole{thm-1-planar}
\begin{theorem}
\label{thm:1-planar}
 Every optimal -planar graph  has a proper \LL-contact representation in 3D and it can be computed in  time.
\end{theorem}

Our algorithm is similar to a recursive procedure used for constructing
box-contact representations of planar graphs in~\cite{FF11,Tho88}. The basic idea is to find
separating -cycles and represent the inner and the outer parts of the graph induced by the cycles
separately. Then these parts are combined together to produce the final representation.
Since the separating -cycles can be nested inside each other, the
running time of our algorithm is dominated by the finding of separating -cycles
and the nested structure among them.
Unlike the early algorithms for box-contact representations of planar graphs~\cite{FF11,Tho88}, our algorithms produce proper contacts between the 3D objects (boxes and \LLs).





\section{Preliminaries}
\label{sec:prelim}

Here we introduce the tools needed to prove our results. In Section~\ref{subsect:def} we define
the known concepts of an \df{ordered path partition} and a \df{Schnyder wood}. In Section~\ref{subsect:corr} we describe new results about the relationship between these two structures for  -connected plane graphs. Section~\ref{sec:ortho} reviews the concept of an \df{orthogonal surface}.

\subsection{Ordered Path Partitions, Canonical Orders and Schnyder Woods}
\label{subsect:def}

The concepts of a Schnyder wood and a canonical order
were initially introduced for maximal plane graphs~\cite{FPP90,Sch90}; later they were generalized
 to -connected plane graphs~\cite{Kan96,FZ08}.
Although the concepts are proved to be equivalent for maximal plane graphs~\cite{FM01},
 they are no longer equivalent after the generalization. Thus Badent~et~al.~\cite{BBC11}
 generalized the notion of a canonical order to an {\em ordered path partition} for a -connected
 plane graph in an attempt to make it equivalent to a Schnyder wood.


Let  be a -connected plane graph with a specified pair of vertices  and a third vertex
, such that , ,  are all on the outer face in that counterclockwise
order. Add the edge  to the outerface of  (if it does not already contain it) and call the augmented graph
. Let  be a partition of the vertices of .
Then  is an \df{ordered path partition} of  if the following conditions hold:


\begin{enumerate}[(1)]


	\item  contains the vertices on the clockwise path from  to  on the outer
		cycle;
		;
	\item for , the subgraph  of  induced by the vertices in
		 is -connected and internally -connected
		(that is, removing two internal vertices of  does not disconnect it); and
		the outer cycle  of  contains the edge ;

	\item for , each vertex on  has at most one neighbor on .
\end{enumerate}

The pair of vertices  forms the \df{base-pair} for  and  is called the \df{head vertex}
of~.
For an ordered path partition  of , we say that a vertex  of  has
\df{level}  if . The \df{predecessors} of  are all the vertices with equal or smaller levels
and the \df{successors} of  are all vertices with equal or larger levels; see Fig.~\ref{fig:can-schB}.



The definition of a \df{canonical order} is similar to that for an ordered path partition, but for Condition (3),
 which is replaced by the following more restricted Condition (3'):

\begin{enumerate}
	\item[(3')] for ,  is a (left-to-right) path , which is a
		subpath of ,
		and each  with  has at least one neighbor in .
If  then each of  and  has one neighbor on  and these are the only
		neighbors of  in .
		If  then  may have multiple neighbors on .
\end{enumerate}


Thus a canonical order is a special ordered path partition and the definition of the \df{base-pair},
 \df{head vertex}, \df{predecessors} and \df{successors} follow from that in an ordered path partition.


A Schnyder wood is defined as follows. Let , ,  be three specified vertices
in that counterclockwise order on the outer face of . For , add a half-edge from 
reaching into the outer face. Then a \df{Schnyder wood} is an orientation and coloring of all the edges of~
(including the added half edges) with the colors  satisfying the following conditions:

\begin{enumerate}[(1)]


	\item every edge  is oriented in either one (\df{uni-directional edge}) or two opposite directions
		(\df{bi-directional edge}). The edges are colored and if  is bi-directional, then the two
		directions have distinct colors;

	\item the half-edge at  is directed outwards and colored ;

	\item each vertex  has out-degree exactly one in each color, and the counterclockwise
		order of edges incident to  is: outgoing in color , incoming in color , outgoing
		in color , incoming in color , outgoing in color , incoming in color ;

	\item there is no interior face whose boundary is a directed cycle in one color.

\end{enumerate}



\begin{figure}[t]
\centering
  \begin{subfigure}[t]{.45\textwidth}
    \centering
    \includegraphics[height=6.8cm]{can-sch}
    \caption{}
    \label{fig:can-schB}
  \end{subfigure}
~
  \begin{subfigure}[t]{.45\textwidth}
    \centering
    \includegraphics[height=6.8cm]{dual}
	\caption{}
    \label{fig:can-schA}
  \end{subfigure}
\caption{(a)~An ordered path partition
and its corresponding Schnyder wood for a -connected graph .
 (b)~The Schnyder woods for the primal and the dual of . The thick solid red, dotted blue and thin
 solid green edges represent the three trees in the Schnyder wood.}
\label{fig:can-sch}
\end{figure}



These conditions imply that for , the edges with color 
induce a tree  rooted at , where all edges of  are directed towards the root.
We denote the Schnyder wood by .
Every -connected plane graph  has a Schnyder wood~\cite{FZ08,BF12}.
From a Schnyder wood of
, one can construct a \df{dual Schnyder wood} (the Schnyder wood for the
dual of ). Consider the dual graph  of  in which the vertex
for the outer face of  has been split into three vertices forming a triangle. These three vertices
represent the three regions between pairs of half edges from the outer vertices of . Then a Schnyder
wood for  is formed by orienting and coloring the edges so that between an edge  in 
and its dual  in , all three colors  are used. In particular, if  is uni-directional
in color , , then  is bi-directional in colors ,  and vice versa;
see Fig.~\ref{fig:can-schA}.


\subsection{Correspondence}
\label{subsect:corr}

Let  be a -connected plane graph with a specified base-pair  and a specified head
vertex  such that , ,  are in that counterclockwise order on the outer face.
It is known that an ordered path partition of  defines a Schnyder wood on , where the three outgoing
edges for each vertex are to its (1)~leftmost predecessor, (2)~rightmost predecessor, and
(3)~highest-level successor~\cite{FZ08,BF12}. We call an ordered path partition and the corresponding Schnyder
wood computed this way to be \df{compatible} with each other.
Badent~et~al.~\cite{BBC11} claim that the converse can also be done, that is, given a Schnyder wood on ,
one can compute an ordered path partition, compatible with the Schnyder wood (and hence,
there is a one-to-one correspondence between the concepts).
However, it turns out that the algorithm in~\cite{BBC11} for converting a Schnyder wood to a compatible ordered path partition is incomplete\footnote{Confirmed by a personal communication with the authors of~\cite{BBC11}.}, that is, the computed ordered path partition is not always
compatible with the Schnyder wood; see Fig.~\ref{fig:non-eq}
for an example. Here we provide a correction for the algorithm.


\begin{figure}[t]
\centering
\includegraphics[width=\textwidth]{non-equivalent}
\caption{Two ordered path partition with their compatible Schnyder woods (left and right) that gives the
 same acyclic graph  after reversing the direction of the edges in two trees and grouping all cyclic
 maximal paths. Thus starting with either Schnyder wood can give either ordered path partition by the
 algorithm in~\cite{BBC11}.}
\label{fig:non-eq}
\end{figure}


Let  be a Schnyder wood of . From here on whenever
we talk about Schnyder woods, we consider a circular order for the indices  so that
 and  are well defined when . By~\cite{FZ08}
there is no cycle of  that is directed in , where
, , denotes the reversed edges of .
Since there are some bi-directional edges in  colored
with  and  (which we call \df{cyclic}),  induces some directed cycles of length . We can form a directed acyclic graph  by \textit{grouping}
each maximal path in  with cyclic bi-colored edges
(call such a path a \textit{cyclic maximal path}) into a single vertex. Here the maximal paths  with
cyclic bi-colored edges are the vertices of , and for
two such paths  and , there is a directed edge from  to  in  whenever
there is a directed (not cyclic) edge  in 
for some  and .
Badent~et~al.~\cite{BBC11} showed that  is acyclic and they suggest
to obtain an ordered path partition by taking a topological order of .
However, the resulting ordered path partition is not necessarily compatible with  and Fig.~\ref{fig:non-eq} shows an example.
Instead, before grouping the cyclic maximal paths, we augment the graph
 with the following directed edges.
For each vertex  of , we add a directed edge from each child of  in 
and  to the parent of  in . The augmented graph remains acyclic
and it is consistent with the partial order defined by  (we call this the \textit{partial order defined by
}).
A topological order of the augmented graph (after grouping all cyclic maximal paths)
induces a compatible ordered path partition.


\begin{lemma}
\label{lem:schny-can}
Let  be a Schnyder wood of a -connected plane graph
 with three specified vertices , ,  in that counterclockwise order on the outer face.
Then for  one can compute in linear time an ordered path partition  compatible with
 such that  has  as the base pair and  as the head.
The ordered path partition is consistent with the partial order defined by
.
\end{lemma}
\begin{proof}
Consider the directed acyclic graph  by grouping each cyclic maximal path of
  into a single vertex.
 is acyclic since by~\cite{FZ08}, no cycle in  is directed in
 .
 Furthermore each vertex in  has at least two predecessors, one in  and one in
 . Therefore one can compute an ordered path partition of  by taking a topological
 ordering  of  and for each vertex  of  assigning  label  where  and
  is the rank of  in ; see~\cite{BBC11} for details. However the ordered path partition obtained
 by this procedure might not be compatible with ; in particular for a vertex  of
 , its parent in  might not be its highest-level successor; see Fig.~\ref{fig:non-eq}.


 In order to ensure compatibility between
 the Schnyder wood and the obtained ordered path partition, we further augment  by adding some
 extra edges. In particular, for each vertex  of , if  is its parent in  and  is it child in
 either  or , then we add a directed edge  in  where ,  are
 cyclic maximal paths in  and , . Call the augmented directed graph .
 We now show that with the addition of the extra edges the directed graph  remains acyclic.
 We prove this claim only for ; for  and  the proofs are analogous.


\begin{figure}[t]
\centering
\includegraphics[width=0.9\textwidth]{face-schny.pdf}\\
(a)\hspace{0.27\textwidth}(b)\hspace{0.27\textwidth}(c)\hspace{0.05\textwidth}
\caption{(a) Coloring and orientation of the edges around a generic face in to a Schnyder wood
 , (b) orientation of the edges in
 , (c) directed paths in
  from children of a vertex  in  and 
 to the parent in .}
\label{fig:face-schny}
\end{figure}


Consider an arbitrary face  of . By~\cite{FZ08}, the edges on the boundary of  can be
 partitioned into at most six consecutive sets in clockwise order around 
 (see Fig.~\ref{fig:face-schny}(a)):


\begin{enumerate}[(i)]
	\item one edge from the set {clockwise in color , counterclockwise in color , bi-colored with
		a clockwise  and counterclockwise }
	\item zero or more edges bi-colored in counterclockwise  and clockwise 
	\item one edge from the set {clockwise in color , counterclockwise in color , bi-colored with
		a clockwise  and counterclockwise }
	\item zero or more edges bi-colored in counterclockwise  and clockwise 
	\item one edge from the set {clockwise in color , counterclockwise in color , bi-colored with
		a clockwise  and counterclockwise }
	\item zero or more edges bi-colored in counterclockwise  and clockwise 

\end{enumerate}


Fig.~\ref{fig:face-schny}(b) shows the direction of the edges of  in . Now consider a vertex 
 of . Let  be its parent in , , ,  its children in  in clockwise order
 around , and , ,  its children in  in counterclockwise order around .
 Thus  appears consecutively around  in that clockwise order.
 Let ,  be the paths from  to its next vertex in  that forms the face of 
 containing these two vertices and ; see Fig.~\ref{fig:face-schny}(c). Similarly let ,  be the paths from  to its previous vertex in  that forms the face of  containing these two
 vertices and . From Fig.~\ref{fig:face-schny}(b) one can see that all the paths  and 
 are directed towards  (possibly with some bi-directional edges) in . This imply that for each
 vertex ,  (resp. , ), there is a directed path (possibly with some
 bi-directional edges) from  (resp. ) to . Thus adding an edge from any  or  to 
 does not create any cycle in  since otherwise replacing the edge with the directed path induces
 a directed cycle in , a contradiction. Since the addition of the
 extra edges does not make any cycle in , the graph  remains acyclic.



Once we add the extra edges, we can compute an ordered path partition consistent with , by taking a
 topological ordering of  and for each vertex  of , assigning  as its label
 where  for some cyclic maximal path  in  and  is the rank of  in the
 topological ordering of . This ordered path partition is compatible with  since
 for each vertex , the parents of  in  and in  are the leftmost and rightmost
 predecessors (due to the embedding) and the parent in  is the highest-level successor
 (due to the addition of the extra edges).


The time complexity of the above algorithm is linear. Computing the directed graph , 
 can be done by traversing the edges of . Addition of the extra
 edges takes  time also. Finally the topological ordering can be
 done by a linear-time Depth-first traversal on the graph , which also has a linear size.
\end{proof}




Fig.~\ref{fig:sch-order} illustrates the three ordered path partitions computed from a Schnyder wood of
 a -connected plane graph using the algorithm described above.



\begin{figure}[htbp]
\centering
\includegraphics[width=\textwidth]{sch-order.pdf}
\caption{(a) A Schnyder wood  in a -connected plane graph , which is not
 compatible with any canonical order, (b)--(d) the three compatible ordered path partition computed from
  using the algorithm described in the proof of Lemma~\ref{lem:schny-can}.}
\label{fig:sch-order}
\end{figure}










\subsection{Elementary Canonical Orders and Elementary Schnyder Woods}


Here we introduce the concepts of \df{elementary} canonical orders and \df{elementary} Schnyder woods,
 which we use in the subsequent section for an alternate proof for Theorem~\ref{thm:box}.


A canonical order  for a -connected plane graph  is \df{elementary}
 if (i)  (or equivalently the base-pair  induces an edge on the outer cycle).
Kant showed that any -connected plane graph , with an edge  and a third vertex ,
 both on the outer cycle, has an elementary canonical order for the base pair  and head 
 and such an elementary canonical order for  can be computed in linear time~\cite{Kan96}.




Not all Schnyder woods are compatible with canonical orders; Fig.~\ref{fig:sch-order}(a) shows an example
 of a Schnyder wood that is not compatible with any canonical order.
 Let  be a canonical order in a -connected plane graph  with
 a specified base-pair  and a specified
 head vertex  such that , ,  are in that counterclockwise order on the outer face.
 Let   be the Schnyder wood compatible with .
 Then by the definition of compatible Schnyder wood, it follows that
for every maximal path  in 
 with each edge  bi-colored in color  and  for , there is no child of the vertices
  in , and the only children of  and  in  (if any) are
 respectively the parent of  in  and the parent of  in .
We call this property for a Schnyder wood  the \df{canonical property} of the
 Schnyder wood for color 3. One can define the canonical property of a Schnyder wood analogously
 for colors  and .
Let  be a Schnyder wood for a -connected plane graph  with three specified
 vertices , ,  in that counterclockwise order on the outer face. Then
  is \textit{elementary} for the color , , if (i)  is an
 edge on the outer cycle, and (ii) the canonical property holds for color  in .




\begin{lemma}
\label{lem:elementary}
 Let  be a -connected plane graph with three specified vertices , ,  in the
 counterclockwise order on the outer face of . Let  be an ordered path partition of 
 with a specified base-pair  and a specified head vertex  for  and let
  be the compatible Schnyder wood for . Then
  is an elementary canonical order if and only if  is an elementary
 Schnyder wood for color .
\end{lemma}
\begin{proof} The lemma follows from the definitions of elementary canonical order and elementary
 Schnyder wood and from the observation that the maximal paths in 
 bi-colored with colors  and  are in one-to-one correspondence with the paths of  with
 the same level in , .

\end{proof}



We end this section with the following lemma.


\begin{lemma}
\label{lem:dual} If a Schnyder wood  for a -connected plane graph  is
 elementary for color , , then its dual Schnyder wood is also elementary for color 
 in the dual graph of .
\end{lemma}
\begin{proof} We prove this lemma only for color , since the proof for color  or  is similar.
By definition the outer cycle for the dual graph of  for constructing the dual Schnyder
 wood is a triangle; hence the base-pair for the dual Schnyder wood are adjacent. Hence for the dual
 Schnyder wood to be elementary it is sufficient that for each 
 with each edge  bi-colored in  and  for , (i) there is no child of the vertices
  in , and (ii) the only child of  and  in  (if any) are
 respectively the parent of  in  and the parent of  in . However, the way
 we define the color and orientation of the edges in the dual, failure to satisfy Condition (i) for some
 path in the dual Schnyder wood implies that Condition (ii) does not hold for some path in the primal
 Schnyder wood and vice versa; see Fig.~\ref{fig:elementary-dual}. Therefore if the primal Schnyder
 wood is elementary in color 3, so is the dual Schnyder wood.
\end{proof}


\begin{figure}[htbp]
\centering
\includegraphics[width=0.37\textwidth]{elementary-dual.pdf}
\caption{Illustration for the proof of Lemma~\ref{lem:dual}.}
\label{fig:elementary-dual}
\end{figure}







\subsection{Orthogonal Surfaces}
\label{sec:ortho}

Here we briefly review the notion of orthogonal surfaces, which we use in the proof for
 Theorem~\ref{thm:box}; see~\cite{FZ08} for more details.
A point  in  \df{dominates} another point  if the coordinate of  is greater than or equal to
  in each dimension;  and  are \df{incomparable} if neither of  nor  dominates the other.
 Given a set  of incomparable points, an \df{orthogonal surface} defined by  is the geometric
 boundary of the set of points that dominate at least one point of .
For two points  and , their \df{join},  is obtained by taking the maximum coordinate of
 ,  in each dimension separately. The \df{minimums} (\df{maximums})
 of an orthogonal surface  are the points of  that dominate (are dominated by) no other
 point of . An orthogonal surface  is \df{rigid} if for each pair of points  and  of  such that
  is on ,  does not dominate any point other than  and .
 An orthogonal surface is \df{axial} if it has exactly three unbounded orthogonal arcs.
 Rigid axial orthogonal surfaces are known to be in one-to-one correspondence with Schnyder woods
 of 3-connected plane graphs~\cite{FZ08} and the rigid axial orthogonal surfaces  and 
 corresponding to a Schnyder wood and its dual coincide with each other, where the maximums of
  are the minimums of  and vice versa.


\section{Primal-Dual Representations of 3-Connected Planar Graphs}
\label{sec:pri-du}

Here we prove Theorem~\ref{thm:box}.


\begin{backInTime}{thm-box}
 \begin{theorem}
  Every -connected planar graph  admits a proper primal-dual box-contact representation in 3D and it can be computed in  time.
 \end{theorem}
\end{backInTime}


Specifically, we describe two different linear-time algorithms that compute
 a box-contact representation for the primal graph and the dual graph separately
 and then fits them together to obtain the desired result.
In the first algorithm we compute the coordinates of the boxes based on a Schnyder wood,
 in the second, we compute them based on an elementary canonical order. Both these algorithms
 guarantee that the boundary for the primal (dual) representation induces an \df{orthogonal surface}
 compatible with the dual (primal) Schnyder wood.
Since the orthogonal surfaces for a Schnyder wood and its dual coincide topologically, we can
fit together the primal and the dual box-contact representation to obtain a desired representation.




To avoid confusion, we denote a connected region in a plane embedding of a graph
 by a \df{face}, and a side of a 3D shape by a \df{facet}. For a 3D box ,
 call the facet with highest (lowest) -coordinate as the -facet (-facet) of .
The -facet, -facet, -facet and -facets of  are defined similarly. For
 convenience, we sometimes denote the -, , -, , - and -facets
 of  as the \df{right}, \df{left}, \df{front}, \df{back}, \df{top} and \df{bottom} facets of , respectively.





\begin{figure}[tb]
\centering
\parbox[t][4.5cm][t]{0.07\textwidth}
{
	\includegraphics[width=0.28\textwidth]{box2}
}
\parbox[t][4.5cm][b]{0.62\textwidth}
{
	\includegraphics[width=0.85\textwidth]{CanSchBox-1}
}
\parbox[t][4.5cm][t]{0.28\textwidth}
{
	\includegraphics[width=0.28\textwidth]{surface2}
}
    \caption{Box-contact representation (a) for the graph in Fig.~\ref{fig:can-sch}
	with its primal-dual Schnyder wood (b) and the associated orthogonal surface (c).
	The thick solid red, dotted blue and thin solid green edges represent the three trees
	in the Schnyder wood.}
    \label{fig:draw}
\end{figure}


\subsection{Construction Based on Schnyder Woods}
\label{sec:mainproof}

We first construct a contact representation  of the primal graph  with boxes in 3D.
 Let ,  and  be three vertices on the outer face of  in counterclockwise order.
 We compute a Schnyder wood  such that for ,  is rooted
 at . By Lemma~\ref{lem:schny-can}, for , one can compute a compatible ordered path
 partition with the base-pair  and head , which is consistent with the partial order
 defined by . Denote by
 ,  and  the three ordered path partitions compatible with , that
 are consistent with , , and , respectively. We use these three ordered path partitions
 to define our box-contact representation for .

 For a vertex , let , , and  be the levels of  in the ordered path
 partitions , , and , respectively. Define ,  and
 , where ,  and  are the parents of  in ,  and ,
 respectively, whenever these parents are defined. For each of the three special vertices ,
 , the parent is not defined in . We assign ,  and
 .


Now for each vertex  of , define a box  representing  as the region
 .
Denote by  the set of all boxes  for the vertices  of . We show in
Lemma~\ref{lem:box-schny} that  yields a box-contact representation for .
Furthermore, for each edge  of , if  is uni-colored then  and 
make a proper contact. Otherwise, assume \WLOG that  is bi-colored
with colors  (oriented from  to ) and  (oriented from  to ).
Then , , ;   and 
make a non-proper contact along the line-segment ,
where . However since  is the only parent of  in 
and  is the only parent of  in , by Lemma~\ref{lem:box-schny},
the -facet of  and the -facet of  do not make a proper contact with any box.
Hence either extending  in the positive -direction or extending  in positive -direction
by some small amount  makes the contact between  and  proper
without creating any overlap or unnecessary contacts between the boxes.
 We thus obtain a proper box-contact representation  for the primal graph .


Now we describe how to construct the box-contact representation for the dual of .
Consider the orthogonal surface induced by . Each vertex  of  corresponds
 to the -corner  of the box for . The three outgoing edges of  in the Schnyder
 wood can be drawn on the surface from  in the directions , , ; see Fig.~\ref{fig:draw}.
Each face of  corresponds to a reflex corner of the orthogonal surface and there is a similar
 (opposite) direction for the outgoing edges in the dual Schnyder wood. The topology of this
 (rigid axial) orthogonal
 surface is uniquely defined by both the Schnyder wood and its dual~\cite{FZ08}.
Thus, we construct the contact representation  for the dual of , then (after possible scaling)
 the boundary of  exactly matches , where  is the bounding box of .
We fit  and  together by replacing the three boxes for the three outer vertices of  with a single shell-box, which forms the boundary of the entire representation.

It is easy to see that the above algorithm runs in  time. A Schnyder wood for  and the dual Schnyder wood
 for the dual of  can be computed in linear time~\cite{FZ08}. For both the primal graph and the dual
 graph, the three ordered path partitions can then be computed in linear time from these Schnyder woods,
 due to Lemma~\ref{lem:schny-can}. The coordinates of the boxes are then directly assigned in constant
 time per vertex. Finally the primal and the dual representation can be combined together by reflecting
 (and possibly scaling) the dual representation, which can also be done in linear time.


Lemma~\ref{lem:box-schny} shows the correctness
for the above construction; specifically,
 it shows that the boxes computed in the construction induce a contact representation for .

\begin{lemma}
\label{lem:box-schny}
The set of all boxes  for the vertices  of  induces a contact representation of , where
 for each vertex , the -, -, -facets of  touch the boxes for the parents of
  in , , , respectively, and the -, -, -facets of  touch
 the boxes for the children of  in , , , respectively. Furthermore for each
 edge  of , the contact between  and  is proper if and only if  is uni-colored.
\end{lemma}
\begin{proof} We prove the lemma by showing the following two claims:
 (i)~for each edge  of , the two boxes  and  make contact in the specified facets,
 (ii)~for any two vertices  and  of , the two boxes  and  are interior-disjoint.

\begin{enumerate}[(i)]
\item Take an edge  of . If  is uni-colored in , assume \WLOG that it has color 3 and is directed from  to .
 By construction, . We now show that  and
 , which implies the correct contact between  and .
 Since  is in  and the ordered path partitions  and  are consistent
 with ,  and .
 To show that , consider the parent  of  in .
 Then . Furthermore, before computing the topological order to find , we added
 the directed edge ; thus . The proof that  is symmetric.

 If  is bi-colored with colors, \WLOG,  (orientated from  to ) and  (from  to ),
 then by construction  and . 
  is bi-colored in , , so . Thus, ,  make
 non-proper contact in the correct facets.

\item Now we show that for any two vertices  and  of ,  and  are
 interior-disjoint. By the properties of Schnyder woods, from any vertex  of ,
 there are three vertex-disjoint paths , ,  where  is 
 a directed path from  to  of edges colored , .
We first claim that for some , , there is a directed path from  to ,
 or from  to  in . The claim holds trivially if 
 is on the directed path , or  is on the directed path , for some .
 Otherwise, assume that  is in the region between , and . Then the path 
 intersects either , and  at some vertex . Assume \WLOG that
  is on . Then the path  that follows  from  to , and then follows
  from  to  is directed in  (here  is the path
  with all the directions reversed). Hence, assume that there is a path from  to  in
 . By definition,  and thus  and  are
 interior-disjoint.
\end{enumerate}\vspace{-0.7cm}
\end{proof}




\subsection{Construction Based on Elementary Canonical Orders}
\label{sec:alternateproof}

Here we provide an alternate proof for Theorem~\ref{thm:box} by a different construction algorithm
 than the one in Section~\ref{sec:mainproof}.
This algorithm is based on an elementary canonical order and builds a representation iteratively
 inserting boxes for the vertices in this order. It is similar to the box-contact representation algorithm
 suggested in~\cite{BEFHK+12} for maximal planar graphs. However in addition to generalizing the
 construction for -connected plane graphs, we maintain a stronger invariant on every iteration,
 in order to accommodate the boxes of the dual graph later on.





We first construct a contact representation  of  with boxes.
Let ,  and  be three vertices on the outer face of  in counterclockwise order
 such that  is an edge on the outer cycle. Compute an elementary canonical
 order  and the compatible Schnyder wood defined by .
Let  be the graph induced by the vertices in . We now add boxes
 representing the vertices in the order defined by .


For step  of our construction, we add two boxes to represent  and 
 such that the -facet of the box for  touches the -facet of the box for .
Their -facets lie on the plane  and their -facets on .
We maintain the invariant that, at the beginning of step ,
the boxes of \df{active vertices}, that
is, the vertices in ,
intersect the plane  in a \df{staircase shape}, that is, where
the minimum  boundary of the intersection is an -monotone
axis-aligned polyline where the convex corners are -corners
of boxes that represent the active vertices consecutively in the order
that they appear in the outer face of .




For step ,
we add boxes for the vertices in 
with their -facets on the plane  and their -facets on
.

 If , let  and  be the leftmost and rightmost neighbors of  in .
We create the box for  so that
its -facet touches box ,
its -facet touches box , and
its -facet covers (and touches if )) the
 box for  ()
(this is possible due to the staircase invariant).
Boxes in this last set are now no longer
active. By the construction of a compatible Schnyder wood, each edge of  added
in this step except possibly the edges with  and 
are colored  and directed towards . The edge 
is colored  and directed towards , and the edge  is
colored  and directed towards ; note that one or both these edges may
also be colored  and directed towards .

If , let  and  be the neighbors of  and
, respectively, on .
We create the boxes for  so that
the -facets or -facets of the boxes for  and
 touch.
The -facet of box  touches box  and the -facet of
box  touches box .
By the construction of compatible Schnyder wood, for ,
the edges  are bi-directional, and the direction from 
to  is colored , while the other direction is colored . The edges
 and  are directed and colored as in the case where .

In both cases:
if  is a bi-directional edge (with colors  and )
we align the -facets of box  and box  (note
 is no longer active); and
if  is a bi-directional edge (with colors  and )
we align the -facets of box  and  (note 
is no longer active).
We have not yet set the coordinate of the -facet of any of the boxes of . We simply
extend the boxes of all active vertices in the  direction, so that the -facet is in
the plane .


An illustration of the representation obtained by this algorithm is shown in Fig.~\ref{fig:draw}.


Now we describe how to construct the box-contact representation for the dual of .
Consider the orthogonal surface induced by . Each vertex  of  corresponds
 to the -corner  of the box for . The three outgoing edges of  in the Schnyder
 wood can be drawn on the surface from  in the directions , , ; see Fig.~\ref{fig:draw}.
Each face of  corresponds to a reflex corner of the orthogonal surface and there is a similar
 (opposite) direction for the outgoing edges in the dual Schnyder wood. The topology of this
 (rigid axial) orthogonal surface is uniquely defined by both the Schnyder wood and its dual~\cite{FZ08}.
Thus, we construct the contact representation  for the dual of  using the same algorithm,
 then (after possible scaling) the boundary of  exactly matches , where 
 is the bounding box of .
We fit  and  together by replacing the three boxes for the three outer vertices of  with a single shell-box, which forms the boundary of the entire representation.



The above construction algorithm takes  time. An elementary canonical order of  can be
 computed in linear time~\cite{Kan96} and one can compute a compatible Schnyder wood using the
 definition in linear time as well.
The ,  and -coordinates of the boxes are computed in constant time for a vertex of a primal
 graph. Hence, the primal representation of the graph can be found in linear time.
Again from the Schnyder wood of , one can compute a dual Schnyder wood for the dual graph
 in linear time~\cite{FZ08} and by Lemma~\ref{lem:schny-can} a compatible elementary canonical order
 for the dual can also be computed in linear time and the same algorithm can be used to construct the
 dual representation in linear time.
 Finally the primal and the dual representation can be combined together by reflecting







The construction for a primal-dual box-contact representation
 of a -connected plane graph  in Theorem~\ref{thm:box} has another interpretation. Start with the orthogonal surface  corresponding to both a Schnyder wood of  and its dual. This orthogonal surface  creates two half-spaces on either side of : extend boxes from  in these half-spaces, so that
(i)~the boxes are interior-disjoint and (ii)~they induce box-contact representations for the
 primal and dual graphs. However, how to extend the corners and why such a construction yields a proper box-contact representation seems to require the same kinds of arguments that we provided in this section.
A ``proof from the book'' for Theorem~\ref{thm:box}, using topological properties of the orthogonal surface, is a nice open problem.











\section{Representations for Optimal 1-Planar Graphs}

In this section we consider contact representation for optimal 1-planar graphs. We first show that
 there exists optimal 1-planar graphs with no box-contact representation. We thus consider contact
 representation for optimal 1-planar graphs with the next simplest axis-aligned 3D object: \LLs.
 We provide a quadratic-time algorithm for representing optimal 1-planar graph with \LLs.



\subsection{Optimal 1-Planar Graphs with no Box-Contact Representation}

Here we prove the following lemma.

\begin{lemma}
\label{lem:K5}{\ \-1em]}
 \begin{itemize}
  \item The subgraph of an embedded optimal 1-planar graph  induced by the non-crossing edges is a plane quadrangulation  with bipartition classes  and .
  \item The induced subgraphs  and  on white and black vertices, respectively, are planar and dual to each other.
  \item The graphs  and  are -connected if and only if  has no separating -cycles.
  \item There exists a simple optimal 1-planar graph with quadrangulation  if and only if  is -connected.
 \end{itemize}
\end{lemma}

\begin{figure}[htb]
\centering
  \begin{subfigure}[t]{.3\textwidth}
    \centering
    \includegraphics[height=3cm]{1-planar-a}
    \caption{}
    \label{fig:opt-1-planar}
  \end{subfigure}
  \hspace{1em}
  \begin{subfigure}[t]{.3\textwidth}
    \centering
    \includegraphics[height=3cm]{1-planar-b}
    \caption{}
    \label{fig:opt-H1}
  \end{subfigure}
  \hspace{2em}
  \begin{subfigure}[t]{.15\textwidth}
    \includegraphics[height=2cm]{1-planar-c}
    \caption{}
    \label{fig:opt-H2}
  \end{subfigure}
  \caption{(\subref{fig:opt-1-planar})~An embedded optimal 1-planar graph, its quadrangulation  (bold) and the partition into white and black vertices.
  (\subref{fig:opt-H1})~The graph  produced by removing the interior of separating -cycle .
  (\subref{fig:opt-H2})~The graph  comprised of the separating -cycle and its interior.}
\end{figure}




Call an optimal 1-planar graph \df{prime} if its quadrangulation has no separating -cycle.

\begin{backInTime}{th-prime}
\begin{corollary}\label{cor:goodCase}
Every prime -planar graph  admits a shelled box-contact representation
in 3D and it can be computed in  time.
\end{corollary}
\end{backInTime}
\begin{proof}
Let  be the quadrangulation of  and let ,  be the bipartition classes of .
 By Lemma~\ref{lem:1-planar-subgraphs},  and  are -connected planar and dual to each other.
 By Theorem~\ref{thm:box}, a primal-dual box-contact representation  of 
 can be computed in linear time.
 We claim that  is a contact representation of .
 Indeed, the edges of  are partitioned into , , .
 Each edge in  is realized by contact of two ``primal'' boxes, in  by contact of ``dual'' boxes, and in  by contact of a primal and a dual box; see Fig.~\ref{fig:big_example}.
\end{proof}


\begin{figure}[htb]
 \centering
 \includegraphics{big_example}
 \caption{Part of an optimal 1-planar graph and its partial proper box contact representation}
 \label{fig:big_example}
\end{figure}




Next, assume that  is any (not necessarily prime) optimal 1-planar graph.
To find an \LL-representation for , we find all separating -cycles in , replace their interiors by a pair of crossing edges and construct an \LL-representation of the obtained smaller 1-planar graph by Corollary~\ref{cor:goodCase}.
We ensure that this \LL-representation has some ``available space'' in which we can place the \LL-representations for the removed subgraph in each separating -cycle, which we construct recursively.
We remark that similar procedures were used before, e.g., for maximal planar graphs and their separating triangles~\cite{FF11,Tho88}.
A separating -cycle is \df{maximal} if its interior is inclusion-wise maximal among all separating -cycles.
A 1-planar graph with at least  vertices is called \df{almost-optimal} if its non-crossing edges induce a quadrangulation  and inside each face of , other than the outer face, there is a pair of crossing edges.





\newcommand{\lt}{\textbf{Let} }
\newcommand{\drawOpt}{algorithm \textbf{L-Contact}}

\hspace{-0.02\textwidth}
\parbox{0.97\textwidth}
{
\begin{algorithm}{L-Contact}[\text{optimal 1-planar graph }G]
	{
}



	Find all separating -cycles in the quadrangulation  of \\

	\qif some inner vertex  of  is adjacent to two outer vertices of \\

		\qthen
		 set of the two -cycles containing  and 3 outer vertices of .
		 \textbf{(Case 1)}

		\qelse
		 set of all maximal separating -cycles in .
		\textbf{(Case 2)}
	\qfi\\

	Take the optimal 1-planar (multi)graph  obtained from  by replacing
	for each -cycle  all vertices strictly inside 
	by a pair of crossing edges; see Fig.~\ref{fig:opt-H1}.
	\label{step:define_G_out}\\

	Compute an \LL-representation of  that has ``some space'' at each -cycle
	. In Case~2, this is done from a shelled box-contact representation
	of  in Corollary~\ref{cor:goodCase}.
	\label{step:G_out}\\

For each , take the almost-optimal 1-planar
	subgraph  of  induced by  and all vertices inside ;
	see Fig.~\ref{fig:opt-H2}. Recursively compute an \LL-representation of 
	and insert it into the corresponding ``space'' in the \LL-representation of .
	\label{step:G_in}

\end{algorithm}
}




\begin{comment}
\begin{figure}[h!]
 \begin{enumerate}
  \item Find all separating -cycles in the quadrangulation  of .
  \item If \textbf{(Case 1)} some inner vertex  of  is adjacent to two outer vertices of  let  be the set of the two -cycles in  consisting of  and three outer vertices of .
  \item Otherwise \textbf{(Case 2)} let  be the set of all maximal separating -cycles in .
  \item Consider the optimal 1-planar (multi)graph  obtained from  by replacing for each -cycle  all vertices strictly inside  by a single pair of crossing edges; see Fig.~\ref{fig:opt-H1}.\label{step:define_G_out}
  \item Compute an \LL-representation of  that has ``some space'' at each -cycle . In Case~2 this is done based on the shelled box-contact representation of  in Corollary~\ref{cor:goodCase}.\label{step:G_out}
  \item Go through all -cycles  and consider the almost-optimal 1-planar subgraph  of  induced by  and all vertices strictly inside ; see Fig.~\ref{fig:opt-H2}. Compute an \LL-representation of  recursively and insert it into the corresponding ``space'' in the \LL-representation of .\label{step:G_in}
 \end{enumerate}
 \caption{Outline of the procedure constructing an \LL-representation for any given optimal 1-planar graph  in linear time.}
 \label{fig:outline}
\end{figure}
\end{comment}


Let us formalize the idea of ``available space'' mentioned in steps~\ref{step:G_out} and~\ref{step:G_in} in the above procedure.
Let  be any \LL-representation of some graph  and  be a -cycle in .
A \df{frame for } is a -dimensional axis-aligned box  together with an injective mapping
of  onto the facets of  such that the two facets without a preimage are adjacent.
Every frame has one of two possible types.
If two opposite vertices of  are mapped onto two opposite facets of , then  has type ;
otherwise,  has type ;  see Fig.~\ref{fig:frames}.
Finally, for an almost-optimal 1-planar graph  with corresponding quadrangulation  and outer face , and a given frame  for , we say that an \LL-representation  of  \df{fits into } if replacing the boxes or \LL's for the vertices in  by the corresponding facets of  yields a proper contact representation of  that is strictly contained in .

Before we prove this Section's main result, namely Theorem~\ref{thm:1-planar}, we need one last lemma addressing the structure of maximal separating -cycles in almost-optimal 1-planar graphs.


\begin{lemma}\label{lem:almost}
 Let  be an almost-optimal 1-planar graph with corresponding quadrangulation .
 Then all maximal separating -cycles of  are interior-disjoint, unless two inner vertices  and  of  are adjacent to two outer vertices of .
\end{lemma}
\begin{proof}
 When two maximal separating -cycles  and  are not interior-disjoint, then some vertex from  lies strictly inside  and some vertex from  lies strictly inside .
 It follows that  is a pair  of two vertices from the same bipartition class of , say , and that some  lies strictly outside  and some  lies strictly outside .
 We have  and that  is a -cycle whose interior strictly contains  and .
 By the maximality of  and  it follows that  is not separating.
 As the vertices  and  lie strictly inside ,  must be the outer cycle of  and  are the desired vertices.
\end{proof}

\begin{figure}[t]
\centering
  \begin{subfigure}[t]{.2\textwidth}
    \centering
    \includegraphics{lemma_almost}
    \caption{}
    \label{fig:sep-pair-b}
  \end{subfigure}
  \hspace{3em}
  \begin{subfigure}[t]{.5\textwidth}
    \centering
    \includegraphics{frames}
    \caption{}
    \label{fig:frames}
  \end{subfigure}
  \caption{(a)~Illustration for the proof of Lemma~\ref{lem:almost}.
  (b)~A frame of type  (left) and of type  (right).}
\end{figure}


\begin{backInTime}{thm-1-planar}
 \begin{theorem}
  Every optimal 1-planar graph  has an \LL-representation and it can be computed in  time.
 \end{theorem}
\end{backInTime}
\begin{proof}
 Fix any 1-planar embedding of  and let  be the corresponding quadrangulation with outer cycle . Following {\drawOpt},
we distinguish two cases.
 If (\textbf{Case~1}) some inner vertex  of  has two neighbors on  we let  be the set of the two -cycles in  that consist of  and 3 vertices of .
 Otherwise (\textbf{Case~2}), let  be the set of all maximal separating -cycles in .
By Lemma~\ref{lem:almost} the cycles in  are interior-disjoint.
 As in step~\ref{step:define_G_out} we define  to be the optimal 1-planar (multi)graph obtained from  by replacing for each  all vertices strictly inside  by a pair of crossing edges.
 Note that in Case~1 the quadrangulation corresponding to  is  with inner vertex .
 We proceed by proving the following claim, which corresponds to step~\ref{step:G_out} in the algorithm.


 \begin{claim}
  Let  be an almost-optimal 1-planar (multi)graph whose corresponding quadrangulation  is either  or has no separating -cycles.
  Let  be a set of facial -cycles of , different from , and  be the graph obtained from  by removing the crossing edges in each .
  Then for any given frame  for the outer cycle  of  one can compute an
 \LL-representation  of  fitting into  such that for every  there is a frame  for  that is interior-disjoint from all boxes and \LL's in .
\label{cl:two-case}
 \end{claim}
 \begin{claimproof}{\ref{cl:two-case}}
  \begin{description}
   \item[Case 1, .] Let  be the inner vertex of .
    Without loss of generality let  and let  be mapped onto the top, back left, bottom and back right facets of .
    We define the \LL for  to be the union of  and .
    Further define four boxes , ,  and , each completely contained in  and disjoint from the \LL for ; see Fig.~\ref{fig:K23}.
Each  is a frame for a -tuple containing  and exactly three vertices of , .
    Thus independent of the type of  and the neighbors of  in , we find a frame for both inner faces of .

    \begin{figure}[htb]
     \centering
     \includegraphics[width=0.35\textwidth]{K23}
     \caption{Construction for Case 1 in the proof of Claim~\ref{cl:two-case}.}
     \label{fig:K23}
    \end{figure}

    \item[Case 2, .]
     Let  and  be the bipartition classes of  and  with  and , .
     Without loss of generality  are mapped onto the back left, back right and top facets of , respectively, and  is mapped onto the bottom facet if (\textbf{Case~2.1})  has type  and onto the front left facet if (\textbf{Case~2.2})  has type .
     Let  be the graph obtained from  by inserting a pair of crossing edges in , leaving  and  on the unbounded region.
     By assumption,  is a prime 1-planar graph and thus by Lemma~\ref{lem:1-planar-subgraphs}  and  are planar -connected and dual to each other.
     We choose  to be the clockwise next vertex after  on the outer face of  and compute (using Corollary~\ref{cor:goodCase}) a shelled box-contact representation  of , in which  is represented as the bounding box , , and  as , , i.e., these boxes constitute the back left, back right and top facets of , respectively.

     Next we show how to create a frame for each facial -cycle .
     Let  be the vertices of  in cyclic order.
     Assume without loss of generality that  and .
     Thus  and  are crossing edges of  and , respectively.
     In the Schnyder wood of  underlying Corollary~\ref{cor:goodCase} exactly one of ,  is uni-directed, say  is uni-directed in tree .
     Then there is a point in  in common with all four boxes in  corresponding to vertices of .
     Moreover, by Lemma~\ref{lem:box-schny} boxes  touch box  with their  facets, respectively; see Fig.~\ref{fig:frame_from_good_case}.
     Now we can increase the lower -coordinate of the box  by some  so that  and  lose contact and between these two boxes a cubic frame  with side length  is created; see again Fig.~\ref{fig:frame_from_good_case}.
     Note that by Lemma~\ref{lem:box-schny} the  facet of  makes contact only with  and hence if  is small enough all other contacts in  are maintained.
     We apply this operation to each  and obtain a shelled box-representation  of .

     \begin{figure}[htb]
\centering
      \includegraphics{frame_from_good_case}
      \caption{Creating a frame  for an inner facial cycle  of  by releasing the contact between  and .}
      \label{fig:frame_from_good_case}
     \end{figure}

     Finally, we show how to modify  to obtain an \LL-representation of  fitting the given frame .
     If (\textbf{Case~2.1})  has type , we define a new box for  to be .
     For each white neighbor of  we union the corresponding box with another box that is contained in  with bottom facet at  so that the result is an \LL-shape.
     For each black neighbor of  we set the lower -coordinate of the corresponding box to ; see Fig.~\ref{fig:good_case_into_frame}.
     (This requires the proper contacts for outer edges of , except for , to be parallel to the -plane, which we can easily guarantee.)
We then apply an affine transformation mapping  onto .
     If (\textbf{Case~2.2})  has type , we define a new box for  to be  and apply an affine transformation mapping  onto .
In both cases we have an \LL-representation of  fitting .
\end{description}\vspace{-0.7cm}
 \end{claimproof}

     \begin{figure}[htb]
      \centering
      \includegraphics{good_case_into_frame}
\caption{Modifying  when  has type  (Case~2.1) to
find a representation fitting .}
      \label{fig:good_case_into_frame}
     \end{figure}


By the claim above we can compute an \LL-representation  of  fitting any given
 frame  for  in  time.
Moreover,  has a set of disjoint frames .
Following step~\ref{step:G_in} of {\drawOpt},
for each , let  be the almost-optimal 1-planar graph given by all vertices
 and edges of  on and strictly inside .
Recursively applying the claim we can compute an \LL-representation  of 
 fitting the frame  for  in .
Clearly,  is an \LL-representation
 of  fitting .
We pick a frame  of arbitrary type for  to complete the construction.
\end{proof}



\begin{comment}

\begin{figure}[t]
  \begin{subfigure}[t]{.4\textwidth}
    \centering
    \includegraphics[height=3.5cm]{cut-box2a}
    \caption{}
    \label{fig:cut-boxa}
  \end{subfigure}
  \hspace{2em}
  \begin{subfigure}[t]{.4\textwidth}
    \centering
    \includegraphics[height=3.5cm]{cut-box2b}
    \caption{}
    \label{fig:cut-boxb}
  \end{subfigure}
    \caption{(a,b) Creating a  open box in a representation of an optimal 1-planar graph.}
\end{figure}

\end{comment}



\section{Conclusion and Open Questions}
In this paper we presented new results about primal-dual contact representations in 3D. In particular, we showed that a 3-connected planar graph and its dual has a box-contact representation and that an optimal 1-planar graph has an \LL-contact representation. Many interesting problems remain open.


\begin{enumerate}
\item Representing graphs with contacts of constant-complexity 3D shapes, such as \LLs, is open for many graph
classes with a linear number of edges, such as -planar graphs, quasi-planar graphs and other nearly planar
graphs. In particular, does there exist an \LL-contact representation of every -planar graph?

\item In 2D, a planar graph has a contact representation with rectangles if and only if it
contains no separating triangle.
Which graphs have 3D box-contact representations?

\item It is known that any planar graph admits a proper contact representation
with boxes in 3D and a non-proper contact representation with cubes (boxes with equal sides).
Does every planar graph admit a proper contact representation with cubes?

\item Given an orthogonal surface  corresponding to the Schnyder wood of a -connected plane graph, how can one extend  into a primal-dual box-contact representation using just topological properties of ?
\end{enumerate}


\section*{Acknowledgments}
We thank Michael Bekos, Therese Biedl, Franz Brandenburg, Michael Kaufmann, Giuseppe Liotta for useful discussions about different variants of these problems.





\begin{thebibliography}{10}

\bibitem{And70}
E.~Andreev.
\newblock On convex polyhedra in lobachevskii spaces.
\newblock {\em Mat. Sbor.}, 123(3):445--478, 1970.

\bibitem{AH95}
D.~Avis and M.~E. Houle.
\newblock Computational aspects of {H}elly's theorem and its relatives.
\newblock {\em International Journal of Computational Geometry and
  Applications}, 5(4):357--367, 1995.

\bibitem{BBGDF+07}
M.~Badent, C.~Binucci, E.~D. Giacomo, W.~Didimo, S.~Felsner, F.~Giordano,
  J.~Kratochv{\'\i}l, P.~Palladino, M.~Patrignani, and F.~Trotta.
\newblock Homothetic triangle contact representations of planar graphs.
\newblock In {\em CCCG}, pages 233--236, 2007.

\bibitem{BBC11}
M.~Badent, U.~Brandes, and S.~Cornelsen.
\newblock More canonical ordering.
\newblock {\em Journal of Graph Algorithms and Applications}, 15(1):97--126,
  2011.

\bibitem{BDEG14}
M.~J. Bannister, W.~E. Devanny, D.~Eppstein, and M.~T. Goodrich.
\newblock The {G}alois complexity of graph drawing: Why numerical solutions are
  ubiquitous for force-directed, spectral, and circle packing drawings.
\newblock In {\em Graph Drawing}, pages 149--161, 2014.

\bibitem{BF12}
O.~Bernardi and E.~Fusy.
\newblock Schnyder decompositions for regular plane graphs and application to
  drawing.
\newblock {\em Algorithmica}, 62(3--4):1159--1197, 2012.

\bibitem{Bez05}
A.~Bezdek.
\newblock On the number of mutually touching cylinders.
\newblock {\em Combinatorial and Computational Geometry}, 52:121--127, 2005.

\bibitem{BR13}
K.~Bezdek and S.~Reid.
\newblock Contact graphs of unit sphere packings revisited.
\newblock {\em Journal of Geometry}, 104(1):57--83, 2013.

\bibitem{BSW84}
R.~Bodendiek, H.~Schumacher, and K.~Wagner.
\newblock {\"U}ber 1-optimale graphen.
\newblock {\em Mathematische Nachrichten}, 117(1):323--339, 1984.

\bibitem{BEGGH+13}
F.~Brandenburg, D.~Eppstein, A.~Glei{\ss}ner, M.~Goodrich, K.~Hanauer, and
  J.~Reislhuber.
\newblock On the density of maximal 1-planar graphs.
\newblock In {\em Graph Drawing}, pages 327--338, 2013.

\bibitem{BEFHK+12}
D.~Bremner, W.~Evans, F.~Frati, L.~Heyer, S.~Kobourov, W.~Lenhart, G.~Liotta,
  D.~Rappaport, and S.~Whitesides.
\newblock Representing graphs by touching cuboids.
\newblock In {\em Graph Drawing}, pages 187--198, 2012.

\bibitem{BGGMT+05}
G.~Brinkmann, S.~Greenberg, C.~Greenhill, B.~McKay, R.~Thomas, and P.~Wollan.
\newblock Generation of simple quadrangulations of the sphere.
\newblock {\em Discrete Math.}, 305(1--3):33--54, 2005.

\bibitem{BGPV08}
A.~L. Buchsbaum, E.~R. Gansner, C.~M. Procopiuc, and S.~Venkatasubramanian.
\newblock Rectangular layouts and contact graphs.
\newblock {\em ACM Transactions on Algorithms}, 4(1), 2008.

\bibitem{CKU13}
S.~Chaplick, S.~G. Kobourov, and T.~Ueckerdt.
\newblock Equilateral {L}-contact graphs.
\newblock In {\em Graph-Theoretic Concepts in Computer Science (WG)}, pages
  139--151, 2013.

\bibitem{CS03}
C.~R. Collins and K.~Stephenson.
\newblock {A circle packing algorithm}.
\newblock {\em Computational Geometry: Theory and Applications},
  25(3):233--256, 2003.

\bibitem{CKU98}
J.~Czyzowicz, E.~Kranakis, and J.~Urrutia.
\newblock A simple proof of the representation of bipartite planar graphs as
  the contact graphs of orthogonal straight line segments.
\newblock {\em Information Processing Letters}, 66(3):125--126, 1998.

\bibitem{FM01}
H.~de~Fraysseix and P.~O. de~Mendez.
\newblock On topological aspects of orientations.
\newblock {\em Discrete Mathematics}, 229(1--3):57--72, 2001.

\bibitem{FM07a}
H.~de~Fraysseix and P.~O. de~Mendez.
\newblock Representations by contact and intersection of segments.
\newblock {\em Algorithmica}, 47(4):453--463, 2007.

\bibitem{FMP91}
H.~de~Fraysseix, P.~O. de~Mendez, and J.~Pach.
\newblock Representation of planar graphs by segments.
\newblock {\em Intuitive Geometry}, 63:109--117, 1991.

\bibitem{FMR94}
H.~de~Fraysseix, P.~O. de~Mendez, and P.~Rosenstiehl.
\newblock On triangle contact graphs.
\newblock {\em Combinatorics, Probability and Computing}, 3:233--246, 1994.

\bibitem{FPP90}
H.~de~Fraysseix, J.~Pach, and R.~Pollack.
\newblock How to draw a planar graph on a grid.
\newblock {\em Combinatorica}, 10(1):41--51, 1990.

\bibitem{FM07}
I.~Fabrici and T.~Madaras.
\newblock The structure of 1-planar graphs.
\newblock {\em Discrete Mathematics}, 307(7--8):854--865, 2007.

\bibitem{FF11}
S.~Felsner and M.~C. Francis.
\newblock Contact representations of planar graphs with cubes.
\newblock In {\em Symposium on Computational Geometry}, pages 315--320, 2011.

\bibitem{FZ08}
S.~Felsner and F.~Zickfeld.
\newblock Schnyder woods and orthogonal surfaces.
\newblock {\em Discrete \& Computational Geometry}, 40(1):103--126, 2008.

\bibitem{GHK10}
E.~R. Gansner, Y.~Hu, and S.~G. Kobourov.
\newblock Visualizing graphs and clusters as maps.
\newblock In {\em IEEE Computer Graphics and Applications}, pages 2259--2267,
  2010.

\bibitem{GLP12}
D.~Gon\c{c}alves, B.~L{\'e}v{\^e}que, and A.~Pinlou.
\newblock Triangle contact representations and duality.
\newblock {\em Discrete \& Computational Geometry}, 48(1):239--254, 2012.

\bibitem{Hli98}
P.~Hlin\v{e}n\'{y}.
\newblock Classes and recognition of curve contact graphs.
\newblock {\em Journal of Combinatorial Theory Series B}, 74(1):87--103, 1998.

\bibitem{HK01}
P.~Hlin\v{e}n\'{y} and J.~Kratochv{\'\i}l.
\newblock Representing graphs by disks and balls (a survey of
  recognition-complexity results).
\newblock {\em Discrete Mathematics}, 229(1--3):101--124, 2001.

\bibitem{HK98}
D.~House and C.~Kocmoud.
\newblock Continuous cartogram construction.
\newblock In {\em Visualization}, pages 197--204, 1998.

\bibitem{Kan96}
G.~Kant.
\newblock Drawing planar graphs using canonical ordering.
\newblock {\em Algorithmica}, 16(1):4--32, 1996.

\bibitem{KUV13}
S.~G. Kobourov, T.~Ueckerdt, and K.~Verbeek.
\newblock Combinatorial and geometric properties of planar {L}aman graphs.
\newblock In {\em Symposium on Discrete Algorithms}, pages 1668--1678, 2013.

\bibitem{Koe36}
P.~Koebe.
\newblock {K}ontaktprobleme der konformen {A}bbildung.
\newblock {\em Berichte {\"u}ber die Verhandlungen der S{\"a}chsischen Akad.
  der Wissen. zu Leipzig. Math.-Phys. Klasse}, 88:141--164, 1936.

\bibitem{KM94}
J.~Kratochv{\'\i}l and J.~Matousek.
\newblock Intersection graphs of segments.
\newblock {\em Journal of Combinatorial Theory Series B}, 62(2):289--315, 1994.

\bibitem{MCP02a}
J.~Michalek, R.~Choudhary, and P.~Papalambros.
\newblock Architectural layout design optimization.
\newblock {\em Engineering Optimization}, 34(5):461--484, 2002.

\bibitem{Moh97}
B.~Mohar.
\newblock Circle packings of maps in polynomial time.
\newblock {\em European Journal of Combinatorics}, 18(7):785--805, 1997.

\bibitem{PT97}
J.~Pach and G.~T{\'o}th.
\newblock Graphs drawn with a few crossings per edge.
\newblock {\em Combinatorica}, 17:427--439, 1997.

\bibitem{Rai34}
E.~Raisz.
\newblock The rectangular statistical cartogram.
\newblock {\em Geographical Review}, 24(2):292--296, 1934.

\bibitem{Rin65}
G.~Ringel.
\newblock Ein sechsfarbenproblem auf der kugel.
\newblock {\em Abhandlungen aus dem Mathematischen Seminar der Universitat
  Hamburg}, 29(1--2):107--117, 1965.

\bibitem{RT86}
P.~Rosenstiehl and R.~E. Tarjan.
\newblock Rectilinear planar layouts and bipolar orientations of planar graphs.
\newblock {\em Discrete \& Computational Geometry}, 1(1):343--353, 1986.

\bibitem{Sch90}
W.~Schnyder.
\newblock Embedding planar graphs on the grid.
\newblock In {\em Symposium on Discrete Algorithms}, pages 138--148, 1990.

\bibitem{Sch86}
H.~Schumacher.
\newblock Zur struktur 1-planarer graphen.
\newblock {\em Math.~Nachrichten}, 125:291--300, 1986.

\bibitem{Shn92a}
B.~Shneiderman.
\newblock Tree visualization with tree-maps: A {2-D} space-filling approach.
\newblock {\em ACM Transactions on Graphics}, 11(1):92--99, 1992.

\bibitem{Suz10}
Y.~Suzuki.
\newblock Re-embeddings of maximum 1-planar graphs.
\newblock {\em SIAM Journal on Discrete Mathematics}, 24(4):1527--1540, 2010.

\bibitem{Tho88}
C.~Thomassen.
\newblock Interval representations of planar graphs.
\newblock {\em Journal of Combinatorial Theory Series B}, 40(1):9--20, 1988.

\bibitem{Tob04a}
W.~Tobler.
\newblock Thirty five years of computer cartograms.
\newblock {\em Annals of the Association of American Geographers}, 94:58--73,
  2004.

\bibitem{Ull84}
J.~D. Ullman.
\newblock {\em Computational Aspects of {VLSI}}.
\newblock Computer Science Press, 1984.

\end{thebibliography}
 


\end{document}
