

\documentclass[preprint,12pt,authoryear]{elsarticle} 

\usepackage{bbm}
\usepackage{fullpage}
\usepackage{amsmath}
\usepackage{mathrsfs}
\usepackage{subfig}
\usepackage[hidelinks]{hyperref}
\hypersetup{colorlinks,bookmarksopen,linkcolor=blue,citecolor=blue,urlcolor=blue}

\newcommand{\bs}[1]{{\boldsymbol{#1}}}
\newcommand{\del}[2]{\mbox{}}
\newcommand{\ppd}[2]{\del{\partial{#1}}{\partial{#2}}}
\newcommand{\od}[2]{\del{{\rm d}\>\!{#1}}{{\rm d}{#2}}}







\usepackage{amssymb}




\journal{arxiv}

\begin{document}
	
\begin{frontmatter}
		




\title{The response of grandstands driven by filtered Gaussian white noise processes}



\author[affiliation1]{Ond\v{r}ej Roko\v{s}\corref{mycorrespondingauthor}}
\cortext[mycorrespondingauthor]{Corresponding author.}
\ead{ondrej.rokos@fsv.cvut.cz}

\author[affiliation1]{Ji\v{r}\'{i} M\'{a}ca}
\ead{maca@fsv.cvut.cz}
		
\address[affiliation1]{Department of Mechanics, Faculty of Civil Engineering, Czech Technical University in Prague, Th\'{a}kurova~7, 166~29 Prague~6, Czech Republic.}
		
\begin{abstract}
This paper presents a semi-analytical estimate of the response of a grandstand occupied by an active crowd and by a passive crowd. Filtered Gaussian white noise processes are used to approximate the loading terms representing an active crowd. Lumped biodynamic models with a single degree of freedom are included to reflect passive spectators occupying the structure. The response is described in terms of the first two moments, employing the It\^{o} formula and the state augmentation method for the stationary time domain solution. The quality of the approximation is compared on the basis of three examples of varying complexity using Monte Carlo simulation based on a synthetic generator available in the literature. For comparative purposes, there is also a brief review of frequency domain estimates.
\end{abstract}
		
\begin{keyword}
grandstand response \sep random vibration \sep active crowd \sep white noise process \sep filtration




\end{keyword}
		
\end{frontmatter}
	




\section{Introduction}
\label{sectintroduction}
The response of a grandstand can be resolved quite easily by linear dynamics methods, if we neglect all the randomness of the system. However, for a more accurate description, at least the most significant uncertainties need to be taken into account. The main uncertainties include
\begin{itemize}
	\item forcing terms resulting from active crowd movements, especially synchronized jumping,
	\item the uncertainties of the parameters in discrete biodynamic models---randomness of stiffness, mass and damping matrices,
	\item the size and spatial distribution of an active crowd and a passive crowd.
\end{itemize}
Further generalizations can take into account various kinds of nonlinearities, e.g. geometrical and material non-linearities and non-linearities of biodynamic models \citep{Huang}. However, the following restrictions will be assumed for the purposes of this paper: the material parameters of the structure are treated as deterministic, since their influence is negligible in comparison with the sources listed above and the scope of the overall response; the spatial distribution of the crowd is fixed; biodynamic models of the passive crowd are treated as deterministic.

The results of measurements carried out on simple structures indicate that an active spectator might be represented as a time dependent force, \textit{cf} \citep{Ellis}. In other words, under certain circumstances, an active spectator does not influence the properties of the structure. Several generators of normalized artificial load processes have been developed in the literature, e.g. \citep{Sim} and \citep{Racic}, which can be supplemented by human body weights, e.g. \citep{Hermanussen}. However, the passive part of the crowd is assumed to be stationary in space and in permanent contact with the structure, excluding accelerations that exceed the gravity of the earth. Thus a passive spectator can be modeled as a biodynamic system. A survey of models of this type can be found in \citep{Sachse}. These remarks allow us to model the structure considered here in terms of random vibrations of linear systems.

Monte Carlo simulation (MC) is one of the ways used in advanced grandstand design procedures for reflecting all kinds of randomness. MC is a general method, but it has considerable disadvantages in its original form, e.g. rather slow convergence when estimating low probabilities, and high computational demands. MC will be employed in this paper mainly for controlling the accuracy and the performance of semi-analytical methods introduced in terms of stochastic differential equations. These simplified methods can be applicable and useful in the preliminary stages of design procedures, when quick and approximate solutions are sufficient.

\section{Stochastic differential equations}
\label{SDE}
Employing the Finite Element Method, and taking into account the above-mentioned assumptions, a mathematical model of the structural system considered here can be written as a set of hyperbolic differential equations

where  and  are  and -valued stochastic processes, and , ,  and  are  and  matrices of mass, damping, stiffness and input distribution. Over-dot denotes a derivative with respect to time . To simplify subsequent expressions, let us apply expectation operator  in equation (\ref{2eq1}) and substract the result from (\ref{2eq1}). We arrive at the set of two equations

for the mean value  and centered process ,  and . As will become apparent in section \ref{appToGrandstands}, under certain conditions process  is approximately normal, and since differential equation (\ref{2eq3}) is linear with deterministic coefficients, it is reasonable to accept Gaussian approximation also for . This consideration leads us to stochastic differential equations and to the It\^{o} calculus. Under Gaussian assumptions, the response will be completely specified by its mean  and covariance . The required quantities can be obtained from the time domain or from the frequency domain.

\subsection{Solution in the time domain}
\label{timedomain}
Equation (\ref{2eq2}) can be solved by direct integration or, more conveniently, by a Fourier series (or Fourier transform) assuming periodic mean , \textit{cf} section \ref{frequencydomain}. Let us rewrite (\ref{2eq3}) as

or, in a more compact form,

where  is an -valued state-space vector stochastic process with zero mean, and  and  are  and -matrices. The solution of this differential equation is given in the form

where  denotes a Green function or the unit impulse response satisfying

 the identity and  can be expressed as a matrix exponential, \textit{cf} \citep{Soong}. Initial conditions  will be set to zero for simplicity. Forcing term  can also satisfy its own stochastic differential equation driven by Gaussian white noise . For example, let  be a continuous-time Gaussian auto-regression scalar process of order , denoted as , \textit{cf} \citep{Brockwell}. Then  where the state vector  satisfies the It\^{o} equation


Processes of this kind are also called filtered white noise processes or colored processes, and they have a specific frequency content. Let us assume that  of   are mutually independent  processes. Then we can merge equations (\ref{2eq4}) and (\ref{2eq8}) to obtain one coupled system

where  are column vectors with the unit in -th position, and  is an -valued Brownian motion. This approach is called a state augmentation method \citep{Grigoriu_stoch}. An extension to the case , where  are mutually independent  processes, is carried out in an obvious manner. This methodology will be employed in section \ref{appToGrandstands} for  processes. Equation (\ref{2eq9}) can again be rewritten in compact form

and employing the It\^{o} formula for semimartingales we arrive at the system of evolutionary equations for the response mean  and covariance 

Since the driving forces  are Gaussian white noise and the coefficients are constant in time, the solution is an Ornstein-Uhlenbeck process with an existing stationary solution. In our case, stationary mean  and covariance  which leads to the so-called continuous Lyapunov equation

For details and further developments, see \citep{Grigoriu_stoch}. Since the stationary matrix  contains only response displacements and velocities, the variances of the acceleration have to be computed through the following formulas which are valid for weakly stationary processes

where  denotes matrix power and  denotes the stationary covariance matrix of velocities and accelerations. Equation (\ref{2eq14a}) is evaluated employing (\ref{2eq13}), which in our special case is simplified to

\subsection{Solution in the frequency domain}
\label{frequencydomain}
Taking the Fourier transform of equation (\ref{2eq2}) leads to

where the FRF (Frequency Response Function)  is

 denotes a complex unit,  denotes angular frequency and  denotes the Fourier transform of function . Assuming  in periodic form  where ,  and  are amplitude, phase shift and angular frequency of the -th harmonic, we obtain the mean response as a solution of the complex linear system of equations.

Employing spectral decomposition of stationary random processes, the response variances are acquired in terms of spectral density matrices  and , , where  is a spectral density estimate of the centered forcing term

 is some weight function, \textit{cf} \citep{Andel}, and  denotes the corresponding periodogram

The diagonal form of  suggests that we treat all input processes as independent. Knowing the spectral density matrix of the input vector stochastic process, we obtain the spectral density matrix of the output process according to \cite{Soong}

where  denotes a Hermitian transpose to . The variance of the stationary scalar process  with two-sided spectral density  or with one-sided spectral density  is evaluated as

and the variance of time derivative 

By analogy for higher time derivatives, applying higher powers of angular frequency .


\subsection{Transformation to modal coordinates}
\label{modalsol}
It is desirable to reduce the system unknowns in equations (\ref{2eq1}), (\ref{2eq2}), (\ref{2eq14}) and (\ref{2eq19}) for MC, mean value, time domain and frequency domain solutions. The usual modal transform technique can be employed directly when a structure with proportional damping is loaded by active spectators only. Some problems arise when the passive part of the crowd is introduced. All matrices are extended according to the added degrees of freedom (dofs). Biodynamic models possess non-proportional damping, so complex eigenvectors should be used. Simultaneously, all these models have the first eigenfrequency close to 5 Hz (and the second eigenfrequency close to 8 Hz if using two-degrees-of-freedom models). This implies that the number of eigenvectors employed in the transform is considerably enlarged. Another possibility is partial modal transformation. The system matrices can be decomposed into four parts according to

where  stands either for mass, stiffness or damping matrix. Sub-matrix  corresponds to the structure,  to the passive crowd, and  or  represents the mutual interactions. Let us note that sub-matrix  is a diagonal or a band, and  is sparse. Computing the eigenvectors corresponding to an empty structure, i.e. corresponding to , arranged in  and writing

all equations can be partially transformed by , as in the standard procedure. Then the dofs inherent to a passive crowd are unchanged, but the structure response is described through several modal coordinates.

Concerning the Lyapunov equation (\ref{2eq14}) partially transformed into modal coordinates, we can further reduce the computational effort by some prior information. Let us assume that the system response corresponds to the equation (\ref{2eq9}) with . Splitting all matrices in (\ref{2eq14}) leads to

where

 and remaining sub-matrices have obvious structure. Dropped subscript at covariance matrix  emphasizes partial transformation to modal coordinates. Since  resp.  are symmetric, in fact diagonal, the solution will be also symmetric, . In the case of  processes, sub-matrix  can be computed explicitly; single  process has uncorrelated state variables ,  and , , based on moment equations, thus  is a diagonal matrix. Introduced considerations reduce the system of four equations (\ref{modal3}) in expanded form, to the set of two coupled equations

for unknowns  and . The set resembles Sylvester and Lyapunov equations respectively with reduced size. Backward transformation  gives covariance matrix for displacement or velocity vector, here  denotes an appropriate sub-matrix of  storing the modal displacements or velocities,  then contains the nodal displacements or velocities.
\subsection{Crossings of Gaussian processes}
\label{upcrossing}
One of the measures of system performance is level crossing. Under some circumstances, in a stationary case, it can be shown that up-crossing is directly connected with the reliability of the system. The -up-crossing rate of Gaussian process  with non-stationary mean value  and stationary variance  is estimated as \citep{Soong}

where  is the -up-crossing rate of level  at time , , , , . The total mean number of upcrossings in time interval  is computed according to

The relations can be generalized to -out-crossings of a -valued stochastic process, where  is some set in .

Another measure of system performance from the point of view of serviceability is the root mean square value estimated as

where  denotes the stationary variance of centered process , and  its mean value. Analogous formulas are valid for velocity and acceleration.
\section{Applications to the response of grandstands}
\label{appToGrandstands}
As was noted in section \ref{sectintroduction}, an active spectator can be treated as a time-dependent process. Figure \ref{3fig1} shows a single realization and the spectral density of a unit process, i.e. of the process with , where  denotes the weight of a spectator, jumping frequency  Hz. Spectral densities were computed from (\ref{2eq17}) with Parzen weight. The realization was generated according to \cite{Sim}.
\begin{figure}
	\centering
  \subfloat[]{\includegraphics[scale=0.7]{fig1a.pdf}}
  \subfloat[]{\includegraphics[scale=0.7]{fig1b.pdf}}
	\caption{Single time history (a) and power spectral density of 10~000 realizations (b) of forcing term  generated according to \cite{Sim}.}
	\label{3fig1}
\end{figure}
Since this function is highly periodic, we will search the mean value in the form

Then vector  of the estimated parameters  can be found by the linear Least Squares Method as

where 

 is a fine enough and equidistant partition of the time interval,  with

are means over  realizations  in time instants . Generating 10 000 trajectories provides the coefficients given in tab. \ref{3tab1}.
\begin{table}
	\centering
	\caption{Coefficients  for an approximation of the mean value in equation (\ref{3eq1}) for  Hz.}
	\begin{tabular}{|c|c|c|c|}\hline
	 & \multicolumn{3}{c|}{0.9958} \\\hline
	 & 0.2939 &  & 1.1170 \\
	 & -0.2471 &  & 0.0984 \\
	 & -0.0037 &  & -0.0153 \\
	 & -0.0008 &  & -0.0001 \\\hline
	\end{tabular}
	\label{3tab1}
\end{table}
Realization centered with the mean value according to equation (\ref{3eq1}) and the coefficients from table \ref{3tab1} is  depicted in figure \ref{3fig2}, together with the spectral density and a normalized histogram, i.e. a histogram of the process  where  is a stationary standard deviation.
\begin{figure}
	\centering
  \subfloat[]{\includegraphics[scale=0.7]{fig2a.pdf}}
  \subfloat[]{\includegraphics[scale=0.7]{fig2b.pdf}}\\
  \subfloat[]{\includegraphics[scale=0.7]{fig2c.pdf}}
  \subfloat[]{\includegraphics[scale=0.7]{fig2d.pdf}}
	\caption{Single centered time history of  (a), the corresponding power spectral density of 10~000 realizations (b), a normalized histogram with standard normal density (c) and a normalized histogram of a non-unit process scaled with  (d).}
	\label{3fig2}
\end{figure}
The centered process resembles non-Gaussian colored noise. To be more precise, for  we have  for the mean,  for the variance,  for the coefficient of skewness, and  for the coefficient of kurtosis. For comparison, the standard Gaussian process has coefficients , ,  and .

Let us briefly analyze the response of a harmonic oscillator with unit mass forced by jumping process  with  Hz, employing MC to justify the normality assumptions. The coefficients of skewness and kurtosis of the state vector  as functions of the oscillator eigenfrequency  for two different values of viscous damping  are depicted in figure \ref{3fig3}.
\begin{figure}
	\centering
  \includegraphics[scale=0.7]{fig3.pdf}
	\caption{Coefficient of skewness  and coefficient of kurtosis  for the normalized displacement and velocity of the harmonic oscillator forced by the jumping process as functions of eigenfrequency . Solid line - viscous damping ; dash-dot line - ; dashed line - corresponding values for the Gaussian random variable.}
	\label{3fig3}
\end{figure}
Note that for frequency range  Hz the response is approximately Gaussian. As was expected, worse convergence is achieved for higher damping values, \textit{cf} the Rosenblatt theorem \citep{Grigoriu_nong}. Normalized histograms of displacement for eigenfrequencies  and ~Hz and both damping values are depicted in figure \ref{3fig4}. Other techniques can be applied for approximations of the response outside this frequency range, e.g. memoryless transformations of Brownian colored noise, but this lies beyond the scope of our paper. Based on heuristic arguments and the Central Limit Theorem, we can assume that the higher the number of active spectators, and the more complex the grandstand geometry is, the more Gaussian the response will be.

\begin{figure}
	\centering
  \subfloat[ Hz, ]{\includegraphics[scale=0.7]{fig4a.pdf}}
  \subfloat[ Hz, ]{\includegraphics[scale=0.7]{fig4b.pdf}}\\
  \subfloat[ Hz, ]{\includegraphics[scale=0.7]{fig4c.pdf}}
  \subfloat[ Hz, ]{\includegraphics[scale=0.7]{fig4d.pdf}}
	\caption{Histograms of normalized displacement  with standard normal density based on 1000 MC realizations.}
	\label{3fig4}
\end{figure}


A spectral density approximation of the forcing process for the frequency domain solution, figure \ref{3fig2} (b), cannot be further simplified. This is because the FRF of the structure has sharp peaks, and thus exact function values are needed. Any approximation employing indicator functions in the vicinities of significant harmonics preserving variance would be inaccurate.

However, we can employ filtered white noise processes , which arise as a solution of the second order It\^{o} equation

with spectral density

where  correspond to the stiffness, mass and damping of a harmonic oscillator. This function has a sharp peak positioned at  when we neglect shifts due to damping effects. For a closer approximation, we assume , where  are mutually independent  processes. Identification leads to a nonlinear optimization problem: find such ,  and , that minimize  norm  of the difference 

where  denotes a spectral density estimate of the centered force term . The problem can be solved by the Nonlinear Least Squares method, by Simulated Annealing etc, with easily estimated initial vector . Optimized coefficients for  of the centered process  are presented in table \ref{3tab2}, and the corresponding spectral density and spectral distribution function are presented in figure \ref{3fig5}. Note that the spectral density is two-sided, and only one half was integrated in the spectral distribution function, thus the variance indicated is .

The performance of the structure was briefly quantified in section \ref{upcrossing}, where the two expressions (\ref{2eq22}) and (\ref{2eq24}) depended on the response variance. Thus an alternative approach is to optimize the response variance directly across some eigenfrequency range of a harmonic oscillator. Such coefficients are summarized in table \ref{3tab2}, with indices , spectral density and distribution function are depicted in figure \ref{3fig5}, frequency range  Hz.
\begin{table}
	\centering
	\caption{Coefficients ,  and  of the six independent  members used for approximation of the centered forcing term  in frequency range  Hz.}
	\begin{tabular}{|c|c|c|c|c|}\hline
	 &  &  &  &  \\\hline
	1 & 90.8657 & 0.3227 & 0.0148 & 2.67 \\
	2 & 35.9464 & 0.1276 & 0.1167 & 2.67 \\
	3 & 283.6701 & 0.2520 & 0.0118 & 5.34 \\
	4 & 74.1544 & 0.0661 & 0.0737 & 5.33 \\
	5 & 913.9890 & 1.1120 & 21.5186 & 4.56 \\
	6 & 228.5270 & 0.0907 & 0.1576 & 7.99 \\\hline
  1 & 91.0909 & 0.3237 & 0.0076 & 2.67 \\
	2 & 40.3066 & 0.1430 & 0.0804 & 2.67 \\
	3 & 281.0462 & 0.2490 & 0.0209 & 5.35 \\
	4 & 83.7399 & 0.0746 & 0.0573 & 5.33 \\
	5 & 914.0015 & 0.9376 & 21.6921 & 4.97 \\
	6 & 228.7642 & 0.0908 & 0.1552 & 7.99 \\\hline
	\end{tabular}
	\label{3tab2}
\end{table}
\begin{figure}
	\centering
	\subfloat[spectral density]{\includegraphics[scale=0.7]{fig5a.pdf}}
  \subfloat[spectral distribution function]{\includegraphics[scale=0.7]{fig5b.pdf}}
	\caption{Spectral density and spectral distribution function of centered forcing  and of its approximation , where  are independent  processes with the coefficients in table~\ref{3tab2} based on spectral and variance optimization.}
	\label{3fig5}
\end{figure}

Let us briefly note the situation when the forcing process  is not a unit process, i.e. . Since we are limiting our considerations to Gaussian approximation, only the first two moments of  will apply. The deterministic weight corresponds to a singular case . Then the forcing term has the form , mean response  and stationary response variance , where  and  are the response mean and the variance of the structure loaded by the unit forcing term . However we should be aware that even when process  was a Gaussian, process  as a product of a random variable with a stochastic process, is not Gaussian. To quantify the influence of such scaling, compare the histograms in figure \ref{3fig2} (c) and \ref{3fig2} (d), where  has normal distribution with mean value  and variance .
\section{Numerical examples and comparison}
\label{examples}
In this section, the quality of the approximation in the time or frequency domain will be compared with MC simulation.
\begin{figure}
	\centering
	\begin{tabular}{ccc}
  \includegraphics[scale=0.6]{fig6a.pdf} &
  \includegraphics[scale=0.3]{fig6b.pdf} &
  \includegraphics[scale=0.4]{fig6c.pdf} \\
  \includegraphics[scale=0.5]{fig6d.pdf} &
  \includegraphics[scale=0.5]{fig6e.pdf} &
  \includegraphics[scale=0.5]{fig6f.pdf}
  \end{tabular}
	\caption{Tested structures, geometries with labeled points of interest and normed histograms of the displacement, structures occupied by an active crowd only.}
	\label{4fig1}
\end{figure}
A total of four mechanical systems will be tested: a harmonic oscillator, a simply supported beam, a simple cantilever grandstand, and a realistic grandstand with 1, 4, 72 and 630 positions for spectators, respectively. Geometries with centered normed response histograms for an active crowd only, based on MC simulation, are depicted in figure \ref{4fig1}, and several lowest eigenfrequencies corresponding to the vertical bending modes are presented in table \ref{4tab1}. Note also that assumptions on convergence to normal distribution are approximately fulfilled. All examples are artificial, not realistic, so the measured responses would be unacceptable. 
\begin{table}
	\centering
	\caption{First eigenvalues corresponding to vertical bending modes [Hz], \textit{cf} figure \ref{4fig1}.}
	\begin{tabular}{|c|c|c|c|c|c|c|c|c|}
		\hline
		Structure & f & f & f & f & f  & f & f & f \\\hline
		beam & 7.5 & 22.5 & 40.3 & --- & --- & --- & --- & --- \\
		cantilever & 5.4 & 7.0 & 8.2 & 25.1 & 28.4 & --- & --- & --- \\
		grandstand & 2.5 & 2.6 & 3.8 & 4.5 & 4.8 & 4.9 & 5.4 & 6.0 \\\hline
	\end{tabular}
	\label{4tab1}
\end{table}
\subsection{Harmonic oscillator}
\label{harmosc}
Let us assume a harmonic oscillator with unit mass, two values of viscous damping  and variable stiffness. The mean value response is presented in figure \ref{4fig2}. The response was acquired by direct integration of (\ref{2eq2}), starting at ~s. The approximation utilizes equation (\ref{3eq1}) and the coefficients in table \ref{3tab1}. A comparison of the total mean up-crossings  in the time interval , ~s, as functions of oscillator eigenfrequency  for two values of viscous damping  and  and for two fixed levels  and ~m are depicted in figure \ref{4fig3}, employing formulas (\ref{2eq22}) and (\ref{2eq23}). The size of time interval  is based on heuristic considerations about the average length of the musical compositions. The time domain solution is based on the sum of six independent  processes with the coefficients in table \ref{3tab2} for  optimization. The stationary response variances are computed according to formulas (\ref{modal5}), (\ref{modal6}) and (\ref{2eq14a}). The frequency domain approximation employs equations (\ref{2eq19}), (\ref{2eq20}) and (\ref{2eq21}). The spectral density estimate  of the centered input process has  values over the frequency range ~Hz, using variable division. Figure \ref{4fig3} shows that the results are roughly in agreement with MC in the frequency range ~Hz. Note that the total number of mean zero-up-crossings for a deterministic periodic function with frequency  Hz is , \textit{cf} figure \ref{4fig3} (c) and (d), where distinct plateaux are found. The results for MC are based on  realizations  s in length.
\begin{figure}
	\centering
	\includegraphics[scale=0.6]{fig7.pdf}
	\caption{Response mean displacement in comparison with an approximation based on the first four harmonics for a harmonic oscillator,  Hz .}
	\label{4fig2}
\end{figure}
\begin{figure}
	\centering
\subfloat[]{\includegraphics[scale=0.7]{fig8a.pdf}}
\subfloat[]{\includegraphics[scale=0.7]{fig8b.pdf}}\\
\subfloat[]{\includegraphics[scale=0.7]{fig8c.pdf}}
\subfloat[]{\includegraphics[scale=0.7]{fig8d.pdf}}\\
\subfloat[]{\includegraphics[scale=0.7]{fig8e.pdf}}
\subfloat[]{\includegraphics[scale=0.7]{fig8f.pdf}}
	\caption{Total mean up-crossings  and acceleration  values as functions of  for a harmonic oscillator, distinct levels  and viscous damping ,  s.}
	\label{4fig3}
\end{figure}
\subsection{A simply supported beam}
\label{beam}
The next example is a simply supported beam with Rayleigh damping  and  for the first two vertical modes and total mass ~kg. Poor approximations are anticipated, since the structure is quite stiff with high eigenfrequencies, see table \ref{4tab1} and the results for the harmonic oscillator. Two cases are studied: a structure occupied by an active crowd only; and a structure occupied by a mixed crowd. In the second case, the two left hand side positions are loaded by forces and the two right hand side positions are occupied by passive spectators. Deterministic biodynamic models according to Coermann are used. These are simple-degree-of-freedom oscillators with mass ~kg, stiffness ~kN/m and viscous damping ~kNs/m, eigenfrequency ~Hz, \textit{cf} \citep{Sachse}. Results are presented only for the labeled point in figure \ref{4fig1}. The total mean up-crossings for the first case  as a function of level  are depicted in figure \ref{4fig4} (a). The  values for acceleration are ~m/s for the MC solution, ~m/s for the frequency domain solution, and ~m/s for the time domain solution. All input processes are treated as independent, so matrix  in equation (\ref{2eq19}) has nonzero only diagonal entries. The mean value response is depicted in figure \ref{4fig4} (d) with a single realization (c). The total up-crossings for a mixed crowd are depicted in figure \ref{4fig4} (b), and the  values are ~m/s for the MC solution, ~m/s for the frequency domain solution, and ~m/s for the time domain solution. Let us also recall our assumption of fixed spatial distribution of the crowd, mass coefficient , where  denotes the total mass of passive spectators, and  denotes the total mass of the structure. Results for MC based on  realizations.
\begin{figure}
	\centering
\subfloat[active crowd only]{\includegraphics[scale=0.7]{fig9a.pdf}}
\subfloat[2 active, 2 passive spectators]{\includegraphics[scale=0.7]{fig9b.pdf}}\\
\subfloat[single realization]{\includegraphics[scale=0.7]{fig9c.pdf}}
\subfloat[response mean]{\includegraphics[scale=0.7]{fig9d.pdf}}
	\caption{Total mean up-crossings  of a simply supported beam as functions of , ~s for an active crowd (a) and for a mixed crowd (b), single realization for an active crowd (c) and mean response for an active crowd (d).}
	\label{4fig4}
\end{figure}
The approximate shape of the total up-crossings, especially for a mixed crowd, differ from the MC simulation, because of the non-Gaussian response due to high structure eigenfrequencies.
\subsection{Cantilever grandstand}
\label{cantilever}
This system has total mass ~t, geometry according to figure \ref{4fig1}, and is loaded with 72 active spectators in the first case, Rayleigh damping with  and  is used for the first two vertical modes. The total up-crossings of the response displacement are depicted in figure \ref{4fig5} (a), and single realization and the mean response are depicted in sub-figures (c) and (d),  accelerations ~m/s for the MC solution, ~m/s for the frequency domain solution, and ~m/s for the time domain solution. For 36 spectators chosen to be passive according to Coermann with uniformly random but fixed positions, the resulting up-crossings are depicted in sub-figure (b), mass coefficient . The acceleration  values in this case appear to be ~m/s for the MC solution, ~m/s for the frequency domain solution, and ~m/s for the time domain solution. The results for MC are again based on  realizations  s in length.
\begin{figure}
	\centering
\subfloat[active crowd only]{\includegraphics[scale=0.7]{fig10a.pdf}}
\subfloat[36 active, 36 passive spectators]{\includegraphics[scale=0.7]{fig10b.pdf}}\\
\subfloat[single realization]{\includegraphics[scale=0.7]{fig10c.pdf}}
\subfloat[response mean]{\includegraphics[scale=0.7]{fig10d.pdf}}
	\caption{Total mean up-crossings of a cantilever grandstand, active crowd (a) and mixed crowd (b), single realization (c) and mean response (d) for an active crowd.}
	\label{4fig5}
\end{figure}
\subsection{Realistic grandstand}
\label{grandstand}
In this concluding example, let us briefly examine the results acquired for a complex structure. The geometry is sketched in figure \ref{4fig1} with the point of interest labeled, total mass ~t, Rayleigh damping with  for the first and  for the sixth vertical mode used here. The results for the structure loaded by an active crowd only are depicted in figure \ref{4fig6} (a) (c) (d). The acceleration  values appear to be ~m/s for the MC solution, ~m/s for the frequency domain solution, and ~m/s for the time domain solution. In the second case,  positions are occupied by passive spectators according to Coermann, mass coefficient . The results for this case are presented in figure \ref{4fig6} (b),  accelerations ~m/s for the MC solution, ~m/s for the frequency domain solution, and ~m/s for the time domain solution. MC simulation based on  realizations.
\begin{figure}
	\centering
\subfloat[active crowd only]{\includegraphics[scale=0.7]{fig11a.pdf}}
\subfloat[315 active, 315 passive spectators]{\includegraphics[scale=0.7]{fig11b.pdf}}\\
\subfloat[single realization]{\includegraphics[scale=0.7]{fig11c.pdf}}
\subfloat[response mean]{\includegraphics[scale=0.7]{fig11d.pdf}}
	\caption{Total mean up-crossings of a realistic grandstand, active crowd (a), mixed crowd (b), single realization (c) and mean response (d) for an active crowd.}
	\label{4fig6}
\end{figure}
\subsection{Comparison and performance}
\label{performance}
The time consumption for the different solution techniques is summarized in table \ref{4tab2}, where the demands of MC simulation are presented for 50 realizations only. This value is based on the convergence tests of total up-crossings presented in figures \ref{4fig7} (a) and (b) for a cantilever grandstand. Obviously these tests are highly sensitive to level , and the  value used can be considered as a lower bound. The size of the time integration step is chosen to be ~s, Newmark integration scheme used. The number of dofs of each system is also mentioned, together with the size of the Lyapunov equation (\ref{2eq14})  for a mixed crowd. Obviously  where  denotes the number of dofs of the empty structure,  is the number of dofs of the passive spectators, and  is the number of active spectators. In the case of a cantilever grandstand we therefore have , since passive spectators are modeled as systems with a single degree of freedom. For the purposes of comparison, the time consumption for the partial modal transform are also summarized together with the number of eigenvectors used  and according highest frequency f. All simulations were performed on core i7 with a 16~GB RAM computer, Matlab\textsuperscript{\textregistered} parallel implementation.
\begin{table}
\centering
\caption{Comparison of computational demands.}
\begin{tabular}{|c|r|r|r|r|}\hline
	Method/Struct. & \multicolumn{1}{c|}{Harm. osc.} & \multicolumn{1}{c|}{Beam} & \multicolumn{1}{c|}{Cant. grand.} & \multicolumn{1}{c|}{Real. grand.} \\\hline
	\multicolumn{5}{|c|}{full system} \\\hline
   & 1 & 29 & 504 & 4\ 068 \\
	 & 14 & 86 & 1512 & 12\ 546 \\	
	MC 50 & 0.601 s & 20.078 s & 380.451 s & 5\ 484 s \\
	Freq. domain & 0.106 s & 2.672 s & 74.639 s & 4\ 178 s \\
	Time domain & 0.112 s & 2.392 s & 38.915 s & 6\ 507 s \\\hline
	\multicolumn{5}{|c|}{partial modal transform, \textit{cf} equations (\ref{modal4}), (\ref{modal5}) and (\ref{modal6})} \\\hline
	 & -- & 7 & 10 & 50 \\
	f & -- &  Hz &  Hz &  Hz \\
	 & -- & 42 & 524 & 4\ 510 \\
	MC 50 & -- & 13.240 s & 36.214 s & 307 s \\
	Freq. domain & -- & 1.764 s & 4.452 s & 734 s \\
	Time domain & -- & 1.588 s & 2.582 s & 67 s \\\hline
\end{tabular}
\label{4tab2}
\end{table}
\begin{figure}
	\centering
	\subfloat[]{\includegraphics[scale=0.7]{fig12a.pdf}}
	\subfloat[]{\includegraphics[scale=0.7]{fig12b.pdf}}
	\caption{MC convergence tests of  for a cantilever grandstand, \textit{cf} figure \ref{4fig5} (b).}
\label{4fig7}
\end{figure}
The results in the table suggest that when 50 realizations are employed, only structures of moderate size can be effectively solved by semi-analytical methods.
\section{Conclusions}
\label{concl}
This paper has presented a study of the vibration of grandstands loaded by an active crowd, using Gaussian approximation of the response. The main results can be summarized as follows:
\begin{enumerate}
	\item A mathematical description of the response of a mechanical system employing spectral and time domain solutions for weakly stationary Gaussian excitations has been recalled. Partial modal transformation due to a passive crowd has been briefly discussed.
	\item A motivating example of a harmonic oscillator has shown that the normalized displacement and velocity have approximately normal distribution under the conditions on eigenfrequencies and damping.
	\item Taking this fact into account, the mean value of the force has been approximated as a truncated Fourier series, and the spectral density of the centered process has been estimated employing the Parzen window for the frequency domain solution. For the time domain solution, we have employed a linear combination of independent auto-regression processes of the second order with coefficients optimized to achieve the least error in the response variance of a harmonic oscillator.
	\item Three different examples of varying complexity have shown the quality of the response approximation in terms of total displacement up-crossings and acceleration  in comparison with Monte Carlo simulation. Limitations following from a simple oscillator have been confirmed on multi-degree-of-freedom systems. 
	\item The computational demands have been measured and summarized in terms of the time needed for solution, and the applicability of the techniques has been proved.
\end{enumerate}
Finally, let us note that these methods are approximations and can be further refined. In particular, the solution can be reformulated for the non-Gaussian processes for which higher moments can be derived. These apply in improving the mean up-crossing rate estimates for stiff structures.
\section*{Acknowledgement}
Financial support for this work from the Czech Technical University in Prague
under project No. SGS12/027/OHK1/1T/11 and from the Czech Science Foundation under project No. GAP105/11/1529 is gratefully acknowledged.





\begin{thebibliography}{11}
	\expandafter\ifx\csname natexlab\endcsname\relax\def\natexlab#1{#1}\fi
	\expandafter\ifx\csname url\endcsname\relax
	\def\url#1{\texttt{#1}}\fi
	\expandafter\ifx\csname urlprefix\endcsname\relax\def\urlprefix{URL }\fi
	
	\bibitem[{And\v{e}l(1976)}]{Andel}
	And\v{e}l, J., 1976. The Statistical Analysis of Time Series (in Czech). SNTL.
	
	\bibitem[{Brockwell et~al.(2007)Brockwell, Davis, and Yang}]{Brockwell}
	Brockwell, P.~J., Davis, R.~A., Yang, Y., 2007. Continuous-time gaussian
	autoregression. Statistica Sinica 17, 63--80.
	
	\bibitem[{Ellis and Ji(1994)}]{Ellis}
	Ellis, B.~R., Ji, T., 1994. Floor vibration induced by dance-type loads:
	Verification. The Structural Engineer 72~(3), 45--50.
	
	\bibitem[{Grigoriu(1995)}]{Grigoriu_nong}
	Grigoriu, M., 1995. Applied Non-Gaussian Processes: Examples, Theory,
	Simulation, Linear Random Vibration, and MATLAB Solutions. Prentice Hall.
	
	\bibitem[{Grigoriu(2002)}]{Grigoriu_stoch}
	Grigoriu, M., 2002. Stochastic Calculus: Applications in Science and
	Engineering. Birkh\"{a}user.
	
	\bibitem[{Hermanussen et~al.(2001)Hermanussen, Danker-Hopfe, and
		Weber}]{Hermanussen}
	Hermanussen, M., Danker-Hopfe, H., Weber, G.~W., 2001. Body weight and the
	shape of the natural distribution of weight, in very large samples of german,
	austrian and norwegian conscripts. International Journal of Obesity 25,
	1550--1553.
	
	\bibitem[{Huang and Griffin(2008)}]{Huang}
	Huang, Y., Griffin, M.~J., 2008. Nonlinear dual-axis biodynamic response of the
	semi-supine human body during longitudinal horizontal whole-body vibration.
	Journal of Sound and Vibration 312, 273--295.
	
	\bibitem[{Racic and Pavic(2010)}]{Racic}
	Racic, V., Pavic, A., 2010. Mathematical model to generate near-periodic human
	jumping force signals. Mechanical Systems and Signal Processing 24, 138--152.
	
	\bibitem[{Sachse et~al.(2003)Sachse, Pavic, and Reynolds}]{Sachse}
	Sachse, R., Pavic, A., Reynolds, P., 2003. Human-structure dynamic interaction
	in civil engineering dynamics: A literature review. The Shock and Vibration
	Digest 35, 3--18.
	
	\bibitem[{Sim(2006)}]{Sim}
	Sim, J.~H., 2006. Human structure interaction in cantilever grandstands.
	
	\bibitem[{Soong and Grigoriu(1993)}]{Soong}
	Soong, T.~T., Grigoriu, M., 1993. Random Vibration of Mechanical and Structural
	Systems. Prentice Hall, New Jersey.
	
\end{thebibliography}

\end{document}