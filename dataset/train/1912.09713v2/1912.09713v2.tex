\documentclass[letterpaper]{article}
\usepackage{iclr2020_conference}
\usepackage{times}
\usepackage{helvet}
\usepackage{courier}


\usepackage{hyperref}
\usepackage{url}
\usepackage{graphicx}
\urlstyle{rm}
\def\UrlFont{\rm}
\usepackage{graphicx}
\frenchspacing
\setlength{\pdfpagewidth}{8.5in}
\setlength{\pdfpageheight}{11in}
\usepackage{alltt}
\usepackage{amsmath}
\usepackage{amssymb}
\usepackage{breqn}
\usepackage{latexsym}
\usepackage{listings}
\usepackage{makecell}
\usepackage{multirow}
\usepackage{textgreek}
\usepackage{xcolor}
\usepackage{rotating}
\usepackage[T1]{fontenc}
\usepackage{makecell}

\graphicspath{ {./images/} }


\newcommand{\myparagraph}[1]{\textbf{#1}~}
\newcommand{\SPARQL}{\textsc{sparql}}
\newcommand{\SCAN}{\textsc{scan}}
\newcommand{\MCD}[1]{MCD}

\hyphenation{Free-base}
\hyphenation{da-ta-set}
\hyphenation{Com-plex-Web-Ques-tions}
\hyphenation{gen-er-a-ted}


\title{Measuring Compositional Generalization:\\ A Comprehensive Method on Realistic Data}

\author{\0.5ex]
Google Research, Brain Team \
    \mathcal{D}_C(V \Vert W) &= 1\,-\, C_{0.1}(\mathcal{F}_C(V) \, \Vert \, \mathcal{F}_C(W)) \\
    \mathcal{D}_A(V \Vert W) &= 1\,-\, C_{0.5}(\mathcal{F}_A(V) \, \Vert \, \mathcal{F}_A(W))
w(G) = \max_{g \in \text{occ}(G)} (1 - \max_{G': g \prec g' \in \text{occ}(G')} P(G'| G)),
where 
 is the set of all occurrences of  in the sample,   denotes the strict subgraph relation, and  is the empirical probability of  occurring as a supergraph of  over the full sample set.

Intuitively, we are trying to estimate how interesting the subgraph  is in the sample. First, for every occurrence  of a subgraph , we look for the supergraph  of  that co-occurs most often with  in the full sample set. The empirical probability of having  as a supergraph of  determines how interesting the occurrence  is -- the higher this probability, the less interesting the occurrence. Thus we compute the weight of the occurrence as the complement of this maximum empirical probability. Then we take the weight of  to be the weight of the most interesting occurrence  of  in the sample.

E.g.\ in the extreme case that  \textit{only} occurs within the context , the weight of  will be 0 in all samples. Conversely, if  occurs in many different contexts, such that there is no single other subgraph  that subsumes it in many cases, then  will be high in all samples in which it occurs. This ensures that when calculating compound divergence based on a weighted subset of compounds, the most representative compounds are taken into account, while avoiding double-counting compounds whose frequency of occurrence is already largely explainable by the frequency of occurrence of one of its super-compounds. 

Returning to our example in Figure~\ref{fig:rules-dag-subgraphs}, suppose that  represents the smallest linear subgraph (yellow area), and suppose that the weight of  in this sample is 0.4. Then this means that there exists some other subgraph  (for instance, the linear subgraph highlighted by the blue area) that is a supergraph of  in 60\% of the occurrences of  across the sample set.


\section{Rules Index}
\label{suppl:rules-index}

Below is a selection of the rules used in the generation of CFQ. Specifically, this includes all rules involved in generating the question ``Who directed [entity]?'' (the same example illustrated in the rule application DAG in Appendix~\ref{suppl:rules-dag}). The format of the rules is discussed in Appendix~\ref{suppl:rules-format}.


\subsection{Grammar rules}

\small{
\begin{flushleft}
\begin{flushleft}
\noindent \textbf{S=WHQ\_F6E9egkQqxj}: \\
\texttt{
S/\_x \\
~ WHQ/\_x
}
\end{flushleft}
\begin{flushleft}
\noindent \textbf{WHQ=NPQ\_INDIRECT\_VP\_INDIRECT\_TXCca9URgVm}: \\
\texttt{
WHQ[\_subject]/DropDependency(\_subject)  DropDependency(RolePair(Subject, SubjectHaver).\_action) \\
~ NPQ\_INDIRECT(is\_what:\_none\_, number:\n, object:yes, subject:\_none\_)/\_action
}
\end{flushleft}
\begin{flushleft}
\noindent \textbf{NPQ\_INDIRECT=WHO\_5ptbPXXbuLZ}: \\
\texttt{
NPQ\_INDIRECT(number:singular)/Person \\
~ 'who'
}
\end{flushleft}
\begin{flushleft}
\noindent \textbf{VP\_INDIRECT=VP\_INDIRECT\_DP\_ZJH4NhRkByc}: \\
\texttt{
VP\_INDIRECT(object:yes)[\_action]/\_action  RolePair(ObjectHaver, Object).\_object \\
~ VP\_INDIRECT(object:\_none\_, subject:\_none\_)/\_action \\
~~~~DP/\_object
}
\end{flushleft}
\begin{flushleft}
\noindent \textbf{VP\_INDIRECT=ACTIVE\_VP\_RX51Tm7RXPe}: \\
\texttt{
VP\_INDIRECT(object\_type:\at)[\_head]/\_head  PredicateWithBoundRolePairs(RolePair(SubjectHaver, Subject), RolePair(Predicate, Agent))  PredicateWithBoundRolePairs(RolePair(ObjectHaver, Object), RolePair(Predicate, Undergoer)) \\
~ ACTIVE\_VP(agent\_type:\ut)/\_head
}
\end{flushleft}
\begin{flushleft}
\noindent \textbf{ACTIVE\_VP=VP\_SIMPLE\_hJqAyjRUYJp}: \\
\texttt{
ACTIVE\_VP(number:singular)[\_head]/\_head \\
~ VP\_SIMPLE(form:past)/\_head
}
\end{flushleft}
\begin{flushleft}
\noindent \textbf{VP\_SIMPLE=VP\_GHWf3fcVRZg}: \\
\texttt{
VP\_SIMPLE(agent\_type:person, undergoer\_type:movie)[\_head]/\_head \\
~ VP(concept\_id:DirectingAFilm)/\_head
}
\end{flushleft}
\begin{flushleft}
\noindent \textbf{VP=DIRECTED\_JkYzNbQyXtv}: \\
\texttt{
VP(concept\_id:DirectingAFilm, form:past)/DirectingAFilm \\
~ 'directed'
}
\end{flushleft}
\begin{flushleft}
\noindent \textbf{DP=ENTITY\_M6fSP5GvRaN}: \\
\texttt{
DP(is\_proper\_noun:yes, number:singular)[\_head]/\_head \\
~ ENTITY/\_head
}
\end{flushleft}
\begin{flushleft}
\noindent \textbf{ENTITY=[ENTITY]\_HSz7QrdGdsX}: \\
\texttt{
ENTITY(number:singular)/Entity(new\_var(V1)) \\
~ '[entity]'
}
\end{flushleft}
\end{flushleft}
}

... (211 grammar rules total)

\subsection{Inference rules}

\small{
\begin{flushleft}
\begin{flushleft}
\noindent \textbf{BOUND\_ROLES\_WITH\_PREDICATE\_OBJECT}: \\
\texttt{
BoundRolePairs(\R, \T, \\existsQ, \A  \\rightarrow\existsS, \A  \B, RolePair(\R), RolePair(\T)):  \\
\\sqcap\existsQ, \A  \\sqcap\existsS, \A
}
\end{flushleft}
\begin{flushleft}
\noindent \textbf{IGNORE\_BOUND\_ROLE\_PAIRS}: \\
\texttt{
\\sqcapX, \\rightarrowA
}
\end{flushleft}
\begin{flushleft}
\noindent \textbf{IGNORE\_DEPENDENCY\_DROPPING}: \\
\texttt{
DropDependency(\\rightarrowX
}
\end{flushleft}
\begin{flushleft}
\noindent \textbf{PREDICATE\_UNREIFICATION}: \\
\texttt{
Role(\P), Role(\P):  \\
RolePair(\P  RolePair(Predicate, \A)  RolePair(\R).\X  \V1)  def2sparql(\V1) ' . ' def2sparql(\V1)
}
\end{flushleft}
\begin{flushleft}
\noindent \textbf{ENTITY\_MID}: \\
\texttt{
ent2sparql(Entity(\\rightarrowX
}
\end{flushleft}
\begin{flushleft}
\noindent \textbf{GET\_SPECIALIZATIONS}: \\
\texttt{
get\_specializations(\\rightarrowX, new\_var(\X, get\_var(\V0)) '\}'
}
\end{flushleft}
\begin{flushleft}
\noindent \textbf{GET\_VAR\_CONJUNCTION}: \\
\texttt{
get\_var(\\sqcapY, \\rightarrowX, get\_var(\V1)), get\_var(\X, \\existsR.\V1)  \X, \X, \\rightarrowV1
}
\end{flushleft}
\begin{flushleft}
\noindent \textbf{PROPERTY\_MAPPING}: \\
\texttt{
FreebasePropertyMapping(\F):  \\
role2sparql(\\rightarrowF
}
\end{flushleft}
\begin{flushleft}
\noindent \textbf{RELATION\_MAPPING\_WITHOUT\_EXCLUSION}: \\
\texttt{
NonExclusiveRolePair(\X, \Y)  \R) \\existsR.\V1)  rel2sparql(\R, ent2sparql(\X, \\rightarrowX
}
\end{flushleft}
\begin{flushleft}
\noindent \textbf{SPECIALIZATION\_OF\_TYPE}: \\
\texttt{
def2sparql(\V1)  \X)
}
\end{flushleft}
\begin{flushleft}
\noindent \textbf{TYPE\_MAPPING}: \\
\texttt{
FreebaseTypeMapping(\F):  \\
type2sparql(\\rightarrowF
}
\end{flushleft}
\end{flushleft}
}

... (21 resolution rules total)

\subsection{Knowledge rules}

\small{
\begin{flushleft}
\texttt{
 BoundRolePairs(DirectingFilm, RolePair(Predicate, Agent), RolePair(Predicate, FilmDirector))
}\\
\texttt{
 BoundRolePairs(DirectingFilm, RolePair(Predicate, Undergoer), RolePair(Predicate, DirectedFilm))
}\\
\texttt{
 BoundRolePairs(PredicateWithBoundRolePairs(RolePair(ObjectHaver, Object), RolePair(Predicate, Undergoer)), RolePair(ObjectHaver, Object), RolePair(Predicate, Undergoer))
}\\
\texttt{
 BoundRolePairs(PredicateWithBoundRolePairs(RolePair(Subject, SubjectHaver), RolePair(Agent, Predicate)), RolePair(Subject, SubjectHaver), RolePair(Agent, Predicate))
}\\
\texttt{
 FreebasePropertyMapping(RolePair(FilmDirector, DirectedFilm), 'ns:film.director.film')
}\\
\texttt{
 FreebaseTypeMapping(Person, 'ns:people.person')
}\\
\texttt{
 NonExclusiveRolePair(FilmDirector, DirectedFilm)
}\\
\texttt{
 Role(DirectedFilm, DirectingFilm)
}\\
\texttt{
 Role(FilmDirector, DirectingFilm)
}\\
\end{flushleft}

... (194 knowledge rules total)
}


\end{document}
