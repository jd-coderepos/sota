

\documentclass[11pt,a4paper]{article}
\usepackage{acl-onecolumn}
\usepackage{times}
\usepackage{url}
\usepackage{latexsym}
\usepackage{amsmath}
\usepackage{breqn}
\usepackage{pgfplotstable}
\usepackage{hhline}
\usepackage{multirow}
\usepackage{multicol}
\usepackage[font=small]{caption}
\usepackage{subcaption}
\usepackage{color}
\usepackage{float}
\usepackage{lipsum,adjustbox}
\usepackage{tikz}
\usepackage{tikz-dependency}
\usepackage{enumitem}
\usepackage{rotating}
\usepackage{xr}
\externaldocument{acl2018}
\usetikzlibrary{shapes,fit,calc,er,positioning,intersections,decorations.shapes,mindmap,trees}
\tikzset{decorate sep/.style 2 args={decorate,decoration={shape backgrounds,shape=circle,
      shape size=#1,shape sep=#2}}}
\newcommand{\oa}[1]{\footnote{\color{red} #1}}
\newcommand{\daniel}[1]{\footnote{\color{blue} #1}}
\newcommand{\com}[1]{}
\DeclareMathOperator*{\argmin}{argmin}
\DeclareMathOperator*{\argmax}{argmax}


\hyphenation{SemEval}
\hyphenation{PARSEVAL}

\title{Multitask Parsing Across Semantic Representations \\ Supplementary Notes}

\begin{document}
\maketitle

\paragraph{Features.}

Table~\ref{tab:features} lists all feature used for the classifier (see \S\ref{sec:classifier}).
Numeric features are taken as they are, whereas categorical features are mapped to real-valued embedding
vectors.
For \texttt{w} features,
we concatenate randomly-initialized and pre-trained word embeddings.
For each node, we select a \textit{head terminal} by traversing the graph according to
a priority order on edge labels, taken from \citet{hershcovich2017a}.

 refers to stack node  from the top, and
 to buffer node .
 and  refer to a 's leftmost and rightmost children, and
 and  to its leftmost and rightmost parents.

\texttt{w} refers to the node's head terminal text,
\texttt{t} to its POS tag, and
\texttt{d} to its dependency relation.
\texttt{h} refers to the node's height,
\texttt{e} to the tag of its first incoming edge,
\texttt{n} and \texttt{c} to the node label and category (used only for AMR),
\texttt{p} to any separator punctuation between  and ,
\texttt{q} to the count of any separator punctuation between  and ,
\texttt{x} to the numeric value of gap type \cite{maier-lichte:2016:DiscoNLP},
\texttt{y} to the sum of gap lengths,
\texttt{P}, \texttt{C}, \texttt{I}, \texttt{E}, and \texttt{M} to the number of
parents, children, implicit children, remote children, and remote parents,
\texttt{N} to the numeric value of the head terminal's named entity IOB indicator,
\texttt{T} to its named entity type,
\texttt{\#} to its word shape (capturing orthographic features, e.g. "Xxxx" or "dd"),
\texttt{\^{}} to its one-character prefix, and
\texttt{\x \to yxya_ii+1CEPs_0xhqyPCIEMN} \\
 & \texttt{wtdencT\#\^{}\s_2xhy} \\
 & \texttt{wtdencT\#\^{}\b_0hPCIEMN} \\
 & \texttt{wtdncT\#\^{}\s_0l, s_0r, s_1l, s_1r, s_0ll, s_0lr, s_0rl, s_0rr, s_1ll, s_1lr, s_1rl, s_1rr} \\
 & \texttt{wen\#\^{}\s_0 \to s_1, s_0 \to b_0s_1 \to s_0, b_0 \to s_0s_0 \to b_0, b_0 \to s_0a_0, a_1F_1F_1F_1EEECAFECPRECCUEECCCCAHFECPRDPAHUEEECAFFCPRECAHUEECAFFCPRECAHU^{++}$ as auxiliaries on the same sentence (bottom).
\label{fig:qualitative}}
\end{sidewaysfigure}

\bibliography{references}
\bibliographystyle{acl_natbib}

\end{document}
