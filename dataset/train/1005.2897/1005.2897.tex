\documentclass{LMCS}

\def\dOi{10(4:8)2014}
\lmcsheading {\dOi}
{1--40}
{}
{}
{Feb.~27, 2014}
{Dec.~\phantom09, 2014}
{}

\ACMCCS{[{\bf Theory of computation}]: Models of
  computation---Computability---Lambda calculus} 
\subjclass{F.4.1
  Lambda calculus and related systems}

\usepackage{needspace}
\usepackage{amsmath,amsfonts,amssymb}
\usepackage{proof}
\usepackage{multirow}
\usepackage{url}
\usepackage[matrix,arrow,curve]{xy}
\usepackage[only,llbracket,rrbracket]{stmaryrd}
\usepackage{hyperref}

\newtheorem{theorem}{Theorem}[section]
\newtheorem{lemma}[theorem]{Lemma}
\newtheorem{remark}[theorem]{Remark}
\newtheorem{corollary}[theorem]{Corollary}
\newtheorem{definition}[theorem]{Definition}
\newtheorem{example}[theorem]{Example}
\newtheorem{convention}[theorem]{Convention}
\newenvironment{Corollary}{\Needspace*{3\baselineskip}\corollary
}{\endtheorem}

\newcommand{\recap}[2]{\medskip\noindent{\bf #1 \ref{#2}.}~}
\newcommand{\App}[1]{The details can be found in Appendix~\ref{proof:#1}}

\newcommand{\define}[1]{{\em #1}}
\newcommand{\void}[1]{}

\newcommand{\olin}{\mbox{\sc lineal}}
\newcommand{\oalg}{\mbox{\sc alg}}
\newcommand{\lalin}{\ensuremath{\lambda_{{\it lin}}}}
\newcommand{\laalg}{\ensuremath{\lambda_{{\it alg}}}}
\newcommand{\xllin}[1]{\ensuremath{\lambda^{\raisebox{0.5ex}{\textrm{\tiny }}}_{{\it lin}}}}
\newcommand{\xlalg}[1]{\ensuremath{\lambda^{\raisebox{0.5ex}{\textrm{\tiny }}}_{{\it alg}}}}
\newcommand{\llinred}{\xllin{\rightarrow}}
\newcommand{\lalgred}{\xlalg{\rightarrow}}
\newcommand{\llineq}{\xllin{=}}
\newcommand{\lalgeq}{\xlalg{=}}

\newcommand{\xto}[1]{\ensuremath{\rightarrow_{#1}}}
\newcommand{\simxto}[1]{\ensuremath{\rightarrow_{#1}^{\hspace{-2.2ex}\textrm{\tiny=}}}}
\newcommand{\ssimxto}[1]{\ensuremath{\rightarrow_{#1}^{\hspace{-2.2ex}\textrm{\tiny=}\ *}}}
\newcommand{\fragment}[2]{\ensuremath{{#2}^{\hspace{-1ex}#1}}}
\newcommand{\toolin}{\xto{\mathcal{L}}}
\newcommand{\stoolin}{\ensuremath{\xto{L}^{\ast}}}
\newcommand{\tolinred}{\xto{\ell}}
\newcommand{\toalgred}{\xto{a}}
\newcommand{\tolineq}{\simxto{\ell}}
\newcommand{\toalgeq}{\simxto{a}}
\newcommand{\fromalgred}{\ensuremath{\leftarrow}_{a}}
\newcommand{\sfromalgred}{\ensuremath{\leftarrow}_{a}^\ast}
\newcommand{\stoalgred}{\ensuremath{\xto{a}^{\ast}}}
\newcommand{\stolinred}{\ensuremath{\xto{\ell}^{\ast}}}
\newcommand{\symclosure}[1]{\,{#1}^{\leftrightarrow}\,}

\newcommand{\tobv}{\xto{\beta_v}}
\newcommand{\tobn}{\xto{\beta_n}}

\newcommand{\toblinred}{\xto{\ell\cup\beta_v}}
\newcommand{\axlin}{{\simeq_{\ell\cup\beta_v}}}
\newcommand{\axalg}{{\simeq_{a\cup\beta_n}}}
\newcommand{\tobalgred}{\xto{a\cup\beta_n}}
\newcommand{\toblineq}{\simxto{{\ell}\cup\beta_v}}
\newcommand{\tobalgeq}{\simxto{a\cup\beta_n}}

\newcommand{\stoblinred}{\ensuremath{\xto{\ell\cup\beta_v}^{\ast}}}
\newcommand{\stobalgred}{\ensuremath{\xto{a\cup\beta_n}^{\ast}}}
\newcommand{\stoblineq}{\ensuremath{\ssimxto{{\ell}\cup\beta_v}}}
\newcommand{\stobalgeq}{\ensuremath{\ssimxto{a\cup\beta_n}}}
\newcommand{\stobv}{\ensuremath{\xto{\beta_v}^{\ast}}}
\newcommand{\stobn}{\ensuremath{\xto{\beta_n}^{\ast}}}
\newcommand{\stolineq}{\ensuremath{\ssimxto{{\ell}}}}
\newcommand{\stoalgeq}{\ensuremath{\ssimxto{a}}}

\newcommand{\admf}[1]{{\it Admin}\,{(#1)}}
\newcommand{\adm}[1]{{\it Admin}{(#1)}}
\newcommand{\toadm}{\to_{{\it adm}}}

\usepackage{accsupp} 

\newcommand*{\llbrace}{\BeginAccSupp{method=hex,unicode,ActualText=2983}\textnormal{\usefont{OMS}{lmr}{m}{n}\char102}\mathchoice{\mkern-4.05mu}{\mkern-4.05mu}{\mkern-4.3mu}{\mkern-4.8mu}\textnormal{\usefont{OMS}{lmr}{m}{n}\char106}\EndAccSupp{}}
\newcommand*{\rrbrace}{\BeginAccSupp{method=hex,unicode,ActualText=2984}\textnormal{\usefont{OMS}{lmr}{m}{n}\char106}\mathchoice{\mkern-4.05mu}{\mkern-4.05mu}{\mkern-4.3mu}{\mkern-4.8mu}\textnormal{\usefont{OMS}{lmr}{m}{n}\char103}\EndAccSupp{}}


\newcommand{\wt}[1]{\llbracket{#1}\rrbracket}
\newcommand{\cps}[1]{\llbrace{#1}\rrbrace}
\newcommand{\pair}[2]{\left\langle #1,#2\right\rangle}

\newcommand{\cont}[1]{\lambda f\,\left(f\right)~#1}
\newcommand{\uncont}[2]{\lambda #1\,({#2}\,#1)}
\newcommand{\tapp}[5]{\lambda{#1}\,({#4})~\lambda{#2}\,({#5})~\lambda{#3}\,(({#2})~{#3})~{#1}}
\newcommand{\tappfgh}{\tapp{f}{g}{h}}
\newcommand{\tappcps}[2]{\lambda{f}\,({#1})~\lambda{g}\,((g)~{{#2}})~f}
\newcommand{\bb}[1]{[#1]}

\newcommand{\s}[1]{{\{\,{#1}\,\}}}
\newcommand{\neu}{\ensuremath{\mathcal{N}}}
\newcommand{\SN}{\ensuremath{\bf SN}}
\newcommand{\red}[1]{\ensuremath{{\rm Red}({#1})}}
\newcommand{\RC}{\ensuremath{\rm RC}}
\newcommand{\cond}[1]{\ensuremath{{\bf RC}_{#1}}}
\newcommand{\lcond}[1]{\ensuremath{{\bf RC}_{#1}^\ell}}
\newcommand{\ca}[1]{\ensuremath{\mathcal{#1}}}
\newcommand{\interp}[1]{\llbracket{#1}\rrbracket}


\newenvironment{absolutelynopagebreak}
  {\par\nobreak\vfil\penalty0\vfilneg
   \vtop\bgroup}
  {\par\xdef\tpd{\the\prevdepth}\egroup
   \prevdepth=\tpd}

\makeatletter 
\def\mynobreakpar{\par\nobreak\@afterheading} 
\makeatother

\allowdisplaybreaks


\begin{document}

\title[CbV, CbN and the vectorial behaviour of the algebraic -calculus]{Call-by-value, call-by-name and the vectorial behaviour of the algebraic -calculus}

\author[A.~Assaf]{Ali Assaf\rsuper a}
\address{{\lsuper{a}}\'Ecole Polytechnique\\
Route de Saclay\\
91120 Palaiseau\\
France}
\address{{\lsuper{a}}INRIA\\
23 avenue d'Italie\\
CS 81321\\
75214 Paris Cedex 13\\
France}
\email{ali.assaf@inria.fr}

\author[A.~D\'iaz-Caro]{Alejandro D\'iaz-Caro\rsuper b}
\address{{\lsuper b}Universidad Nacional de Quilmes\\
Roque S\'aenz Pe\~na 352\\
1876 Bernal, Buenos Aires\\
Argentina}
\email{alejandro@diaz-caro.info}
\thanks{\lsuper{b} Partially supported by the French-Argentinian Laboratory in Computer Science INFINIS}

\author[S.~Perdrix]{Simon Perdrix\rsuper c}
\address{{\lsuper c}CNRS \& LORIA\\
615, rue du Jardin Botanique\\
BP-101\\
54602 Villers-l\`es-Nancy\\
France}
\email{simon.perdrix@loria.fr}

\author[C.~Tasson]{Christine Tasson\rsuper d}
\address{{\lsuper{d,e}}PPS, Universit\'e Paris-Diderot -- Paris 7\\
CNRS UMR 7126\\
75205 Paris Cedex 13\\
France}
\email{christine.tasson@pps.univ-paris-diderot.fr, benoit.valiron@monoidal.net}

\author[B.~Valiron]{Beno\^it Valiron\rsuper e}
\address{{\lsuper{e}} I2M, Universit\'{e} Aix-Marseille,\\
CNRS UMR 7373, 
Campus de Luminy, case 907, F–13288 Marseille\\
France}
\thanks{{\lsuper{e}} Partially supported by the ANR project ANR-2010-BLAN-021301 LOGOI.}

\keywords{algebraic -calculus; linear-algebraic -calculus; CPS simulation}

\begin{abstract} 
  We examine the relationship between the {\em algebraic
  -calculus}, a fragment of the differential
  -calculus and the \emph{linear-algebraic
  -calculus}, a candidate -calculus for quantum
  computation. Both calculi are algebraic: each one is equipped with
  an additive and a scalar-multiplicative structure, and their set of
  terms is closed under linear combinations. However, the two
  languages were built using different approaches: the former is a
  call-by-name language whereas the latter is call-by-value; the
  former considers algebraic equalities whereas the latter approaches
  them through rewrite rules.

  In this paper, we analyse how these different approaches relate to one
  another. To this end, we propose four canonical languages based
  on each of the possible choices: call-by-name versus call-by-value,
  algebraic equality versus algebraic rewriting. We show that the
  various languages simulate one another.
Due to subtle interaction between beta-reduction and algebraic
  rewriting, to make the languages consistent some additional
  hypotheses such as confluence or normalisation might be required. We carefully devise
  the required properties for each proof, making them general enough
  to be valid for any sub-language satisfying the corresponding
  properties. 
\end{abstract}

\maketitle

\section{Introduction}

Two algebraic versions of the -calculus arise independently in distinct contexts: the algebraic -calculus ()~\cite{VauxMSCS09} and the linear algebraic -calculus ()~\cite{ArrighiDowekRTA08}. 
Both languages are extensions of -calculus where linear combinations of terms are also terms.
The former has been introduced in the context of linear logic as a fragment of the differential -calculus~\cite{EhrhardRegnierTCS03}: the algebraic structure allows to gather in a non deterministic manner different terms, {\em i.e.}\ each term in the linear combination represents one possible execution. The latter has been introduced as a candidate -calculus for quantum computation: in , a linear combination of terms reflects the phenomenon of superposition, {\em i.e.}\ the capability for a quantum system to be in two or more states at the same time.
Our purpose is to study the connections between the two systems.

In both languages, functions which are linear combinations of terms are interpreted pointwise: , where ``'' denotes the scalar multiplication. The two languages differ in the treatment of the arguments. In , 
in order to deal with the algebraic structure, any function is considered as a linear map: , reflecting the fact that any quantum evolution is a linear map. It reflects a call-by-value behaviour in the sense that the argument is evaluated until one has a base term.
Conversely,  has a call-by-name evolution: , without any restriction on . As a consequence, the evolutions are different as illustrated by the following example. In ,  while in , . 

Because they were designed for different purposes, another difference
appears between the two languages: the way the algebraic part of the
calculus is treated. In , the algebraic structure is captured
with a rewrite system, whereas in  terms are considered up to
algebraic equivalence.

The two choices -- call-by-value versus call-by-name and algebraic
equality versus algebraic reduction -- allow one to construct four
possible calculi. We name them \llinred, \llineq, \lalgred, and \lalgeq. See Figure~\ref{tab:linalg} where they
are presented according to their evaluation policy and the
way they take care of the algebraic part of the language.
\begin{figure}[b]
  \def\arraystretch{1.5}
  \begin{tabular}{|c|c|c|}
    \hline
    & call-by-name  &  call-by-value \\
    \hline
    algebraic & \multirow{2}{*}{\lalgred} & \multirow{2}{*}{\llinred}\-0.2cm]
    equality & & \\
    \hline
  \end{tabular}\1.5ex]
  \small
   means `` is simulated by ''
  \caption{Relations between the languages. }
  \label{fig:relation}
\end{figure}

\subsection*{Plan of the paper.}
In Section~\ref{sec:originals}, we present the original calculi  and .
In Section~\ref{sec:alglam}, we define the set of terms and the rewrite systems we consider in the paper.
In Section~\ref{sec:relation-orig}, we establish the relation between the original setting and the setting used in this paper.
In Section~\ref{sec:consistency}, we discuss the confluence 
of the algebraic rewrite systems. 
Section~\ref{sec:sim} is concerned with the actual simulations. In Section~\ref{sec:redeq} we consider the correspondence between algebraic reduction and algebraic equality whereas in Sections~\ref{subsec:lintoalg} to~\ref{subsec:algtolin-completeness} we consider the distinction call-by-name versus call-by-value. In Section~\ref{subsec:compose}, we show how the simulations can compose to obtain the correspondence between any two of the four languages. 
In Section~\ref{sec:concl} we conclude by providing some paths for future work.
Omitted and sketched proofs are fully developed in the appendix.


\section{The languages}
In this section we present all the languages: the original setting, and our standardised versions of the algebraic calculi.
\subsection{The original setting}\label{sec:originals}
The language \olin\ was first presented in~\cite{ArrighiDowekRTA08} as
summarised in Figure~\ref{fig:olin}. The rewrite system is defined by
structual induction on the left-hand-side. The factorisation and
application rules ask for a particular subterm of the left-hand-side
to {\em not} reduce (conditions (*) and (**)). Because of the
inductive definition, this is indeed well-defined.
These conditions (*) and (**) in particular ensure confluence: we
refer the reader to the original paper~\cite{ArrighiDowekRTA08} for
more details.


\begin{figure}
  {
    \hrule\vspace{1pt}\hrule
    

    The rewrite system is defined inductively on the size of the left-hand-side.\\begin{array}[t]{l@{\hspace{1.5cm}}r@{\ ::=\quad}l}
      \textrm{\itshape Terms:} & M,N,L & x\ |\ \lambda x\,M\ |\ (M)~N\ |\ 0\ |\ \alpha.M\ |\ M+N\\
    \end{array}\begin{array}{rcll}
    M,N,L&::=& V ~|~ (M)~N ~|~ \alpha.M ~|~M+N&\textrm{(terms),}\\
    U,V,W&::=& 0~|~B~|~\alpha.V~|~V+W&\textrm{(values),}\\
    B&::=& x~|~ \lambda x\,M&\textrm{(basis terms),}
\end{array}-1ex]
	\multicolumn{3}{c}{\textrm{Call-by-name ()}}
	&
	\multicolumn{3}{c}{\textrm{\sc Linearity of the application ()}}\1.5ex]
	\hline
	\hline
	\multicolumn{6}{c}{\rule{0ex}{4ex}{\textrm{\sc Specific rules for  and }}}\\
	\hline\1.5ex]
	(\lambda x\,M)~B &\to& M[x:=B] 
	& 		
	\multicolumn{3}{c}{\infer{(V)~M\to (V)~M'}{M\to M'}}\\
	\multicolumn{6}{c}{\textrm{\sc Linearity of the application}}\1.5ex]
	(M+N)~V &\to& (M)~V + (N)~V 
	&
	(B)~(M+N) &\to& (B)~M + (B)~N\\
	(\alpha.M)~V &\to& \alpha.(M)~V
	&
	(B)~(\alpha.M) &\to& \alpha.(B)~M\\
	(0)~V &\to& 0 
	&
	(B)~0 &\to& 0\-1ex]
	\multicolumn{6}{c}{\textrm{\sc Ring rules ()}}\1.5ex]
	M+(N+L) &\to& (M+N)+L
	&
	M+N &\to& N+M\\
	(M+N)+L &\to& M+(N+L)
	&
	\1.5ex]
	\alpha.M + \beta.M &\to& (\alpha+\beta).M
	&
	\alpha.(M+N) &\to& \alpha.M + \alpha.N\\
	\alpha.M + M &\to& (\alpha + 1).M
	&
	1.M &\to& M\\
	M + M &\to& (1 + 1).M
	& 
	0.M &\to& 0\\
	\alpha.(\beta.M) &\to& (\alpha\beta).M
	&
	\alpha.0 &\to& 0\\
	&&
	&
	0 + M &\to& M\1.5ex]
	\multicolumn{2}{c}{\infer{(M)~N\to (M')~N}{M\to M'}}
	&
	\infer{M + N\to M' + N}{M\to M'}
	&
	\multicolumn{2}{c}{\infer{M + N\to M + N'}{N\to N'}}
	&
	\infer{\alpha.M\to \alpha.M'}{M\to M'}\\begin{array}{rcl@{\hspace{0.75cm}}rcl@{\hspace{0.75cm}}rcl}
      \toalgred& := &A \cup L \cup \xi &\tolinred&:=&A_l \cup A_r \cup L \cup \xi \cup \xi_{\lalin} &\tobv&:=&\beta_v \cup \xi \cup \xi_{\lalin}\\
      \toalgeq&:=&\symclosure{(\toalgred)}&\tolineq&:=&\symclosure{(\tolinred)}&\tobn&:=&\beta_n \cup \xi
  \end{array}
  M \toalgeq M+0
  \toalgeq  M+(Y_{N-M}-Y_{N-M})
  \tobalgeq  M+((N-M+Y_{N-M})-Y_{N-M})
  \stoalgeq  N

  \begin{array}{rcl@{\hspace{1cm}}rcl}
    \wt{x} &=& \cont x,&
    \wt{0} &=& 0,\\
    \wt{\lambda x\,M} & =&\lambda f\,(f)~\lambda x\,\wt M,&
    \wt{(M)~N} &=& \tappfgh{\wt M}{\wt N},\\
    \wt{\alpha.M} &=&\lambda f\,(\alpha.\wt M)~f,&
    \wt{M+N}&=&\lambda f\,(\wt M +\wt N)~f.
  \end{array}

    \begin{array}{r@{\,=\,}l}
      0:K&0\\
      B:K&(K)~\Psi(B)\\
      \alpha.M:K&\alpha.(M:K)\\
      M+N:K&M:K+N:K\\
    \end{array}~~~
    \begin{array}{r@{\,=\,}l}
      (0)~N:K&0\\
      (B)~N:K&N:\lambda f\,((\Psi(B))~f)~K\\
      (\alpha.M)~N:K&\alpha.(M)~N:K \\
      (M+N)~L:K&((M)~L+(N)~L):K \\
      ((M)\,N)\,L:K&(M)\,N{:}\lambda\,g\,(\wt{L})\,\lambda\,h\,((g)\,h)\,K \\
    \end{array}
  
    \wt{(\textsf{copy})~(B_1+B_2) }=
    \lambda f\, (\wt{\textsf{copy}})\,
    \lambda g\, (\wt{B_1+B_2})\lambda h\, ((g)\,h)\,f\ ,
  
    (\wt{(\textsf{copy})~(B_1+B_2)})\,k
    &\tobalgred
    (\wt{\textsf{copy}})\,
    \lambda g\, (\wt{B_1+B_2})\lambda h\,((g)\,h)\,k
    \\
    &=
    (\lambda f\, (f)\,\lambda x\, \wt{\pair x x})\,
    \lambda g\, (\wt{B_1+B_2})\lambda h\, ((g)\,h)\,k
    \\
    &\tobalgred
    (\lambda g\, (\wt{B_1+B_2})\, \lambda h\, ((g)\,h)\,k)\,
    \lambda x\, \wt{\pair x x}
    \\
    &\tobalgred
    (\wt{B_1+B_2})\,\lambda h\, ((\lambda x\, \wt{\pair x x})\,h)\,k
    \\
    &\tobalgred
    (\wt B_1  + \wt B_2)\,
    \,\lambda h\, ((\lambda x\, \wt{\pair x x})\,h)\,k\\
    &\tobalgred
    (\wt B_1)\,
    \,\lambda h\, ((\lambda x\, \wt{\pair x x})\,h)\,k
    +
    (\wt B_2)\,
    \,\lambda h\, ((\lambda x\, \wt{\pair x x})\,h)\,k
    \\
    &\stobalgred
    (\lambda h\, ((\lambda x\, \wt{\pair x x})\,h)\,k)\, \Psi(B_1)
    +
    (\lambda h\, ((\lambda x\, \wt{\pair x x})\,h)\,k)\, \Psi(B_2)
    \\
    &\stobalgred
    ((\lambda x\, \wt{\pair x x})\,\Psi(B_1))\,k 
    +
    ((\lambda x\, \wt{\pair x x})\,\Psi(B_2))\,k 
    \\
    &\stobalgred
    (\wt{\pair x x}[x:= \Psi(B_1)])\,k 
    +
    (\wt{\pair x x}[x:= \Psi(B_2)])\,k  
    \\
    &\stobalgred
    (\wt{\pair {B_1}{B_1}})\,k 
    +
    (\wt{\pair {B_2}{B_2}})\,k
    \quad \textrm{(Lemma \ref{lem:substitution2})}
    \\
    &\stobalgred
    (k) \, \Psi({\pair {B_1}{B_1}})
    +
    (k) \, \Psi({\pair {B_2}{B_2}})\\
    &=
    ({\pair {B_1}{B_1}} + {\pair {B_2}{B_2}}):k
   (\wt M)~k \stoblinred V:k  (\wt M)~k \stoblineq V:k 
  \begin{array}{rcll}
    C & ::= & (K)~B \mid ((B_{1})~B_{2})~K \mid (T)~K & \mbox{(base computations)}\\
    D & ::= & C\mid0\mid\alpha.D\mid D_{1}+D_{2} & \mbox{(computation combinations)}\\
    \\
    S & ::= & \lambda k\,C & \mbox{(base suspensions)}\\
    T & ::= & S\mid0\mid\alpha.T\mid T_{1}+T_{2} & \mbox{(suspension combinations)}\\
    \\
    K & ::= & k \mid \lambda b\,((B)~b)~K \mid \lambda{b_{1}}\,(T)~\lambda{b_{2}}\,((b_{1})~b_{2})~K & \mbox{(continuations)}\\
    \\
    B & ::= & x \mid \lambda x\,S & \mbox{(CPS-values)}
  \end{array}

  \begin{array}{rclrcl}
    \overline{(K)~B} & = & \underline{K}[\psi(B)] & \sigma(\lambda k\,C) & = & \overline{C}\\
    \overline{((B_{1})~B_{2})~K} & = & \underline{K}[(\psi(B_{1}))~\psi(B_{2})] & \sigma(0) & = & 0\\
    \overline{(T)~K} & = & \underline{K}[\sigma(T)] & \sigma(\alpha.T) & = & \alpha.\sigma(T)\\
    \overline{0} & = & 0 & \sigma(T_{1}+T_{2}) & = & \sigma(T_{1})+\sigma(T_{2})\\
    \overline{\alpha.D} & = & \alpha.\overline{D}\\
    \overline{D_{1}+D_{2}} & = & \overline{D_{1}}+\overline{D_{2}}\\
    &  &  & \underline{k}[M] & = & M\\
    \psi(x) & = & x & \underline{\lambda b\,((B)~b)~K}[M] & = & \underline{K}[(\psi(B))~M]\\
    \psi(\lambda x\,S) & = & \lambda x\,\sigma(S) & \underline{\lambda{b_{1}}\,(T)~\lambda{b_{2}}\,((b_{1})~b_{2})~K}[M] & = & \underline{K}[(M)~\sigma(T)]
  \end{array}

  \begin{array}{rcl@{\hspace{1cm}}rcl}
    \cps{x} &=&x,&
    \cps{0} &=& \lambda f\, (0)~f,\\
    \cps{\lambda x\,M} & =&\lambda f\,(f)~\lambda x\,\cps M,&
    \cps{(M)~N} &=& \tappcps{\cps M}{\cps N},\\
    \cps{\alpha.M} &=& \lambda f\, (\alpha.\cps M)~f,&
    \cps{M+N}&=&\lambda f\,(\cps M +\cps N) ~f.
  \end{array}

    \begin{array}{r@{\,=\,}l}
      0:K&0\\
      x:K&(x)~K\\
      \lambda\,x\,M:K&(K)~\Phi(\lambda\,x\,M)\\
      \alpha.M:K&\alpha.(M:K)\\
      M+N:K&M:K+N:K\\
    \end{array}~~~~
    \begin{array}{r@{\,=\,}l}
      (0)~N:K&0\\
      (x)~N:K&x:\lambda\,f\,((f)\,\cps{N})\,K\\
      (\lambda\,x\,M)~N:K&((\Phi(\lambda\,x\,M))~\cps N)~K \\
      (\alpha.M)~N:K&\alpha.(M)~N:K \\
      (M+N)~L:K&((M)~L+(N)~L):K \\
      ((M)\,N)\,L:K&(M)\,N{:}\lambda\,f\,((f)\,\cps{L})\,K
  \end{array}(\cps M)~k\stobalgred V:k(\cps M)~k\stobalgeq V:k
    \begin{array}{rll}
      (\cps{(\textsf{copy})~(B_1+B_2)})\,k&\toblinred&
      (\cps{\textsf{copy}})\,
      \lambda g\, ((g) \, \cps{B_1+B_2})\,k
      \\
      &=&
      (\lambda f\, (f)\,\lambda x\, \cps{\pair x x})\,
      \lambda g\, ((g) \, \cps{B_1+B_2})\,k
      \\
      &\toblinred&
      (\lambda g\, ((g) \, \cps{B_1+B_2})\,k)\,
      \lambda x\, \cps{\pair x x}
      \\
      &\toblinred&
      ((\lambda x\, \cps{\pair x x}
      ) \, \cps{B_1+B_2})\,k
      \\
      \textrm{(Lemma \ref{3:lem:base})}
      &\toblinred&
      (\cps{\pair x x}[x:=  \cps{B_1+B_2}]
      )\,k
      \\
      \textrm{(Lemma \ref{lem:substitution-cps})}
      &=&
      (\cps{\pair {B_1+B_2} {B_1+B_2}})\,k
      \\
      &\toblinred&
      (k) \, \Phi({\pair {B_1+B_2} {B_1+B_2}}) 
      \\
      &=&
      {\pair {B_1+B_2} {B_1+B_2}}:k
    \end{array}
  
  \begin{array}{rcll}
    C & ::= & (K)~B \mid ((B)~S)~K \mid (T)~K & \mbox{(base computations)}\\
    D & ::= & C \mid 0 \mid \alpha.D \mid D_{1}+D_{2} & \mbox{(computation combinations)}\\\\
    S & ::= & x \mid \lambda k\,C & \mbox{(base suspensions)}\\
    T & ::= & S \mid 0 \mid \alpha.T \mid T_{1}+T_{2} & \mbox{(suspension combinations)}\\\\
    K & ::= & k \mid \lambda b\,((b)~S)~K & \mbox{(continuations)}\\\\
    B & ::= & \lambda x\,S & \mbox{(CPS-values)}
  \end{array}

  \begin{array}{rclrcl}
    \overline{(K)~B} & = & \underline{K}[\phi(B)] & \sigma(x) & = & x\\
    \overline{((B)~S)~K} & = & \underline{K}[(\phi(B))~\sigma(S)] & \sigma(\lambda k\,C) & = & \overline{C}\\
    \overline{(T)~K} & = & \underline{K}[\sigma(T)] & \sigma(0) & = & 0\\
    \overline{0} & = & 0 & \sigma(\alpha.T) & = & \alpha.\sigma(T)\\
    \overline{\alpha.D} & = & \alpha.\overline{D} & \sigma(T_{1}+T_{2}) & = & \sigma(T_{1})+\sigma(T_{2})\\
    \overline{D_{1}+D_{2}} & = & \overline{D_{1}}+\overline{D_{2}}\\
    &  &  & \underline{k}[M] & = & M\\
    \phi(\lambda x\,S) & = & \lambda x\,\sigma(S) & \underline{\lambda b\,((b)~S)~K}[M] & = & \underline{K}[(M)~\sigma(S)]
  \end{array}



To prove the completeness of the simulation we need analogous lemmas.
Their proofs are similar, but we need to account for the changes mentioned
above. 

\begin{lemma}
  \label{lem:inverse-term-a} For any term , .
\end{lemma}
\begin{proof}
  By induction on . \App{inverse-term-a}.
\end{proof}

\begin{lemma}
  \label{lem:inverse-value-a} For any value , .
\end{lemma}
\begin{proof}
  By induction on . \App{inverse-value-a}.
\end{proof}

\begin{lemma}
  \label{lem:inverse-step-a} For any computation , if 
  then . Also, if 
  then .
\end{lemma}
\begin{proof}
  The proof of this lemma is very similar to the one of
  Lemma~\ref{lem:inverse-step}. It follows by induction, involving several
  intermediary results similar to the ones in the proof of
  Lemma~\ref{lem:inverse-step}, where:  is replaced with
   and  with  in the four first equalities; then:
   is replaced with  and  is
  replaced with . In other words:
  \begin{itemize}
    \item  The following equalities hold.
      \begin{enumerate}
	\item {}
	\item 
	\item {}
	\item 
      \end{enumerate}
    \item For all terms  and continuations
       and , 
      .
    \item For all  and
      , .
    \item For any continuation  and term
      , if  then .
    \item For any continuation , scalar  and terms , 
      and , the following relations hold.
      \begin{itemize}
	\item 
	\item 
	\item 
      \end{itemize}
    \item For any suspension , if 
      then .
  \end{itemize}
  \App{inverse-step-a}.
\end{proof}

Finally, we can prove the completeness theorem using the previous lemmas.

\begin{proof}[\bf Proof of Theorem \ref{thm:completeness-a}]
  Using Lemma \ref{lem:inverse-step-a} for each step of the reduction,
  we get that . By Lemma
  \ref{lem:inverse-term-a} and Lemma \ref{lem:inverse-value-a}, this
  implies .
\end{proof}


\subsection{The remaining simulations}\label{subsec:compose}

In Figure~\ref{fig:relation}, some arrows are missing, for example from \llinred to \lalgeq. We now show that the already existing arrows ``compose'' well. The first two simulations are  and  and do not require confluence.
\begin{theorem}
  For any term  and variable ,
  if  (respectively ) where  is a value, then  (respectively ).
\end{theorem}
\begin{proof}
  Given that , by Theorem~\ref{th:sim}, , which by Theorem~\ref{th:redeq} implies .
  Analogously, given that , by Theorem~\ref{th:sim2}, we have
  , which by
  Theorem~\ref{th:redeq2} implies that   .
\end{proof}


The other two simulations are  and
 and they do require confluence.
\begin{theorem}
  For any term  in a confluent fragment of 
  (respectively ) and variable , if  (respectively ) then we have
   with  
  (respectively  with  ).
\end{theorem}
\begin{proof}
  Given that  and that  is in a confluent fragment, Theorem~\ref{thm:eqred} states that  with . In addition, Theorem~\ref{th:sim} states that .  
  The other result is similar using
  Theorems~\ref{thm:eqred2} and \ref{th:sim2}. 
\end{proof}

Finally, we show that we can also compose with the
inverse translations to give the completeness of the remaining simulations.
This time however, the requirements of confluence are reversed.

\begin{theorem}
  For any term  and variable , if  (respectively )
  then for any fragment  of  (respectively )
  such that , we have  (respectively ) in .\end{theorem}
\begin{proof}
  If  then by Theorem \ref{thm:completeness}
   which implies . Similarly if 
  then by Theorem \ref{thm:completeness-a}  which
  implies .\end{proof}

\begin{theorem}
  Let  be a confluent fragment of  (respectively ).
  For any term  and variable  such that  (respectively )
  is in , if  (respectively )
  then  with  (respectively 
  with ).\end{theorem}
\begin{proof}
  If  then by Theorem \ref{thm:eqred}
   with . By Lemma \ref{lem:inverse-step},
  we get 
  with , which by Lemmas \ref{lem:inverse-term}
  and \ref{lem:inverse-value} imply  with
  .\\
  Similarly, if  then by Theorem \ref{thm:eqred2}
   with . By Lemma \ref{lem:inverse-step-a},
  we get 
  with , which by Lemmas \ref{lem:inverse-term-a}
  and \ref{lem:inverse-value-a} imply 
  with .
\end{proof}


\section{Discussion and perspectives}\label{sec:concl}

\subsection{Simulating cloning}
As we discussed in the introduction,  is a language whose
original purpose was to emulate quantum superpositions of states with
linear combinations of terms. However, we saw in Example \ref{ex:2}
that we can emulate the ``cloning'' operation
 in
 using a CPS encoding. How does this relate to the no-cloning
theorem \cite{WoottersZurekNATURE82} stating that a quantum state
cannot be duplicated?

Since  is a higher-order
language, a term both represents a quantum operator (i.e. a linear
map) and a state of the system (i.e. a vector in the space of states).
The choice of a call-by-value reduction strategy enforces this
philosophy: an application  is really ,
and there is no reduction under lambda's, making lambda-abstractions
correspond to ``pieces of code'' to be executed only when applied. So
the term  really stands for a piece of code that
would input a base vector  and produce (possibly after some
process) a superposition . But in itself, the lambda-abstraction
is a base vector -- it is not a linear combination. If we were to use
it, say as argument of , it would
actuallly get duplicated.

On the contrary, the term  is the linear
combination of two operators\,: one that inputs  and produces ,
the other one that inputs  and produces . Fed to the same term
, the distributivity of addition
over application would take precedence: the term  behaves as a
linear operator and  is really
.

We can see the same pattern in the term \,:
the argument to the (linear) operator  is a linear
combination of  and , therefore in  the term
 really corresponds to  and not to .  Along the CPS
transformation from  to , recall from
Section~\ref{subsec:algtolin} that the argument  is
transformed into . We are
therefore not anymore in presence of a linear combination, but of
a program that eventually produces a superposition. But this program
is not a superposition: it is a base state that will be fed unchanged
to an operator. In particular, if this operator is duplicating its
argument, the code  will be
duplicated. But instead of duplicating a superposition of terms, we
are really duplicating the description of a program eventually
producing a superposition.


\subsection{Conclusion}
In this paper we have shown the relation between two algebraic
-calculi,  and , via four canonical languages.
These canonical algebraic -calculi account for all the
different choices we can make between call-by-value versus
call-by-name and algebraic reduction versus algebraic equality. We
showed how each language can simulate the other, by taking care of
marking where confluence was used.

This study opens the door to other questions.  The calculus 
admits finiteness spaces as a model
\cite{EhrhardMSCS05,EhrhardLICS10}. What is the structure of the model
of the linear algebraic -calculus induced by the
continuation-passing style translation in finiteness spaces? The
algebraic lambda-calculus can be equipped with a differential
operator.  What is the corresponding operator in call-by-value through
the translation?  The linear-algebraic lambda-calculus can encode
quantum programs \cite{ArrighiDiazcaroValiron13}. Can this translation
help elucidate the relation between quantum computing and finiteness
spaces?

\paragraph*{Acknowledgements} 
We would like to thank Pablo Arrighi and Lionel Vaux for fruitful discussions and suggestions.

\bibliographystyle{alpha}
\bibliography{biblio}
\appendix

\setlist{topsep=0pt,parsep=0pt,partopsep=0pt,
  labelindent=0ex,labelwidth=1.5ex,labelsep=1.5ex,
leftmargin=3ex,itemindent=0ex}


\newenvironment{myenumerate}{\begin{enumerate}[labelindent=0ex,labelwidth=3ex,labelsep=1.5ex,leftmargin=4.5ex,itemindent=0ex]}{\end{enumerate}}

\begin{absolutelynopagebreak}

  \section{Detailed proofs}\label{sec:detailled-proofs}

  \subsection{Proof of Theorem~\ref{thm:LLINtoLIN}}\label{proof:LLINtoLIN}~

  \recap{Theorem}{thm:LLINtoLIN} 
  If  is closed and strongly normalising in  and ,
  then there exists  such that  and .
\end{absolutelynopagebreak}
\begin{proof}
  We proceed by induction on the  rewrite relation.
  The differences between  rules and  rules are only in the conditions of rules  and , the three first factorisation rules and the context rule . Hence, if , we can just take . So, it suffices to consider only these different rules, when they do not coincide with those in . 
  \begin{myenumerate}
    \item\label{case:leftLin} , with  not normal in  (it is closed by assumption). Let  be the normal form in  of . Cases:
      \begin{itemize}
	\item  with  and . Then we have
	  ,
	  and also
	  .
	\item , with  and . Then we have
	  ,
	  and also
	  .
	\item  with  and . Then we have
	  
	  ,
	  and also
	   .
	\item Cases  with  and , and  with  and  are analogous to the previous case.
      \end{itemize}
    \item , with  not normal in . This case is analogous to case~\ref{case:leftLin}.
    \item\label{case:leftScalarLin} , with  not normal in . Let  be the normal form in  of . Then 
      
      and
      
      .

    \item , with  not normal in . This case is analogous to case~\ref{case:leftScalarLin}.

    \item . Notice that .

    \item . Notice that .

    \item\label{thm:LLINtoLIN:it:fact} , when  is not normal. Let  be the normal form in  of . Then , and .
    \item  and , when  is not normal. Analogous to case \ref{thm:LLINtoLIN:it:fact}.
    \item , with . By the induction hypothesis, there exists  such that  and . Hence we have  and also . \qedhere
  \end{myenumerate}
\end{proof}

\subsection{Proof of Theorem~\ref{thm:LINtoLLIN}}\label{proof:LINtoLLIN}~
\mynobreakpar
\recap{Theorem}{thm:LINtoLLIN}
If  is closed and strongly normalising in  and  , then there exists  such that  and .\mynobreakpar
\begin{proof}
  We only need to verify the seven differing rules between the two languages.
  Notice that the normal form of a closed term is a value .
  \begin{myenumerate}
    \item , with  closed normal. 
      Let , then we have
      ,
      and also
      .
    \item . A value  can be  or a linear combination of base terms.
      \begin{itemize}
	\item Let  be the normal form of . Hence,
	  
	  and
	  
	\item Let  be the normal form of . Then
	  
	   
	   
	   
	   
	  ,
	  and
	  
	   
	   .
      \end{itemize}

    \item . Let , then 
      
      ,
      and also
      .

    \item .  
      \begin{itemize}
	\item Let  be the normal form of . Hence,
	  
	  and
	  .
	\item Let  be the normal form of . Then 
	  
	  
	  
	  
	  and
	  
	  
	  
	  .
      \end{itemize}

    \item . Let  be the normal form of . Then
      .

    \item .
      \begin{itemize}
	\item Let  be the normal form of , then
	  .
	\item Let  be the normal form of . Then
	  
	  
	  
	  
	  
	  .
      \end{itemize}

    \item , with . Let  be the normal form of . Then
      
      
      and
      .
      \qedhere
  \end{myenumerate}
\end{proof}

\subsection{Proof of Theorem~\ref{thm:eqred}}\label{proof:eqred}~
\mynobreakpar
\recap{Theorem}{thm:eqred}
For any term  in a confluent fragment of , if , then , with .
\begin{proof}
  First note that a value can only reduce to another value. This
  follows by direct inspection of the rewriting rules.
We proceed by induction on the length of the reduction.
  \begin{itemize}
    \item If , then choose  and note that
      .
    \item Assume the result true for : there is a value
       such that  and . Let
      . Case distinction:
      \begin{itemize}
	\item , then  which implies
	  .
	\item , then either , and then this
	  case is analogous to the previous one, or . Due to
	  the confluence of the subset, there exists a term  such that
	   and , implying that  is a
	  value, thus . Then we have  and
	  , so , closing the case.\qedhere
      \end{itemize}
  \end{itemize}
\end{proof}


\subsection{Proof of Lemma~\ref{lem:substitution2}}\label{proof:substitution2}~

\recap{Lemma}{lem:substitution2}
 with  a base term.
\begin{proof}
  Structural induction on .
  \begin{itemize}
    \item . Cases:
      \begin{itemize}
	\item . Then , and so  and this is equal to .
	\item . Then .
      \end{itemize}
    \item . Then .
    \item . Analogous to previous case.
    \item . Then 
      
      
      , 
      which by the induction hypothesis is 
      
      
      .
    \item . Then
      
      \!
      \!
      ,
      which, by the induction hypothesis, is equal to
      ,
      which can be rewritten as
      
      
      .
    \item . Then 
      
      
      
      which is equal to
      ,
      and this, by the induction hypothesis is
      
      
      
      .
    \item . Then 
      
      \!
      \!
      ,
      which, by the induction hypothesis, is equal to
      
      \!
      
      .
      \qedhere
  \end{itemize}
\end{proof}

\subsection{Proof of Lemma~\ref{lem:lemma2}}\label{proof:lemma2}~

\recap{Lemma}{lem:lemma2}
If  is a base term, then for any term , .
\begin{proof}
  Structural induction on .
  \begin{itemize}
    \item . Then  and by definition of  this is equal to .
    \item . Then .
    \item . Then  which -reduces to   which -reduces by the induction hypothesis to .
    \item . Then  which -reduces to  and this, by the induction hypothesis, -reduces to . 
    \item . Then  which -reduces to . 
      Since  is a value, by the induction hypothesis 
      
      reduces to .
      We do a second induction, over , to prove that .
      \begin{itemize} 
	\item If , then .
	\item If  is a base term, then  is by definition equal to the term   which by the main induction hypothesis -reduces to  , and this is equal to .
	\item If , then the term  is equal to  which by the second induction hypothesis -reduces to .
	\item If , then  is equal to the term  which is equal to  which -reduces by the second induction hypothesis to .
	\item If  then .\qedhere
      \end{itemize}
  \end{itemize}
\end{proof}

\subsection{Proof of Lemma~\ref{lem:lemma3alg}}\label{proof:lemma3alg}~

\recap{Lemma}{lem:lemma3alg}
If  then for all  base term, .
\begin{proof}
  Case by case on the rules .
  \begin{description}
    \item[Rules ]~
      \begin{itemize}
	\item , with  being a base term. Then .
	\item , with  a base
	  term. Then  is equal to the term
	   , which is .
	\item , with  a base term. Then .
      \end{itemize}
    \item[Rules ]~\mynobreakpar
      \begin{itemize}
	\item , with  being a value. Then .
	\item , with  being a value. Then .
	\item , with  a value. Then .
      \end{itemize}
    \item[Rules  and ]~
      \begin{itemize}
	\item . Then .
	\item . Then .
	\item . Then  .
	\item . Then  .
	\item . Then  .
	\item . Then   which -reduces to .
	\item . Then .
	\item . Then  .
	\item . Then  .
      \end{itemize}
    \item[Rules  and ]~
      \begin{itemize}
	\item . Then .
	\item . Then .
      \end{itemize}
    \item[Rules  and ] Assume , and assume that for all  base term, . We show that the result also holds for each contextual rule.
      \begin{itemize}
	\item . Then .
	\item , analogous to previous case.
	\item . Then .
	\item . Case by case:
	  \begin{itemize}
	    \item .  Then  which -reduces by the induction hypothesis to .
	    \item . Then .
	    \item . Then   which -reduces by the induction hypothesis to .
	    \item . Then   which -reduces by the induction hypothesis to .
	  \end{itemize}
	\item  Case by case:
	  \begin{itemize}
	    \item . Absurd since a base term cannot reduce.
	    \item . Case by case on the possible -reductions of :
	      \begin{itemize}
		\item  with . Then  which by the induction hypothesis -reduces to .
		\item  and . Then .
		\item  and . Then .
		\item  and . Then  which \toalgred-reduces to .
		\item  and . Then .
		\item  and . Then .
	      \end{itemize}
	    \item . Case by case on the possible -reductions of :
	      \begin{itemize}
		\item  with . Then  which by the induction hypothesis -reduces to .
		\item  with . Analogous to previous case.
		\item  and . Then  and this -reduces to .
		\item  and . Analogous to previous case.
		\item . Then .
		\item ,  and . Then  which \toalgred-reduces to .
		\item ,  and . Analogous to previous case.
		\item  and . Analogous to previous case.
	      \end{itemize}
	    \item . Absurd since  does not reduce.
	    \item . Then the term  is equal to , which by the induction hypothesis -reduces to .
	      We do a second induction, over , to prove that .
	      \begin{itemize} 
		\item If , then  is equal to .
		\item  cannot be a base term since from  it is not possible to arrive to a base term using only .
		\item If , then  is equal to the term  which -reduces by the induction hypothesis to .
		\item If , then the term  is equal to  which is equal to  which -reduces by the induction hypothesis to .
		\item If  then  is equal  and this to .\qedhere
	      \end{itemize}
	  \end{itemize}
      \end{itemize}
  \end{description} 
\end{proof}

\subsection{Proof of Lemma~\ref{lem:lemma3}}\label{proof:lemma3}~

\recap{Lemma}{lem:lemma3}
If  then for all  base term, .
\begin{proof}
  Case by case on the rules .
  \begin{description}
    \item[Rule ]
      
      
      
      
      
      ,
      which, by Lemma~\ref{lem:substitution2}, is equal to
      , and this, by Lemma~\ref{lem:lemma2},
      \stobalgred-reduces to
      .
    \item[Algebraic rules] If , then by Lemma~\ref{lem:lemma3alg}  which implies that .
    \item[Rules  and ] If , then we use Lemma~\ref{lem:lemma3alg} to close the case. 
      Assume , and assume that for all  base term, . We show that the result also holds for each contextual rule.
      \begin{itemize}
	\item . Then .
	\item , analogous to previous case.
	\item . Then .
	\item . Case by case:
	  \begin{itemize}
	    \item .  Then  which -reduces by the induction hypothesis to .
	    \item . Then .
	    \item . Then   which -reduces by the induction hypothesis to .
	    \item . Then   which -reduces by the induction hypothesis to .
	  \end{itemize}
	\item  Case by case:
	  \begin{itemize}
	    \item . Absurd since a base term cannot reduce.
	    \item . The only possible -reduction from  is  with . Then  which by the induction hypothesis -reduces to .
	    \item . Case by case on the possible -reductions of :
	      \begin{itemize}
		\item  with . Then  which by the induction hypothesis -reduces to .
		\item  with . Analogous to previous case.
	      \end{itemize}
	    \item . Absurd since  does not reduce.
	    \item . Then the term  is equal to , which -reduces, by the induction hypothesis, to .
	      We do a second induction, over , to prove that  -reduces to .
	      \begin{itemize} 
		\item If , then  is equal to .
		\item If  is a base term, then the term  is equal to  which -reduces to   which, by Lemma~\ref{lem:lemma2}, -reduces .
		\item If , then  is equal to the term  which -reduces by the induction hypothesis to .
		\item If , then the term  is equal to  which is equal to  which -reduces by the induction hypothesis to .
		\item If  then  is equal to .\qedhere
	      \end{itemize}
	  \end{itemize}
      \end{itemize}
  \end{description} 
\end{proof}

\subsection{Proof of Lemma~\ref{lem:inverse-term}}\label{proof:inverse-term}~

\recap{Lemma}{lem:inverse-term} For any term , .

\begin{proof}
  By induction on .
  \begin{itemize}
    \item Case . Then .
    \item Case . Then 
      
      
      
      
      
      ,
      which, by the induction hypothesis, is equal to
      .

    \item Case . Then 
      
      
      which is equal to
      
      
      which is equal to
      
      ,
      which, by the induction hypothesis, is equal to
      .

    \item Case . Then 
    \item Case . Then 
      
      
      
      
      
      ,
      which, by the induction hypothesis, is equal to
      .

    \item Case . Then
      
      
      
      
      
      ,
      which, by the induction hypothesis, is equal to
      .
      \qedhere
  \end{itemize}
\end{proof}

\subsection{Proof of Lemma \ref{lem:inverse-value}}\label{proof:inverse-value}~

\recap{Lemma}{lem:inverse-value} For any value , .
\begin{proof}
  By induction on .
  \begin{itemize}
    \item Case . Then .
    \item Case . Then 
      by Lemma \ref{lem:inverse-term}. 
    \item Case . Then 
      by the induction hypothesis.
    \item Case . Then 
      by the induction hypothesis.
      \qedhere
  \end{itemize}
\end{proof}

\subsection{Proof of Lemma~\ref{lem:inverse-step}}\label{proof:inverse-step}~

\recap{Lemma}{lem:inverse-step} For any computation , if 
then . Also, if 
then .

\bigskip
To prove this lemma, we need several intermediary results.

\begin{lemma}\label{lem:substitution-lemma}
  The following equalities hold.
  \begin{myenumerate}
    \item 
    \item 
    \item 
    \item 
  \end{myenumerate}
\end{lemma}
\begin{proof}
  We prove simultaneously the four properties by induction on the structure of , , , and .
  \begin{myenumerate}
    \item Cases for .
      \begin{itemize}
	\item Case . Then .
	\item Case . Then .
	\item Case . Then 
	  
	  ,
	  which, by the induction hypothesis, is equal to
	  
	  .
      \end{itemize}
    \item Cases for .
      \begin{itemize}
	\item Case . Then 
	  
	  ,
	  which, by the induction hypothesis, is equal to
	  
	  .

	\item Case . Then .
	\item Case . Then 
	  
	  ,
	  which, by the induction hypothesis, is equal to
	  
	  .

	\item Case . Then 
	  
	  ,
	  which, by the induction hypothesis, is equal to
	  
	  .
      \end{itemize}
    \item Cases for .
      \begin{itemize}
	\item Case . Then 
	  
	  ,
	  which, by the induction hypothesis, is equal to
	  ,
	  which, by the induction hypothesis, is 
	  
	  .

	\item Case . Then
	  
	  ,
	  which, by the induction hypothesis, is 
	  ,
	  which, by the induction hypothesis, is equal to
	  
	  .

	\item Case . Then 
	  
	  ,
	  which, by the induction hypothesis, is equal to
	  ,
	  which, by the induction hypothesis, is equal to
	  
	  .
      \end{itemize}
    \item Cases for .
      \begin{itemize}
	\item Case . Then .
	\item Case . Then 
	  
	  ,
	  which, by the induction hypothesis, is 
	  ,
	  which, by the induction hypothesis, is equal to
	  
	  .

	\item Case . Then
	  
	  is the term
	  ,
	  which, by the induction hypothesis, is equal to
	  ,
	  which, by the induction hypothesis, is equal to
	  
	  and this is finally equal to the term
	  .
	  \qedhere
      \end{itemize}
  \end{myenumerate}
\end{proof}

\begin{lemma}\label{lem:continuation-composition}
  For all terms  and continuations
   and , 
  .
\end{lemma}
\begin{proof}
  By induction on the structure of . 
  \begin{itemize}
    \item Case . Then .
    \item Case . Then 
      
      ,
      which, by the induction hypothesis, is 
      
      .

    \item Case . Then 
      
      which is equal to
      ,
      which, by the induction hypothesis, is 
      , which is equal to the term
      .
      \qedhere
  \end{itemize}
\end{proof}

\begin{lemma}\label{lem:continuation-substitution}
  For all  and ,
  .
\end{lemma}
\begin{proof}
  By induction on the structure of , using Lemma \ref{lem:continuation-composition}
  where necessary.
  \begin{itemize}
    \item Case . Then 
      
      ,
      which, by Lemma \ref{lem:continuation-composition}, is equal to
      
      .

    \item Case . Then 
      
      ,
      which, by Lemma \ref{lem:continuation-composition}, is equal to
      
      .

    \item Case . Then 
      
      ,
      which, by Lemma \ref{lem:continuation-composition}, is equal to
      
      .
      \qedhere
  \end{itemize}
\end{proof}

\begin{lemma}\label{lem:continuation-step}
  For any continuation  and term
  , if , then .
\end{lemma}
\begin{proof}
  By induction on the structure of .
  \begin{itemize}
    \item Case . Then .
    \item Case . Then  since 
      is a base term, and 
      
      ,
      which, by the induction hypothesis, \toblinred-reduces to
      
      .

    \item Case . Then we have
      . Hence, we have
      
      ,
      which, by the induction hypothesis, \toblinred-reduces to
      
      .
      \qedhere
  \end{itemize}
\end{proof}

\begin{lemma}\label{lem:continuation-linearity}
  The following relations hold.
  \begin{itemize}
    \item 
    \item 
    \item 
  \end{itemize}
\end{lemma}
\begin{proof}
  We prove each statement by induction on , using Lemma \ref{lem:continuation-step}
  where necessary. We prove only the first statement, as the others
  are similar.
  \begin{itemize}
    \item Case . Then .
    \item Case . Then 
      
      ,
      which, by Lemma \ref{lem:continuation-step}, \tolinred-reduces to
      ,
      which, by the induction hypothesis, \stolinred-reduces to
      
      .

    \item Case . Then
      the term
      
      is equal to
      
      which, by Lemma \ref{lem:continuation-step}, \tolinred-reduces to
      ,
      which, by the induction hypothesis, \stolinred-reduces to
      
      which is equal to
      .
      \qedhere
  \end{itemize}
\end{proof}

\begin{lemma}\label{lem:suspension-step}
  For any suspension , if 
  then .
\end{lemma}
\begin{proof}
  By induction on the reduction rule. Since  terms do not contain
  applications, the only cases possible are , which are common
  to both languages.
  \begin{itemize}
    \item Case . Using linearity of . We give the following example.
      
      
      
      .

    \item Case . Using linearity and the induction hypothesis. We give
      the following example. Consider the case 
      with . Then 
      
      ,
      which, by the induction hypothesis, \tolinred-reduces to
      
      .
      \qedhere
  \end{itemize}
\end{proof}

We now have the tools to prove the Lemma~\ref{lem:inverse-step}.

\begin{proof}[\bf Proof of Lemma~\ref{lem:inverse-step}]
  By induction on the reduction rule, using Lemmas \ref{lem:substitution-lemma},
  \ref{lem:continuation-substitution}, \ref{lem:continuation-step},
  \ref{lem:continuation-linearity} and \ref{lem:suspension-step} where
  necessary.
  \begin{itemize}
    \item Case . There are several sub-cases.
      \begin{itemize}
	\item Case . Then 
	  
	  is equal to
	  
	  
	  .

	\item Case . Then we have that
	  
	  
	  
	  
	  .

	\item Case . Then
	  
	  ,
	  which, by Lemma \ref{lem:continuation-substitution}, is equal to
	  

	\item Case . Then  is a
	  base term, and hence ,
	  so
	  
	  ,
	  which, by Lemma \ref{lem:continuation-step}, \toblinred-reduces to
	  ,
	  which, by Lemma \ref{lem:substitution-lemma}, is equal to
	  
	  .
      \end{itemize}
    \item Case . Since  and  are base terms, the only term that
      can match the rules is . There are three sub-cases. 
      \begin{itemize}
	\item Case . Then 
	  
	  ,
	  which, by Lemma \ref{lem:continuation-linearity}, \stolinred-reduces to
	  
	  .
	\item Case . Then 
	  ,
	  
	  which, by Lemma \ref{lem:continuation-linearity}, \stolinred-reduces to
	  
	  .
	\item Case . Then 
	  ,
	  
	  which, by Lemma \ref{lem:continuation-linearity}, \stolinred-reduces to
	  
	  .
      \end{itemize}
    \item Case . Since the rules in  are common to both languages and
      the inverse translation  distributes linearly over
      the computations, the proof for these cases is straightforward. We
      give the following example. Consider .
      Then
      
      ,
      and this \tolinred-reduces to
      
      .

    \item Case . There are 4 sub-cases.
      \begin{itemize}
	\item Case  and . Then 
	  by Lemma \ref{lem:suspension-step}, therefore 
	  
	  ,
	  by Lemma \ref{lem:continuation-step}, \tolinred-reduces to
	  
	  .

	\item The other three cases are similar to each other. We give the following
	  example. Consider  and .
	  Then by the induction hypothesis ,
	  therefore 
	  
	  
	  
	  .
	  \qedhere
      \end{itemize}
  \end{itemize}
\end{proof}

\subsection{Proof of Lemma~\ref{lem:substitution-cps}}\label{proof:substitution-cps}~

\recap{Lemma}{lem:substitution-cps}
.
\begin{proof}
  Structural induction on .
  \begin{itemize}
    \item . Then .
    \item . Then .
    \item . Analogous to previous case.
    \item . Then
      
      ,
      which, by the induction hypothesis, is equal to
      
      
      
      .
    \item . Then 
      
      ,
      which is equal to the\break term
      ,
      and this, by the induction hypothesis, is equal to
       
      
      
      
      
      .
    \item . Then
      
      
      ,
      which, by the induction hypothesis, is equal to
      
      
      
      
      .
    \item . Then 
      
      
      which is equal to the term
      ,
      which, by the induction hypothesis, is 
      
      , which is
      
      
      .
      \qedhere
  \end{itemize}
\end{proof}

\subsection{Proof of Lemma~\ref{lem:lemma2cps}}\label{proof:lemma2cps}~

\recap{Lemma}{lem:lemma2cps}
If  is a base term,  for any term 
.
\begin{proof}
  Structural induction on .
  \begin{itemize}
    \item . Then .
    \item . Then  and by definition of  this is equal to .
    \item . Then .
    \item . Then   which -reduces by the induction hypothesis to .
    \item . Then  which -reduces to  and this, by the induction hypothesis, -reduces to . 
    \item . Then  which -reduces to . Note that  is a value, so by the induction hypothesis the above term reduces to . 
      We do a second induction, over , to prove that .
      \begin{itemize} 
	\item If , then .
	\item If  then .
	\item If  then  which -reduces to .
	\item If , then   which -reduces by the induction hypothesis to .
	\item If , then  which is equal to  which -reduces by the induction hypothesis to .
	\item If  then .\qedhere
      \end{itemize}
  \end{itemize}
\end{proof}

\subsection{Proof of Lemma~\ref{lem:lemma3cps}}\label{proof:lemma3cps}~

\recap{Lemma}{lem:lemma3cps}
If  then for all  base term, 
\begin{proof}
  Case by case on the rules of .
  \begin{description}
    \item[Rule ]
      
      
      .
      Since  is a base term, this last term \toblinred-reduces to
      ,
      which, by Lemma~\ref{lem:substitution-cps}, is equal to
      ,
      and this, by Lemma~\ref{lem:lemma2cps}, \stoblinred-reduces to
      .
    \item[Rules ]~
      \begin{itemize}
	\item Let  . .
	\item Let  . 
	\item Let  .  
      \end{itemize}
    \item[Rules  and ]~
      \begin{itemize}
	\item . Then .
	\item . Then .
	\item . Then  .
	\item . Then  .
	\item . Then  .
	\item . Then   which -reduces to .
	\item . Then .
	\item . Then  .
	\item . Then  .
      \end{itemize}
    \item[Rules  and ]~
      \begin{itemize}
	\item . Then .
	\item . Then .
      \end{itemize}
    \item[Rules ] Assume , and that for all  base term, . We show that the result also holds for each contextual rule.
      \begin{itemize}
	\item . Then .
	\item , analogous to previous case.
	\item . Then .
	\item  Case by case:
	  \begin{itemize}
	    \item . Absurd since a base term cannot reduce.
	    \item . Case by case on the possible -reductions of :
	      \begin{itemize}
		\item  with . Then   which, by the induction hypothesis, -reduces to .
		\item  and . Then .
		\item  and . Then .
		\item  and . Then  and this \toblinred-reduces to .
		\item  and . Then .
		\item  and . Then .
	      \end{itemize}
	    \item . Case by case on the possible -reductions of :
	      \begin{itemize}
		\item  with . Then  which by the induction hypothesis -reduces to .
		\item  with . Analogous to previous case.
		\item  and . Then  which -reduces to .
		\item  and . Analogous to previous case.
		\item . Then .
		\item ,  and . Then .
		\item ,  and . Analogous to previous case.
		\item  and . Analogous to previous case.
	      \end{itemize}
	    \item . Absurd since  does not reduce.
	    \item . Then , which by the induction hypothesis -reduces to .
	      We do a second induction, over , to prove that  \stoblinred-reduces to .
	      \begin{itemize} 
		\item If , then .
		\item If  then .
		\item If  then  is \
		  .
		\item If , then   which -reduces by the induction hypothesis to .
		\item If , then  which is equal to  which -reduces by the induction hypothesis to .
		\item If  then .\qedhere
	      \end{itemize}
	  \end{itemize}
      \end{itemize}
  \end{description} 
\end{proof}

\subsection{Proof of Lemma \ref{lem:inverse-term-a}}\label{proof:inverse-term-a}~
\mynobreakpar
\recap{Lemma}{lem:inverse-term-a} For any term , .\mynobreakpar
\begin{proof}
  By induction on .\mynobreakpar
  \begin{itemize}
    \item Case . Then .
    \item Case . Then 
      
      
      
      
      
      ,
      which, by the induction hypothesis, is equal to
      .

    \item Case . Then 
      
      
      which is equal to
      
      
      , equal to
      ,
      which, by the induction hypothesis, is equal to
      .

    \item Case . Then 
    \item Case . Then 
      
      , which is equal to
      the term
      
      
      
      ,
      which, by the induction hypothesis, is equal to
      .

    \item Case . Then
      
      
      
      
      
      ,
      which, by the induction hypothesis, is equal to
      .
      \qedhere
  \end{itemize}
\end{proof}


\subsection{Proof of Lemma \ref{lem:inverse-value-a}}\label{proof:inverse-value-a}~

\recap{Lemma}{lem:inverse-value-a} For any value , .
\begin{proof}
  By induction on .
  \begin{itemize}
    \item Case . Then .
    \item Case . Then .
    \item Case . Then 
      by Lemma \ref{lem:inverse-term-a}.
    \item Case . Then 
      by the induction hypothesis.
    \item Case . Then 
      by the induction hypothesis.
      \qedhere
  \end{itemize}
\end{proof}


\subsection{Proof of Lemma \ref{lem:inverse-step-a}}\label{proof:inverse-step-a}~

\recap{Lemma}{lem:inverse-step-a} For any computation , if 
then . Also, if 
then .

\bigskip
In order to prove this lemma, we need intermediary results similar to Lemmas \ref{lem:substitution-lemma}, \ref{lem:continuation-composition}, \ref{lem:continuation-substitution}, \ref{lem:continuation-step}, \ref{lem:continuation-linearity}, and \ref{lem:suspension-step}.

\begin{lemma}
  \label{lem:substitution-lemma-a} The following equalities hold.
  \begin{myenumerate}
    \item {}
    \item 
    \item {}
    \item 
  \end{myenumerate}
\end{lemma}
\begin{proof}
  We prove simultaneously the four properties by induction on the structure of , ,  and .
  \begin{myenumerate}
    \item Cases for .
      \begin{itemize}
	\item Case . Then 
	  
	  ,
	  which, by the induction hypothesis, is equal to
	  
	  .
      \end{itemize}
    \item Cases for .
      \begin{itemize}
	\item Case . Then .
	\item Case . Then .
	\item Case . Then 
	  
	  ,
	  which, by the induction hypothesis, is equal to
	  
	  .

	\item Case . Then .
	\item Case . Then 
	  
	  ,
	  which, by the induction hypothesis, is equal to
	  
	  .

	\item Case . Then 
	  
	  ,
	  which, by the induction hypothesis, is equal to
	  
	  .
      \end{itemize}
    \item Cases for .
      \begin{itemize}
	\item Case . Then 
	  
	  ,
	  which, by the induction hypothesis, is equal to
	  ,
	  which, by the induction hypothesis, is equal to
	  
	  .

	\item Case . Then
	  
	  ,
	  which, by the induction hypothesis, is equal to
	  ,
	  which, by the induction hypothesis, is equal to
	  
	  .

	\item Case . Then 
	  
	  ,
	  which, by the induction hypothesis, is equal to
	  ,
	  which, by the induction hypothesis, is equal to
	  
	  
      \end{itemize}
    \item Cases for .
      \begin{itemize}
	\item Case . Then .
	\item Case . Then 
	  
	  ,
	  which, by the induction hypothesis, is 
	  ,
	  which, by the induction hypothesis, is equal to
	  
	  .
	  \qedhere
      \end{itemize}
  \end{myenumerate}
\end{proof}

\begin{lemma}
  \label{lem:continuation-composition-a} For all terms  and continuations
   and , we have,
  .
\end{lemma}
\begin{proof}
  By induction on .
  \begin{itemize}
    \item Case . Then .
    \item Case . Then 
      
      ,
      which, by the induction hypothesis, is equal to
      
      .
      \qedhere
  \end{itemize}
\end{proof}

\begin{lemma}
  {\label{lem:continuation-substitution-a} For all  and
  , }.
\end{lemma}
\begin{proof}
  By induction on the structure of , using Lemma \ref{lem:continuation-composition-a}
  where necessary.
  \begin{itemize}
    \item Case . Then 
      
      ,
      which, by Lemma \ref{lem:continuation-composition-a}, is equal to
      
      .

    \item Case . Then 
      
      ,
      which, by Lemma \ref{lem:continuation-composition-a}, is equal to
      
      .

    \item Case . Then 
      
      ,
      which, by Lemma \ref{lem:continuation-composition-a}, is equal to
      
      .
      \qedhere
  \end{itemize}
\end{proof}

\begin{lemma}
  \label{lem:continuation-step-a}For any continuation  and term
  , if  then .
\end{lemma}
\begin{proof}
  By induction on the structure of .
  \begin{itemize}
    \item Case . Then .
    \item Case . Then we have , and 
      
      ,
      and this, by the induction hypothesis, \tobalgred-reduces to 
      
      .
      \qedhere
  \end{itemize}
\end{proof}

\begin{lemma}
  \label{lem:continuation-linearity-a}
  For any continuation , scalar  and terms ,  and , the following relations hold.
  \begin{itemize}
    \item 
    \item 
    \item 
  \end{itemize}
\end{lemma}

\begin{proof}
  We prove each statement by induction on , using Lemma \ref{lem:continuation-step-a}
  where necessary.We prove only the first statement, as the others
  are similar.
  \begin{itemize}
    \item Case . Then .
    \item Case . Then 
      
      ,
      which, by Lemma \ref{lem:continuation-step}, \toalgred-reduces to
      ,
      and this, by the induction hypothesis, \stoalgred-reduces to
      
      
      \qedhere
  \end{itemize}
\end{proof}

\begin{lemma}
  \label{lem:suspension-step-a} For any suspension , if 
  then .
\end{lemma}
\begin{proof}
  By induction on the reduction rule. Since  terms do not contain
  applications, the only cases possible are , which are common
  to both languages.
  \begin{itemize}
    \item Case . Using linearity of . We give the following example.
      
      
      
      .

    \item Case . Using linearity and the induction hypothesis. We give
      the following example. Consider the case 
      with . Then 
      
      ,
      which, by the induction hypothesis, \toalgred-reduces to
      
      .
      \qedhere
  \end{itemize}
\end{proof}

We now have the tools to prove the Lemma~\ref{lem:inverse-step-a}.

\begin{proof}[\bf Proof of Lemma~\ref{lem:inverse-step-a}]
  By induction on the reduction rule, using Lemmas \ref{lem:substitution-lemma-a},
  \ref{lem:continuation-substitution-a}, \ref{lem:continuation-step-a},
  \ref{lem:continuation-linearity-a} and \ref{lem:suspension-step-a}
  where necessary. The rules  and
   are not applicable since arguments in the target language
  are always base terms.
  \begin{itemize}
    \item Case . There are several sub-cases.
      \begin{itemize}
	\item Case . Then 
	  
	  which is equal to
	  
	  
	  .

	\item Case . Then
	  
	  ,
	  which, by Lemma \ref{lem:continuation-substitution-a}, is equal to
	  .

	\item Case . Then ,
	  and
	  
	  ,
	  which, by Lemma \ref{lem:continuation-step-a}, \tolinred-reduces to
	  ,
	  which, by Lemma \ref{lem:substitution-lemma-a}, is equal to
	  
	  .
      \end{itemize}
    \item Case . Since  and  are base terms, the only term that
      can match the rules is . There are three sub-cases. 
      \begin{itemize}
	\item Case . Then 
	  
	  ,
	  which, by Lemma \ref{lem:continuation-linearity-a}, \stoalgred-reduces to
	  
	  .
	\item Case . Then 
	  ,
	  ,
	  which, by Lemma \ref{lem:continuation-linearity-a}, \stoalgred-reduces to
	  
	  .
	\item Case . Then 
	  
	  ,
	  which, by Lemma \ref{lem:continuation-linearity-a}, \stoalgred-reduces to
	  
	  .
      \end{itemize}
    \item Case . Since the rules in  are common to both languages and
      the inverse translation  distributes linearly over
      the computations, the proof for these cases is straightforward. We
      give the following example. Consider .
      Then
      
      
      ,
      which is equal to
      .

    \item Case . There are 4 sub-cases.
      \begin{itemize}
	\item Case  and . Then 
	  by Lemma \ref{lem:suspension-step-a}, therefore 
	  
	  ,
	  which, by Lemma \ref{lem:continuation-step-a}, \toalgred-reduces to
	  
	  .

	\item The other three cases are similar to each other. We give the following
	  example. Consider  and .
	  Then by the induction hypothesis ,
	  therefore 
	  
	  
	  
	  .
	  \qedhere
      \end{itemize}
  \end{itemize}
\end{proof}



\end{document}
