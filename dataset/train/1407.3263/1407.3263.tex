We present our new relaxation for the capacitated facility location problem in this section. Let us first define some notation to be used in the rest of this paper. Let $\cF$ be the set of facilities and $\cD$ be the set of clients. Each facility $i\in\cF$ has opening cost $o_i$, and cannot be assigned more number of clients than its capacity $U_i$. We are also given a metric cost $c$ on $\cF\cup\cD$ as a part of the input: $c_{ij}$ denotes the distance between $i\in\cF$ and $j\in\cD$.

The variables of our relaxation is the pair $(\vec{x}, \vec{y})$ where we refer to $\vec{x}\in
[0,1]^{\cF \times \cD}$ as the \emph{assignment variables} and to $\vec{y}\in [0,1]^{\cF}$ as the \emph{opening variables}. These variables naturally encode the decisions to which facility a client is connected
and which facilities that are opened. Indeed, the intended integral solution is that $x_{ij} =1$ if
client $j$ is connected to facility $i$ and $x_{ij} = 0$ otherwise; $y_i=1$ if facility $i$ is
opened and $y_i = 0$ otherwise. The idea of our relaxation is that every partial assignment of
clients to facilities should be extendable to a complete assignment while only using the assignments
of $\vec{x}$ and openings of $\vec{y}$. To this end let us first describe the partial assignments
that we shall consider. We then define the constraints of our linear program which will be
feasibility constraints of multi-commodity flows.

A partial fractional assignment $\vec{g}=\{g_{ij}\}_{i\in \cF, j \in \cD}$ of clients to facilities is  \emph{valid} if
$$
\forall j\in
\cD: \sum_{i\in \cF} g_{ij} \leq 1   \mbox{,} \qquad  \forall i \in \cF: \sum_{j\in \cD}
g_{ij} \leq U_i \qquad \mbox{and} \qquad  \forall i \in \cF, j\in \cD: g_{ij}\geq 0.
$$
The first condition says that each client should be fractionally assigned at most once and the
second condition says that no facility should receive more clients than its capacity. We emphasize
that we allow clients to be fractionally assigned, \ie $\vec{g}$ is not assumed to  be integral. As we
shall see later (see Lemma~\ref{lem:bing}), this does not change the strength of our relaxation but it will
be convenient in the analysis of our rounding algorithm in Section~\ref{sec:algorithm}. We also remark that
the above inequalities are exactly the $b$-matching polytope of the complete bipartite graph
consisting of the clients on the one side and the facilities on the other side; each client can be
matched to at most one facility and each facility $i$ can be matched to at most $U_i$ clients.

The constraints of our relaxation will be that, no matter how we partially assign the clients according to a valid
$\vec{g}$, the solution $(\vec{x},\vec{y})$ should support a multi-commodity flow where each client $j$ becomes the source of its own commodity $j$, and the demand of this commodity is equal to the amount by which $j$ is ``not assigned'' by $\vec g$, $1-\sum_{i\in \cF} g_{ij}$. The flow network, whose arc capacities are given as a function of $\vec{g}$ and the solution $(\vec{x},\vec{y})$, is defined as follows (see also Figure~\ref{fig:MFN}):
\begin{definition}[Multi-commodity flow network]
\label{def:mfn}
For a valid partial assignment $\vec{g}$,  assignment variables $\vec{x} = \{x_{ij}\}_{i\in \cF, j\in \cD}$, and opening variables $ \vec{y}=
\{y_i\}_{i\in \cF}$, let $\MFN(\vec{g},\vec{x},\vec{y})$ be a
multi-commodity flow network with $|\cD|$ commodities, defined as follows. Note that some arcs may have zero capacities.
\begin{enumerate}[(a)]
\item Each client $j\in \cD$ is associated with commodity $j$ of demand $d_j := 1 - \sum_{i\in \cF} g_{ij}$, and its source-sink pair is $(j^s, j^t)$.  \item Each facility $i\in \cF$ has two nodes $i$ and $i'$ with an arc $( i,i')$ of capacity $y_i\cdot( U_i - \sum_{j\in \cD} g_{ij})$.  \item For each client $j$ and facility $i$, there is an arc $( j^s, i)$ of capacity $x_{ij}$, an arc
  $( i, j^s) $ of capacity $g_{ij}$, and an arc $( i', j^t) $ of capacity $y_i d_j$.
\end{enumerate}
\end{definition}


\begin{figure}[tb]
\begin{center}
\begin{tikzpicture}

\node[client] (j1) at (-1.5, 0)  {$j^s_1$};
\node[client] (j2) at (-1.5, -1.5)  {$j^s_2$};
\node at (-1.5,-3) {$\vdots$};
\node[client] (jn) at (-1.5, -4.5)  {$j^s_n$};

\node (d) at (-1.7, 1.5) {\small demand $d_{j_1} = 1 - \sum_{i\in \cF} g_{ij_1}$};

\draw (d) edge[dotted] (j1);

\node[facility] (i1) at (1.5,-0.5) {$i_1$};

\node[facility] (i2) at (1.5,-1.75) {$i_2$};

\node at (1.5,-2.9) {$\vdots$};
\node[facility] (im) at (1.5,-4) {$i_m$};

\node[facility] (i'1) at (6.5,-0.5) {$i'_1$};

\node[facility] (i'2) at (6.5,-1.75) {$i'_2$};

\node at (6.5,-2.9) {$\vdots$};
\node[facility] (i'm) at (6.5,-4) {$i'_m$};

\node[client] (t1) at (9.7,0 )  {$j^t_1$};
\node[client] (t2) at (9.7,-1.5 )  {$j^t_2$};
\node at (9.7,-3 ) {$\vdots$};
\node[client] (tn) at (9.7,-4.5 )  {$j^t_n$};



\draw (j1) edge[->,ultra thick, bend left] node[above] {\small capacity $x_{i_1j_1}$}   (i1);
\draw[gray] (j1) edge[<->] (i2);
\draw[gray] (j1) edge[<->] (im);

\draw[gray] (j2) edge[<->] (i1);
\draw[gray] (j2) edge[<->] (i2);
\draw[gray] (j2) edge[<->] (im);

\draw[gray] (jn) edge[<->] (i1);
\draw[gray] (jn) edge[<->] (i2);
\draw[gray] (jn) edge[->] (im);

\draw (jn) edge[<-, ultra thick, bend right] node[below] {\small capacity $g_{i_mj_n}$}  (im);


\draw (i1) edge[ultra thick,->] node[above] {\small\begin{tabular}{c}capacity\\$y_{i_1} \cdot (U_{i_1} -  \sum_{j\in \cD} g_{i_1 j})$\end{tabular}} (i'1);
\draw[gray] (i2) edge[->] (i'2);
\draw[gray] (im) edge[->] (i'm);

\draw (i'1) edge[ultra thick,->] node[above=0.2] {\small capacity $ y_{i_1} d_{j_1}$}(t1);
\draw[gray] (i'1) edge[->] (t2);
\draw[gray] (i'1) edge[->] (tn);


\draw[gray] (i'2) edge[->] (t1);
\draw[gray] (i'2) edge[->] (t2);
\draw[gray] (i'2) edge[->] (tn);

\draw[gray] (i'm) edge[->] (t1);
\draw[gray] (i'm) edge[->] (t2);
\draw[gray] (i'm) edge[->] (tn);


  \begin{pgfonlayer}{background}
    \filldraw [line width=4mm,join=round,black!10]
      (-1.8,0.7)  rectangle (1.8,-5.2);
  \end{pgfonlayer}

\end{tikzpicture}
 \end{center}

\caption{A depiction of the multi-commodity flow network $\MFN(\vec{g},\vec{x},\vec{y})$.}
\label{fig:MFN}
\end{figure}

\begin{remark}
Intuitively, the bipartite subgraph induced by $\{j^s\}_{j\in\cD}\cup\{i\}_{i\in\cF}$, marked with a shaded box in Figure~\ref{fig:MFN}, is the interesting part of the flow network. $\{i'\}_{i\in\cF}$ and $\{j^t\}_{j\in\cD}$ are added to this bipartite graph purely in order to state that $i$ is a sink with ``double'' capacities: a commodity-oblivious capacity $y_i\cdot( U_i - \sum_{j\in \cD} g_{ij})$ and a commodity-specific capacity $y_i d_j$ for each client $j\in \cD$.


\end{remark}

Let us give some intuition on the definition of $\MFN(\vec{g},\vec{x},\vec{y})$. As already noted,
the demand $d_j = 1 - \sum_{i\in \cF} g_{ij}$ of a client $j$ equals the amount by which $j$ is not
assigned by the partial assignment $\vec{g}$. This demand should only be assigned to opened
facilities. Therefore, facility $i$ can accept at most $y_i d_j$ of $j$'s demand which is either
$d_j$ or $0$ in an integral solution.
Observe that such a constraint, for each client and facility, cannot be imposed by a single-commodity flow problem.
Multi-commodity flow problems, on the other hand, allows us to express this constraint as a commodity-specific capacity of $y_id_j$, as denoted by arc $(i',j^t)$ in Figure~\ref{fig:MFN}.

Now consider the \emph{commodity-oblivious} capacities of the facilities. The \emph{total} demand an opened facility $i$ can accept
is its capacity minus the amount of clients assigned to it in the partial assignment $\vec{g}$; and
a closed facility can accept no demand. Therefore, the total demand a facility $i$ can accept is at
most $y_i (U_i - \sum_{j\in \cD} g_{ij})$. The arc capacity $x_{ij}$ of an arc $(j,i)$ says that
client $j$ should be connected to facility $i$ only if $x_{ij} =1$. The reason for having arcs of
the form $(i,j)$ of capacity $g_{ij}$ is discussed in Section~\ref{sec:ourcontrib}: these allow the alternating paths for routing the remaining demand and are
necessary for the formulation to be a relaxation.


We are now ready to formally state our relaxation \MFNLP of the  capacitated facility
location problem in Figure~\ref{fig:MFN-LP}. Note that the only variables of our relaxation are the assignment variables $\vec x$ and the opening variables $\vec y$. While it is natural to formulate each of the multi-commodity flow problem using auxiliary variables denoting the flow, our algorithm will utilize the equivalent formulation obtained via projecting out the flow variables. This projected formulation is a relaxation where the only variables are assignment variables $\vec x$ and the opening variables $\vec y$.
\begin{figure}[h!]
\begin{equation*}
\boxed{\begin{minipage}{10cm}\begin{align*}
            \textrm{minimize}\qquad & c(\vx,\vy):= \sum_{i\in \cF} o_i\cdot y_i  + \sum_{i\in \cF, j\in \cD} c_{ij} \cdot x_{ij},\\
            \textrm{subject to}\qquad & \MFN(\vec{g}, \vec{x}, \vec{y}) \mbox{ is feasible}\quad \forall \vec g\textrm{ valid};\\[2mm]
            &\vec{x}\in [0,1]^{\cF\times\cD},\vec{y}\in[0,1]^{\cF}.
          \end{align*}
          \vspace{-0.4cm}
        \end{minipage}}
\end{equation*}
\caption{Our relaxation of \cfl.}
\label{fig:MFN-LP}
\end{figure}

In Lemma~\ref{lem:bing} we show that the constraints of \MFNLP can equivalently be formulated over
the subset of valid partial assignments that are integral. \MFNLP can therefore be seen as the
intersection of the feasible regions
of finitely many multi-commodity flow linear programs and is therefore itself a linear program.
At first sight, however, it may not be clear that \MFNLP is a relaxation, or how we can separate it.
We will answer these questions in the rest of this section.

\subsection{Integral Partial Assignments and Separation}

We first present a useful lemma that allows us to consider only the valid assignments $\vec{g}$ that are integral, i.e., $\{0,1\}$-matrices. This lemma follows from the integrality of the $b$-matching polytope.

\begin{lemma}
\label{lem:bing}
For any $(\vec{x}, \vec{y})$, $\MFN(\vec{g}, \vec{x}, \vec{y})$ is feasible for all valid $\vec{g}$
if and only if $\MFN(\vec{\hat g}, \vec{x}, \vec{y})$ is feasible for all valid $\vec{\hat g}$  that
are integral.
\end{lemma}
\begin{proof}
  It is clear that if the flow network is feasible for all valid $\vec{g}$ then it is also feasible
  for the subset that are integral. We  show the harder side. Suppose $\MFN(\vec{\hat g}, \vec{x},
  \vec{y})$ is feasible for all valid $\vec{\hat g}$  that are integral and consider an arbitrary valid assignment $\vec{g}$
  that may be fractional. We will show that $\MFN(\vec{g}, \vec{x}, \vec{y})$ is feasible.

  Construct a complete bipartite graph with vertices $\cF \cup \cD$ and interpret $\vec{g}$ as the
 weights on the edges of this complete bipartite graph. As $\vec{g}$ is valid, we have $\sum_{j\in \cD}
  g_{ij} \leq U_i$ for each $i\in \cF$ and
  $\sum_{i\in \cF} g_{ij} \leq 1$ for each $j\in \cD$. In other words, $\vec{g}$ is a fractional
  solution to the $b$-matching polytope.  By the integrality of the $b$-matching polytope~(see e.g. \cite{S2003}),  we
  can write $\vec{g}$ as a convex combination of valid integral assignments $\vec{\hat g}^1, \vec{\hat
    g}^2, \ldots, \vec{\hat g}^r$, \ie there exist $\lambda_1, \lambda_2, \ldots, \lambda_r \geq
  0$ such that $\sum_{k=1}^r \lambda_k = 1$ and $\vec{g} = \sum_{k=1}^r \lambda_k\vec{\hat g}^k$.

  Now, let $\vec f^k$ denote the feasible flow for $\MFN(\vec{\hat g}^k,
  \vec{x},\vec{y})$, and choose $\vec f=\sum_{k} \lambda_k \vec f^k$. Observe that $\vec f$ is a feasible solution to $\MFN(\vec{g}, \vec{x}, \vec{y})$, since all the capacities and demands of $\MFN(\cdot,\vec x,\vec y)$ are given as linear functions of $\vec g$.
\end{proof}


A natural question is whether \MFNLP can be separated in polynomial time. While we currently do not
know if this is the case, we will establish in this paper that the feasibility constraint of
$\MFN(\vec { g},\vec x,\vec y)$ can be separated for any fixed $\vec{ g}$, and that this is
sufficient to find a fractional solution whose cost is within a constant factor from the optimum: in
a sense, this oracle enables us to extract the power of our strong relaxation within a constant
factor. The following lemma states the oracle. It follows from known characterizations using
LP-duality of multi-commodity flows and its proof can be found in Appendix~\ref{ap:lp}.


\begin{lemma}
\label{lem:linprog}
Given $\vec{ g^{\star}}$ in addition to $(\vec x^{\star},\vec y^{\star})$ such that $\MFN(\vec{ g^{\star}},\vec x^{\star},\vec y^{\star})$ is infeasible, we can find in polynomial time a violated inequality, i.e., an inequality that is valid for \MFNLP but violated by $(\vec x^{\star},\vec y^{\star})$. Moreover, the number of bits needed to represent each coefficient of this inequality is polynomial in $|\cF|$, $|\cD|$, and $\log U$, where $U:=\max_{i\in\cF}U_i$.
\end{lemma}


\subsection{\MFNLP is a Relaxation of the Capacitated Facility Location Problem}

We show in this subsection that \MFNLP is indeed a relaxation.

\begin{lemma}
\MFNLP is a relaxation of the capacitated facility location problem.
\end{lemma}

\begin{proof}
  Consider an arbitrary integral solution $(\vec{x}^{\star}, \vec{y}^{\star})$ to the facility location
  problem. By Lemma~\ref{lem:bing} we only need to verify that $\MFN(\vec{ g}, \vec{x}^{\star},
  \vec{y}^{\star})$ is feasible for each valid integral assignment $\vec{ g}$. Let
  $\vec{\hat g}$ be an arbitrary valid integral assignment.

Now we consider a directed bipartite graph $G=(V,A)$, of which one side of the vertex set is $\cD$, and on the other side, each facility $i\in\cF$ appears in $y^{\star}_i\cdot U_i$ duplicate copies. Consider the following two matchings $M_1$ and $M_2$ on these vertices.\begin{itemize}
\item For each client $j$, $M_1$ has an edge between $j$ and (a copy of) $i$ for which $x^{\star}_{ij}=1$. There will always be a copy of $i$ since $y_i^{\star}\geq x_{ij}^{\star}=1$. We will also ensure that a single copy of a facility does not have more than one incident edge: this is possible due to the capacity constraints on $(x^{\star},y^{\star})$.
\item For each $(i,j)$ such that $\hat g_{ij}=1$ and $y^{\star}_i=1$, $M_2$ has an edge between a copy of facility $i$ and client $j$. Note that no client will have more than one incident edge since $\sum_{i\in\cF}\hat g_{ij}\leq 1$. We will also ensure that a single copy of a facility does not have more than one incident edge. This is possible since $\sum_{j\in\cD}\hat g_{ij}\leq U_i$.
\end{itemize}Now we orient every edge in $M_1$ from clients to facilities; edges in $M_2$ are oriented in the opposite direction. $A$ is defined as the union of these two directed matchings. Since both $M_1$ and $M_2$ are matchings, every vertex in $G$ has indegree of at most one and outdegree of at most one. Hence, we can decompose $A$ into a set of maximal paths and cycles. Moreover, since $M_1$ matches every client, none of these maximal paths will end at a client. Reinterpret these paths as paths on $\cD$ and $\cF$, instead of on the duplicate copies of facilities. Let $\cP$ denote the set of these (nonempty) paths.

We will now construct a feasible multi-commodity flow on $\MFN(\vec{\hat g},\vec x^{\star},\vec y^{\star})$. We consider each $P\in\cP$. If $P$ starts from a facility, ignore it; otherwise let $j$ be the starting point of $P$ and $i$ the ending point: $P=( j,i_1,j_2,i_2, \ldots,j_k, i)$. If $d_j=0$, we ignore $P$; otherwise, we push one unit of flow of commodity $j$ along $P$, staying within the shaded area of Figure~\ref{fig:MFN}: i.e., the flow is pushed along $( j^s,i_1,j_2^s,i_2, \ldots,j_k^s, i)$. When we arrive at $i$, further push this flow along $( i,i',j^t) $, draining the flow at $j^t$: this is legal since the flow is of commodity $j$. We repeat this until we have considered all paths in $\cP$. We claim that this procedure yields a feasible multi-commodity flow.

First, note that each arc in $A$ maps to an edge of capacity 1 in $\MFN(\vec{\hat g},\vec
x^{\star},\vec y^{\star})$.
Since $\cP$ is a decomposition of
(a subset of) $A$, capacity constraints on $( j^s,i)$ and $( i,j^s)$ are
satisfied from the construction. Now consider the capacity of $( i,i')$. Each time we
encounter a path $P\in\cP$ that starts at some client and ends at $i$, one unit of additional flow
is sent over this arc. If $y^{\star}_i=0$, there will be no such path in $\cP$. If $y^{\star}_i=1$,
there are at most $U_i-\sum_{j\in\cD}\hat g_{ij}$ paths in $\cP$ ending at $i$, since $M_2$ matches
exactly $\sum_{j\in\cD}\hat g_{ij}$ copies of $i$ out of $U_i$ in total. This verifies that the
capacity constraint on $( i,i')$ is also satisfied. Finally, arc $( i',j^t)$
is used only when we process $P\in\cP$ that starts from $j$ and ends at $i$. This is true for at
most one path in $\cP$ since there is at most one path starting from each client (note that there
are no duplicate copies of clients in $G$); moreover, $P$ can end at $i$ only if $y^{\star}_i=1$
(otherwise, there are no copies of $i$ in $G$). The capacity constraint on $( i',j^t)$
is therefore also satisfied.

Demand constraints are also satisfied: suppose $d_j=1$ for some $j\in\cD$. This means $\vec{\hat g}$ does not assign $j$ to any facility, and therefore $M_2$ does not match $j$. Hence $j$ has indegree of zero and outdegree of one in $G$, and thus $\cP$ contains exactly one path that starts from $j$.
\end{proof}

Intuitively, the above proof can also be interpreted as follows: given an arbitrary partial assignment and integral solution, consider the shaded area of Figure~\ref{fig:MFN}. By saturating every arc in this area, we obtain a feasible single-commodity flow where every client generates a unit flow either at its original position or at the facility it is assigned to by $\vec g$. While this flow satisifies every commodity-oblivious capacity, it may not be immediately clear why it also satisfies the commodity-specific capacities; here we can appeal to the integrality of $\vec y^\star$, because in this case every facility with nonzero commodity-oblivious capacity will automatically have the full commodity-specific capacity of $1$. Such an argument, however, would not extend to a fractional solution (to the standard LP for example), which illustrates the strength of our relaxation.

\subsection{Comparing \MFNLP to Standard LP and  Knapsack-Cover Inequalities}\label{sec:knapsack}
In order to facilitate our understanding of the new relaxation, we demonstrate how it relates to
other formulations for the capacitated facility location problem.

\paragraph{Standard LP.} We shall show that the constraint that $\MFN(\vec g,\vec x,\vec y)$ is feasible for $\vec g=\vec
0$ already is sufficient to see that our relaxation is no worse than the standard LP relaxation. The
standard LP relaxation uses the same variables $(\vec{x},\vec{y})$ as \MFNLP and is formulated as
follows:
\begin{center}
\boxed{\addtolength{\linewidth}{-2\fboxsep}\addtolength{\linewidth}{-2\fboxrule}\begin{minipage}{10cm}\begin{align}
&&&\textrm{minimize} \quad  \rlap{$\displaystyle\sum_{i\in \cI} o_i \cdot y_i + \sum_{i\in \cF, j\in \cD} c_{ij} \cdot x_{ij}$,}\nonumber\\[2mm]
       &&&     \textrm{subject to} & x_{ij}&\leq   y_i & \forall i\in \cF, j\in \cD;\label{e:standard:1}\\[1mm]
               &&& &\sum_{i\in \cF} x_{ij}&= 1 & \forall  j\in \cD;\label{e:standard:2}\\[1mm]
             &&&  & \sum_{j\in \cD} x_{ij}&\leq y_iU_i & \forall  i\in \cF;\label{e:standard:3}\\[1mm]
          &&& & 0 \leq \vec{x},&\vec{y}\leq 1
          \end{align}
\end{minipage}}
\end{center}

Consider arbitrary $(\vec x^{\star},\vec y^{\star})$ that makes $\MFN(\vec 0,\vec
x^{\star},\vec y^{\star})$ feasible. Since $\vec g=\vec 0$, the support of $\MFN(\vec g,\vec
x^{\star},\vec y^{\star})$ is acyclic and the flow of each commodity can be decomposed into paths
with no cycles. In particular, every path for commodity $j\in\cD$ will be in the form of
$j^s-i-i'-j^t$ for some $i\in\cF$. Observe that this implies that the only commodity that has
nonzero flow on $( j^s,i)$ is $j$; let this flow be $\bar x_{ij}$. Now we claim that
$(\vec {\bar x},\vec y^{\star})$ is a feasible solution to the standard LP: \eqref{e:standard:2}
follows from $d_j=1$; \eqref{e:standard:1} follows from the capacity constraint on $(
i',j^t)$. Finally, $\sum_{j\in\cD}\bar x_{ij}$ equals the total (regardless the commodity)
incoming flow to $i$; this in turn is bounded from above by $y^{\star}_i U_i$ from the capacity
constraint on $( i,i')$. This shows that $(\vec {\bar x},\vec y^{\star})$ is feasible to
the standard assignment LP. Observe that $\vec x^{\star}$ dominates $\vec{\bar x}$ and therefore the
lower bound on the optimum given by \MFNLP is always no worse than the standard assignment LP.

\paragraph{Knapsack-cover inequalities.}
Consider a special case where the metric on $\cF$ and $\cD$ is constantly zero: i.e., every facility
and every client are ``on the same spot''. As there is no connection cost, the problem reduces to
simply selecting a set of facilities that as a whole has enough capacity while minimizing the total
cost: this is the minimum knapsack problem. Each facility $i$ corresponds to an item with weight $U_i$ and cost $o_i$; $|\cD|$
corresponds to the demand of the knapsack problem. Using the notation of the capacitated facility
location problem, the knapsack cover inequalities due to Carr et al.~\cite{CFLP2000} are written as follows:\begin{equation*} \sum_{i\in\cF\setminus A} \min
  (U_i,|\cD|-{\textstyle\sum_{i\in A}U_i})\cdot y_i\geq |\cD|-{\textstyle\sum_{i\in A}U_i},\quad
  \forall A\subset\cF\textrm{ such that }{\textstyle\sum_{i\in A}U_i}\leq |\cD| .\end{equation*}

Now each of these inequalities are implied by our relaxation. Let $S$ be a set of any
${\textstyle\sum_{i\in A}U_i}$ clients and $R:=\cD\setminus S$. Consider $g$ that fully assigns
every client in $S$ to the facilities in $A$, thereby saturating those facilities, and does not
assign any clients in $R$. Note that $d_j$ will be zero for every $j\in S$ and one for every $j\in
R$. Now we choose a feasible solution $(\vec{z},\vec{\ell})$ to \eqref{e:seplp:1}-\eqref{e:seplp:2} as follows: for
each facility $i\in A$, if $U_i> |\cD|-{\textstyle\sum_{i\in A}U_i}$, set $l_{(
  i',j^t)}:=1$ for all $j\in R$; if $U_i\leq |\cD|-{\textstyle\sum_{i\in A}U_i}$, set
$\ell_{( i,i')}:=1$. All other $\ell$'s are set to zero. $z_j:=1$ for all $j\in R$; $z_j:=0$
for all $j\in S$. Now \eqref{e:mfdual} implies the knapsack cover
inequalities.
 
\paragraph{Example.}

We give a simple integrality gap example for the standard LP and show how our linear program strengthens the linear program to \emph{cut off} the fractional solution.
Consider the following instance of the capacitated facility location problem. Here we have two facilities $i_1$ and $i_2$ each with capacity $n$ and opening costs are $0$ and $1$ respectively. There are $n+1$ clients $j_1,\ldots,j_{n+1}$. The distance between any two points, either facility or client, is zero. Thus all facilities and clients sit at the same point. Consider the following fractional solution $(\vx^*,\vy^*)$ where we have $y^{\star}_{i_1}=1$ and $y^{\star}_{i_2}=\frac{1}{n}$,  $x^{\star}_{i_1j_r}=\frac{n}{n+1}$ for each $1\leq r \leq n+1$ and $x^{\star}_{i_2j_r}=\frac{1}{n+1}$ for each $1\leq r \leq n+1$. It is quite simple to verify that $(\vx^{\star},\vy^{\star})$ is a feasible solution to the standard LP and costs $\frac{1}{n+1}$ while the cost of the optimal solution is $1$ giving us an unbounded integrality gap for large $n$. We now show how this fractional solution can be \emph{cut off} using our stronger LP. We also note that, for this instance, the knapsack cover inequalities are enough to obtain a good approximation.


Consider the partial assignment $\vg^{\star}$ defined as follows. We let $g^{\star}_{i_1j_r}=1$ for
each $1\leq r\leq n$. Thus we have assigned $n$ clients to facility $i_1$ and saturating it. The
only facility that can serve the unassigned client is $i_2$. Consider the flow network defined by
this instance. The capacity of arc $(i_2',j_{n+1}^t)$ is $y_{i_2}$ and demand of client $j_{n+1}$ is
$1-\sum_{i\in \cF} g^{\star}_{ij_{n+1}}=1$. Since all flow reaching $j^t$ must go on this arc, we
must have $y_{i_2}\geq 1$. Thus the fractional solution must cost at least one.
 
