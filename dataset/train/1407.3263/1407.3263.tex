We present our new relaxation for the capacitated facility location problem in this section. Let us first define some notation to be used in the rest of this paper. Let  be the set of facilities and  be the set of clients. Each facility  has opening cost , and cannot be assigned more number of clients than its capacity . We are also given a metric cost  on  as a part of the input:  denotes the distance between  and .

The variables of our relaxation is the pair  where we refer to  as the \emph{assignment variables} and to  as the \emph{opening variables}. These variables naturally encode the decisions to which facility a client is connected
and which facilities that are opened. Indeed, the intended integral solution is that  if
client  is connected to facility  and  otherwise;  if facility  is
opened and  otherwise. The idea of our relaxation is that every partial assignment of
clients to facilities should be extendable to a complete assignment while only using the assignments
of  and openings of . To this end let us first describe the partial assignments
that we shall consider. We then define the constraints of our linear program which will be
feasibility constraints of multi-commodity flows.

A partial fractional assignment  of clients to facilities is  \emph{valid} if

The first condition says that each client should be fractionally assigned at most once and the
second condition says that no facility should receive more clients than its capacity. We emphasize
that we allow clients to be fractionally assigned, \ie  is not assumed to  be integral. As we
shall see later (see Lemma~\ref{lem:bing}), this does not change the strength of our relaxation but it will
be convenient in the analysis of our rounding algorithm in Section~\ref{sec:algorithm}. We also remark that
the above inequalities are exactly the -matching polytope of the complete bipartite graph
consisting of the clients on the one side and the facilities on the other side; each client can be
matched to at most one facility and each facility  can be matched to at most  clients.

The constraints of our relaxation will be that, no matter how we partially assign the clients according to a valid
, the solution  should support a multi-commodity flow where each client  becomes the source of its own commodity , and the demand of this commodity is equal to the amount by which  is ``not assigned'' by , . The flow network, whose arc capacities are given as a function of  and the solution , is defined as follows (see also Figure~\ref{fig:MFN}):
\begin{definition}[Multi-commodity flow network]
\label{def:mfn}
For a valid partial assignment ,  assignment variables , and opening variables , let  be a
multi-commodity flow network with  commodities, defined as follows. Note that some arcs may have zero capacities.
\begin{enumerate}[(a)]
\item Each client  is associated with commodity  of demand , and its source-sink pair is .  \item Each facility  has two nodes  and  with an arc  of capacity .  \item For each client  and facility , there is an arc  of capacity , an arc
   of capacity , and an arc  of capacity .
\end{enumerate}
\end{definition}


\begin{figure}[tb]
\begin{center}
\begin{tikzpicture}

\node[client] (j1) at (-1.5, 0)  {};
\node[client] (j2) at (-1.5, -1.5)  {};
\node at (-1.5,-3) {};
\node[client] (jn) at (-1.5, -4.5)  {};

\node (d) at (-1.7, 1.5) {\small demand };

\draw (d) edge[dotted] (j1);

\node[facility] (i1) at (1.5,-0.5) {};

\node[facility] (i2) at (1.5,-1.75) {};

\node at (1.5,-2.9) {};
\node[facility] (im) at (1.5,-4) {};

\node[facility] (i'1) at (6.5,-0.5) {};

\node[facility] (i'2) at (6.5,-1.75) {};

\node at (6.5,-2.9) {};
\node[facility] (i'm) at (6.5,-4) {};

\node[client] (t1) at (9.7,0 )  {};
\node[client] (t2) at (9.7,-1.5 )  {};
\node at (9.7,-3 ) {};
\node[client] (tn) at (9.7,-4.5 )  {};



\draw (j1) edge[->,ultra thick, bend left] node[above] {\small capacity }   (i1);
\draw[gray] (j1) edge[<->] (i2);
\draw[gray] (j1) edge[<->] (im);

\draw[gray] (j2) edge[<->] (i1);
\draw[gray] (j2) edge[<->] (i2);
\draw[gray] (j2) edge[<->] (im);

\draw[gray] (jn) edge[<->] (i1);
\draw[gray] (jn) edge[<->] (i2);
\draw[gray] (jn) edge[->] (im);

\draw (jn) edge[<-, ultra thick, bend right] node[below] {\small capacity }  (im);


\draw (i1) edge[ultra thick,->] node[above] {\small\begin{tabular}{c}capacity\\\end{tabular}} (i'1);
\draw[gray] (i2) edge[->] (i'2);
\draw[gray] (im) edge[->] (i'm);

\draw (i'1) edge[ultra thick,->] node[above=0.2] {\small capacity }(t1);
\draw[gray] (i'1) edge[->] (t2);
\draw[gray] (i'1) edge[->] (tn);


\draw[gray] (i'2) edge[->] (t1);
\draw[gray] (i'2) edge[->] (t2);
\draw[gray] (i'2) edge[->] (tn);

\draw[gray] (i'm) edge[->] (t1);
\draw[gray] (i'm) edge[->] (t2);
\draw[gray] (i'm) edge[->] (tn);


  \begin{pgfonlayer}{background}
    \filldraw [line width=4mm,join=round,black!10]
      (-1.8,0.7)  rectangle (1.8,-5.2);
  \end{pgfonlayer}

\end{tikzpicture}
 \end{center}

\caption{A depiction of the multi-commodity flow network .}
\label{fig:MFN}
\end{figure}

\begin{remark}
Intuitively, the bipartite subgraph induced by , marked with a shaded box in Figure~\ref{fig:MFN}, is the interesting part of the flow network.  and  are added to this bipartite graph purely in order to state that  is a sink with ``double'' capacities: a commodity-oblivious capacity  and a commodity-specific capacity  for each client .


\end{remark}

Let us give some intuition on the definition of . As already noted,
the demand  of a client  equals the amount by which  is not
assigned by the partial assignment . This demand should only be assigned to opened
facilities. Therefore, facility  can accept at most  of 's demand which is either
 or  in an integral solution.
Observe that such a constraint, for each client and facility, cannot be imposed by a single-commodity flow problem.
Multi-commodity flow problems, on the other hand, allows us to express this constraint as a commodity-specific capacity of , as denoted by arc  in Figure~\ref{fig:MFN}.

Now consider the \emph{commodity-oblivious} capacities of the facilities. The \emph{total} demand an opened facility  can accept
is its capacity minus the amount of clients assigned to it in the partial assignment ; and
a closed facility can accept no demand. Therefore, the total demand a facility  can accept is at
most . The arc capacity  of an arc  says that
client  should be connected to facility  only if . The reason for having arcs of
the form  of capacity  is discussed in Section~\ref{sec:ourcontrib}: these allow the alternating paths for routing the remaining demand and are
necessary for the formulation to be a relaxation.


We are now ready to formally state our relaxation \MFNLP of the  capacitated facility
location problem in Figure~\ref{fig:MFN-LP}. Note that the only variables of our relaxation are the assignment variables  and the opening variables . While it is natural to formulate each of the multi-commodity flow problem using auxiliary variables denoting the flow, our algorithm will utilize the equivalent formulation obtained via projecting out the flow variables. This projected formulation is a relaxation where the only variables are assignment variables  and the opening variables .
\begin{figure}[h!]

            \textrm{minimize}\qquad & c(\vx,\vy):= \sum_{i\in \cF} o_i\cdot y_i  + \sum_{i\in \cF, j\in \cD} c_{ij} \cdot x_{ij},\\
            \textrm{subject to}\qquad & \MFN(\vec{g}, \vec{x}, \vec{y}) \mbox{ is feasible}\quad \forall \vec g\textrm{ valid};\
          \vspace{-0.4cm}
        \end{minipage}}

&&&\textrm{minimize} \quad  \rlap{,}\nonumber\1mm]
               &&& &\sum_{i\in \cF} x_{ij}&= 1 & \forall  j\in \cD;\label{e:standard:2}\1mm]
          &&& & 0 \leq \vec{x},&\vec{y}\leq 1
           \sum_{i\in\cF\setminus A} \min
  (U_i,|\cD|-{\textstyle\sum_{i\in A}U_i})\cdot y_i\geq |\cD|-{\textstyle\sum_{i\in A}U_i},\quad
  \forall A\subset\cF\textrm{ such that }{\textstyle\sum_{i\in A}U_i}\leq |\cD| .

Now each of these inequalities are implied by our relaxation. Let  be a set of any
 clients and . Consider  that fully assigns
every client in  to the facilities in , thereby saturating those facilities, and does not
assign any clients in . Note that  will be zero for every  and one for every . Now we choose a feasible solution  to \eqref{e:seplp:1}-\eqref{e:seplp:2} as follows: for
each facility , if , set  for all ; if , set
. All other 's are set to zero.  for all ; 
for all . Now \eqref{e:mfdual} implies the knapsack cover
inequalities.
 
\paragraph{Example.}

We give a simple integrality gap example for the standard LP and show how our linear program strengthens the linear program to \emph{cut off} the fractional solution.
Consider the following instance of the capacitated facility location problem. Here we have two facilities  and  each with capacity  and opening costs are  and  respectively. There are  clients . The distance between any two points, either facility or client, is zero. Thus all facilities and clients sit at the same point. Consider the following fractional solution  where we have  and ,   for each  and  for each . It is quite simple to verify that  is a feasible solution to the standard LP and costs  while the cost of the optimal solution is  giving us an unbounded integrality gap for large . We now show how this fractional solution can be \emph{cut off} using our stronger LP. We also note that, for this instance, the knapsack cover inequalities are enough to obtain a good approximation.


Consider the partial assignment  defined as follows. We let  for
each . Thus we have assigned  clients to facility  and saturating it. The
only facility that can serve the unassigned client is . Consider the flow network defined by
this instance. The capacity of arc  is  and demand of client  is
. Since all flow reaching  must go on this arc, we
must have . Thus the fractional solution must cost at least one.
 
