



\documentclass[11pt]{article}

\usepackage{graphicx, amssymb, latexsym, amsfonts, amsmath, lscape, amscd,
amsthm, color, epsfig, mathrsfs, tikz, enumerate}




\setlength{\topmargin}{-1.5cm}
\setlength{\textheight}{23cm} \setlength{\textwidth}{16cm}    \setlength{\oddsidemargin}{0cm} \setlength{\evensidemargin}{0cm} 




\vfuzz2pt \hfuzz2pt \newtheorem{theorem}{Theorem}[section]
\newtheorem{conjecture}[theorem]{Conjecture}
\newtheorem{corollary}[theorem]{Corollary}
\newtheorem{example}[theorem]{Example}
\newtheorem{lemma}[theorem]{Lemma}
\newtheorem{proposition}[theorem]{Proposition}
\newtheorem{question}[theorem]{Question}
\newtheorem{problem}[theorem]{Problem}
\newtheorem{defn}[theorem]{Definition}
\newtheorem{rem}[theorem]{Remark}
\newtheorem{claim}{Claim}
\newtheorem{observation}{Observation}

\newcommand\DELETE[1]{}
\newcommand\ES[1]{\textcolor{red}{#1}}
\newcommand\ESOK[1]{#1}

\newcommand\THREESATPB{\textsc{-Satisfiability}}
\newcommand\MONNAESATPB{\textsc{Monotone Not-All-Equal -Satisfiability}}
\newcommand\SCLIQUEPB{\textsc{Sclique Signing}}
\newcommand\SSCLIQUEPB{\textsc{Signed Clique Edge-Color Assignment}}
\newcommand\Pclass{\textsf{P}}
\newcommand\NPclass{\textsf{NP}}





\begin{document}

\title{{\bf On chromatic number of colored mixed graphs}}
\author{
{\sc Sandip Das\footnote{{\small sandipdas@isical.ac.in}}}, {\sc Soumen Nandi\footnote{{\small soumen.nandi\_r@isical.ac.in}}}, {\sc Sagnik Sen\footnote{{\small sen007isi@gmail.com}}}\\
\mbox{}\\
{\small Indian Statistical Institute, Kolkata, India}\\
}




\date{\today}

\maketitle

\begin{abstract}
An -colored mixed graph  is a graph  with its arcs having one of the  different colors and edges having one of the  different colors. A homomorphism  of an -colored mixed graph    to an  -colored mixed graph  is a vertex mapping such that if  is an arc (edge) of color  in , then  is an arc (edge) of  color  in . The \textit{-colored mixed chromatic number}  of an   -colored mixed graph  is the 
order (number of vertices)
of the smallest homomorphic image of . This notion was introduced by 
Ne\v{s}et\v{r}il and Raspaud (2000,  J. Combin. Theory, Ser. B 80, 147--155). 
They showed that  where  is a -acyclic colorable graph.  We proved the tightness of this bound. 
We also showed that the acyclic chromatic number of a graph is bounded by  if its 
-colored mixed chromatic number is at most .  
Furthermore, using probabilistic method, we showed that for graphs with maximum degree  its -colored mixed chromatic number is at most 
. 
In particular, the last result directly improves the upper bound   of oriented chromatic number of 
graphs with  maximum degree , obtained by Kostochka, Sopena and Zhu (1997, J. Graph Theory 24, 331--340)  to . 
We also show that there exists a graph with maximum degree  and -colored mixed chromatic number at least . 
 \end{abstract}


\noindent \textbf{Keywords:} colored mixed graphs, acyclic chromatic number, graphs with bounded maximum degree, arboricity, chromatic number.


\section{Introduction}
An \textit{-colored mixed graph}  is a graph  with set of vertices , set of arcs  and set of edges  where each arc is 
 colored by one of the  colors  and each edge is  colored by one of the  colors . We denote  the number of vertices and the number of edges of the underlying graph of  by   and , respectively.
Also, we will consider only those 
-colored mixed graphs for which the underlying undirected graph is simple.
Ne\v{s}et\v{r}il and Raspaud~\cite{raspaud_and_nesetril} generalized the notion of vertex coloring and chromatic number 
 for -colored mixed graphs
 by definining colored homomorphism. 

  
Let  and 
be two -colored mixed graphs. 
A colored homomorphism of  to  is a function  satisfying
 

and the color of the arc or  edge linking  and  is the same as the color of the arc or the edge linking  and ~\cite{raspaud_and_nesetril}.
  We write  whenever there exists a 
 homomorphism of  to .


Given an -colored mixed graph  let  be an -colored mixed graph with minimum \textit{order} (number of vertices) such that . 
Then the order of  is the \textit{-colored mixed chromatic number}  of .
For an undirected simple graph , the maximum -colored mixed chromatic number taken over all -colored mixed
graphs having underlying undirected simple graph  is denoted by  .
Let  be a family of undirected simple graphs. 
Then   is the maximum of  
taken over all .




Note that a -colored mixed graph   is nothing but an undirected simple graph while  
 is the ordinary chromatic number. 
Similarly, the study of  
is  the study of oriented chromatic number which is considered by several researchers in the last two decades (for details please check the recent updated survey~\cite{sopena_updated_survey}).
Alon and Marshall~\cite{Marshall-edgecoloring}  studied the homomorphism of -colored mixed graphs with a particular focus on . 

A simple graph  is \textit{-acyclic colorable} if we can color its vertices with  colors such that each color class induces an independent set and any two color class induces a forest. 
The \textit{acyclic chromatic number}  of a simple graph  is the minimum   
such that  is -acyclic colorable. 
Ne\v{s}et\v{r}il and Raspaud~\cite{raspaud_and_nesetril} showed that  where  is a -acyclic colorable graph. 
As planar graphs are -acyclic colorable due to Borodin~\cite{Borodinacyclic}, the same authors implied 
  for the family  of planar graphs as a corollary.
This result, in particular, implies  and  (independently proved 
before in~\cite{planar80} and~\cite{Marshall-edgecoloring}, respectively). 


 Let  be the family of graphs with acyclic chromatic number at most . Ochem~\cite{Ochem_negativeresults} showed that the upper bound     is tight.  
We generalize it for all  to show that the upper bound 
 obtained by Ne\v{s}et\v{r}il and Raspaud~\cite{raspaud_and_nesetril}
is tight. 
This implies that the upper bound  cannot be improved using
the upper bound of .  



The arboricity  of a graph  is the minimum  such that the edges of  can be decomposed into  forests. 
Kostochka, Sopena and Zhu~\cite{Kostochka97acyclicand} showed that given a simple graph , the acyclic chromatic number  of  is also bounded by a function of  . 
We generalize this result for all  by showing that for a graph  with 
 we have 
 where . Our bound slightly improves the bound obtained by  Kostochka, Sopena and Zhu~\cite{Kostochka97acyclicand} for .
For achieving this result we first establish some relations among arboricity of a graph, -colored mixed chromatic number and 
acyclic chromatic number. 



 Let  be the family of graphs with maximum degree  .
Kostochka, Sopena and Zhu~\cite{Kostochka97acyclicand} proved that .
We improve this result in a generalized setting by proving  for all   where 
.



\section{Preliminaries}

A \textit{special 2-path}  of an -colored mixed graph  is a 2-path satisfying one of the following conditions:

\begin{itemize}
\item[(i)]  and  are edges of different colors,

\item[(ii)]  and  are arcs (possibly of the same color),

\item[(iii)]  and  are arcs of different colors,

\item[(iv)]  and  are arcs of different colors,

\item[(v)] exactly one of  and  is an edge and the other is an arc.
\end{itemize}




\begin{observation}\label{special}
The endpoints of a special 2-path must have different image under any homomorphism of .
\end{observation}

\begin{proof}
Let  be a special 2-path in an -colored mixed graph . Let  be a colored homomorphism of  to an
-colored mixed graph  such that . Then  and   will induce parallel edges in the underlying graph of .
But as we are dealing with -colored mixed graphs with underlying simple graphs, this is not possible. 
\end{proof}




Let  be an -colored mixed graph. Let  be an arc of  with color  for some . 
Then  is a
\textit{-neighbor} of  and  is a \textit{-neighbor} of . The set of all -neighbors 
and -neighbors of  is denoted by  and , respectively. Similarly, 
let  be an edge of  with color  for some . Then  is a
\textit{-neighbor} of  and the set of all -neighbors of  is denoted by .
Let  be a \textit{-vector} such that 
 where . 
Let  be a \textit{-tuple} (without repetition) of  vertices  from . Then we define the set 
. 
Finally, we say that  has property  if for each -vector  and each -tuple  we have 
 where  and  is an integral function. 
  
  
  \section{On graphs with bounded acyclic chromatic number}\label{sec acyclic}
  First we will construct examples of -colored mixed graphs  with acyclic chromatic number at most  and 
   for all  and for all . This, along with the upper bound established by Ne\v{s}et\v{r}il and Raspaud~\cite{raspaud_and_nesetril}, will imply the following result:
  
  
  \begin{theorem}\label{acyclic-chromatic}
  Let  be the family of graphs with acyclic chromatic number at most . 
  Then  for all  and for all . 
  \end{theorem}
  
\begin{proof}
First we will construct an -colored mixed graph , where , as follows. 
Let  be the set of all -vectors. 
Thus, .  

Define  as a set of  vertices  for all  such that  when . The vertices of 's are called \textit{bottom} vertices for each .
Furthermore, let  be a -tuple.  

After that define the set of  vertices  
for all .  
The vertices of 's are called \textit{top} vertices for each . 
Observe that there are  vertices in  for each .

Note that the definition of  already implies some colored arcs and edges between the set of vertices 
 and  for all   . 


As  it is possible to construct a special 2-path. 
Now for each  pair of vertices  and  (), construct a special 2-path  and  call these new vertices  as 
\textit{internal} vertices for all .
This so obtained graph is .

Now we will show that . 
Let  be two distinct 
-vectors. 
Assume that the  co-ordinate of  and   is different. 
Then note that  is a special 2-path. 
Therefore,  and  must have different homomorphic image under any homomorphism. 
Thus, all the vertices in  must have distinct homomorphic image under any homomorphism. 
Moreover, as a vertex of  is connected by a special 2-path with a vertex of  for all , all the top vertices must have distinct 
homomorphic image under any homomorphism. It is easy to see that  for all   .
Hence .


 Then we will show that . From now on, by , we mean the underlying undirected simple graph of the 
 -colored mixed graph . We will provide an acyclic coloring of this graph with .
 Color all the vertices of  with  for all . Then color all the vertices of  with distinct  colors 
 from the set  of colors for all .
Note that each internal vertex have exactly two neighbors. Color each internal vertex with a color different from its neighbors.
It is easy to check that this is an acyclic coloring. 

Therefore, we showed that  
while, on the other hand, Ne\v{s}et\v{r}il and Raspaud~\cite{raspaud_and_nesetril} showed that  for all  and for all . 
  \end{proof}



Consider a complete graph  . Replace all its edges  by a 2-path to obtain the graph . 
For all , it is possible to assign colored edges/arcs to 
the edges of  such that it becomes an -colored mixed graph with  vertices that are pairwise connected by 
a special 2-path. Therefore, by Observation~\ref{special} we know that  whereas, it is easy to note that  has arboricity 2. 
Thus, the -colored mixed chromatic number  is not bounded by any function of arboricity.
Though the reverse type of bound exists. Kostochka, Sopena and Zhu~\cite{Kostochka97acyclicand}  proved such a bound for . We generalize their result for all . 





\begin{theorem}\label{chromatic-arboricity}
Let  be an -colored mixed graph with  where . Then . 
\end{theorem}

\begin{proof}
Let  be an arbitrary labeled subgraph of  consisting  vertices and  edges. We know from 
Nash-Williams' Theorem~\cite{nash1page}  that the arboricity  of any graph  is equal to the maximum of 
 over all subgraphs  of . So it is sufficient to prove that for any subgraph  of , 
. 
As  is a labeled graph, so there are  different -colored mixed graphs with underlying graph . As , there exits a homomorphism from  to a -colored mixed graph  which has the complete graph on  vertices as its underlying graph. 
Note that the number of possible homomorphisms of  to  is at most .
For each such homomorphism of   to  there are at most  different -colored mixed graphs with underlying labeled graph   as there are  choices of . 
Therefore,

  which implies 
    

If , then . Now let . We know that . So



Therefore, . 
\end{proof}


We have seen that the -colored mixed chromatic number of a graph  is bounded by a function of the acyclic chromatic number of . Here we show that
 it is possible to bound the acyclic chromatic number of a graph in terms of its -colored mixed chromatic number and arboricity.
Our result is a generalization of a similar result proved for  by  Kostochka, Sopena and Zhu~\cite{Kostochka97acyclicand}. 


\begin{theorem}\label{arboricity.chromatic-acyclic}
Let  be an -colored mixed graph with  and  where .  Then 
.
\end{theorem}




\begin{proof}
First we rename the following symbols:  
.

Let  be a graph with  where .
Let  be some ordering of the vertices of . 
Now consider the -colored mixed graph  with underlying graph  such that for any  we have 
 
whenever  is 
an edge of . 

Note that the edges of   can be covered by  edge disjoint forests  as . 
Let  be the number  expressed with base 
for all . Note that  can have at most  digits. 


  Now we will construct a sequence of -colored mixed graphs  each having underlying graph . 
  For a fixed  we will describe the construction of .   
  Let  and  is an edge of . 
  Suppose  is an edge of the forest  for some .
  Let the  digit of   be . Then   is constructed in a way such that 
  we have  in .
  
  Note that there is a homomorphism   for each  such that  is 
  an -colored mixed graph on  vertices. 
  Now we claim that  for each  is an acyclic coloring of . 
  
  For adjacent vertices  in  clearly we have  as . 
  Let  be a cycle in . We have to show that at least 3 colors have been used to color this cycle with respect to the coloring given by . 
  Note that in  there must be two incident edges  and  such that they belong to different forests, 
  say,  and , respectively.
 Now suppose that  received two colors with respect to . Then we must have . In particular we must have 
 . 
 To have that we must also have  for some  in . 
 Let   and  differ in their  digit. Then in  we have  and 
 for some . Then we must have .  Therefore, we also have . Thus, the cycle  cannot be colored with two colors under the coloring . So  is indeed an acyclic coloring of . 
    \end{proof}


Thus, combining   Theorem~\ref{chromatic-arboricity} and~\ref{arboricity.chromatic-acyclic} we have 
 for  where . 
However,  we managed to obtain the following better bound.


\begin{theorem}\label{chromatic-acyclic}
Let  be an -colored mixed graph with   where .  Then 
.
\end{theorem}



\begin{proof}
Let  be the maximum real number such that there exists a subgraph  of  with  and .
Let  be the biggest subgraph of  with . Thus, by maximality of , .

Let . Hence .
By maximality of , for each subgraph  of , we have .


If , then . If , then . So . Therefore,  for each subgraph  of .

By Nash-Williams' Theorem~\cite{nash1page}, there exists  forests  which covers all the edges of . We know from Theorem~\ref{arboricity.chromatic-acyclic}  where . 


Using  inequality~(\ref{eqn wlog}) we get . Therfore




Hence .
\end{proof}



Our bound, when restricted to the case of , slightly improves the existing bound~\cite{Kostochka97acyclicand}.

















\section{On graphs with bounded maximum degree}
Recall that   is the family of graphs with maximum degree  . 
It is known that ~\cite{Kostochka97acyclicand}. 
Here we prove that  for all  and . 
Our result, restricted 
to the case , slightly improves the upper bound of Kostochka, Sopena and Zhu~\cite{Kostochka97acyclicand}.

\begin{theorem}\label{chromatic-degree}
For the family   of graphs with maximum degree  
 we have   for all  and for all .
\end{theorem}

If every subgraph of a graph  have at least one vertex with degree at most , then  is \textit{-degenerated}.  
Minimum such  is the \textit{degeneracy} of . To prove the above theorem we need the following result.

\begin{theorem}\label{chromatic-degree.degeneracy}
Let   be the family of graphs with maximum degree   and degeneracy . Then 
   for all  and for all .
\end{theorem}

 
To prove the above theorem we need the following lemma. 

\begin{lemma}\label{key-lemma}
There exists an  -colored complete mixed graph  with property  on  vertices
where  and . 
\end{lemma} 


\begin{proof}
Let  be a random -colored  mixed graph with underlying complete graph. Let  be two vertices of  and the events 
for  are equiprobable and independent  with probability . 
We will show that the probability of  not having property  is strictly less than 1 when 
. Let  denote the probability of the event  where  is a -tuple of  and  is a -vector for some . Call such an event a \textit{bad event}. Thus,




Let  denote the probability of the occurrence of at least one bad event. 
To prove this lemma it is enough to  show that . Let  denote the set of all -tuples and  denote the set of all -vectors.  Then 



Consider the function . Observe that  is the  
summand  of the last sum from equation~(\ref{eq2}). 
Now


As ,   

Furthermore,  
 
 Adding the above two inequalities we get 
 

Hence . Thus, using inequality~(\ref{eq2}) we get 
.  This implies
 
\medskip 
 




\medskip

\textbf{Case.1:} .



Now, we observe that



So from the inequality (\ref{eq4}), we can say that  for .


\medskip

\textbf{Case.2:} .



Observe that, .

Now, we see that 


So from the inequality (\ref{eq6}), we can say that  for .
\end{proof}

Now we are ready to prove Theorem~\ref{chromatic-degree.degeneracy}.

\medskip

\noindent \textit{Proof of Theorem~\ref{chromatic-degree.degeneracy}.} Suppose that  is an -colored mixed graph with maximum degree  and degeneracy 
.
By Lemma~\ref{key-lemma} we know that there exists an -colored mixed graph  with property 
 on  vertices
where  and . We will show that  admits a homomorphism to . 


As  has degeneracy , we can provide an ordering  of the vertices of  in such a way that each vertex  has at most  neighbors with lower indices.
Let  be the  -colored mixed graph induced by the vertices  from  for . 
Now we will recursively construct a  homomorphism  with the following properties:

\begin{itemize}
\item[] The partial mapping  is a homomorphism of   to  for all .

\item[]  For each , all the neighbors of  with indices less than or equal to  has different images with respect to the mapping .  
\end{itemize} 

Note that the base case is trivial, that is, any partial mapping  is enough. 
Suppose that the function  satisfies the above properties for all  where  is  fixed. 
Now assume that  has  neighbors with indices greater than . 
Then  has at most  neighbors with  indices less than . 
Let  be the set of neighbors of    with  indices greater than .
 Let  be the set of vertices   with indices at most  and with at least one neighbor in . 
 Note that as each vertex of  is a neighbor 
 of  and has at most  neighbors with lesser indices, . 
 Let  be the set of possible  options for  such that the partial mapping 
 is a homomorphism of  to . 
 As  has property  we have . 
 So the set  is non-empty. 
 Thus, choose any vertex from  as the image . 
 Note that this partial mapping satisfies the required conditions.  \hfill 
 
 


\medskip


Finally, we are ready to prove Theorem~\ref{chromatic-degree}.

\medskip

\noindent \textit{Proof of Theorem~\ref{chromatic-degree}.}  First we will prove the lower bound. 
Let  be a  regular graph on  vertices. Thus,  has   edges. Then 
we have  
using inequality~(\ref{eqn wolog}) (see Section~\ref{sec acyclic}). If  for some , then we are done. 
Otherwise,  is bounded. In that case, if  is sufficiently large, then  as 
 is a positive integer.


\medskip

Let  be a connected -colored mixed graph with maximum degree  and . 
If  has a vertex of degree at most  then it has degeneracy at most . In that case  by Theorem~\ref{chromatic-degree}
we are done. 

Otherwise,  is  regular. In that case, remove an edge  of  to obtain the graph . Note that  has 
maximum degree at most   and has degeneracy at most . Therefore, by Theorem~\ref{chromatic-degree} there exists an 
-colored complete mixed graph  on 
 vertices to which  admits a  homomorphism to. 
Let  be the graph obtained by deleting the vertices  and  of . Note that the homomorphism  restricted to  is 
 a homomorphism  of  to . Now include two new vertices  and  to  and obtain a new graph . 
Color the edges or arcs between the vertices of  and  in such a way so that we can extend the homomorphism  to a homomorphism
 
of  to   where ,  and  for all . 
It is easy to note that the above mentioned process is possible. 

Thus, every connected -colored mixed graph with maximum degree  admits a homomorphism to . 
\hfill 

   

\bibliographystyle{abbrv}
\bibliography{NSS14}



 
 
\end{document}
