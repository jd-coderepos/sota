The next lemmas state important observations of a maximal -set packing and are key components in the correctness of our algorithm.

\begin{lemma}\label{maximalisaksolution}
Let  be a maximal -set packing. If , then  is a -set packing.
\end{lemma}

\begin{proof}
Assume otherwise that  is not a -set packing. This would be only possible if there is at least one pair of sets  in  for which  but in that case  would not be a  maximal -set packing. 
\end{proof}

\begin{lemma}\label{intersectionLemma}
Given an instance  of -Set Packing with -Overlap, where  is well-conditioned, let  and  be a -set packing and a maximal -set packing, respectively. 
For each ,  shares at least one element with at least one . 
\end{lemma}

\begin{proof}
If , the lemma simply follows. 
Assume by contradiction that there is a set  such that  and there is no set  -conflicting with . However,  we could add  to , contradicting its maximality. 
Thus, there exists at least one  -conflicting with .
Since  is well-conditioned, by Definition \ref{alphaCondition} (ii) . \qed
\end{proof}


Our Bounded Search Tree algorithm (abbreviated as \texttt{BST--algorithm}) for -Set Packing with -Overlap has three main components: \texttt{Initialization}, \texttt{Greedy}, and \texttt{Branching}. We start by computing a maximal -set packing  of . If  then  is a -set packing and the \texttt{BST--algorithm} stops (Lemma \ref{maximalisaksolution}). Otherwise, we create a search tree  where at each node , there is a collection of sets  with  for some . The goal is \emph{to complete}  to a solution, if possible. That is, to find  sets  of , such that  for  and  is a -set packing. 

The children of the root of  are created according to a procedure called \texttt{Initialization}. After that for each node  of , a routine called \texttt{Greedy} will attempt to complete  to -set packing. If \texttt{Greedy} succeeds then the \texttt{BST--algorithm} stops. Otherwise, the next step is to create children of the node  using the procedure \texttt{Branching}. The \texttt{BST--algorithm} will repeat \texttt{Greedy} in these children. 
Eventually, the \texttt{BST--algorithm} either finds a solution at one of the leaves of the tree or determines that it is not possible to find one. 

We next explain the three main components of the \texttt{BST--algorithm} individually. Let us start with the \texttt{Initialization} routine. By Lemma \ref{intersectionLemma}, if there is a solution  each  contains at least one element of . 
Notice that each element of  could be in at most  sets of . 
Thus, we create a set  that contains  copies of each element in . That is, per each element  there are  copies  in  and . The root will have a child  for each possible combination of  elements from . A set of  is initialized with one element of that combination. For example, if the combination is , . After that, we remove the indices from the elements in , e.g., . 

At each node , the \texttt{Greedy} routine returns a collection of sets . 
Initially,  and . 
At iteration , \texttt{Greedy} searches for a set  that contains  (the th set of ) subject to two conditions (**): (1)  is not already in  and (2)  does not -conflict with any set in  (i.e.,  for each ). 
If such set  exists, \texttt{Greedy} adds  to , i.e.,  and  continues with iteration . 
Otherwise, \texttt{Greedy} stops executing and returns .
If , then  is a -set packing and the \texttt{BST--algorithm} stops.
If  cannot be completed into a solution (Lemma \ref{terminationLemma}), \texttt{Greedy} returns .
\texttt{Greedy} searches for the set  in the collection  which is obtained as follows: add a set  to , if  does not -conflict with any set in  and  is distinct of each set in .

The \texttt{Branching} procedure executes every time that \texttt{Greedy} does not return a -set packing but  could be completed into one. That is,  and  .
Let  and  be the th set in . 
\texttt{Greedy} stopped at  because each set  either it was already contained in , or  it -conflicts with at least one set in  (see **). 
We will use the conflicting elements between  and  to create children of the node .
Let  be the set of those conflicting elements. 
\texttt{Branching} creates a child  of the node  for each element . The collection  of child  is the same as the collection  of its parent  with the update of the set  as , i.e., . 
The set  is obtained as  for each pair  and  that -conflict () or that . The pseudocode of all these routines is detailed in the Appendix. 

\subsubsection{Correctness.}

With the next series of lemmas we establish the correctness of the \texttt{BST--algorithm} for any well-conditioned function .


A collection  is a \emph{partial-solution} of a -set packing  if and only if , for . The next lemma states the correctness of the \texttt{Initialization} routine and it follows because we created a node for each selection of  elements from , i.e., . 

\begin{lemma}\label{InitializationLemma}
If there exists at least one -set packing of , at least one of the children of the root will have a partial-solution.
\end{lemma}

\begin{proof}
By Lemma \ref{intersectionLemma}, every set in  contains at least one element of . 
It is possible that the same element be in at most  different sets of . Therefore, we replicated  times each element in  collected in . 
Since we created a node for each selection of  elements from , i.e., , the lemma follows. \qed
\end{proof}

The next lemma states that the \texttt{BST--algorithm} correctly stops attempting to propagate a collection .
Due to the (i) property of a well-conditioned , we can immediately discard a collection , if it has a pair of sets that -conflicts. In addition, the collection  contains all sets from  that are not -conflicting with any set in  (excluding ). So again, if  is a partial-solution, due to the (i) property,  cannot be empty. 

\begin{lemma}\label{terminationLemma}
Assuming  is well-conditioned,  is not a partial solution either: i. if there is a pair of distinct sets in  that -conflict, or ii. if for some  .
\end{lemma}

\begin{proof}
(i) Suppose otherwise that  is a partial-solution, but  -conflict. 
Since  is a partial-solution,  and  where  ,  and  is a -set packing. 

The pair , does not -conflict, otherwise,  would not be a solution.
However,  is hereditary, , and , thus,  and  do not -conflict either.

(ii) To prove the second part of the lemma, we will prove the next stronger claim. 

\indent \begin{cl}\label{FeasibleSponsorsLemma}
If  is a partial-solution then  for each .
\end{cl}

\begin{proof}
Assume by contradiction that  is a partial-solution but  for some .

If  is a partial-solution,  and ) is a partial-solution as well. 
The set  and  (see Algorithm \ref{FeasibleSponsorsAlgorithm} for the computation of ). 
The only way that  would not be in  is if there is at least one set  in  that -conflicts with  or if  is equal to a set in  but then  would not be a partial-solution a contradiction to (i). \qed
\end{proof} 
\qed
\end{proof}


\texttt{Branching} creates at least one child whose collection is a partial-solution, if the collection of the parent is a partial-solution as well. Recall that  is computed when \texttt{Greedy} stopped its execution at some , i.e., it could not add a set that contains  to . If  is a partial solution,  and . Given property (ii) for a well-conditioned ,  must be intersecting in at least one element with at least on set in . Therefore, at least one element of  will be in .

\begin{lemma}\label{BranchingLemma}
If  is a partial-solution then there exists at least one  such that  is a partial-solution.
\end{lemma}

\begin{proof}
Assume to the contrary that  is a partial-solution but that there exists no element  such that  is a partial-solution. 
This can only be possible if .

First, given that  is a partial-solution  (Claim \ref{FeasibleSponsorsLemma}).

In addition, either  is already in  (i.e, it is equal to some set ) or  must be -conflicting with at least one set ; otherwise,  would have been selected by \texttt{Greedy}. 
Any of these situations implies that  (Definition \ref{alphaCondition} (ii)).
By the computation of  (Algorithm \ref{BranchRoutine}), .

Now it remains to show, that at least one element of  is in . This will guarantee that the set  will be increased by one element at the next level of the tree. This immediately follows if .

Thus, we will show it for the case that  but  -conflicts with .
Suppose that  by contradiction.
Recall that  contains some set  of  (for some ).
Furthermore, ; otherwise  would not have been selected by \texttt{Greedy}.

If  is -conflicting with  but  is not -conflicting with  (otherwise  would not have been in ), then  by property (ii) in Definition \ref{alphaCondition}. \qed
\end{proof} 


\begin{theorem}
The \texttt{BST--algorithm} finds a -set packing of , if  has at least one and  is well-conditioned.
\end{theorem}

\subsubsection{Running Time.}
The number of children of the root is given by   and the height of the tree is at most .  The number of children of each node at level  is equivalent to the size of  at each level . The number of elements in  is at most , thus, . Therefore, the size of the tree is given by:  which is . In addition,   is computable in  for some constant  (property (iii), Definition \ref{alphaCondition}).

\begin{theorem}\label{runningtime}
The -Set Packing with -Overlap problem can be solved in \\  time, when  is well-conditioned.
\end{theorem}