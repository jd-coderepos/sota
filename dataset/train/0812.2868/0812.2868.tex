\documentclass[runningheads]{llncs}

\newcommand{\Oh}[1]
    {\ensuremath{\mathcal{O} \hspace{-.5ex} \left( {#1} \right)}}

\begin{document}

\title{Minimax Trees in Linear Time}
\author{Pawe\l\ Gawrychowski\inst{1}
    \and Travis Gagie\inst{2}\fnmsep\thanks
    {This paper was written while the second author was at the University of Eastern Piedmont, Italy, supported by Italy-Israel FIRB Project ``Pattern Discovery Algorithms in Discrete Structures, with Applications to Bioinformatics''.}}
\authorrunning{P. Gawrychowski and T. Gagie}
\institute{Institute of Computer Science\\
    University of Wroclaw, Poland\\
    \email{gawry1@gmail.com}\\\mbox{}\\
    \and Research Group in Genome Informatics\\
    University of Bielefeld, Germany\\
    \email{travis.gagie@gmail.com}}
\maketitle

\begin{abstract}
A minimax tree is similar to a Huffman tree except that, instead of minimizing the weighted average of the leaves' depths, it minimizes the maximum of any leaf's weight plus its depth.  Golumbic (1976) introduced minimax trees and gave a Huffman-like, -time algorithm for building them.  Drmota and Szpankowski (2002) gave another -time algorithm, which checks the Kraft Inequality in each step of a binary search.  In this paper we show how Drmota and Szpankowski's algorithm can be made to run in linear time on a word RAM with -bit words.  We also discuss how our solution applies to problems in data compression, group testing and circuit design.
\end{abstract}

\section{Introduction} \label{sec:intro}

In a minimax tree for a multiset  of weights, each leaf has a weight , each internal node has weight equal to the maximum of its children's weights plus 1, and the weight of the root is as small as possible.  In other words, if  is the depth of the leaf with weight , then  is minimized.  The weight of the root is called the minimax cost of , denoted .  Golumbic~\cite{Gol76} showed that if we modify Huffman's algorithm~\cite{Huf52} to repeatedly replace the two nodes with smallest weights  and , by a node with weight  instead of , then it builds a minimax tree instead of a Huffman tree.  Like Huffman's algorithm, it takes  time and can build trees of any degree.  Our results in this paper also generalize to higher degrees and larger code alphabets but, for the sake of simplicity, we henceforth consider only binary trees and alphabets.  Golumbic, Parker~\cite{Par79} and Hoover, Klawe and Pippenger~\cite{HKP84} showed how to use Golumbic's algorithm to restrict circuits' fan-in and fan-out without greatly increasing their sizes or depths.  Drmota and Szpankowski~\cite{DS02,DS04} pointed out that, if  is a probability distribution and each , then a minimax tree for  is the code-tree for a prefix code with minimum maximum pointwise redundancy with respect to .  (As we are considering only binary trees in this paper, by  we always mean .)  They gave another -time algorithm for building minimax trees and, by analyzing it, proved bounds on the redundancy of arithmetic coding, which Baer~\cite{Bae08} recently improved by analyzing Golumbic's algorithm.  Drmota and Szpankowski start with a \mbox{Shannon} code~\cite{Sha48} for , in which the codeword for the th character has length , for each ; they sort the logarithms by their fractional parts, i.e., ; and they use binary search to find the largest value  such that  obey the Kraft Inequality~\cite{Kra49}.  In a previous paper~\cite{Gag04} (see also~\cite{Gag07,KN??}) we noted that minimax trees built with Golumbic's algorithm have the same Sibling Property~\cite{Fal73,Gal78} as Huffman trees, and turned the Faller-Gallager-Knuth algorithm~\cite{Knu85} for dynamic Huffman coding into an algorithm for dynamic Shannon coding.  Intriguingly, although static Huffman coding is optimal and static Shannon coding is not, dynamic Shannon coding has a better worst-case bound than dynamic Huffman coding does.

Hu, Kleitman and Tamaki~\cite{HKT79} gave an -time algorithm for building alphabetic minimax trees, in which the leaves' weights, from left to right, must be in the given order.  Kirkpatrick and Klawe~\cite{KK85} and Coppersmith, Klawe and Pippenger~\cite{CKP86} gave an algorithm (or, more precisely, two algorithms that are equivalent when trees are binary) that builds an alphabetic minimax tree for integer weights in  time, and showed how to use it to restrict circuits' fan-in and fan-out without greatly increasing their sizes or depths and without changing the numbers of edge crossings (and, thus, preserving planarity).  Kirkpatrick and Klawe also showed how to combine their algorithm with binary search in order to build alphabet minimax trees for real weights in  time.  We note that, if their algorithm for integer weights is viewed as an alphabetic analogue of the Kraft Inequality --- as it was by Yeung~\cite{Yeu91} and Nakatsu~\cite{Nak91}, who independently rediscovered it --- then their algorithm for real weights is the alphabetic analogue of Drmota and Szpankowski's.  Kirkpatrick and Przytycka~\cite{KP90} gave an -time, -processor algorithm for integer weights in the CREW PRAM model.  In another previous paper~\cite{Gag??} we used a data structure due to Kirkpatrick and Przytycka and a technique for generalized selection due to Klawe and Mumey~\cite{KM95}, to make Kirkpatrick and Klawe's algorithm for real weights run in  time, where  is the number of distinct values .  In this paper we prove a conjecture we made then, that a similar modification can make Drmota and Szpankowski's algorithm run in  time.

\section{Applications} \label{sec:apps}

In the full version of this paper, we will consider all of the following problems:
\begin{enumerate}
\renewcommand{\theenumi}{\Alph{enumi}}
\item \label{prob:ptwise}
    build a prefix code with minimum maximum pointwise redundancy;
\item \label{prob:code}
    given a good estimate of the distribution over an alphabet, build a good prefix code;
\item \label{prob:test}
    given a good estimate of the distribution over a set, design a good group test to find the unique target;
\item \label{prob:reals}
    build a minimax tree for a multiset of real weights;
\item \label{prob:shannon}
    build a Shannon code;
\item \label{prob:depths}
    build a tree whose leaves have at most given depths;
\item \label{prob:circuit}
    restrict a circuit to have bounded fan-in or fan-out;
\item \label{prob:ints}
    build a minimax tree for a multiset of integer weights.
\end{enumerate}
The authors cited in the introduction have already shown, however, that Problem~\ref{prob:ptwise} takes  more time than \ref{prob:reals}, \ref{prob:shannon} than \ref{prob:depths}, and \ref{prob:depths} and \ref{prob:circuit} than \ref{prob:ints}.  Therefore, in the current version of this paper, we consider only Problems~\ref{prob:code}, \ref{prob:test}, \ref{prob:reals} and \ref{prob:ints}.  In the remainder of this section we define what we mean by ``good'' in Problems~\ref{prob:code} and \ref{prob:test}, and show they take  more time than \ref{prob:reals}.  Problems~\ref{prob:code} and \ref{prob:test} are, in fact, equivalent to each other and to \ref{prob:ptwise}, and analogous to a problem we considered in our paper~\cite{Gag??} on building alphabetic minimax trees.  In Section~\ref{sec:ints} we give two -time algorithms for Problem~\ref{prob:ints}.  Finally, in Section~\ref{sec:reals} we show how to use either of those algorithms to obtain an algorithm for Problem~\ref{prob:reals} that takes  time on a word RAM with -bit words.  It follows that all the problems listed above take  time.

Suppose we want to build a good prefix code with which to compress a file, but we are given only a sample of its characters.  Let  be the normalized distribution of characters in the file, let  be the normalized distribution of characters in the sample and suppose our codewords are .  An ideal code for  assigns the th character a codeword of length  (which may not be an integer), and the average codeword's length using such a code is , where  is the entropy of  and  is the relative entropy between  and .  The entropy measures our expected surprise at a character drawn uniformly at random from the file, given ; the relative entropy (also known as the informational divergence or Kullback-Leibler pseudo-distance) measures the increase in our expected surprise when we estimate  by , and is often used to quantify how well  approximates  (see, e.g.,~\cite{CT06}).

Consider the best worst-case bound we can achieve, given only , on how much the average codeword's length exceeds .  A result by Katona and Nemetz~\cite{KN76} implies we do not generally achieve a constant bound on the difference when  is a Huffman code for .  (Given , of course, the best bound we could achieve on how much the average codeword's length exceeds , would be the redundancy of a Huffman code for .)  For example, if  are proportional to , where  denotes the th Fibonacci number (i.e.,  and  for , then the codewords' lengths are  in any Huffman code for .  If  is sufficiently close to 1, then

but the average codeword's length , so for large  the difference is about , where  is the golden ratio.

As long as  whenever , the average codeword's length

(if  but  for some , then  is infinite). Notice each  is the length of a branch in the code-tree for . Therefore, the best bound we can achieve is

which is less than 1, by inspection of Drmota and Szpankowski's algorithm.  (Recall that  denotes the minimax cost of , i.e., the weight of the root of a minimax tree for .)  Moreover, we achieve this bound when the code-tree for  has the same shape as a minimax tree for .  In other words, Problem~\ref{prob:code} takes  more time than \ref{prob:reals}.

Now suppose we want to design a good group test (see, e.g.,~\cite{AW87,Aig88}) to find the unique target in a set, given only an estimate  --- presumably gained from past experience or experimentation --- of the probability distribution  according to which the target is chosen.  A group test allows us to choose, repeatedly, a subset of the elements and check whether the target is among them.  We can represent a group test as a decision tree in which each leaf is labelled with an element and each internal node is labelled with the concatenation of its children's labels.  Because such a decision tree can be viewed as the code-tree for a prefix code, and vice versa, the expected number of checks we make exceeds  by as little as possible when the decision tree for our group test has the same shape as a minimax tree for .  In other words, Problem~\ref{prob:test} is equivalent to \ref{prob:code} and, therefore, also takes  more time than \ref{prob:reals}.

We are currently studying whether either Drmota and Szpankowski's solution to Problem~\ref{prob:ptwise} or our solution to~\ref{prob:code} can give us an intuitive explanation of why dynamic Shannon coding has a better worst-case bound than dynamic Huffman coding does.  On the one hand, worst-case bounds (especially for online algorithms; see, e.g.,~\cite{BE98}) are often proven by considering a game between the algorithm and an omniscient adversary, and minimizing the maximum pointwise redundancy at each step seems somehow related (more than just by name) to the minimax strategy for the algorithm.  On the other hand, dynamic prefix coding can be viewed as a procedure in which we repeatedly build a prefix code based on a sample --- i.e., the characters already encoded.

\section{Minimax Trees for Integer Weights} \label{sec:ints}

In this section we give two -time algorithms for building a minimax tree for a multiset of integer weights, both based on the following lemma (which we note applies to any weights, not only integers) and corollary:

\begin{lemma} \label{lem:replacement}
If  is a multiset of weights and

then .  Moreover, any minimax tree for  becomes a minimax tree for  when we replace the leaves' weights equal to  by the weights in  less than or equal to , in any order.
\end{lemma}

\begin{proof}
Consider a minimax tree  for .  Without loss of generality, we can assume  is strictly binary --- i.e., that every internal node has exactly two children --- and, therefore, that it has height at most .  (Recall that, for simplicity, we consider only binary trees.)  If , then .  Otherwise, all the leaves have depth at least 1, so .  Consider any leaf (if one exists) with weight less than  and depth .  Since , increasing that leaf's weight to  and updating its ancestors' weights, does not change the weight  of the root.  It follows that .

Now consider a minimax tree  for .  If we replace the leaves' weights equal to  by the weights in  less than or equal to  and update all the nodes' weights, then the weight  of the root cannot increase nor, by definition, decrease to less than .  Since , it follows that the re-weighted tree is a minimax tree for . \qed
\end{proof}

\begin{corollary} \label{cor:sorting}
When all the weights in  are integers, we can sort  in  time.
\end{corollary}

\begin{proof}
When all the weights in  at least  are integers, all the weights in  are integers in the interval .  Since this interval has length , we can sort  in  time using either direct addressing, which takes  extra space, or radix sort, which takes no extra space~\cite{FMP07}. \qed
\end{proof}

For our first algorithm, we build and sort ; build a minimax tree for  using a implementation of Golumbic's algorithm that takes  time when the weights are already sorted; and replace the leaves' weights equal to  by the weights in  less than or equal to .  We note that Van Leeuwen~\cite{VLe76} showed how to implement Huffman's algorithm to take  when the weights are already sorted.  We could implement Golumbic's algorithm analogously, but we think the implementation below is simpler.

\begin{lemma} \label{lem:sorted}
Golumbic's algorithm can be implemented to take  time when the weights are already sorted.
\end{lemma}

\begin{proof}
We start with the weights stored in a linked list in nondecreasing order, and set a pointer to the head of the list.  We then repeat the following procedure until there is only one node left in the list, which is the root of a minimax tree for the given weights: we move the pointer along the list to the last weight less than or equal to the maximum of the first two weights plus 1; remove the first two nodes from the list; make those nodes the children of a new node with weight equal to the maximum of their weights plus one; and insert the new node immediately to the right of the pointer.  Notice we remove two nodes for each one we insert, so the total number of nodes is .  Therefore, since the pointer passes over each node once, this implementation takes  time.
\qed
\end{proof}

\noindent Building and sorting  takes  time, by Corollary~\ref{cor:sorting}; building a minimax tree for  takes  time, by Lemma~\ref{lem:sorted}; replacing the leaves' weights equal to  by the weights in  less than or equal to  takes  time, because it can be done in any order.  By Lemma~\ref{lem:replacement}, the resulting tree is a minimax tree for .

\begin{theorem} \label{thm:ints}
Given a multiset  of  integer weights, we can build a minimax tree for  in  time.
\end{theorem}

Our second algorithm differs in its second step: instead of using Golumbic's algorithm to build a minimax tree for , we use Kirkpatrick and Klawe's -time algorithm for integer weights to build an alphabetic minimax tree for the sequence  consisting of the weights in  in non-increasing order.  The algorithm's correctness follows from the Kraft Inequality:

\begin{theorem}[Kraft, 1949] \label{thm:kraft}
If there exists a binary tree whose leaves have depths , then .  Conversely, if  and , then there exists an ordered binary tree whose leaves, from left to right, have depths .
\end{theorem}

\noindent By the latter part of Theorem~\ref{thm:kraft} and a standard exchange argument --- i.e., if a minimax tree contains two leaves such that the deeper one has a higher weight than the shallower one, then we can swap their weights --- there exists a minimax tree for  in which the leaves' weights are non-increasing from left to right.  Therefore, by definition, any alphabetic minimax tree for  is a minimax tree for .

\section{Minimax Trees for Real Weights} \label{sec:reals}

Strictly speaking, Drmota and Szpankowski's algorithm works only when given a multiset of weights equal to  for some probability distribution .  For any value , however, if  and  then, by definition,  and any minimax tree for  becomes a minimax tree for  when we subtract  from each leaf's weight.  In particular, if  then ; therefore,  for some probability distribution  and we can use Drmota and Szpankowski's algorithm to build minimax trees for  and, thus, for .  Without loss of generality, we henceforth assume the given multiset  of weights is equal to  for some probability distribution  (so each .

\begin{theorem}[Drmota and Szpankowski, 2002] \label{thm:D&S}
If  is a multiset of weights,  and  is the largest element in  such that

then any minimax tree for  becomes a minimax tree for  when we replace each leaf's weight  or  by .
\end{theorem}

If  and  then, by Theorem~\ref{thm:D&S},  is the largest index such that  satisfies the Kraft Inequality.  To build a minimax tree for  with Drmota and Szpankowski's algorithm, we compute and sort ; use binary search to find , in each round testing whether the Kraft Inequality holds; build a minimax tree for  ; and replace each leaf's weight  or  by .  Our version differs in three ways: we use generalized selection instead of sorting and binary search; we use a new data structure to test the Kraft Inequality; and we use either of our algorithms from Section~\ref{sec:ints} to build the minimax tree for .  In the remainder of this section we first show how to use generalized selection to find  in  time, excluding the time needed to test the Kraft Inequality; we then show how to perform all the necessary tests in a total of  time on a word RAM with -bit words, using our new data structure.  Since each of our algorithms from Section~\ref{sec:ints} takes  time, it follows that we can build a minimax tree for  in  time.

To find  in  time with general selection, we start with the multiset  and repeat the following procedure until we reach the empty set: in the th round, we use the linear-time selection algorithm due to Blum {\it et al.}~\cite{BFP+73} to find the current multiset 's median , then test whether

if so, we remove those elements of  that are less than or equal to  and recurse on the resulting multiset; if not, we remove those elements of  that are greater than or equal to  and recurse.  The element  is the largest median we consider for which the test is positive.  Since the size of the multisets decreases by a factor of at least 2 in each round, we use  rounds and we find all the medians in a total of  time.

By the same arguments we used to prove Lemma~\ref{lem:replacement}, we can assume, without loss of generality, that  for each .  To test the Kraft Inequality, we use a data structure consisting of two -bit binary fractions,  and , each broken into -bit blocks and initially set to 0.  For , adding  to either fraction takes  amortized time, for the same reason that incrementing a binary counter takes  amortized time (see, e.g.,~\cite[Section 17.3]{CLRS01}).  On a word RAM with -bit words, nondestructively testing whether  takes  time, because adding each corresponding pair of blocks takes  time and, by induction, the number carried from each pair to the next is at most 1; resetting either fraction to 0 takes  time for each block, i.e.,  time in total.

Before starting to search for , we set  in  time.  Throughout our generalized selection, we maintain the invariant that, at the beginning of the th round,

and .  In the th round, we set

in  time.  Since

we can test the Kraft Inequality in  time by checking whether .  If the test is positive, then we add  to  in  time; if the test is negative, then we do not change .  In either case, straightforward calculation shows that, afterwards,

so the first part of our invariant is maintained.  Finally, we reset  in  time, so the second part of our invariant is maintained.  Since , the th round takes a total of  time.  Since  and we use  rounds, it follows that our whole generalized selection takes  time.  This completes the proof of our main result:

\begin{theorem} \label{thm:reals}
Given a multiset  of  real weights, we can build a minimax tree for  in  time on a word RAM with -bit words.
\end{theorem}

\bibliographystyle{plain}
\bibliography{iwoca}

\end{document} 