

\documentclass[a4paper]{article}
\usepackage[T1]{fontenc} \usepackage{RR}
\usepackage{hyperref}
\usepackage{color}
\usepackage{cite}
\usepackage{graphicx}
\usepackage{algorithm}
\usepackage{algorithmic}
\usepackage{ntheorem}


\usepackage[cmex10]{amsmath}



\newcommand{\TODO}[1]{\textcolor{blue}{\em \bf  #1 }}
\newcommand{\power}{P_T}
\newcommand{\T}{\mathcal{T}}
\newcommand{\M}{\mathcal{M}}
\newcommand{\X}{\mathcal{X}}
\newcommand{\V}{\mathcal{V}}
\newcommand{\Orig}{\mathcal{O}}
\newcommand{\D}{\mathcal{D}}
\newcommand{\E}{\mathcal{E}}
\newcommand{\R}{\mathcal{R}}
\newcommand{\Nout}[2]{ \overrightarrow{ \mathcal{N}_{#1}^{#2} } }
\newcommand{\Nin}[2]{ \overleftarrow{ \mathcal{N}_{#1}^{#2} } }
\newcommand{\Fout}[3]{\overrightarrow{ F_{#1}^{#2}}(#3)} 
\newcommand{\Mmatrix}{M}
\newcommand{\Qmatrix}{Q}
\newcommand{\Rmatrix}{D}
\newcommand{\Nmatrix}{M_F}
\newcommand{\Smatrix}{S_M}
\newcommand{\Msrc}{\Mmatrix_S}
\newcommand{\Qsrc}{\Qmatrix_S}
\newcommand{\Rsrc}{\Rmatrix_S}
\newcommand{\Qe}[4]{\Qmatrix_{#1#2}^{#3#4}} 
\newcommand{\Qd}[4]{\Rmatrix_{#1#2}^{#3#4}} 



\RRdate{September 2010}


\RRauthor{Katia Jaffr\`es-Runser
\thanks[sfn1]{\'Equipe SWING} 
  \thanks[sfn2]{Stevens Institute of Technology, Hoboken, NJ, USA}\and
Cristina Comaniciu
\thanksref{sfn2}
    \and
Jean-Marie Gorce
\thanksref{sfn1}
}
\authorhead{Jaffr\`es-Runser, Gorce \& Comaniciu}
\RRtitle{De la prise en compte des interf\'erences et de la nature broadcast du canal radio pour l'\'evaluation de performance des r\'eseaux sans-fils}
\RRetitle{On the performance evaluation of wireless networks with broadcast and interference-limited channels}
\titlehead{On the performance evaluation of wireless networks}
\RRnote{This paper has been submitted to IEEE INFOCOM 2011}
\RRresume{
Ce rapport de recherche pr\'esente un mod\`ele multiobjectif d'\'evaluation des performances d'un r\'eseau ad hoc sans-fils. Les crit\`eres tels que la capacit\'e, la robustesse, l'\'energie et le d\'elais de transmission sont optimis\'es conjointement.
Gr\^ace \`a ce mod\`ele il est possible de d\'eterminer une borne en performance Pareto-optimale et les diff\'erents param\`etres du r\'eseau qui permettent d'obtenir de telles bornes. 
L'originalit\'e de cette approche r\'eside dans le fait qu'elle mod\`ele finement la nature broadcast intrins\`eque du canal radio et prend en compte de fa\,con pr\'ecise la distribution des interf\'erences dans le r\'eseau. 
Il est ici possible d'optimiser les d\'ecisions de routage et d'ordonnancement des paquets dans un r\'eseau quand plusieurs flots sont transmis dans le r\'eseau.
La complexit\'e de calcul de l'ensemble des solutions Pareto-optimale n'augmente |pas avec le nombre de flots pr\'esents dans le r\'eseau. 
La contribution majeure de ces travaux est la pr\'esentation d'un nouvelle formulation analytique des mesures de performance. 
Cette formulation se base sur une repr\'esentation matricielle des contraintes impos\'ees par la nature broadcast du canal et la distribution des interf\'erences dans le r\'eseau. 
Du fait de la similarit\'e de cette matrice avec une matrice Markovienne, il est possible d'exploiter des r\'esultats classiques de la th\'eorie des cha\^ines de Markov pour d\'eduire des m\'etriques de capacit\'e, robustesse, \'energie et d\'elais, le tout pour un r\'egime permanent.  
}
\RRabstract{
In this report we propose a MultiObjective (MO) performance evaluation framework for wireless ad hoc networks where criteria such as capacity, robustness, energy and delay are optimized concurrently.
Within such a framework, we can determine both the Pareto-optimal performance bounds and the networking parameters that provide these bounds. 
The originality of this approach is that it accounts for the inherent broadcast properties of the transmission and finely models the interference distribution.
In the proposed model, the network performance can be optimized when several flows (source-destination transmissions) exist. One benefit of our approach is that the complexity does not grow with the number of flows.
The other major contribution of this paper is the new analytical formulation of the performance metrics. It relies on a matrix representation of the constraints imposed by the interference-limited and broadcast wireless channel. Because of the similarity of this matrix with a Markovian transition matrix, we can exploit classical results from Markov chains theory to derive steady state performance metrics relative to capacity, robustness, energy and delay.    
Another very interesting feature of these new metrics is that the Pareto-optimal solutions related to them provide a tight bound on capacity, robustness, energy and delay.  
}

\RRmotcle{r\'eseaux ad hoc sans-fil, optimisation multi-objectifs, \'evaluation de performances}
\RRkeyword{wireless ad hoc networks, multiobjective optimization, performance evaluation}
\RRprojets{SWING}
\RRdomaine{1} \RRtheme{Mod\`ele pour l \'evaluation des performance d'un r\'eseaux ad hoc sans-fils}
\URRhoneAlpes \RCGrenoble 

\begin{document}
\RRNo{7379}
\makeRR   



\section{Introduction}\label{sec:introduction}
Wireless ad hoc and sensor networks are many times operating in difficult environments and require several performance criteria to be satisfied, related to timely, reliable, and secure information transfer.Routing and resource allocation protocols are key elements  for ensuring the information transfer across the network.
To better understand the capabilities of such protocols for a given network topology, it would be very helpful to know the bounds that can be achieved with respect to multiple performance criteria. These bounds illustrate the interdependence between multiple performance metrics and can capture the tradeoffs between the various operating points for the network. As a consequence, defining a unified multi-objective design framework capable of capturing the above mentioned tradeoffs for various possible operating points for the network becomes of premier importance.

However, globally optimizing capacity for two-dimensional networks where routing and resource allocation (i.e. frequency, time or power assignment) are performed concurrently is a very hard problem that has triggered a comprehensive research effort under various conditions since the seminal work of Gupta and Kumar \cite{Gupta2000}. Several works provided ways for increasing the asymptotic capacity bound of  \cite{Gupta2000,Wang2008} by for instance accounting for mobility \cite{Grossglauser2002}.
In these works two important assumptions are made: unicast communications and threshold-based interference models.
In a unicast communication, packets are sent to a specific receiver on a multi-hop chain. In a threshold-based model, nodes interfere with other nodes within a fixed range while beyond that range, no node is interfered. 
These results have been extended in \cite{Mhatre2009} by accounting for realistic additive interference or in \cite{Comaniciu2006} by considering multi-user detection techniques. 
Toumpis et al. \cite{toumpis2003} have also proposed an interesting model to derive the capacity region by properly scheduling transmissions and hence accounting for a temporal multiplexing directly in a  dimensional network model. While the model is appealing, it is not scalable with respect to the network size and the number of flows transmitted.

The broadcast nature of the wireless channel is mostly assumed as the main source of interference in these works and hence negatively impacts the network performance. However, this broadcast property can serve the capacity if the receiving nodes are able to cooperate and coordinate to transmit the data flow \cite{Dana2006}.  It has been shown to be beneficial in the context of opportunistic routing \cite{Jacquet-JSAC2009} as well.


In this paper we consider the broadcast nature of the channel as a possible way to increase the capacity, and we propose realistic interference models. This work does not aim to assess the performance of a specific routing protocol, but more generally to extract the boundary of the feasible performance region of the network. Further, we consider that capacity is not the only interesting objective. More specifically, it is the trade-off between different performance criteria that should be considered to properly select a transmission strategy in the network. 
For instance, increasing the number of parallel paths improves capacity but at the expense on an increased energy due to the multiplication of packets traveling in the network. Thus, various criteria related to transmission delay \cite{Brand2008r}, energy consumption \cite{Vassileva2007} or fairness \cite{Eryilmaz2006} should be considered in addition to the main design goal of reliable information transmission. 
As a consequence, the assessment of networking protocols usually relies on various criteria which may be evaluated analytically or through network simulations.
In this work, we concentrate on three criteria: capacity, energy and latency. As we will see in Section \ref{sec:steadystate}, robustness is related to capacity with a redundancy factor. 

We consider a network as a set of nodes comprised of relays, sources and receivers. In this work, we do not look for the optimal single path between a source and a destination, but for a set of relays deciding to forward packets or not. This approach is more closely related to a diffusion mechanisms than to a strict single path routing protocol. 
However, our more general formulation encompasses single path routing solutions. 
This concept has been inspired by the pioneering works on optic and electrostatic inspired routing \cite{Jacquet2004-MobiHoc,Toumpis2008-ComputNet,Altman2008-AdhocNow}. 
However, our work is not directly related to a physics inspired phenomenon. 
We provide here a more practical framework that relies on a time and spatial multiplexed approach: we model the decision of each relay to transmit, receive or just sleep with a given probability associated to a time slot. These forwarding probabilities define a steady state of the network.
Then, we propose an analytical framework to obtain an estimate of overall capacity, latency and energy depending on the forwarding probabilities associated to each node and time slot. 
 Since interference is modeled for a network steady state, the results obtained scale with respect to the number of flows being transmitted in the network. 

The paper is organized as follows. Section \ref{sec:preliminaries} gives some preliminary notations. The network model is detailed in Section \ref{sec:networkmodel} and Section \ref{sec:problemstatement} describes the MO optimization problem. The steady-state performance evaluation tools are described in Section \ref{sec:steadystate} and Section \ref{sec:conclusion} concludes the paper. 

\section{Preliminaries}\label{sec:preliminaries}

\subsection{Notations}
Throughout this paper, upper case letters (e.g., , , ) usually denote vectors or matrices. 
Lower case letters (e.g., , , ) denote probability values. 
The transpose of a vector or a matrix is shown by . 
Subscripts denote the indeces of nodes of the network. For instance,  could denote a vector relative to node  and  a probability related to both nodes  and .
Superscripts are introduced to specify the index of a time slot. For instance,  could denote a probability value relative to time slot  and . 
Sets are denoted by calligraphic alphabets (e.g., , , ) and the cardinality of set  is denoted by . The complement of a set  is shown by .

A flow of symbols coming \emph{out} of a node  on time slot  is denoted by  and a flow of symbols coming \emph{into} node  from node  in time slot  is denoted by  .  
Table~\ref{tab:notations} summarizes our notations.


\begin{table} 
\caption{Some important notations in this paper} 
\centering
\label{tab:notations}
\begin{tabular}{| c | l |}
\hline
			& Set of nodes \\
		& Set of edges \\
		& Set of source nodes \\
		& Set of destination nodes\\
		& Set of possible relay nodes\\
				& Number of possible relay nodes\\
				& Set of time slots composing one frame\\
 		& Set of incoming edges at node   on time slot \\
 		& Set of outcoming edges at node  on time slot \\  
		& \emph{Transmission rate} for node  in time slot  \\
			& Discrete set of possible values of  \\			
 		& \emph{Channel probability} on edge  for slot  \\ 
		& \emph{Forwarding probability} for a packet sent by node  in \\
				&  time slot  to be forwarded by node  in time slot \\
			& Set of active transmission in time slot \\
				& Set of all actives transmission \\
		& Redundancy, delay and energy criteria, resp. \\
\hline

\end{tabular}
\end{table}

\subsection{Definitions of graph theory}

In this part, we briefly review the concepts and definitions from graph theory considered in this paper \cite{graph}. 
A complete graph  has vertex set  and edge set . Without loss of generality, let
.

We assume that the graph is finite, i.e., . 
For each node ,  and  are the set of edges leaving from and the set of edges going into , respectively. Formally 


A complete graph  of  vertices is a graph having the maximum number of edges. 






\section{Network model}\label{sec:networkmodel}

We model the wireless ad hoc network by a finite complete graph  for two reasons: modeling the \emph{broadcast nature} of the wireless channel and finely accounting for \emph{interference}.
Transmission in the network is time multiplexed and a frame of  time slots is repeated in each epoch . 
Each edge  for a particular time slot  represents an interference-limited channel which is modeled by the probability of a symbol or a packet to be correctly transmitted. This probability is referred to as the \emph{channel probability} in the following. It models interference as an additive noise and is computed considering the distribution of the bit error rates (BER) or the packet error rates (PER) as shown hereafter. As a consequence, there are   orthogonal interference-limited channels for each edge  as illustrated on Fig.~\ref{fig:networkmodel}.
Each channel is assumed to be in a hald-duplex mode, i.e. a node cannot transmit and receive a packet at the same time. 



A set of sources  and destinations  is defined. A flow of packets is transmitted between any source-destination combination . 
We make the assumption that source and destination nodes do not relay the information. As a consequence, the network we are modeling is composed of a set of relay nodes . In the following, we consider that the number of relays in the network is . We also assume that a relay can not differentiate packets. As a consequence, all packets are treated as being unique by a relay.

A \emph{transmission} is defined as the couple  and represents the fact that node  is transmitting in a time slot .



\subsection{Transmission rate}
We consider that a node  transmits a flow of symbols in time slot . The symbols are the realization of a random variable  chosen from an alphabet  ( for instance). With this definition, we consider that a node  transmits the same symbol in a time slot  on all its outgoing edges . This constraint incorporates the broadcast property of wireless communications. Time multiplexing is here accounted for in our model since a node can transmit different symbols on each available time slot. 
 
A flow of symbols  coming into node  from node  on time slot  is modeled as a random variable . The symbols are transmitted by node  on the edge  in time slot  and consequently, experience the probability  of being received successfully in . Depending on the interference temperature of the network, symbols are received successfully or not at . 

Having this, let  be the random variable giving all the symbols that can be received on the incoming channels of node  for all time slots where . 
Let  be the random variable giving all the symbols that can be transmitted on the outgoing channels of node  for all the time slots. 
The relation between the \emph{outgoing} random variables  and the \emph{incoming} random variables   defines a coding scheme for the network.
 
A node has one main decision to take upon receiving a symbol or a packet: whether it should process it or not. If it decides to process it, the next steps are to


 Decide on the coding for the symbol or packet to be sent. 

 Decide on which channel to transmit it. 

These decisions strongly influence the quality of the transmission and aim at mitigating the transmission errors due to fading and interference. 
Depending on these decisions, the transmission rate of a node  on a channel  varies. For instance, if a node decides to drop one packet out of two received on a same time slot , the transmission rate on channel  would become half the rate at which it received packets on the same time slot. 
For instance, it is possible that a node transmits one packet every two received packets on time slot  because it is applying a coding scheme for which two packets are combined into a single transmitted packet. In both cases, the node is adapting the transmission rate on each channel to combat errors.   

\begin{figure}
\begin{center}
  \scalebox{0.45}{
  \begin{picture}(0,0)\includegraphics{NetworkModel.eps}\end{picture}\setlength{\unitlength}{3947sp}\begingroup\makeatletter\ifx\SetFigFont\undefined \gdef\SetFigFont#1#2#3#4#5{\reset@font\fontsize{#1}{#2pt}\fontfamily{#3}\fontseries{#4}\fontshape{#5}\selectfont}\fi\endgroup \begin{picture}(6712,2485)(2611,-4145)
\put(2851,-2836){\makebox(0,0)[lb]{\smash{{\SetFigFont{25}{30.0}{\rmdefault}{\mddefault}{\updefault}{\color[rgb]{0,0,0}}}}}}
\put(5701,-2011){\makebox(0,0)[lb]{\smash{{\SetFigFont{25}{30.0}{\familydefault}{\mddefault}{\updefault}{\color[rgb]{0,0,0}}}}}}
\put(5701,-2611){\makebox(0,0)[lb]{\smash{{\SetFigFont{25}{30.0}{\familydefault}{\mddefault}{\updefault}{\color[rgb]{0,0,0}}}}}}
\put(3751,-2611){\makebox(0,0)[lb]{\smash{{\SetFigFont{25}{30.0}{\familydefault}{\mddefault}{\updefault}{\color[rgb]{0,0,0}}}}}}
\put(3751,-2011){\makebox(0,0)[lb]{\smash{{\SetFigFont{25}{30.0}{\familydefault}{\mddefault}{\updefault}{\color[rgb]{0,0,0}}}}}}
\put(3751,-3211){\makebox(0,0)[lb]{\smash{{\SetFigFont{25}{30.0}{\familydefault}{\mddefault}{\updefault}{\color[rgb]{0,0,0}}}}}}
\put(8326,-4036){\makebox(0,0)[lb]{\smash{{\SetFigFont{20}{24.0}{\familydefault}{\mddefault}{\updefault}{\color[rgb]{0,0,0}Node }}}}}
\put(2626,-4036){\makebox(0,0)[lb]{\smash{{\SetFigFont{20}{24.0}{\familydefault}{\mddefault}{\updefault}{\color[rgb]{0,0,0}Node }}}}}
\put(5701,-3211){\makebox(0,0)[lb]{\smash{{\SetFigFont{25}{30.0}{\familydefault}{\mddefault}{\updefault}{\color[rgb]{0,0,0}}}}}}
\put(8551,-2836){\makebox(0,0)[lb]{\smash{{\SetFigFont{25}{30.0}{\rmdefault}{\mddefault}{\updefault}{\color[rgb]{0,0,0}}}}}}
\end{picture} }
  \caption{Relay network model: transmission rate and channel probabilities}
\label{fig:networkmodel}
\end{center}
\end{figure}

Based on this statement, we characterize the behavior of a node  by the rate at which it is transmitting in each time slot . The \emph{transmission rate} of node  in time slot  is denoted by . Having this, a vector of transmission rates for each time slot can be defined
  

The transmission rate can be interpreted as the percentage of time a node is transmitting packets in a given time slot. For instance, if a node  transmits at a rate of  in time slot , it transmits a packet every two time slots, if one time slot permits to transmit exactly one packet. 
We assume that when a node is not transmitting on a channel, which happens with probability , it is listening to packets. This is to comply with the half duplex assumption. In other words, the \emph{listening rate} is equal to . The model could be extended to also account for the time a node may be sleeping, but this option has been disregarded so far because it would increase the dimension of our search space. However, it would be worth investigating such an option when minimal energy consumption prevails as it is the case for wireless sensor networks.
The proposed network model is consequently defined in a steady state mode, where we know which nodes are transmitting or not packets, continuously.

The average rate at which all the symbols are coming into node  is given by 

for  the probability to receive a symbol correctly. 
The rate at which symbols are being transmitted by node  is given by: 


Let  be the matrix of the transmission rates assigned to all the nodes of the network for all time slots. A particular instance of  belongs to the set of possible transmission rate matrices shown by .
 is feasible if Properties 1 and 2 hold for each node: 

{\sc Property 1:} {\it Flow conservation}.  The rate of all outgoing flows is lower or equal to the rate of all incoming flows, i.e.
  

{\sc Property 2:} {\it Half duplex.} A node  is able to receive a message on a time slot  the proportion of time it is not transmitting on that same time slot. 
As a consequence, the proportion of packets being actually received on time slot  is given by
  
\noindent where  stands for the incoming rate in time slot . The conditions of \eqref{eq:halfduplex} are always true for the definition of the transmission and listening rates. However, this constraint becomes more meaningful if a sleeping rate variable is defined for a node. 
  
Now that we have defined the transmission rate matrix, we can define the set of active transmissions  corresponding to . A transmission  is said to be active if . As a consequence, the set  is given by:

Similarly, the set  refers to the set of transmission being active in a same time slot . We have 

\subsection{Channel probability}
Knowing , it is straightforward to derive the interference temperature of the network. We model interference as an additive noise computed at the end of each link  of the network. 
Let  be the probability for a symbol transmitted by node  to arrive \emph{successfully} at node  in time slot . We denote  as the channel probability.
It is a function of the statistical distribution of the Signal to Noise and Interference Ratio (SINR) at the location of the destination node . 
This quantity can either be defined using a Packet Error Rate (PER) or a Bit Error Rate (BER) depending if transmissions on the network are packetized or not.  
In the following we consider that transmissions are packetized.
Before giving the exact expression of the channel probability, a few preliminary definitions and notations are given hereafter:

\paragraph*{Pathloss attenuation factor and transmission power}
 reflects the attenuation due to propagation effects between nodes  and . In our simulations, the simple isotropic propagation model is considered.
We consider that all the nodes use the same transmission power denoted as .



\paragraph*{Interference}
Since we consider time-multiplexed channels, interference only occurs between transmissions using the same time slot.
Let   be the power of the interference on the link  on time slot  and computed at node . 
If we denote by  a set of nodes of the network interfering at node  (not including ) in time slot , interference power is defined by:

\noindent where  gives the number of the node interfering at .
 
\paragraph*{SINR}
The SINR between any two nodes  and  in resource  is given by the following equation:

where  is the interference power on the link and  the noise power density.

\paragraph*{Packet error rate (PER)}
For a specific value of SINR , the packet error rate  can be computed according to:

\noindent where  is the number of bits of a data packet and  is the bit error rate for the specified SINR per bit  which depends on the physical layer technology and the statistics of the channel.
Results are given for an AWGN channel and a BPSK modulation without coding where .

\paragraph*{Interfering sets}
A node  is said to be active in the network if . We recall that  designates the set of active transmissions in time slot . 
We give now a more precise definition of an interfering set. An interfering set  on a link  in time slot  is any subset of the set  made of  elements, .
If , let  refer to the set of all possible interfering sets of cardinality .

\vspace{\baselineskip}

Equation \eqref{eq:linkProbaPER} details the derivation of the channel probability  as the average of the PER experienced for all possible interfering sets  on time slot  referred to as :

\noindent where , where  is the SINR experienced on the link  in time slot  if the nodes of interfering set  are active.  

 is the probability for the interfering set  to be active and create interference on the link  on time slot . More specifically, it is the probability that the nodes of the interfering set  are transmitting concurrently and the others are not as specified in the following:

In \eqref{eq:probaPER},  gives the probability that the  active nodes of the interfering set  are transmitting and  the probability that the  other active nodes are not.

Similarly to the vector of transmission rates, we define the vector of channel probabilities for a link :


Let  be the matrix giving all incoming channel probabilities at node 

and  the matrix of all channel probabilities in the network 







\subsection{Forwarding and scheduling}

Now, lets introduce the forwarding and scheduling decisions of the nodes relative to the transmission of a packet. This decision is represented by the probability  of a node  to transmit on time slot  a packet coming from node  on time slot . We will refer in the rest of the text to the forwarding probability  . 
For each node of the network, we can define a -by- matrix giving all the forwarding probabilities relative to any node  of the network as follows. This forwarding matrix  is given by:


where each matrix   provides the scheduling probabilities of a flow of packets coming from node  on its output times slots, depending on the time slot the packets are received on.


The matrix of forwarding probabilities is related to the matrix of transmission rates  and the matrix of channel probabilities  with the following set of  equations

\noindent where  is the probability that a packet sent by  on time slot  arrives and  is the probability that node  is listening on channel . These equations introduce strict constraints on the choices of the forwarding probabilities. 

The forwarding probabilities represent the decisions of the nodes to either  retransmit all the packets or symbols received or  reduce the output rate by dropping or re-encoding them together. 
From now on, we will refer to the set of all forwarding probabilities of the complete network using a matrix  of size -by- defined by:


\noindent where  is the set of all possible matrix instances.







\section{MO optimization problem}\label{sec:problemstatement}

The scope of this section is now to take advantage of the previously described network model to derive a framework capable of extracting the set of Pareto-optimal transmission strategies with respect to various performance criteria (e.g. capacity, delay, energy ).
A Pareto-optimal set is composed of all the non-dominated solutions of the MO problem with respect to the performance metrics considered. The definition of dominance is:

{\sc Definition 1:} A solution  dominates a solution  for a objective MO problem if  is at least as good as  for all the objectives and  is strictly better than  for at least one objective. Mathematically, we have for a minimization problem:

\noindent Notation  means that  strictly dominates .
In the space of the evaluation functions, for the case of a minimization of all the optimization objectives, the set of Pareto-optimal solutions for an example 2-objective problem is illustrated in Fig.~\ref{fig:dominance}.

\begin{figure}
\begin{center}
\scalebox{0.5}{
\begin{picture}(0,0)\includegraphics{dominance.eps}\end{picture}\setlength{\unitlength}{3947sp}\begingroup\makeatletter\ifx\SetFigFont\undefined \gdef\SetFigFont#1#2#3#4#5{\reset@font\fontsize{#1}{#2pt}\fontfamily{#3}\fontseries{#4}\fontshape{#5}\selectfont}\fi\endgroup \begin{picture}(5137,3777)(4561,-5032)
\put(4576,-1486){\makebox(0,0)[lb]{\smash{{\SetFigFont{14}{16.8}{\rmdefault}{\mddefault}{\updefault}{\color[rgb]{0,0,0}}}}}}
\put(9451,-4936){\makebox(0,0)[lb]{\smash{{\SetFigFont{14}{16.8}{\rmdefault}{\mddefault}{\updefault}{\color[rgb]{0,0,0}}}}}}
\end{picture} }
\caption{Non-dominated solutions are highlighted with red circles for a 2-function minimization problem.}
\label{fig:dominance}
\end{center}
\end{figure}

\subsection{Solution definition}
A solution of our MO optimization problem is given by the set of all the forwarding probabilities represented by matrix  defined in \eqref{eq:xmatrix}. As a consequence, the set  becomes the problem search space. 
Let   and  be the performance criteria relative to capacity, robustness, delay and energy, respectively. Capacity and robustness are maximized, while energy and delay are minimized. The derivation of these metrics is provided in the next section. 
Our goal is to solve the following multiobjective optimization problem by finding the set of Pareto-optimal solutions :

\noindent where . The dominance relation is defined in \eqref{eq:dominance} and adapted for criteria  and .

The cardinality of the search space  as defined grows exponentially with  and . However, considering the complete search space may be meaningless depending on the network connectivity.  
As a consequence, it is reasonable to define the search space for a maximum number of relays .     
That is why we define  as the subset of  made of at most  active relays and refer in that case to the -relay problem. In this case, we are looking for the Pareto-optimal combinations  of  relays in the set .  

There is a whole set of constraints for a solution  to be valid related to Properties 1 and 2. As a consequence, the solutions of    that do not follow these properties are dropped before being evaluated by the search algorithm since they are considered as not feasible. We recall that Property 1 ensures that the flow conservation constraint at each node holds and that Property 2 ensures that the half duplex constraint is met.  

\subsection{Change of variable}

The derivation of the transition rate matrix  knowing a solution  is intractable. In a nutshell, to compute the transmission rates of a node ,  and the incoming transmission probabilities   have to be known yet the elements of  are a function of the transmission rates matrix, creating a circular dependency between the variables.

Nevertheless, being able to model exactly the interference temperature of the network is appealing and to keep the same formalism, we propose a reverse approach where a solution of the optimization problem is defined by the set  of the transmission rates for all the nodes and time slots. From , it is possible to derive the channel probabilities matrix  according to \eqref{eq:probaPER} since the activity of all nodes on each time slot is known. 

Only instances of  that meet the constraints relative to Property 1 and 2 are further considered as valid. 
Now that we have a valid , we can derive all the forwarding matrices  that verify the constraints of equation \eqref{eq:fwfprobaflowconservation}. 
There are  constraints, each one constraining the choice of the  for all nodes and time slots of the network with respect to .
Let  be the subset of  that verifies \eqref{eq:fwfprobaflowconservation} with respect to the transmission rate matrix . 
Each solution  can be evaluated according to  and . 

The MO optimization problem of \eqref{eq:MOproblem} stays unchanged, however, the way the search space is constructed has changed and permits to select in a first stage a subset of valid and interesting solutions . 
This feature is very interesting in terms of optimization. In fact, if it is possible to determine knowing  if the set of corresponding  is promising or not, it is possible to apply pruning techniques that disregard non-promising 's instances. 

\subsection{Problem complexity}
We recall that each transmission rate  takes its values in the continuous closed set . However, to be able to use common MO optimization heuristics, we formulate the problem as a combinatorial MO optimization problem. Therefore,  transmission rates take their values in a discretized set  of  values.

The set  of possible  solutions has a cardinality of . The set of feasible solutions is reduced by Property 1 but since the relation between  and  is complex, it is difficult to derive analytically the exact number of feasible solutions. We can just say that  is bounded by .

The set  of possible forwarding probabilities matrices has a cardinality of . The subsets  of feasible  variables is also reduced because of Property 1. Let  be the set of feasible  solutions. 
 is the union of the  for all feasible  instances. 
Similarly to , it is difficult to evaluate the size of . Thus, .




\section{Steady state performance evaluation}\label{sec:steadystate}

This section provides a framework for the definition of performance criteria for a particular solution  of our MO optimization problem. 
It is an important contribution because it permits a fast performance derivation for end-to-end criteria such as capacity, robustness, delay and energy. Moreover, this framework also permits to reduce the complexity of the problem by pruning solutions where the transmission rate matrix  does not meet sufficient performance constraints. 
Our framework relies on the definition of a transition matrix which is composed of the probabilities for a flow of packets to be re-transmitted by the nodes of the network. Such a formulation is inspired by the theory of Markov chains. However, we want to stress that we do not model the network using a Markov chain. We define instead a transition matrix that has the properties needed to be able to re-use some results from the theory of Markov chains.
Table \ref{tab:notations2} gives notations specific to this section. 

\begin{table} 
\caption{Main notations in this section} 
\centering
\label{tab:notations2}
\begin{tabular}{| c | l |}
\hline
	& Flow vector at time epoch  \\
		& Transition matrix \\
		& Relaying matrix \\
		& Arrival matrix \\
		& Fundamental matrix \\
		& Source matrix \\
\hline

\end{tabular}
\end{table}
  
  
\subsection{Flow vector}
Let  represent the probability of node  to transmit a packet in time slot  at time epoch . We consider that at the beginning of a time epoch, a frame of  time slots starts and all nodes having a packet to send can transmit it in their respective time slots. A new epoch starts when all nodes have finished receiving their packets and are ready to send new ones if needed. 
Let  be a vectorial representation of these probabilities for each node and each time slot at a given time epoch . 
 is referred to as the \emph{flow vector} and is shown by:

\noindent where .

The vector  stands for the flows received at the destinations nodes of set . 
It is given by  where  gives the flows received in each time slot by destination  .

The flow vector is a  matrix when all the possible transmissions are accounted for. However, if a relay  never transmits in a time slot  (i.e. ), we have .    
As a consequence, the size of the flow vector can be reduced to the size of the only transmissions that are active. Hence, the size of  can be reduced to a .

\subsection{Transition matrix}
The propagation of the packets and the forwarding decisions of the nodes can be represented by a transition matrix .
Each element of the matrix  gives the probability for a packet transmitted by one relay in one of the time slots to be transmitted by any other relay in the network in one of the time slots.   
When it is applied to the flow vector of time epoch , the new flow vector of time epoch  is obtained. Formally we have:
  

The transition matrix  has the following structure:
 
\noindent For  and ,  is a non-zero -by- matrix,  is a non-zero -by- matrix,  is a -by- identity matrix and  a  -by- zero matrix.  is referred to as the relaying matrix and  as the arrival matrix. 

Transition matrix  has a similar canonical structure than a finite absorbing Markov chain \cite{markov}. Here, we have  transient states, i.e. relay nodes that are forwarding packets using active transmissions. The relaying matrix  gives the probabilities for the packets sent at time epoch  to be retransmitted at time epoch  by the relays of the networks. 
We have  absorbing states where the destination nodes of  receive packets on each time slot and keep them. The identity matrix represents the fact that packets received by a destination are never forwarded, but absorbed.  Matrix  is composed of the probabilities from going from one transient state to one absorbing state, i.e. the probability for packets to be received at the destinations when being transmitted from any relay or source in the network. 

\emph{The relaying matrix } is structured as follows:

\noindent 0 is an -by- zero matrix representing the fact that a node  does not forward a packet coming from itself. The matrix  is a -by- matrix that gives the probabilities of node  to transmit a packet sent by node  for all possible combinations of time slot. Its structure is:

\noindent where  is the probability for a node  to retransmit on channel  a packet that has been transmitted by node  on time slot . With respect to our network model, it is equal to:
 
It is important to note that the powers of matrix  tend to 0 if its elements verify the strict inequality . For this property to hold, we have to ensure that . This is usually the case for a wireless transmission where perfect channels exist in very few unrealistic cases.

\emph{The arrival matrix } is given by:

\noindent  is a -by- diagonal matrix whose diagonal elements  give the probabilities for a packet transmitted by a node  in time slot  to arrive at destination .  Consequently,  is the probability of reception:

\noindent since  for any destination node .

\emph{The source matrix } is needed to compute the initial flow vector , naming the probabilities of all the nodes to transmit a packet sent by a source.  is a -by- matrix. An element  gives the probability of the source  to transmit in time slot . An initial flow matrix is obtained from  with

       
\noindent where  is a -by- transition matrix defined as  with  and  the relaying and arrival matrix for the packets sent by the source, respectively. We have 
       
\noindent and 
       
\noindent where  follows the pattern given by \eqref{eq:relaying2}. Since destinations do not retransmit packets,  is a -by- diagonal matrix whose diagonal elements are .

The flow matrix  created at epoch  is a -by- matrix of flows. The initial flow vector  relative to source  is given by the  line of . 
A very interesting feature of the steady state performance evaluation and network model proposed herein is that the matrix  and  can be applied to any type of source-destination communication (unicast, multicast, anycast, ...). Indeed, this is a direct consequence of the fact that the inherent broadcast and interference-limited properties of the radio channel are accounted for in the structure of .  
For instance, for a multicast transmission, the source matrix  is composed of a single source and the arrival matrix   is defined for   destinations. The matrix  stays unchanged.
Or when multiple sources are transmitting to a same destination (a sink for instance in a wireless sensor network), a source matrix of  is composed of several nodes. Each one of them is propagated in the network using the same  to a unique destination modeled by .  

For clarity purposes, we have presented all matrices using all  possible relays. However, when considering a specific solution , these matrices are derived only for the set  of active transmissions. For instance, the size of the square matrix  is reduced from  to .  

\subsection{Fundamental matrix}

We really want to stress that the proposed transition matrix definition is not the definition of a Markov chain. Indeed, the sum of the probabilities on a line of  is higher than . However, having  ensures that  tends to zero as the number of time epochs  tends to infinity. 
As a consequence, the following theorem still holds:

{\sc Theorem 1:}  has an inverse, and


{\sc Proof}
The complete proof is omitted for conciseness purposes but can be found in \cite{markov}. Even though  is not a Markovian transition matrix, the only property needed for Theorem 1 to hold in our case is that  as . 

\hspace{4.2in}{\textit{q.e.d.}

The inverse of  is defined as the \emph{fundamental matrix} and denoted .
In our case, the fundamental matrix gives us the mean of the total number of transmissions that have been done until all packets be received at the destination. 

\subsection{Optimization criteria}

The definition of our optimization criteria are directly derived from the fundamental matrix. Indeed, if  is invertible,  gives a measure of the performance of one solution  when the number of time epochs tends to infinity. The very interesting feature is that it also accounts for all the possible cycles in the graph and consequently models the broadcast property of the wireless channel.  
The criteria here after are defined for one source-destination flow. They will be extended in future work to multiple flows which is possible since the complete framework can model concurrent flows on the network.  

\paragraph{Redundancy, capacity and robustness} It is possible to derive for one flow the number of packets received at destination for one transmitted packet  () with 
  
\noindent where  is a vector of  zeros followed by  ones that accumulate all packets being received at the destinations into . 
This model shows that a solution  provides  packets at the destination for one sent packet. However, since relays are not able to discriminate packets, the set of  packets may be composed in the worst case of  copies of a packet by the source. What we can say here is that:

 if , there may be at most  different packets. Consequently,  provides both an upper bound on the network capacity and redundancy. Let  be the bound on the real capacity of the network. It is given by
  This bound can be reached if all the packets received at the source are orthogonal. This may be possible if an optimal network coding strategy is found where all transmitted packets are encoded combinations of the original information. 

 if ,  can be interpreted as a robustness criterion giving the probability to receive one message at destination.      

This criterion can be used to remove solutions that do not guaranty robustness. For instance, solutions having  can be disregarded if robustness wants to be guaranteed.

\paragraph{Delay criterion}
Let  be the probability for a transmission to be done in  hops. Thus 

We assume that a relay introduces a delay of 1 unit. Consequently, a -hop transmission introduces a delay of  units.  
It is given by the infinite sum: 

Let  be equal to . We can show that  and consequently
 
{\sc Proof}
We can substitute  by  in the definition of . Then we get by factorizing :

We now have . Consequently .

\hspace{4.2in}{\it q.e.d.}

\paragraph{Energy criterion}

The energy is measured by the number of packets being transmitted in the network. It can be simply derived by:


\noindent where  is a -by- vector of ones that sums all the packets sent by the relays that have participated in the transmission.

\subsection{Example: 1-relay network}

In this example, we consider a basic 1-relay network as the one depicted in Fig.~\ref{fig:1relay}.
\begin{figure}
\begin{center}
  \scalebox{0.5}{
  \begin{picture}(0,0)\includegraphics{1Relay.eps}\end{picture}\setlength{\unitlength}{3947sp}\begingroup\makeatletter\ifx\SetFigFont\undefined \gdef\SetFigFont#1#2#3#4#5{\reset@font\fontsize{#1}{#2pt}\fontfamily{#3}\fontseries{#4}\fontshape{#5}\selectfont}\fi\endgroup \begin{picture}(6739,1876)(2941,-3995)
\put(6001,-3886){\makebox(0,0)[lb]{\smash{{\SetFigFont{20}{24.0}{\familydefault}{\mddefault}{\updefault}{\color[rgb]{0,0,0}}}}}}
\put(7726,-2836){\makebox(0,0)[lb]{\smash{{\SetFigFont{20}{24.0}{\familydefault}{\mddefault}{\updefault}{\color[rgb]{0,0,0}}}}}}
\put(3526,-3886){\makebox(0,0)[lb]{\smash{{\SetFigFont{20}{24.0}{\familydefault}{\mddefault}{\updefault}{\color[rgb]{0,0,0}}}}}}
\put(6076,-2911){\makebox(0,0)[lb]{\smash{{\SetFigFont{20}{24.0}{\familydefault}{\mddefault}{\updefault}{\color[rgb]{0,0,0}}}}}}
\put(6526,-2386){\makebox(0,0)[lb]{\smash{{\SetFigFont{20}{24.0}{\familydefault}{\mddefault}{\updefault}{\color[rgb]{0,0,0}}}}}}
\put(4426,-2836){\makebox(0,0)[lb]{\smash{{\SetFigFont{20}{24.0}{\familydefault}{\mddefault}{\updefault}{\color[rgb]{0,0,0}}}}}}
\end{picture} }
   \caption{1-Relay network}
\label{fig:1relay}
\end{center}
\end{figure}
Interference is completely mitigated in this network if we use  times slots and  transmits in time slot 1 and relay  in time slot 2. As a consequence we have  and . The forwarding probabilities matrix is:
  
where the values of the  are derived according to \eqref{eq:fwfprobaflowconservation}. 
The derivation of  provides:

Since we have only one relay, matrix  does not exist and we have . The criteria  equals:

\noindent where  is obtained for . The bound on capacity of this 1-relay network can be obtained for . 
Thus, we can derive the value of  that guaranties  and verifies: 

\noindent It shows that the optimal transmission rate of the relay with respect to capacity is the rate at which the transmissions on the link  compensate the losses on the link . 
However, this bound may not be achievable when the sum of the information coming into  and  is lower than 1. 
Typically, applying a cut-set between  and the set , leads to a maximal flow .
Further, this capacity bound can only be achieved if the relay knows which packet have not been received by  directly from  or, more interestingly, by performing a network coding to introduce proper diversity in the flow.    



\section{Conclusion and discussion}\label{sec:conclusion}
The work presented in this paper has derived a flexible and powerful framework for evaluating the performance of a wireless ad hoc network with respect to several performance criteria. It has been designed to account for the broadcast nature of wireless communications and for an accurate interference temperature characterization for the network. The proposed framework allows for the determination of Pareto-optimal solutions with respect to capacity, redundancy, delay and energy optimization criteria. 
The solutions defined herein represent a steady state of the network where all active relays are constantly transmitting data. 
  
The complexity of the proposed framework is not a function of the number or the structure of the source-destination flows and it is scalable with respect to the number of flows in the network. As a consequence, it can model various types of transmission: one-to-one, one-to-many, many-to-one or many-to-many.

This flexibility comes at the price of an extended problem search space. However, the structure of this search space is very interesting from the optimization point of view since there is a dependency between the set of feasible transmission rate matrices  and the set of feasible forwarding probabilities matrices . If it is possible to disregard a feasible value of  because it is not promising enough for the MO search, then the worst case complexity of the problem can be reduced from  to  .

The definition of the fundamental matrix permits to derive an upper bound on the capacity of the network. This upper bound is provided in terms of the average number of packets being received at the destination. With our framework, we can not differentiate packets at the destination and hence can not state whether a received packets is a copy of an already received packet of not. However, we state that using proper network coding techniques, it is possible to get an network effective capacity close enough from this bound.  

This paper has focused on presenting theoretically our MO performance evaluation framework. It is only illustrated by a basic example. In future work, we will have to derive proper MO optimization search algorithms \cite{siarry} to efficiently compute the set of Pareto-optimal solutions. This task is challenging but as mentioned in the paper, it is often enough to address the -relaying subproblem (with  the maximum number of active relays in the network). Next, it will be of major interest to analyze the Pareto-optimal bounds and its corresponding solutions. The knowledge of such bounds will provide a powerful tool to assess the quality of distributed routing and resource allocation algorithms. Further, and this is of great interest in our opinion, we can take advantage of the steady state Pareto-optimal solutions to design distributed protocols that get close to our MO performance bound.       


\section*{Acknowledgments}
This work was supported in part by the Marie Curie IOF Action of the European Community's Sixth Framework Program (DistMO4WNet project). This article only reflects the authors' views and the European Community are liable for any use that may be made of the information contained herein.


\bibliographystyle{IEEEtran}
\bibliography{IEEEabrv,basebiblio}







\tableofcontents

\end{document}
