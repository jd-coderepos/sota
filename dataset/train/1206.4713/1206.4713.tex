\documentclass[12pt]{article}\usepackage{amsmath}
\usepackage{mitpress}
\usepackage{amsfonts}
\usepackage{amssymb}
\usepackage{graphicx}\setcounter{MaxMatrixCols}{30}
\providecommand{\U}[1]{\protect\rule{.1in}{.1in}}
\newtheorem{theorem}{Theorem}
\newtheorem{acknowledgement}[theorem]{Acknowledgement}
\newtheorem{algorithm}[theorem]{Algorithm}
\newtheorem{axiom}[theorem]{Axiom}
\newtheorem{case}[theorem]{Case}
\newtheorem{claim}[theorem]{Claim}
\newtheorem{conclusion}[theorem]{Conclusion}
\newtheorem{condition}[theorem]{Condition}
\newtheorem{conjecture}[theorem]{Conjecture}
\newtheorem{corollary}[theorem]{Corollary}
\newtheorem{criterion}[theorem]{Criterion}
\newtheorem{definition}[theorem]{Definition}
\newtheorem{example}[theorem]{Example}
\newtheorem{exercise}[theorem]{Exercise}
\newtheorem{lemma}[theorem]{Lemma}
\newtheorem{notation}[theorem]{Notation}
\newtheorem{problem}[theorem]{Problem}
\newtheorem{proposition}[theorem]{Proposition}
\newtheorem{remark}[theorem]{Remark}
\newtheorem{solution}[theorem]{Solution}
\newtheorem{summary}[theorem]{Summary}
\newenvironment{proof}[1][Proof]{\textbf{#1.} }{\ \rule{0.5em}{0.5em}}
\newdimen\dummy
\dummy=\oddsidemargin
\addtolength{\dummy}{72pt}
\marginparwidth=.5\dummy
\marginparsep=.1\dummy
\begin{document}

\title{Universal Regular Autonomous Asynchronous Systems: Fixed Points, Equivalencies
and Dynamic Bifurcations}
\author{Serban E. Vlad\\str. Zimbrului, nr. 3, bl. PB68, ap. 11, 410430, Oradea, Romania, E-mail: serban\_e\_vlad@yahoo.com}
\maketitle

\begin{abstract}
The asynchronous systems are the non-deterministic models of the asynchronous
circuits from the digital electrical engineering. In the autonomous version,
such a system is a set of functions 
called states ( is the time set). If an asynchronous system is
defined by making use of a so called generator function  then it is called regular. The property of
universality means the greatest in the sense of the inclusion.

The purpose of the paper is that of defining and of characterizing the fixed
points, the equivalencies and the dynamical bifurcations of the universal
regular autonomous asynchronous systems. We use analogies with the dynamical
systems theory.

\end{abstract}

\section{Preliminaries}

\begin{definition}
We denote by  the \textbf{binary Boole algebra}, endowed
with the discrete topology and with the usual laws.
\end{definition}

\begin{definition}
Let be the Boolean function  and  We define 
by 

\end{definition}

\begin{remark}
 represents the function resulting from  when this one is
not computed, in general, on all the coordinates 
if  then  is not computed, 
and if  then  is computed, 
\end{remark}

\begin{definition}
Let be the sequence  The functions  are defined iteratively by 

\end{definition}

\begin{definition}
The sequence  is
called \textbf{progressive} if
The set of the progressive sequences is denoted by 
\end{definition}

\begin{remark}
Let be  When  is progressive, each coordinate  is computed infinitely many times in the sequence  . This is the meaning of
the progress property, giving the so called 'unbounded delay model' of
computation of the Boolean functions.
\end{remark}

\begin{definition}
The \textbf{initial value}, denoted by  or  and the \textbf{final
value}, denoted by  or  of the function  are defined by

\end{definition}

\begin{definition}
The function  is called
(\textbf{pseudo})\textbf{periodical with the period}  if

a)  does not exist and

b) 
\end{definition}

\begin{definition}
The \textbf{characteristic function}  of the set  is defined in the following way:

\end{definition}

\begin{notation}
We denote by  the set of the real sequences 
which are unbounded from above.
\end{notation}

\begin{remark}
The sequences  act as time sets. At this level of generality
of the exposure, a double uncertainty exists in the real time iterative
computations of the function 
we do not know precisely neither the coordinates  of  that are
computed, nor when the computation happens. This uncertainty implies the non
determinism of the model and its origin consists in structural fluctuations in
the fabrication process, the variations in ambiental temperature and the power
supply etc.
\end{remark}

\begin{definition}
A \textbf{signal} (or \textbf{signal}) is a function  of the form
with  The set of the signals is denoted by 
\end{definition}

\begin{remark}
The signals  model the electrical signals from the digital
electrical engineering. They have by definition initial values and they avoid
'Dirichlet type' properties (called Zeno properties by the engineers) such as
because these properties cannot characterize the inertial devices.
\end{remark}

\begin{notation}
We denote by  the set of the non-empty subsets of a set.
\end{notation}

\begin{definition}
The \textbf{autonomous asynchronous systems} are the non-empty sets 
\end{definition}

\begin{example}
\label{Exa13}We give the following simple example that shows how the
autonomous asynchronous systems model the asynchronous circuits. In Figure
\ref{preliminaries1} we have drawn the (logical) gate NOT with the\begin{figure}
[ptb]
\begin{center}
\fbox{\includegraphics[
height=0.7308in,
width=1.4451in
]{preliminaries1.eps}}\caption{Circuit with the logical gate NOT}\label{preliminaries1}\end{center}
\end{figure}
input  and the state (the output)  For
 and
the state  represents the computation of the negation of  and it is of
the form
where  is the initial value of  and  is
arbitrary. As we can see,  depends on  only and it is
independent on 

In Figure \ref{preliminaries2},\begin{figure}
[ptb]
\begin{center}
\fbox{\includegraphics[
height=0.9548in,
width=1.5031in
]{preliminaries2.eps}}\caption{Circuit with feedback with the logical gate NOT}\label{preliminaries2}\end{center}
\end{figure}
we have
thus this circuit is modeled by the autonomous asynchronous system

\end{example}

\begin{definition}
The \textbf{progressive functions} 
are by definition the functions
where  and  The set of the progressive functions is denoted by 
\end{definition}

\begin{definition}
\label{Def20}For  and  like at (\ref{pre2}), we define  by 

\end{definition}

\begin{remark}
The previous equation reminds the iterations of a discrete time real dynamical
system. The time is not exactly discrete in it, but some sort of intermediate
situation occurs between the discrete and the real time; on the other hand the
iterations of  do not happen on all the coordinates (synchronicity), but
on some coordinates only, such that any coordinate  is computed
infinitely many times,  (asynchronicity) when .
\end{remark}

\section{Discrete time}

\begin{notation}
We denote by
the discrete time set.
\end{notation}

\begin{definition}
Let be  and  We define the function
 by 

\end{definition}

\begin{notation}
Let us denote

\end{notation}

\begin{definition}
The equivalence of  is defined by:  such that (\ref{pre2}) and
are true.
\end{definition}

\begin{definition}
The 'canonical surjection'  is by
definition the function 
where  is the only sequence such that  exists, making the equation (\ref{pre2}) true.
\end{definition}

\begin{remark}
The relation between the continuous and the discrete time is the following:
for any  and any   and  exist making the equation (\ref{pre2}) true
and we have
Equivalent progressive functions  (i.e.
) give 'equivalent' functions  in the sense that the computations of
 are the same, but the time flow is piecewise faster or slower in the
two situations.
\end{remark}

\section{Regular autonomous asynchronous systems}

\begin{definition}
The \textbf{universal regular autonomous asynchronous system}  that is generated by the function  is defined by


\end{definition}

\begin{definition}
An autonomous asynchronous system  is called
\textbf{regular}, if  exists such that  In this
case  is called the \textbf{generator function\footnote{The terminology
of 'generator function' is also used in \cite{bib4}, page 18 meaning the
vector field of a discrete time dynamical system. In \cite{bib6} the
terminology of 'generator' (function) of a dynamical system is mentioned too.
Moisil called  'network function' in a non-autonomous, discrete time
context; for Moisil, 'network' means 'system' or 'circuit'.}} of .
\end{definition}

\begin{remark}
In the last two definitions, the attribute 'regular' refers to the existence
of a generator function  and the attribute 'universal' means maximal
relative to the inclusion.

For a regular system,  is not unique in general.
\end{remark}

\begin{example}
For any  and  the autonomous
systems    and  are regular.

For  the system  is regular.

Another example of universal regular autonomous asynchronous system is given
by  the constant function, for which .
\end{example}

\begin{remark}
These examples suggest several possibilities of defining the systems
 which are not universal. For example by putting
appropriate supplementary requests on the functions  one could
rediscover the 'bounded delay model' of computation of the Boolean functions.
\end{remark}

\section{Orbits and state portraits}

\begin{definition}
Let be  Two things are understood by \textbf{orbit}, or
(\textbf{state}, or \textbf{phase}) \textbf{trajectory} (\cite{bib4}, page 19;
\cite{bib3}, page 3; \cite{bib1}, page 8; \cite{bib2}, page 24; \cite{bib5},
page 2) \textbf{of}  \textbf{starting at} :

a) the function 

b) the set  representing
the values of the previous function.

Sometimes (\cite{bib3}, page 4; \cite{bib6}, page 91; \cite{bib2}, page 24;
\cite{bib5}, page 2) the function from a) is called the \textbf{motion} (or
the \textbf{dynamic}) of  through 
\end{definition}

\begin{definition}
The equivalent properties

and

are called of \textbf{accessibility}; the points  are said to be \textbf{accessible}.
\end{definition}

\begin{remark}
\label{Rem32}The orbits are the curves in , parametrized by
 and  On the other hand  
imply  and we see the truth of
the implication

\end{remark}

\begin{definition}
The \textbf{state} (or the \textbf{phase})\textbf{\ portrait} of 
is the set of its orbits (\cite{bib3}, page 4; \cite{bib6}, page 92;
\cite{bib1}, page 10; \cite{bib5}, page 2).
\end{definition}

\begin{example}
\label{Exa34}The function  is
defined by the following table
The state portrait of  is:
This set is drawn in Figure \ref{ph1},\begin{figure}
[ptb]
\begin{center}
\fbox{\includegraphics[
height=1.1052in,
width=1.4131in
]{phaseportrait.eps}}\caption{The state portrait of the system from Example \ref{Exa34}.}\label{ph1}\end{center}
\end{figure}
where the arrows show the increase of time. One might want to put arrows from
 to itself and from  to itself.
\end{example}

\section{Nullclins}

\begin{definition}
Let be  For any  the \textbf{nullclins} of  are the sets
If  then the coordinate  is said to be \textbf{not
excited}, or \textbf{not enabled}, or \textbf{stable} and if  then it is called \textbf{excited}, or
\textbf{enabled}, or \textbf{unstable}.
\end{definition}

\begin{remark}
Sometimes, instead of indicating  by a table like previously, we can
replace Figure \ref{ph1} by Figure \ref{ph2},\begin{figure}
[ptb]
\begin{center}
\fbox{\includegraphics[
height=1.0966in,
width=1.4096in
]{phaseportrait1.eps}}\caption{The state portrait of the system from Example \ref{Exa34}, version}\label{ph2}\end{center}
\end{figure}
where we have underlined the unstable coordinates. For example in Figure
\ref{ph2},  means that 
 means that  etc.

In fact Figure \ref{ph2} results uniquely from Figure \ref{ph1}, one could
know by looking at Figure \ref{ph1} which coordinates should be underlined and
which should be not.
\end{remark}

\section{Fixed points}

\begin{definition}
A point  that fulfills  is called a
\textbf{fixed point} (an \textbf{equilibrium point}, a \textbf{critical
point}, a \textbf{singular point}) (\cite{bib4}, page 43; \cite{bib3}, page 4;
\cite{bib6}, page 92; \cite{bib1}, page 9; \cite{bib2}, page 24; \cite{bib5},
page 2), shortly an \textbf{equilibrium} of  A point that is not fixed
is called \textbf{ordinary}.
\end{definition}

\begin{theorem}
\label{The38}The following statements are equivalent for 

\end{theorem}

\begin{proof}
(\ref{equ1})(\ref{equ2}) We take  in the
following way
with  For the sequence
from  we can prove by induction on  that
wherefrom


(\ref{equ2})(\ref{equ1}) From (\ref{equ2}) we have the
existence of  and  with the property that
(\ref{equ7}) is true, thus (\ref{equ6}) is true. We denote
and we have from (\ref{equ6}):








with the conclusion that
Some  exists with the property that
thus (\ref{equ1}) is true.

(\ref{equ1})(\ref{equ2_}) Let be
with  and  arbitrary.
It is proved by induction on  the validity of (\ref{equ6}) and this implies
the truth of (\ref{equ7}).

(\ref{equ2_})(\ref{equ1}) This is true because (\ref{equ2_})(\ref{equ2}) and (\ref{equ2})(\ref{equ1})
are true.

(\ref{equ2})(\ref{equ3}) and (\ref{equ2_})(\ref{equ3_}) are obvious.

(\ref{equ1})(\ref{equ3__})  and...and  and...and 
\end{proof}

\begin{definition}
If  then  the orbit  is called \textbf{rest position}.
\end{definition}

\section{Fixed points vs. final values of the orbits}

\begin{theorem}
\label{The8}(\cite{bib7}, Theorem 49) The following fixed point property is
true

\end{theorem}

\begin{proof}
Let  be
arbitrary and fixed. Some  exists such that 
and from Theorem \ref{The38}, (\ref{equ2})(\ref{equ1}) we
have 
\end{proof}

\begin{remark}
Theorem \ref{The8} shows that the final values of the states of a system are
fixed points of .
\end{remark}

\begin{theorem}
\label{The9}(\cite{bib7}, Theorem 50) We have 

\end{theorem}

\begin{proof}
For arbitrary  we suppose that and We have 

\end{proof}

\begin{remark}
As resulting from Theorem \ref{The9}, the accessible fixed points are final
values of the states of the systems.

The properties of the fixed points that are expressed by Theorems \ref{The38},
\ref{The8}, \ref{The9} give a better understanding of Example \ref{Exa34}.
\end{remark}

\section{Transitivity}

\begin{definition}
\label{Def108}The system  (or ) is \textbf{transitive}
(\cite{bib4}, page 22; \cite{bib3}, page 3), or \textbf{minimal} (\cite{bib4},
page 23) if one of the following non-equivalent properties holds true:

\end{definition}

\begin{remark}
The property of transitivity may be considered one of surjectivity or one of accessibility.

If  is transitive, then it has no fixed points.
\end{remark}

\begin{example}
The property (\ref{tt1}) of transitivity is exemplified in Figure
\ref{minimality1}
\begin{figure}
[ptb]
\begin{center}
\fbox{\includegraphics[
height=0.9202in,
width=1.3422in
]{minimality1.eps}}\caption{Transitivity}\label{minimality1}\end{center}
\end{figure}
and the property (\ref{tt2}) of transitivity is exemplified in Figure
\ref{minimality2}.
\begin{figure}
[ptb]
\begin{center}
\fbox{\includegraphics[
height=0.9115in,
width=1.3327in
]{minimality2.eps}}\caption{Transitivity}\label{minimality2}\end{center}
\end{figure}

\end{example}

\section{The equivalence of the dynamical systems}

\begin{notation}
Let  and  be some functions. We denote by  the function

\end{notation}

\begin{remark}
If  and  is expressed
by
then

\end{remark}

\begin{notation}
For  and  we denote by 
the sequence 
\end{notation}

\begin{notation}
Let be  arbitrary and we denote for 

\end{notation}

\begin{notation}
We denote by  the set of the functions  that fulfill

i)  is bijective;

ii) 

iii) 

\end{notation}

\begin{theorem}
\label{The59}a)  is group relative to the composition  of the functions;

b) 

c) 
\end{theorem}

\begin{proof}
a) The fact that   and
 is obvious.

b) Let  and  be arbitrary. We denote for 
and we remark that


Case 


Case 


c) Let us take arbitrarily some  and a function 
where  and  We have
Because  taking into account b), we conclude
that 
\end{proof}

\begin{theorem}
\label{The60}Let be the generator functions  of the systems  and the
bijections  The following statements are equivalent:

a)  the diagrams
are commutative;

b) 


c) 

\end{theorem}

\begin{proof}
a)b) It is sufficient to prove that 
since this is equivalent with b).

We fix arbitrarily some  and some  and we use the induction on
. For  the statement is proved, thus we suppose that it is true for
 and we prove it for :


b)c) For arbitrary  and 
 we have that
is an element of  (see Theorem \ref{The59} c)) and


c)a) Let  be arbitrary and fixed
and we consider 
with  fixed too. We have
But
and from (\ref{eds1}), for  we obtain

\end{proof}

\begin{definition}
\label{Def160}We consider the generator functions  and the universal asynchronous systems
 . If two bijections  exist such that one of the equivalent
properties a), b), c) from Theorem \ref{The60} is satisfied, then  are called \textbf{equivalent} (\cite{bib4}, page 35;
\cite{bib6}, page 102; \cite{bib1}, page 40; \cite{bib2}, page 32;
\cite{bib5}, page 6) and  are called \textbf{conjugated}. In this
case we denote 
\end{definition}

\begin{definition}
\label{Def161}We fix  The fact that  exists such that the
previous property holds, makes us say that  is \textbf{structurally
stable} (Peixoto \cite{bib6}, page 121).  is called an
\textbf{admissible} (or \textbf{allowable})\textbf{\ perturbation of} .
\end{definition}

\begin{remark}
The equivalence of the universal regular autonomous asynchronous systems is
indeed an equivalence and it should be understood as a change of coordinates.
Thus  and  are indistinguishable.
\end{remark}

\begin{example}
 are given by, see Figure
\ref{echiv1}\begin{figure}
[ptb]
\begin{center}
\fbox{\includegraphics[
height=1.4313in,
width=3.1981in
]{echiv1.eps}}\caption{Equivalent systems}\label{echiv1}\end{center}
\end{figure}

and the bijection  is
The diagram
commutes for  and for  we have
the assignments
We denote 
For  we have
and for  the assignments are
respectively.  and  are conjugated.
\end{example}

\begin{example}
The functions  are given
in the following table
and the state portraits of the two systems are given in Figure \ref{echiv2}.
 and  are equivalent.\begin{figure}
[ptb]
\begin{center}
\fbox{\includegraphics[
height=1.3612in,
width=3.1981in
]{echiv2.eps}}\caption{Equivalent systems}\label{echiv2}\end{center}
\end{figure}

\end{example}

\begin{theorem}
If  and  are conjugated, then the following possibilities exist:

a) 

b)  and 
\end{theorem}

\begin{proof}
We presume that  In the
equation
we put  and we have
thus  and finally

\end{proof}

\begin{theorem}
We suppose that  and  are equivalent and let be
 such that .

a) If  is a fixed point of  then  is a fixed point of


b) For any  and any  if  is periodical with the period , then  is periodical with the period .

c) If  is transitive, then  is transitive.
\end{theorem}

\begin{proof}
a) The commutativity of the diagram
for  gives


b) The hypothesis states that 
and in this situation


c) Let  be arbitrary and fixed. The
hypothesis (\ref{tt1}) states that
wherefrom
The situation with (\ref{tt2}) is similar.
\end{proof}

\section{Dynamic bifurcations}

\begin{remark}
Let be the generator function   that depends on
the parameter . Intuitively speaking (Ott,
\cite{bib3}, page 137) a dynamic bifurcation is a qualitative change in the
dynamic of the system  that occurs at the variation
of the parameter .
\end{remark}

\begin{definition}
If for any parameters  the systems
 and  are
equivalent, then  is called \textbf{structurally stable} (\cite{bib6},
page 117; \cite{bib2}, page 43; \cite{bib5}, page 9); the existence of
 such that  and
 are not equivalent is called a
\textbf{dynamic} \textbf{bifurcation} (\cite{bib1}, page 57; \cite{bib5}, page 9).

Equivalently, let us fix an arbitrary  If
,  is
an admissible perturbation of  (Definition \ref{Def161}),
then  is said to be \textbf{structurally stable}, otherwise we say that
 has a \textbf{dynamic bifurcation}.
\end{definition}

\begin{remark}
If 
the bijections  exist such that  the diagram
commutes, then  is structurally stable, otherwise we have a dynamic bifurcation.
\end{remark}

\begin{example}
In Figure \ref{bifurcation3} (),
\begin{figure}
[ptb]
\begin{center}
\fbox{\includegraphics[
height=1.1537in,
width=3.0588in
]{bifurcation3.eps}}\caption{Structural stability}\label{bifurcation3}\end{center}
\end{figure}
 is structurally stable and the bijections  are defined
accordingly to the following table:

\end{example}

\begin{example}
In Figure \ref{bifurcation4} (),\begin{figure}
[ptb]
\begin{center}
\fbox{\includegraphics[
height=1.1078in,
width=2.9888in
]{bifurcation4.eps}}\caption{Dynamic bifurcation}\label{bifurcation4}\end{center}
\end{figure}
 has a dynamic bifurcation.
\end{example}

\begin{definition}
The \textbf{bifurcation diagram} (\cite{bib1}, page 61) is a partition of the
set of systems  in
classes of equivalence given by the equivalence of the systems, together with
representative state portraits for each class of equivalence.
\end{definition}

\begin{example}
Figure \ref{bifurcation4} is a bifurcation diagram.
\end{example}

\begin{definition}
The \textbf{bifurcation diagram} (\cite{bib3}, page 5) is the graph that gives
the position of the fixed points depending on a parameter, such that a
bifurcation exists.
\end{definition}

\begin{remark}
Such a(n informal) definition works for calling Figure \ref{bifurcation4} a
bifurcation diagram, since there fixed points exist. However for Figure
\ref{bifurcation5}
\begin{figure}
[ptb]
\begin{center}
\fbox{\includegraphics[
height=1.1087in,
width=2.9862in
]{bifurcation5.eps}}\caption{Dynamic bifurcation}\label{bifurcation5}\end{center}
\end{figure}
this definition does not work, because a bifurcation exists there, but no
fixed points.
\end{remark}

\begin{definition}
Let be . The families of systems  and 
are called \textbf{equivalent} (\cite{bib5}, pages 7, 17) if there exists a
bijection  such that
 and  are equivalent in the sense of
Definition \ref{Def160}.
\end{definition}

\begin{thebibliography}{9}                                                                                                

\bibitem {bib4}Constanta-Dana Constantinescu, Chaos, fractals and
applications, the Flower Power publishing house, Pitesti, 2003 (in Romanian).

\bibitem {bib3}Marius-Florin Danca, Logistic map: dynamics, bifurcation and
chaos, the publishing house of the Pitesti University, Pitesti, 2001 (in Romanian).

\bibitem {bib6}Adelina Georgescu, Mihnea Moroianu, Iuliana Oprea, Bifurcation
theory, principles and applications, the publishing house of the Pitesti
University, Pitesti, 1999 (in Romanian).

\bibitem {bib1}Yuri A. Kuznetsov, Elements of Applied Bifurcation Theory,
Second Edition, Springer, 1997.

\bibitem {bib2}Mihaela Sterpu, Dynamic and bifurcation for two generalized van
der Pol models, the publishing house of the Pitesti University, Pitesti, 2001
(in Romanian).

\bibitem {bib5}Mariana P. Trifan, Dynamic and bifurcation in the mathematical
study of the cancer, the Pamantul publishing house, Pitesti, 2006 (in Romanian).

\bibitem {bib7}Serban E. Vlad, Boolean dynamical systems, Romai Journal, Vol.
3, Nr. 2, 2007.
\end{thebibliography}


\end{document}