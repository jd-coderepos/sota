Abbreviations of Behaviour Change Theories (BCT):

\begin{itemize}
	\item SDT - Self Determination Theory \cite{Deci:1985km}
	\item SCT - Social Cognitive Theory \cite{bandura1977self}
	\item TBP - Theory of Planned Behaviour \cite{ajzen1991theory}
	\item TTM - Transtheoretical model \cite{Prochaska:1982jn}
	\item FBC - Fogg's Behaviour Change Model \cite{fogg2009behavior}
	\item FP - Fogg Persuasive \cite{fogg2002persuasive}
	\item SET - Self Efficacy Theory \cite{bandura1977self}
\end{itemize}

\begin{landscape}

\setlength{\extrarowheight}{0pt}
	\tiny
\begin{longtable}{
	p{0.1\linewidth-2\tabcolsep-1.5\fboxrule} p{0.17\linewidth-2\tabcolsep-1.5\fboxrule} p{0.17\linewidth -2\tabcolsep -1.5\fboxrule} p{0.16\linewidth-2\tabcolsep-1.5\fboxrule} p{0.07\linewidth -2\tabcolsep-1.5\fboxrule} p{0.09\linewidth-2\tabcolsep-1.5\fboxrule} p{0.17\linewidth-2\tabcolsep-1.5\fboxrule} p{0.05\linewidth-2\tabcolsep-1.5\fboxrule}} \hline
Paper & Summary & Sensors and Feedback & Feedback & N & Duration & Results & BCT\\\hline\hline
\endhead
\multicolumn{7}{c}{Motivation through Gamification} 
\\\hline
\cite{bleecker2007mobzombies}
	&Game utilising movements from wearable in a game where avatar runs from zombies and player gets points
	&Accelerometer, GPS for movement data and position
	&game points
	&-
	&-
	&-
	&

\\\hline
\cite{Payton:2011fp}, \cite{Doran:2010va}
	&World of Workout exergame to motivate physical activity with quests that the user must complete by beating step count goals set up either by them or the app. 
	&iPhone shake sensor
	&new quests on a virtual quest map
	&10 students
	&2 quests
	&users liked the game and found it fun
	&

\\\hline
\cite{Stanley:2014gq}
	&Pervasive Accumulated Context Exergame (PACE) which passively collects activity during the day, and rewards the participant in a later sedentary computer game 
	&phone accelerometer, location (if on campus by looking at wifi SSID), bluetooth proximity
	&received game advantages in computer game, and alerts during the day 
	&24 players (two rounds of 12 players each)
	&9 days
	&

	&

\\\hline
\cite{clawson2010dancing}
	&Mobile and wearable health game which uses 2 wireless accelerometers worn around the users ankles as input into a social dancing game played by groups of users at the same time. 49 people were satisfied with the game but there was a lot of users finding the sensors difficult to use or that they weren’t agreeing with what they were doing.
	&Accelerometer for each leg
	&dancing game with encouraging messages
	&50
	&2 songs
	&found system fun and challenging, users were satisfied with experience, 
	&

\\\hline
\cite{ali2006fitster}
	&Game with an online dashboard showing physical activity of the group members and offer capability to challenge other members of the group. 
	&phone accelerometer for step data
	&online leaderboard
	&-
	&-
	&-
	&

\\\hline
\cite{ahtinen2010let}
	&Designed and modelled a mobile phone game that looked at using social and play aspects to encourage physical activity called 'into'. The game works on the analogy of the user going on 'virtual trips' using their distance traveled to win rewards. People can combine and work together. 
	&phone accelerometer for step data
	&challenges within the group, presentation of the walked distance on a map
	&37 in groups of 2-6
	&1 week
	&users liked the app, found it understandable, and appreciated the similarity of game world
	&

\\\hline
\cite{chen2014healthytogether}
	&Developed a game to study and observe how users interact in different group gamification settings - competition, cooperation, or hybrid. App included a messaging service to allow a pair of people to talk to each other and help or taunt each other.
	&Fitbit pedometer
	&mobile dashboard
	&36 in pairs  (collaboration, competition, hybrid groups)
	&2 day warm-up, 1 week control 1 week experimental session
	&significant increase of activity in all groups, cooperation and hybrid effective in motivation more activity, competition lead to a negative correlation between increase of step count between pair members
	&

\\\hline
\cite{chuah2012wifitreasurehunt}
	&A location based alternate reality game that encourages users to stay physically active. The app includes ways for the user to go on tours to find hidden rewards on a real world map and also on group tours and share to social networks.
	&phone WiFi module to log into WiFi access points
	&mobile phone all with a map, access points and progress
	&-
	&-
	&-
	&

\\\hline
\cite{macvean2012ifitquest}
	&iFitQuest is a mobile location-aware, alternate reality exergame using google maps. Made up of mini games like ‘collect the coins’ and ‘escape the ghost’ where the user must physically move to avoid or collect things shown on screen.
	&mobile phone GPS and compass for locationing and navigation
	&Escape the Ghost: Map showing a virtual ghost avatar the players must escape by moving in the physical world; Collect the Coins: The users must collect coins shown on the virtual map while avoiding the ghosts.
	&25
	&30 minutes
	&users found both games enjoyable, boys enjoyed the game significantly more than girls
	&

\\\hline		 
\cite{gorgu2012freegaming}
	&Freegaming is a location-based augmented exergame. The player is navigated on a virtual map by following augmented reality clues. The status off each playerrs progress can be seen on a map in the mobile phone app. 
	&GPS for locationing; camea for augmented reality clues
	&augmented reality clues, a game map with the status of each player
	&-
	&-
	&-
	&

\\\hline
\cite{Zuckerman:2014fl}
	&StepByStep mobile application to promote regular walking. Quantified version with continuous measurement, daily goals and feedback on daily progression for self-motivated reflection. Gamified version with virtual rewards and social comparison.
	&accelerometer in phone for step count
	&1st study mobile app showed active minutes and progress towards goal
	&40
	&2 weeks
	&significant increase of walking minutes; app raised the awareness for own activity
	&

\\\hline
\cite{Zuckerman:2014fl}
	&StepByStep mobile application to promote regular walking. Quantified version with continuous measurement, daily goals and feedback on daily progression for self-motivated reflection. Gamified version with virtual rewards and social comparison.
	&accelerometer in phone for step count
	&2nd study mobile app showed active minutes, progress towards goal and gamification elements of points and a leaderboard (in the leaderboard version)
	&59
	&10 days
	&significant correlation between walking goal and active minutes for QS and points version, but not for leaderboards
	&

\\\hline


\multicolumn{7}{c}{Social Influence } \\\hline


\cite{buttussi2008mopet}
	&Generated a mobile personal trainer (MOPET), the MOPET takes in real time data from sensors and knowledge from professional trainers to provide motivation and health and safety advice. · To interact with the user there is a 3d embodied agent that can talk
	&GPS for locationing, heart rate monitor with a 3D accelerometer
	&exercise recommendations from provessional
	&-
	&-
	&-
	&

\\\hline
\cite{lin2006fish}
	&Fish’n’Steps is a social computer game which links the players daily activity count to the growth of animated fish characters. 
	&Accelerometer for step count data
	&animated fish character in a bowl in the phone app
	&19
	&

	&Fish‘n’Steps study indicates that participants either rose in the levels of the transtheoretical model or increased the number of daily steps
	&

\\\hline
\cite{anderson2007shakra}
	&Looks at using a mobile phone as a health promotion tool. The app tracks the daily exercise levels of users by analysing their movement. This data is then shared amongst the users group of friends. A short study found that this sharing of data encouraged the user to reflect upon the data more.
	&GSM cell signal strength to detect cells and movement
	&app shows current progress and peers progress towards goal
	&9
	&10 days
	&application was perceived well by participants, no study on effect of behaviour change
	&

\\\hline
\cite{Cercos:2013fk}
	&fight sedentary through social play and collective awareness, team data shown on semi-public display, they utilise power of social relationships to change behaviour, promote shared reflective view of players, a fictional player (10k guy) to promote shared goal of 10.000 steps, utilises social comparison and showed discrepancy between goal and actual performance
	&Fitbit: accelerometer and altimeter
	&semi-public display with visualisation of step data of all participants, 2D line graph is shown
	&15
	&8 weeks
	&observational results of preliminary study: when display was hanging, more and more people were interested in participating, also new conversation spaces through discussions
	&SDT, SCT, TTM
\\\hline
\cite{lim2011pediluma}
	&Pediluma is a wearable device that is worn on the user's foot, the more steps a user takes, the more the device lights up. A study was conducted and found that the use of the device did encurage an increase in physical activity but there are issues around discreetness of the device.
	&shoe accessory with pedometer 
	&LED in accessory lights up when wearer moves, dims when wearer is stationary
	&18
	&2 weeks (1 week detecting baseline, 1 week with wearable feedback)
	&the device was able to increase the step count and physical activity, but people felt not too comfortable around strangers seeing the device
	&FP
\\\hline
\cite{Foster:2010bd}
	&StepMatron is a Facebook application to provide social and competitive environment to increase physical activity at the workplace 
	&Pedometer
	&non-social version: participant can see own step data in online dashboard, social: participant can additionally see step data from group members and can comment on them.
	&10 nurses
	&3w
	&significant increased step count in social version compared to non-social 
	&

\\\hline
\cite{toscos2006chick}
	&Chick Clique is a Mobile phone app that allows girls to talk track their step count as a group and talk about their progress. Social element worked and encouraged girls to talk to each other about health, something they would not normally do.
	&accelerometer as pedometer
	&app with group activity overview
	&2 groups of friends (one 4 (15-17), the other 3 (13 years))
	&

	&comparison just tracker - tracker and app for group awareness: group one performed better with app, group two without (no significance)
	&

\\\hline
\cite{mauriello2014social}
	&Created a set of wearable devices to support group fitness, which displayed important run data to every group member on e int displays located on the back of the runners t-shirts. 
	&accelerometer (determine pace and distance) and heart rate
	&e-ink screen on back of the shirt show progress to other runners
	&52
	&1 run
	&display motivated to perform better; displays improve awareness of individual and group performance, helps groups stay together, and improves in-situ motivation
	&

\\\hline
\cite{lu2014reducing}
	&Designed an app called 'UOIFit' aimed at increasing physical activity levels in adolescents. The app incorporated social aspects into its design such as friending, sharing progress and collaboratively exercising with friends either in person or remotely, the fitfeed tab of the app displaying all this data. Studies were conducted with the app and found social aspects to have a positive impact on the amount of activity the user did and their BMI.
	&accelerometer in phone for tracking exercises
	&app shows fitness activity and offers social activity functions
	&35
	&6 weeks
	&all participants reduced their BMI to a healthier range, strong correlation between use of social activity features ad BMI reduction
	&

\\\hline
\cite{Lane:2014iu}
	&Created a set of wearable devices to support group fitness, which displayed important run data to every group member on e int displays located on the back of the runners t-shirts. 
	&phone’s accelerometer and microphone to detect sleep patterns and quality, physical activity, and social activity to calculate wellbeing score
	&baseline group: web-dashboard showing wellbeing score: multidimensional-group: ambient phone display with fish swimming in a bowl
	&27
	&19 days
	&liked the ambient display, positive behaviour change, but study short 
	&

\\\hline
\cite{burns2012activmon}
	&Developed a low-complexity low- engagement interface to motivate physical activity. Activimon is a wrist worn device that has a light display that shows the user when other members of their group are being active by flashing. In studies, users were divided in their opinion about the device, some liked it but others felt it didnt provide them with enough information.
	&Movement of users arm, step count
	&wristband shows light flashing when peers are active
	&5
	&2 weeks
	&usability study
	&

\\\hline

\multicolumn{7}{c}{Context-Aware Interfaces and Feedback} \\\hline



\cite{Oliver:2006df},
\cite{deOliveira:2008gm}
	&MPTrain app plays adaptive music to the runners speed, TrippleBeat app influences music based on exercise performance, advancement over MPTrain: TippleBeat considers optimal training zones, virtual competition with others, motivation through scores, glance-able interface showing 
	&ecg, accelerometer
	&MPTrain: adaptive changes in music, TrippleBeat: music changes, glancable phone screen with information
	&10 runners
	&-
	&compared MPTrain and TrippleBeat: increased time in trainings zone with Tripple Beat (57.1\% vs 82.8\%), all participants spend more time in optimal trainings zone, competition was valued by users, participants clearly preferred trippleBeat
	&

\\\hline
\cite{carroll2013food}
	&Looks at modifying the behaviour that people have with regards to emotional eating. Users used a mobile phone application to track emotions and to receive interventions - emotree. This helped them to find the emotions most felt when eating occurred. Then made a bra that could sense these emotions.
	&electrocardiogram sensor, electrodermal activity sensor, gyroscope, accelerometer, user's food and mood input to detect emotional eating patterns
	&

	&-
	&-
	&3 studies: 2 on interventions for snacking; 1 on feasibility of wearable emotion recognition
	&SCT
\\\hline
\cite{Kranz:2013ho}
	&Looked into the feasibility of using a smart as a personalised fitness trainer. The 'GymSkill' app involves exercise descriptions, sensor data logging, activity recognition and on-top skill assessment to present data as valuable as that of a personal trainer, tailored to the users ability. The gym skill app was specific to balance board training, as the user was on the balance board, they would simply place their Andriod smart phone in the middle of the board. Testing of the system showed its potential for ensuring long-term retention in this type of application. People particularly liked the personalised feedback and suggestions more than other features.
	&mobile phone accelerometer and gyroscope 
	&Personalised feedback on exercises and suggestions on the phone
	&6
	&5 days
	&people especially liked the personal feedback and exercise suggestions.
	&SCT
\\\hline \cite{fortmann2014waterjewel}
	&Waterjewel is a device to encourage its wearer to drink more water daily. It has a light up display which indicates how much of their daily goal the user has achieved but also acts as a reminder every two hours to the user that they should drink more. LEDs in bracelet would light up red or green according to users drinking behaviour. Made using Arduino lily pad and linked to an android app called 'Carbodroid'. Studies showed that people drank more when wearing the bracelet than not. So the bracelet was successful in promoting good drinking behaviour.
	&manual input in phone app
	&progress is visualised as LEDs on the wristband, progress also shown in mobile app
	&6
	&4 weeks (2 weeks in each condition)
	&participants drank more and more regularly with the wearable compared to the mobile app alone
	&

\\\hline \cite{pels2014fatbelt}
	&The Fat belt is a wearable device that uses physical feedback through inflating around the stomach as a response to calorie overconsumption, simulating the long-term weight-gain associated with over-eating – isomorphism.
	&input of calories in mobile app
	&inflating belt mimicking weight gain in the stomach area
	&12
	&2 days
	&significant decrease in consumption over a baseline period of the same length. Seen as an extension of the users own body – gave the wearable more emotional power over the user.
	&

\\\hline
\cite{rajanna2014step}
	&A context aware health assistant system – A mobile application that encourages the user to adopt a healthy life style by performing simple and contextually suitable physical exercises. The mobile app promotes brief physical exercise after prolonged periods of inactivity by sending ‘nudges’ to the user.
	&accelerometer and GPS to determine activity
	&Nudges sent to user through their smart phone to remind them to be more active. Can be vibrations, visual or auditory.
	&-
	&-
	&-
	&FBC
\\\hline
\cite{Lin:2011cg}
	&Developed a context aware recommendation system called 'Motivate', which takes into consideration an individuals location and other features to offer personalised advice. The user downloads the motivate app onto their smartphone and sets up a profile for the app to base its advice around. The advice has constraints such as weather and time and models its response around these. the user then tells the app whether they intend take the advice.
	&GPS or GSM localisation; further information from weather services, 
	&App interface with recommendations for activities based on weather, location and personal preferences
	&6
	&5 weeks
	&Studys found the reception of this app to be mixed, with only 50\% of people replying yes to advice given to them within the app.
	&

\\\hline

\multicolumn{7}{c}{Support for Self-Monitoring and Reflection} \\\hline

\cite{Lin:2012jz}
	&mobile phone app BeWell+ to monitor sleep, physical activity and social activity to generate wellbeing score, which is presented to users. Compared two versions in their study: baseline with web dashboard and version with additional ambient display. Social feature allows to compare own wellbeing score to peers and identify role models
	&phone’s accelerometer and microphone
	&baseline group: web-dashboard: multidimensional-group: ambient phone display with fish swimming in a bowl
	&27
	&19 days
	&liked the ambient display, positive behaviour change, but study short 
	&

\\\hline
\cite{consolvo2008activity}
	&UbiFit is an application that uses on-body sensing and machine learning to infer people’s activities, using a personal, mobile display to encourage physical activity. The display uses the metaphor of a garden that blooms throughout the week as the user performs physical activities. 	
	&accelerometer and barometer to infer physical activity
	&glancable screen on phone with a blooming garden depending on activity levels
	&12
	&21-25 days
	&The technology worked reasonably well within the field study, recognising most activities correctly. Participants mentioned that the garden was motivating, often surprisingly so – worked as a constant representation of their data. For others it helped them focus on planning or simply finding time for physical activity.
	&TTM
\\\hline
\cite{Oliver:2006df}
	&Health gear consists of physiological sensors wirelessly connected to a mobile phone via Bluetooth. The data from these sensors are then manipulated and displayed to the user in a relevant way. There was an 100\% success rate in recognising cases of sleep apnea. Issues like security and privacy need to be addressed.
	&wearable blood oximeter for heart rate and blood oxygen levels
	&shows heart rate and oxygen on phone. 
	&20
	&1 night
	&shows heart rate and oxygen on phone. 
	&

\\\hline
\cite{How:2013et}
	&Android app to help people in rehabilitation (eg. after stroke) to improve their step symmetry. The app has three modes where the user can train, capture regular walks and compare his improvements in a history. Data can be shared with therapist via email.
	&Android phone accelerometer to detect step pattern
	&Dashboard in app with scores for symmetry of walk and stats like tome and rating of the walk
	&-
	&-
	&-
	&

\\\hline
\cite{Nerino:2013fd}
	&A wireless body sensor network for monitoring the exercises of rehabilitation patients of knee surgery. This supports unassisted rehabilitation of motor functions.
	&accelerometer, gyroscope to determine leg position and movement
	&GUI on a tablet for patients with feedback on correct positioning during exercises 
	&-
	&-
	&-
	&

\\\hline
\cite{zhang2013see}
	&Created a system that contains a wearable UV sensor and a pair of AR glasses. The glasses would change the colour of the users skin to make it look more red dependant on the amount of time the user spends out in the UV rays.
	&UV sensor
	&Augmented reality glasses showing a change in skin colour after too much sun exposure to simulate sunburn
	&6
	&-
	&Studies conducted found the visualisation to have a positive effect on users.
	&

\\\hline
\cite{Madan:2010fg}
	&They used mobile phone social sensing and self-assessments to identify correlations between mobile data and illness symptoms.
	&Bluetooth proximity to other phone users, WLAN for rough location, Call \& SMS records, daily symptoms self-assessment
	&no feedback
	&70 residents of a dormitory  
	&-
	&it is possible to determine the health status of individuals
using information gathered by mobile phones alone,
without having actual health measurements about the subject
	&

\\\hline
\cite{Spelmezan:2009jy}
	&Custom sensor and mobile phone application (phone for hosting and computation) for learning and training of physical activities. Used snowboarding beginners as example.
	&force sensitive resistors in shoes, Shake sensors to measure upper body rotation, BendShort sensors to measure knee flexion, 
	&vibro-tactile feedback through actuators placed at different places
	&8
	&1 snowboarding session
	&identified movements accurately, participants perceived very well tactile instructions (87as compared to corresponding audio instructions (97played back over earplugs while snowboarding.
	&

\\\hline 
\cite{Chi:2004gp}
	&Study Looked at putting sensors into a martial art competition (taekwondo) to see when a significant impact had been delivered to either competitor.
	&Force sensors, Amount of force on competitors body protector
	&Displays score to user on screen
	&4 national champions in Taekwondo
	&2 hours
	&participants and jury agreed with the scoring provided by the system; participants gave positive feedback
	&

\\\hline
\cite{sanches2010mind}
	&Different sensors brought together with a mobile app with different visualizations to help users track their stress and reflect.
	&Accelerometer, GSR, ECG integrated into clothing to detect movements and stress
	&mobile phone app, stress history view for reflection and manual pattern detection for the user
	&-
	&-
	&-
	&

\\\hline
\cite{Kocielnik:2013dt}
	&Framework for measuring stress in real-life conditions continuously and unobtrusively to help the users reflect their stress states and develop relief patterns
	&Philips DTI-2 sensor (GSR, skin temperature, ambient temperature, lightning, accelerometer), calendar entries, self-assessments
	&no feedback during the study, LifelogExporer showed graphs of data at the end of study
	&10
	&4 weeks
	&data presented to the user after the 4 weeks, LifelogExplorer to generate overview of all data 
semi-structured interviews on the meaningfulness, usefulness, and triggering of healthier behaviour; the participants were positive, that new unobtrusive sensors for long-term data measurement can help user to get feedback of their stress levels in real work environments
	&

\\\hline
\cite{Stahl:2009fm}
	&A digital diary for user-written notes as well as body sensor data and mobile phone media data to help people reflect their emotions. The tablet app showed a timeline with photos and information on the current affective state, the presence of others, and phone activity like calls and SMS. 
	&Bodymedia biosensor collecting skin conductance data, mobile phone photos, mobile phone usage data (SMS, calls), bluetooth proximity to detect presence of others
	&a tablet app allows the presentation to the collected data in a timeline. It shows photos and symbolised data on detected bluetooth device, phone calls and SMS, and figures with colour codes for current emotional state 
	&4
	&2 to 4 weeks
	&Their qualitative study showed that participants used the diary in very different ways to interpret and make sense of the data. They also concluded, that the measurements did not always represent the experienced feelings.  
	&

\\\hline
\cite{Reitberger:2014gs}
	&Nutriflect system with a ambient display in kitchen, shows information on healthiness of bought food
	&camera to scan EAN or NFC to identify foods
	&ambient display in form of tablet showing family process of health eating
	&21
	&2 weeks pre, 4 weeks study, 
	&participants liked it and rated it positively
	&

\\\hline

\multicolumn{7}{c}{Social Support} 
\\\hline


\cite{Paredes:2011be}
	&They measured stress levels of participants and offered different type of intervention to help participants relax: games, guided breathing with haptic feedback and social support from loved ones
	&Heart Rate sensor and GSR sensor to detect stress
	&3 intervention types: gaming; haptic feedback for guided breathing; emotional, social support
	&20
	&1 experimental setup
	&no significant difference between interventions
	&

\\\hline
\cite{polzien2007efficacy}
	&Studied whether using technology with personal counselling or using purely technology will result in a bigger weight loss. Used the Sensewear pro armband was used to determine the energy expenditure. The group that used solely technology lost the most weight, followed by the people that used solely counselling.
	&Sensewear pro armband for determining energy expenditure, self-monitored calorie intake
	&Inteructions on calorie intake and exercise regime, the technology group also had access to their energy expenditure data from Sensewear
	&58
	&12 weeks
	&participants using the Sensewear lost ~2kg more weight than group without technology support (not significant)
	&SET
\\\hline

\end{longtable}

\end{landscape}

