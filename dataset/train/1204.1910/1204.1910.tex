

\documentclass{ifacconf}\usepackage{graphicx}
\usepackage{natbib}
\usepackage{amssymb}
\usepackage{amsmath}
\usepackage{amsfonts}
\usepackage{multirow}
\usepackage{epsfig}\setcounter{MaxMatrixCols}{30}
\begin{document}
\begin{frontmatter}
\title{Multi-intersection Traffic Light Control Using Infinitesimal Perturbation
Analysis\thanksref{footnoteinfo}}
\thanks[footnoteinfo]{The authors' work is supported in part by NSF under
Grant EFRI-0735974, by AFOSR under grant FA9550-09-1-0095, by DOE under grant
DE-FG52-06NA27490, and by ONR under grant N00014-09-1-1051.}
\author[First]{Yanfeng Geng}
\author[First]{Christos G. Cassandras}
\address[First]{Division of Systems Engineering and Center for Information and Systems Engineering, 	Boston University, Brookline, MA, 02446 (e-mail: gengyf@bu.edu, cgc@bu.edu)}
\begin{abstract}                We address the traffic light control problem for multiple intersections in tandem by
viewing it as a stochastic hybrid system and developing a Stochastic Flow
Model (SFM) for it. Using Infinitesimal Perturbation Analysis (IPA), we derive
on-line gradient estimates of a cost metric with respect to the controllable
green and red cycle lengths. The IPA estimators obtained require counting
traffic light switchings and estimating car flow rates only when specific
events occur. The estimators are used to iteratively adjust light cycle
lengths to improve performance and, in conjunction with a standard
gradient-based algorithm, to obtain optimal values which adapt to changing
traffic conditions. Simulation results are included to illustrate the approach.
\end{abstract}
\begin{keyword}
Traffic Light Control, SFM, IPA.
\end{keyword}
\end{frontmatter}



\section{Introduction}

The Traffic Light Control (TLC) problem aims at dynamically controlling the
flow of traffic at an intersection through the timing of green/red light
cycles with the objective of reducing congestion, hence also the delays
incurred by drivers. The more general problem involves a set of intersections
and traffic lights with the objective of reducing overall congestion over an
area covering multiple urban blocks. Since the control of one intersection
influences the traffic flow from or towards others, this is a complex problem
further complicated by the fact that traffic flows constantly change depending
on the time of day, accidents, weather conditions, etc. Recent technological
developments involving better, inexpensive sensors and wireless sensor
networks have enabled the collection of data (e.g., counting vehicles in a
specific road section) which can be used to optimally select traffic light
cycles over specific time intervals in a day or even to dynamically control
them based on real-time data. Thus, methodologies that would not be possible
to implement not long ago are now becoming feasible. The approach proposed in
this paper to the TLC problem is specifically intended to exploit these recent developments.

Several different approaches have been proposed to solve the TLC problem. It
is formulated as a Mixed Integer Linear Programming (MILP) problem in
\cite{Dujardin11}, and as an Extended Linear Complementary Problem (ELCP) in
\cite{Schutter99}. A Markov Decision Process (MDP) approach has been proposed
in \cite{Yu06} and Reinforcement Learning (RL) was used in \cite{Thorpe97},
with several extensions found in \cite{Prashanth11, Wiering04}. A game
theoretic viewpoint is given in \cite{Alvarez10}, while a hybrid system
formulation is presented in \cite{Zhao03}. Due to its complexity when viewed
as an optimization problem, fuzzy logic is often used in both a single
(isolated) junction \cite{Murat05} and multiple junctions \cite{Choi02}.
Expert systems \cite{Findler92} and evolutionary algorithms \cite{Taale98}
have also been applied to develop a traffic light controller for a single
intersection. Perturbation analysis techniques were used in \cite{Head96} and
a formal approach using Infinitesimal Perturbation Analysis (IPA) to solve the
TLC problem was presented in \cite{Panayiotou05} for a single intersection.

In \cite{GengCDC12}, we study the TLC problem for a single intersection using
a Stochastic Flow Model (SFM) and Infinitesimal Perturbation Analysis (IPA).
In this paper, we extend our analysis to two tandem intersections. We still
adopt a stochastic hybrid system modeling framework (see \cite{Cassandras08},\cite{Cassandras06}), since the problem involves both event-driven dynamics
in the switching of traffic lights and time-driven dynamics that capture the
flow of vehicles through an intersection. Although one can also view this as a
purely Discrete Event System (DES) with the intersection area as a
\textquotedblleft server\textquotedblright\ processing \textquotedblleft
users\textquotedblright\ (vehicles), the fact that a vehicle does not
exclusively occupy this area makes a flow-based viewpoint a more accurate way
to model such a process. While in most traditional flow models the flow rates
involved are treated as deterministic parameters, a SFM as introduced in
\cite{Cassandras02} treats them as stochastic processes. In the TLC problem,
this is consistent with continuously and randomly varying traffic flows,
especially in heavy traffic conditions where the problem is most interesting.
With only minor technical assumptions imposed on the properties of such
processes, a general IPA theory for stochastic hybrid systems was recently
presented in \cite{Wardi10},\cite{Cassandras10} through which one can estimate
on line gradients of certain performance measures with respect to various
controllable parameters. These estimates may be incorporated in standard
gradient-based algorithms to optimize system parameter settings. IPA was
originally developed as a technique for evaluating gradients of sample
performance functions in queueing systems and using them as unbiased gradient
estimates of performance metrics. However, IPA estimates become biased when
dealing with aspects of queueing systems such as multiple user classes,
blocking due to limited resource capacities, and various forms of feedback
control. The use of IPA in stochastic hybrid systems, however, circumvents
these limitations and yields simple unbiased gradient estimates (under mild
technical conditions) of useful metrics even in the presence of blocking and a
variety of feedback control mechanisms (see \cite{Yao11}.)

In Section 2, we formulate the TLC problem for two intersections and construct
a SFM. In Section 3, we derive an IPA estimator for a cost function gradient
with respect to a controllable parameter vector defined by green and red cycle
lengths. This is then used to iteratively adjust these cycle lengths to
improve performance and, under proper conditions, obtain optimal parameter
values. Simulation-based examples are given in Section 4 and we conclude with
Section 5.

\section{Problem Formulation}

In this paper, we concentrate on solving the TLC problem for two coupled
intersections, as shown in Fig. \ref{junction}. There are four roads and four
traffic lights, with each traffic light controlling the\ associated incoming
traffic flow. The traffic in road 1 of intersection  flows into road 3
of . For simplicity, we make the following assumptions:  Left-turn
and right-turn traffic flows are not considered, i.e., traffic lights only
control vehicles going straight.  A YELLOW light is combined with a RED
light (therefore, the YELLOW light duration is not explicitly controlled).
 Road 3 is long enough that cars accumulated in it do not influence the
departure process of road 1.

\begin{figure}[tbh]
\centering
\includegraphics[scale = 0.4]{TwoIntersections.jpg} \caption{Two tandem
traffic intersections}\label{junction}\end{figure}

The system involves a number of stochastic processes which are all defined on
a common probability space . Each of the four roads is
considered as a queue with a random \emph{arrival} flow process , where  is the instantaneous vehicle
arrival rate at time . When the traffic light corresponding to road  is
GREEN, the \emph{departure} flow process is denoted by  Let the GREEN light duration in a cycle of queue  be
, and the controllable parameter vector of interest is
. We define a state vector
 where  is the content of queue . We use the notation
 to emphasize the dependence of the queue content on ; however, for notational simplicity, we will write  when no
confusion arises. We also define a left-continuous \textquotedblleft
clock\textquotedblright\ state variable , , associated
with the GREEN light cycle for queue  as follows:

where  is the index of the road perpendicular to road  at the same
intersection. We set . Thus,  measures
the time since the last switch from RED to GREEN of the traffic light for
queue . It is reset to  as soon as the GREEN cycle length 
is reached and remains at this value while the light is GREEN for queue
; as soon as that cycle ends, i.e., , then  and the process repeats.

To simplify notation, we set  if the Boolean expression
used in (\ref{dzdt}), i.e.,  or , is true (light is GREEN) and 
otherwise. We can now write the dynamics of each state variable  as
follows:
where

In (\ref{departRate}),  describes the departure process if the road
is not empty. According to assumption ,  is independent
of . However,   depends on the departure process
 of queue 1; in particular,

Combining (\ref{dxdt}) through (\ref{alpha3}), we have the dynamics of queue
3:



The operation of the intersection can be viewed as a hybrid system with the
time-driven dynamics described by (\ref{dxdt})-(\ref{f3}) and event-driven
dynamics dictated by GREEN-RED light switches and by events causing some
 to switch from positive to zero or vice versa.

\begin{figure}[tbh]
\centering
\includegraphics[scale = 0.30]{Automaton.jpg} \caption{The Stochastic Hybrid
Automaton model}\label{Automaton}\end{figure}

Using the standard definition of a Stochastic Hybrid Automaton (SHA) (e.g.,
see \cite{Cassandras08}), we may obtain a SHA model for queue 1,2 and 4 which
is similar to \cite{GengCDC12}. Here, we concentrate on the SHA for the
operation of queue 3 as shown in Fig. \ref{Automaton}. This reflects the fact
that a typical sample path of any one of the queue contents (as shown in Fig.
\ref{SamplePath}) consists of intervals over which , which we call
Non-Empty Periods (NEPs), followed by intervals where , which we
call Empty Periods (EPs). Thus, the entire sample path consists of a series of
alternating NEPs and EPs. The event set that
affects any queue  is 
where  is a switch in the sign of  from
non-positive to strictly positive,  is a switch in the sign of
 from  to strictly positive,  is the queue content
becoming empty, i.e., , which terminates a NEP (and initiates an EP),
 switches a light from RED to GREEN, and  switches a light from
GREEN to RED. For easier reference, we label  as \textquotedblleft \textquotedblright\ for the end of NEP events,  as
\textquotedblleft\textquotedblright\ and  as \textquotedblleft \textquotedblright\ for the light switching events. The resulting
start of a NEP is an event \textquotedblleft induced\textquotedblright\ by
either  or  or  which we will refer to as an
\textquotedblleft\textquotedblright\ event. For queue 3, the event set
includes all those events that cause a jump in the value of 
in (\ref{f3}). As we can see from Fig. \ref{Automaton}, every event of
 also affects the dynamics of queue 3. Thus, we have .

Returning to Fig. \ref{SamplePath}, the th NEP in a sample path of queue 3,
, is denoted by , i.e., ,
 are the occurrence times of the th  and  event
respectively at this queue. During the th NEP, ,
, denotes the time when an event occurs.

\begin{figure}[tbh]
\centering
\includegraphics[scale = 0.4]{SamplePath.jpg}\caption{A typical sample path of
traffic light queue 3}\label{SamplePath}\end{figure}

\begin{figure}[tbh]
\centering
\includegraphics[scale = 0.6]{StartofNEP.jpg} \caption{Three ways for starting
a NEP}\label{StartofNEP}\end{figure}

Our objective is to select  so as to minimize a cost function that
measures a weighted mean of the queue lengths over a fixed time interval
. In particular, we define the sample function
where  is a cost weight associated with queue  and  are
given initial conditions. It is obvious that since  during EPs of
queue , we can rewrite (\ref{costfun1}) in the form

where  is the total number of NEPs during the sample path of queue .
For convenience, we also define
to be the sample cost associated with the th NEP of queue . We can now
define our overall performance metric as
Since we do not impose any limitations on the processes 
and , it is infeasible to obtain a closed-form expression of
. The only assumption we make is that ,
 are piecewise continuous w.p. 1. The value of IPA as developed
for general stochastic hybrid systems in \cite{Cassandras10} is in providing
the means to estimate the performance metric gradient , by
evaluating the sample gradient . As shown elsewhere (e.g.,
see \cite{Cassandras10}), these estimates are unbiased under mild technical
conditions. Moreover, an important property of IPA estimates is that they are
often independent of the unknown processes  and
 or they depend on values of  or  at specific event times only. Such robustness properties of IPA
(formally established in \cite{YaoCgc11}) make it attractive for estimating on
line performance sensitivities with respect to controllable parameters such as
 in our case. One can then use this information to either improve
performance or, under appropriate conditions, solve an optimization problem
and determine an optimal  through an iterative scheme:

where  is an estimate of  based on the information obtained from the sample path denoted by
, and  is the stepsize at the th iteration. Next we
will focus on how to obtain , . We may then also
obtain  through (\ref{iteration}), provided that the random
processes  and  are stationary over
. We will assume that the derivatives , exist for all
 w.p. 1 (if this is violated, IPA is still
possible by considering one-sided derivatives; see \cite{Cassandras02}.)

\section{Infinitesimal Perturbation Analysis (IPA)}

Consider a sample path of the system as modeled in Fig. \ref{Automaton} over
 and let  denote the occurrence time of the th
event (of any type), where we stress its dependence on . To simplify
notation, we define the derivatives of the states  and
 and event times  with respect to
, , as follows:

Taking derivatives with respect to  in (\ref{costfun}), we obtain
Since, at the start and end of a NEP ,
this reduces to
where the last equality follows from the definition (\ref{Lnm}).

By assumption ,  is independent of . Therefore,
 for  and . It follows that
 for  and  can be obtained by the analysis
of a single isolated intersection in \cite{GengCDC12}. Since equation
(\ref{dxdt}) can still be applied for , we can obtain , , similar to a queue in an isolated intersection.
Therefore, in what follows, we focus on obtaining  and
hence .

\subsection{State Derivatives}

Observe that the determination of the sample derivatives in (\ref{dLi})
depends on the state derivatives . The purpose of IPA is
to evaluate these derivatives as functions of observable sample path
quantities. We pursue this next, using the framework established in
\cite{Cassandras10} where, for arbitrary stochastic hybrid systems, it is
shown that the state and event time derivatives in (\ref{IPAnotation}) can be
obtained from three fundamental \textquotedblleft IPA
equations\textquotedblright. For the sake of self-sufficiency, these equations
are rederived here as they pertain to our specific SFM. Looking at
(\ref{dxdt}) and Fig. \ref{Automaton}, note that the dynamics of 
are fixed over any interevent interval  and we write
 to represent the appropriate expression on the
right-hand-side of (\ref{dxdt}) over this interval. We have

and taking derivatives with respect to , we get

Letting  and since  and  from (\ref{dxdt}), we
obtain

Moreover, taking derivatives with respect to  in (\ref{xprime}), we get,
for all ,

Again,  and we get .
Therefore,  remains constant over all :

Thus, focusing on a NEP of , the queue content derivative is
piecewise constant with jumps occurring according to (\ref{jumps}). The final
step is to obtain the event time derivatives  appearing
in (\ref{jumps}), which we do next.

\subsection{Event Time Derivatives}

Clearly  depends on the type of event occurring at time
. Following the framework in \cite{Cassandras10}, there are three
types of events for a general stochastic hybrid system. For the purpose of
these definitions, let the continuous state component of the hybrid system be
, ,
and let .

\begin{enumerate}
\item \textbf{Exogenous Events.} An event is \emph{exogenous} if it causes a
discrete state transition at time  independent of the controllable
parameter . Thus, it satisfies



\item \textbf{Endogenous Events.} An event is occurring at time  is
\emph{endogenous} if there exists a continuously differentiable function
 such that

where the function  normally corresponds to a guard condition in a
hybrid automaton. Taking derivatives with respect to ,
, it is straightforward to obtain



\item \textbf{Induced Events.} An event at time  is \emph{induced}
if it is triggered by the occurrence of another event at time . In this case,  depends on the derivative
 (details can be found in \cite{Cassandras10}.)
\end{enumerate}

In the following, we consider each of the event types at queue  that were
identified in the previous section and derive the corresponding event time
derivatives. Based on these, we can then also derive the state derivatives
through (\ref{jumps}) and (\ref{xprime(t)}).

\textbf{(1)} \emph{Event }\emph{ ends a NEP of queue 1}. This is an
endogenous event that occurs when . Thus, when such an
event occurs at , let .
Using (\ref{dtaudtheta}), we get . Looking at (\ref{f3}), we have
either  and
 when
, or  and
 when . In
both cases, . Using these values in (\ref{jumps}) along
with  above we get
As we can see,  explicitly depends on
.

\textbf{(2)} \emph{Event }\emph{ ends a NEP of queue 3}. This is an
endogenous event that occurs when . Thus, when such an
event occurs at , let .
Using (\ref{dtaudtheta}), we get . According to (\ref{f3}), we have
. Using these values in (\ref{jumps}) along with
 above we get
Thus, at the end of a NEP  of queue  we have
indicating that these state derivatives are always reset to  upon ending a NEP.

\textbf{(3)} \emph{Event }\emph{, i.e., the GREEN light of queue 1
switches to RED}. This is an endogenous event that occurs when .  is
determined by the following lemma.

\begin{lem}
Let  be the total number of  events that have occurred before or at , and  be the total number of  events that have occurred before or at . Then, , ,  and 
\label{lemma1}
\end{lem}

The proof of this lemma can be found in \cite{GengCDC12}. According to
(\ref{f3}), we have either  (from (5.1)-(5.3), or (5.4)-(5.6)), or
 (from
(5.2)-(5.3), or (5.5)-(5.6)). From (\ref{departRate}), we can combine these
two situations and simply so that . According to (\ref{jumps}), we get



\textbf{(4)} \emph{Event }\emph{, i.e., the GREEN light of queue 3
switches to RED}. This is an endogenous event that occurs when .  is
determined by the following lemma.

\begin{lem}
Let  be the total number of  events that have occurred before or at , and  be the total number of  events that have occurred before or at . Then, , ,  and 
\label{lemma1}
\end{lem}

From (\ref{f3}), if , we have  (from (5.4)-(5.1), or
(5.5)-(5.2), or (5.6)-(5.3)). According to (\ref{jumps}), the state derivative
is

If ,  (from (5.3)-(5.1), or (5.3)-(5.2)). Then,



\textbf{(5)} \emph{Event }\emph{, i.e., the RED light of queue 1
switches to GREEN}. This is an endogenous event that occurs when
. 
is determined by Lemma 1. Similar to the analysis of a  event, we
have 
(from (5.3)-(5.1), or (5.6)-(5.4), or (5.3)-(5.2), or (5.6)-(5.5)). Thus the
state derivative is



\textbf{(6)}\emph{Event }\emph{, i.e., the RED light of queue 3
switches to GREEN}. This is an endogenous event that occurs when
. 
is determined by Lemma 2. From (\ref{f3}), we have  (from (5.1)-(5.4), or
(5.2)-(5.5), or (5.3)-(5.6)). The state derivative is



\textbf{(7)} \emph{Event }\emph{ starts a NEP of queue 1} As already
mentioned, this is an event induced by a  event () or a switch
of  from zero to a strictly positive value () occurring
during a RED cycle, or a switch of  from a
non-positive to a strictly positive value () occurring during a GREEN
cycle (see Fig. \ref{StartofNEP}). Consequently, there are three possible
cases to consider as follows.

\textbf{Case (7a)}: \emph{A NEP of queue 1 starts right after a }\emph{ event}. This is an endogenous event and was analyzed in Case
\textbf{(3)}. Since ,  (from (5.2)-(5.3) or from
(5.5)-(5.6) in (\ref{f3})). We get



\textbf{Case (7b)}: \emph{ A NEP of queue 1 starts while }\emph{,  }. This is an exogenous event occurring during
a RED cycle for queue  and is due to a change in 
from a zero to a strictly positive value. Therefore, .
We then have



\textbf{Case (7c)}: \emph{A NEP of queue 1 starts while }\emph{,  }. This is an exogenous event occurring during
a GREEN cycle for queue  due to a change in  or
 that results in  switching from a non-positive to a strictly positive value. The
analysis is exactly the same as Case (\textbf{7b}) above and (\ref{7b}) applies.

\textbf{(8)} \emph{Event }\emph{ starts a NEP of queue 3}. This is
similar to Case \textbf{(7)}, and there are also three possible cases to consider.

\textbf{Case (8a)}: \emph{A NEP of queue 3 starts right after a }\emph{ event}. This is an endogenous event and was analyzed in Case
\textbf{(4)}. Since , we have , and (\ref{type4di2})
applies. Suppose that this is the th NEP, i.e., . We
have already shown in (\ref{E_event}) that . In addition, we have  over the interval , thus  for all  and we get  , the state derivative in
(\ref{type4di2}) becomes



\textbf{Case (8b)}: \emph{ A NEP of queue 3 starts while }\emph{, \ }. This is due to a change in  from a zero to a strictly positive value. It also happens in
two ways. First,  becomes positive
because a  event occurs. Then (\ref{type3di}) applies, where
. Second,  becomes
positive because either  or  switches
from 0 to a strictly positive value, which is an exogenous event. Therefore,
the state derivative is



\textbf{Case (8c)}: \emph{A NEP of queue 3 starts while }\emph{, \ }. This is due to a change in  from a zero to a strictly positive value,
which may happen in two ways. First, it becomes positive because a 
event occurs, which makes  larger. Then (\ref{type3di})
applies, where . Second, it is due to a
change of value in either  or  or
, which are all exogenous events.The state derivative is
the same as in (\ref{8b}).

This completes the derivation of all state and event time derivatives required
to evaluate the sample performance derivative in (\ref{dLi}). Using the
definition of  in (\ref{Lnm}), note that we can decompose
(\ref{dLi}) into its NEPs and evaluate the derivatives  as shown next.

\subsection{Cost Derivatives}

By virtue of (\ref{xprime(t)}),  is piecewise constant
during a NEP and its value changes only at an event point ,
. Therefore, we have

Clearly, the state derivative at each event point is determined by
(\ref{jumps}) which in turn depends on the event type at ,
 and is given by the corresponding expression in
(\ref{type1}) through (\ref{8b}). An explicit closed-form expression of
 may be obtained in this manner but becomes
complicated. A simple algorithm that updates  after every observed event is simple to implement. More
importantly, note that this IPA derivative depends on:  the number of
events in each NEP ,  the number of total  events
,  the number of total  events ,
 the event times ,  and , and 
the arrival and departure rates , 
at an event time \emph{only}. The quantities in  are easily observed
through counters and timers. The rates in  may be obtained through simple
estimators, emphasizing that they are only needed at a finite number of
observed event times.

\section{Simulation Results}


We describe how the IPA estimator derived for the SFM can be used to determine
optimal light cycles for two intersections modeled as a DES. We apply the IPA
estimator using actual data from an observed sample path of this DES (in this
case, by simulating as a pure DES).

We assume cars arrive according to a Poisson process with rate ,  (as already emphasized, our results are independent of this
distribution.). We also assume cars depart at a rate  which we fix
to be a constant  when the road is not empty. We also constrain , to
take values in .

For the simulated DES model, we use a brute-force (BF) method to find an
optimal : we discretize all real values of 
and for  combinations we run  sample paths
to obtain the average total cost. The value of  is the one
generating the least average cost, to be compared to ,
the IPA-based method. In our simulations, we estimate  through  by
counting car arrivals  over a time window  before or after
;  is similarly estimated.

In the results reported here, we set ,
,    and the sample length . Fig. \ref{Trajectory} shows the trajectories of  and
 using the IPA-based method where  and initial  . More results are
shown in Table 1. As we can see,  is approaching the
optimal value obtained by the BF method. Notice that BF method becomes
impractical when there are more controlling parameters, or when the range of
the parameter is large. However, the IPA method is still effective in such
situations. Moreover, we notice that the value of  is similar to . This indicates that the two intersections tend to have the sample traffic light switching cycle to balance traffic flows.

\begin{figure}[tbh]
\centering
\includegraphics[scale = 0.4]{Trajectory3.jpg}\caption{IPA-based TLC result I }\label{Trajectory}\end{figure}

\begin{table}[tbh]
\caption{IPA vs BF method result I}\label{table1}	
\begin{center}\begin{tabular}
[c]{|c|c|c||c|c|}\hline
\multirow{2}{*}{w} & \multicolumn{2}{|c||}{BF} & \multicolumn{2}{|c|}{IPA}\\\cline{2-5}
 &  &  &  & \\\hline
{[1,1,1,1]}  & [15,15,15,15] & 5.4 & [15,15,15,15] & 5.4\\\hline
{[10,1,1,1]}  & [27,15,15,29] & 16.6 & [28.8,15,15,27.8] & 17.5\\\hline
{[1,5,5,1]}  & [15,23,17,21] & 12.6 & [15.1,18.6,15.6,18.5] & 13.2\\\hline
{[5,1,1,10]}  & [25,15,15,25] & 22.0 & [22.1,15,15,22.9] & 22.5\\\hline
{[1,10,1,1]}  & [15,29,15,29] & 16.3 & [15, 31.2,18.1,26.6] & 17.2\\\hline
\end{tabular}
\end{center}
\end{table}

Based on this observation, we also do simulations by setting  and , which
indicates that we set the \textquotedblleft GREEN plus RED\textquotedblright \ cycle to be fixed  for each intersection. With this constraint, we only need to find optimal  and , since  and . We first let , which restricts the two intersections to have the same traffic light switching cycle. Table 2 shows the simulation results.  and  are set to be the value obtained from Table 1. For example, when ,  and  in Table 1. We then set  in Table 2, and restrict . Comparing the results in Table 2 with Table 1, we find that it supports the results where we allow independent .

In Fig. \ref{T2}, we set  and change  to obtain  while keeping . As we can see, the minimum  is achieved when , which also matches the observations under independent . 


\begin{figure}[tbh]
\centering
\includegraphics[scale = 0.4]{T2.jpg}\caption{ at different }\label{T2}\end{figure}

\begin{table}[tbh]
\caption{IPA vs BF method result II}\label{table2}
\begin{center}\begin{tabular}
[c]{|c|c|c|c||c|c|}\hline
\multirow{2}{*}{w} & \multirow{2}{*}{ } & \multicolumn{2}{|c||}{BF}
& \multicolumn{2}{|c|}{IPA}\\\cline{3-6}
&  &  &  &  & \\\hline
{[1,1,1,1]} & [30,30] & [15,15] & 5.4 & [15,15] & 5.4\\\hline
{[10,1,1,1]} & [44,44] & [31,20] & 16.2 & [30.4,21.6] & 17.6\\\hline
{[1,5,5,1]} & [39,39] & [15,16] & 12.2 & [15,15.6] & 14.1\\\hline
{[5,1,1,10]} & [40,40] & [25,15] & 24.3 & [23.1.6,12.5] & 24.3\\\hline
{[1,10,1,1]} & [44,44] & [15,16] & 17.5 & [15,15.2] & 17.6\\\hline
\end{tabular}
\end{center}
\end{table}

We are also interested to see the optimal control parameters under different traffic intensities. We set ,  and , i.e., we operate under different arrival rate of queue 1. Fig. \ref{arrivalRate} shows the optimal cost and optimal  and  while  varies. It is clear to see that  increases as  decreases. This indicates more GREEN light duration is assigned to queue  as more cars are accumulated in queue 1 because of the fast arrival rate.  also increases because more cars flow into queue 3. 

\begin{figure}[tbh]
\centering
\includegraphics[scale = 0.4]{ArrivalRate.jpg}\caption{ and  at different }\label{arrivalRate}\end{figure}

It must be pointed out that the BF method does not provide a \textquotedblleft true\textquotedblright\ optimal, since the DES model of the traffic system is as much an approximation as the SFM based on which IPA operates. Thus, the comparative results should be interpreted accordingly. 

\section{Conclusions and Future work}

We have developed a SFM for a traffic light control problem with two
coupledintersections, based on which we derive an IPA\ gradient estimator of a
cost metric with respect to the controllable green and red cycle lengths. The
estimators are used to iteratively adjust light cycle lengths to improve
performance and, under proper conditions, obtain optimal values which adapt to
changing traffic conditions. The analysis in the paper can be readily extended
to  intersections in tandem. Future work will extend our method to solving
the TLC problem over multiple junctions without assumption , i.e.,
allowing a finite car capacity between intersections to cause blocking effects.




\bibliographystyle{plain}
\bibliography{yf}

























\end{document}