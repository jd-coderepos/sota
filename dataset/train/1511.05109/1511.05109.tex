\documentclass[10pt]{llncs}

\usepackage[]{geometry}
\newgeometry
{
    left=2.0cm,
    right=2.0cm,
    top=3.2cm,
    bottom=3.2cm
}

\usepackage{amsmath}
\usepackage{amsfonts}
\usepackage{amssymb}

\spnewtheorem{observation}{Observation}{\bfseries}{\itshape}
\usepackage[ruled, vlined, linesnumbered]{algorithm2e}

\usepackage[utf8]{inputenc}
\usepackage{microtype}
\input{glyphtounicode}
\pdfgentounicode=1
\usepackage{graphicx}

\usepackage{enumerate}

\newcommand{\calA}{\mathcal{A}}
\newcommand{\calB}{\mathcal{B}}
\newcommand{\calC}{\mathcal{C}}
\newcommand{\calD}{\mathcal{D}}
\newcommand{\calE}{\mathcal{E}}
\newcommand{\calF}{\mathcal{F}}
\newcommand{\calG}{\mathcal{G}}
\newcommand{\calH}{\mathcal{H}}
\newcommand{\calI}{\mathcal{I}}
\newcommand{\calJ}{\mathcal{J}}
\newcommand{\calK}{\mathcal{K}}
\newcommand{\calL}{\mathcal{L}}
\newcommand{\calM}{\mathcal{M}}
\newcommand{\calN}{\mathcal{N}}
\newcommand{\calO}{\mathcal{O}}
\newcommand{\calP}{\mathcal{P}}
\newcommand{\calQ}{\mathcal{Q}}
\newcommand{\calR}{\mathcal{R}}
\newcommand{\calS}{\mathcal{S}}
\newcommand{\calT}{\mathcal{T}}
\newcommand{\calU}{\mathcal{U}}
\newcommand{\calV}{\mathcal{V}}
\newcommand{\calW}{\mathcal{W}}
\newcommand{\calX}{\mathcal{X}}
\newcommand{\calY}{\mathcal{Y}}
\newcommand{\calZ}{\mathcal{Z}}

\title
{Minimum Eccentricity Shortest Paths in some Structured Graph Classes\thanks
    {Results of this paper were partially presented at WG~2015\,\cite{DraganLeiter2015}.
    }
}

\author
{Feodor F. Dragan 
    \and 
    Arne Leitert
}

\institute{
    Department of Computer Science, \\
    Kent State University, Kent, Ohio, USA  \\
    \email{dragan@cs.kent.edu}, 
    \email{aleitert@cs.kent.edu} 
}

\makeatletter
\newcommand{\ie}{i.\,e.\@ifnextchar{,}{}{~}}
\newcommand{\eg}{e.\,g.\@ifnextchar{,}{}{~}}
\makeatother

\DeclareMathOperator{\pg}{pg}
\DeclareMathOperator{\pl}{pl}
\DeclareMathOperator{\tl}{tl}
\DeclareMathOperator{\ld}{ld}
\DeclareMathOperator{\ecc}{ecc}
\DeclareMathOperator{\diam}{diam}

\DeclareRobustCommand{\qedClaim}
{\ifmmode \lozenge \else \leavevmode\unskip\penalty9999 \hbox{}\nobreak\hfill \quad\hbox{}\fi }

\SetKw{KwDownTo}{downto}

\begin{document}
\pagestyle{plain}
\maketitle

\begin{abstract}
We investigate the \emph{Minimum Eccentricity Shortest Path} problem in some structured graph classes. 
It asks for a given graph to find a shortest path with minimum eccentricity. 
Although it is NP-hard in general graphs, we demonstrate that a minimum eccentricity shortest path can be found in linear time for distance-hereditary graphs (generalizing the previous result for trees) and give a generalised approach which allows to solve the problem in polynomial time for other graph classes.
This includes chordal graphs, dually chordal graphs, graphs with bounded tree-length, and graphs with bounded hyperbolicity.
Additionally, we give a simple algorithm to compute an additive approximation for graphs with bounded tree-length and graphs with bounded hyperbolicity.
\end{abstract}

\section{Introduction}

The \emph{Minimum Eccentricity Shortest Path} problem asks for a given graph  to find a shortest path~ such that for each other shortest path~,  holds.
Here, the eccentricity of a set  in  is .
This problem was introduced in~\cite{DrLei2015}.
It may arise in determining a ``most accessible'' speedy linear route in a network  and can find applications in communication networks, transportation planning, water resource management and fluid transportation.
It was also shown in~\cite{DKL2014,DrLei2015} that a minimum eccentricity shortest path plays a crucial role in obtaining the best to date approximation algorithm for a minimum distortion embedding of a graph into the line.
Specifically, every graph~ with a shortest path of eccentricity~ admits an embedding~ of  into the line with distortion at most , where  is the minimum line-distortion of  (see~\cite{DrLei2015} for details).
Furthermore, if a shortest path of  of eccentricity~ is given in advance, then such an embedding~ can be found in linear time.
Note also that every graph has a shortest path of eccentricity at most .

Those applications motivate investigation of the Minimum Eccentricity Shortest Path problem  in general graphs and in particular graph classes.
Fast algorithms for it will imply fast approximation algorithms for the minimum line distortion problem.
Existence of low eccentricity shortest paths in structured graph classes will imply low approximation bounds for those classes.
For example, all AT-free graphs (hence, all interval, permutation, cocomparability graphs) enjoy a shortest path of eccentricity at most 1~\cite{COS-SICOMP}, all convex bipartite graphs enjoy a shortest path of eccentricity at most 2~\cite{DKL2014}.

In \cite{DrLei2015}, the Minimum Eccentricity Shortest Path problem was investigated in general graphs.
It was shown that its decision version is NP-complete (even for graphs with vertex degree at most 3).
However, there are efficient  approximation algorithms: a 2-approximation, a 3-approximation, and an 8-approximation for the problem can be computed in  time, in  time, and in linear time, respectively.
Furthermore, a shortest path of minimum eccentricity~ in general graphs can be computed in  time. 
Paper~\cite{DrLei2015} initiated also the study of the Minimum Eccentricity Shortest Path problem in special graph classes by showing that a minimum eccentricity shortest path in trees can be found in linear time.
In fact, every diametral path of a tree is a minimum eccentricity shortest path. 

In this paper, we design efficient algorithms for the Minimum Eccentricity Shortest Path problem in distance-hereditary graphs, in chordal graphs, in dually chordal graphs, and in more general graphs with bounded tree-length or with bounded hyperbolicity.
Additionally, we give a simple algorithm to compute an additive approximation for graphs with bounded tree-length and graphs with bounded hyperbolicity.

Note that our Minimum Eccentricity Shortest Path problem is close but different from the \emph{Central Path} problem in graphs introduced in~\cite{Slater}. 
It asks for a given graph~ to find a path~ (not necessarily shortest) such that any other path of  has eccentricity at least .
The Central Path problem generalizes the Hamiltonian Path problem and therefore is NP-hard even for chordal graphs \cite{haiko}.
Our problem is polynomial time solvable for chordal graphs. 

\section{Notions and Notations}
All graphs occurring in this paper are connected, finite, unweighted, undirected, loopless and without multiple edges. For a graph~, we use  and  to denote the cardinality of the vertex set and the edge set of~.
 denotes the \emph{induced subgraph} of  with the vertex set~. 

The \emph{length} of a path from a vertex~ to a vertex~ is the number of edges in the path. 
The \emph{distance}~ of two vertices  and~ is the length of a shortest path connecting  and~. 
The distance between a vertex~ and a set~ is defined as . 
The \emph{eccentricity}~ of a vertex~ is .
For a set~, its eccentricity is .
For a vertex pair , a shortest -path~ has \emph{minimal eccentricity}, if there is no shortest -path~ with .
Two vertices  and  are called \emph{mutually furthest} if . 
A vertex~ is \emph{-dominated} by a vertex~ (by a set~), if  (, respectively).

The \emph{diameter} of a graph~ is . 
The diameter~ of a set~ is defined as . 
A pair of vertices  of  is called a \emph{diametral  pair} if .
In this case, every shortest path connecting  and~ is called a \emph{diametral path}. 

For a vertex~,  is called the \emph{open neighborhood}, and  the \emph{closed neighborhood} of . 
 denotes the \emph{disk} of radius~ around vertex~. 
Additionally,  denotes the vertices with distance~ from~. 
For two vertices  and~,  is the \emph{interval} between  and~. 
The set~ is called a \emph{slice} of the interval from  to~. 
For any set~ and a vertex~,  denotes the \emph{projection} of  on~. 

A \emph{chord} in a path is an edge connecting two non-consecutive vertices of the path. 
A set of vertices~ is a \emph{clique} if all vertices in  are pairwise adjacent. 
A graph is \emph{chordal} if every cycle with at least four vertices has a chord. 
A graph is \emph{distance-hereditary} if the distances in any connected induced subgraph are the same as they are in the original graph. 
A graph is \emph{dually chordal} if it is the intersection graph of maximal cliques of a chordal graph.
For more definitions of these classes and relations between them see~\cite{BrLeSpinGraphClasses}.

\section{A Linear-Time Algorithm for Distance-Hereditary Graphs}
    \label{sec:DistHered}

\emph{Distance-hereditary graphs} can be defined as graphs where each chordless path is a shortest path~\cite{howorka}. 
Several interesting characterizations of distance-hereditary graphs in terms of metric and neighborhood properties, and forbidden configurations were provided by \textsc{Bandelt} and \textsc{Mulder}~\cite{BM-dhg}, and by \textsc{D'Atri} and \textsc{Moscarini}~\cite{D-AM-dhg}. 
The following proposition  lists  the basic information on distance-hereditary graphs that is needed in what follows. 

\begin{proposition}
    [\cite{BM-dhg,D-AM-dhg}] 
    \label{prop:dhg} 
For a graph~ the following conditions are equivalent:
\begin{enumerate}[(1)]
    \item
         is distance-hereditary;
    \item
        The house, domino, gem (see Fig.~\ref{fig:dhg}) and the cycles~ of length~ are not induced subgraphs of ;
    \item
        For an arbitrary vertex~ of  and every pair of vertices , that are in the same connected component of the graph , we have .
    \item
        (4-point condition)
        For any four vertices  of  at least two of the following distance sums are equal:
        ; ; .
        If the smaller sums are equal, then the largest one exceeds the smaller ones at most by 2.
\end{enumerate}
\end{proposition}

\begin{figure}
    [htb]
    \centering
    \includegraphics[]{fig_house}\hspace*{1cm}\includegraphics[]{fig_domino}\hspace*{1cm}\includegraphics[]{fig_gem}\caption
    {
        Forbidden induced subgraphs in a distance-hereditary graph. 
    }
    \label{fig:dhg} \end{figure}

As a consequence of statement (3) of Proposition~\ref{prop:dhg} we get. 

\begin{corollary}
    \label{cor:pr-size} 
Let  be a shortest path in a distance-hereditary graph~ connecting vertices  and~, and  be an arbitrary vertex of~. 
Let  be a vertex of  that is closest to , and let  be a vertex of  that is closest to~. 
Then  and there must be a vertex  in  adjacent to both  and~ and at distance  from~. 
\end{corollary}

As a consequence of statement (4) of Proposition \ref{prop:dhg} we get. 

\begin{corollary}
    \label{cor:two-paths} 
Let  be arbitrary vertices of a distance-hereditary graph~ with , , and , then . 
That is, if two shortest paths share ends of length at least~, then their union is a shortest path. 
\end{corollary} 

\begin{proof}
Consider distance sums ,  and . 
Since , we have . 
Then, either  or  and . 
If the latter is true, then  and a contradiction arises.
Thus,  and we get .
\qed
\end{proof}


\begin{lemma}
    \label{lm:dhg} 
Let  be a diametral pair of vertices of a distance-hereditary graph~, and  be the minimum eccentricity of a shortest path in . 
If for some shortest path , connecting  and ,  holds, then . 
Furthermore, if  then there is a shortest path  between  and~ with . 
\end{lemma} 

\begin{proof} 
Consider a vertex  with . 
Let  be a vertex of  closest to , and  be a vertex of  closest to . 
By Corollary~\ref{cor:pr-size},  and there must be a vertex  in  adjacent to both  and  and at distance  from . 
Let  and  be subpaths of  connecting vertices  and vertices , respectively. 
Consider also an arbitrary shortest path  connecting  and  in . 
By choices of  and , no chords in  exist in paths  and . 
Hence, those paths are shortest in . 
Since  is a diametral pair, we have . 
That is, .
Similarly, . 
Combining both inequalities and taking into account that , we get  . 
Furthermore, we have  if  and  if .
Also, if  then , ,  and .

Now assume that . 
Consider sets  and . 
Let  be a vertex of  that -dominates the maximum number of vertices in . 
Consider a shortest path  connecting vertices  and  and passing through vertex . 
We will show that . 
Let  () be the neighbor of  in subpath of  connecting  with  (with , respectively).

Assume there is a vertex  in  such that . 
As in the first part of the proof, one can show that , \ie,  and . 
Furthermore, , \ie, . 
Also, vertex , that is adjacent to ,  and at distance  from , must belong to . 
Since  but , by choice of , there must exist a vertex  such that  and . 
Since ,  must equal  and both  and  must equal .  

Since  and , we must have a chord between vertices of a shortest path  connecting  with  and vertices of a shortest path  connecting  with . 
If no chords exist or only chord  is present, then , contradicting with . 
So, consider a chord  with , , , and  is minimum. 
We know that  must hold since  and .
To avoid induced cycles of length ,  must hold.
But then, vertices  form either an induced cycle~, when  and  are not adjacent, or a house, otherwise.
Note that, by distance requirements, edges , , , and  are not possible. 

Contradictions obtained show that such a vertex  with  is not possible, \ie, . 
\qed
\end{proof}

\begin{lemma}
    \label{lm:mf} 
In every distance-hereditary graph there is a minimum eccentricity shortest path  where  and~ are two mutually furthest vertices. 
\end{lemma}

\begin{proof} 
Let  be the minimum eccentricity of a shortest path in . 
Let  be a shortest path of  of eccentricity~ with maximum~, that is, among all shortest paths with eccentricity~,  is a longest one. 
Assume, without loss of generality, that  is not a vertex most distant from . 
Let  be the smallest index such that subpath  of  has also the eccentricity~. 
By choice of , there must exist a vertex  in  which is -dominated only by vertex  of , \ie,  and . 
Let  be an arbitrary shortest path of  connecting  with .
By choice of , no vertex of  is adjacent to a vertex of . 
Hence, path obtained by concatenating  with  is chordless and, therefore, shortest in , and has eccentricity~, too. 
Note that  is now a most distant vertex from , \ie, . 
Since , a contradiction with maximality of  arises. 
\qed
\end{proof}

The main result of this section is the following.

\begin{theorem}
    \label{th:dhg} 
Let  be a diametral pair of vertices of a distance-hereditary graph~, and  be the minimum eccentricity of a shortest path in . 
Then, there is a shortest path~ between  and~ with .  
\end{theorem}

\begin{proof} 
We may assume that for some shortest path  connecting  and ,  holds (otherwise, there is nothing to prove).
Then, by Lemma~\ref{lm:dhg}, we have .

Let  be a shortest path of  of eccentricity  such that  and  are two mutually furthest vertices (see Lemma~\ref{lm:mf}). 
Consider projections of  and  to . 
We distinguish between three cases: 
 is completely on the left of  in ;  and  have a common vertex ; and the remaining case (see Corollary~\ref{cor:pr-size}) when  and  for some index .

\medskip

\noindent
\emph{Case 1:  is completely on the left of  in .}    

\medskip
\noindent
Let  be a vertex of  closest to  and  a vertex of  closest to . 
Consider an arbitrary shortest path  of  connecting vertices  and , an arbitrary shortest path  of  connecting vertices  and , and a subpath  of  between vertices  and . 
We claim that the path  of  obtained by concatenating  with  and then with  is a shortest path of eccentricity~. 

Indeed, by choice of , no edge connecting a vertex in  with a vertex in  can exist in . 
Similarly, no edge connecting a vertex in  with a vertex in  can exist in . 
Since we also have ,  and , no edge connecting a vertex in   with a vertex in  can exist in . 
Hence, chordless path  is a shortest path of .

Consider now an arbitrary vertex  of . 
We want to show that . 
Since , . 
Consider the projection of  to . 
We may assume that  and, without loss of generality, that vertices of  are closer to  than vertex . 
Let  be a vertex of  closest to .
As before, by choices of  and , paths  and  are chordless and, therefore, are shortest paths of  (here  is an arbitrary shortest path of  connecting  with ). 
Since , by Corollary~\ref{cor:two-paths}, . 
Hence, from ,  and , we obtain .

\medskip
\noindent
\emph{Case 2:  and  have a common vertex .} 

\medskip
\noindent
In this case, we have . 
Earlier we assumed also that .
Hence,  and the statement of the theorem follows from Lemma~\ref{lm:dhg}. 

\medskip
\noindent
\emph{Case 3: Remaining case when  and  for some index .}

\medskip
\noindent
In this case, we have . 
By Lemma~\ref{lm:dhg}, we can assume that , \ie, .

Let  and  be subpaths of  connecting vertices  and  and vertices  and , respectively. 
Pick an arbitrary shortest path  connecting  with . 
Since no chords are possible between  and  and between  and , we have
 and .
Inequalities  and  imply  and . 
If both  and  equal , then  contradicting with . 
Hence, we may assume, without loss of generality, that . 
We will show that shortest path  has eccentricity  (here,  is an arbitrary shortest path of  connecting  with ). 

Consider a vertex  in  and assume that  is strictly contained in . 
Denote by  the vertex of  that is closest to . 
Let  be an arbitrary shortest path connecting  and . 
As before,  is a chordless path and therefore .
Since  is a most distant vertex from , . 
Hence, , \ie, . 

Consider a vertex  in  and assume now that  is strictly contained in . 
Denote by  the vertex of  that is closest to . 
Let  be an arbitrary shortest path connecting  and . 
Again,  is a chordless path and therefore .
Since  is a most distant vertex from , . 
Hence, , \ie, . 

Thus, all vertices of  are -dominated by . 
\qed
\end{proof}

It is known~\cite{DrNi} that a diametral pair of a distance-hereditary graph can be found in linear time. 
Hence, according to Theorem~\ref{th:dhg}, to find a shortest path of minimum eccentricity in a distance-hereditary graph in linear time, one needs to efficiently extract a best eccentricity shortest path for a given pair of end-vertices. 
In what follows, we demonstrate that, for a distance-hereditary graph, such an extraction can be done in linear time as well. 

We will need few auxiliary lemmas. 

\begin{lemma}
    \label{lem:dhgGateVertex}
In a distance-hereditary graph~, for each pair of vertices  and , if  is on a shortest path from  to  and , then .
\end{lemma}

\begin{proof}
Let  and~ be two vertices in  and .
By statement~(3) of Proposition~\ref{prop:dhg}, . 
Thus, each vertex  on a shortest path from  to  with  (which is in  by definition) is adjacent to all vertices in , \ie, .
\qed
\end{proof}

\begin{lemma}
    \label{lem:dhgSliceJoin}
In a distance-hereditary graph~, let  and  be two consecutive slices of an interval . 
Each vertex in  is adjacent to each vertex in .
\end{lemma}

\begin{proof}
Consider statement~(3) of Proposition~\ref{prop:dhg} from perspective of~.
Thus,  for each vertex~. 
Additionally, from perspective of ,  for each vertex~.
\qed
\end{proof}

\begin{lemma}
    \label{lem:dhgProjSliceInter}
In a distance-hereditary graph~, if a projection  intersects two slices of an interval , each shortest -path intersects .
\end{lemma}

\begin{proof}
Because of Lemma~\ref{lem:dhgGateVertex}, there is a vertex  with  and . 
Thus,  intersects at most two slices of interval  and those slices have to be consecutive, otherwise  would be a part of the interval.
Let  and  be these slices. 
Note that . 
Thus, by statement~(3) of Proposition~\ref{prop:dhg},  for each . 
Therefore, , \ie, each shortest path from  to~ intersects .
\qed
\end{proof}

From the lemmas above, we can conclude that, for determining a shortest -path with minimal eccentricity, a vertex~ is only relevant if  and the projection of  on the interval  only intersects one slice.
Algorithm~\ref{algo:MinEccDistHere} uses this.

\begin{algorithm}
    [htb]
    \caption
    {
        \label{algo:MinEccDistHere}
        Computes a shortest -path~ with minimal eccentricity for a given distance-hereditary graph~ and a vertex pair~.
    }

\KwIn
{
    A distance-hereditary graph~ and two distinct vertices  and~.
}

\KwOut
{
    A shortest path~ from  to~ with minimal eccentricity.
}

Compute the sets  for .

Each vertex  gets a pointer~ initialised with  if , and  otherwise.

\For
{
     \KwTo 
    \label{line:dhgFindGateLoop}
}
{
    For each , select a vertex~ and set .
    \label{line:dhgFindGateIteration}
}

\ForEach
{
    
}
{
    If  intersects only one slice of , flag  as \emph{relevant}. 
    \label{line:dhgFlagRelevant}
}

Set .

\For
{
     \KwTo 
    \label{line:dhgSelectPLoop}
}
{
    Find a vertex~ for which the number of \emph{relevant} vertices in  is maximal.

    Add  to .
    \label{line:dhgSelectPAddV}
}

\end{algorithm}


\begin{lemma}
    \label{lm:path-pair} 
For a distance-hereditary graph~ and an arbitrary vertex pair~, Algorithm~\ref{algo:MinEccDistHere} computes a shortest -path with minimal eccentricity in linear time.
\end{lemma}

\begin{proof}
The loop in line~\ref{line:dhgFindGateLoop} determines for each vertex~ outside of the interval~ a \emph{gate vertex}~ such that  and  (see Lemma~\ref{lem:dhgGateVertex}).
From Lemma~\ref{lem:dhgProjSliceInter} and Lemma~\ref{lem:dhgSliceJoin}, it follows that for a vertex~ which is not in  or its projection to  is intersecting two slices of ,  for every shortest path  between  and .
Therefore, line~\ref{line:dhgFlagRelevant} only marks  if  and its projection  intersects only one slice. 
Because only one slice is intersected and each vertex in a slice is adjacent to all vertices in the consecutive slice (see Lemma~\ref{lem:dhgSliceJoin}), in each slice the vertex of an optimal (of minimum eccentricity) path  can be selected independently from the preceding vertex.
If a vertex~ of a slice~ has the maximum number of \emph{relevant} vertices in , then  is good to put in .
Indeed, if  dominates all relevant vertices adjacent to vertices of , then  is a perfect choice to put in .
Else, any vertex  of a slice  is a good vertex to put in . 
Hence,  is optimal if the number of \emph{relevant} vertices adjacent to  is maximal.
Thus, the path selected in line~\ref{line:dhgSelectPLoop} to~line~\ref{line:dhgSelectPAddV} is optimal.
\qed
\end{proof}

Running Algorithm~\ref{algo:MinEccDistHere} for a diametral pair of vertices of a distance-hereditary graph~, by Theorem~\ref{th:dhg}, we get a shortest path of  with minimum eccentricity.
Thus, we have proven the following result. 

\begin{theorem}
    \label{tm:opt-path} 
A shortest path with minimum eccentricity of a distance-hereditary graph  can be computed in  total time.
\end{theorem}

\section{A Polynomial-Time Algorithm for Tree-Structured Graphs}

\subsection{Projection Gap}

In a graph~, consider a shortest path~ which starts in a vertex~.
Each vertex~ has a projection~.
In case of a tree this is a single vertex.
However, in general,  can contain multiple vertices and does not necessarily induce a connected subgraph.
In this case, there are two vertices  and~ in  such that all vertices~ in the subpath~ between  and~ are not in~.
Formally, , , and .

Now, assume the cardinality of  is at most~, \ie,  for each , ,  and~.
We refer to  as the \emph{projection gap} of .

\begin{definition}
    [Projection Gap]
In a graph~, let  be a shortest path with .
The \emph{projection gap} of  is ,  for short, if, for every vertex~ of~ and every two vertices ,  implies that there is a vertex~ with .
\end{definition}

Based on this definition, we can make the following observation.

\begin{lemma}
    \label{lem:valProp}
In a graph  with , let  be a shortest path starting in~,  be a subpath of~, ,  and~ be two vertices in  such that , and  be an arbitrary vertex in~.
If , then .
\end{lemma}

\begin{proof}
Assume that  and .
Without loss of generality, let  for all  with .
Let  be the subpath of  from  to~.
Note that  and .
Thus, .
This contradicts with .
\qed
\end{proof}

Informally, Lemma~\ref{lem:valProp} says that, when exploring a shortest path~, if the distance to a vertex~ did not decrease during the last  vertices of~, it will not decrease when exploring the remaining subpath.
Based on this, we will show that a minimum eccentricity shortest path can be found in polynomial time if  is bounded by some constant.

For the rest of this section, we assume we are given a graph~ with  containing a vertex~.
We will need the following notions and notations:
\begin{itemize}
\item
 and~ are subpaths of length~ of some shortest paths starting in~.
They do not need to be subpath of the same shortest path.
Let  and  be the two vertices such that  and  .
Without loss of generality, let .
We say,  is \emph{compatible} with  (with respect to~) if ,  is adjacent to~, and .
Let  denote the set of subpaths compatible with~.

\item
 is the set of vertices~ such that there is a shortest path from  to~ containing  (or ).

\item
 are the vertices that are on a shortest path from  to~ (or in~).

\item
 is the set of vertices~ which are closer to~ than to all other vertices in~.
Thus, given a shortest path~ containing~ and starting in~, expanding  beyond  will not decrease the distance from  to~.
\end{itemize}
Note that  and .

\subsection{Algorithm}

\begin{lemma}
    \label{lem:kValeyVert}
For each vertex~ in~,  or .
\end{lemma}

\begin{proof}
Assume,  and .
Then, there is a vertex~ and a vertex~ with  and .
Because , , and  are on a shortest path starting in~ and , this contradicts Lemma~\ref{lem:valProp}.
\qed
\end{proof}

\begin{lemma}
    \label{lem:kValeyVertSubset}
If  is compatible with , then .
\end{lemma}

\begin{proof}
Assume that , \ie, there is a vertex~.
Then, .
Thus, by Lemma~\ref{lem:kValeyVert}, .
Because , .
Since , .
Also, because , .
Thus, .
On the other hand, because , , and a contradiction arises.
\qed
\end{proof}

For a subpath , let  denote the set of shortest paths~ which start in~ such that .
Then, we define  as follows:

Consider a subpath  for which , \ie, a shortest path containing  cannot be extended any more.
Then, .
Therefore, for any path , .

\begin{lemma}
    \label{lem:valEpsilon}
If  is not empty, then

\end{lemma}

\begin{proof}
By definition,

Let  be compatible with~.
Because, by Lemma~\ref{lem:kValeyVertSubset}, , we can partition  into  and .
Thus, 

Note that we changed the definition of~ from  to , \ie,  may not contain the last vertex of  any more.

If , then .
Thus, by Lemma~\ref{lem:kValeyVert}, .
Note that, by definition of~, .
Therefore,  and

For simplicity, we define

Note that  does not depend on .
Therefore, because ,

If , then .
Therefore,

Thus,

\qed
\end{proof}

Based on Lemma~\ref{lem:valEpsilon}, Algorithm~\ref{algo:valMESP} computes a shortest path starting in~ with minimal eccentricity.
The algorithm has two parts.
First, it computes the pairwise distance of all vertices and  for each vertex pair  and~ where, similarly to , .
This allows to easily determine if a vertex~ is in~.
Second, it computes  for each subpath~.
For this, the algorithm uses dynamic programming.
After calculating  for all subpaths with distance~ to~, the algorithm uses Lemma~\ref{lem:valEpsilon} to calculate  for all subpaths~ which  is compatible with.

\begin{algorithm}
    [htb!]
    \caption
    {
        Determines, for a given graph~ with  and a vertex~, a minimal eccentricity shortest path starting in~.
    }
    \label{algo:valMESP}

\KwIn
{
    A graph~, an integer~, and a vertex~.
}

\KwOut
{
    A shortest path~ starting in~ with minimal eccentricity.
}

Determine the pairwise distances of all vertices.
\label{line:pairwDistance}

\ForEach
{
    
}
{
    Set .
    \label{line:defaultR}
}

\For
{
     \KwDownTo 
    \label{line:compRsStart}
}
{
    \ForEach
    {
        
    }
    {
        \ForEach
        {
            
        }
        {
            \ForEach
            {
                
            }
            {
                Set .
                \label{line:distRv}
            }
        }
    }
}

\For
{
     \KwTo 
    \label{line:QjLoop}
}
{
    \ForEach
    {
         with 
        \label{line:selectQj}
    }
    {
        \ForEach
        {
            
            \label{line:xQjLoop}
        }
        {
            Let  be the vertex in  with the largest distance to~.
            If , add  to  and store .
            \label{line:computeVsOj}
        }

        \If
        {
            
        }
        {
            
            \label{line:compStartEpsilon}
        }
        \Else
        {
            
            \label{line:QjEpsilonInf}
        }

        \ForEach
        {
            
            \label{line:QiLoop}
        }
        {
            
            \label{line:compEpsilonPrime}

            \If
            {
                
                \label{line:comareEpsilon}
            }
            {
                Set  and .
                \label{line:setP}
            }
        }
    }
}

Find a subpath~ such that a shortest path containing  cannot be extended any more and for which  is minimal.
\label{line:findMinQj}

Construct a path~ from  to  using the -pointers and output it.
\label{line:constructP}
\end{algorithm}

\begin{theorem}
For a given graph~ with  and a vertex~, Algorithm~\ref{algo:valMESP} computes a shortest path starting in~ with minimal eccentricity.
It runs in  time if , in  time if , and in  time if .
\end{theorem}

\begin{proof}
    [Correctness]
The algorithm has two parts.
The first part (line~\ref{line:pairwDistance} to line~\ref{line:distRv}) is a preprocessing which computes  for each vertex pair  and~.
The second part computes  which is then used to determine a path with minimal eccentricity.

For the first part, without loss of generality, let , , and let  be an arbitrary vertex.
By definition of ,  implies , \ie, .
Therefore,  is correct for all vertices~ with  after line~\ref{line:defaultR}.
By induction, assume that  is correct for all vertices~.
Because , .
Therefore, line~\ref{line:distRv} correctly computes~.

The second part of Algorithm~\ref{algo:valMESP} iterates over all subpaths~ in increasing distance to~.
Line~\ref{line:computeVsOj} checks if a given vertex~ is in .
By definition,  where  is the vertex in  with the largest distance to~.
Thus, .
By definition of ,  if and only if .
Therefore,  if and only if , \ie, line~\ref{line:computeVsOj} computes  correctly.

Recall the definition of :

If , .
Therefore,  as computed in line~\ref{line:compStartEpsilon}.
Note that there is no subpath~ which is compatible with , if .
Therefore, the loop starting in line~\ref{line:QiLoop} is skipped for these~.
Thus, the algorithm correctly computes , if .

By induction, assume that  is correct for each .
Thus, Lemma~\ref{lem:valEpsilon} can be used to compute .
This is done in the loop starting in line~\ref{line:QiLoop}.
Therefore, at the beginning of line~\ref{line:findMinQj},  is computed correctly for each subpath~.

Recall, if  and , then  and, therefore, .
Thus,  implies that  is the minimal eccentricity of all shortest paths starting in~ and containing~.
Therefore, if  is picked by line~\ref{line:findMinQj}, then  is the minimal eccentricity of all shortest paths starting in~.
\qed
\end{proof}

\begin{proof}
    [Complexity]
First, we will analyse line~\ref{line:pairwDistance} to line~\ref{line:distRv}.
Line~\ref{line:pairwDistance} runs in  time.
This allows to access the distance between two vertices in constant time.
Thus, the total running time for line~\ref{line:defaultR} is~.
Because line~\ref{line:distRv} is called at most once for each vertex~ and edge~, implementing line~\ref{line:compRsStart} to line~\ref{line:distRv} can be done in  time.

For the second part of the algorithm (starting in line~\ref{line:QjLoop}), if , let all subpaths be stored in a trie as follows:
There are  layers of internal nodes.
Each internal node is an array of size~ (one entry for each vertex) and each entry points to an internal node of the next layer representing ~subtrees.
This requires  memory.
Leafs are objects representing a subpath.

If , a subpath is a single edge, and, if , a subpath is a single vertex.
Thus, no extra data structure is needed for these cases.
In all three cases, a subpath can be accessed in  time.

Next, we analyse the runtime of line~\ref{line:xQjLoop} to line~\ref{line:QjEpsilonInf} for a single subpath~.
Accessing  can be done in  time.
Line~\ref{line:computeVsOj} requires at most  time for a single call and is called at most  times.
Line~\ref{line:compStartEpsilon} requires  time and line~\ref{line:QjEpsilonInf} runs in constant time.
Therefore, for a given subpath, line~\ref{line:xQjLoop} to line~\ref{line:QjEpsilonInf} require  time.

For line~\ref{line:compEpsilonPrime} to line~\ref{line:setP}, consider a given pair of compatible subpaths  and~.
Accessing both subpaths can be done in  time.
Assuming the vertices in  and~ are sorted and stored with their distance to  and~, line~\ref{line:compEpsilonPrime} requires at most  time.
Note that  and  intersect in  vertices.
Thus,  where  is the vertex in  closest to~.
Line~\ref{line:comareEpsilon} and line~\ref{line:setP} run in constant time.
Therefore, for a given pair of compatible subpaths, line~\ref{line:compEpsilonPrime} to line~\ref{line:setP} require  time.

Let  be the number of subpaths and  be the number of pairs of compatible subpaths.
Then, the overall runtime for line~\ref{line:QjLoop} to line~\ref{line:setP} is  time,  time for line~\ref{line:findMinQj}, and  time for line~\ref{line:constructP}.
Together with the first part of the algorithm, the total runtime of Algorithm~\ref{algo:valMESP} is .

Because a subpath contains  vertices, there are up to  subpaths and up to  compatible pairs if , \ie,  and .
Therefore, if , Algorithm~\ref{algo:valMESP} runs in  time.

If , a subpath is a single edge and there are at most  compatible pairs of subpaths, \ie,  and .
For the case when , a subpath is a single vertex () and a pair of compatible subpaths is an edge ().
Therefore, Algorithm~\ref{algo:valMESP} runs in  time if , and in  time if .
\qed
\end{proof}

Note that Algorithm~\ref{algo:valMESP} computes a shortest path starting in a given vertex~.
Thus, a shortest path with minimum eccentricity among all shortest paths in  can be determined by running Algorithm~\ref{algo:valMESP} for all start vertices~, resulting in the following:

\begin{theorem}
For a given graph~ with , a  minimum eccentricity shortest path can be found in  time if , in  time if , and  time if .
\end{theorem}

\subsection{Projection Gap for some Graph Classes}

Above, we have shown that a minimum eccentricity shortest path can be found in polynomial time if the projection gap is bounded by a constant.
In this subsection, we will determine the projection gap for some graph classes.

\subsubsection{Chordal Graphs and Dually Chordal Graphs.}

The class of chordal graphs is a well known class which can be recognised in linear time~\cite{TarjanYannak1984}.
Due to the strong tree structure of chordal graphs, they have the following property known as -convexity:

\begin{lemma}
     [\cite{FaberJamison1986}]
     \label{lem:ChordalMConvex}
Let  be a chordal graph.
If, for two distinct vertices~ in a disk~, there is a path~ connecting them with , then  and~ are adjacent.
\end{lemma}

\begin{lemma}
If  is a chordal graph, then .
\end{lemma}

\begin{proof}
Assume .
Then, there is a shortest path~ and a vertex~ with  and .
By Lemma~\ref{lem:ChordalMConvex},  and  are adjacent.
This contradicts with .
\qed
\end{proof}

\begin{corollary}
For chordal graphs, a minimum eccentricity shortest path can be found in  time.
\end{corollary}

Dually chordal graphs where introduced in~\cite{BraDraCheVol1998}.
They are closely related to chordal graphs.

\begin{lemma}
If  is a dually chordal graph, then .
\end{lemma}

\begin{proof}
Assume there is a shortest path~ and a vertex~ with .
To show that , we will show that  implies there is a vertex~ with .

Consider a family of disks  where .
Let  be the intersection graph of ,  be the vertex in~ representing ,  representing  (for ),  representing , and  representing~.
Because the intersection graph of disks of a dually chordal graph is chordal~\cite{BraDraCheVol1998},  is chordal, too.
 contains the edges  and~, , , and  for all .
Note that, if ,  and~ are not adjacent in~.
However, the path  connects  and~.
Therefore, because  is chordal and by Lemma~\ref{lem:ChordalMConvex}, there is a  with  such that  is adjacent to~ in~.
Thus, , \ie, .
\qed
\end{proof}

\begin{corollary}
For dually chordal graphs, a minimum eccentricity shortest path can be found in  time.
\end{corollary}

\subsubsection{Graphs with bounded Tree-Length or Tree-Breadth.}

As defined by \textsc{Robertson} and \textsc{Seymour}~\cite{RobertSeymou1986}, a \emph{tree-decomposition} of a graph  is a tree  with the vertex set~ where each vertex of , called bag, is a subset of~ such that:
(i)~, (ii)~for each edge~, there is a bag~ with , and (iii)~for each vertex~, the bags containing  induce a subtree of~.

The \emph{length} of a tree~decomposition is smaller than or equal to~ if for each bag~, .
A graph~ has \emph{tree-length}~, if there exist a tree-decomposition~ for~ such that  has length~.
Similarly, the \emph{breadth} of a tree~decomposition is smaller than or equal to~ if for each bag~ there is a vertex~ with .
A graph~ has \emph{tree-breadth}~, if there exist a tree-decomposition~ for~ such that  has breadth~.

For these graphs, we use a concept called \emph{layering partition}.
It was introduced in~\cite{BranChepDrag1999,ChepoiDragan2000}.
The idea is to first partition the vertices of a given graph in distance layers~ with respect to a given vertex~.
Second, partition each layer~ into \emph{clusters} in such a way that two vertices  and~ are in the same cluster if they are connected by a path~ such that , \ie,  does not contain vertices of layers closer to~ than  and~.

Unfortunately, computing the tree-length of a graph is an NP-hard problem~\cite{Lokshtanov2010}.
However, for our needs, an approximation of it would suffice.

\begin{lemma}
If  has tree-length~ or tree-breadth~, a factor~ can be computed in  time such that  or , respectively.
\end{lemma}

\begin{proof}
To compute , first determine the pairwise distances of all vertices.
Then, compute a layering partition for each vertex~.
Let  be the maximum diameter of all clusters of all layering partitions.

The diameter of each cluster is at most  if  has tree-length~ and at most  if  has tree-breadth~~\cite{DouDraGavYan2007,DraganKohler2014}.
Therefore, for each shortest path~,  and , respectively.
Thus,  and .

Computing the pairwise distances of all vertices can be done in  time.
A layering partition can be computed in linear time~\cite{ChepoiDragan2000}.
For a given layering partition, the diameter of each cluster can be computed in  time if the pairwise distances of all vertices are known.
Thus,  can be computed in  time.
\qed
\end{proof}

Note that it is not necessary to know the tree-length or tree-breath of~ to compute~.
Thus, by computing~ and then running Algorithm~\ref{algo:valMESP} for each vertex in , we get:

\begin{corollary}
For graphs with tree-length~ or tree-breadth~, a minimum eccentricity shortest path can be found in  time or  time, respectively.
\end{corollary}


\subsubsection{-Hyperbolic Graphs.}

A graph has \emph{hyperbolicity}~ if for any four vertices , , , and~, the  two  larger of the sums , , and  differ by at most~.

\begin{lemma}
    [\cite{CheDraEstHab2008}]
    \label{lem:deltaHyper}
Let , , , and  be four vertices in a -hyperbolic graph.
If , then .
\end{lemma}

\begin{lemma}
If  is -hyperbolic, then .
\end{lemma}

\begin{proof}
Consider two vertices  and~ such that  for some vertex~ and shortest path~.
Let  be a vertex such that  and , \ie,  is a middle vertex on the subpath from  to~.

Assume, .
Thus,  and .
Therefore, by Lemma~\ref{lem:deltaHyper}, .
This contradicts that .
Hence, the diameter of a projection is at most  and, therefore, .
\qed
\end{proof}

\begin{corollary}
For -hyperbolic graphs, a minimum eccentricity shortest path can be found in  time.
\end{corollary}

\section{Approximation for Graphs with Bounded Tree-Length and Bounded Hyperbolicity}

In the last sections, we have shown how to find a shortest path with minimum eccentricity~ for several graph classes.
For graphs with tree-length~, this can require up to  time.
In this section, we will show that, for graphs with tree-length~, we can find a shortest path with eccentricity at most  in at most  time and, for graphs with hyperbolicity~, we can find a shortest path with eccentricity at most  in at most ~time.

\begin{lemma}
    \label{lem:mutualDistHyper}
Let  be a graph with hyperbolicity~.
Two vertices  and~ in  with  can be found in  time.
\end{lemma}

\begin{proof}
Let  and~ be two vertices in  such that .
For an arbitrary vertex~ and for , let  be vertices in~ such that  and .
To prove Lemma~\ref{lem:mutualDistHyper}, we will show that there is no vertex~.

Because , .
Therefore, by Lemma~\ref{lem:deltaHyper},  and, thus, .
Since , there is no vertex~ with , otherwise .
Therefore, a vertex pair~ with  can be found in ~time as follows:
Pick an arbitrary vertex~ and find a vertex~ with  using a BFS.
Next, find a vertex~ such that .
Repeat this (at most  times) until .
\qed
\end{proof}

Note that, if a graph has tree-length~, its hyperbolicity is at most ~\cite{CheDraEstHab2008}.
Thus, it follows:

\begin{corollary}
    \label{cor:mutualDistTreeLen}
Let  be a graph with tree-length~.
Two vertices  and~ in  with  can be found in  time.
\end{corollary}

The next lemma will show that, in a graph with bounded tree-length, a shortest path between two mutually furthest vertices gives an approximation for the MESP-problem.

\begin{lemma}
    \label{lem:ApproxTreeLen}
Let  be a graph with tree-length~ having a shortest path with eccentricity~.
Also, let  and~ be two mutually furthest vertices, \ie, .
Each shortest path from  to~ has eccentricity less than or equal to .
\end{lemma}

\begin{proof}
Let  be a shortest path from  to~ with eccentricity~ and  be a shortest path from  to~.
Consider a tree-decomposition~ for  with length~.
We distinguish between two cases:
(1)~There is a bag in~ containing a vertex of  and a vertex of  and (2)~there is no such bag in~.

\paragraph{Case 1: There is a bag in~ containing a vertex of  and a vertex of~.}
We define bags  and  as follows:
Both contain a vertex of~ and a vertex of~,  is a bag closest to a bag containing~,  is a bag closest to a bag containing~, and the distance between  and~ in~ is maximal.
Let  be a subpath of the shortest path from  to~ in~ such that  is a bag closest to a bag containing~,  is a bag closest to a bag containing~,  is adjacent to~ in  (), and the distance~ between  and~ is maximal.
Without loss of generality, let .
Let  be the vertex in  which is closest to~ in~ and let  be the vertex in  which is closest to~ in~.
Figure~\ref{fig:TreeDecoEg} gives an illustration.

\begin{figure}
    [htb]
    \centering
    \includegraphics[]{fig_treeDeco}\caption
    {
        Example for a possible tree-decomposition.
    }
    \label{fig:TreeDecoEg}
\end{figure}

\begin{claim}
For each vertex~ with , .
\end{claim}

\begin{proof}
    [Claim]
There is a vertex set , where  for all positive .
Because  for , .
Thus, because  is a shortest path, for all  with  there is a vertex~ with  such that .
By definition of , each bag~ () contains a vertex~, \ie,  ().
Therefore, for all  with  there is a vertex~ with  such that .
\qedClaim
\end{proof}

Consider an arbitrary vertex~ in~.
Let  be a vertex in~ closest to~ and let  be a shortest path from  to~.
If  is between  and , \ie, , by the claim above, .
If  intersects a bag containing a vertex~, .

Next, consider the case when  does not intersect a bag containing a vertex of  and (without loss of generality) .
In this case, each path from  to~ intersects~.

\begin{claim}
There is a vertex~ such that .
\end{claim}

\begin{proof}
    [Claim]
Let  be a vertex in~ that is closest to~ and let  be a shortest path from  to~.
If  intersects~, there is a vertex  with .

If  does not intersect~, there is a subpath of~ starting at~, containing~, and ending in a vertex~.
Because , .
Therefore, .
\qedClaim
\end{proof}

Let , , and~ be vertices in~ such that ,  is on a shortest path from  to~, and .
Because , .
Also, by the triangle inequality,  and .
Because  and ,  and therefore .

Thus, if there is a bag in~ containing a vertex of~ and a vertex of~,  for all vertices~ in~.

\paragraph{Case 2: There is no bag in~ containing vertices of  and~.}
Because there is no such bag,  contains a bag~ such that each path from  and~ to~ intersects~ and there is a vertex~.

Consider an arbitrary vertex~.
If there is a shortest path~ from  to~ which intersects~, then .
If there is no such path, each path from  to~ intersects~.
Let  be a vertex on a shortest path from  to~ and let  be a vertex on a shortest path from  to~.
Note that .

Because , .
Also, by the triangle inequality,  and .
Because  and ,  and therefore .

Thus, if there is no bag in~ containing vertices of~ and~,  for all vertices~ in~.
\qed
\end{proof}

In~\cite{CheDraEstHab2008}, it was shown that an -vertex -hyperbolic graph has tree-length at most~.

\begin{corollary}
    \label{cor:ApproxHyperbolic}
Let  be a graph with hyperbolicity~ having a shortest path with eccentricity~.
Also, let  and~ be two mutually furthest vertices, \ie, .
Each shortest path from  to~ has eccentricity less than or equal to k + .
\end{corollary}

Lemma~\ref{lem:mutualDistHyper}, Lemma~\ref{lem:ApproxTreeLen}, Corollary~\ref{cor:mutualDistTreeLen}, and Corollary~\ref{cor:ApproxHyperbolic} imply our main result of this section:

\begin{theorem}
Let  be a graph having a shortest path with eccentricity~.
If  has tree-length~, a shortest path with eccentricity at most  can be found in ~time.
If  has hyperbolicity~, a shortest path with eccentricity at most  can be found in ~time.
\end{theorem}

A graph is chordal if and only if it has tree-length~~\cite{Gavril1974}.

\begin{corollary}
If  is a chordal graph and has a shortest path with eccentricity~, a shortest path in~ with eccentricity at most  can be found in linear time.
\end{corollary}

Figure~\ref{fig:chordalEg} gives an example that, for chordal graphs,  is a tight upper bound for the eccentricity of the determined shortest path.

\begin{figure}
    [htb]
    \centering
    \includegraphics[]{fig_chordalEg}\caption
    {
        A chordal graph~.
        A shortest path from  to~ passing  has eccentricity~ which is the minimum for all shortest paths in~.
        The diametral path from  to~ passing  has eccentricity~ because of its distance to~.
    }
    \label{fig:chordalEg}
\end{figure}

\section{Conclusion}

We have investigated the Minimum Eccentricity Shortest Path problem for some structured graph classes.
For these classes, we were able to present linear or polynomial time algorithms.
Additionally, we presented a simple algorithm which gives an additve approximation in linear time for chordal graphs, in  time for graphs with tree-length~, and in  time for graphs with hyperbolicity~.
Table~\ref{tbl:Results} gives an overview of our results.

\begin{table}
    [htb]
    \caption
    {Runtime for solving the Minimum Eccentricity Shortest Path problem for some graph classes. Also, if the solution is not optimal, the maximal difference to an optimal solution is shown.
    }
    \label{tbl:Results}
    \centering
    \begin{tabular}{@{\quad}l@{\quad}||@{\quad}l@{\quad}@{\quad}||@{\quad}l@{\quad}}
        Graph class & Runtime & Approx. \\
        \hline
        \hline
        distance-hereditary & linear \\
        \hline
        chordal &  \\
                & linear &  \\
        \hline
        dually chordal &  \\
        \hline
        tree-length~ &  \\
                              &   &  \\
        \hline
        tree-breadth~ &  \\
        \hline
        -hyperbolic &  \\
                            &  & \\
    \end{tabular}

\end{table}

One reason why the runtime to find an optimal path for distance-hereditary graphs is linear is that we can determine the start and end vertices of an optimal path in linear time for these graphs.
For the other classes, the algorithm iterates over all possible start vertices~.
We know that, for general graphs, the problem remains NP-complete even if a start-end vertex pair is given (see the reduction in~\cite{DrLei2015}).
Also, we have shown that there is a shortest path with minimum eccentricity between every diametral pair of vertices of a distance-hereditary graph (Theorem~\ref{th:dhg}).
This leads to the following question:
How hard is it to determine the start and end vertices of an optimal path?
This question applies to general graphs as well as to special graph classes like chordal graphs.

Another interesting question is, for which other graph classes the problem remains NP-complete or can be solved in polynomial time.
The NP-completeness proof in~\cite{DrLei2015} uses a reduction from SAT.
There is a planar version of 3-SAT (see~\cite{Lichtenstein1982}).
Does this imply that the problem remains NP-complete for planar graphs?

\subsubsection{Acknowledgement:}
This work was partially supported by the NIH grant R01 GM103309.


\begin{thebibliography}{99}

\bibitem{BM-dhg} 
    H.-J. Bandelt and H.M. Mulder, 
    Distance-hereditary graphs, 
    \emph{J. Comb. Theory, Ser. B} 41 (1986) 182--208.

\bibitem{BranChepDrag1999}
    A. Brandstädt, V. Chepoi, and F.F. Dragan,
    Distance approximating trees for chordal and dually chordal graphs.
    \emph{Journal of Algorithms} 30 (1999) 166--184.

\bibitem{BraDraCheVol1998}
    A. Brandstät, F.F. Dragan, V. Chepoi and V. Voloshin,
    Dually chordal graphs,
    \emph{SIAM J. Discrete Math} 11 (1998) 437--455.

\bibitem{BrLeSpinGraphClasses}
    A. Brandstädt, V.B. Le and J. Spinrad,
    Graph Classes: A Survey,
    SIAM, Philadelphia, 1999.

\bibitem{ChangNemhau1984}
     G.J. Chang and G.L. Nemhauser, 
     The k-domination and k-stability problems on sun-free chordal graphs, 
     SIAM J. Algebraic Discrete Math. 5 (1984) 332--345.

\bibitem{ChepoiDragan2000}
    V. Chepoi and F.F. Dragan,
    A note on distance approximating trees in graphs.
    \emph{European Journal of Combinatorics} 21 (2000) 761--766.

\bibitem{CheDraEstHab2008}
    V.D. Chepoi, F.F. Dragan, B. Estellon, M. Habib, and Y. Vaxes,
    Diameters, centers, and approximating trees of -hyperbolic geodesic spaces and graphs.
    \emph{Proceedings of the 24th Annual ACM Symposium on Computational Geometry (SoCG 2008)} (2008) 59--68.

\bibitem{COS-SICOMP}
    D.G.\,Corneil, S.\,Olariu, L.\,Stewart,
    Linear Time Algorithms for Dominating Pairs in Asteroidal Triple-free Graphs,
    \emph{SIAM J. Computing} 28 (1997), 292--302.

\bibitem{D-AM-dhg}
    A.\,D'Atri and M.\,Moscarini,
    Distance-hereditaxy graphs, Steiner trees and connected domination, 
    \emph{SIAM J. Computing} 17 (1988) 521--538.

\bibitem{DeogunKratsch1995}
    J.S.\,Deogun, D.\,Kratsch. 
    Diametral path graphs,
    \emph{Lecture Notes in Computer Science} 1017 (1995) 344--357 

\bibitem{DouDraGavYan2007}
    Y. Dourisboure, F.F. Dragan, C. Gavoille, and C. Yan,
    Spanners for bounded tree-length graphs.
    \emph{Theoretical Computer Science} 383 (2007) 34--44.

\bibitem{DraganKohler2014}
    F.F. Dragan and Ekkehard Köhler,
    An approximation algorithm for the tree t-spanner problem on unweighted graphs via generalized chordal graphs.
    \emph{Algorithmica}, 69 (2014) 884--905.

\bibitem{DKL2014}
    F.F.\,Dragan, E.\,Köhler and A.\,Leitert,
    Line-Distortion, Bandwidth and Path-Length of a Graph,
    SWAT 2014: \emph{14th Scandinavian Symposium and Workshops on Algorithm Theory}, July 2--4 2014. Copenhagen, Denmark, \emph{Lecture Notes in Computer Science} 8503, 2014, pp. 146--257.

\bibitem{DrLei2015}
    F.F.\,Dragan and A.\,Leitert,
    On the Minimum Eccentricity Shortest Path Problem,
    WADS 2015, \emph{Lecture Notes in Computer Science} 9214 (2015) 276--288.

\bibitem{DraganLeiter2015}
    F.F.\,Dragan and A.\,Leitert,
    Minimum Eccentricity Shortest Paths in some Structured Graph Classes,
    WG 2015, To appear in \emph{Lecture Notes in Computer Science} (2015).

\bibitem{DrNi} 
    F.F.\,Dragan and F.\,Nicolai,
    LexBFS-orderings of distance-hereditary graphs with application to the diametral pair problem, 
    \emph{Discrete Appl. Math.} 98 (2000), 191--207.

\bibitem{FaberJamison1986}
    M.\,Farber and R.\,E.\,Jamison,
    Convexity in graphs and hypergraphs,
    \emph{SIAM Journal on Algebraic and Discrete Methods} 7 (1986), 433–444.

\bibitem{Gavril1974}
     F. Gavril,
     The intersection graphs of subtrees in trees are exactly the chordal graphs,
     \emph{Journal of Combinatorial Theory} 16 (1974), 47--56.

\bibitem{howorka} 
    E.\,Howorka,
    A characterization of distance--hereditary graphs, 
    \emph{Quart. J. Math. Oxford Ser.} 2, 28 (1977) 417--420.

\bibitem{Kratsch2000}
    D.\,Kratsch. 
    Domination and total domination on asteroidal triple-free graphs, 
    \emph{Discrete Applied Mathematics} 99 (2000) 111--123

\bibitem{Lichtenstein1982}
    D.\,Lichtenstein,
    Planar formulae and their uses,
    \emph{SIAM Journal of Computing} 11 (1982), 329--343.

\bibitem{Lokshtanov2010}
    D. Lokshtanov,
    On the complexity of computing treelength.
    \emph{Discrete Applied Mathematics}, 158 (2010), 820--827.

\bibitem{haiko}
    H.\,Müller,
    Hamiltonian circuits in chordal bipartite graphs,
    \emph{Discrete Mathematics} 156 (1996), 291--298.

\bibitem{RobertSeymou1986}
    N. Robertson and P.D. Seymour,
    Graph minors. II. Algorithmic aspects of tree-width,
    \emph{J. Algorithms} 7 (1986) 309--322.

\bibitem{Slater}
    P.J.\,Slater,
   Locating central paths in a graph,
    \emph{Transportation Science} 16 (1982), 1--18.

\bibitem{TarjanYannak1984}
    R.E. Tarjan and M. Yannakakis,
    Simple linear-time algorithms to test chordality of graphs, test acyclicity of hypergraphs, and selectively reduce acyclic hypergraphs,
    \emph{SIAM Journal on Computing} 13 (1984), 566--579.

\end{thebibliography}

\end{document}