\documentclass{article}
\pdfoutput=1
\usepackage[utf8]{inputenc}
\usepackage{subfig,ifthen,graphicx,enumerate}
\usepackage{amsmath,amsthm,amssymb}
\usepackage{xspace,paralist}
\usepackage{calc}
\usepackage{aliascnt,hyperref}
\usepackage{cleveref}
\usepackage[round]{natbib}
\let\cite\citep


\ifx\pgfversion\undefined
  \input{pgfexternal.tex}
\else

  \usetikzlibrary{shapes.geometric}
  \usetikzlibrary{shapes.callouts}
  \usetikzlibrary{calc,fit}
  \pgfdeclarelayer{background}
  \pgfsetlayers{background,main}

  \tikzstyle{commgraphnode}=[circle,minimum size=20pt]
  \tikzstyle{commgraphedge}=[]
  \tikzstyle{commgraphhyperarc}=[ellipse,fill=gray,opacity=.3,inner sep=0pt]

  \ifthenelse{\lengthtest{\pgfversion cm<1.18cm}}{
  }{
    \ifx\regeneratepgf\undefined\pgfrealjobname{\jobname}
    \else\pgfrealjobname{\regeneratepgf}
    \fi
  }
\fi

\title{Common Knowledge in Interaction Structures}
\author{Krzysztof R. Apt and Andreas Witzel and Jonathan A. Zvesper} 

\ifx\pdfoutput\undefined\else\makeatletter\pdfinfo{
  /Title (\@title)
  /Author (\@author)
}\makeatother\fi

\ifx\crefformat\undefined
\newcommand{\crefformats}[7]{}
\else
\newcommand{\crefformats}[7]{\crefformat{#3}{##2\ifthenelse{\equal{#4}{}}{}{#4~}#1##1#2##3}
  \Crefformat{#3}{##2\ifthenelse{\equal{#5}{}}{}{#5~}#1##1#2##3}
  \crefrangeformat{#3}{\ifthenelse{\equal{#6}{}}{}{#6~}##3#1##1#2##4--##5#1##2#2##6}
  \Crefrangeformat{#3}{\ifthenelse{\equal{#7}{}}{}{#7~}##3#1##1#2##4--##5#1##2#2##6}
  \crefmultiformat{#3}{\ifthenelse{\equal{#6}{}}{}{#6~}##2#1##1#2##3}{ and~##2#1##1#2##3}{, ##2#1##1#2##3}{ and~##2#1##1#2##3}
  \Crefmultiformat{#3}{\ifthenelse{\equal{#7}{}}{}{#7~}##2#1##1#2##3}{ and~##2#1##1#2##3}{, ##2#1##1#2##3}{ and~##2#1##1#2##3}
  \crefrangemultiformat{#3}{\ifthenelse{\equal{#6}{}}{}{#6~}##2#1##1#2##3}{ and~##2#1##1#2##3}{, ##2#1##1#2##3}{ and~##2#1##1#2##3}
  \Crefrangemultiformat{#3}{\ifthenelse{\equal{#7}{}}{}{#7~}##2#1##1#2##3}{ and~##2#1##1#2##3}{, ##2#1##1#2##3}{ and~##2#1##1#2##3}
}

\makeatletter
\@ifclassloaded{llncs}{
  \crefformats{}{}{section}{Sect.}{Section}{Sects.}{Sections}
  \crefformats{}{}{appendix}{App.}{Appendix}{App.}{Appendices}
  \crefformats{}{}{subsection}{Sect.}{Section}{Sects.}{Sections}
  \crefformats{}{}{subsubsection}{Sect.}{Section}{Sects.}{Sections}
  \crefformats{}{}{chapter}{Chap.}{Chapter}{Chaps.}{Chapters}
  \crefformats{}{}{part}{Part}{Part}{Parts}{Parts}
  \crefformats{}{}{figure}{Fig.}{Figure}{Figs.}{Figures}
  \crefformats{}{}{table}{Table}{Table}{Tables}{Tables}
}{
   \crefformats{}{}{section}{Section}{Section}{Sections}{Sections}
   \crefformats{}{}{appendix}{Appendix}{Appendix}{Appendices}{Appendices}
   \crefformats{}{}{subsection}{Section}{Section}{Sections}{Sections}
   \crefformats{}{}{subsubsection}{Section}{Section}{Sections}{Sections}
   \crefformats{}{}{chapter}{Chapter}{Chapter}{Chapters}{Chapters}
   \crefformats{}{}{part}{Part}{Part}{Parts}{Parts}
   \crefformats{}{}{figure}{Figure}{Figure}{Figures}{Figures}
   \crefformats{}{}{table}{Table}{Table}{Tables}{Tables}
}
\makeatother
\fi

\ifx\finalversion\undefined
\newcommand{\marginlabel}[2]{\mbox{}\marginpar[\raggedleft\hspace{0pt}#1]{\raggedright\hspace{0pt}#2}}
\ifx\tikzstyle\undefined
\newcommand{\todoar}[2][]{\todo[#1]{#2}}
\else\tikzstyle{todoarrow}=[opacity=0.4,gray,-stealth]
\newcommand{\todoar}[2][]{\marginlabel{\small #2}{\tikz[remember picture,overlay,baseline=(todoarrowstart.220)]\node(todoarrowstart){}; \small #2}\ifthenelse{\equal{#1}{}}{}{{\color{red}[}#1{\color{red}]}}\tikz[remember picture,overlay]\node[inner sep=2pt](todoarrowend){};\tikz[remember picture,overlay]\path(todoarrowstart)edge[todoarrow,out=190,in=-45](todoarrowend);}
\fi
\newcommand{\todo}[2][]{\marginlabel{{\small #2} }{ \small #2}\ifx\color\undefined \ifthenelse{\equal{#1}{}}{}{{[}#1{]}}\else \ifthenelse{\equal{#1}{}}{}{{\color{red}[}#1{\color{red}]}}\fi }
\fi

\newcounter{autoexternalpgf}
\setcounter{autoexternalpgf}{-1}
\newenvironment{exttikzpicture}[1][]{\addtocounter{autoexternalpgf}{1}\beginpgfgraphicnamed{autoexternalpgf-\arabic{autoexternalpgf}}\begin{tikzpicture}[#1]}{\end{tikzpicture}\endpgfgraphicnamed}
 

\newcommand{\NAT}{\ensuremath{\mathbb{N}}\xspace}
\newcommand{\NATp}{\ensuremath{\mathbb{N}^+}\xspace}
\newcommand{\REAL}{\ensuremath{\mathbb{R}}\xspace}
\newcommand{\abs}[1]{\lvert#1\rvert}
\newcommand{\suchthat}{\,|\,}
\newenvironment{mainclaim}{\begin{center}}{\end{center}}





\makeatletter
\@ifclassloaded{beamer}{}{\newcommand{\newtheoremwithalias}[3]{\ifx\newaliascnt\undefined
    \newcounter{#1}
    \else
    \newaliascnt{#1}{#2}
    \fi
    \newtheorem{#1}[#1]{#3}
    \ifx\aliascntresetthe\undefined\else
    \aliascntresetthe{#1}
    \fi
  }
  \ifx\creflastconjunction\undefined\else
  \renewcommand{\creflastconjunction}{ and }
  \fi

  \@ifclassloaded{llncs}{
    \if@envcntsame\errmessage{cleveref naming doesn't work because no aliascntrs used in llncs.cls}
\else\if@envcntsect
    \newtheorem{observation}{Observation}[section]
    \newtheorem{fact}{Fact}[section]
    \else
    \newtheorem{observation}{Observation}
    \newtheorem{fact}{Fact}
    \fi\fi
    \newcommand{\qedhere}{\qed}
    \crefformats{(}{)}{equation}{}{Equation}{}{Equations}
  }{
    \newtheorem{theorem}{Theorem}[section]
    \newtheoremwithalias{lemma}{theorem}{Lemma}
    \newtheoremwithalias{proposition}{theorem}{Proposition}
    \newtheoremwithalias{corollary}{theorem}{Corollary}
    \newtheoremwithalias{observation}{theorem}{Observation}
    \newtheoremwithalias{fact}{theorem}{Fact}
    \newtheoremwithalias{conjecture}{theorem}{Conjecture}
    \newtheoremwithalias{numberednote}{theorem}{Note}
    \newtheoremwithalias{numberedremark}{theorem}{Remark}

    \ifx\theoremstyle\undefined\else\theoremstyle{remark}\fi
    \newtheorem*{note}{Note}
    \newtheorem*{remark}{Remark}

    \ifx\theoremstyle\undefined\else\theoremstyle{definition}\fi
    \newtheoremwithalias{definition}{theorem}{Definition}
    \newtheoremwithalias{example}{theorem}{Example}

    \crefformats{(}{)}{equation}{Equation}{Equation}{Equations}{Equations}

}

  \crefformats{}{}{theorem}{Theorem}{Theorem}{Theorems}{Theorems}
  \crefformats{}{}{lemma}{Lemma}{Lemma}{Lemmas}{Lemmas}
  \crefformats{}{}{proposition}{Proposition}{Proposition}{Propositions}{Propositions}
  \crefformats{}{}{corollary}{Corollary}{Corollary}{Corollaries}{Corollaries}
  \crefformats{}{}{observation}{Observation}{Observation}{Observations}{Observations}
  \crefformats{}{}{fact}{Fact}{Fact}{Facts}{Facts}
  \crefformats{}{}{conjecture}{Conjecture}{Conjecture}{Conjectures}{Conjectures}
  \crefformats{}{}{numberednote}{Note}{Note}{Notes}{Notes}
  \crefformats{}{}{numberedremark}{Remark}{Remark}{Remarks}{Remarks}
  \crefformats{}{}{definition}{Definition}{Definition}{Definitions}{Definitions}
  \crefformats{}{}{example}{Example}{Example}{Examples}{Examples}
}
\makeatother



 \clearpage{}

\newcommand{\Searrow}{\begin{turn}{45}    \end{turn}}
\newcommand{\Swarrow}{\begin{turn}{135}   \end{turn}}
\newcommand{\myparpic}[1]{\parpic{{\darkgray{#1}}}}
\newcommand{\bfe}[1]{\begin{bfseries}\emph{#1}\end{bfseries}\index{#1}}
\newcommand{\oldbfe}[1]{\begin{bfseries}\emph{#1}\end{bfseries}}
\newcommand{\inv}{\invisible}
\newcommand{\IF}{\mbox{{\bf if}\ }}
\newcommand{\FI}{\mbox{{\bf fi}}}
\newcommand{\DO}{\mbox{{\bf do}\ }}
\newcommand{\OD}{\mbox{{\bf od}}}
\newcommand{\WHILE}{\mbox{{\bf while}\ }}
\newcommand{\END}{\mbox{{\bf end}}}
\newcommand{\THEN}{\mbox{\ {\bf then}\ }}
\newcommand{\ELSE}{\mbox{\ {\bf else}\ }}
\newcommand{\T}{\mbox{{\bf true}}}
\newcommand{\F}{\mbox{{\bf false}}}
\newcommand{\FAIL}{{\bf fail}}
\newcommand{\ES}{\mbox{}}

\newcommand{\myla}{\mbox{}}
\newcommand{\myra}{\mbox{}}
\newcommand{\da}{\mbox{}}
\newcommand{\ua}{\mbox{}}
\newcommand{\La}{\mbox{}}
\newcommand{\Ra}{\mbox{}}
\newcommand{\tra}{\mbox{}}
\newcommand{\lra}{\mbox{}}
\newcommand{\bu}{\mbox{}}
\newcommand{\up}{\mbox{}}

\newcommand{\A}{\mbox{}}
\newcommand{\bigA}{\mbox{}}
\newcommand{\Orr}{\mbox{}}
\newcommand{\U}{\mbox{}}
\newcommand{\I}{\mbox{}}
\newcommand{\sse}{\mbox{}}

\newcommand{\po}{\mbox{}}
\newcommand{\spo}{\mbox{}}
\newcommand{\rpo}{\mbox{}}
\newcommand{\rspo}{\mbox{}}


\newcommand{\Mo}{\mbox{}}
\newcommand{\mtwo}{\mbox{}}
\newcommand{\mthree}{\mbox{}}
\newcommand{\Mf}{\mbox{}}
\newcommand{\Mt}{\mbox{}}
\newcommand{\Mwt}{\mbox{}}
\newcommand{\Mp}{\mbox{}}

\newcommand{\PR}{\mbox{}}

\newcommand{\fa}{\mbox{}}
\newcommand{\te}{\mbox{}}
\newcommand{\fai}{\mbox{}}
\newcommand{\tei}{\mbox{}}

\newcommand{\calA}{\mbox{}}
\newcommand{\calG}{\mbox{}}
\newcommand{\calB}{\mbox{}}

\newcommand{\LLn}{\mbox{}}
\newcommand{\LL}{\mbox{}}

\newcommand{\WP}{deterministic program}
\newcommand{\WPs}{deterministic programs}
\newcommand{\AW}{\mbox{{\bf await}-statement}}
\newcommand{\AWs}{\mbox{{\bf await}-statements}}

\newcommand{\IFP}{\mbox{}}
\newcommand{\DOP}{\mbox{}}
\newcommand{\DIP}{\mbox{}}
\newcommand{\IFPa}{\mbox{}}
\newcommand{\DOPa}{\mbox{}}

\newcommand{\MS}[1]{\mbox{}}
\newcommand{\newMS}[1]{\mbox{}}
\newcommand{\newhat}[1]{\langle {#1} \rangle}
\newcommand{\M}[2]{\mbox{}}
\newcommand{\MF}[1]{\mbox{}}
\newcommand{\MT}[1]{\mbox{}}
\newcommand{\MP}[1]{\mbox{}}
\newcommand{\MWT}[1]{\mbox{}}
\newcommand{\NS}[1]{\mbox{}}

\newcommand{\ITE}[3]{\mbox{}}
\newcommand{\IT}[2]{\mbox{}}
\newcommand{\WD}[1]{\mbox{}}
\newcommand{\WDD}[2]{\mbox{}}
\newcommand{\W}[1]{\mbox{}}
\newcommand{\AT}[1]{\mbox{}}
\newcommand{\ATE}[2]{\mbox{}}
\newcommand{\ATOM}[1]{\mbox{}}
\newcommand{\RU}[2]{\mbox{}}

\newcommand{\TF}[1]{\mbox{}}
\newcommand{\TFn}{\mbox{}}

\newcommand{\HT}[3]{\mbox{}}

\newcommand{\sg}{{\mbox{}}}
\newcommand{\Sg}{{\mbox{}}}
\newcommand{\om}{{\mbox{}}}
\newcommand{\al}{{\mbox{}}}
\newcommand{\ba}{{\mbox{}}}

\newcommand{\B}[1]{\mbox{}}       \newcommand{\C}[1]{\mbox{}}           

\newcommand{\UPDATE}{\mbox{}}
\newcommand{\INIT}{\mbox{}}
\newcommand{\SCH}{\mbox{}}
\newcommand{\FAIR}{\mbox{}}
\newcommand{\RORO}{\mbox{}}
\newcommand{\QUEUE}{\mbox{}}
\newcommand{\INV}{\mbox{}}
\newcommand{\VAR}{\mbox{\it Var}}
\newcommand{\unfold}{\mbox{\it unfold}}
\newcommand{\fold}{\mbox{\it fold}}
\newcommand{\ol}[1]{\mbox{}}
\newcommand{\NI}{\noindent}
\newcommand{\HB}{\hfill{}}
\newcommand{\VV}{\vspace{5 mm}}
\newcommand{\III}{\vspace{3 mm}}
\newcommand{\II}{\vspace{2 mm}}
\newcommand{\PP}{\mbox{}}

\newcommand{\X}[1]{\mbox{}}
\newcommand{\lX}[1]{\mbox{}}





\newenvironment{mydef}{\begin{boxitpara}{box 0.95 setgray fill}}{\end{boxitpara}}

\newcommand{\vect}[1]{{\bf #1}}
\newcommand{\D}[1]{\mbox{}}           \newcommand{\hb}[1]{{\Theta_{#1}}}
\newcommand{\hi}{{\cal I}}
\newcommand{\ran}{\rangle}
\newcommand{\lan}{\langle}
\newcommand{\dom}{{\it Dom}}
\newcommand{\var}{{\it Var}}
\newcommand{\pred}{{\it pred}}
\newcommand{\sem}[1]{{\mbox{}}}
\newcommand{\false}{{\it false}}
\newcommand{\true}{{\it true}}
\newtheorem{Property}{Property}[section]
\newcommand{\hsuno}{\hspace{ .5in}}
\newcommand{\vsuno}{\vspace{ .25in}}
\newcommand{\cons}{}
\newcommand{\nil}{[\,]}
\newcommand{\restr}[1]{\! \mid \! {#1}} 
\newcommand{\range}{{\it Ran}}
\newcommand{\Range}{{\it Range}}
\newcommand{\PC}{\mbox{}}
\newcommand{\TC}{\mbox{}}
\newcommand{\IO}{\mbox{}}
\newcommand{\size}{{\rm size}}
\newcommand{\Der}[2]
        {\; |\stackrel{#1}{\!\!\!\longrightarrow _{#2}}\; }

\newcommand{\Sder}[2]
        {\; |\stackrel{#1}{\!\leadsto_{#2}}\; }



\newcommand{\szkew}[1]{\relax \setbox0=\hbox{\kern -24pt \kern 0pt }\box0}
{\catcode`\@=11 \global\let\ifjusthvtest@=\iffalse}


\newcommand{\iif}{\mbox{}}
\newcommand{\longra}{\mbox{}}
\newcommand{\Longra}{\mbox{}}
\newcommand{\res}[3]{\mbox{}}
\newcommand{\reso}[4]{\mbox{
}}
\newcommand{\nextres}[2]{\mbox{}}
\newcommand{\preres}[2]{\mbox{}}
\newcommand{\Preres}[2]{\mbox{}}

\newcounter{oldmycaption}
\newcommand{\oldmycaption}[1]
{\begin{center}\addtocounter{oldmycaption}{1}
{{\bf Program \theoldmycaption} : #1}\end{center}} 

\newcommand{\mycite}[1]{\cite{#1}\glossary{#1}}

\newcommand{\mycaption}[1]
{\begin{center}{{\bf Program:} }
\index{#1} \end{center}} 

\newcommand{\ass}{{\cal A}}
\newcommand{\defi}{{\stackrel{\rm def}{=}}}





\newcommand{\Seq}[1]{{\bf #1}}
\newcommand{\BSeq}[2]{{{#1}_1},\ldots,{{#1}_{#2}}}
\newcommand{\OSeq}[2]{{{#1}_0},\ldots,{{#1}_\alpha},\ldots}
\newcommand{\Var}[1]{{\it Var}({#1})}
\newcommand{\FreeVar}[1]{{\it FreeVar}({#1})}
\newcommand{\Dom}[1]{{\it Dom}({#1})}
\newcommand{\Ran}[1]{{\it Ran}({#1})}
\newcommand{\Neg}{\neg}
\newcommand{\Fail}{{\it fail}}
\newcommand{\Success}{\Box}
\newcommand{\Flounder}{{\it flounder}}
\newcommand{\Prune}{{\it prune}}
\newcommand{\cut}{\mbox{\it cut}}


\newcommand{\p}[2]{\langle #1 \ ; \ #2 \rangle}

\newcommand{\ceiling}[1]{\lceil #1 \rceil}
\newcommand{\adjceiling}[1]{\left \lceil #1 \right \rceil}
\newcommand{\floor}[1]{\lfloor #1 \rfloor}
\newcommand{\adjfloor}[1]{\left \lfloor #1 \right \rfloor}




\newcommand{\mytitle}[1]
{\begin{center}\begin{Sbox} {\bf #1}\end{Sbox}\shadowbox{\TheSbox}\end{center}}

\newcommand{\myctitle}[1]
{\textRed\begin{center}\begin{Sbox} {\bf #1}\end{Sbox}\shadowbox{\TheSbox}\end{center}\textBlack}













%
\clearpage{}

\title{Common Knowledge in Interaction Structures\footnote{To appear in Proceedings of TARK 2009}}
\author{Krzysztof R. Apt
\and
Andreas Witzel
\and
Jonathan A. Zvesper
\
M_w &:= \C{(\cdot,A,\cdot) \in M \mid \setof w\subseteq A}\\
\Facts(M) &:= \C{p \mid (\cdot, \cdot,p) \in M}.

 \varphi ::= p \mid \neg \varphi \mid \varphi \land \varphi \mid \varphi \lor \varphi \mid \ck G \varphi,

  \state  &\vDash_H p                   && \iff p \in V, \\
  \state  &\vDash_H \neg \varphi            && \iff \state  \nvDash_H \varphi, \\
  \state  &\vDash_H \varphi \lor \psi       && \iff \state  \vDash_H \varphi\text{ or }\state  \vDash_H \psi, \\
  \state  &\vDash_H \varphi \land \psi      && \iff \state  \vDash_H \varphi\text{ and }\state  \vDash_H \psi, \\
  \state  &\vDash_H \ck G \varphi             && \iff
  \begin{aligned}[t]
    &\state['] \vDash_H \varphi\\
    &\text{for each -compliant \state[']}\\
    &\text{with } \state \sim_{G} \state[']\enspace.
  \end{aligned}

  \label{equ:K}\state \vDash \ck G \varphi\text{ iff }\state \vDash \knows w \varphi\text{ for all }w\in G^*\enspace.
  \tag{}

  \state\vDash_H\knows i\neg \knows j p\enspace.
  
  \state\vDash_H\ck G\neg \knows j p\enspace,
  
    \textup{if } \state\vDash_H\varphi, \textup{ then } \state['] \vDash_H\varphi.

  \state['''] := (V_i\cup\textstyle\bigcup_{j\neq i}V_j'',M_i)\enspace.
  
\mbox{ iff .}

  M'\restriction_H:=\{\msg{\cdot}{A}{\cdot} \in M' \mid A\in H\}\enspace.
  
\mbox{ iff .}

    \state\sim_i \state[']&\text{ and }\state[']\nvDash\varphi_1\enspace\text{, as well as}\\
    \state\sim_i\state['']&\text{ and }\state['']\nvDash\varphi_2\enspace.
  
  \state[''']:=( V_i\cup\textstyle\bigcup_{j\neq i}(V_j'\cap V_j''),M_i)\enspace.
  
    \Facts(M''')&\subseteq\Facts(M)\cap\Facts(M')\cap\Facts(M'')\\
    &\subseteq V\cap V'\cap V''\\
    &=\textstyle\bigcup_{i\in N}(V_i\cap V_i'\cap V_i'')\\
    &\subseteq V'''\enspace.
  
\mbox{ iff .}

    \state&=(\{p\},\{\msg{k}{\{i,k\}}{p}\})\\
    \varphi&=\knows i(\knows j p\vee\neg(\knows j p\vee \knows j\neg p))\enspace,
  
\mbox{ iff  and .}

  \state \vDash_H \knows i \neg \knows k p
  
  \state \nvDash_H \knows i \neg \knows k p\enspace.
  
cl(m) := \C{m' \in M \mid m \leadsto^{*} m'}.

\mbox{ iff there is  with .}

\mbox{ iff .}

\mbox{ iff .}

      \state \sim_i L(V_i, M_i).

  \mbox{ iff .}
  
\state[''] := L(V_i, M_i)

    \state['']\nvDash_H \textstyle\bigvee_{j=1}^{k} p_j.

    \textup{if } (V', M')\vDash_H\varphi, \textup{ then } \state \vDash_H\varphi.

  \state \nvDash \knows i \knows k p,
  
  \state \vDash_H \knows i \knows k p,
  
\mbox{ iff .}

  \state\vDash \ck G \psi&\iff\state\vDash\textstyle\bigwedge_{j=1}^{k} \bigvee_{l=1}^{m_j} \ck G p_{j,l}\\
  \intertext{and}
  \state\vDash_H \ck G \psi&\iff\state\vDash_H\textstyle\bigwedge_{j=1}^{k} \bigvee_{l=1}^{m_j} \ck G p_{j,l}\enspace.\\
  \intertext{But by \cref{result:ck-of-h-doesnt-matter-for-facts}, for all  and  we have}
  \state\vDash \ck G p_{j,l}&\iff\state\vDash_H \ck G p_{j,l}\enspace.

  \mbox{ iff .}
  
  M =
  \C{(n,\C{n,k},p), (k,\C{k,j},p), (j,\C{j,i},p)}\enspace.
  
  \state \vDash_H \knows i(\knows k p \vee \knows l p),
  
  but neither
  
  nor
  
  holds. Informally, player~ knows that either player~ or player~ knows~
  but he does not know which one of them knows~.
\end{example}

As noticed already after the proof of \cref{result:ck-disjunction-distributes},
the  operator does not distribute over negation either;
the same example applies here.

Finally, reconsider \cref{thm:permutation}. It is straightforward to see that it does not hold in 
the present setting, even for two players. Indeed, reconsider \cref{ex:ck-of-h-does-matter}.
We showed there that . However, it is easy to see that 
 since . 

\section{Conclusions and related work}
\label{sec:conclusions}

In this paper we studied various aspects of common knowledge in two
simple frameworks concerned with synchronous communication. It is
useful to clarify that our two impossibility results concerning
the attainment of common knowledge amongst players
(\cref{result:knowledge-chain-phi-only-through-msg,thm:group1})
differ from the customary impossibility results.

For example, \citet{HM90} formalize the epistemic aspects of the
celebrated Coordinated Attack Problem that consists in
achieving common knowledge (a `common plan of action').
They show (in Section~8) that in a
distributed system in which communication is not guaranteed, common
knowledge is not attainable. When communication is guaranteed, they
show the same result when there is no bound on message delivery
times. In both situations the proof assumes the existence of clocks and
point-to-point communication.

The close correspondence between simultaneous events (in our system a
broadcast to the whole group) and common knowledge is pointed out by
\citet{FHMV99}. Their model of a distributed system consists of a
set of linear `runs' (histories), while we only assume a partial
ordering () between messages broadcast to groups, which are
the only possible actions.
We have shown that in our framework, common knowledge of a positive formula
is indeed inseparably related to group communication, which corresponds to simultaneous events.
However, as we have seen, this does not hold of negative formulas,
so the relationship is not as obvious as it may seem.
The results of \citet{FHMV99} may be seen to correspond to our \cref{cor:iff},
though we allow broadcasts instead of just point-to-point communication.

\citet{chandy_processes_1986} consider the flow of information
in distributed systems with asynchronous communication.
They study how processes `learn' about states of other processes and how knowledge evolves.
The main difference is that with asynchronous communication,
hypergraphs are equivalent to mere point-to-point graphs.
Without guarantees on the delivery time, and without temporal reasoning,
from the knowledge point of view
sending an asynchronous group message has the same effect as
sending a separate message to each group member.

Our study concerning the consequences of the assumption whether the underlying
hypergraph is commonly known among the players brings our paper
somewhat closer to the area of social networks
(see, e.g., \citet{Jac08}).
Within logic, the relevance of epistemic issues in communication networks
has been recognized by a number of authors, e.g.~\citet{van_benthem_one_2006_}.
However, to our knowledge the only work that addresses these issues
is \citet{pacuit_reasoning_2007} and, to some extent,~\citet{roelofsen_exploring_2005}.
We now briefly discuss these frameworks and relate them to our own.

\Citet{pacuit_reasoning_2007} use a history-based model
to study diffusion of information in a communication graph,
starting from facts initially known to individual players.
Communicative acts are assumed to consist in
a player~ `reading' an arbitrary propositional formula from another player~,
with the precondition that~ \emph{knows} that the formula holds.
Communicative acts are restricted to a commonly known, static, directed graph,
and, unlike in our case, are assumed to go \emph{unnoticed by~}.
The paper formalizes what conclusions,
beyond the mere factual content of messages,
can be drawn using knowledge of the communication graph and, consequently,
knowledge of the possible routes along which certain information can have flown.

\Citet{roelofsen_exploring_2005} uses a model based on Dynamic
Epistemic Logic (DEL) to describe how some initial epistemic
model evolves in a communication situation.  Communication is
among subgroups and can contain arbitrary epistemic formulas.
Further, communication is assumed to be truthful and is restricted to
occur along a hypergraph.
However the hypergraph is explicitly encoded in the model, and thus
(knowledge of it) is subject to change.


While under certain circumstances history-based modeling and DEL are
equivalent~\cite{van_benthem_merging_2007}, our approach is more
in the spirit of~\citet{pacuit_reasoning_2007}.
Indeed, we also study how specific information may have spread.
Also, all possible communications are included in the model
and suspicions about them are not explicitly formed.
Finally, the underlying graph (in our case hypergraph) is static
and not included in the model.

On a technical level, our approach differs from~\citet{pacuit_reasoning_2007}
in that we use sets of messages instead of sequences
and, when dealing with forwarding, employ a more general structure than histories
by considering messages partially ordered by the relation .
On the other hand, our messages are simpler:
\citet{pacuit_reasoning_2007} allow disjunctions of facts, while we allow only facts.

What distinguishes our approach on a more conceptual level
is that our focus lies on identifying natural conditions that allow us to
prove stronger results about knowledge, such as distributivity over
disjunctions, or irrelevance of (common) knowledge of the underlying
hypergraph.



\section{Extensions}
\label{sec:extensions}













We conclude by listing a number of natural extensions of the considered framework 
that are worthy of further study:
\begin{itemize}

\item We could equip the players with theories that their parts of
  valuations, , have to satisfy.  In this extension we would
  assume that each player  has a propositional theory 
  built from facts in  that he adheres to. The theories 
  where  then form a common knowledge among the players. So
  each player  can assume that player  considers  such that
   is a model of .

\item We could consider more complex messages than simple atomic
facts, for example propositional formulas, or even
epistemic formulas.
Also, we could study asynchronous communication,
messages from unknown senders or to an unknown group of recipients,
and a counterpart of the blind copy feature familiar from e-mails.
  
\item In \cref{sec:forwarding} we relaxed the assumption that in a
  message  it has to be the case that , but we
  did still insist on the \emph{truthfulness} of messages, requiring
  that .  We could further relax this assumption, by
  insisting only that . This way we would model messages
  that consist of possibly false (but credible) information.  This
  would lead to a study of beliefs (which can be false) rather than
  knowledge (which cannot) and common beliefs rather than common
  knowledge.
  
\item We could consider in this framework belief revision, by assuming
  that the theory  of player  consists of his beliefs, which
  would then be revised in view of received information.
  Alternatively,  could be the certain knowledge of player 
  against which received information would be revised.
  
\item We could assume that the players have different knowledge of the
  underlying hypergraph, by assuming that for all  we have , where  is the underlying hypergraph and  is its
  approximation known to , and that players learn  by exchanging
  messages.  The messages would contain information about which
  hyperarcs do \emph{not} belong to .
\item Alternatively, we could study a setup in which each player has
an indistinguishability relation over hypergraphs. This
would allow us to model players' partial knowledge of the
underlying hypergraph.
\end{itemize}



We use the setting of the first item in~\cite{apt_strategy_2009} to reason
about iterated elimination of strategies in \oldbfe{strategic games with interaction structures}.
These are strategic games in which there is a hypergraph over the set
of players (an interaction structure) and
the players can communicate about their preferences, initially only known to themselves,
so that within each hyperarc players can obtain common knowledge of each other's
preferences.






\section*{Acknowledgements}
\label{sec:acknowledgements}

We thank Rohit Parikh, Willemien Kets, Aaron Ar\-cher, Henry Landau,
and three anonymous referees for discussion and helpful suggestions.
The second and third authors were supported by a GLoRiClass fellowship
funded by the European Commission (Early Stage Research Training
Mono-Host Fellowship MEST-CT-2005-020841).



\begin{thebibliography}{11}
\providecommand{\natexlab}[1]{#1}
\providecommand{\url}[1]{\texttt{#1}}
\expandafter\ifx\csname urlstyle\endcsname\relax
  \providecommand{\doi}[1]{doi: #1}\else
  \providecommand{\doi}{doi: \begingroup \urlstyle{rm}\Url}\fi

\bibitem[Apt et~al.(2009)Apt, Witzel, and Zvesper]{apt_strategy_2009}
K.~R. Apt, A.~Witzel, and J.~A. Zvesper.
\newblock Strategy elimination in games with interaction structures.
\newblock Manuscript, 2009.

\bibitem[Chandy and Misra(1986)]{chandy_processes_1986}
K.~M. Chandy and J.~Misra.
\newblock How processes learn.
\newblock \emph{Distributed Computing}, 1\penalty0 (1):\penalty0 40--52, Mar.
  1986.

\bibitem[Fagin et~al.(1995)Fagin, Halpern, Vardi, and Moses]{FHMV_RAK}
R.~Fagin, J.~Halpern, M.~Vardi, and Y.~Moses.
\newblock \emph{Reasoning about knowledge}.
\newblock MIT Press, Cambridge, MA, USA, 1995.

\bibitem[Fagin et~al.(1999)Fagin, Halpern, Moses, and Vardi]{FHMV99}
R.~Fagin, J.~Y. Halpern, Y.~Moses, and M.~Y. Vardi.
\newblock Common knowledge revisited.
\newblock \emph{Annals of Pure and Applied Logic}, 96\penalty0 (1--3):\penalty0
  89--105, 1999.

\bibitem[Halpern and Moses(1990)]{HM90}
J.~Y. Halpern and Y.~Moses.
\newblock Knowledge and common knowledge in a distributed environment.
\newblock \emph{Journal of the ACM}, 37\penalty0 (3):\penalty0 549--587, 1990.

\bibitem[Jackson(2008)]{Jac08}
M.~O. Jackson.
\newblock \emph{Social and Economic Networks}.
\newblock Princeton University Press, Princeton, 2008.

\bibitem[Pacuit and Parikh(2007)]{pacuit_reasoning_2007}
E.~Pacuit and R.~Parikh.
\newblock Reasoning about communication graphs.
\newblock In \emph{Interactive Logic}, volume~1 of \emph{Texts in Logic and Games}, pages
  135--157, London, 2007. Amsterdam University Press.

\bibitem[Roelofsen(2005)]{roelofsen_exploring_2005}
F.~Roelofsen.
\newblock Exploring logical perspectives on distributed information and its
  dynamics.
\newblock Master's thesis, ILLC, University of Amsterdam, 2005.

\bibitem[van Benthem(2006)]{van_benthem_one_2006_}
J.~van Benthem.
\newblock One is a lonely number: Logic and communication.
\newblock In \emph{Logic Colloquium 2002}, volume~27 of \emph{Lecture Notes in Logic}, pages 96--129.
  A~K~Peters, 2006.

\bibitem[van Benthem et~al.(2007)van Benthem, Gerbrandy, and
  Pacuit]{van_benthem_merging_2007}
J.~van Benthem, J.~Gerbrandy, and E.~Pacuit.
\newblock Merging frameworks for interaction: {DEL} and {ETL}.
\newblock In \emph{Proceedings of the 11th conference on
  Theoretical aspects of rationality and knowledge}, pages 72--81, Brussels,
  Belgium, 2007. {ACM}.

\end{thebibliography}
















\end{document}
