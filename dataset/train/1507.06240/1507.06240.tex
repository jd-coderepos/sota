\documentclass{article}[11pt,letter]


\usepackage[T1]{fontenc}
\usepackage[english]{babel}
\usepackage[cmex10]{amsmath}
\usepackage[utf8]{inputenc}
\usepackage{amssymb}
\usepackage{amsthm}
\usepackage{mathtools}
\usepackage{xspace}
\usepackage[symbol*]{footmisc}
\usepackage[hyperfootnotes=false,breaklinks,bookmarks=false]{hyperref}
\usepackage{algorithmic}
\usepackage{algorithm}
\usepackage{color}
\usepackage{authblk}
\usepackage{fullpage}
\usepackage{cite}
\usepackage{todonotes}


\newtheorem{definition}{Definition}[section]
\newtheorem{lemma}[definition]{Lemma}
\newtheorem{theorem}[definition]{Theorem}
\newtheorem{proposition}[definition]{Proposition}
\newtheorem{example}[definition]{Example}
\newtheorem{observation}[definition]{Observation}
\newtheorem{corollary}[definition]{Corollary}


\newcommand{\bigo}{\mathcal{O}}
\newcommand{\oh}{\bigo}
\renewcommand{\O}{\bigo}
\newcommand{\E}{\mathbb{E}}
\newcommand{\symb}{R}


\newcommand{\etal}{{et~al.}\xspace}
\newcommand{\cf}{{cf.}\xspace}
\newcommand{\eg}{{e.g.}\xspace}
\newcommand{\ie}{{i.e.}\xspace}
\newcommand{\offset}{\mathsf{offset}}
\newcommand{\name}{\mathsf{name}}
\newcommand{\rank}{\mathsf{rank}}
\newcommand{\select}{\mathsf{select}}
\renewcommand{\P}{\mathcal P}

\newcommand{\cost}{\delta}
\newcommand{\da}{\delta'}
\newcommand{\length}{\ell}
\newcommand{\encoding}{\mathsf{label}}






\begin{document}

\author[1]{Pawe\l{} Gawrychowski}
\affil[1]{Institute of Informatics, University of Warsaw, Poland}
\author[2]{Adrian Kosowski}
\affil[2]{Inria Paris and IRIF, Universit\'e Paris Diderot, France}
\author[3]{Przemys\l{}aw Uznański}
\affil[3]{Department of Computer Science, ETH Z\"{u}rich, Switzerland}


\date{}

\title{Sublinear-Space Distance Labeling using Hubs}

\maketitle

\begin{abstract}
A distance labeling scheme is an assignment of bit-labels to the vertices of an undirected, unweighted graph such that the distance between any pair of vertices can be decoded solely from their labels. We propose a series of new labeling schemes within the framework of so-called hub labeling (HL, also known as landmark labeling or 2-hop-cover labeling), in which each node  stores its distance to all nodes from an appropriately chosen set of hubs . For a queried pair of nodes , the length of a shortest -path passing through a hub node from  is then used as an upper bound on the distance between  and .

We present a hub labeling which allows us to decode exact distances in sparse graphs using labels of size sublinear in the number of nodes.  For graphs with at most  nodes and average degree , the tradeoff between label bit size  and query decoding time  for our approach is given by , for any . Our simple approach is thus the first sublinear-space distance labeling for sparse graphs that simultaneously admits small decoding time (for constant , we can achieve any  while maintaining ), and it also provides an improvement in terms of label size with respect to previous slower approaches.

By using similar techniques, we then present a -additive labeling scheme for general graphs, i.e., one in which the decoder provides a 2-additive-approximation of the distance between any pair of nodes. We achieve almost the same label size-time tradeoff , for any . To our knowledge, this is the first additive scheme with constant absolute error to use labels of sublinear size. The corresponding decoding time is then small (any  is sufficient).

We believe all of our techniques are of independent value and provide a desirable simplification of previous approaches.
\end{abstract}

\thispagestyle{empty}


\newpage

\section{Introduction}
Distance labeling schemes, popularized by Gavoille et al.~\cite{Gavoille:2004:DLG:1036161.1036165}, are among the most fundamental distributed data structures for graph data. The design problem combines two major challenges. First of all, distance labelings serve the role of a \emph{distance oracle}, i.e., a data structure which for a given undirected graph  can answer queries of the form: ``what is the distance between the nodes ?''. Throughout most of this paper, we will assume that  is an unweighted graph with  nodes and  edges. The efficiency of a distance oracle is measured by the interplay between the \emph{space} requirement of the data structure representation, the \emph{encoding time} required to set up the oracle for a given graph, and perhaps more importantly, its \emph{decoding time}, that is, the time of processing a  distance query. Moreover, a distance labeling scheme is defined more restrictively than a distance oracle, as an assignment of a binary string (label)  to each node , so that the graph distance between  and  is uniquely determined by the pair of labels:  and . The \emph{size} of a distance labeling scheme is now the maximum length of a node label in the graph. In this way, distance labelings add an extra layer of complexity to the graph distance decoding problem, by imposing a distributed representation of information in the labels . Whereas the concatenation of all  labels in a distance labeling forms a centralized distance oracle, distance labelings can also be applied in a distributed setting, in which the label of each node is stored at a distinct location in the network. This is the case, for instance, in applications in compact routing protocols, where the goal is to find a shortest path from a source node to a target node with a known label~\cite{ChepoiDEHVX12}.

An interesting characteristic of the problem of distance oracle design for sparse graph is its inherent link to an underlying set intersection task. On the side of lower bounds, this is most clearly observed, following Pătraşcu and Roditty~\cite{PatrascuR14}, when we consider a pair of vertices belonging to the same partition of a bipartite graph. The distance between them is  if and only if the sets of their neighbors intersect, and at least  otherwise. Consequently, assuming a plausible conjecture on the space required to decide intersection of a set of small sets, it follows that any oracle for graphs with  maximum degree, which admits constant decoding time, requires  space. (Here, the  and  notation disregards polylogarithmic factors in .) By contrast, many efficient algorithms for answering distance queries in real-world scenarios rely on the premise that the distance between a pair of nodes can be computed using an intersection-type query on a pair of small sets. In the basic framework of hub labelings, see \cite{Abraham:2012:HHL:2404160.2404164}, (introduced in~\cite{Cohen:2003:RDQ:942270.944300} under the name of 2-hop covers, and also referred to as landmark labelings~\cite{Abraham11onapproximate}), each node  stores the set of its distances to some subset  of other nodes of the graph. Then, the computed distance value  for a queried pair of nodes  is returned as:

where  denotes the shortest path distance function between a pair of nodes. The computed distance between all pairs of nodes  and  is exact if set  contains at least one node on some shortest  path. This property of the family of sets  is known as \emph{shortest path cover}. The hub-based method of distance computation is in practice effective for two reasons. First of all, for transportation-type networks it is possible to show bounds on the sizes of sets , which follow from the network structure. Notably, Abraham et al.~\cite{doi:10.1137/1.9781611973075.64} introduce the notion of highway dimension  of a network, which is presumed to be a small constant e.g.\ for road networks, and show that an appropriate cover of all shortest paths in the graph can be achieved using sets  of size . Moreover, the order in which elements of sets  and  is browsed when performing the minimum operation is relevant, and in some schemes, the operation can be interrupted once it is certain that the minimum has been found, before probing all elements of the set. This is the principle of numerous heuristics for the exact shortest-path problem, such as contraction hierarchies and algorithms with arc flags~\cite{Kohler06fastpoint-to-point,Bauer:2010:SFR:1498698.1537599}.

In this work, we make use of the hub set techniques to obtain better (distributed) distance labelings. Whereas  is a lower bound of the size of a hub set for general graphs, we provide hub-based schemes using smaller sets for specific case, leading to labels which can be encoded on  bits. Our scheme provides a shortest-path cover in the class of sparse graphs (with average degree  subpolynomial in ). This construction is overviewed in more detail in Section~\ref{sec:our_results}.

The implications of our result can be seen as twofold. First of all, our approach directly leads to labeling of smaller size (and smaller decoding time) for exact distance queries in sparse graphs than all previous distance labeling approaches. Additionally, an corollary of our result concerns the case of \emph{-additive} approximate distance labeling, in which the distance decoder is required to return an upper bound on the shortest path length which is within an additive factor of at most  from the optimum. So far, no way to construct -additive distance labels using labels of sublinear size in  was known for any constant . (This question was considered previously in, e.g.,~\cite{DBLP:conf/soda/AlstrupGHP16}). We provide a way to construct a -additive distance labeling in general graphs using distance labels of size . (This result is essentially the best possible, since a -additive distance labeling requires distance labels of size at least  already on the class of bipartite graphs.)

In our approaches, the size of the obtained distance labels for the considered cases is improved with respect to the state-of-the-art by up to a logarithmic multiplicative factor. Rather than seeing this as ``gaining'' a logarithm, we rather see this as ``not losing'' a logarithm. Indeed, the basic ingredient of the hub sets in previous approaches was a subset of nodes, sampled independently at random from ~\cite{BCE05,Sublinear}. The constructions then relied on the probabilistic method to guarantee that the hubs would have the shortest-path cover properties, based on the premise that for each pair of nodes the constructed hubs provide a shortest path cover with sufficiently high probability. The derandomization of this process resulted in a loss of a logarithmic factor in the analysis of the size of labels. Our approach shows how to avoid this issue: when constructing labelings for sparse graphs, we do away with randomization altogether, relying on simple structural results to replace the random subset of nodes.

\subsection{Related Work}

\paragraph{Distance Labelings.}
The distance labeling problem in undirected graphs was first investigated by Graham and Pollak~\cite{pollak}, who provided the first labeling scheme with labels of size . The decoding time for labels of size  was subsequently improved to  by Gavoille \etal~\cite{Gavoille:2004:DLG:1036161.1036165} and to  by Weimann and Peleg~\cite{WP11}. Finally, Alstrup \etal~\cite{DBLP:conf/soda/AlstrupGHP16} present a scheme for general graphs with decoding in  time using labels of size  bits.\footnote{For the sake of sanity of the notation, we define .} This matches up to low order terms the space of the currently best known distance oracle with  time and  total space in a \emph{centralized} memory model, due to Nitto and Venturini~\cite{NV08}.

The notion of -preserving distance labeling, first introduced by Bollob\'as \etal~\cite{BCE05}, describes a labeling scheme correctly encoding every distance that is at least . \cite{BCE05} presents such a -preserving scheme of size .
This was recently improved by Alstrup \etal~\cite{Sublinear} to a -preserving scheme of size . Together with an observation that all distances smaller than  can be stored directly, this results in a labeling scheme of size , where . For sparse graphs, this is .

For specific classes of graphs, Gavoille \etal~\cite{Gavoille:2004:DLG:1036161.1036165} described a  distance labeling for planar graphs, together with  lower bound for the same class of graphs. Additionally,  upper bound for trees and  lower bound for sparse graphs were given.

\paragraph{Distance Labeling with Hub Sets.} For a given graph , the computational task of minimizing the sizes of hub sets  for exact distance decoding is relatively well understood. A -approximation algorithm for minimizing the average size of a hub set having the sought shortest path cover property was presented in Cohen \etal~\cite{Cohen:2003:RDQ:942270.944300}, whereas a -approximation for minimizing the largest hub set at a node was given more recently in Babenko \etal~\cite{DBLP:conf/icalp/BabenkoGGN13}. Rather surprisingly, the structural question of obtaining bounds on the size of such hub sets for specific graph classes is wide open. For example, for the class of graphs of constant maximum degree, there is a large gap between the hub sets in our construction (of size ) and the generic lower bound of .





\paragraph{Distance Oracles.}

A centralized version of distance labeling problem is \emph{distance oracle} problem, where one asks for a centralized data structure allowing for querying a distance between pair of vertices. There usually one asks for what type of tradeoffs are possible between size of the structure, time of the query and allowed error (multiplicative stretch). Sommer~\etal \cite{sommer2009distance} proved that any constant time, constant stretch oracle must be superlinear in . Thorup and Zwick \cite{Thorup:2001:CRS:378580.378581} proved that distance oracles of stretch 2 require  space, and of stretch 3 require  space. Pătraşcu and Roditty \cite{PatrascuR14} strengthened the lower bound for stretch 2, proving a lower bound of  on the size of oracles with constant query time. For general weighted graphs, Thorup and Zwick \cite{Thorup:2001:CRS:378580.378581} designed a distance oracle of size , stretch- and  time. The query time has been improved to  time by Wulff-Nilsen \cite{doi:10.1137/1.9781611973105.39}, and to constant time in Chechik \cite{Chechik:2014:ADO:2591796.2591801}. The size of the distance oracle from \cite{Thorup:2001:CRS:378580.378581} is optimal assuming girth-conjecture. For sparse graphs, \cite{PatrascuR14} design distance oracle of size  and stretch 2 in constant time. Also in \cite{PatrascuR14}, a conditional lower bound of  bits for a constant time distance oracle is provided. Cohen and Porat \cite{DBLP:journals/corr/abs-1006-1117} extended this result to sparse graphs.  An up-to-date survey of results on approximate distance oracles is provided in~\cite{DBLP:reference/algo/Roditty15a}.

\subsection{Our Results and Organization of the Paper}\label{sec:our_results}

We start by introducing the necessary conventions in Section~\ref{sec:preliminaries}. We also describe the basic building block for encoding distance labels, namely, an efficient method of storing the hub set of a node, together with corresponding distances, in its distance label.

In Section~\ref{sec:sparse}, we show how to construct an exact distance labeling scheme for graphs of bounded maximum degree. This relies on hub sets which consist, for a vertex  of the union of all nodes from a small ball around vertex , and all nodes from a selection of equally-spaced levels of the breadth-first-search tree of . We then apply a trick, known from the previous work of \cite{Stretch}, to reduce the problem of constructing a labeling scheme for a graph with bounded average degree to that of constructing a labeling scheme for a~bounded-degree graph on twice as many nodes. For graphs with at most  nodes and average degree , the tradeoff between label bit size  and query decoding time  for our approach is given by , for any .
In particular, setting , we obtain labels of size , which improves previously best result \cite{Sublinear} by a factor of , keeping the  decoding time. On the other end, setting  we obtain first sublinear size distance labeling that achieves almost-constant decoding time.

In Section~\ref{sec:general}, we adapt our approach to general graphs, using a variant of the proposed labeling scheme for sparse graphs to achieve 2-additive approximation of distances. As before, we achieve a tradeoff between label size  and time  of the form , for any . This 2-additive distance labeling scheme can be easily transformed into an exact one, by encoding the difference between the estimation and the true distances. Since this difference is always from , we achieve labels of size  (with any  decoding time), or of size  (with  decoding time), for any . Our approach almost matches the size of the best known distance labeling schemes~\cite{DBLP:conf/soda/AlstrupGHP16}, which make use of labels of size  to achieve  decoding time. Arguably, our approach may be considered simpler.

We remark that all our results apply to unweighted graphs, in which each edge has unit length. For sparse graphs, in which each edge has an integer weight from some interval , we can use the same hub sets with an appropriately modified encoding to achieve a time-label tradeoff of . For the additive scheme, by subdividing each edge of length  into a chain of unweighted edges (of length 1), we achieve a conversion of the 2-additive distance labeling scheme into a -additive-distance scheme for weighted graphs.

\section{Preliminaries}\label{sec:preliminaries}

\paragraph{Notation and Conventions.}
Even though we are mainly interested in unweighted graphs, for technical reasons in Sections~\ref{sec:sparse} and~\ref{sec:general} we will work in a more general setting where every edge of a graph has a fixed cost from the set .  denotes the cost of a cheapest path
connecting a pair of nodes  and , and  denotes the smallest number of edges on such a~path. We will require the constructed distance labeling to return the value of . The degree of a node  is denoted by . When analyzing the complexity of the decoding, we assume standard word RAM with logarithmic word size, where we are allowed to access  consecutive bits of the stored binary string in constant time.

From now on, we assume that the graph is connected. This is enough because we can always include the identifier of its connected component in the label of every node, and return  if  and  belong to different connected components; this only induces additive  overhead to the label size.

\paragraph{Encoding Distances and Identifiers.}

The basic procedure for encoding a hub set in a label exploits some ideas from~\cite{DBLP:conf/soda/AlstrupGHP16}; we provide a self-contained exposition for completeness. We fix an arbitrary spanning tree of the graph and assign preorder numbers in the tree to the nodes, i.e., node numbered  corresponds to the root and so on. The preorder number of a node  is denoted by . Such a~numbering has the following useful property.

\begin{lemma}
\label{lem:sum}
Let  be the preorder sequence of all nodes.
Then, for any node , .
\end{lemma}

\begin{proof}
Consider an Euler tour corresponding a traversal of the chosen spanning tree. Every node is visited at least
once there, and the total length of the tour is at most . Consequently, we can cut the tour
into paths connecting node  with node , for every . The total length
of all these paths is at most  and the claim follows.
\end{proof}

The following lemma is used for encoding a hub set  using  bits.

\begin{lemma}
\label{lem:set_encoding}
For a fixed  and set  such that , set  and all of the distances  for  can be stored in  bits. For any constant , the representation
can be augmented with  additional bits so that all elements of  can
be extracted one-by-one in  total time and given any  we can check if 
(and if so, extract ) in  time.
\end{lemma}


\begin{proof}
Let , where . We store  and then the differences . Every difference is encoded using the Elias  code (see Elias~\cite{EliasGamma}), and the encodings are concatenated to form one binary
string. We are storing up to  integers whose absolute values sum up to at most , so by Jensen's inequality this takes  bits in total. Similarly, we store  and then the differences . By Lemma~\ref{lem:sum} we are again storing up to  numbers whose absolute values sum up to at most , which takes  bits.

All  can be extracted one-by-one in  time each with standard bitwise operations.
To facilitate checking if  in  time, we observe that it is enough to store a bit-vector
, where the -th bit is set to {\bf 1}, for every . Then checking if
 reduces to two  queries. A  query counts {\bf 1}s in the specified prefix of the bit-vector and a  query returns the position of the -th {\bf 1} in the bit-vector. By
the result of P\v{a}tra\c{s}cu~\cite{Succincter}, for any constant , a bit-vector of length  containing 
{\bf 1}s can be stored using

bits so that any rank or select query can be answered in  time.
This allows us to check if  and calculate  such that  in  time. To retrieve
, we store two additional bit-vectors  and . Each of them contains exactly
 {\bf 1}s and up to  {\bf 0}s. The bit-vectors are defined as follows. For each 
we consider the difference . If , we append
 to  and  to . Otherwise, we append
 to  and  to . By Lemma~\ref{lem:sum}, each of these
two bit-vectors contains at most  {\bf 0}s, so they can be stored using
 bits so that any rank or select query can be answered
in  time. To recover , we need to sum up all the differences. This reduces
to summing up all positive and all negative differences separately, which can be done using the
corresponding bit-vector with one  and one  query in  total time.
\end{proof}

We remark that the above encoding and decoding scheme is efficient for sets of size . For smaller sets, we will simply use an explicit encoding of all distances in , requiring  bits.


\section{Exact Distance Labeling in Sparse Graphs}\label{sec:sparse}

\subsection{Graphs of Bounded Maximum Degree}\label{sec:bdg}

In this subsection, we assume that  for every node . We consider distance labeling schemes characterized by a time parameter . Intuitively, in the construction,  will be a threshold parameter, distinguishing small distances from large distances in the graph --- a node will be able to afford to explicitly store the distances and identifiers of all nodes up to some distance  from itself in its distance label. Although this case is of independent interest, we are considering it as a building block for construction of labeling in graphs of bounded average degree. Thus graphs considered here are weighted with edge weights from , for the reason explained in Section~\ref{sec:bounded_average}.

The rest of this subsection is devoted to the proof of the following Theorem.

\begin{theorem}\label{thm:bounded}
Fix any value   and let . In bounded-degree graphs, there is a labeling scheme of size  and decoding time .
\end{theorem}

Let us denote . Since , we can bound .
Consider a node . The ball of radius  centered at , denoted , is the set of nodes
which can be reached from  by following at most  edges. Because the degrees of all nodes
are bounded by , . The -th layer centered at ,
denoted , consists of all nodes  such that . Because
the layers are disjoint, there exists an  such that
.

\paragraph{Definition of the Labeling.} We define the hub set of node , to which it stores all its distances, as , see Fig.~\ref{fig:layers}. Formally, the label of  consists of the following:
\begin{enumerate}
\item  and ,
\item  and  for every ,
\item  and  for every .
\end{enumerate}

\begin{figure}[t]
  \centering
    \vspace*{-2mm}
    \includegraphics[scale=1]{layers}
    \vspace*{-2mm}
    \caption{Shortest path from  to  goes through  which belongs to both  and .}
    \label{fig:layers}
\end{figure}


\paragraph{Computing .} For reasons of efficiency, we will not perform the distance decoding following Eq.~\eqref{eq:distance} directly, but we will treat the two components of the hub set of each node separately. Given  and , we can determine
 as follows. First we check if  and if so return the stored .
Otherwise, we iterate through all nodes  and check if .
If so, we know . We return the smallest such sum.

For the proof of correctness of the distance decoder, it is clear that  for any , so it remains to argue that either  or there exists  such that  and
. Consider a shortest path  where  and  such that . If  ,  and there is nothing to prove,
so we can assume that .  Observe that for any , ,
so in particular  for any integer .
We choose  and .
Then ,  by the choice of , and 
because  lies on a shortest path connecting  and , so indeed we are able to correctly
determine .

\paragraph{Encoding and Size of the Scheme.} Encoding  and  takes  bits.
The set  with corresponding distances is stored explicitly, while set  together with the corresponding distances is stored using Lemma~\ref{lem:set_encoding}, using  and
 bits, respectively.
Hence the total size of the scheme is 

where we have used the fact that for any  the claimed label size is the same, thus we can assume .

\paragraph{Complexity of the Decoding.} Checking if  and retrieving the encoded  takes  time.
Similarly,  iterating through all , checking if  and
if so retrieving the encoded  takes, by Lemma~\ref{lem:set_encoding},  time per single ,  thus  total time.
All in all, we can compute  in 
total time.
\qed

\paragraph{Smaller values of .} For the sake of completeness, we consider the special case of . Consider labeling where the label of a node
 consists of , , and all values  for  stored using
Lemma~\ref{lem:set_encoding}. This takes  bits, with  decoding time, and matches claimed bounds from Theorem~\ref{thm:bounded}.

\ \\
We also observe that our result applies not only to distance labels, but also as a size upper bound of hub sets for sparse graphs. Indeed, by fixing , and observing that , we have the following:
\begin{corollary}
In bounded-degree graphs, there is a hub set construction of size  vertices per node.
\end{corollary}

\subsection{Graphs of Bounded Average Degree}
\label{sec:bounded_average}

We now allow for bounded average degree by reduction to the approach from Subsection~\ref{sec:bdg}. Given a graph , let . We will create a new graph by splitting nodes of high degree. Following the formulation from~\cite[Lemma 4.2]{Stretch} (cf. Figure \ref{fig:splitting}), we can obtain a graph  on at most  nodes and at most  edges, such that the degree of every node is bounded by  and the distance between two nodes in the original graph  is exactly the same as the distance between their corresponding nodes in the new graph .
\begin{figure}[t]
  \centering
    \vspace*{-2mm}
    \includegraphics[scale=1.8]{splitting}
    \vspace*{-2mm}
    \caption{Example of subdividing of a large degree node (on the left) into a family of nodes of small degree, connected by edges of weight 0 (dashed edges).}
    \label{fig:splitting}
\end{figure}
We can now directly apply the scheme from Theorem~\ref{thm:bounded} to graph , and exactly the same distance labels will work for the corresponding nodes of graph . In this way, we obtain a scheme of size:

which returns  in  time given the labels of  and . The correctness of the this reduction is guaranteed by the fact that Theorem~\ref{thm:bounded} allows for edge weights from .

\begin{theorem}
Fix any   and , and let . There exists an exact distance labeling for graphs with average degree  using labels of size 
 and a corresponding decoding scheme requiring time .\qed
\end{theorem}

It is easy to see that this reduction preserves bounds on the size of hub sets, so we have the following:
\begin{corollary}
In graphs with average degree , there is a hub set construction of size  vertices per node.
\end{corollary}

\section{2-Additive Distance Labeling in General Graphs}\label{sec:general}

We will apply a similar distance labeling scheme as for sparse graphs, obtaining a 2-additive approximation of the distance between any pair of  with label sizes of  per node. In this approximate scheme, the hub sets will have the following property.  The label of each node  will provide an encoding of the node identifiers of a subset  and of the distances from  to all elements of . The sets  will be defined so that for any pair , there exists a node , such that either  or a neighbor of  lies on the shortest path from  to  in . We will decode the approximate distance as before, using Eq.~\eqref{eq:distance}; clearly, .

The construction of sets  is performed as follows. Let  be an threshold value of vertex degree, to be chosen later. Let , and let  be a minimal dominating set for , i.e., a subset of  with the property: . By a straightforward application of the probabilistic method~(cf.\cite[proof of Theorem 1.2.2]{probmethod}), we have that there is  such that , and it can be easily constructed in polynomial time (a deterministic construction by a folklore greedy algorithm gives set of size ). For every , we define  as the set of nodes of the ball of radius  around  in the subgraph .  Finally, we define  and let , , and  be defined as in Section~\ref{sec:bdg}, and let  be a minimal subset of  such that for every  adjacent the boundary of , i.e. , we have . Such  can be easily constructed in polynomial time, and moreover, since there are at most  vertices adjacent to the boundary, we have .

The approximate distance label of  now consists of the following elements:
\begin{enumerate}
\item  and ,
\item  and  for every ,
\item  and  for every .
\item  and  for every ,
\item  and  for every .
\end{enumerate}
The separation of  into  and  in the label is done to allow
efficient decoding.

\paragraph{Computing .}
To show the correctness of this approximate labeling scheme, fix a pair of vertices . If there exists a vertex  lying on a fixed shortest path  between  and  such that  or , then the labeling scheme finds the shortest path distance between  and  as in Section~\ref{sec:bdg}. Otherwise, let  be the nearest vertex to  lying in ; it follows from the construction that . Then, there exists  such that . In this case, the distance  is a -additive approximation of .

\paragraph{Size of the Scheme.}
The size of the label of a node  in the scheme can be bounded as follows: , , . Overall, the total size is , thus using Lemma~\ref{lem:set_encoding} to store the sets and the corresponding distances we obtain labels of
size .

\paragraph{Complexity of the Decoding.}
To perform the distance decoding, for a given pair , it suffices to minimize  over all  belonging to  which are also encoded in the label of . Hence, distance decoding is possible in time . Overall, setting , we obtain the following main result of the section.

\begin{theorem}
There is a -additive distance labeling scheme for general graphs, which achieves decoding time  using labels of size , for any .\qed
\end{theorem}

Finally, we remark on some implications of our result. By a standard argument, converting a -additive approximate distance labeling into an exact one requires an additional label of size  bits per node (and an additional  overhead in the space, which is negligible), with each node  encoding the difference between the approximate and real distance value, , for all . The time overhead of the corresponding decoding is . In an analogous manner, converting a -additive approximate distance labeling into an -additive approximate one requires an additional label of size  bits per node. Thus we convert our scheme into an exact distance labeling scheme or -additive scheme achieving  decoding time using labels of size respectively  or , for any .


Thus, setting  as an arbitrarily small increasing function of , for any desired decoding time  we can make use of labels of size ,  and  respectively for -additive, -additive and exact distances. Moreover, using this scheme,  decoding time can be achieved for labels of size ,  and , for any absolute constant .

While a slightly stronger in terms of decoding time schemes were presented in  Alstrup \etal~\cite{DBLP:conf/soda/AlstrupGHP16} (achieving  decoding time and labels of size  and  for exact and -additive distances), we believe that presented here schemes are of independent value due to the simplification of the construction.







\section*{Acknowledgments}
Most of the work was done while PU was affiliated to Aalto University, Finland.
Research partially supported by the National Science Centre, Poland - grant number 2015/17/B/ST6/01897.

\bibliographystyle{abbrv}
\bibliography{biblio}

\end{document}
