\newcommand{\ov}[1]{\overline{#1}}
\newcommand{\srewRule}[3]{#1\srew{#3}#2}
\newcommand{\srew}[1]{\stackrel{#1}{\longmapsto}}
\newcommand{\csrew}[1]{\stackrel{#1}{\longmapsto}_{\sf c}}
\renewcommand{\wr}[3]{( #1 \into #2)^{#3}}


The Calculus of Wrapped Compartments (CWC)
(see~\cite{preQAPL2010,CDDGGT_TCSB11}) is based on a nested structure of ambients delimited by membranes with specific proprieties. Biological entities
like cells, bacteria and their interactions can be easily described in CWC.

\subsection{Syntax}
\label{CWC_formalism - syntax}

Let $\AT$ be a set of  \emph{atomic elements} (\emph{atoms} for
short), ranged over by $a$, $b$, ..., and  $\LT$ a set of \emph{compartment types} represented as \emph{labels} ranged over by $\ell,\ldots$.
A \emph{term} of CWC is a multiset $\ov{t}$ of  \emph{simple terms} where a simple term  is either an atom $a$ or a compartment $(\overline{a}\into
\overline{t'})^\ell$ consisting of a \emph{wrap}
 (represented by a multiset of atoms $\overline{a}$)\footnote{the wrap of a compartment contains the elements characterizing the membrane enclosing it. So we assume that it cannot contain other compartments but only atomic elements (for instance proteins).}, represented by a (possibly empty) and a \emph{content},represented by a term
$\overline{t'}$) and a \emph{type}, represented by the label $\ell$. An empty multiset is represented symbolically with ``$\emptyseq$''.

\begin{figure*}\centering
\subfigure[] {
\includegraphics[height=28mm]{Figures/term1.jpg}
}
\centering
\subfigure[] {
\includegraphics[height=28mm]{Figures/term2.jpg}
}
\subfigure[] {
\includegraphics[height=28mm]{Figures/term3.jpg}
}

\caption{\textbf{(a)} represents $(a \conc b \conc c \into \emptyseq)^\ell$; \textbf{(b)} represents $(a \conc b \conc c  \into (d \conc e \into
\emptyseq)^{\ell'})^\ell$; \textbf{(c)} represents $(a \conc b \conc c \into (d \conc e \into \emptyseq)^{\ell'} \conc f \conc g)^\ell$}
\label{fig:example CWM}
\end{figure*}
As usual, the notation $n*t$ denotes $n$ occurrences of the simple term $t$. An example of term is $2\!*\!a \conc b \conc (c \conc d \into e \conc
f)^\ell$ representing a multiset (multisets are denoted by listing the elements separated by a space) consisting of two occurrences of $a$, one occurrence
of $b$ (e.g. three molecules) and an $\ell$-type compartment $(c \conc d \into e \conc f)^\ell$ which, in turn, consists of a wrap (a membrane) with two
atoms $c$ and $d$ (e.g. two proteins) on its surface, and containing the atoms $e$ (e.g. a molecule) and $f$ (e.g. a DNA strand). See
Figure~\ref{fig:example CWM} for some other examples with a simple graphical representation.



\subsection{Rewriting Rules}
\label{CWC_formalism - rewriting}

System transformations are defined by rewriting rules written over an
extended set of terms including variables. The l.h.s. component  of a rewrite rule $\ov{\LeftPat}$ is called  \emph{pattern} and  the r.h.s. component $ \ov{\RightPat}$ of a rewrite rule is called \emph{open term}.  Patterns and open terms are multiset of \emph{simple patterns} $p$ and \emph{simple open terms} $o$ defined by the following syntax:
$$
\begin{array}{lcl}
\LeftPat  & \;\qqop{::=}\; & a \agr (\overline{a} \conc x \into \overline{\LeftPat}\conc X)^\ell \\
   \RightPat & \;\qqop{::=}\; & a \agr (\ov{a}  \conc x \into \ov{o})^\ell   \agr X
\end{array}
$$

where $\overline{a}$ is a multiset of atoms, $x$ and $X$ are
variables that can be instantiated, respectively by a multiset of atoms (on a wrap) or by a term (in the  compartment contents). The label $\ell$ is called the \emph{type of the pattern}. Patterns are intended to match, via substitution of variables, with simple terms and compartments occurring as subterms of the term representing the whole system. Note that we force \emph{exactly} one variable to occur in each
compartment content and wrap.
 This prevents ambiguities in the instantiations needed to match a given compartment.
 \footnote{
The linearity condition, in biological terms, corresponds to the exclusion of transformations depending on the presence of two (or more)
 identical (and generic) components in different compartments (see also~\cite{OP11}).}

A \emph{rewrite rule} is a triple $(\ell, \ov{\LeftPat},\ov{\RightPat} $, denoted by $\ell:  \overline{\LeftPat} \srew{}    \ov{\RightPat})$, where $\ov{\LeftPat}$ and $\ov{\RightPat}$ are such that the variables occurring in $\ov{\RightPat}$ are a subset of the variables occurring in $\ov{\LeftPat}$.
The application of a rule $\ell:  \ov{\LeftPat} \red \ov{\RightPat}$ to a term $\ov{t}$ is performed in the following way: 1) Find in $\ov{t}$ (if it
exists) a compartment of type $\ell$ with content $\ov{u}$ and a substitution $\sigma$ of variables by ground terms (i.e. terms without occurrences of variables) such that $\ov{u} = \sigma( \ov{\LeftPat} \conc
X)$, where $X$ is a variable not occurring in $\ov{p},~ \ov{o}$; and 2) Replace in $\ov{t}$ the
subterm $u$ with $\sigma(\ov{\RightPat}\conc X)$. We write  $\ov{t} \red \ov{t'}$ if $\ov{t'}$ is obtained  by applying a rewrite rule to  $\ov{t}$.


\begin{comment}
A \emph{rewrite rule} is a pair $(\LeftPat,\RightPat)$, denoted by $\srewRule{\LeftPat}{\RightPat}{}$, where $\LeftPat= (\overline{a} \conc x \into
\overline{b} \conc \overline{\LeftPat}\conc X)^\ell$ is a pattern and  $\RightPat$ is a compartment of the same type of $\LeftPat$ such that the variables occurring in it are a subset of the variables occurring in $\LeftPat$.
The application of a rule $\LeftPat \red \RightPat$ to a term
$\ov{t}$ is performed in the following way: 1) Find (if it exists) a compartment $\ov{u}$ in $\ov{t}$ and a substitution $\sigma$ of variables by ground terms
such that $\ov{u} = \sigma(\LeftPat)$; and 2) Replace in $\ov{t}$ the subterm $\ov{u}$ with $\sigma(\RightPat)$. We write  $\ov{t} \red \ov{t'}$ if
$\ov{t'}$ is obtained  by applying a rewrite rule to  $\ov{t}$.

 A rewrite rule $\srewRule{\LeftPat}{\RightPat}{}$ then states that a compartment $\sigma(\LeftPat)$,
obtained by instantiating variables in $\LeftPat$ by some instantiation function $\sigma$, can be transformed into the compartment $\sigma(\RightPat)$
with the same label of $\LeftPat$. Linearity is not required in the r.h.s. of a rule, thus allowing duplication.

In many cases rules involve only the content of a compartment. We therefore introduce the following notation:
$$\ell: \overline{a} \conc \overline{\LeftPat}  \srew{}   \ov{o}$$
 as a short for
 $\wr{x}{\overline{a} \conc \overline{\LeftPat}\conc X}{\ell}   \srew{}  \wr{x}{ \ov{o}  \conc X }{\ell} $
where $x$ and $X$ are canonically chosen variables not occurring in $\overline{\LeftPat}$ and $\overline{o}$.


For instance, the rewrite rule
  $\ell:~a \; b \red c$
means that in all compartments of type $\ell$ an occurrence of $a \conc b$  can be replaced by $c$ (the implicit variable $x$ matches with the whole membrane
and the implicit variable $X$ matches with all the remaining part of the compartment content).
\end{comment}


For instance, the rewrite rule
  $\ell:~a \; b \red c$
means that in all compartments of type $\ell$ containing an occurrence of $a \conc b$, it can be replaced by $c$. While the rewrite rule  $\ell: (x \into a \conc b \conc X)^{\ell_1} \red ( x \into c \conc X)^{\ell_2}$ means that, if contained in a compartment of type $\ell$, all compartments of type $\ell_1$ and containing an $a \conc b$ in their content can change their type to $\ell_2$ and an occurrence of $a \conc b$  can be replaced by $c$.


 \noindent \emph{Remark.} For  uniformity we assume that the term representing the whole
system is always a single compartment labelled $\top$ with an empty wrap, i.e., all systems are represented by a term of the shape
$\wr{\emptyseq}{\ov{t}}{\top}$, which we will also write as $\ov{t}$ for simplicity.


\subsection{Stochastic Simulation}\label{SECT:STO_SEM}

A stochastic simulation model for biological systems can be defined
along the lines of the one
presented by Gillespie in \cite{G77}, which is, \emph{de facto}, the standard way to model quantitative aspects of biological systems. The basic idea of
Gillespie's algorithm is that a rate function is associated with each considered chemical reaction which is used as the parameter of an exponential distribution modelling the probability that the reaction takes place. In the standard approach this reaction rate is obtained by multiplying the kinetic
constant of the reaction by the number of possible combinations of reactants that may occur in the region in which the reaction takes place, thus modelling the law of mass action. In \cite{preQAPL2010}, the reaction rate is defined in a more general way by associating to each reduction rule a function which can also define rates based on different principles as, for instance, the Michaelis-Menten nonlinear kinetics.

In the examples of this paper it is reasonable to assume a linear dependence of the reaction rate on the number of possible combinations of interacting components, so we will follow the standard approach in defining reaction rates. Each reduction rule is then enriched by the kinetic constant $k$ of the reaction that it represents (notation $\ell:\LeftPat \srewrites{k} \RightPat $).
For instance in evaluating the application rate of the stochastic rewrite rule $R=
\ell: a \conc b \conc X \srewrites{k} c \conc X$ to the term $\ov{t}=a\conc a \conc b \conc b$ in a compartment of type $\ell$ we must consider the number of the possible combinations of reactants of
the form $a\conc b$ in $\ov{t}$. Since each occurrence of $a$ can react with each occurrence of $b$, this number is 4. So the application rate of $R$ is
$k\cdot 4$.





\subsection{The CWC simulator}
\label{sec:simulator}


The CWC simulator~\cite{HCWC_SIM} is a tool under development at the Computer science Department of Turin University, based on Gillespie's direct method
algorithm~\cite{G77}. It treats CWC models with different rating semantics (law of mass action, Michaelis-Menten kinetics, Hill equation) and
it can run independent stochastic simulations over CWC models, featuring deep parallel optimizations for multi-core platforms on the top of
FastFlow~\cite{fastflow:web}. It also performs online analysis by a modular statistical framework.





