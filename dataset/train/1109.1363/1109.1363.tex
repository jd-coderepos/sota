\newcommand{\ov}[1]{\overline{#1}}
\newcommand{\srewRule}[3]{#1\srew{#3}#2}
\newcommand{\srew}[1]{\stackrel{#1}{\longmapsto}}
\newcommand{\csrew}[1]{\stackrel{#1}{\longmapsto}_{\sf c}}
\renewcommand{\wr}[3]{( #1 \into #2)^{#3}}


The Calculus of Wrapped Compartments (CWC)
(see~\cite{preQAPL2010,CDDGGT_TCSB11}) is based on a nested structure of ambients delimited by membranes with specific proprieties. Biological entities
like cells, bacteria and their interactions can be easily described in CWC.

\subsection{Syntax}
\label{CWC_formalism - syntax}

Let  be a set of  \emph{atomic elements} (\emph{atoms} for
short), ranged over by , , ..., and   a set of \emph{compartment types} represented as \emph{labels} ranged over by .
A \emph{term} of CWC is a multiset  of  \emph{simple terms} where a simple term  is either an atom  or a compartment  consisting of a \emph{wrap}
 (represented by a multiset of atoms )\footnote{the wrap of a compartment contains the elements characterizing the membrane enclosing it. So we assume that it cannot contain other compartments but only atomic elements (for instance proteins).}, represented by a (possibly empty) and a \emph{content},represented by a term
) and a \emph{type}, represented by the label . An empty multiset is represented symbolically with ``''.

\begin{figure*}\centering
\subfigure[] {
\includegraphics[height=28mm]{Figures/term1.jpg}
}
\centering
\subfigure[] {
\includegraphics[height=28mm]{Figures/term2.jpg}
}
\subfigure[] {
\includegraphics[height=28mm]{Figures/term3.jpg}
}

\caption{\textbf{(a)} represents ; \textbf{(b)} represents ; \textbf{(c)} represents }
\label{fig:example CWM}
\end{figure*}
As usual, the notation  denotes  occurrences of the simple term . An example of term is  representing a multiset (multisets are denoted by listing the elements separated by a space) consisting of two occurrences of , one occurrence
of  (e.g. three molecules) and an -type compartment  which, in turn, consists of a wrap (a membrane) with two
atoms  and  (e.g. two proteins) on its surface, and containing the atoms  (e.g. a molecule) and  (e.g. a DNA strand). See
Figure~\ref{fig:example CWM} for some other examples with a simple graphical representation.



\subsection{Rewriting Rules}
\label{CWC_formalism - rewriting}

System transformations are defined by rewriting rules written over an
extended set of terms including variables. The l.h.s. component  of a rewrite rule  is called  \emph{pattern} and  the r.h.s. component  of a rewrite rule is called \emph{open term}.  Patterns and open terms are multiset of \emph{simple patterns}  and \emph{simple open terms}  defined by the following syntax:


where  is a multiset of atoms,  and  are
variables that can be instantiated, respectively by a multiset of atoms (on a wrap) or by a term (in the  compartment contents). The label  is called the \emph{type of the pattern}. Patterns are intended to match, via substitution of variables, with simple terms and compartments occurring as subterms of the term representing the whole system. Note that we force \emph{exactly} one variable to occur in each
compartment content and wrap.
 This prevents ambiguities in the instantiations needed to match a given compartment.
 \footnote{
The linearity condition, in biological terms, corresponds to the exclusion of transformations depending on the presence of two (or more)
 identical (and generic) components in different compartments (see also~\cite{OP11}).}

A \emph{rewrite rule} is a triple , denoted by , where  and  are such that the variables occurring in  are a subset of the variables occurring in .
The application of a rule  to a term  is performed in the following way: 1) Find in  (if it
exists) a compartment of type  with content  and a substitution  of variables by ground terms (i.e. terms without occurrences of variables) such that , where  is a variable not occurring in ; and 2) Replace in  the
subterm  with . We write   if  is obtained  by applying a rewrite rule to  .


\begin{comment}
A \emph{rewrite rule} is a pair , denoted by , where  is a pattern and   is a compartment of the same type of  such that the variables occurring in it are a subset of the variables occurring in .
The application of a rule  to a term
 is performed in the following way: 1) Find (if it exists) a compartment  in  and a substitution  of variables by ground terms
such that ; and 2) Replace in  the subterm  with . We write   if
 is obtained  by applying a rewrite rule to  .

 A rewrite rule  then states that a compartment ,
obtained by instantiating variables in  by some instantiation function , can be transformed into the compartment 
with the same label of . Linearity is not required in the r.h.s. of a rule, thus allowing duplication.

In many cases rules involve only the content of a compartment. We therefore introduce the following notation:

 as a short for
 
where  and  are canonically chosen variables not occurring in  and .


For instance, the rewrite rule
  
means that in all compartments of type  an occurrence of   can be replaced by  (the implicit variable  matches with the whole membrane
and the implicit variable  matches with all the remaining part of the compartment content).
\end{comment}


For instance, the rewrite rule
  
means that in all compartments of type  containing an occurrence of , it can be replaced by . While the rewrite rule   means that, if contained in a compartment of type , all compartments of type  and containing an  in their content can change their type to  and an occurrence of   can be replaced by .


 \noindent \emph{Remark.} For  uniformity we assume that the term representing the whole
system is always a single compartment labelled  with an empty wrap, i.e., all systems are represented by a term of the shape
, which we will also write as  for simplicity.


\subsection{Stochastic Simulation}\label{SECT:STO_SEM}

A stochastic simulation model for biological systems can be defined
along the lines of the one
presented by Gillespie in \cite{G77}, which is, \emph{de facto}, the standard way to model quantitative aspects of biological systems. The basic idea of
Gillespie's algorithm is that a rate function is associated with each considered chemical reaction which is used as the parameter of an exponential distribution modelling the probability that the reaction takes place. In the standard approach this reaction rate is obtained by multiplying the kinetic
constant of the reaction by the number of possible combinations of reactants that may occur in the region in which the reaction takes place, thus modelling the law of mass action. In \cite{preQAPL2010}, the reaction rate is defined in a more general way by associating to each reduction rule a function which can also define rates based on different principles as, for instance, the Michaelis-Menten nonlinear kinetics.

In the examples of this paper it is reasonable to assume a linear dependence of the reaction rate on the number of possible combinations of interacting components, so we will follow the standard approach in defining reaction rates. Each reduction rule is then enriched by the kinetic constant  of the reaction that it represents (notation ).
For instance in evaluating the application rate of the stochastic rewrite rule  to the term  in a compartment of type  we must consider the number of the possible combinations of reactants of
the form  in . Since each occurrence of  can react with each occurrence of , this number is 4. So the application rate of  is
.





\subsection{The CWC simulator}
\label{sec:simulator}


The CWC simulator~\cite{HCWC_SIM} is a tool under development at the Computer science Department of Turin University, based on Gillespie's direct method
algorithm~\cite{G77}. It treats CWC models with different rating semantics (law of mass action, Michaelis-Menten kinetics, Hill equation) and
it can run independent stochastic simulations over CWC models, featuring deep parallel optimizations for multi-core platforms on the top of
FastFlow~\cite{fastflow:web}. It also performs online analysis by a modular statistical framework.





