\pdfoutput=1


\documentclass[11pt]{article}

\usepackage[]{acl2021}

\usepackage{times}
\usepackage{latexsym}


\usepackage[T1]{fontenc}


\usepackage[utf8]{inputenc}

\usepackage{microtype}






\renewcommand{\UrlFont}{\ttfamily\small}



\usepackage{amsfonts}
\usepackage{amsmath}
\usepackage{graphicx}
\usepackage{booktabs}





\newcommand\BibTeX{B\textsc{ib}\TeX}

\title{Code Synonyms Do Matter: \\ Multiple Synonyms Matching Network for Automatic ICD Coding}

\author{
Zheng Yuan\thanks{Work done at Alibaba DAMO Academy.} \space\space\space
Chuanqi Tan \space\space
Songfang Huang \space\space\\
Tsinghua University \space\space\space\space
Alibaba Group\\
\texttt{yuanz17@mails.tsinghua.edu.cn}\\
\texttt{\{chuanqi.tcq,songfang.hsf\}@alibaba-inc.com}
}

\date{}

\begin{document}
\maketitle
\begin{abstract}
Automatic ICD coding is defined as assigning disease codes to electronic medical records (EMRs).
Existing methods usually apply label attention with code representations to match related text snippets.
Unlike these works that model the label with the code hierarchy or description, we argue that the code synonyms can provide more comprehensive knowledge based on the observation that the code expressions in EMRs vary from their descriptions in ICD. 
By aligning codes to concepts in UMLS, we collect synonyms of every code. Then, we propose a multiple synonyms matching network to leverage synonyms for better code representation learning, and finally help the code classification. 
Experiments on the MIMIC-III dataset show that our proposed method outperforms previous state-of-the-art methods.

\end{abstract}

\section{Introduction}
International Classification of Diseases (ICD) is a classification and terminology that provides diagnostic codes with descriptions for diseases\footnote{\url{who.int/standards/classifications/classification-of-diseases}}.
The task of ICD coding refers to assigning ICD codes to electronic medical records (EMRs) which is highly related to clinical tasks or systems including patient similarity learning \cite{suo2018deep}, medical billing \cite{sonabend2020automated}, and clinical decision support systems \cite{sutton2020overview}. Traditionally, healthcare organizations have to employ specialized coders for this task, which is expensive, time-consuming, and error-prone. As a result, many methods have been proposed for automatic ICD coding since the 1990s \cite{de1998hierarchical}.






Recent methods treat this task as a multi-label classification problem \cite{xie2018neural,li2020icd,zhou2021automatic}, which learn deep representations of EMRs with an RNN or CNN encoder and predict codes with a multi-label classifier.
Recent state-of-the-art methods propose label attention that uses the code representations as attention queries to extract the code-related representations\footnote{``Label'' equals to ``code'' in some contexts of this paper.} \cite{mullenbach-etal-2018-explainable}.
Many works further propose using code hierarchical structures \cite{falis2019ontological,xie2019ehr,cao2020hypercore} and descriptions \cite{ijcai2020-556,kim2021read} for better label representations.

In this work, we argue that the synonyms of codes can provide more comprehensive information. For example, the description of code \textit{244.9} is ``Unspecified hypothyroidism'' in ICD. 
However, this code can be described in different forms in EMRs such as ``low t4'' and ``subthyroidism''. Fortunately, these different expressions can be found in the Unified Medical Language System \cite{Bodenreider2004}, a repository of biomedical vocabularies that contains various synonyms for all ICD codes. Therefore, we propose to leverage synonyms of codes to help the label representation learning and further benefit its matching to the EMR texts. 





To model the synonym and its matching to EMR text, we further propose a \textbf{M}ultiple \textbf{S}ynonyms \textbf{M}atching \textbf{N}etwork (\textbf{MSMN})\footnote{Our codes and model can be found at \url{https://github.com/GanjinZero/ICD-MSMN}.}.
Specifically, we first apply a shared LSTM to encode EMR texts and each synonym. Then, we propose a novel multi-synonyms attention mechanism inspired by the multi-head attention \cite{vaswani2017attention}, which considers synonyms as attention queries to extract different code-related text snippets for code-wise representations.
Finally, we propose using a biaffine-based similarity of code-wise text representations and code representations for classification.

We conduct experiments on the MIMIC-III dataset with two settings: full codes and top-50 codes. Results show that our method performs better than previous state-of-the-art methods.



\section{Approach}
Consider free text  (usually discharge summaries) from EMR with
words .
The task is to assign a binary label  based on .
Figure~\ref{arch} shows an overview of our method.



\subsection{Code Synonyms}

We extend the code description  by synonyms from the medical knowledge graph (i.e., UMLS Metathesaurus).
We first align the code to the Concept Unique Identifiers (CUIs) from UMLS.
Then we select corresponding synonyms of English terms from UMLS with the same CUIs and add additional synonyms by removing hyphens and the word ``NOS'' (Not Otherwise Specified).
We denote the code synonyms as  in which each code synonym  is composed of words .


\subsection{Encoding}
Previous works \cite{JI2021104998,pascual-etal-2021-towards} have shown that pretrained language models like BERT \cite{devlin-etal-2019-bert} cannot help the ICD coding performance, hence we use an LSTM \cite{hochreiter1997long} as our encoder.
We use pre-trained word embeddings to map words   to .
A -layer bi-directional LSTM layer takes word embeddings as input to obtain text hidden representations .

For code synonym , we apply the same encoder with a max-pooling layer to obtain representation .


\subsection{Multi-synonyms Attention}
To interact text with multiple synonyms, we propose a multi-synonyms attention inspired by the multi-head attention  \cite{vaswani2017attention}.
We split  into  heads :

Then, we use code synonyms  to query . We take the linear transformations 
of  and  to calculate attention scores .
Text related to code synonym  can be represented by .
We aggregate code-wise text representations  using max-pooling of  since the text only needs to match one of the synonyms.





\begin{figure}
\centering
\includegraphics[width=3.0in]{figure/main.pdf}
\caption{The architecture of our proposed MSMN. 
Different colors indicate different code synonyms. We also split hidden representations into different heads for multi-synonyms attention.
}
\label{arch}
\end{figure}

\subsection{Classification}
We classify whether the text  contains code  based on the similarity between code-wise text representation  and code representation.
We aggregate code synonym representations  to code representation  by max-pooling.
We then propose using a biaffine transformation to measure the similarity for classification:

Previous works \cite{mullenbach-etal-2018-explainable,ijcai2020-461-vu}  classify codes via\footnote{We omit the biases in all equations for simplification.}:

Their work need to learn code-dependent parameters  for classification, which suffers from training rare codes.
On the contrary, our biaffine function that uses  instead of  only needs to learn code-independent parameters .









\subsection{Training} We optimize the model using binary cross-entropy between predicted probabilities  and labels :




\begin{table}[ht]
    \centering
    \small
    \begin{tabular}{lccc}
    \toprule
    & Train & Dev & Test  \\
    \multicolumn{4}{c}{MIMIC-III Full} \\
    \midrule
    \# Doc. & 47,723 & 1,631 & 3,372 \\
    Avg \# words per Doc. & 1,434 & 1,724 & 1,731 \\  
    Avg \# codes per Doc. & 15.7 & 18.0 & 17.4 \\ 
    Total \# codes & 8,692 & 3,012 & 4,085\\ 
    \midrule
    \multicolumn{4}{c}{MIMIC-III 50} \\
    \midrule
    \# Doc. & 8,066 & 1,573 & 1,729 \\
    Avg \# words per Doc. & 1,478 & 1,739 & 1,763 \\
    Avg \# codes per Doc. & 5.7 & 5.9 & 6.0 \\ 
    Total \# codes & 50 & 50 & 50 \\
    \bottomrule
    
    \end{tabular}
    \caption{Statistics of MIMIC-III dataset under full codes and top-50 codes settings.}
    \label{tab:dataset}
\end{table}




\section{Experiments}


\subsection{Dataset}



MIMIC-III dataset \cite{johnson2016mimic} contains deidentified discharge summaries with human-labeled ICD-9 codes.
We list the document counts, average word counts per document, average codes counts per document, and total codes of the MIMIC-III dataset in Table~\ref{tab:dataset}.
We use the same splits with previous works \cite{mullenbach-etal-2018-explainable,ijcai2020-461-vu}
with two settings as full codes (MIMIC-III full) and top-50 frequent codes (MIMIC-III 50).
We follow the preprocessing of \citet{xie2019ehr} and \citet{ijcai2020-461-vu} to truncate discharge summaries at 4,000 words.
We measure the results using macro AUC, micro AUC, macro , micro  and precision@k ( for MIMIC-III 50,  and  for MIMIC-III full).







\subsection{Implementation Details}



We sample  and  synonyms per code for MIMIC-III full and MIMIC-III 50 respectively.
We sample synonyms fully randomly from the synonyms set. If some ICD codes do not have enough synonyms, we just repeat these synonyms.
We use the same word embeddings as \citet{ijcai2020-461-vu} which are pretrained on the MIMIC-III discharge summaries using CBOW \cite{mikolov2013efficient} with a hidden size of 100.
We apply R-Drop with  \cite{liang2021rdrop} to regularize the model to prevent over-fitting. 
We apply the dropout with a ratio of 0.2 after the word embedding layer and before the classification layer.
For text encoding, we add a linear layer upon the LSTM layer (the output dimension of the linear layer refers to LSTM output dim. in Table~\ref{hyper para}).
We train MSMN with AdamW \cite{loshchilov2017decoupled} with a linear learning rate decay.
We optimize the threshold of classification using the development set.
For the MIMIC-III 50 setting, we train with one 16GB NVIDIA-V100 GPU.
For the MIMIC-III full setting, we train with 8 32GB NVIDIA-V100 GPUs.
We list the detailed training hyper-parameters in Table~\ref{hyper para}.


\begin{table}
\centering
\small
\begin{tabular}{lcc}
\toprule
Parameters & Full & Top 50 \\
\midrule
Emb. dim. & 100 & 100 \\
Emb. dropout & 0.2 & 0.2 \\
LSTM Layer () & 2 & 1 \\
LSTM hidden dim. & 256 & 512 \\
LSTM output dim. () & 512 & 512 \\
Synonyms count () & 4 & 8 \\
Rep. dropout & 0.2 & 0.2 \\
R-Drop weight & 5.0 & 5.0 \\
Epoch & 20 & 20  \\
Peak lr. & 5e-4 & 5e-4 \\
Batch size & 16 & 16 \\
Adam  & 1e-8 & 1e-8 \\
Weight decay & 0.01 & 0.01 \\
Clipping grad. & 1.0 & 1.0 \\
\bottomrule
\end{tabular}
\caption{Hyper-parameters used for training MIMIC-III full setting and MIMIC-III 50 setting.
}
\label{hyper para}
\end{table}


\subsection{Baselines}


\noindent\textbf{CAML} \cite{mullenbach-etal-2018-explainable} uses CNN to encode texts and proposes label attention for coding.

\noindent\textbf{MSATT-KG} \cite{xie2019ehr} applies multi-scale attention and GCN to capture codes relations.

\noindent\textbf{MultiResCNN} \cite{li2020icd} encodes text using multi-filter residual CNN.

\noindent\textbf{HyperCore} \cite{cao2020hypercore} embeds ICD codes into the hyperbolic space to utilize code hierarchy and uses GCN to leverage the code co-occurrence.

\noindent\textbf{LAAT} \& \noindent\textbf{JointLAAT} \cite{ijcai2020-461-vu} propose a hierarchical joint learning mechanism to relieve the imbalanced labels, which is our main baseline since it is most similar to our work.


\begin{table*}[h]
    \small 
    \centering
    \begin{tabular}{lcccccc}
    \toprule
    & \multicolumn{2}{c}{AUC}  & \multicolumn{2}{c}{} & \multicolumn{2}{c}{Precision@N}\\
    & Macro & Micro & Macro & Micro & P@8 & P@15 \\
    \midrule
    CAML \cite{mullenbach-etal-2018-explainable} & 89.5 & 98.6 & 8.8 & 53.9 &  70.9 & 56.1 \\
    MSATT-KG \cite{xie2019ehr} & 91.0 & \textbf{99.2} & 9.0 & 55.3  & 72.8 & 58.1 \\
    MultiResCNN \cite{li2020icd} & 91.0 & 98.6 & 8.5 & 55.2  & 73.4 & 58.4 \\
    HyperCore \cite{cao2020hypercore} & 93.0 & 98.9 & 9.0 & 55.1  & 72.2 & 57.9 \\
    LAAT \cite{ijcai2020-461-vu} & 91.9 & 98.8 & 9.9 & 57.5  & 73.8 & 59.1 \\
    JointLAAT \cite{ijcai2020-461-vu} & 92.1 & 98.8 & \textbf{10.7} & 57.5  & 73.5 & 59.0  \\
\midrule
MSMN & \textbf{95.0} & \textbf{99.2}& 10.3& \textbf{58.4} & \textbf{75.2}& \textbf{59.9} \\
\bottomrule
    \end{tabular}
    \caption{Results on the MIMIC-III full test set.}
    \label{tab:full}
\end{table*}

\begin{table}
    \small 
    \centering
    \begin{tabular}{p{1.58cm}ccccc}
    \toprule
    & \multicolumn{2}{c}{AUC}  & \multicolumn{2}{c}{} &\\
     & Macro & Micro & Macro & Micro & P@5 \\
    \midrule
    CAML & 87.5 & 90.9 & 53.2 & 61.4 & 60.9 \\
    MSATT-KG & 91.4 & 93.6 & 63.8 & 68.4 & 64.4  \\
    MultiResCNN & 89.9 & 92.8 & 60.6 & 67.0 & 64.1 \\
    HyperCore & 89.5 & 92.9 & 60.9 & 66.3 & 63.2  \\
    LAAT & 92.5 & 94.6 & 66.6 & 71.5 & 67.5  \\
    JointLAAT & 92.5 & 94.6 & 66.1 & 71.6 & 67.1\\
\hline
MSMN & \textbf{92.8} & \textbf{94.7} & \textbf{68.3} & \textbf{72.5} & \textbf{68.0} \\
    \bottomrule \\
    \end{tabular}
    \caption{Results on the MIMIC-III 50 test set.}
    \label{tab:50}
\end{table}

\subsection{Main Results}



Table~\ref{tab:full} and \ref{tab:50} show the main results under the MIMIC-III full and MIMIC-III 50 settings, respectively. 
Under the full setting, our MSMN achieves 95.0 (+2.0), 99.2 (+0.0), 10.3 (-0.4), 58.4 (+0.9), 75.2 (+1.4), and 59.9 (+0.8) in terms of macro-AUC, micro-AUC, macro-, micro-, P@8, and P@15 respectively (parentheses shows the differences against previous best results), which shows that MSMN obtains state-of-the-art results in most metrics.
Under the top-50 codes setting, MSMN performs better than LAAT  in all metrics and achieves state-of-the-art scores of 92.8 (+0.3), 94.7 (+0.1), 68.3 (+1.7), 72.5 (+0.9), 68.0 (+0.5) on macro-AUC, micro-AUC, macro-, micro-, and P@5, respectively. 
We notice that the macro  has a large variance in every epoch under the MIMIC-III full setting since it is more sensitive in a long tail problem.





\subsection{Discussion}



To explore the influence of leveraging different numbers of code synonyms, we search  among  on the MIMIC-III 50 dataset. Results are shown in Table \ref{tab:ab}.
Compared with  that we only use the original ICD code descriptions, leveraging more synonyms from UMLS consistently improves the performance. Using  achieves the best performance in terms of AUC, and  achieves the best performance in terms of  and P@5. In addition, the median and mean count of UMLS synonyms are 5.0 and 5.4 respectively, which echoes why the results of  or  are better.






To evaluate the effectiveness of our proposed biaffine-based similarity function, we compare it with the baseline LAAT in Table \ref{tab:ab}. We also provide a simple function by removing  to  in Equation~\ref{eq:biaffine}. Results show that the biaffine-based similarity scoring performs best among others.



To better understand what MSMN learns from the multi-synonyms attention, we plot the synonym representations  under MIMIC-III 50 setting via t-SNE \cite{JMLR:v9:vandermaaten08a} in Figure~\ref{fig:example}.
We observe for some codes like \textit{585.9} (``chronic kidney diseases''), all synonym representations cluster together, which indicates that synonyms extract similar text snippets.
However, codes like \textit{410.71} (``subendocardial infarction initial episode of care'' or ``subendo infarct, initial'') and \textit{403.90} (``hypertensive chronic kidney disease, unspecified, with chronic kidney disease stage i through stage iv'' or ``unspecified orhy kid w cr kid i iv'') with very different synonyms learn different representations, which benefits to match different text snippets.
Furthermore, we observe it has similar representations for sibling codes \textit{37.22} (``left heart cardiac catheterization'') and \textit{37.23} (``rt/left heart card cath''), which indicates the model can also implicitly capture the code hierarchy.



\begin{table}
    \small 
    \centering
    \begin{tabular}{lccccc}
    \toprule
    & \multicolumn{2}{c}{AUC}  & \multicolumn{2}{c}{} & \\
    & Macro & Micro & Macro & Micro & P@5 \\
    \midrule
 & 92.1 & 94.2 & 67.4 & 71.0 & 67.0 \\
 & 92.6 & 94.6 & 67.6 & 71.7 & 67.2 \\
 & \textbf{92.8} & \textbf{94.7} & 67.9 & 71.9 & 67.7 \\
     & \textbf{92.8} & \textbf{94.7} & \textbf{68.3} & \textbf{72.5} & \textbf{68.0} \\
 & 92.5 & 94.6 & 66.9 & 71.5 & 67.6 \\
\midrule
     & \textbf{92.8} & \textbf{94.7} & \textbf{68.3} & \textbf{72.5} & \textbf{68.0} \\
 & 92.5& 94.5& 67.1& 71.2& 67.1 \\
     & 91.5& 94.1& 65.1& 70.8& 66.3 \\
    \bottomrule
    \end{tabular}
    \caption{Results of different settings including synonyms counts and scoring functions on MIMIC-III 50 dataset.
    Underlined setting denotes the default parameters used in MSMN.}
    \label{tab:ab}
\end{table}

\begin{figure}[t]
\centering
\includegraphics[width=2.6in]{figure/1_v2.pdf}
\caption{T-SNE visualization of code synonym representations learned from MIMIC-III 50.}
\label{fig:example}
\end{figure}

\subsection{Memory Complexity}
The memory usage of our proposed MSMN is dominated by Equation~\ref{eq:sim0} and Equation~\ref{eq:max}.
We suppose batch size as , word count as , label count as  and synonyms count as .
Calculating Equation~\ref{eq:sim0} for all  simultaneously requires calculating Einstein summation \cite{daniel2018opt} among tensors with shape  and shape  to shape .
Calculating Equation~\ref{eq:max} requires calculating Einstein summation among tensors with shape  and shape  to shape . The memory complexities of these two equations are linearly proportional to .

\section{Related Work}
Automatic ICD coding is an important task in the medical NLP community.
Earlier works use machine learning methods for coding \cite{larkey1996combining,pestian-etal-2007-shared,perotte2014diagnosis}.
With the development of neural networks, many recent works consider ICD coding as a multi-label text classification task. They usually apply RNN or CNN to encode texts and use the label attention mechanism to extract and match the most relevant parts for classification.
The label attention relies on the label representations as attention queries.
\citet{li2020icd,ijcai2020-461-vu} randomly initialize the label representations which ignore the code semantic information.
\citet{cao2020hypercore} use the average of word embeddings as label representations to leverage the code semantic information.
\citet{xie2019ehr,cao2020hypercore} use GCN to fuse hierarchical structures of ICD codes for label representations.
\citet{kim2021read} use CNN for codes representations.
Compared with previous works, 
we use synonyms instead of a single description to represent the code, which can provide more comprehensive expressions of codes.

Biomedical entity linking is a related task to automatic ICD coding. The task requires standardizing given terms to a pre-defined concept dictionary. There are two differences between biomedical entity linking and automatic ICD coding: (1) They have different target concepts. ICD coding map EMRs to ICD codes, while biomedical entity linking usually map terms to a larger dictionary like SNOMED-CT or UMLS. (2) They have different input formats. Entity linking task has labeled entities in texts, while ICD coding only provides texts. Synonyms have also been used in biomedical entity linking \cite{sung-etal-2020-biomedical,yuan2021coder}.
BioSYN \cite{sung-etal-2020-biomedical} uses marginalization to sum the probabilities of all synonyms as the similarity between a term and a concept.
However, we consider multi-synonyms attention to extracting different parts of clinical texts to interact with synonyms.

\section{Conclusions}
In this paper, we propose MSMN to leverage code synonyms from UMLS to improve the automatic ICD coding.
Multi-synonyms attention is proposed for extracting different related text snippets for code-wise text representations.
We also propose a biaffine transformation to calculate similarities among texts and codes for classification.
Experiments show that MSMN outperforms previous methods with label attention and achieves state-of-the-art results in the MIMIC-III dataset.
Ablation studies show the effectiveness of multi-synonyms attention and biaffine-based similarity.


\section*{Acknowledgements}
We would like to thank the anonymous reviewers for their helpful comments and suggestions.
We thank Fuli Luo, Shengxuan Luo, Hongyi Yuan, Xu Chen, and Jiayu Li for their help.
This work was supported by Alibaba Group through Alibaba Research Intern Program.



\bibliographystyle{acl_natbib}
\bibliography{anthology,acl2021}















\end{document}
