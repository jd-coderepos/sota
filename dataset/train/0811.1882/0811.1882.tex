




\documentclass[11pt]{article}
\usepackage{amsmath, amssymb, amsthm}
\usepackage{graphicx}

\usepackage{mathrsfs}

\renewcommand{\baselinestretch}{1.25}

\textwidth=6.5 in \textheight=9 in \hoffset=-.7 in \voffset=-.5 in

\newtheorem{thm}{Theorem}[section]
 \newtheorem{cor}[thm]{Corollary}
 \newtheorem{lem}[thm]{Lemma}
 \newtheorem{prop}[thm]{Proposition}
 \newtheorem{obs}[thm]{Observation}
 \theoremstyle{definition}
 \newtheorem{defn}[thm]{Definition}
 \newtheorem{exmp}[thm]{Example}
 \newtheorem{exer}{ Exercise}[section]
 \newtheorem{hardexer}[exer]{ Exercise}
 \theoremstyle{remark}
 \newtheorem{rem}[thm]{Remark}
 \newtheorem*{soln}{{\it\textbf{Solution}}}

\numberwithin{equation}{section}

\newcommand{\address}[1]
 { \vspace{-2em}\begin{center}
  \footnotesize{#1}
   \end{center}}

  \newcommand{\email}[1]
   {\vspace{-2em} \begin{center}
   \footnotesize{{\it E-mail address:} \texttt{#1}}
   \end{center}}

  \newcommand{\subjclass}[1]{\footnotesize{{\em 2000 AMS Mathematics
 Subject Classification:} {\bf #1}}}
 \newcommand{\keywords}[1]{\renewcommand{\baselinestretch}{1}
 \footnotesize{\noindent {\em Key words and phrases:} {#1}}}

\newcommand{\subjclassetc}[2]{\begin{figure}[b]
\footnoterule

\vspace{0.5em} \subjclass{#1}

\vspace{0.25em} \keywords{#2}
\end{figure}}



\newcommand{\C}{\circ}
 \newcommand{\To}{\longrightarrow}
 \newcommand{\Map}[3]{#1\, :\, #2\To #3}
 \newcommand{\A}{\alpha}
 \newcommand{\B}{\beta}
  \newcommand{\LL}{\mathcal{L}}
 \newcommand{\J}{\mathcal{J}}
 \newcommand{\D}{\mathcal{D}}
 \newcommand{\R}{\mathcal{R}}
 \newcommand{\HH}{\mathcal{H}}
\newcommand{\lcm}{\textrm{lcm}}
 \newcommand{\Real}{\mathbb{R}}
 \newcommand{\Rat}{\mathbb{Q}}
 \newcommand{\Nat}{\mathbb{N}}
 \newcommand{\Int}{\mathbb{Z}}
 \newcommand{\Comp}{\mathbb{C}}
 \newcommand{\abs}[1]{\left\vert#1\right\vert}
 \newcommand{\set}[1]{\left\{#1\right\}}
 \newcommand{\Set}[2]{\set{#1\ \vert\ #2}}
 \renewcommand{\atop}[2]{\genfrac{}{}{0pt}{}{#1}{#2}}
 \newcommand{\EQ}[1]{}






 \newcommand{\Hugecross}{\mbox{\Huge }}
 \newcommand{\hugecross}{\mbox{\huge }}
 \newcommand{\LARGEcross}{\mbox{\LARGE }}
 \newcommand{\Largecross}{\mbox{\Large }}
 \newcommand{\largecross}{\mbox{\large }}

 \newcommand{\Hugecrossbf}{\boldsymbol{\Hugecross}}
 \newcommand{\hugecrossbf}{\boldsymbol{\hugecross}}
 \newcommand{\LARGEcrossbf}{\boldsymbol{\LARGEcross}}
 \newcommand{\Largecrossbf}{\boldsymbol{\Largecross}}
 \newcommand{\largecrossbf}{\boldsymbol{\largecross}}
 
 \newcommand{\Bigcross}{\raisebox{-.6ex}{}}
 \newcommand{\bigcross}{\mathop{\Bigcross}}
 
















\newcommand{\Mod}[3]{#1\equiv #2 \pmod{#3}}
\newcommand{\nMod}[3]{#1\not\equiv #2\ (\textrm{mod}\ #3)}
\newcommand{\Div}[2]{#1\,\big|\, #2}
\newcommand{\cgrp}[1]{\langle\,#1\,\rangle}
\newcommand{\BOX}[1]{\texttt{box}\,(#1)}



\title{\Large\bf Ferrers Dimension and Boxicity}

\author{Soumyottam Chatterjee and Shamik Ghosh}





\begin{document}

\maketitle

\vspace{-0.5em}
\address{Department of Electronics and Tele-Communications Engineering and Department of Mathematics,\\
Jadavpur University,
Kolkata - 700 032, India.}

\vspace{0.25em}
\email{soumyottamchatterjee@gmail.com, sghosh@math.jdvu.ac.in}



\renewcommand{\baselinestretch}{1}
\begin{abstract}
This note explores the relation between the boxicity of undirected graphs and the Ferrers dimension of digraphs.

\vspace{0.5em}\noindent
{\footnotesize {\bf Keywords:}\ \ Interval Graph, Ferrers digraph, Ferrers dimension, Boxicity.}
\end{abstract}



\renewcommand{\baselinestretch}{1.25}



\section{Introduction}

An undirected graph  is an {\em interval graph} if and only if it is the intersection graph of a family of intervals on the real line. Each vertex is assigned an interval and two vertices are adjacent if and only if their corresponding intervals intersect. Motivated by theoretical as well as practical considerations, graph theorists have tried to generalize the concept of interval graphs in many ways. In many cases, representation of a graph as the intersection graph of a family of geometric objects, which are generalizations of intervals is sought. An example is the concept of boxicity introduced by F. S. Roberts in 1969 \cite{F}. For a graph , its {\em boxicity}  is the minimum positive integer  such that  can be represented as the intersection graph of axis-parallel b-dimensional boxes. Here a b-dimensional box is a Cartesian product  where each  is a closed interval on the real line. The boxicity of a complete graph may be assumed to be zero and since a one-dimensional box is a closed interval on the real line, graphs of boxicity at most 1 are exactly the interval graphs.

\vspace{1 em} Introduced independently by Guttman \cite{G} and Riguet \cite{R}, a {\em Ferrers digraph}  is a directed graph (in short, digraph) whose successor sets are linearly ordered by inclusion, where the successor set of  is its set of out-neighbors . It is easy to see that the successor sets are linearly ordered by inclusion if and only if the analogously defined predecessor sets are linearly ordered by inclusion, and that both are equivalent to the transformability of the adjacency matrix by independent row and column permutations to a -matrix in which the 1's are clustered in a corner in the shape of a Ferrers diagram (hence the term `Ferrers digraph'). It is well-known that every digraph  is the intersection of a finite number of Ferrers digraphs and the minimum such number is its {\em Ferrers dimension}. It is known \cite{R} that a digraph D is a Ferrers digraph if and only if its adjacency matrix does not contain any  permutation matrix:

The digraphs of Ferrers dimension at most 2 were characterized by Cogis \cite{C}. He called every  permutation matrix a {\em couple} and defined an undirected graph , the graph {\em associated to a digraph}  whose vertices correspond to the 0's of its adjacency matrix with two such vertices joined by an edge if and only if the corresponding 0's belong to a couple. Cogis \cite{C} proved that  is of Ferrers dimension at most 2 if and only if  is bipartite. In the general case, if , then there exist Ferrers digraphs , , such that  can be expressed as . Zeros belonging to any particular  do not form any couple among themselves and consequently form an independent set in . Thus  where  is the chromatic number of . No instance has been found yet for which the inequality is a strict one and it is not known whether  for all digraphs . But from the above inequality, it follows that  whenever  contains . In fact,  if , as  cannot exceed the number of 's of the adjacency matrix of .

\vspace{1em} Let  be a graph (directed or undirected). We denote the adjacency matrix of  by . For convenience, an entry of  corresponding to, say, the vertex  in the row and the vertex  in the column will be denoted by, simply, . The graph whose adjacency matrix is obtained by interchanging 's and 's of  will be denoted by . Note that if  has loops at all vertices (i.e., all principal diagonal elements of  are ), then  is a graph without loops (i.e., all principal diagonal elements of  are ) and vice-versa. Again for a digraph  with adjacency matrix , we denote the digraph whose adjacency matrix is  (the transpose of the matrix ) by .

\vspace{1em} Now we explore some nice relations between Ferrers digraphs and interval graphs. A digraph  is {\em oriented} if every arc of  has a unique direction (i.e.,  for any ). An oriented digraph  is {\em transitively oriented} if . An undirected graph  is {\em transitively orientable} if each edge of  can be assigned a one-way direction in such a way that the resulting digraph is transitively oriented. We call this digraph as a {\em transitive orientation} of . A transitive digraph  without loop at any vertex is an {\em interval order} digraph if . The class of interval order digraphs are transitive digraphs  such that  are interval graphs \cite{Fi} or equivalently, loopless Ferrers digraphs \cite{P}. Also if  is a Ferrers digraph without loops, then  is a Ferrers digraph with loop at every vertex. Further we know that an undirected graph is an interval graph if and only if it does not contain  (the cycle of length ) as an induced subgraph and its complement (called {\em co-interval} graph) is transitively orientable.\cite{A} Finally since every orientation of  is isomorphic to  (cf. Figure \ref{fig:d1}), we have the following observations:

\begin{obs}\label{obs:alpha}
Let  be an undirected graph (with loop at every vertex). Then  is an interval graph if and only if there exists a Ferrers digraph  (with loop at every vertex) such that .
\end{obs}

\begin{obs}\label{lem:beta}
A digraph without loop at any vertex is a Ferrers digraph if and only if it is transitively oriented and does not contain  as an induced subdigraph.
\end{obs}

\begin{figure}[h]
\begin{center}
\includegraphics*[scale=0.5]{arrow.eps}
\caption{The digraph }\label{fig:d1}
\end{center}
\end{figure}

Note that  itself is a transitively oriented digraph without loops  and the following is an example of an oriented digraph (without loops) which does not contain  as an induced subdigraph, but it is not a Ferrers digraph as it is not transitively oriented:

\begin{figure}[h]
\begin{center}
\includegraphics*[scale=0.5]{arrow3.eps}
\end{center}
\end{figure}

\noindent Moreover the following digraph is transitively oriented and does not contain  as an induced subdigraph, though it is not a Ferrers digraph.

\begin{figure}[h]
\begin{center}
\includegraphics*[scale=0.5]{arrow4.eps}
\end{center}
\end{figure}

The following are some interesting consequences of the above observations:

\begin{cor}\label{cor:chi}
Every transitive orientation of a co-interval graph (without loops) is a Ferrers digraph.
\end{cor}

\begin{cor}\label{cor:delta}
An undirected graph  (with loop at every vertex) is an interval graph if and only if  has an orientation of a Ferrers digraph (without loops). 
\end{cor}

The {\em intersection digraph}  of a family of ordered pairs of sets  is the digraph such that  if and only if . An {\em Interval digraph} is an intersection digraph of a family of ordered pairs of intervals on the real line. A bipartite graph (in short, {\em bigraph})  is an {\em intersection bigraph} if there exist a family  of sets such that  () if and only if . An {\em interval bigraph} is such when each  is an interval on the real line. The submatrix of the adjacency matrix of  consisting of the rows corresponding to one partite set and the columns corresponding to the other is known as the {\em biadjacency matrix} of . It should be noted that the two concepts intersection digraph and  intersection bigraph are basically equivalent \cite{P}. Indeed the bigraph whose biadjacency matrix is the adjacency matrix of a digraph corresponds to the digraph. Also given any bigraph, if the number of vertices of the sets  and  are not equal, we can make them equal by properly introducing some isolated vertices (correspondingly adding the required number of rows and columns consisting of all zeros in the biadjacency matrix of the bigraph) and then convert it into the adjacency matrix of a digraph. The bigraph corresponding to a Ferrers digraph is known as {\em Ferrers bigraph} and the {\em Ferrers dimension} of a bigraph  is the minimum number of Ferrers bigraphs whose intersection is . Now we shall observe an interesting relation between Ferrers bigraphs and interval graphs.

\begin{defn} 
Let  be a bigraph with biadjacency matrix . Then the graph with the following adjacency matrix is denoted by :

Clearly the graph  is obtained from the bigraph  by joining edges so that the partite sets of  become cliques and by adding loops at all vertices.
\end{defn}

Let  be a symmetric  matrix with 's in the principal diagonal. Then  is said to satisfy the {\em quasi-linear property for ones} if 's are consecutive right to and below the principal diagonal. It is known \cite{M} that an undirected graph  (with loop at every vertex) is an interval graph if and only if rows and columns of  can be suitably permuted (using the same permutation for rows and columns) in such a way that it satisfies the quasi-linear property for ones. Now if  is a Ferrers bigraph, then it is interesting to note that  is an interval graph (with loop at every vertex), as its adjacency matrix has quasi-linear property for ones.

\begin{figure}[h]
\begin{center}
\includegraphics*[scale=0.25]{fig4.eps}
\end{center}
\end{figure}

Conversely, consider an interval graph  whose vertices are covered by two disjoint cliques, say  and . We call such an interval graph, a {\em -clique interval graph}. Now since  is an interval graph, its maximal cliques are consecutively ordered. Let  be a consecutive linear ordering of maximal cliques. Assign (closed) intervals  to the each vertex  according to its first and last appearance in above sequence of maximal cliques. Let  and . 

\vspace{1em} Suppose . Now for every  and for all . We may restrict the right end point of each  up to  whenever it is exceeding  as every  contains  and each  has already a common point, namely , for every other vertex in . Similarly restrict the left end points of  up to  whenever it is lower than . 

\vspace{1em} With this new assignment of intervals for the interval graph , we go for further reduction. Now since for every , the left end point of  is  and  for all , safely we may fix all the left end points of  to  and similarly all right end points of  to .

\vspace{1em} Finally we arrange all the vertices of  in its adjacency matrix according to the lexicographic ordering (dictionary order) of the above constructed intervals. Thus the adjacency matrix of  (with loop at every vertex) takes the following form:

Moreover, in this matrix,  if and only if  which gives us  for all  and  for all . That is in each row of the submatrix , every  has only  to its right and every  has only  below it. What this says is nothing but the bigraph corresponding to the biadjacency matrix  is a Ferrers bigraph. The case for  is similar. In this case the vertices of  would come before those in  in the adjacency matrix of . Thus we have the following result:

\begin{obs}\label{t:fchar}
A bigraph  is a Ferrers bigraph if and only if  is a (-clique) interval graph.
\end{obs}

It is interesting to note that every -clique interval graph  is necessarily an {\em indifference graph}\footnote{Equivalently, a {\em proper interval graph} (an interval graph with an interval representation where no interval properly contains another) or a {\em unit interval graph} (which has an interval representation with all the intervals are of same length) or an interval graph which does not contain an induced copy of .} as  does not contain an induced  (since among any three vertices of , two of them must be in the same clique). 
Also since the bigraph complement (also called the {\em converse}) of a Ferrers bigraph is again a Ferrers bigraph, the above observation immediately gives the following:

\begin{cor}\label{c:fcomp}
A bigraph  is a Ferrers bigraph if and only if its graph complement is a -clique interval graph (with loop at every vertex).\footnote{The result is analogous to a known one which states that a bigraph  is of Ferrers dimension at most  if and only if its graph complement is a -clique circular-arc graph (with loop at every vertex).}
\end{cor}

In this note, we relate the two concepts - one corresponding to undirected graphs and the other to directed graphs - those of boxicity and Ferrers dimension respectively and propose a new construction for determining the Ferrers dimension of a digraph in the general case.

\section{Relating boxicity with Ferrers dimension}

An application to Observation \ref{obs:alpha} leads to the following theorem. Henceforth we denote the Ferrers dimension of a digraph  [bigraph ] by  [resp. ].

\begin{thm} \label{thm:alpha}
Let  be an undirected graph with loop at every vertex. Then there exists a digraph  such that  and . In general,  for any digraph  such that . Cosequently, 

\end{thm}

\begin{proof}Let . Then , where each  is an interval graph with loop at every vertex for . Also by Observation \ref{obs:alpha}, for each ,  for some Ferrers digraph  (with loop at every vertex). Then , where . As  can be expressed as the intersection of  Ferrers digraphs, . We show that  is exactly equal to . If possible, let  where . Then there exist Ferrers digraphs , for , for which . Now , where  for . Again since the graph  has loops at all the vertices and , the digraph  and hence each  also has loops at all its vertices. Then by Observation \ref{obs:alpha}, each 's is an interval graph. So  can be expressed as the intersection of  interval graphs, where , contrary to the fact that the boxicity of  is . Hence .

\vspace{1em} Moreover from the above deduction, it follows that, whenever  for some digraph , we have . This completes the proof.
\end{proof}

On the other hand the following is a consequence of Observation \ref{t:fchar}:

\begin{thm}
Let  be a bipartite graph. Then .
\end{thm}

\begin{proof}
Suppose the bigraph  is of Ferrers dimension . Then  for some Ferrers bigraphs , {\small ()} which implies . Since each  is an interval graph by Observation \ref{t:fchar}, we have , if the graph  has boxicity . 

\vspace{1em} Conversely, if  is the boxicity of , then  where each  is an interval graph. Also since their intersection (the graph ) has two cliques covering all the vertices, each  also contains same cliques for those vertices, i.e., each of them is a -clique interval graphs and the two cliques are consisting of the partite sets of . Thus it follows from Observation \ref{t:fchar} that  is the intersection of  Ferrers bigraphs,  such that  for all . Therefore , as required.
\end{proof}

As an immediate consequence of the above theorem we obtain certain  characterizations of bigraphs of Ferrers dimension  and interval bigraphs.

\begin{cor}
A bipartite graph  is of Ferrers dimension at most  if and only if  is a -clique rectangular graph.\footnote{A {\em rectangular graph} is an intersection graph of rectangles in . A {\em -clique rectangular graph} is a rectangular graph whose vertices are covered by two disjoint cliques.}
\end{cor}

\begin{cor}
A bipartite graph  is an interval bigraph if and only if  is a -clique rectangular graph such that there is a rectangular representation of  in which for every pair of rectangles, their projections intersect on at least one of the axes. 
\end{cor}

\begin{proof}
The proof follows from the fact that

\vspace{1em} \begin{tabular}{cl}
 &  is an interval bigraph\\
 &  where  and  are two Ferrers bigraphs whose union is complete \cite{M}\\
 &  for two Ferrers bigraphs,  with  is complete\\
 &  where  and  are (-clique) interval graphs whose union is complete.
\end{tabular}

\end{proof}

Let  be an undirected graph. Denote the corresponding (symmetric) digraph with the same adjacency matrix as that of  by .

\begin{thm} \label{thm:beta}
Let  be an undirected graph  (with loop at every vertex) such that . Let  be the corresponding digraph with the same adjacency matrix as that of  and . Then 

and the bounds are tight.
\end{thm}

\begin{proof}
Since ,  can be expressed as , where each  is an interval graph and so  for some Ferrers digraphs (with loop at every vertex) for . Then  which implies , i.e., . The limit is reached in the case of  as  and from the following adjacency matrix of  it is clear that  and hence .



\vspace{1em}As for the upper bound, let . Then . Since  is symmetric, . Now . Also  as  has loops at all its vertices and hence  is complete. So  which implies . Thus  where  for . Since each  has loop at every vertex, we have each  is an interval graph by Observation \ref{obs:alpha}. Therefore .

\vspace{1em} This limit is reached for  (the cycle of length 6). Since  is not an interval graph, but it can be easily obtained as an intersection graph of -dimensional boxes, we have . Now  where , are Ferrers digraphs as represented below:

\vspace{1em}\noindent {\small }

\noindent So . Again  where  the following submatrix of :


\noindent Therefore  and hence , as required.
\end{proof}

\section{A construction to determine the Ferrers dimension of a directed graph}

Let  be a digraph and  be the graph associated to . The following example shows that not every color class in a given coloring of  forms a Ferrers digraph.

\begin{exmp}
Let us consider the digraph  whose adjacency matrix  is given below:

The associated graph  is given by:
\begin{figure}[h]
\begin{center}
\includegraphics*[scale=0.5]{counter.eps}
\end{center}
\end{figure}

\noindent Consider the -coloring of  with color classes  and . Now zeros of  corresponding to the color class ,  forms the digraph whose adjacency matrix,

shows that it is not a Ferrers digraph.
\end{exmp}

Thus it is clear that if a color class has to correspond a Ferrers digraph, it must contain all the zeros, which, among themselves, ensure the absence of couples. More precisely, if zeros  and  () are in the same color class, either  or  or both must also be in that same color class. In view of this observation, we modify the construction of Cogis and introduce the directed graph  instead of the undirected graph  corresponding to a digraph .

\begin{defn}Let  be a digraph. We define a digraph  with vertex set  (i.e. the arcs of ) and there is an arc from  to  if and only if  and . \footnote{ may not be all distinct.}
\end{defn}

It is clear that for any subdigraph  of ,  is also a subdigraph of . Also  becomes an induced one whenever  is an induced subdigraph of .

\begin{exmp}Let us consider the following digraph . Then the corresponding  is obtained as follows:
\begin{figure}[h]
\begin{center}
\includegraphics*[scale=0.4]{d.eps}\hspace{1in}
\includegraphics*[scale=0.4]{jdd.eps}

 \hspace{3in} 
\end{center}
\end{figure}
\end{exmp}

\vspace{-2em} Certainly not all induced subdigraphs of  are of the form  for some subdigraph  of . For example, the induced subdigraph of  with the vertex set  is not of the form  for any subdigraph  of .

\begin{defn} A subdigraph  of  with vertex set  is called an {\em ideal} subdigraph if

We note that for any subdigraph  of ,  is an ideal subdigraph of .
\end{defn}

\begin{defn} An ideal subdigraph  of  is called {\em total} if for any  in , we have  or  or both (i.e., ). Let . Then the {\em total covering number} of  is the minimum number of total subdigraphs of  needed to cover , i.e., the vertex set of .
\end{defn}

For a digraph , it is not known (as we mentioned above) that whether , i.e., the clique covering number of . But we have the following result:
\begin{thm} \label{thm:beta2}
Let  be a digraph. Then the Ferrers dimension of  is equal to the total covering number of .
\end{thm}

\begin{proof}
Let  and the total covering number of  be . Then  for some Ferrers digraphs , . Hence . Now we consider the subdigraph  of . We have,  is an ideal subdigraph of  as  is a subdigraph of . We claim that  is a total subdigraph of , whence it will follow that .

\vspace{1em}Let , . Then  and so there are zero entries in the positions  and  the adjacency matrix of . We have the following three cases:


For the last two cases,  and in the first case, since  is a Ferrers digraph,  or  (or both) must be equal to zero. Thus  or  (or both) , which implies  or  (or both) in . Therefore  is a total subdigraph of  and the claim is verified.

\vspace{1em}Next, let  be a total covering of . For each , we define the subdigraph  of  with the vertex set same as that of  and edges which are belonging to the vertex set of , i. e., , where . We show that  is a Ferrers digraph by the method of contradiction. We assume that  is not a Ferrers digraph so that there is a couple

in the adjacency matrix of . Then , where . Since  is total, we have,  or  (or both) must belong to  as  or  in . This contradiction proves our assertion. Also since this covering covers all vertices of , i.e., all the edges of , we have , where each  is a Ferrers digraph, so that . Finally, since the complement of a Ferrers digraph is again a Ferrers digraph, we have . This completes the proof.
\end{proof} 

One final remark is that the undirected graph obtained from  by ignoring the directions of the arcs is the same as the complement of the graph . Now the chromatic number of  is the clique covering number of , which is less than or equal to the total covering number of  as every total subdigraph of , made undirected by ignoring the direction of the arcs, is a clique in , but the converse may not be true.

\vspace{2em}
\begin{thebibliography}{10}\label{bibliography}

\bibitem{C}
O. Cogis, \emph{A characterization of digraphs with Ferrers dimension 2}, Rapport de Recherche, \textbf{19}, G. R. CNRS no. 22, Paris, 1979.

\bibitem{Fi} P. C. Fishburn, \emph{Interval orders and interval graphs}, John Wiley \& Sons, New York, 1985.

\bibitem{A}
A. Ghouil-Houri, \emph{Caractrisation des graphes non orientes dont on peut orienter les artes de manire  obtenir le graphe d'une relation d'ordre}, C. R. Acad. Sci. Paris \textbf{254}, 1370-1371.

\bibitem{G}
L. Guttman, \emph{A basis for scaling quantitative data}, Am. Social. Rev. \textbf{9} (1944), 139-150.

\bibitem{H}
G. Hajs, \emph{ber eine Art von Graphen}, \textit{Int. Math. Nachr.}, \textbf{11}, Problem 65, 1957.

\bibitem{Ma}
E. Marczewski, \emph{Sur deux proprits des classes d'ensembles}, Fund. Math. \textbf{33} (1945), 303-307.

\bibitem{MR}
B. G. Mirkin and S. N. Rodin, \emph{Graphs and Genes}, Springer-Verlag, New York, 1984.

\bibitem{P} E. Prisner, A journey  through intersection graph county, Web
manuscript.

\bibitem{R}
J. Riguet, \emph{Les Relations des Ferrers}, C. R. Acad. Sci. Paris \textbf{232} (1951), 1729.

\bibitem{F}
F. S. Roberts, On the boxicity and cubicity of a graph in ``Recent Progresses in Combinatorics'' (W. T. Tutte, ed.) pp. 301-310, Academic Press, New York, 1969.

\bibitem{M}
M. Sen, S. Das, A. B. Roy and D. B. West, \emph{Interval Digraphs: An Analogue of Interval Graphs}, J. Graph Theory, \textbf{13} (1989), 189--202.

\end{thebibliography}

\end{document}
