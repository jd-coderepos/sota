\documentclass[conference,twocolum,final]{IEEEtran}


\usepackage{bbm}
\usepackage{algorithmic}
\usepackage{algorithm} 
\usepackage{amsthm}
\usepackage{amsmath,amssymb}
\usepackage{subfigure}
\hyphenation{op-tical net-works semi-conduc-tor IEEEtran pro-duct}

\ifCLASSINFOpdf 
\usepackage[pdftex]{graphicx} 
\usepackage[pdftex,colorlinks,
bookmarks=true,pdftitle=Optimal\ Geographic\ Caching\ In\ Cellular Networks,pdfauthor=B.\ Blaszczyszyn\ -\ A.\ Giovanidis]{hyperref}  
\usepackage[numbers,sort&compress]{natbib} 
\usepackage{hypernat} 
\else 
\newcommand\hypertarget[2]{#2} 
\newcommand\hyperlink[2]{#2} 
\usepackage[dvips]{graphicx} 
\usepackage[dvips,colorlinks,bookmarks=true,pdftitle=Optimal\ Geographic\ Caching\ In\ Cellular Networks,pdfauthor=B.\ Blaszczyszyn\ -\ A.\ Giovanidis]{hyperref}  
 \usepackage[numbers,sort&compress]{natbib} 
\fi 
\hypersetup{
   bookmarksnumbered,
   pdfstartview={FitH},
   citecolor={blue},
   linkcolor={red},
   urlcolor=[rgb]{0,0.55,0},pdfpagemode={UseOutlines}} 



\newtheorem{Pro}{Proposition}
\newtheorem{Exm}{Example}
\newtheorem{Rem}{Remark}
\newtheorem{Lem}{Lemma}
\newtheorem{Cor}{Corollary}
\newtheorem{Fact}{Fact}
\newtheorem{Def}{Definition}
\newtheorem{Quest}{Question}
\newtheorem{Theorem}{Theorem}
\newtheorem*{proof*}{Proof}
\newtheorem{Prob}{Problem}




\addtolength{\abovedisplayskip}{-2.5pt} \addtolength{\belowdisplayskip}{-2.5pt} \abovecaptionskip3pt \belowcaptionskip-12pt

\begin{document}



\title{Optimal Geographic Caching In Cellular Networks}

\author{Bart{\l}omiej~B{\l}aszczyszyn and Anastasios~Giovanidis\
\label{CovNum}
p_m:=\mathbb{P}\left[\mathcal{N}=m\right],\qquad m=0,1,\ldots

\label{SumPm}
\sum_{m=0}^{M} p_m = 1.

\label{PMFzipf}
a_j  =  A^{-1}j^{-\gamma}.

\label{sumAJ}
\sum_{j=1}^{J}a_j  =  1,
b_j:=\mathbf{P}\left(c_j\in\Xi^{(i)}\right)
\label{sumBKa}
\sum_{j=1}^{J}b_j \leq K, & \\
\label{sumBKb}
0\leq b_j\leq 1, & \forall j.
\mathbb{E}\left[\sum_{j=1}^J \mathbbm{1}(c_j\in\Xi)\right]=\sum_{j=1}^J\mathbf{P}\left(c_j\in\Xi\right)=\sum_{j=1}^{J}b_j\,.
\label{PerfMe}
f\left(b_1,\ldots,b_J\right) := 1-\sum_{j=1}^{J}a_j  \sum_{m=0}^{\infty} p_m\left(1-b_j\right)^{m}.

\label{CS1}
\mathcal{F}_1 & := & \left\{ (b_1,\ldots,b_J)|\ b_1+\ldots+b_J\leq K,\right.\nonumber\\
& & \left. \ \& \ b_j\in\left[0,1\right], \ \forall j\right\}

\label{EqConst}
b_1^*+\ldots+b_J^*  =  K.

\label{LagPerfMe}
L\left(b_1,\ldots,b_J,\mu\right) & = & \sum_{j=1}^Ja_j\left(1-\sum_{m=0}^{\infty}p_m\left(1-b_j\right)^{m}\right)+\nonumber\\
& +& \mu\left(K-\sum_{j=1}^Jb_j\right),

\label{MinMaxL}
f^*:=\max_{\mathcal{F}_1} f\left(b_1,\ldots,b_J\right) & = & \min_{\mu\geq 0}\max_{\mathcal{F}_2}L\left(b_1,\ldots,b_J,\mu\right).\nonumber\\
 & = & f\left(b_1^*,\ldots,b_J^*\right).

\label{SolveSubJJ}
b_j\left(\mu^*\right) & = & \left\{
\begin{tabular}{l l}
, & if \\
, & if \\
, & if  
\end{tabular}.
\right.

\label{Solve2}
a_j\sum_{m=1}^M p_m m (1-b_j)^{m-1}=\mu^*.

\label{Solve3}
b_1\left(\mu^*\right)+\ldots+b_J\left(\mu^*\right)  =  K.

\label{OFD2++}
b_1^* =\left\{\begin{tabular}{l l}
, & if .\\
, & otherwise
\end{tabular}\right.

\label{SINRdef}
\mathrm{SINR}(x_i):=\frac{S_{i}/\ell(r_i)}{W+I-S_i/\ell(r_i)}.

\label{NT}
\mathcal{N}\left(T\right)  =  \sum_{x_i\in\Phi}\mathbf{1}[\mathrm{SINR}(x_i)>T].

\label{SINRT}
M= \left \lceil{\frac{1}{T}}\right \rceil  \Leftrightarrow  m< 1+1/T.

\label{pSINR}
p_m^{\mathrm{SINR}} &:=  \mathbf{P}\left[\mathcal{N}\left(T\right)=m\right]\\
&=\sum_{n=k}^{\infty}  (-1)^{n-k}{n\choose k}\mathcal{S}_n(T)\,,\nonumber
\mathcal{S}_n(T)=\textstyle{\left(\frac{T}{1-(n-1)T}\right)^{-2n/\beta}}  \mathcal{I}_{n,\beta}(Wa^{-\beta/2})  \mathcal{J}_{n,\beta}\textstyle{\left(\frac{T}{1-(n-1)T}\right)}\label{In}
\mathcal{I}_{n,\beta}(x)=\frac{2^n
\int_0^{\infty} u^{2n-1}e^{-u^2-u^\beta x\Gamma(1-2/\beta)^{-\beta/2}} du
}{\beta^{n-1}(C'(\beta))^n(n-1)!}

 C'(\beta)=\frac{2\pi}{\beta\sin(2\pi/\beta)}=
\Gamma(1-2/\beta)\Gamma(1+2/\beta).
\label{e.Jn}
 \mathcal{J}_{n,\beta}(x)=\int_{[0,1]^{n-1}}
 \frac{   \prod\limits_{i=1}^{n-1}   v_i^{i(2/\beta+1)-1}(1-v_i)^{2/\beta}
  }{\prod\limits_{i=1}^{n-1} (x+\eta_i)} dv_1\dots
dv_{n-1}

\label{pBoo}
p_m^B  =  \frac{\nu^m}{m!}e^{-\nu}.

\label{2Net}
\mathbf{p}^{2NET} = \mathbf{p}^{(1)}*\mathbf{p}^{(2)} \ \Rightarrow \  p_m^{2NET} = \sum_{n=0}^M p_n^{(1)}p_{m-n}^{(2)}.



\subsection{Performance of the Content Placement Policies}
In this section 
we show the performance benefits of our scheme compared to a standard policy, the one that places in the cache memory of size , always  Most Popular Contents [MPC]. For the [MPC] policy,    and the objective function is always equal to . 
As shown in the following plots, when the user has significant probability to access more than one cache, the [MPC] is suboptimal. This result is intuitive, because a user covered by  BSs, can search in  memory slots instead of . 

 \begin{figure}[t!]    
\centering  
\label{CacheEval}
\subfigure{          
           \includegraphics[trim = 7mm 30mm 10mm 25mm, clip, width=0.3\textwidth]{CacheVScover.pdf}
           \label{CacheEval:2CPb}
           }
	   \subfigure{          
           \includegraphics[trim = 7mm 30mm 8mm 25mm, clip, width=0.3\textwidth]{HitProbVScover.pdf}
           \label{CacheEval:2CPf}
           }
\caption{Case [2PC]: (a) The optimal caching policy  and (b) The maximum hit probability  (objective function), with respect to the coverage ratio . The evaluation is done for different values of . In (b) the optimal hit probability value is compared with the one when [MPC] policy is applied.}
\end{figure} 



\subsubsection{Simple Scenario [2CP]}
We start by the solution of the [2CP] provided in Example~\ref{Cor1}.  In the simulated example, we assume  to be the probability that the typical user is not covered by any BS. Hence . A general picture of the way the optimal caching policy  varies w.r.t. the coverage ratio  and for different values of the popularity , is given in Fig. \ref{CacheEval:2CPb}. Here, the ratio  varies from  and we find for each value the optimal , given . We can deduce from the figure, that when each location is covered with high probability by 2 BSs, it is optimal to cache with probability . When each location is covered with high probability by a single BS, it is optimal to cache with [MPC], i.e. . 


In Fig. \ref{CacheEval:2CPf} we plot the objective function  of [2CP], given the solution  and for different values of . We compare the solution to the value of the objective function under the [MPC], which is always equal to , irrespective of the values of . From the figure we can observe a considerable performance improvement in the total hit probability, which is especially large when  is small (comparable popularities) and when  is small. 



\begin{figure*}[th!]    
\centering  
\label{CacheEval2}        
           \subfigure[Hit probability (Boolean).]{          
           \includegraphics[trim = 5mm 30mm 10mm 30mm, clip, width=0.3\textwidth]{FinalBoolHitVST.pdf}
           \label{CacheEval2:Bool}
           }
	   \subfigure[Hit probability ().]{          
           \includegraphics[trim = 7mm 30mm 10mm 30mm, clip, width=0.3\textwidth]{FinalSINRHitVST3.pdf}
           \label{CacheEval2:SINR}
           }
	   \subfigure[Hit probability (2NET).]{          
           \includegraphics[trim = 7mm 30mm 10mm 30mm, clip, width=0.3\textwidth]{FinalCONVHitVST.pdf}
           \label{CacheEval2:2NET}
           }
\caption{Evaluation of the optimal policy [GCP] and comparison with the [MPC] policy for the three different coverage models.}
\end{figure*}


\subsubsection{Boolean,  and Overlaid 2-Network Coverage}

We further evaluate the general problem [GCP] for the three coverage models suggested in Section \ref{secII}. We consider a content library of size  and cache memories of size . Since in all three models coverage depends on the threshold ratio , we use the latter as the variable on the x-axis. In all cases, increasing the service threshold  reduces the probability of coverage by  BSs, and consequently increases . Another important aspect is the relation of  with the transmission rate . These two are related through the Shannon formula 
,
where the transmission bandwidth is considered here equal to  MHz for typical applications.

In Fig. \ref{CacheEval2:Bool} we evaluate the total hit probability under the Boolean model, for which the values of  are calculated as in (\ref{pBoo}). We choose . The evaluation spans the threshold values . Compared to [MPC], we observe considerable gains in hit probability until , which corresponds to rate service of  Kbits/sec. Given that  Kbits/sec is a very high \textit{video} quality from YouTube \cite{YouTubeEncode}, our approach can realistically improve the backhaul cellular network  traffic under this model.


In Fig. \ref{CacheEval2:SINR} the same performance evaluation is done for the  model, where the probability of coverage is found in closed form in \cite{KeelerBartek13}. We use the software developed for MATLAB and available in \cite{KeelerMATLABk} to get the numerical values of . These are used as input to solve the [GCP]. We chose to evaluate the interference-limited case, i.e. , thus we consider . For numerical integration reasons, the minimum threshold is taken to be , which from (\ref{SINRT}) refers to at most  BSs covering a planar point. The maximum threshold value is  because due to (\ref{SINRT}) at most  BS can cover a planar point when . From the figure we see that the benefits are not very important and appear until , or equivalently  Kbits/sec. This rate refers to \textit{audio files} rather than video files, given that the a high quality encoded audio file has a rate of  Kbits/sec. The main reason for the poor performance is the generally low probability of coverage by more than one BS (around  at best). We conclude that in the  model without frequency reuse, it is optimal to use [GCP] for low bit rate content (audio) and [MPC] for high bit rate content (video).

Finally, Fig. \ref{CacheEval2:2NET} illustrates the performance gains when the [GCP] is applied to the case of coverage by 2 independent overlaid networks (2NET). The coverage probability  is given in (\ref{2Net}). 
For both vectors of the convolution, we use the same numerical values from  as calculated in the single  network above. Due to the convolution, the coverage probability for  is now increased, since most planar areas will be covered by at least two BSs. In such case the [GCP] policy has impressive benefits in the entire domain of , compared to the [MPC]. More than any other, this case emphasises the great potentials of optimal geographic caching of content.



\section{Conclusions}
In this work, we have revisited the problem of optimal content placement in caches within a cellular network. We exploited the fact that certain areas are covered by multiple BSs. An optimal policy is derived which suggests that when multi-coverage areas are significant, it is not optimal to cache the most popular contents everywhere. The total hit probability of the policy is evaluated in plots for three different coverage models (Boolean, , Overlaid 2-Network) and the results are highly in favour of our approach.


\label{secVI}


\bibliographystyle{unsrt}
\footnotesize
\bibliography{CacheStoGeoB.bib}

\end{document}
