\documentclass{llncs}
\usepackage{amsfonts}
\usepackage{amsmath}
\usepackage{amssymb}
\usepackage{algorithmicx}
\usepackage{algpseudocode}
\usepackage{graphicx}
\usepackage{gastex}
\usepackage{array}

\usepackage{listings}
\usepackage{hyperref}

\setcounter{page}{1}
\tolerance=1000

\newcommand{\lvec}[1]{\overset{{}_{\leftarrow}}{#1}}
\def\add{{\sf add}}
\def\eertree{{\sf eertree}}
\def\maxsuf{{\sf maxSuf}}
\def\maxpref{{\sf maxPref}}
\def\len{{\sf len}}
\def\link{{\sf link}}
\def\go{{\sf to}}
\def\occ{{\sf occ}}
\def\occas{{\sf occAsMax}}
\def\argmax{{\sf argmax}}
\def\sufc{{\sf sufCount}}
\def\prefc{{\sf prefCount}}
\def\flag{{\sf flag}}
\def\pop{{\sf pop}}
\def\quick{{\sf quickLink}}
\def\series{{\sf seriesLink}}
\def\direct{{\sf directLink}}
\def\richa{{\sf richAfterAdd}}
\def\addv{{\sf addVersion}}
\def\search{{\sf searchTree}}
\def\pred{{\sf pred}}
\def\symb{{\sf symb}}
\def\ans{{\sf ans}}
\def\diff{{\sf diff}}
\def\getmin{{\sf getMin}}
\def\res{{\sf res}}
\def\ddp{{\sf dp}}
\def\newv{{\sf newNode}}
\def\nod{{\sf node}}
\def\cT{{\cal T}}





\begin{document}

\thispagestyle{empty}

\title{EERTREE: An Efficient Data Structure for Processing Palindromes in Strings} 
\author{{Mikhail Rubinchik and Arseny M. Shur \\Ural Federal University}}

\author{Mikhail Rubinchik \and Arseny M. Shur}

\institute{Ural Federal University, Ekaterinburg, Russia\\ \email{mikhail.rubinchik@gmail.com}, \email{arseny.shur@urfu.ru}}
\maketitle

\vspace*{-4mm}
\begin{abstract}
We propose a new linear-size data structure which provides a fast access to all palindromic substrings of a string or a set of strings. This structure inherits some ideas from the construction of both the suffix trie and suffix tree. Using this structure, we present simple and efficient solutions for a number of problems involving palindromes.\ \label{eq:occ}
\occ[v]=\occas[v]+\sum_{u: \link[u]=v} \occ[u]\,.

\sum_{j=1}^{n-1} \sufc[\maxsuf[j]]\cdot \prefc[\maxpref[j{+}1]].
3pt]
\centerline{
\setlength{\extrarowheight}{3pt}
\begin{tabular}{c|c}
Problem&Solution\\
\hline
\begin{minipage}[t]{5.6cm}
\centering
Find the number of subpalindromes, common to all  given strings.	
\end{minipage}
&
\begin{minipage}[t]{6.2cm}
Build  and count the nodes having only 's in the  array.
\end{minipage}\3pt]
\hline
\begin{minipage}[t]{5.6cm}
For strings  and  find the number of palindromes  having more occurrences in  than in .
\end{minipage}
&
\begin{minipage}[t]{6.2cm}
Build , computing  and  in its nodes (see {\sf Palindromic Refrain} in Sect.~\ref{palrefrain}). Return the number of nodes  such that .
\end{minipage}\
\underbrace{\add(a), \ldots, \add(a),}_{n/3 \text{ times}}
\underbrace{\add(b), \pop(), \add(b), \pop(), \ldots, \add(b), \pop()}_{n/3 \text{ times}}
3pt]
\centerline{
\begin{tabular}{c|c|c|c}
Method
&
Time for  calls
&
Time for one call
&
Space for one node\\
\hline
basic &  &  but  & \\
quickLink &  &  but  &  \\
directLink &  &  & \\
\hline
\end{tabular} }

\smallskip
The basic version is the simplest one and uses the smallest amount of memory. Quick and direct links work somewhat faster, but their main advantage is that any single call is cheap, and thus can be reversed without much pain. Hence, one can easily maintain an eertree for a string with both operations  and . Indeed, let  push to a stack the node containing  and, if  is a new palindrome, the node containing  such that . This takes  additional time and space. Then  reads this information from the stack and restores the previous state of the eertree in constant time.

The table above also suggests the question whether some further optimization of the obtained algorithms is possible.

\begin{question}
Is there an online algorithm which builds an eertree spending  time and  space for any call to ? 
\end{question}


\subsection{Enumerating rich strings}

By Lemma~\ref{nnodes}, the number of distinct subpalindromes in a length  string is at most . Such strings with exactly  palindromes are called \emph{rich}. Rich strings possess a number of interesting properties; see, e.g., \cite{DJP01,GJWZ09}. The sequence A216264 in the Online Encyclopedia of Integer Sequences \cite{Slo} is the growth function of the language of binary rich strings, i.e., the th term of this sequence is the number of binary rich strings of length . J. Shallit computed this function up to , thus enumerating several millions of rich strings. Using the results of Sect.~\ref{ssec:del}, we were able to raise the upper bound to , enumerating several \emph{trillions} of rich strings in 10 hours on an average laptop. The new numerical data shows that this sequence grows much slower than it was expected before.

Proposition~\ref{pro:rich} below serves as the theoretic basis for such a breakthrough in computation. It is based on the following obvious corollary of Lemma~\ref{nnodes}.

\begin{lemma}\label{rich}
Any prefix of a rich string is rich.
\end{lemma}

\begin{proposition} \label{pro:rich}
Suppose that  is the number of -ary rich strings of length , for some fixed  and . Then the trie built from all these strings can be traversed in time . 
\end{proposition}

\begin{proof}
For simplicity, we give the proof for the binary alphabet. The extension to an arbitrary fixed alphabet is straightforward. Consider the following code, using an eertree on a string with deletions.

\begin{lstlisting}
void calcRichString(i) 
	ans[i]++
	if (i < n)
		if (add('0') )
			calcRichString(i + 1)
		pop()
		if (add('1') )
			calcRichString(i + 1)
		pop()
\end{lstlisting}

Here  is the length of the currently processed rich string. Recall that  appends  to the current eertree and returns the number of new palindromes, which is 0 or 1. Hence the modified string is rich if and only if  returns 1. Note that any added symbol will be deleted back with . So we exit every particular call to  with the same string as the one we entered this call. As a result, the call  traverses depth-first the trie of all binary rich strings of length . 

As was mentioned in Sect.~\ref{ssec:del}, the  operation works in constant time. For  we use the method with direct links. Since the alphabet is constant-size, the array  can be copied in  time. Hence,  also works in  time. The number of 's equals the number of 's, and the latter is twice the number of rich strings of length . The number of other operations is constant per call of , so we have the total  time bound.\qed 
\end{proof}

\begin{remark}
Visit \texttt{http://pastebin.com/4YJxVzep} for an implementation of the above algorithm. In 10 hours, it computed the first 58 terms of the sequence A216264. To increase the number of terms to 60, we used a few optimization tricks which reduce the constant in the -term. We do not discuss these tricks here, because they make the code less readable.
\end{remark}


\subsection{Persistent eertrees}

In Sect.~\ref{ssec:del} we build an eertree supporting deletions from a string. A natural generalization of this approach leads to \emph{persistent} eertrees. Recall that a persistent data structure is a set of ``usual'' data structures of the same type, called \emph{versions} and ordered by the time of their creation. A call to a persistent structure asks for the access or update of any specific version. Existing versions are neither modified nor deleted; any update creates a new (latest) version.
 
Consider a \emph{tree of versions}  whose nodes, apart from the root, are labeled by symbols. The tree represents the set of versions of some string : each node  represents the string read from the root to . Recall that we denote a node of a data structure by the same letter as the string related to it. Note that some versions can be identical except for the time of their creation (i.e., for the number of a node). The problem we study is maintaining an eertree for each version of . More precisely, the function  to be implemented adds a new child  labeled by  to the node  of  and computes . The data structure which performs the calls to , supporting the eertrees for all nodes of , will be called a \emph{persistent eertree}. Surprisingly enough, this complicated structure can be implemented efficiently in spite of the fact that the current string cannot be addressed directly for symbol comparisons.

\begin{proposition}
The persistent eertree can be implemented to perform each call to  in  time and space.
\end{proposition}

\begin{proof}
We use the method with direct links and build, as in Sect.~\ref{common}, a joint eertree for all versions. Each node of the tree  stores links to the palindromes of the corresponding version of . Overall, the node  of  contains the following information: a binary search tree , containing links to all subpalindromes of ; link  to the maximal suffix-palindrome of ; array , whose th element is the link to the predecessor  of  such that the distance between  and  in  is  (); and the symbol  added to the parent of  to get . All listed parameters except for  use  space. For search trees we use, as in Sect.~\ref{ssec:del}, the persistent tree \cite{DSST89}, reducing both time and space for copying the tree and inserting one element to  . (Recall that another persistent tree is used inside the eertree for storing direct links of all nodes.)

Now we implement  in time . Note that for any  the symbol  can be found in  time. Indeed, this symbol is , where  is the predecessor of  such that the distance between  and  is . Using the binary representation of , we can reach  from  in at most  steps following the appropriate  links. 

Let  be the current number of versions (at any time). Creating a new version  with the parent , we increment  by one and compute all parameters for . First we compute . This can be done in  time because  and  for .

To compute the palindrome , we call  for the string . Let  be the parent of  in the eertree. Then  if  is preceded by  in  and  otherwise. Hence, to compute  we access exactly one symbol of . Further, if  is not in the eertree, a new node of the eertree should be created for . It is easy to see that . Next,  is copied from , with one element replaced by . To find this element, we need to know the letter of  preceding . Therefore, to find  and modify eertree if necessary, we need  time for accessing a constant number of symbols in  and  time for the rest of computation in . Finally, we create a version of the search tree for , updating the version for  with  (if  is in the search tree for , this tree is copied to the new version without changes). This operation takes  as well. The proposition is proved. The code for  is given below.\qed
\end{proof}

\begin{lstlisting}[escapeinside={/*@}{@*/}]
void getpred(v, par)
	pred[v][0] = par
	i = 1
	while (pred[v][i] > 0)
		pred[v][i + 1] = pred[ pred[v][i] ][i]
		i++
int addVersion(v, c)
	t++ // /*@the number of versions, initialized by 0@*/
	u = t	
	symb[u] = c
	pred[u] = getpred(u, v)
	if (c == v[len[v] - len[maxSuf[v]]])
		x = maxSuf[v] 
	else 
		x = directLink[maxSuf[v]][c]
	maxSuf[u] = /*@to[x][c] //created if does not exist@*/
	searchTree[u] = insert(searchTree[v], maxSuf[u])
	return u
\end{lstlisting}



\section{Factorizations into Palindromes}\label{splittingSection}

As was mentioned in the introduction, the -factorization problem can be solved online in  time for the length  string and any  \cite{KRS15}. In this section we are aimed at solving this problem in time independent of . This setting is motivated by the fact that the expected palindromic length of a random string is  \cite{Rav03}, and the  asymptotics is quite bad for such big values of . On the positive side, the palindromic length of a string , which is the minimum  such that a -factorization of  exists, can be found in  time \cite{FGKK14}. 



\subsection{Palindromic length vs -factorization}

\begin{lemma} \label{kplus2}
Given a -factorization of a length  string , it is possible, in  time, to factor  into  palindromes for any positive integer  such that .
\end{lemma}

\begin{proof}
Let  be palindromes, , . It is sufficient to show how to factor  into  palindromes. If  for some , then we split  into three palindromes: the first letter, the last letter, and the remaining part. Otherwise, there are some ,  of length 2, each of which can be split into two palindromes.\qed
\end{proof}

Thus, -factorization problem is reduced in linear time to two similar problems: factor a string into the minimum possible odd (resp. even) number of palindromes. We solve these two problems using an eertree. To do this, we first describe an algorithm, based on an eertree and finding the palindromic length in time . While its asymptotics is the same as of the algorithm of \cite{FGKK14}, its constant under the -term is much smaller (see Remark~\ref{ficiConst} below) and its code is simpler and shorter.

\begin{proposition}
Using an eertree, the palindromic length of a length  string can be found online in time . 
\end{proposition}

\begin{proof}
For a length  string  we compute online the array  such that  is the palindromic length of . Note that any -factorization of  can be obtained by appending a suffix-palindrome  of  to a -factorization of . Thus, .

To compute  efficiently, we store two additional parameters in the nodes of the eertree: \emph{difference}  and \emph{series link} , which is the longest suffix-palindrome of  having the difference unequal to . Series links are similar to quick links, which are not suitable for the problem studied. Clearly, the difference is computable in  time and space on the creation of a node; the following code shows that the same is true for the series link.

\begin{lstlisting}
if (diff[v] == diff[link[v]])
	seriesLink[v] = seriesLink[link[v]]
else 
	seriesLink[v] = link[v]
\end{lstlisting}

The following ``naive'' implementation computes  in  time.
\begin{lstlisting} [escapeinside={/*@}{@*/}]
ans[n] = /*@@*/ 
for (v = maxSuf; len[v] > 0; v = link[v])
	ans[n] = min(ans[n], ans[n - len[v]] + 1)
\end{lstlisting}

With series links, the same idea can be rewritten as follows:
\begin{lstlisting} [escapeinside={/*@}{@*/}]
int getMin(u)
	res = /*@@*/ 
	for (v = u; len[v] > len[seriesLink[u]]; v = link[v])
		res = min(res, ans[n -	 len[v]] + 1)
	return res
ans[n] = /*@@*/ 
for (v = maxSuf; len[v] > 0; v = seriesLink[v])
	ans[n] = min(ans[n], getMin(v))
\end{lstlisting}

The  function has linear-time worst-case complexity, and we are going to speed it up to a constant time. By the \emph{series} of a palindrome  we mean the sequence of nodes in the suffix path of  from  (inclusive) to  (exclusive). Note that  loops through the series of . Comparing  and , we can check whether the series of  contains just one palindrome. If this is the case, then  can be computed in  time. Hence, below we are interested in series of at least two elements. A suffix-palindrome  of  is called \emph{leading} if either  or  for some suffix-palindrome  of . We need four auxiliary lemmas.

\begin{lemma}\label{intersect}
If a palindrome  of length  is both a prefix and a suffix of a string , then  is a palindrome.
\end{lemma}

\begin{proof}
Let . Then , i.e.,  is a palindrome by definition.\qed
\end{proof}

\begin{lemma}\label{prevocc}
Suppose  is a leading suffix-palindrome of a string  and  belongs to the series of . Then  occurs in  exactly two times: as a suffix and as a prefix.\end{lemma}

\begin{proof}
Let . Then . Since , we have , so that the two mentioned occurrences of  touch or overlap. If there exist  such that  and , then  is a palindrome by Lemma~\ref{intersect}. This palindrome is a proper suffix of  and is longer than , which is impossible. \qed
\end{proof}

\begin{lemma} \label{maxseries} 
Suppose  is a leading suffix-palindrome of a string  and  belongs to the series of . Then  is a leading suffix-palindrome of .
\end{lemma}

\begin{proof}
If  is not leading, then the string  has a suffix-palindrome  with  and . Since  is both a prefix and a suffix of  and , clearly . Then  is a palindrome by Lemma~\ref{intersect}. Assume that  has a suffix-palindrome  which is longer than . Then  begins with , and this occurrence of  is neither prefix nor suffix of , contradicting Lemma~\ref{prevocc}. Therefore,  and , which is impossible because  is leading. This contradiction proves that  is leading.\qed
\end{proof}

\begin{lemma} \label{seriesway}
In an eertree, a path consisting of series links has length .
\end{lemma}

\begin{proof}
Follows from \cite[Lemma 6]{KRS15}, since any leading suffix-palindrome is also leading in terms of \cite{KRS15}. 
\end{proof}

By Lemma~\ref{seriesway}, the function  calls   times. Now consider an  time implementation of . Recall that it is enough to analyze non-trivial series of palindromes; they look like in Fig.~\ref{fig:series}. The first positions of all palindromes in the depicted series of  and  match (because ) except for the last palindrome in the series of .

\begin{figure}[hbt!]
\vspace*{-4mm}
\centering
\includegraphics[trim=140 625 140 129,clip]{series.pdf}
\vspace*{-2mm}
\caption{Series of a palindrome  in  and of  in . Leading palindromes of the next series are shown by dash lines. The function  returns the minimum of the values of  in the marked positions, plus one.}
\label{fig:series}
\end{figure}
\vspace*{-3mm}

We see that  steps before we already computed the minimum of all but one required numbers. If we memorize the minimum at that moment, we can use it now to obtain  in constant time. We store such a minimum as an additional parameter  of the node of the eertree, updating it each time the palindrome represented by a node becomes a leading suffix-palindrome. Lemmas~\ref{prevocc} and~\ref{maxseries} ensure that when we access  to compute , it is exactly the value computed  steps before. The computations with  can be performed inside the  function:

\begin{lstlisting}
int getMin(v)
	dp[v] = ans[n - (len[seriesLink[v]] + diff[v])] //last 
	if (diff[v] == diff[link[v]])// non-trivial series
		dp[v] = min(dp[v], dp[link[v]])
	return dp[v] + 1
\end{lstlisting}

Here  is initialized by the value of  in the position preceding the last element of the series of . It is nothing to do if this series does not have other elements; if it has, the minimum value of  in the corresponding positions is available in .
\end{proof}

\begin{remark} \label{seriestree}
Series links can replace quick links in the construction of eertrees. Recall that in the method of quick links (Sect.~\ref{ssec:del}) after checking the symbols in  preceding  and  we assign  to  and repeat the process until the required symbol is found or the node  is reached. With series links, the termination condition is the same, but the process is a bit different. We first put  and check the symbol before . Then we keep repeating two operations: check the symbol preceding  and assign  to . In this way, all ``skipped'' symbols, including the symbol preceding , equal the symbol preceding the previous value of . (This is due to periodicity of ; for details see, e.g., \cite[Sect.\,2]{KRS15}.) The number of iterations of the cycle equals the number of series of suffix-palindromes of , which is  by Lemma~\ref{seriesway}.
\end{remark}


\begin{remark}\label{ficiConst}
Let  be the number of series of suffix-palindromes for the string . Our computation of palindromic length\footnote{See {\tt http://ideone.com/xE2k6Y} for an implementation.} performs, on each step, the following operations. For the eertree: at most  symbol comparisons (Remark~\ref{seriestree}) and one ()-time access to a dictionary. For palindromic length:  calls to , which fills one cell in  and one cell in .

The algorithm by Fici et al. \cite[Figure 8]{FGKK14} on each step builds three arrays (), each containing  triples of numbers; totally  cells to be filled. So, our algorithm should work significantly faster. 
\end{remark}

Now we return to the -factorization problem.

\begin{proposition}
Using an eertree, the -factorization problem for a length  string can be solved online in time . 
\end{proposition}

\begin{proof}
The above algorithm for palindromic length can be easily modified to obtain both minimum odd number of palindromes and minimum even number of palindromes needed to factor a string. Instead of  and , one can maintain in the same way four parameters: , , , , to take parity into account. Now  (resp., ) uses  (resp., ), while  (resp., ) uses  (resp., ). The reference to Lemma~\ref{kplus2} finishes the proof.
\end{proof}


\subsection{Towards a linear-time solution}

A big question is whether palindromic length can be found faster than in  time. First of all, it may seem that the bound  for our algorithm is imprecise. Indeed, for building an eertree we scan only  suffix palindromes even when we use just suffix links (see the proof of Proposition~\ref{nlogsigma}). For palindromic length, on each step we run through all suffix-palindromes, but possibly skipping many of them due to the use of series links. Can this number of scanned palindromes be  as well? As was observed in \cite{FGKK14}, the answer is ``yes'' on average, but ``no'' in the worst case: processing any length  prefix of the famous \emph{Zimin word}, one should analyze   series of palindromes (all of them 1-element, but this does not help).

Below we design an  offline algorithm for building an eertree of a length  string  over the alphabet , getting rid of the  factor in online algorithms. Then we discuss ideas which may help to obtain the palindromic length from an eertree in linear time. The offline algorithm consists of four steps.



\noindent
1. Using Manacher's algorithm, compute arrays  and , where   (resp. ) is the radius of the longest subpalindrome of  with the center  (resp., ).\2pt]
3. Build the suffix array  and the  array for ; for the alphabet , this can be done in linear time \cite{PST07}. Recall that  is the length of the longest common prefix of  and .\2pt] 
4. Recall from Sect.~\ref{ssec:prop} that an eertree consists of two tries, containing right halves of odd-length and even-length palindromes, respectively. Build each of them using a variation of the algorithm, constructing a suffix tree from a suffix array and its  array \cite{KLAAP01}. The algorithm for odd-length palindromes is given below; the algorithm for even lengths is essentially the same, so we omit it.

\begin{lstlisting}[escapeinside={/*@}{@*/}]
path = (-1) // /*@stack for the current branch of the trie@*/
for (i = 1; i /*@@*/ n; i++)
	k = SA[i] // start processing palindromes centered at k
	while (path.size() > LCP[i] + 1)
		path.pop()
	for (j = path.size(); j /*@@*/ oddR[k]; j++) //can be empty
		path.push( newNode(path.top(), S[k + j - 1]) )
	for (j = 1; j /*@@*/ C[k].size(); j++)
		(rank, r) = C[k][j]
		node[rank][r] = path[r - k + 1]
\end{lstlisting}

The function  returns a new node attached to the node  with the edge labeled by . Array  (resp., ) contains links to the longest (resp., second longest) palindromes ending in given positions. Now estimate the working time of this algorithm. The outer cycle works  time plus the time for the inner cycles. The number of pop operations is bounded by  the number of pushes, and the latter is the same as the number of nodes in the resulting eertree, which is . The total number of iterations of the third inner cycle is the number of palindromes stored in the whole array ; this is exactly , see step 2. Thus, the algorithm works in  time. 

After running both the above code and its modification for even-length palindromes, we obtain the eertree without suffix links and the arrays , . From these arrays the suffix links can be computed trivially:

\begin{lstlisting}[escapeinside={/*@}{@*/}]
for (i = 1; i /*@@*/ n; i++)
	link[ node[1][i] ] = node[2][i];
\end{lstlisting}

Thus we have proved

\begin{proposition}
The eertree of a length  string over the alphabet  can be built offline in  time. 
\end{proposition}

Now return to the palindromic length. Even with an  preprocessing for building the eertree, we still need  time for factorization. Note that in \cite{KRS15} an  algorithm for -factorization was transformed into a  algorithm using bit compression (the so-called \emph{method of four Russians}). That algorithm produced a  bit matrix (showing whether a th prefix of the string is -factorable), so such a speed up method was natural. In our case we work with integers, so the direct application of a bit compression is impossible. However, we have the following property.

\begin{lemma}\label{splitting}
If  is a string of palindromic length  and  is a symbol, then the palindromic length of  is , or .
\end{lemma}

\begin{proof}
Any -factorization of  plus the substring  give a -factorization of . Suppose  has a -factorization  for a smaller . Then  has length . Hence, either  and  has the -factorization  or  for a palindrome  and  has the -factorization . The result now follows. \qed
\end{proof}

Consider a  bit matrix  such that  if and only if  is -factorable. For th column, we have to compute just two values: in the rows  and , where  is the palindromic length of  (if , we write  by Lemma~\ref{splitting}). For each value we should apply the OR operation to  bit values, to the total of  bit operations. If we will be able to arrange these operations naturally in groups of size , we will use the bit compression to get just  operations. So we end the paper with the following conjecture.


\begin{conjecture}
Using Lemma~\ref{splitting}, eertree and the method of four Russians it is possible to find palindromic length of a string in  time online and in  time offline.
\end{conjecture}

\section{Conclusion}

In this paper, we proposed a new tree-like data structure, named eertree, which stores all palindromes occurring inside a given string. The eertree has linear size (even sublinear on average) and can be built online in nearly linear time. We proposed some advanced modifications of the eertree, including the joint eertree for several strings, the version supporting deletions from a string, and the persistent eertree.

Then we provided a number of applications of the eertree. The most important of them are the new online algorithms for -factorization, palindromic length, the number of distinct palindromes, and also for computing the number of rich strings up to a given length.

For further research we formulated a conjecture on the linear-time factorization into palindromes and an open problem about the optimal construction of the eertree.

\paragraph*{Acknowledgments.} The authors thank A.~Kul'kov, O.~Merkuriev and G.~Nazarov for helpful discussions. 

\bibliographystyle{splncs03}
\bibliography{my_bib}

\end{document}
