\documentclass[secthm,amsthm,english]{article}

\usepackage{amsmath, amssymb, graphics}

\usepackage{xr}
\usepackage{amsthm}

\usepackage{enumerate}

\usepackage{graphicx}
\usepackage{psfrag}
\usepackage{epsf}
\usepackage{epsfig}


\externaldocument[II-]{partII}

\usepackage{anysize}
\marginsize{3.5cm}{3cm}{2cm}{3cm}


\newtheorem{theorem}{Theorem}
\newtheorem{observation}{Observation}
\newtheorem{lemma}{Lemma}
\newtheorem{proposition}{Proposition}
\newtheorem{conjecture}{Conjecture}
\newtheorem{corollary}{Corollary}

\theoremstyle{definition}
\newtheorem{definition}{Definition}

\theoremstyle{remark}
\newtheorem{remark}{Remark}


\begin{document}



\title{Cayley configuration spaces of 2D mechanisms\\
Part I: extreme points, continuous motion paths and minimal representations}


\author{Meera Sitharam, Menghan Wang, Heping Gao}







\maketitle


\begin{abstract}

We consider longstanding questions concerning configuration spaces of
\emph{1-degree-of-freedom tree-decomposable linkages} in 2D, an algebraically
well-defined class that commonly occurs in mechanical computer aided
design. By employing the notion of a \emph{Cayley configuration space}, i.e., a
set of intervals of realizable
distance-values for an independent non-edge, we answer the following.
 (1) How to measure the complexity of the configuration space and efficiently compute configuration
spaces of low algebraic
complexity? (2) How to define realization or orientation types that restrict the Cayley
configuration space to be a single interval? (3) How to efficiently obtain continuous motion paths between realizations?
(4) How to find a bijective representation of the Cartesian
realization space as a curve in an ambient space of minimum dimension? Such a representation would provide an efficient notion of
distance between connected components of
the realization space (5) How robust is the complexity measure from (1) and how to
efficiently classify and characterize linkages according to
this complexity measure?

In Part I of this paper, we deal with problems (1)--(4) by introducing the
notions of (a) Cayley size, i.e., the
number of intervals in the Cayley configuration space, (b) Cayley
computational complexity of computing the interval endpoints, as a function of the
number of intervals, and (c) Cayley
(algebraic) complexity of describing the interval endpoints.
Specifically
(i) We give an algorithm to determine the interval endpoints of 
a Cayley configuration space by characterizing the Cartesian realizations cor
responding to these endpoints. 
For graphs with low Cayley (algebraic) complexity, we give the following. 
(ii) A natural,
minimal realization type, i.e. a minimal set of local orientations, whose specification guarantees Cayley size
of 1 and
$O(|V|^2)$ Cayley computational complexity. Specifying fewer local
orientations
results in a superpolynomial blow-up of both the Cayley size
and computational complexity, provided P is different from NP. 
(iii) An algorithm--for generic linkages (as defined here)--to
find a path of continuous motion
(provided one exists) between two given realizations, in time linear in a natural measure
of the length of the path. We show that the number of such paths
is at most two. 
(iv)
A canonical
bijective representation of the Cartesian realization space in minimal ambient dimension,
also for generic linkages.
We provide a comparison of our results with relevant previous work.

\end{abstract}

\section{Introduction}
\label{sec:introduction}



A {\em linkage} $(G,\bar{l})$, is a graph $G=(V,E)$ with fixed
length bars as edges, i.e. $\bar{l}: E \rightarrow \mathbb{R}$. 
A 2D (Cartesian) \emph{realization} $G(p)$ of $G$ is an assignment of points $p \in \mathbb{R}^2$ to the vertices of $G$, 
satisfying the bar lengths in $\bar{l}$, i.e., for all edges $(u,v) \in G$, $\| p_u - p_v \| = \bar{l}(u,v)$ where $p_u,p_v \in \mathbb{R}^2$.
Note that a linkage may or may not have a 2D realization.

We will use standard and well-known terminology from geometric constraint solving and combinatorial rigidity. 
For detailed background in these fields we refer the reader to, for example,
\cite{bib:Graver} and \cite{bib:survey}.
 The \emph{degrees of freedom (dofs)} of a linkage
 is the minimum number of new bars that need to be added 
in order to ensure that a generic realization of the linkage (with the new bars) is \emph{rigid}, 
i.e., its bar lengths permit no motions other than the Euclidean or rigid body motions (translations or rotations), 
otherwise the linkage is \emph{flexible}. 






Describing and analyzing realization spaces of 1-dof linkages
or \emph{mechanisms} in 2D is 
a  difficult problem with a long history. 
In mechanical computer aided design, it represents a key underlying barrier
to understanding underconstrained geometric constraint system.
In fact, even for rigid linkages, the number of realizations can be exponential in the number of vertices
and not easy to estimate \cite{bib:Borcea}. 
One way to classify realizations is to use \emph{realization types} \cite{bib:navigation,bib:FudHo97},  
which uniquely determines a realization of a rigid linkage. 
For flexible linkages, 
a well-known early result \cite{kempe1875} 
shows that an arbitrary algebraic curve can be traced by the motion of a linkage joint. 
One outstanding example is the Peaucellier-Lipkin linkage, which transforms planar rotary motion into straight-line motion \cite{bib:Kempe}. 
For polygonal linkages, recent results on the variants of Carpenter's rule problem and pseudotriangulations yield spaces of non-crossing realizations and expansive motions \cite{bib:streinu05,streinu2000combinatorail,bib:straightening,bib:rote03}.
Versions of the problem play an important role in Computer-Aided-Design (CAD), 
robotics and molecular geometry \cite{bib:sacks10,bib:Yang,bib:survey}, but few results are known beyond individual or specific families of linkages \cite{bib:JoanArinyo03,bib:hilderick06,bib:henneberg,bib:ZhangGao06}. 

There are numerous examples of algorithms and software dealing with 1-dof linkages, 
such as Geometry Expressions, SAM,  Phun, Sketchpad, Geogebra, D-cubed, 
the algorithm in \cite{bib:hidalgo2011reachability}, etc. 
They have the following major functionalities: 
(i) designing 1-dof tree-decomposable linkages for tracing out specific curves, 
especially by building new mechanisms based on a library of existing ones;
(ii) accepting user-specified parameters, ranges and realization types to generate 
continuous motion of the linkages. 
However, the following issues still exist. 
(a) Currently, 
the realization space is typically represented as separate curves in 2D that are 
traced by each vertex of the linkage. 
In fact, 
a realization actually corresponds to a tuple of points, one on each of these curves.
I.e., the realization space is bijectively represented by a curve in the full \emph{ambient dimension} of $2|V|-3$ after factoring out rigid transformations, 
where $|V|$ is the number of vertices in the linkage. 
(b) Currently, 
for two realizations in different connected components, 
there is no method to find out how ``close" they can get towards each other by continuous motion, 
using a meaningful definition of ``distance" between connected components. 
(c) Currently, in order to generate continuous motion, 
the user must specify a range of a parameter containing the parameter value at the given realization.
Then either a single connected component is generated
for a subset of the specified range, 
or multiple segments of the realization space, under only the given realization type,  
are generated within the specified range. 
We discuss this issue in more detail in Section \ref{sub:contributions}.


\medskip


We study the \emph{Cayley configuration space}, first introduced in \cite{bib:GaoSi05,bib:GaoSitharam08a},
which is a set of intervals of possible distance-values for an independent non-edge. 
In both parts of this paper, we restrict ourselves to 1-dof linkages 
obtained by dropping a bar from minimally rigid, \emph{tree-decomposable linkages},
which are widely used in engineering and CAD, 
because they are \emph{quadratically-radically solvable} (\emph{QRS}, also called ruler-and-compass realizable),
i.e., the realizations are solutions to a
triangularized quadratic system with coefficients in $\mathbb{Q}$
(i.e. the solution coordinates belong to an extension field over
$\mathbb{Q}$ obtained by nested square-roots). 




Let $G  = (V,E)$. The graph $G\cup f$, is the graph $(V,E\cup\{f\})$, 
where the end
points of the new edge $f$ are known to be in $V$.
A reasonable way to describe the space of 2D realizations of a 1-dof
linkage $(G,\bar{l})$ is to take a pair of vertices whose distance is not fixed by the bars,
i.e., an {\em independent non-edge} $f$ with $G\cup f$ being minimally rigid, 
and ask for all the possible lengths $l_f$ that the non-edge $f$ can attain (i) over all the realizations of $(G, \bar{l})$; 
(ii) over all realizations of $(G, \bar{l})$ of a particular \emph{$T$-realization type}. A \emph{$T$-realization type} of a 2D realization $G(p)$,
where $T$ consists of triples of points in $p$, 
is a set $\sigma$  of \emph{local orientations} (chirality) $\sigma_t \in \{+1,-1,0\}$, each denoting the local orientation of a specific 
triple of points $t \in T$ (see Definition \ref{def:solution-type}). 


For (i) (resp. (ii)), we call each realizable length $l_f$ as a (resp. \emph{T-oriented}) \emph{Cayley configuration}, 
and the set of all such configurations as the (resp. \emph{T-oriented}) \emph{Cayley configuration space} of the linkage $(G,\bar{l})$ on $f$, parameterized by the distance $l_f$. 
It is a set of disjoint closed intervals on the real line.
Aside: the Cayley configuration space over $f$ 
is actually the \textsl{projection} of the Cayley-Menger semi-algebraic set \cite{bib:cayley}
associated with the linkage $(G,\bar{l})$ on the Cayley non-edge length parameter $f$. 


\begin{figure}[h]
	\psfrag{1}{$v_{1}$}\psfrag{2}{$v_{3}$}\psfrag{3}{$v_{2}$}\psfrag{4}{$v_{4}$}\psfrag{5}{$v_{5}$}\psfrag{6}{$v_{6}$}\psfrag{7}{$v_{7}$}\psfrag{8}{$v_{8}$}
	
	\begin{centering}
	\includegraphics[width=0.4\textwidth]{img/exponent-graph}
	\par\end{centering}
	
	\caption{A 1-dof linkage: adding any of the edges $(v_i,v_{1+2})$ would make the linkage rigid. 
	In Figure \ref{F:interval}, we show the Cayley configuration space over the dotted non-edge $(v_1,v_3)$}

\label{F:intro} 
\end{figure}

For example, the linkage $(G,\bar{l})$ in Figure \ref{F:intro} is 1-dof, 
and adding a bar between any of the pairs $(v_i, v_{i+2})$ would make it rigid.
In Figure \ref{F:interval},  
we choose $(v_1,v_3)$ as the non-edge $f$ of $(G,\bar{l})$ and
give the Cayley configuration space of this linkage over $f$,
consisting of the three intervals shown. 
It is important to note that Cayley configuration is not always a bijective representation, 
since each Cayley configuration can (generically) correspond to (finitely) many realizations: 
the figure shows arbitrarily chosen realizations corresponding to six Cayley configurations.
In Figure \ref{F:tracking}, 
we take \emph{forward realization types} (a special T-realization type defined in Definition \ref{def:solution-type})
$\sigma$ of realization (C1) from Figure \ref{F:interval} and $\tau$ of realization (C2) from Figure \ref{F:interval}, 
and show the two corresponding \emph{forward-oriented Cayley configuration spaces} of $(G,\bar{l})$
over $f$.  
Each of these oriented Cayley configuration spaces consists of a single interval. 




\begin{figure}[h]

	\begin{centering}
	\includegraphics[width=\textwidth]{screenshots/configs}
	\par\end{centering}
	
	\caption{
	The circled part is the Cayley configuration space of the 1-dof linkage in Figure \ref{F:intro} over $f = (v_1,v_3)$ 
	(demonstrated using our new CayMos software \protect\cite{bib:caymos}. 
	See also a demonstration video at http://www.cise.ufl.edu/\~{}menghan/caymos/definitions.avi.).  
	Each Cayley configuration corresponds to many realizations. 
	Arbitrarily chosen realizations for Cayley configurations: (A) $l_f=2$, (B) $l_f=3$, (C) $l_f=5.5$, (D) $l_f = 5.3$, (E) $l_f = 8$, (F) $l_f = 9$. }

\label{F:interval} 
\end{figure}

\begin{figure}[h]
	\psfrag{1}{$v_{1}$}\psfrag{2}{$v_{2}$}\psfrag{3}{$v_{3}$}\psfrag{4}{$v_{4}$}\psfrag{5}{$v_{5}$}\psfrag{6}{$v_{6}$}\psfrag{7}{$v_{7}$}\psfrag{8}{$v_{8}$}\psfrag{9}{$v_{9}$} 
	\psfrag{s1}{$\sigma$}\psfrag{s2}{$\tau$}
	
	\begin{centering}
	\includegraphics[width=0.7\textwidth]{img/tracking}
	\par\end{centering}
	
	\caption{Two oriented Cayley configuration spaces of the 1-dof linkage in Figure \ref{F:intro}, and   
unique realizations for oriented Cayley configurations (A1)(A2) $l_f = 5.3$, (B1)(B2) $l_f = 5.5$ (realizations (C1)(C2) from Figure \ref{F:interval}), (C) $l_f = 5.9$, (D) $l_f = 6.3$. }

\label{F:tracking} 
\end{figure}

\subsection{Questions}
\label{subsec:questions}

In this paper, we consider various natural questions about a well-known class of 1-dof linkages.

\begin{itemize}

	\item (In Part I)
	(1) Is there a robust measure of complexity of the Cayley configuration space of a linkage $(G,\bar{l})$ over $f$,
	 that depends only on the graph $G$ and the non-edge $f$, and not on the bar lengths $\bar{l}$? 	
	(2) How can we obtain a Cayley configuration space? 
	(3) How to define realization of orientation types 
	that restrict the corresponding oriented Cayley configuration space to be a single interval?
(4) How can we use a Cayley configuration space to obtain paths of continuous motion between realizations? 
	(5) How to obtain a canonical bijective representation of the Cartesian realization space using minimum ambient dimension?
	
	\item (In Part II) 
	(6) Is there a robust measure of complexity for Cayley configuration spaces,
	that does not even depend on the choice of non-edge $f$?
	(7) Is there a natural characterization and/or efficient algorithmic characterization 
	of graphs $G$ that have Cayley configuration spaces of low complexity?

\end{itemize}

\noindent In order to state our contributions more precisely, in the next section, we define several natural complexity measures for Cayley configuration spaces.
\subsection{Complexity measures for Cayley configuration spaces}


Consider the (oriented) Cayley configuration spaces of 1-dof linkages with underlying graph $G$ over a non-edge $f$. 
We take the following as measures of complexity:

\begin{enumerate}[(a)]
	\item \emph{Cayley size} of $G$, i.e. the maximum number of intervals in the complete description of the (oriented) Cayley configuration space, 
			over all possible linkages $(G, \bar{l})$ of $G$. 
	
	\item \emph{Cayley Computational Complexity} of $G$, the maximum overall time complexity of obtaining the complete  (oriented) Cayley configuration space, 
			over all possible linkages $(G, \bar{l})$ of $G$. 
			It can be regarded as a function of Cayley size.

	\item \emph{Cayley algebraic complexity} of $G$, i.e. the maximum algebraic descriptive complexity of each endpoint in the  (oriented) Cayley configuration space,  
			over all possible linkages $(G, \bar{l})$ of $G$. 
			Specifically, if the bar lengths $\bar{l}$ are in $\mathbb{Q}$, 
			it is desirable if the endpoints are solutions to a triangularized quadratic system with coefficients in $\mathbb{Q}$ 
			(i.e. the endpoints belong to an extension field over $\mathbb{Q}$ obtained by nested square-roots). 
			Such values are called \emph{quadratically-radically solvable}, or \emph{QRS}.
\end{enumerate}


Before investigating these complexity measures, 
we discuss two desirable requirements 
on the Cayley configuration space. 
First, for each Cayley configuration $l_f$,
there should exist only finitely many (could be exponential in $|G|$) 
realizations of $(G\cup f,\bar{l})$. 
This is guaranteed if the linkage $(G\cup f, \bar{l})$ is rigid. 
Second, with a specified realization type, 
there should exist a linear time algorithm to convert 
from a Cayley configuration $l_f$ to a corresponding Cartesian realization, 
As an example, the linkage in Figure \ref{F:intro} satisfies both requirements 
when we choose any non-edge $f=(v_i, v_{i+2})$ as the Cayley parameter, 
since there exists a simple ruler and compass realization of the linkage $(G \cup f, \bar{l})$ from any such $f$. 
For linkages with such a realization, the coordinate values of realizations are QRS, 
and we call such linkages \emph{QRS linkages}, and the underlying graphs \emph{QRS graphs}. 


With these two requirements in mind, 
we focus on a natural class of 1-dof linkages called \emph{1-dof tree-decomposable linkages}. 
The underlying graphs are obtained by dropping an edge from so-called \emph{tree-decomposable graphs} (formally defined in Section \ref{sec:Basic-properties}). 
Tree-decomposable graphs are 
minimally rigid and well-studied, for example, in geometric constraint solving for CAD, because they are QRS \cite{bib:FudHo97}. 
Conversely, QRS has
been shown \cite{bib:Owen02} to generically imply tree-decomposability
in the case of  planar graphs, and the implication is strongly conjectured for
all graphs. 
Our initial example in Figure \ref{F:intro} is a 1-dof tree-decomposable linkage.


\subsubsection{Model of Computation}

Our complexity measures are based on a model of computation that uses exact representation
of numbers in any quadratic extension field of the rational numbers. In other words,
we assume that all arithmetic operations, over extraction of square roots and comparison are
constant time, exact operations. This model of computation is not as strong as the real RAM
model that is normally used in computational geometry, that permits exact representation
of arbitrary algebraic numbers \cite{bib:loos1983computing}. Issues in exact geometric computation such as efficient
and robust implementation of such a representation, for example using interval arithmetic,
are beyond the scope of this manuscript.


\subsection{Contributions and novelty}
\label{sub:contributions}


\noindent\textbf{Contributions of Part I}:  
In Part I we only consider \emph{generic} 1-dof, tree-decomposable linkages. 
By a \emph{generic} linkage, we mean that no bar length is zero, all bars have distinct lengths and 
at most one pair of adjacent bars can be collinear in any realization. 
We answer the following questions posed in Section \ref{subsec:questions}. 


\begin{enumerate}[(1)]

	\item How to obtain a Cayley configuration space?
	
	We answer this question by giving two algorithms for obtaining a Cayley configuration space. 
	One algorithm works for any 1-dof tree-decomposable linkage. 
	The other only works for linkages whose underlying graphs have \emph{low Cayley complexity} -- 
	i.e. all interval endpoints in the corresponding (oriented) Cayley configuration spaces being tree-decomposable.

	\item Given a linkage whose underlying graph has low Cayley complexity, 
	can we define realization types whose specification
	restricts the corresponding oriented Cayley configuration space to be a single interval?

	We answer this question by giving a natural \emph{minimal realization type}, i.e. a set of local orientations, whose specification guarantees Cayley size of 1 (i.e., the set of realizable distances for the chosen non-edge is a single interval) 
	and  Cayley computational complexity of $O(|V|^2)$.
	We also show that specifying fewer local orientations than those contained in the minimal realization type results in a
	superpolynomial blow-up of  both the Cayley size and computational complexity, provided $P$ is different
	from $NP$.

	\item From (1) and (2) we can
	immediately answer the following questions for linkages whose underlying graphs have low Cayley complexity:	
	given two realizations or two Cayley configurations of a linkage, can we determine if there exists a path of continuous motion between them? How do we obtain such a path if it exists? 
	How can we bijectively represent the realization space using minimum ambient dimension?
	
	

	In our paper, we  show that for generic 1-dof tree-decomposable linkages with low Cayley complexity, 
	the path between two realizations is at most two, and can be directly obtained from the oriented Cayley configuration spaces. 
	Provided a path exists, it can be found in time linear in the number of interval endpoints of oriented Cayley configuration spaces that the path contains. 
	Moreover, when the two realizations have the same minimal realization type as in (2) above, 
	it is guaranteed that there exists a path between them  staying within the same minimal realization type as these two realizations; 
	and the time complexity of finding that path is $O(1)$. 
    In addition, we give a canonical bijective representation of the realization space in minimal ambient dimension.
    This representation allows us to meaningfully visualize the realization space, as well as define 
    a canonical distance between different connected components of the realization space. 
	

\end{enumerate}




\medskip

\noindent \textbf{Contributions of Part II}:  
Part II of the paper answers the following questions from Section \ref{subsec:questions}. 

Consider the Cayley configuration space of a 1-dof linkage with a underlying graph $G$ over any non-edge $f$, 
such that $G \cup f$ is tree-decomposable. 
Does the Cayley complexity  depend on the choice of $f$? We answer this question in the negative. 
Specifically, 
we show that if the Cayley configuration space over some choice of $f$ has low Cayley complexity, then 
then the Cayley configuration space over any choice of $f$ also has low Cayley complexity.
This shows robustness of the Cayley complexity measure for 1-dof tree-decomposable graphs, 
and yields an algorithm that runs in time $O(|V|^2)$ to determine low Cayley complexity.

Finally, we answer the question:  can we combinatorially characterize 1-dof
tree-decomposable graphs $G$ with \emph{low Cayley algebraic complexity}, 	i.e. all interval endpoints in the corresponding (oriented) Cayley configuration spaces being QRS, 
without checking every such endpoint? We show  a surprising result that (graph) planarity is equivalent
to low Cayley algebraic complexity for a natural subclass of 1-dof tree-decomposable graphs. 
While this is a finite forbidden minor graph characterization of
low Cayley algebraic complexity, we provide counterexamples showing impossibility of
such finite forbidden minor characterizations when the above subclass is
enlarged.

\bigskip
\noindent 
Implementation of the algorithms developed in Part I, Part II and further functionalities is part of our new CayMos software,
whose architecture is described in \cite{bib:caymos} and 
web-accessible at http://www.cise.ufl.edu/\~{}menghan/caymos/. 
See also demonstration videos at this website.
A different manuscript \cite{bib:beest} describes Cayley and Cartesian configuration space analysis and motion analysis 
of common and well-known mechanisms using CayMos.


\subsubsection{Comparison with related work}

A key feature of our results is the essential and judicious \emph{interplay
between simple algebraic geometry properties} (Part I of this paper)
\emph{and simple graph theoretic properties} (Part II of this paper)
of these linkages.
To the best of our knowledge, the only known result in this area that has a similar flavor
of combinatorially capturing algebraic complexity is the result of \cite{bib:Owen02} that relates quadratic
solvability and tree-decomposability for planar graphs.

Concerning the use of Cayley parameters or non-edges for parametrizing the configuration
space, the papers \cite{bib:JoanArinyo03,bib:survey,bib:ZhangGao06} study how to obtain ``completions'' 
of underconstrained
graphs $G$, i.e, a set of non-edges $F$ whose addition makes $G$ minimally rigid. 
All are motivated by the need to efficiently obtain realizations of flexible linkages. 
In particular \cite{bib:JoanArinyo03} also guarantees that the completion ensures tree-decomposability.
However, they do not even attempt to address the question of how to find realizable
distance values for the completion edges. Nor do they address
algebraic complexity of the set of distance values that these completion non-edges
can take, nor the complexity of obtaining a description of this configuration space,
nor a combinatorial characterization of graphs
for which this complexity is low. 
The latter factors however are crucial for tractably analyzing and decomposing 
the realization space in order to obtain the corresponding realizations. 
On the other hand, 
the paper \cite{bib:hilderick06} gives a collection of useful observations and heuristics for computing
the interval endpoints in the configuration space descriptions of certain linkages that
arise in real CAD applications, by decomposing the linkages into subproblems.
However, it relies on a complete list of solutions for all possible subproblems.
Nor does it address the complexity issues mentioned above.

	\begin{figure}[hbtp]
	\centering
	\includegraphics[width=8cm]{img/diagram}
	\caption{Complete case analysis of continuous motion paths between two realizations $p_1$ and $p_2$. 
}
	\label{fig:diagram}
	\end{figure}

	We know that the Cayley configuration is not always a bijective representation of the realization space. 
A non-oriented Cayley interval, being a union of multiple oriented Cayley intervals, 
	could correspond to multiple connected components of the realization space, 
	as in Figure \ref{F:interval}. 
Although an oriented Cayley interval corresponds to a unique connected component, 
	the mapping is not bijective, since that same connected component could contain more than one oriented interval.
	Figure \ref{fig:diagram} summarizes the different cases when 
	determining existence of a continuous motion path between two realizations. 
	There are two cases (2 and 3) where there may or may not exist a continuous motion path (a and b).
	
	Previous algorithms and software, except for \cite{bib:hidalgo2011reachability},
	generate continuous motion within a specified Cayley interval, 
	or multiple segments of continuous motion, 
	each corresponding to different oriented Cayley interval with the same realization type. 
Thus they cannot consistently distinguish Case 2a from Case 2b, or Case 3a from Case 3b. The algorithms in \cite{bib:hidalgo2011reachability} can distinguish between these four types. 
	However, it deals with general tree-decomposable linkages, relies on exhaustive searching
	 and could have exponential time complexity.









\subsection{Organization of Part I}

In Section \ref{sec:Basic-properties}, we give basic definitions 
related to 1-dof tree-decomposable graphs. 

In Section \ref{sec:Combinatorial-interpretation-of}, we associate a special so-called \emph{extreme graph} with each interval endpoint of a (oriented) Cayley configuration space. We also give the precise definition of low Cayley complexity.  

In Section \ref{sec:Find-Cayley-configuration}, we prove that 
$O(1)$ Cayley size and $O(|V|^2)$ Cayley computational complexity will be guaranteed
when a natural \emph{minimal realization type} -- i.e. a minimal set of local orientations -- is fixed, 
for a graph with low Cayley complexity. 
We also show that specifying fewer local orientations than those contained in the minimal realization type results in a
superpolynomial blow-up of  both the Cayley size and computational complexity, provided $P$ is different
from $NP$.

In Section \ref{sec:cont-path}, 
we give an algorithm to find a path of continuous motion between two given realizations from the oriented Cayley configuration spaces, 
and show that there are at most two such paths.  
We also show that when those two realizations have the same minimal realization type, 
a path staying within the same minimal realization type can always be found in constant time. 
In Section \ref{sec:ambient}, we give a canonical bijective representation the the realization space, which yields a meaningful visualization of the realization space as curves in an ambient space of minimal dimension.


\section{Definitions and basic properties of 1-dof tree-decomposable graphs and linkages}
\label{sec:Basic-properties}

\begin{definition}
\label{def:t-decomp}
A graph $G$ is {\emph{tree-decomposable}} if: 

\begin{itemize}
	\item it is a single edge; or 

	\item it can be divided into three {\emph{tree-decomposable components}}, namely tree-decomposable subgraphs $G_{1}$,
	$G_{2}$ and $G_{3}$, such that $G=G_{1}\cup G_{2}\cup G_{3}$, $G_{1}\cap G_{2}=(\{v_{3}\},\emptyset)$,
	$G_{2}\cap G_{3}=(\{v_{2}\},\emptyset)$ and $G_{1}\cap G_{3}=(\{v_{1}\},\emptyset)$,
	with $v_{1}$, $v_{2}$ and $v_{3}$ being distinct vertices
	(see Figure~\ref{F:triangleDecomposition}(a)). 
\end{itemize}
A graph $G$ is a {\emph{1-dof tree-decomposable graph}}
if there exists a non-edge $f$ such that $G \cup f$ is tree-decomposable.
Such an $f$ is called a {\emph{base non-edge}}
of $G$ and a {\emph{base edge}} of $G\cup f$.
\end{definition}


 For example, in Figure \ref{F:triangleDecomposition}, a tree-decomposable graph is decomposed into three tree-decomposable components, and $G_{1}$ is decomposed into $G_{11}$, $G_{12}$ and $G_{13}$.

\smallskip

\noindent\textbf{Note.}
A 1-dof tree-decomposable graph $G$ can have many base non-edges. 
That is, $G$ may have non-edges $f'\ne f$ such that both $G \cup f$ and $G\cup f'$ are tree-decomposable graphs. 
We emphasize that this is different from deleting a different
edge $h$ from $G\cup f$, which  gives an entirely different 1-dof tree-decomposable graph from $G$.  

\begin{figure}[h]
	\psfrag{A}{$v_{1}$} \psfrag{B}{$v_{2}$} \psfrag{C}{$v_{3}$}\psfrag{D}{$v_{4}$}
	\psfrag{E}{$v_{5}$} 
	\psfrag{1}{$G_{1}$} \psfrag{2}{$G_{2}$} \psfrag{3}{$G_{3}$}
	\psfrag{11}{$G_{11}$} \psfrag{12}{$G_{12}$} \psfrag{13}{$G_{13}$} 
	\psfrag{(a)}{(a)} \psfrag{(b)}{(b)}
	
	\begin{centering}
	\includegraphics[width=0.6\textwidth]{img/triangleDecomposition} 
	\par\end{centering}
	
	\caption{A graph is tree-decomposable if it can be divided into three
	tree-decomposable components. }

\label{F:triangleDecomposition} 
\end{figure}



\begin{definition}\label{def:td-construction} 
Any 1-dof tree-decomposable graph $G$ can be constructed iteratively as follows, starting from a given base non-edge $f$. 
At the $k^{th}$ {\emph{construction step}}, two new maximal tree-decomposable subgraphs
$T_{1}$ and $T_{2}$ sharing a single {\emph{step vertex}} $v_{k}$
are appended to the previously constructed graph $G_f(k-1)$: $T_{1}$
and $T_{2}$ each has exactly one shared vertex, $u_k$ and $w_k$ respectively, with $G_f(k-1)$, 
where $u_k \ne w_k$
(see Figure \ref{F:construction}). 
Vertices $u_k$, $w_k$ are called the {\emph{base pair of vertices}}{\emph{at Step}} $k$. 
We denote this construction
step by $v_{k}\triangleleft(u_k\in T_{1},w_k\in T_{2})$, or simply $v_{k}\triangleleft(u_k,w_k)$.
We call these maximal tree-decomposable subgraphs $T_i$ the {\emph{clusters}} of $G$. 
If a vertex $v$ is shared by $m$ distinct clusters, we say {$cdeg(v)=m$}. 

A tree-decomposable graph can be constructed in a similar way from a given base edge.
\end{definition}

For example, refer to Figure \ref{F:construction}(a). Construction steps
from base non-edge $(v_{0},v_{0}')$ are: $v_{1}\triangleleft(v_{0}\in T_{1},v_{0}'\in T_{2})$, $v_{2}\triangleleft(v_{0}\in T_{3},v_{0}'\in T_{4})$,
and $v_{3}\triangleleft(u_1\in T_{5},v_{0}'\in T_{6})$. 
We have $cdeg(v_0') = 3$, $cdeg(v_3) = 2$. 

\begin{figure}[h]
\psfrag{1}{$v_{1}$} \psfrag{2}{$v_{2}$} \psfrag{3}{$v_{3}$} \psfrag{4}{$u_1$} \psfrag{5}{$u_2$} \psfrag{6}{$u_3$} \psfrag{0}{$v_{0}$} \psfrag{0'}{$v_{0}'$} \psfrag{t1}{$T_{1}$} \psfrag{t2}{$T_{2}$} \psfrag{t3}{$T_{3}$} \psfrag{t4}{$T_{4}$} \psfrag{t5}{$T_{5}$} \psfrag{t6}{$T_{6}$}
	
	\begin{centering}
	\includegraphics[width=0.6\textwidth]{img/construction} 
	\par\end{centering}
	
	\caption{ (a) A 1-dof tree-decomposable graph with three construction steps from base non-edge $(v_{0},v_{0}')$.
	(b) The extreme graph of (a) corresponding to the $3^{rd}$ construction step. 
	}

\label{F:construction} 
\end{figure}


\noindent\textbf{Note.} (i) The decomposition of a 1-dof tree-decomposable graph $G$ into clusters is unique \cite{bib:FudHo97proof}. 

\noindent (ii) It follows from the definition that for any construction step $v_k \triangleleft (u_k,w_k)$, 
the base pair of vertices $u_k$ and $w_k$ must lie in two different clusters $T_{u}$ and $T_{w}$ in $G_{f}(k-1)$. 
We call $T_{u}$ and $T_{w}$ the \emph{base pair of clusters} at Step $k$.

\smallskip





A linkage $(G,\bar{l})$ with a 1-dof tree-decomposable underlying graph $G$ generically has one degree of freedom,
and a  Cayley configuration space with parameter $f$.
Given $l_f$, the graph construction of $G \cup f$ from $f$ clearly yields a QRS realization sequence of $(G \cup f, \bar{l} \cup l_f)$ from $f$, where $\bar{l} \cup l_f: E \cup {f} \rightarrow \mathbb{R}$. 
Specifically, for each construction step $v \triangleleft (u,w)$,
given  $p_u$, $p_w$ and the lengths $\bar{l}(v,u)$, $\bar{l}(v,w)$,
$p_v$ can be determined by a corresponding simple ruler and compass algebraic solution. 
The realization may not be unique, since for each realization step we may have two possible \emph{local orientations}.

\begin{definition} \label{def:solution-type}
A construction step  $v \triangleleft (u,w)$ of a 1-dof tree-decomposable graph $G$
can be associated with a {\emph{local orientation}} for the corresponding realization step of a linkage $(G,\bar{l})$.
This local orientation $\sigma_{k}$ takes a value in $\{+1,-1,0\}$, 
which represents the sign of the determinant $\varDelta = \left|\begin{array}{c}p_w-p_u\\p_v-p_u\end{array}\right|$. 
A {\emph{forward realization type}} is a $T$-realization type, 
where $T$ is the set of triples of points corresponding to vertices $(v,u,w)$, 
where $v \triangleleft (u,w)$ represent the construction steps of $G$ from $f$.
A {\emph{forward-oriented Cayley configuration space}} is a $T$-oriented Cayley configuration space with respect to forward realization type. 
\end{definition}

\noindent\textbf{Note.}
In the following discussion, 
when referring to oriented Cayley configuration spaces of 1-dof tree-decomposable graphs,
unless otherwise specified, 
we always mean the forward-oriented Cayley configuration spaces. 
\smallskip

For example, refer to Figure \ref{F:direction} (a) and (b). The graph
is 1-dof tree-decomposable with base non-edge $f=(v_{0},v_{0}')$. The realization
step $v_{3}\triangleleft(v_{1},v_{2})$ can choose from two possible local orientations:
(a) has $\sigma_3 < 0$, while (b) has $ \sigma_3 > 0$ by flipping $v_{3}$ to the other side of $(v_{1},v_{2})$. 

\begin{figure}[h]
	\psfrag{1}{$v_{1}$} \psfrag{2}{$v_{2}$} \psfrag{3}{$v_{3}$}
	\psfrag{0}{$v_{0}$} \psfrag{0'}{$v_{0}'$} 
	
	\begin{centering}
	\includegraphics[width=0.8\textwidth]{img/direction}
	\par\end{centering}
	
	\caption{ (a) (b): Two choices exist for local orientation of
	realization step $v_{3}\triangleleft(v_{1},v_{2})$ from base non-edge $f=(v_{0},v_{0}')$. 
	(c) (d): realizations for extreme linkage $(\hat{G}_{f}(3),\bar{l}^{\max})$. 
	(e) (f): realizations for extreme linkage $(\hat{G}_{f}(3),\bar{l}^{\min})$.}
	
	
	\label{F:direction} 
\end{figure}


\begin{remark}
Since the total number of forward realization types is exponential in the number of construction steps,
only by knowing the forward realization type 
can we guarantee linear time complexity for realizing a tree-decomposable linkage, i.e., 
for obtaining a realization from a Cayley configuration of a 1-dof tree-decomposable linkage.
\end{remark}







\medskip
\noindent 
A video demonstrating the basic definitions of 
1-dof tree-decomposable linkages and Cayley configuration spaces 
can be found at http://www.cise.ufl.edu/\~{}menghan/caymos/definitions.avi.



\section{Extreme graphs and interval endpoints of Cayley configuration space}
\label{sec:Combinatorial-interpretation-of}

We use the notion of \emph{extreme graphs} and \emph{extreme linkages}
to describe the endpoints of the Cayley configuration space of a 
1-dof tree-decomposable linkage $(G, \bar{l})$ over $f$.


\begin{definition} \label{def:extreme-graph}
The $k^{th}$ {\emph{extreme graph}}
for $f$ of a 1-dof tree-decomposable graph $G$,
where the $k^{th}$ construction step of $G$ from $f$ is $v_{k}\triangleleft(u_k,w_k)$, 
is the  graph obtained by adding a new edge $e_{k}=(u_k,w_k)$ in $G_{f}(k-1)$. 
We denote this extreme graph by $\hat{G}_{f}(k)$.
We call $(u_k,w_k)$ the {\emph{extreme edge}} of the extreme graph $\hat{G}_{f}(k)$, 
and an {\emph{extreme non-edge}} of $G$.
\end{definition}

\noindent\textbf{Note.}
It is easy to verify using Laman's theorem \cite{bib:Laman70} that any extreme graph
of a 1-dof tree-decomposable graph is minimally rigid.
\smallskip


For example, refer to Figure \ref{F:construction}. 
The $3^{rd}$ construction step in (a) is $v_{3}\triangleleft(u_1,v_0')$.
Connecting $e_{3}=(u_1,v_0')$ in $G_{f}(2)$, we get the extreme
graph $\hat{G}_{f}(3)=G_{f}(2)\cup e_{3}$ in (b). 




\begin{definition}\label{def:extreme-linkage}For a 1-dof tree-decomposable linkage $(G,\bar{l})$,
the $k^{th}$ {\emph{extreme linkages}}\emph{ }are $(\hat{G}_{f}(k),\bar{l}^{\min})$
and $(\hat{G}_{f}(k),\bar{l}^{\max})$, where $\min$ and $\max$ represent
the two possible extreme extensions of $\bar{l}$ for the extreme edge
$e_{k}=(u,w)$, obtained from triangle inequalities: 
$\bar{l}^{\min}(e_{k}):=|\bar{l}(u,v_{k})-\bar{l}(v_{k},w)|$,
$\bar{l}^{\max}(e_{k}):=\bar{l}(u,v_{k})+\bar{l}(v_{k},w)$. 
\end{definition}



\noindent\textbf{Note.} In realizations of $(\hat{G}_{f}(k),\bar{l}^{\min})$
and $(\hat{G}_{f}(k),\bar{l}^{\max})$, the local orientation $\sigma_k = 0$.
These realizations are sometimes called \emph{unyielding} realizations. 




\begin{figure}[h]
	\psfrag{1}{$v_{0}$} \psfrag{2}{$v_{0}'$} \psfrag{3}{$v_{1}$}
	\psfrag{4}{$v_{2}$} \psfrag{5}{$v_{3}$} \psfrag{6}{$v_{4}$}
	\psfrag{7}{$v_{5}$} \psfrag{8}{$v_{6}$} \psfrag{9}{$v_{7}$}
	\psfrag{10}{$v_{8}$} 
	
	\begin{centering}
	\includegraphics[width=0.8\textwidth]{img/equal_distance}
	\par\end{centering}
	
	\caption{Example showing the importance of genericity: when $p(v_{5})$ and $p(v_{6})$ are coincident, $l(v_{5},v_{8})$
	is not a function of $l(v_{0},v_{0}')$.}
	
	\label{F:equal-distance} 
\end{figure}





The next theorem associates a linkage realization with each endpoint of an (oriented) Cayley configuration space. Note that if $(G, \bar{l})$ is generic, then for a given Cayley configuration $l_f$ and forward realization type $\sigma$, there exists at most one 2D realization:  
as demonstrated in Figure~\ref{F:equal-distance}, 
for any vertex $v$ of $G$, the point $p_v$ is not unique only if for $v$'s realization step $v \triangleleft (u,w)$,
$p_u$ and $p_w$ are coincident and $\bar{l}(v,u) = \bar{l}(v,w)$.

\begin{theorem}[structure of Cayley configuration space]\label{lem:algebraic}
 For a generic 1-dof tree-decomposable linkage $(G,\bar{l})$ with base non-edge $f=(v_{0},v_{0}')$, 
the following hold:

 \begin{enumerate}
 	\item The (oriented) Cayley configuration space over $f$ is a set of disjoint closed real intervals or empty. 
 	
 	\item Any interval endpoint in the (oriented) Cayley configuration space corresponds to the length of $f$ in a realization of an extreme linkage. 
 	\item For any vertex $v$, $p_v$ is a continuous
 	 	function of $l_f$ on each closed interval of the oriented Cayley configuration space.
Consequently, for any  non-edge $(u,w)$, $l(u,w)$ is a continuous function of $l_f$ on each closed interval of the oriented Cayley configuration space.\end{enumerate}
 
\end{theorem}

\begin{remark} \label{rmk:endpoint}
While Theorem \ref{lem:algebraic} (2) states that every endpoint of the unoriented Cayley configuration space
corresponds to an extreme linkage, 
the converse is not true. 
As an example, refer to Figure \ref{F:interval}. 
Realization (D) is an extreme linkage, 
but it is not an endpoint in the unoriented Cayley configuration space.
However we will see in Lemma \ref{lem:endpoint} that the converse is true for oriented Cayley configuration spaces. 
\end{remark}

The proof of Theorem \ref{lem:algebraic} follows from elementary algebraic geometry and can be considered folklore. 
However, for completeness, a proof is provided in Appendix \ref{sec:Proof-for-lemma}. 

\begin{observation} \label{obs:NP-hard}
\begin{enumerate}[(i)]
\item Theorem \ref{lem:algebraic} gives a straightforward algorithm  called {\emph{ELR (extreme linkage realization)}} 
to obtain the (oriented) Cayley configuration space for a generic 1-dof tree-decomposable linkage $(G, \bar{l})$. 
The algorithm could take time exponential in the Cayley size.

\item Deciding whether the Cayley configuration space over $f$ is non-empty is a NP-hard problem, and
the Cayley computational complexity is superpolynomial unless $P = NP$. 
\end{enumerate}

\end{observation}

\begin{proof}
(i) The ELR algorithm works by realizing all the extreme linkages for $f$ consistent with each forward realization type. 
Note that since the extreme graphs may not be QRS, realizing each extreme linkage can take time exponential in $|V|$ 
(requiring the solution of a general multi-variable system of quadratic equations).
Additionally, the overall time complexity could be exponential in the actual Cayley size, 
since many candidate endpoints generated during this procedure could finally lead to dead ends. 
The detailed version of this algorithm is in Appendix \ref{sec:not-low}. 



(ii) The problem of determining the existence of a realization of a tree-decomposable linkage 
is NP-complete by early results \cite{bib:saxe79}. This problem can be reduced to our problem of finding the Cayley configuration space over $f$, 
whose decision version is whether the Cayley configuration space is non-empty. 
Clearly, a realization exists if and only if the Cayley configuration space is not empty. 
Therefore, the problem of finding the Cayley configuration space over $f$ is NP-hard, and the Cayley computational complexity is superpolynomial unless $P = NP$.
\end{proof}



By Theorem \ref{lem:algebraic}, we can obtain a specific measure of Cayley complexity in terms of extreme graphs. 
Given the promise that there exists a realization of $(G,\bar{l})$ corresponding to 
a specific extreme linkage  consistent with a given forward realization type, 
we now consider the following question: 
what is the algebraic complexity of $l_f$, a potential endpoint of the Cayley configuration space?
I.e., what is the Cayley complexity of $G$ on $f$? 
The answer depends on whether the extreme graph is QRS or not. 
If it is not QRS, not only would this adversely affect the Cayley complexity, 
the Cayley computational complexity could also be exponential in $|V|$  
(solving general quadratic systems). 

\begin{definition}\label{def:low}A 1-dof tree-decomposable $G$ is said to have {\emph{low Cayley algebraic complexity}} on base non-edge $f$ if all  extreme graphs of $G$ for $f$ are QRS. 
Since tree-decomposable graphs are an example of QRS graphs, 
a 1-dof tree-decomposable $G$ is said to have {\emph{low Cayley complexity}} on base non-edge $f$ if all  extreme graphs of $G$ for $f$ are tree-decomposable. 
\end{definition}

Note that for planar graphs, QRS and tree-decomposability have been shown \cite{jackson2012radically} 
to be equivalent, and the equivalence has been strongly conjectured for all graphs. 

\begin{figure}[h]
	\psfrag{1}{$v_{0}$} \psfrag{2}{$v_{0}'$} \psfrag{3}{$v_{1}$}
	\psfrag{4}{$v_{2}$} \psfrag{5}{$v_{3}$} \psfrag{6}{$v_{4}$}
	\psfrag{7}{$v_{5}$}
	
	\begin{centering}
	\includegraphics[width=0.6\textwidth]{img/initial}
	\par\end{centering}
	
	\caption{The graph in (b) has low Cayley complexity on $(v_0,v_0')$, while the graph in (a) does not. }
	
	\label{F:not_low} 
\end{figure}

For example, refer to Figure \ref{F:not_low}. The graph in (b) has low Cayley complexity on $(v_0,v_0')$, 
while the graph in (a) does not,  since the extreme graph corresponding to the construction step of $v_5$
is not tree-decomposable. 
We can also verify that the 1-dof tree-decomposable graphs in Figure \ref{F:intro}, Figure \ref{F:construction} and Figure \ref{F:direction}  have low Cayley complexity on the given base non-edge, 
while the graph in Figure \ref{F:equal-distance} does not.





\section{Finding Cayley configuration spaces for linkages with low Cayley complexity} 
\label{sec:Find-Cayley-configuration}


Suppose we are given a 1-dof tree-decomposable graph $G$ with low Cayley complexity. 
For any corresponding linkage $(G, \bar{l})$, 
each interval endpoint in its Cayley configuration spaces can be computed essentially using
a sequence of solutions of one quadratic equation at a time. 
We may ask: are we also guaranteed to have small Cayley size and low Cayley
computational complexity even for the oriented Cayley configuration spaces? 
The answer is yes only for a fixed \emph{minimal realization type} 
(defined below, which is more restrictive than the forward realization type). 

\begin{definition}
For a graph $G$ with low Cayley complexity on $f$, each extreme graph $\hat{G}_{f}(k)$
can be constructed with the extreme edge $e_{k}$ as base edge. We call this a {\emph{reverse construction}}. 
Each realization of $(G,\bar{l})$ corresponds to a {\emph{reverse realization type}}, a sequence of local orientations for each reverse realization step.
A {\emph{minimal realization type}} consists of both a forward realization type and a reverse realization type.
\end{definition}





For example,  for the linkage in Figure \ref{F:direction},  
realizations in (a) and (b) have different forward realization types (thus different minimal realization types). 
Moreover,  since the underlying graph has low Cayley complexity on base non-edge $f=(v_{0},v_{0}')$, 
the extreme graph $\hat{G}_{f}(3)$ has reverse construction $v_{0}\triangleleft(v_{1},v_{2})$, $v_{0}'\triangleleft(v_{1},v_{2})$,  
where (c)(e) and (d)(f) correspond to different reverse  realization types (thus different minimal realization types):
(c)(e) have $v_{0}$ and $v_{0}'$ on the same side of $(v_{1},v_{2}),$
while (d)(f) have them on opposite sides of $(v_{1},v_{2})$. 

\medskip

Observation \ref{obs:NP-hard} can be extended to show that in order
to guarantee small Cayley size and computational complexity, we need to specify the minimal realization type.

\begin{observation} \label{obs: no-orientation}
For a 1-dof tree-decomposable graph $G$ with low Cayley complexity on a non-edge $f$, 
when the reverse realization type is specified but the forward realization type is unspecified, 
resp. when the forward realization type is specified but the reverse realization type is unspecified, 
the problem of obtaining the complete description of Cayley configuration space
(decomposition of the Cayley configuration space into a union of oriented Cayley configuration spaces)
of a linkage $(G, \bar{l})$ over $f$ can take time exponential  in $|V|$.
\end{observation}

\begin{proof}
When the reverse realization type is unspecified, the Cayley size can be exponential in $|V|$.
Symmetrically, when the forward realization type is unspecified, 
the number of non-empty oriented Cayley configuration spaces can be exponential in $|V|$. 
In Appendix \ref{sec:Exponential}, we use our initial example, the linkage in Figure \ref{F:intro} 
to demonstrate this exponential blow-up in Cayley size and computational complexity.
Existing examples \cite{bib:Borcea} can also be adapted to show the exponential blow-up. 
Therefore, in both cases, 
obtaining the decomposition of the Cayley configuration space into a union of oriented Cayley configuration spaces
takes time exponential  in $|V|$.
\end{proof}


On the other hand, when the minimal realization type is fixed, for graphs with low Cayley complexity, 
we show below that the Cayley size is $O(1)$ and the computational complexity is $O(|V|^2)$.

\subsection{Finding Cayley configuration spaces when the minimal realization type is fixed}

\begin{theorem}[fixed minimal realization type]\label{obs:k-path}
For a 1-dof tree-decomposable graph with low Cayley complexity on a non-edge $f$, 
if the minimal realization type is fixed, then
the Cayley size is $O(1)$ and the Cayley computational complexity is $O(|V|^2)$. 
\end{theorem}


\noindent \textbf{Idea of the Proof}. 
We first prove the theorem for a special subclass of graphs called \emph{1-path} graphs. 
Informally, these tree-decomposable graphs have a linearly ordered construction sequence in a well-defined sense (Definition \ref{def:1-path}). 
In the case of general tree-decomposable graphs, this linear order generalizes to a partial order. 
The proof for 1-path graphs will serve as induction basis for the general case. 
For the proof of the 1-path case, we utilize Lemma \ref{II-the:chain} from Part II (Recursive Structure Lemma) concerning the structure of 1-path graphs with low Cayley complexity. 
Based on this lemma, we obtain a \emph{quadrilateral interval mapping (QIM)} algorithm that correctly finds the Cayley configuration space (Lemma \ref{lem:QIM}, \ref{lem:QIM-1-path}), yielding the proof for the 1-path case (Proposition \ref{obs:1-path}).

Next, we prove the multi-path case, by doing induction on the number of \emph{paths} of the graph. 
This gives a generalization of the QIM algorithm, which however works only when the minimal realization type is fixed (Lemma \ref{lem: QIM-k-path}). 
From this algorithm, we get the proof of the main theorem. 

The structure of the proof is schematically shown in Figure \ref{F:structure}. 



\begin{figure}[h] 
\psfrag{quad}{\small Lemma \ref{lem:QIM} (Four-cycle Lemma)} 
	\psfrag{(4-cycle)}{\footnotesize Theorem \ref{II-the:four-cycle} (2), Part II} 
	\psfrag{qim}{\small Lemma \ref{lem:QIM-1-path}} \psfrag{1path}{\small Proposition \ref{obs:1-path}} 
\psfrag{qim2}{\small Lemma \ref{lem: QIM-k-path}}
	\psfrag{main}{\small Theorem \ref{obs:k-path}}
\psfrag{q}{\small QIM(1-path)} \psfrag{q2}{\small QIM(multi-path)} \psfrag{1p}{1-path case}
	
	\begin{centering}
	\includegraphics[width= 12cm]{img/structure} 
	\par\end{centering}
	
	\caption{Structure of proof of Theorem \ref{obs:k-path}}
	
\label{F:structure}
\end{figure}


\noindent We first give the definition of \emph{1-path} graphs. 

\begin{definition} \label{def:last-level} 
Given a 1-dof tree-decomposable graph $G$, 
a vertex $v$ is in $G$'s {\emph{last level} $L_{t}$} if: 
(a) $cdeg(v)=2$, i.e. there are exactly two clusters $T_{1}$ and $T_{2}$ sharing $v$; 
(b) each of $T_{1}$ and $T_{2}$ has only one shared vertex with the rest of the graph 
$G'=G\setminus(T_{1}\cup T_{2}) = G\setminus\{v\}$. 
\end{definition}



\begin{definition}\label{def:1-path}
A 1-dof tree-decomposable graph $G$ has a {\emph{1-path construction}} from base non-edge $f=(v_{0},v_{0}')$ 
if there is only one vertex $v$ in the last level $L_{t}$, other than possibly $v_{0}$ and $v_{0}'$. 
As long as there exists a base non-edge permitting 1-path construction, we say the 1-dof tree-decomposable
graph $G$ is a {\emph{1-path}} graph. See Figure \ref{F:obstruction} for an example. 
\end{definition}

\begin{figure}[h]
	\psfrag{0}{$v_0$} \psfrag{0'}{$v_0'$}
	\psfrag{1}{$v_{1}$} \psfrag{2}{$v_{2}$} \psfrag{3}{$v_{3}$}
	\psfrag{4}{$v_{4}$} \psfrag{5}{$v_{5}$} \psfrag{6}{$v_{6}$} 
	\psfrag{a}{$a$} \psfrag{b}{$b$} 
	
	\begin{centering}
	\includegraphics[width=0.6\textwidth]{img/obstruction} 
	\par\end{centering}
	
	\caption{(a) A 1-path graph with $v_5$ in the last level $L_t$. (b) A multi-path graph.}
\label{F:obstruction} 
\end{figure}



The following lemma from Part II gives a key structural property of 1-path, 1-dof, tree-decomposable graphs with low Cayley complexity.


\begin{definition}[Definition \ref{II-def:four-cycle} from Part II]
A {\emph{four-cycle}} of clusters consists of four clusters $T_{1},T_{2},T_{3},T_{4}$ such that each consecutive pair has exactly one shared vertex. 
In other words, we have $T_{1}\cap T_{2}=\{v_{1}\}$, $T_{2}\cap T_{3}=\{v_{2}\}$, $T_{3}\cap T_{4}=\{v_{3}\}$, $T_{4}\cap T_{1}=\{v_{4}\}$, where $v_{1}$, $v_{2}$, $v_{3}$, $v_{4}$ are distinct vertices.
Vertices pairs $(v_1,v_3)$ and $(v_2,v_4)$ are called the \emph{diagonal pairs} of the four-cycle. 
For any other vertices $u_i$, $u_{i+1}$ belonging to adjacent clusters $T_i$, $T_{i+1}$, 
$(u_i, u_{i+1})$ is called a {\emph{chordal pairs}} of the four-cycle. 
\end{definition}


\begin{lemma}[Four-cycle Lemma, Theorem \ref{II-the:four-cycle} (2) from Part II] \label{lem:QIM}
Let $G$ be a 1-path tree-decomposable graph with low Cayley complexity on $f$. 
For any two distinct base pairs of vertices that we encounter consecutively in the construction of $G$ from $f$, 
there exists a four-cycle of clusters such that: 

\begin{enumerate}[(a)]
\item The two consecutive base pairs are the two diagonal pairs of the four-cycle, or:

\item One  base pair is a diagonal pair of the four-cycle, while the other base pair is a chordal pair of the four-cycle. 
\end{enumerate}

\end{lemma}







\begin{figure}[h]
	\psfrag{v0u0}{$v_0(u_0)$} \psfrag{v0'u0'}{$v_0'(u_0')$} 
	\psfrag{v1u1}{$v_1(u_1)$} \psfrag{v2u1'}{$v_2(u_1')$} 
	\psfrag{v3}{$v_{3}$}
	\psfrag{v4}{$v_{4}$} \psfrag{v1}{$v_{1}$} \psfrag{v2}{$v_{2}$}
	\psfrag{u1}{$u_1$} \psfrag{u1'}{$u_1'$} 
	\psfrag{u2}{$u_2$} \psfrag{u2'}{$u_2'$}
	\psfrag{T1}{$T_{1}$} \psfrag{T2}{$T_{2}$} \psfrag{T3}{$T_{3}$} \psfrag{T4}{$T_{4}$}  
	\psfrag{T5}{$T_{5}$} \psfrag{T6}{$T_{6}$} \psfrag{T7}{$T_{7}$} \psfrag{T8}{$T_{8}$}
	\psfrag{(a)}{(a)} \psfrag{(b)}{(b)}
	
	\begin{centering}
	\includegraphics[width=0.7\textwidth]{img/quadrilaterals} 
	\par\end{centering}
	
	\caption{(a) the two pairs $(u_1,u_1')$ and $(u_2,u_2')$ shows Case (a) of Lemma \ref{lem:QIM}.
	(b) the two pairs $(u_1,u_1')$ and $(u_2,u_2')$ shows Case (b) of Lemma \ref{lem:QIM}.}
\label{F:quadrilaterals} 
\end{figure}

\subsubsection{The Quadrilateral Interval Mapping (QIM) algorithm} \label{sub:QIM}

From Lemma \ref{lem:QIM}, we obtain a \emph{quadrilateral interval mapping }(QIM) algorithm for finding the Cayley configuration space over $f$.
We will see the advantage of the QIM algorithm over the ELR algorithm (described in Section \ref{sec:Combinatorial-interpretation-of}):  
only by using the QIM algorithm can we attain linear Cayley computational complexity for 1-path graphs of low Cayley complexity, 
when the minimal realization type is fixed.

\begin{figure}[h]
	\psfrag{min}{\tiny {$l^{l}(e_1)$}} 
	\psfrag{max}{\tiny {$l^{r}(e_1)$}}
	\psfrag{pe1}{\tiny {$p_{\min}(e_1)$}}	\psfrag{pe2}{\tiny {$p_{\max}(e_1)$}}
	\psfrag{pf1}{\tiny {$p_{\min}(e_2)$}}	\psfrag{pf2}{\tiny {$p_{\max}(e_2)$}}
	\psfrag{1}{{$l_{l}^{r}(e_2)$}} \psfrag{2}{{$l_{r}^{r}(e_2)$}}
	\psfrag{3}{{$l_{l}^{l}(e_2)$}} \psfrag{4}{{$l_{r}^{l}(e_2)$}}
	\psfrag{5}{{$\bar{l}^{\min}(e_2)$}} \psfrag{6}{{$\bar{l}^{\max}(e_2)$}}
	\psfrag{e}{{$l(e_1)$}} \psfrag{f}{{$l(e_2)$}}{\tiny \par}
	\psfrag{(a)}{(a)} \psfrag{(b)}{(b)} \psfrag{(c)}{(c)} \psfrag{(d)}{(d)}  
	
	\begin{centering}
	{\tiny \includegraphics[width=\textwidth]{img/elipse} }
	\par\end{centering}{\tiny \par}
	
	\caption{ (a) an example of ellipse $\mathcal{C}$ used in QIM; (b)(c)(d) various cases when mapping with ellipse $\mathcal{C}$}
	

\label{F:ellipse} 
\end{figure}

\noindent \textbf{Idea of the QIM algorithm}. 
Consider a quadrilateral with four sides $s_{1}$, $s_2$, $s_3$, $s_{4}$ and two diagonals $e_1$, $e_2$. 
Since the linkage lives in $\mathbb{R}^2$, 
the volume of the tetrahedron formed by $\{s_{1}$, $s_2$, $s_3$, $s_{4}$, $e_1$, $e_2\}$ must equal zero. 
If we know lengths $l(s_{1})$, $l(s_2)$, $l(s_3)$, $l(s_4)$, 
we can get from the volume equation (a so-called Cayley-Menger determinant) 
an implicit ellipse $\mathcal{C}$ relating attainable $l(e_1)$ and $l(e_2)$ values. 
See Figure \ref{F:ellipse} (a). 
So from the attainable interval $[l^{l}(e_1),l^{r}(e_1)]$ of one diagonal $e_1$, 
we can obtain the attainable intervals of $l(e_2)$ by mapping $[l^{l}(e_1),l^{r}(e_1)]$ on the curve $\mathcal{C}$, and vice versa. 

The curve $\mathcal{C}$ used in the algorithm has the following useful properties: 
(1) one specific value of $l(e_1)$ 
can map to up to 2 distinct corresponding values of $l(e_2)$.
Figure \ref{F:ellipse} (b)(c)(d) illustrates several cases in determining
the interval for $l(e_2)$ from $[l^{l}(e_1),l^{r}(e_1)]$.
This mapping can take a set of intervals for $l(e_1)$ into double the number of intervals for $l(e_2)$. 
(2) the overall maximum and minimum points of the curve, $p_{\min}(e_1)$, $p_{\min}(e_2)$, $p_{\max}(e_1)$ and $p_{\max}(e_2)$, each corresponds to a change in the minimal realization type. 
For example, the upper
left segment of $\mathcal{C}$, from $p_{\min}(e_2)$ to $p_{\max}(e_1)$,
corresponds to the realization type that the vertices of $e_1$ lie on different sides of the line specified by $e_2$, 
and the vertices of $e_2$ lie on the same side of the line specified by $e_1$. 

\noindent\textbf{Algorithm (QIM)}: 
given a linkage $(G, \bar{l})$, where $G$ is a 1-path graph with low Cayley complexity on base non-edge $f$, 
we start from the interval $[\bar{l}^{\min}(e_k),\bar{l}^{\max}(e_k)]$ obtained by triangle inequality from extreme linkages of the last extreme graph $\hat{G}_f(k)$, and map back to obtain a set $S_0$ of intervals for $l_f$. 

\smallskip

\indent $S_{k}$ $\leftarrow$ $\{[\bar{l}^{\min}(e_k),\bar{l}^{\max}(e_k)]\}$ \\
\indent \textbf{for} $i = k-1$ \textbf{to} $0$ \hspace{10pt} \\
\indent \indent [\textit{Map the interval of $l(e_{i+1})$ back to $l(e_i)$}]\\
\indent \indent \textbf{if} $e_{i+1}$ and $e_i$ satisfy the requirement in Lemma \ref{lem:QIM} (a) \\
\indent \indent \indent Obtain $S_{i}$ from $S_{i+1}$ using mapping with the ellipse $\mathcal{C}$  \\
\indent \indent \indent [\textit{$e_{i+1}$ and $e_i$ are two diagonals of a quadrilateral}] \\
\indent \indent \textbf{else} \hspace{10pt} [\textit{$e_{i+1}$ and $e_i$ satisfy the requirement in} Lemma \ref{lem:QIM} (b)] \\
\indent \indent \indent \textbf{if} $e_{i+1}$ and $e_i$ connect the same pair of adjacent clusters in the four-cycle \\
\indent \indent \indent \indent Obtain $S_{i}$ from $S_{i+1}$ by the law of cosines \\
\indent \indent \indent \indent [\textit{see Remark \ref{rmk:qim}(a) in Appendix \ref{app:qim}}] \\
\indent \indent \indent \textbf{else}\\ \indent\indent\indent\indent Obtain $S_{i}$ from $S_{i+1}$ by the law of cosines and mapping with $\mathcal{C}$ \\
\indent \indent \indent \indent [\textit{see Remark \ref{rmk:qim}(b) in Appendix \ref{app:qim}}] \\
\indent \textbf{return} $S_0$










\begin{lemma}[QIM for 1-path] \label{lem:QIM-1-path}
For a 1-path tree-decomposable linkage $(G,\bar{l})$ where $G$ has low Cayley complexity on $f$, the set $S_0$ of intervals for $l_f$ generated by the QIM algorithm is exactly the Cayley configuration space over $f$. 
\end{lemma}

The proof of Lemma \ref{lem:QIM-1-path} is given in Appendix \ref{app:qim}. 

\smallskip

\noindent We give two examples to demonstrate how the QIM algorithm works. 

\noindent \textbf{Example 1: } 

To obtain the Cayley configuration space on $f=(v_{0},v_0')$ for $G$ in 
Figure~\ref{F:obstruction}(a): 


\begin{itemize}
\item \textbf{Step 1:} 
Obtain the interval of $l(v_{4},v_2)$ in $\triangle v_2v_{4}v_{5}$ by triangle inequality; 

\item \textbf{Step 2:} In quadrilateral $v_0'v_2v_3v_4$, obtain
the interval of $l(v_0',v_{3})$ from the interval of $l(v_{4},v_2)$;

\item \textbf{Step 3:} Similarly, in quadrilateral $v_0'v_{1}v_{3}v_2$, 
we have $l(v_0',v_{3}) \rightarrow l(v_1,v_2)$;

\item \textbf{Step 4:} In quadrilateral $v_0v_2v_0'v_1$, 
we have $l(v_1,v_2) \rightarrow l(v_0,v_0')$. 
\end{itemize}

\smallskip

\noindent \textbf{Example 2: } 

To obtain the Cayley configuration space on $f=(v_{0},v_0')$ for $G$ in 
Figure~\ref{F:1PathOnly}: 

\begin{itemize}
\item \textbf{Step 1:} 
Obtain the interval of $l(u_2,v_3)$ in $\triangle v_4v_{3}u_2$ by triangle inequality; 

\item \textbf{Step 2:} In four-cycle $T_2'T_3T_3'T_1'$,
we have $l(u_2,v_3) \rightarrow l(u_1,u_1')$; 

\item \textbf{Step 3:}  In four-cycle $T_2'T_1'T_1T_2$,
we have $l(u_1,u_1') \rightarrow l(v_0,v_0')$.  	
\end{itemize}



\begin{figure}[h]
	\psfrag{0}{$v_0$} \psfrag{0'}{$v_0'$}
	\psfrag{1}{$v_{1}$} \psfrag{2}{$v_{2}$} \psfrag{3}{$v_{3}$} \psfrag{4}{$v_{4}$} \psfrag{5}{$v_{5}$} \psfrag{6}{$v_{6}$} 
	\psfrag{u1}{$u_1$} \psfrag{u1'}{$u_1'$} \psfrag{u2}{$u_2$} 
	\psfrag{T1}{$T_1$} \psfrag{T1'}{$T_1'$} \psfrag{T2}{$T_2$} \psfrag{T2'}{$T_2'$}
	\psfrag{T3}{$T_3$} \psfrag{T3'}{$T_3'$} \psfrag{T4}{$T_4$} \psfrag{T4'}{$T_4'$}
	
	\begin{centering}
	\includegraphics[width=0.4\textwidth]{img/1PathOnly} 
	\par\end{centering}
	
	\caption{ Example 2:  QIM on a 1-path graph. See also a demonstration video at http://www.cise.ufl.edu/\~{}menghan/caymos/qim.avi. }
\label{F:1PathOnly} 
\end{figure}

The worst-case time complexity of the QIM algorithm is exponential in the number of construction steps, since each mapping could possibly double the number of intervals. 
Nevertheless, when the minimal realization type is fixed, using the QIM algorithm, we can obtain the Cayley configuration space of a 1-path graph in linear time. 

\begin{proposition}[fixed minimal realization type for 1-path]\label{obs:1-path}
For a 1-path, 1-dof, tree-decomposable graph $G$
with low Cayley complexity on $f$, if the minimal realization type is fixed, 
for any linkage $(G, \bar{l})$, 
the Cayley configuration space over $f$ contains at most one interval, and can be found in $O(|V|)$ time. 
\end{proposition}

\begin{proof}
When the minimal realization type is fixed, 
when applying QIM to any linkage with underlying graph $G$, 
we are restricted to a monotonic segment of the ellipse $\mathcal{C}$. We start from a single interval for the last extreme edge of $G$ and map back to $f$. 
Since we are always mapping using monotonic segment  of $\mathcal{C}$, the image is at most one interval. 
Therefore, $l_f$, as well as each of the distances between base pairs of vertices, 
takes values in at most one attainable interval (the interval depends on the base pair). 
So each construction step takes constant time.  
Since the number of construction steps is $O(|V|)$, the Cayley configuration space can be computed in $O(|V|)$ time. 
\end{proof}










\noindent Next, we will look at the general case, 
where a graph can have more than one path, namely, 
more than one vertex at the last level $L_t$ (other than the endpoints of the given base non-edge). 
We say that each such last level vertex corresponds to a \emph{path}. 




\begin{lemma}[QIM for multi-path] \label{lem: QIM-k-path}
For a 1-dof tree-decomposable graph $G$ with low Cayley complexity on $f=(v_0,v_0')$, 
if the minimal realization type is fixed, 
we can use the QIM algorithm to correctly find the Cayley configuration space over $f$ for any linkage $(G, \bar{l})$,  
and this Cayley configuration space is a single interval.  
\end{lemma}

\begin{proof}
Consider a linkage $(G, \bar{l})$ with $k+1$ paths. 
For any vertex $v_i$ in the last level $L_t$ different from $v_0$ and $v_0'$, 
there exists a sequence of construction steps from $f$ to $v_i$, 
such that $v_i$ depends on all these construction steps. 
We call this construction sequence \emph{the $i^{th}$ path $p_i$ of $G$}, 
which is in fact a 1-path subgraph. By Proposition \ref{obs:1-path}, for each $p_i$, 
we can apply the QIM algorithm for 1-path linkages to find its Cayley configuration space 
over $f$, which is a single interval $I_i$. 
Then we take the intersection $I = \bigcap \limits_{1 \le i \le k} I_i$, which is also a single interval. 

By induction hypothesis, each path $(p_i, \bar{l})$ is realizable if and only if $l_f$ is in $I_i$, so if $l_f \notin I$, the linkage is not realizable. 
On the other hand, since the minimal realization type is fixed, when $l_f \in I$, all paths are simultaneously realizable 
(each value of $l_f$ corresponds to a unique realization of the entire linkage). 
Therefore, $I$ is the Cayley configuration space of $(G, \bar{l})$ over $f$.  
\end{proof}





Finally, we can prove the main theorem. 

\begin{proof}[Proof of Theorem \ref{obs:k-path}]
By Lemma \ref{lem: QIM-k-path}, for any linkage with underlying graph $G$, 
the Cayley configuration space over $f$ contains at most a single interval. 
The QIM algorithm takes $O(|V|)$ time on each path, and there can be at most $O(|V|)$ paths, so 
the Cayley computational complexity is  $O(|V|^2)$. 
\end{proof}

\begin{remark} This result shows the advantage of the QIM algorithm over the ELR algorithm we mentioned in Section \ref{sec:Combinatorial-interpretation-of}. First, realizing each extreme linkage takes $O(|V|)$ time, 
therefore realizing all $O(|V|)$ extreme linkages from $f$ takes $O(|V|^2)$ time. 
Moreover, note that when the reverse realization type is fixed, an interval endpoint can also arise from a change in reverse realization type. 
This is not an extreme linkage for the given base non-edge, but an extreme linkage for an extreme non-edge. 
So the ELR algorithm must consider extreme linkages for all these $O(|V|)$ possible base non-edges, 
 and the overall time complexity is $O(|V|^{3})$.
\end{remark} 





\medskip

Why does the QIM algorithm fails as is,  
if the minimal realization type is not fixed? 
Notice that when the minimal realization type is not fixed, 
we cannot simply take intersection of intervals as in proof of Lemma \ref{lem: QIM-k-path}, since realizations requiring different minimal realization types can generate the same  $l_f$. For example, refer to Figure \ref{F:obstruction} (b). 
There are two paths, $p_0$ and $p_1$, 
corresponding to vertices $v_5$ and $v_6$ in the last level respectively. 
For certain $\bar{l}$, 
$p_1$ may require $v_1$ and $v_4$ to be on the same side of $(v_3,v_0')$ such that $v_6$ can be realized. 
At the same time, 
$p_0$ may require $v_1$ and $v_4$ to be on different sides of $(v_3,v_0')$ such that $v_5$ can be realized.
These two different realization types can generate the same $l_f$. If we just take intersection of intervals of values for $l_f$, even the intersection is non-empty, the linkage may still be unrealizable since $p_0$ and $p_1$ require conflicting realization types. 
An alternative strategy would be to follow the construction steps, 
and perform $O(|V|)$ mappings altogether. 

The above discussion leads to the following conjecture: 

\begin{conjecture}
Even when the minimal realization type is not fixed, 
the QIM algorithm can be adapted to work for 1-dof tree-decomposable linkages whose underlying graphs have low Cayley complexity 
When the minimal realization type is fixed, 
this adapted algorithm can be used to obtain the Cayley configuration space in $O(|V|)$ time. 
\end{conjecture}

\medskip
\noindent 
Implementation of the algorithms discussed in this paper 
for finding Cayley configuration spaces is part of our new CayMos software,
whose architecture is described in \cite{bib:caymos} and 
web-accessible at http://www.cise.ufl.edu/\~{}menghan/caymos/. 
See also a video demonstrating QIM algorithm 
at http://www.cise.ufl.edu/\~{}menghan/caymos/qim.avi.
A different manuscript \cite{bib:beest} describes Cayley and Cartesian configuration space analysis and motion analysis 
of common and well-known mechanisms using CayMos.



\section{Finding a path of continuous motion between two realizations}
\label{sec:cont-path}

As we will show below, 
we can easily find a path of continuous motion between two given realizations, 
given the oriented Cayley configuration spaces of a linkage 
whose underlying graph has low Cayley complexity. 

\begin{theorem}[Continuous motion path Theorem] \label{theo:path}
For a generic linkage $(G, \bar{l})$ where $G$ has low Cayley complexity on $f$, 
\begin{enumerate}[(i)]
	\item Given two realizations, 
	there exist \emph{at most two} paths of continuous motion between them, 
	and the time complexity of finding such a path (provided one exists) is linear in the number of  
	interval endpoints of oriented Cayley configuration spaces that the path contains. 
	
	\item Given two realizations with the same minimal realization type, 
	there exists a unique path of continuous motion between them, 
	staying within the same minimal realization type, 
	and the time complexity of finding that path is $O(1)$. 
\end{enumerate}
\end{theorem}

\noindent \textbf{Idea of the proof. } 
The genericity is very important here. 
We analyze the continuous motion of the linkage in Lemma \ref{lem:endpoint} (continuous motion inside an interval) 
and Lemma \ref{lem:reachable} (change of realization type),
proving that the realization type can only be changed via extreme linkages at interval endpoints of oriented Cayley configuration spaces,
and (i) directly follows. 
Then we apply Theorem \ref{obs:k-path} (fixed minimal realization type) to get (ii). 



We mentioned in Remark \ref{rmk:endpoint} that
an extreme linkage realization may be an internal point 
of an interval in the unoriented Cayley configuration space. 
We now show that every extreme linkage realization corresponds to an interval endpoint 
of the oriented Cayley configuration space. 

\begin{lemma}[continuous motion inside an interval] \label{lem:endpoint}
(i) Between two points lying in the same interval of an oriented Cayley configuration space, 
there always exists a unique path of continuous motion staying within that interval. \\
(ii) No extreme linkage realization can correspond to an internal point of an interval in any oriented Cayley configuration space. 
\end{lemma}



\begin{lemma}[change of realization type] \label{lem:reachable}
During continuous motion, 
a linkage whose underlying graph has low Cayley complexity can only change forward realization type via endpoints of oriented Cayley configuration spaces, 
and only one entry of the forward realization type is switched by such a change. 
\end{lemma}

The proofs of Lemma \ref{lem:endpoint} and \ref{lem:reachable} are given in Appendix \ref{app:path}.




\begin{proof}[Proof of Theorem \ref{theo:path}]
For (i), the following algorithm finds a path of continuous motion between two realizations in time linear in the number of endpoints along that path: 

From the starting realization $G(p_s)$ with forward realization type $\sigma$, 
we take the oriented Cayley configuration space for realization type $\sigma$ and find the interval $I_\sigma$ that $G(p_s)$ is in. 
By Lemma \ref{lem:endpoint} (i), inside $I_\sigma$ there always exists a path of continuous motion. 
Take one endpoint $l_f = l_0$ of $I_\sigma$. 
In the corresponding realization $G(p_0)$, exactly one entry, say entry $i$, of the forward realization type is $0$. 
By Lemma \ref{lem:reachable}, 
the next immediately reachable oriented Cayley configuration space 
is uniquely determined, 
since its realization type $\tau$ should be the same as $\sigma$ except having the opposite sign in entry $i$.
Since intervals in an oriented Cayley configuration space are all disjoint, 
there is at most one interval in the oriented Cayley configuration space for realization type $\tau$ with $l_0$ as an endpoint. So 
we can find at most one interval $I_\tau$ immediately reachable. 
We repeat the process from the other endpoint of $I_\tau$ until we reach the interval containing the target realization, or we may go back to the starting interval $I_\sigma$, 
in that case there is no continuous motion path between the two given realizations. 
Each endpoint encountered leads to at most one next immediately reachable interval, so backtracking is never necessary.
Therefore the time complexity is linear in the number of endpoints along the path we found.
Since both endpoints of $I_\sigma$ could potentially lead to the target realization, 
there are \emph{at most two paths} between two given realizations. 


For (ii), by Theorem \ref{obs:k-path}, the two realizations lie in a single interval $I_\sigma$ in the corresponding oriented Cayley configuration space. 
By Lemma \ref{lem:endpoint} (i),
within a single interval of an oriented Cayley configuration space, there is always a continuous motion path.  
Therefore, there always exists a unique path maintaining the minimal realization type between the two realizations within $I_\sigma$. 
The time complexity of finding this path is $O(1)$. 
\end{proof}


Figure \ref{F:tracking} gives an example running 
of the algorithm described in the proof above, 
finding a path of continuous motion 
from realization (B1) with realization type $\sigma$ to (B2) with realization type $\tau$. 
We start from the interval $I_\sigma$ containing (B1) and take one endpoint of $I_\sigma$, which corresponds to extreme linkage realization (A1).  
Taking the entry of $\sigma$ which is 0 in (A1), and reversing its sign in $\sigma$, 
we get the next realization type $\tau$. 
Now we go from (A1) to (A2), 
which is essentially the same realization but contained in the oriented Cayley configuration space for realization type $\tau$, 
and the immediately reachable interval $I_\tau$ with (A2) as an endpoint realization is uniquely determined. 
Since $I_\tau$ contains the target realization (B2), a continuous path is successfully found. 


\begin{remark}
1. For Theorem \ref{theo:path} (ii), 
a second path of continuous motion could exist, 
if we remove the requirement that the path should stay within the same minimal realization type.  
In that path, the linkage leaves $I_\sigma$ from the endpoint 
closer to the starting realization than the target realization, 
takes various minimal realization types along the path, 
and reaches the target realization via the other endpoint of $I_\sigma$. 
By Theorem \ref{theo:path} (i), the time complexity of finding this path is  linear in the number of interval endpoints contained in it.

2. Not much improvement of the length of the paths is possible beyond  Theorem \ref{theo:path} (i), 
if the two realizations have either the same forward realization type or the same reverse realization type.  
When the two realizations have the same forward realization type, 
the  two realizations may belong to different intervals of the corresponding oriented Cayley configuration space. 
When the two realizations have the same reverse realization type, the two realizations may belong to different oriented Cayley configuration spaces. 
In both cases, it is not guaranteed that a path of continuous motion exists, 
and even if such a path exists, the number of interval endpoints contained in it is hard to determine. 
\end{remark}

\begin{corollary}[continuous motion paths between Cayley configurations]
\label{corollary:path_between_cayley}
To obtain a continuous path between two Cayley configurations where their forward realization types are unspecified, 
we run the algorithm given by Theorem \ref{theo:path} (i) for each candidate forward realization type of the starting Cayley configuration and each candidate forward realization type of the target Cayley configuration.
\end{corollary}

\noindent\textbf{Note.} For each pair of starting and target realizations there are at most two paths, but the number of such pairs could be exponential in the size of the linkage. 



\bigskip
\noindent 
Implementation of the algorithms  discussed above 
for finding a continuous motion paths is part of our new CayMos software,
whose architecture is described in \cite{bib:caymos} and 
web-accessible at http://www.cise.ufl.edu/\~{}menghan/caymos/. 
See also a demonstration video 
at \\http://www.cise.ufl.edu/\~{}menghan/caymos/motion.avi, and the screen-shot in Figure \ref{F:caymos_path}. 
A different manuscript \cite{bib:beest} describes Cayley and Cartesian configuration space analysis and motion analysis 
of common and well-known mechanisms using CayMos.




\begin{figure}[hbtp]
\begin{center}
	\includegraphics[width= .6\linewidth]{screenshots/path}
\end{center}

\caption{Finding a continuous motion path using our new CayMos software \protect\cite{bib:caymos} between two realizations. 
(A) The start and end realizations.
(B) The current realization, moving as the user traces the continuous motion path. 
(C) The 3D projection of the current connected component in the motion space. 
(D) The intervals of the oriented Cayley configuration spaces encountered along the path. }
\label{F:caymos_path}
\end{figure}

\subsection{Canonical bijective representation of the realization space as curves in an ambient space of minimum/minimal dimension}
\label{sec:ambient}

In this section, we deal with the problem of bijectively representing 
the realization space of 1-dof tree-decomposable linkages with low Cayley complexity,
which yields a meaningful visualization of the realization space and continuous motion.
In \cite{bib:beest}, we have given a bijective representation 
of the realization spaces using \emph{complete Cayley vectors}.
In this paper, we prove  Theorem \ref{thm:parameterize}
which greatly reduce the dimension of complete Cayley vectors. 
Specifically, for linkages with 1-path underlying graphs, 
we achieve the minimum possible dimension of the complete Cayley vector, which is 2.

\noindent\textbf{Important Note}: the bijective representation of minimal dimension requires the assumption 
that the clusters are globally rigid, and 
clusters sharing only two vertices with the rest of the graph be reduced into edges. 
\smallskip

Given a 1-dof tree-decomposable linkage $(G,\bar{l})$, 
where $G$ has low Cayley complexity as well as a 1-path construction from base non-edge $f=(v_0,v_0')$, with $v_n$ as the last constructed vertex, 
we define a minimum \emph{complete Cayley vector} $F$ as follows.
(1) If $G$ has only one construction step, $F = \left<f\right>$.
(2) If $G$ has two or more construction steps, $F=\left<f,f_1\right>$, 
where $f_1 = (v_0,v_n)$ if $(v_0,v_n)$ is a non-edge of $G$, otherwise $f=(v_0',v_n)$. 

The definition of a minimal \emph{complete Cayley vector}
for a general 1-dof tree-decomposable linkage 
is given  in Appendix \ref{app:cayley_vector}.

The minimal \emph{complete Cayley distance vector} of a realization $G(p)$
is the vector of distances for the non-edges in the complete Cayley vector. 

\smallskip

\begin{lemma}[1-path global rigidity] \label{lem:globally_rigid_1path}
For a 1-path, 1-dof tree-decomposable graph $(G,\bar{l})$ with low Cayley complexity, 
adding the non-edges in the minimum complete Cayley vector as edges 
makes $G$ \emph{globally rigid}; i.e., 
any 2D linkage with $G$ as the underlying graph has at most one realization. 
\end{lemma}


The proof is given in Appendix \ref{app:cayley_vector}.


\smallskip

\begin{lemma}[multi-path global rigidity] \label{lem:globally_rigid_multi}
For a 1-dof tree-decomposable graph $(G,\bar{l})$ with low Cayley complexity, 
adding the non-edges in the minimal complete Cayley vector as edges 
makes $G$ \emph{globally rigid}; i.e., 
any 2D linkage with $G$ as the underlying graph has at most one realization. 
\end{lemma}

The proof is given in Appendix \ref{app:cayley_vector}.


\smallskip

\begin{proposition}[minimality of representation]
\noindent (1) 
For a 1-path, 1-dof tree-decomposable graph with low Cayley complexity,
 the minimum complete Cayley vector $F$ contains the minimum number of edges that make $G$ globally rigid.

\noindent (2) 
For a 1-dof tree-decomposable graph with low Cayley complexity,
either the minimal complete Cayley vector $F$ is a minimal set of edges that makes $G$ globally rigid, 
or $F \setminus \{f\}$ is a minimal set of edges that makes $G$ globally rigid,
where $f$ is the given base non-edge of the graph.
\end{proposition}

The proof is straightforward.




\smallskip

\begin{theorem}[bijectivity of representation] \label{thm:parameterize}
\noindent (1)
For a generic 1-path, 1-dof tree-decomposable linkage with low Cayley complexity, 
there exists a bijective correspondence between the set of Cartesian realizations and 
points on a curve in $\mathbb{R}^2$, whose points are the minimum complete Cayley distance vectors. 

\noindent (2)
For a generic 1-dof tree-decomposable linkage with low Cayley complexity, 
there exists a bijective correspondence between the set of Cartesian realizations and 
points on a curve in $n$-dimension, whose points are the minimal complete Cayley distance vectors, 
where $n$ is the number of last level vertices of the underlying graph. 
\end{theorem}


\vspace{-10pt}

\begin{proof}
Given a Cartesian realization, 
the minimum or minimal complete Cayley distance vector is  uniquely determined.
Conversely, given a realizable complete Cayley distance vector, 
using Lemma \ref{lem:globally_rigid_1path} and 
 \ref{lem:globally_rigid_multi}, 
we  obtain a unique Cartesian realization
since the linkage is generic. 
Therefore we have a bijective correspondence between the set of Cartesian realizations 
and the set of minimum or minimal complete Cayley distance vectors. 
\end{proof}




Given a generic 1-dof tree-decomposable linkage with low Cayley complexity,
we can visualize the realization space using the \emph{canonical Cayley curve} \cite{bib:beest},
which is formed by the set of complete Cayley distance vectors 
representing each realization in the realization space.
The complete Cayley distance vector also enables us
to define a canonical distance between two connected components,
and distance separating two realizations without continuous motion path between them \cite{bib:beest}.


\bigskip
\noindent 
Implementation of the algorithms  discussed above 
is part of our new CayMos software,
whose architecture is described in \cite{bib:caymos} and 
web-accessible at http://www.cise.ufl.edu/\~{}menghan/caymos/. 
See also a demonstration video at http://www.cise.ufl.edu/\~{}menghan/caymos/motion.avi, 
and the screen-shot in Figure \ref{F:caymos_path}. 
A different manuscript \cite{bib:beest} describes Cayley and Cartesian configuration space analysis and motion analysis 
of common and well-known mechanisms using CayMos.






\section{Conclusion}

In Part I of this paper, 
we investigated the structure of Cayley configuration spaces for 1-dof tree-decomposable linkages, 
introduced the complexity measures associated with the underlying graphs, 
and formally gave algorithms to obtain such Cayley configuration spaces. 
For linkages whose underlying graphs have low Cayley complexity, 
we gave sufficient and necessary condition for small Cayley size and low Cayley computational complexity. 
We also gave an efficient algorithm to find a path of continuous motion for such linkages.


Implementation of the algorithms developed in Part I and Part II and further functionalities is part of our new CayMos software, described in \cite{bib:caymos} 
and web-accessible at \\http://www.cise.ufl.edu/\~{}menghan/caymos/. 
A different manuscript \cite{bib:beest} includes Cayley and Cartesian configuration space analysis and motion analysis of common and well-known mechanisms. 



\bibliographystyle{plain}
\bibliography{bib}

\appendix

\section{Proof of Theorem \ref{lem:algebraic} (structure of Cayley configuration space)}
\label{sec:Proof-for-lemma}

In the following, we denote the Cayley configuration space of a 1-dof tree-decomposable linkage $(G, \bar{l})$ over $f$ by $\Phi_{f}(G,\bar{l})$, and the oriented Cayley configuration space with forward realization type $\sigma$ by $\Phi_f(G, \bar{l}, \sigma)$.

\begin{proof} 
We first prove for a fixed forward realization type $\sigma$,  
that the theorem holds for the oriented Cayley configuration space, 
by induction on the number of construction steps from $f$. 

In the base case, $G$ has only one construction step $v_{1}\triangleleft(v_{0}\in T_{1},v_{0}'\in T_{2})$.
The distances $\bar{l}(v_{1},v_{0})$ and $\bar{l}(v_{1},v_{0}')$ are
fixed by clusters $T_1$ and $T_2$ respectively, 
and by triangle inequality, 
$\Phi_{f}(G,\bar{l},\sigma)$
is a single closed interval $[|\bar{l}(v_{1},v_{0})-\bar{l}(v_{1},v_{0}')|,\bar{l}(v_{1},v_{0})+\bar{l}(v_{1},v_{0}')]$. 
Clearly,  (1) and (2) hold. 
For (3),  without loss of generality, 
let $p(v_0)$ be the origin, $p(v_0')$ lie on the $x$-axis, 
and $p(v_{1})=(x_{1},y_{1})$ has positive $y$-coordinate. 
Let $R_{1}=\bar{l}(v_{1},v_{0})$,
$R_{2}=\bar{l}(v_{1},v_{0}')$ and $R_{3}=l(v_{0},v_{0}')=l_f$.
In $\triangle v_0v_0'v_1$, we have

\[x_{1}=\frac{R_{1}^{2}+R_{3}^{2}-R_{2}^{2}}{2R_{3}} \cdot \sigma_1\]
\[y_{1}=\frac{\sqrt{(R_{1}+R_{2}+R_{3})(R_{1}+R_{2}-R_{3})(R_{1}-R_{2}+R_{3})(-R_{1}+R_{2}+R_{3})}}{2R_{3}}\]
where $\sigma_1$ is the entry corresponding to the first construction step in the forward realization type $\sigma$.  

Since the linkage is generic, $R_{3}$, namely $l_f$, cannot be $0$, 
so both $x_{1}$ and $y_{1}$ are continuous functions of $l_f$. 
Moreover, since internal realizations of both $T_{1}$ and $T_{2}$ are uniquely specified, 
the coordinates of all other vertices in $T_{1}$ and $T_{2}$ are continuous functions
of coordinates of $v_{1}$, $v_{0}$ and $v_{0}'$, thus continuous
functions of $l_f$. 

\smallskip{}




As induction hypothesis, 
assume that the theorem holds for linkages whose underlying graph $G_f(k-1)$ has $k-1$ construction steps.
Consider a graph $G_f(k)$ with $k$ construction steps, 
obtained by adding one more construction step $v_{k}\triangleleft(u_{k}\in T_{1},w_{k}\in T_{2})$
to $G_{f}(k-1)$. 
For any linkage $(G_f(k), \bar{l})$, according to Statement (3) of the induction hypothesis, 
$l(u_k,w_k)$ is a continuous function of $l_f$, say $l(u_k,w_k)=g(l_f)$. 
By triangle inequality,  $l(u_{k},w_{k})$ is restricted to the interval $[\min,\max]$
where $\min=|\bar{l}(u_{k},v_{k})-\bar{l}(w_{k},v_{k})|$ and $\max=\bar{l}(u_{k},v_{k})+\bar{l}(w_{k},v_{k})$.
This restriction may create new candidate interval endpoints in $\Phi_{f}(G_{f}(k),\bar{l}, \sigma)$,
namely $g^{-1}(\min)$ and $g^{-1}(\max)$, as shown in Figure~\ref{F:RccStepN}. 
A candidate endpoint is actually a new interval endpoint, 
only if its corresponding extreme linkage realization 
$p(\hat{G}_{f}(k),\bar{l}^{\min},\sigma)$ (resp. $p(\hat{G}_{f}(k),\bar{l}^{\max},\sigma)$) does exist. 
So (1) and (2) also hold for $(G_{f}(k),\bar{l})$.

\begin{figure}[h]
	\psfrag{uw}{\small $l(e_{k})$}\psfrag{f}{\small $l_f$}
	\psfrag{t1}{\small $I_1$} \psfrag{t2}{\small $I_2$}
	
	\begin{centering}
	\includegraphics[width=0.35\textwidth]{img/RccStepN} 
	\par\end{centering}
	
	\caption{For Theorem \ref{lem:algebraic}. New constraint on $l(u_{k},w_{k})$
	creating interval endpoints in $\Phi_{f}(G_{f}(k),\bar{l}).$ $\bullet$:
	$g^{-1}(\min)$; { } $\circ$: $g^{-1}(\max)$. }

\label{F:RccStepN} 
\end{figure}

To prove (3), take any vertex $v$ in $G_{f}(k)$. 
By induction hypothesis, if $v\notin (T_{1}\cup T_{2})$, $p(v)$ is a continuous function of $l_f$. 
For $v\in (T_{1}\cup T_{2})$,  we first consider $p(v_{k})$. 
For convenience, first rotate and translate the coordinate system
so that $p(u_{k})$ is at the origin,  $p(w_{k})$ is on the $x$-axis, 
and $p(v_{k}) = (x_k, y_k)$ have positive $y$-coordinate. Let $R_{1}=\bar{l}(v_{k},u_{k})$,
$R_{2}=\bar{l}(v_{k},w_{k})$ and $R_{3}=l(u_{k},w_{k})$.
In $\triangle u_kw_kv_k$, we have

\[x_{k}=\frac{R_{1}^{2}+R_{3}^{2}-R_{2}^{2}}{2R_{3}} \cdot \sigma_k\]

\[y_{k}=\frac{\sqrt{(R_{1}+R_{2}+R_{3})(R_{1}+R_{2}-R_{3})(R_{1}-R_{2}+R_{3})(-R_{1}+R_{2}+R_{3})}}{2R_{3}}\]
where $\sigma_k$ is the entry corresponding to the $k^{th}$ construction step in the forward realization type $\sigma$.  

Since the linkage is generic, $R_{3}>0$. So both $x_{k}$ and $y_{k}$ are continuous functions of $l_f$. 
Moreover, since internal realizations for both $T_{1}$ and $T_{2}$
are specified, the coordinates of any other $v\in (T_{1}\cup T_{2})$ 
are continuous functions of $p(u_{k})$, $p(v_{k})$ and $p(w_{k})$, 
thus continuous functions of $l_f$. 
Consequently, for any non-edge $(u,w)$, 
$l(u,w)$ is also a continuous function of $l_f$.
Note that this continuity is not affected even if we transform back to the original coordinate system. 

For the complete Cayley configuration space, (1) and (2) still hold 
since $\Phi_f(G,\bar{l})$ is just the union of  oriented Cayley configuration spaces 
over all possible forward realization types.
\end{proof}


\section{Finding Cayley configuration spaces by realizing all extreme linkages (ELR)}
\label{sec:not-low}


Given a linkage $(G, \bar{l})$ and a forward realization type $\sigma$, 
in the ELR algorithm, 
we use a set ${\cal I}_{\sigma}$ to store candidate intervals of the oriented Cayley configuration space, 
which is initially the entire $\mathbb{R}^1$. 
For each realization step $i$, 
we update ${\cal I}_{\sigma}$ by considering restrictions on $l_f$ 
from all extreme linkages realizations of $\hat{G}_f(i)$ with forward realization type $\sigma$. 
After we have done this for every realization step, ${\cal I}_{\sigma}$ is the oriented Cayley configuration space. 


\noindent \textbf{Algorithm (ELR):} 

\indent ${\cal I}_{\sigma}$ $\leftarrow$ $(-\infty, +\infty)$ \\
\indent \textbf{for} $i=1$ \textbf{to} $k$ \textbf{do}
		\indent [\textit{$k$ is the number of $G$'s  construction steps}] \\
	\indent \indent $S \gets \emptyset$
		\indent [\textit{set of candidate interval endpoints}] \\
	\indent \indent \textbf{for} every extreme linkage realization $p$ of $\hat{G}_f(i)$ with forward realization type $\sigma$\\
		\indent\indent\indent \textbf{if} $(G \cup f, \bar{l}\cup l_f)$ is realizable \\
			\indent\indent\indent\indent add $l_f$ value of $p$ to $S$ \\
	\indent \indent \textbf{for} each candidate endpoint $l_0$ in $S$ \\
	\indent \indent \indent UPDATE(${\cal I}_{\sigma}$, $l_0$) 
				\indent [\textit{see following discussion}] \\
\indent \textbf{return} ${\cal I}_{\sigma}$

When updating ${\cal I}_{\sigma}$, we need to notice that 
not every candidate Cayley configuration $l_0$ in $S$
actually creates new restriction on ${\cal I}_{\sigma}$.
Recall from the proof of Theorem \ref{lem:algebraic} that
a realization step $v_{i}\triangleleft(u_{i},w_{i})$
restricts $l(u_{i},w_{i})=g(l_f)$ in $[\min,\max]$. 
As shown in Figure~\ref{F:endpoints}, there are three possible cases for a candidate configuration $l_0$: 
(a) both the left and the right neighborhood of $l_0$ fall into ${\cal I}_{\sigma}$; 
(b) the left neighborhood of $l_0$ falls into ${\cal I}_{\sigma}$ but the right does not, and symmetrically,
the right neighborhood falls into ${\cal I}_{\sigma}$ but the left does not;
(c) neither the left nor the right neighborhood of $l_0$ falls into ${\cal I}_{\sigma}$, meaning that $l_f$ itself is the only realization in the neighborhood. 
In (b), $l_0$ creates an new endpoint in ${\cal I}_{\sigma}$. 
In (c), $l_0$ creates an isolated point in ${\cal I}_{\sigma}$. 
In (a), $l_0$ does not create any interval endpoint in ${\cal I}_{\sigma}$. 



\begin{figure}[h]
	\psfrag{xx}{$l_f$}\psfrag{yy}{$l(u_{k},w_{k})$}
	\psfrag{y1}{$g(l_f)$} \psfrag{y2}{$=$}
	\psfrag{y3}{$l(u_{k},w_{k})$} \psfrag{aa}{$\diamond$}
	\psfrag{bb}{$\bullet$} \psfrag{cc}{$\circ$} 
	
	\psfrag{I1}{$I_{1}$} \psfrag{I2}{$I_{2}$}
	
	\begin{centering}
	\includegraphics[width=0.9\textwidth]{img/endpoints} 
	\par\end{centering}
	
	\caption{$\min=|\bar{l}(v_{k},u_{k})-\bar{l}(v_{k},w_{k})|$, $\max=\bar{l}(v_{k},u_{k})+\bar{l}(v_{k},w_{k})$.
	(a) candidate $l_f$ values $\diamond$ that are internal points of some interval in ${\cal I}_{\sigma}$, not endpoints;
	(b) candidate $l_f$ values $\bullet$ that are new interval endpoints
	in ${\cal I}_{\sigma}$, creating intervals $I_{1}$ and $I_{2}$; 
	(c) candidate $l_f$ value $\circ$ that creates an isolated point in ${\cal I}_{\sigma}$. }


\label{F:endpoints} 
\end{figure}


So in the UPDATE procedure, 
for each candidate Cayley configuration $l_0$, 
we check if there is any realization, with $l_f$ value between $l_0$
and the immediately preceding (resp. immediately succeeding) candidate interval endpoint 
in ${\cal I}_{\sigma}$. 



\noindent \textbf{Algorithm: UPDATE(${\cal I}_{\sigma}$, $l_f^{\min}$, $l_f^{\max}$)} 

\indent $prev \gets \max \{ l \in {\cal I}_{\sigma} | l < l_0 \}$, {   }  $next \gets \min \{ l \in {\cal I}_{\sigma} | l > l_0 \}$ \\ \indent $p \gets (prev+l_0)/2$, {   } $n \gets (next+l_0)/2$ \\ \indent $P \gets true$ if $p$ has corresponding realization, $false$ otherwise \\
\indent $N \gets true$ if $n$ has corresponding realization, $false$ otherwise \\
\indent \textbf{if} exactly one of $P$ and $N$ is $true$  \\
	\indent \indent add $l_0$ as an endpoint in ${\cal I}_{\sigma}$ \\
\indent \textbf{elseif} both $P$ and $N$ are $false$ \\
	\indent \indent add $l_0$ as an (isolated) endpoint to ${\cal I}_{\sigma}$ 



\noindent To obtain the complete Cayley configuration space, we just take $\Phi_{f}(G,\bar{l})=\bigcup\limits _{\sigma}{\cal I}_{\sigma}$.

\section{Exponential blow-up of Cayley size and Cayley complexity when the minimal realization type is not specified}
\label{sec:Exponential}

We provide in this section an example, 
that the Cayley size of a graph with low Cayley complexity can be exponential in $|V|$, 
even when a forward realization type is specified. 
One can use a symmetric example to show that if we just fix the reverse realization type, there can be exponentially many non-empty oriented Cayley configuration spaces. 

\begin{observation} \label{obs:interval-expo} 
The Cayley size of a graph with low Cayley complexity can be exponential
in the number of construction steps if we only fix the forward realization type.
\end{observation}

\begin{proof}
We give an example of a 1-dof tree-decomposable linkage $(G, \bar{l})$
with a fixed forward realization type $\sigma$ and low Cayley complexity, 
which has exponential Cayley size. 
See Figure \ref{F:exponent-graph}. 
The base non-edge is $f = f_{1}=(v_{1},v_{3})$.
For convenience we slightly abuse our notation to let 
the construction step number start from $0$, 
so the $0^{th}$ and $1^{st}$ construction steps are 
$v_{4}\triangleleft(v_{1},v_{3})$
and $v_{2}\triangleleft(v_{1},v_{3})$ respectively. 
They form the outermost quadrilateral $Q_{1}=v_{4}v_{3}v_{2}v_{1}$.

For every $k>1$, 
the $k^{th}$ construction step  is $v_{k+3}\triangleleft(v_{k+2},v_{k})$,
which appends one vertex and two edges to the graph, and forms a nested
quadrilateral $Q_{k}=v_{k+3}v_{k+2}v_{k+1}v_{k}$. 
We denote the four
edges of $Q_{k}$ as $(v_{k+3},v_{k+2})=s_{k,1}$, $(v_{k+2},v_{k+1})=s_{k,2}$,
$(v_{k+1},v_{k})=s_{k,3}$ and $(v_{k},v_{k+3})=s_{k,4}$, 
and two diagonals $(v_{k+3},v_{k+1})=f_{k+1}$, $(v_{k+2},v_{k})=f_{k}$.
Notice that $Q_{k}$ shares two edges with $Q_{k-1}$: $s_{k,2}=s_{k-1,1}$, $s_{k,3}=s_{k-1,2}$.
The forward realization type $\sigma$ is assigned such that $v_{k+3}$ and $v_{k+1}$ lies on
different side of $f_{k}$. 

Clearly, $G$ is 1-path with low Cayley complexity. 
So we use the QIM algorithm introduced in Section
\ref{sub:QIM} to compute $\Phi_{f}(G,\bar{l})$. 
We start from the last extreme edge, 
and repetitively map $l(f_k)$ to get intervals for $l(f_{k-1})$, 
until we obtain the intervals for $l(f_1)$. 

\begin{figure}[h]
	\psfrag{1}{$v_{1}$}\psfrag{2}{$v_{3}$}\psfrag{3}{$v_{2}$}\psfrag{4}{$v_{4}$}\psfrag{5}{$v_{5}$}\psfrag{6}{$v_{6}$}\psfrag{7}{$v_{7}$}\psfrag{8}{$v_{8}$}\psfrag{9}{$v_{9}$}
	
	\begin{centering}
	\includegraphics[width=0.35\textwidth]{img/exponent-graph}
	\par\end{centering}
	
	\caption{A 1-dof tree-decomposable
	graph that can have exponential Cayley size for base non-edge $f_{1}=(v_{1},v_{3})$.
	The graph is a series of nested quadrilaterals and each cluster is
	an edge.}


\label{F:exponent-graph} 
\end{figure}

\noindent \textbf{Example C.1}. 
We choose $\bar{l}$ such that for $Q_{1}$, $\bar{l}(s_{1,1})=\bar{l}(v_{4},v_{3})=8$,
$\bar{l}(s_{1,2})=\bar{l}(v_{3},v_{2})=8.1$, $\bar{l}(s_{1,3})=\bar{l}(v_{2},v_{1})=7.9$,
$\bar{l}(s_{1,4})=\bar{l}(v_{4},v_{1})=1$. 
The ellipse $\mathcal{C}$ relating
the two diagonals of $Q_{1}$ is shown in Figure \ref{F:diagonal-curve}. Only
the upper half of the curve (shown in solid line) corresponds to realizations
with forward realization type $\sigma$. 

For each $k>1$,
we assign $\bar{l}(s_{k,1})$ and $\bar{l}(s_{k,4})$
by the following: 
(1) Observe the ellipse $\mathcal{C}$ of $Q_{k-1}$, as shown in Figure \ref{F:diagonal-curve}. 
Denote the leftmost point on $\mathcal{C}$ by $p_{\min}(f_{k-1})$, the rightmost point by $p_{\max}(f_{k-1})$, the topmost point by $p_{\max}(f_{k})$. 
Let $l_2$ be the length of $f_k$ at $p_{\max}(f_{k})$, 
and $l_1$ be the larger one of the lengths of $f_k$ at $p_{\min}(f_{k-1})$ and $p_{\max}(f_{k-1})$. 
The interval $[l_1,l_2]$ is attainable by $l(f_k)$ in $G_f(k-1)$. 
(2) Assign $\bar{l}(s_{k,1})$ and $\bar{l}(s_{k,4})$ 
such that $\bar{l}(s_{k,1})-\bar{l}(s_{k,4})=\bar{l}^{\min}(f_k)=(1+\epsilon)l_1$,
$\bar{l}(s_{k,1})+\bar{l}(s_{k,4})=\bar{l}^{\max}(f_k)=(1-\epsilon)l_2$, 
where $\epsilon$ is a positive value small enough so that $\bar{l}(s_{k,1})$
and $\bar{l}(s_{k,4})$ have positive solutions. 
In this way, $l(f_k)$ is restricted to an interval $[\bar{l}^{\min}(f_k),\bar{l}^{\max}(f_k)]$
slightly tighter than $[l_1,l_2]$.


For example, we want to assign $\bar{l}$ for the $2^{nd}$ realization step $v_{5}\triangleleft(v_{4},v_{3})$.
As shown in Figure \ref{F:diagonal-curve}, in $Q_{1}$, $l_2=7.9+1=8.9$. 
For $p_{\min}(f_{k-1})$, $l(f_1) = 8-1 = 7$, the corresponding $l(f_2) \approx 8.36$. For $p_{\max}(f_{k-1})$, $l(f_1) = 8+1 = 9$, the corresponding $l(f_2) \approx 7.42$. 
Therefore $l_1=\max[8.36, 7.42] = 8.36$.
So let $\bar{l}^{\min}(f_2)=\bar{l}(v_{5},v_{4})-\bar{l}(v_{5},v_{3})=(1+10^{-5})l_1 \approx 8.364$,
$\bar{l}^{\max}(f_2)=\bar{l}(v_{5},v_{4})+\bar{l}(v_{5},v_{3})=(1-10^{-5})l_2 \approx 8.900$.
We assign $\bar{l}(v_{5},v_{4}) \approx 8.632$, $\bar{l}(v_{5},v_{3}) \approx 0.268$.
The two new extreme linkages $(\hat{G}_{f}(2),\bar{l}^{\min})$ and
$(\hat{G}_{f}(2),\bar{l}^{\max})$ each has two realizations, and each
of these realizations creates a new endpoint in  $\Phi_{f}(G,\bar{l})$:
$(\hat{G}_{f}(2),\bar{l}^{\max})$ corresponds to realizations in Figure
\ref{F:diagonal-curve} (b) and (d) , $(\hat{G}_{f}(2),\bar{l}^{\min})$
corresponds to realizations in Figure \ref{F:diagonal-curve} (a)
and (c) (point $b$, $d$, $a$ and $c$ in the left graph respectively).
Since $l_{b}(f_{1})\approx7.00$, $l_{d}(f_{1})\approx8.52$,
$l_{a}(f_{1})\approx7.49$, $l_{c}(f_{1})\approx7.51$,
$\Phi_{f}(G,\bar{l})$ contains two intervals $I_{1}=[7.00,7.49]$
and $I_{2}=[7.51,8.52]$, 
corresponding to two different reverse realization types. 


\begin{figure}[h]
	\psfrag{I1}{$I_{1}$}\psfrag{I2}{$I_{2}$}\psfrag{3}{$v_{2}$}\psfrag{4}{$v_{4}$}\psfrag{1}{$v_{1}$}\psfrag{2}{$v_{3}$}
	
	\psfrag{e}{\small $f_2=(v_{4},v_{2})$} \psfrag{f}{\small $f_1=(v_{1},v_{3})$} 
	\psfrag{er}{\small $\bar{l}^{\max}(f_2)$} \psfrag{el}{\small $\bar{l}^{\min}(f_2)$}
	
	\begin{centering}
	\includegraphics[width=0.85\textwidth]{img/diagonal-curve} 
	\par\end{centering}
	
	\caption{Example C.1. Ellipse $\mathcal{C}$ for quadrilateral $Q_1 = v_{4}v_{3}v_{2}v_{1}$.
	Length of extreme edge $(v_{2},v_{4})$ is restricted by the 
	$2^{nd}$ realization step, and of $l(v_{1},v_{3})$ has
	2 intervals. }

\label{F:diagonal-curve} 
\end{figure}



Table \ref{T:edge-length} shows $\bar{l}$ for
the subsequent construction steps, computed by the procedure described above. 
Figure \ref{F:diagonal-curve-2} shows $\Phi_{f}(G_{f}(6),\bar{l})$. 
The single interval of $l(v_{5},v_{3})$ maps to 2 intervals
for $l(v_{2},v_{4})$: $I_{1}=[8.36,8.48]$, $I_{2}=[8.49,8.74]$,
and 4 intervals for the base non-edge $l(v_{1},v_{3})$:
$I_{11}=[7.000,7.008]$, $I_{21}=[7.010,7.121]$, $I_{22}=[8.039,8.391]$,
$I_{12}=[8.403,8.524]$. 

\begin{centering}
\begin{table}
	\begin{tabular}{|c|c|c|c|c|c|}
	\hline 
	$k$ & $\bar{l}(s_{k,1})$ & $\bar{l}(s_{k,2})$ & $\bar{l}(s_{k,3})$ & $\bar{l}(s_{k,4})$ & number of intervals for $l(v_{1},v_{3})$ after step $k$ \\
	\hline
	\hline 
	2 & 8.632 & 8 & 8.1 & 0.268 & 2 \\
	\hline 
	3 & 8.306 & 8.632 & 8 & 0.062 & 4 \\
	\hline 
	4 & 8.044 & 8.306 & 8.632 & 0.017 & 8\\
	\hline 
	5 & 8.645 & 8.044 & 8.306 & 0.004 & 16\\
	\hline 
	6 & 8.310 & 8.645 & 8.044 & 0.001 & 32\\
	\hline 
	7 & 8.045 & 8.310 & 8.645 & 0.0003 & 64\\
	\hline 
	8 & 8.645 & 8.045 & 8.310 & 0.00006 & 128\\
	\hline 
	9 & 8.310 & 8.645 & 8.045 & 0.00001 & 256\\
	\hline 
	10 & 8.045 & 8.310 & 8.645 & 0.000004 & 512\\
	\hline
	\end{tabular}
\medskip	
\caption{Example C.1. Edge lengths of quadrilateral $Q_{k}$ for construction step $2$ to $10$.}
\label{T:edge-length}
\end{table}
\end{centering}

\begin{figure}[h]
	\psfrag{I1}{\small $I_{1}$} \psfrag{I2}{\small $I_{2}$} 
	\psfrag{I11}{\tiny $I_{11}$} \psfrag{I12}{\tiny $I_{12}$}\psfrag{I21}{\tiny$I_{21}$}\psfrag{I22}{\tiny$I_{22}$}
	
	\psfrag{e=52}{\small $f_3=(v_{5},v_{3})$} \psfrag{f=34}{\small $f_2=(v_{2},v_{4})$}
	\psfrag{e=34}{\small $f_2=(v_{2},v_{4})$} \psfrag{f=12}{\small $f_1=(v_{1},v_{3})$}
	
	\begin{centering}
	\includegraphics[width=0.7\textwidth]{img/diagonal-curve2} 
	\par\end{centering}
	
	\caption{Example C.1. Ellipse $\mathcal{C}$ for quadrilaterals (a) $Q_2 = v_{5}v_{4}v_{3}v_{2}$
	and (b) $Q_1 = v_{4}v_{3}v_{2}v_{1}$ after $v_{6}$ is realized. 
	The Cayley configuration space over $(v_1,v_3)$ is divided into 4 intervals $I_{11}$,
	$I_{21}$, $I_{22}$, $I_{12}$. }

\label{F:diagonal-curve-2} 
\end{figure}


In general, the $k^{th}$ realization step  $(k > 1)$ produces one interval for $l(f_k)$
which maps to 2 intervals for $l(f_{k-1})$, 4 intervals for $l(f_{k-2})$, and finally
$2^{k-1}$ intervals for $f_{1}=(v_{1},v_{3})$. 
Notice that there is no overlapping since 
the curve is monotonic in each interval.
\end{proof}









\section{Details of the QIM Algorithm}
\label{app:qim}

\begin{remark}
\label{rmk:qim}
(a) Refer to $(u_2,u_2')$ and $(u_1,u_1')$ in Figure \ref{F:quadrilaterals} (a). 
Consider the two triangles $\triangle v_3u_1u_1'$ and $\triangle v_3u_2u_2'$. 
Since $T_5$ and $T_6$ are fixed clusters, the lengths of triangle edges $(v_3,u_1)$, $(v_3,u_1')$, $(v_3,u_2)$ and $(v_3,u_2')$ are fixed. 
Moreover, if we know one of the two angles, $\angle u_1v_3u_1'$ and $\angle u_2v_3u_2'$, we can easily obtain the other. 
So from a value of $l(u_1,u_1')$, by the law of cosines, we can obtain $\angle u_1v_3u_1'$ and thus $\angle u_2v_3u_2'$, from which we can get a unique corresponding value of $l(u_2,u_2')$. Symmetrically, each value of $l(u_2,u_2')$ corresponds to a unique value of $l(u_1,u_1')$. 


\noindent(b) Refer to $(u_2,u_2')$ and $(u_1,u_1')$ in Figure \ref{F:quadrilaterals} (b). 
By (i), there is a one-to-one correspondence between $l(v_3,u_0')$ and $l(u_2,u_2')$ with constant time cost. 
Notice that $(v_3,u_0')$ and $(u_1,u_1')$ are the two diagonals of quadrilateral $v_3u_1u_0'u_1$. 
Therefore we can map to $l(u_1,u_1')$. 
\end{remark}

\begin{proof}[Proof of Lemma \ref{lem:QIM-1-path} (QIM for 1-path)]
We prove by induction on the number of construction steps of $G$. 

In the base case, there is only one construction step, and the Observation is vacuously true. 

As the induction hypothesis, we assume that the algorithm correctly generates the Cayley configuration space for linkages with less that $k$ construction steps. 
For a graph $G$ with $k$ construction steps, by the Recursive Structure Lemma, 
there are only two construction steps directly based on $f=(u_0,u_0')$, 
and $G' = G \setminus \{u_0,u_0'\}$ (or $G' = G \setminus \{u_0\}$) is a 1-path, 
1-dof, tree-decomposable graph with low Cayley complexity on $(u_1,u_1')$, 
and less than $k$ construction steps.

By induction hypothesis, 
$(G', \bar{l})$ is realizable if and only if $l(u_1,u_1')$ is in the set $S_1$ of intervals generated by applying QIM on $(G',\bar{l})$. 
For QIM on $(G, \bar{l})$, we do an additional mapping to get $S_0$ from $S_1$. 
When $l_f$ belongs to  an interval of $S_0$, the first four-cycle $v_0u_1'v_0'u_1$ is clearly realizable; 
moreover, $l(u_1,u_1')$ is in $S_1$ since $S_0$ is generated by mapping from $S_1$, so $(G',\bar{l})$ is also realizable. 
Thus for all $l_f$ values in $S_0$, $(G,\bar{l})$ is realizable. 
On the other hand, when $l_f$ is not in any interval of $S_0$, either the first four-cycle is not realizable, or $(G',\bar{l})$ is not realizable, 
so all realizable $l_f$ values are contained in $S_0$. 
Therefore $S_0$ is the Cayley configuration space of $(G,\bar{l})$ over $f$.
\end{proof}

\section{Proofs for the Continuous motion path Theorem} \label{app:path}


\begin{proof}[Proof of Lemma \ref{lem:endpoint} (continuous motion inside an interval)]
\noindent (i) 
Let $l_f = l_1$ and $l_f = l_2$ be in the same interval $I_\sigma$ in oriented Cayley configuration space $\Phi_f(G, \bar{l}, \sigma)$, and the corresponding realizations be $G(p_1)$ and $G(p_2)$ respectively. 
By definition of oriented Cayley configuration space,  
there exists a ruler and compass realization with realization type $\sigma$ for
every Cayley configuration in $I_\sigma$ between $l_1$ and $l_2$.
One can easily verify that this series of realizations gives a path of continuous motion between $G(p_1)$ and $G(p_2)$. 


\noindent (ii) Assume some extreme graph $\hat{G}_f(k)$ has a corresponding extreme linkage realization $G(p)$, 
with $f$'s length $l_f = l_0$ being an internal point of an interval $I_\sigma$ in oriented Cayley configuration space $\Phi_f(G, \bar{l}, \sigma)$.
Since the length $l_e^*$ of the extreme edge $e$ in $G(p)$ is the maximum (resp. minimum) possible value for $e$'s length $l(e)$, 
it follows that for all small enough $\epsilon$, the oriented Cayley configurations $l_f = l_0 - \epsilon$ and $l_f = l_0 + \epsilon$ in $I_\sigma$ both correspond to realizations with $l(e) < l_e^*$ (resp. $l(e)>l_e^*$). 

By the proof of Theorem \ref{obs:k-path} (fixed minimal realization type), $l(e)$  would be a monotonic function of $l_f$ if the minimal realization type is unchanged.
Since $l(e)$ increases before $l_f$ reaches $l_0$ and decreases after $l_f$ reaches $l_0$, 
the minimal realization type must change at $l_0$. 
Since the forward realization type is unchanged, 
we can conclude that the reverse realization type must change at $l_0$. 
Namely, there are two pairs of collinear bars in $G(p)$, 
one corresponds to the $k^{th}$ forward construction step, the other corresponds to the change in the reverse realization type. 
However this violates our assumption of genericity of linkages that  at most two bars can be collinear in any realization. 
\end{proof}

\begin{proof}[Proof of Lemma \ref{lem:reachable} (change of realization type)]
Suppose the linkage changes forward realization type at $l_f = l_0$ during the continuous motion. 
In other words, the linkage realization has forward realization type $\sigma$ immediately before reaching $l_0$,
and has forward realization type $\tau$ immediately after reaching $l_0$, where $\sigma \ne \tau$. 
To guarantee continuous motion, the forward realization type at $l_0$ must be compatible with both $\sigma$ and $\tau$. 
Due to the genericity assumption on linkages, 
this is possible only if $\sigma$ and $\tau$ differ at exactly one entry which is $0$ at $l_0$. 
Therefore the realization at $l_0$ is a realization of an extreme linkage. 
By Lemma \ref{lem:endpoint} (ii), $l_0$ must be an interval endpoint of both $\Phi_f(G, \bar{l},\sigma)$ and $\Phi_f(G, \bar{l},\tau)$.
\end{proof}

\section{Details of the complete Cayley vector}
\label{app:cayley_vector}

\subsection{Definition of minimal complete Cayley vector for general 1-dof graphs of low Cayley complexity}

Given a general 1-dof tree-decomposable linkage $(G,\bar{l})$ with low Cayley complexity on base non-edge $f = (v_0,v_0')$,
we define a minimal \emph{complete Cayley vector} $F$ using the following iterative procedure
of adding edges to $G$ until it becomes triconnected and redundantly rigid (i.e. globally rigid).

\noindent (1) 
Add the edge $(v_0,v_0')$ to $G$, and call the obtained graph $G^+$. 
As long as $(v_0,v_0')$ remains a two-separator of $G^+$, 
we take two arbitrary vertices $v_k$ and $v_k'$,
which are in the last level of $G$ and are separated by $(v_0,v_0')$ in $G^+$, 
and add the edge $(v_k,v_k')$ to $G^+$. 

\noindent (2) While there are remaining vertices in the last level of $G^+$, we repeat the following.
(a) If there are two or more vertices in the last level of $G^+$, we take an arbitrary pair from them and add an edge between.
(b) If there is only one vertex $v_k$ in the last level of $G^+$, 
we add one of the following three edges to $G^+$: 
(i) the edge $e_1 = (v_0,v_k)$, if $(v_0, v_k)$ is a non-edge in $G$; 
(ii) the edge $e_2 = (v_0',v_k)$, if $e_1$ is not added and $(v_0',v_k)$ is a non-edge in $G$;
(iii) the edge $e_3 = (v_k,v_k')$ where $v_k'$ is any vertex in last level vertex in $G$ different from $v_k$, if neither $e_1$ nor $e_2$ is added.

\noindent (3) While $G^+$ is not 3-connected, we repeat the following. 
Take any two separator $(v_m,v_n)$ of $G^+$. 
Take any last level vertex $v_k$ of $G$ that cannot be constructed from $f$ in $G$ without constructing both $v_m$ and $v_n$. 
If $(v_0,v_k)$ is a non-edge, we add the edge $(v_k,v_0)$ to $G^+$, otherwise we add the edge $(v_0',v_k)$ to $G^+$. 

The minimal complete Cayley vector $F$ consists of all the edges we added in this procedure above. 


\subsection{Proofs of global rigidity}

\begin{proof}[Proof of Lemma \ref{lem:globally_rigid_1path} (1-path global rigidity) ]
The 3-connectivity is clear to verify.

For redundant rigidity, we use $G^+$ to denote the graph obtained by adding the non-edges
in  the complete Cayley vector as edges to $G$. 
Without loss of generality, assume the edges added are $(v_0,v_0')$ and $(v_0,v_n)$.
For contradiction, assume that after removing an edge $(x,y)$ from $G^+$, 
the remaining graph $G'$ is not rigid. 
Clearly $(x,y)$ is not the edge $(v_n,v_0)$.

Since we assume that each cluster of $G$ is globally rigid, here we only consider the case that
$(x,y)$ is not in any non-trivial cluster of $G$. 

Since $G'$ is not rigid and $G'$ satisfies $|E'|=2|V'|-3$, 
there must exist a subgraph $G_S$ of $G$
with $|E_S|>2|V_S|-3$. 
The subgraph $G_S$ must also contain the edge $(v_n,v_0)$, 
since the tree-decomposable graph $G \cup f$ is minimally rigid thus cannot contain a subgraph satisfying such a condition. 

So $G_S = G_f \cup (v_n,v_0)$, where $G_f$, having $|E_f|=2|V_f|-3$,
is a minimally rigid subgraph of the tree-decomposable graph $G \cup f$, 
and contains both $v_0$ and $v_n$. 
We have the following cases:

\noindent (a) $G_f$ is a single edge, that is $(v_n,v_0)$. 
This contradicts our construction of the complete Cayley vector that $(v_0,v_n)$ should be a non-edge of $G$. 

\noindent (b) $G_f$ is a cluster that is not reduced to an edge.  
Since $G_f$ contains $v_n$ and $v_n$ is a last level vertex, 
this contradicts our assumption that each cluster not reduced to an edge
has at least 3 shared vertices with the rest of the graph.

\noindent (c) $G_f$ contains the edge $(v_0,v_0')$, and does not contain both $x$ and $y$. 
Since $G_f$ contains $v_n$,  there exists a construction sequence from $(v_0,v_0')$ to $v_n$. 
So $v_n$ can be constructed in $G$ from $(v_0,v_0')$ without first constructing both $x$ and $y$, 
contradicting the 1-path property of $G$.
\end{proof}





\begin{proof}[Proof of Lemma \ref{lem:globally_rigid_multi} (multi-path global rigidity) ]
The 3-connectivity is clear from the construction. 

For redundant rigidity, 
we use $G^+$ to denote the graph obtained by adding the non-edges
in  the complete Cayley vector as edges to $G$. 
For contradiction, assume that after removing an edge $(x,y)$ from $G^+$, 
the remaining graph $G'$ is not rigid. 
Clearly $(x,y) \notin F \setminus f$.


Let $v_k$ be any last level vertex in $G$ which depends on $(x,y)$, 
namely $v_k$ cannot be constructed from $f$
without constructing both $x$ and $y$ (here $v_k$ can be any last level vertex if $(x,y) = f$).
We have the following cases:


\noindent (1) The graph $G^+$ contains the edge $(v_0,v_k)$ (or $(v_0',v_k)$). 

Starting from $G \cup f$, 
we remove all last level construction steps, till
$v_k$ becomes the only last level vertex in the remaining graph $P$
(other than possibly $v_0$ and $v_0'$).
Let $P^+ = P \cup (v_k,v_0)$. 

$P$ is a 1-path tree-decomposable graph with base edge $f$. 
By Lemma \ref{lem:globally_rigid_1path}, $P^+ \setminus (x,y)$ is rigid. 
So $G^+ \setminus (x,y)$ is also rigid, since adding back the construction steps preserves rigidity. 


\noindent (2) The graph $G^+$ does not contain the edge $(v_0,v_k)$ or $(v_0',v_k)$. 
So there must be another last level vertex $v_j$ of $G$, such that $G^+$ contains the edge $(v_j,v_k)$.

Starting from $G \cup f$, 
we remove all last level construction steps, till
$v_k$ and $v_j$ become the only last level vertices in the remaining graph $P$
(other than possibly $v_0$ and $v_0'$). 
Let $P^+ = P \cup (v_j,v_k)$. 
We can use an argument similar to the proof of Lemma \ref{lem:globally_rigid_1path} to 
prove the rigidity of $P^+$.
\end{proof}


\end{document}
