\documentclass[times, doublespace]{rncauth}
\usepackage{moreverb}
\usepackage[dvips,colorlinks,bookmarksopen,bookmarksnumbered,citecolor=red,urlcolor=red]{hyperref}
\newcommand\BibTeX{{\rmfamily B\kern-.05em \textsc{i\kern-.025em b}\kern-.08em
		T\kern-.1667em\lower.7ex\hbox{E}\kern-.125emX}}
\usepackage{amsmath, enumerate, amsfonts, amssymb, flushend, empheq, cite, amsthm}
\newcommand\ignore[1]{{}}
\newcommand{\iqc}{QC}
\newcommand{\hb}[1]{\hat{\bar{#1}}}
\newcommand{\rank}{\mathrm{rank}}
\newcommand{\intinf}{\int_{-\infty}^{\infty}}
\newcommand{\intinfb}{\int_{t-\beta}^{t+\beta}}
\newcommand{\sgn}{\mathrm{sgn}}
\usepackage{graphicx}
\def\volumeyear{2016}

\newtheorem{theorem}{Theorem}
\newtheorem{prop}{Proposition}
\newtheorem{lemma}{Lemma}
\newtheorem{example}{Example}
\newtheorem{corollary}{Corollary}
\newtheorem{definition}{Definition}
\newtheorem{assumption}{Assumption}
\newtheorem{remark}{Remark}
\begin{document}
\runningheads{A. Chakrabarty, G. T. Buzzard, S. H. \.Zak, F. Zhu, A. E. Rundell}{State and Exogenous Input Reconstruction using BL-SMO}

\title{Simultaneous state and exogenous input estimation for nonlinear systems using boundary-layer sliding mode observers}

\author{Ankush Chakrabarty\affil{1}\corrauth,
	Gregery T. Buzzard\affil{2}, Stanis\l aw H. \.Zak\affil{1},\\ Fanglai Zhu\affil{3}, Ann E. Rundell\affil{4}}

\address{\affilnum{1}School of Electrical and Computer Engineering, Purdue University, West Lafayette, IN, USA\break \affilnum{2}Department of Mathematics, Purdue University, West Lafayette, IN, USA\break \affilnum{4} Weldon School of Biomedical Engineering at Purdue University, West Lafayette, IN, USA\break
\affilnum{3} College of Electronics and Information Engineering, Tongji University, Shanghai, P. R. China
}

\corraddr{465 Northwestern Avenue, School of Electrical and Computer Engineering, Purdue University, West Lafayette, IN, USA. E-mail: \texttt{chakraa@purdue.edu}. Alternate E-mail: \texttt{chak.ankush@gmail.com}}

\begin{abstract}
While sliding mode observers (SMOs) using discontinuous relays are widely analyzed, most SMOs are implemented computationally using a continuous approximation of the discontinuous relays. This approximation results in the formation of a boundary layer in a neighborhood of the sliding manifold in the observer error space. Therefore, it becomes necessary to develop methods for attenuating the effect of the boundary layer and guaranteeing performance bounds on the resulting state estimation error. In this paper, a method is proposed for constructing boundary-layer SMOs (BL-SMOs) with prescribed state estimation error bounds. The BL-SMO formulation is then extended to simultaneously estimate exogenous inputs (disturbance signals in the state and output vector fields), along with the system state. Two numerical examples are presented to illustrate the effectiveness of the proposed approach.
\end{abstract}

\keywords{Unknown input observers, low-pass filtering, incremental quadratic constraints, sliding mode, descriptor systems, multiplier matrix, linear matrix inequalities}
\maketitle

\section{Introduction}\label{sec:intro}
Estimation of system states and \textit{exogenous} inputs (disturbance inputs in the state and output vector fields) for nonlinear systems is a critical problem in many applications. These applications include: estimating actuator faults in mechanical systems, unmodeled disturbances in biomedical systems, and attacks in the measurement channels of cyberphysical systems~\cite{Chen1999, Chakrabarty2014, Pasqualetti2013}. The presence of exogenous inputs generally degrade closed-loop system performance. Therefore, it is imperative to design observers that simultaneously estimate state and exogenous inputs for implementation of high-performance closed-loop control systems.

The application of sliding modes~\cite{Utkin1977, DeCarlo1988, Kachroo1996, Rundell1998, DeCarlo2000, Utkin2009} to state and unknown input observer design has been widely developed in the context of linear systems. In~\cite{Edwards2000} and~\cite{Tan2002}, an equivalent output error injection term is proposed to recover the state and measurement disturbance signals. Linear matrix inequalities for the construction of the observer gains and the reconstruction of the state disturbances are proposed in~\cite{zak05, Kalsi2010, Kalsi2011,Witczak2015} for linear systems. An extension to Lipschitz nonlinear systems has been proposed in~\cite{Tan2003, Ha2004, Yan2007, Raoufi2010, Raoufi2010b, Teh2013, Veluvolu2011, Veluvolu2014}, and one-sided Lipschitz nonlinear systems in~\cite{Li2014}. These observers are only applicable for a limited class of nonlinear systems and furthermore, most of the above papers deal with the estimation of states and unknown inputs acting in the state vector field, in the absence of measurement disturbances.

Descriptor systems provide an attractive approach for simultaneous estimation of the states and exogenous disturbances~\cite{Darouach1995, Hou1999}. Sliding mode observer based on descriptor systems is proposed in~\cite{Gao2006}. Some recent papers~\cite{Zhu2012, Zhu2014} also discuss reconstruction of the unknown signals using second-order sliding modes. However, the classes of nonlinearities considered in the current literature are restricted to Lipschitz or quasi-Lipschitz nonlinearities, which may introduce conservativeness in the design. Additionally, the SMOs are formulated with a discontinuous injection term, but are implemented with a continuous (generally sigmoidal) injection term. This is because the system response with the discontinuous injection term is computationally taxing, and difficult to implement in practice.

We ameliorate some of these open problems in the present paper. Our \textbf{contributions} include the following: (i) we propose a method to simultaneously reconstruct the system state and exogenous inputs (both disturbances in the state and output vector fields); (ii) we provide ultimate bounds on the observer estimation error based off a tractable continuous approximation of the discontinuous relay term (this is sometimes called a `boundary-layer', see for example~\cite{Chouinard1985, Barmish1983, Corless1981, Zak_txtbk}); (iii) we extend current SMO formulations to a wider class of nonlinearities using incremental multiplier matrices, proposed in~\cite{iqs_corless}; and, (iv) we demonstrate the utility of smooth window functions in recovering the exogenous inputs to within a prescribed accuracy.

The rest of the paper is organized as follows. In Section~\ref{sec:notation}, we provide our notation. In Section~\ref{sec:prob}, we define the class of nonlinear systems considered and formally state the objective of this paper. Subsequently, an observer architecture is presented and sufficient conditions are provided which, if satisfied, specify performance bounds on the observation error of the plant state and unknown output disturbance signal. In Section~\ref{sec:lpf}, we leverage smooth window functions to reconstruct the unknown state disturbance signal to within a prescribed accuracy. In Section~\ref{sec:ex}, we test our proposed observer scheme on two numerical examples, and offer conclusions in Section~\ref{sec:conc}.

\ignore{}
\section{Notation}\label{sec:notation}
We denote by  the set of real numbers, and  denotes the set of natural numbers. Let . For a function , we denote  the space of -times differentiable functions. The function  if  and  if .
For every , we denote , where  is the transpose of . The sup-norm or -norm is defined as . We denote by  the smallest eigenvalue of a square matrix . The symbol  indicates positive (negative) definiteness and  implies  for  of appropriate dimensions. Similarly,  implies positive (negative) semi-definiteness. The operator norm is denoted  and is defined as the maximum singular value of . For a symmetric matrix, we use the  notation to imply symmetric terms, that is,
 
For Lebesgue integrable functions , we use the symbol  to denote the convolution operator, that is,


\section{Problem Statement and Proposed Solution}\label{sec:prob}
We begin by describing the class of systems considered in the paper.  
\subsection{Plant model and problem statement}
We consider a nonlinear plant modeled by

	\dot x(t) &= Ax(t)+B_f f(t,u,y,q) + B_g g(t,u,y) +  G w_x(t),\label{eq:sys_nom_a}\\
	y(t) &= C x(t) + D w_y(t).
	
Here,  is the state vector,  is the control action vector,  is the vector of measured outputs. The nonlinear function  models nonlinearities in the system \textit{whose arguments are available} at each time instant . 


Let the function  denote the system nonlinearities \textit{whose argument  has to be estimated}, where  and . 

The signal  is the \textbf{unknown state disturbance}, for example: unmodeled dynamics, actuator faults or attack vectors. The signal  models \textbf{unknown measurement/sensor disturbances}, for example: cyber-attacks on the measurement channel or sensor faults. We refer to the vectors  and  as the \textbf{exogenous input}. The matrices , , , ,  and  are of appropriate dimensions. 

To proceed, we make the following assumptions.
\begin{assumption}\label{ass:local_lipz}
The right-hand-side of~\eqref{eq:sys_nom_a} is locally Lipschitz. 
\end{assumption}
\begin{assumption}\label{asmp:ranks}
The matrices  and  have full column rank, that is,  and . 
\end{assumption}
\begin{assumption}\label{asmp:faults_bdd}
The state disturbance input  is Lebesgue integrable.
\end{assumption}

Finally, we make an assumption on the classes of nonlinearities considered in this paper. To this end, we need the following definition.
\begin{definition}[Incremental Multiplier Matrix]
A matrix  is an \textbf{incremental multiplier matrix} if it satisfies an \textbf{incremental quadratic constraint}~(\iqc)
	
	
		\delta q \triangleq q_1 - q_2 \in\mathbb{R}^{n_q}
		
		\delta f \triangleq f(t, u, y, q_1) - f(t, u, y, q_2) \in\mathbb{R}^{n_f}
		
	for all .
\end{definition}
To illustrate the concept of the incremental quadratic constraint, we provide the following examples.
\begin{example}
Consider the nonlinearity . Since

we have 

that is,

Hence, an incremental multiplier matrix for  is

for any .
\end{example}
\begin{example}
Consider the nonlinearity , which is not globally Lipschitz. The nonlinearity  satisfies the inequality

for any . This 
can be rewritten as

  Hence, an~incremental multiplier matrix~for  is 

for any .
\end{example}
\begin{remark}
Clearly, if a nonlinearity has a non-zero  incremental multiplier matrix, it  is not unique.
Any positive scalar multiplier of an incremental multiplier matrix is also an~incremental multiplier matrix.
\end{remark}
The class of nonlinearities satisfying~\iqc~contains a wide class of nonlinearities, including globally- and one-sided Lipschitz nonlinearities, incrementally sector-bounded nonlinearities, incrementally positively real nonlinearities, and nonlinearities with derivatives lying in cones or polytopes. For more details regarding the construction of the incremental multiplier matrix for different categories of nonlinearities, we refer the reader to the Appendix. For the more interested reader, we refer to~\cite{iqs_corless,acikmese11obs}.
\begin{figure}[!ht]
	\centering
	\includegraphics[width=0.65\columnwidth]{overview.eps}
	\caption{Overview of proposed state and exogenous input estimation scheme. The \textbf{unknown} exogenous inputs  acting on the nonlinear plant are shown in red. The estimated state  and estimated exogenous inputs  are shown in blue.}
	\label{fig:overview}
\end{figure}

\subsection{Objective and overview of the proposed solution}
Our \textbf{objective} is to design a boundary-layer sliding mode observer (BL-SMO) for the nonlinear system~\eqref{eq:sys_nom} that can simultaneously reconstruct the state , and the unknown exogenous inputs . We first rewrite the nonlinear plant~\eqref{eq:sys_nom} as a generalized descriptor system whose state is the augmented state . Subsequently, we use the descriptor system as a platform to design a BL-SMO and guarantee ultimate bounds on the estimation error of the augmented state, that is, the estimation error of  and . We conclude this section by demonstrating that the BL-SMO error dynamics converge to the boundary-layer sliding manifold in finite time, which will be used in the sequel to reconstruct the state disturbance input .
For convenience, an overview of the scheme is provided in Figure~\ref{fig:overview}.
\subsection{Generalized descriptor formulation}
Let

be an augmented state vector. Also let 
Then we can represent the nonlinear plant~\eqref{eq:sys_nom} as a descriptor system

	\label{eq:descr_sys_a}
	\bar E\dot{\bar x}(t) &= \bar A \bar x(t)+ B_f f(t,u,y, C_q\bar E\bar x) + B_g g(t,u,y) + G w_x(t),\\
	y(t) &= \bar C \bar x(t).
	
We illustrate this with an example.
\begin{example}
Suppose we have the nonlinear system

Clearly, we can write this in the form~\eqref{eq:sys_nom} with

Thus, the descriptor system can be written as
\qed
\end{example}
\begin{remark}
Note that the descriptor system~\eqref{eq:descr_sys} is constructed for developing the observer. It is not computationally implemented. For simulations, we use the original nonlinear plant~\eqref{eq:sys_nom}.
\end{remark}
To proceed, we require the following technical result.
\begin{lemma}\label{prop:T1_T2}
	Suppose the number of measured outputs is greater than or equal to the number of sensor disturbances; that is, . Then there exist two matrices  and  such that
	
\end{lemma}
\begin{proof}
	Let  and  
	
	Computing  reduces to solving the linear equation . By Assumption~\ref{asmp:ranks}, we know that  has full column rank, which implies  has full column rank. Hence, a left inverse of  exists. We denote  as a left inverse of , that is, . Clearly,  is a solution to .
	
	Therefore,  can be computed by taking the first  columns of  and  is the matrix constructed using the last  columns of . This concludes the proof.
\end{proof}

\begin{remark}
	A particular choice of such a left inverse is the Moore-Penrose pseudoinverse, that is, 
	\qed	\end{remark}
\subsection{Proposed BL-SMO}
Let  We propose the following \textbf{boundary layer sliding mode observer} architecture to estimate the plant states  and the exogeneous inputs  and :

	\dot z &= Q z + (L_1 - Q T_2) y + T_1 B_g g + T_1 B_f \hat f + T_1 G \hat w_x^\eta\\
		\label{eq:obs_b}
	\hb x &= z - T_2 y\\
	\label{eq:obs_c}
	\hat f &= f(t, u, y, C_q\bar E\hb x + L_2 e_y)\\
	\label{eq:obs_d}\hat w^\eta_x &= \begin{cases}
	\rho\;Fe_y/\|Fe_y\| & \text{if }\\
	\rho \; Fe_y/\eta & \text{if },
	\end{cases}
	
where  is an available (measured) output,  is a~\textbf{continuous injection term} for the sliding mode observer parametrized by the \textbf{smoothing coefficent}  and 

The signal  will be used in the sequel to recover the state disturbance input . 


The observer is parameterized by four gain terms: (i) the linear gain , (ii) the innovation term  which improves the estimate of the known nonlinearity  by adding a degree of freedom in the design methodology, (iii) the matrix  and, (iv) the scalar . 
\begin{remark}\label{rk:abs_cont_e}
	Assumption~\ref{ass:local_lipz} implies that the observer ODEs also have unique classical solutions as , and hence, the functions  are absolutely continuous.
\qed	\end{remark}
\subsection{Derivation of error dynamics}
We investigate the error dynamics of the proposed observer. To this end, we first require the following result which is easily proven by verification.
\begin{lemma}\label{prop:Q_zero}
	Let  and , where  are constructed as described in Lemma~\ref{prop:T1_T2}. Then \qed
\end{lemma}

We define the observer error to be  Using~\eqref{eq:T1_T2} and~\eqref{eq:obs}, the observer \textbf{error dynamics} are given by

Using Lemma~\ref{prop:Q_zero} yields

Our \textbf{objective} is to design the observer gains ,  and  to ensure that the error dynamical system~\eqref{eq:err_dyn} is ultimately bounded and the effect of the unknown input  is attenuated.

\subsection{Ultimate boundedness of observer error dynamics}
In order to investigate the stability properties of the observer error~\eqref{eq:err_dyn}, we need the following technical lemma.
\begin{lemma}\label{lem:1}
	Suppose  is an incremental multiplier matrix (see Definition~1) for the nonlinearity  and let  where  is defined in~\eqref{eq:obs_c}. Then the condition
	
	holds for any .
\end{lemma}
\begin{proof}
	Recall that . From~\eqref{eq:descr_sys_a} and~\eqref{eq:obs_c}, we have 
	
	Let , , , and . 
	Hence, we obtain .
	Now, we can write
	
	Recalling that the matrix  is an incremental multiplier matrix of , and substituting~\eqref{eq:temp1} into the incremental quadratic constraint~\eqref{eq:iqc}, we obtain the desired matrix inequality.
\end{proof}

Herein, we present sufficient conditions in the form of matrix inequalities for the observer design.
\begin{theorem}\label{thm:obs_design}
Let  and . Suppose there exist matrices , , , , an incremental multiplier matrix  for the nonlinearity , and scalars , which satisfy
	
		\Xi + \Phi^\top M \Phi &\preceq 0,\label{eq:thm1_a}\\
		G^\top T_1^\top P &= F \bar C,\label{eq:thm1_b}\\
		\begin{bmatrix} P & I \\ I & \mu \end{bmatrix} &\succeq 0\label{eq:thm1_c}\\
		\rho &\ge \rho_{x},
		
	where 
	
	then the error trajectories of the BL-SMO~\eqref{eq:obs} with gains , , , and  satisfies
	
	where  is the smoothing coefficient of the continuous injection term .
\end{theorem}

\begin{proof}
With , we get

Now, we consider a quadratic function of the form  Herein, for readability, we omit the argument of . Then, the time derivative of  evaluated on the trajectories of the error dynamical system~\eqref{eq:err_dyn} is given by
	
	
	Let . Then from~\eqref{eq:thm1_a}, we get
	
	From Lemma~\ref{lem:1}, we know that . Hence,
	
	For error states satisfying , we have
	
	Hence, recalling the definition of  from~\eqref{eq:obs_d} and condition~\eqref{eq:thm1_b}, we get
	
	By choosing , we can ensure 
	 which implies global exponential stability of the observer error  to the set  with decay rate ; see for example,~\cite{Barmish1983, Corless1993}, for global exponential stability to a set.
	
	Now consider error states that satisfy . Then, from~\eqref{pf:1_a}, we obtain
	
Using~\eqref{eq:thm1_b} yields
	 
	Recalling that , we obtain
	
	Summarizing, we write
	
	The above implies that for any , the inequality~\eqref{pf1:b} holds.
	
	Note that taking Schur complements of~\eqref{eq:thm1_c} yields . Hence . We use this inequality and the Bellman-Gr\"onwall inequality on~\eqref{pf1:b}. This yields
	
	Taking the limit superior concludes the proof.
\end{proof}
\begin{remark}
With  and  fixed, the conditions in Theorem~\ref{thm:obs_design} devolve into a convex programming problem in , , , ,  and . Additionally, solving the convex problem  subject to the constraints~\eqref{eq:thm1} with fixed  and  results in tighter ultimate bounds on .
\end{remark}
Methods for converting the matrix inequality~\eqref{eq:thm1_a} into LMIs without pre-fixing  are provided in~\cite{acikmese11obs}.
	
\begin{remark}
Suppose the conditions of Theorem~\ref{thm:obs_design} are satisfied. Let  and . 
\limsup_{t\to\infty}\|x(t) - \hat x(t)\|\le \sqrt{\mu\eta\rho_x/\alpha},

\limsup_{t\to\infty} \|w_y(t) - \hat w_y(t)\| \le \sqrt{\mu\eta\rho_x/\alpha}.

For a fixed ,  and , we can minimize  over the space of feasible solutions. This attenuates the effect of , thereby producing more accurate estimates of the state and measurement disturbance/sensor attack vectors.
\qed\end{remark}

\begin{remark}
	Note that  as , which implies that under ideal sliding () the matched disturbance can be completely rejected, and exact estimates of the plant state  and output disturbance  can be obtained.
	\qed\end{remark}


Summarizing, we have discussed a method to obtain estimates of the state  and measurement disturbance  to a specified degree of accuracy. However, certain applications such as fault detection~\cite{yan07} and attack detection~\cite{Teixeira2010,Pasqualetti2013,Mo2014}, require the estimation of the unknown state disturbance input . The following subsection provides a crucial ingredient for the simultaneous recovery of  along with  and .
\subsection{Finite time convergence to the boundary-layer sliding manifold}
We will demonstrate that the trajectories of the error system~\eqref{eq:err_dyn} are driven to the boundary layer sliding manifold in finite time. We begin with the following assumption on the plant states.
\begin{assumption}\label{asmp:plant_bdd}
The state vector  and sensor disturbance  of the nonlinear plant~\eqref{eq:sys_nom} are bounded, and known.
\end{assumption}
The boundedness of plant states is reasonable for any practical system. We believe the restriction placed on the output disturbance is also not conservative, as measurement channels will transmit bounded signals, and attack vectors will be of finite magnitude.

Herein, for brevity, we consider  to be a nonlinear function with the single argument . We use the following definition from~\cite[p. 406]{Aronszajn1956} to proceed with the development of technical results in this section. 
\begin{definition}[Minimal Modulus of Continuity]
	The minimal modulus of continuity for any nonlinearity  is given by
	 
	for all .
\end{definition}
\begin{remark}\label{rk:mod_of_cont_dec}
	An important property of the modulus of continuity is that it is a non-decreasing function, that is, if  then . This follows from the definition of the supremum.\qed
	\end{remark}
We also pose a restriction on the class of nonlinearities considered.
\begin{assumption}\label{ass:uc_r_to_zero}
	The nonlinearity  considered in the plant~\eqref{eq:sys_nom} is uniformly continuous on .
\end{assumption}

\begin{remark}\label{rk:unif_cont}
	Most nonlinearities encountered in practical applications adhere to Assumption~\ref{ass:uc_r_to_zero}.
	For example, if a function  is continuously differentiable with bounded derivative, H\"older continuous with exponent , or globally Lipschitz continuous (which is a very common assumption in the literature), then  is also uniformly continuous. Hence, our assumption is not restrictive.\qed
	\end{remark}
We now present the following technical result.
\begin{lemma}\label{lem:4}
	If  is uniformly continuous on  then  as .
\end{lemma}
\begin{proof}
	Let . By uniform continuity, there exists  such that  forces . This implies  which, in turn, implies that . Since  is non-decreasing, we have  for all , which concludes the proof.
\end{proof}


We are now ready to state and prove the following theorem which provides conditions for the observer error trajectories to converge to the boundary-layer sliding manifold

in finite time.
\begin{theorem}\label{thm2}
	Let 
	 
	, , and
	 Suppose Assumptions \ref{ass:local_lipz}--\ref{ass:uc_r_to_zero} hold. If there exists matrices , , ,  and scalars  which satisfy the conditions~\eqref{eq:thm1}. If  is chosen to satisfy
	
	then the BL-SMO~\eqref{eq:obs} with gains , ,  and  generates error trajectories  that converge to  in finite time.
\end{theorem}
\begin{proof}
	If , we are done. Hence, for the remainder of this proof, we consider error trajectories satisfying . It is enough to show that  for some  in order to prove finite-time convergence to , as argued in~\cite{Slotine1991}. To this end,
	
	from~\eqref{eq:thm1_b}. From~\cite{Zhu2014}, we know that for~\eqref{eq:thm1_b} to have a solution,  must be full column rank. Hence 
	 
	since .
	
	Recalling that  is the minimal eigenvalue of the symmetric positive definite matrix , we get
	
	We claim that for every , we can select  large enough to ensure . To prove our claim, we first demonstrate that
	
	Using the triangle inequality, we have
	
	since  and  are bounded by Assumptions~\ref{asmp:faults_bdd} and~\ref{asmp:plant_bdd} and by~\eqref{thm1:E}. 
From~\eqref{rk:exp_decay}, we also know that  decays exponentially when . This implies that  is bounded and decreasing with increasing . By Remark~\ref{rk:mod_of_cont_dec}, this implies that  also decreases with increasing , since  is bounded. Hence  is bounded. As all the terms are bounded, thereby finite, the condition~\eqref{eq:pf3a} holds and the gain~ selected using~\eqref{eq:thm2} is well-defined.
\end{proof}
\section{Recovering the state disturbance using smooth window functions}
\label{sec:lpf}
In this section, we discuss a filtering method to reconstruct the state disturbance . Specifically, we will show that any piecewise uniformly continuous state disturbance input can be reconstructed to prescribed accuracy by filtering the injection term~\eqref{eq:obs_d} of the BL-SMO with a smooth window function.
 
To begin, we present our notion of a smooth window function.
\begin{definition}[Smooth Window Function]
\label{def:swf}
	A \textbf{smooth window} function  satisfies the following conditions:
	\begin{enumerate}[(i)]
		\item , that is,  is smooth;
		\item  for  and  elsewhere;
		\item .
	\end{enumerate}
\end{definition}
In the following proposition, we demonstrate that a uniformly continuous function can be approximated arbitrarily closely by filtering with smooth window functions.
\begin{prop}
	\label{lem:trig_poly}
	Let  be a smooth window function, and define
	
	Also, suppose  and  is uniformly continuous. Then for every , there exists a  such that
	
	for every . Here, `' denotes the convolution operator.
\end{prop}
\begin{proof}
	We begin by noting that conditions (i)--(iii) in Definition~\ref{def:swf} imply that the function  is non-negative on the compact support , and,
	
	Using the definition of convolution, we have
	
	Applying~\eqref{pf:prop_a} to the above yields
	
	Since by definition,  is uniformly continuous on , for every  there exists a  such that for any  satisfying , we obtain .
	
	Therefore, we get the estimate
	
	since  is non-negative. Applying~\eqref{pf:prop_a} on the above, we get~\eqref{eq:acc_uio}. This concludes the proof.
\end{proof}

We wish to extend the result in Proposition~\ref{lem:trig_poly} to a more general class of functions, namely piecewise uniformly continous function, defined next.
\begin{definition}[Piecewise uniformly continuous]
	\label{def:pw_cont}
	A signal  is \textbf{piecewise} (or sectionally) \textbf{uniformly continuous} if
	\begin{enumerate}[(i)]
		\item the signal  exhibits finite (in magnitude) jump discontinuities at abscissae of discontinuity denoted  where . Specifically,  is the set of integers  satisfying  for some , where  may be  and  may be , and  whenever .
		\item there exists a scalar  such that 
		for every ;
		\item the signal  is uniformly continuous on the closure of each open interval in  and this uniformity is independent of the interval.  More formally, for every  there exists  such that if  are in  satisfying  and such that there is no  with , then .
	\end{enumerate}
\end{definition}
\begin{assumption}\label{ass:unif_cont_uio}
	The state disturbance input  is piecewise uniformly continuous.
\end{assumption}
\begin{remark}
	Assumption~\ref{ass:unif_cont_uio} ensures that unknown input signals do not exhibit Zeno behavior (infinite number of jumps in finite time intervals), which is a reasonable assumption for state disturbances such as actuator faults or unmodeled inputs in physiological systems.\qed
	\end{remark}

We are now ready to extend Proposition~\ref{lem:trig_poly} to piecewise uniformly continuous .

\begin{prop}\label{prop:filt_pw_cont}
	Let  satisfy Assumption~\ref{ass:unif_cont_uio} and . Let the function  be defined as in~\eqref{eq:h_beta} and let 
	denote the union of closed neighborhoods around each abscissa of discontinuity of .  Then for every , there exists a  such that
	
	As before, `' denotes the convolution operator and  is the  norm.
\end{prop}

\begin{proof}
	We begin by noting that the existence of the constant  in Definition~\ref{def:pw_cont} implies that  is a set of Lebesgue measure zero. Hence, the convolution integrals over  are well-defined.
	
	First, fix  and recall the definition of  in~\eqref{eq:h_beta}. Since the function  has compact support , the convolution integral is evaluated over the window of length . Thus, we can directly apply Proposition~\ref{lem:trig_poly} with  to obtain~\eqref{eq:acc_uio} for any interval in . By (iii) in Definition~\ref{def:pw_cont}, we can select  independent of . From this, we conclude
	
	for .
	
	However, the same cannot be said for the points  because the function  is not uniformly continuous across the point of discontinuity. Since , we know that the function jumps just once in the interval . Then we write
	
	since  by construction. This concludes the proof.
\end{proof}
Proposition~\ref{prop:filt_pw_cont} implies that for piecewise uniformly continuous , the filtering approach using smooth windows leads to high accuracy reconstructions in all but neighborhoods of the points of jump discontinuity.

The following theorem is the main result of this section. It is an extension of a low-pass filtering method proposed in~\cite{Hui2013}, which was for linear systems with sliding manifolds of co-dimension one. 

\begin{theorem}\label{thm:act_fault_unstable}
	Suppose Assumptions 1--\ref{ass:unif_cont_uio} hold, and there exists a feasible solution  satisfying the conditions~\eqref{eq:thm1} in Theorem~\ref{thm:obs_design}. Let  be selected as in~\eqref{eq:thm2} and  be defined as in~\eqref{eq:I_beta}. 
	Then for a given , there exist scalars , a sufficiently large , a sufficiently small  and a low-pass filter 
	
	such that
	
	for all .
\end{theorem}
\begin{proof}
	We begin by fixing  and choosing  for  such that  satisfies~\eqref{eq:prop1} for the th component of . We define
	
	where  denotes the  norm.
	
	Let , , and , with  denoting the operator norm. We note that for a given 's and , we can choose  sufficiently small to ensure that
	
	where 
	as defined in~\eqref{eq:lambda_1}. Recall that , as . Note that by construction , which implies
	
	Let , and  be the time at which the error trajectories enter the boundary layer sliding manifold. We know that  as  satisfies~\eqref{eq:thm2} in Theorem~\ref{thm2}. Furthermore, we know that  is an absolutely continuous function (see Remark~\ref{rk:abs_cont_e}). 
	
	Therefore, we can apply integration by parts for  and use the compact support and smoothness of  to obtain
	
	which implies
	
	Let  be defined as in~\eqref{eq:f_minus_fhat}. Replacing the error-derivative  using~\eqref{eq:err_dyn} gives
	
	
	We now rewrite the last term in~\eqref{eq:pf2e} as
	
From Theorem~\ref{thm:obs_design}, we know that 
	 and from Theorem~\ref{thm2} we get  for . Thus, for a given  and  chosen as in~\eqref{eq:chi}, there exists a sufficiently small , and sufficiently large  for which
	
	for all . 
	
	The inequality~\eqref{eq:e_sup} along with Lemma~\ref{lem:4} implies
	
	
	Therefore, using~\eqref{pf:prop_a} and~\eqref{eq:pf2d}, we upper bound the right hand side terms in~\eqref{eq:pf2f}. That is,
	
		\nonumber\left\|\intinf \dot h_\beta(t-\tau)S\bar e(\tau)\, d\tau\right\| &\le \|\mathcal H_\beta\| \sup_{\tau\in[t-\beta, t+\beta]}\|S\bar e(\tau)\|\\
		&\le \chi_1 \varepsilon_1,\\
		\nonumber\left\|\intinf h_\beta(t-\tau)ST_1 B_f\left(f(q(\tau))- f(\hat q(\tau))\right)\,d\tau\right\|&\le \sup_{\tau\in[t-\beta, t+\beta]}\|ST_1 B_f\|\|f(q(\tau))-f(\hat q(\tau))\|\\
		&\le \chi_3 \gamma_f(\|C_q\bar E - L_2\bar C\|\varepsilon_1),\\
		\nonumber\left\|\intinf h_\beta(t-\tau) S(T_1 \bar A -L_1\bar C)\bar e(\tau)\,d\tau\right\|&\le \|S(T_1 \bar A -L_1\bar C)\|\sup_{\tau\in[t-\beta, t+\beta]}\|\bar e(\tau)\|\\
		&\le\chi_2 \varepsilon_1,
		
	for .
	We know that  is symmetric positive definite, and hence, . Therefore,
	
	Applying to the above~\eqref{eq:pf2f} and~\eqref{eq:pf2_g} produces
	
	By construction of  in~\eqref{eq:chi}, we get
	
	
	Recall the definition of  from~\eqref{eq:I_beta}. We now use~\eqref{eq:prop1} in Proposition~\ref{prop:filt_pw_cont} and~\eqref{pf:bd2} to obtain
	
	for  and  sufficiently small. This concludes the proof.
\end{proof}
Theorem~\ref{thm:act_fault_unstable} implies the existence a bank of smooth window filters capable of reconstructing the vector-valued signal  up to arbitrary accuracy in all but neighborhoods of jump discontinuities.

\section{Simulation Results}\label{sec:ex}
In this section, the performance of the proposed observer-based state and exogenous input estimation formalism is tested on two numerical examples. The first example is a practical system, where the nonlinearity is globally Lipschitz continuous and there is one state disturbance and one output disturbance signal. The second example is a randomly generated system (to demonstrate the non-conservativeness of our approach) with multiple exogenous inputs and a non-Lipschitz nonlinearity.
\subsection{Example 1}
We use the single joint flexible robot described in~\cite{Zhu2014} to test our observer design methodology. The nonlinear plant is modeled as in~\eqref{eq:sys_nom} with system matrices


\begin{figure}[!ht]
	\centering
	\includegraphics[width=\columnwidth]{fig1b.eps}
	\caption{Simulation Results. (Top left) The unmeasured variable  is shown in blue, and the dashed red line is the estimated trajectory . We note that the estimate is satisfactorily close to the actual. (Top right) The error  is plotted in blue with the dashed black lines showing the error bound computed to be 0.082. (Bottom Left) The estimate of the state disturbance input  shown after 40~s. Note that the low pass filtered estimate is highly accurate. }
	\label{fig:ex1}
\end{figure}
Here, the nonlinearity  is globally Lipschitz and its argument, , is not measured directly. The function  is known at all  because  is a measured output. The control input is set to zero.
Hence , , ,  and  and . Thus  with . From~\eqref{eq:imm2}, we deduce that this nonlinearity has an incremental multiplier matrix  where .

We select  and . Using CVX~\cite{cvx}, we obtain a feasible solution\footnote{We find that minimizing the norm of , hence , usually enables faster runtimes using MATLAB's \texttt{ode15s} or \texttt{ode23s}.} to the LMIs in~\eqref{eq:thm1}, namely  and , the matrix

the observer gain

and the sliding surface matrix

For simulation purposes, we consider a randomly generated initial condition  and the exogenous inputs are chosen to be  and . Hence, . The observer is initialized at  and the sliding mode gain is set at . Finally, the boundary layer sliding mode injection term  is computed using . From Theorem~\ref{thm:obs_design}, we get the error state bound
 A 9th-order Butterworth low-pass filter with window length ~s is used to obtain the actuator fault signal estimate. The corresponding MATLAB implementation is \verb|butter(9,0.12,'low')|. The simulation results are shown in Figure~\ref{fig:ex1}. We compute the experimental mean squared error  from , which verifies that our reconstruction is highly accurate.
\subsection{Example 2}
To demonstrate that our assumptions are not restrictive, we will test our method on a randomly generated system of the form~\eqref{eq:sys_nom} with multiple unknown inputs and a non-Lipschitz nonlinearity. Here,

, ,

and the nonlinearity is . We set . Since the nonlinearity is globally Lipschitz, we know from~\eqref{eq:imm1} that it is characterized by an incremental multiplier matrix of the form 

for some . We fix ,  and use CVX to obtain , ,


and

\begin{figure*}[!ht]
	\centering
	\includegraphics[width=\columnwidth]{fig2b.eps}
	\caption{Simulation Results. (Top left) The actual (blue) and estimated (red dashed) trajectories of the unmeasured state  are shown. (Top right) Zoom-in of error trajectory and the computed plant state error bound. We note that the bound (black dashed) is not conservative. (Bottom) We illustrate that the exogenous inputs are estimated with high accuracy.}
	\label{fig:ex2}
\end{figure*}
We generate a random initial condition 
 and the exogenous inputs are chosen to be 
 and  respectively. Hence,  and ; therefore, these exogenous inputs are of significant magnitude. The observer is initialized at  and the sliding mode gain is chosen . The continuous injection term  is computed with .
Therefore, from Theorem~\ref{thm:obs_design}, we get the error state bound

Two different smoothing filters with ~s, ~s are used to obtain estimates of the unknown state input . The MATLAB command to implement these smooth window filters is:~\texttt{smooth(injectionTerm, 'loess');}. The simulation results are shown in Figure~\ref{fig:ex2}. We note that although the unknown exogenous inputs are reconstructed with high accuracy, the sawtooth input  exhibits overshoots and undershoots at the points of jump discontinuities, as predicted by Theorem~\ref{thm:act_fault_unstable}.

\section{Conclusions}\label{sec:conc}
We developed a methodology for constructing implementable boundary layer sliding mode observers for a wide class of nonlinear systems using incremental multiplier matrix as a unifying tool for the design procedure. We formulate linear matrix inequalities which, if satisfied, can be used to construct the observer with pre-specified ultimate bounds on the reconstruction error of plant states and unknown output disturbances. We also demonstrate the utility of smooth window functions in recovering the unknown state disturbance signal and provide an upper bound on the exogenous input estimation error for state disturbance inputs exhibiting jump discontinuities, which has not been investigated previously. The proposed methodology has a variety of applications including fault detection and reconstruction for mechanical systems, high confidence estimation in cyberphysical systems and secure communication.

\acks
The authors would like to thank Professor Martin J. Corless of the School of Aeronautics and Astronautics, Purdue University, West Lafayette, for his useful comments and suggestions.
This research was supported by a National Science Foundation (NSF) grant DMS-0900277.
\vspace{1em}
\bibliographystyle{unsrt}
\bibliography{refs}
\appendix
\section{Incremental Multiplier Matrices for Common Nonlinearities}\label{app}
We present systematic methods for the computation of incremental multiplier matrices for a variety of nonlinearities analyzed in this paper and encountered in practical systems. We refer the reader to~\cite[Section 6]{iqs_corless} for a detailed discussion of methods used to compute incremental multiplier matrices and corresponding derivations of these matrices.

We begin by recalling the definition of  and  given in~\eqref{eq:del_qp}.
\subsection{Incrementally sector bounded nonlinearities}
An incrementally sector bounded nonlinearity satisfies the inequality

for some fixed matrices  and for all , where  is a set of matrices.
After representing the nonlinearity in the form~\eqref{eq:inc_sb_nonlin}, the incremental quadratic constraint (IQC) in~\eqref{eq:iqc} is satisfied by choosing

where,


\subsection{Incrementally positively real nonlinearities}
For a class of incrementally positively real nonlinearities, that is, nonlinearities satisfying  the corresponding incremental multiplier matrix is given by

with .

\subsection{Globally Lipschitz nonlinearities}
For a globally Lipschitz nonlinearity that satisfies  for some , we write 
and inequality~\eqref{eq:inc_sb_nonlin} is satisfied by choosing

with .
\subsection{Quasi-Lipschitz nonlinearities}
Another class of nonlinearities considered in this paper is the so-called `one-sided' or `quasi' Lipschitz nonlinearities that satisfy

for some ,  and . An incremental multiplier matrix for this class of nonlinearities is given by

with .
\subsection{Nonlinearities with derivatives residing in a polytope}
Suppose we have a nonlinearity  that satisfies

where  is a polytope with vertex matrices . In other words, 

where , and  satisfies  for all  and . Then a corresponding incremental multiplier matrix

satisfies the matrix inequalities

for all . An example of this class of nonlinearity is , whose derivative is  which lies in a polytope  with vertices

Another example that falls into this category is the Takagi-Sugeno fuzzy model, proposed in~\cite{Sugeno1985}.
\subsection{Nonlinearities with derivatives residing in a cone}
Suppose we have a nonlinearity  that satisfies

where  is a cone with vertex matrices . In other words, 

where , and  satisfies  for all . Then a corresponding incremental multiplier matrix of the form~\eqref{eq:imm_con} satisfies the matrix inequalities

for all . An example of this class of nonlinearity is , whose derivative is  which lies in a cone  with vertices

\end{document}
