\documentclass[12pt,a4paper]{article}
\usepackage{fullpage}
\usepackage{amssymb,amsmath,stmaryrd,amsthm}
\usepackage[mathscr]{eucal}
\usepackage{pstricks,pst-node,pst-text,pst-3d}


\theoremstyle{definition}
\newtheorem{definition}{Definition}
\newtheorem{theorem}{Theorem}
\newtheorem{proposition}{Proposition}
\newtheorem{lemma}{Lemma}
\newtheorem{corollary}{Corollary}
\newtheorem{notation}{Notation}
\newtheorem{property}{Property}
\newtheorem{example}{Example}
\theoremstyle{remark}
\newtheorem{remark}{Remark}

\renewcommand{\labelenumi}{(\roman{enumi})}



\def \RR {\mathbb{R}}
\def \1{{\mathbf{1}}}
\def \0{{\mathbf{0}}}
\def \C {\mathscr{C}}
\def\cA{\mathcal{A}}
\def\cC{\mathcal{C}}
\def\cG{\mathcal{G}}
\def\cM{\mathcal{M}}
\def \NC {\mathscr{NC}}
\def \SNC {\mathscr{SNC}}
\def \S {\mathscr{S}}


\begin{document}


\title{\bf\Large Exact bounds of the M\"obius inverse of monotone set functions}

\author{Michel GRABISCH\thanks{Corresponding author. Paris School of Economics,
          University of Paris I,
         106-112, Bd. de l'H\^opital, 75013 Paris, France. Tel. (33)
        144-07-82-85, Fax (33)-144-07-83-01. Email:
         \texttt{michel.grabisch@univ-paris1.fr}}       \and Pedro
         MIRANDA\thanks{Complutense University of Madrid. Plaza de Ciencias, 3,
           28040 Madrid, Spain Email:
       \texttt{pmiranda@mat.ucm.es}}
}

\date{\today}

\maketitle


\begin{abstract}
We give the exact upper and lower bounds of the M\"obius inverse of monotone and
normalized set functions (a.k.a. normalized capacities) on a finite set of 
elements. We find that the absolute value of the bounds tend to
 when  is large. We establish also the exact bounds of
the interaction transform and Banzhaf interaction transform, as well as the
exact bounds of the M\"obius inverse for the subfamilies of -additive normalized
capacities and -symmetric normalized capacities.
\end{abstract}


\noindent {\bf Keywords:} M\"obius inverse, monotone set function, interaction

\noindent AMS Classification: 05, 06, 91

\section{Introduction}
The M\"obius function is a well-known tool in combinatorics and partially
ordered sets (see, e.g., \cite{aig79,lin97,rot64}). In the field of
decision theory, the M\"obius inverse of a monotone set function (called a
capacity) is a fundamental concept permitting to derive simple expressions of
nonadditive integrals and to analyze the core of capacities (set of probability
measures dominating a capacity) \cite{chja89}. Set functions can also be seen as
pseudo-Boolean functions, and it is well known that the M\"obius inverse
corresponds to the coefficients of the polynomial representation of a
pseudo-Boolean function.  In particular, monotone and normalized
  pseudo-Boolean functions correspond to semicoherent structure functions in
  reliability theory (see, e.g., Marichal and Mathonet \cite{mama13}, Marichal
  \cite{mar14}). 

Consider  and a monotone set function
 with the property  and
 (normalized capacity).  In optimization problems involving capacities
or monotone pseudo-Boolean functions (as in reliability) it is often
useful to know the bounds of the M\"obius inverse to use algorithmic
  methods (see Crama and Hammer \cite{crha11}, Chapter~13). This is the case
for example when dealing with -additive measures, which are best represented
through their M\"obius inverse (see below); then, when solving optimization
problems like model fitting, algorithms usually need to fix an interval where
the searched values lay, and the upper and lower bounds are the natural limits
of these intervals. Surprisingly, although  takes values in , the
exact bounds of its M\"obius inverse grow rapidly with , approximately in
 when  is large. The aim of the paper is to
establish this result, correcting wrong bounds obtained in a previous paper by
the authors \cite{migr99a}, and providing a complete proof of the result. We
extend this result to the interaction transform, another useful linear
invertible transform of set functions, and we consider also specific subclasses
of capacities, like -additive and -symmetric capacities.

\section{Preliminaries}
Let . A \textit{capacity} on  is a set function
 satisfying  and monotonicity:
 implies . A capacity is
\textit{normalized} if in addition . We denote respectively by 
and  the set of capacities and normalized capacities on . The set
 is a convex closed polytope, whose extreme points are all
-valued normalized capacities (as the polytope of normalized capacities
is an order polytope, this result has been shown by Stanley \cite{sta86}. For a
direct proof, see \cite{rad98}). We denote by
 the set of all -valued normalized capacities.

Consider a set function  on  such that . The
\textit{monotonic cover} of  is the smallest capacity  such that
. We denote it by , and it is given by




Consider now a set function . The linear system

has always a unique solution, known as the \textit{M\"obius inverse}
\cite{rot64}, and is given by

Since  is also a set function, we view now the M\"obius inverse as a transform on
the set of set functions:

We call  the \textit{M\"obius transform} of . Remark that it is a linear invertible
transform.

We introduce another linear invertible transform, which is useful in decision
making, called the \textit{interaction transform}. To this end we introduce the
derivative of a set function . Let  and . The
\textit{derivative} of  w.r.t.  at  is defined by . Derivatives w.r.t. sets are defined recursively by

with , , and
. For , we obtain

 Also, observe that 

The interaction transform  computes a weighted
average of the derivatives:

where . Its expression through the M\"obius transform is much
simpler:

while the inverse relation uses the Bernoulli numbers :

(see \cite{degr96,grmaro99a} for details). Another related transform is the
\textit{Banzhaf interaction transform} , which is the
(unweighted) average of the derivatives:



\medskip

Lastly, we introduce two specific families of normalized capacities. A
normalized capacity  is said to be \textit{at most -additive}
() if  for every set  such that
 \cite{gra96f}. 1-additive capacities are ordinary additive capacities,
i.e., satisfying  for disjoint sets . Note that
by (\ref{eq:4}),  can be replaced by  in the above definition.

We denote by  the set of at most -additive capacities
on . It is a convex closed polytope (see \cite{micogi06} for a study of its properties).

Another family of interest is the family of -symmetric capacities
\cite{migrgi02}. A capacity  is \textit{symmetric} if 
whenever .  We denote by  the set of symmetric normalized
capacities. This notion can be generalized as follows. A nonempty subset
 is a \textit{subset of indifference} for  if for all
 with , we have 
for every . The \textit{basis} of the capacity is the
coarsest partition of  into subsets of indifference. It always exists and is
unique \cite{migr03a}. Now,  is \textit{-symmetric} with respect to the
partition  if this partition is its basis. Symmetric
games are therefore 1-symmetric games (with respect to the basis ). We
denote by  the set of normalized
capacities such that  are subsets of indifference. It is a
convex closed polytope (again, see \cite{micogi06} for a study of its properties).

Lastly, we mention a combinatorial result on the binomial coefficients:

for any positive integer .

\section{Exact bounds of the M\"obius inverse}
We present in this section the main result of the paper.

\begin{theorem}\label{th:2.boundM}
For any normalized capacity , its M\"obius transform satisfies for any
, :

with

and for :

and  if .
These upper and  lower bounds are attained by the normalized capacities
, respectively:

for any .
\end{theorem}

We give in Table~\ref{tab:2.mobb} the first values of the bounds.
\begin{table}[htb]
\begin{tabular}{|l|rrrrrrrrrrrr|}\hline
 & 1 & 2 & 3 & 4 & 5 & 6 & 7 & 8 & 9 & 10 & 11 & 12\\ \hline
u.b. of  & 1 & 1 & 1 & 3 & 6 & 10 & 15 & 35 & 70 & 126 & 210 & 462\\ \hline
l.b. of  &  &  &  &  &  &  &  &  &  &
 &  &  \\ \hline
\end{tabular}
\caption{Lower and upper bounds for the M\"obius transform of a normalized
  capacity}
\label{tab:2.mobb}
\end{table}
Using the well-known Stirling's approximation
 for , we
deduce that

when  tends to infinity.

\begin{proof}
Let us prove the result for the upper bound when .
 We consider the group  of permutations on . For any 
and any capacity , we define the capacity  by
 for any .

We observe that the target function  is invariant under
permutation. Indeed, 

For every set function  on , define its symmetric part , which is a symmetric function. By
convexity of , if , then so is , and by linearity
of the M\"obius inverse, we have

It is therefore sufficient to maximize  on the set of symmetric
normalized capacities . But this set is also a convex polytope, whose
extreme points are the following -valued capacities
 defined by

Indeed, if  is symmetric, it can be written as a convex combination of
these capacities:

It follows that the maximum of  is attained on one of these
capacities, say . We compute

where the third equality is obtained by letting  and the last one
follows from (\ref{eq:1.comb2}). 
Therefore  must be even.  If  is even, the maximum of 
for  even is attained for  if this is an even number,
otherwise . If  is odd, the maximum of  is
reached for  and ,
among which the even one must be chosen. As it can be checked (see
Table~\ref{tab:1} below), this amounts to taking

that is,  as defined in (\ref{eq:boundm}), and we have defined the capacity

which is  as defined in the theorem.


For establishing the upper bound of  for any ,
  remark that the value of  depends only on the subsets of . It
  follows that applying the above result to the sublattice , the set
  function  defined on  by

yields an optimal value for . It remains to turn this set function
into a capacity on , without destroying optimality. This can be done since 
is monotone on , so that taking the monotonic cover of  by
(\ref{eq:mc}) yields an optimal capacity, given by

which is exactly  as desired. Note however that this is not the only
optimal solution in general, since values of the capacty on the sublattice
 are irrelevant. 


One can proceed in a similar way for the lower bound. In this case however, as
it can be checked, the
capacity must be equal to 1 on the  first lines of the lattice ,
with  (see
Table~\ref{tab:1}).
\end{proof}
 \begin{table}[htb]
\begin{center}
\begin{tabular}{c|rrrrrrrrrrrr}
 &  &  & & & & & & & & & &
  \\ \hline
 & \red{1} &&&&&&&&&&& \\
 & \red{1} & \blue{} &&&&&&&&&& \\ 
 & \red{1} & \blue{} & 1 &&&&&&&&& \\ 
 & 1 & \blue{} & \red{3} &  &&&&&&&& \\ 
 & 1 &  & \red{6} & \blue{} & 1 &&&&&&& \\ 
 & 1 &  & \red{10} & \blue{} & 5 &  &&&&&& \\ 
 & 1 &  & \red{15} & \blue{} & 15 &  & 1 &&&&& \\ 
 & 1 &  & 21 & \blue{} & \red{35} &  & 7 &  &&&& \\ 
 & 1 &  & 28 &  & \red{70} & \blue{} & 28 &  & 1 &&& \\ 
 & 1 &  & 36 &  & \red{126} & \blue{} & 84 &  & 9 &  && \\ 
 & 1 &  & 45 &  & \red{210} & \blue{} & 210 &  & 45 &  & 1 &\\ 
 & 1 &  & 55 &  & 330 & \blue{} & \red{462} &  & 165 &  & 11
  & \\ 
\end{tabular}
\end{center}
\caption{Computation of the upper (red) and lower (blue) bounds. The value of
  the capacity  is 1 for the  first
lines of the lattice . Each entry  equals , as given by
(\ref{eq:boundm1}).} 
\label{tab:1}
\end{table}

\section{Exact bounds of the interaction transforms}
We begin by establishing a technical lemma which will permit to get the results
easily from Theorem~\ref{th:2.boundM}.
\begin{lemma}\label{lem:1}
Let , , be disjoint sets. Then

and the maximum is attained for .
\end{lemma}
\begin{proof}
The function we have to maximize is simply the derivative .  As
this is a linear function in  and  is a polytope, its maximum is
attained on a vertex, i.e. a -valued capacity. If , then by monotonicity of  we get . Since this
is clearly not the maximum of the derivative, we can discard such capacities
 from the analysis. Assuming then , we define a
capacity  by

Observe that if  is -valued, then necessarily  and , hence (\ref{eq:10}) collapses to , for any , and  is -valued and normalized too.
Moreover, any -valued normalized capacity on  can be obtained from a
-valued normalized capacity on  by the latter equality. 
  On the other hand, remark that for any 

since . In summary, we have

the last equality coming from Theorem~\ref{th:2.boundM}. Hence
(\ref{eq:max}) is established, the value of the maximum is given by
Theorem~\ref{th:2.boundM}, as well as the capacity attaining the maximum.
\end{proof}
A similar result can be established for the lower bound.
\begin{corollary}\label{cor:1}
Consider  The upper and lower bounds for the interaction
transform  are the same as
for , and they are obtained for the capacities 
and  of Theorem~\ref{th:2.boundM}.
\end{corollary}
\begin{proof}
We will obtain the upper bound, the proof for the lower bound being
similar. From Lemma~\ref{lem:1}, we see that the maximum of  does
not depend on . Thus, from (\ref{eq:3}), letting
 we obtain

\end{proof}

Similarly, we obtain the exact bounds for the Banzhaf interaction index.
\begin{corollary}
Consider  The upper and lower bounds for  are the same as
for  These upper and lower bounds are obtained for the capacities 
and  of Theorem~\ref{th:2.boundM}.
\end{corollary}
\begin{proof}
Proceeding as for Corollary~\ref{cor:1}, the result follows from the identity .
\end{proof}

\section{Exact bounds for -additive and -symmetric capacities}
We show in this section that the results established for the bounds of the
M\"obius and interaction transforms on the set of normalized capacities are
still valid when one restricts to -additive capacities and -symmetric capacities.

\begin{proposition}\label{prop:1}
For any nonempty , the normalized capacities 
given in Theorem~\ref{th:2.boundM} are at most -additive for any
. Therefore, the upper and lower bounds for the
M\"obius transform, the interaction transform and the Banzhaf interaction
transform, are valid:

for ,
and similarly for .
\end{proposition}
\begin{proof}
Given a nonempty , it suffices to show that  are at
most -additive for . Take  such that 
Then, . On the other hand, observe that for
  any , 

for any . It follows that  for any  as soon as
. Taking , by (\ref{eq:md}), we conclude that
 if , as desired.
\end{proof}
 \begin{remark} Proposition~\ref{prop:1} tells us what is the maximum
    achieved by  for the set of -additive capacities when
    , but says nothing when . The question
    appears to be very complex, because in general  will not be
    -additive, and the vertices of the polytope of -additive capacities
    are not known, except for  and 2. In particular, it is known that many
    vertices are \textit{not} -valued as soon as  (see
    \cite{micogi06}).
\end{remark}
\begin{proposition}
For any  and any partition  of ,


and similarly for .
\end{proposition}
\begin{proof}
Consider the capacities defined by

 Observe that
 for any 

Therefore , 
On the other hand,  and  are symmetric capacities, whence
they are -symmetric for any  and any partition of indifference.
\end{proof}

\section{{\bf Acknowledgements}}

We address all our thanks to the anonymous referees, for their constructive
comments, which have permitted to correct and improve the paper, in particular
the proof of Theorem~1 which is now much shorter and more clear.  This work was
partially supported by Grant MTM2012-33740.



\bibliographystyle{plain}
\bibliography{../BIB/fuzzy,../BIB/grabisch,../BIB/general}

\end{document}
