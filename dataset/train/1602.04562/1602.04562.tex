\documentclass[12pt]{article}

\usepackage{wrapfig, amsmath, amsthm, amssymb, fullpage, placeins}

\usepackage{subfigure}
\usepackage{array}
\usepackage[ruled,linesnumbered,vlined]{algorithm2e}

\clearpage{}\usepackage{rotating,graphics,psfrag,epsfig,subfigure, color}
\usepackage{fancyvrb}

\usepackage{listings}
\lstdefinelanguage{maple}
	{morekeywords={true, false, try, catch, return, break, error, 
	               module, export, local, option, in, use,
                 and, or, not, xor, xnor,
                 if, then, elif, else, fi,
                 while, for, from, by, to, do, od,
                 proc, nargs, local, global, end, NULL}}
\lstset{language=maple,numbers=left,basicstyle=\footnotesize,numberstyle=\tiny, breakatwhitespace=false,frame=single,morecomment=[l]{\#},stringstyle=\ttfamily}

\usepackage[usenames,dvipsnames]{xcolor}
\definecolor{MapleRed}{rgb}{1,0,0}
\definecolor{MapleBlue}{rgb}{0,0,1}
\definecolor{MaplePink}{rgb}{1,0,1}



\renewcommand{\arraystretch}{1.25}
\renewcommand{\vec}{\boldsymbol}



\theoremstyle{definition}
\newtheorem{notation}{Notation}


\renewcommand{\implies}{\Rightarrow}
\newcommand{\imp}{\textbf}
\newcommand{\dom}[1]{\textsf{dom} (#1)}
\newcommand{\cod}[1]{\textsf{cod}(#1)}
\newcommand{\arr}[1]{\mathcal{#1}\mathrm{-arrows}}
\newcommand{\obj}[1]{\mathcal{#1}\mathrm{-objects}}
\newcommand{\state}[1]{\item[#1] \textcolor{white}{.} \\ \textcolor{white}{.} \\ }
\newcommand{\Lstate}[1]{\item[#1] \nl}
\newcommand{\id}[1]{id_{#1}}
\newcommand{\func}[3]{#1 : #2 \rightarrow #3}
\newcommand{\CC}{\mathbb{C}}
\newcommand{\DD}{\mathcal{D}}
\newcommand{\point}[1]{\begin{itemize} \item[] #1 \end{itemize}}
\newcommand{\blank}{\textcolor{white}{-}}
\newcommand{\fieldt}{\mathbb{Z}_q[t]}
\newcommand{\fieldtx}{\mathbb{Z}_q[t][x]}
\newcommand{\impl}{\Rightarrow}
\newcommand{\eps}{\varepsilon}
\newcommand{\brac}[1]{\left( #1 \right)}
\newcommand{\cbrac}[1]{\left\{ #1 \right\}}
\newcommand{\sbrac}[1]{\left[ #1 \right]}
\newcommand{\vbrac}[1]{\left< #1 \right>}


\newcommand{\x}{\boldsymbol{x}}
\renewcommand{\H}{\boldsymbol{H}}
\newcommand{\zero}{\boldsymbol{0}}
\newcommand{\z}{\boldsymbol{z}}
\newcommand{\F}{\boldsymbol{F}}
\newcommand{\p}{\boldsymbol{p}}
\newcommand{\q}{\boldsymbol{q}}
\newcommand{\alp}{{\boldsymbol{\alpha}}}
\newcommand{\bet}{{\boldsymbol{\beta}}}

\newcommand{\V}{\mathbf{V}}
\newcommand{\W}{\mathbf{W}}
\newcommand{\M}{\mathbf{M}}
\newcommand{\A}{\mathbf{A}}
\newcommand{\B}{\mathbf{B}}
\newcommand{\C}{\mathbf{C}}
\newcommand{\X}{\mathbf{X}}
\newcommand{\xx}{\boldsymbol{x}}
\newcommand{\0}{\mathbf{0}}
\newcommand{\II}{\mathbf{I}}
\newcommand{\ZZ}{\mathbb{Z}}
\newcommand{\QQ}{\mathbb{Q}}
\newcommand{\floor}[1]{\left\lfloor #1 \right\rfloor}
\newcommand{\eval}[2]{ \left.#1\right |_{#2}}
\newcommand{\xinit}{x_{\text{init}}}
\newcommand{\idnty}{\text{Id}}

\newcommand{\pdiff}[2]{\frac{\partial #1}{\partial #2}}

\newcommand{\FF}{\mathbf{F}}
\newcommand{\JJ}{\mathbf{J}}
\newcommand{\XX}{\mathbf{X}}

\newcommand{\Xinit}{\XX_{\text{init}}}
\newcommand{\Ginit}{G_{\text{init}}}

\renewcommand{\mod}{\text{ mod }}
\newcommand{\Mod}{\hspace{2mm}\mod}

\newcommand{\rem}{\text{ rem }}
\newcommand{\quo}{\text{ quo }}
\renewcommand{\div}{\text{ div }}
\newcommand{\MM}{\textsf{M}}

\def\MapleInput#1{\noindent{\tt{[}\color{MapleRed}{\tt{>}\tt{#1}}}}
\def\MapleOutput#1{{\begin{center} \emph{\color{MapleBlue}{#1}} \end{center}}}
\def\MapleWarning#1{\noindent{{\small {\tt \color{MaplePink}{#1} }}}}


\DefineVerbatimEnvironment {MapleInputs}{Verbatim}{formatcom=\color{MapleRed}}

\renewcommand{\FancyVerbFormatLine}[1]{\makebox[0.43cm][l]{\color{black}{[}\color{MapleRed}{>}}#1}

\newcommand{\PutEnd}{\begin{center}
\vspace{\fill}\ \newline
{\tiny \rm Typeset by: Paul Vrbik }
{\tiny \rm Compiled on \today }
\end{center}
}
\clearpage{}

\theoremstyle{theorem}
\newtheorem{theorem}{Theorem}

\theoremstyle{definition}
\newtheorem*{definition}{Definition}
\newtheorem*{remark}{Remark}
\newtheorem*{example}{Example}




\newcommand{\RR}{\mathcal{R}}
\newcommand{\DFT}{\text{DFT}}
\newcommand{\TFT}{\text{TFT}}
\newcommand{\w}{\omega}
\renewcommand{\a}{\mathbf{a}}
\newcommand{\xs}[1]{\mathbf{x_{#1}}}
\renewcommand{\thefootnote}{\fnsymbol{footnote}}


\title{An Illustrated Introduction to the Truncated Fourier Transform}
\author{Paul Vrbik. \\
School of Mathematical and Physical Sciences \\
The University of Newcastle\\
Callaghan, Australia\\
\tt{paulvrbik@gmail.com}}
\date{\today}

\begin{document}
\maketitle


\begin{abstract}
The Truncated Fourier Transform (\textsc{tft}) is a variation of the Discrete Fourier Transform (\textsc{dft}/\textsc{fft}) that allows for input vectors that do \emph{not} have length  for  a positive integer. We present the univariate version of the \textsc{tft}, originally due to Joris van der Hoeven, heavily illustrating the presentation in order to make these methods accessible to a broader audience.
\end{abstract}



\section{Introduction}


In 1965 Cooley and Tukey developed the \emph{Discrete Fourier Transform} (\textsc{dft}) to recover continuous functions from discrete samples. This was a landmark discovery because it allowed for the digital manipulation of analogue signals (like sound) by computers. Soon after a variant called the \emph{Fast Fourier Transform} (\textsc{fft}) eclipsed the \textsc{dft} to the extent that \textsc{fft} is often mistakenly substituted for \textsc{dft}. As its name implies, the \textsc{fft} is a method for computing the \textsc{dft} faster.






\textsc{fft}s have an interesting application in Computer Algebra. Let  be a ring with  a unit. If  has a primitive th root of unity  with  (i.e. ) then the \textsc{fft} computes the product of two polynomials  with  in  operations in . Unfortunately, when  is sufficiently far from a power of two \emph{many} computations wasted. This deficiency was addressed by the signal processing community using a method called \textsc{fft}-pruning \cite{SigProcessing}. However, the difficult inversion of this method is due to van der Hoeven \cite{TFT2}\cite{TFT1}.\smallskip

In Section \ref{Section::DFT} we outline the \textsc{dft}, including a method for its non-recursive implementation. In Section \ref{Section::TFT} we develop the ``pruned'' variant, called Truncated Fourier Transform (\textsc{tft}). Finally, in Section \ref{Section::InvTFT}, we show how the \textsc{tft} can be inverted and outline the algorithm for doing so in Section \ref{Section::AlgTFT}.

\section{The Discrete Fourier Transform}\label{Section::DFT}
For this paper let  be a ring with  a unit and  an th root of unity. The Discrete Fourier Transform\footnote{In signal processing community this is called the ``decimation-in-time'' variant of the \textsc{fft}.}, with respect to , of vector 

is the vector  with 

Alternatively we can view these -tuples as encoding the coefficients of polynomials from  and define the \textsc{dft} with respect to  as the mapping 
We let the relationship between  and its coefficients be implicit and write 

when .

The \textsc{dft} can be computed efficiently using binary splitting. This method requires evaluation only at  for , rather than at all . To compute the \textsc{dft} of  with respect to  we write

and recursively compute the \textsc{dft} of  and  with respect to :

Finally, we construct  according to



This description has a natural implementation as a recursive algorithm, but in practice it is often more efficient to implement an in-place algorithm that eliminates the overhead of creating recursive stacks.

\begin{definition}
Let  and  be a positive integers and let  for . The length- bitwise reverse of  is given by

\end{definition}


\begin{example}  and  because

and
 
Notice if we were to write 3, 24, 11, and 26 as a binary numbers to five digits we have 00011 reverses to 11000 and 01011 reverses to 11010 --- in fact this is the inspiration for the name ``bitwise reverse.''
\end{example}

For the in-place non-recursive \textsc{dft} algorithm, we require \emph{only one vector} of length . Initially, at step zero, this vector is

and is updated (incrementally) at steps  by the rule

where  and for all ,  . Note that two additions and \emph{one} multiplication are done in (\ref{EqButterfly}) as one product is merely the negation of the other.\medskip

We illustrate the dependencies of the  values in (\ref{EqButterfly}) with \begin{center}
   \includegraphics[width=0.65\textwidth]{tst.35}
\end{center}
We call this a ``butterfly'' after the shape it forms and may say  controls the width --- the value of which decreases as  increases. By placing these butterflies on a  grid  (Figure \ref{Butterfly}) we can see which values of  are required to compute particular entires of  (and vice-versa). For example
\begin{center}
   \includegraphics[width=0.4\textwidth]{tst.36}\;\;\;\;\;\;\;\;\end{center}
denotes that  and  are required to determine  and  (and vice-versa).\medskip

\begin{figure}[htbp]
\begin{center}
\includegraphics[width=\textwidth]{tst.0}
\caption{Schematic representation of Equation (\ref{EqButterfly}) using butterflies to illustrate the value dependencies at various steps . The grid has rows  and columns  for illustrating the  values.}
\label{Butterfly}
\end{center}
\end{figure}


Using induction over ,

for all  and  \cite{TFT2}. In particular, when  and , we have

for all . That is,  is a (specific) permutation of  as illustrated in Figure \ref{FFT1}.

The key property of the \textsc{dft} is that it is straightforward to invert, that is to recover  from :

since  whenever . This yields a polynomial multiplication algorithm that does  operations in  (see \cite[\S4.7]{AFCA} for the outline of this algorithm).


\begin{figure}[htbp]
\begin{center}
\includegraphics[width=\textwidth]{tst.1}
\caption{The Discrete Fourier Transform for . The top row, corresponding to , are the initial values . The bottom row, corresponding to , is a permutation of  (the result of the DFT on ).}
\label{FFT1}
\end{center}
\end{figure}

\FloatBarrier

\section{The Truncated Fourier Transform}\label{Section::TFT}
The motivation behind the Truncated Fourier Transform (\textsc{tft}) is the observation that many computations are wasted when the length of  (the input) is not a power of two.\footnote{The \textsc{tft} is exactly equivalent to a technique called ``\textsc{fft} pruning'' in the signal processing literature \cite{SigProcessing}.} This is entirely the fault of the strategy where one ``completes'' the -tuple  by setting  when  to artificially extend the length of  to the nearest power of two (so the \textsc{dft} can be executed as usual).

However, despite the fact that we may only want  components of , the \textsc{dft} will calculate \emph{all} of them. Thus computation is wasted. We illustrate this in Figures \ref{OverFFT} and \ref{TFT0}. This type of wasted computation is relevant when using the \textsc{dft} to multiply polynomials --- their products are rarely of degree one less some power of two.

The definition of the \textsc{tft} is similar to that of the \textsc{dft} with the exception that the input and output vector ( resp.\ ) are not necessarily of length some power of two.  More precisely the \textsc{tft} of an -tuple  is the -tuple 
where ,  (usually ) and  a th root of unity. 


\begin{figure}[htbp]
\begin{center}
\includegraphics[width=\textwidth]{tst.4}
\caption{The \textsc{dft} with ``artificial'' zero points (large black dots).}
\label{OverFFT}
\end{center}
\end{figure}

\begin{figure}[htbp]
\begin{center}
\includegraphics[width=\textwidth]{tst.5}
\caption{Removing all unnecessary computations from Figure \ref{OverFFT} gives the schematic representation of the \textsc{tft}.}
\label{TFT0}
\end{center}
\end{figure}

\begin{remark}
van der Hoeven \cite{TFT2} gives a more general description of the \textsc{tft} where one can choose an initial vector  and target vector . Provided the 's are distinct one can carry out the \textsc{tft} by considering the full \textsc{dft} and removing all computations not required for the desired output. In this paper, we restrict our discussion to that of the scenario in Figure \ref{TFT0} (where the input and output are the same initial segments) because it can be used for polynomial multiplication, and because it yields the most improvement. 
\end{remark}

If we only allow ourselves to operate in a size  vector it is straightforward to modify the in-place \textsc{dft} algorithm from the previous section to execute the \textsc{tft}. (It should be emphasized that this only saves computation and not space. For a ``true'' in-place \textsc{tft} algorithm that operates in an array of size , see Harvey and Roche's \cite{InPlaceTFT}.) At stage  it suffices to compute 
 with  where .\footnote{This is a correction to the bound given in \cite{TFT1} as pointed out in \cite{TFT2}.}

\begin{theorem}
Let ,  and  be a primitive th root of unity in . The \textsc{tft} of an -tuple  with respect to  can be computed using at most  additions and  multiplications of powers of .
\end{theorem}

\begin{proof}
Let ; at stage  we compute . So, in addition to  we compute  more values. Therefore, in total, we compute at most 

values . The result follows.
\end{proof}

\section{Inverting The Truncated Fourier Transform}\label{Section::InvTFT}
Unfortunately, the \textsc{tft} cannot be inverted by merely doing another \textsc{tft} with  and adjusting the output by some constant factor like inverse of the \textsc{dft}. Simply put: we are missing information and must account for this.

\begin{example}
Let , , with  a th primitive root of unity. Setting , the \textsc{tft} of  at 5 is

Now, to show the \textsc{tft} of this with respect to  is \emph{not} , define 

The \textsc{tft} of  with respect to  is

which is \emph{not} a constant multiple of .

This discrepancy is caused by the completion of  to  --- we should have instead completed  to .\end{example}

To invert the \textsc{tft} we 
use the fact that whenever two values among  are known, that the other values can be deduced. That is, if two values of some butterfly are known then the other two values can be calculated using (\ref{EqButterfly}) as the relevant matrix is invertible. Moreover, these relations only involve shifting (multiplication and division by two), additions, subtractions and multiplications by roots of unity --- an ideal scenario for implementation.

As with \textsc{dft}, observe that  can be calculated from . This is because all the butterfly relations necessary to move up like this never require  for any  and . This is illustrated in Figure \ref{boxes}. More generally, we have that

is sufficient information to compute
 
provided that . (In Algorithm 1, that follows, we call this a ``self-contained push up''.)\\

\begin{figure}[h]
\begin{center}
\includegraphics[width=\textwidth]{tst.9}
\caption{The computations in the boxes are self contained.}
\label{boxes}
\end{center}
\end{figure}

\section{Inverse \textsc{tft} Algorithm}\label{Section::AlgTFT}

Finally, in this section, we present a simple recursive description of the inverse \textsc{tft} algorithm for the case we have restricted ourselves to (all zeroes packed at the end). The algorithm operates in a length  array  for which we assume access; here  corresponds to , a th primitive root of unity. Initially, the content of the array is

where  is the result of the \textsc{tft} on .

In keeping with our ``Illustrated'' description we use pictures, like Figure \ref{PicPic},
to indicate what values are known (solid dots ) and what value to calculate (empty dot ). For instance, ``push down  with Figure \ref{PicPic}'', is shorthand for: use  and  to determine . We emphasize with an arrow that this new value should also overwrite the one at . This calculation is easily accomplished using (\ref{EqButterfly}) with a caveat: the values  and  are not explicitly known. What \emph{is} known is , and therefore , and some array position . Observe that  is recovered by  (the quotient of ).
\begin{figure}[htp]
  \begin{center}
    \includegraphics[width=0.7\textwidth]{tst.27}
  \end{center}
  \caption{Overwrite  with .}
\label{PicPic}
\end{figure}

The full description of the inverse \textsc{tft} follows in Algorithm \ref{InvTFT}; note that the initial call is {\bf InvTFT}. A visual depiction of Algorithm \ref{InvTFT} is given in Figure \ref{BigInvTFT}. A sketch of a proof of its correctness follows.



\newcommand{\head}{\text{head}}
\newcommand{\tail}{\text{tail}}
\newcommand{\last}{\text{last}}
\newcommand{\LM}{\text{LeftMiddle}}
\newcommand{\RM}{\text{RightMiddle}}

\begin{algorithm}

\setlength{\parskip}{0.65ex}

\caption{\textbf{InvTFT}}
\label{InvTFT}

\SetKw{Break}{break}\SetKw{AND}{and}\SetKwInOut{Input}{Initial call}\SetKwInOut{Output}{Output\,}
\Input{ InvTFT; }



\BlankLine
middle \1em]

\If {\rm } { 
	Base case---do nothing\;
	\Return null;
}


\ElseIf {\rm } {
	Push up the self-contained region  to \;
	
	\mbox{Push down  to  with }\raisebox{-,9em}{\includegraphics[width=.06\textwidth]{tst.24}}~\;
	
	{\bf InvTFT}\;
	\;

	\mbox{Push up (in pairs)  to } 		\mbox{with }\raisebox{-.9em}{\includegraphics[width=.07\linewidth]{tst.25}}~\;
	
	
}
\ElseIf {\rm} {


	\mbox{Push down  to  with }\raisebox{-.9em}{\includegraphics[width=.06\textwidth]{tst.26}}\;
	
	{\bf InvTFT}\;
	
\mbox{Push up  to  with }\raisebox{-.9em}{\includegraphics[width=.06\textwidth]{tst.23}}\;

}



\end{algorithm}
 

\begin{theorem}
Algorithm \ref{InvTFT}, initially called with  and given access to the zero-indexed length  array

will terminate with

where (\ref{forward}) is the result of the \textsc{tft} on (\ref{backward}).
\end{theorem}

\begin{proof}[Termination]
Let , , and  be the values of head, tail, and last at the th recursive call. Consider the {integer} sequences given by


If  then we have termination. Otherwise, either branch (7) executes giving

and thus , or branch (13) executes, giving

and thus .

Neither branch can run forever since  causes termination and  means either  strictly decreases or condition (13) fails, forcing termination.
\end{proof}

\begin{proof}[Sketch of correctness]
Figure \ref{MoreHalf} and Figure \ref{LessHalf} demonstrate that self contained regions can be exploited to obtain the  initial values required to complete the inversion. That is to say, for , that

can always be calculated from

\end{proof}

{\center
\begin{figure}[p]
\caption{ (i.e. at least half the values are at ).}
\label{MoreHalf}
\subfigure[Line (8): push up the self contained (dashed) region. This yields values sufficient to push down at line (9). ]{
\includegraphics[width=.98\textwidth]{tst.28}
}
\subfigure[This enables us to make a recursive call on the dashed region (line (12)). By our induction hypothesis this brings all points at  to .]{
\includegraphics[width=.98\textwidth]{tst.29}
}
\subfigure[Sufficient points at  are known to move to  at line (13).]{
\includegraphics[width=.98\textwidth]{tst.30}
}
\end{figure}
}

{\center
\begin{figure}[p]
\begin{center}
\label{LessHalf}
\subfigure[Initially there is sufficient information to push down at line (14).]{
\includegraphics[width=.98\textwidth]{tst.31}
}
\subfigure[This enables us to make the prescribed recursive call at line (15). ]{
\includegraphics[width=.98\textwidth]{tst.32}
}
\subfigure[By the induction hypothesis this brings the values in the dashed region to , leaving enough information to move up at line (16).]{
\includegraphics[width=.98\textwidth]{tst.33}
}
\end{center}
\caption{ (i.e. less than half the values are at ).}
\end{figure}
}





\begin{figure}[p]
\subfigure[Initial state of the algorithm. Grey dots are the result of the forward \textsc{tft}; larger grey dots are zeros.]{
\includegraphics[width=.48\textwidth]{tst.10}
\label{fig:6a}
}~
\subfigure[tailLeftMiddle. Push up; calculate  from  (contained region). \textbf{Then} push down.]{
\includegraphics[width=.48\textwidth]{tst.11}
\label{fig:6b}
}
\subfigure[Recursive call on right half.]{
\includegraphics[width=.48\textwidth]{tst.12}
\label{fig:6c}
}~
\subfigure[tailLeftMiddle. Push down with.]{
\includegraphics[width=.48\textwidth]{tst.13}
\label{fig:6d}
}
\subfigure[Recursive call on left half.]{
\includegraphics[width=.48\textwidth]{tst.14}
\label{fig:6e}
}~
\subfigure[tailLeftMiddle. Push up the contained (dashed) region then push down.]{
\includegraphics[width=.48\textwidth]{tst.15}
\label{fig:6f}
}
\subfigure[Recursive call on right half.]{
\includegraphics[width=.48\textwidth]{tst.16}
\label{fig:6g}
}~
\subfigure[Hiding details. The result of (g).]{
\includegraphics[width=.48\textwidth]{tst.18}
\label{fig:6h}
}
\subfigure[Finish step (e) by pushing up.]{
\includegraphics[width=.48\textwidth]{tst.19}
\label{fig:6i}
}~
\subfigure[Finish step (c) by pushing up.]{
\includegraphics[width=.48\textwidth]{tst.20}
\label{fig:6j}
}
\subfigure[Resolve the original call by pushing up.]{
\includegraphics[width=.48\textwidth]{tst.21}
\label{fig:6k}
}~
\subfigure[Done.]{
\includegraphics[width=.48\textwidth]{tst.22}
\label{fig:6l}
}
\caption[Optional caption for list of figures]{Schematic representation of the recursive computation of the inverse \textsc{tft} for  and . }
\label{BigInvTFT}
\end{figure}

\section{Conclusions}
The Truncated Fourier Transform is a novel and elegant way to reduce the number of computations of a \textsc{dft}-based computation by a possible factor of two (which may be significant). Additionally, with the advent of Harvey and Roche's paper \cite{InPlaceTFT}, it is possible to save as much space as computation. The hidden ``cost'' of working with the \textsc{tft} algorithm is the increased difficulty of determining the inverse \textsc{tft}. Although in most cases this is still less costly than the inverse \textsc{dft}, the algorithm is no doubt more difficult to implement.

\section*{Acknowledgements}
The author wishes to thank Dr.\ Dan Roche and Dr.\ \'Eric Schost for reading a draft of this paper and offering suggestions. 

\bibliographystyle{plain}
\bibliography{allbib}

\end{document}