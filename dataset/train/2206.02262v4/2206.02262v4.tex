
\documentclass{article} \usepackage{iclr2023_conference,times}

\def\eqD{\mathrel{\ooalign{\raisebox{1.5\height}{}\cr\hidewidth\hidewidth\cr}}}

\usepackage{amsmath,amsfonts,bm,amsthm,amssymb,mathrsfs,algorithm,algorithmic}
\usepackage{hyperref,url,booktabs,enumitem,multicol,graphicx,caption,subcaption}

\newcommand{\figleft}{{\em (Left)}}
\newcommand{\figcenter}{{\em (Center)}}
\newcommand{\figright}{{\em (Right)}}
\newcommand{\figtop}{{\em (Top)}}
\newcommand{\figbottom}{{\em (Bottom)}}
\newcommand{\captiona}{{\em (a)}}
\newcommand{\captionb}{{\em (b)}}
\newcommand{\captionc}{{\em (c)}}
\newcommand{\captiond}{{\em (d)}}

\newcommand{\newterm}[1]{{\bf #1}}


\def\figref#1{figure~\ref{#1}}
\def\Figref#1{Figure~\ref{#1}}
\def\twofigref#1#2{figures \ref{#1} and \ref{#2}}
\def\quadfigref#1#2#3#4{figures \ref{#1}, \ref{#2}, \ref{#3} and \ref{#4}}
\def\secref#1{section~\ref{#1}}
\def\Secref#1{Section~\ref{#1}}
\def\twosecrefs#1#2{sections \ref{#1} and \ref{#2}}
\def\secrefs#1#2#3{sections \ref{#1}, \ref{#2} and \ref{#3}}
\def\eqref#1{(\ref{#1})}
\def\Eqref#1{Equation~\ref{#1}}
\def\plaineqref#1{\ref{#1}}
\def\chapref#1{chapter~\ref{#1}}
\def\Chapref#1{Chapter~\ref{#1}}
\def\rangechapref#1#2{chapters\ref{#1}--\ref{#2}}
\def\algref#1{algorithm~\ref{#1}}
\def\Algref#1{Algorithm~\ref{#1}}
\def\twoalgref#1#2{algorithms \ref{#1} and \ref{#2}}
\def\Twoalgref#1#2{Algorithms \ref{#1} and \ref{#2}}
\def\partref#1{part~\ref{#1}}
\def\Partref#1{Part~\ref{#1}}
\def\twopartref#1#2{parts \ref{#1} and \ref{#2}}

\def\ceil#1{\lceil #1 \rceil}
\def\floor#1{\lfloor #1 \rfloor}
\def\1{\bm{1}}
\newcommand{\train}{\mathcal{D}}
\newcommand{\valid}{\mathcal{D_{\mathrm{valid}}}}
\newcommand{\test}{\mathcal{D_{\mathrm{test}}}}

\def\eps{{\epsilon}}


\def\reta{{\textnormal{}}}
\def\ra{{\textnormal{a}}}
\def\rb{{\textnormal{b}}}
\def\rc{{\textnormal{c}}}
\def\rd{{\textnormal{d}}}
\def\re{{\textnormal{e}}}
\def\rf{{\textnormal{f}}}
\def\rg{{\textnormal{g}}}
\def\rh{{\textnormal{h}}}
\def\ri{{\textnormal{i}}}
\def\rj{{\textnormal{j}}}
\def\rk{{\textnormal{k}}}
\def\rl{{\textnormal{l}}}
\def\rn{{\textnormal{n}}}
\def\ro{{\textnormal{o}}}
\def\rp{{\textnormal{p}}}
\def\rq{{\textnormal{q}}}
\def\rr{{\textnormal{r}}}
\def\rs{{\textnormal{s}}}
\def\rt{{\textnormal{t}}}
\def\ru{{\textnormal{u}}}
\def\rv{{\textnormal{v}}}
\def\rw{{\textnormal{w}}}
\def\rx{{\textnormal{x}}}
\def\ry{{\textnormal{y}}}
\def\rz{{\textnormal{z}}}

\def\rvepsilon{{\mathbf{\epsilon}}}
\def\rvtheta{{\mathbf{\theta}}}
\def\rva{{\mathbf{a}}}
\def\rvb{{\mathbf{b}}}
\def\rvc{{\mathbf{c}}}
\def\rvd{{\mathbf{d}}}
\def\rve{{\mathbf{e}}}
\def\rvf{{\mathbf{f}}}
\def\rvg{{\mathbf{g}}}
\def\rvh{{\mathbf{h}}}
\def\rvi{{\mathbf{i}}}
\def\rvj{{\mathbf{j}}}
\def\rvk{{\mathbf{k}}}
\def\rvl{{\mathbf{l}}}
\def\rvm{{\mathbf{m}}}
\def\rvn{{\mathbf{n}}}
\def\rvo{{\mathbf{o}}}
\def\rvp{{\mathbf{p}}}
\def\rvq{{\mathbf{q}}}
\def\rvr{{\mathbf{r}}}
\def\rvs{{\mathbf{s}}}
\def\rvt{{\mathbf{t}}}
\def\rvu{{\mathbf{u}}}
\def\rvv{{\mathbf{v}}}
\def\rvw{{\mathbf{w}}}
\def\rvx{{\mathbf{x}}}
\def\rvy{{\mathbf{y}}}
\def\rvz{{\mathbf{z}}}

\def\erva{{\textnormal{a}}}
\def\ervb{{\textnormal{b}}}
\def\ervc{{\textnormal{c}}}
\def\ervd{{\textnormal{d}}}
\def\erve{{\textnormal{e}}}
\def\ervf{{\textnormal{f}}}
\def\ervg{{\textnormal{g}}}
\def\ervh{{\textnormal{h}}}
\def\ervi{{\textnormal{i}}}
\def\ervj{{\textnormal{j}}}
\def\ervk{{\textnormal{k}}}
\def\ervl{{\textnormal{l}}}
\def\ervm{{\textnormal{m}}}
\def\ervn{{\textnormal{n}}}
\def\ervo{{\textnormal{o}}}
\def\ervp{{\textnormal{p}}}
\def\ervq{{\textnormal{q}}}
\def\ervr{{\textnormal{r}}}
\def\ervs{{\textnormal{s}}}
\def\ervt{{\textnormal{t}}}
\def\ervu{{\textnormal{u}}}
\def\ervv{{\textnormal{v}}}
\def\ervw{{\textnormal{w}}}
\def\ervx{{\textnormal{x}}}
\def\ervy{{\textnormal{y}}}
\def\ervz{{\textnormal{z}}}

\def\rmA{{\mathbf{A}}}
\def\rmB{{\mathbf{B}}}
\def\rmC{{\mathbf{C}}}
\def\rmD{{\mathbf{D}}}
\def\rmE{{\mathbf{E}}}
\def\rmF{{\mathbf{F}}}
\def\rmG{{\mathbf{G}}}
\def\rmH{{\mathbf{H}}}
\def\rmI{{\mathbf{I}}}
\def\rmJ{{\mathbf{J}}}
\def\rmK{{\mathbf{K}}}
\def\rmL{{\mathbf{L}}}
\def\rmM{{\mathbf{M}}}
\def\rmN{{\mathbf{N}}}
\def\rmO{{\mathbf{O}}}
\def\rmP{{\mathbf{P}}}
\def\rmQ{{\mathbf{Q}}}
\def\rmR{{\mathbf{R}}}
\def\rmS{{\mathbf{S}}}
\def\rmT{{\mathbf{T}}}
\def\rmU{{\mathbf{U}}}
\def\rmV{{\mathbf{V}}}
\def\rmW{{\mathbf{W}}}
\def\rmX{{\mathbf{X}}}
\def\rmY{{\mathbf{Y}}}
\def\rmZ{{\mathbf{Z}}}

\def\ermA{{\textnormal{A}}}
\def\ermB{{\textnormal{B}}}
\def\ermC{{\textnormal{C}}}
\def\ermD{{\textnormal{D}}}
\def\ermE{{\textnormal{E}}}
\def\ermF{{\textnormal{F}}}
\def\ermG{{\textnormal{G}}}
\def\ermH{{\textnormal{H}}}
\def\ermI{{\textnormal{I}}}
\def\ermJ{{\textnormal{J}}}
\def\ermK{{\textnormal{K}}}
\def\ermL{{\textnormal{L}}}
\def\ermM{{\textnormal{M}}}
\def\ermN{{\textnormal{N}}}
\def\ermO{{\textnormal{O}}}
\def\ermP{{\textnormal{P}}}
\def\ermQ{{\textnormal{Q}}}
\def\ermR{{\textnormal{R}}}
\def\ermS{{\textnormal{S}}}
\def\ermT{{\textnormal{T}}}
\def\ermU{{\textnormal{U}}}
\def\ermV{{\textnormal{V}}}
\def\ermW{{\textnormal{W}}}
\def\ermX{{\textnormal{X}}}
\def\ermY{{\textnormal{Y}}}
\def\ermZ{{\textnormal{Z}}}

\def\vzero{{\bm{0}}}
\def\vone{{\bm{1}}}
\def\vmu{{\bm{\mu}}}
\def\vtheta{{\bm{\theta}}}
\def\va{{\bm{a}}}
\def\vb{{\bm{b}}}
\def\vc{{\bm{c}}}
\def\vd{{\bm{d}}}
\def\ve{{\bm{e}}}
\def\vf{{\bm{f}}}
\def\vg{{\bm{g}}}
\def\vh{{\bm{h}}}
\def\vi{{\bm{i}}}
\def\vj{{\bm{j}}}
\def\vk{{\bm{k}}}
\def\vl{{\bm{l}}}
\def\vm{{\bm{m}}}
\def\vn{{\bm{n}}}
\def\vo{{\bm{o}}}
\def\vp{{\bm{p}}}
\def\vq{{\bm{q}}}
\def\vr{{\bm{r}}}
\def\vs{{\bm{s}}}
\def\vt{{\bm{t}}}
\def\vu{{\bm{u}}}
\def\vv{{\bm{v}}}
\def\vw{{\bm{w}}}
\def\vx{{\bm{x}}}
\def\vy{{\bm{y}}}
\def\vz{{\bm{z}}}
\def\vphi{{\bm{\phi}}}
\def\vepsilon{{\bm{\epsilon}}}
\def\vmu{{\bm{\mu}}}
\def\vsigma{{\bm{\sigma}}}
\def\piv{{\boldsymbol{\pi}}}
\def\evalpha{{\alpha}}
\def\evbeta{{\beta}}
\def\evepsilon{{\epsilon}}
\def\evlambda{{\lambda}}
\def\evomega{{\omega}}
\def\evmu{{\mu}}
\def\evpsi{{\psi}}
\def\evsigma{{\sigma}}
\def\evtheta{{\theta}}
\def\eva{{a}}
\def\evb{{b}}
\def\evc{{c}}
\def\evd{{d}}
\def\eve{{e}}
\def\evf{{f}}
\def\evg{{g}}
\def\evh{{h}}
\def\evi{{i}}
\def\evj{{j}}
\def\evk{{k}}
\def\evl{{l}}
\def\evm{{m}}
\def\evn{{n}}
\def\evo{{o}}
\def\evp{{p}}
\def\evq{{q}}
\def\evr{{r}}
\def\evs{{s}}
\def\evt{{t}}
\def\evu{{u}}
\def\evv{{v}}
\def\evw{{w}}
\def\evx{{x}}
\def\evy{{y}}
\def\evz{{z}}

\def\mA{{\bm{A}}}
\def\mB{{\bm{B}}}
\def\mC{{\bm{C}}}
\def\mD{{\bm{D}}}
\def\mE{{\bm{E}}}
\def\mF{{\bm{F}}}
\def\mG{{\bm{G}}}
\def\mH{{\bm{H}}}
\def\mI{{\bm{I}}}
\def\mJ{{\bm{J}}}
\def\mK{{\bm{K}}}
\def\mL{{\bm{L}}}
\def\mM{{\bm{M}}}
\def\mN{{\bm{N}}}
\def\mO{{\bm{O}}}
\def\mP{{\bm{P}}}
\def\mQ{{\bm{Q}}}
\def\mR{{\bm{R}}}
\def\mS{{\bm{S}}}
\def\mT{{\bm{T}}}
\def\mU{{\bm{U}}}
\def\mV{{\bm{V}}}
\def\mW{{\bm{W}}}
\def\mX{{\bm{X}}}
\def\mY{{\bm{Y}}}
\def\mZ{{\bm{Z}}}
\def\mBeta{{\bm{\beta}}}
\def\mPhi{{\bm{\Phi}}}
\def\mLambda{{\bm{\Lambda}}}
\def\mSigma{{\bm{\Sigma}}}

\DeclareMathAlphabet{\mathsfit}{\encodingdefault}{\sfdefault}{m}{sl}
\SetMathAlphabet{\mathsfit}{bold}{\encodingdefault}{\sfdefault}{bx}{n}
\newcommand{\tens}[1]{\bm{\mathsfit{#1}}}
\def\tA{{\tens{A}}}
\def\tB{{\tens{B}}}
\def\tC{{\tens{C}}}
\def\tD{{\tens{D}}}
\def\tE{{\tens{E}}}
\def\tF{{\tens{F}}}
\def\tG{{\tens{G}}}
\def\tH{{\tens{H}}}
\def\tI{{\tens{I}}}
\def\tJ{{\tens{J}}}
\def\tK{{\tens{K}}}
\def\tL{{\tens{L}}}
\def\tM{{\tens{M}}}
\def\tN{{\tens{N}}}
\def\tO{{\tens{O}}}
\def\tP{{\tens{P}}}
\def\tQ{{\tens{Q}}}
\def\tR{{\tens{R}}}
\def\tS{{\tens{S}}}
\def\tT{{\tens{T}}}
\def\tU{{\tens{U}}}
\def\tV{{\tens{V}}}
\def\tW{{\tens{W}}}
\def\tX{{\tens{X}}}
\def\tY{{\tens{Y}}}
\def\tZ{{\tens{Z}}}


\def\gA{{\mathcal{A}}}
\def\gB{{\mathcal{B}}}
\def\gC{{\mathcal{C}}}
\def\gD{{\mathcal{D}}}
\def\gE{{\mathcal{E}}}
\def\gF{{\mathcal{F}}}
\def\gG{{\mathcal{G}}}
\def\gH{{\mathcal{H}}}
\def\gI{{\mathcal{I}}}
\def\gJ{{\mathcal{J}}}
\def\gK{{\mathcal{K}}}
\def\gL{{\mathcal{L}}}
\def\gM{{\mathcal{M}}}
\def\gN{{\mathcal{N}}}
\def\gO{{\mathcal{O}}}
\def\gP{{\mathcal{P}}}
\def\gQ{{\mathcal{Q}}}
\def\gR{{\mathcal{R}}}
\def\gS{{\mathcal{S}}}
\def\gT{{\mathcal{T}}}
\def\gU{{\mathcal{U}}}
\def\gV{{\mathcal{V}}}
\def\gW{{\mathcal{W}}}
\def\gX{{\mathcal{X}}}
\def\gY{{\mathcal{Y}}}
\def\gZ{{\mathcal{Z}}}

\def\sA{{\mathbb{A}}}
\def\sB{{\mathbb{B}}}
\def\sC{{\mathbb{C}}}
\def\sD{{\mathbb{D}}}
\def\sF{{\mathbb{F}}}
\def\sG{{\mathbb{G}}}
\def\sH{{\mathbb{H}}}
\def\sI{{\mathbb{I}}}
\def\sJ{{\mathbb{J}}}
\def\sK{{\mathbb{K}}}
\def\sL{{\mathbb{L}}}
\def\sM{{\mathbb{M}}}
\def\sN{{\mathbb{N}}}
\def\sO{{\mathbb{O}}}
\def\sP{{\mathbb{P}}}
\def\sQ{{\mathbb{Q}}}
\def\sR{{\mathbb{R}}}
\def\sS{{\mathbb{S}}}
\def\sT{{\mathbb{T}}}
\def\sU{{\mathbb{U}}}
\def\sV{{\mathbb{V}}}
\def\sW{{\mathbb{W}}}
\def\sX{{\mathbb{X}}}
\def\sY{{\mathbb{Y}}}
\def\sZ{{\mathbb{Z}}}

\def\emLambda{{\Lambda}}
\def\emA{{A}}
\def\emB{{B}}
\def\emC{{C}}
\def\emD{{D}}
\def\emE{{E}}
\def\emF{{F}}
\def\emG{{G}}
\def\emH{{H}}
\def\emI{{I}}
\def\emJ{{J}}
\def\emK{{K}}
\def\emL{{L}}
\def\emM{{M}}
\def\emN{{N}}
\def\emO{{O}}
\def\emP{{P}}
\def\emQ{{Q}}
\def\emR{{R}}
\def\emS{{S}}
\def\emT{{T}}
\def\emU{{U}}
\def\emV{{V}}
\def\emW{{W}}
\def\emX{{X}}
\def\emY{{Y}}
\def\emZ{{Z}}
\def\emSigma{{\Sigma}}

\newcommand{\etens}[1]{\mathsfit{#1}}
\def\etLambda{{\etens{\Lambda}}}
\def\etA{{\etens{A}}}
\def\etB{{\etens{B}}}
\def\etC{{\etens{C}}}
\def\etD{{\etens{D}}}
\def\etE{{\etens{E}}}
\def\etF{{\etens{F}}}
\def\etG{{\etens{G}}}
\def\etH{{\etens{H}}}
\def\etI{{\etens{I}}}
\def\etJ{{\etens{J}}}
\def\etK{{\etens{K}}}
\def\etL{{\etens{L}}}
\def\etM{{\etens{M}}}
\def\etN{{\etens{N}}}
\def\etO{{\etens{O}}}
\def\etP{{\etens{P}}}
\def\etQ{{\etens{Q}}}
\def\etR{{\etens{R}}}
\def\etS{{\etens{S}}}
\def\etT{{\etens{T}}}
\def\etU{{\etens{U}}}
\def\etV{{\etens{V}}}
\def\etW{{\etens{W}}}
\def\etX{{\etens{X}}}
\def\etY{{\etens{Y}}}
\def\etZ{{\etens{Z}}}



\newcommand{\pdata}{p_{\rm{data}}}
\newcommand{\ptrain}{\hat{p}_{\rm{data}}}
\newcommand{\Ptrain}{\hat{P}_{\rm{data}}}
\newcommand{\pmodel}{p_{\rm{model}}}
\newcommand{\Pmodel}{P_{\rm{model}}}
\newcommand{\ptildemodel}{\tilde{p}_{\rm{model}}}
\newcommand{\pencode}{p_{\rm{encoder}}}
\newcommand{\pdecode}{p_{\rm{decoder}}}
\newcommand{\precons}{p_{\rm{reconstruct}}}

\newcommand{\laplace}{\mathrm{Laplace}} 

\newcommand{\E}{\mathbb{E}}
\newcommand{\Ls}{\mathcal{L}}
\newcommand{\R}{\mathbb{R}}
\newcommand{\emp}{\tilde{p}}
\newcommand{\lr}{\alpha}
\newcommand{\reg}{\lambda}
\newcommand{\rect}{\mathrm{rectifier}}
\newcommand{\softmax}{\mathrm{softmax}}
\newcommand{\sigmoid}{\sigma}
\newcommand{\softplus}{\zeta}
\newcommand{\KL}{D_{\mathrm{KL}}}
\newcommand{\Var}{\mathrm{Var}}
\newcommand{\standarderror}{\mathrm{SE}}
\newcommand{\Cov}{\mathrm{Cov}}
\newcommand{\normLzero}{L^0}
\newcommand{\normLone}{L^1}
\newcommand{\normLtwo}{L^2}
\newcommand{\normLp}{L^p}
\newcommand{\normLmax}{L^\infty}

\newcommand{\normlzero}{\ell^0}
\newcommand{\normlone}{\ell^1}
\newcommand{\normltwo}{\ell^2}
\newcommand{\normlp}{\ell^p}
\newcommand{\normlmax}{\ell^\infty}

\newcommand{\parents}{Pa} 

\DeclareMathOperator*{\argmax}{arg\,max}
\DeclareMathOperator*{\argmin}{arg\,min}

\DeclareMathOperator{\sign}{sign}
\DeclareMathOperator{\Tr}{Tr}
\let\ab\allowbreak

\newcommand{\abs}[1]{\left\vert#1\right\rvert}
\newcommand{\norm}[1]{\left\Vert#1\right\Vert}
\newcommand{\cbr}[1]{\left\{#1\right\}}
\newcommand{\br}[1]{\left(#1\right)}
\newcommand{\sbr}[1]{\left[#1\right]}
\newcommand{\given}{\,|\,}

\newcommand\indep{\protect\mathpalette{\protect\independenT}{\perp}}
\def\independenT#1#2{\mathrel{\rlap{}\mkern2mu{#1#2}}}

\newcommand{\zd}[1]{\textcolor{blue}{[zhengdong: #1]}}
\newcommand{\hj}[1]{\textcolor{cyan}{[huangjie: #1]}}
\newcommand{\mz}[1]{{\color{purple}{[mz: #1]}}}
\newcommand{\wz}[1]{{\color{red}{[wz: #1]}}}

\newcommand{\rebuttal}[1]{{\color{blue}{#1}}}

\newcommand{\probx}{\mbox{Prob}(\gX)} 
\usepackage{hyperref}
\usepackage{url}
\usepackage{amsmath}
\usepackage{amssymb}
\usepackage{mathtools}
\usepackage{amsthm}
\usepackage{graphicx}
\usepackage{cleveref}
\usepackage{multirow}
\usepackage[font=small]{caption}

\usepackage{tabularray}
\UseTblrLibrary{booktabs}

\usepackage{color, colortbl}
\definecolor{hl_color}{gray}{0.925}
\newcommand{\redcell}{\cellcolor{red}}

\theoremstyle{plain}
\newtheorem{theorem}{Theorem} \newtheorem{proposition}[theorem]{Proposition}
\newtheorem{lemma}[theorem]{Lemma}
\newtheorem{corollary}[theorem]{Corollary}
\theoremstyle{definition}
\newtheorem{definition}[theorem]{Definition}
\newtheorem{assumption}[theorem]{Assumption}
\theoremstyle{remark}
\newtheorem{remark}[theorem]{Remark}


\title{Diffusion-GAN: Training GANs with Diffusion}

\author{
  Zhendong Wang, Huangjie Zheng, Pengcheng He, Weizhu Chen, Mingyuan Zhou \\
  The University of Texas at Austin, Microsoft Azure AI \\ \textit{\{zhendong.wang, huangjie.zheng\}@utexas.edu, \{penhe,wzchen\}@microsoft.com} \\ 
  \textit{mingyuan.zhou@mccombs.utexas.edu }
}


\newcommand{\fix}{\marginpar{FIX}}
\newcommand{\new}{\marginpar{NEW}}

\iclrfinalcopy \begin{document}


\maketitle


\begin{abstract}



Generative adversarial networks (GANs) are challenging to train stably, and a promising remedy of injecting instance noise into the discriminator input has not been very effective in practice.  In this paper, we propose Diffusion-GAN, a novel GAN framework that leverages a forward diffusion chain to generate Gaussian-mixture distributed instance noise. Diffusion-GAN consists of three components, including an adaptive diffusion process, a diffusion timestep-dependent discriminator, and a generator. Both the observed and generated data are diffused by the same adaptive diffusion process. At each diffusion timestep, there is a different noise-to-data ratio and the timestep-dependent discriminator learns to distinguish the diffused real data from the diffused generated data. The generator learns from the discriminator's feedback by backpropagating through the forward diffusion chain, whose length is adaptively adjusted to balance the noise and data levels. We theoretically show that the discriminator's timestep-dependent strategy gives consistent and helpful guidance to the generator, enabling it to match the true data distribution. We demonstrate the advantages of Diffusion-GAN over strong GAN baselines on various datasets, showing that it can produce more realistic images with higher stability and data efficiency than state-of-the-art GANs.

\end{abstract}


\section{Introduction} \label{sec:intro}

Generative adversarial networks (GANs) \citep{goodfellow2014generative} and their variants \citep{brock2018large,karras2019style,karras2020training,zhao2020differentiable} have achieved great success in synthesizing photo-realistic high-resolution 
images.  
GANs in practice, however, are known to suffer from a variety of issues ranging from non-convergence and training instability to mode collapse \citep{ Arjovsky2017TowardsPM,mescheder2018training}.
As a result, a wide array of analyses and modifications has been proposed for GANs, including 
improving the network architectures \citep{karras2019style,Radford2016UnsupervisedRL,sauer2021projected,zhang2019self}, gaining theoretical understanding of GAN training \citep{Arjovsky2017TowardsPM, heusel2017gans, Mescheder2017TheNO,mescheder2018training}, changing the objective functions \citep{arjovsky2017wasserstein,bellemare2017cramer,deshpande2018generative,li2017mmd,nowozin2016f-gan,zheng2021exploiting,yang2021data}, regularizing the weights and/or gradients \citep{arjovsky2017wasserstein,fedus2018many,mescheder2018training,Miyato2018SpectralNF,roth2017stabilizing,salimans2016improved}, utilizing side information \citep{wang2018high,zhang2017stackgan,Zhang2020Variational}, adding a mapping from the data to latent representation \citep{donahue2016adversarial,dumoulin2016adversarially,li2017alice}, and applying differentiable data augmentation \citep{karras2020training,Zhang2020Consistency,zhao2020differentiable}. 


A simple technique to stabilize GAN training is to inject instance noise, , to add noise to the discriminator input, which can widen the support of both the generator and discriminator distributions and prevent the discriminator from overfitting \citep{Arjovsky2017TowardsPM,sonderby2017amortised}. 
However, this technique is hard to implement in practice, as finding a suitable noise distribution is challenging \citep{Arjovsky2017TowardsPM}. \citet{roth2017stabilizing} show that adding instance noise to the high-dimensional discriminator input does not work well, and propose to approximate it by adding a zero-centered gradient penalty on the discriminator. This approach is theoretically and empirically shown to converge in \citet{mescheder2018training}, who also demonstrate that adding zero-centered gradient penalties to non-saturating GANs can result in stable training and better or comparable generation quality compared to WGAN-GP \citep{arjovsky2017wasserstein}. However, \citet{brock2018large} caution that zero-centered gradient penalties and other similar regularization methods may stabilize training at the cost of generation performance. To the best of our knowledge,  there has been no existing work that is able to empirically demonstrate the success of using instance noise in GAN training on high-dimensional image data.

To inject proper instance noise
that can facilitate
GAN training, we introduce Diffusion-GAN, which uses a diffusion process to generate Gaussian-mixture distributed instance noise. We show a graphical representation of Diffusion-GAN in \Cref{fig:cda_diagram}. In Diffusion-GAN, the input to the diffusion process is either a real or a generated image, and the diffusion process consists of a series of steps that gradually add noise to the image. The number of diffusion steps is not fixed, but depends on the data and the generator. We also design the diffusion process to be differentiable, which means that we can compute the derivative of the output with respect to the input. This allows us to propagate the gradient from the discriminator to the generator through the diffusion process, and update the generator accordingly. Unlike vanilla GANs, which compare the real and generated images directly, Diffusion-GAN compares the noisy versions of them, which are obtained by sampling from the Gaussian mixture distribution over the diffusion steps, {with the help of our timestep-dependent discriminator.} This distribution has the property that its components have different noise-to-data ratios, which means that some components add more noise than others. By sampling from this distribution, we can achieve two benefits: first, we can stabilize the training by  easing the problem of vanishing gradient, which occurs when the data and generator distributions are too different; second, we can augment the data by creating different noisy versions of the same image, which can improve the data efficiency and the diversity of the generator. We provide a theoretical analysis to support our method, and show that the min-max objective function of Diffusion-GAN, which measures the difference between the data and generator distributions, is continuous and differentiable everywhere. This means that the generator in theory can always receive a useful gradient from the discriminator, and improve its performance.


\begin{figure}[!t]
    \centering
    \includegraphics[width=.7\textwidth]{figs/new_depict.pdf}
     \vspace{-3mm}
    \caption{Flowchart for Diffusion-GAN. The top-row images represent the forward diffusion process of a real image, while the bottom-row images represent the forward diffusion process of a generated fake image. The discriminator learns to distinguish a diffused real image from a diffused fake image at all diffusion steps. }
    \label{fig:cda_diagram}
    \vspace{-4mm}
\end{figure}


Our main contributions include: 1) We show both theoretically and empirically how the diffusion process can be utilized to provide a model- and domain-agnostic differentiable augmentation, enabling data-efficient and leaking-free stable GAN training. 
2) 
Extensive experiments show
that Diffusion-GAN 
boosts the stability and generation performance of  strong baselines, 
including StyleGAN2 \citep{karras2020analyzing},  Projected GAN \citep{sauer2021projected}, and InsGen \citep{yang2021data},  achieving state-of-the-art results in synthesizing 
photo-realistic images, as measured by both the Fr\'echet Inception Distance (FID) \citep{heusel2017gans} and Recall score \citep{kynkaanniemi2019improved}.  






\section{Preliminaries: GANs and diffusion-based generative models}
GANs \citep{goodfellow2014generative} are a class of generative models that aim to learn the data distribution  of a target dataset 
by setting up a min-max game between two neural networks: a generator and a discriminator. The generator  takes as input a random noise vector  sampled from a simple prior distribution , such as a standard normal or uniform distribution, and tries to produce realistic-looking samples  that resemble the data. The discriminator  receives either a real data sample  drawn from  or a fake sample  generated by , and tries to correctly classify them as real or fake. The goal of  is to fool  into making mistakes, while the goal of  is to accurately distinguish  from . The min-max objective function of GANs is given by

In practice, this vanilla objective function is often modified to improve the stability and performance of GANs\citep{goodfellow2014generative,Miyato2018SpectralNF,fedus2018many}, but the general idea of adversarial learning between  and  remains the same.



Diffusion-based generative models \citep{ddpm,SohlDickstein2015DeepUL,scorematching} 
assume
, where  are latent variables of the same dimensionality as the data . There is a forward diffusion chain that gradually adds noise to the data  in  steps with pre-defined variance schedule  and variance :

A notable property 
is that  at an
arbitrary time-step  can be sampled in closed form as 


A variational lower bound \citep{blei2017variational} is then used to optimize 
the reverse diffusion chain as 


\section{Diffusion-GAN: Method and Theoretical Analysis}\label{sec:methodology}

\label{sec:method}


To construct Diffusion-GAN, we describe how to inject instance noise via diffusion, how to train the generator by backpropagating through the forward diffusion process, and how to adaptively adjust the diffusion intensity. We further provide theoretical analysis illustrated with a toy example.  
\subsection{Instance noise injection via diffusion} 

We aim to generate realistic samples  from a generator network  that maps a latent variable  sampled from a simple prior distribution  to a high-dimensional data space, such as images. The distribution of generator samples  ,  is denoted by . To make the generator more robust and diverse, we inject instance noise into the generated samples  by applying a diffusion process that adds Gaussian noise at each step. The diffusion process can be seen as a Markov chain that starts from the original sample  and gradually erases its information until reaching a noise level  after  steps. 


We define a mixture distribution  that models the noisy samples  obtained at any step of the diffusion process, with a mixture weight  for each step . The mixture components  are Gaussian distributions with mean proportional to  and variance depending on the noise level at step . We use the same diffusion process and mixture distribution for both the real samples  and the generated samples . More specifically, the diffusion-induced mixture distributions are expressed as 
where   is a -component mixture distribution, the mixture weights  are non-negative and sum to one, and the mixture components  are obtained via diffusion as in  \Cref{eq:samplext}, expressed as

Samples from this mixture can be drawn as . 

By sampling  from this mixture distribution, we can obtain noisy versions of both real and generated samples with varying degrees of noise. The more steps we take in the diffusion process, the more noise we add to  and the less information we preserve from . We can then use this diffusion-induced mixture distribution to train a timestep-dependent discriminator  that distinguishes between real and generated noisy samples, and a generator  that matches the distribution of generated noisy samples to the distribution of real noisy samples. Next we introduce Diffusion-GAN that trains its discriminator and generator with the help of the diffusion-induced mixture distribution.



\subsection{Adversarial Training} 

The Diffusion-GAN trains its generator and discriminator by solving a min-max game objective as 

Here,  is the true data distribution,  is a discrete distribution that assigns different weights  to each diffusion step , and  is the conditional distribution of the perturbed sample  given the original data  and the diffusion step . {By \Cref{eq:qy_xt}, with Gaussian reparameterization, the perturbation function could be written as , where  is the cumulative noise level at step },  is a scale factor, and  is a Gaussian noise. 

The objective function in \Cref{eq:loss} encourages the discriminator to assign high probabilities to the perturbed real data and low probabilities to the perturbed generated data, for any diffusion step~. The generator, on the other hand, tries to produce samples that can deceive the discriminator at any diffusion step . Note that the perturbed generated sample  can be rewritten as . This means that the objective function in \Cref{eq:loss} is differentiable with respect to the generator parameters, and we can use gradient descent to optimize it with back-propagation.



The objective function \Cref{eq:loss} is similar to the one used by the original GAN \citep{goodfellow2014generative}, except that it involves the diffusion steps and the perturbation functions. We can show that this objective function also minimizes an approximation of the \textit{Jensen--Shannon (JS) divergence} between the true and the generated distributions, but with respect to the perturbed samples and the diffusion steps, as follows:

The JS divergence measures the dissimilarity between two probability distributions, and it reaches its minimum value of zero when the two distributions are identical. The proof of the equality in \Cref{eq:loss_jsd} is given in \Cref{sec:appendix_derivation}. A natural question that arises from this result is whether minimizing the JS divergence between the perturbed distributions implies minimizing the JS divergence between the original distributions, , whether the optimal generator for \Cref{eq:loss} is also the optimal generator for . We will answer this question affirmatively and provide a theoretical justification in \Cref{sec:theory}.




\subsection{Adaptive diffusion} 

{With the help of the perturbation function and timestep dependency, we have a new strategy to optimize the discriminator.}
We want the discriminator  to have a challenging task, {neither too easy to allow overfitting the data
\citep{karras2020training, zhao2020differentiable} nor too hard to impede learning}. Therefore, we adjust the intensity of the diffusion process, which adds noise to both  and , depending on how much  can distinguish them. When the diffusion step  is larger, the noise-to-data ratios are higher and the task is harder. We use {} to measure the intensity of the diffusion, which increases as  grows. To control the diffusion intensity, we  
adaptively modify the maximum number of steps .


Our strategy is to make the discriminator learn from the easiest samples first, which are the original data samples, and then gradually increase the difficulty by feeding it samples from larger . To do this, we use a self-paced schedule for , which depends on a metric  that estimates how much the discriminator overfits to the data:

where  is the same as in \citet{karras2020training} and  is a constant. We calculate  and update  every four minibatches.
We have two options for the distribution  that we use to sample  for the diffusion process:

The `priority' option gives more weight to larger , which means the discriminator will see more new samples from the new steps when  increases. This is because we want the discriminator to focus on the new and harder samples that it has not seen before, as this indicates that it is confident about the easier ones. Note that even with the `priority' option, the discriminator can still see samples from smaller , because  is a mixture of Gaussians that covers all steps of the diffusion chain.




To avoid sudden changes in  during training, we use an exploration list  that contains  values sampled from . We keep  fixed until we update , and we sample  from  to generate noisy samples for the discriminator. This way, the model can explore each  sufficiently before moving to a higher . We give the details of training Diffusion-GAN in \Cref{alg:algo} in \Cref{sec:appendix_algo}.


\subsection{Theoretical analysis with Examples} 
\label{sec:theory}



To better understand the theoretical properties of our proposed method, we present two theorems that address two important questions about the use of diffusion-based instance noise injection for training GANs. The proofs of these theorems are deferred to \Cref{sec:appendix_proof}. The first question, denoted as \textbf{(a)}, is whether adding noise to the real and generated samples in a diffusion process can facilitate the learning.
The second question, denoted as \textbf{(b)}, is whether minimizing the JS divergence between the joint distributions of the noisy samples and the noise levels,  and , can lead to the same optimal generator as minimizing the JS divergence between the original distributions of the real and generated samples,  and . 

To answer \textbf{(a)}, we prove that for any choice of noise level  and any choice of convex function , the -divergence \citep{nowozin2016f-gan} between the marginal distributions of the noisy real and generated samples,  and , is a smooth function that can be computed and optimized by the discriminator. This implies that the diffusion-based noise injection does not introduce any singularity or discontinuity in the objective function of the GAN. The JS divergence is a special case of -divergence, where .


\begin{theorem}[Valid gradients anywhere for GANs training] \label{theorem:d-jsd}
Let  be a fixed distribution over  and  be a random noise over another space . Denote  as a function with parameter  and input  and   as the distribution of . Let , where  and .
Let  and . Then, , if function  is continuous and differentiable,  the f-divergence  is continuous and differentiable with respect to . 
\end{theorem}

\Cref{theorem:d-jsd} shows that with the help of diffusion noise injection by , ,  and  are defined on the same support space, the whole , and  is continuous and differentiable everywhere. Then, one natural question is what if  keeps a near constant value and hence provides little useful gradient. 
Hence, we empirically show that by injecting noise through a mixture defined over all steps of the diffusion chain, there is always a good chance that a sufficiently large  is sampled to provide a useful gradient, via the toy example below. 

\textbf{Toy example. }
We use the same simple example from \citet{arjovsky2017wasserstein} to illustrate our method. Let  be the real data and  be the data generated by a one-parameter generator, where  is a uniform random variable in . The JS divergence between the real and the generated distributions, , is discontinuous: it is 0 when  and  otherwise, so it does not provide a useful gradient to guide  towards zero.




We introduce diffusion-based noise to both the real and the generated data, as shown in the first row of \Cref{fig:wgan_example}. The noisy data,  and , have supports that cover the whole space  and their densities overlap more or less depending on the diffusion step . In the second row, left, of \Cref{fig:wgan_example}, we plot how the JS divergence between the noisy distributions, , varies with  for different  values. The black line with  is the original JS divergence, which has a discontinuity at . As  increases, the JS divergence curves become smoother and have non-zero gradients for a larger range of . However, some values of , such as  in this example, still have flat regions where the JS divergence is nearly constant. To avoid this, we use a mixture of all steps to ensure that there is always a high chance of getting informative gradients.





For the discriminator optimization, as shown in the second row, right, of \Cref{fig:wgan_example}, the optimal discriminator under the original JS divergence is discontinuous and unattainable. With diffusion-based noise, the optimal discriminator changes with : a smaller  makes it more confident and a larger  makes it more cautious. Thus the diffusion acts like a scale to balance the power of the discriminator. This suggests the use of a differentiable forward  diffusion chain that can provide various levels of gradient smoothness to help the generator training.




\begin{theorem}[Non-leaking {noise injection}] \label{theorem:equality}
Let  and , where  is the transition density. Given certain , if  could be reparameterized into , where  is a known distribution, and both  and  are one-to-one mapping functions, then we could have .
\end{theorem}

To answer question \textbf{(b)}, we present \Cref{theorem:equality}, which shows a sufficient condition for the equality of the original and the augmented data distributions. 
{By \Cref{theorem:equality}, the function  maps each  to a unique , the function  maps each  to a unique noise term, and the distribution of  is known and independent of .} Under these assumptions, the theorem proves that the distribution of  is the same as the distribution of , if and only if the distribution of  is the same as the distribution of . {If we take  as the  introduced in the theorem, then for , \Cref{eq:qy_xt} fits the assumption made.} This means that, by minimizing the divergence between  and , which is the same as minimizing the divergence between  and , we are also minimizing the divergence between  and . This implies that the noise injection does not affect the quality of the generated samples, and we can safely use {our noise injection} to improve the training of the generative model.





\begin{figure}[!t]
    \centering
    \includegraphics[width=0.96\textwidth]{figs/scatter_t_old.pdf}\\
    \includegraphics[width=0.48\textwidth]{figs/toy_jsd_1.pdf}
    \includegraphics[width=0.48\textwidth]{figs/toy_jsd_3.pdf} 
    \vspace{-2mm}
    \caption{The toy example inherited from \citet{arjovsky2017wasserstein}. The first row plots the distributions of data with diffusion noise injected for . The second row shows the JS divergence and the optimal discriminator value with and without our noise injection. }
    \label{fig:wgan_example}
    \vspace{-2mm}
\end{figure}

\subsection{Related work}\label{sec:related_work}
The proposed Diffusion-GAN can be related to previous works on stabilizing the GAN training, building diffusion-based generative models, and constructing differential augmentation for data-efficient GAN training. A detailed discussion on these related works is deferred to Appendix \ref{appendix:related_work}.


\section{Experiments} \label{sec:experiment}

We conduct extensive experiments to answer the following questions: \textbf{(a)} Will Diffusion-GAN outperform state-of-the-art GAN baselines on benchmark datasets?
\textbf{(b)} Will the diffusion-based noise injection help the learning of GANs in domain-agnostic tasks?
\textbf{(c)} Will our method improve the performance of data-efficient GANs trained with a very limited amount of data?

\textbf{Datasets. } We conduct experiments on image datasets ranging from low-resolution (, ) to high-resolution (, ) and from low-diversity to high-diversity: CIFAR-10 \citep{krizhevsky2009learning}, STL-10 \citep{coates2011analysis}, LSUN-Bedroom \citep{yu2015lsun}, LSUN-Church \citep{yu2015lsun}, AFHQ(Cat/Dog/Wild) \citep{Choi2020StarGANVD}, and FFHQ \citep{karras2019style}. More details on these benchmark datasets are provided in \Cref{sec:data}.



\textbf{Evaluation protocol. } We measure image quality using FID \citep{heusel2017gans}. Following \citet{karras2019style, karras2020analyzing}, we measure FID using 50k generated samples, with the full training set used as reference. 
We use the number of real images shown to the discriminator to evaluate convergence \citep{karras2020training,sauer2021projected}. Unless  specified otherwise, all models are trained with 25 million images to ensure convergence (these trained with more or fewer images are specified in table captions).
We further report the improved \textit{Recall} score introduced by 
\citet{kynkaanniemi2019improved} to measure the sample diversity of generative models. 

\textbf{Implementations and resources. } We build Diffusion-GANs based on the code of StyleGAN2 \citep{karras2020analyzing}, ProjectedGAN \citep{sauer2021projected}, and InsGen \citep{yang2021data} to answer questions \textbf{(a)}, \textbf{(b)}, and \textbf{(c)}, respectively.
Diffusion GANs inherit from their corresponding base GANs all their network architectures and training hyperparamters, whose details are provided in \Cref{sec:hyperparameter}.  
Specifically for StyleGAN2 and InsGen, we construct the discriminator as , where  is injected via their mapping network. 
For ProjectedGAN, we empirically find  in the discriminator could be ignored to simplify the implementation and minimize the modifications to ProjectedGAN. More implementation details are provided in \Cref{sec:implementation}.
By applying our diffusion-based noise injection, we denote our models as Diffusion StyleGAN2/ProjectedGAN/InsGen. 
In the following experiments, we train related models with their official code if the results are unavailable, while others are all reported from references and marked with .
We run all our experiments with either 4 or 8 NVIDIA V100 GPUs depending on the demands of the inherited training configurations. 


\subsection{Comparison to state-of-the-art GANs} \label{sec:experiment_sota}

\begin{table}[!t]
\caption{\small Image generation results on benchmark datasets: CIFAR-10, CelebA, STL-10, LSUN-Bedroom, LSUN-Church, and FFHQ. We highlight the best and second best results in each column with  and , respectively. Lower FIDs indicate better fidelity, while higher Recalls indicate better diversity.}
    \vspace{-2mm}
    \label{tab:main_results}
    \centering
    \resizebox{\textwidth}{!}{
    \begin{tblr}{
      colspec = {l *{6}{*{2}{c}}},
      cell{1}{1} = {r=3}{c},
      cell{1}{2, 4, 6, 8, 10, 12} = {c=2}{c},
      cell{2}{2, 4, 6, 8, 10, 12} = {c=2}{c},
    }
    \toprule
    \textbf{Methods}  & \textbf{CIFAR-10} & & \textbf{CelebA} & & \textbf{STL-10}  & & \textbf{LSUN-Bedroom} & & \textbf{LSUN-Church} & & \textbf{FFHQ} & \\
    & {()} & & {()}   &  &  {()}  & & {()} & & {()} & & {()} & \\
    & FID & Recall & FID & Recall & FID & Recall & FID & Recall & FID & Recall & FID & Recall \\
    \midrule
    StyleGAN2 \citep{karras2020training}  &  8.32 & 0.41 & \underline{2.32} & \underline{0.55} & \underline{11.70} & \underline{0.44} & \underline{3.98} & \underline{0.32} & \underline{3.93} & \underline{0.39} & \underline{4.41} & \underline{0.42} \\
    StyleGAN2 + DiffAug \citep{zhao2020differentiable}  &  5.79 & 0.42 & 2.75 & 0.52 & 12.97 & 0.39 & 4.25 & 0.19 & 4.66 & 0.33 & 4.46 & 0.41 \\
    StyleGAN2 + ADA  \citep{karras2020training}  &  \textbf{2.92} & \underline{0.49} & 2.49 & 0.53 & 13.72 & 0.36 & 7.89 & 0.05 & 4.12 & 0.18 & 4.47 & 0.41 \\
    Diffusion StyleGAN2  &  \underline{3.19} & \textbf{0.58} & \textbf{1.69} & \textbf{0.67} & \textbf{11.43} & \textbf{0.45} & \textbf{3.65} & \textbf{0.32} & \textbf{3.17} & \textbf{0.42} & \textbf{2.83} & \textbf{0.49} \\
    \bottomrule
    \end{tblr}}
\end{table}

\begin{figure}[!t]
    \centering
    \begin{subfigure}[b]{0.16\textwidth}
        \includegraphics[width=\textwidth]{figs/cifar_h.jpg}
        \caption{\small CIFAR-10}
    \end{subfigure}
    \begin{subfigure}[b]{0.40\textwidth}
        \includegraphics[width=\textwidth]{figs/celeba_h.jpg}
        \caption{\small CelebA}
    \end{subfigure}
    \begin{subfigure}[b]{0.40\textwidth}
        \includegraphics[width=\textwidth]{figs/stl_h.jpg}
        \caption{\small STL-10}
    \end{subfigure} \\
    \begin{subfigure}[b]{0.482\textwidth}
       \includegraphics[width=\textwidth]{figs/bedroom_h.jpg}
        \caption{\small LSUN-Bedroom}
    \end{subfigure}
    \begin{subfigure}[b]{0.482\textwidth}
       \includegraphics[width=\textwidth]{figs/church_h.jpg}
        \caption{\small LSUN-Church}
    \end{subfigure} \\
    \begin{subfigure}[b]{0.97\textwidth}
       \includegraphics[width=\textwidth]{figs/ffhq_h.jpg}
        \caption{\small FFHQ}
    \end{subfigure}
     \vspace{-1mm}
    \caption{\small Randomly generated images from Diffusion StyleGAN2 trained on CIFAR-10, CelebA, STL-10, LSUN-Bedroom, LSUN-Church, and FFHQ datasets. }
    \vspace{-2mm}
    \label{fig:show_image}
\end{figure}


We compare Diffusion-GAN with its state-of-the-art GAN backbone, StyleGAN2 \citep{karras2020training}, and to evaluate its effectiveness from the data augmentation perspective, we compare it with both StyleGAN2 + DiffAug \citep{zhao2020differentiable} and StyleGAN2 + ADA \citep{karras2020training}, in terms of both sample fidelity (FID) and sample diversity (Recall) over extensive benchmark datasets. 

We present the quantitative and qualitative results in \Cref{tab:main_results} and \Cref{fig:show_image}. Qualitatively, these generated images from Diffusion StyleGAN2 are all photo-realistic and have good diversity, ranging from low-resolution (32 
32) to high-resolution (1024  1024). Additional randomly generated images  can be found in \Cref{sec:gen_images}. Quantitatively, Diffusion StyleGAN2 outperforms all the GAN baselines in generation diversity, as measured by Recall, on all 6 benchmark datasets and outperforms them in FID by a clear margin on 5 out of the 6 benchmark datasets. 



From the data augmentation perspective, we observe that Diffusion StyleGAN2 always clearly outperforms the backbone model StyleGAN2 across various datasets, which empirically validates our \Cref{theorem:equality}. By contrast, both the ADA \citep{karras2020analyzing} and Diffaug \citep{zhao2020differentiable} techniques could sometimes impair the generation performance on sufficiently large datasets, , LSUN-Bedroom and LSUN-Church, which is also observed by \citet{yang2021data} on FFHQ. This is possibly because their risk of leaking augmentation overshadows the benefits of data augmentation.


To investigate how the adaptive diffusion process works during training, we illustrate in \Cref{fig:covergence_pg} the convergence of the maximum timestep  in our adaptive diffusion and discriminator outputs. We see that  is adaptively adjusted: The  for Diffusion StyleGAN2 increases as the training goes while the  for Diffusion ProjectedGAN first goes up and then goes down. Note that the  is adjusted according to the overfitting status of the discriminator. The second panel shows that trained with the diffusion-based mixture distribution, the discriminator is always well behaved and provides useful learning signals for the generator, which validates our analysis in \Cref{sec:theory} and \Cref{theorem:d-jsd}. 

{\textbf{Memory and time costs. } Generally speaking, the memory and time costs of a Diffusion-GAN are comparable to those of the corresponding GAN baseline. More specifically, switching from ADA  \citep{karras2020training} to our diffusion-based augmentation, the added memory cost is negative, the added training time cost is negative, and the added inference time cost is zero. For example, for CIFAR-10, with four NVIDIA V100 GPUs, the training time for each 4k images is around 8.0s for StyleGAN2, 9.8s for StyleGAN2-ADA, and 9.5s for Diffusion-StyleGAN2.}


\begin{figure}[t]
    \centering
\includegraphics[width=0.34\textwidth]{figs/T_cda_projected_gan.pdf} ~~~~~~~~~~~~~~~
    \includegraphics[width=0.34\textwidth]{figs/D_cifar10_cda_stylegan2.pdf}
    \vspace{-1mm}
    \caption{\small Plot of adaptively adjusted maximum diffusion steps  and discriminator outputs of Diffusion-GANs.}
    \label{fig:covergence_pg}
    \vspace{-2mm}
\end{figure}




\subsection{Effectiveness of Diffusion-GAN for domain-agnostic augmentation} \label{sec:domain_agnostic_exp}

To verify whether our method is domain-agnostic, we apply Diffusion-GAN onto the input feature vectors of GANs. We conduct experiments on both low-dimensional and high-dimensional feature vectors, for which commonly used image augmentation methods are no longer applicable.

\textbf{25-Gaussians Example. } We conduct experiments on the popular 25-Gaussians generation task. The 25-Gaussians dataset is a 2-D toy data, generated by a mixture of 25 two-dimensional Gaussian distributions.  Each data point is a 2-dimensional feature vector.  We train a small GAN model, whose generator and discriminator are both parameterized by multilayer perceptrons (MLPs), with two 128-unit hidden layers and LeakyReLu nonlinearities. 

The training results are shown in \Cref{fig:toy_examples}. We observe that the vanilla GAN exhibits severe mode collapsing, capturing only a few modes. Its discriminator outputs of real and fake samples depart from each other very quickly. This implies a strong overfitting of the discriminator happened so that the discriminator stops providing useful learning signals for the generator. However, Diffusion-GAN successfully captures all the 25 Gaussian modes and the discriminator is under control to continuously provide useful learning signals. We interpret the improvement from two perspectives: First, non-leaking augmentation helps provide more information about the data space; 
Second, the discriminator is well behaved given the adaptively adjusted diffusion-based noise injection. 

\begin{figure}[ht]
    \centering
    \includegraphics[width=0.19\textwidth]{figs/Diffused_by_T=0.pdf}
    \includegraphics[width=0.19\textwidth]{figs/nocda_samples.pdf}
    \includegraphics[width=0.19\textwidth]{figs/nocda_metrics.pdf}
    \includegraphics[width=0.19\textwidth]{figs/cda_samples.pdf}
    \includegraphics[width=0.19\textwidth]{figs/cda_metrics.pdf}
    \vspace{-1mm}
    \caption{ The 25-Gaussians example. We show the true data samples, the generated samples from vanilla GANs, the discriminator outputs of the vanilla GANs, the generated samples from our Diffusion-GAN, and the discriminator outputs of Diffusion-GAN.}
    \label{fig:toy_examples}
\end{figure}

\textbf{ProjectedGAN. } To verify that our adaptive diffusion-based noise injection could benefit the learning of GANs on high-dimensional feature vectors, we directly apply it to the discriminator feature space of ProjectedGAN \citep{sauer2021projected}. ProjectedGANs generally leverage pre-trained neural networks to extract meaningful features for the adversarial learning of the discriminator and generator. 
Following \citet{sauer2021projected}, we adaptively diffuse the feature vectors extracted by EfficientNet-v0 and keep all the other training parts unchanged. We report the performance of Diffusion ProjectedGAN on several benchmark datasets in \Cref{tab:projected_gan}, which verifies that our augmentation method is domain-agnostic. Under the ProjectedGAN framework, we see that with noise properly injected into the high-dimensional feature space, Diffusion ProjectedGAN shows clear improvement in terms of both FID and Recall. 
We reach state-of-the-art FID results with Diffusion ProjectedGAN on STL-10 and LSUN-Bedroom/Church datasets.


\begin{table}[ht]
    \caption{Domain-agnostic experiments on ProjectedGAN. }
    \vspace{-2mm}
    \label{tab:projected_gan}
    \centering
    \resizebox{\textwidth}{!}{
    \begin{tblr}{
      colspec = {l *{4}{*{2}{c}}},
      cell{1}{1} = {r=2}{c},
      cell{1}{2, 4, 6, 8} = {c=2}{c},
}
    \toprule
    \textbf{Domain-agnostic Tasks} & \textbf{CIFAR-10 {()}} & & \textbf{STL-10 {()}}  & & \textbf{LSUN-Bedroom {()}} & & \textbf{LSUN-Church {()}} & \\
    & FID & Recall & FID & Recall & FID & Recall & FID & Recall \\
    \midrule
    ProjectedGAN \citep{sauer2021projected} & 3.10 & 0.45 & 7.76 & 0.35 & 2.25 & 0.55 & 3.42 & 0.56 \\
    Diffusion ProjectedGAN & \textbf{2.54} & \textbf{0.45} & \textbf{6.91} & \textbf{0.35} & \textbf{1.43} & \textbf{0.58} & \textbf{1.85} & \textbf{0.65} \\
    \bottomrule
    \end{tblr}}
\end{table}

\subsection{Effectiveness of Diffusion-GAN for limited data}

We evaluate whether Diffusion-GAN can provide data-efficient GAN training. 
We first generate five FFHQ () dataset splits, consisting of 200, 500, 1k, 2k, and 5k images, respectively, where 200 and 500 images are considered to be extremely limited data cases. We also consider AFHQ-Cat, -Dog, and -Wild (512  512), each with as few as around 5k images. Motivated by the success of InsGen \citep{yang2021data} on small datasets, we build our Diffusion-GAN upon it. We note on limited data, InsGen convincingly outperforms both StyleGAN2+ADA and +DiffAug, and currently holds the state-of-the-art performance for data-efficient GAN training. The results in Table~\ref{tab:limited_data2} show that our Diffusion-GAN method can help further boost the performance of InsGen in limited data settings.

\begin{table}[ht]
    \caption{\small {FFHQ} (1024  1024) FID results with 200, 500, 1k, 2k, and 5k training samples; AFHQ (512  512) FID results. To ensure convergence, all models are trained across 10M images for FFHQ and 25M images for AFHQ. We bold the best number in each column. }
    \label{tab:limited_data2}
     \vspace{-2mm}
    \centering
    \resizebox{\textwidth}{!}{
\begin{tblr}{
    colspec = { l c c c c c c c c},
    row{1} = {font=\bfseries}
    }
    \toprule
    \textbf{ Models} & {FFHQ (200)}  & {FFHQ (500)}  & {FFHQ (1k)} & {FFHQ (2k)} & {FFHQ (5k)} & Cat & Dog & Wild \\
    \midrule
    InsGen \citep{yang2021data} & 102.58 & 54.762 & 34.90 & 18.21 & 9.89 & 2.60 & 5.44 & 1.77 \\ 
    Diffusion InsGen & \textbf{63.34} & \textbf{50.39} & \textbf{30.91} & \textbf{16.43} & \textbf{8.48} & \textbf{2.40} & \textbf{4.83} & \textbf{1.51} \\
    \bottomrule
    \end{tblr}}
\end{table}





\section{Conclusion}

We present Diffusion-GAN, a novel GAN framework that uses a variable-length forward diffusion chain with a Gaussian mixture distribution to generate instance noise for GAN training. This approach enables model- and domain-agnostic differentiable augmentation that leverages the advantages of diffusion without requiring a costly reverse diffusion chain. We prove theoretically and demonstrate empirically that Diffusion-GAN can prevent discriminator overfitting and provide non-leaking augmentation. We also demonstrate that Diffusion-GAN can produce high-resolution photo-realistic images with high fidelity and diversity, outperforming its corresponding state-of-the-art GAN baselines on standard benchmark datasets according to both FID and Recall.

\subsection*{Acknowledgements}

Z. Wang, H. Zheng, and M. Zhou acknowledge the support of NSF-IIS 2212418 and IFML.



\bibliography{references.bib}
\bibliographystyle{plainnat}
\newpage
\appendix

\begin{center}{\Large{\textbf{Appendix}}}\end{center}


\section{Related work} \label{appendix:related_work}


\textbf{Stabilizing GAN training. } 

A root cause of training difficulties
in GANs is often attributed to the JS divergence that GANs intend to minimize. 
This is because when the data and generator distributions have non-overlapping supports, which are often the case for high-dimensional data supported by low-dimensional manifolds, the gradient of the JS divergence may provide no useful guidance to optimize the generator \citep{Arjovsky2017TowardsPM,arjovsky2017wasserstein, mescheder2018training, roth2017stabilizing}.  For this reason, \citet{arjovsky2017wasserstein} propose to instead use 
the Wasserstein-1 distance, 
which in theory
can provide useful gradient for the generator even if the two distributions have disjoint supports.
However, Wasserstein GANs often require the use of a critic function under the 1-Lipschitz constraint, which is difficult to satisfy in practice and hence realized with heuristics such as weight clipping \citep{arjovsky2017wasserstein}, gradient penalty \citep{gulrajani2017improved}, and spectral normalization~\citep{Miyato2018SpectralNF}.


While the divergence minimization perspective has played an important role in motivating the construction of Wasserstein GANs and gradient penalty-based regularizations, cautions should be made on purely relying on it 
to understand GAN training, due to not only the discrepancy between the divergence in theory and the actual min-max objective function used in practice, but also the  potential confounding between different divergences and different training  and regularization strategies \citep{fedus2018many,mescheder2018training}. E.g.,   \citet{mescheder2018training} have provided a simple example where in theory the Wasserstein GAN is predicted to succeed while the vanilla GAN is predicted to fail, but in practice the Wasserstein GAN with a finite number of discriminator updates per generator update fails to converge while the vanilla GAN 
with the non-saturating loss can slowly converge. 
\citet{fedus2018many} provide a rich set of empirical evidence to discourage viewing GANs purely from the perspective of minimizing a specific divergence at each training step and emphasize the important role played by gradient penalties on stabilizing GAN training. 




\textbf{Diffusion models. } 
Due to the use of a forward diffusion chain, the proposed Diffusion-GAN can be related to 
diffusion-based (or score-based) deep generative models \citep{ddpm,SohlDickstein2015DeepUL,scorematching} that employ both a forward (inference) and a reverse (generative) diffusion chain.
These diffusion-based generative models  are stable to train and can generate high-fidelity photo-realistic images \citep{Dhariwal2021DiffusionMB,ddpm,nichol2021glide,ramesh2022hierarchical,scorematching,song2021scorebased}. However, they are notoriously slow in generation due to the need to traverse the reverse diffusion chain, which involves 
going through the same U-Net-based generator network hundreds or even thousands of times \citep{song2021denoising}. For this reason, a variety of methods have been proposed to reduce the generation cost of diffusion-based generative models \citep{kong2021fast,luhman2021knowledge,pandey2022diffusevae,san2021noise,song2021denoising,xiao2021tackling,zheng2022truncated}.



A key distinction is that Diffusion-GAN  needs a reverse diffusion chain during neither training nor  generation.
More specifically, its generator  
maps the noise to a generated sample in a single step.
Diffusion-GAN can train and generate as quickly as a vanilla GAN does with the same generator size. 
{For example, it takes around 20 hours to sample 50k images of size 32 × 32 from a DDPM \citep{ddpm} on an Nvidia 2080 Ti GPU, but would take less than a minute to do so from Diffusion-GAN.}

 





\textbf{Differentiable augmentation. }
As Diffusion-GAN 
transforms both the data and generated samples before sending them to the discriminator, we can also  relate it 
to differentiable augmentation  \citep{karras2020training, zhao2020differentiable} 
proposed for data-efficient GAN training. 
\citet{karras2020training} 
introduce a stochastic augmentation pipeline with 18 transformations and develop an adaptive mechanism for controlling the augmentation probability. \citet{zhao2020differentiable} propose to use \textit{Color + Translation + Cutout} as differentiable augmentations for both generated and real images.


While providing good empirical results on some datasets, these augmentation methods are developed with domain-specific knowledge and
have the risk of leaking augmentation into generation \citep{karras2020training}.
As observed in our experiments, they sometime worsen the results when applied to a new dataset, likely because the risk of augmentation leakage overpowers the benefits of enlarging the training set, which could happen especially if the training set size is already sufficiently large.


By contrast,
Diffusion-GAN  
uses a differentiable  forward diffusion process to stochastically transform the data and can be considered as both a domain-agnostic and a model-agnostic augmentation method. In other words, Diffusion-GAN can be applied to non-image data or even latent features, for which appropriate data augmentation is difficult to be defined, and easily plugged into an existing GAN to improve its generation performance. Moreover, 
we prove in theory and show in experiments that 
augmentation leakage is not a concern for
Diffusion-GAN.  
\citet{Tran2021OnDA} provide a theoretical analysis for deterministic non-leaking transformation with differentiable and invertible mapping functions. 
\citet{Bora2018AmbientGANGM} show similar theorems to us for specific stochastic transformations, such as Gaussian Projection, Convolve+Noise, and stochastic Block-Pixels, while our \Cref{theorem:equality} includes more satisfying possibilities as discussed in  \Cref{sec:appendix_proof}. 



\section{Proof} \label{sec:appendix_proof}

\begin{proof}[Proof of \Cref{theorem:d-jsd}]
For simplicity, let , , , ,  and . Then, 




Since  is assumed to be an isotropic Gaussian distribution, for simplicity, in what follows we show the proof in uni-variate Gaussian, which could be easily extended to multi-variate Gaussian by the production rule. We first show that under mild conditions, the  and  are continuous functions over . 

where  and  are constants. Hence,  is a continuous function defined on . The proof of continuity for  is exactly the same proof. Then, given  is also a continuous function, it is clear to see that  is a continuous function over . 

Next, we show that  is differentiable. By the chain rule, showing  to be differentiable is equivalent to show ,  and  are differentiable. Usually,  is defined with differentiability \citep{nowozin2016f-gan}. 


where  and  are constants. Hence,  and  are differentiable, which concludes the proof. 
\end{proof}

\begin{proof}[Proof of \Cref{theorem:equality}] We have  and . \\
 If , then  \\
 Let  and . Given the assumption on , we have


Since  and  are one-to-one mapping functions,  and  are identifiable, which indicates . By the property of moment-generating functions (MGF), given  is independent with , we have for 

where  denotes the MGF of random variable  and the others follow the same form. By the moment-generating function uniqueness theorem, 
given  and , we have . Then, we could obtain . Thus, , which concludes the proof.

\paragraph{Discussion.} Next, we discuss which  fits the assumption we made on it. We follow the discussion of reparameterization of distributions as used in \citet{Kingma2014AutoEncodingVB}. Three basic approaches are:
\begin{enumerate}
    \item Tractable inverse CDF. In this case, let , and  be the inverse CDF of . From , if , for example,  and , then \Cref{theorem:equality} holds.
    \item Analogous to the Gaussian example, . For any ``location-scale'' family of distributions we can choose the standard distribution (with location = 0, scale = 1) as the auxiliary variable , and let . Examples: Laplace, Elliptical, Student’s t, Logistic, Uniform, Triangular, and Gaussian distributions.
    \item Implicit distributions.  could be modeled by neural networks, which implies , where  and  are one-to-one nonlinear transformations. 
\end{enumerate}
\end{proof}


\section{Derivations} \label{sec:appendix_derivation}

\paragraph{Derivation of equality in JSD}




\section{Details of toy example}

Here, we provide the detailed analysis of the JS divergence toy example. 



\paragraph{Notation.} Let  be a compact metric set (such as the space of images ) and  denote the space of probability measures defined on . Let  be the target data distribution and  \footnote{For notation simplicity,  and  both denote the generator network in GANs in this paper.} be the generator distribution. The JSD between the two distributions  is defined as:

where  is the mixture  and  denotes the Kullback-Leibler divergence, , . More generally, the -divergence \citep{nowozin2016f-gan} between  and  is defined as:

where the generator function  is a convex and lower-semicontinuous function satisfying
. We refer to \citet{nowozin2016f-gan} for more details.

We recall the typical example introduced in \citet{Arjovsky2017TowardsPM} and follow the notations.
\paragraph{Example.} Let  be the uniform distribution on the unit interval. Let  be the distribution of , which contains a  on the x-axis and a random variable Z on the y-axis. Let  be the distribution of , where  is a single real parameter. In this case, the  is not continuous, 

which can not provide a usable gradient for training. The derivation is as follows: 


\begin{figure}[!t]
    \centering
    \includegraphics[width=0.96\textwidth]{figs/scatter_t_old.pdf}\\
    \includegraphics[width=0.48\textwidth]{figs/toy_jsd_1.pdf}
    \includegraphics[width=0.48\textwidth]{figs/toy_jsd_3.pdf} 
    \caption{\small We show the data distribution and . }
    \label{fig:wgan_example_old}
\end{figure}

Although this simple example features distributions with disjoint supports, the same conclusion holds when the supports have a non empty intersection contained in a set of measure zero \citep{Arjovsky2017TowardsPM}. This happens to be the case when two low dimensional manifolds intersect in general position \citep{Arjovsky2017TowardsPM}. To avoid the potential issue caused by having non-overlapping distribution supports, a common remedy is to use Wasserstein-1 distance which in theory can still provide usable gradient \citep{Arjovsky2017TowardsPM,arjovsky2017wasserstein}. In this case, the Wasserstein-1 distance is . 



\paragraph{Diffusion-based noise injection} In general, with our diffusion noise injected, we could have,



For the previous example, we have  and  such that, 

where ,  and  are abbreviations for  and . The supports of  and  are both the whole metric space  and they overlap with each other depending on , as shown in \Cref{fig:wgan_example}. As  increases,  the high density region of  and  get closer since the weight  is decreasing towards 0. Then, we derive the JS divergence,

which is clearly continuous and differentiable. 

We show this  with respect to increasing  values and a  grid in the second row of \Cref{fig:wgan_example}. As shown in the left panel, the black line with  shows the origianl JSD, which is not even continuous, while as the diffusion level  increments, the lines become smoother and flatter. It is clear to see that these smooth curves provide good learning signals for . Recall that the Wasserstein-1 distance is  in this case. 
Meanwhile, we could observe with an intense diffusion, , , the curve becomes flatter, which indicates smaller gradients and a much slower learning process. This motivates us that an adaptive diffusion could provide different level of gradient smoothness and is possibly better for training. The right panel shows the optimal discriminator outputs over the space . With diffusion, the optimal discriminator is well defined over the space and the gradient is smooth, while without diffusion the optimal discriminator is only valid on two star points. Interestingly, we find that smaller  drives the optimal discriminator to become more assertive while larger  makes discriminator become more neutral. The diffusion here works like a scale to balance the power of the discriminator. 


\section{Dataset descriptions}\label{sec:data}
The CIFAR-10  dataset consists of 50k  training images in 10 categories. The STL-10  dataset originated from ImageNet \citep{deng2009imagenet} consists of 100k unlabeled images in 10 categories, and we resize them to  resolution. For LSUN datasets, we sample 200k images from LSUN-Bedroom, use the whole 125k images from LSUN-Church, and resize them to  resolution for training. The AFHQ datasets includes around 5k  images per category for dogs, cats, and wild life; we train a separate network for each of them. The FFHQ contains 70k images crawled from Flickr at  resolution and we use all of them for training. 


\section{Algorithm} \label{sec:appendix_algo}

We provide the Diffusion-GAN algorithm in \Cref{alg:algo}.

\begin{algorithm}[t]
  \caption{\small Diffusion-GAN} 
  \label{alg:algo}
  
  \begin{algorithmic}
    \WHILE{i  number of training iterations}
        \STATE \textit{Step I: Update discriminator}
        \begin{itemize}
            \item Sample minibatch of  noise samples .
            \item Obtain generated samples  by .
            \item Sample minibatch of  data examples .
            \item Sample  from  list uniformly with replacement. 
            \item For , sample  and 
            \item Update discriminator by maximizing \Cref{eq:loss}. 
        \end{itemize}
        \STATE
        \STATE \textit{Step II: Update generator}
        \begin{itemize}
            \item Sample minibatch of  noise samples 
            \item Obtain generated samples  by .
            \item Sample  from  list with replacement. 
            \item For , sample 
            \item Update generator by minimizing \Cref{eq:loss}. 
        \end{itemize}
        \STATE
        \STATE \textit{Step III: Update diffusion}
        \IF{i mod 4 == 0}
        \STATE Update T by \Cref{eq:modify_T}
        \STATE Sample , where .  is in \Cref{eq:pi_t}. \COMMENT{ has 64 dimensions.}
        \ENDIF
    \ENDWHILE
  \end{algorithmic}
\end{algorithm}

\section{Hyperparameters} \label{sec:hyperparameter}

{Diffusion-GAN is built on GAN backbones, so we keep the learning hyperparameters of the original GAN backbones untouched. Diffusion-GAN introduces four new hyperparameters: noise standard deviation , ,  increasing threshold , and  sampling distribution .

The  is fixed as 0.05 for images (pixel values rescaled to [-1 ,1]) in all our experiments and it shows good performance.  could be fixed as 500 or 1000, which depends on the diversity of the dataset. We recommend a large  for diverse datasets.  is usually fixed as 0.6, which does not influence much about the performance.  has two choices, `uniform' and `priority'. Generally,  is a good starting point for a new dataset. 
}

{In our experiment,} we find StyleGAN2-based models are not sensitive to the values of , so we set  for them across all dataset, only except that we set  for FFHQ ( for FFHQ is slightly better than  in FID). We report  of Diffusion ProjectedGAN for our experiments in \Cref{tab:d_target}. We also evaluated two  sampling distribution , [`priority', `uniform'], defined in \Cref{eq:pi_t}. In most cases, `priority' works slightly better, while in some cases, such as FFHQ, `uniform' is better. Overall, we didn't modify anything in the model architectures and training hyperparameters, such as learning rate and batch size. The forward diffusion configuration and model training configurations are as follows. 


\paragraph{Diffusion config.} For our diffusion-based noise injection, we set up a linearly increasing schedule for , where . For pixel level injection in StyleGAN2, we follow \citet{ddpm} and set  and . We adaptively modify  ranging from  to . The image pixels are usually rescaled to  so we set the Guassian noise standard deviation . For feature level injection in Diffusion ProjectedGAN, we set , , , , and . We list all these values in \Cref{tab:diffusion_config}

\paragraph{Model config.} For StyleGAN2-based models, we borrow the config settings provided by \citet{karras2020training}, which include [`auto', `stylegan2', `cifar', `paper256', `paper512', `stylegan2']. We create the `stl' config based on `cifar' with a small modification that we change the gamma term to be~0.01. For ProjectedGAN models, we use the recommended default config \citep{sauer2021projected}, which is based on FastGAN \citep{liu2020towards}. We report the config settings  used for our experiments in \Cref{tab:model_config}. 

\begin{table}
    \centering
    \resizebox{0.5\textwidth}{!}{
    \begin{tabular}{l | c}
        \toprule
         Datasets &  \\
         \midrule
         CIFAR-10 (, 50k images) & 0.45 \\
         STL-10 (, 100k images) & 0.6 \\
         LSUN-Church (, 120k images) & 0.2 \\
         LSUN-Bedroom (, 200k images) & 0.2 \\
         \bottomrule
    \end{tabular}}
    \vspace{1mm}
    \caption{\small  for Diffusion ProjectedGAN}
    \label{tab:d_target}
\end{table}

\begin{table}
    \centering
    \resizebox{\textwidth}{!}{
    \begin{tabular}{l|l}
        \toprule
         Diffusion config for pixel, priority &  , , , , \\
         Diffusion config for pixel, uniform &  , , , ,  \\
         \midrule
         Diffusion config for feature & , , , ,   \\
         \bottomrule
    \end{tabular}}
    \vspace{1mm}
    \caption{\small  Diffusion config.}
    \label{tab:diffusion_config}
\end{table}

\begin{table}[t]
    \centering
    \resizebox{\textwidth}{!}{
    \begin{tabular}{l | l | c | c}
        \toprule
         Dataset & Models & Config & Specification  \\
         \midrule
         \multirow{4}{*}{CIFAR-10 } & StyleGAN2  & cifar & - \\
        & Diffusion StyleGAN2    &  cifar & diffusion-pixel, , `priority' \\
        & ProjectedGAN & default & diffusion-feature \\
        & Diffusion ProjectedGAN & default & diffusion-feature \\
        \midrule
         \multirow{4}{*}{STL-10 } & StyleGAN2  & stl & - \\
        & Diffusion StyleGAN2    &  stl & diffusion-pixel, , `priority' \\
        & ProjectedGAN & default & diffusion-feature \\
        & Diffusion ProjectedGAN & default & diffusion-feature \\
        \midrule
         \multirow{4}{*}{LSUN-Bedroom } & StyleGAN2  & paper256 & - \\
        & Diffusion StyleGAN2    &  paper256 & diffusion-pixel, , `priority' \\
        & ProjectedGAN & default & diffusion-feature \\
        & Diffusion ProjectedGAN & default & diffusion-feature \\
        \midrule
        \multirow{4}{*}{LSUN-Church } & StyleGAN2  & paper256 & - \\
        & Diffusion StyleGAN2    &  paper256 & diffusion-pixel, , `priority' \\
        & ProjectedGAN & default & diffusion-feature \\
        & Diffusion ProjectedGAN & default & diffusion-feature \\
        \midrule
        \multirow{4}{*}{AFHQ-Cat/Dog/Wild } & StyleGAN2  & paper512 & - \\
        & Diffusion StyleGAN2    &  paper512 & diffusion-pixel, , `priority' \\
        & InsGen    &  default & - \\
        & Diffusion InsGen & paper512 & diffusion-pixel, , `uniform' \\
        \midrule
        \multirow{4}{*}{FFHQ } & StyleGAN2  & stylegan2 & - \\
        & Diffusion StyleGAN2    &  stylegan2 & diffusion-pixel, , `uniform' \\
        & InsGen    &  default & - \\
        & Diffusion InsGen & stylegan2 & diffusion-pixel, , `uniform' \\
         \bottomrule
    \end{tabular}}
    \caption{\small  The config setting of StyleGAN2-based models and ProjectedGAN-based models. For StyleGAN2-based models, we borrow the config settings provided by \citet{karras2020training}, which includes [`auto', `stylegan2', `cifar', `paper256', `paper512', `paper1024']. We create the `stl' config based on 'cifar' with small modifications that we change the gamma term to be 0.01. For ProjectedGAN models, we use the recommended default config \citep{sauer2021projected}, which is based on FastGAN.  }
    \label{tab:model_config}
\end{table}

\section{Implementation details} \label{sec:implementation}

We implement an additional diffusion sampling pipeline, where the diffusion configurations are set in \Cref{sec:hyperparameter}. The  in the forward diffusion process is adaptively adjusted and clipped to . As illustrated in \Cref{alg:algo}, at each update step, we sample  from  for each data point , and then use the analytic Gaussian distribution at diffusion step  to sample . Next, we use  and  instead of  for optimization. 

\paragraph{Diffusion StyleGAN2. } We inherit all the network architectures from StyleGAN2 implemented by \citet{karras2020training}. We modify the original mapping network, which is there for label conditioning and unused for unconditional image generation tasks, inside the discriminator to inject . Specifically, we change the original input of mapping network, the class label , to our discrete value timestep . Then, we train the generator and discriminator with diffused samples  and .

\paragraph{Diffuson ProjectedGAN. } To simplify the implementation and minimize the modifications to ProjectedGAN, we construct the discriminator as , where  is ignored. Our method is plugged in as a data augmentation method. The only change in the optimization stage is that the discriminator is fed with diffused images  instead of original images .

\paragraph{Diffuson InsGen. } To simplify the implementation and minimize the modifications to InsGen, we keep their contrastive learning part untouched. We modify the original discriminator network to inject  similarly to Diffusion StyleGAN2. Then, we train the generator and discriminator with diffused samples  and .

 

\section{Ablation on the mixing procedure and  adaptiveness}

Note the mixing procedure described in \Cref{eq:pi_t}, referred to as ``priority mixing'' in what follows, is designed based on our intuition. Here we conduct an ablation study on the mixing procedure by comparing the priority mixing with uniform mixing on three representative datasets. We report in \Cref{tab:mixing_ablation} the FID results, which suggest that uniform mixing could work better than priority mixing in some dataset, and hence Diffusion-GAN may be further improved by optimizing its mixing procedure according to the training data. While optimizing the mixing procedure is beyond the focus of this paper, it is worth further investigation in future studies.


\begin{table}[H]
 \caption{ Ablation study on the mixing procedure. ``Priority Mixing'' refers to the mixing procedure in \Cref{eq:pi_t} and ``Uniform Mixing'' refers to sample  uniformly at random. }
    \label{tab:mixing_ablation}
    \centering
    \resizebox{0.5\textwidth}{!}{
    \begin{tabular}{c|ccc}
         \toprule
         & CIFAR-10 & STL-10 & FFHQ  \\
         \midrule
         Priority Mixing& \textbf{3.19} & \textbf{11.43}  & 3.22 \\
         Uniform  Mixing& 3.44 & 11.75  & \textbf{2.83} \\
         \bottomrule
    \end{tabular}}
\end{table}

{We further conduct ablation study on whether the  needs to be adaptively adjusted. As shown in \Cref{fig:adaptive_ablation}, we observe with adaptive diffusion strategy, the training curves of FIDs converge faster and reach lower final FIDs. }

\begin{figure}
    \centering
    \includegraphics[width=0.6\textwidth]{figs/adaptive_ablation.pdf}
    \caption{Ablation study on the  adaptiveness.}
    \label{fig:adaptive_ablation}
\end{figure}

\section{More GAN variants}

To further validate our noise injection via diffusion-based mixtures, we add our diffusion-based training into two more representative GAN variants: DCGAN \citep{radford2015unsupervised} and SNGAN \citep{miyato2018spectral}, which have quite different GAN architectures compared to StyleGAN2. We provide the FIDs for CIFAR-10 in \Cref{tab:more_gan_variants}. We observe that both Diffusion-DCGAN and Diffusion-SNGAN clearly outperform their corresponding baseline GANs.


\begin{table}[H]
\caption{ FIDs on CIFAR-10 for DCGAN, Diffusion-DCGAN, SNGAN, and Diffusion-SNGAN. }
    \centering
    \resizebox{\textwidth}{!}{
    \begin{tabular}{l|cc|cc}
         \toprule
         & DCGAN \citep{radford2015unsupervised} & Diffusion-DCGAN & SNGAN \citep{miyato2018spectral} & Diffusion-SNGAN \\
         \midrule
         CIFAR-10 & 28.65 & \textbf{24.67} & 20.76 & \textbf{17.23} \\
         \bottomrule
    \end{tabular}}
    \label{tab:more_gan_variants}
\end{table}

\section{{Inception Score for CIFAR-10}}


{
We report the Inception Score (IS) \citep{salimans2016improved} of Diffusion StyleGAN2 for CIFAR-10 dataset in \Cref{tab:inception_score} and also include other state-of-the-art GANs and diffusion models as baselines. Note CIFAR-10 is a well-known dataset and tested by almost all baselines, so we pick CIFAR-10 here and we reference the reported IS values from their original papers for a fair comparison.}

\begin{table}[!t]
    \centering
    \resizebox{0.6\textwidth}{!}{
    \begin{tabular}{l|cccc}
        \toprule
         Method & IS  & FID  & Recall  & NFE  \\
         \hline
         DDPM \citep{Ho2020DenoisingDP} & 9.46 & 3.21 & 0.57 & 1000 \\
         DDIM \citep{song2020denoising} & 8.78 & 4.67 & 0.53 & 50 \\
         Denoising Diffusion GAN \citep{xiao2021tackling} & 9.63 & 3.75 & 0.57 & 4  \\
         StyleGAN2 \citep{karras2020training} & 9.18 & 8.32 & 0.41 & 1 \\
         StyleGAN2 + DiffAug \citep{zhao2020differentiable}  & 9.40 & 5.79 & 0.42   & 1 \\
         StyleGAN2 + ADA \citep{karras2020training}& 9.83 & \textbf{2.92} & 0.49 & 1 \\
         Diffusion StyleGAN2 & \textbf{9.94} & 3.19 & \textbf{0.58} & 1 \\
         \bottomrule
    \end{tabular}}
    \vspace{1mm}
    \caption{\small Inception Score for CIFAR-10.  For sampling time, we use the number of function evaluations (NFE).}
    \label{tab:inception_score}
\end{table}

\section{More generated images} \label{sec:gen_images}

We provide more randomly generated images for LSUN-Bedroom, LSUN-Church, AFHQ, and FFHQ datasets in \Cref{fig:bedroom_and_church_big}, \Cref{fig:afhq_big}, and \Cref{fig:ffhq_big}.

\begin{figure}
    \centering
    \includegraphics[width=\textwidth]{figs/bedroom_big.jpg} \\
    \quad \\
    \includegraphics[width=\textwidth]{figs/church_big.jpg}
    \caption{More generated images for LSUN-Bedroom (FID 1.43, Recall 0.58) and LSUN-Church (FID 1.85, Recall 0.65) from Diffusion ProjectedGAN. }
    \label{fig:bedroom_and_church_big}
\end{figure}

\begin{figure}
    \centering
    \includegraphics[width=\textwidth]{figs/cat_h.jpg} \\
    \includegraphics[width=\textwidth]{figs/dog_h.jpg} \\
    \includegraphics[width=\textwidth]{figs/wild_h.jpg}
    \caption{More generated images for AFHQ-Cat (FID 2.40), AFHQ-Dog (FID 4.83) and AFHQ-Wild (FID 1.51) from Diffusion InsGen. }
    \label{fig:afhq_big}
\end{figure}

\begin{figure}
    \centering
    \includegraphics[width=\textwidth]{figs/ffhq_big.jpg}
    \caption{More generated images for FFHQ from Diffusion StyleGAN2 (FID 3.71, Recall 0.43). }
    \label{fig:ffhq_big}
\end{figure}

\end{document}