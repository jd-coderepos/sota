\documentclass[10pt,twocolumn,letterpaper]{article}

\usepackage{iccv}
\usepackage{times}
\usepackage{epsfig}
\usepackage{graphicx}
\usepackage{amsmath}
\usepackage{amssymb}
\usepackage{subfigure}
\usepackage{color}
\usepackage{multirow}
\usepackage{textcomp, gensymb}
\usepackage{bbding}
\usepackage{booktabs}
\usepackage{bm, eucal}
\usepackage{pifont}
\usepackage{enumitem}

\usepackage{caption, multirow, overpic, textpos}
\usepackage{booktabs}
\usepackage[table]{xcolor}

\usepackage{xspace}
\usepackage{balance}
\usepackage{numprint}
\usepackage{sidecap}
\usepackage{float}
\usepackage{siunitx}
\usepackage{pifont}

\usepackage[pagebackref=true,breaklinks=true,letterpaper=true,colorlinks,bookmarks=false]{hyperref}

\newcommand{\titlepicture}[2][]{\renewcommand\placetitlepicture{\includegraphics[#1]{#2}\par\medskip
  }}
\newcommand{\placetitlepicture}{} 


\newcommand{\minus}{\scalebox{0.75}[1.0]{}}


\usepackage[capitalize]{cleveref}
\crefname{section}{Sec.}{Secs.}
\Crefname{section}{Section}{Sections}
\Crefname{table}{Table}{Tables}
\crefname{table}{Tab.}{Tabs.}

\iccvfinalcopy 

\def\iccvPaperID{2663} \def\httilde{\mbox{\tt\raisebox{-.5ex}{\symbol{126}}}}



\def\Ours{{Zolly}\xspace}
\def\Oursp{{}\xspace}

\newcommand{\name}{Zolly\xspace}

\newcommand{\cP}{\mathcal{P}}

\newcommand{\misscite}{\textcolor{red}{[C]~}}
\newcommand{\missref}{\textcolor{red}{[R]~}}
\newcommand{\missvalue}{\textcolor{red}{[V]~}}
\newcommand{\needcheck}[1]{\textcolor{red}{#1}}
\newcommand{\todo}[1]{\textcolor{red}{#1}}


\begin{document}



\title{Zolly: Zoom Focal Length Correctly for Perspective-Distorted \\Human Mesh Reconstruction}

\author{Wenjia Wang \quad \; Yongtao Ge \quad \; Haiyi Mei \quad \; Zhongang Cai \quad \; \\ Qingping Sun \quad \; Yanjun Wang \quad \;  Chunhua Shen \quad \; Lei Yang \quad \; Taku Komura\
\small
\begin{bmatrix}
  f&  & \\
  &f  & \\
  &  &1
\end{bmatrix}\begin{bmatrix}
 x+T_{x}\\
 y+T_{y}\\
z+T_{z}
\end{bmatrix} = \begin{bmatrix}
 f(x+T_{x})\\
 f(y+T_{y})\\
T_{z},
\end{bmatrix}\text{, },

\small
\begin{bmatrix}
f(x+T_{x})/T_{z} \\
f(y+T_{y})/T_{z}
\end{bmatrix}
= \begin{bmatrix}
s(x+t_{x})  \\
s(y+t_{y}) 
\end{bmatrix}.\\

\small
s \times T_{z} = f\text{, }T_{x} = t_{x}\text{, }T_{y} = t_{y}.
\label{equation:f_sz}

\small
\begin{bmatrix}
x_{2D} \\
y_{2D}
\end{bmatrix} = \begin{bmatrix}
f(x+T_{x})/(z+T_{z}) \\
f(y+T_{y})/(z+T_{z})
\end{bmatrix}
= \begin{bmatrix}
s(x+t_{x})  \\
s(y+t_{y}) 
\end{bmatrix}.
\label{equation:pers_proj}

\small
\mathcal{L}_{Transl} = \lambda_{IUV}\mathcal{L}^{2}_{IUV} + \lambda_{d}\mathcal{L}^{2}_{d}  + \lambda_{z}\mathcal{L}^{1}_{z},

\small
\begin{split}
\mathcal{L}_{Mesh} = \lambda_{J_{3D}}\mathcal{L}^{1}_{J_{3D}} + \lambda_{J_{2D}^{P}}\mathcal{L}^{1}_{J_{2D}^{P}} + \lambda_{J_{2D}^{W}}\mathcal{L}^{1}_{J_{2D}^{W}}\\ + \lambda_{V}(\mathcal{L}^{1}_{V^{''}}+\mathcal{L}^{1}_{V^{'}}+\mathcal{L}^{1}_{V}) ,
\end{split}

\small
 \mathit{K}_{W} =\begin{bmatrix}
 f_{W} &  & h/2\\
  &   f_{W} & h/2\\
  &  & 1
\end{bmatrix}\text{, }
\mathit{T_{W}} = \begin{bmatrix}
   t_{x}  \\
  t_{y} \\
 2f_{W}/sh
\end{bmatrix}.

\small
    \hat{J}_\mathit{2D}^{W} = K_{W}(\hat{J}_\mathit{3D}^{\otimes} + T_{W}),
\\
\mathcal{L}_{\mathit{2D}}^{W} = \sum_{i=1}^{N_{j}} \frac{1}{d_{J[i]}}\| \hat{J}_\mathit{2D}^{W}[i] \; - \; J_\mathit{2D}^{im}[i] \|_F^1,
\label{eq:weak_loss}

\small
\mathit{K}_{P} =\begin{bmatrix}
 f_{P} &  & H/2\\
  &   f_{P}  & H/2\\
  &  & 1
\end{bmatrix}, 
\hat{J}_\mathit{2D}^{P} = K_{P}(\hat{J}_\mathit{3D} + T_{P}^{\otimes}),
\label{eq:project}

\small
\begin{bmatrix}  
\frac{w}{2}\left[ s_{x}(x+t_{x})) +1\right]+c_{x}-\frac{w}{2}\6pt]
\frac{H}{2}\left[ S_{H}(y+T_{y}) + 1\right] \end{bmatrix}.

\small
\begin{bmatrix}
T_{x}\\ 
T_{y}\\ 
1
\end{bmatrix}=\begin{bmatrix}
 1& 0 & (2c_{x}-W)/ws_{x}\\ 
 &1  & (2c_{y}-H)/hs_{y}\\ 
 &  & 1
\end{bmatrix}\begin{bmatrix}
t_{x}\\ 
t_{y}\\ 
1
\end{bmatrix}

 terms the bounding-box center coordinate in original image.  terms the cropped image size, where  terms original image size. During training, we will expand every bounding box of the human body to a square and resize the cropped image to  pixels.

\noindent\textbf{Analysis of dolly zoom.}
The dolly zoom is an optical effect performed in-camera, whereby the camera moves towards or away from a subject while simultaneously zooming in the opposite direction.
It was first proposed in the film JAW~\cite{jaws1975}. In this section, we simulate the effect on CMU-MOSH~\cite{mosh} data.
First, we get all the vertices in CMU-MOSH~\cite{mosh} by feeding the SMPL~\cite{smpl} parameters to the body model. 
We further obtain 3D joints by multiplying the joint regressor matrix to the vertices. 
We then apply weak-perspective camera parameters  while adjusting the distance to approximate the human body's location and size, producing increased distortion as the camera approaches.
The weak-perspectively projected 2D joints are , where  are the corresponding 3D joint coordinates. We set the image height to 224 pixels and re-project the 2D joints and compare the error between the weak-perspective and perspective projection results.
As shown in \cref{tab:supp_dolly}, when the subject is located over 4 meters away, the re-projected error is only 1.76 pixels, which is negligible on a 224-pixel image. When the subject is further than 8 meters, the error is less than 1 pixel, indicating non-distortion of the images. 




 \begin{table}
    \centering
    \scalebox{0.7}{\begin{tabular}{l|p{20pt}
<{\raggedright}p{20pt}<{\raggedright}p{20pt}<{\raggedright}p{20pt}<{\raggedright}p{20pt}<{\raggedright}p{20pt}<{\raggedright}p{20pt}<{\raggedright}p{20pt}<{\raggedright}p{20pt}<{\raggedright}p{20pt}<{\raggedright}}\toprule
  &      0.5 &0.75&1.0 & 2.0 & 4.0 & 8.0 & 12.0 & 16.0 &20       \\\midrule
  &    3.06&1.69&1.41&1.16&1.07&1.04&1.02&1.02&1.01          \\
  & 30.56&10.46&7.38&3.54&1.76&0.88&0.59&0.44&0.35              \\

\bottomrule       
\end{tabular} }
    \caption{Distortion and re-projection error caused by distance in CMU-MOSH~\cite{mosh}. The re-projected error is measured in pixels.}
    \label{tab:supp_dolly}
    \vspace{-10pt}
\end{table}

\begin{figure}[t]
    \centering
    \includegraphics[width=0.9\linewidth]{pictures/supp_dollyzoom.pdf}
    \caption{Distortion and re-projection error caused by distance. The vertical axis is measured in pixels, and the horizontal axis is measured in meters.}
    \label{fig:supp_dolly}
    \vspace{-5pt}
\end{figure}  

\section{More quantitative results}























\noindent\textbf{Full results on PDHuman:} As shown in~\cref{tab:sota_supp_pdhuman} and \cref{tab:sota_supp_pdhuman2}, we report results on all 5 protocols in the PDHuman test dataset. Our proposed methods, \Ours(H48) and \Oursp (R50) outperform the other methods in all metrics by a large margin. 


\noindent\textbf{Full results on SPEC-MTP:} As illustrated in~\cref{tab:sota_supp_specmtp}, we report the results of all the 3 protocols in SPEC-MTP dataset. In this real-world dataset, \Ours (H48) largely outperforms other methods in all metrics. In column , Note that our re-implemented SPEC achieves higher performance than the official implementation.
 
\noindent\textbf{Full results on HuMMan:} As shown in \cref{tab:sota_sup_humman}, \Ours (H48) largely outperforms other methods in all metrics. By contrast, although CLIFF~\cite{cliff} performs comparably well on the HuMMan dataset, it demonstrates poor performance on the PDHuman dataset. We conjecture the focal length assumption of CLIFF is suitable for datasets captured by fixed and similar camera settings, \eg HuMMan dataset, while not valid for the PDHuman dataset with varied camera settings.


\begin{table*}[hb]
    \centering
    \scalebox{0.61}{\begin{tabular}{lccccc|ccccc|ccccc}
\toprule
\multirow{2}{*}{\textbf{Methods}} & \multicolumn{5}{c}{PDHuman~()}  & \multicolumn{5}{c}{PDHuman~()}
&\multicolumn{5}{c}{PDHuman~()} \\ \cmidrule{2-16}
&PA-MPJPE & MPJPE  & PVE & mIoU &P-mIoU& PA-MPJPE & MPJPE & PVE & mIoU &P-mIoU& PA-MPJPE & MPJPE & PVE &  mIoU  & P-mIoU \\ \midrule 

\rule{0pt}{10pt} HMR~(R50)~\cite{hmr}  & 62.5 & 91.5 & 106.6 & 48.9& 21.7 & 59.9 &87.8 & 102.4 & 50.0 & 22.5 & 57.4 & 84.0 & 98.1 & 51.4 & 23.6 \\

\rule{0pt}{10pt} HMR-~(R50)~\cite{hmr}& 61.6 & 90.2 & 105.5 & 45.2 & 20.4 & 59.2 & 86.6 & 101.3 & 46.5 & 21.4 &56.8 & 82.9 & 97.2 & 48.1 & 22.7  \\

\rule{0pt}{10pt} SPEC~(R50)~\cite{spec}  &  65.8 & 94.9 & 109.6 & 43.4 & 19.6 & 63.2 & 91.5 & 105.8 &43.3 & 19.5 &60.6 & 87.3 & 101.3 & 42.2 & 18.7   \\

\rule{0pt}{10pt} CLIFF~(R50)~\cite{cliff} & 66.2 & 99.2 & 115.2 & 51.4 & 24.8 & 63.4 & 94.4 & 109.8 & 52.7 & 25.9 & 60.6 & 89.6 & 104.3 & 54.2 & 27.1 \\



\rule{0pt}{10pt} PARE~(H48)~\cite{pare} & 66.3 & 95.9 & 116.7 & 48.2 & 20.9 &63.6& 92.3 & 112.7 & 49.3 & 21.7 & 60.6 & 88.7 & 108.6 & 50.7 & 22.7   \\

\midrule
\rule{0pt}{10pt} GraphCMR~(R50)   &  62.1 & 85.8 & 98.4 & 47.9 & 21.5 & 59.5 & 82.6 & 94.8 & 49.1 & 22.4 & 56.8 & 78.8 & 90.4 & 50.5 & 23.6\\

\rule{0pt}{10pt} FastMetro(H48)~\cite{fastmetro}   &58.6 & 83.6 & 95.4 & 50.1 & 22.5 & 55.8 & 79.9 & 91.4 & 51.4 & 23.5  & 53.1 & 75.9 & 86.7 & 52.9 & 24.9 \\

\midrule
\rule{0pt}{10pt}  (R50)  &  54.3 & 80.9 & 93.9 & \textbf{54.5} & \textbf{27.4} & 52.4 & 77.5 & 90.2 & \textbf{55.7} & 28.5 & 50.0 & 74.0 & 86.4 & \textbf{56.9} & \textbf{29.5} \\
\rule{0pt}{10pt} \Ours (R50)  &  54.3 & 76.4&87.6 & 51.4 & 24.0 & 51.8 & 73.3 & 84.1 & 52.4 & 24.8 &  49.3 & 70.1 & 80.6 & 53.3 & 25.7 \\
\rule{0pt}{10pt} \Ours (H48)  & \textbf{49.7} & \textbf{70.2} & \textbf{81.2} & 50.5 & 23.8 & \textbf{47.6} & \textbf{64.3} & \textbf{74.4} & 55.3 & \textbf{28.5} & \textbf{44.9} & \textbf{64.3} & \textbf{74.7} & 55.3 & 28.5     \\
\bottomrule

\end{tabular}
 }
    
    \caption{Results of SOTA methods on PDHuman (, ,  protocols). HMR- terms HMR~\cite{hmr} model trained with same focal length as \Ours.}
    \label{tab:sota_supp_pdhuman}
\end{table*}

\begin{table*}[h]
    \centering
    \scalebox{0.61}{\begin{tabular}{lp{57pt}<{\centering}p{47pt}<{\centering}p{47pt}<{\centering}p{47pt}<{\centering}p{47pt}<{\centering}|p{57pt}<{\centering}p{47pt}<{\centering}p{47pt}<{\centering}p{47pt}<{\centering}p{47pt}<{\centering}}
\toprule
\multirow{2}{*}{\textbf{Methods}} & \multicolumn{5}{c}{PDHuman~()} & \multicolumn{5}{c}{PDHuman~()}
 \\ \cmidrule{2-11}
&PA-MPJPE & MPJPE  & PVE & mIoU &P-mIoU& PA-MPJPE & MPJPE & PVE & mIoU &P-mIoU \\ \midrule 

\rule{0pt}{10pt} HMR~(R50)~\cite{hmr} & 53.9 &  79.0 & 92.4 & 53.6 & 25.1 & 49.2 & 73.3 & 85.9 & 57.3 & 28.2\\

\rule{0pt}{10pt} HMR-~(R50)~\cite{hmr}&  53.4 & 78.3 & 91.8 & 50.4 & 24.6  & 48.8 & 72.7 & 85.3 & 54.6 & 28.1 \\

\rule{0pt}{10pt} SPEC~(R50)~\cite{spec}  &  56.8 & 81.8 & 95.1 & 40.1 & 17.1 & 51.8 & 75.4 & 87.9 & 37.4 & 15.3   \\

\rule{0pt}{10pt} CLIFF~(R50)~\cite{cliff} & 56.7 & 83.6 & 97.3 & 56.5 & 29.1 & 51.6 & 76.9 & 89.7 & 60.2 & 32.7\\



\rule{0pt}{10pt} PARE~(H48)~\cite{pare}  & 56.8 & 83.9 & 103.0 & 52.8 & 24.3 & 51.8 & 78.5 & 96.6 & 56.6 & 27.5 \\

\midrule
\rule{0pt}{10pt} GraphCMR~(R50)   &  53.2 & 74.2 & 85.2 & 52.7 & 25.3 & 48.7 & 69.1 & 79.4 & 56.4 & 28.6 \\

\rule{0pt}{10pt} FastMetro(H48)~\cite{fastmetro}   &  49.4 & 71.1 & 81.1 & 55.5 & 27.0 & 45.0 & 65.8 & 75.2 & 59.7 & 31.0  \\

\midrule
\rule{0pt}{10pt}  (R50)  & 47.1 & 69.8 & 81.6 & \textbf{58.7} & \textbf{30.7} & 43.2 & 65.2 & 76.5 & \textbf{61.3} & \textbf{32.6}  \\
\rule{0pt}{10pt} \Ours (R50)  & 45.9 & 66.0 & 75.9 & 54.8 & 26.8 & 41.9 & 61.5 & 70.9 & 57.2 & 28.2 \\
\rule{0pt}{10pt} \Ours (H48)  & \textbf{42.1} & \textbf{60.7} & \textbf{70.4} & 56.8 & 29.5 & \textbf{39.4} & \textbf{56.6} & \textbf{69.6} & 58.3 & 29.9    \\
\bottomrule
\end{tabular}
 }
    
    \caption{Results of SOTA methods on PDHuman (,  protocols).}
    \label{tab:sota_supp_pdhuman2}
\end{table*}


\begin{table*}[h]
    \centering
    \scalebox{0.61}{\begin{tabular}{lccccc|ccccc|ccccc}
\toprule
\multirow{2}{*}{\textbf{Methods}} & \multicolumn{5}{c}{SPEC-MTP~()}& \multicolumn{5}{c}{SPEC-MTP~()}
&\multicolumn{5}{c}{SPEC-MTP~()}\\ \cmidrule{2-16}
&PA-MPJPE & MPJPE  & PVE & mIoU &P-mIoU& PA-MPJPE & MPJPE & PVE & mIoU &P-mIoU& PA-MPJPE & MPJPE & PVE &  mIoU  & P-mIoU \\ \midrule 

\rule{0pt}{10pt} HMR~(R50)~\cite{hmr}  & 73.9 & 121.4 & 145.6 & 48.8 & 16.0 & 73.1 & 112.5 & 135.7 & 51.1 & 20.0 & 69.6 & 111.8 & 135.7 & 50.5 & 21.8\\

\rule{0pt}{10pt} HMR-~(R50)~\cite{hmr}& 72.7 & 123.2 & 145.1 & 52.3 & 21.0 & 72.1 & 113.3 & 135.5 & 51.9 & 21.9 & 69.1 & 112.8 & 136.3 & 52.5 & 24.8 \\

\rule{0pt}{10pt} SPEC~(R50)~\cite{spec}  &   76.0 & 125.5 & 144.6 & 49.9 & 18.8 & 72.4 & 114.0 & 134.3 & 49.3 & 19.5 & 67.4 & 110.6& 132.5 & 49.1 & 21.2  \\

\rule{0pt}{10pt} SPEC~*~(R50)~\cite{spec}  &  -& -& -& -& -& -& -& -& -& -& 71.8& 116.1&  136.4 & -& - \\


\rule{0pt}{10pt} CLIFF~(R50)~\cite{cliff} & 74.3 & 115.0 & 132.4 & 53.6 & 23.7 & 70.2 & 107.0 & 126.8 & 52.0 & 22.1 & 67.4 & 108.7 & 130.4 & 51.9 & 23.4\\



\rule{0pt}{10pt} PARE~(H48)~\cite{pare} & 74.2 & 121.6 & 143.6 & 55.8 & 23.2 & 71.6 & 112.7 & 137.2 & 55.1 & 22.4 & 68.5 & 113.5 & 139.6 & 55.3 & 25.1  \\

\midrule
\rule{0pt}{10pt} GraphCMR~(R50)   & 76.1 & 121.1 & 133.1 & 56.3 & 23.4 & 74.4 & 114.9 & 129.5 & 52.6 & 20.8 &  70.2 & 112.7 & 127.8 & 51.7 & 22.0\\

\rule{0pt}{10pt} FastMetro(H48)~\cite{fastmetro}   & 75.0   & 123.1  & 137.0 & 53.5  & 20.5 & 70.8 & 112.3 & 128.0 & 52.4 & 20.6 & 66.3 & 110.2 & 126.5 & 51.8 & 22.6   \\

\midrule
\rule{0pt}{10pt}  (R50)  & 72.9 & 117.7 & 138.2 & 54.7 & 22.4 & 70.5 & 108.1 & 129.4 & 53.9 & 21.5 &  68.4 & 110.2 & 134.3 & 54.7 & 24.2 \\
\rule{0pt}{10pt} \Ours (R50)  & 74.0 & 122.1 & 135.6 & 58.9 & 24.9 & 70.3 & 111.1 & 126.0 & 56.9 & 22.0 & 66.9 & 109.6 & 124.4 & 56.5 & 23.4 \\
\rule{0pt}{10pt} \Ours (H48)  & \textbf{67.4} & \textbf{114.6} & \textbf{126.7} & \textbf{62.6} & \textbf{30.4}   & \textbf{66.5} & \textbf{106.1} & \textbf{120.1} & \textbf{59.9} & \textbf{26.6} & \textbf{65.8} & \textbf{108.2} & \textbf{121.9} & \textbf{58.5} & \textbf{27.0} \\
\bottomrule
\end{tabular}
 }
    
    \caption{Results of SOTA methods on SPEC-MTP (, ,  protocols). SPEC-MTP~() indicates the original SPEC-MTP~\cite{spec} dataset. SPEC~* terms the results reported in SPEC~\cite{spec}.}
    \label{tab:sota_supp_specmtp}
\end{table*}



\begin{table*}[h]
    \centering
    \scalebox{0.61}{\begin{tabular}{lccccc|ccccc|ccccc}
\toprule
\multirow{2}{*}{\textbf{Methods}} & \multicolumn{5}{c}{HuMMan~()}& \multicolumn{5}{c}{HuMMan~()}
&\multicolumn{5}{c}{HuMMan~()}\\ \cmidrule{2-16}
&PA-MPJPE & MPJPE  & PVE & mIoU &P-mIoU& PA-MPJPE & MPJPE & PVE & mIoU &P-mIoU& PA-MPJPE & MPJPE & PVE &  mIoU  & P-mIoU \\ \midrule 

\rule{0pt}{10pt} HMR~(R50)~\cite{hmr} & 30.2 & 43.6 & 52.6 & 65.1 & 39.5 & 31.9 & 45.0 & 39.5 & 66.6 & 39.9 & 30.0 & 44.1 & 50.7 & 66.6 & 39.5 \\

\rule{0pt}{10pt} HMR-~(R50)~\cite{hmr}& 29.9 & 43.6 & 53.4 & 62.7 & 34.9 & 31.3 & 45.0 & 53.3 & 66.6 & 39.9 & 29.8 & 44.1 & 50.7 & 66.6 & 39.5 \\

\rule{0pt}{10pt} SPEC~(R50)~\cite{spec}  & 31.4 & 44.0 & 54.2 & 51.4 & 24.6 & 33.1 & 46.1 & 41.7 & 46.0 & 19.2 &   31.2 & 44.8 & 51.6 & 42.2 & 16.6 \\

\rule{0pt}{10pt} CLIFF~(R50)~\cite{cliff} &  28.6 & 42.4 & 50.2 & 68.9 & 44.7 & 30.3 & 43.3 & 51.2 & 70.2 & 44.9 & 28.3 & 42.3 & 48.5 & 70.6 & 44.5 \\



\rule{0pt}{10pt} PARE~(H48)~\cite{pare} & 32.6 & 53.2 & 65.5 & 64.5 & 38.3  & 33.6 & 53.3 & 66.2 & 65.1 & 38.0 & 32.2 & 53.1 & 64.6 & 65.0 & 37.6\\

\midrule
\rule{0pt}{10pt} GraphCMR~(R50)   & 29.5 & 40.6 & 48.4 & 61.6 & 37.5 & 30.3 & 40.6 & 48.2 & 62.6 & 37.6 & 29.3 & 40.2 & 46.3 & 62.8 & 37.0 \\

\rule{0pt}{10pt} FastMetro(H48)~\cite{fastmetro}   & 26.3 & 38.8 & 45.6 & 68.3 & 45.2 &   27.8 & 39.9 & 46.6 & 69.9 & 45.7 & 26.5 & 38.5 & 43.6 & 70.0 & 45.3 \\

\midrule
\rule{0pt}{10pt}  (R50)  &  24.4 & 36.7 & 45.9 & 70.4 & \textbf{45.5} & 26.2 & 37.6 & 45.6 & 70.4 & 45.3 & 25.6 & 37.7 & 43.7 & 70.8 & 45.2 \\
\rule{0pt}{10pt} \Ours (R50)  &  25.5 & 36.7 & 43.4 & 67.0 & 38.4 & 25.6 & 36.5 & 42.5 & 70.4 & 42.7 & 24.2 & 35.2 & 40.4 & 70.7 & 42.4 \\
\rule{0pt}{10pt} \Ours (H48)  & \textbf{22.3} & \textbf{32.6} & \textbf{40.0} & \textbf{71.2} & 45.1 & \textbf{24.1} & \textbf{33.8} & \textbf{40.7} & \textbf{72.2} & \textbf{47.9} & \textbf{23.0} & \textbf{33.0} & \textbf{38.7} & \textbf{73.2} & \textbf{47.4}    \\
\bottomrule
\end{tabular}
 }
    
    \caption{Results of SOTA methods on HuMMan (, ,  protocols).}
    \label{tab:sota_sup_humman}
\end{table*}

\section{Qualitative results.}



\noindent\textbf{Qualitative results on Human3.6M~\cite{h36m} dataset.}
We show qualitative results of \Ours on Human3.6M dataset in \cref{fig:h36m}
\begin{figure}[ht]
    \centering
    \includegraphics[width=1.\linewidth]{pictures/h36m.pdf}
    \caption{Qualitative results on Human3.6M dataset. The number under each image represents predicted/ground-truth focal length , FoV angle, and z-axis translation . Our method could predict an approximate translation for non-distorted images as well.}
    \label{fig:h36m}
    \vspace{-10pt}
\end{figure}



\begin{figure}[h]
    \centering
    \includegraphics[width=1\linewidth]{pictures/failure_cases.pdf}
    \caption{Failure cases. The left part is input, and the right part is our prediction.}
    \label{fig:failure}
    \vspace{-10pt}
\end{figure}

\noindent\textbf{Failure cases.}
Although our methodology is generally effective, it has trouble under certain extreme circumstances. As demonstrated in~\cref{fig:failure}, due to the lack of training data containing characters with large hands ((1), (2), and (6)), and large feet ((7) and (8)), \Ours produce sub-optimal results on such images. 
Similarly, our approach may not perform well on characters with exceptional body shapes, as exemplified by (2), (3), and (7), where the athletes have muscular bodies. Additionally, it is difficult for \Ours to reconstruct self-occluded human bodies, as depicted in (4). We are actively exploring strategies to address these limitations and improve the robustness of our methodology.

\noindent\textbf{More qualitative results on distorted images.}
We show more qualitative results of \Ours comparing with SOTA methods for perspective-distorted images on PDHuman (\cref{fig:supp_pdhuman_sota}), Web images (\cref{fig:supp_real_sota}), and SPEC-MTP (\cref{fig:supp_specmtp_sota}).

\clearpage



\begin{figure*}
    \centering
    \includegraphics[width=1.0\linewidth]{pictures/supp_pdhuman_sota.pdf}
    \caption{Qualitative results on PDHuman dataset. The number under each image represents predicted/ground-truth focal length , FoV angle, and z-axis translation . }
    \label{fig:supp_pdhuman_sota}
\end{figure*}


\begin{figure*}[htp]
    \includegraphics[width=1.0\linewidth]{pictures/supp_real.pdf}
\caption{Qualitative results on in-the-wild images. The number under each image represents the predicted focal length , FoV angle, and z-axis translation . Images are collected from \url{https://pexels.com} and \url{https://yandex.com}.
}
\label{fig:supp_real_sota}
\end{figure*}

\begin{figure*}
    \includegraphics[width=1.0\linewidth]{pictures/supp_specmtp.pdf}
    \caption{Qualitative results on SPEC-MTP dataset. The number under each image represents predicted/ground-truth focal length , FoV angle, and z-axis translation . The ground-truth  and focal length  for SPEC-MTP are pseudo labels.}
    \label{fig:supp_specmtp_sota}
\end{figure*}





%
 
\end{document}
