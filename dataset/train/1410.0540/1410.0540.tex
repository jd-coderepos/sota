\documentclass[11pt,a4paper]{article}

\usepackage{fullpage}
\usepackage[utf8x]{inputenc}
\usepackage{amsmath, amsthm}
\usepackage{wrapfig,graphicx,amssymb,textcomp,array,amsmath}
\usepackage{algpseudocode} 
\usepackage{enumitem}
\algtext*{EndWhile}\algtext*{EndIf}\usepackage{algorithm}
\usepackage{color}

\setlength{\arraycolsep}{0in}


\newcommand{\MC}{M_\times}
\newcommand{\MNC}{M_=}
\newcommand{\MOPT}{M^*}
\newcommand{\bt}{\lambda}
\newcommand{\btopt}{\lambda^*}
\newcommand{\cw}{cw}
\newcommand{\kTD}[2]{}
\newcommand{\kDG}[2]{}
\newcommand{\kGG}[2]{}
\newcommand{\kRNG}[2]{}
\newcommand{\WS}[1]{\text{WS}}
\newcommand{\Tg}{T_{gst}}
\newcommand{\GC}{G_{\mathcal{C}}}
\newcommand{\CD}[2]{D[#1,#2]}
\newcommand{\OD}[2]{D(#1,#2)}
\newcommand{\CIRC}[2]{C(#1,#2)}

\newcommand{\require}{\textbf{Input: }}
\newcommand{\ensure}{\textbf{Output: }}


\title{Matching in Gabriel Graphs\thanks{Research supported by NSERC.}}

\author{
Ahmad Biniaz\thanks{School of Computer Science, Carleton University, 
                    Ottawa, Canada.}
\and 
Anil Maheshwari\footnotemark[2]
\and 
Michiel Smid\footnotemark[2]
}
\date{\today}
\newtheorem{lemma}{Lemma}
\newtheorem{corollary}{Corollary}
\newtheorem{theorem}{Theorem}
\newtheorem{observation}{Observation}
\begin{document}

\maketitle

\begin{abstract}
Given a set  of  points in the plane, the order- Gabriel graph on , denoted by \kGG{k}{}, has an edge between two points  and  if and only if the closed disk with diameter  contains at most  points of , excluding  and . We study matching problems in \kGG{k}{} graphs. We show that a Euclidean bottleneck perfect matching of  is contained in \kGG{10}{}, but \kGG{8}{} may not have any Euclidean bottleneck perfect matching. In addition we show that \kGG{0}{} has a matching of size at least  and this bound is tight. We also prove that \kGG{1}{} has a matching of size at least  and \kGG{2}{} has a perfect matching. Finally we consider the problem of blocking the edges of \kGG{k}{}.
\end{abstract}

\section{Introduction}
Let  be a set of  points in the plane. For any two points , let  denote the closed disk which has the line segment  as diameter. Let  be the Euclidean distance between  and .
The {\em Gabriel graph} on , denoted by , is defined to have an edge between two points  and  if  is empty of points in . Let  denote the circle which has  as diameter. Note that if there is a point of  on , then . That is,  is an edge of  if and only if 

Gabriel graphs were introduced by Gabriel and Sokal \cite{Gabriel1969} and can be computed in  time \cite{Matula1980}. Every Gabriel graph has at most  edges, for , and this bound is tight \cite{Matula1980}. 

A {\em matching} in a graph  is a set of edges without common vertices. A {\em perfect matching} is a matching which matches all the vertices of . 
In the case that  is an edge-weighted graph, a {\em bottleneck matching} is defined to be a perfect matching in  in which the weight of the maximum-weight edge is minimized. For a perfect matching , we denote the {\em bottleneck} of , i.e., the length of the longest edge in , by . For a point set , a {\em Euclidean bottleneck matching} is a perfect matching which minimizes the length of the longest edge. 

In this paper we consider perfect matching and bottleneck matching admissibility of higher order Gabriel Graphs. The {\em order- Gabriel graph} on , denoted by \kGG{k}{}, is the geometric graph which has an edge between two points  and  iff  contains at most  points of . The standard Gabriel graph, , corresponds to \kGG{0}{}. It is obvious that \kGG{0}{} is plane, but \kGG{k}{} may not be plane for . Su and Chang \cite{Su1990} showed that \kGG{k}{} can be constructed in  time and contains  edges. In \cite{Bose2013}, the authors proved that \kGG{k}{} is -connected.
\subsection{Previous Work}
For any two points  and  in , the {\em lune} of  and , denoted by , is defined as the intersection of the open disks of radius  centred at  and .
The {\em order- Relative Neighborhood Graph} on , denoted by \kRNG{k}{}, is the geometric graph which has an edge  iff  contains at most  points of .  
The {\em order- Delaunay Graph} on , denoted by \kDG{k}{}, is the geometric graph which has an edge  iff there exists a circle through  and  which contains at most  points of  in its interior. 
It is obvious that 

The problem of determining whether a geometric graph has a (bottleneck) perfect matching is quite of interest. Dillencourt showed that the Delaunay triangulation (\kDG{0}{}) admits a perfect matching \cite{Dillencourt1990}. Chang et al. \cite{Chang1992} proved that a Euclidean bottleneck perfect matching of  is contained in \kRNG{16}{}.\footnote{They defined \kRNG{k}{} in such a way that  contains at most  points of .} This implies that \kGG{16}{} and \kDG{16}{} contain a (bottleneck) perfect matching of . In \cite{Abellanas2009} the authors showed that \kGG{15}{} is Hamiltonian which implies that \kGG{15}{} has a perfect matching. 

Given a geometric graph  on a set  of  points, we say that a set  of points {\em blocks}  if in  there is no edge connecting two points in , in other words,  is an independent set in .
Aichholzer et al.~\cite{Aichholzer2013} considered the problem of blocking the Delaunay triangulation (i.e. \kDG{0}{}) for  in general position. They show that  points are sufficient to block DT() and at least  points are necessary. To block a Gabriel graph,  points are sufficient, and  points are sometimes necessary \cite{Aronov2013}.

In a companion paper \cite{Biniaz2014}, we considered the matching and blocking problems in triangular-distance Delaunay (TD-Delaunay) graphs. The {\em order- TD-Delaunay graph}, denoted by \kTD{k}{}, on a point set  is the graph whose convex distance function is
defined by a fixed-oriented equilateral triangle. Then,  is an edge in \kTD{k}{} if there exists an equilateral triangle which has  and  on its boundary and contains at most  points of . We showed that \kTD{6}{} contains a bottleneck perfect matching and \kTD{5}{} may not have any. As for maximum matching, we proved that \kTD{1}{} has a matching of size at least  and \kTD{2}{} has a perfect matching (when  is even). We also showed that  points are necessary and  points are sufficient to block \kTD{0}{}. In \cite{Babu2013} it is shown that \kTD{0}{} has a matching of size . 

\subsection{Our Results}
In this paper we consider the following three problems: (a) for which values of  does every \kGG{k}{} have a Euclidean bottleneck matching of ? (b) for a given value , what is the size of a maximum matching in \kGG{k}{}? (c) how many points are sufficient/necessary to block a \kGG{k}{}? In Section~\ref{preliminaries} we review and prove some graph-theoretic notions. In Section~\ref{bottleneck-section} we consider the problem (a) and prove that a Euclidean bottleneck matching of  is contained in \kGG{10}{}. In addition, we show that for some point sets, \kGG{8}{} does not have any Euclidean bottleneck matching. In Section~\ref{max-matching-section} we consider the problem (b) and give some lower bounds on the size of a maximum matching in \kGG{k}{}. We prove that \kGG{0}{} has a matching of size at least , and this bound is tight. In addition we prove that \kGG{1}{} has a matching of size at least  and \kGG{2}{} has a perfect matching. In Section~\ref{blocking-section} we consider the problem (c). We show that at least  points are necessary to block a Gabriel graph and this bound is tight. We also show that at least  points are necessary and  points are sufficient to block a \kGG{k}{}. The open problems and concluding remarks are presented in Section~\ref{conclusion}.

\section{Preliminaries}
\label{preliminaries}
Let  be an edge-weighted graph with vertex set  and weight function . Let  be a minimum spanning tree of , and let  be the total weight of . 

\begin{lemma}
\label{not-mst-edge}
Let  be a cycle in  which contains an edge . Let  be the set of edges in  which do not belong to  and let  be the largest edge in . Then, .
\end{lemma}

\begin{proof}
Let  and let  and  be the two trees obtained by removing  from . Let  be an edge in  such that one of  and  belongs to  and the other one belongs to . By definition of , we have . Let . Clearly,  is a spanning tree of . If  then ; contradicting the minimality of . Thus, , which completes the proof of the lemma. 
\end{proof}
 
For a graph  and , let  be the subgraph obtained from  by removing all vertices in , and let  be the number of odd components in , i.e., connected components with an odd number of vertices. The following theorem by Tutte~\cite{Tutte1947} gives a characterization of the graphs which have perfect matching: 

\begin{theorem}[Tutte~\cite{Tutte1947}] 
\label{Tutte} 
 has a perfect matching if and only if  for all .
\end{theorem}

Berge~\cite{Berge1958} extended Tutte’s theorem to a formula (known as the Tutte-Berge formula) for the maximum size of a matching in a graph. In a graph , the {\em deficiency}, , is . Let .

\begin{theorem}[Tutte-Berge formula; Berge~\cite{Berge1958}] 
\label{Berge} 
The size of a maximum matching in  is 
\end{theorem}

For an edge-weighted graph  we define the {\em weight sequence} of , \WS{G}, as the sequence containing the weights of the edges of  in non-increasing order. A graph  is said to be less than a graph  if \WS{G_1} is lexicographically smaller than \WS{G_2}.

\section{Euclidean Bottleneck Matching}
\label{bottleneck-section}
Given a point set , in this section we prove that \kGG{10}{} contains a Euclidean bottleneck matching of . We also present a configuration of a point set  such that \kGG{8}{} does not contain any Euclidean bottleneck matching of . We use a similar argument as in \cite{Abellanas2009, Chang1991}. First consider the following lemma of \cite{Abellanas2009}:
\begin{lemma}[Abellanas et al.~\cite{Abellanas2009}]
\label{cone-lemma}
Let . Let  be a cone with apex , bounding rays 
and  emanating from  and angle  computed clockwise from  to . Given two points  and a constant . If  and , then  or .
\end{lemma}

\begin{figure}[htb]
  \centering
  \includegraphics[width=.6\columnwidth]{fig/cone0.pdf}
 \caption{Illustration for Theorem~\ref{10-GG-thr}.}
  \label{10-GG-fig}
\end{figure}

\begin{theorem}
 \label{10-GG-thr}
For every point set , \kGG{10}{} contains a Euclidean bottleneck matching of .
\end{theorem}
\begin{proof}
Let  be the set of all perfect matchings through the points of . Define a total order on the elements of  by their weight sequence. If two elements have exactly the same weight sequence, break ties arbitrarily to get a total order.
Let  be a perfect matching in  with minimal weight sequence. It is obvious that  is a Euclidean bottleneck matching for . We will show that all edges of  are in \kGG{10}{}. Consider any edge  in  and its corresponding disk . Suppose that  contains  points of . Let  represent the points inside , and  represent the points where . We will show that . Let  be the radius of . 

{\em Claim 1}: For each , . To prove this, assume that  and let  be the perfect matching obtained from  by deleting , and adding . The two new edges are smaller than the old ones. Thus,  which contradicts the minimality of .

Let  and  respectively be the open disks with radius  centered at  and . By Claim 1, we may assume that no point of  lies inside . In other words all points of  are contained in .

{\em Claim 2}: For each pair  and  of points in , . To prove this, assume that . Let  be the perfect matching obtained from  by deleting  and adding . Since ,  which contradicts the minimality of .

Let  be the center of . Consider a decomposition of the plane into 10 cones  of angle  with apex at . See Figure~\ref{10-GG-fig}. By contradiction, we will show that each cone , contains at most one point of . Suppose that a cone  where  contains two points . It is obvious that 




{\em Claim 3}: Each cone  where  and  contains at most one point of . Suppose that  contain two points . By Claim 1, all points of  are contained in . Consider the disk  with radius  centred at , as shown in Figure~\ref{10-GG-fig}. Since ,  and  are outside , i.e.,  and . By Lemma~\ref{cone-lemma},  or . By inequality~(\ref{equ1}),  which contradicts Claim 2.

\begin{figure}[htb]
  \centering
\setlength{\tabcolsep}{0in}
  
  \caption{(a) The angle  is smaller than the angle , and hence (b) .}
\label{cone-fig}
\end{figure}

{\em Claim 4}: Each of  and  contains at most one point of . Let  be the partition of  which lies inside  and  as shown in Figure~\ref{10-GG-fig}. Because of symmetry, we only prove the claim for . Suppose that  contains two points .
For the rest of the proof, refer to Figure~\ref{cone-fig}. 
W.l.o.g. assume that  is further from  than  and  is to the left of  (i.e.,  is to the left of the line through
 and  oriented from  to ). If  then  and . Then, by Lemma~\ref{cone-lemma} and Claim 2 we have a contradiction. Therefore, assume that . Let  and  be the two rays defining . Let  be the intersection of  and . Let  be the intersection of the boundaries of  and  which is inside . Define the point  on  such that  and . See Figure~\ref{cone-fig}(a). The triangle  is isosceles, and hence . This implies that . On the other hand, in triangle , , which implies that . Thus . In addition  and hence . Therefore in the triangle  we have  Let  be the circle with radius  having  as diameter, and let  be the ray emanating from  which goes through  as shown in Figure~\ref{cone-fig}(b). The intersection of  with  which lies to the right of  is completely inside . Thus, if  is to the right of , , which contradicts Claim 2. Therefore  lies to the left of . If  is in the interior of , rotate  counter-clockwise around  until  lies on . Since  is to the left of , the point  is still in the interior of . Let  be the intersection of the new  with . Note that  and hence  is contained in . In addition  and  are outside  and to the left of the line through  and . Therefore,  and hence  which contradicts Claim 2. 

By Claim 3 and Claim 4 each cone  where  contains at most one point of . Thus, , and  is an edge of \kGG{10}{}.
\end{proof}

\begin{figure}[htb]
  \centering
  \includegraphics[width=.8\columnwidth]{fig/9-GG.pdf}
 \caption{A set of 20 points such that \kGG{8}{} does not contain any Euclidean bottleneck matching.}
  \label{8-GG-fig}
\end{figure}
Now, we will show that for some point sets, \kGG{8}{} does not contain any Euclidean bottleneck matching.
Consider Figure~\ref{8-GG-fig} which shows a configuration of a set  of 20 points. The closed disk  is centred at  and has diameter one, i.e., .  contains 9 points  which lie on a circle with radius  which is centred at . Nine points in  are placed on a circle with radius 1.5 which is centred at  in such a way that , , , and  for  and . Consider a perfect matching  where each point  is matched to its closest point . It is obvious that , and hence the bottleneck of any bottleneck perfect matching is at most . We will show that any Euclidean bottleneck matching of  contains . By contradiction, let  be a Euclidean bottleneck matching which does not contain . In ,  is matched to a point . If , then . If , w.l.o.g. assume that . Thus, in  the point  is matched to a point  where . Since  is the closest point to  and , . In both cases , which is a contradiction. Therefore,  contains . Since  contains 9 points of , . Therefore \kGG{8}{} does not contain any Euclidean bottleneck matching of . 


\section{Maximum Matching}
\label{max-matching-section}
Let  be a set of  points in the plane. In this section we will prove that \kGG{0}{} has a matching of size at least ; this bound is tight. We also prove that \kGG{1}{} has a matching of size at least  and \kGG{2}{} has a perfect matching (when  is even).

First we give a lower bound on the number of components that result after removing a set  of vertices from \kGG{k}{}. Then we use Theorem~\ref{Tutte} and Theorem~\ref{Berge}, respectively presented by Tutte~\cite{Tutte1947} and Berge~\cite{Berge1958}, to prove a lower bound on the size of a maximum matching in \kGG{k}{}. 

\begin{figure}[htb]
  \centering
\setlength{\tabcolsep}{0in}
  
  \caption{The point set  of 16 points is partitioned into open/closed disks, open/closed squares, and crosses. (a) The graph , (b) The set  of straight-line edges corresponding to  is in bold, and the set  of their corresponding disks.}
\label{partition-fig}
\end{figure}

Let  be a partition of the points in . For two sets  and  in  define the distance  as the smallest Euclidean distance between a point in  and a point in , i.e., .
Let  be the complete edge-weighted graph with vertex set . For each edge  in , let . This edge  is defined by two points  and , where  and . Therefore, an edge  corresponds to a straight line edge  in ; see Figure~\ref{partition-fig}(a). Let  be a minimum spanning tree of . It is obvious that each edge  in  corresponds to a straight line edge  in . Let  be the set of all these straight line edges. Let  be the set of disks which have the edges of  as diameter, i.e., . See Figure~\ref{partition-fig}(b). 



\begin{observation}
 \label{T-plane}
 is a subgraph of a minimum spanning tree of , and hence  is plane.
\end{observation}

\begin{lemma}
 \label{D-empty}
A disk  does not contain any point of .
\end{lemma}
\begin{proof}
  Let  be the edge in  corresponding to . Note that . By contradiction, suppose that  contains a point . Three cases arise: (i) , (ii) , (iii)  where  and . In case (i) the edge  between  and  is smaller than ; contradicting that  in .  In case (ii) the edge  between  and  is smaller than ; contradicting that  in . In case (iii) the edge  (resp. ) between  and  (resp.  and ) is smaller than ; contradicting that  is an edge in . 
\end{proof}


\begin{lemma}
\label{center-in-lemma}
 For each pair  and  of disks in ,  (resp. ) does not contain the center of  (resp ).
\end{lemma}

\begin{proof}
 Let  and  respectively be the edges of  which correspond to  and . Let  and  be the circles representing the boundary of  and . W.l.o.g. assume that  is the bigger circle, i.e., . By contradiction, suppose that  contains the center  of . Let  and  denote the intersections of  and . Let  (resp. ) be the intersection of  (resp. ) with the line through  and  (resp. ). Similarly, let  (resp. ) be the intersection of  (resp. ) with the line through  and  (resp. ). 

\begin{figure}[htb]
  \centering
  \includegraphics[width=.6\columnwidth]{fig/center-in.pdf}
 \caption{Illustration of Lemma~\ref{center-in-lemma}:  and  intersect, and  contains the center of .}
  \label{center-in-fig}
\end{figure}

As illustrated in Figure~\ref{center-in-fig}, the arcs , , , and  are the potential positions for the points , , , and , respectively. First we will show that the line segment  passes through  and . The angles  and  are right angles, thus the line segment  goes through . Since  (resp. ), for any point  (resp. ). Therefore, 
Consider triangle  which is partitioned by segment  into  and . It is easy to see that  in  is equal to  in , and the segment  is shared by  and . Since  is inside  and , the angle . Thus,  in  is smaller than  (and hence smaller than  in ). That is,   in  is smaller than  in . Therefore,



By symmetry . Therefore . In addition  is a cycle and at least one of  and  does not belong to . This contradicts Lemma~\ref{not-mst-edge} (Note that by Observation~\ref{T-plane},  is a subgraph of a minimum spanning tree of ).
\end{proof}

Now we show that four disks in  cannot intersect mutually. In other words, every point in the plane cannot lie in more than three disks in . In Section~\ref{proof-section} we prove the following theorem, and in Section~\ref{lower-bounds-section} we present the lower bounds on the size of a maximum matching in \kGG{k}{}.

\begin{theorem}
 \label{four-circle-theorem}
For every four disks , .
\end{theorem}
\subsection{Proof of Theorem~\ref{four-circle-theorem}}
\label{proof-section}
Let   and let  be a point in . Let  be the edge in  which corresponds to , let  be the center of , and let  denote the boundary of , where . Denote the angle  by , where . Since  is a diameter of  and  lies in , . First we prove the following observation.
\begin{observation}
\label{inclusion-exclusion}
 For  where , the angles  and  are disjoint or one is completely contained in the other.
\end{observation}
\begin{proof}
The proof is by contradiction. Suppose that  and  share some part and w.l.o.g. assume that  is in the cone which is defined by  and  is in the cone which is defined by . Three cases arise:
\begin{itemize}
 \item . In this case  is inside  which contradicts Lemma~\ref{D-empty}.
 \item . In this case  is inside  which contradicts Lemma~\ref{D-empty}.
 \item  and . In this case  intersects  which contradicts Observation~\ref{T-plane}.
\end{itemize}
\end{proof}

We call  a {\em blocked angle} if  is contained in an angle  where , otherwise we call  a {\em free angle}.

\begin{lemma}
\label{not-all-free-angles}
At least one , where , is blocked.
\end{lemma}
\begin{proof}
Suppose that all angles , where , are free. This implies that the s are pairwise disjoint and . If , we obtain a contradiction to the fact that the sum of the disjoint angles around  is at most . If , then the four edges  where , form a cycle which contradicts the fact that  is a subgraph of a minimum spanning tree of .
\end{proof}

\begin{figure}[htb]
  \centering
\setlength{\tabcolsep}{0in}
  
  \caption{(a) The point  should be inside the arc . (b) The  which consists of two almond-shaped regions known as  and .}
\label{trap-fig}
\end{figure}

By Lemma~\ref{not-all-free-angles} at least one of the angles is blocked. Hereafter, assume that  is blocked by  where  and . W.l.o.g. assume that  is a vertical line segment and the point  (which belongs to ) is to the left of . Thus,  and  are to the right of . This implies that . See Figure~\ref{trap-fig}(a). By Lemma~\ref{center-in-lemma},  cannot be inside , thus either  or , but not both. W.l.o.g. assume that . Let  be the circle with radius  which is centred at . Let  denote the intersection of  with  which is to the right of . Consider the circle  with radius  centred at . Let  be the closed arc of  to the left of  as shown in Figure~\ref{trap-fig}(a).

We show that  cannot be outside . By contradiction suppose that  is outside  (and to the left of ). Let  and  respectively be the perpendicular bisectors of  and . Let  and  respectively be the intersection of  and  with  and let  be the intersection point of  and . Since  is outside , the intersection point  is to the left of (the vertical line through)  and inside triangle . If  is below  then  and  contains  which contradicts Lemma~\ref{center-in-lemma}. If  is above  then  and  contains  which contradicts Lemma~\ref{center-in-lemma}. Thus,  is above  and below , and (by the initial assumption) to the right of . That is,  is in triangle . Since ,  lies inside  which contradicts Lemma~\ref{center-in-lemma}. Therefore,  is contained in . 

By symmetry  can intersect  and/or  can be to the left of  as well. Therefore, if  blocks , the point  can be in  or any of the symmetric arcs. For an edge  we denote the union of these arcs by  which is shown in Figure~\ref{trap-fig}(b). For each disk , let  where  is the edge in  corresponding to . Therefore  is contained in  which implies that  Note that  consists of two almond-shaped symmetric regions; for simplicity we call them  and , i.e., .

\begin{lemma}
\label{angle-in-trap}
 For any point , .
\end{lemma}
\begin{proof}
 See Figure~\ref{trap-fig}(a). The angle , which implies that . Thus, for any point  on the arc , , and hence for any point  in , . This implies that in , . On the other hand , which proves the lemma.
\end{proof}

\begin{figure}[htb]
  \centering
\setlength{\tabcolsep}{0in}
  
  \caption{Illustration of Lemma~\ref{intersecting-trap}.}
\label{trap-intersection-fig}
\end{figure}


\begin{lemma}
\label{intersecting-trap}
For any two disks  and  in , .
\end{lemma}
\begin{proof}
 We prove this lemma by contradiction. Suppose  and w.l.o.g. assume that  as shown in Figure~\ref{trap-intersection-fig}. Connect  to , , , and  ( may be identified with ). As shown in the proof of Lemma~\ref{angle-in-trap}, . Two configurations may arise: 
\begin{itemize}
 \item . In this case . W.l.o.g. assume that  which implies that ; see Figure ~\ref{trap-intersection-fig}(a). Clearly , and hence . Thus,  contains  which contradicts Lemma~\ref{center-in-lemma}.
  \item . In this case  and , hence  and . Three configurations arise:

\begin{itemize}
 \item , in this case  and hence  contains . See Figure~\ref{trap-intersection-fig}(b).
  \item , in this case  and hence  contains . 
  \item  and , in this case w.l.o.g. assume that . Thus  which implies that  contains . See Figure~\ref{trap-intersection-fig}(b).
\end{itemize}
All cases contradict Lemma~\ref{D-empty}. 
\end{itemize}
\end{proof}

Recall that each blocking angle is representing a trap. Thus, by Lemma~\ref{not-all-free-angles} and Lemma~\ref{intersecting-trap}, we have the following corollary:

\begin{corollary}
\label{one-blocked-angle}
Exactly one , where , is blocked.
\end{corollary}
Recall that  is blocked by ,  is vertical line segment,  is to the right of , and . As a direct consequence of Corollary~\ref{one-blocked-angle}, , , and  are free angles, where  and . In addition,  and  are to the left of . It is obvious that 

\begin{figure}[htb]
  \centering
  \includegraphics[width=.5\columnwidth]{fig/trap2.pdf}
 \caption{Illustration of Lemma~\ref{intersecting-trap2}.}
  \label{trap2-fig}
\end{figure}

\begin{lemma}
\label{intersecting-trap2}
For a blocking angle  and free angles  and , . 
\end{lemma}
\begin{proof}
Since  is a blocking angle and ,  are free angles,  and  are on the same side of .
 By contradiction, suppose that . See Figure~\ref{trap2-fig}. It is obvious that , , and . By Lemma~\ref{angle-in-trap}, . In addition . Thus, . Hence, , , and . Therefore, . In addition  is a cycle and at least one of ,  and  does not belong to . This contradicts Lemma~\ref{not-mst-edge}. 
\end{proof}
Thus, ; which complete the proof of Theorem~\ref{four-circle-theorem}.

\subsection{Lower Bounds}
\label{lower-bounds-section}

In this section we present some lower bounds on the size of a maximum matching in \kGG{2}{}, \kGG{1}{}, and \kGG{0}{}.

\begin{theorem}
 \label{matching-2GG}
For a set  of an even number of points, \kGG{2}{} has a perfect matching.
\end{theorem}
\begin{proof}
First we show that by removing a set  of  points from \kGG{2}{}, at most  components are generated. Then we show that at least one of these components must be even. Using Theorem~\ref{Tutte}, we conclude that \kGG{2}{} has a perfect matching.

Let  be a set of  vertices removed from \kGG{2}{}, and let  be the resulting  components, where  is a function depending on . Actually  and  is a partition of the vertices in . 

{\bf\em  Claim 1.} . Let  be the complete graph with vertex set  which is constructed as described above. Let  be the set of all edges in  corresponding to the edges of  and let  be the set of disks corresponding to the edges of . It is obvious that  contains  edges and hence . Let  be the set of all (point, disk) pairs where , , and  is inside . By Theorem~\ref{four-circle-theorem} each point in  can be inside at most three disks in . Thus, .
Now we show that each disk in  contains at least three points of  in its interior.  
Consider any disk  and let  be the edge of  corresponding to . By Lemma~\ref{D-empty},  does not contain any point of . Therefore,  contains at least three points of , because otherwise  is an edge in \kGG{2}{} which contradicts the fact that  and  belong to different components in . Thus, each disk in  has at least three points of . That is, . Therefore, , and hence .

{\bf \em Claim 2}: . By Claim 1, . If , then . Assume that . Since , the total number of vertices of  is equal to . Consider two cases where (i)  is odd, (ii)  is even. In both cases if all the components in  are odd, then  is odd; contradicting our assumption that  has an even number of vertices. Thus,  contains at least one even component, which implies that .

Finally, by Claim 2 and Theorem~\ref{Tutte}, we conclude that \kGG{2}{} has a perfect matching.
\end{proof}

\begin{theorem}
\label{matching-1GG}
For every set  of  points, \kGG{1}{} has a matching of size at least .
\end{theorem}

\begin{proof}
Let  be a set of  vertices removed from \kGG{1}{}, and let  be the resulting  components. Actually  and  is a partition of the vertices in . Note that .
Let  be a maximum matching in \kGG{1}{}. By Theorem~\ref{Berge}, 



where


Define , , , and  as in the proof of Theorem~\ref{matching-2GG}. By Theorem~\ref{four-circle-theorem}, .
By the same reasoning as in the proof of Theorem~\ref{matching-2GG}, each disk in  has at least two points of  in its interior. Thus, . Therefore, , and hence

 

In addition, , and hence



By Inequalities~(\ref{ineq1}) and ~(\ref{ineq2}), 



Thus, by (\ref{align1}) and (\ref{ineq3})



where the last equation is achieved by setting  equal to , which implies . Finally by substituting (\ref{align2}) in Equation (\ref{align0}) we have

\end{proof}

By similar reasoning as in the proof of Theorem~\ref{matching-1GG} we have the following Theorem.

\begin{figure}[htb]
  \centering
  \includegraphics[width=.6\columnwidth]{fig/tight-0GG.pdf}
 \caption{A \kGG{0}{} of   points with a maximum matching of size  (bold edges). The dashed edges do not belong to the graph because any of their corresponding closed disks has a point on its boundary.}
  \label{tight-0GG}
\end{figure}

\begin{theorem}
\label{matching-0GG}
For every set  of  points, \kGG{0}{} has a matching of size at least .
\end{theorem}

The bound in Theorem~\ref{matching-0GG} is tight, as can be seen from the graph in Figure~\ref{tight-0GG}, for which the maximum matching has size . Actually this is a Gabriel graph of maximum degree four which is a tree. The dashed edges do not belong to \kGG{0}{} because any closed disk which has one of these edges as diameter has a point on its boundary. Observe that each edge in any matching is adjacent to one of the vertices of degree four.

\begin{paragraph}{Note:}For a point set , let  and  respectively denote the size of a maximum matching and a maximum independent set in \kGG{k}{}. For every edge in the maximum matching, at most one of its endpoints can be in the maximum independent set. Thus,
By combining this formula with the results of Theorems ~\ref{matching-0GG}, \ref{matching-1GG}, \ref{matching-2GG}, respectively, we have , , and . The \kGG{0}{} graph in Figure~\ref{tight-0GG} has an independent set of size , which shows that this bound is tight for \kGG{0}{}. On the other hand, \kGG{0}{} is planar and every planar graph is 4-colorable; which implies that . There are some examples of \kGG{0}{} in \cite{Matula1980} such that , which means that this bound is tight as well.
\end{paragraph}

\section{Blocking Higher-Order Gabriel Graphs}
\label{blocking-section}
In this section we consider the problem of blocking higher-order Gabriel graphs. Recall that a point set  blocks \kGG{k}{(P)} if in \kGG{k}{(P\cup K)} there is no edge connecting two points in . 

\begin{theorem}
\label{blocking-thr1}
For every set  of  points, at least  points are necessary to block \kGG{0}{(P)}.
\end{theorem}
\begin{proof}
Let  be a set of  points which blocks \kGG{0}{(P)}. Let  be the complete graph with vertex set . Let  be a minimum spanning tree of  and let  be the set of closed disks corresponding to the edges of . It is obvious that . By Lemma~\ref{D-empty} each disk  does not contain any point of , thus,  . To block each edge of , corresponding to a disk in , at least one point is necessary. By Theorem~\ref{four-circle-theorem} each point in  can lie in at most three disks of . Therefore, , which implies that at least  points are necessary to block all the edges of  and hence \kGG{0}{(P)}.
\end{proof}
\begin{figure}[htb]
  \centering
\setlength{\tabcolsep}{0in}
  
  \caption{(a) \kGG{0}{} graph of  points (in bold edges) which is blocked by  white points, (b) dashed edges do not belomg to \kGG{0}{}.}
\label{blocking-fig}
\end{figure}

Figure~\ref{blocking-fig}(a) shows a \kGG{0}{} with  (black) points which is blocked by  (white) points. Note that all the disks, corresponding to the edges of every cycle, intersect at the same point in the plane (where we have placed the white points). As shown in Figure~\ref{blocking-fig}(b), the dashed edges do not belong to \kGG{0}{}. Thus, the lower bound provided by Theorem~\ref{blocking-thr1} is tight. It is easy to generalize the result of Theorem~\ref{blocking-thr1} to higher-order Gabriel graphs. Since in a \kGG{k}{} we need at least  points to block an edge of  and each point can be inside at most three disks in , we have the following corollary:

\begin{corollary}
For every set  of  points, at least  points are necessary to block \kGG{k}{(P)}.
\end{corollary}

In \cite{Aronov2013} the authors showed that every Gabriel graph can be blocked by a set  of  points by putting a point slightly to the right of each point of , except for the rightmost one. Every disk with diameter determined by two points of  will contain a point of . Using a similar argument one can block a \kGG{k}{} by putting  points slightly to the right of each point of , except for the rightmost one. Thus,

\begin{corollary}
 For every set  of  points, there exists a set of  points that blocks \kGG{k}{(P)}.
\end{corollary}

Note that this upper bound is tight, because if the points of  are on a line, the disks representing the minimum spanning tree are disjoint and each disk needs  points to block the corresponding edge.

\section{Conclusion}
\label{conclusion}
In this paper, we considered the bottleneck and perfect matching admissibility of higher-order Gabriel graphs. We proved that
\begin{itemize}
  \item \kGG{10}{} contains a Euclidean bottleneck matching of  and \kGG{8}{} may not have any.
  \item \kGG{0}{} has a matching of size at least  and this bound is tight.
  \item \kGG{1}{} has a matching of size at least .
  \item \kGG{2}{} has a perfect matching.
    \item  points are necessary to block \kGG{0}{} and this bound is tight.
  \item  points are necessary and  points are sufficient to block \kGG{k}{}.
\end{itemize}
We leave a number of open problems:
\begin{itemize}
  \item Does \kGG{9}{} contain a Euclidean bottleneck matching of ?
  \item What is a tight lower bound on the size of a maximum matching in \kGG{1}{}?
\end{itemize}

\bibliographystyle{abbrv}
\bibliography{GG-matching.bib}


\end{document}
