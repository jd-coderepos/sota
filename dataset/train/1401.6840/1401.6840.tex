\subsection{Zero-Reachability, Case~I}
\label{sec-case1}

For the rest of this section, let us fix a (non-labeled) 
pMC  of dimension  and 
a configuration . 

Our aim is to identify the conditions under which
. To achieve that, we 
first consider a (non-labeled) finite-state Markov chain
 where  iff 

Here  is the probability assignment
for the rules defined as follows (we write 
instead of ):
\begin{itemize}
\item For every rule  where 
  we put .
\item ,
  where  is the total weight of all rules of the form
  .
\end{itemize}
Intuitively, a state  of  captures the behavior of 
configurations  where all components of  are
positive. 

Further, we partition the states of 
 into SCCs  according to~. Note that every 
run  eventually \emph{stays} in precisely
one , i.e., there is precisely one  
such that for some , the control state of every ,
where , belongs to~. 
We use  to denote 
the set of all  that stay in~. 
Obviously,


  For any  denote by  the probability that a run  initiated in  satisfies the following for every :  does not belong to any BSCC of  and . The following lemma shows that  decays exponentially fast.

\begin{lemma}\label{lem:F_A-BSCC}
  For any  we have  where  is the minimal positive transition probability in .
  In particular, for any non-bottom SCC  of  we have 
  .
\end{lemma}
\begin{proof}
 The lemma immediately follows from the fact that for every configuration  there is a path (in ) of length at most  to a configuration  satisfying either  or  for some BSCC  of .
\end{proof}


Now, let  be a BSCC of . For every , let 
 be a -dimensional vector of \emph{expected counter changes}
given by 

Note that  can be seen
as a finite-state irreducible Markov chain, and hence there exists the
unique \emph{invariant distribution}  on the states of~ 
(see, e.g.,~\cite{KS:book}) satisfying

The \emph{trend} of  is a -dimensional vector   defined by

Further, for every  and every , we denote
by  the \emph{least}  such that 
for every configuration  where ,
there is \emph{no}  where
counter~ is zero in the last configuration of  and all counters
stay positive in every , where .
If there is no such , we put .
It is easy to show that if , then ; and if , then  for all . Moreover, if , then there is a -safe finite path of length at most  from  to a configuration with -th counter equal to 0, where  and  for . In particular, the number  is computable in time polynomial in .

We say that counter~ is
\emph{decreasing} in  if  for some (and
hence all) .

\begin{definition}
  Let  be a BSCC of  with trend , and let 
  . We say that counter~ is \emph{diverging}
  in  if either , or  and the counter~ 
  is not decreasing in~.
\end{definition}
Intuitively, our aim is to prove that  iff
all counters are diverging in  and  can reach a configuration 
 (via a -safe finite path) where all components of 
 are ``sufficiently large''.
To analyze the individual counters, for every 
we introduce a~(labeled) \emph{one-dimensional} pMC
which faithfully simulates the behavior of counter~
and ``updates'' the other counters just symbolically in the labels.

\begin{definition} 
  Let , and let 
  be an -labeled pMC of dimension one such that 
  \begin{itemize}\itemsep1ex
  \item  \ iff \ 
        ;
  \item  \ iff \ 
        ;
  \item .
  \item .
  \end{itemize}
  Here, 
  .
\end{definition}

\noindent
Observe that the symbolic updates of the counters different from~ 
``performed'' in the labels of  mimic the real updates performed
by  in configurations where all of these counters are
positive. 

Given a run  in  and , we denote by  the vector , and given , we denote by  the number  (i.e., the -th component of ).

Let  be a function which for a given run 
 of
 returns a run

of  where the label  
corresponds to the update in the abstracted counters performed in
the transition ,
i.e., . 
The next lemma is immediate.
\begin{lemma}\label{prop:one-counter-runs}
For all  and 
 we have that
\begin{itemize}
\item ,
\item .
\end{itemize}
Further, for every measurable set 
 we have that  is measurable and
\end{lemma}

\noindent
Now we examine the runs of  where  is a BSCC of
 such that some counter is not diverging in~. A proof of
the next lemma can be found in Appendix~\ref{app-sec1}.

\begin{lemma}
\label{lem:not-diverging}
  Let  be a BSCC of .
  If some counter is not diverging in , then .
\end{lemma}

It remains to consider the case when  is a BSCC of 
where all counters are diverging. Here we use the results of
\cite{BKK:pOC-time-LTL-martingale} which allow to derive
a bound on divergence probability in one-dimensional pMC.
These results are based on designing and analyzing a suitable
martingale for one-dimensional pMC. 

\begin{lemma}
\label{lem-divergence}
  Let  be a -dimensional pMC, let  be a BSCC of  
  such that the trend  of the only counter in  is positive and let  where  is the smallest non-zero transition probability in .
  Then for all  and  we have that  
  , where  and .
\end{lemma}
\begin{proof}
Denote by  the probability that a run initiated in  visits a configuration with zero counter value for the first time in exactly  steps. By Proposition~7 of \cite{BKK:pOC-time-LTL-martingale-arxiv} we obtain for all ~\footnote{The precise bound on  is given in Proposition~7~\cite{BKK:pOC-time-LTL-martingale-arxiv}.},

where  for ~\footnote{The bound on  is given in Proposition 6~\cite{BKK:pOC-time-LTL-martingale-arxiv}.}.

Thus

\end{proof}



\begin{definition}
  Let  be a BSCC of  where all counters are diverging,
  and let . We say that a configuration 
  is \emph{above} a given  if  for every
   such that , and 
   for every
   such that . 
\end{definition}


\begin{lemma}
\label{lem-diverging}
  Let  be a BSCC of  where all counters are diverging.
  Then  iff there is a -safe finite 
  path of the form 
  where ,  is above , ,
  and  for every  such that . 


\end{lemma}
\begin{proof}
  We start with ``''. 
  Let  be the trend of . We show that for almost
  all  and all , one of
  the following conditions holds:
  \begin{enumerate}
  \item[(A)]  and ,
  \item[(B)]  and  
    for all 's large enough.
  \end{enumerate}
  First, recall that  is also a BSCC of , and realize that
  the trend of the (only) counter in the BSCC  of  is~.

  Concerning~(A), it follows, e.g., from the results of
  \cite{BKK:pOC-time-LTL-martingale}, that almost all runs
   that stay
  in  and do not visit a configuration with zero counter 
  satisfy .
  In particular, this means that almost all 
  
  satisfy this property.
  Hence, by Lemma~\ref{prop:one-counter-runs}, for almost all 
   we have that
  \mbox{}.

  Concerning (B), note that almost all runs 
  satisfying 
  for infinitely many 's eventually visit zero in some counter 
  (there is a path of length at most  from each such
   to a configuration with zero in counter , or in one of the
  other counters).  

  The above claim immediately implies that for every , 
  almost every run of  visits a configuration 
   above~. Hence, there must be a -safe 
  path of the form 
  with the required properties.


  ``'': If there is a -safe 
  path of the form 
  where ,  is above , ,
  and  for every  such that , then
   can a reach a configuration  above~ for an 
  arbitrarily large  via a -safe path.

  By Lemma~\ref{lem-divergence}, there exists
   such that for every  where 
   and every ,
  the probability of all  that 
  visit a configuration with zero counter is strictly smaller than 
  \mbox{}. Let  be a configuration above~ reachable
  from  via a -safe path (the existence of such 
  a  follows from the existence of
  ). It suffices
  to show that . For every 
   where ,
  let  be the set of all  such that
   for some  and
  all counters stay positive in all  where . 
  Clearly, , and thus we obtain
  
\end{proof}

The following lemma shows that it is possible to decide, whether for a given  a configuration above  can be reached via a -safe path. Its proof uses the results of~\cite{BG:VASS-coverability} on the coverability problem in (non-stochastic) VASS.

\begin{lemma}
\label{lem:cover-short-path}
 Let  be a BSCC of  where all counters are diverging and let . There is a -safe finite path of the form  with  is above some  iff there is a -safe finite path of length at most  of the form  with  is above . Moreover, the existence of such a path can be decided in time  where  is a fixed constant independent of  and .
\end{lemma}
\begin{proof}
  We employ a decision procedure of \cite{BG:VASS-coverability} for
  VASS coverability. Since we need to reach  above  via
  a -safe finite path, we transform  into a
  (non-probabilistic) VASS  whose control states and rules are
  determined as follows: for every rule 
  of , we add to  the control states  together with two
  auxiliary fresh control states , and we also add the rules
  , ,
  . Hence,  behaves like , but when
  some counter becomes zero, then  is stuck (i.e., no transition
  is enabled except for the self-loop). Now it is easy to check that
   can reach a configuration  above  via a
  -safe finite path in  iff  can reach a
  configuration  above  via \emph{some} finite path in
  , which is exactly the coverability problem for VASS. 
  Theorem~1 in~\cite{BG:VASS-coverability} shows that such a
  configuration can be reached iff there is configuration 
  above  reachable via some finite path of length at most . (The term 
  represents the number of control states of .) This path
  induces, in a natural way, a -safe path from  to
   in  of length at most . Moreover, Theorem~2
  in~\cite{BG:VASS-coverability} shows that the existence of such a
  path in  can be decided in time
  , which proves
  the lemma. \end{proof}



\begin{theorem}
\label{thm:qual-all-algorithm}
  The qualitative -reachability problem for \mbox{-dimensional}
  pMC is decidable in time , where  is 
  a fixed constant independent of  and . 
\end{theorem}
\begin{proof}
  Note that the Markov chain  is computable in time polynomial
  in  and , and we can efficiently identify all diverging
  BSCCs of . For each diverging BSCC , we need to check the
  condition of Lemma~\ref{lem-diverging}. By applying Lemma~2.3.{}
  of \cite{RY:VASS-JCSS}, we obtain that if there exist \emph{some}  
  above~ and a -safe finite path of the form 
   such that 
  and  for every  where ,
  then such a path exists for \emph{every}  above
   and its length is bounded by . Here
   is a fixed constant independent of  and~ (let us
  note that Lemma~2.3.{} of \cite{RY:VASS-JCSS} is formulated for 
  vector addition
  systems without states and a non-strict increase in every counter, 
  but the corresponding result for VASS is easy
  to derive; see also Lemma~15 in \cite{BJK:VASS-games-arxiv}). 
  Hence, the existence
  of such a path for a given  can be decided in 
   time. It remains to check whether 
  can reach a configuration  above   via
  a -safe finite path. By Lemma~\ref{lem:cover-short-path} this can be done in time  for another constant . This gives us the desired complexity bound.
\end{proof}




















\smallskip

Note that for every fixed dimension , the qualitative 
\mbox{-reachability} problem is solvable in polynomial time.









Now we show that  can be effectively 
approximated up to an arbitrarily small absolute/relative error 
. A full proof of Theorem~\ref{thm:approx-general}
can be found in Appendix~\ref{app-approx}.

\begin{theorem}
 \label{thm:approx-general}
 For a given -dimensional pMC  and its initial configuration
 , the probability  can be
 approximated up to a given absolute error  in time
 .
\end{theorem}
\begin{proof}[Proof sketch]
  First we check whether    
  (using the algorithm of Theorem~\ref{thm:qual-all-algorithm}) and 
  return  if it is the case. Otherwise, we first show how
  to approximate  under the 
  assumption that  is in some diverging BSCC of , and 
  then we show how to drop this assumption.

  So, let  be a diverging BSCC of  such that 
  , and let us assume that . 
  We show how to compute  such that 
   in time 
  . We proceed by induction
  on~. The key idea of the inductive step is to find a sufficiently
  large constant~ such that if some counter reaches~, it can
  be safely ``forgotten'', i.e., replaced by , without influencing
  the probability of reaching zero in some counter by more than 
  . Hence, whenever we visit a configuration 
  where some counter value in  reaches , we can
  apply induction hypothesis and approximate the probability or reaching 
  zero in some counter from  by ``forgetting'' the large
  counter a thus reducing the dimension. Obviously, there are only 
  finitely many configurations where all counters are below~, and
  here we employ the standard methods for finite-state Markov chains. 
  The number  is computed by using
  the bounds of Lemma~\ref{lem-divergence}. 

  Let us note that the base (when ) is handled by relying only
  on Lemma~\ref{lem-divergence}. Alternatively, we could employ
  the results of \cite{ESY:polynomial-time-termination}. This would
  improve the complexity for , but not for higher 
  dimensions. 

  Finally, we show how to approximate 
  when the control state  does not belong to a BSCC of~.
  Here we use the bound of Lemma~\ref{lem:F_A-BSCC}.
\end{proof}

Note that if , then this probability is
at least  where  is the least positive
transition probability in  and  is the maximal component 
of . Hence, Theorem~\ref{thm:approx-general} can also be
used to approximate  up to a given
\emph{relative} error .

