\subsection{Zero-Reachability, Case~I}
\label{sec-case1}

For the rest of this section, let us fix a (non-labeled) 
pMC $\A = (Q,\gamma,W)$ of dimension $d \in \Nset^+$ and 
a configuration $p\vec{v}$. 

Our aim is to identify the conditions under which
$\calP(\run(p\vec{v},\neg\Zall)) > 0$. To achieve that, we 
first consider a (non-labeled) finite-state Markov chain
$\F_{\A} = (Q,\btran{},\Prob)$ where $q \btran{x} r$ iff 
\[
  x \quad = \quad 
  \sum_{(q,\vec{\alpha},\emptyset,r) \in \gamma} P_\emptyset(q,\vec{\alpha},\emptyset,r) 
  \quad > \quad 0.
\]
Here $P_\emptyset : \gamma \rightarrow [0,1]$ is the probability assignment
for the rules defined as follows (we write $P_\emptyset(q,\vec{\alpha},\emptyset,r)$
instead of $P_\emptyset((q,\vec{\alpha},\emptyset,r))$):
\begin{itemize}
\item For every rule $(p,\vec{\alpha},c,q)$ where $c \neq \emptyset$
  we put $P_\emptyset(p,\vec{\alpha},c,q) = 0$.
\item $P_\emptyset(p,\vec{\alpha},\emptyset,q) = W((p,\alpha,\emptyset,q))/T$,
  where $T$ is the total weight of all rules of the form
  $(p,\vec{\alpha}',\emptyset,q')$.
\end{itemize}
Intuitively, a state $q$ of $\F_{\A}$ captures the behavior of 
configurations $q \vec{u}$ where all components of $\vec{u}$ are
positive. 

Further, we partition the states of 
$Q$ into SCCs $C_1,\ldots,C_m$ according to~$\hookrightarrow$. Note that every 
run $w \in \run(p\vec{v})$ eventually \emph{stays} in precisely
one $C_j$, i.e., there is precisely one $1\leq j \leq m$ 
such that for some $k \in \Nset$, the control state of every $w(k')$,
where $k' \geq k$, belongs to~$C_i$. 
We use $\runzc{p\vec{v}}{C_j}$ to denote 
the set of all $w \in \run(p\vec{v},\neg\Zall)$ that stay in~$C_j$. 
Obviously,
\[
  \run(p\vec{v},\neg\Zall) = 
  \run(p\vec{v},C_1) \uplus \cdots \uplus \runzc{p\vec{v}}{C_m}.
\]

  For any $n \in \Nset$ denote by $P_n$ the probability that a run $w$ initiated in $p\vec{v}$ satisfies the following for every $0\leq i \leq n$: $\state(w(i))$ does not belong to any BSCC of $\F_\A$ and $Z(w(i))=\emptyset$. The following lemma shows that $P_n$ decays exponentially fast.

\begin{lemma}\label{lem:F_A-BSCC}
  For any $n \in \Nset$ we have $$P_n \leq (1 - \pmin^{|Q|})^{\lfloor\frac{n}{|Q|}\rfloor},$$ where $\pmin$ is the minimal positive transition probability in $\M_\A$.
  In particular, for any non-bottom SCC $C$ of $\F_\A$ we have 
  $\calP(\run(p\vec{v},C)) = 0$.
\end{lemma}
\begin{proof}
 The lemma immediately follows from the fact that for every configuration $p\vec{v}$ there is a path (in $\A$) of length at most $|Q|$ to a configuration $q\vec{u}$ satisfying either $Z(q\vec{u})\neq \emptyset$ or $q\in D$ for some BSCC $D$ of $\F_\A$.
\end{proof}


Now, let $C$ be a BSCC of $\F_{\A}$. For every $q \in C$, let 
$\change^q$ be a $d$-dimensional vector of \emph{expected counter changes}
given by 
\[
   \change^q_i = 
   \sum_{(q,\vec{\alpha},\emptyset,r) \in \gamma} 
     P_{\emptyset}(q,\vec{\alpha},\emptyset,r) \cdot \vec{\alpha}[i] \,.
\]
Note that $C$ can be seen
as a finite-state irreducible Markov chain, and hence there exists the
unique \emph{invariant distribution} $\mu$ on the states of~$C$ 
(see, e.g.,~\cite{KS:book}) satisfying
\[
\mu(q)\quad =\quad \sum_{r\btran{x} q} \mu(r)\cdot x \,.
\]
The \emph{trend} of $C$ is a $d$-dimensional vector $\vec{t}$  defined by
\[
  \vec{t}[i] \quad = \quad \sum_{q \in C} \mu(q) \cdot \change^q_i \,.
\]
Further, for every $i \in \{1,\ldots,d\}$ and every $q\in C$, we denote
by $\bottomfin_i(q)$ the \emph{least} $j \in \Nset$ such that 
for every configuration $q \vec{u}$ where $\vec{u}[i] = j$,
there is \emph{no} $w \in \fpath_{\M_\A}(q \vec{u})$ where
counter~$i$ is zero in the last configuration of $w$ and all counters
stay positive in every $w(k)$, where $0 \leq k < \len(w)$.
If there is no such $j$, we put $\bottomfin_i(q) = \infty$.
It is easy to show that if $\bottomfin_i(q) < \infty$, then $\bottomfin_i(q)
\leq |C|$; and if $\bottomfin_i(q) = \infty$, then $\bottomfin_i(r)
= \infty$ for all $r \in C$. Moreover, if $\bottomfin_i(q) < \infty$, then there is a $\Zminusi{i}$-safe finite path of length at most $|C|-1$ from $q\vec{u}$ to a configuration with $i$-th counter equal to 0, where $\vec{u}[i]=\bottomfin_i(q)-1$ and $\vec{u}[\ell]=|C|$ for $\ell\neq i$. In particular, the number $\bottomfin_i(q)$ is computable in time polynomial in $|C|$.

We say that counter~$i$ is
\emph{decreasing} in $C$ if $\bottomfin_i(q) = \infty$ for some (and
hence all) $q \in C$.

\begin{definition}
  Let $C$ be a BSCC of $\F_{\A}$ with trend $\vec{t}$, and let 
  $i \in \{1,\ldots,d\}$. We say that counter~$i$ is \emph{diverging}
  in $C$ if either $\vec{t}[i] > 0$, or $\vec{t}[i] = 0$ and the counter~$i$ 
  is not decreasing in~$C$.
\end{definition}
Intuitively, our aim is to prove that $\calP(\run(p\vec{v},C))>0$ iff
all counters are diverging in $C$ and $p\vec{v}$ can reach a configuration 
$q\vec{u}$ (via a $\Zall$-safe finite path) where all components of 
$\vec{u}$ are ``sufficiently large''.
To analyze the individual counters, for every $i \in \{1,\ldots,d\}$
we introduce a~(labeled) \emph{one-dimensional} pMC
which faithfully simulates the behavior of counter~$i$
and ``updates'' the other counters just symbolically in the labels.

\begin{definition} 
  Let $L = \{-1,0,1\}^{d-1}$, and let $\B_i = (Q,\hat{\gamma},\hat{W})$
  be an $L$-labeled pMC of dimension one such that 
  \begin{itemize}\itemsep1ex
  \item $(q,j,\emptyset,\vec{\beta},r) \in \hat{\gamma}$ \ iff \ 
        $(q,\langle \vec{\beta},j\rangle_i,\emptyset,r) \in \gamma$;
  \item $(q,j,\{1\},\vec{\beta},r) \in \hat{\gamma}$ \ iff \ 
        $(q,\langle \vec{\beta},j\rangle_i,\{i\},r) \in \gamma$;
  \item $\hat{W}(q,j,\emptyset,\vec{\beta},r) = 
         W(q,\langle \vec{\beta},j \rangle_i,\emptyset,r)$.
  \item $\hat{W}(q,j,\{1\},\vec{\beta},r) = 
         W(q,\langle \vec{\beta},j \rangle_i,\{i\},r)$.
  \end{itemize}
  Here, 
  $\langle (j_1,\ldots,j_{d-1}),j\rangle_i = 
   (j_1,\ldots,j_{i-1},j,j_i,\ldots,j_{d-1})$.
\end{definition}

\noindent
Observe that the symbolic updates of the counters different from~$i$ 
``performed'' in the labels of $\B_i$ mimic the real updates performed
by $\A$ in configurations where all of these counters are
positive. 

Given a run $w\equiv p_0(v_0) \, \vec{\alpha}_0 \, 
p_1(v_1) \, \vec{\alpha}_1 \, p_2(v_2) \, \vec{\alpha}_2 \, \ldots$ in $\run_{\M_{\B}}(p_0(v_0))$ and $k\in \Nset$, we denote by $\totalrew{}{w}{0}{k}$ the vector $\sum_{n=0}^{k-1} \vec{\alpha}_n$, and given $j\in \{1,\ldots,d\}\smallsetminus \{i\}$, we denote by $\totalrew{j}{w}{0}{k}$ the number $\sum_{n=0}^{k-1} \vec{\alpha}_n[j]$ (i.e., the $j$-th component of $\sum_{n=0}^{k-1} \vec{\alpha}_n$).

Let $\Upsilon_i$ be a function which for a given run 
$w \equiv p_0\vec{v}_0\,p_1\vec{v}_1\, p_2\vec{v}_2\ldots$ of
$\run_{\M_\A}(p\vec{v},\neg\Zminusi{i})$ returns a run
$\Upsilon_i(w) \equiv p_0(\vec{v}_0[i]) \, \vec{\alpha}_0 \, 
p_1(\vec{v}_1[i]) \, \vec{\alpha}_1 \, p_2(\vec{v}_2[i]) \, \vec{\alpha}_2 \, \ldots$
of $\run_{\M_{\B_i}}(p(\vec{v}[i]))$ where the label $\vec{\alpha}_j$ 
corresponds to the update in the abstracted counters performed in
the transition $p_j\vec{v}_j \tran{} p_{j+1}\vec{v}_{j+1}$,
i.e., $\vec{v}_{j+1} - \vec{v}_j = 
\langle \vec{\alpha}_j,\vec{v}_{j+1}[i] - \vec{v}_j[i] \rangle_i$. 
The next lemma is immediate.
\begin{lemma}\label{prop:one-counter-runs}
For all $w \in \run_{\M_\A}(p\vec{v},\neg\Zminusi{i})$ and 
$k \in \Nset$ we have that
\begin{itemize}
\item $\state(w(k))=\state(\Upsilon_i(w)(k))$,
\item $\cval(w(k)) = \langle \totalrew{}{\Upsilon_i(w)}{0}{k}, 
  \cval_1(\Upsilon_i(w)(k))\rangle_i$.
\end{itemize}
Further, for every measurable set 
$R\subseteq \run_{\M_\A}(p\vec{v},\neg\Zminusi{i})$ we have that $\Upsilon_i(R)$ is measurable and
\begin{equation}
\calP(R) \ = \
  \calP(\Upsilon_i(R)) 
\label{eq-project}
\end{equation}\end{lemma}

\noindent
Now we examine the runs of $\run(p\vec{v},C)$ where $C$ is a BSCC of
$\F_\A$ such that some counter is not diverging in~$C$. A proof of
the next lemma can be found in Appendix~\ref{app-sec1}.

\begin{lemma}
\label{lem:not-diverging}
  Let $C$ be a BSCC of $\F_{\A}$.
  If some counter is not diverging in $C$, then $\calP(\run(p\vec{v},C)) = 0$.
\end{lemma}

It remains to consider the case when $C$ is a BSCC of $\F_{\A}$
where all counters are diverging. Here we use the results of
\cite{BKK:pOC-time-LTL-martingale} which allow to derive
a bound on divergence probability in one-dimensional pMC.
These results are based on designing and analyzing a suitable
martingale for one-dimensional pMC. 

\begin{lemma}
\label{lem-divergence}
  Let $\B$ be a $1$-dimensional pMC, let $C$ be a BSCC of $\F_\B$ 
  such that the trend $t$ of the only counter in $C$ is positive and let $\delta=2|C|/x_{\min}^{|C|}$ where $x_{\min}$ is the smallest non-zero transition probability in $\M_\B$.
  Then for all $q \in C$ and $k > 2\delta/t$ we have that  
  $\calP(q(k),\neg\Z) \geq 1-\left(a^k/(1+a)\right)$, where $\Z = \{1\}$ and $a=\exp\left(-t^2\, /\, 8(\delta+t+1)^2\right)$.
\end{lemma}
\begin{proof}
Denote by $[q(k){\downarrow},\ell]$ the probability that a run initiated in $q(k)$ visits a configuration with zero counter value for the first time in exactly $\ell$ steps. By Proposition~7 of \cite{BKK:pOC-time-LTL-martingale-arxiv} we obtain for all $\ell\geq h = 2\delta/t$~\footnote{The precise bound on $h$ is given in Proposition~7~\cite{BKK:pOC-time-LTL-martingale-arxiv}.},
\[
[q(k){\downarrow},\ell]\quad \leq\quad a^\ell
\]
where $a=\exp\left(-t^2\, /\, 8(\delta+t+1)^2\right)$ for $\delta\leq 2|C|/x_{\min}^{|C|}$~\footnote{The bound on $\delta$ is given in Proposition 6~\cite{BKK:pOC-time-LTL-martingale-arxiv}.}.

Thus
\[
\calP(q(k),\neg\Z)\geq 1-\sum_{\ell=k}^{\infty} [q(k){\downarrow},\ell]=1-\frac{a^k}{1+a}
\]
\end{proof}



\begin{definition}
  Let $C$ be a BSCC of $\F_{\A}$ where all counters are diverging,
  and let $q \in C$. We say that a configuration $q\vec{u}$
  is \emph{above} a given $n \in \Nset$ if $\vec{u}[i] \geq n$ for every
  $i$ such that $\vec{t}[i] > 0$, and 
  $\vec{u}[i] \geq \bottomfin_i(q)$ for every
  $i$ such that $\vec{t}[i] = 0$. 
\end{definition}


\begin{lemma}
\label{lem-diverging}
  Let $C$ be a BSCC of $\F_{\A}$ where all counters are diverging.
  Then $\calP(\run(p\vec{v},C)) > 0$ iff there is a $\Zall$-safe finite 
  path of the form $p\vec{v} \tran{}^* q\vec{u} \tran{}^* q\vec{z}$
  where $q \in C$, $q\vec{u}$ is above $1$, $\vec{z} - \vec{u} \geq \vec{0}$,
  and $(\vec{z} - \vec{u})[i] > 0$ for every $i$ such that $\vec{t}[i] > 0$. 


\end{lemma}
\begin{proof}
  We start with ``$\Rightarrow$''. 
  Let $\vec{t}$ be the trend of $C$. We show that for almost
  all $w \in \run(p\vec{v},C)$ and all $i \in \{1,\ldots,d\}$, one of
  the following conditions holds:
  \begin{enumerate}
  \item[(A)] $\vec{t}[i]>0$ and $\liminf_{k\rightarrow \infty} \cval_i(w(k))=\infty$,
  \item[(B)] $\vec{t}[i]=0$ and $\cval_i(w(k))\geq \bottomfin_i(\state(w(k)))$ 
    for all $k$'s large enough.
  \end{enumerate}
  First, recall that $C$ is also a BSCC of $\F_{\B_i}$, and realize that
  the trend of the (only) counter in the BSCC $C$ of $\F_{\B_i}$ is~$\vec{t}[i]$.

  Concerning~(A), it follows, e.g., from the results of
  \cite{BKK:pOC-time-LTL-martingale}, that almost all runs
  $w' \in \run_{\M_{\B_i}}(p(1))$ that stay
  in $C$ and do not visit a configuration with zero counter 
  satisfy $\liminf_{k\rightarrow \infty} \cval_1(w'(k))=\infty$.
  In particular, this means that almost all 
  $w' \in \Upsilon_i(\run(p\vec{v},C))$
  satisfy this property.
  Hence, by Lemma~\ref{prop:one-counter-runs}, for almost all 
  $w \in \run(p\vec{v},C)$ we have that
  \mbox{$\liminf_{k\rightarrow \infty} \cval_i(w(k))=\infty$}.

  Concerning (B), note that almost all runs $w \in \run(p\vec{v},C)$
  satisfying $\cval_i(w'(k)) < \bottomfin_i(\state(w(k)))$
  for infinitely many $k$'s eventually visit zero in some counter 
  (there is a path of length at most $|C|$ from each such
  $w(k)$ to a configuration with zero in counter $i$, or in one of the
  other counters).  

  The above claim immediately implies that for every $k \in \Nset$, 
  almost every run of $\run(p\vec{v},C)$ visits a configuration 
  $q\vec{u}$ above~$k$. Hence, there must be a $\Zall$-safe 
  path of the form $p\vec{v} \tran{}^* q\vec{u} \tran{}^* q\vec{z}$
  with the required properties.


  ``$\Leftarrow$'': If there is a $\Zall$-safe 
  path of the form $p\vec{v} \tran{}^* q\vec{u} \tran{}^* q\vec{z}$
  where $q \in C$, $q\vec{u}$ is above $1$, $\vec{z} - \vec{u} \geq \vec{0}$,
  and $(\vec{z} - \vec{u})[i] > 0$ for every $i$ such that $\vec{t}[i] > 0$, then
  $p\vec{v}$ can a reach a configuration $q\vec{y}$ above~$k$ for an 
  arbitrarily large $k \in \Nset$ via a $\Zall$-safe path.

  By Lemma~\ref{lem-divergence}, there exists
  $k \in \Nset$ such that for every $i \in \{1,\ldots,d\}$ where 
  $\vec{t}[i] > 0$ and every $n \geq k$,
  the probability of all $w \in \run_{\M_{\B_i}}(q(n))$ that 
  visit a configuration with zero counter is strictly smaller than 
  \mbox{$1/d$}. Let $q\vec{y}$ be a configuration above~$k$ reachable
  from $p\vec{v}$ via a $\Zall$-safe path (the existence of such 
  a $q\vec{y}$ follows from the existence of
  $p\vec{v} \tran{}^* q\vec{u} \tran{}^* q\vec{z}$). It suffices
  to show that $\calP(\run(q\vec{y},\Zall)) < 1$. For every 
  $i \in  \{1,\ldots,d\}$ where $\vec{t}[i] > 0$,
  let $R_i$ be the set of all $w \in \run(q\vec{y},\Zall)$ such that
  $\cval_i(w(k)) = 0$ for some $k \in \Nset$ and
  all counters stay positive in all $w(k')$ where $k' < k$. 
  Clearly, $\run(q\vec{y},\Zall) = \bigcup_{i} R_i$, and thus we obtain
  \begin{equation*}
    \calP(\run(q\vec{y},\Zall)) \leq \sum_{i} \calP(R_i) =
    \sum_{i} \calP(\Upsilon_i(R_i)) <  d \cdot \frac{1}{d} = 1
  \end{equation*}
\end{proof}

The following lemma shows that it is possible to decide, whether for a given $n\in\Nset$ a configuration above $n$ can be reached via a $\Zall$-safe path. Its proof uses the results of~\cite{BG:VASS-coverability} on the coverability problem in (non-stochastic) VASS.

\begin{lemma}
\label{lem:cover-short-path}
 Let $C$ be a BSCC of $\F_{\A}$ where all counters are diverging and let $q\in C$. There is a $\Zall$-safe finite path of the form $p\vec{v}\tran{}^* q\vec{u}$ with $q\vec{u}$ is above some $n\in \Nset$ iff there is a $\Zall$-safe finite path of length at most $(|Q|+|\gamma|)\cdot(3+n)^{(3d)!+1}$ of the form $p\vec{v}\tran{}^* q\vec{u'}$ with $q\vec{u'}$ is above $n$. Moreover, the existence of such a path can be decided in time $(|\A|\cdot n)^{c'\cdot 2^{d\log(d)}}$ where $c'$ is a fixed constant independent of $d$ and $\A$.
\end{lemma}
\begin{proof}
  We employ a decision procedure of \cite{BG:VASS-coverability} for
  VASS coverability. Since we need to reach $q\vec{u'}$ above $n$ via
  a $\Zall$-safe finite path, we transform $\A$ into a
  (non-probabilistic) VASS $\A'$ whose control states and rules are
  determined as follows: for every rule $(p,\vec{\alpha},\emptyset,q)$
  of $\A$, we add to $\A'$ the control states $p,q$ together with two
  auxiliary fresh control states $q',q''$, and we also add the rules
  $(p,\vec{-1},q')$, $(q',\vec{1},q'')$,
  $(q'',\vec{\alpha},q)$. Hence, $\A'$ behaves like $\A$, but when
  some counter becomes zero, then $\A'$ is stuck (i.e., no transition
  is enabled except for the self-loop). Now it is easy to check that
  $p\vec{v}$ can reach a configuration $q\vec{u}$ above $n$ via a
  $\Zall$-safe finite path in $\A$ iff $p\vec{v}$ can reach a
  configuration $q\vec{u}$ above $n$ via \emph{some} finite path in
  $\A'$, which is exactly the coverability problem for VASS. 
  Theorem~1 in~\cite{BG:VASS-coverability} shows that such a
  configuration can be reached iff there is configuration $q\vec{u'}$
  above $n$ reachable via some finite path of length at most $m =
  (|Q|+|\gamma|)\cdot(3+n)^{(3d)!+1}$. (The term $(|Q|+|\gamma|)$
  represents the number of control states of $\A'$.) This path
  induces, in a natural way, a $\Zall$-safe path from $p\vec{v}$ to
  $q\vec{u'}$ in $\A$ of length at most $m/2$. Moreover, Theorem~2
  in~\cite{BG:VASS-coverability} shows that the existence of such a
  path in $\A'$ can be decided in time
  $(|Q|+|\gamma|)\cdot(3+n)^{2^{\mathcal{O}(d\log(d))}}$, which proves
  the lemma. \end{proof}



\begin{theorem}
\label{thm:qual-all-algorithm}
  The qualitative $\Zall$-reachability problem for \mbox{$d$-dimensional}
  pMC is decidable in time $|\A|^{\kappa \cdot 2^{d\log(d)}}$, where $\kappa$ is 
  a fixed constant independent of $d$ and $\A$. 
\end{theorem}
\begin{proof}
  Note that the Markov chain $\F_\A$ is computable in time polynomial
  in $|\A|$ and $d$, and we can efficiently identify all diverging
  BSCCs of $\F_\A$. For each diverging BSCC $C$, we need to check the
  condition of Lemma~\ref{lem-diverging}. By applying Lemma~2.3.{}
  of \cite{RY:VASS-JCSS}, we obtain that if there exist \emph{some} $q\vec{u}$ 
  above~$1$ and a $\Zall$-safe finite path of the form 
  $q\vec{u} \tran{}^* q\vec{z}$ such that $\vec{z} - \vec{u} \geq \vec{0}$
  and $(\vec{z} - \vec{u})[i] > 0$ for every $i$ where $\vec{t}[i] > 0$,
  then such a path exists for \emph{every} $q\vec{u}$ above
  $|\A|^{c\cdot d}$ and its length is bounded by $|\A|^{c\cdot d}$. Here
  $c$ is a fixed constant independent of $|\A|$ and~$d$ (let us
  note that Lemma~2.3.{} of \cite{RY:VASS-JCSS} is formulated for 
  vector addition
  systems without states and a non-strict increase in every counter, 
  but the corresponding result for VASS is easy
  to derive; see also Lemma~15 in \cite{BJK:VASS-games-arxiv}). 
  Hence, the existence
  of such a path for a given $q \in C$ can be decided in 
  $\calO(|\A|^{c\cdot d})$ time. It remains to check whether $p\vec{v}$
  can reach a configuration $q\vec{u}$ above  $|\A|^{c\cdot d}$ via
  a $\Zall$-safe finite path. By Lemma~\ref{lem:cover-short-path} this can be done in time $(|\A|\cdot |\A|^{c\cdot d})^{c'\cdot 2^{d\log(d)}}$ for another constant $c'$. This gives us the desired complexity bound.
\end{proof}




















\smallskip

Note that for every fixed dimension $d$, the qualitative 
\mbox{$\Zall$-reachability} problem is solvable in polynomial time.









Now we show that $\calP(\run(p\vec{v},\Zall))$ can be effectively 
approximated up to an arbitrarily small absolute/relative error 
$\varepsilon > 0$. A full proof of Theorem~\ref{thm:approx-general}
can be found in Appendix~\ref{app-approx}.

\begin{theorem}
 \label{thm:approx-general}
 For a given $d$-dimensional pMC $\A$ and its initial configuration
 $p\vec{v}$, the probability $\calP(\run(p\vec{v},\Zall))$ can be
 approximated up to a given absolute error $\eps>0$ in time
 $(\exp(|\A|)\cdot \log(1/\eps))^{\calO(d\cdot d!)}$.
\end{theorem}
\begin{proof}[Proof sketch]
  First we check whether   $\calP(\run(p\vec{v},\Zall))  = 1$ 
  (using the algorithm of Theorem~\ref{thm:qual-all-algorithm}) and 
  return $1$ if it is the case. Otherwise, we first show how
  to approximate $\calP(\run(p\vec{v},\Zall))$ under the 
  assumption that $p$ is in some diverging BSCC of $\F_\A$, and 
  then we show how to drop this assumption.

  So, let $C$ be a diverging BSCC of $\F_\A$ such that 
  $\calP(\run(p\vec{v},C))<1$, and let us assume that $p \in C$. 
  We show how to compute $\nu >0$ such that 
  $|\calP(\run(p\vec{v},\Zall))-\nu|\leq d\cdot\eps$ in time 
  $(\exp(|\A|)\cdot \log(1/\eps))^{\calO(d!)}$. We proceed by induction
  on~$d$. The key idea of the inductive step is to find a sufficiently
  large constant~$K$ such that if some counter reaches~$K$, it can
  be safely ``forgotten'', i.e., replaced by $\infty$, without influencing
  the probability of reaching zero in some counter by more than 
  $\varepsilon$. Hence, whenever we visit a configuration $q\vec{u}$
  where some counter value in $\vec{u}$ reaches $K$, we can
  apply induction hypothesis and approximate the probability or reaching 
  zero in some counter from $q\vec{u}$ by ``forgetting'' the large
  counter a thus reducing the dimension. Obviously, there are only 
  finitely many configurations where all counters are below~$K$, and
  here we employ the standard methods for finite-state Markov chains. 
  The number $K$ is computed by using
  the bounds of Lemma~\ref{lem-divergence}. 

  Let us note that the base (when $d=1$) is handled by relying only
  on Lemma~\ref{lem-divergence}. Alternatively, we could employ
  the results of \cite{ESY:polynomial-time-termination}. This would
  improve the complexity for $d=1$, but not for higher 
  dimensions. 

  Finally, we show how to approximate $\calP(\run(p\vec{v},\Zall))$
  when the control state $p$ does not belong to a BSCC of~$\F_\A$.
  Here we use the bound of Lemma~\ref{lem:F_A-BSCC}.
\end{proof}

Note that if $\calP(\run(p\vec{v},\Zall)) > 0$, then this probability is
at least $\pmin^{m \cdot |Q|}$ where $\pmin$ is the least positive
transition probability in $\M_\A$ and $m$ is the maximal component 
of $\vec{v}$. Hence, Theorem~\ref{thm:approx-general} can also be
used to approximate $\calP(\run(p\vec{v},\Zall))$ up to a given
\emph{relative} error $\varepsilon > 0$.

