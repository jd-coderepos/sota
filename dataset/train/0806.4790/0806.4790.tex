\def\draft{0}  \documentclass[proceedings]{stacs}
\stacsheading{2010}{119-130}{Nancy, France}
 \firstpageno{119}

\usepackage{latexsym}
\usepackage{amssymb}
\usepackage{times}
\usepackage{amsmath,amsfonts,amsthm}
\usepackage{enumerate}
\usepackage{clrscode}
\usepackage{url}



\theoremstyle{plain}\newtheorem{satz}[thm]{Satz}
\theoremstyle{definition}\newtheorem{crucial}[thm]{Crucial Definition}
\def\eg{{\em e.g.}}
\def\cf{{\em cf.}}

\ifnum\draft=1
\newcommand{\Lnote}[1]{{[\bf Liu's Note: #1]}}
\newcommand{\Cnote}[1]{{[\bf Chung's Note: #1]}}
\newcommand{\Bnote}[1]{{[\bf Braverman's Note: #1]}}
\else
\newcommand{\Lnote}[1]{{}}
\newcommand{\Cnote}[1]{{}}
\newcommand{\Bnote}[1]{{}}
\fi

\newcommand{\vecv}{{\vec{v}} }
\newcommand{\barS}{{\bar{S}} }
\newcommand{\bfp}{{\mathbf p} }
\newcommand{\bfq}{{\mathbf q} }
\newcommand{\bfa}{{\mathbf a} }
\newcommand{\bfb}{{\mathbf b} }
\newcommand{\bfc}{{\mathbf c} }
\newcommand{\bfd}{{\mathbf d} }
\newcommand{\Var}{{\mathrm{Var}} }
\newcommand{\E}{{\mathrm{E}} }


\begin{document}

\title[AMS Without -Wise Independence on Product Domains]{AMS Without -Wise Independence on Product Domains\footnote{This paper is a merge from the  work of \cite{BO01, BO02, CLM}}}

\author[lab1]{V. Braverman}{Vladimir Braverman}
\address[lab1]{University of California Los Angeles. Supported in part by NSF grants 0716835, 0716389, 0830803, 0916574
and Lockheed Martin Corporation.}  \email{vova@cs.ucla.edu}  \urladdr{\url{http://www.cs.ucla.edu/~vova}}  

\author[lab2]{K. Chung}{Kai-Min Chung}
\address[lab2]{Harvard School of Engineering and Applied Sciences. Supported by US-Israel BSF grant 2006060 and NSF grant CNS-0831289.}	\email{kmchung@fas.harvard.edu}  \urladdr{\url{http://people.seas.harvard.edu/~kmchung/}}  

\author[lab3]{Z. Liu}{Zhenming Liu}
\address[lab3]{Harvard School of Engineering and Applied Sciences. Supported in part by NSF grant CNS-0721491. The work was finished during an internship in Microsoft Research Asia.}
\email{zliu@fas.harvard.edu}  \urladdr{\url{http://people.seas.harvard.edu/~zliu/}}  

\author[lab4]{M. Mitzenmacher}{Michael Mitzenmacher}
\address[lab4]{Harvard School of Engineering and Applied Sciences. Supported in part by NSF grant CNS-0721491 and research grants from Yahoo!, Google, and Cisco.}
\email{michaelm@eecs.harvard.edu}  \urladdr{\url{http://www.eecs.harvard.edu/~michaelm/}}  

\author[lab5]{R. Ostrovsky}{Rafail Ostrovsky}
\address[lab5]{University of California Los Angeles. Supported in part by IBM Faculty Award, Lockheed-Martin Corporation
Research Award, Xerox Innovation Group Award, the Okawa Foundation Award,
Intel, Teradata, NSF grants 0716835, 0716389, 0830803, 0916574 and U.C.
MICRO grant.}  \email{rafail@cs.ucla.edu}  \urladdr{\url{http://www.cs.ucla.edu/~rafail}}  



\keywords{Data Streams, Randomized Algorithms, Streaming Algorithms, Independence, Sketches}
\subjclass{F.2.1, G.3 }




\begin{abstract}
In their seminal work, Alon, Matias, and Szegedy introduced
several sketching techniques, including showing that -wise
independence is sufficient to obtain good approximations of the second
frequency moment.  In this work, we show that their sketching
technique can be extended to product domains  by using the
product of -wise independent functions on .
Our work extends that of Indyk and McGregor, who showed the result
for .  Their primary motivation was the problem of identifying correlations in data streams.
In their model, a stream of pairs  arrive,
giving a joint distribution , and they find approximation
algorithms for how close the joint distribution is to the product of
the marginal distributions under various metrics, which naturally
corresponds to how close  and  are to being independent.
By using our technique, we obtain a new result for the problem of
approximating the  distance between the
joint distribution and the product of the marginal distributions for -ary
vectors, instead of just pairs, in a single pass.
Our analysis gives a randomized algorithm that is a  approximation (with probability ) that
requires space logarithmic in  and  and proportional to .
\vspace{-0.6cm}
\end{abstract}

\maketitle

\section{Introduction}


In their seminal work, Alon, Matias and Szegedy \cite{ams} presented
celebrated sketching techniques and showed that -wise independence
is sufficient to obtain good approximations of the second frequency
moment.  Indyk and McGregor \cite{IM08} make use of this technique in
their work introduce the problem of measuring independence in the
streaming model.  There they give efficient algorithms for
approximating pairwise independence for the  and  norms.
In their model, a stream of pairs  arrive, giving a
joint distribution , and the notion of approximating pairwise
independence corresponds to approximating the distance between the
joint distribution and the product of the marginal distributions for
the pairs.  Indyk and McGregor state, as an explicit open question in
their paper, the problem of whether one can estimate -wise
independence on -tuples for any .  In particular, Indyk and McGregor show that, for the  norm, they
can make use of the product of -wise independent functions on 
in the sketching method of Alon, Matias, and Szegedy.  We extend their
approach to show that on the product domain , the sketching
method of Alon, Matias, and Szegedy works when using the product of
 copies of -wise independent functions on .  The cost is that the
memory requirements of our approach grow exponentially with ,
proportionally to .



Measuring
independence and -wise independence is a fundamental problem with
many applications (see e.g., Lehmann \cite{stat}). Recently, this problem was also addressed in other models by, among
others, Alon, Andoni, Kaufman, Matulef, Rubinfeld and Xie
\cite{k-wise_independenc}; Batu, Fortnow, Fischer, Kumar, Rubinfeld
and White \cite{batu_independence}; Goldreich and Ron \cite{ind1};
Batu, Kumar and Rubinfeld \cite{ind2}; Alon, Goldreich and Mansour
\cite{ind3}; and Rubinfeld and Servedio \cite{ind4}.
Traditional non-parametric methods of testing independence over empirical data
usually require space complexity that is polynomial to either the support
size or input size. The scale of contemporary data sets often
prohibits such space complexity.  It is therefore natural to ask
whether we will be able to design algorithms to test for independence
in streaming model. Interestingly, this specific problem appears not
to have been introduced until the work of Indyk and McGregor.  While
arguably results for the  norm would be stronger than for
the  norm in this setting, the problem for  norms
is interesting in its own right. The problem for the  norm has been
recently resolved by Braverman and Ostrovsky in \cite{BO03}. They gave an -approximation algorithm
that makes a single pass over a data stream and uses polylogarithmic memory.

\subsection{Our Results}
In this paper we generalize the ``sketching of sketches''
result of Indyk and McGregor.
Our specific theoretical contributions can be summarized as follows:

\ \

\noindent
\emph{\textbf{Main Theorem. }}


\noindent
Let  be a vector with entries  for .
Let  be independent copies of 4-wise independent hash functions;  that is,  are -wise independent hash functions for each , and  are mutually independent.  Define , and the sketch .

We prove that the sketch  can be used to give an efficient approximation for ; our result is stated formally in Theorem \ref{tm:main1}. Note that  is not -wise independent.

\ \ \



As a corollary, the main application of our main theorem is to extend the result of Indyk and McGregor \cite{IM08} to detect the dependency of  random variables in streaming model.

\begin{corollary}
For every  and , there exists a randomized algorithm that computes,
given a sequence  of -tuples, in one pass and using
 memory bits,
a number  so that the probability  deviates from the 
distance between product and joint distribution by more than a factor of
 is at most .
\end{corollary}

\subsection{Techniques and a Historical Remark}

This paper is merge from \cite{BO01, BO02, CLM}, where the same result was obtained with different
proofs. The proof of \cite{CLM} generalizes the geometric approach of
Indyk and McGregor \cite{IM08} with new geometric observations.  The
proofs of \cite{BO01,BO02} are more combinatorial in nature. These papers
offer new insights, but due to the space limitation, we focus
on the proof from \cite{BO02} in this paper. Original papers are
available on line and are recommended to the interested reader.

\section{The Model}
We provide the general underlying model.  Here we mostly follow the notation of \cite{BO01,IM08}.

Let  be a stream of size  with elements , where
.
(When we have a sequence of elements that are themselves vectors, we
denote the sequence number by a subscript and the vector entry by a
superscript when both are needed.)
The stream  defines an
\emph{empirical} distribution over  as follows: the frequency
 of an element  is defined as the number
of times it appears in , and the empirical distribution is


Since  is a vector of size , we
may also view the streaming data as defining a joint distribution over
the random variables  corresponding to the values in
each dimension. (In the case of , we write the random variables
as  and  rather than  and .) There is a natural way of
defining marginal distribution for the random variable : for
, let  be the number of times
 appears in the th coordinate of an element of ,
or

The empirical marginal distribution   for the th coordinate is defined as


Next let  be the vector in  with  for all . Our goal is to approximate the value

This represent the  norm between the tensor of the marginal distributions and the joint distribution,
which we would expect to be close to zero in the case where the  were truly independent.

Finally, our algorithms will assume the availability of 4-wise independent hash functions.
For more on 4-wise independence, including efficient implementations, see \cite{BCH, TZ04}.
For the purposes of this paper, the following simple definition will suffice.

\begin{definition}\emph{(4-wise independence)}
A family of hash functions  with domain  and range  is \emph{4-wise independent} if for any distinct values
 and any , the following equality holds,

\end{definition}
\begin{remark}In \cite{IM08}, the family of 4-wise independent hash functions  is called 4-wise independent random vectors. For consistencies within our paper, we will always view the object  as a hash function family.
\end{remark}

\section{The Algorithm and its Analysis for } \label{sec:kis2}
We begin by reviewing the approximation algorithm and associated proof
for the  norm given in \cite{IM08}. Reviewing this result
will allow us to provide the necessary notation and frame the setting
for our extension to general .  Moreover, in our proof, we find
that a constant in Lemma 3.1 from \cite{IM08} that we subsequently
generalize appears incorrect.  (Because of this, our proof is slightly
different and more detailed than the original.)  Although the error is
minor in the context of their paper (it only affects the constant
factor in the order notation), it becomes more important when
considering the proper generalization to larger , and hence it
is useful to correct here.


In the case , we assume that the sequence  arrives an item by an item.
Each  (for ) is an element in . The random variables  and  over  can be expressed
as follows:

We simplify the notation and use
, ,
.
and
.

Indyk and McGregor's algorithm proceeds in a similar fashion
to the streaming algorithm presented in \cite{ams}. Specifically let  and .
The algorithm computes  random variables  and
outputs their median. The output is the algorithm's estimate on the norm of  defined in Equation~\ref{eqn:alpha}.
Each  is the average of  random variables : , where  are independent, identically
distributed random variables. Each of the variables  can be computed from the algorithmic routine shown in
Figure~\ref{fig:algx}.


\begin{figure}[htp]
\begin{codebox}
\Procname{}
\li Independently generate 4-wise independent random functions  from  to .
\li \For  \To 
\li \Do Let the th item 
\li , , .
\End
\li Return .
\end{codebox}
\caption{ The procedure for generating random variable  for .}
\label{fig:algx}
\end{figure}

\noindent
By the end of the process , we have 
,
, and . Also, when a vector is in , its indices can be represented by . In what follows, we will use a bold letter to represent the index of a high dimensional vector, e.g., .
The following Lemma shows that the expectation of  is  and the variance of  is at most  because .

\begin{lemma}\label{lem:varx}(\cite{IM08}) Let  be two independent instances of 4-wise independent hash functions from  to . Let  and . Let us define . Then
 and , which implies .
\end{lemma}

\begin{proof}
We have .  For all
, we know . On the other
hand, .  The probability
that  is . The last equality holds is because
 is equivalent to saying either all
these variables are 1, or exactly two of these variables are -1, or all
these variables are -1. Therefore, . Consequently, .

Now we bound the variance. Recall that , we bound


Also . The quantity
 if and only if the following relation holds,

Denote the set of 4-tuples  that satisfy the above relation by .
We may also view each 4-tuple as an ordered set that consists of 4 points in .
Consider the unique smallest axes-parallel rectangle in 
that contains a given 4-tuple in  (i.e. contains the
four ordered points).  Note this could either be a (degenerate) line segment
or a (non-degenerate) rectangle, as we discuss below.
Let  be the function
that maps an element  to the smallest rectangle
 defined by . Since a
rectangle can be uniquely determined by its diagonals, we may write
,
where , 
and the corresponding rectangle is understood to be the one with
diagonal . Also, the
inverse function 
represents the pre-images of 
in .  is degenerate if either  or , in which case the rectangle (and
its diagonals) correspond to the segment itself, or  and , and the rectangle is just a
single point.


\begin{example}
Let , ,
, and . The tuple is in  and its corresponding bounding rectangle is
a non-degenerate rectangle. The function .
\end{example}

\begin{example} Let 
and . The tuple is also in  and
minimal bounding rectangle formed by these points is an interval . The function
.
\end{example}

To start we consider the non-degenerate cases.
Fix any  with  and .
There are in total  tuples  in  with .  Twenty-four of these tuples
correspond to the setting where none of  are equal, as there
are twenty-four permutations of the assignment of the labels  to the four points. (This corresponds to the first example).  In this
case the four points form a rectangle, and we have
.
Intuitively, in these cases, we assign the ``weight'' of the tuple to the diagonals.

The remaining twelve tuples in  correspond to intervals. (This corresponds to the second example.)
In this case two of  correspond to one endpoint of the interval, and the other two labels
correspond to the other endpoint.
Hence we have either  or , and there are six tuples for each case.

Therefore for any 
and  we have:


The analysis is similar for the degenerate cases, where the constant 18 in the bound above is now quite loose.
When exactly one of
 or  holds, the size of
 is , and the resulting intervals correspond to vertical or horizontal
lines.  When both  and , then
. In sum, we have
Following the same analysis as for the non-degenerate cases, we find







Finally, we have  and
.
\end{proof}

We emphasize the geometric interpretation of the above proof as follows. The goal is to bound the variance by a constant times , where the index set is the set of all possible lines in plane  (each line appears twice). We first show that ,
where the 4-tuple index set corresponds to a set of rectangles in a natural way. The main idea of \cite{IM08} is to use inequalities of the form  to assign the ``weight'' of each -tuple to the diagonals of the corresponding rectangle.  The above analysis shows that  copies of all lines are sufficient to accommodate all 4-tuples. While similar inequalities could also assign the weight of a -tuple to the vertical or horizontal edges of the corresponding rectangle, using vertical or horizontal edges is problematic. The reason is that there are  -tuples but only  vertical or horizontal edges, so some lines would receive  weight, requiring  copies. This problem is already noted in \cite{BO01}.



Our bound here is , while in
\cite{IM08} the bound obtained is . There appears to have been an error in the derivation in \cite{IM08};
some intuition comes from the following example.
We note that  is at
least . (This counts the number of non-degenerate -tuples.) Now if we
set  for all , we have , which
suggests . Again, we emphasize this
discrepancy is of little importance to \cite{IM08};  the point there is that
the variance is bounded by a constant factor times the square of the expectation.
It is here, where we are generalizing to , that the exact
constant factor is of some importance.

Given the bounds on the expectation and variance for the ,
standard techniques yield a bound on the performance
of our algorithm.


\begin{theorem}\label{thm:case2} For every  and , there exists a randomized algorithm
that computes, given a sequence , in one pass and using 
memory bits, a number  so that the probability  deviates from  by more than  is at most .
\end{theorem}
\begin{proof}
Recall the algorithm described in the beginning of Section \ref{sec:kis2}: let  and .
We first computes  random variables  and
outputs their median , where each  is the average of  random variables :  and  are independent, identically
distributed random variables computed by Figure~\ref{fig:algx}. By Chebyshev's inequality, we know that for any fixed ,

Finally, by standard Chernoff bound arguments (see for example Chapter 4 of \cite{MU05}), the probability that more than 
of the variables  deviate by more than  from  is at most . In case this does not happen,
the median  supplies a good estimate to the required quantity  as needed.
\end{proof}


\section{The General Case }\label{sec:estimator}

Now let us move to the general case where . Recall that  is a vector in  that maintains certain statistics of a data stream, and we are interested in estimating its  norm . There is a natural generalization for Indyk and McGregor's method for  to construct an estimator for : let  be independent copies of 4-wise independent hash functions (namely,  are -wise independent hash functions for each , and  are mutually independent.).  Let . The estimator  is defined as .

Our goal is to show that  and  is reasonably small so that a streaming algorithm maintaining multiple independent instances of estimator  will be able to output an approximately correct estimation of  with high probability. Notice that when  represents the  distance between the joint distribution and the tensors of the marginal distributions, the estimator can be computed efficiently in a streaming model similarly to as in  Figure~\ref{fig:algx}. We stress that our result is applicable to a broader class of -norm estimation problems, as long as the vector  to be estimated has a corresponding efficiently computable estimator  in an appropriate streaming model. Formally, we shall prove the following main lemma in the next subsection.



\begin{lemma} \label{lem:main_lemma} Let  be a vector in , and  be independent copies of 4-wise independent hash functions. Define , and . We have  and .
\end{lemma}

We remark that the bound on the variance in the above lemma is tight. One can verify that when the vector  is a uniform vector (i.e., all entries of  are the same), the variance of  is . With the above lemma, the following main theorem mentioned in the introduction immediately follows by a standard argument presented in the proof of Theorem \ref{thm:case2} in the previous section.

\begin{theorem}\label{tm:main1}
Let  be a vector in  that maintains an arbitrary statistics in a data stream of size , in which every item is from . Let  be real numbers. If there exists an algorithm that maintains an instance of  using  memory bits, then
there exists an algorithm  such that:
\begin{enumerate}
\item With probability  the algorithm  outputs a value between  and
\item the space complexity of  is .
\end{enumerate}
\end{theorem}

As discussed above, an immediate corollary is the existence of a one-pass space efficient streaming algorithm to detect the dependency of  random variables in -norm:

\begin{corollary}For every  and , there exists a randomized algorithm that computes,
given a sequence  of -tuples, in one pass and using
 memory bits,
a number  so that the probability  deviates from the square of the 
distance between product and joint distribution by more than a factor of
 is at most .
\end{corollary}














\subsection{Analysis of the Sketch }\label{sec:analysis}

This section is devoted to prove Lemma \ref{lem:main_lemma}, where the main challenge is to bound the variance of .  The geometric approach of Indyk and McGregor \cite{IM08} presented in Section \ref{sec:kis2} for the case of  can be extended to analyze the general case. However, we remark that the generalization requires new ideas. In particular, instead of performing ``local analysis'' that maps each rectangle to its diagonals, a more complex ``global analysis'' is needed in higher dimensions to achieve the desired bounds. The alternative proof we present here utilizes similar ideas, but relies on a more combinatorial rather than geometric approach.

For the expectation of , we have

where the last equality follows by , and  for .

Now, let us start to prove . By definition, , so we need to understand the following random variable:

The random variable  is a sum of terms indexed by pairs  with . At a very high level, our analysis consists of two steps. In the first step, we group the terms in  properly and simplify the summation in each group. In the second step,  we expand the square of the sum in  according to the groups and apply Cauchy-Schwartz inequality three times to bound the variance.





We shall now gradually introduce the necessary notation for grouping the terms in  and simplifying the summation. We remind the reader that vectors over the reals (e.g., ) are denoted by , and vectors over  are denoted by  and referred as \emph{index vectors}.  We use  to denote a subset of , and let . We use  to denote the \emph{Hamming distance} of index vectors , i.e., the number of coordinates where  and  are different. 



\begin{definition}\label{def:projection} \emph{(Projection and inverse projection)} Let  be an index vector and  a subset. We define the \emph{projection of  to }, denoted by , to be the vector  restricted to the coordinates in .
Also, let  and  be index vectors. We define the \emph{inverse projection of  and  with respect to }, denoted by , as the index vector  such that  and .
\end{definition}



We next define \emph{pair groups} and use the definition to group the terms in .

\begin{definition}\label{def:pairs projection} \emph{(Pair Group)}
Let  be a subset of size . Let  be a pair of index vectors with  (i.e., all coordinates of  and  are distinct.). The \emph{pair group}  is the set of pairs  such that (i) on coordinate ,  and , and (ii) on coordinate ,  and  are the same, i.e., . Namely,

\end{definition}

To give some intuition for the above definitions, we note that for every , there is a unique pair  with , and so . On the other hand, for every pair  with , there is a unique non-emtpy  such that  and  are distinct on exactly coordinates in . Therefore,  belongs to exactly one pair group . It follows that we can partition the summation in  according to the pair groups:


We next observe that for any pair , since  and  agree on coordinates in , the value of the product  depends only on ,  and  . More precisely,

which depends only on ,  and   since   and . This motivates the definition of \emph{projected hashing}.

\begin{definition}\label{def:projectedhashing} \emph{(Projected hashing)}
Let  be a subset of , where .
Let . We define the \emph{projected hashing} .
\end{definition}
We can now translate the random variable  as follows:


Fix a pair group , we next consider the sum . Recall that for every , there is a unique pair  with . The sum can be viewed as the inner product of two vectors of dimension  with entries indexed by . To formalize this observation, we introduce the definition of \emph{hyper-projection} as follows.

\begin{definition}\label{def:hyperproj} \emph{(Hyper-projection)} Let , , and . The \emph{hyper-projection}  of  (with respect to  and ) is a vector  in  such that  for all .
\end{definition}

Using the above definition, we continue to rewrite the  as


Finally, we consider the product  again and introduce the following definition to further simplify  the .



\begin{definition}\label{def:similar pairs} \emph{(Similarity and dominance)}
Let  be a positive integer.
\begin{itemize}
\item Two pairs of index vectors  and  are \emph{similar} if for all , the two sets  and  are equal. We denote this as .

\item Let  and  be two index vectors. We say  \emph{is dominated by}  if  for all . We denote this as . Note that .
\end{itemize}
\end{definition}

Now, note that if , then  since the value of the product   depends on the values  only as a set. It is also not hard to see that   is an equivalence relation, and for every equivalent class , there is a unique  with . Therefore, we can further rewrite the  as


We are ready to bound the term  by expanding the square of the sum according to Equation (\ref{eq:final_err}). We first show in Lemma \ref{lem:vanish} below that all the cross terms in the following expansion vanish.




\begin{lemma} \label{lem:vanish} Let  and  be subsets of , and  and  index vectors. We have .
Furthermore, we have

\noindent
 iff .
\end{lemma}
\begin{proof} Recall that  are independent copies of -wise independent uniform random variables over . Namely, for every ,  are -wise independent, and  are mutually independent. Observe that for every , there are at most  terms out of  appearing in the product . It follows that all distinct terms appearing in  are mutually independent uniform random variable over . Therefore, the expectation is either 0, if there is some  that appears an odd number of times, or 1, if all  appear an even number of times. By inspection, the latter case happens if and only if  .
\end{proof}

By the above lemma, Equation (\ref{eq:expand_var}) is simplified to




We next apply the Cauchy-Schwartz inequality three times to bound the above formula. Consider a subset  and a pair . Note that there are precisely  pairs  such that . Thus, by the Cauchy-Schwartz inequality:
\bfa, \bfb)\sim (\bfc, \bfd)}} \langle\Upsilon_{S,\bfa}(\vecv), \Upsilon_{S,\bfb}(\vec v)\rangle\right)^2 &\leq& 2^{|S|} \sum_{\substack{(\bfa, \bfb) \in [n]^{|S|}\\ (\bfa, \bfb)\sim (\bfc, \bfd)}} (\langle\Upsilon_{S,\bfa}, \Upsilon_{S,\bfb}\rangle)^2 \\
 &\leq & 2^{|S|}\sum_{\substack{(\bfa,\bfb) \in [n]^{|S|}\\ (\bfa, \bfb)\sim (\bfc,\bfd)}} \langle\Upsilon_{S,\bfa}(\vecv), \Upsilon_{S,\bfa}(\vecv)\rangle\cdot\langle\Upsilon_{S,\bfb}, \Upsilon_{S, \bfb}(\vecv)\rangle.

& & \sum_{\bfc \prec \bfd \in [n]^{|S|}} \ \ \ \ \  \left(\sum_{\substack{(\bfa, \bfb) \in [n]^{|S|}\\ (\bfa, \bfb)\sim (\bfc, \bfd)}} \langle\Upsilon_{S,\bfa}(\vec v), \Upsilon_{S,\bfb}(\vec v)\rangle\right)^2 \\
& \leq & 2^{|S|}\sum_{\bfc \prec \bfd \in [n]^{|S|}} \ \ \ \sum_{\substack{(\bfa,\bfb) \in [n]^{|S|}\\ (\bfa, \bfb)\sim (\bfc,\bfd)}} \langle\Upsilon_{S,\bfa}(\vec v), \Upsilon_{S,\bfa}(\vec v)\rangle\cdot\langle\Upsilon_{S,\bfb}(\vec v), \Upsilon_{S, \bfb}(\vec v)\rangle \\
& = & 2^{|S|}\sum_{\substack{\bfc, \bfd \in [n]^{|S|}\\ \mathrm{Ham}(\bfc, \bfd) = |S|}} \langle\Upsilon_{S,\bfc}(\vec v), \Upsilon_{S,\bfc}(\vec v)\rangle\cdot\langle\Upsilon_{S, \bfd}(\vec v), \Upsilon_{S, \bfd}(\vec v)\rangle \\
& \leq & 2^{|S|}\sum_{\bfc,\bfd \in [n]^{|S|} } \langle\Upsilon_{S,\bfc}(\vec v), \Upsilon_{S,\bfc}(\vec v)\rangle\cdot\langle\Upsilon_{S, \bfd}(\vec v), \Upsilon_{S, \bfd}(\vec v)\rangle \\
& = & 2^{|S|}\left(\sum_{\bfc \in [n]^{|S|}} \langle\Upsilon_{S,\bfc}(\vec v), \Upsilon_{S,\bfc}(\vec v)\rangle\right)^2.
\Var[Y] \leq \sum_{S \subseteq [k], S \neq \emptyset} 2^{|S|} \cdot \mathrm E[Y]^2 = \mathrm E[Y]^2 \sum_{i=1}^k {k \choose i}2^i = (3^k-1)\mathrm E[Y]^2,
which finishes the proof of Lemma \ref{lem:main_lemma}.

\section{Conclusion}

There remain several open questions left in this space.  Lower bounds,
particularly bounds that depend non-trivially on the dimension ,
would be useful.  There may still be room for better algorithms for
testing -wise independence in this manner using the  norm.
A natural generalization would be to find a
particularly efficient algorithm for testing -out-of--wise
independence (other than handling each set of  variable
separately).  More generally, a question given in \cite{IM08}, to
identify random variables whose correlation exceeds some threshold
according to some measure, remains widely open.




\bibliographystyle{plain}
{\footnotesize
\bibliography{braverman}
}
\vspace{-1cm}

\end{document}
