\documentclass{LMCS}




\usepackage{enumerate}
\usepackage{hyperref}
\usepackage{amsmath}


\theoremstyle{plain}
\newtheorem{lemma}[thm]{Lemma}
\newtheorem{example}[thm]{Example}

\newcommand\ignore[1]{}
\newcommand\ifshortpaper[1]{#1}
\newcommand\iffullpaper[1]{}
\newcommand{\comment}[1]{}
\newcommand\subterms[1]{{{\tt Subterms}(}{#1}{)}}
\newcommand\Pos{{\tt Pos}}
\newcommand{\F}{{\Sigma}} \newcommand{\X}{{\mathcal X}}
\newcommand{\TFX}{{\mathcal T}(\F,\X)}
\def\Tau{{\mathcal T}}
\def\Vars{{\mathcal V}}
\newcommand{\tor}{\to_R}
\newcommand{\tors}{\to_R^*}
\newcommand{\tofrs}{\leftrightarrow_R^*}




\def\doi{6 (3:8) 2010}
\lmcsheading {\doi}
{1--20}
{}
{}
{Dec.~\phantom04, 2009}
{Aug.~25, 2010}
{}   

\begin{document}

\title[Termination of Rewriting]{Termination of Rewriting with Right-Flat Rules Modulo Permutative Theories\rsuper*}

\author[L.~Bargu\~n\'o]{Luis Bargu\~n\'o\rsuper a}	\address{{\lsuper{a,b,c}}Universitat Polit\`ecnica de Catalunya, Jordi Girona 1, Barcelona, Spain}
\email{luisbargu@gmail.com, ggodoy@lsi.upc.edu, eduard.hl@gmail.com}  \thanks{{\lsuper a}Supported by Spanish Ministry of Education and Science by the
FORMALISM project (TIN2007-66523).}

\author[G.~Godoy]{Guillem Godoy\rsuper b}	\address{\vskip-6 pt}
\thanks{{\lsuper b}Supported by Spanish Ministry of Education and Science by the
FORMALISM project (TIN2007-66523) and the
LOGICTOOLS-2 project (TIN2007-68093-C02-01).}

\author[E.~Huntingford]{Eduard Huntingford}	\address{\vskip-6 pt}


\author[A.~Tiwari]{Ashish Tiwari\rsuper c}	\address{{\lsuper c}SRI International, Menlo Park, CA 94205}	\email{tiwari@csl.sri.com}  \thanks{{\lsuper c}Supported in part by the National Science Foundation under grants
CNS-0720721 and CSR-0917398.}	





\keywords{term rewriting, termination, decidability, complexity}
\subjclass{F.4.2}
\titlecomment{{\lsuper*}A preliminary version~\cite{DBLP:conf/rta/GodoyHT07}
containing some of the results
appeared in the Proceedings of the 18th International Conference
on Rewriting Techniques and Applications, RTA 2007.}




\begin{abstract}
\noindent 
We present decidability results for termination of classes
of term rewriting systems modulo permutative theories.
Termination and innermost termination modulo permutative theories
are shown to be decidable for term rewrite systems (TRS) whose right-hand
side terms are restricted to be shallow (variables
occur at depth at most one) and linear (each variable occurs at most once).
Innermost termination modulo permutative theories is also shown to be decidable
for shallow TRS. 
We first show that a shallow TRS can be transformed
into a flat (only variables and constants occur at depth one) 
TRS while preserving termination and innermost
termination.
The decidability results are then proved by showing that 
(a) for right-flat right-linear (flat) TRS, 
non-termination (respectively, innermost non-termination) implies
non-termination starting from flat terms, and
(b) for right-flat TRS,
the existence of non-terminating derivations
starting from a given term is decidable.
On the negative side, we show PSPACE-hardness of termination and innermost
termination for shallow right-linear TRS,
and undecidability of termination
for flat TRS.\end{abstract}

\maketitle

\section{Introduction}

\noindent Termination is an important property of
computing systems and it has
generated significant renewed interest in 
recent years.  
There has been progress on both the theoretical
and practical aspects of proving
termination of many different computing 
paradigms - such as 
term rewrite systems (TRS), 
functional programs, and 
imperative programs.
Innermost termination refers to termination of rewriting
restricted to the innermost strategy, which forces
the complete evaluation of all the subterms before
a rule is applied at any position. It
corresponds to the ``call by value'' computation of programming languages.
A typical example of a TRS that is innermost terminating
but not terminating is the following~\cite{Toy87b}:

The non-terminating derivation

is not possible with innermost rewriting, since  has to
be normalized before a rule can be applied at the root position 
to reduce .

Often, a term rewrite system contains rules
that are trivially non-terminating (like commutativity:
) and one desires to ensure a weaker
notion than termination, namely 
termination of a term rewrite system  modulo a theory 
(for example, when ).
Althought  could be non-terminating, in some
cases the important question is to determine if there is
a derivation with  that has infinitely many rewrite
steps with rules in .

While termination is
undecidable for general TRS
and string rewrite systems~\cite{HuetLankford78},
several subclasses with decidable termination
problem have been identified.  
Termination is decidable for ground TRS~\cite{HuetLankford78};
in fact, in polynomial time~\cite{Plaisted93:RTA}.
Termination is decidable for right-ground
TRS~\cite{Dershowitz81:ICALP} and also for the more general class
that also has collapsing (right-variable) rules~\cite{GodoyTiwari04:IJCAR}. 
Later, it was shown that termination is decidable for
TRS that contain any combination of right-ground, collapsing, and
shallow right-linear rewrite rules~\cite{GodoyTiwari05:CADE}.
There are further known decidability results about
shallow left-linear and shallow right-linear TRS~\cite{Sakai06}.

This paper focuses on termination and innermost termination of TRS
for rewriting modulo permutative theories.
Here we extend the results of our conference
paper~\cite{DBLP:conf/rta/GodoyHT07} by generalizing from
plain rewriting to rewriting modulo permutative theories.
Moreover, 
we provide extended proofs of our earlier results, and
a new PSPACE-hardness result.

The main contributions of the paper are as follows:
\\ (1)
In Section~\ref{sec-right-flat}, we prove that
termination {\em starting from a given fixed term}
is decidable for right-shallow TRS and rewriting modulo permutative theories.
This result is used to obtain subsequent results. \\ (2)
In Section~\ref{sec-innermost},
we consider innermost rewriting modulo permutative
theories and show that termination is decidable for shallow TRS.
\\ (3)
In Section~\ref{sec-right-flat-linear}, we show that 
termination (and innermost termination as well) is decidable
for rewriting modulo permutative theories using TRS whose
right-hand side terms are both shallow and linear.
There is no restriction on the left-hand
side terms. Thus, right-ground TRS and shallow
right-linear TRS are both contained in our class.
\\ (4)
In Section~\ref{sec-hardness}, we prove that termination,
as well as innermost termination, is PSPACE-hard for
flat (and hence shallow) right-linear TRS.
\\ (5)
In Section~\ref{sec-undec},
we show undecidability of termination for
flat TRS and plain rewriting, and
undecidability of termination for right-shallow TRS
and innermost rewriting.


Uchiyama, Sakai and Sakabe~\cite{Sakai} have recently also 
generalized the results of our conference 
paper~\cite{DBLP:conf/rta/GodoyHT07} by replacing
syntactic restrictions on the rewrite rules by 
syntactic restrictions on the {\em{dependency pairs}}.
Specifically, termination and innermost termination were
shown to be decidable for TRS whose {dependency
pairs} are right-linear and right-shallow; and
innermost  termination was shown to be decidable for
TRS whose dependency pairs are shallow. 


\section{Preliminaries}\label{sec-preliminaries}


\noindent We use standard notation from the
term rewriting literature~\cite{Allthat}.
A signature  is a (finite) set of function
symbols with arity, which is partitioned as 
such that  if the arity of  is .
Symbols in ,
called {\em constants},
are denoted by , with possible subscripts.
The elements of a set  of 
variable symbols are denoted by  with possible subscripts. 
The set

of {\em terms} over  and ,
is the smallest set containing  and
such that  is in 
whenever ,
and . A
{\it position\/} is a sequence of positive integers.
The set of positions of a term , denoted ,
is defined recursively as follows. If  is a variable
then  is , where 
represents the empty sequence. If  is of the
form , then  is
.
If  is a
position and  is a term, then by  we denote the {\em subterm
of  at position \/}: we have  (where 
denotes the empty sequence) and  if
 (and is undefined if ).
By  we denote the length of a position .
We also write 
to denote the term obtained by replacing in  the subterm at
position  by the term .
More formally,  is , and
 is
.
For example, if  is
, then , and .
Note that , and that the equality 
implies .
The set of all subterms of a term  is
denoted by .
The symbol occurring at the root of a term 
is denoted as .
We write  (equivalently, )
and say  is below 
(equivalently,  is above ) if  is a proper
prefix of , that is,  for some non-empty .
In this case, by  we denote .
By  we denote that either  or  hold.
Positions  and  are {\em parallel}, denoted ,
if  and
 hold.
By  we denote the set of all variables occurring in a term .
The {\tt height} of a term  is  if  is a variable or a constant,
and  if .
The {\tt depth} of an occurrence at position  of a term  in
a term  is .
Sometimes we will denote a term  by
the simplified form  when the arity of  is , and  by
 when  is clear from the context or not important.

A {\it substitution\/}  is a mapping from variables to terms.
It can be homomorphically extended to a function from
terms to terms:
 denotes the result of
simultaneously replacing in  every  by .
For example, if  is
, then
 is .

A \emph{rewrite rule} over 
is a pair of terms  of , denoted
by , with left-hand side  and right-hand side .
We make the usual assumptions for the rules, i.e.\  is not a variable,
and all variables occurring in the term  also occur in the term .
A \emph{term rewrite system} (TRS)  over  is a finite set of
rewrite rules over . We often assume  as implicit
when talking about a TRS .
We say that  rewrites to  in one step at position  (by ),
denoted by , if
 and , for some 
and substitution .
We also denote such a rewrite step by

if we make explicit the used rule 
and substitution .
If , then the rewrite step 
is said to be applied at the root.
Otherwise, it is denoted by .

If  is a binary relation on a set , then
 is its symmetric closure,
 is its transitive
closure,  is its inverse, and
 is its reflexive-transitive closure.

A \emph{(rewrite) derivation} (from ) is a sequence of
rewrite steps (starting from ), that is, a sequence
. With  we
denote that  is -reachable from , or a concrete derivation
from  to , depending on the context.
A term  is context-reachable from  with  (with a non-empty
context) if
there exists a derivation of the form  where
 is a (proper) subterm of .
The length of a derivation , denoted
, is its number of rewrite steps.
We denote this derivation as , 
and  when this number is , , and
 or , respectively.
A TRS  is {\em terminating from }
if there are no {\em -derivations},
 with infinitely
many rewrite steps.
If  is terminating from every term, then  is 
said to be {\em terminating}.
A term  is -\emph{irreducible} (or, in {\em -normal form}) 
if there is no term 
 such that .
When there is a unique normal form reachable from a given term
 using , it is denoted by .
When  is singleton, say , then
 will also be written as .


A term  is called {\em ground} if  contains no variables.
It is called {\em shallow} if all variable positions in  are
at depth  or .
It is {\em flat} if its height is at most .
It is {\em linear} if every variable occurs at most once.



A rule  is called {\em ground} ({\em flat}, {\em shallow},
{\em linear})
if both  and  are. A rule  is called
{\em left-ground} ({\em left-flat}, {\em left-shallow}, {\em
left-linear}) if  is.
A rule  is called
{\em right-ground} ({\em right-flat}, {\em right-shallow},
{\em right-linear}) if  is.
A rule  is called {\em collapsing} if  is a variable.

A TRS  is called (left-,right-){\em ground} ({\em flat}, {\em shallow},
{\em linear}) if all its rules are. A TRS  is called {\em collapsing}
if it contains a collapsing rule.

A rewrite step

is an {\em innermost rewrite step} if
 is -irreducible, for all .
The concepts of reachability and termination
can be naturally defined for innermost
rewriting.

A set  of pairs of terms is a set of equations if
whenever a pair, again written as , belongs to , 
the pair  also belongs to .
Given a TRS  and a set of equations ,
a term  rewrites into a term  with  modulo  in one step,
denoted , if  holds.
Note that  is equivalent to existence
of a derivation of the form  with
at least one rewrite step with .
A {\em permutative rule} is a linear flat rewrite rule 
satisfying  and .
When  contains just permutative rules we say that 
is a {\em permutative theory}.
In the rest of the paper
we will always assume that  is a permutative theory
defined over the same signature as .

The notion of innermost rewriting is extended to rewriting
modulo in the following natural way. A term  is a normal
form with respect to  if no  rewrite step can
be applied on . A term  {\em{innermost rewrites}} to 
with  if there exist terms  and a position  satisfying
 and such that any proper subterm
of  is a normal form with respect to .

The notion of termination for  is naturally defined
as the non-existence of a  derivation with infinitely
many rewrite steps.
Note that this is equivalent to the non-existence of
a derivation with  where infinitely many
of the involved steps use .

\section{Flattening and Other Simplifying Assumptions}
\label{sec-flatten}

\noindent In this section we present some standard transformations
on the signature and TRS~\cite{GodoyTiwari05:CADE,DBLP:conf/rta/GodoyHT07}, 
and argue that they preserve
termination and innermost termination modulo permutative theories.
In particular, we show that we can assume without any loss of generality
that \\
(A1) the signature contains exactly one function symbol with nonzero arity
\\
(A2) all shallow terms are in fact flat.
\\
Readers who believe these claims can jump to the next section.

The discussion is
written for general termination, but it is also valid
when we interpret termination as innermost termination.
To this end, in the innermost case we assume that
for a given TRS, all the rules 
such that  has a proper subterm that is not a normal
form have been removed. Note that these rules can not
be used in an innermost derivation.
Thus, when considering innermost rewriting, we assume that
\\
(A3) if  is a rule in , then all
proper subterms of  are in normal form.

We will always assume that all terms
are constructed over a given fixed signature
 that contains several constants and
only one non-constant function symbol .
If this was not the case,
we can define a transformation  from
terms over  into terms over a new signature 
as follows. Let  be the
maximum arity of a symbol in  plus . We choose
a new function symbol  with arity  and define the new
signature  as
 and .
Note that all symbols of  appear also in 
but with arity .
Now, we recursively define

as  and  for constants  and variables ,
and 
for terms headed with .
We denote  as 
for a given TRS .
Note that the size of  is at most  times the
size of , and hence, this transformation can be
easily performed in polynomial time.
Note also that  is a TRS over , and that 
is a TRS over . As we mentioned in the preliminaries,
we will not explicitly state the signature of each TRS.

\begin{lemma}\label{lemma-simplifying1}
Let  be a TRS. Let  be a permutative theory.
Then,  is (innermost) terminating if and only if 
is (innermost) terminating.
\end{lemma}

\proof
It is straightforward to see that, for any terms 
of ,

and  hold.
Thus, non-termination of  trivially implies non-termination
of .

For the left-to-right direction,
we define the 
transformation  as the following extension of the inverse of 
on the image. 
Since  is not surjective, we will use two new function symbols,
\#m0T'(t)tTT':\Tau(\Sigma',\X)\rightarrow \Tau(\Sigma\cup\{\
as follows:


It is easy to see that any rewrite step
 can be transformed into
a rewrite step ,
and any rewrite step
 can be transformed into
a rewrite step .
Thus, non-termination of  implies non-termination
of  for the signature
,\#\}T'(T(R))T'(T(E))RER/E\Sigma\cup\{\. But, note that
non-termination (and non-termination modulo) of a TRS
does not depend on symbols in the signature that
do not occur in the rules. Hence,  is non-terminating
over the original signature, and we are done.
\qed


In the case where  is left-shallow, we will also assume
that  is, indeed, left-flat. If this was not the case,
we proceed by applying several times the following
transformation step a), until  is left-flat.

\medskip

\noindent
{\bf step a)} If there is a non-constant ground term  that is
a proper subterm of a left-hand side of a rule in , then create
a new constant , replace all occurrences of  in
the left-hand sides of the rules of  by , and add the
rule  to . Formally,
the new TRS  is
.
Note that, as a consequence of Assumption~(A3),
when considering innermost rewriting,  is
necessarily a normal form.

\medskip

We will also assume that all rules in  are right-flat.
If this was not the case, as before
we proceed by applying several times the following
transformation step b), until the obtained TRS is right-flat.


\medskip

\noindent
{\bf step b)} If there is a non-constant ground term  that is a proper
subterm of a right-hand side of a rule in , then create
a new constant , replace all occurrences of  in
the right-hand sides of the rules of  by , and add the
rule  to . Formally,
the new TRS  is
.

\medskip

Every step (a or b) decreases the total sum of the number of positions at
depth more than one in all the left-hand and right-hand sides of .
Moreover, also at every step, the total size of the TRS
increases by at most the size of two constants.
Hence, this process terminates in linear time and the size of
the resulting flat TRS is within a constant factor of the size of
the original shallow TRS.



\begin{lemma}\label{lemma-simplifying2}
Let  be a TRS. Let  be a permutative theory.
Let  be obtained from  by applying step a).
Then,  is (innermost) terminating if and only if
 is (innermost) terminating.
\end{lemma}
\proof
For the right-to-left direction, we first observe
that each rewrite step  can be transformed
into a derivation of the form ,
since the application of a rewrite rule 
can be simulated by several applications of 
and one application of .
Thus, any derivation of  with infinitely many
rewrite steps of  and starting from a certain term 
can be transformed into a derivation of 
with infinitely many rewrite steps of  and starting from .
In the case of innermost rewriting,
we have that  is a normal form and hence, 
the transformed derivation is also innermost.

For the left-to-right direction, we first observe the
following two facts:
\begin{enumerate}[]
\item The existence of a rewrite step  implies
.
\item For each rule  of , 
if , then, for each rewrite step
, it holds that
.
\end{enumerate}
From the above facts, it follows that any rewrite step
 can be transformed into a derivation 

with  or  steps. Note that this is not enough to
argue that a derivation of  with infinitely
many steps with , and starting
from a term , can be transformed into a derivation
of  with infinitely many steps with  starting
from . This is because
rewrite steps with  are, in fact, removed. 
However, it suffices to additionally argue that
a derivation of 
with infinitely many steps of  cannot exist.
This is a consequence of the fact
that rules of  preserve the size, and  decreases
the size.
Finally, in case of innermost rewriting, using the facts
that  is a normal form,  is an innermost
step, and , we infer that 
the transformed derivation is an innermost derivation.
\qed


The preservation of termination for the case of step b)
is proved analogously.

\begin{lemma}\label{lemma-simplifying3}
Let  be a TRS. Let  be a permutative theory.
Let  be obtained from  by applying step b).
Then,  is (innermost) terminating if and only if 
is (innermost) terminating.\qed
\end{lemma}


\section{Right Flat TRS}\label{sec-right-flat}



\noindent In this section, we will show that, given a 
right-flat TRS  and a term ,
it is decidable if  is terminating from .
In particular, this implies that
non-termination is semi-decidable for
right-flat TRS.
We will show that termination is undecidable
for right-flat TRS in Section~\ref{sec-undec}.


The proofs of this section are
written for general termination, but they are also valid
when we interpret termination as innermost termination,
reachability as innermost reachability, and so on.

An important property of a right-flat TRS  is that
whenever  holds, then every subterm of  is
reachable from either a constant or some subterm of .
This result, stated as Lemma~\ref{lemma-reach1}, is 
used extensively later.  It is proved 
by inductively 
marking each position of a term (in the above derivation) by
a term from . The idea of
the marking  at each position  of  is that
it satisfies , and moreover,
for a position  the corresponding marking
 of  at  is context-reachable
from .

\begin{defi}
Let  be a right-flat TRS.
Let  be a derivation with .
A {\em Marking} of this derivation is a sequence
 of  functions

defined inductively as follows:
\begin{enumerate}[]
\item For each  in , we define .
\item For  we assume that  
is defined. Let  be the 'th
rewrite step of the derivation above more explicitly written. Then,
we define  as follows:
\begin{enumerate}[(i)]
\item For each  in 
satisfying , we define .
\item For each  in 
satisfying , , and  is a constant, 
we define .
\item For each  in  
satisfying , , , and  is a variable,
we define , where
 is any position in  such that .
\end{enumerate}
\end{enumerate}
\end{defi}


\noindent Recall that we are assuming that every variable on the right-hand
side also appears on the left-hand side; if not, then the TRS
is trivially non-terminating.
The following example illustrates the definition of marking and
also shows that markings need not be unique.

\begin{example}[Marking]
\label{example-marking}
Let  and consider
the derivation
.
A marking for this derivation is given by:
 for all ,
 for all ,
,
 for all ,
, 
,
, and
.
Note that if we redefined  so that
, then the resulting functions would still be a marking.
Hence, there can be multiple markings for the same derivation.
\end{example}




Now we will state and prove some useful properties about markings.
Henceforth, let us fix  to be a right-flat TRS,
 to be a (innermost) derivation 
and
 to be a marking of this derivation.
The properties below will capture the intuition
that, if , then the term  
is reachable from the term .

\begin{lemma}\label{lemma-reach0}
.
Moreover, if  is not a constant, then,
for each  in  we have .
\end{lemma}

\proof
The claim is proved by induction on .
For , by definition of marking, we have
. Moreover, if  is not a constant,
for each  in  we have
.

For the induction step,
suppose .
By induction hypothesis, we know that  and
whenever  is not a constant then, for each ,
 holds.
The fact that  follows from the fact
that  and 
holds, since Case (i) of the definition of marking applies for
.
Under the assumption that  is not a constant, we note
that Case (ii) defines  as a constant, and cases (i)
and (iii) define  for  in  as
 for some  in .
Thus, from the assumption
that , it follows that .\qed


A second property of markings is that
 is always reachable from .
\begin{lemma}\label{lemma-reach1}
For each  in ,
 is (innermost) reachable from  .
\end{lemma}
\proof
The claim is proved by induction on .
For , by definition of marking, we have  for
each  in . Thus,
 in  steps follows trivially.

For the induction step,
suppose  is the
-th (innermost) rewrite step.
By induction hypothesis,  holds for each .
Consider a fixed . We prove
 as follows:

\begin{enumerate}[]
\item If , then we have
. 
Note that, since  (innermost) rewrites to , it follows that
 (innermost) rewrites to .

\item If  and  hold, and  is a constant, then,
by definition of marking we have  is ,
from which 
in  steps follows trivially.

\item If  and  hold,
and  is a variable,
then, for some ,
 and
 hold, and
 holds
by induction hypothesis. Thus,
 follows.

\item If , then the claim holds by induction
hypothesis again as
 and  hold.
\end{enumerate}
Thus, for each position , we proved that
 is (innermost) reachable from  .
\qed


\begin{cor}\label{corollary-reach1}
If  is a constant, then all subterms of 
are (innermost) reachable from a constant.\qed
\end{cor}

Another property of markings is that 
 is context-reachable from  
for all .
\begin{lemma}\label{lemma-creach1}
For each  satisfying ,
 is (innermost) context-reachable from .
Moreover, if 
 and  are both constants,
then  is (innermost) context-reachable from 
with a non-empty context.
\end{lemma}
\proof
The claim is proved by induction on .
For , by definition of marking we have  for
each  in . Since, for each
 satisfying , 
 holds, then we also have
. Thus, the statement
trivially follows for the base case.

For the induction step,
suppose  is
the 'th (innermost) rewrite step. Consider two fixed positions
 satisfying .
We distinguish the following cases.
\begin{enumerate}[]
\item If  or , then we have
 and .
Thus, the statement follows by induction hypothesis.

\item If , , and  is a constant, then
 holds.
By Lemma~\ref{lemma-reach1},
 is reachable from .
Note that  is a proper subterm of .
Hence,
 is context reachable from 
with a non-empty context (independently of whether
 is a constant or not).

\item If ,  and  is a variable,
then, for some , .
We distinguish two cases. 
(a) If , then 
  and, by induction hypothesis,
 is context reachable from 
(with a non-empty context if both  and
 are constants),
which is the same as saying that
 is context reachable from 
(with a non-empty context if both  and
 are constants).
(b) If  holds, then
 holds for some 
and, by induction hypothesis,
 is context reachable from

(with a non-empty context if both  and
 are constants).
This is the same as saying that
 is context reachable from 
(with a non-empty context if both  and
 are constants).
\end{enumerate}
Thus, in all cases, the claim follows. \qed


We illustrate Lemma~\ref{lemma-creach1} by an example below.
\begin{example}[Lemma~\ref{lemma-creach1}]
Consider again the derivation and marking defined in
Example~\ref{example-marking}.
We note that, on the term ,
we had the marking  defined so that
 and .
By Lemma~\ref{lemma-creach1},  should be context-reachable
from , and indeed we have .
\end{example}

Finally, another observation about a marking is 
that positions below  are always
marked by constants.
\begin{lemma}\label{lemma-creach2}
For each  such that ,
 is a constant.
\end{lemma}
\proof
The claim is proved by induction on .
For , note that  holds and hence all 
satisfy . Thus, the claim is vacuously true.

For the induction step,
suppose  is
the 'th (innermost) rewrite step. Consider any position  satisfying
.
\begin{enumerate}[]
\item If  or ,
then  holds, and by induction hypothesis
 is a constant.
\item
If ,  and  is a constant, 
then  holds, which is a constant.
\item
If ,  and  is a variable, then
 for some , and
 holds since left-hand sides of 
are not variables and  is right-flat.
Hence, the induction hypothesis is applicable and
we can conclude that , and therefore
, is a constant.
\end{enumerate}
Thus, for all  s.t. ,  is a constant.
This completes the proof. \qed


An important consequence of Lemma~\ref{lemma-creach1}
and Lemma~\ref{lemma-creach2}
is that, if  is terminating from ,
then the height of terms reachable from  is
bounded by the height of  plus the number of
constants in .
\begin{cor}\label{corollary-bounded}
Let  be a right-flat TRS.
Let  be a permutative theory.
Let  be a term
such that  is (innermost) terminating from .
Then for any term  (innermost) reachable from  with ,
we have .
\end{cor}
\proof
We proceed by contradiction by assuming
 and .
Recall that the derivation  can be seen
as a derivation .
Let  be a marking of this derivation
.
By Lemma~\ref{lemma-creach2}, each position in  that
is deeper than  is marked with a constant.
Since
 holds,
by pigeon-hole principle, there are two positions
 such that
 and  hold, and  is a constant, say .
By Lemma~\ref{lemma-creach1}, it follows that
 is context reachable from  with a non-empty context.
Moreover, since  is a permutative theory,
 is context reachable from  with a derivation using
at least one rewrite step with a rule of .
Furthermore,
by Lemma~\ref{lemma-reach0} the position  of every term in
a derivation is marked with .
Using Lemma~\ref{lemma-creach1} again,
we infer that , is also context reachable from .
Thus, we can construct a derivation
 with infinitely many
steps with .
Hence, there is a derivation starting from  using 
with infinitely many rewrite steps, a contradiction. \qed


Using the above corollary, we can show that 
the existence of non-terminating derivations 
starting from a term is decidable for right-flat TRS. 
\begin{thm}\label{theorem-c}
Termination (innermost termination) of a right-flat TRS 
modulo a permutative theory  from
a given term is decidable. Hence, non-termination 
(innermost non-termination) is
semi-decidable for right-flat TRS modulo permutative theories.
\end{thm}
\proof
Let  be any term.
We enumerate all (innermost) derivations starting from .
If we reach a term with height greater than
,
then by Corollary~\ref{corollary-bounded} we know
that  is non-terminating from .
Otherwise, we will get only {\em finitely} many
reachable terms.  If there is a derivation that 
cycles among these terms, then  is non-terminating
from . If not, then  is terminating from .  \qed



\noindent
{\em Remark:}
We can use an argument similar to the one used in the proof of 
Theorem~\ref{theorem-c} to prove that,
for any class  of TRS's that are effectively
regularity preserving,
termination of a TRS  of 
from a term , where both  and  are given as input, 
is decidable.
While we do not use this observation here,
we nevertheless note that, using recent results on
regularity preserving TRSs~\cite{TakaiKajiSeki00:RTA},  
we immediately get very simple proofs of known decidability results, such as for
right-ground TRS~\cite{Dershowitz81:ICALP}: a right-ground
TRS is regularity preserving, and is non-terminating iff it is 
non-terminating from some right-hand side, which can
be checked for every right-hand side term using the above
observation.

\ignore{
We state below a related new (up to our knowledge) theorem that
uses a similar argument as the proof
of Theorem~\ref{theorem-c}.
\begin{thm}\label{theorem-unused}
Let  be a class of TRS's that are effectively
regularity preserving.
Termination of a TRS  of 
from a term , where both  and  are given as input, is decidable.
\end{thm}
\proof
Note that  is a regular tree language.
Since  is effectively regularity preserving,
we can compute a tree automaton  recognizing the
set of terms reachable from .
The size of the language recognized by  can be
checked to be infinite, in which case we know that there exists
a derivation starting from  with infinitely many
rewrite steps. Otherwise, we have a finite
number of terms reachable from . By producing all possible
derivations starting from  we will either detect a cycle,
thus concluding non-termination, or will halt, thus concluding
termination.\qed



We shall not use Theorem~\ref{theorem-unused} in this paper.
However, we note here that, using recent results on
regularity preserving TRS~\cite{TakaiKajiSeki00:RTA},  
we immediately get very simple
proofs of known decidability results, such as for
right-ground TRS~\cite{Dershowitz81:ICALP}: a right-ground
TRS is regularity preserving, and is non-terminating iff it is 
non-terminating from some right-hand side, which can
be checked for every one using Theorem~\ref{theorem-unused}.
\endignore}

\section{Innermost Termination of Flat TRS's}\label{sec-innermost}

\noindent In this section, we show that innermost termination 
of flat TRS modulo permutative theories is decidable. In sharp contrast,
general termination is undecidable for 
flat TRS (Section~\ref{sec-undec}).

Let  be a flat TRS, and let  be a permutative theory.
We show decidability of innermost termination of
 by showing that 
if  is not innermost terminating, then
there will be an infinite  derivation starting 
from a ground flat term.
Using Theorem~\ref{theorem-c}, we know that this latter 
check is decidable.

\begin{lemma}\label{lemma-flat-innermost}
Let  be a flat TRS.
Let  be a permutative theory.
Suppose that  is not
innermost terminating. Then, there is an
innermost derivation starting from
a ground flat term with infinitely many innermost rewrite steps.
\end{lemma}
\proof
We assume that
there is no innermost derivation with infinitely many
innermost rewrite steps and starting from a constant, and we
show that there is one from a ground flat term with height .

Since  is not innermost terminating,
there exists an
innermost derivation

with infinitely many innermost rewrite steps using ,
whose first step is at position . 
We first prove that for every ,
{\em every subterm at depth  of  is either
reachable from a constant, or a normal form.}
First note that no term
 is a constant, by our initial assumption.
Moreover, since we use innermost rewriting,
all proper subterms of  are normal forms. 
By Lemma~\ref{lemma-reach1}, all subterms at depth 1
of  are innermost reachable from either constants or proper
subterms of . Hence
they are innermost reachable from constants, or they are normal forms.

Now, we note that there exists at least one constant,
call it , that is
a normal form.  If not, any ground term can be 
innermost rewritten to another ground term, and 
hence there will be innermost derivations starting from
constants with infinitely many innermost rewrite steps,
which contradicts our initial assumption.

We construct a new innermost derivation

by defining each  to be as  but replacing every
subterm at depth  that is not innermost reachable from any constant
by the constant  chosen above. We need to show that 
the new derivation is ``correct'', that is,
there is an innermost rewrite step from  to .
Consider the corresponding innermost rewrite step 
.
\begin{enumerate}[]
\item If  is not , then  is
of the form  for some  in 
and some position . Since  is rewritten,
it is not a normal form. Thus it is innermost reachable from a constant,
and hence,  and  coincide with
 and , respectively.
Therefore, the same innermost rewrite step can be applied on 
to produce .
\item If  is , then,
by our initial assumption, both  and  are not constants.
Moreover,  cannot be a variable, since, 
otherwise,  would be a normal form
since we have innermost rewriting (and the derivation would
be finite).
Hence,  is of the form
.
Recall that, since  is flat, each  and each 
is either a constant or a variable.
If  is the substitution used in this innermost rewrite step,
then define  to be as  except for the cases where 
 is not innermost reachable from a constant, in which case we
define . With these definitions,

is an innermost rewrite step.
\end{enumerate}
The derivation  is again innermost,
has infinitely many innermost rewrite steps with , and the
initial term  satisfies that all its
subterms at depth  are innermost reachable from constants.
Therefore, there exists a ground flat term  with height  such that
 is an innermost derivation,
and hence, there exists
an innermost derivation with infinitely many innermost rewrite
steps starting from a ground flat term 
with height .  \qed




\begin{thm}\label{theorem-innermost}
Innermost termination modulo permutative theories is decidable
for shallow TRS's.
\end{thm}
\proof
By Lemmas~\ref{lemma-simplifying1}, \ref{lemma-simplifying2}
and~\ref{lemma-simplifying3}
innermost termination of shallow TRS modulo permutative theories can
be reduced to the particular case where  is flat
and where the signature contains just one non-constant function symbol.

Since there are only finitely many ground
flat terms, using Theorem~\ref{theorem-c}, we can check
if a given flat  is not innermost terminating starting from
one of these terms. By Lemma~\ref{lemma-flat-innermost},
we will find a witness for non-termination this way
iff  is not innermost terminating.
\qed



\section{Termination and Innermost Termination of Right-Flat Right-Linear TRS's}
\label{sec-right-flat-linear}

\noindent In this section, we show decidability of termination and
innermost termination for right-flat right-linear TRS.
Again, the proofs of this section are written for
general rewriting, but they remain valid for
innermost rewriting.

The proof of decidability of (innermost) termination for
right-flat right-linear TRS depends on two key
observations. The first one is Lemma~\ref{lemma-reach1},
which says that for any (innermost) derivation 
 using a right-flat TRS ,
every {\em proper} subterm of
 is (innermost) reachable from either a constant or a 
{\em proper} subterm of .
The second key lemma is stated by first
defining the following {\em measure} of a term t: 

Note that  depends on
whether we are dealing with general or innermost rewriting.

Let us fix  to be a right-flat right-linear TRS and
 to be a permutative theory.
The first lemma below uses right-linearity of .

\begin{lemma}\label{decreasing}
If , then .
Moreover, if  rewrites to 
at position  with a rule ,
and , then, for every  in ,
if  is not reachable from a constant, then
 is a variable.
\end{lemma}
\proof
Let  be the rewrite
step of the lemma.
We prove the first statement by constructing
an injective map, from positions  of 
such that  is not reachable from a constant, to
positions  of  such that  is not reachable
from a constant, as follows. If  or ,
then we let . If , then  can be written
in the form  where  is a height  term.
In fact,  cannot be a constant since otherwise
 would be a constant. Hence,  is a variable.
We choose a position  such that  is the
same variable as  and define . The injectivity
of the map follows by right-linearity of . Hence,
 holds.

For the second statement,
we assume , that
 is of the form , and  is
of the form . If a certain  is not
reachable from a constant, but  is not a variable, then
 is not in the image of the previous mapping, and hence
 holds, contradicting .
Therefore, all such 's are variables.  \qed


Note that since  is linear and flat, Lemma~\ref{decreasing}
applies to rewrite steps with  too.
In the next lemma, if  is non-terminating,
we establish the existence of 
a non-terminating derivation with certain properties.

\begin{lemma}\label{lemma-main-aux}
If  is (innermost) non-terminating  and
there is no (innermost) non-terminating derivation starting
from a constant, then there is an infinite derivation

with infinitely many rewrites with  and
with the following properties:
\begin{enumerate}[\em(a)]
\item there is no infinite derivation starting from a proper
subterm of 

\item there is no rewrite with a collapsing rule at position 

\item there are infinitely many rewrites at position 
\end{enumerate}
\end{lemma}
\proof
Since  is non-terminating,
there exists a
derivation 
with infinitely many rewrite steps with .
We pick the derivation that has minimal height for .
We claim this derivation has all the properties mentioned above.

It has Property~(a) due to our choice of the infinite derivation.
Next assume that 
 is the first 
application of a collapsing rule at  in the derivatin.
Then,
by Lemma~\ref{lemma-reach0} and
Lemma~\ref{lemma-reach1}, all proper subterms of  are reachable from either a constant or
a proper subterm of . Since  is a proper subterm
of , it is reachable from either a constant or
a proper subterm of . In either case we infer the existence of
a derivation starting from a term with height smaller
than , and involving infinitely many rewrite steps with ,
which contradicts our choice of .
Hence, the infinite derivation we picked has Property~(b).

Finally, we show that 
there are infinitely many rewrite steps at position .
Suppose not. Let 
be the last rewrite step at position . Then, 
there is a derivation starting from
some subterm at depth  of  with infinitely many
rewrite steps with .
As before, this subterm is reachable
from either a constant or a proper subterm of .
Again, this implies the existence of an infinite derivation
that starts from a term with height smaller than .
This contradicts the minimality of .\qed


The idea of the decidability proof is the same as that
for Theorem~\ref{theorem-innermost}, that is, we show
that if  is non-terminating, then it is non-terminating
from a ground flat term.

\begin{lemma}\label{lemma-main}
If  is non-terminating (innermost non-terminating), then
there exists an (innermost) derivation starting from
a ground flat term with infinitely many rewrite steps.
\end{lemma}
\proof
Assume that
there is no infinite derivation starting from a constant.
We will show that there is one from a ground flat term.





Using Lemma~\ref{lemma-main-aux}, we know there is
an infinite derivation,
,
with Properties~(a),~(b) and~(c) from Lemma~\ref{lemma-main-aux}.
All the infinitely many rewrite steps at position 
in this derivation necessarily are done using rules of the form 
,
where the height of  is greater than or equal to .
(If not, then there will be a constant in the derivation.)
By Lemma~\ref{decreasing},  for all .
Since this relation can not be indefinitely decreasing,
for some  we have .
From the derivation 
we construct a new derivation
 with also infinitely many rewrite steps as follows.
Analogously to the
proof of Lemma~\ref{lemma-flat-innermost}, 
we can deduce the existence of 
at least one constant  that is a normal form.
For every , we construct  to be equal to 
except for the subterms at depth  that are not reachable
from constants, which are replaced by . Formally,
 if 
are the subterms at depth  in  that are not reachable
from constants.

We show that the new derivation is correct by analyzing each
rewrite step  and its corresponding
step . 
\begin{enumerate}[(1)]
\item If 
is done at a position inside a subterm at depth  of 
that is reachable from a constant, then, the same rewrite
step can be applied on  to produce .
\item
If 
is done at a position inside a subterm, say ,
at depth  of 
that is not reachable from a constant, then,
 is neither reachable from a constant.
This follows from 
and the fact that, by Lemma~\ref{corollary-reach1},
if a term is reachable from a constant, then all
its subterms are. Thus,  holds, and
hence,  holds.
\item
If 
is done at position , then, by Lemma~\ref{decreasing},
if  and  are the rule
and substitution applied,
then  is a variable for 
every position  such that  
is not reachable from a constant.
We define a new substitution 
to be equal to  except for such variables ,
for which we define . The same rule
 applied to 
at position  and with substitution 
produces .
\end{enumerate}
Since every rewrite step 
at position  corresponds to a rewrite step
, and there are infinitely many of
such steps, it follows that the derivation

has infinitely many rewrite steps.

Note that all subterms at depth  in 
are reachable from constants. Therefore,
there exists a ground flat term  with height  such that
 holds, and hence, there exists
an infinite derivation from a ground flat term .
To finish the proof, we only need to prove that this
infinite derivation has infinitely many rewrite steps with .

We proceed by contradiction
by assuming that
 contains only finitely many rewrite steps
with .  Hence, there exists an  such that
the derivation
 contains {\em{no}} steps with .
Call this derivation .
We can observe the following properties about the
corresponding old derivation
, which we name :
\\
(a) All rewrite steps at position  in 
the derivation  are performed with :  
if there was a rewrite step 
 in , then
we would have had 
 in , which 
contradicts the fact that there are no rewrite steps 
with  in .
\\
(b) In , there are infinitely many rewrite steps of
the form  where  is
not reachable from a constant:
we know that there are infinitely many rewrite steps with 
in , but there are no rewrite steps with  in ,
and hence, all the (infinitely many) rewrite steps with 
in  have to fall in Case~(2) above.

From facts~(a) and~(b), it follows that there is a subterm
 that is not reachable from a constant and
such that there is an infinite derivation
starting from  that uses infinitely many rewrites with . 
This is because any subterm at depth  in the derivation 
that is not reachable from a constant is either
(i) left unchanged by a rewrite step in  (it may be moved to
another position at depth ), or
(ii) it is rewritten into a subterm at depth  that is also not
reachable from a constant (because of the choice of  and the fact
that ). A subterm that is reachable from a constant can not
be rewritten into a term that is not reachable from a constant.

As before, the subterm  is reachable
from either a constant or a proper subterm of .
Hence, there is an infinite derivation with infinitely
many rewrite steps with  starting from a constant or
a proper subterm of , contradicting the
minimality of .\qed 





Now, the main result follows immediately
from Lemmas~\ref{lemma-simplifying1}, \ref{lemma-simplifying2},
\ref{lemma-simplifying3}, \ref{lemma-main} and 
Theorem~\ref{theorem-c}.

\begin{thm}
Termination and innermost termination are both
decidable for rewriting with right-shallow right-linear TRS 
modulo permutative theories.\qed
\end{thm}

\section{Termination is PSPACE-hard for flat right-linear TRS}\label{sec-hardness}

\noindent The algorithms of the previous sections decide termination by 
essentially generating
all terms reachable from ground flat terms up to a height
linearly bounded by the size of TRS . Thus, these algorithms
run in double exponential time, since there are that
many different reachable terms to consider. In this section
we give a lower bound for the time complexity of
these problems.

\begin{thm}
The termination and innermost termination are PSPACE-hard for
flat right-linear TRS.
\end{thm}
\proof
We reduce from the following automata intersection
problem, which is
well-known to be PSPACE-complete~\cite{Kozen77}, to non-termination:
\\
\begin{tabular}{rl}
{\bf Input}: &  finite (word) automata .\\
{\bf Question}: & ?
\end{tabular}

Let  be ,
respectively, more explicitly written. 
Here 
 is the set of states,
 is the alphabet,
 is the initial state,
 is the set of final states and
 is the set of transitions of the -th automaton.
Without loss of generality, we assume that  is .

Our goal is to construct a TRS  satisfying that
 is non-terminating if and only if
 holds.
It is easy to check whether the empty word  is
accepted by all . In the affirmative case we may generate,
as the result of our reduction, a trivially non-terminating TRS.
Thus, from now on, assume that  is not in
.

The idea behind the construction of  is as follows.
A word , say , is encoded by terms, either
 or .
We will include rules in  so that if ,
then  can -reach every possible representation of .
If 
,
then we would like to get a nonterminating derivation
 using the rules
 and 
 in .
To ensure that ``all other rules'' of  are terminating,
the constant  will not reach all terms in , but
only terms of a bounded length.

Let  be .
Let  be .
Formally,  is defined over the following alphabet.

 is defined to contain the following rules:

Now, we prove that
 is non-terminating if and only if
 holds.


:
Suppose that
 is not empty.
In this case, it is well-known that
there exists a word 
 
with size bounded by .
Thus, there exists a term  with height bounded
by , with  in all its internal nodes, and
whose sequence of leaves is .
It is clear that  reaches
.
Moreover, by using the rule ,
this term reaches . Therefore, . 
Hence,  is nonterminating.

: Suppose that 
 is empty.
In order to prove termination of ,
it suffices to prove termination of  starting from any 
right-hand side term of .
Suppose  does not terminate starting from the term .
\\
(a) First, we observe that  terminates from
all constants of  independently of
the form of . 
Hence, .
\\
(b) Consider the case when  is .
But, the fact that 
is empty ensures that  is also terminating from , and hence
.
\\
(c)
If  is , then either there is a derivation
with infinitely many rewrite steps
starting from some  or there is a derivation 
with infinitely many rewrite steps and starting
from .  We argued above that none of these cases is possible.
\\
(d)
If  is ,
then, since there is no rule with left-hand side rooted by ,
there is a derivation with infinitely many rewrite
steps starting from one of the arguments.
We argued above that there are no derivations
with infinitely many rewrite steps and starting from
constants.

We finish the proof by noting that the size of  is 
,
which is polynomial in the size 
of the automata intersection problem.
\qed


\ignore{ \proof
We reduce from the following problem, which is
well-known to be PSPACE-complete, to non-termination:

\noindent
{\bf Input}:  finite (word) automata
.\\
{\bf Question}: ?

Let  be ,
respectively, more explicitly written.
Without loss of generality, we assume that  is .

Our goal is to construct a TRS  satisfying that
 is non-terminating if and only if
 holds.
It is easy to check whether the empty word  is
accepted by all . In the affirmative case we may generate,
as the result of our reduction, a trivially non-terminating TRS.
Thus, from now on, assume that  is not in
.

The idea behind the construction of  is as follows:
 will include rules such that if word ,
then every representation of  as a term is generated from .
Let  be .
Let  be .
We define the following alphabet for .



Intuitively, the goal of each symbol is the following:
 generates a term of the form
,
each  generates the representation of any word
in  whose size is bounded by .
This is done with a binary tree with height
bounded by . The internal binary symbol is ,
and each  is a generator of the
representation of a word 
with size bounded by  such that, starting the execution of  
from , the state  is reached.


 is defined to contain the following rules:



plus the set



Now, we prove that
 is non-terminating if and only if
 holds.
\begin{enumerate}[]
\item Suppose that
 is not empty.
In this case, it is well-known that
there exists a word  with size bounded by  in
.
Thus, there exists a term  with height bounded
by , with  in all its internal nodes, and
whose sequence of leaves is .
It is clear that  reaches
.
Moreover, by using the rules  and ,
this term reaches . Therefore,  reaches  in more than
 -steps. Hence,  is not terminating.

\item Suppose that 
 is empty.
By Lemma~\ref{lemma-main}, in order to prove termination of ,
it suffices to prove termination of  starting from any ground flat term.
We proceed by contradiction by assuming the existence of
a ground flat term  minimal in size from which  does not terminate.

First, suppose that  is a constant.  terminates from
all constants of  independently of
the form of . Thus,  must be .
But the fact that 
is empty ensures that  is also terminating from , a
contradiction.

Second, suppose that  is
of the form  where  and 
are constants. If the
rule  is not used
in a derivation starting from  and with infinitely many
rewrite steps, then there exists a
derivation with infinitely many rewrite steps
and starting from either  or , thus contradicting
the minimality of . Otherwise, if the rule  is used
in this derivation with infinitely many rewrite steps,
then  reaches a term of the form
, which is then rewritten to . But  is reachable
from both  and . This contradicts the minimality
of  again. 

Third, in the case where  is rooted by , a contradiction can
be reached analogously as above. The difference is that,
when  is used,  reaches  as an intermediate
term of the derivation. Thus, there exists a
derivation starting from  and with infinitely many rewrite steps,
which is not possible by our
previous argument.

Finally, suppose that  is of the form .
Since there is no rule with left-hand side rooted by ,
there is a derivation starting from either
 or  with infinitely many rewrite steps,
contradicting again the minimality of .
\end{enumerate}\qed
\endignore}

\section{Undecidability of termination for flat TRS}\label{sec-undec}

\noindent In this section, we prove undecidability of termination for 
flat TRS, and undecidability of innermost termination for 
right-flat TRS. This is done by
a reduction from the Post correspondence problem (PCP)
restricted to nonempty strings, which is defined as:\\
\begin{tabular}{rl}
{\bf Input}: &  pairs of strings 
  s.t. 
  for all 
\\
{\bf Question}:  &
Does there exist  and  s.t.
 and
\\ &
 ?
\end{tabular}


Since decidability of termination for flat TRS is equivalent to
decidability of termination for shallow TRS (Lemmas~\ref{lemma-simplifying2}
and~\ref{lemma-simplifying3}), we will prove 
undecidability of termination for shallow TRS.
Since PCP is not decidable but it is semi-decidable,
and non-termination is semi-decidable for shallow
TRS (Theorem~\ref{theorem-c}), we will
reduce PCP to non-termination of shallow TRS:
a reduction to just termination is not possible.
The reduction is given in the proof of 
Theorem~\ref{theorem-shallow-undec}, but
to provide an intuition, we first illustrate it via a small example.

Consider a PCP instance

over a signature . 
The 'th symbol of  and , whenever it exists,
is denoted by  and  respectively.
For example, 
 
is a PCP instance over .  It has a
solution  since .
We want to define a rewrite system  such that  is non-terminating
iff there is such a solution.  
Let  be the number of rules in the PCP instance
and let  be the maximum size of a string in the PCP instance.
We define  over a signature  where
where

A potential solution, say , to the PCP instance is encoded
by a pair of terms  where

Concretely, the solution  is encoded by the pair

Here the notation 
 serves as a shorthand for the term
. 
This convention allows us to view strings as (parts of) terms.
We need to construct a rewrite system  whose non-termination 
implies that  indeed correspond
to a solution of the PCP.  Hence, we need to check that
\\
(1)  and  are of the form above,
\\
(2) the indices sequence  
in  and  are the same, and
\\
(3) the words  and  are the same.

To check~(1), we introduce the following rules in :

We note that  and
.
Hence we can check~(1) by checking if 

and
.
But this does not still check that the sequence 
(sequence  in the example) used
in  is the same as the one used in .

To check~(2), we make  and  rewrite to
.  Hence, we introduce the following rules in :

Now, using these new rules,
we note that  and  are joinable if
they use the same sequence of indices .
In fact, both  and  rewrite to the term
, where

Moreover, using , 
 can rewrite to either  or .
Thus, we can check~(2) by checking for the joinability of 
 and  to a term that can reach both  and .

Finally, to check~(3), we introduce the following rules in :

Using these rules,
 and  can both rewrite to a common term 
() if
the strings 
 
and
 are equal (to ).
In our example, .
Moreover, the common reachable term () can then rewrite to either  or .
Hence, we can check~(3) by checking for joinability of  and
 to a term that can reach both  and .

We can put everything together by introducing three more rules in :

If  is generated from a solution of the PCP instance, then
we can immediately get a nonterminating derivation using :

The following theorem formally describes and proves this reduction.

\ignore{ and
constructs several sets of rules. At
first look, it is very difficult to figure out the intuition behind
each set. For this reason, in order to help the reader, we
provide a first subsection where simpler
reductions are considered in order to see why they fail.
We give a sequence of simpler but incorrect reductions, in increasing
order of complexity, in order to make each new reduction to
avoid flaws of the previous ones, thus showing the necessity
of the new constructions.

In all the constructions, consider that
 is
the given PCP instance, and that
 is ,
and the 'th symbol of  and , whenever it exists,
is denoted by  and  respectively.

\subsection{Incorrect reductions}\label{subsection-incorrect}

We construct  over a signature  given by


The TRS  is defined as follows:


The idea of the reduction is the following.
For a solution  of the given PCP instance,
we can construct the following non-terminating derivation of ,
starting from the ground term , where

We rewrite  using   to get the
term 
.
Now, we do not touch positions  and  of . 
We rewrite position  using  to , 
where .
We rewrite position  using  to . 
We rewrite position  with  
to .  
Similarly, 
we rewrite position  with  to .
As a result, we reach the term
,
which is rewritten by  to the starting term
.

Thus, the idea is to represent sequences of indexes
with  and , respectively,
to check that both sequences are identical using
 and equality between positions 5 and 6 of
rule , and
to check that both sequences generate the same word
using  and equality between positions 3 and 4
of rule ,

But the above reduction is too simple, and fails for
several reasons. First of all, the subset of rules

is always non-terminating. The problem is that we are not ensuring
that the left and right child of the starting term are constructed using
symbols , respectively. To this end, we propose the
following variation. We add four constant symbols  and the
following rules.



But also to replace  by:



Now, it seems that the loop forces to start with terms
in 
and ,
respectively. The reduction preserves the possitive answer
by redefining  and  as follows.

The reduction fails again for several reasons. For example,
by defining ,  and  as
, 
and ,
respectively, there always exists the non-terminating derivation
.
In this example, the rule  is not checking identity of the words generated
by the index . Thus, we need to force the words
in the positions with 's to be joined by rules in ,
and to force the words in positions with 's to be joined with
. To this end, we add two new constants  and ,
remove the constant  and add the following rules:



We also replace  and  by the following definitions:



But the reduction still fails. For example,
by defining ,  and  as
, 
and ,
respectively, there always exists the non-terminating derivation
.
In the step , we assure that positions  and 
contain a common term reachable from  and , respectively,
and without using . Similarly, we assure that
positions  and 
contain a common term reachable from  and , respectively,
and without using . The problem is that the terms  and ,
respectively, already satisfy this requirement. We would like to check
equality between positions  and , and between positions  and
, but before reducing those terms to  and . To this end
we add four new constants  and replace 
the sets of rules  by:



The current reduction works, as it is shown in the next subsection.


\subsection{The correct reduction}\label{subsection-correct}

\endignore}

\begin{thm}\label{theorem-shallow-undec}
Termination of shallow TRS is undecidable.
\end{thm}
\proof
Consider an instance 
 of the
restricted PCP, that is,
 are nonempty strings over alphabet .
We construct a shallow TRS  such that
this PCP instance  has a solution iff  is non-terminating.

Let .
We construct  over a signature , where  is defined in
Equation~\ref{eqn-sigma}.
The TRS  is defined as follows:

where 
 are defined in Equation~\ref{eqn-ru},
 are defined in Equation~\ref{eqn-rp},
 are defined in Equation~\ref{eqn-ra}
and
 is defined in Equation~\ref{eqn-rf}.

\ignore{
In order to see how this works, we will call
 to the term ,
 to the term ,
 to the term ,

We want to see that, if there is a solution of the original PCP
problem, then the resulting TRS is not terminating.
\endignore}

:
We first show that if the PCP instance has a solution, then
 is non-terminating.
Let 
be a solution of the PCP instance,
i.e.\  holds.
Then, we have the infinite derivation, shown in Equation~\ref{eqn-infinite-derivation},
starting from the ground term , where
 and  are defined in Equation~\ref{eqn-suu-svv}.


:
Suppose  does not terminate. We need to show that the PCP 
instance has a solution.
To this end we define the concept of {\em -variant}.
We say that a term  is a -variant of a term , if
 can be obtained from  by applying several rewrite
steps using rules from the subset
 of .
Note that, since none of  or  is  in 
the original PCP instance,
 and  have the same number
of occurrences of symbols of .

Now, note that since all rules in  are height-preserving or 
height-decreasing,
there is a derivation with infinitely many rewrite steps at the top.
We pick such a derivation, but with minimal height for
the initial term . 
Then, the root symbol of  has to be one of the
's: otherwise, only a finite number of rewrite
steps can be done at the top
and preserving the height.
Therefore, we have a derivation of the form
 with infinitely many rewrite
steps at the top. We can assume that we start
with a term of the form . By observing the  rules,
one can deduce that  and  reach , and that
 reaches  and that  reaches .
This is possible only if the terms  and  are
-variants of terms of the form

where .
But, moreover, these terms have to be joinable to a term of the form
 ,
and also of the form . 
(Note here that since  are not , terms like
 can not rewrite to  and hence
the indices  will be preserved
in any joinability proof.)
Hence,  and  for all  in . But moreover,
 and  have to be joinable to a term of the form
. Hence,
 and there is a solution
of the original PCP.  \qed


\noindent
{\em Remark:}
It is important to keep  and  (and  and ) as two different
constants in the above proof.  If we reuse  in place of  and 
(respectively,  in place of  and ),
then terms that satisfy Check~(1), but do not satisfy Check~(2) 
(respectively, Check~(3)), such as, 
 and , 
which do {\em{not}} correspond to a solution of the PCP,
would generate infinite derivations  starting from .

Combining Theorem~\ref{theorem-shallow-undec} with 
Lemmas~\ref{lemma-simplifying2}
and~\ref{lemma-simplifying3}, we have the following result.


\begin{thm}\label{theorem-flat-undec}
Termination of flat TRS is undecidable.\qed
\end{thm}

For the case of innermost rewriting, we have
seen that termination is decidable for flat TRS.
However, in the innermost case we have the following result.

\begin{thm}\label{thm-inner-right-flat-undecidable}
Innermost termination
of right-flat TRS is undecidable.
\end{thm}
\proof
Given an instance 

of Post correspondence problem,
we generate the TRS
.
Here  is a constant representing the empty string.
Note that  is right-flat.
It is easy to see that the PCP instance has a solution
iff  is innermost non-terminating.
\qed


\noindent
{\em Remark:}
A reduction similar to the one in the proof of 
Theorem~\ref{thm-inner-right-flat-undecidable}
was given in Definition 5.3.6 of~\cite{TeReSe} for
showing undecidability of termination for 
(general) term rewriting systems.

\section{Conclusions}

\noindent We showed that termination and innermost termination 
of right-shallow right-linear term rewriting systems
is decidable.
This result also holds when we consider rewriting modulo permutative
theories.
We also showed that innermost termination of flat TRSs
is decidable, whereas termination of flat TRSs is undecidable.
For the decidable problems,
the complexity of the given algorithms
is doubly exponential, whereas we have also
provided a PSPACE-hardness lower bound.
It is unclear whether both upper and lower bounds can be improved
in some way. As further work it would be interesting
to fix the exact complexity of these problems, but also to
consider other classes of TRS,  for example, classes defined 
by imposing syntactic restrictions not on the original TRS,
but on the dependency pairs of the TRS~\cite{Sakai06,Sakai}.



\begin{thebibliography}{BKdV03}

\bibitem[BKdV03]{TeReSe}
M.~Bezem, J.~W. Klop, and R.~de~Vrijer, editors.
\newblock {\em Term Rewriting Systems}.
\newblock Cambridge Tracts in Theoretical Computer Science 55. Cambridge
  University Press, 2003.

\bibitem[BN98]{Allthat}
F.~Baader and T.~Nipkow.
\newblock {\em Term Rewriting and All That}.
\newblock Cambridge University Press, New York, 1998.

\bibitem[Der81]{Dershowitz81:ICALP}
N.~Dershowitz.
\newblock Termination of linear rewriting systems.
\newblock In {\em Proc. 8th Colloquium on Automata, Languages and Programming,
  ICALP}, volume 115 of {\em LNCS}, pages 448--458, 1981.

\bibitem[GHT07]{DBLP:conf/rta/GodoyHT07}
G.~Godoy, E.~Huntingford, and A.~Tiwari.
\newblock Termination of rewriting with right-flat rules.
\newblock In {\em Proc. 18th Intl. Conf. on Rewriting Techniques and
  Applications, RTA}, volume 4533 of {\em LNCS}, pages 200--213, 2007.

\bibitem[GT04]{GodoyTiwari04:IJCAR}
G.~Godoy and A.~Tiwari.
\newblock Deciding fundamental properties of right-(ground or variable) rewrite
  systems by rewrite closure.
\newblock In {\em Proc. Intl. Joint Conf. on Automated Deduction, IJCAR},
  volume 3097 of {\em LNAI}, pages 91--106. Springer, July 2004.

\bibitem[GT05]{GodoyTiwari05:CADE}
G.~Godoy and A.~Tiwari.
\newblock Termination of rewrite systems with shallow right-linear, collapsing,
  and right-ground rules.
\newblock In {\em Proc. 20th Intl. Conf. on Automated Deduction, CADE}, volume
  3632 of {\em LNCS}, pages 164--176. Springer, July 2005.

\bibitem[HL78]{HuetLankford78}
G.~Huet and D.~S. Lankford.
\newblock {\em On the uniform halting problem for term rewriting systems}.
\newblock INRIA, Le Chesnay, France, 1978.
\newblock Technical Report 283.

\bibitem[Koz77]{Kozen77}
D.~Kozen.
\newblock Lower bounds for natural proof systems.
\newblock In {\em Proc. 18th Symp. on the Foundations of Computer Science},
  pages 254--266, 1977.

\bibitem[Pla93]{Plaisted93:RTA}
D.~A. Plaisted.
\newblock Polynomial time termination and constraint satisfaction tests.
\newblock In {\em Proc. 5th Intl. Conf. on Rewriting Techniques and
  Applications, RTA}, volume 690 of {\em LNCS}, pages 405--420, 1993.

\bibitem[TKS00]{TakaiKajiSeki00:RTA}
T.~Takai, Y.~Kaji, and H.~Seki.
\newblock Right-linear finite path overlapping term rewriting systems
  effectively preserve recognizability.
\newblock In {\em Proc. 11th Intl. Conf. on Rewriting Techniques and
  Applications, RTA}, volume 1833 of {\em LNCS}, pages 246--260, 2000.

\bibitem[Toy87]{Toy87b}
Y.~Toyama.
\newblock Counterexamples to termination for the direct sum of term rewriting
  systems.
\newblock {\em Information Processing Letters}, 25:141--143, 1987.

\bibitem[USS10]{Sakai}
K.~Uchiyama, M.~Sakai, and T.~Sakabe.
\newblock Decidability of termination and innermost termination for term
  rewriting systems with right-shallow dependency pairs.
\newblock {\em {IEICE} Trans. on Information and Systems}, E93-D(5):953--962,
  2010.

\bibitem[WS06]{Sakai06}
Y.~Wang and M.~Sakai.
\newblock Decidability of termination for semi-constructor trss, left-linear
  shallow trss and related systems.
\newblock In {\em Proc. 17th Intl. Conf. on Rewriting Techniques and
  Applications, RTA}, volume 4098 of {\em LNCS}, pages 343--356. Springer,
  2006.

\end{thebibliography}


\end{document}
