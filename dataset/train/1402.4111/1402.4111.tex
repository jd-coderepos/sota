In this section, we present a new LP relaxation for \textsc{Non preemptive minimum energy scheduling ($\alpha$)} on a single processor. 
\begin{thm}\label{thm:intgap}
The linear program  \textbf{LP1} has integrality gap at most $\lpgap$.
\end{thm}
As a corollary, we obtain the following theorem. 
 

\begin{thm}\label{thm:existsingproc}
  There exists a polynomial-time algorithm which computes a $\singleprocgap$-approximation to the \textsc{Non preemptive minimum energy scheduling ($\alpha$)} problem on a single processor
\end{thm}
\begin{proof}
Since our proof of the integrality gap of \textbf{LP1} is algorithmic, it is straightforward to obtain the claimed ratio:

Given an instance of \textsc{Non preemptive minimum energy scheduling ($\alpha$)}, write the linear program \textbf{LP1} corresponding to this instance, solve it to obtain a fractional solution and then use Algorithm \ref{alg:singleproc} to obtain an integral solution of value at most $\lpgap$ times the value of the fractional solution and output it. Since the fractional solution obtained was optimal, by Lemma \ref{lem:discretization}, it has value at most $(1+\varepsilon)^{\alpha-1}$ times the energy consumed by the optimum schedule. Thus, the integral solution we output attains the claimed bound.
\qed \end{proof}

\subsection{Linear programming formulation}

To model the problem, we start from 0-1 variables $x_{I,j}$ indexed by a job $j$ and an execution interval $I$ which indicate whether $j$ is assigned to $I$. To bound the number of variables, we use the following result of Huang and Ott \cite{Huang_Ott} which allows us to restrict our attention to schedules where all execution intervals begin and end in some set  $T$ of time points such that $|T|$ is  polynomial in the input.



\begin{lem}[Discretization of time]\label{lem:discretization}\cite{Huang_Ott}
    Let $r_1,\ldots,r_{2n}$ be the release dates and deadlines of jobs.
    For each $1 \le i < 2n$, create $n^2 (1 + \frac{1}{\varepsilon}) -1$ equally-spaced ``landmarks'' in the interval $[r_i, r_{i+1}]$.
    Let $S$ be a solution of minimal cost such that for each job $j$ and each consecutive landmarks $t_i, t_{i+1}$, either
    job $j$ is executed during the whole interval $[t_i, t_{i+1}]$ or not at all.
    Then $E(S) \le (1 + \varepsilon)^{\alpha - 1} \text{OPT}$.
\end{lem}


Thus, we consider the set  ${\cal I}$ of all the  intervals with both endpoints in a landmark to be   the set of the allowed execution intervals.
Since $j$ must be scheduled somewhere, $\sum_{I} x_{I,j} = 1$. Since at any time $t$, at most one job is being processed, 
$\sum_j \sum_{I\ni t} x_{I,j}\leq 1$. Our LP, which we now state, contains an additional constraint (\ref{lp1-non-preempt}) capturing non-premption:
if some job $j$ is scheduled  during some interval $I$ or a subinterval thereof, then no other job can be scheduled during an interval that contains $I$. This holds for non-preemptive schedules but not necessarily for preemptive schedules, and in that sense this new constraint ``captures" non-preemption. The constraint is necessary to bound the integrality gap: without it, there exist instances and  fractional solutions that have much lower value than their integer counterpart (Lemma~\ref{lem:gap_lp1}, proved in below using  the instance described Figure~\ref{fig:exmpl_LP}.)   
\begin{alignat}{2}
    \textbf{\text{LP1:}} \quad \quad \text{minimize }   &E(\vec{x}) = \sum_{j \in J} \sum_{I \in \mathcal{I}} x_{I,j} \left(\frac{w_j}{|I|}\right)^{\alpha}  |I| &\\
    \text{subject to } & \sum_{I \in \mathcal{I}} x_{I,j} \ge 1 & \forall j \in J \label{lp1-job-assigned}\\
                       & \sum_{j \in J} \sum_{\substack{I \in \mathcal{I}\\ t \in I}} x_{I,j} \le 1 & \forall \text{ landmark } t \label{lp1-no-overlap}\\
                       & \sum_{\substack{I' \in \mathcal{I}\\ I' \cap I \neq \emptyset}} x_{I',j}  + \sum_{\substack{I'' \in \mathcal{I}\\ I \subseteq I'', j' \in J}} x_{I'',j'} \le 1& \forall I \in \mathcal{I}, \forall j \in J \label{lp1-non-preempt}\\
                       & x_{I,j} \ge 0 & \forall j \in J, \forall I \in \mathcal{I}(j) \label{lp1-non-negativity}\\
                       & x_{I,j} = 0 & \forall I \notin \mathcal{I}(j)
\end{alignat}



\begin{lem}\label{lem:gap_lp1}
Without constraint~(\ref{lp1-non-preempt}), \textbf{LP1} has integrality gap at least $\Omega (n^{\alpha-1})$.
\end{lem}
\begin{proof}
We construct an instance on which the integrality gap is at least $\Omega (n^{\alpha -1})$.
Let us define $n+1$ jobs. We create $n$ \emph{small} jobs and a \emph{big} job.
The $i^{\text{th}}$ small job has release date $i-1$ and deadline $i$ and processing requirement 1.
The big job has release date 0 and deadline $n$ and a processing requirement of $n$.
More details about this instance are given figure \ref{fig:exmpl_LP}.

An integral solution will have to process the big job between two consecutive small jobs and thus,
will have a cost of at least $(\frac{n+2}{2})^{\alpha} \cdot 2 = \Omega (n^{\alpha})$.

Now, consider the following assignment of the variables. For any job $j$, let $L_j^1$ and $L_j^2$ be respectively the
first and second halves of its life interval.
We set $x_{L_j^1,j} = x_{L_j^2,j} = 1/2$ and the other $x_{I,j}$ to 0. Hence constraint \ref{lp1-job-assigned} is satisfied.
Consider now a time $t$ which is not a release date or a deadline. At this time, we can process a small job and the big job.
For both of them, we are either on their first or second halves and so, the sum of the $x_{I,j}$ such that $I$ contains $t$ is
at most 1. It follows that constraint \ref{lp1-no-overlap} is also satisfied and that the assignment described is a solution for
the linear program without constraint \ref{lp1-non-preempt}.

By doing this, each job costs $2^{\alpha -1}$, hence the optimal fractional solution has value at most $n \cdot (2^{\alpha -1}) = \mathcal{O}(n)$.
Therefore, the integrality gap is at least $\Omega (n^{\alpha -1})$.
\qed\end{proof}

\begin{figure}
   \begin{center}
        \includegraphics[scale=0.9]{IG_LP_wo_cst3.pdf}
   \end{center}
   \caption{An instance on which LP1 has an integrality gap of at least $\Omega (n^{\alpha -1})$.
   The instance contains a \emph{big} job and $n$ \emph{small} jobs. The figure shows the life intervals of the jobs
   and, above them, their work volume.}
   \label{fig:exmpl_LP}
\end{figure}


The remainder of this section is devoted to proving Theorem~\ref{thm:intgap}.

\subsection{Overview}

We show that any fractional solution can be transformed into an integral solution without increasing the value of the solution by too much, in three steps. 

We first divide the time into zones and transform the fractional solution so all (non-zero) fractional execution intervals are inside a zone. Then, each zone is divided into nested subzones and we further transform our fractional solution so that all fractional execution intervals are inside a subzone and the life interval of the corresponding job contains that subzone. Finally, we build a weighted bipartite graph from the transformed fractional solution whose edges represent the possible allocation of execution intervals to subzones. Similarly to~\cite{Shmoys_Tardos}, we find an integral (weighted) matching in this graph and translate this solution to an integral schedule.

We then show that the cost of the integral solution we built is at most $\lpgap$ times the cost of the original fractional solution.



\subsection{Building an integral solution from a fractional solution}

We now give the detailed description for our procedure to transform a fractional in three steps. The first step is derived from Antoniadis and Huang's algorithm \cite{Antoniadis_Huang}. 

\subsubsection{Splitting execution intervals on deadlines}

Our first transformation turns a fractional solution into a fractional solution where $x_{I,j}$ is 0 for any execution interval $I$ that any points in a set of deadlines we pick. The deadlines we pick are the deadlines of a good independent set.

\begin{lem}\label{lem:deadline_zones}
  Let \textbf{LP1} be the LP obtained from an instance of \textsc{Non preemptive minimum energy scheduling ($\alpha$)} and $\cal{J}$ be any good independent set for this instance.

  In polynomial time, we can transform any fractional solution $\vec{x}$ to \textbf{LP1} to a fractional solution $\vec{y}$ of value at most $2^{\alpha-1}E(\vec{x})$ where $y_{I,j}=0$ if $I$ crosses a deadlines of $\cal{J}$.
\end{lem}

To prove this lemma, we will simply ``shift'' some of the values of $\vec{x}$.
\begin{defn}
  By \emph{shifting} $y_{I,j}$ to $y_{I',j}$ for a fractional solution $\vec{y}$ to \textbf{LP1}, we mean to increase $y_{I',j}$ by $y_{I,j}$ and decrease $y_{I,j}$ to 0.
\end{defn}

\begin{proof}
  First note that $x_{I,j}=0$ if $I$ crosses \emph{two} deadlines of $\cal{J}$. Indeed, otherwise, $I$ contains the execution interval of some job $j'$ of the independent set and $\vec{x}$ doesn't satisfy constraint (\ref{lp1-no-overlap}).

  We build $\vec{y}$ iteratively, starting with $\vec{y}=\vec{x}$.

  For each $x_{I,j}>0$ and $I=[s,e]$ crosses a deadline $d$ in $\cal{J}$ (so $s<d<e$), $I'$ the larger of the two intervals $[s,d]$, $[d,e]$ has size at least half the size of $I$. We shift $y_{I,j}$ to $y_{I',j}$. 

  This process shrink the size of each execution interval in the fractional solution by a factor of at most 2 and thus $E(\vec{y})$ is at most $2^{\alpha-1}E(\vec{x})$.

  Constraint (\ref{lp1-job-assigned}) is still satisfied as shifting $x_{I,j}$ from $y_{I,j}$ to $y_{I',j}$ preserves the sum of probabilities over $j$ (and does not affect the constraint for any other job $j' \ne j$).
  Constraint (\ref{lp1-no-overlap}) is still satisfied as $I'$ is contained in $I$ (so fewer (fractional) execution intervals cross each deadline in $\cal{J}$).
  Again, since $I'$ is contained in $I$, we see that constraint (\ref{lp1-non-preempt}) is satisfied since an inequality in constraint (\ref{lp1-non-preempt}) containing the term $x_{I',j}$ contains the term $x_{I,j}$.
\qed\end{proof}




\subsubsection{Further splits}

We now proceed to our second transformation and show that we can further split the execution intervals of a fractional solution. Now that all execution intervals
(in the support of $\vec{x}$) lie between two consecutive deadlines (which we now call a ``zone''), we can further partition each zone so the first half
is dedicated to jobs whose life interval ends in that zone and the second half is dedicated to the others (namely, jobs whose life interval starts
in that zone or jobs whose life interval contains the zone).


\begin{lem}\label{lemma:subzones}
  Let \textbf{LP1} be obtained from an instance of \textsc{Non preemptive minimum energy scheduling ($\alpha$)} and $\cal{J}$ be any good independent set for this instance.

  In polynomial time, we can transform any fractional solution $\vec{y}$ where $y_{I,j}=0$ if $I$ crosses a deadlines of $\cal{J}$ to \textbf{LP1} to a fractional solution $\vec{z}$ of value at most $2^{\alpha-1}$ times the value of $\vec{y}$  where $z_{I,j}>0$ implies
  \begin{enumerate}
  \item
    $I \subseteq [d_s, d_s + \frac{1}{2^{k}}(d_e-d_s) ] \subseteq L_j$ for some consecutive deadlines $d_s, d_e$  of $\cal{J}$ and $k \ge 1$, or
  \item
    $I \subseteq [d_e - \frac{1}{2^{k}}(d_e-d_s), d_e ] \subseteq L_j$ for some consecutive deadlines $d_s, d_e$  of $\cal{J}$ and $k \ge 1$.
  \end{enumerate}
\end{lem}

We let ${\cal Z}$ consists of all intervals of the form $[d_s, d_s + \frac{1}{2^{k}}(d_e-d_s) ]$ and $[d_e - \frac{1}{2^{k}}(d_e-d_s), d_e]$ 
for consecutive deadlines $d_s, d_e$ of $\cal{J}$.

Though they do not partition the timeline, we still refer to ${\cal Z}$ as \emph{subzones}. We now prove the following refinement of the above lemma where we make ${\cal Z}$ explicit.

\begin{proof}
  First set $\vec{z}$ to be $\vec{y}$.

  Let $[d_s,d_e]$ be a zone defined by two consecutive deadlines. No life interval $L_j$ is contained in $[d_s,d_e]$ (or $\cal{J}$ is not a good independent set as we could add $L_j$ and still obtain an independent set). Thus, we can partition jobs whose life interval intersects $[d_s,d_e]$ into jobs $E$ whose life interval starts (at or) before $d_s$ and jobs $S$ whose life interval end (at or) after $d_e$ (putting jobs that can go into both into either set).

  If $z_{I,j}>0$ and $I$ intersects $[d_s,d_e]$ then $I$ is a subset of $L_j$ which also intersects $[d_s,d_e]$ so $j$ is in $S$ or $E$.

  We now describe how to shift $z_{I,j}$ with $j$ in $E$ (the shift for jobs in $S$ is symmetric). We simulaneously shift all $z_{I,j}>0$ with $I=[s,e]$. Since $I$ does not cross $d_s$ or $d_e$, it ends in $[d_s,d_e]$ and $e$ is in $[d_s + \frac{1}{2^k}, d_s + \frac{1}{2^{k-1}}(d_e-d_s)]$ for some $k$. We shift $z_{I,j}$ to $z_{I',j}$ where $I=[s-\frac{1}{2}(s-d_s) , e-\frac{1}{2}(e-d_s)$.

  Shifts for $z_{I,j}$ with $j$ in $S$ are defined symmetrically (by reversing the timeline).

  Since each execution interval of $\vec{y}$ is shifted to an execution interval exactly half its original size, by Lemma \ref{lemma:half}, $E(\vec{z})$ is at most $2^{\alpha-1}E(\vec{y})$.
  Since we only shifted intervals, constraint (\ref{lp1-job-assigned}) remains satisfied. Constraint (\ref{lp1-no-overlap}) is satisfied as we simply compressed the entire region $[d_s, d_e]$ to $[d_s, d_e-\frac{1}{2}(d_e-d_s)]$ for jobs in $S$ and the entire region $[d_s, d_e]$ to $[d_s +\frac{1}{2}(d_e-d_s), d_e]$ for jobs in $E$. Finally, constraint (\ref{lp1-non-preempt}) is satisfied by $\vec{z}$ since again these constraint were satisfied by $\vec{y}$ and we only compressed some block of execution intervals into disjoint regions of the timeline.
\qed\end{proof}

\subsubsection{Building a weighted bipartite matching}

As a result of Lemma \ref{lemma:subzones}, for each $z_{I,j}>0$, $I$ is contained in some subzone $Z$ and furthermore, the life interval of $j$ contains $Z$ so we can freely shift $I$ to another interval (of the same length) inside $Z$.
Thus, we will only remember the length of the fractional execution intervals and the subzone $Z$ in which they belong. I.e., we think of $\vec{z}$ as a fractional assignment of lengths $\ell_i$ for each job to $Z$.

\begin{lem}\label{lemma:earliest_deadline_first}
  If for each subzone $Z$, the lengths $\ell(e)$ assigned to $Z$ and all subzones included in $Z$ is at most $|Z|$ then there is a feasible schedule where each job is given their assigned length in $Z$.
\end{lem}
\begin{proof}
We greedily assign $j$ an interval of length $\ell(e)$ to the leftmost possible empty spot if $Z$ is of the form $[d_s, d_s + \frac{1}{2^{k}}(d_e-d_s) ]$. In fact, at each intermediate step, all jobs currently assigned to $Z$ take up a contiguous region starting from the left endpoint $d_s$ of $Z$. If $Z$ is of the form $[d_e - \frac{1}{2^{k}}(d_e-d_s), d_e ]$, we assign $j$ the rightmost possible interval of length $\ell(e)$.
We see by induction on the number of subzones contained in $Z$ that every job is processed in $Z$.
\end{proof}



We now desire an integral assignment of lengths to each $Z$ where the total of all lengths assigned to $Z$ does not exceed $|Z|$. Note that this constraint is satisfied by the fractional solution derived from $\vec{z}$ (as $\vec{z}$ satisfies constraint (\ref{lp1-no-overlap})).

To obtain such an integral assignment from our fractional assignment derived from $\vec{z}$, we build a weighted bipartite graph $G(\vec{z})$ where an assignments correspond to matchings and the weight of a matching correspond to the energy cost (of the matching interpreted as a schedule). We will then obtain an integral matching from the derived fractional matching (whose weight is exactly $E(\vec{z})$.

We now describe $G(\vec{z})$ with bipartition $(A,B)$ and weight $w(e)$ for each edge $e \in E(G)$. We also keep a \emph{length} $\ell(e)$ for each edges which will be used in the very last step of our proof (but in no way affects the weighted bipartite matching we look for).

\begin{itemize}
\item
  $A$ contains one vertex for each job. I.e., $A = \{a_j | j \in J\}$
\item
  $B$ consists of vertices for subzones. However, $B$ may contain more than one vertex for each subzone $Z$. In fact, it contains the ceiling of the sum of fractional value of all lengths assigned to $Z$. I.e.,
  \[
  B = \left\{b_{Z,i} | Z \in {\cal Z}, i \in 1,\ldots, \left\lceil \sum_{j \in J} \sum_{I \subseteq Z } z_{I,j} \right\rceil \right\}
  \]


\item
  The edges are constructed as followed. Start with all edges $a_jb_{Z,i}$ for all $i$ if $z_{I,j}>0$ for some $I \subseteq Z$. We now delete some edges to obtain the edges of $G(\vec{z})$ and assign weights and length of the remaining edges.

  Sort the lengths assigned to $Z$ by $\vec{z}$ in decreasing order of length. For each such length $\ell_k$ for job $j$ of fractional value $z_{I,j}$, set $w(a_jb_{Z,i})$ to $\frac{w_j^\alpha}{\ell_k^{\alpha-1}}$ where $i$ is the ceiling of the partial sum of all jobs previously considered for $Z$ (i.e., $i=\lceil \sum_{q=1}^{k-1} \ell_{q} \rceil$). Set $\ell(a_jb_{Z,i})$ to $\ell_k$. Also set $w(a_jb_{Z,i+1})$ to $\frac{w_j^\alpha}{\ell_k^{\alpha-1}}$ and $\ell(a_jb_{Z,i+1})$ to $\ell_k$ if adding $j$ to the ceiling of the partial sum increases it by 1. Delete all other edges of the form $a_jb_{Z,t}$.
\end{itemize}

$\vec{z}$ naturally gives the following fractional matching $M(\vec{z})$ of $G(\vec{z})$ with total weight $E(\vec{z})$: we pick each edge with weight exactly $z_k$ (or $z_k$ split into two as follows if adding $z_k$ increased the ceiling of the partial sum by 1. Whatever we need to add to the partial sum to make it an integer is the fraction we choose of the first edge, and the rest of $z_k$ for the second edge).

To complete the description of our final transformation, we apply the following two technical lemmas.


\begin{lem}\label{lemma:integer-matching}\cite{Lovasz_Plummer_matching,Shmoys_Tardos}
  In a weighted bipartite graph, there exists an (integral) matching of same weight as any fractional matching\footnote{in a fractional matching edges can be selected with a fractional value as long as the total value of edges incident to any vertex is at most 1}.
\end{lem}

\begin{lem}\label{lemma:matching-to-schedule}
  Let $G(\vec{z})$ be the bipartite graph built from a transformed fractional solution $\vec{z}$.
  For any matching $M$ saturating $A$ of $G(\vec{z})$, we can obtain a schedule whose energy consumption is at most $3^{\alpha-1}$ times the weight of $M$. \end{lem}
\begin{proof}

\textbf{Schedule construction}.
Since $M(\vec{z})$ has weight $E(\vec{z})$ and saturates all of $A$, by Lemma \ref{lemma:integer-matching} there exists an integer matching $M'$ of the same weight that saturates $A$.

We build a schedule $S$ from $M'$ as follows. Consider subzones in order of containment starting with subzones containing no other subzones. For each edge $e=(a_j, b_{Z,i}) \in M'$ give $j$ an execution interval of length $\ell(e)/3$ in $Z$. By Lemma \ref{lemma:matching-to-schedule}, we only need to verify that for all subzones $Z$, the sum of lengths for all subzones contained in $Z$ does not exceed $|Z|$.

\textbf{Feasibility}.
We now check that for each subzone $Z$, the sum of lengths for all subzones contained in $Z$ does not exceed $|Z|$.

\begin{figure}[ht!]
   \begin{center}
        \includegraphics{lp-matching-fig.pdf}
   \end{center}
   \caption{An illustration of the feasibility proof. The fractional solution is represented in green with $z_{I,j}$ corresponding to the height of a rectangle. (b) Recall that $G(\vec{z})$ is built by first ordering the execution intervals by length. (c) In the worst case, the matching picked the longest edge incident to each vertex.}
   \label{fig:LP_matching}
\end{figure}

Let $v(Z)$ be the sum of all lengths assigned to $Z$ by the fractional solution, weighted by $z_{I,j}$ (i.e., $v(Z) = \sum_{j, I, I \subseteq Z \subseteq L_j} z_{I,j} |I|$). Since $\vec{z}$ satisfies constraint (\ref{lp1-no-overlap}), we have
\[
\sum_{Z' \subseteq Z} v(Z') \le |Z|
\]
for all $Z$.



Let $mi(Z,i)$ be the minimum length of all edges incident to $b_{Z,i}$ and $ma(Z,i)$ be the maximum length of all edges incident to $b_{Z,i}$. Since we considered lengths in non-increasing order of their $z$ values, $mi(Z,i) \ge ma(Z,i+1)$ for all $i$ and $Z$.


Let $n(Z)$ is the number of copies of vertex $b_{Z,i}$ and $\ell(Z)$ be the sum of lengths of all edges of $M$ incident to $\{b_{Z,i}\}_{i=1}^{n(Z)}$.
For all $Z$,
\[
\ell(Z) \le \sum_{i=1}^{n(Z)} ma(Z,i) \le ma(Z,1) + \sum_{i}^{n(Z)-1} mi(Z,i) \le |Z| + v(Z).
\]
and therefore the sum of lengths assigned to $Z$ in our schedule is at most $\frac{1}{3}(|Z| + v(Z))$.


Now for each $Z$, the sum of all lengths assigned to $Z$ and all subzones included in $Z$ in our schedule is at most
\[
\frac{1}{3}\left(\sum_{Z' \subseteq Z} |Z'|+ \sum_{Z' \subseteq Z} v(Z')\right) \le \frac{1}{3}\left(\left(\sum_{q=0}^\infty \frac{|Z|}{2^i}\right) + |Z|\right) \le \frac{1}{3}(3|Z|)
\]
and so our assigned lengths are feasible.
\qed\end{proof}

\begin{proof}(of Theorem \ref{thm:intgap})
  To prove the integrality gap of \textbf{LP1}, we simply need to apply each lemma in this section in turn.

  Given an optimal fractional solution $\vec{x}$ to \textbf{LP1}, find a good independant set $\cal{J}$ (to the instance which generate the LP) and apply Lemma \ref{lem:deadline_zones} to $\vec{x}$ and $\cal{J}$ to obtain a fractional solution $\vec{y}$ of value at most $2^{\alpha-1}E(\vec{x})$ where no execution interval crosses a deadline in $\cal{J}$.

  Then apply Lemma \ref{lemma:subzones} to $\vec{y}$ to obtain $\vec{z}$ of value at most $2^{\alpha-1}E(\vec{y}) \le 4^{\alpha-1}E(\vec{x})$ where for any non-zero $z_{I,j}$, $I$ is contained in a ``subzone'' of the form $[d_s, d_s + \frac{1}{2^{k}}(d_e-d_s)]$ or $[d_e - \frac{1}{2^{k}}(d_e-d_s), d_e ]$ for some consecutive deadlines $d_s, d_e$ and $k \ge 1$ and furthermore $L_j$ contains this subzone.

  Now build $G(\vec{z})$ and interpret $\vec{z}$ as a fractional matching in $G(\vec{z})$. By Lemma \ref{lemma:integer-matching}, $G(\vec{z})$ has a matching $M$ of same weight as this fractional matching and by Lemma \ref{lemma:matching-to-schedule}, we can build a schedule from $M$ whose energy consumption is at most $3^{\alpha-1}E(\vec{z}) \le 12^{\alpha-1}E(\vec{x})$.

  Such a schedule is of course a solution to \textbf{LP1} of same value, thus completing the proof.
\qed\end{proof}









\subsubsection{Algorithm summary}

We can summarize our algorithm from transforming any fractional solution $\vec{x}$ to an integral solution.

\begin{alg}\label{alg:singleproc}
~
\begin{enumerate}
\item Apply the transformation of Lemma \ref{lem:deadline_zones} and then Lemma \ref{lemma:subzones} to the fractional solution to obtain a new fractional solution $\vec{z}$.
\item Construct the weighted bipartite graph $G(\vec{z}) = (A,B)$.
\item Find a minimum weight matching $M$ that matches every node in $A$.
\item For each edge $e = (a_j, b_{Z, i}) \in M$, schedule job $j$ in the subzone $Z$ with an interval of length $\frac{\ell(e)}{3}$. Use an earliest deadline first schedule for all jobs in $Z$ if $Z$ is in the first half of a zone and an latest release date first schedule if $Z$ is in the second half of a zone.
\end{enumerate}

\end{alg}
