In this section, we present a new LP relaxation for \textsc{Non preemptive minimum energy scheduling ()} on a single processor. 
\begin{thm}\label{thm:intgap}
The linear program  \textbf{LP1} has integrality gap at most .
\end{thm}
As a corollary, we obtain the following theorem. 
 

\begin{thm}\label{thm:existsingproc}
  There exists a polynomial-time algorithm which computes a -approximation to the \textsc{Non preemptive minimum energy scheduling ()} problem on a single processor
\end{thm}
\begin{proof}
Since our proof of the integrality gap of \textbf{LP1} is algorithmic, it is straightforward to obtain the claimed ratio:

Given an instance of \textsc{Non preemptive minimum energy scheduling ()}, write the linear program \textbf{LP1} corresponding to this instance, solve it to obtain a fractional solution and then use Algorithm \ref{alg:singleproc} to obtain an integral solution of value at most  times the value of the fractional solution and output it. Since the fractional solution obtained was optimal, by Lemma \ref{lem:discretization}, it has value at most  times the energy consumed by the optimum schedule. Thus, the integral solution we output attains the claimed bound.
\qed \end{proof}

\subsection{Linear programming formulation}

To model the problem, we start from 0-1 variables  indexed by a job  and an execution interval  which indicate whether  is assigned to . To bound the number of variables, we use the following result of Huang and Ott \cite{Huang_Ott} which allows us to restrict our attention to schedules where all execution intervals begin and end in some set   of time points such that  is  polynomial in the input.



\begin{lem}[Discretization of time]\label{lem:discretization}\cite{Huang_Ott}
    Let  be the release dates and deadlines of jobs.
    For each , create  equally-spaced ``landmarks'' in the interval .
    Let  be a solution of minimal cost such that for each job  and each consecutive landmarks , either
    job  is executed during the whole interval  or not at all.
    Then .
\end{lem}


Thus, we consider the set   of all the  intervals with both endpoints in a landmark to be   the set of the allowed execution intervals.
Since  must be scheduled somewhere, . Since at any time , at most one job is being processed, 
. Our LP, which we now state, contains an additional constraint (\ref{lp1-non-preempt}) capturing non-premption:
if some job  is scheduled  during some interval  or a subinterval thereof, then no other job can be scheduled during an interval that contains . This holds for non-preemptive schedules but not necessarily for preemptive schedules, and in that sense this new constraint ``captures" non-preemption. The constraint is necessary to bound the integrality gap: without it, there exist instances and  fractional solutions that have much lower value than their integer counterpart (Lemma~\ref{lem:gap_lp1}, proved in below using  the instance described Figure~\ref{fig:exmpl_LP}.)   




\begin{lem}\label{lem:gap_lp1}
Without constraint~(\ref{lp1-non-preempt}), \textbf{LP1} has integrality gap at least .
\end{lem}
\begin{proof}
We construct an instance on which the integrality gap is at least .
Let us define  jobs. We create  \emph{small} jobs and a \emph{big} job.
The  small job has release date  and deadline  and processing requirement 1.
The big job has release date 0 and deadline  and a processing requirement of .
More details about this instance are given figure \ref{fig:exmpl_LP}.

An integral solution will have to process the big job between two consecutive small jobs and thus,
will have a cost of at least .

Now, consider the following assignment of the variables. For any job , let  and  be respectively the
first and second halves of its life interval.
We set  and the other  to 0. Hence constraint \ref{lp1-job-assigned} is satisfied.
Consider now a time  which is not a release date or a deadline. At this time, we can process a small job and the big job.
For both of them, we are either on their first or second halves and so, the sum of the  such that  contains  is
at most 1. It follows that constraint \ref{lp1-no-overlap} is also satisfied and that the assignment described is a solution for
the linear program without constraint \ref{lp1-non-preempt}.

By doing this, each job costs , hence the optimal fractional solution has value at most .
Therefore, the integrality gap is at least .
\qed\end{proof}

\begin{figure}
   \begin{center}
        \includegraphics[scale=0.9]{IG_LP_wo_cst3.pdf}
   \end{center}
   \caption{An instance on which LP1 has an integrality gap of at least .
   The instance contains a \emph{big} job and  \emph{small} jobs. The figure shows the life intervals of the jobs
   and, above them, their work volume.}
   \label{fig:exmpl_LP}
\end{figure}


The remainder of this section is devoted to proving Theorem~\ref{thm:intgap}.

\subsection{Overview}

We show that any fractional solution can be transformed into an integral solution without increasing the value of the solution by too much, in three steps. 

We first divide the time into zones and transform the fractional solution so all (non-zero) fractional execution intervals are inside a zone. Then, each zone is divided into nested subzones and we further transform our fractional solution so that all fractional execution intervals are inside a subzone and the life interval of the corresponding job contains that subzone. Finally, we build a weighted bipartite graph from the transformed fractional solution whose edges represent the possible allocation of execution intervals to subzones. Similarly to~\cite{Shmoys_Tardos}, we find an integral (weighted) matching in this graph and translate this solution to an integral schedule.

We then show that the cost of the integral solution we built is at most  times the cost of the original fractional solution.



\subsection{Building an integral solution from a fractional solution}

We now give the detailed description for our procedure to transform a fractional in three steps. The first step is derived from Antoniadis and Huang's algorithm \cite{Antoniadis_Huang}. 

\subsubsection{Splitting execution intervals on deadlines}

Our first transformation turns a fractional solution into a fractional solution where  is 0 for any execution interval  that any points in a set of deadlines we pick. The deadlines we pick are the deadlines of a good independent set.

\begin{lem}\label{lem:deadline_zones}
  Let \textbf{LP1} be the LP obtained from an instance of \textsc{Non preemptive minimum energy scheduling ()} and  be any good independent set for this instance.

  In polynomial time, we can transform any fractional solution  to \textbf{LP1} to a fractional solution  of value at most  where  if  crosses a deadlines of .
\end{lem}

To prove this lemma, we will simply ``shift'' some of the values of .
\begin{defn}
  By \emph{shifting}  to  for a fractional solution  to \textbf{LP1}, we mean to increase  by  and decrease  to 0.
\end{defn}

\begin{proof}
  First note that  if  crosses \emph{two} deadlines of . Indeed, otherwise,  contains the execution interval of some job  of the independent set and  doesn't satisfy constraint (\ref{lp1-no-overlap}).

  We build  iteratively, starting with .

  For each  and  crosses a deadline  in  (so ),  the larger of the two intervals ,  has size at least half the size of . We shift  to . 

  This process shrink the size of each execution interval in the fractional solution by a factor of at most 2 and thus  is at most .

  Constraint (\ref{lp1-job-assigned}) is still satisfied as shifting  from  to  preserves the sum of probabilities over  (and does not affect the constraint for any other job ).
  Constraint (\ref{lp1-no-overlap}) is still satisfied as  is contained in  (so fewer (fractional) execution intervals cross each deadline in ).
  Again, since  is contained in , we see that constraint (\ref{lp1-non-preempt}) is satisfied since an inequality in constraint (\ref{lp1-non-preempt}) containing the term  contains the term .
\qed\end{proof}




\subsubsection{Further splits}

We now proceed to our second transformation and show that we can further split the execution intervals of a fractional solution. Now that all execution intervals
(in the support of ) lie between two consecutive deadlines (which we now call a ``zone''), we can further partition each zone so the first half
is dedicated to jobs whose life interval ends in that zone and the second half is dedicated to the others (namely, jobs whose life interval starts
in that zone or jobs whose life interval contains the zone).


\begin{lem}\label{lemma:subzones}
  Let \textbf{LP1} be obtained from an instance of \textsc{Non preemptive minimum energy scheduling ()} and  be any good independent set for this instance.

  In polynomial time, we can transform any fractional solution  where  if  crosses a deadlines of  to \textbf{LP1} to a fractional solution  of value at most  times the value of   where  implies
  \begin{enumerate}
  \item
     for some consecutive deadlines   of  and , or
  \item
     for some consecutive deadlines   of  and .
  \end{enumerate}
\end{lem}

We let  consists of all intervals of the form  and  
for consecutive deadlines  of .

Though they do not partition the timeline, we still refer to  as \emph{subzones}. We now prove the following refinement of the above lemma where we make  explicit.

\begin{proof}
  First set  to be .

  Let  be a zone defined by two consecutive deadlines. No life interval  is contained in  (or  is not a good independent set as we could add  and still obtain an independent set). Thus, we can partition jobs whose life interval intersects  into jobs  whose life interval starts (at or) before  and jobs  whose life interval end (at or) after  (putting jobs that can go into both into either set).

  If  and  intersects  then  is a subset of  which also intersects  so  is in  or .

  We now describe how to shift  with  in  (the shift for jobs in  is symmetric). We simulaneously shift all  with . Since  does not cross  or , it ends in  and  is in  for some . We shift  to  where .

  Shifts for  with  in  are defined symmetrically (by reversing the timeline).

  Since each execution interval of  is shifted to an execution interval exactly half its original size, by Lemma \ref{lemma:half},  is at most .
  Since we only shifted intervals, constraint (\ref{lp1-job-assigned}) remains satisfied. Constraint (\ref{lp1-no-overlap}) is satisfied as we simply compressed the entire region  to  for jobs in  and the entire region  to  for jobs in . Finally, constraint (\ref{lp1-non-preempt}) is satisfied by  since again these constraint were satisfied by  and we only compressed some block of execution intervals into disjoint regions of the timeline.
\qed\end{proof}

\subsubsection{Building a weighted bipartite matching}

As a result of Lemma \ref{lemma:subzones}, for each ,  is contained in some subzone  and furthermore, the life interval of  contains  so we can freely shift  to another interval (of the same length) inside .
Thus, we will only remember the length of the fractional execution intervals and the subzone  in which they belong. I.e., we think of  as a fractional assignment of lengths  for each job to .

\begin{lem}\label{lemma:earliest_deadline_first}
  If for each subzone , the lengths  assigned to  and all subzones included in  is at most  then there is a feasible schedule where each job is given their assigned length in .
\end{lem}
\begin{proof}
We greedily assign  an interval of length  to the leftmost possible empty spot if  is of the form . In fact, at each intermediate step, all jobs currently assigned to  take up a contiguous region starting from the left endpoint  of . If  is of the form , we assign  the rightmost possible interval of length .
We see by induction on the number of subzones contained in  that every job is processed in .
\end{proof}



We now desire an integral assignment of lengths to each  where the total of all lengths assigned to  does not exceed . Note that this constraint is satisfied by the fractional solution derived from  (as  satisfies constraint (\ref{lp1-no-overlap})).

To obtain such an integral assignment from our fractional assignment derived from , we build a weighted bipartite graph  where an assignments correspond to matchings and the weight of a matching correspond to the energy cost (of the matching interpreted as a schedule). We will then obtain an integral matching from the derived fractional matching (whose weight is exactly .

We now describe  with bipartition  and weight  for each edge . We also keep a \emph{length}  for each edges which will be used in the very last step of our proof (but in no way affects the weighted bipartite matching we look for).

\begin{itemize}
\item
   contains one vertex for each job. I.e., 
\item
   consists of vertices for subzones. However,  may contain more than one vertex for each subzone . In fact, it contains the ceiling of the sum of fractional value of all lengths assigned to . I.e.,
  


\item
  The edges are constructed as followed. Start with all edges  for all  if  for some . We now delete some edges to obtain the edges of  and assign weights and length of the remaining edges.

  Sort the lengths assigned to  by  in decreasing order of length. For each such length  for job  of fractional value , set  to  where  is the ceiling of the partial sum of all jobs previously considered for  (i.e., ). Set  to . Also set  to  and  to  if adding  to the ceiling of the partial sum increases it by 1. Delete all other edges of the form .
\end{itemize}

 naturally gives the following fractional matching  of  with total weight : we pick each edge with weight exactly  (or  split into two as follows if adding  increased the ceiling of the partial sum by 1. Whatever we need to add to the partial sum to make it an integer is the fraction we choose of the first edge, and the rest of  for the second edge).

To complete the description of our final transformation, we apply the following two technical lemmas.


\begin{lem}\label{lemma:integer-matching}\cite{Lovasz_Plummer_matching,Shmoys_Tardos}
  In a weighted bipartite graph, there exists an (integral) matching of same weight as any fractional matching\footnote{in a fractional matching edges can be selected with a fractional value as long as the total value of edges incident to any vertex is at most 1}.
\end{lem}

\begin{lem}\label{lemma:matching-to-schedule}
  Let  be the bipartite graph built from a transformed fractional solution .
  For any matching  saturating  of , we can obtain a schedule whose energy consumption is at most  times the weight of . \end{lem}
\begin{proof}

\textbf{Schedule construction}.
Since  has weight  and saturates all of , by Lemma \ref{lemma:integer-matching} there exists an integer matching  of the same weight that saturates .

We build a schedule  from  as follows. Consider subzones in order of containment starting with subzones containing no other subzones. For each edge  give  an execution interval of length  in . By Lemma \ref{lemma:matching-to-schedule}, we only need to verify that for all subzones , the sum of lengths for all subzones contained in  does not exceed .

\textbf{Feasibility}.
We now check that for each subzone , the sum of lengths for all subzones contained in  does not exceed .

\begin{figure}[ht!]
   \begin{center}
        \includegraphics{lp-matching-fig.pdf}
   \end{center}
   \caption{An illustration of the feasibility proof. The fractional solution is represented in green with  corresponding to the height of a rectangle. (b) Recall that  is built by first ordering the execution intervals by length. (c) In the worst case, the matching picked the longest edge incident to each vertex.}
   \label{fig:LP_matching}
\end{figure}

Let  be the sum of all lengths assigned to  by the fractional solution, weighted by  (i.e., ). Since  satisfies constraint (\ref{lp1-no-overlap}), we have

for all .



Let  be the minimum length of all edges incident to  and  be the maximum length of all edges incident to . Since we considered lengths in non-increasing order of their  values,  for all  and .


Let  is the number of copies of vertex  and  be the sum of lengths of all edges of  incident to .
For all ,

and therefore the sum of lengths assigned to  in our schedule is at most .


Now for each , the sum of all lengths assigned to  and all subzones included in  in our schedule is at most

and so our assigned lengths are feasible.
\qed\end{proof}

\begin{proof}(of Theorem \ref{thm:intgap})
  To prove the integrality gap of \textbf{LP1}, we simply need to apply each lemma in this section in turn.

  Given an optimal fractional solution  to \textbf{LP1}, find a good independant set  (to the instance which generate the LP) and apply Lemma \ref{lem:deadline_zones} to  and  to obtain a fractional solution  of value at most  where no execution interval crosses a deadline in .

  Then apply Lemma \ref{lemma:subzones} to  to obtain  of value at most  where for any non-zero ,  is contained in a ``subzone'' of the form  or  for some consecutive deadlines  and  and furthermore  contains this subzone.

  Now build  and interpret  as a fractional matching in . By Lemma \ref{lemma:integer-matching},  has a matching  of same weight as this fractional matching and by Lemma \ref{lemma:matching-to-schedule}, we can build a schedule from  whose energy consumption is at most .

  Such a schedule is of course a solution to \textbf{LP1} of same value, thus completing the proof.
\qed\end{proof}









\subsubsection{Algorithm summary}

We can summarize our algorithm from transforming any fractional solution  to an integral solution.

\begin{alg}\label{alg:singleproc}
~
\begin{enumerate}
\item Apply the transformation of Lemma \ref{lem:deadline_zones} and then Lemma \ref{lemma:subzones} to the fractional solution to obtain a new fractional solution .
\item Construct the weighted bipartite graph .
\item Find a minimum weight matching  that matches every node in .
\item For each edge , schedule job  in the subzone  with an interval of length . Use an earliest deadline first schedule for all jobs in  if  is in the first half of a zone and an latest release date first schedule if  is in the second half of a zone.
\end{enumerate}

\end{alg}
