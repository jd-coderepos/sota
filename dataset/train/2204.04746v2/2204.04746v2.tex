

1. A critical study on surgical action triplet recognition in form a deep learning challenge.

2. A summary of deep learning methods presented at the challenge.

3. An in-depth methodological comparative analysis, providing a comprehensive overview of possible strategies for tackling the surgical action triplet problem.

4. Simple but effective ensemble algorithms setting a new baseline with an improvement of +4.3\% AP to the challenge methods for the surgical action triplet recognition.

5. An in-depth quantitative and qualitative result analysis considering multiple metrics to cover all aspects of the triplet recognition problem.



\onecolumn
\section*{Highlights}


\begin{enumerate}
    \item A comprehensive study on surgical action triplet recognition directly from videos.
    \item A new attention mechanism that is guided by weakly-supervised class activation maps for more precise detection.
    \item A Transformer-inspired neural network that utilizes self- and cross-attention {\red mechanisms} for action triplet recognition.
    \item Introduction of CholecT50, a large endoscopic dataset for action triplet recognition.
    \item Extensive comparison and ablation study showing state-of-the-art results and discussion of clinical relevance.
\end{enumerate}

\thispagestyle{empty}


\newpage
\section*{Suggested Editors}
\begin{enumerate}
    \item \textbf{Kensaku Mori}\\ Nagoya University Graduate School of Engineering Department of Chrystalline Materials Science, Nagoya, Japan
    \item \textbf{Jocelyne Troccaz}\\Techniques for Biomedical Engineering and Complexity Management Informatics Mathematics and Applications Grenoble Laboratory, La Tronche cedex, France
    \item \textbf{Tom Vercauteren} \\ King's College London, London, United Kingdom
    
    
\end{enumerate}



\newpage
\section*{Suggested Reviewers}
\begin{enumerate}
    
    
\item 
\textbf{Name}: Prof. Raphael Sznitman \\
\textbf{Institution}: ARTORG Center, University of Bern\\
\textbf{Country}: Switzerland\\
\textbf{E-Mail}:  raphael.sznitman@artorg.unibe.ch\\
\textbf{Reason for the choice}: The professor leds a research in computational vision, probabilistic methods and statistical learning, applied to applications in Ophthalmology and biomedical imaging. He has multiple publications that use deep learning method for the modeling of surgical activities. Thus, he will provide an in-depth review of the submitted work.



\item 
\textbf{Name}: Dr. Max Allan \\
\textbf{Institution}: Intuitive Surgical, Inc\\
\textbf{Country}: United State of America\\
\textbf{E-Mail}: max.allan@intusurg.com\\
\textbf{Reason for the choice}: Max Allan is a researcher with both academic and industrial experience. He work on applying Computer Vision and Machine Learning to the da Vinci surgical robot and has multiple publications on surgical tool detection and activity recognition in Minimally Invasive Surgery.



\item 
\textbf{Name}: Darko Katic\\
\textbf{Institution}: Karlsruhe Institute of Technology\\
\textbf{Country}: Germany\\
\textbf{E-Mail}: darko.katic@kit.edu\\
\textbf{Reason for the choice}: He is a pioneer researcher on fine-grained surgical action recognition. In fact, his paper is one of the first that describe surgical actions as triplets of instrument, verb, and target. He is a prominent researcher in the computer-assisted intervention. Very easy to find on Google scholar on the topic in the manuscipt.

\end{enumerate}




