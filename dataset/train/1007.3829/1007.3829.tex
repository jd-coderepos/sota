\documentclass{tlp}
\pdfoutput=1

\usepackage{url,alltt,amsmath,epsfig,epsf,amssymb,stmaryrd}

\bibliographystyle{acmtrans}

\usepackage{fancyvrb}

\SaveVerb{SVNId}||


\usepackage[utf8]{inputenc}
\usepackage{enumerate}
\usepackage[all]{xy}
\usepackage{todo}
\usepackage{rotating}


\newtheorem{theorem}{Theorem}

\newtheorem{example}{Example}[section]
\newtheorem{definition}{Definition}[section]
\newtheorem{remark}{Remark}[section]

\newcommand{\A}{\ensuremath{\mathcal{A}}}
\newcommand{\B}{\ensuremath{\mathbb{B}}}
\newcommand{\C}{\ensuremath{\mathbb{C}}}
\newcommand{\G}{\ensuremath{\mathbb{G}}}
\newcommand{\bL}{\ensuremath{\mathbb{L}}}
\newcommand{\mcP}{\ensuremath{\mathcal{P}}}
\renewcommand{\S}{\ensuremath{\mathbb{S}}}
\newcommand{\T}{\ensuremath{\mathbb{T}}}
\newcommand{\bP}{\ensuremath{\mathbb{P}}}
\newcommand{\V}{\ensuremath{\mathbb{V}}}
\newcommand{\CT}{\ensuremath{\mathcal{CT}}}

\newcommand{\mcG}{\ensuremath{\mathcal{G}}}
\newcommand{\mcS}{\ensuremath{\mathcal{S}}}
\newcommand{\bbB}{\ensuremath{\mathbb{B}}}
\newcommand{\bbC}{\ensuremath{\mathbb{C}}}
\newcommand{\bbG}{\ensuremath{\mathbb{G}}}
\newcommand{\bbT}{\ensuremath{\mathbb{T}}}
\newcommand{\bbV}{\ensuremath{\mathbb{V}}}

\newcommand{\subxt}{\left[X/t\right] }
\newcommand{\subXt}{\left[X/t\right] }
\newcommand{\psubxt}{\left[X\wr t\right] }
\newcommand{\subxy}{\left[X/Y\right] }
\newcommand{\Xet}{X\doteq t }
\newcommand{\xey}{X\doteq Y }
\newcommand{\bs}{{\bar s}}
\newcommand{\by}{{\bar y}}
\newcommand{\project}{\ensuremath{\pi}}
\newcommand{\bang}{\ensuremath{!}}

\DeclareMathOperator{\chr}{chr}
\DeclareMathOperator{\id}{id}
\DeclareMathOperator{\vars}{vars}

\def\tuple#1{\langle #1 \rangle}
\newcommand{\stesq}[3]{\ensuremath{\tuple{#1; #2; #3}}}
\newcommand{\stbang}[4]{\ensuremath{\tuple{#1; #2; #3; #4}}}
\newcommand{\tstate}[6]{\ensuremath{\tuple{#1; #2; #3; #4}_{#5}^{#6}}}
\newcommand{\pstate}[6]{\ensuremath{\tuple{#1; #2; #3; #4}_{#5}^{#6}}}

\newcommand{\obang}{\ensuremath{{\omega_\bang}}}
\newcommand{\oesq}{\ensuremath{{\omega_e}}}

\newcommand{\Sbang}{\ensuremath{{\Sigma_{\bang}}}}
\newcommand{\Sesq}{\ensuremath{{\Sigma_e}}}
\newcommand{\sbang}{\ensuremath{\sigma_{\bang}}}
\newcommand{\sesq}{\ensuremath{\sigma_e}}
\newcommand{\tbang}{\ensuremath{\tau_{\bang}}}
\newcommand{\tesq}{\ensuremath{\tau_e}}

\newcommand{\ebang}{\ensuremath{\equiv_{\bang}}}
\newcommand{\eesq}{\ensuremath{\equiv_e}}

\newcommand{\der}{\ensuremath{\rightarrowtail}}
\newcommand{\dert}{\ensuremath{\der_t}}
\newcommand{\derr}{\ensuremath{\der_r}}
\newcommand{\derp}{\ensuremath{\der_p}}
\newcommand{\derbang}{\ensuremath{\der_{\bang}}}
\newcommand{\deresq}{\ensuremath{\der_e}}


\newcommand{\sinc}{\ensuremath{\sqsubseteq}}

\newcommand{\applyL}{\textrm{\textbf{ApplyLinear}}}
\newcommand{\applyP}{\textrm{\textbf{ApplyPersistent}}}

\newcommand{\ok}{\ensuremath{\quad(\checkmark)\quad}}

\begin{document}
\title{A Complete and Terminating Execution Model for Constraint Handling Rules}

\author[Hariolf Betz, Frank Raiser and Thom Fr{\"u}hwirth]{HARIOLF BETZ,
FRANK RAISER \and THOM FR{\"U}HWIRTH\\
Faculty of Engineering and Computer Science, Ulm University, Germany\\
\email{firstname.lastname@uni-ulm.de}}

\maketitle

\textbf{Note: } This article has been published in \textit{Theory and Practice
of Logic Programming}, 10(4--6), 597--610, \copyright Cambridge University Press

\begin{abstract}
We observe that the various formulations of the operational semantics of
Constraint Handling Rules proposed over the years fall into a spectrum ranging
from the analytical to the pragmatic.
While existing analytical formulations facilitate program
analysis and formal proofs of program properties, they cannot be implemented as is.
We propose a novel operational semantics~, which has a strong
analytical foundation, while featuring a terminating execution model.
We prove its soundness and completeness with respect to existing analytical
formulations and we provide an implementation in the form of a source-to-source
transformation to CHR with rule priorities.
\end{abstract}

\begin{keywords}
Constraint Handling Rules, Operational Semantics, Execution Model, Persistent
Constraints
\end{keywords}

\section{Introduction}
\label{sec:intro}
Constraint Handling Rules \cite{fruehwirth09} (CHR) is a declarative, multiset-
and rule-based programming language suitable for concurrent execution and
powerful program analysis. While it is known as a language that combines
efficiency with declarativity, publications in the field display a tendency to
favor one of these aspects over the other. We observe a spectrum of research
directions ranging from the \emph{analytical} to the \emph{pragmatic}.

On the analytical end of the spectrum, emphasis is put on CHR as a mathematical
formalism, declarativity, and the understanding of its logical foundations and
theoretical properties. Several formalizations of the operational semantics,
found in \cite{Fruhwirth1993,fruehwirth98} and \cite{fruehwirthabdennadher03},
belong to this side of the spectrum. Notable results building on these analytical
formalizations include decidable criteria for operational equivalence
\cite{abdennadherfruehwirth99} and confluence
\cite{abdennadherfruehwirthmeuss99}, a strong foundation of CHR in linear
logic~\cite{betzfruehwirth05}, as well as weak and strong parallelization, as
presented in \cite{fruehwirth05} and further developed toward concurrency in
\cite{Sulzmann2007,Sulzmann2008}.

A recent analytical formalization is the operational semantics~, given
in \cite{Raiser2009a}. It consists in a rewriting system of equivalence classes
of states based on an axiomatic formulation of equivalence. It has been shown to
coincide with the operational semantics~, which has been introduced
in \cite{fruehwirth09} to set a standard for all other operational semantics to
build upon.

On the downside, these operational semantics are detached from practical
implementation in that they are oblivious to questions of efficiency and
termination. Particularly, the class of rules called \emph{propagation rules}
causes trivial non-termination in both of them. Hence, it is safe to say that
the existing analytical formalizations of the operational semantics lack a 
terminating execution model.

This contrasts with most work on the pragmatic side of the spectrum, which
emphasizes practical implementation and efficiency over formal reasoning. It
originates with \cite{abdennadher97}, where a token-based approach is proposed in
order to avoid trivial non-termination: Every propagation rule is applicable only
once to a specific combination of constraints.  This is realized by keeping a
\emph{propagation history} -- sometimes called \emph{token store} -- in the CHR
state. Thus, we gain a terminating execution model for the full segment of CHR.

Building upon \cite{abdennadher97}, a plethora of operational semantics has been
brought forth, such as the token-based operational semantics~ and its
refinement  \cite{Duck2004}. The latter reduces non-determinism for a
gain in efficiency and sets the current standard for CHR implementations. Another
notable exponent is the priority-based operational semantics~
\cite{dekoninckschrijversdemoen07}.

On the downside, token stores break with declarativity: Two states that differ
only in their token stores may exhibit different operational behavior while
sharing the same logical reading. Therefore, we consider token stores as
\emph{non-declarative elements} in CHR states.

Recent work on linear logical algorithms \cite{Simmons2008} and the close
relation of CHR to linear logic \cite{betzfruehwirth05} suggest a novel approach
that emphasizes aspects from both sides of the spectrum to a useful degree:
In this work, we introduce the notion of \emph{persistent constraints} to CHR, a
concept reminiscent of unrestricted or ``banged'' propositions in linear logic.
Persistent constraints provide a finite representation of the result of any
number of propagation rule firings.

We furthermore introduce a state transition system based on persistent
constraints, which is explicitly irreflexive. In combination, the two ideas
solve the problem of trivial non-termination while retaining declarativity and
preserving the potential for effective concurrent execution. This state
transition system requires no more than two rules. As every transition step
corresponds to a CHR rule application, it facilitates formal reasoning over
programs.

In this work, we show that the resulting operational semantics~ is sound
and complete with respect to . We show that  can be faithfully
embedded into the operational semantics~, thus effectively providing an
implementation in the form of a source-to-source transformation. All operational
semantics developed with an emphasis on pragmatic aspects lack this completeness
property. Therefore, this work is the first to show that it is possible to
implement CHR soundly and completely with respect to its abstract foundations,
whilst featuring a terminating execution model.

\begin{example}\label{ex:trans}
Consider the following straightforward CHR program for computing the transitive
hull of a graph represented by edge constraints~: 
This most intuitive formulation of a transitive hull is not a suitable
implementation in most existing operational semantics. In fact, for goals
containing cyclic graphs it is non-terminating in all aforementioned existing
semantics. In this work we show that execution in our proposed
semantics~ correctly computes the transitive hull whilst guaranteeing
termination.
\end{example}

The remainder of this paper is structured as follows: We state the syntax of CHR
and summarize the existing operational semantics  and  in
Sect.~\ref{sec:opsems}. In Sect.~\ref{sec:ouromega}, we present our semantics
, originally proposed in \cite{Betz2009}, and we state results concerning
its soundness and completeness with respect to . In
Sect.~\ref{sec:implementation}, we show how  can be implemented by means
of a faithful source-to-source transformation into . In
Sect.~\ref{sec:discussion}, we discuss the termination behavior of  as
well as related work, before we conclude in Sect.~\ref{sec:conclusion}.
Proofs of the theorems presented in this work can be found in the accompanying
technical report \cite{Betz2010} \footnote{\cite{Betz2010} is available from
\url{http://vts.uni-ulm.de/doc.asp?id=7193}}, and will be omitted here.

\section{Preliminaries}
\label{sec:opsems}

We first introduce the syntax of CHR and the equivalence-based operational
semantics~, which offers a foundation for all other semantics, although it
lacks a terminating execution model. We furthermore present its refinements
 and .

\subsection{The Syntax of CHR}

Constraint Handling Rules distinguishes two kinds of constraints:
\emph{user-defined constraints} (or \emph{CHR constraints}) and \emph{built-in
constraints}. Reasoning on built-in constraints is possible through a
satisfaction-complete and decidable constraint theory~\CT.

CHR is a programming language that offers advanced rule-based multiset rewriting.
Its eponymous rules are of the form  where  and  are multisets of user-defined constraints,
called the \emph{kept head} and \emph{removed head}, respectively. The
\emph{guard}~ is a conjunction of built-in constraints and the
 \emph{body} consists of a conjunction of built-in constraints~ and a
 multiset of
user-defined constraints~. The \emph{rule name}  is optional and may be
omitted along with the  symbol. Note that throughout this paper, we omit the
curly braces around sets and multisets where there is no ambivalence. This
applies especially to CHR rules and states.

In this work, we put special emphasis on the class of rules where , called \emph{propagation rules}. Propagation rules can be written
alternatively as 

A \emph{variant} of a rule~ with variables~ is a rule of the form  for any sequence of pairwise
distinct variables~. For any rule~, the \emph{local variables}~ are
defined as . A rule where
 is called \emph{range-restricted}.

A CHR program~ is a set of rules. A \emph{range-restricted} CHR program is
a set of range-restricted rules.

\subsection{Equivalence-based Operational Semantics }

In this section, we recall the \emph{equivalence-based operational
semantics~\oesq} \cite{Raiser2009a}. It is operationally close to the very
abstract semantics , but we prefer it for its concise formulation
and the explicit distinction of global variables, user-defined, and built-in
constraints.

\begin{definition}[ State]\label{def:m_state}
An {\em \oesq\ state} is a tuple~. The \emph{user-defined
(constraint) store}~ is a multiset of CHR constraints. The {\em built-in
(constraint) store}~ is a conjunction of built-in constraints.  is a set
of variables called the \emph{global variables}. We use~ to denote the set
of all  states. A variable~ is called a \emph{strictly local
variable} iff .
\end{definition}

The operational semantics~\oesq\ is founded on equivalence classes of
states, based on the following definition of state equivalence.

\begin{definition}[ State Equivalence]
\label{def:m:equiv}
Equivalence between \oesq~states is the smallest equivalence relation~
over \oesq~states that satisfies the following conditions:

\begin{enumerate}
\item \label{cond:m:subst}

\item \label{cond:m:ct} If
 where  are the strictly local variables of , respectively, then 
\item \label{cond:m:global} 
If  is a variable that does not occur in  or  then

\item \label{cond:m:fail}

\end{enumerate}
\end{definition}

\begin{definition}[\oesq~Transitions]
	\label{def:opsem_classes}
For a CHR program~, the state transition system~
is defined as follows. The transition is based on a variant of a rule~ in
 such that its local variables are disjoint from the variables occurring in
the pre-transition state.

\begin{center}
\begin{tabular*}{9.5cm}{c}
 \ \sigma = \stesq{e(A,B),e(B,A)}{\top}{\emptyset}

\begin{array}{ll}
\sigma \eesq & \stesq{e(A',B'), e(B',C')}{A=A' \land B=B' \land
A=C'}{\emptyset}\\
\der^t & \stesq{e(A',B'), e(B',C'), e(A',C')}{A=A' \land B=B'
\land A=C'}{\emptyset}\\
\eesq & \stesq{e(A,B), e(B,A), e(A,A)}{\top}{\emptyset}
\end{array}
p :: r\ @\
  H_1' \setminus H_2' \Leftrightarrow G \mid B
	\begin{array}{lclcl}
	\text{r1} & @ & a & \Longrightarrow & b\\
	\text{r2} & @ & b & \Leftrightarrow & c
	\end{array}
  
	\stbang{\bL}{\bP}{\Xet\wedge\B}{\V} \ebang
	\stbang{\bL\subXt}{\bP\subXt}{\Xet\wedge\B}{\V}

	\stbang{\bL}{\bP}{\B}{\V} \ebang \stbang{\bL}{\bP}{\B'}{\V}

	\stbang{\bL}{\bP}{\B}{\{X\}\cup\V} \ebang \stbang{\bL}{\bP}{\B}{\V}

	\stbang{\bL}{\bP}{\bot}{\V} \ebang \stbang{\bL'}{\bP'}{\bot}{\V'}

	\stbang{\bL}{P\uplus P\uplus\bP}{\B}{\V} \ebang
	\stbang{\bL}{P\uplus\bP}{\B}{\V}
-.05cm]
\hline
\\

\end{tabular*}
\end{center}
\textbf{ApplyPersistent:}
\begin{center}
\begin{tabular*}{8.25cm}{c}
\
\begin{array}{ll}
\sigma \ebang & \stbang{e(A',B'), e(B',C')}{\emptyset}{A=A' \land B=B'
\land A=C'}{\emptyset}\\
\der^t & \stbang{e(A',B'), e(B',C')}{e(A',C')}{A=A' \land B=B'
\land A=C'}{\emptyset}\\
\ebang & \stbang{e(A,B), e(B,A)}{e(A,A)}{\top}{\emptyset} = \sigma'
\end{array}

\begin{array}{ll}
\sigma' \ebang & \stbang{e(A',B'), e(B',C')}{e(A,A)}{A=A' \land B=B'
\land A=C'}{\emptyset}\\
\not \der^t & \stbang{e(A',B'), e(B',C')}{e(A,A), e(A',C')}{A=A'
\land B=B' \land A=C'}{\emptyset}\\
\ebang & \stbang{e(A,B), e(B,A)}{e(A,A)}{\top}{\emptyset} = \sigma'
\end{array}

  	3:: l(H_1^l)\uplus p(H_1^p)\uplus p(H_2^p)\setminus l(H_2^l) \Leftrightarrow
  	G\mid l(B_c), B_b
  
  	3:: l(H_1^l)\uplus p(H_1^p)\uplus p(H_2) \Longrightarrow G\mid c(B_c),B_b
  
	3:: \{c(p,\bar t)\}\uplus H_1\setminus H_2 \Leftrightarrow \bar t={\bar
	t}'\wedge G\mid B
  
    \begin{array}{l}
	  1:: c(p, \bar t) \backslash c(c, \bar t) \Leftrightarrow \top\\
	  2:: c(c, \bar t) \Leftrightarrow c(p, \bar t)
	\end{array}
  
\begin{array}{lclcl}
t & @ & e(X,Y), e(Y,Z) & \Longrightarrow & e(X,Z)
\end{array}

\begin{array}{ccrcl}
3 & :: & e(l,X,Y), e(l,Y,Z) & \Longrightarrow & e(c,X,Z) \\
3 & :: & e(l,X,Y), e(p,Y,Z) & \Longrightarrow & e(c,X,Z) \\
3 & :: & e(p,X,Y), e(l,Y,Z) & \Longrightarrow & e(c,X,Z) \\
3 & :: & e(p,X,Y), e(p,Y,Z) & \Longrightarrow & e(c,X,Z) \\
\\
3 & :: & e(p,X,Y) & \Longrightarrow & X=Y \wedge Y=Z \mid e(c,X,Z) \\
\\
1 & :: & e(p,X,Y) \backslash e(c,X,Y) & \Leftrightarrow & \top\\
2 & :: & e(c,X,Y) & \Leftrightarrow & e(p,X,Y)
\end{array}

 \stbang{G}{\emptyset}{B}{\bbV} \derbang^*
  	\stbang{\bL}{\bP}{\bbB}{\bbV} \not \derbang \textrm{ in }
  
 \exists \bbT,n. \tstate{l(G), B}{\emptyset}{\top}{\emptyset}{0}{\bbV} \derp^*
 \tstate{\emptyset}{l(\bL)\uplus p(\bP)}{\bbB}{\bbT}{n}{\bbV} \not \derp \textrm{ in }
   
\begin{array}{ll}
& 	\tstate{e(l,A,B),e(l,B,A)}{\emptyset}{\top}{\emptyset}{0}{\{A,B\}}\\
\derp^* &
	\tstate{\emptyset}{e(l,A,B)\#0,e(l,B,A)\#1}{\top}{\emptyset}{2}{\{A,B\}}\\
\derp^{*} &
	\tstate{\emptyset}{e(l,A,B)\#0,e(l,B,A)\#1,e(c,A,A)\#2}{\top}{\ldots}{3}{\{A,B\}}\\
\derp^{*} &
	\tstate{\emptyset}{e(l,A,B)\#0,e(l,B,A)\#1,e(p,A,A)\#3}{\top}{\ldots)}{4}{\{A,B\}}\\
\derp^{*} &
	\tstate{\emptyset}{e(l,A,B)\#0,e(l,B,A)\#1,e(p,A,A)\#3,e(c,B,B)\#4}{\top}{\ldots}{5}{\{A,B\}}\\
\derp^{*} &
	\tstate{\emptyset}{e(l,A,B)\#0,e(l,B,A)\#1,e(p,A,A)\#3,e(p,B,B)\#5}{\top}{\ldots)}{6}{\{A,B\}}\\
\derp^{*} &
	\tstate{\emptyset}{e(l,A,B)\#0,e(l,B,A)\#1,e(p,A,A)\#3,e(p,B,B)\#5,e(c,A,B)\#6}{\top}{\ldots}{7}{\{A,B\}}\\
\derp^{*} &
	\tstate{\emptyset}{e(l,A,B)\#0,e(l,B,A)\#1,e(p,A,A)\#3,e(p,B,B)\#5,e(p,A,B)\#7}{\top}{\ldots)}{8}{\{A,B\}}\\
\derp^{*} &
	\tstate{\emptyset}{e(l,A,B)\#0,e(l,B,A)\#1,e(p,A,A)\#3,e(p,B,B)\#5,e(p,A,B)\#7,e(c,B,A)\#8}{\top}{\ldots}{9}{\{A,B\}}\\
\derp^{*} &
	\tstate{\emptyset}{e(l,A,B)\#0,e(l,B,A)\#1,e(p,A,A)\#3,e(p,B,B)\#5,e(p,A,B)\#7,e(p,B,A)\#9}{\top}{\ldots)}{10}{\{A,B\}}\\
\derp^{*} &
	\tstate{\emptyset}{e(l,A,B)\#0,e(l,B,A)\#1,e(p,A,A)\#3,e(p,B,B)\#5,e(p,A,B)\#7,e(p,B,A)\#9}{\top}{\ldots)}{24}{\{A,B\}}\\
\not\derp
\end{array}

\begin{array}{ll}
\sigma \ebang & \stbang{e(A,B), e(B,A)}{\emptyset}{\top}{\{A,B\}}\\
\derbang^t & \stbang{e(A,B),e(B,A)}{e(A,A)}{\top}{\{A,B\}}\\
\derbang^t & \stbang{e(A,B),e(B,A)}{e(A,A),e(B,B)}{\top}{\{A,B\}}\\
\derbang^t & \stbang{e(A,B),e(B,A)}{e(A,A),e(B,B),e(A,B)}{\top}{\{A,B\}}\\
\derbang^t &
\stbang{e(A,B),e(B,A)}{e(A,A),e(B,B),e(A,B),e(B,A)}{\top}{\{A,B\}}\\
\not \derbang
\end{array}
 
This example also demonstrates how  streamlines execution which in turn
facilitates formal reasoning over derivations: the whole computation consists of
4 state transitions in , whereas the corresponding computation in
 requires 60 state transitions.
\end{example}

The presented source-to-source transformation satisfies conditions for an
\emph{acceptable encoding} according to \cite{Gabbrielli2009}, modulo the
necessary distinction between linear and persistent constraints in the
translation.

\section{Discussion}
\label{sec:discussion}

In this section, we discuss our insights on the behavior of  in
comparison with existing operational semantics.

\subsection{Termination Behavior}\label{sec:termination}

Our proposed operational
semantics~ exhibits a termination behavior different from ,
, and . Compared to , we have solved the problem of
trivial non-termination of propagation rules, whereas any program terminating in
 also terminates in . With respect to  and ,
we found programs that terminate in  but not in  and
, and vice versa.

We have seen in Example~\ref{ex:trans_omega_e} and Example~\ref{ex:trans_obang}
that the transitivity rule displays different behavior in  and .
The program's termination behavior in  and  has been
investigated in \cite{Pilozzi2009}, where it is shown to terminate for acyclic
graphs. However, states containing cyclic graphs entail non-terminating behavior
(cf. \cite{Betz2010}). Contrarily, we show in the accompanying
technical report \cite{Betz2010} that in the operational semantics~, 
the computation of the transitive hull terminates for every possible input.
At the same place, we present a CHR program that terminates in  and
, but not in .

\subsection{Related Work}
\label{sec:related_work}

In \cite{Sarna-Starosta2007} the set-based semantics~ has been
introduced. Its development was, among other considerations, driven by the
intention to eliminate the propagation history. Besides addressing the problem of
trivial non-termination in a novel manner, it reduces non-determinism similarly
to the refined operational semantics~ \cite{Duck2004}. In
, a propagation rule cannot be fired infinitely often for a
possible matching. However, multiple firings are possible, the exact number
depending on the built-in store. 

The authors of \cite{Sarna-Starosta2007} justify their set-based approach by the
following statement:
\begin{quotation} 
``When working with a multi-set-based constraint store, it appears that
propagation history is essential to provide a reasonable semantics.''
\end{quotation}
Our approach can be understood as a compromise since we avoid
a propagation history by imposing an implicit set semantics on persistent
constraints. The distinction between linear and persistent
constraints, however, allows us to restrict the set behavior to those
constraints, whereas the multiset semantics is
preserved for linear constraints.

Linear logical algorithms \cite{Simmons2008} (LLA) is a programming language
based on bottom-up reasoning in linear logic, inspired by logical algorithms
\cite{Ganzinger2002}. The first implementation of logical algorithms was realized
in CHR with rule priorities \cite{DeKoninck2009}. 

Our proposed operational semantics~ is related to LLA \cite{Simmons2008},
but displays significant differences: Firstly, the notion of a constraint theory
with built-in constraints is absent in LLA. Secondly, LLA rules are restricted
such that persistent propositions cannot be derived multiple times, whereas
 makes no such restriction and solves this problem via the irreflexive
transition system. Thirdly, LLA requires a strict separation of propositions into
linear and persistent ones. In  a CHR constraint can occur in the linear
store, in the persistent store, or both.

On the other hand, the separation of
propositions in LLA allows the corresponding rules to freely mix linear and persistent
propositions in bodies. This is not directly possible with our approach, as CHR
constraints in a body are either added as linear or persistent constraints.

\section{Conclusion and Future Work}
\label{sec:conclusion}

The main motivation of this work was the observation that CHR research spans a
spectrum ranging from an analytical to a pragmatic end: on the analytical side of
the spectrum, emphasis is put on the formal aspects and properties of the
language while on the pragmatic side, it is put on implementation and efficiency.
A variety of operational semantics has been brought forth in the past, each
aligning with one side of the spectrum. In this work we proposed the novel
operational semantics~, heeding both analytical and pragmatic aspects.

Unlike other operational semantics with a strong analytical foundation, 
thus provides a terminating execution model and may be implemented as is. We
provided evidence to this claim by presenting a sound and complete encoding of
 into , which can be used to implement  by
source-to-source transformation.

Our operational semantics~ is based on the concept of persistent
constraints. These are finite representations of an arbitrarily large number of
syntactically equivalent constraints. They enable us to subsume trivially
non-terminating computations in a single derivation step.

We proved soundness and completeness of our operational semantics~ with
respect to . The latter stands exemplarily for analytical
formalizations of the operational semantics, thus providing a strong analytical
foundation for . This facilitates program analysis and formal proofs of
program properties.

In its current formulation,  is only applicable to range-restricted CHR
programs -- a limitation we plan to address in the future. Furthermore, similar
to  being the basis for numerous extensions to CHR
\cite{chr_survey_tplp08}, we plan to investigate the effect of building these
extensions on .

In a concurrent environment, some kind of conflict resolution is required for the
case that multiple rules try to remove the same constraint. For example, in
\cite{Sulzmann2008} a transaction-based approach is used, leading to a rollback,
if the first evaluated rule application removed the constraint. The formulation
of the \textbf{ApplyPersistent} transition reveals that for persistent
constraints, no such conflicts have to be taken into account. A closer
investigation of potential benefits of the persistent constraint approach in
concurrent settings remains to be conducted.

\bibliography{bibliography}

\end{document}