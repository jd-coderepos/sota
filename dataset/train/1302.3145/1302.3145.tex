\documentclass[11pt]{article}
\usepackage{fullpage}
\usepackage{latexsym}
\usepackage{amsthm}
\usepackage{amsfonts}
\usepackage{amssymb}
\usepackage[compact]{titlesec}
\usepackage{amsmath}
\usepackage{nicefrac}
\usepackage{epsfig}
\usepackage{ifpdf,color}


\definecolor{Darkblue}{rgb}{0,0,0.4}
\definecolor{Brown}{cmyk}{0,0.81,1.,0.60}
\definecolor{Purple}{cmyk}{0.45,0.86,0,0}
\newcommand{\mydriver}{hypertex}
\ifpdf
 \renewcommand{\mydriver}{pdftex}
\fi
\usepackage[breaklinks,\mydriver]{hyperref}
\hypersetup{colorlinks=true,citebordercolor={.6 .6 .6},linkbordercolor={.6 .6 .6},citecolor=Darkblue,urlcolor=black,linkcolor=Darkblue,pagecolor=black}
\newcommand{\lref}[2][]{\hyperref[#2]{#1~\ref*{#2}}}

\usepackage{float}


\newtheorem{theorem}{Theorem}[section]
\newtheorem{claim}[theorem]{Claim}
\newtheorem{proposition}[theorem]{Proposition}
\newtheorem{lemma}[theorem]{Lemma}
\newtheorem{corollary}[theorem]{Corollary}
\newtheorem{conjecture}[theorem]{Conjecture}
\newtheorem{observation}[theorem]{Observation}
\newtheorem{fact}[theorem]{Fact}
\theoremstyle{definition}
\newtheorem{example}[theorem]{Example}
\newtheorem{algorithm}[theorem]{Algorithm}
\newtheorem{definition}[theorem]{Definition}
\newtheorem{remark}[theorem]{Remark}
\newtheorem{problem}[theorem]{Problem}

\newcommand{\sa}{\textsf{Sherali-Adams}}
\newcommand{\ls}{\textsf{Lov\'asz-Schrijver}}
\newcommand{\la}{\textsf{Lasserre}}
\newcommand{\setc}{\textsc{Set Cover}}
\newcommand{\iLP}{\textsf{LP}}
\newcommand{\iSDP}{\textsf{SDP}}
\newcommand{\eps}{\epsilon}
\def\distr{\mathcal{D}}
\newcommand{\pr}{\mathop{\mathbb P}\displaylimits}
\newcommand{\Exp}{\mathop{\mathbb E}\displaylimits}
\def\bo {{\bf 0}}
\def\be {{\bf e}}
\def\bu {{\bf u}}
\def\bv {{\bf v}}
\def\bw {{\bf w}}
\def\bx {{\bf x}}
\def\by {{\bf y}}
\def\bz {{\bf z}}
\def\bff {{\bf f}}
\def\bg {{\bf g}}
\def\bb {{\bf b}}
\def\bc {{\bf c}}
\def\ba {{\bf a}}
\def\C{{\mathcal C}}
\def\A{{B}}
\def\E{{\rm E}}
\def\Re{{\mathbb R}}
\def\Pr{{\rm Pr}}

\newcommand{\OPT}{\textsc{OPT}}

\newcommand{\factor}{\ensuremath{O(\smash{\frac{\log n}{\log\log n}})}}

\makeatletter
 \setlength{\parindent}{0pt}
 \addtolength{\partopsep}{-2mm}
 \setlength{\parskip}{5pt plus 1pt}
\addtolength{\abovedisplayskip}{-3mm}
\makeatother

\newcounter{note}[section]
\renewcommand{\thenote}{\thesection.\arabic{note}}
\newcommand{\agnote}[1]{\refstepcounter{note}{\bf Anupam's Comment~\thenote:} {\sf #1}\marginpar{\tiny\bf AG~\thenote}}
\newcommand{\mnote}[1]{\refstepcounter{note}{\bf Mohit's Comment~\thenote:} {\sf #1}\marginpar{\tiny\bf MS~\thenote}}
\newcommand{\zfnote}[1]{\refstepcounter{note}{\bf Zac's Comment~\thenote:} {\sf #1}\marginpar{\tiny\bf ZF~\thenote}}

\newcommand{\initOneLiners}{\setlength{\itemsep}{0pt}
    \setlength{\parsep }{0pt}
    \setlength{\topsep }{0pt}
}
\newenvironment{OneLiners}[1][\ensuremath{\bullet}]
    {\begin{list}
        {#1}
        {\initOneLiners}}
    {\end{list}}


\def\polhk#1{\setbox0=\hbox{#1}{\ooalign{\hidewidth\lower1.5ex\hbox{`}\hidewidth\crcr\unhbox0}}}

\newenvironment{comment}{

\begin{quote}
  \footnotesize
\noindent{\bf Comment:}}
{\end{quote}

}

\title{An Improved Integrality Gap for Asymmetric TSP Paths\thanks{An
    extended abstract of this paper appears in the \emph{Proceedings of
      the 16th Conference on Integer Programming and Combinatorial
      Optimization, 2013}.}}
\author{
Zachary Friggstad\thanks{Department of Computing Science, University of Alberta.}
\and
Anupam Gupta\thanks{Department of Computer Science, Carnegie Mellon University, Pittsburgh
    PA 15213, and Microsoft Research SVC, Mountain View, CA
    94043. Research was partly supported by NSF awards CCF-0964474 and
    CCF-1016799.}
\and
Mohit Singh\thanks{Microsoft Research, Redmond.}
}


\begin{document}

\maketitle

\begin{abstract}

  \bigskip
  The Asymmetric Traveling Salesperson Path Problem (ATSPP) is one
  where, given an \emph{asymmetric} metric space  with specified
  vertices  and , the goal is to find an - path of minimum
  length that passes through all the vertices in .

  \medskip
  This problem is closely related to the Asymmetric TSP (ATSP),
  which seeks to find a tour (instead of an - path) visiting all
  the nodes: for ATSP, a -approximation guarantee implies an
  -approximation for ATSPP. However, no such connection is
  known for the \emph{integrality gaps} of the linear programming
  relaxations for these problems: the current-best approximation
  algorithm for ATSPP is , whereas the best bound
  on the integrality gap of the natural LP relaxation (the subtour
  elimination LP) for ATSPP is .

  \medskip
  In this paper, we close this gap, and improve the current best bound
  on the integrality gap from  to . The
  resulting algorithm uses the structure of narrow - cuts in the
  LP solution to construct a (random) tree spanning tree that can be cheaply augmented
  to contain an Eulerian - walk.

  \medskip
  We also build on a result of Oveis Gharan and Saberi and show
  a strong form of Goddyn's conjecture about thin spanning trees
  implies the integrality gap of the subtour elimination LP relaxation for ATSPP is bounded by a constant.
  Finally, we give a simpler family of instances showing the
  integrality gap of this LP is at least .
\end{abstract}





\section{Introduction}
\label{sec:introduction}

In the Asymmetric Traveling Salesperson Path Problem (ATSPP), we are
given an \emph{asymmetric} metric space  (i.e., one where the
distances satisfy the triangle inequality, but potentially not the
symmetry condition), and also specified source and sink vertices  and
, and the goal is to find an - Hamilton path of minimum length.

ATSPP is a close relative of Asymmetric TSP
(ATSP), where the goal is to find a Hamilton tour instead of an -
path.  For ATSP, the -approximation of Frieze,
Galbiati, and Maffioli~\cite{FGM} from 1982 was the best result known
for more than two decades, until it was finally improved by constant
factors in~\cite{Blaser08,KLSS05,FS07}. A breakthrough on this problem
was an -approximation due to Asadpour, Goemans,
M{\polhk{a}}dry, Oveis Gharan, and Saberi~\cite{AGMSS}; they also
bounded the integrality gap of the subtour elimination linear
programming relaxation for ATSP by the same factor.

Somewhat surprisingly, the study of ATSPP has been of a more recent
vintage: the first approximation algorithms appeared only around
2005~\cite{LN08,CP07,FS07}. It is easily seen that the ATSP reduces to
ATSPP in an approximation-preserving fashion (by guessing two
consecutive nodes on the tour).  In the other direction, Feige and Singh~\cite{FS07}
show that a -approximation for ATSP implies an
-approximation for ATSPP. Using the above-mentioned
-approximation for ATSP~\cite{AGMSS}, this implies an
-approximation for ATSPP as well.

The subtour elimination linear program generalizes simply to ATSPP
and is given in \lref[Section]{sec:rounding-algorithm}. However,
prior to our work, the best integrality gap known for this LP for ATSPP was still ~\cite{FSS10}. In this paper we show the following result.

\begin{theorem}\label{thm:main}
  The integrality gap of the subtour elimination linear program for
  ATSPP is .
\end{theorem}

We also explore the connection between integrality gaps for ATSPP
and the so-called ``thin trees conjecture''. In particular, if Goddyn's conjecture
regarding thin trees holds with strong-enough quantitative bounds
then the integrality gap of the subtour elimination LP for ATSPP
is bounded by a constant. The precise statement of the conjecture and of
our result can be found in \lref[Section]{sec:thin}. This is analogous to
a similar statement made by Oveis Gharan and Saberi regarding
the integrality gap of the subtour elimination LP for ATSP~\cite{GS11}.

Finally, we give a simple construction showing that the integrality gap of
this LP is at least~; this example is simpler than previous known
integrality gap instance showing the same lower bound, due to Charikar,
Goemans, and Karloff~\cite{CGK06}.

Given the central nature of linear programs in approximation algorithms,
it is useful to understand the integrality gaps for linear programming
relaxations of optimization problems. Not only does this study give us a
deeper understanding into the underlying problems, but upper bounds
on the integrality gap of LPs are often useful in approximating related problems.
For example, the polylogarithmic approximation guarantees in
the work of Nagarajan and Ravi~\cite{NR07} for Directed Orienteering and
Minimum Ratio Rooted Cycle, and those in the work of Bateni and
Chuzhoy~\cite{BC10} for Directed -Stroll and Directed -Tour were
all improved by a factor of  following the improved bound of
 on the integrality gap of the subtour LP relaxation for ATSP.
We emphasize that these improvements required the integrality gap
bound improvement for ATSP, not merely improved approximation guarantees.



\subsection{Our Approach}
\label{sec:approach}


Our approach to bound the integrality gap for ATSPP is similar to that
for ATSP~\cite{AGMSS,GS11}, but with some crucial differences.
To prove \lref[Theorem]{thm:main}, we sample a
random spanning tree in the underlying undirected multigraph
and then augment the directed version of this tree
to an integral circulation using Hoffman's circulation theorem while
ensuring the - edge is only used once. The support of this circulation
is weakly connected, so it can be used to obtain an Eulerian circuit with no greater cost.
Deleting the - edge from this walk results in a spanning - walk.

However, the non-Eulerian nature of ATSPP makes it difficult
to satisfy the cut requirements in Hoffman's circulation theorem if we
sample the spanning tree directly from the distribution given by the LP
solution. It turns out that the problems come from the - cuts 
that are nearly-tight: i.e., which satisfy  for some small constant  --- these give rise to problems
when the sampled spanning tree includes more than one edge across this
cut. Such problems also arise in the symmetric TSP paths case (studied
in the recent papers of An, Kleinberg, and Shmoys~\cite{AKS12} and Seb\H{o}~\cite{Sebo13}): their
approach is again to take a random tree directly from the distribution
given by the optimal LP solution, but in some cases they need to boost
the narrow cuts, and they show that the loss due to this boosting is
small.

In our case, the asymmetry in the problem means that boosting the narrow
cuts might be prohibitively expensive. Hence, our idea is to preprocess
the distribution given by the LP solution to \emph{tighten} the narrow
cuts, so that we never pick two edges from a narrow cut. Since the
original LP solution lies in the spanning tree polytope, lowering the
fractional value on some edges means we need to raise the fractional value on other edges. This would cause
the costs to increase, and the technical heart of the paper is to ensure
this can be done with a small increase in the cost.

Our approach for proving an  integrality gap bound
under the thin trees conjecture is similarly inspired by related work
for ATSP~\cite{GS11},
but, again, we must be careful to ensure that the thin tree
crosses each narrow cut exactly once.
We do this by finding a cheap thin tree ``between'' narrow cuts
(which we will prove are nested) and then chaining these thin together
trees by selecting a single edge across each narrow cut.
The resulting tree will have the desired thinness properties.








\subsection{Other Related Work}
\label{sec:related-work}

The first non-trivial approximation for ATSPP was an
-approximation by Lam and Newman~\cite{LN08}. This was
improved to  by Chekuri and P\'al~\cite{CP07}, and the
constant was further improved in~\cite{FS07}. The paper~\cite{FS07} also
showed that a -approximation algorithm for ATSP can be used
to obtain an -approximation algorithm for ATSPP.
All these results are combinatorial and do not bound
integrality gap of ATSPP. A bound of  on the integrality
gap of ATSPP was given by Nagarajan and Ravi~\cite{NR-direct-latency},
and was improved to  by Friggstad, Salavatipour and
Svitkina~\cite{FSS10}. Note that there is still no result known that
relates the integrality gaps of subtour elimination relaxations for ATSP and ATSPP in a
black-box fashion.

In the symmetric case (where the problems become TSPP and TSP
respectively), constant factor approximations and integrality gaps have
long been known. We do not survey the rich body of literature on TSP
here, instead pointing the reader to, e.g., the recent paper on
graphical TSP by Seb\H{o} and Vygen~\cite{SV12}. An exception is a
result of An, Kleinberg, and Shmoys~\cite{AKS12}, who give an upper
bound of  on integrality gap of the LP relaxation for the TSPP
problem; their algorithm also proceeds via studying the narrow -
cuts, and the connections to our work are discussed in
\lref[Section]{sec:approach}.  This bound on the integrality gap was
subsequently improved to  via a more refined analysis by
Seb\H{o}~\cite{Sebo13}.

\subsection{Notation and Preliminaries}

Given a directed graph , and two disjoint sets , let . We use the
standard shorthand that ,
and . When the set  is a
singleton (say ), we use  or 
instead of  or . For undirected
graph , we use  to denote edges crossing
between  and , and  to denote the edges with exactly
one endpoint in  (which is the same as .
For any subset  we let  denote , the set of
arcs with both endpoints in .
If we are discussing subsets of arcs  of , we add subscripts
to the  notation to indicate we only consider those arcs crossing
the cut that in are .
For example,  denotes 
and so on. A collection of subsets of , say  is a partition if each element of  occurs in exactly one part of . Given a graph  and a partition
 of , we let  to be the set of edges in  which have endpoints in different sets of .

For a digraph , a set of arcs  is \emph{weakly
  connected} if the undirected version of  forms a connected graph
that spans all vertices in .

For values  for all , and a set of arcs , we let  denote the sum .

Given an undirected graph  and a subset of edges
, we let  denote
the characteristic vector . The spanning
tree polytope is the convex hull of . See, e.g.,~\cite[Chapter~50]{Schrijver-book} for
several equivalent linear programming formulations of this polytope.
We sometimes abuse notation and call a set of directed arcs 
a tree if the undirected version of  is a tree in the usual sense.


A \emph{directed metric graph} on vertices  has arcs 
where the non-negative arc costs satisfy the triangle
inequality  for all . However,
arcs  and  need not have the same cost. An instance of ATSPP
is a directed metric graph along with distinguished vertices .



\section{The Rounding Algorithm}
\label{sec:rounding-algorithm}


In this section, we give the linear programming relaxation for
ATSPP, and show how to round a feasible solution  to this LP to get a path of
cost  times the cost of .
We then give the proof, with some of the details being deferred to the
following sections.

Given a directed metric graph  with arc costs , we use the following standard linear programming relaxation for
ATSPP which is also known as the subtour elimination linear program.
  


Constraints (4) can be separated over in polynomial time using
standard min-cut algorithms, so this LP can be solved
in polynomial time using the ellipsoid method.
We begin by solving the above LP to obtain an optimal solution .
Consider the undirected (multi)graph  obtained by removing
the orientation of the arcs of . That is, create precisely two edges
between every two nodes  in , one having cost  and
the other having cost . (Hence, .) For a point , let  denote the corresponding point in ,
and view  as the ``undirected'' version of .



We will use the following definition: An \emph{- cut} is a subset
 such that . The following fact will be used throughout the paper.

\begin{claim} \label{claim:lp}
Let  be a feasible solution to LP (\ref{eq:lp}). For any  cut , .
Also,  for every nonempty .
\end{claim}
\begin{proof}
For any nonempty subset of vertices  we have

If  is an  cut, then the first sum in the last expression is  and the second sum is  by Constraints (1), (2), and (3). If ,
then both sums are equal to  by Constraints (3).
\end{proof}


\begin{definition}[Narrow cuts]
  Let . An - cut  is \emph{-narrow} if
   (or equivalently, ).
\end{definition}

The main technical lemma is the following:
\begin{lemma}\label{lem:fix_sol}
  For any , one can find, in polynomial-time, a vector
   (over the directed arcs) such that:
  \begin{enumerate}
  \item[(a)] its undirected version  lies in the spanning tree
    polytope for ,
  \item[(b)]  (where the inequality
    denotes component-wise dominance), and
  \item[(c)]  and  for every
    -narrow - cut .
\end{enumerate}
\end{lemma}

Before we prove the lemma (in \lref[Section]{sec:proof-of-lem}), let us
sketch how it will be useful to get a cheap ATSPP solution. Since  (or more
correctly, its undirected version ) lies in the spanning tree
polytope, it can be represented as a convex combination of spanning
trees.  Using some recently-developed algorithms (e.g., those due
to~\cite{AGMSS,CVZ10}) one can choose a (random) spanning tree that crosses
each cut only  times more than the LP solution. Finally, we can
use  times the LP solution to patch this tree to get an -
path. Since the LP solution is ``weak'' on the narrow cuts and may
contribute very little to this patching (at most ), it is crucial
that by property~(c) above, this tree will cross the narrow cuts
\emph{only once}, and that too, it crosses in the ``right'' direction,
so we never need to use the LP when verifying the cut conditions of
Hoffman's circulation theorem on narrow cuts. The details of these
operations appear in \lref[Section]{sec:patching}.

We will assume  to ensure all of our arguments work. For ,
we use the known integrality gap bound of 
from~\cite{FSS10} to ensure the gap is bounded for all .

\subsection{The Structure of Narrow Cuts}
\label{sec:proof-of-lem}

We now prove \lref[Lemma]{lem:fix_sol}: it says that we can take the LP
solution  and find another vector  such that if an - cut is
narrow in  (i.e. ), then . Moreover,
the undirected version of  can be written as a convex combination of
spanning trees, and  is not much larger than  for any arc .

The undirected version of  itself can be written as a
convex combination of spanning trees, so if we force  to cross the
narrow cuts to an extent less than  (loosely, this reduces the
connectivity), we had better increase the value on other arcs. To show we
can perform this operation without changing any of the coordinates by
very much, we need to study the structure of narrow cuts more closely.
(Such a study is done in the \emph{symmetric} TSP path paper of An et
al.~\cite{AKS12}, but our goals and theorems are somewhat different.)

First, say two - cuts  and  \emph{cross} if 
and  are
non-empty. 

\begin{lemma}
  \label{lem:nocross}
  For , no two -narrow - cuts cross.
\end{lemma}
\begin{proof}
  Suppose  and  are crossing -narrow - cuts. Then
  
  where the last inequality follows from the first three terms being
  cuts excluding  and hence having at least unit -value crossing them (by the
  LP constraints), the fourth term being non-negative, and the last term
  being the -value of a subset of the arcs in  and remembering that  and  are -narrow.
  However, this contradicts .
\end{proof}

\lref[Lemma]{lem:nocross} says that the -narrow cuts form a chain
 with .
For . let . We also define
 and . Let  and .
For the rest of this paper, we will use  to denote a value in the range .
Ultimately, we will set  for the final bound but we state the lemmas
in their full generality for .


Next, we show that out of the (at most)  mass of  across
each -narrow cut , most of it comes from the ``local'' arcs
in .


\begin{lemma} \label{lem:li+}
  For each ; . \end{lemma}
\begin{proof}
  If  then  so in fact . In this case, 
  and  and the LP constraints clearly imply .

  Now consider the case .
  For , since  we have  from the LP
  constraints. We also have  because  is
  -narrow, and therefore . A similar argument for  shows
  . So it remains to consider
  .
Define the following quantities, some of which can be zero.
  \begin{OneLiners}
  \item 
  \item 
  \item 
  \end{OneLiners}
  We have
   because  and  is -narrow.  Similarly
  
  Summing these two inequalities yields  where we have used . Rearranging shows .
\end{proof}






Now, recall that  denotes the assignment of arc weights to
the graph  from the previous section obtained by ``removing''
the directions from arcs in .  We prove that the restriction of
 to any  almost satisfies the partition inequalities
that characterize the convex hull of connected spanning subgraphs of . This characterization was given by Edmonds~\cite{Edmonds70b}; see also Chapter 50, Corollary 50.8(a) in Schrijver~\cite{Schrijver-book}. We state it here for completeness.


\begin{theorem}\cite{Edmonds70b}
Let  be a graph. Then the convex hull of all connected spanning subgraphs of  is given by . Moreover, the convex hull of spanning trees of  is given by
.
\end{theorem}






For a partition  of a subset of , we let
 denote the set of edges whose endpoints lie in two different
sets in the partition. To be clear,  does not contain any edge that has at least one endpoint in .

\begin{lemma}\label{lem:parts}
For any  and any partition  of , we have .
\end{lemma}
\begin{proof}
Since  and , there is nothing to prove for  or . So, we suppose .

Consider the quantity .
On one hand  because neither  nor  lie in  for any , so
. On the other hand,  counts each arc between two parts in  exactly twice and each
arc with one end in  and the other not in  precisely once. So, .

Notice that  and  are disjoint subsets of .
So, since both  and  are -narrow, .
This shows  which, after rearranging, is what we wanted to show.
\end{proof}





\begin{corollary}\label{cor:parts}
  For any partition  of , we have
  .
\end{corollary}
\begin{proof}
From \lref[Lemma]{lem:parts}, we have  for any .
\end{proof}



Finally, to efficiently implement the arguments in the proof of \lref[Lemma]{lem:fix_sol}, we need to be able to efficiently find
all -narrow cuts .
This is done by a standard recursive algorithm that exploits the fact that the cuts are nested.
\begin{lemma} \label{lem:efficient}
There is a polynomial-time algorithm to find all -narrow  cuts.
\end{lemma}
\begin{proof}
Consider following recursive algorithm.
As input, the routine is given a directed graph  with arc weights 
and distinct nodes   where both  and  are -narrow.
Say a -narrow cut  in  is non-trivial if  and .
The claim is that the procedure will find all non-trivial -narrow  cuts of , provided that they are nested.

The procedure works as follows.
If there are non-trivial -narrow  cuts in ,
then there are nodes  such that
some -narrow  cut  has .
So, the procedure tries all  pairs of distinct nodes , contracts both  and 
to a single node and determines if the minimum cut separating these contracted nodes
has -capacity less than . If such a cut  was found for some , the algorithm makes two recursive calls,
one with the contracted graph  with start node being the contraction of  and end node being ,
and the other with the contracted graph  with start node  and end node being the contraction of .
After both recursive calls complete, the algorithm returns all -narrow
cuts found by these two recursive calls (of course, after expanding the contracted nodes)
and the -narrow cut  itself. If such a cut  was not found over all choices of , then the algorithm returns
nothing because there are no non-trivial -narrow  cuts in .

It is easy to see that a non-trivial -narrow cut in either contracted graph corresponds to a -narrow cut in .
On the other hand, if the -narrow  cuts are nested in , then every non-trivial -narrow 
cut apart from  itself corresponds to a non-trivial -narrow cut in exactly one of  or
. Also, the -narrow cuts in both contracted graphs remain nested.
So, the recursive procedure finds all non-trivial -narrow cuts of . The number of recursive calls
is at most the number of non-trivial -narrow cuts, and this is at most  because the cuts are nested
so it is an efficient algorithm. We call this algorithm initially with graph , start node  and end node .
\lref[Lemma]{lem:nocross} implies the -narrow  cuts of  are nested
so the recursive procedure finds all non-trivial -narrow cuts of . Adding these to
 and  gives all -narrow cuts of .
\end{proof}








\begin{proof}[Proof of Lemma \ref{lem:fix_sol}]
  The claimed vector  can be described by linear constraints: indeed,
  consider the following polytope on the variables .


Consider the vector  given as follows.


Constraints~(\ref{cons4}) and (\ref{cons5}) are satisfied by
construction. Constraint~(\ref{cons2}) follows from \lref[Lemma]{lem:li+}
for edges in  and by construction for rest of the
edges. For constraint~(\ref{cons3}), note that


Next we show Constraints~(\ref{cons1}) holds.  It suffices to
show that  can be decomposed as a convex combination of characteristic
vectors of connected graphs. For
, let  denote the restriction of  to
edges whose endpoints are both contained in . Then
\lref[Corollary]{cor:parts}, Constraints~(\ref{cons5}), and
\cite[Corollary~50.8a]{Schrijver-book} imply that  can be
decomposed as a convex combination of integral vectors, each of which
corresponds to an edge set that is connected on .  Next, let 
denote the restriction of  to edges whose endpoints are both
contained in some common . Since the sets  are disjoint, we have that  (where the
addition is component-wise). Furthermore,
, being the sum of the  vectors,
can be decomposed as a convex combination of integral vectors
corresponding to edge sets  such that the connected components of
the graph  are precisely the sets
.


Next, let  denote the restriction of  to edges contained
in one such .  We also note that the sets
 are disjoint.  By
construction, we have  for each  so we may decompose  as a convex-combination of integral
vectors, each of which includes precisely one edge across each
.

Adding any integral point  in the decomposition of  to any
integral point  in the decomposition of  results
in an integral vector that corresponds to a connected graph:
each  is connected by  and consecutive  are connected
by . By construction of , we have 
so we may write  as a convex combination of characteristic vectors
of connected graphs, each of which satisfies Constraints~(\ref{cons1}).

Finally, we modify  slightly to ensure constraint (\ref{cons1-1}) holds while maintaining
the other constraints. From~\cite[Corollary~50.8a]{Schrijver-book} and
Constraints (\ref{cons1}) and (\ref{cons5}),
 lies in the convex hull of incidence vectors corresponding to connected
(multi)graphs. Decompose  into a convex combination of such vectors,
drop edges from the corresponding connected graphs to get spanning tree,
and recombine these spanning trees to get a point in the spanning
tree polytope. Note that we only decreased  values
so Constraints (\ref{cons2}) and (\ref{cons4}) continue to hold.
Finally, since  now lies in the spanning tree polytope
then each tree must still cross each narrow cut, so Constraints (\ref{cons3}) still hold.

Such a vector can be found efficiently because Constraints (\ref{cons1}) admit
an efficient separation oracle~\cite[Corollary~51.3b]{Schrijver-book}.
\end{proof}








\section{Obtaining an - Path}
\label{sec:patching}


Having transformed the optimal LP solution  into the new vector 
(as in \lref[Lemma]{lem:fix_sol}) without increasing it too much in any
coordinate, we now sample a random tree such that it has a small total
cost, and that the tree does not cross any cut much more than prescribed
by . Finally we add some arcs to this tree (without increasing its
cost much) so that every  has equal indegree and outdegree
while ensuring that  has outdegree 1 and indegree 0.
This gives us an Eulerian - walk.

By the triangle inequality,
shortcutting this walk past repeated nodes yields a Hamiltonian  path
of no greater cost. While this general approach is similar
to that used in~\cite{AGMSS}, some new ideas are required because we are
working with the LP for ATSPP---in particular, only one unit of flow is
guaranteed to cross - cuts, which is why we needed to deal with
narrow cuts in the first place. The details appear in the rest of this section.

\subsection{Sampling a Tree}\label{sec:sample}

For a digraph  and a collection of arcs , we say  is
\emph{-thin with respect to } if  for every . The set  is also
\emph{-approximate with respect to } if the total cost of
all arcs in  is at most  times the cost of , i.e.,
.  The reason
we are deviating from the undirected setting used in~\cite{AGMSS} to the directed setting is that the
orientation of the arcs across each -narrow cut will be important
when we sample a random ``tree''.


\begin{lemma} \label{lem:findtree}
  Let . Let  and .
For , there is a randomized,
  polynomial time algorithm that, with probability at least 1/2, finds
  an -thin and -approximate (with respect to )
  collection of arcs  that is weakly connected and satisfies  and  for each
  -narrow - cut .
\end{lemma}
\begin{proof}
  Let  be a vector as promised by \lref[Lemma]{lem:fix_sol}. From
  , randomly sample a set of arcs  whose undirected
  version  is a spanning tree on . This should be done
  from any distribution with the following two properties:
  \begin{itemize}
  \item[(i)] \emph{(Correct Marginals)} 
  \item[(ii)] \emph{(Negative Correlation)} For any subset of edges , 
  \end{itemize}
  This can be obtained using, for example, the swap rounding approach
  for the spanning tree polytope given by Chekuri et al.~\cite{CVZ10}.
  As in~\cite{AGMSS}, the negative correlation property implies the following
  theorem. The proof is found in \lref[Section]{sec:chernoff}.
  \begin{theorem}\label{thm:chernoff}
  For , the tree  is -thin with probability at least .
  \end{theorem}

  By \lref[Lemma]{lem:fix_sol}(b), property~(i) of the random sampling,
  and Markov's inequality, we get that  (from \lref[Lemma]{lem:findtree}) is
  -approximate with respect to  with probability
  at least 2/3. By a trivial union bound, for  we have
  with probability at least 1/2 that  is both -thin and -approximate with
  respect to . It is also weakly connected---i.e., the undirected
  version of  (namely, ) connects all vertices in .

  The statement for -narrow - cuts follows from the fact
  that  satisfies \lref[Lemma]{lem:fix_sol}(c). That is,  contains
  no arcs of , since  (for  being a
  -narrow - cut). But since  is a spanning tree,
   must contain at least one arc from . Finally, since
   is exactly 1, then any set of arcs supported by this
  distribution we use must have precisely one arc from .
\end{proof}

\subsection{Augmenting to an Eulerian - Walk}

We wrap up by augmenting the set of arcs  to an Eulerian - walk.
Specifically, we prove the following for general .
\begin{theorem}\label{thm:augment}
Suppose we are given a collection of arcs  that is weakly connected, -thin, and satisfies
 and  for any -narrow  cut .
We can find a Hamiltonian  path with cost at most
 in polynomial time.
\end{theorem}

For this, we use Hoffman's circulation theorem, as
in~\cite{AGMSS}, which we recall here for convenience (see,
e.g,~\cite[Theorem~11.2]{Schrijver-book}):
\begin{theorem}
  \label{thm:hoffman}
  Given a directed flow network , with each arc having a
  lower bound  and an upper bound  (and ), there exists a circulation  satisfying  for all arcs  if and only if
   for all .
  Moreover, if the  and  are integral, then the circulation 
  can be taken integral.
\end{theorem}


\begin{proof}[Proof of Theorem \ref{thm:augment}]
Set lower bounds  on the arcs by:

For now, we set an upper bound of 1 on arc  and leave all other arc
upper bounds at . We compute the minimum cost circulation
satisfying these bounds (we will soon see why one must exist). Since the
bounds are integral and since  is weakly connected, this circulation
gives us a directed Eulerian graph. Furthermore, since , the  arc must appear exactly once in this Eulerian
graph. Our final Hamiltonian - path is obtained by following an
Eulerian circuit, removing the single  arc from this circuit to get
an Eulerian - walk, and finally shortcutting this walk past
repeated nodes. The cost of this Hamiltonian path will be, by the
triangle inequality, at most the cost of the circulation minus the cost
of the  arc.

Finally, we need to bound the cost of the circulation (and also to prove
one exists). To that end, we will impose stronger upper bounds  as follows:

We use Hoffman's circulation theorem to show that a circulation 
exists satisfying these bounds  and  (The calculations appear
in the next paragraph.) Since  is no longer integral, the circulation
 might not be integral, but it does demonstrate that a circulation
exists where each arc  is assigned at most
 more flow in the circulation than the number
of times it appears in . Consequently, it shows that the minimum
cost circulation  in the setting where we only had a non-trivial
upper bound of  on the arc  can be no more expensive (since there
are fewer constraints), and that circulation  can be chosen to be
integral. The cost of circulation  is at most the cost of , which
is at most
 Subtracting the cost of the  arc (since
we drop it to get the Hamilton path),
we get that the final Hamiltonian path has cost at most






One detail remains: we need to verify the conditions of
\lref[Theorem]{thm:hoffman} for the bounds  and .  Firstly, it is
clear by definition that  for each arc . Now we need
to check  for each cut
. This is broken into four cases.

\begin{enumerate}
\item  is a -narrow - cut. Then , since  contains only one arc in . But .

\item  is an - cut, but not -narrow. Then by the
  -thinness of  and \lref[Claim]{claim:lp},
   On the other hand,
  
  where the last inequality used the fact that .

\item  is a - cut. Then by the -thinness of  and \lref[Claim]{claim:lp},
  
  the last inequality using that .  Moreover
   Then .

\item  does not separate  from . Then
  
\end{enumerate}
\end{proof}

The proof of our main result, \lref[Theorem]{thm:main}, follows immediately from
\lref[Theorem]{thm:augment} and \lref[Lemma]{lem:findtree} and setting .
Furthermore, this proof also shows that there is a randomized polynomial time algorithm
that constructs a Hamiltonian  path witnessing this integrality gap bound with probability at least 1/2.



\section{Guaranteeing -Thinness}
\label{sec:chernoff}
We prove \lref[Theorem]{thm:chernoff} in this section.  Recall that
-thin means the number of arcs chosen from 
should not exceed  (so a directed version).
Let  where the logarithm is the natural logarithm. Recall
that  is the set of arcs found with corresponding undirected
spanning tree .  By the first property of the distribution
(preservation of marginals on singletons) we have for each  that .

We have negative correlation on subsets of items, so we can apply standard concentration bounds.
Specifically, we use the following inequality.
\begin{theorem}{\cite[Theorem 3.4]{PS97}}\label{thm:ps}
Let  be given 0-1 random variables with  and  such that for all ,
. Then for any  we have

\end{theorem}

For notational simplicity, let .
\lref[Theorem]{thm:ps} immediately shows

Let  (again using the natural logarithm) and use \lref[Theorem]{thm:ps} with .
For , the above expression is bounded (in a manner similar to \cite{AGMSS}) by

However, for any graph, there are at most  cuts whose capacity
is at most  times the capacity of the minimum
cut~\cite{KS96}. Note that the minimum cut with capacities  is 1, so
there are at most  cuts of the undirected graph  with capacity (under ) at most .
Another way to view this is that there are at most  cuts whose capacity is between  and .
For each such cut , the previous analysis shows that
probability that  is at most .
Thus, by the union bound, the probability that 
for some  is bounded by


Since , then we have just seen that with probability at least  that there is no 
with . This is close to what we want, except we should bound  against the -capacity of .
That is, we ultimately want to show  for every . To do this, we consider two cases.


\begin{itemize}
\item If either  or  is a -narrow  cut. Then we ignore the above analysis and simply note that by the properties of  guaranteed by \lref[Lemma]{lem:fix_sol}
either  (if ) or  (if ), both of which are bounded by .
\item Otherwise, either  or  is an  cut that is not -narrow, or  does not separate  from . In any case, we have
 (by Claim \ref{claim:lp}) and .
Since , then  so . So,

Summarizing, for  we have with probability at least  that

That is,  is -thin with high probability.

\end{itemize}

\section{Improved Bounds from Thin Tree Conjectures}
\label{sec:thin}



In \lref[Section]{sec:patching}, we defined thinness of a set of directed arcs with respect to an LP solution. Here, we define it for undirected graphs
with respect to the original graph itself.
\begin{definition}
Let  be an undirected graph. A spanning tree  of  is said to be -thin if for every cut 
we have .
\end{definition}

The following conjecture was given by Goddyn~\cite{goddyn}.
\begin{conjecture}\label{conj:thin}
There is some constant  such for any , any undirected -edge connected graph has a -thin spanning tree.
\end{conjecture}

Oveis-Gharan and Saberi~\cite{GS11} show that assuming \lref[Conjecture]{conj:thin} is true, there is an -approximation for the ATSP problem by bounding the integrality gap of the subtour elimination LP. We generalize the result for ATSPP in \lref[Theorem]{thm:thin}. While the proof follows the same outline,  there are some technicalities that must be overcome in the case of ATSPP which we outline.

\begin{theorem}\label{thm:thin}
If \lref[Conjecture]{conj:thin} is true, then the integrality gap of the subtour elimination LP for ATSPP is at most .
\end{theorem}

\lref[Theorem]{thm:thin} follows immediately from \lref[Theorem]{thm:augment} once we show the following. For notational simplicity, we will set the value of  to 1/4 for the remainder of this section.

\begin{lemma}\label{lem:thin}
If \lref[Conjecture]{conj:thin} is true, then we can find a -thin collection of arcs  of cost at most  satisfying the
requirements of \lref[Theorem]{thm:augment}.
\end{lemma}

First we recall a result by Oveis Gharan and Saberi~\cite{GS11} which will play an important role in our proof. We state a more specific
form of their proposition.
\begin{proposition}{\cite{GS11}}\label{prop:thintree}
If \lref[Conjecture]{conj:thin} is true, then every -edge connected graph  with edge costs  has a
-thin spanning tree with cost at most .
\end{proposition}


\begin{proof}[Proof of Lemma \ref{lem:thin}]
Let  be an optimum LP solution. We cannot invoke \lref[Proposition]{prop:thintree} directly on a scaled version of 
(as was done for ATSP in \cite{GS11}) since the resulting thin tree might cross the narrow cuts more than once or, perhaps, in the wrong direction.
Our solution will be to sample a thin tree on the subgraphs between narrow cuts and chain these together using arcs that cross the narrow cuts
to ensure the resulting tree crosses the narrow cuts exactly once.

Recall the definition of -narrow cuts (again, we use  here)
and let  be the sets defined in \lref[Section]{sec:proof-of-lem}.
For every , let  denote the restriction of  to . That is,  if  and  otherwise.
Recall by \lref[Corollary]{cor:parts} that  for any .

For each  with , create an undirected graph  where  will contain many copies of edges between nodes in .
Similar to the proof of Theorem 5.3 in~\cite{GS11}, round down each  value to its nearest multiple of  and call this value .
Add  copies of the undirected version of arc  to  for each , each with cost .
Since , for every cut  of  we have .
Therefore, we have  for every cut  of .

By \lref[Proposition]{prop:thintree}, we may find a -thin spanning tree  of  with cost bounded by

Let  be the original (directed) arcs of the graph  that are used by .

Next, for each , let  denote the cheapest arc in . By \lref[Lemma]{lem:li+}
with , .

Finally, let  and note that because the cost of  was
charged to the LP cost for arcs in  and the cost of  was charged to the LP cost for edges in , then
 (clearly \lref[Conjecture]{conj:thin} can only hold for ).

From construction,  and  for any -narrow cut .
Since  is formed by chaining together weakly connected subgraphs in each  using edges in , it is weakly connected.

We finish by showing that  is -thin with respect to . Consider any cut  of .
If  or  is a -narrow cut then  by construction of  and feasibility of  as a solution to the subtour elimination LP.

Otherwise, let  and let  for each  with .
Note that .

For each  we have


Finally, we bound . If  then it cannot be the case that  nor can it be the case that .
So, at least one of the three following cases must hold:
\begin{enumerate}
\item ; we charge the occurrence of  to the quantity 
(cf.\ \lref[Corollary]{cor:parts}).
\item ; we charge the occurrence of  to the quantity .
\item  and  and therefore, . In this case, we charge the occurrence of  to the quantity 
(cf.\ \lref[Lemma]{lem:li+}).
\end{enumerate}
In each of the cases, the edges whose  values were charged all lie in  or . Furthermore, every edge is charged at most twice
this way. If  then it is charged at most once (for ), if  then it is charged at most once for  and at most
once for . Overall, we see .

Considering all of these bounds, we have

\end{proof}

The collection of arcs  is -thin and has cost at most .
Furthermore,  satisfies  and  for every  -narrow  cut .
From \lref[Theorem]{thm:augment}, we can then obtain a ATSPP solution with cost at most

This completes the proof of \lref[Theorem]{thm:thin}.

We have not attempted to optimize the constants in our analysis. For example, a more careful scaling of  to get the  values
in the above proof will improve the constants.



\section{A Simple Integrality Gap Example}
\label{sec:int-gap}

In this section, we show that the integrality gap of the subtour
elimination LP~(\ref{eq:lp}) is at least~. This result can also be
inferred from the integrality gap of  for the ATSP tour
problem~\cite{CGK06}, but our construction is relatively simpler.

For a fixed integer , consider the directed graph 
defined below (and illustrated in \lref[Figure]{fig:gap}). The vertices of
 are ;
the arcs are as follows:
\begin{OneLiners}
\item , each with cost 1,
\item , each with cost 0,
\item ,
  each with cost 1,
\item and , each with cost 0.
\end{OneLiners}
Let  denote the ATSPP instance obtained from the metric completion
of .


\begin{figure}
 \centering
\scalebox{0.65}{\psfig{figure=gap_2-thisone.pdf}}
  \caption{The graph  with . The solid arcs have cost 1 and
    the dashed arcs have cost 0.}\label{fig:gap}
\end{figure}

\begin{lemma} \label{lem:intgap}
  The integrality gap of the LP~\ref{eq:lp} on the instance  is
  at least .
\end{lemma}

\begin{proof}
  It is easy to verify that assigning  to each arc that originally
  appeared in  is a valid LP solution. Indeed, the degree constraints are immediate,
  and there are two edge-disjoint paths from  to every other node in
   (so there must be at least 2 arcs exiting any subset containing )
  so the cut constraints are also satisfied. The total cost of
  this LP solution is .

  On the other hand, we claim that the cost of any Hamiltonian -
  path in , which corresponds to a spanning - walk  in
  , is at least . This shows an integrality gap of
  .

  To lower-bound the length of any spanning - walk, we first argue
  that the walk  can avoid using at most one of the unit cost arcs
  of the form  or . Indeed, any - walk
  must use arcs  for every . Similarly, every
  - walk must use all arcs of the form . One of
   and  is visited before the other, so either all of the
   arcs or all of the  arcs are used by
  . Now suppose, without loss of generality, that  does not use
  the arcs  and  for .  Every
  - walk uses arc  and every 
  walk uses arc . Since one of  or  must be
  visited by  before the other, then  cannot avoid both
   and  which contradicts our assumption.

  Thus,  must use all but at most one of the  unit cost arcs
  in . Moreover,  must also use one of the arcs exiting  and one
  of the arcs entering , so the cost of  is at least .
  (In fact, the walk

 is of length exactly , so this argument is tight.)
\end{proof}


\section{Conclusion}



In this paper we showed that the integrality gap for ATSPP is .
We also show that a constant integrality gap bound follows
from the form of Goddyn's conjecture used in \cite{GS11}
to get an analogous ATSP integrality gap bound.
We also showed a simpler construction achieving a lower
bound of  for the subtour elimination LP.
One of the main open questions following this work is to
show a more general reduction: does an  integrality
gap bound for ATSP directly imply an  integrality gap bound
for ATSPP without any further assumptions?

















\subsubsection*{Acknowledgments.} We thank V.~Nagarajan for enlightening
discussions in the early stages of this project. Z.F.\ and A.G.\ also
thank A.~Vetta and M.~Singh for their generous hospitality. Part of this work was done when
Z.F.\ was a postdoctoral fellow in the Department of Combinatorics and Optimization at the University of
Waterloo, when A.G.\ was visiting the IEOR Department at Columbia
University, and  when M.S.\ was at McGill University.
Finally, we thank anonymous reviewers for many helpful comments and the suggestion to obtain better bounds through thin tree conjectures.

\bibliographystyle{abbrv}
{\small \bibliography{atsp}}








\end{document}
