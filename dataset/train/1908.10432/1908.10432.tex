\documentclass{article}
\usepackage{amsmath,graphicx,mlspconf}
\usepackage{comment}
\usepackage{amssymb}
\usepackage{color}


\copyrightnotice{U.S.\ Government work not protected by U.S.\ copyright}

\copyrightnotice{978-1-7281-0824-7/19/\31.00 {\copyright}2019 European Union}

\copyrightnotice{978-1-7281-0824-7/19/\\underline{\boldsymbol{T}} \in \mathbb{R}^{N \times M \times K}\underline{\boldsymbol{T}}_{:,j,k} N\underline{\boldsymbol{T}}\boldsymbol{T}_1\boldsymbol{T}_2\boldsymbol{T}_3\underline{\boldsymbol{T}}\circ\boldsymbol{\underline{X}}\boldsymbol{U}\boldsymbol{\underline{X}}\times_n \boldsymbol{U}n^{\text{th}}
\boldsymbol{\underline{Y}}=\boldsymbol{\underline{X}}\times_n \boldsymbol{U} \leftrightarrow \boldsymbol{Y}_n=\boldsymbol{U}\boldsymbol{X}_n, 
\boldsymbol{X}_n\boldsymbol{Y}_n\boldsymbol{\underline{X}}\boldsymbol{\underline{Y}}\boldsymbol{u}\underline{\boldsymbol{X}}\boldsymbol{u}KRR<<K\gammaW[l]Hann_l[l]Hamm_m[l]Hamm_n[l]Hann_l[l]\delta[l]W[n]\frac{\surd{\pi\sigma}}{2|m|} e^{\frac{-\pi^2\sigma l^2}{4n^2}}\frac{cosh^{-2\gamma}} {\sum_{l}^{}{cosh^{-2\gamma }l}}W[l+n]W[l-n]\delta[l]\boldsymbol{X}\in \mathbb{R}^{T\times K}K\underline{\boldsymbol{X}}\in \mathbb{R}^{T\times F \times K}\underline{\boldsymbol{X}}R\boldsymbol{a}_r\boldsymbol{b}_r\boldsymbol{c}_r\boldsymbol{a}_r\boldsymbol{A}\boldsymbol{B}\boldsymbol{C}\boldsymbol{A}\boldsymbol{B}\boldsymbol{C}R\underline{\boldsymbol{X}}RRRR\boldsymbol{z}\underline{\boldsymbol{X}}\boldsymbol{C}\boldsymbol{z}\underline{\boldsymbol{X}}\boldsymbol{C}\Tilde{\boldsymbol{z}}=\underset{\Tilde{\boldsymbol{z}}}{\text{argmin}}\|\boldsymbol{z}-\boldsymbol{C}\Tilde{\boldsymbol{z}}\|_2^2\Tilde{\boldsymbol{z}}(\boldsymbol{C}^T\boldsymbol{C})^{-1}\boldsymbol{C}^T\boldsymbol{z}KR\boldsymbol{P}=(\boldsymbol{C}^T\boldsymbol{C})^{-1}\boldsymbol{C}^T\small{\Tilde{\boldsymbol{\underline{X}}}=\underline{\boldsymbol{X}}\times_3 P}\Tilde{\boldsymbol{\underline{X}}}\boldsymbol{\underline{X}}\boldsymbol{P}_{r,:}r^{\text{th}}\boldsymbol{P} \boldsymbol{P}\boldsymbol{P}KRK>>R\boldsymbol{P}=(\boldsymbol{C}^T\boldsymbol{C})^{-1}\boldsymbol{C}^T\boldsymbol{P}\Tilde{\boldsymbol{\underline{X}}}SoftMax2\times2\alpha=0.8\beta=0.02Fs/4Fs=256\|.\|_F15[0.2,\ 0.3]2\times23\times3n2\times23\times3FS2\times23\times33\times3\alpha\beta\delta\thetaaverage \ powermeanstandard \ deviation>85\%89.63\%86.17\%$). Tensor considers all of the channels together and is capable of capturing temporal  and  spectral correlations in  addition to dependencies of different channels over its third way. While PCA works on each TF image separately and it is prone to ignoring the correlations between different channels. Moreover, as Fig. \ref{res_all} shows, the tensor-based dimensionality reduction (TF-tenosr-CNN) framework, due to its capability of reducing the redundancies and handling artifacts, outperforms the TF-CNN framework
without dimensionality reduction.    \\
\indent Comparing our result (89\% of accuracy) with previous studies (less than 86\% accuracy), our algorithm has improved the results of cross-patient seizures detection in CHB-MIT dataset \cite{Tzallas,Thodoroff}.


\vspace{-3mm}
\section{Conclusion}
\vspace{-5pt}
In this study, we proposed a new tensor-based framework to enhance the classification accuracy and efficiency of the deep learning models, specifically convolutional neural networks (CNNs), for EEG signals. We proposed a tensor decomposition-based dimentionality reduction of time-frequency (TF) inputs of CNN model to improve its performance in terms of storage space and running time. Our proposed method transforms a large set of slices of the input tensor to a concise set of \emph{super-slices}, which is capable of not only handling the artifacts and redundancies of the EEG data but also reducing the dimension of the CNNs training inputs. We also considered different TF approaches and evaluated their performances to provide a comprehensive comparison of different TF methods for this classification problem. We implemented our proposed method on a publicly available dataset (CHB-MIT). Our results showed the superiority of our scheme compared to the state-of-the-art methods and recent studies. 
\vspace{-9pt}
\begin{figure}[t]
\centerline{\includegraphics[scale=0.8]{res-all.pdf}}
\vspace{-3mm}
\caption{\small{Comparison of the classification accuracy of cross-patient seizure detection on CHB-MIT EEG dataset. Each box plot shows 10 iterations of 10 cross validation of the predictive model for the associated method.}}
\vspace{-3mm}
\label{res_all}
\end{figure}








\begin{small}


\begin{thebibliography}{00}
\vspace{-2mm}

\bibitem{Game}
Taherisadr, M. and Dehzangi, O., 2019. EEG-Based Driver Distraction Detection via Game-Theoretic-Based Channel Selection. In Advances in Body Area Networks I (pp. 93-105). Springer, Cham.
\vspace{-5pt}

\bibitem{Feature}
Dehzangi, O. and Taherisadr, M., 2019. EEG Based Driver Inattention Identification via Feature Profiling and Dimensionality Reduction. In Advances in Body Area Networks I (pp. 107-121). Springer, Cham.
\vspace{-5pt}



\bibitem{spectro_intro}
Tsipouras, M.G., Spectral information of EEG signals with respect to epilepsy classification. EURASIP Journal on Advances in Signal Processing, 2019(1), p.10.
\vspace{-5pt}

\bibitem{Wavelet_Intro}
Taherisadr, Mojtaba, Omid Dehzangi, and Hossein Parsaei. "Single channel EEG artifact identification using two-dimensional multi-resolution analysis." Sensors 17, no. 12 (2017): 2895.
\vspace{-5pt}


\bibitem{emd_Intro}
Gupta, Vipin, Anurag Nishad, and Ram Bilas Pachori. "Focal EEG signal detection based on constant-bandwidth TQWT filter-banks." 2018 IEEE International Conference on Bioinformatics and Biomedicine (BIBM). IEEE, 2018.
\vspace{-5pt}


\bibitem{HMM_Intro}
Mumtaz, Maheen, Mubashira Afzal, and Aleem Mushtaq. "Sensorimotor Cortex EEG signal classification using Hidden Markov Models and Wavelet Decomposition." 2018 IEEE International Symposium on Signal Processing and Information Technology (ISSPIT). IEEE, 2018.
\vspace{-5pt}


\bibitem{SVM_Intro}
Manshouri, Negin, and Temel Kayikcioglu. "A Comprehensive Analysis of 2D \& 3D Video Watching of EEG Signals by Increasing PLSR and SVM Classification Results." arXiv preprint arXiv:1903.05636 (2019).
\vspace{-5pt}


\bibitem{NN_Intro}
Yasmeen, Shaguftha, and Maya V. Karki. "Neural network classification of EEG signal for the detection of seizure." 2017 2nd IEEE International Conference on Recent Trends in Electronics, Information \& Communication Technology (RTEICT). IEEE, 2017.
\vspace{-5pt}

\bibitem{eye_blink}
Dehzangi, O., Melville, A. and Taherisadr, M., 2019. Automatic EEG Blink Detection Using Dynamic Time Warping Score Clustering. In Advances in Body Area Networks I (pp. 49-60). Springer, Cham.
\vspace{-5pt}

\bibitem{CNN_Intro}
Craik, Alexander, Yongtian He, and Jose Luis Pepe Contreras-Vidal. "Deep learning for Electroencephalogram (EEG) classification tasks: A review." Journal of neural engineering (2019).
\vspace{-14pt}

\bibitem{CNN1_Intro}
Yannick, Roy, et al. "Deep learning-based electroencephalography analysis: a systematic review." arXiv preprint arXiv:1901.05498 (2019).
\vspace{-5pt}


\bibitem{tensor}
Cong F, Lin QH, Kuang LD, Gong XF, Astikainen P, Ristaniemi T. Tensor decomposition of EEG signals: a brief review. Journal of neuroscience methods. 2015 Jun 15;248.
\vspace{-5pt}

\bibitem{tensor_noise}
Triantafyllopoulos D, Megalooikonomou V. ''Eye blink artifact removal in EEG using tensor decomposition." InIFIP International Conference on Artificial Intelligence Applications and Innovations 2014 Sep 19 (pp. 155-164). Springer, Berlin.
\vspace{-5pt}









\bibitem{joneidi16}
Joneidi, M., Ahmadi, P., Sadeghi, M. and Rahnavard, N., 2016. ''Union of low-rank subspaces detector". IET Signal Processing, 10(1), pp.55-62.
\vspace{-3pt}
\bibitem{tari19}
Salimitari, M., Joneidi, M. and Chatterjee, M., 2019. AI-enabled Blockchain: An Outlier-aware Consensus Protocol for Blockchain-based IoT Networks. IEEE Global Communications Conference (GLOBECOM) 2019. arXiv preprint is available online, arXiv:1906.08177.
\bibitem{ashk16}
Esmaeili, A., Amini, A. and Marvasti, F., 2016, December. Fast methods for recovering sparse parameters in linear low rank models. In 2016 IEEE Global Conference on Signal and Information Processing (GlobalSIP) (pp. 1403-1407). IEEE.
\vspace{-5pt}

\bibitem{Victor}
Boashash, Boualem, and Victor Sucic. "Resolution measure criteria for the objective assessment of the performance of quadratic time-frequency distributions." IEEE Transactions on Signal Processing 51.5 (2003).
\vspace{-5pt}


\bibitem{ava}
Pedersen, Flemming. "Joint time frequency analysis in digital signal processing." (1997).
\vspace{-5pt}



\bibitem{Boashash.}
Boashash, Boualem. Time-frequency signal analysis and processing: a comprehensive reference. Academic Press, 2015.
\vspace{-4pt}

\bibitem{kolda}
Kolda TG, Bader BW. Tensor decompositions and applications. SIAM review. 2009 Aug 5;51(3):455-500.
\vspace{-4pt}

\bibitem{karpathy2016cs231n}
Karpathy, Andrej. "Cs231n: Convolutional neural networks for visual recognition." Neural networks 1 (2016).
\vspace{-3pt}


\bibitem{zhong2014sensor}
Zhong, Yu, and Yunbin Deng. "Sensor orientation invariant mobile gait biometrics." Biometrics (IJCB), 2014 IEEE International Joint Conference on. IEEE, 2014.
\vspace{-3pt}




\bibitem{BoualemB}
Boashash, Boualem. Time-frequency signal analysis and processing: a comprehensive reference. Academic Press, 2015.
\vspace{-3pt}













\bibitem{changal}
Dehzangi, O., Taherisadr, M. and ChangalVala, R., 2017. IMU-based gait recognition using convolutional neural networks and multi-sensor fusion. Sensors, 17(12), p.2735.
\vspace{-3pt}




\bibitem{Subasi}
Subasi, Abdulhamit. "Automatic recognition of alertness level from EEG by using neural network and Wavelet coefficients." Expert systems with applications 28.4 (2005): 701-711.
\vspace{-3pt}

\bibitem{hinton}
LeCun, Y., Bengio, Y. and Hinton, G., 2015. Deep learning. nature, 521(7553), p.436.
\vspace{-3pt}


\bibitem{shoeb}
Goldberger AL, Amaral LA, Glass L, Hausdorff JM, Ivanov PC, Mark RG, Mietus JE, Moody GB, Peng CK, Stanley HE. PhysioBank, PhysioToolkit, and PhysioNet: components of a new research resource for complex physiologic signals. Circulation. 2000 Jun 13;101(23):e215-20.
\vspace{-3pt}


\bibitem{Tzallas}
Tzallas AT, Tsipouras MG, Tsalikakis DG, Karvounis EC, Astrakas L, Konitsiotis S, Tzaphlidou M. Automated epileptic seizure detection methods: a review study. InEpilepsy-histological, electroencephalographic and psychological aspects 2012 Feb 29. IntechOpen.
\vspace{-3pt}

\bibitem{Thodoroff}
Thodoroff P, Pineau J, Lim A. Learning robust features using deep learning for automatic seizure detection. InMachine learning for healthcare conference 2016 Dec 10 (pp. 178-190).
\vspace{-14pt}










\bibitem{kuncheva}
Kuncheva, Ludmila I. Combining pattern classifiers: methods and algorithms. John Wiley , Sons, 2004.
\vspace{-5pt}


\end{thebibliography}
\end{small}

\bibliographystyle{IEEEbib}
\bibliography{refs}

\end{document}
