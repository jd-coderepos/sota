
\documentclass[11pt]{article}

\usepackage{tabularx,booktabs,multirow,delarray,array}
\usepackage{graphicx,amssymb,amsmath,amssymb}
\usepackage{latexsym}
\usepackage{wrapfig}
\usepackage{enumerate}
\usepackage{todonotes}

\def\qed{\hbox{\rlap{}}}
\def\calP{\mathcal{P}}
\def\calT{\mathcal{T}}
\def\calE{\mathcal{E}}
\def\calR{\mathcal{R}}
\def\calF{\mathcal{F}}

\newtheorem{propo}{Proposition}
\newtheorem{observation}{Observation}
\newtheorem{cor}{Corollary}
\newtheorem{theorem}{Theorem}[section]
\newtheorem{lemma}{Lemma}[section]
\newenvironment{proof}{\par\noindent{\bf Proof:}}{\mbox{}\hfill\\}

\setlength{\topmargin}{-0.25in}
\setlength{\textwidth}{6.50in}
\setlength{\oddsidemargin}{0.0in}
\setlength{\textheight}{8.7in}

\newcommand{\ignore}[1]{ }

\begin{document}
\title{A Polynomial Time Algorithm to Compute an Approximate Weighted Shortest Path}

\author{R. Inkulu\thanks{R.~Inkulu's research was supported in part by NBHM under grant 248(17)2014-R\&D-II/1049.}\footnote{
  Department of Computer Science \& Engineering; IIT Guwahati, India; \texttt{rinkulu@iitg.ac.in}
}
\and
Sanjiv Kapoor\footnote{
  Department of Computer Science; IIT Chicago, USA; \texttt{kapoor@iit.edu}
}}
\date{}
\maketitle

\begin{abstract}
We devise a polynomial-time approximation scheme for the classical geometric problem of finding an approximate short path amid weighted regions.
In this problem, a triangulated region  comprising of  vertices, a positive weight associated with each triangle, and two points  and  that belong to  are given as the input.
The objective is to find a path whose cost is at most OPT where OPT is the cost of an optimal path between  and .
Our algorithm initiates a discretized-Dijkstra wavefront from source  and progresses the wavefront till it strikes .
This result is about a cubic factor (in ) improvement over the Mitchell and Papadimitriou '91 result \cite{journals/jacm/MitchellP91}, which is the only known polynomial time algorithm for this problem to date.
Further, with polynomial time preprocessing of , a map is computed which allows answering approximate weighted shortest path queries in polynomial time.
\end{abstract}

\section{Introduction}
\label{sect:intro}

The problem of computing a shortest path in polygonal subdivisions is important and well-studied due to its applications in geographic information systems, VLSI design, robot motion planning, etc.
A survey of various shortest path problems and algorithms may be found in Mitchell \cite{hb/cg/Mitch98}.
In this paper, we devise an algorithm for the {\it weighted shortest path problem} \cite{journals/jacm/MitchellP91}: given a triangulation  with  faces, each face associated with a positive weight, find a path between two input points  and  (both belonging to ) so that the path has minimum cost among all possible paths joining  and  that lie on . 
The cost of any path  is the sum of costs of all line segments in , whereas the cost of a line segment is its Euclidean length multiplied by the weight of the face on which it lies.
The weighted shortest path problem helps in modeling region specific constraints in planning motion.

To compare the time complexities of various algorithms in the literature, we use the following notation -
: number of vertices defining ; : length of the longest edge bounding any face of ; : maximum coordinate value used in describing ; : maximum non-infinite weight associated with any triangle; : minimum weight associated with any triangle; : minimum among the internal face angles of ; and, : ratio of  to .
(Note that the same notation is used in later parts of the paper as well.)

Mitchell and Papadimitriou \cite{journals/jacm/MitchellP91} presented an algorithm that finds an approximate weighted shortest path in  time.
Their algorithm essentially builds a shortest path map for  by progressing continuous-Dijkstra wavefront in  using the Snell's laws of refraction.
By introducing  equi-spaced Steiner points on each edge of  and building a graph spanner over these points, Mata and Mitchell \cite{conf/compgeom/MataM97} devised a preprocessing algorithm to construct a graph spanner in  time, where , on which -approximate weighted shortest path queries are performed.
Lanthier et~al. \cite{journals/algorithmica/LanthierMS01} independently devised an  time approximation algorithm with an additive error of  by choosing .
Instead of uniform discretization (as in \cite{conf/compgeom/MataM97}), Aleksandrov et~al. \cite{conf/swat/AleksandrovLMS98,journals/jacm/AleksandrovMS05} used logarithmic discretization and devised an  time approximation algorithm.
Sun and Reif \cite{journals/jal/SunR06} provided an approximation algorithm, popularly known as BUSHWHACK, with time complexity .
Their algorithm dynamically maintains for each Steiner point , a small set of incident edges of  that may contribute to an approximate weighted shortest path from  to .
More recently, Cheng et al. \cite{conf/soda/ChengJV15} devised an approximation algorithm that takes  time, where  is the smallest integer such that the sum of the  smallest angles in  is at least .
The query version of this problem is addressed in \cite{conf/swat/AleksandrovLMS98,journals/jacm/AleksandrovMS05,journals/jal/SunR06,journals/jacm/MitchellP91,journals/siamcomp/ChengNVW10,journals/dcg/AleksandrovDGMNS10}.
Further, algorithms in Cheng et~al. \cite{journals/siamcomp/ChengNVW10} handle the case of measuring the cost of path length in each face with an asymmetric convex distance function. 


\subsubsection*{Our contribution}

The time complexities of each of the above mentioned solutions, except for \cite{journals/jacm/MitchellP91}, are polynomial in  as well as in parameters such as  and .
Hence, strictly speaking, these algorithms are not polynomial.
Like \cite{journals/jacm/MitchellP91}, this paper devises an algorithm that is polynomial in time complexity.
The time complexity of our algorithm is , which is about a cubic factor improvement from \cite{journals/jacm/MitchellP91}.
As established in \cite{journals/jacm/MitchellP91}, there are   events that need to be handled in order to find the interaction of shortest path map with the .
Our algorithm takes first steps to provide a solution that is sub-quadratic in the number of events.

This result uses several of the characterizations from \cite{journals/jacm/MitchellP91} in order to design a simple and more efficient algorithm.
Every ray is a simple path in .
Our approach discretizes the wavefront in the continuous-Dijkstra's approach by a set  of rays whose origin is source .
These rays are distributed uniformly around .
As the discrete wavefront propagates, a subset of the rays in  are progressed (traced) further while following the Snell's laws of refraction.
Each of these subsets of rays is guided by two extreme rays from that subset, i.e., all the rays in that subset lie between these two special rays. 
Essentially, each such subset represents a section of the wavefront and is called a {\em bundle}.
For any vertex  in , whenever such a section of the wavefront strikes a vertex , we initiate another discrete-wavefront (set of rays) from .
We continue doing this until the wavefront (approximately) strikes .
In summary our contributions are:
\begin{enumerate}
\item A discretized approach to propagating wavefronts using bundles.
\item
An algorithm that computes an -approximate weighted shortest path from  to  in  time.
\item
Further, we preprocess  in  time to compute a data structure for answering single-source approximate weighted shortest path queries in  time.
\end{enumerate}

Section~\ref{sect:prelim} lists relevant propositions from \cite{journals/jacm/MitchellP91} and \cite{journals/jal/SunR06}, and defines terminology required to describe the algorithm. 
Section~\ref{sect:algooutline} outlines the algorithm while introducing few structures used in the algorithm.
In Section~\ref{sect:boundrays}, we bound the number of rays.
The details of the algorithm are provided in Section~\ref{sect:algodetails}.
Section~\ref{sect:interpol} describes an interpolation scheme to improve the time complexity of the algorithm. 
Section~\ref{sect:analysis} argues for the correctness and analyzes the time complexity of the algorithm.
And, the conclusions are given in Section~\ref{sect:conclu}.

\section{Preliminaries}
\label{sect:prelim}

We define the problem using the terminology from \cite{journals/jacm/MitchellP91}.

We assume a planar subdivision , that is polygonal and specified by triangular faces.
Each face  has a weight  associated with it. 
We denote the weight of a face  (resp. edge ) with  (resp. ).
For an edge  shared by faces  and , the weight of  is defined as .

A {\it path} is a continuous image of an interval, say , in the plane.
A {\it geodesic path} is a path that is locally optimal and cannot, therefore, be shortened by slight perturbations.
An {\it optimal path} is a geodesic path that is globally optimal.
The general form of a weighted geodesic path is a simple (that is, not self-intersecting) piecewise linear path that goes through zero or more vertices while possibly crossing a zero or more edges.
The Euclidean length of a line segment  is denoted by .
Let  be a geodesic path with line segments  such that  lies on face , for every  in ; then the {\it weighted Euclidean distance} (also termed the {\it cost}) of path  is defined as the .\\ \\
{\bf Weighted Shortest Path problem:}
{\em Given a finite triangulation  in the plane with two points  (source) and  (destination) located on , an assignment of positive integral weights to faces of , and an error tolerance , the {\it approximate weighted shortest path} problem is to determine a path  from  to  that lies on  such that the cost of  is at most  times the cost of an optimal path from  to .}\\

We assume that all input parameters of the problem are specified by integers.
In particular, all vertices have non-negative integer coordinates.
Further, we assume that  and  are two vertices of .
A {\it ray} in our algorithm is a piecewise linear simple path in  such that endpoints of each of the line segments that it contains lie on the edges of .
Let  be the first face traversed by a ray  and let  be the first face traversed by a ray .
The angle between  and  is defined as the angle between the vectors induced by  and .
Unless specified otherwise, all angles are acute.
Let ,  and  be three unit vectors that originate from a point  in  such that the vectors  lie in the same half-plane defined by the line induced by .
The {\it cone}  is the set comprising of all the points that are positive linear combinations of  and .

\subsection*{A few facts from the literature} 

First we list few propositions, definitions and descriptions from Mitchell and Papadimitriou \cite{journals/jacm/MitchellP91} tailored for our purpose.

A sequence of edge-adjacent faces is a list, , of two or more faces such that, for every , face  shares edge  with face .
Further, the corresponding sequence of edges  is referred as an {\it edge sequence}.
When a geodesic path  crosses edges in  in the order specified by  and without passing through any vertex, then  is the {\it edge sequence of path }.
A geodesic path  in , is termed a {\it ray}  as it behaves similar to a  ray of light.
Further, when the geodesic path is extended to add a  point  in , by the  line segment , the ray  is said to be {\it traced} (or, {\it progressed}) to .
Any point or line segment in  from which at least one ray is initiated is termed a {\it source}.

	\begin{wrapfigure}{r}{0.5\textwidth}
	\centering
	\begin{minipage}[t]{\linewidth}
	\centering
	\includegraphics[totalheight=0.65in]{figs/refract.pdf}
	\vspace{-0.15in}
	\caption{\footnotesize Illustrating refraction}
	\label{fig:refraction}
	\end{minipage}
	\end{wrapfigure}

Let  and  be two faces with shared edge .
(See Fig. \ref{fig:refraction}.)
Let  be two successive line segments along a ray  with  lying on face  and  lying on face  with point .
Let  be a vector entering  and containing  and let  be a vector with origin  and containing .
Also, let  be a vector normal to edge , passing through point  to some point in face . 
The angle  between  and  is known as the {\it angle of incidence} of  onto .
And, the angle  between  and  is known as the {\it angle of refraction} of  from .

	\begin{wrapfigure}{r}{0.5\textwidth}
	\centering
	\begin{minipage}[t]{\linewidth}
	\centering
	\includegraphics[totalheight=0.7in]{figs/critrefr.pdf}
	\caption{\footnotesize Illustrating critical reflection}
	\label{fig:critrefl}
	\end{minipage}
	\end{wrapfigure}

When , the {\it critical angle of }, denoted by  is .
(When  and  are understood from the context, the critical angle of  is also denoted by .)
(See Fig. \ref{fig:critrefl}.)
If , then the angles  and  are related by Snell's law of refraction with .
Since there does not exist a geodesic weighted shortest path with , we only need to consider the case in which  equals to . 
Let  be the point of incidence of ray  from face  onto  with angle of incidence .
Also, let  be a vector normal to edge , passing through point  to some point in face .
Then the geodesic path travels along  for some positive distance before exiting edge  back into face , say at a point  located in the interior of , while making an angle  with . 
We say that the path  is {\it critically reflected} by edge  and the line segment  is termed a {\it critical segment} of path  on .
Sometimes, we also say  is a critical segment corresponding to the critcial incidence of a ray at  from face .
The point  (the closer of the two points  to ) is known as a {\it critical point of entry} of path  to edge  and  is known as the corresponding {\it critical point of exit} of path  from edge .

The following propositions from the literature are useful for our algorithm.

\begin{propo}[Lemma~3.7, \cite{journals/jacm/MitchellP91}]
\label{prop:betwcrit}
Let  be a geodesic path.
Then either (i) between any two consecutive vertices on , there is at most one critical point of entry to an edge , and at most one critical point of exit from an edge  (possibly equal to ); or
(ii) the path  can be modified in such a way that case (i) holds without altering the length of the path. 
\end{propo}

Let  and  be two faces with shared edge .
For any point , a {\it locally -free path} strikes  from the exterior of face  and is locally optimal.

\begin{propo}[Lemma~7.1, \cite{journals/jacm/MitchellP91}]
\label{prop:edgeseqlen}
For a face  of , let  be a shortest locally -free path.
Let  be a sub-path of  such that  goes through no vertices or critical points.
Then,  can cross an edge  at most  times.
Thus, in particular, the cardinality of any edge sequence of path  is .
\end{propo}

\begin{propo}[Lemma~7.4, \cite{journals/jacm/MitchellP91}]
\label{prop:numcritsrc}
There are at most  critical points of entry on any given edge .
\end{propo}
Finally we have a  proposition that provides us the {\it non-crossing property of (weighted) shortest paths}.
\begin{propo}[Lemma~1, \cite{journals/jal/SunR06}]
\label{prop:noncrossing}
Any two geodesic shortest paths that originate from the same point in  cannot intersect in the interior of any weighted region of .
\end{propo}

\section{Algorithm outline}
\label{sect:algooutline}

In our algorithm, we progress a discretized-Dijkstra wavefront, as the traditional approach of
progressing a continuous-Dijkstra wavefront results in a complicated algorithm, 
This wavefront is defined and expanded using rays.
Every {\it ray} is a geodesic path in  from its point of origin to a point on the wavefront.
The wavefront that defines the locus of points at weighted Euclidean distance  from the source is identified by points on the the rays that are at weighted Euclidean distance  from the source.
Note that geodesic paths in  are piece-wise linear.
We {\it initiate} a set  of rays whose origin is .
The rays in  are ordered according to their counterclockwise angle with the positive -axis and are uniformly distributed around .
These rays together describe  a {\it (discrete) wavefront} initiated at .
Let  and  be two rays in . Two rays are termed {\it successive} when
they are adjacent in the ordering of rays around any point from which those rays are initiated.
The number of rays in  is defined by the angle  between successive rays; by expressing the value of  in terms of  later (in Section~\ref{sect:boundrays}), our algorithm ensures an -approximation.
When the wavefront strikes any vertex , an ordered set  of rays are initiated from , unless vertex  is the destination  itself.

\subsection{Types of rays}

We next discuss the types of rays that form part of our wavefront.
Only a subset of the  rays that have been initiated are  used in the propagation of the wavefront; a ray that is considered for propagation is said to have been {\it traced}.
We propagate the rays as described below:
let  be a common edge between faces  and .
Consider a ray  that is traced along face  and suppose it strikes  at a point .

If the angle of incidence of  onto  is less than , then  refracts onto face  with the angle of refraction defined by Snell's law of refraction.
The case in which the angle of incidence of  onto  is greater than  occurs only when .
In this case, we do not propagate  further as  would not be part of a weighted shortest path to .


	\begin{wrapfigure}{r}{0.5\textwidth}
	\centering
	\begin{minipage}[t]{\linewidth}
	\centering
	\includegraphics[totalheight=0.8in]{figs/critrefl.pdf}
	\vspace{-0.15in}
	\caption{\footnotesize Illustrating critical reflection}
	\label{fig:critreflequispaced}
	\end{minipage}
	\end{wrapfigure}

If the angle of incidence of  onto  is equal to , then a weighted shortest path that uses the ray  propagates along  for a positive distance before critically reflecting back into face  itself. 
Let  be the critical segment corresponding to this critical incidence of  onto  where  is the point of incidence of  on .
Since a weighted shortest path can be reflected back from any point on the critical segment  of , 
our algorithm initiates rays from a discrete set of evenly spaced points on .
The number and position of points from which these rays are generated is again a function of .
Let  be a vector normal to edge , passing through point  to some point in face .
These rays critically reflect from  back into face  while making an angle  with . 
(See Fig. \ref{fig:critreflequispaced}.)
We let  be the ordered set of rays that originate from , and are ordered by the distance from . 
Two rays  and  in  are {\it successive} whenever there is no ray in  that occurs between  and  in the linear ordering along .

	\begin{wrapfigure}{r}{0.5\textwidth}
	\centering
	\begin{minipage}[t]{\linewidth}
	\centering
	\includegraphics[totalheight=1.3in]{figs/initcritsrc.pdf}
	\caption{\footnotesize Illustrating rays initiated from a critical source  and the corresponding sibling pair }
	\label{fig:initcritsrc}
	\end{minipage}
	\end{wrapfigure}

We next introduce an additional category of rays. To account for the divergence of rays and to ensure that
the wavefront is adequately represented, we further initiate an angle ordered set of {\it Steiner rays}, denoted by , from each critical point of entry  onto face . 
(See Fig. \ref{fig:initcritsrc}.)
Hence, a critical point of entry is also termed as a {\it critical source}.
These rays are motivated as follows: a pair of successive rays when traced along an edge sequence can diverge non-uniformly, with the divergence being large especially at angles close to critical angles.
To establish an approximation bound, instead of having a set of rays that is more dense from every source, we fill the gaps by generating Steiner rays from critical points of entries.
This helps in reducing the overall time complexity.
Let  and  be the endpoints of .
For any source , let  be successive rays in  such that  is refracted and  is critically reflected.
Let  (resp. ) be the point at which  (resp. ) is incident to .
Let  be the angle of refraction of .
Also, let  be a vector with origin at  and that makes an angle  with . 
As a result of discretization, there may not exist rays in  that intersect the cone .
Hence, to account for this region devoid of rays, an ordered set  of rays are initiated from  that lie in the cone , ordered by the angle each ray makes with respect to . 
Every ray in  is termed a {\it Steiner ray}.

To recapitulate, there are three sets  of rays that are used to define the discrete wavefront:
\begin{itemize}
\item \{ \hspace{0.005in}  \hspace{0.005in} \}
\item \{ \hspace{0.005in}  \hspace{0.005in}  is a critical segment obtained  when a ray   strikes an edge  at a critical angle\} 
\item \{ \hspace{0.005in}  \hspace{0.005in}  is a critical point of entry for a ray  for some \}
\end{itemize}


\subsection{Ray bundles}

We approximate the expansion of the continuous-Dijkstra wavefront by tracing rays in pairs (angle between the rays being less than ) so that each pair represents a section of the continuous-Dijkstra wavefront
that lies between the pair at corresponding distance  along the chosen pair of rays.
Appropriately chosen  pairs suffice to represent the behavior of this section.
The rays could be initiated from either of these: (i) a vertex , (ii) a critical source , or (iii) a critical segment .
Let  be one such source of rays.
We partition the set of rays with origin  as follows: 
Let  be a maximal set of successive rays in  such that all rays in  cross the same edge sequence  when traced from the source to  the current state of the discrete wavefront.
Then  is said to be a {\it ray bundle} of .
Furthermore,  is the edge sequence associated with .
Let  and  be two rays in  intersecting an edge  such that the line segment defined by the points of intersection of  and  with  intersects every other ray from  (when traced).
Then the rays  and  are extremal rays of the bundle and are termed as the {\it sibling pair} of ray bundle .
For any ray bundle , instead of tracing all the rays in , we trace only the sibling pair of .
This helps in reducing the number of event points, hence the time complexity.

\begin{wrapfigure}{r}{0.5\textwidth}
\begin{minipage}[t]{\linewidth}
\vspace{-20pt}
\hspace{-10pt}
\begin{center}
\includegraphics[totalheight=1.0in]{figs/bundsplit.pdf}
\end{center}
\vspace{-20pt}
\caption{\footnotesize Split of a ray bundle }
\label{fig:bundsplit}
\vspace{-10pt}
\end{minipage}
\end{wrapfigure}

The ray bundles and corresponding sibling pairs are updated as the rays progress further.
For the sibling pair  of ray bundle , when the sequence of edges intersected by  changes from the sequence of edges intersected by  for the first time, we need to split the bundle of rays into two and determine new sibling pairs.  
A binary search among the rays in  is used to trace the appropriate rays across the edge sequence of  and form new sibling pairs. 
Let  be a face  in .
(See Fig. \ref{fig:bundsplit}.)
Suppose rays  and  are siblings in a bundle  before they strike edge  of . 
However, when they are traced further, suppose  strikes edge  and  strikes edge .
At this instance of wavefront progression, the sequence of edges that are intersected by   and  differ for the first time, and hence  and  do not belong to the same bundle from there on.
Using binary search over the rays in , a pair of successive rays, say  and , are found to determine new sibling pairs:  and .
Further, the ray bundle  is {\it split} accordingly.
And, the wavefront is said to {\em strike} . 
The details of the algorithm to find rays  and  is described later.

The sets of rays in  and , where  is a critical point of entry and  is a critical segment, are handled similarly.
The rays in  have the characteristic that the paths of any two rays in  are parallel between any two successive edges in the edge sequence associated with the corresponding ray bundle.

\subsection{Tree of rays}

Let  be a vertex in . And let  be the set of rays initiated from . 
A ray may lead to the initiation of critical sources as described before.
Let  be the set of critical sources such that for any source  in , Steiner rays are initiated from  when the discrete wavefront that  originates at  strikes .
Further, let  be the sets of critical sources such that for every  and , critical source  is initiated due to the critical incidence of a Steiner ray from a source in .
We organize the sources in the set  into a {\it tree of rays}, denoted by .
Each node of  corresponds to a source in :
more specifically,  is the root node and every other node in  is a distinct critical source from . 
For any two nodes  in ,  is the parent of  in  if and only if the critical source , located on an edge , is initiated when a ray from  strikes  at the critical angle for edge  (and the point of critical incidence is ). 
The in-order traversal of the tree  provides a natural order on the set of rays. 
A subset  of rays are termed {\it successive} whenever rays in  are a contiguous subsequence of the ordered set of rays

\subsection{Ray Bundles and Sibling Pairs}
In order to partition the rays in
, where  is  a vertex, or in , where  is a critical segment,
we generalize the definitions of  sibling pair and ray bundles.

We consider the rays in . 
Let  be an edge sequence.
For any , we say  is a {\it suffix of edge sequence }. 
Two rays belonging to  are {\it siblings} whenever the edge sequence associated with one of them is a suffix of the edge sequence of the other. 
A maximal set  of successive rays initiated from the nodes of  that are siblings to each other is a {\it ray bundle}.
For two rays  belonging to a ray bundle  and points , the rays  and  are termed as the {\it sibling pair of } whenever the line segment  intersects every ray in .
A similar definition holds for rays in .
 
Therefore, there are two  kinds of sibling pairs possible:
Either (i) both the rays in a sibling pair originate from the same source or the
 sources of both the rays in a sibling pair belong to a tree of rays,  or
(iii) both the rays in a sibling pair originate from the same critical segment.
A ray bundle of the first category is denoted as {\it a ray bundle of }. 


Let  be a sibling pair of  rays in a bundle  in .
Let  be the respective origins of  and . 
Also, let  be the least common ancestor of  and  in .
We define  to be the {\em root} of .
The path  in tree  from  to  comprising of a  sequence of critical sources is the {\it critical ancestor path} of  with respect to .  
Note that the critical ancestor path of a ray is always defined with respect to a sibling pair.
Given that rays in the bundle  are successive rays,
nodes on the critical ancestor paths from  and  to 
in the subtree  provide all the rays in the bundle .

We next show a property of the rays in a bundle. We define two rays to be {\em pairwise divergent} during traversal of an edge sequence
 if they do not intersect each other during the traversal. A bundle of rays is {\em divergent} during traversal of 
if all rays in the bundle are pairwise divergent during the traversal.

\begin{lemma}
Let  be a ray bundle of  and let  be the edge sequence associated with rays in .
The rays in  are pairwise divergent whey they traverse across the edge sequence .
\end{lemma}
\begin{proof}
Let  be the root of a bundle. The rays in  that start from , termed , clearly diverge during the  traversal of  the sequence of edges , since divergence is preserved on refraction. Additionally, consider
rays in  that are generated from a critical source, , termed . Each pair of rays, a steiner ray  originating at  and a ray in , are pairwise divergent, by construction.
Similarly, every pair of ray in  is pairwise divergent.
These rays remain pairwise divergent  after refraction during propagation across edges in the sequence .
In general, let  be  a bundle of
rays that is divergent during traversal of a subsequence of . These rays are  generated from a vertex and/or  a set of critical sources. Suppose the
rays in the bundle  strike an edge  and refract, and also generate
a critical source  on . Then , where  is the set of Steiner rays generated from ,
is a set of rays with every pair of rays being pairwise divergent during traversal of  since  every  steiner ray in
 is pairwise divergent from each ray in .
Thus these rays will not intersect each other as they strike edges .
\end{proof}

The consequence of this Lemma is that  rays in a bundle  can be angle ordered. 
As described later, this ordering and the critical ancestor path of a ray will be useful in efficiently splitting a sibling pair of .




\section{Bounding the number of initiated and traced rays}
\label{sect:boundrays}

Before detailing the algorithm, we bound the number of initiated and traced rays in .
An approximate weighted shortest path from source  to  can be split into sub-paths such that each such sub-path goes between a pair of vertices. 
Further, each of these sub-paths are classified into following types:

\begin{itemize}
\item
a {\it Type-I path} from one vertex to another vertex that does not use any critical segment in between, and

\item
a {\it Type-II path} from one vertex to another vertex that critically reflects and uses at least one critical segment in-between.
\end{itemize}

We next bound the approximation achieved for both of these types of paths by establishing properties regarding the rays.
We show that given a Type-I (resp. Type-II) weighted shortest path  from a vertex, say , to another vertex, say , there exists a Type-I (resp. Type-II) path  using only rays in our discretization such that  closely approximates .
We first consider the divergence of successive rays that are within the same bundle. 
This divergence is due to refraction of rays. 


\begin{lemma}
\label{lem:depsilon}
Let  be two successive rays in a bundle in .
Also, let  and  be points on rays  and  respectively, at weighted Euclidean distance  from  and lying on the same face.
If the angle between rays  and  is upper bounded by , then the weighted Euclidean distance of the line segment joining  and  is upper bounded by . 
Here,  is a small constant (expressed in terms of input parameters).
\end{lemma}

\begin{wrapfigure}{r}{0.5\textwidth}
\centering
\begin{minipage}[b]{.4\textwidth}
\centering{\includegraphics[totalheight=1in]{figs/refrangles.pdf}}
\caption{\footnotesize Illustrating the angle of refractions of rays  and }
\label{fig:refrangles}
\end{minipage}
\end{wrapfigure}

\begin{proof}
We first show a  bound on the divergence between a pair of successive rays  in .
Let the sibling pair of rays  in  traverse across the (same sequence of) faces  with weights  respectively.
Let  be the angle between  and .
Let  be the edge sequence associated with this sibling pair, with  being the edge
adjacent to faces  and  for every .
Let  be the angles at which the ray  is incident on edges  respectively.
(See Fig. \ref{fig:refrangles}.)
Let  be the angles at which the ray  refracts at edges  respectively. 
Similarly, let  be the angles at which the ray  is incident on edges  respectively.
And, let  be the angles at which the ray  refracts at edges  respectively. 
(See Fig. \ref{fig:refrthm}.)
For every , if a critical angle exists for edge , assume that both  and  are less than that critical angle.
Further, for every , assume that  is a small positive number less than 
(we will justify this assumption later). We will first provide a bound on  in terms of .  \hfil\break
\hfil\break
For every integer , using Snell's law,
 
and 
 
Since ,  
Both  and  are assumed to be very small; thus we approximate , and  with the first term of the series expansion (an analysis with higher order  terms reveals no additional benefit) and thus:

Letting , the above leads to  


Let  be the critical angle corresponding to faces  and .
If  is close to the critical angle  then  grows unbounded. 
We thus restrict  to , where  is a constant defined as . 
And, we note that  is upper bounded by .
This ensures that every refracted ray is at an angle less than  with respect to the normal at edge .
Thus .
The error introduced due to this assumption will be bounded later.

\begin{figure}[h]
\begin{minipage}[t]{\linewidth}
\begin{center}
\includegraphics[totalheight=1.6in]{figs/refrthmnew.pdf}
\caption{\footnotesize
Illustrating the construction in proving Lemma \ref{lem:depsilon}
\scriptsize{}
(Except for , all the distances' shown in the figure are weighted Euclidean distances.)
\normalsize{}
}
\label{fig:refrthm}
\end{center}
\end{minipage}
\vspace*{-0.1in}
\end{figure}

To compute the weighted Euclidean distance between two points at weighted Eulcidean distance  from the source, first we restrict our attention to two points that are on successive rays in the same bundle at weighted Eulcidean distance  from the source and that lie on the same face. 
Let the two points,  and , both on the same face , be such that they
(i) lie on a pair of successive rays  and  in the same bundle in , and (ii) are at equal weighted Euclidean distance from  .
Let  be the point of incidence of  on an edge  (which is common to  and ) with an angle of incidence  and let it refract from  at an angle of refraction . 
Further, let the ray  be incident on  at  with an angle of incidence  and refract from  at an angle of refraction . 
(See Fig. \ref{fig:refrthm}.)
W.l.o.g., assume that the weighted Euclidean distance from vertex  to  is larger than the distance from  to . 
Consider an additional point  located on ray  such that both  and  are in the face , and the weighted Euclidean distance from  to either of these points is . 
Let  be the weighted Euclidean distance between  and .
Also, points  are located on rays  respectively such that  are in the region  with weight .
Further, we let  be the weighted Euclidean distance from  to either of these points and let  be the weighted Euclidean distance between  and .

We establish a bound on  for .
For every , we let .
We first show that if  then . 
Clearly, , i.e., when .

We first consider the case when points  and  are on the same face .
Let  be the point that is at weighted Euclidean distance  from .
Let  be the Euclidean distance between  and  along edge .
Also, let  be the weighted Euclidean distance between  and .

Now by triangle inequality

By assumption  and, by triangle inequality

The last inequality follows from the assumption that .

Next consider the weighted Eulcidean distance between  and , where  is at weighted Euclidean distance  from , and  is at weighted Euclidean distance  from .

using the small angle approximation of distance along a circular arc subtending a small angle.
Thus, since ,

Since ,  and , if we choose  then the weighted Euclidean distance  is bounded by .
\end{proof}

The following corollary to the above eliminates the restriction for points  and  being on the same face.
\begin{cor}
\label{lem:depsilonC}
Let  be two successive rays in a bundle  in .
Also, let  and  be two points on  and  respectively, at weighted Euclidean distance  from .
If the angle between rays  and  is upper bounded by , then the weighted Euclidean distance of the line segment joining  and  is upper bounded by . 
Here,  is a constant (expressed in terms of input parameters).
\end{cor}
\begin{proof}
Let  be the angle between  and .
Consider the locus of points at weighted Euclidean distance  from the root of  that lie in the region bounded by rays  and .
Let  be the sequence  of points of intersections of this locus with the edges of .
Consider the rays in the bundle  that strike points in .
Further, for , let  be the angle between rays that strike points  and  of . 
Then the Lemma~\ref{lem:depsilon} shows that the weighted Euclidean distance between  and  is  for every .
The overall bound follows by summing up the weighted Euclidean distances' between successive points in .
\end{proof}

Before we proceed further, for simplicity, we establish another restriction on the angle that two successive rays in  make at their source . To ensure that each triangle in our planar subdivision gets intersected by at least two rays, we further restrict angle between any two successive rays so that .
Here,  is the largest weighted Euclidean distance in ;  and  are respectively the lengths of edges with maximum and minimum Euclidean lengths in .
Thus this establishes another condition on , i.e.

We can now prove the following.

\begin{lemma}
\label{lem:errorX}
Let  be a vertex and let  be a point on an edge  in  such that the weighted Euclidean distance from  to  via a Type-1 ray in  bundle  is .
If  is not a vertex then there exists at least one traced ray in  that is incident onto a point  belonging to edge  such that . 
Here,  and  is the maximum value of the critical angle at any edge in . 
\end{lemma}

\begin{proof}
Consider the Type-1 ray  from  to , which is not traced and a ray  neighboring  in the angular order of rays originated at  is traced by the algorithm. 
(See Fig. \ref{fig:refrthmc}.)

\begin{wrapfigure}{r}{0.55\textwidth}
\centering
\includegraphics[totalheight=0.7in]{figs/refrthmnewC.pdf}
\caption{\footnotesize Illustrating the construction in proving the Lemma \ref{lem:errorX} }
\label{fig:refrthmc}
\end{wrapfigure}

\ignore {
\begin{center}
\includegraphics[totalheight=1.0in]{figs/refrthmnewCb.pdf}
}

Recall that the angle between any two successive rays is . 
We consider the weighted Euclidean distance of points on the segment .
If  then we are done.
If not, then  consider the point  on the ray ,  at distance  from .
First, assume that  and  are on a common face. 
Let  be the weight of the common face.
Thus using Lemma~\ref{lem:depsilon},  , since the two rays  and  belong to the same bundle. Moreover,
the Euclidean distance between  and  is upper bounded by .
Note that the angle of incidence  is less than or equal to the critical angle of . 
Hence, the  angle is greater than .
Let  be the projection of  onto .
Then .
Thus, 
and 
.

Next consider the case in which  and  lie on different faces. 
Let the line segment  be crossing faces .
(Here,  is the face containing .)
We assume w.l.o.g. that , so that rays  and  diverge as they progress across these faces.
By Corollary~\ref{lem:depsilonC}, the weighted Euclidean distance between  and  is upper bounded by ;
and the Euclidean distance between  and  is bounded by . 
Furthermore, the line segment  is incident to  at an angle larger than the critical angle of . 
By similar arguments as above, the bound follows.
\end{proof}

\begin{cor}
\label{cor:errorX}
Let  be a vertex and let  be a point on an edge  in  such that the weighted Euclidean distance from  to  via a Type-1 ray in  bundle  is .
If  is a vertex then there exists at least one traced ray in  that is incident onto a point  belonging to an edge  incident to  such that .
Here,  and  is the maximum value of the critical angle at any edge in . 
\end{cor}

The above Lemma and Corollary are used to bound the approximation of the distance from  to a point with the condition that the point lies between any two successive rays in the same  bundle .
A Type-I weighted shortest path from  to an arbitrary vertex , however, could use a critical source and hence use multiple bundles.
The following theorem consider these types of paths in ensuring an -approximation as well as the time complexity. 

\begin{lemma}
\label{lem:refrnoncritical}
If the angle between a pair of successive rays is as specified in Lemma~\ref{lem:depsilon}, then a Type-I weighted shortest path from any vertex  to another vertex  in  can be approximated to within a factor of  using rays in , where .
\end{lemma}
\begin{proof}
We first consider when an optimum weighted shortest path from  to  traverses a point  such that a weighted shortest path to  using the rays in  has a critical source  in-between.
Consider two such successive rays  and  that originate from  and that lie
in the same bundle  (refer Fig. \ref{fig:refrthmb}) such that
  is incident to a point  located on edge , at an angle less than the critical angle of 
and   is incident to a point  located on edge , at an angle equal to the critical angle of .  
Due to inequality (\ref{eqnepsilon}), the condition imposed on  in the assumption of this theorem, and the construction of the rays, such rays are guaranteed to exist in the same bundle.

Let  represent the distance between points  and  along path . 
Suppose an optimal weighted shortest path  intersects edge  between  and  at .
Let  refract at point  with  as the angle of refraction. 
Consider a path  that approximates : it uses ray  from  to  and then uses a weighted shortest path from  to some point . Then the modified path has length specified by  where  is a weighted shortest path from  to  in the discretized space.
By Lemma~\ref{lem:errorX},

with .
Further, .
Applying  Lemma~\ref{lem:errorX} again, 
 .
Thus .
Since the cardinality of any edge sequence of a path is  ,
repeating the above analysis for all edges that the ray might encounter,
results in an error factor of .
The case when no critical source is in-between is handled by  Lemma~\ref{lem:errorX}.
This completes the proof of Lemma~\ref{lem:refrnoncritical}.
\begin{figure}[h]
\begin{minipage}[t]{\linewidth}
\begin{center}
\includegraphics[totalheight=0.8in]{figs/refrthmb.pdf}
\caption{\footnotesize Illustrating the construction in proving Lemma~\ref{lem:refrcritical} }
\label{fig:refrthmb}
\end{center}
\end{minipage}
\vspace*{-0.1in}
\end{figure}
\end{proof}

We next consider Type-II paths.
\begin{lemma}
\label{lem:refrcritical}
If the angle between any pair of successive rays is as specified in Lemma~\ref{lem:depsilon} and rays generated from a critical segment are  Euclidean distance apart, then a Type-II weighted shortest path from any vertex  to another vertex  in  can be approximated to within a factor of  using rays in , where .
\end{lemma}
\begin{proof}
An optimal Type-II path  can be partitioned into two: a path  from  to  and a path  from  to .
Let the critical segment  have as a critical point of entry,  on edge .
We first show that a good approximation to the path  from  to  can be found.
Let  be a weighted shortest path that originates at a point  on  and strikes .
This path get critically reflected by edge ; hence, exits .
Since rays are generated  apart, there exist a sibling pair of rays that originate from two
points  and  adjacent to  and strike edges incident to .
These rays are parallel to  and a proof similar to Lemma~\ref{lem:errorX} shows that 
 is approximated to well within a factor .
Furthermore, the points  and  can be discovered by a binary search for the point , with an additive error of .
Finally, using Lemma~\ref{lem:refrnoncritical}, it is clear that the weighted shortest distance from  to  is approximated to within a factor of  where  is the optimal weighted Euclidean distance from  to . 
\end{proof}

We finally finish the analysis with the observation that an approximate weighted shortest path from  to  can be split into at most  approximate weighted shortest paths.
Thus we have the following Theorem to summarize.
\begin{theorem}
Let  be a weighted triangulated polygonal domain with vertex set . Let 
where 
and  is the maximum critical angle of any edge in .
A weighted shortest path from  to  in  can be approximated to within a factor of  using rays in .
\end{theorem}

\section{Details of the algorithm}
\label{sect:algodetails}

The algorithm is event-driven, where the events are considered by their weighted Euclidean distance from .
The algorithm starts with initiating a set  of rays that are uniformly distributed around .
With each traced ray , we save the point of its origin, points of refraction, critical points of entry and critical points of exit along the path traced by   i.e., as these event points occur in order.
As  is traced, we append the event points to the list associated with .
The various types of events that need to be both determined and handled are described in the following Subsections.

\subsection{Initiating rays from a vertex}
\label{subsect:initraysvert}

This procedure is invoked to initiate rays from a given vertex, say , when the discrete wavefront strikes .
Since  is also a vertex of , this procedure is also used in initiating rays from  as well. 

Let  be the face along which a ray has been determined to strike a vertex  on the face.
Let  be the collection of all the faces incident to  except for .
The set  of rays are initiated from , and all these rays lie on the set  of faces.
Due to Proposition~\ref{prop:noncrossing} (non-crossing property of weighted shortest paths), we do not initiate rays from  over the face . 
Further, the angle between any two successive rays in  at   are bounded by  (whose value is bounded as described in Section \ref{sect:boundrays}).

\begin{wrapfigure}{r}{0.5\textwidth}
\begin{minipage}[t]{\linewidth}
\begin{center}
\includegraphics[totalheight=0.9in]{figs/initverta.pdf}
\caption{\footnotesize Illustrating successive rays (shown in blue color) striking  from  and a ray bundle being initiated on a face ;  is the sibling pair of .  When  along  and/or  along  strike  and  edges respectively, a discrete wavefront is initiated from .}
\label{fig:initverta}
\end{center}
\end{minipage}
\end{wrapfigure}

Consider any face .
Let  be the edges bounding . 
(See Fig. \ref{fig:initverta}.)
The set  of rays that lie on face  is a ray bundle:
every ray in  strikes  before striking any other edge in .
For every such ray bundle , we find a sibling pair corresponding to .
Let  (resp. ) be the ray in  that strikes  at (resp.  such that there does not exist a ray in  that strikes  between  (resp. ) and  (resp. ).
We do binary search (with respect to edges  and ) among the rays in  to find both the rays  and . We trace the sibling pair  and  as long as both the rays refract along the same sequence of edges  until the bundle is forced to split.
Let  be the last edge in the edge sequence  at this stage.
Let  and  be the points on edge  to which rays  and  are incident when they are traced.
Let  be the weighted Euclidean distance from  to  i.e., when the discrete wavefront struck .
The following sets of event points are pushed to the event heap:
the event point corresponding to tracing the ray  (resp. ) to  (resp. ) that occurs at distance equal to the weighted Euclidean distance  added with the weighted distance along  (resp. ) from .
Further, the corresponding sibling pair is saved with each event point.
The splitting of sibling pairs is further detailed in Subsection \ref{subsect:splitsibpair}.

At the initialization step, let  be two adjacent faces in .
(See Fig. \ref{fig:initverta}.)
Let  (resp. ) be the triangle defining  (resp. ).
Let  be the ray in  that lies on  and let  be the ray in  that lies on face  such that no ray in  lies between  and .
Also, let  (resp. ) be the point on edge  (resp. ) to which ray  (resp. ) is incident when traced.
Then the event point for initiating a discrete wavefront from  is pushed to the event heap with the key value .

To improve the time complexity, we use the non-crossing property of (weighted) shortest paths (Proposition~\ref{prop:noncrossing}): whenever a ray bundle  that originates at a vertex  strikes another vertex , we save that information with  so that whenever another ray bundle  that has originated from the same vertex  splits at vertex  at an event corresponding to a larger distance, we do not split  in order to propagate from .
To achieve this, whenever a vertex  is struck by a ray bundle , we save the origin of  with  in a set.
Further, whenever a different ray bundle  splits due to vertex , we check this set and (i) update the bundle that first strikes , say  (ii) eliminate progressing any bundle (including ) whose origin is same as  if it crosses the bundle .

\subsection{Handling the ray striking an edge critically}
\label{subsect:crit}

Let  be a sibling pair and let  be a common edge to faces  and .
Consider the following event point: A ray  is traced along face  and is critically incident to  at point .
When this event occurs, we initiate two kinds of rays:
set  of rays that originate from the critical segment  (denoted by );
set  of Steiner rays that originate from the critical source . 
In the following Subsections, we describe algorithms to both initiate these sets of rays and to set up ray bundles. 

\subsubsection{Initiating rays from a critical segment}
\label{subsect:initrayscritseg}

\begin{figure}[h]
\begin{minipage}[t]{0.49\linewidth}
\begin{center}
\includegraphics[totalheight=1.2in]{figs/initcritseg.pdf}
\caption{\footnotesize Illustrating a sibling pair  originated at a critical segment that does not require splitting.} 
\label{fig:initcritseg}
\end{center}
\end{minipage}
\hspace*{0.04in}
\begin{minipage}[t]{0.49\linewidth}
\begin{center}
\includegraphics[totalheight=1.2in]{figs/splitsib-critseg.pdf}
\caption{\footnotesize Illustrating a sibling pair  originated at a critical segment that does require splitting.} 
\label{fig:splitsib-critseg}
\end{center}
\end{minipage}
\vspace*{-0.1in}
\end{figure}

We describe the procedure to initiate rays from the critical segment  first.
Given that a geodesic shortest path can be reflected back onto face  from any point on , we initiate a discrete wavefront from .
Let  be a vector normal to edge , passing through point  to some point in face .
The initiated rays get critically reflect back into face  from  while making an angle  with .  
Note that the rays in the discrete wavefront originating from  are parallel to each other; further, to achieve an -approximation, the distance between any two such successive rays along  is upper bounded by  (here,  is defined as in Lemma~\ref{lem:refrnoncritical}).

Let  be the point located on  at Euclidean distance   from  such that  is located between  and .
Let  and  be two critically reflected (parallel) rays (making an angle  with ), that originate from points  and  respectively.

If both the rays are incident to the same edge, say  of , we set  as a sibling pair and all the rays between  and  that potentially cross the same edge sequence in the future together with  is a ray bundle. 
(See Fig. \ref{fig:initcritseg}.)
Let  (resp. ) be the point at which the ray  (resp. ) is incident onto edge . 
This sibling pair is traced further across the polygonal domain.

If the rays in a sibling pair  are incident to distinct edges of , we need to find two rays  in , to respectively pair up with  and , for forming two new sibling pairs which define two corresponding ray bundles.
(See Fig. \ref{fig:splitsib-critseg}.)
Further, we also need to store an event point in the event heap that corresponds to the current shortest distance to  if it is via the rays  and . 
At that stage, a discrete wavefront from  is initiated. 
The algorithm to split a sibling pair that originates from a critical segment is described in Subsection \ref{subsubsect:splitcritseg}.

\subsubsection{Initiating rays from a critical source}
\label{subsect:initrayscritsrc}

There are three cases to consider based on the origin of the sibling pair,  and , of a bundle  that represents  the discrete wavefront from the critical source  on an edge :

\begin{enumerate}[(i)]
\item  have originated from some critical segment
\item  have originated from some vertex
\item  have originated from (possibly distinct) nodes of a tree of rays, say 
\end{enumerate}

We first rule out case (i) as it  cannot occur, since from  Proposition~\ref{prop:betwcrit}, 
a critical point of exit and a critical point of entry cannot occur in succession along a geodesic path.
Since case (iii) is a generalization of case (ii), herewith we explain event handling for case (iii).

Let  be the origin of .
Note that  may be  itself or a distinct critical source in .
Let  be successive rays in  such that  is refracted and  is critically reflected.
Let  be the point at which  is incident to .
A node corresponding to critical source  is inserted as a child of  in ; 
further, a set  of rays are initiated from . 
Let  be the angle at which  refracts (measured with respect to the vector  normal to edge ).
Let  be the face  onto which ray  is refracted into.
Also, let  and  be two vectors in face  with origin  such that they respectively make  and  angles with respect to . 
(See Fig. \ref{fig:initcritsrc}.)
The rays in  are uniformly distributed in the cone .
A ray bundle and a sibling pair corresponding to that ray bundle are determined:
Let  be the ray in  that subtends the minimum required angle,  with edge .
And let  be the ray that originates from  and is parallel to ray  in . 
Then  are extremal pairs of rays in .
The bundle  is now modified to  comprise all the rays that lie between  and , with  and  together forming a sibling pair of .
This sibling pair is traced over the face .

When traced, if both  and  are incident to same edge of face , then they together will continue to be a sibling pair.
Otherwise, to form two sibling pairs, new rays to pair up with  and  are found from the sets of rays initiated in ; the procedure is detailed in Subsection \ref{subsubsect:splittreeofrays}.
These sibling pairs' are traced and the corresponding event points are pushed to the event heap.

\subsection{Extending rays  across a region}
\label{subsect:extendrays}

 Given the bundles of rays that strike an edge , the rays need to be extended across the region they enter.
 Let  be the triangular region. The rays that refract after striking , as well as the critical rays that 
 originate from , need to be extended to determine their strike points on the other two edges  and .
 
 Extending a bundle may involve splitting a sibling pair. 
 While the split operation details will be defined subsequently,
 we discuss how to manage the extension of the set of bundles. 
 Note that only one of the bundles that strike , 
 will lead to a weighted shortest path from the origin  to the vertex  from amongst the bundles that strike edge . 
Furthermore, from a pair of bundles that cross each other, only one will be retained, due to the non-crossing property of shortest paths; and, from a  pair of bundles that split at vertex  only one bundle need be split, i.e., the one that determines the shortest distance to .

 Determining a weighted shortest path to  via rays contained in a bundle  will be detailed later. 
 Given a weighted shortest path to  via each bundle in the current set of bundles  that strike , one can determine a weighted shortest path to  with respect to . The bundle  that determines
 a weighted shortest path is maintained. Note that  changes as bundles strike edge .
 The bundle  partitions the bundles in  into two sets of bundles, one that strike
the edge   and the other set that strikes  . 
 These bundles are traced after processing their strike on their corresponding edge.
 To ensure that the relevant bundles in  are traced across an edge, say , we determine a shortest path to the edge  via rays in bundle .
 When this event occurs, the bundle is traced across edge  if it does not violate non-crossing property. 
 A weighted shortest path to the edge is determined by a binary search over the space of rays in the bundle. 
This binary search procedure is similar to the process of determining a weighted shortest path to a specific vertex by a binary search.


Note that the above  methodology is true for all bundles, independent of the source being a vertex or a critical segment.
We next determine how to efficiently determine the splitting of sibling pairs that define a bundle.

\subsection{Splitting a sibling pair}
\label{subsect:splitsibpair}

When a sibling pair needs to be split i.e.,  a ray bundle needs to be partitioned into two, then, based on the origin of the two rays in the sibling pair being considered, the following procedures are invoked.

\subsubsection{Pair that originates from a critical segment}
\label{subsubsect:splitcritseg}

\begin{figure}[h]
\begin{minipage}[t]{\linewidth}
\begin{center}
\includegraphics[totalheight=1.1in]{figs/split-origcrigseg.pdf}
\caption{\footnotesize Illustrating a ray bundle split when that ray bundle is originated from a critical segment ; two new sibling pairs are also shown}
\label{fig:split-origcrigseg}
\end{center}
\end{minipage}
\vspace*{-0.2in}
\end{figure}

Consider a sibling pair  that originates from a critical segment . 
(See Fig. \ref{fig:splitsib-critseg}.)
Let  be the edge sequence of  and  that the two rays have in common till they are incident to an edge  of face  and let  be the corresponding ray bundle.
Let  be incident to  and let  be incident to .
Given that  and  are parallel between any two successive edges in , following are the possibilities:
 and  refract from  and  respectively at the same angle;
 and  critically reflect from  and .
Since the latter kind of rays are not geodesic paths (Proposition \ref{prop:betwcrit}), we focus on the former.
If the rays refracted from  and  are both incident to the same edge of , then there is no need to split the sibling pair .
Otherwise, as explained below, we find new rays  and  that originate from  to respectively pair with  and .

Let  be the vertices of  where the endpoints of  are . 
(See Fig. \ref{fig:split-origcrigseg}.)
Let  be a line segment on face  such that it is parallel to line segment  ( refracted at ) and passes through vertex  of face . 
Let  be its other endpoint.
We interpolate over rays in  to find a ray  (resp. )) such that the section of  (resp. ) on  lies between  and the section of  (resp. ) on .
Then the ray bundle  splits into: a bundle with sibling pair  and another with  as its sibling pair.
Since between any successive edges in , rays  and  are parallel, an interpolation in possible.
Based on the ratio of  to , we interpolate to find a point  on  so that a critically reflected ray  from  reaches .
The ray  (resp. ) is the one whose origin is closest to  among all the rays between  and  (resp. ).
These sibling pairs' are pushed to the event heap with the key values being the respective weighted Euclidean distances' from  to the points at which these pairs strike the edges  and . 

Let the ray  be incident to  at  and let  be  incident to  at .
An event point to initiate a discrete wavefront from  is pushed  to the heap with the key value  where  is the weighted Euclidean distance from  to  and  is the weighted Euclidean distance from  to .

\subsubsection{Pair that originates from a tree of rays}
\label{subsubsect:splittreeofrays}

Let  be the edges bounding face .
Also, let  be a sibling pair of  such that both of them strike edge  of .
With further expansion of the wavefront, let  strike edge  at  and let  strike edge  at .
This requires us to split the ray bundle, , corresponding to sibling pair .
Let  (resp. ) be the critical ancestor path of  (resp. ).
Further, let  be the open region bounded by rays along critical ancestor paths , the rays  and the line segment .
Let  (resp. ) be the set of rays such that a ray  (resp. ) if and only if  originates from a critical point of entry or a critical source located on the critical ancestor path  (resp. ) and the ray  lies in .
With binary search over the rays in , we find a ray  and with binary search over  we find a ray  such that  intersects ,  intersects , and  and  are either successive rays originating from the same origin or adjacent origins on a critical ancestor path. The binary search
is performed over the nodes on the critical path, and at each node  the two extreme rays in the set of rays  that originate  at that node are used to decide whether the two rays  and  lie within  or not. 
This leads to splitting ray bundle   into two  and  
with  and  sibling pairs, respectively.
Events corresponding to the ray bundles created are pushed to the event queue reflecting the split.
Further, we also push the event corresponding to initiating a discrete wavefront from .
Note that splitting a sibling pair that originates from a vertex is just a special case of the procedure listed above. 

\section{Improving the time complexity: interpolating versus tracing rays}
\label{sect:interpol}

To improve the time complexity of our algorithm while obtaining a weighted shortest path with  multiplicative error, instead of tracing a ray  across an edge sequence to find its point (and angle) of incidence onto an edge , we show that it suffices to interpolate the position and the angle of refraction of  from the last edge of .
For this, we need an assumption (albeit mild) that the angle between any two successive rays in any ray bundle is upper bounded by .

\begin{wrapfigure}{r}{0.5\textwidth}
\centering
\begin{minipage}[b]{.4\textwidth}
\centering{\includegraphics[totalheight=1.5in]{figs/interptrace.pdf}}
\caption{\footnotesize Illustrating the error in interpolating versus tracing a ray across an edge sequence }
\label{fig:interptrace}
\end{minipage}
\end{wrapfigure}

From Lemma~\ref{lem:depsilon} and Lemma~\ref{lem:refrcritical}, we note that in order to find an -approximate weighted shortest path, the cardinality of the set  of rays that originate from a vertex, or a critical source, is chosen to be .
Let  be an edge sequence of a sibling pair , and let  be the last edge of .
In doing binary search over  to find a ray , we need to trace  rays in  to .
Taking into account the cardinality of  (which is  from the Proposition~\ref{prop:edgeseqlen}), the time it takes for binary search to find a ray from the source is .
As detailed below, we reduce the time involved in this by interpolation. 

Let  and  be two sibling rays in a bundle  in , with  being the angle between them.
(See Fig. \ref{fig:interptrace}.)
Consider a ray  with origin  and that lies between  and .
Also, let  be an edge that is intersected by all three rays  and , respectively at points  and .
We denote  with .
(This ratio is known if the target point  on  is known even though  itself may not be known.)
The interpolation of ray  is denoted with  and it is characterized as follows:

\begin{wrapfigure}{r}{0.5\textwidth}
\centering
\begin{minipage}[b]{.4\textwidth}
\centering{\includegraphics[totalheight=1in]{figs/interpolangles.pdf}}
\caption{\footnotesize Illustrating the angle of incidence and refraction of rays , and  at an edge }
\label{fig:interpolangles}
\end{minipage}
\end{wrapfigure}

\begin{itemize}
\vspace{-0.1in}
\item[(i)]
Let  be the face bounded by  and .
The angle between the vectors induced by  and  is upper bounded by .

\item[(ii)]
The Euclidean distance of interpolated point of incidence  of ray  from  is .
\vspace{0.05in}
\end{itemize}

\begin{lemma}
\label{lem:interpolate}
Let  be the edge sequence associated with a ray bundle  in , having siblings  and  ; and, let  be the last edge in .
Let  be a ray that lies in between  and  and let  be the point of incidence of  on edge  when traced across .
Also, let  be the point due to interpolation. 
Given that the angle  between  and  is upper bounded by , the weighted distance  between  and  is approximated by  within a multiplicative error of , and the angle of incidence of  is -approximated with respect to .
\end{lemma}

\begin{proof}
Let  be the edge sequence in .
As in Lemma~\ref{lem:depsilon}, angles of incidence and angles of refractions of rays  and  are as mentioned here.
Let  be the angles at which the ray  is incident on edges  respectively.
(See Fig. \ref{fig:interpolangles}.)
Let  be the angles at which the ray  refracts at edges  respectively.
Similarly, let  be the angles at which the ray  is incident on edges  respectively.
And, let  be the angles at which the ray  refracts at edges  respectively.
For every , if a critical angle exists for edge , assume that both  and  are less than that critical angle.
Using Lemma~\ref{lem:depsilon}, we derive a bound on  as follows: 



Let  be the angles at which the ray  is incident on edges  respectively.
Also, let  be the angles at which the ray  refracts from edges  respectively.
Then analogous to the above, we can lower bound the angle at which ray  strikes edge  as follows: 
.
For every , let  be the face bounded by  and .
Also, for every , let  be the angle between the vectors induced by  and .
The angle  is upper bounded by 
.
Hence, the ratio of the  error in the angle  versus the angle , denoted with  equals to  

Therefore, to keep the error ratio less than , we choose  since . 
(In the last inequality, we used the assumption that ).
In fact, we choose  less than or equal to , so that to upper bound  with  as well as to upper bound .

Also, let  respectively be the points of incidence of interpolated ray  on edges  in .
Considering that the ray  is traced across the edge sequence of cardinality , the  is bounded recursively.
Let  be the distance along ray  from  to  and let  be equal to .
Since the error in angle accumulates,  is upper bounded by .
\end{proof}

In the interpolation method, the binary search is performed on the ordered set of rays, , at vertex  until the angle between the rays guiding the binary search is less than or equal to .
This results in  search steps, where each step would require tracing a ray from the source.
Moreover, this method can also be used to approximate the shortest distance to an edge via rays in a ray bundle.
We summarize the discussion in the following Lemma.
\begin{theorem}
\label{thm:interpol}
The interpolation method determines the shortest distance to a given point  from a source  to within a multiplicative approximation factor of  in  time using rays from a ray bundle. 
Furthermore, an -approximation to the shortest distance from  to an edge   can be determined in  time using rays from a ray bundle.
\end{theorem}

\section{Analysis}
\label{sect:analysis}

The analysis is based on showing the following facts:
(i) bundles of rays are correctly maintained, and
(ii) the shortest distances to vertices, and edges similarly, are correctly computed using the bundles. 

Bundles are initiated from vertices and propagated across the faces of the domain as they strike the 
edges of the domain. The propagation is evidently correct, the modifications to the bundles
being (a) bundle splits and (b) elimination of bundles due to crossing of bundles.
As described in Subsection \ref{subsect:extendrays}, elimination of bundles is
determined at vertices when more than one of the bundles split at that vertex.
Note that bundles are propagated until they split and further propagation of the split parts of the bundles occurs when the shortest distance event, of weighted distance , corresponding to the bundle, say ,  striking an edge is determined by the heap. 
When two bundles cross each other completely, one of them is eliminated from further consideration.
Thus for the rest of the proof, we will assume w.l.o.g., that bundles are correctly maintained.

To determine approximate weighted shortest paths to vertices correctly, we consider two categories of paths, Type-I and Type-II. 
Recall that Type-I paths do not have critically reflected segments in between, while Type-II paths does have. 
Analogous arguments prove the correctness involved in computing approximate weighted shortest paths to edges.

\begin{lemma}
\label{lem:type1find}
The algorithm correctly determines sibling pair in  for every source , and computes an -approximation to a Type-I weighted shortest path from  to , for any vertex  in .
\end{lemma}
\begin{proof}
Suppose that there exists a Type-I weighted shortest path between  and , traversing an edge sequence .
Lemma~\ref{lem:refrnoncritical} shows that an  approximation can be found using rays within
a bundle. Bundles are maintained using sibling pair.
We show that the algorithm maintains sibling pair for edges in  that determine bundles, the sibling pairs being traced rays such that these rays can be used to find an -approximate weighted shortest path from  to .

Initially when rays are generated from a vertex, , siblings are computed correctly and each ray is refracted correctly.
Consider the procedure that computes a sibling pair  closest to a vertex .
The rays  and  are thus two successive rays and the distance to  via rays  and  is updated.
Suppose both the rays  and  have the same origin.
Using binary search with interpolation, we find the siblings correctly.
Otherwise, the origin of  is different from the origin of .
In this case, we find a sibling pair via the binary search with interpolation on the rays in the critical ancestor paths of two siblings, say  and  such that the rays  and   are part of the set of rays lying in between  and .
The correctness of the interpolation method is shown in Lemma~\ref{lem:interpolate}.
\end{proof}

\begin{lemma}
\label{lem:refrcritical-find}
Let  be a Type-II weighted shortest path from a vertex  to another vertex  on  with a critical segment  in-between.
Then the algorithm determines a pair of traced rays  in  that can approximate a weighted shortest path, . 
\end{lemma}
\begin{proof}
An optimal path  can be partitioned into sub-paths, each such sub-path going from one vertex  to another vertex . 
Suppose a sub-path uses a critical segment  and is partitioned as follows: a path  from  to  and a path  from  to .
Let  be the edge on a face  such that the critical segment  lies  on  and reflects rays back onto face . 
Also, let  be the critical point of entry into .
The correct determination of the critical point of incidence from a bundle of rays follows from
Lemma~\ref{lem:refrnoncritical}.
If  is the endpoint of  then we are done since the distance to the endpoint from the critical source is included in consideration.
Otherwise rays are generated from   that are parallel and separated by a small weighted Euclidean distance  less than .
Let  be sibling rays that originate from  such that  lies in between  and .
Lemma~\ref{lem:refrcritical} shows that the shortest distance from  to  can
be found by tracing the rays  and  and interpolating between them to find the  point
on  closest to .
\end{proof}

\begin{theorem}
\label{thm:timecompl} 
The algorithm computes an -approximate weighted shortest path from  to  in 
 time. 
\end{theorem}

\begin{proof}
The correctness follows from Lemmas~\ref{lem:type1find} and \ref{lem:refrcritical-find}.

We first bound the number of ray bundles.
The number of bundles that are initiated from vertex sources are . 
By Proposition 3, the number of bundles that are initiated from critical segments is bounded by .
The processing of such bundles will be handled separately.

The number of bundles increase when split. 
The increase is charged to the vertex  that causes the split. 
Only one bundle, the bundle that determines an approximate weighted shortest distance to  is split and thus the total number of bundles split are .
Bundles are propagated further into adjacent faces after striking an edge.

We thus need to analyze the time involved in finding an approximate weighted shortest paths and splitting ray bundles.
Let  be the set of rays from a vertex source .
To search for  a  ray in  that is closest to the vertex or provides the shortest distance to the last edge  of an edge sequence , the algorithm utilizes the combination of binary search and interpolation (as described in Section~\ref{sect:interpol}).
Using Theorem~\ref{thm:interpol}, the complexity of determining the shortest distance  when  rays from a specific vertex are considered
is equal to .
Note that  edges will be encountered as rays are traced.


Next, let us consider the  case in which the sibling pair is not from a vertex but arises from a  critical source
contained in a tree of rays, say .
Let  be the set consisting of all the critical points of entry in any critical ancestor path  in .
A binary search on the critical path determines vertex  such that the ray to be determined is in .
Since  is  in size and since it is ensured that no two ray bundles cross, binary search can be used to find a vertex   in  such that an approximate weighted shortest path lies in  with  ray tracings.
A search on the rays from that specific vertex in  follows as described in the earlier paragraph. 

A similar search is required to determine the ray that identifies the approximate weighted shortest distance to an edge, which is required when a bundle splits.
The procedure and time complexity of this is similar to the determination of a weighted shortest path to a vertex. 

Since the total number of vertices and critical sources is , the total work in splitting and initiating ray bundles from these sources is , as claimed in Theorem~\ref{thm:interpol}.

Determining the ray originating from a critical segment and that strikes  a vertex can be determined via interpolation (or a binary search) on the space of the parallel rays that are part of the bundle that originates from a  critical segment.
The work involved in tracing a ray that originates at a critical segment and traverses an edge sequence takes  time (Proposition~\ref{prop:edgeseqlen}).
The specific pair of successive rays of interest can be found by interpolation, taking  time.
As there are  vertices and  critical segments (Proposition~\ref{prop:numcritsrc}), the time complexity is , including the binary search involved.
\end{proof}

\subsection*{Single-source approximate weighted shortest path queries}
\label{subsect:sssp}

Here we devise an algorithm to preprocess  to construct a {\it shortest path map} so that for any given query point  in  an approximate weighted shortest path from  can be computed efficiently.
As part of preprocessing, using the non-crossing property of shortest paths (Proposition~\ref{prop:noncrossing}), we compute for every edge  in the subdivision a minimum cardinality set  of bundles such that for every point  an approximate weighted shortest path can be computed from a bundle .
Further, for every edge , the set  of bundles are saved with .
For a query point , exploiting Lemma~\ref{lem:errorX} and Corollary~\ref{cor:errorX}, we locate  in the triangulation and determine an approximate weighted shortest path from  to  via one of the bundles associated to edges of the triangle containing .

As mentioned, let  be the maximum coordinate value used in describing .
Since the number of bundles is  and since it takes  time to determine the bundles that need to be associated to an edge and there  edges, it requires  time for preprocessing.
The query phase takes  time: there could be  bundles associated to edges of the triangle containing  and finding an approximate weighted shortest path via any one such bundle takes  time.

\section{Conclusions}
\label{sect:conclu}

In this paper we have presented a polynomial-time algorithm for finding an approximate weighted shortest path between two given points  and .
The main ideas of this algorithm rely on progressing the discretized wavefront from  to . 
The time complexity of our algorithm is .
Significantly, our algorithm is polynomial with respect to input parameters.
This result is about a cubic factor (in ) improvement over the Mitchell and Papadimitriou's '91 result \cite{journals/jacm/MitchellP91} in finding a weighted shortest path between two given points, which is the only known polynomial time algorithm for this problem to date.
In addition, we extend our algorithm to answer single-source weighted shortest path queries.
Further, with minor modifications, our algorithm appears extendable to determine geodesic shortest paths on the surface of a -manifold whose faces are associated with positive weights. 
Since the number of events in the problem stand at  (from \cite{journals/jacm/MitchellP91}), it would be interesting to explore further improvements in devising a more efficient polynomial time approximation scheme.

\bibliographystyle{plain}

\bibliography{weireg-sp}

\end{document}
