\documentclass[preprint,12pt]{elsarticle}
\usepackage{amsthm}
\usepackage{amsmath}
\usepackage{amsfonts}
\usepackage{amssymb}
\usepackage{graphicx}
\usepackage[section]{algorithm}
\usepackage{algorithmic}
\usepackage{float}




\usepackage{graphicx}


\usepackage{amssymb}
\usepackage{amsmath}
\usepackage{amsthm}
\newtheorem{thm}{Theorem}[section]
\newtheorem{lem}{Lemma}[section]
\newtheorem{defn}{Definition}[section]





\DeclareGraphicsExtensions{.eps,.jpg}

\begin{document}

\begin{frontmatter}





\title{An efficient parallel algorithm for the longest path problem in meshes}

\address[label1]{Corresponding author: fatemeh.keshavarz@aut.ac.ir
}
\address[label2]{ ar\_bagheri@aut.ac.ir}


\author{Fatemeh Keshavarz-Kohjerdi \fnref{label1}}
\address{Department of Computer Engineering,\\ Islamic Azad University,
    North Tehran Branch, Tehran, Iran.\\}

\author{Alireza Bagheri \fnref{label2}}

\address{Department of Computer Engineering \& IT,\\ Amirkabir University of Technology, Tehran, Iran.}

\begin{abstract}
In this paper, first we give a sequential linear-time algorithm for
the longest path problem in meshes. This algorithm can be considered
as an improvement of \cite{FAA:ALAFFLPIRGG}. Then based on this
sequential algorithm, we present a constant-time parallel algorithm
for the problem which can be run on every parallel machine.
\end{abstract}

\begin{keyword} grid graph, longest path
\sep meshes \sep sequential and parallel algorithms.

MSC: ; ; .

\end{keyword}

\end{frontmatter}








\section{Introduction}\label{IntroSect}
The longest path problem, i.e. the problem of finding a simple path
with the maximum number of vertices, is one of the most important
problems in graph theory. The well-known NP-complete Hamiltonian
path problem, i.e. deciding whether there is a simple path that
visits each vertex of the graph exactly once, is a special case of
the longest path problem and has many applications \cite{D:GT,
GJ:CAI}. \par Only few polynomial-time algorithms are known for the
longest path problem for special classes of graphs. This problem for
trees began with the work of Dijkstra around 1960, and was followed
by other people \cite{BSZVGF:OCALPIAT, G:FALPIACMD, LMN:TLPPIPOIG,
MC:ASPAFTLPPOCG, UU:OCLPISGC}.
In the area of approximation
algorithms it has been shown that the problem is not in APX, i.e.
there is no polynomial-time approximation algorithm with constant
factor for the problem unless P=NP \cite{G:FALPIACMD}. Also, it has
been shown that finding a path of length  is not
possible in polynomial-time unless P=NP \cite{KMR:OATLPIAG}. For the
backgrround and some known result about approximation algorithms, we
refer the reader to \cite{BH:FAPOSL, GN:FLPCAC, ZL:AFLPIG}.\par A
grid graph is a graph in which vertices lie only on integer
coordinates and edges connect vertices that are separated by a
distance of once. A solid grid graph is a grid graph without holes.
The rectangular grid graph  is the subgraph of 
(infinite grid graph) induced by , where  and  are
respectively  and  coordinates of  (see Figure ~\ref{bb}).
A mesh  is a rectangular grid graph . Grid graphs
can be useful representation in many applications. Myers \cite{M}
suggests modeling city blocks in which street intersection are
vertices and streets are edges. Luccio and Mugnia \cite{LM:HPOARC}
suggest using a grid graph to represent a two-dimensional array type
memory accessed by a read/write head moving up, down or across. The
vertices correspond to the center of each cell and edges connect
adjacent cells. Finding a path in the grid corresponds to accessing
all the data.\par Itai \textit{et al.} \cite{IPS:HPIGG} have shown
that the Hamiltonian path problem for general grid graphs, with or
without specified endpoints, is NP-complete. The problem for
rectangular grid graphs, however, is in P requiring only
linear-time. Later, Chen \textit{et al.} \cite{CST:AFAFCHPIM}
improved the algorithm of \cite{IPS:HPIGG} and presented a parallel
algorithm for the problem in mesh architecture. There is a
polynomial-time algorithm for finding Hamiltonian cycle in solid
grid graphs \cite{LU:HCISGG}. Also, the authors in \cite{CT:HPOGG}
presented sufficient conditions for a grid graph to be Hamiltonian
and proved that all finite grid graphs of positive width have
Hamiltonian line graphs.\par Recently the Hamiltonian cycle (path)
and longest path problem of a grid graph has received much
attention. Salman \textit{et al.} \cite{AEB:SSAG} introduced a
family of grid graphs, i.e. alphabet grid graphs, and determined
classes of alphabet grid graphs that contain Hamiltonian cycles.
Islam \textit{et al.} \cite{Imnrx:hcihgg} showed that the
Hamiltonian cycle problem in hexagonal grid graphs is NP-complete.
Also, Gordon \textit{et al.} \cite{vyf:hpotgg} proved that all
connected, locally connected triangular grid graphs are Hamiltonian,
and gave a sufficient condition for a connected graph to be fully
cycle extendable and also showed that the Hamiltonian cycle problem
for triangular grid graphs is NP-complete.\par Moreover, Zhang and
Liu \cite{wqz} gave an approximation algorithm for the longest path
problem in grid graphs and their algorithm runs in quadratic time.
Also the authors in \cite{FAA:ALAFFLPIRGG} has been studied the
longest path problem for rectangular grid graphs and their algorithm
is based on divide and conquer technique and runs in linear time.
Some results of the grid graphs are investigated in
\cite{kb:hpiscogg, 6}. \par In this paper, we present a sequential
and a parallel algorithms for finding longest paths between two
given vertices in rectangular grid graphs (meshes). Our algorithm
has improved the previous algorithm \cite{FAA:ALAFFLPIRGG} by
reducing the number of partition steps from  to only a
constant.
\par The organization of the paper as follow: In Section 2, we review some necessary definitions
and results that we will need. A sequential algorithm for the
longest path problem is given in Section 3. In Section 4, a parallel
algorithm for the problem is introduced which is based on the
mentioned sequential algorithm. Conclusions is given in Section 5.
\begin{figure}[tb]
  \centering
  \includegraphics[scale=1]{17}
  \caption[]{\small The rectangular grid graph .}
\label{bb}
\end{figure}
\section{Preliminary results}
In this section, we give a few definitions and introduce the
corresponding notations. We then gather some previously established
results on the Hamiltonian and the longest path problems
in grid graphs which have been presented in \cite{CST:AFAFCHPIM, IPS:HPIGG, FAA:ALAFFLPIRGG}.\\
The \textit{two-dimensional integer grid}  is an infinite
graph with vertex set of all the points of the Euclidean plane with
integer coordinates. In this graph, there is an edge between any two
vertices of unit distance. For a vertex  of this graph, let
 and  denote  and  coordinates of its
corresponding point (sometimes we use  instead of ).
We color the vertices of the two-dimensional integer grid as black
and white. A vertex  is colored \textit{white} if
 is even, and it is colored
\textit{black} otherwise. A \textit{grid graph}  is a finite
vertex-induced subgraph of the two-dimensional integer grid. In a
grid graph , each vertex has degree at most four. Clearly,
there is no edge between any two vertices of the same color.
Therefore,  is a bipartite graph. Note that any cycle or path
in a bipartite graph alternates between black and white vertices. A
\textit{rectangular grid graph}  (or  for short) is a
grid graph whose vertex set is . In the figures we
assume that  is the coordinates of the vertex in the upper
left corner. The size of  is defined to be .  is
called \textit{odd-sized} if  is odd, and it is called
\textit{even-sized} otherwise. In this paper without loss of
generality, we assume  and all rectangular grid graphs
considered here are oddodd, evenodd and
eveneven.  is called a \textit{n-rectangle}.\\
The following lemma states a result about the Hamiltonicity of
even-sized rectangular graphs.
\begin{lem}
\label{Lemma:1} \cite{CST:AFAFCHPIM}  has a Hamiltonian
cycle if and only if it is even-sized and .
\end{lem}

\begin{figure}[tb]
  \centering
  \includegraphics[scale=1]{1}
  \caption[]{\small A Hamiltonian cycle for the rectangular grid graph .}
\label{fig:hamcycle}
\end{figure}

\par Figure \ref{fig:hamcycle} shows a
Hamiltonian cycle for an even-sized rectangular grid graph, found by
Lemma \ref{Lemma:1}. Every Hamiltonian cycle found by this lemma
contains all the boundary edges on the three sides of the
rectangular grid graph. This shows that for an even-sized
rectangular graph , we can always find a Hamiltonian cycle, such
that it contains all the boundary edges, except of exactly one side of  which contains an even number of vertices.\par
Two different vertices  and  in  are
called \textit{color-compatible} if either both  and
 are white and  is odd-sized, or  and
 have different colors and  is even-sized. Let
 denote the rectangular grid graph  with two
specified distinct vertices  and . Without loss of generality,
we assume .\\  is called
\textit{Hamiltonian} if there exists a Hamiltonian path between 
and  in . An even-sized rectangular grid graph contains
the same number of black and white vertices. Hence, the two
end-vertices of any Hamiltonian path in the graph must have
different colors. Similarly, in an odd-sized rectangular grid graph
the number of white vertices is one more than the number of black
vertices. Therefore, the two end-vertices of any Hamiltonian path in
such a graph must be white. Hence, the color-compatibility of 
and  is a necessary condition for  to be
Hamiltonian. Furthermore, Itai \textit{et al.} \cite{IPS:HPIGG}
showed that if one of the following conditions hold, then
 is not Hamiltonian:
\begin{itemize}
\item [(F1)]  is a 1-rectangle and either  or  is not a corner
vertex (Figure \ref{RecFig}(a))
\item[(F2)]  is a 2-rectangle
and  is a nonboundary edge, i.e.  is an edge and it is
not on the outer face (Figure \ref{RecFig}(b)).
\item [(F3)]  is
isomorphic to a 3-rectangle grid graph  such that  and
 is mapped to  and  and all of the following three conditions hold:
\begin{enumerate}
\item  is even,
\item  is black,  is white,
\item   and  (Figure \ref{RecFig}(c)) or
 and  (Figure \ref{RecFig}(d)).
\end{enumerate}
\end{itemize}
Also by \cite{IPS:HPIGG} for a rectangular graph  with two
distinct vertices  and ,  is Hamiltonian if and
only if  and  are color-compatible and ,  and 
do not satisfy any of conditions ,  and .
In the following we use  to indicate the problem of
finding a longest path between vertices  and  in a rectangular
grid graph ,  to show the length of longest
paths between  and  and  to indicate the upper
bound on the length of longest paths between  and .
\par The authors in
\cite{FAA:ALAFFLPIRGG} showed that the longest path problem between
any two given vertices  and  in rectangular grid graphs satisfies one of the
following conditions:
\begin{itemize}
\item [(C0)]  and  are color-compatible and none of (F1)- (F3) hold.
\item [(C1)] Neither (F1) nor (F) holds and either
\begin{enumerate}
\item  is even-sized and  and  are same-colored or
\item  is odd-sized and  and  are different-colored.
\end{enumerate}
\item [(C2)]
\begin{enumerate}
\item  is odd-sized and  and  are black-colored and neither (F1) nor (F) holds, or
\item  and  are color-compatible and (F3) holds.
\end{enumerate}
\end{itemize}
Where (F) is defined as follows:
\begin{itemize}
\item [(F)]  is a 2-rectangle and  or  and .
\end{itemize}
They also proved some upper bounds on the length of longest paths as
following:\\ (F1)(F2^*)(C0)(C1)(C2)
\begin{thm} \label{theorem:2}\cite{FAA:ALAFFLPIRGG} Let  be the upper bound on the length of
longest paths between  and  in  and let
 be the length of longest paths between  and .
In a rectangular grid graph , a longest path between any two
vertices  and  can be found in linear time and its length
i.e.,  is equal to .
\end{thm}
\begin{figure}[tb]
  \centering
  \includegraphics[scale=1]{2}
  \caption[]{\small Rectangular grid graph in which there is no Hamiltonian path between  and .}
  \label{RecFig}
\end{figure}

\section{\bf The sequential algorithm}
In this section, we present a sequential algorithm for finding a
longest path between two vertices in rectangular grid graphs. This
algorithm is the base of our parallel algorithm which is introduced
in Section 4. First, we solve the problem for 1-rectangles and
2-rectangles.
\begin{lem} \label{Lemma:5} \cite{FAA:ALAFFLPIRGG} Let  be a longest
path problem with  or , then
.
\end{lem}
\begin{proof}
For a 1-rectangle obviously the lemma holds for the single possible
path between  and  (see Figure \ref{a3}(a)). For a
2-rectangle, if removing  and  splits the graph into two
components, then the path going through all vertices of the larger
component has the length equal to  (see Figure
\ref{a}(b)). Otherwise, let  be the vertex adjacent to 
and  be the vertex adjacent to  such that   and . Then we make a path from  to
 and a path from  to  as shown in Figure
\ref{a3}(c), (d), and connect  to  by a path such that
at most one vertex remains out of the path as depicted in this
figure.
\end{proof}
\begin{figure}[tb]
  \centering
  \includegraphics[scale=1]{3}
  \caption[]{
 (a) Longest path between  and  in a 1-rectangle,
  (b) Longest path between  and  in a 2-rectangle,
  (c) and (d) A path with length  and  for a 2-rectangle, respectively.}
  \label{a3}
\end{figure}
From now on, we assume that , so one of conditions (C0),
(C1) and (C2) should hold. Following the technique used in
\cite{CST:AFAFCHPIM} we develop an algorithm for finding longest
paths.\\

\noindent\textbf{Definition 3.1.} \cite{FAA:ALAFFLPIRGG} A
\textit{separation} of a rectangular grid graph  is a partition
of R into two disjoint rectangular grid graphs  and , i.e.
, and .\\

\noindent\textbf{Definition 3.2.} \cite{IPS:HPIGG} Let 
and  be two distinct vertices in . If
 and , then  and
 are called \textit{antipodes}.
\\

\noindent\textbf{Definition 3.3.} \cite{CST:AFAFCHPIM} Partitioning
a rectangular grid graph  into five disjoint rectangular grid
subgraphs  that is done by two horizontal and two
vertical separations are called \textit{peeling operation}, if the following two conditions hold:
\begin{enumerate}
\item  and  and  are antipodes.
\item  Each of four rectangular grid subgraphs  is an
even-sized rectangular grid graph whose boundary sizes are both
greater than one, or is empty.
\end{enumerate}
Generally the two vertical separation of a peeling are done before
the two horizontal separation. However, for an oddodd or
oddeven rectangular grid graph with , this
order is reversed in order to guarantee that the boundary sizes of
 and  are greater than one. Figure
\ref{a} shows a peeling on  where  is  and  is .\\
\begin{figure}[htb]
  \centering
  \includegraphics[scale=1]{4}
  \caption[]{\small A peeling on .}
  \label{a}
\end{figure}
The following lemma can be obtained directly from Definition 3.3.
\begin{lem} \label{Lemma:5} \cite{CST:AFAFCHPIM}
Let  be the resulting rectangular grid subgraph
of a peeling on , where . Then
\begin{enumerate}
\item  remain the same color in  as in ; and
\item  has the same parity as , that is, , and .
\end{enumerate}
\end{lem}


\noindent\textbf{Definition 3.4.} A peeling operation on  is
called \textit{proper} if
,
where  denotes the number of vertices of
.
\begin{lem} \label{Lemma:6} For the longest path problem , any peeling on  is proper if:
\begin{enumerate}
\item The condition  holds and  i.e.  is  or , or
\item One of the conditions  and  hold and  is  or .
\end{enumerate}
\end{lem}
\begin{proof}
The lemma has been proved for the case that (C0) holds (see
\cite{CST:AFAFCHPIM}). So, we consider conditions (C1) and (C2).
From Lemma \ref{Lemma:5}, we know that  and  are still
color-compatible, and we are
going to prove that  is not in cases  and .\\
By Lemma \ref{Lemma:5}, when  is an 
rectangular grid graph,  is also an  rectangular grid graph,  and  have the same color as in
, and hence  is not a 2-rectangle. If 
is a 1-rectangle, then  or  and then we
have the two following cases:\par Case1. (C1) holds and both  and
 are different color. In this case, one of  and 
( and ) is even and the other is odd. Considering that
 and  are antipodes and  is oddodd, one of 
and  must be at the corner and exactly one of the vertex goes out
of the path.
\par Case2. (C2) holds and both  and  are black color. In this case, all
, ,  and  are even. Hence, vertices 
and  are before corner vertices and exactly two vertices go out
of the path.\par In the similar way, when  is an  rectangular grid graph ((C1) holds),  is
also an  rectangular grid graph, and hence
 is not a 2-rectangle. If  is a
1-rectangle, then . In this case,  and 
are both odd or even. Hence,  or  are at the
corner and exactly one vertex goes out of the path. \\
Therefore by Theorem \ref{theorem:2},
 and any peeling of
 is always proper.
\end{proof}
Nevertheless, a peeling operation in an 
rectangular grid graph  may not be proper, and
, see Figure \ref{bs}
where the dotted-lines represent a peeling operation. In the two
following cases a peeling operation is not proper:
\begin{enumerate}
\item [(F)]  is black,  is even (or
odd),  and 
\item [(F)] is white,  is even (or odd),  and
;
\end{enumerate}
\begin{figure}[tb]
  \centering
  \includegraphics[scale=1]{5}
  \caption[]{\small Rectangular grid graph in which a peeling operation is not proper.}
  \label{bs}
\end{figure}
\begin{figure}[tb]
  \centering
  \includegraphics[scale=1]{6}
  \caption[]{\small }
  \label{bj}
\end{figure}
\begin{lem} \label{Lemma:7} For the longest path problem , where
 is an  rectangular grid graph, a peeling
operation on  is proper if and only if  is
not cases in  and .
\end{lem}
When a peeling operation is not proper it can be made proper by
adjustment the peeling boundaries. In that case, if  and  are empty, then  is 2-rectangle that
is in case (F). Therefore, without loss of generality, we
assume  or  is not empty. If
rectangular grid subgraphs  and  are empty, then we
move one column (or two columns when  or  is a
2-rectangle) from  or  to  such that  or
 is still even-sized rectangular grid graphs; see Figure
\ref{bj}(a). If  and  (or  and
) is not empty, then we move one row (or two rows when
 or  is 2-rectangle) from  or  to
 (Figure \ref{bj}(b)), or move the bottom row to 
(Figure \ref{bj}(c)) or move the upper row to  (Figure
\ref{bj}(d)), such that  or  is still even-sized
rectangular grid graphs.
\par After a peeling operation on , we construct longest paths
in . Consider the following cases for :
\begin{itemize}
\item[(a)] .
\item[(b)]  are even, and either  or ;
\item[(c)]  are odd, and either  or ;
\item[(d)]  is even and  is odd, and either  or .
\end{itemize}
For case (a), we showed that when  the problem can be solved
easily. For  the longest paths of all the possible problems
are depicted in Figure \ref{k} (the isomorphic cases are omitted).
 \begin{figure}
  \centering
\includegraphics[scale=1]{7}
  \caption[]{\small For , (a)  and  are white, then there is Hamiltonian path,
   (b)  and  have different colors, then there is a path with 
   and (c)  and  are black, then there is a path with .}
  \label{k}
\end{figure}
\par For cases (b), (c) and (d) we use
the definition of trisecting.\\


\noindent\textbf{Definition 3.5.} \cite{CST:AFAFCHPIM} Two
separations of  that partition it into three rectangular grid
subgraphs ,  and  is called
\textit{trisecting}, if
\begin{itemize}
\item[.]  and  are a 2-rectangle, and
\item[.]  and .
\end{itemize}

A trisecting can be done by two ways horizontally and vertically. If
 or , then trisecting is done
horizontally, if , then trisecting is done vertically. \\

\noindent\textbf{Definition 3.6.} A corner vertex on the boundary of
 resp.,  facing  is called a
\textit{junction vertex} of  resp.,  if
either
\begin{itemize}
\item [(i)] The condition  holds and it has different color from  and
, or
\item [(ii)] One of the conditions  or  hold and
.
Where  is one of the corner vertices of ,  is one of
the corner vertices of , and  and  are two of the
corner vertices of  facing  and ,
respectively.
\end{itemize}

In Figure \ref{ca},  and ,  and , ,
,  and  are junction vertices in ,
 and , respectively.
\begin{figure}[tb]
  \centering
  \includegraphics[scale=1]{8}
  \caption[]{\small A trisecting on .}
  \label{ca}
\end{figure}
Existence of junction vertices has been proved for condition 
in \cite{CST:AFAFCHPIM}, in this paper we only consider conditions
 and .
\begin{lem} \label{Lemma:7} Performing a trisecting on , where , assuming condition  or 
holds, if , and  and  facing the common border of
 and , then there is no junction vertex for
 and , otherwise  and 
have at least one junction vertex.
\end{lem}
\begin{proof}
Consider Figure \ref{cc}(a) and (b), where  and two vertices
 and  facing the common border  and . In
this case, the only two vertices  and  may be junction
vertices. By Theorem \ref{theorem:2}, there exists a Hamiltonian
path from  to  and from  to  in  and
, respectively, and
.
Hence, neither  nor  has a junction vertex at
all. Now for other cases, we show that  and 
have at least one junction vertex.\\
In case (b),  and  have the same color (white or black), and
two corner vertices on the boundary of  (resp.,
) facing  are different color and also
 is a rectangle that  is empty or  and
even. We consider the following three cases for  and :
\par Case 1. Both  and  are the corner vertices on the boundary
of  and  facing ; see Figure
\ref{cc}(c). By Theorem \ref{theorem:2}, there exists a Hamiltonian
path from  to  and from  to  in  and
, respectively, and a path from  to  which does not pass through a vertex in . Therefore,

and hence both  and  have a unique junction
vertex.
\par Case
2.  is the corner vertex on the boundary of  facing
; see Figure \ref{cc}(d). By Theorem \ref{theorem:2},
there exists a Hamiltonian path from  to , from  to
 and a path from  to  which does not pass through a
vertex, or  a Hamiltonian path from  to , form   to  and a path
from  to  which does not pass through a vertex. Therefore,

and

and hence  has a unique junction vertex and  have
two junction vertices (the same argument is also applied to ). In
this case, where , both  and  have a unique
junction vertex; see Figure \ref{cc}(e).
\par Case 3.  and  are not the corner vertices on the
boundary of  and  facing ; see
Figure \ref{cc}(f). By Theorem \ref{theorem:2}, there exists a
Hamiltonian path from  to ,  to  and a path from
 to  which does not pass through a vertex, or a Hamiltonian path from  to
, from  to  and a path from  to  which does
not pass through a vertex, or a Hamiltonian path form  to , from  to 
and a path from  to  which does not pass through a vertex.
Therefore,
,

and

and hence both  and  have two junction
vertices.\\
\begin{figure}[tb]
  \centering
  \includegraphics[scale=1]{9}
  \caption[]{\small A trisecting on , .}
  \label{cc}
\end{figure}
In case (c),  and  are black or different color, and two
corner vertices on the boundary of  (resp., )
facing  are black and also  is a rectangle
that  and odd. There are three cases for  and :
\par Case 1. Both  and  are the corner vertices on the boundary
of  and  facing ; see Figure
\ref{ccv}(a). Then  and  are black. By Theorem
\ref{theorem:2}, there exists a path from  to , from
 to  which does not pass through a vertex in 
and , respectively, and a Hamiltonian path from 
to  in . Therefore,

and hence both  and  have a unique junction
vertex.
\par Case 2.  is the corner vertex on the boundary of 
facing , then  is black and  is black or white; see
Figure \ref{ccv}(b). By Theorem \ref{theorem:2}, there exists a path
from  to  and from  (or ) to , where 
is black, which does not pass through a vertex, and a Hamiltonian
path from and 
to  (or from  to ), or a Hamiltonian path from  (or  to , where  is white and 
to  (or from  to ) and a path
from  to  which does not pass through a vertex . Therefore,

and

and hence  has a unique junction vertex and  have
two junction vertices (the same argument is also applied to ). In
this case, where , both  and  a unique
junction vertex; see Figure \ref{ccv}(c).
\par Case 3.  and  are not the corner vertices on the
boundary of  and  facing ; see
Figure \ref{ccv}(d). By Theorem \ref{theorem:2}, there exists a
Hamiltonian path from  to ,  to  and  to ,
where  (or ) is white, and a path form  to  and  to
 which does not pass through a vertex where  (or ) is
black,  is  or ,  is  or ,  is  or  and  is  or . Therefore,

and hence both  and  have two junction
vertices.\\
In case (d), if , the trisecting is performed
horizontally, and the claim is proved by applying the same argument
for case (b); see Figure \ref{ccv}(e). If , the trisecting
is performed vertically
and also
two corner vertices on the boundary of  facing
 are black. Therefore, the claim is proved by applying
the same argument for case (c).
\end{proof}

\begin{figure}[htb]
  \centering
  \includegraphics[scale=1]{10}
  \caption[]{\small A trisecting on ,  and .}
  \label{ccv}
\end{figure}
After trisecting, we construct a longest path in ,
 and  between  and ,  and  and 
and , respectively. In the case that none of  and
 have junction vertices (when  and both  and
 facing the common border  and ), we
construct a longest path in  (resp., ) between
 (resp.,  and a none-corner vertex of the boundary facing
 (resp., ; see Figure \ref{cd1}. At the end,
the longest paths in  are combined through the junction
vertices; see Figures \ref{cc} and \ref{ccv}.\par Then we construct
Hamiltonian cycles in rectangular grid subgraphs  to , by
Lemma \ref{Lemma:1}; see Figure \ref{cd}. Then combine all
Hamiltonian cycles to a single Hamiltonian cycle. \par Two
non-incident edges  and  are parallel, if each end vertex
of  is adjacent to some end vertex of . Using two parallel
edges  and  of two Hamiltonian cycles (or a Hamiltonian
cycle and a longest path), such as two darkened edges of Figure
\ref{cm}(a), we can combine them as illustrated in Figure
\ref{cm}(b) and obtain a large Hamiltonian cycle.\par Combining the
resulted Hamiltonian cycle with
the longest path of  is done as in Figure \ref{cn}. \\

\begin{figure}[htb]
  \centering
  \includegraphics[scale=1]{15}
  \caption[]{\small the Longest path in .}
  \label{cd1}
\end{figure}

\begin{figure}[htb]
  \centering
  \includegraphics[scale=1]{11}
  \caption[]{\small Hamiltonian cycles in  to .}
  \label{cd}
\end{figure}

\begin{figure}[htb]
  \centering
  \includegraphics[scale=1]{12}
  \caption[]{\small Combining two Hamiltonian cycles.}
  \label{cm}
\end{figure}

\begin{figure}[htb]
  \centering
  \includegraphics[scale=1]{13}
  \caption[]{\small The longest path between  and .}
  \label{cn}
\end{figure}

Considering all of the above, we get the algorithm of finding a
longest path in rectangular grid graphs, as shown in Algorithm
\ref{alg:1}.
\newcommand{\LET}{\STATE \textbf{let }}
\newcommand{\SET}{\STATE \textbf{set }}
\newcommand{\PROC}{\STATE \textbf{procedure }}
\newcommand{\RET}{\STATE \textbf{return }}
\begin{algorithm}
\caption{The longest path algorithm} \label{alg:1}
\begin{algorithmic}
\PROC LongestPath
\end{algorithmic}
\begin{algorithmic}
{\small \STATE \textbf{Step 1.} By a peeling operation, 
partitions into five disjoint rectangular grid subgraphs  to
, such that  \STATE \textbf{Step 2.} Finding
longest path between  and  in . \STATE \textbf{Step 3.} Construct Hamiltonian cycles in rectangular
grid subgraphs  to .
 \STATE \textbf{Step 4.} Construct a longest
path between  and  by combine all Hamiltonian cycles and a
longest path.}
\end{algorithmic}
\end{algorithm}

Consider the pseudo-code of our algorithm in Algorithm \ref{alg:1}.
The step 1 dose only a constant number of partitioning, during the
peeling operation, which is done in constant time. The step 2
trisects  which requires also a constant number of
partitioning. Then finds a longest path in  by merging paths of
the partitions which can be done in linear time. The step 3 finds
Hamiltonian cycles of  to  which is done in linear time.
The step 4 which combines the Hamiltonian cycles and the longest
path requires only constant time. Therefore, in total our sequential
algorithm has linear-time complexity.
\section{The parallel algorithm}
In this section, we present a parallel algorithm for the  longest
path problem. This algorithm is based on the sequential algorithm
presented in the previous sections. Our parallel algorithm runs on
every parallel machine, we do not need any inter-processor
connection in our algorithm. We assume there are  processors and
they work in SIMD mode. For simplicity, we use a two-dimensional
indexing scheme. Each vertex  of the given rectangular grid graph
 is mapped to processor . Each processor knows
its index, coordinates  and , and  and . \par The
peeling phase is parallelized easily, every processor calculates the
following four variables, in parallel \cite{CST:AFAFCHPIM}:

 
  \\


  \\


  \\


  \\
  \par
Where variables , ,  and  correspond to the
right-most column number of ,  the left-most column number of
, the bottom row number of , and the top row number of
, respectively. Then a processor can identify its
subrectangular by comparing its coordinates with these four
variables. In case (F1) and (F2), the boundary
adjustment can be done by simply decrementing , ,
 or  or incrementing  or .


\par The trisecting phase is also parallelized in a similar manner. In the following we describe
how we parallelized the horizontal trisecting, in two cases when
 is evenodd (or oddodd) and when it is
eveneven. In case 
is evenodd or oddodd, every processor simultaneously calculate the following two variables:\\
\\
\par
Where variables  and  correspond to the bottom row number of
 (resp. ), and the top row number of
 (resp. ), respectively.\par In case 
is eveneven, every processor simultaneously calculate the
following two variables:
\\
\\
\\

A similar method can be used to parallelize the vertically
trisecting.\\
After peeling and trisecting, all processors in the same
subrectangles simultaneously construct either a longest path,
Hamiltonian path or cycle according to the pattern associated with
the subrectangle. For constructing a Hamiltonian path in a
rectangular grid graph, we use the constant-time algorithm of
\cite{CST:AFAFCHPIM}. For constructing a Hamiltonian cycle in an
even-sized rectangle, we use the constant-time algorithm of
\cite{SST:EPPRAOM} in which every processor computes its successor
in the cycle. This algorithm is given in Algorithm \ref{alg:2}; see
Figure \ref{e}(a) .\par For constructing a longest path, parallel
algorithms can be easily developed for each different pattern shown
in  Figure \ref{e}(b), (c). As two examples, for constructing a
longest path between vertices  and  in an
oddodd rectangular grid graph , and vertices
 and  in an eveneven rectangular grid
graph . We have developed the simple algorithms Algorithm
\ref{alg:3} and \ref{alg:4}, respectively. The algorithms for other
patterns can be derived in the similar way. \par Then combining
phase is parallelized as follows.  The two processors at the two
endpoints of a corner edge in a Hamiltonian cycle  check
whether a neighboring Hamiltonian cycle  exists or not. If
 exists, then their successors are changed to the adjacent
processors in . Similarly, the two processors at the endpoints
of a corner edge in the longest path  in  also check the
existence of the adjacent edge in the Hamiltonian cycle , and
change their successors. Thus, the combining phase can be
parallelized in constant steps without inter-processor
communication.

\begin{figure}[htp]
  \centering
  \includegraphics[scale=1]{16}
  \caption[]{(a) A Hamiltonian cycle in , (b) and (c) two patterns of longest path in  and .}
  \label{e}
\end{figure}
\begin{algorithm}
\caption{The Hamiltonian cycle parallel algorithm for an even-sized
rectangular grid graphs} \label{alg:2}
\begin{algorithmic}
\PROC LongestPath
\end{algorithmic}
\begin{algorithmic}[1]
{\small \STATE \textbf{for} each processor  in 
\textbf{do} in parallel \STATE \textbf{if} , \textbf{then}
successor    \STATE \textbf{elseif}
(,  is odd and ) or ( and  even),
\textbf{then} successor   \STATE
\textbf{slesif}  is even and , \textbf{then} successor
  
 \STATE \textbf{else}  is odd and , \textbf{then} successor   }
\end{algorithmic}
\end{algorithm}
\begin{algorithm}
\caption{The longest path parallel algorithm for oddodd
rectangular grid graphs} \label{alg:3}
\begin{algorithmic}
\PROC LongestPath
\end{algorithmic}
\begin{algorithmic}[1]
{\small \STATE \textbf{for} each processor  in 
\textbf{do} in parallel \STATE \textbf{if}  and  or 
and , \textbf{then} successor   nill
\STATE \textbf{elseif}  is odd and , \textbf{then} successor
  \STATE \textbf{slesif}  is odd
and , \textbf{then} successor  
\STATE \textbf{elseif}  is even and , \textbf{then}
successor   
 \STATE \textbf{else}  is even and , \textbf{then} successor   }
\end{algorithmic}
\end{algorithm}


\begin{algorithm}
\caption{The longest path parallel algorithm for eveneven
rectangular grid graphs} \label{alg:4}
\begin{algorithmic}
\PROC LongestPath
\end{algorithmic}
\begin{algorithmic}[1]
{\small \STATE \textbf{for} each processor  in 
\textbf{do} in parallel \STATE \textbf{if}  and ,
\textbf{then} successor   nill \STATE
\textbf{elseif} ( is odd and ), ( is even and ) or
( and  is even)  \textbf{then} successor 
 \STATE \textbf{slesif} ( is odd and
), ( and  is even), ( and  is odd)
\textbf{then} successor   
 \STATE \textbf{elseif}  is even and  \textbf{then} successor   
 \STATE \textbf{elseif}  and  is odd \textbf{then} successor   
 }

\end{algorithmic}
\end{algorithm}

\section{Conclusion and future work} \label{ConclusionSect}
We presented a linear-time sequential algorithm for finding a longest path in a
rectangular grid graph between any two given vertices. Since the
longest path problem is NP-hard in general grid graphs
\cite{IPS:HPIGG}, it remains open if the problem is polynomially
solvable in solid grid
graphs. Based on the sequential algorithm a constant-time parallel algorithm
is introduced for the problem, which can be run on every parallel machine.\1cm]


\begin{thebibliography}{00}

\bibitem{BH:FAPOSL}
A. Bj\"{o}rklund and T. Husfeldt, Finding a path of superlogarithmic
length, SIAM J. Comput., 32(6):1395-1402, 2003.

\bibitem{BSZVGF:OCALPIAT}
R. W. Bulterman, F. W. van der Sommen, G. Zwaan, T. Verhoeff, A. J.
M. van Gasteren and W. H. J. Feijen, On computing a longest path in
a tree, Information Processing Letters, 81(2):93-96, 2002.

\bibitem{CST:AFAFCHPIM}
S. D. Chen, H. Shen and R. Topor, An efficient algorithm for
constructing Hamiltonian paths in meshes, J. Parallel Computing,
28(9):1293-1305, 2002.

\bibitem{SST:EPPRAOM}
S. D. Chen, H. Shen, R. W. Topor, Efficient parallel
permutation-based range-join algorithms on meshconnected computers,
in: Proceedings of the 1995 Asian Computing Science Conference,
Pathumthani, Thailand, Springer-Verlag, 225-238, 1995.

\bibitem{D:GT}
R. Diestel, Graph Theory, Springer, New York, 2000.

\bibitem{GN:FLPCAC}
H. N. Gabow and S. Nie, Finding long paths, cycles and circuits,
19th annual International Symp. on Algorithms and Computation
(ISAAC), LNCS, 5369:752-763, 2008.

\bibitem{GJ:CAI}
M. R. Garey and D. S. Johnson, Computers and Intractability: A Guide
to the Theory of NP-completeness, Freeman, San Francisco, 1979.

\bibitem{vyf:hpotgg}
{V. S. Gordon, Y. L. Orlovich and F. Werner, Hamiltonian properties
of triangular grid graphs, \em Discrete Math.}, 308 (2008)
6166-6188.

\bibitem{G:FALPIACMD}
G. Gutin, Finding a longest path in a complete multipartite digraph,
SIAM J. Discrete Math., 6(2):270-273, 1993.

\bibitem{Imnrx:hcihgg}
{K. Islam, H. Meijer, Y. Nunez, D. Rappaport and H. xiao,
Hamiltonian Circuts in Hexagonal Grid Graphs, \em CCCG}, (2007)
20-22.
\bibitem{IPS:HPIGG}
A. Itai, C. Papadimitriou and J. Szwarcfiter, Hamiltonian paths in
grid graphs, SIAM J. Comput., 11(4):676-686, 1982.


\bibitem{KMR:OATLPIAG}
D. Karger, R. Montwani and G. D. S. Ramkumar, On approximating the
longest path in a graph,  Algorithmica, 18(1):82-98, 1997.

\bibitem{FAA:ALAFFLPIRGG}
F. Keshavarz-Kohjerdi, A. Bagheri and A. Asgharian-Sardroud, A
Linear-time Algorithm for the Longest Path Problem in Rectangular
Grid Graphs, {\em Discrete Applied Math.}, 160(3): 210-217, 2012.

\bibitem{kb:hpiscogg}
{F. Keshavarz-Kohjerdi and A. Bagheri, Hamiltonian Paths in Some
Classes of Grid Graphs},  {\em Journal of Applied Mathematics}, accepted.

\bibitem{LU:HCISGG}
W. Lenhart and C. Umans, Hamiltonian Cycles in Solid Grid Graphs,
Proc. 38th Annual Symposium on Foundations of Computer Science (FOCS
'97), 496-505, 1997.

\bibitem{LMN:TLPPIPOIG}
K. Loannidou, G. B. Mertzios and S. Nikolopoulos, The longest path
problem is polynomial on interval graphs, Proc. of 34th Int. Symp.
on Mathematical Foundations of Computer Science, Springer-Verlag,
Novy Smokovec, High Tatras, Slovakia, 5734:403-414, 2009.

\bibitem{LM:HPOARC}
F. Luccio and C. Mugnia, Hamiltonian paths on a rectangular
chessboard, Proc. 16th Annual Allerton Conference, 161-173, 1978.

\bibitem{MC:ASPAFTLPPOCG}
G. B. Mertzios and D. G. Corneil, A simple polynomial algorithm for
the longest path problem on Cocomparability Graphs, J. Comput. Sci.,
Submitted 2010.

\bibitem{M}
{B. R. Myers, Enumeration of tours in Hamiltonian rectangular latice
graphs, \em Mathematical Magazine}, 54(1) (1981), 19-23.

\bibitem{6}
{M. Nandi, S. Parui and A. Adhikari, The domination numbers of
cylindrical grid graphs, \em Applied Mathematics and Computation},
217(10) (2011) 4879-4889.

\bibitem{AEB:SSAG}
A. N. M. Salman, Contributions to Graph Theory, Ph.D. Thesis,
University of Twente, (2005).

\bibitem{UU:OCLPISGC}
R. Uehara and Y. Uno, On Computing longest paths in small graph
classes, Int. J. Found. Comput. Sci., 18(5):911-930, 2007.

\bibitem{CT:HPOGG}
C. Zamfirescu and T. Zamfirescu, Hamiltonian Properties of Grid
Graphs, {\em  SIAM J. Math.}, 5(4) (1992) 564-570.

\bibitem{ZL:AFLPIG}
Z. Zhang and H. Li, Algorithms for long paths in graphs, Theoretical
Comput. Sci., 377(1-3):25-34, 2007.

\bibitem{wqz}
W. Q. Zhang and Y. J. Liu, Approximating the longest paths in grid
graphs, Theoretical Computer Science, 412(39): 5340-5350, 2011.

\end{thebibliography}

\end{document}
