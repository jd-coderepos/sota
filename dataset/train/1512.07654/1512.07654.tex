Sporadic Servers (SS)~\cite{Sprunt90} and Priority Inheritance
Bandwidth-preserving Servers (PIBS)~\cite{DanishLiWe11} are the two scheduling
models used in the Quest real-time operating system~\cite{quest-webpage}.
Sporadic Servers are specified using a budget capacity, , and period
. By default, the Sporadic Server with the smallest period is given highest
priority, which follows the rate-monotonic policy~\cite{LiuLa73}. The main
tasks in Quest run on Sporadic Servers, thereby guaranteeing them a minimum
share of CPU time every real-time period.  Replenishment lists are used to
track the consumption of CPU time and when it is eligible to be re-applied to
the corresponding server.

PIBS uses a much simpler scheduling method which is more appropriate for the
short execution times associated with interrupt bottom half threads.  A PIBS
is specified by a utilization, .  A PIBS always runs on behalf of a
Sporadic Server and inherits both the priority and period of the Sporadic
Server.  For example, the PIBS running in response to a device interrupt would
run on behalf of the Sporadic Server that requested the I/O action to be
performed.  The capacity of a PIBS is calculated as , where
 is the period of the Sporadic Server.

As with Sporadic Servers, PIBS uses replenishments but instead of a list there
is only a {\em single} replenishment.  When a PIBS has executed
, its next replenishment is set to ,
where  is the time the PIBS started its most recent execution.  A PIBS cannot
execute again until the next replenishment time regardless of whether it has
utilized its entire budget or not. Since a PIBS uses only one replenishment
value rather than a list, it is beneficial for scheduling short-lived interrupt
service routines that would otherwise fragment a Sporadic Server's budget into
many small replenishments. The replenishment method of a PIBS limits its maximum
utilization within any sliding window of size  to .  This
occurs when the PIBS first runs for  and then again for
.  This is demonstrated in Figure~\ref{fig:PIBS}:



\begin{figure}[h]
  \centering
  \includegraphics[width=0.5\textwidth]{figures/PIBS.eps}
  \caption{PIBS Server Utilization}
  \label{fig:PIBS}
\end{figure}

The interaction between Sporadic Servers and PIBS is depicted in
Figure~\ref{fig:vcpus}.  First, the Sporadic Server initiates an I/O related
system call (Step 1).  The system call invokes the associated device driver,
which programs the device (Step 2).  The device will eventually initiate an
interrupt which is handled by the top half interrupt handler (Step 3).  The
top half interrupt handler will acknowledge the interrupt and wake up one of
the PIBS to handle the majority of the work associated with the interrupt
(Step 4).  Note that although the figure shows PIBS run at kernel-level, they
could just as well be associated with user-space threads if the system granted
such privileges.  Finally, after a PIBS finishes executing it will wake up the
corresponding Sporadic Server, assuming the Sporadic Server was blocked on an
I/O request (Step 5).  A more detailed description of the scheduling of PIBS
and Sporadic Servers can be found in prior work (2011)~\cite{DanishLiWe11}.

\begin{figure}[h]
  \vspace{0.1in}
  \centering
  \includegraphics[width=0.44\textwidth]{figures/vcpus.eps}
  \vspace{0.1in}
  \caption{Sporadic Server and PIBS Interaction}
  \label{fig:vcpus}
  \vspace{-0.3in}
\end{figure}

The advantages of using PIBS for bottom half interrupt handling include lower
scheduling overhead and no delayed execution due to replenishment list
fragmentation.  The short execution time of bottom half interrupt handlers can
cause a Sporadic Server to complete execution before exhausting its available
capacity. This leads to a fragmented replenishment list. In practice, the lists
are finite in length because of memory constraints and to limit scheduling
overhead. When a replenishment list is full, items are merged to make space for
new replenishments. This causes the available budget to be
deferred~\cite{StanovichBaWa10}, and the effective utilization of the Sporadic
Server can drop below its specified value. This in turn results in deadlines
being missed.  In contrast, PIBS have only a single replenishment list item and
a different policy for how the replenishment is posted, which prevents a drop in
their effective utilization and lower scheduling overhead.

Figure~\ref{fig:gantt_ss_only} shows an example of replenishment list
fragmentation.  The first task, , begins execution at time  and
continues execution for eight time units.   utilized its entire
capacity at  so a single single replenishment item is posted for 8 time
units of capacity at time .  The replenishment is posted at 
because  started execution at time  and has a period of 16.
Right at the completion of its execution  initiates an I/O related
event, e.g. a read.  Suppose this I/O event causes four interrupts to
occur.  Each interrupt initiates a bottom half thread that takes one time unit
of computation to complete.   will require all four bottom half
interrupt handlers to complete execution before it can continue execution,
e.g. it is blocking on the read.   is the task responsible for
handling these bottom half interrupt handlers.  The first interrupt occurs at
time  and is immediately handled by .  Note that at time
 the time of the head replenishment list item is updated from zero to
nine.  This is to ensure that when a replenishment item is posted after the
task blocks or depletes its budget the replenishment item is posted at the
correct time.  Once  completes execution of the bottom half interrupt
handler it blocks as it waits for another I/O interrupt to occur.

When  blocks it posts a replenishment item for the capacity that it
used.  Since it used 1 time unit of capacity and started executing at time
 a replenishment item of 1 time unit is posted at time .  At
time  another interrupt occurs, waking up  for another time
unit.  The time of the first replenishment list item is updated to 11 to
reflect that the Sporadic Server started execution at time .  After
handling the bottom half interrupt handler, another replenishment item for one
time unit is posted, this time at time .  When the third interrupt
occurs  again executes for 1 time unit.  However, when 
attempts to post a replenishment item for the one unit of capacity used it
cannot since its replenishment list is full.\footnote{For the sake of this
example the replenishment list size is three.  In practice, a larger
replenishment list size would be chosen but, regardless, fragmentation and
capacity postponement can occur~\cite{DanishLiWe11}.}  In order to ensure that
 does not adversely affect other running tasks, its remaining capacity
of one time unit is merged with the next replenishment list item, which in
this example is at .  This results in the available capacity for
 being zero, leaving it unable to immediately handle the interrupt
that occurs at time .  Instead, the execution of the interrupt is
delayed and completes only at time .  Meanwhile, , which had
the capacity to execute at time , is blocking waiting for completion of
the fourth interrupt handler.   begins execution at time  but
that leaves only six time units until the deadline at time , instead
of the eight required to complete execution.

\begin{figure}[h]
  \centering
  \includegraphics[width=0.5\textwidth]{figures/gantt-ss-only.eps}
  \vspace{0.05in}
  \caption{Example Task and I/O Scheduling using Sporadic Servers}

  \label{fig:gantt_ss_only}
\end{figure}

Figure~\ref{fig:gantt_ss_pibs} shows a similar scheduling scenario. However,
this time the interrupt bottom halves are handled by a PIBS.  As with the
previous scenario,  initiates an I/O related event at time  and
blocks until the completion of the event.  The first interrupt occurs at time
 and is immediately handled by PIBS.  As with the Sporadic Server, the
time in the replenishment list item is updated to reflect when the PIBS
started execution.  Once the event is handled, the PIBS posts a single
replenishment item at time .  This is because  is running on
behalf of  so it inherits both the priority and period of .
Since  executed for only 25\% of its available four time units of
capacity the replenishment is posted 25\% of its period from when it started
execution.  The second interrupt occurs at time  but its execution is
deferred until  has available capacity.  At time  the third
interrupt arrives and  has the capacity to handle both it and the
previous interrupt.  Finally, the fourth interrupt arrives at time ,
which can also be handled by .  Since  has executed for 75\%
of its available capacity a replenishment is posted twelve time units after it
started execution, at .  This permits  to continue execution
at time .  The pattern then repeats itself.  This simple example
demonstrates the advantages of PIBS for bottom half threads compared to
Sporadic Servers.  Finally, note that even if the replenishment list in the
first example had been long enough to avoid the delayed budget, the Sporadic
Server would have experienced twice as much context switching overhead
compared to the equivalent PIBS.

\begin{figure}[h]
  \vspace{-0.15in}
  \centering
  \includegraphics[width=0.5\textwidth]{figures/gantt-ss-pibs.eps}
  \vspace{0.05in}
  \caption{Example Task and I/O Scheduling using Sporadic Servers \& PIBS}

  \label{fig:gantt_ss_pibs}
  \vspace{-0.15in}
\end{figure}

Note that in the first example, if a different policy for handling a full
Sporadic Server replenishment list had been chosen,  might have
completed in time for  to finish before its deadline.  For example, if
the later replenishment items were merged instead of the head replenishment
item,  would have had one remaining time unit of capacity to handle
the last bottom half interrupt handler.  However, as more interrupts occur,
this temporary fix will no longer work as more capacity is delayed further in
time.

\subsection{Response Time Analysis for SS and PIBS}

In order to perform an Adaptive Mixed-Criticality analysis for a combined
Sporadic Server and PIBS system, the response time analysis equation of the
system must be derived.  First, under the assumption that a Sporadic Server
can be treated as an equivalent periodic task~\cite{Sprunt90}, the response
time equation for task  in a system of only Sporadic Servers is the
following:

where  is the set of tasks of equal or higher
priority than task . Second, due to the worst-case phasing of a
combined system of PIBS and Sporadic Servers, a PIBS utilization bound of
 cannot repeatedly occur.  The worst case phasing can
result in at most an additional capacity (i.e., execution time) of  for PIBS  assigned to the Sporadic Server
. This is only possible if PIBS blocks waiting on I/O before consuming
its full budget capacity.
Therefore, a
tighter upper-bound on the interference a PIBS can cause is:

This can be incorporated into the response time analysis of Sporadic Server
, in a system consisting of both Sporadic Servers and PIBS, in the
following way:

Where  is the set of all PIBS and \mbox{}, i.e. the set
containing  and all tasks with equal or higher priority than task
.  This is necessary as the PIBS can be running on behalf of task
.  In general, there is no a-priori knowledge about which PIBS runs for
which Sporadic Server.  Therefore, the Sporadic Server,  that maximizes
the interference caused by the PIBS must be considered.  If such a-priori knowledge
existed, it could be used to reduce the possible set of Sporadic Servers on
  behalf of which a PIBS could be executing. However, without such knowledge all possible
Sporadic Server tasks of equal or higher priority must be considered.

The response time analysis for a PIBS is therefore dependent on the associated
Sporadic Server.  The response time analysis for PIBS  when assigned to
Sporadic Server  is:

Note that  is the maximum execution time of the PIBS
over a time window of , i.e.  .  Besides the first terms differing,
Equation~\ref{eq:pibs_rta} differs from Equation~\ref{eq:ss_rtb} in that
 is used instead of 
for the set of Sporadic Servers.  This is because Sporadic Server 
must be included as it has an equal priority to PIBS  when  is
running on behalf of .  Also, the summation over all PIBS does not
include PIBS  when determining its response time.  If
, for each and every Sporadic Server  that 
can be assigned to, then  will never miss a deadline.













