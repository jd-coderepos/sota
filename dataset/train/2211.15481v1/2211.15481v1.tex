A discutir -------------------------------------------------
- Si no crees que tenga algùn valor el experimenrto con la RN creo que es mejor no poner eso .. porque no tiene sentido hacer un experimento que se supone que no funcione, y se muestra que no funciona, porque eso n aporta conocimiento ... casi eso es lo que saldrìa de la frase " Then it is not surprising that given the complexity
of the presented dataset the models did not obtained good results" ... en mi opinion, enfocar el trabajo al dataset, poniendo màs informaciòn sobre èl, podrìa quedar mejor que asi, porque la parte de RN creo que le bajara puntos .. a menos que haya algo interesante que mostrar de los resultados

- En el resumen creo que hay que resaltar lo que significa que sea "continuo" porque eso es lo que da el matiz novedoso.
- En la seccion 2 creo que conviene separar en lo posible lo otro existente (lo previo) del tuyo, quizas indicando luego de exponer lo previo la necesidad del tuyo, y ahi decir sus singularidades, reforzado con la tabla.

- hablas de "trigrams" aqui y no es una base de datos en que el texto sea la informacion primaria... si es relevante, creo que merece explicarlo porque creo que queda claro la utilidad de eso aqui

- Entiendo la necesidad de la nueva metrica, pero seguro los revisores exigiran una mayor justificacion y el problema de las otras mètricas ... incluso, creo que debes dejar claro si considerar que las comparaciones con este baseline que das siempre debe hacerse respecto a esta nueva metrica, o habrìa otra que tendria sentido para la comparacion


A RESOLVER ------------------------------------------

- En la primera linea de 1.1 tratar de evitar que quede dividido "LSA-T"
https://tex.stackexchange.com/questions/182569/how-to-manually-set-where-a-word-is-split

- EN la tabla. Queda Raro FHD.  Se solucionaría si ponemos Language= Argentinian ?