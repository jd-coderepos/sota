We designed our experimental analysis with the aim of running a thorough evaluation of the impact of style transfer data augmentation on domain generalization. Besides observing how this data augmentation can improve the standard learning baseline model, and how it compares with the most recent state of the art DG methods, we are also interested in the effectiveness of their combination. 
In the following we provide details on the chosen data testbeds and sota models, describing how the data augmentation strategy is integrated in each approach.

\subsection{Datasets}
We consider three standard benchmark datasets which differ in number of classes and covered domains. 

\paragraph{PACS~\cite{hospedalesPACS}} contains images of 7 object classes spanning 4 visual domains: Photo, Art Painting, Cartoon, Sketch. Given that the visual domains go from real world representations to artistic images, the style variability is quite large. We follow the original experimental protocol by training on the train splits of three source domains (using the validation splits for model selection), and then testing on the whole left out domain which acts as unknown target.  
\paragraph{OfficeHome~\cite{venkateswara2017Deep}} 
is similar to PACS, it covers 4 domains (Art, Clipart, Product and Real-World) but shows a much larger set of 65 object classes. We adopt the same experimental protocol of~\cite{Antonio_GCPR18}: a random 90-10 train-val split is used to select the training images for the 3 source domains (once again the validation images are used for model selection) and testing is performed on the whole left out target domain.
\paragraph{VLCS~\cite{TorralbaEfros_bias}} is built upon 4 different datasets: PASCAL VOC 2007, Labelme, Caltech and SUN and contains 5 object categories. Differently from the other considered testbeds, all the domains are composed of real world photos with the shift mainly due to camera type, illumination conditions, point of view,  \etc. Moreover, while Caltech is composed by object-centered images, the other three domains  contain scene images.
We apply the same experimental protocol of~\cite{jigsawCVPR19}: the predefined full training data is randomly partitioned in train and validation sets with a 90-10 ratio. The training is performed on the train splits of the 3 source domains while the validation splits are used for model selection. At the end the model is tested on the predefined test split of the left out domain. This split has been defined randomly by selecting 30\% of images of the overall dataset.

All our results are obtained by performing an average over 3 runs. In the case of both OfficeHome and VLCS the random 90-10 train-val split was repeated for each run.


\subsection{Comparison methods}
For our study we consider as main \emph{Baseline} a classification model learned on all the source data and na\"{\i}vely applied on the target. We indicate with \emph{Original} the standard data augmentation with horizontal flippling and random cropping, while we use \emph{Stylized} to specify the cases where we add style transfer data augmentation.
The behavior of four among the most recent DG methods is evaluated under both these augmentation settings. We dedicate a particular attention to the integration of the style transfer data augmentation strategy with each of the considered approaches. The goal is getting the most out of them without undermining their nature. In particular, considering that the style transfer leads to domain mixing, it is important to not integrate it in procedures that need a separation among source domains.
\paragraph{DG-MMLD~\cite{dg_mmld}} this approach exploits clustering and domain adversarial feature alignment. Since it does not need the source domain labels, the integration of the proposed style transfer data augmentation is straightforward: styles of random images are applied to each content images (inside a batch) with probability $p$, exactly as done for the Baseline.
\paragraph{Epi-FCR~\cite{episodic_hospedales}} is a meta-learning method which splits the network in two modules, each one is trained by pairing it with a partner that is badly tuned for the domain considered in the current learning episode. The modules are the feature extractor and the classifier which alternatively cover the two roles of learning part and bad reference. After this phase, a final model is learned by integrating the trained modules together with a random classifier used as regularizer. In the first stage, knowing the source domain labels is crucial to choose and set the two network modules, thus mixing the domains with style transfer augmentation could degrade its performance. In the ending stage instead, all the source data are considered together: we applied here the style data augmentation.  
\paragraph{DDAIG}~\cite{zhou2020deep} is a data augmentation strategy based on a transformation network which is trained so that every synthesized sample keeps the same label of the original image, but fools a domain classifier. In the learning procedure the transformation module, the label classifier and the domain classifier are iteratively updated. In particular the label classifier is trained on all the source data, both original and synthetic: we further extended this set with style transfer augmented data. 
\paragraph{Rotation~\cite{lopez_rotation}} it has been shown that self-supervised knowledge supports domain generalization when combined with supervised learning in a multi-task model. In particular we focused on rotation recognition, where the orientation angle of each image should be recognized among $\{0^\circ, 90^\circ, 180^\circ, 270^\circ\}$. The model minimizes a linear combination of the supervised and self-supervised loss with weight $\eta$ generally kept lower than 1 to let the supervised model guide the learning process.
In this case the domain labels are not used during training, so the application of the source augmentation by style transfer is straightforward.

An approach related to data augmentation, originally defined to improve generalization in standard in-domain learning, is \emph{Mixup}~\cite{zhang2018mixup}: it interpolates samples and their labels, regularizing a neural network to favor a simple linear behavior between training examples. Its hyper-parameter $\gamma \in \{0,\infty\}$ controls the strength of interpolation between data pairs, recovering the Baseline for $\gamma=0$. In our study we consider Mixup as further reference, and in particular we tested data mixing both at pixel and at feature level~\cite{xu2020adversarial}.

\subsection{Training setup}

Our style transfer model $A$ is trained on source data before training the classification model $C$. As already mentioned, $A$ is implemented by AdaIN~\cite{Huang_2017_ICCV_adain} and is therefore based on a VGG backbone. It is trained for 20 epochs with a learning rate equal to 5e-5. The hyperparameters $\alpha$ and $p$ used in each experiment are specified in the caption of the respective result tables and in depth analysis on the sensitivity of the method to them is presented in Section \ref{sec:sensitivity}.

For the classification model $C$ we use AlexNet and ResNet18 backbones. Specifically, \emph{Baseline}, \emph{Rotation} and \emph{Mixup} are trained using SGD with $0.9$ momentum for $30k$ iterations. We set the batch size to $32$ images per source domain: since in all the testbed there are three source domains each data batch contains $96$ images. The learning rate and the weigh decay are respectively fixed to $0.001$ and $0.0001$. Regarding the hyperparameters of the individual algorithms, we empirically set the \emph{Rotation} auxiliary weight to $\eta = 0.5$ and for \emph{Mixup} $\gamma= 0.4$.

We implement \emph{Rotation} by adding a rotation recognition branch to our Baseline.
For \emph{DG-MMLD}, \emph{Epi-FCR} and \emph{DDAIG}, we use the code provided by the authors integrating different datasets/backbones where needed. The training setup for these experiments is the one defined in their papers for both the \emph{Original} and \emph{Stylized} version. 
We report the previously published results whenever possible. In the following we will indicate with a star ($^*$) the results we obtained by running the authors' code.

\subsection{Results analysis}
\begin{table}[]
    \centering
    \caption{PACS classification accuracy (\%). We used AdaIN with $\alpha=1.0$ and $p=0.75$ for AlexNet-based experiments and AdaIN with $\alpha=1.0$ and $p=0.90$ for those based on ResNet18.} \vspace{-2mm}
    \resizebox{0.5\textwidth}{!}{
    \begin{tabular}{c@{~~}c@{~~}|c@{~~}c@{~~}c@{~~}c@{~~}|c}
    \hline
    \multicolumn{7}{c}{ AlexNet } \\
    \hline
    & & Painting & Cartoon & Sketch & Photo & Average \\
    \hline
    \multirow{5}{*}{ Original }& Baseline & 66.83 & 70.85 & 59.75 & 89.78 & 71.80 \\
    & Rotation & 65.66 & 71.89 & 62.15 & 89.88 & 72.39 \\
    & DG-MMLD & 69.27 & 72.83 & 66.44 & 88.98 & 74.38 \\
    & Epi-FCR & 64.70 & 72.30 & 65.00 & 86.10 & 72.03 \\
    & DDAIG* & 62.77 & 67.06 & 58.90 & 86.82 & 68.89 \\
    \hline
    \multirow{5}{*}{ Stylized } & Baseline & 71.96 & 72.47 & 76.47 & 88.34 & \textbf{77.31} \\
    
    & Rotation & 71.74 & 73.39 & 75.98 & 89.22 & 77.59 \\
    & DG-MMLD & 70.50 & 70.84 & 75.39 & 88.43 & 76.29 \\
    & Epi-FCR & 65.19 & 69.54 & 71.97 & 83.43 & 72.53 \\
    & DDAIG & 69.35 & 71.10 & 70.99 & 87.70 & 74.79 \\
    \hline 
    \multirow{2}{*} {Mixup}
    & pixel-level & 66.03 & 68.00 & 51.18 & 88.90 & 68.53 \\
    & feature-level & 67.04 & 69.10 & 55.40 & 88.88 & 70.11 \\
    \hline
    \multicolumn{7}{c}{ ResNet18 } \\
    \hline
    \multirow{5}{*}{ Original } & Baseline & 77.28 & 73.89 & 67.01 & 95.83 & 78.50 \\
    & Rotation & 78.16 & 76.64 & 72.20 & 95.57 & 80.64 \\
    & DG-MMLD & 81.28 & 77.16 & 72.29 & 96.06 & 81.83 \\
    & Epi-FCR & 82.10 & 77.00 & 73.00 & 93.90 & 81.50 \\
    & DDAIG* & 79.41 & 74.81 & 69.29 & 95.22 & 79.68 \\
    \hline
    \multirow{6}{*}{ Stylized } & Baseline & 82.73 & 77.97 & 81.61 & 94.95 & \textbf{84.32} \\
    & Rotation & 79.51 & 79.93 & 82.01 & 93.55 & 83.75 \\
    & DG-MMLD & 80.85 & 77.10 & 77.69 & 95.11 & 82.69 \\
    & Epi-FCR & 80.68 & 78.87 & 76.57 & 92.50 & 82.15 \\
    & DDAIG & 81.02 & 78.75 & 79.67 & 95.07 & 83.63 \\
    \hline 
    \multirow{2}{*} {Mixup}
    & pixel-level & 78.09 & 71.08 & 66.58 & 93.85 & 77.40 \\
    & feature-level  & 81.20 & 76.41 & 69.67 & 96.31 & 80.90 \\
    \hline
    \end{tabular} 
    }
    \label{tab:pacs}\vspace{-5mm}
\end{table} \begin{table}[]
    \centering
    \caption{OfficeHome classification accuracy (\%). We used AdaIN with parameters $\alpha=1.0$ and $p=0.1$.} \vspace{-2mm}
    \resizebox{0.5\textwidth}{!}{
    \begin{tabular}{c@{~~}c@{~~}|c@{~~}c@{~~}c@{~~}c@{~~}|c}
    \hline
    \multicolumn{7}{c}{ ResNet18 } \\
    \hline
    & & Art & Clipart & Product & Real World & Average \\
    \hline
    \multirow{5}{*}{ Original } & Baseline & 57.14 & 46.96 & 73.50 & 75.72 & 63.33 \\
    & Rotation & 55.94 & 47.26 & 72.38 & 74.84 & 62.61 \\
    & DG-MMLD* & 58.08 & 49.32 & 72.91 & 74.69 & 63.75 \\
    & Epi-FCR* & 53.34 & 49.66 & 68.56 & 70.14 & 60.43 \\
    				
    & DDAIG* & 57.79 & 48.32 & 73.28 & 74.99 & 63.59 \\
    \hline
    \multirow{5}{*}{ Stylized } & Baseline & 58.71 & 52.33 & 72.95 & 75.00 & \textbf{64.75} \\
    & Rotation & 57.24 & 52.15 & 72.33 & 73.66 & 63.85 \\
    & DG-MMLD & 59.24 & 49.30 & 73.56 & 75.85 & 64.49 \\
    & Epi-FCR & 52.97 & 50.14 & 67.03 & 70.66 & 60.20 \\
    & DDAIG & 58.21 & 50.26 & 73.81 & 74.99 & 64.32 \\
    \hline
    Mixup & feature-level & 58.33 & 39.76 & 70.96 & 72.07 & 60.28 \\
    \hline
    \end{tabular} 
    }
    \label{tab:officehome} \vspace{-4mm}
\end{table}
 \begin{table}[]
    \centering
    \caption{VLCS classification accuracy (\%). We used AdaIN with parameters are $\alpha=1.0$ and $p=0.75$.}\vspace{-2mm}
    \resizebox{0.5\textwidth}{!}{
    \begin{tabular}{c@{~~}c@{~~}|c@{~~}c@{~~}c@{~~}c@{~~}|c}
    \hline
    \multicolumn{7}{c}{ AlexNet } \\
    \hline
    & & CALTECH & LABELME & PASCAL & SUN & Average \\
    \hline
    \multirow{5}{*}{ Original } & Baseline & 94.89 & 59.14 & 71.31 & 64.64 & 72.49 \\
    & Rotation & 94.50 & 61.27 & 68.94 & 63.28 & 72.00 \\
    & DG-MMLD* & 96.94 & 59.10 & 68.48 & 62.06 & 71.64 \\
    & Epi-FCR* & 91.43 & 61.36 & 63.44 & 60.07 & 69.07 \\
    & DDAIG* & 95.75 & 60.18 & 65.48 & 60.78 & 70.55 \\
    \hline
    \multirow{5}{*}{ Stylized } & Baseline & 96.86 & 60.77 & 68.18 & 63.42 & 72.31 \\
    & Rotation & 96.86 & 60.77 & 68.18 & 63.42 & 72.31 \\
    & DG-MMLD & 97.49 & 61.02 & 64.23 & 62.37 & 71.28 \\
    & Epi-FCR & 92.69 & 58.18 & 62.59 & 57.87 & 67.83 \\
    & DDAIG & 97.48 & 60.48 & 65.19 & 62.57 & 71.43 \\
    \hline
    Mixup & feature-level & 94.73 & 62.15 & 69.82 & 62.98 & 72.42 \\
    \hline
    \end{tabular}
    }
    \label{tab:vlcs}\vspace{-4mm}
\end{table} Table \ref{tab:pacs} shows results on PACS benchmark with both AlexNet and ResNet18 backbones. 
We get two main outcomes. (1) There is an evident improvement of more than 5 percentage points in the Baseline performance when using the stylized augmented source data with respect to the original case.
Looking at the results for the different domains we can see that improvement is higher for Art Painting, Cartoon and Sketch, than in Photo. 
(2) All the considered state of the art DG methods benefit from the source augmentation. Indeed in absolute terms their performance grows, but at the same time they lose in effectiveness as they cannot outperform the Baseline any more.


Table \ref{tab:officehome} shows results on OfficeHome dataset with ResNet18 backbone. Even if in this case the improvement produced by the source augmentation by style transfer is more limited, the results confirm what we have already observed for PACS. The Stylized Baseline obtains the best accuracy outperforming the competitor state of the art methods, even when those are improved using the same source augmentation.

Table \ref{tab:vlcs} reports results on VLCS benchmark with AlexNet backbone. This dataset is particularly challenging and shows a fundamental limit of tackling DG through style transfer data augmentation. Since the domain shift is not originally due to style differences in this testbed, source augmentation by style transfer does not support generalization.

As a final remark, we focus on Mixup. The results over all the considered datasets show that it is not able to generalize across domains and it might perform even worse that the Original Baseline. Between the two considered pixel and feature variants, only the second shows some advantage on PACS, so we focused on it in the other tests. Still, its results remain lower than those obtained by the DG methods both with and without style based data augmentation. 

\subsection{Analysis of AdaIN hyperparameters}
\label{sec:sensitivity}
In Figures \ref{fig:varying_alpha} and \ref{fig:varying_p} we see how the PACS AlexNet results change when varying either $\alpha$ or $p$ by keeping the other fixed. With a low value of $\alpha$ the style transfer is too weak to produce an effective appearance change of the source sample and introduce extra variability. In general the best results are obtained using $\alpha = 1$ regardless of the specific value of $p$. 

For what concerns the value of $p$ we can see that, if $\alpha$ is high enough, even a small $p$ allows to obtain good performance with the best results obtained with $p=0.5$ or $p=0.75$.

\begin{figure}
    \centering
    \includegraphics[width=0.95\linewidth]{images/Parameters/Accuracy_alpha.pdf}\vspace{-5mm}
    \caption{Average accuracy on PACS AlexNet with different values of $p$ when varying $\alpha$.} 
    \label{fig:varying_alpha}\vspace{-5mm}
\end{figure}
\begin{figure}
    \centering
    \includegraphics[width=0.95\linewidth]{images/Parameters/Accuracy_p.pdf}\vspace{-5mm}
    \caption{Average accuracy on PACS AlexNet with different values of $\alpha$ when varying $p$.}
    \label{fig:varying_p} \vspace{-5mm}
\end{figure}


\subsection{Style transfer from external data vs source data} 
The described procedure for the application of AdaIN differs from what appeared in previous works. Indeed, both the original approach~\cite{Huang_2017_ICCV_adain} and its use for data augmentation in~\cite{zhang2020learning}, exploit the style transfer model trained on MS-COCO~\cite{mscoco} as content images, and paintings mostly collected from WikiArt~\cite{wikiart} as style images. In our study we did not allow extra datasets besides those directly involved in the domain generalization task as source domains. The reason is twofold: first, we want to keep the method as simple as possible, without the need of relying on external data; second, to perform a fair benchmark with the competitors DG methods all of them should have access to the same source information. 

Still, the interested reader may wonder what would be the effect of using the original AdaIN model trained on MSCOCO and WikiArt. Figure \ref{fig:transfer_comparison} shows one example obtained in this way. Specifically we consider a dog image drawn from the PACS Photo domain and we analyse the images obtained by borrowing the style form the Art Painting guitar image. 
We compare the stylized sample produced with the MSCOCO-WikiArt AdaIN model against the outcomes of the four AdaIN variants trained on the source with every one of the four domains used as target. 

As can be observed, the obtained results in terms of image quality are not so different. We also run a quantitative analysis: in Table \ref{tab:adain_training} we compare the performance of the our Stylized Baseline on PACS AlexNet with the analogous Baseline trained using the augmented data produced with the AdaIN MSCOCO-WikiArt pretrained model. The last one shows a slightly better accuracy which is though not significant if we consider the related standard deviation.

\begin{figure}
\centering
\begin{tabular}{ccc}

\includegraphics[width=0.25\linewidth]{images/adain_wiki_source/056_0024.jpg} &
\includegraphics[width=0.25\linewidth]{images/adain_wiki_source/pic_001.jpg} &
\includegraphics[width=0.25\linewidth]{images/adain_wiki_source/056_0024-to-pic_001_wiki.jpg}
\end{tabular}\\
\begin{tabular}{cccc}
\includegraphics[width=0.2\linewidth]{images/adain_wiki_source/056_0024-to-pic_001_no_art_painting.jpg} &
\includegraphics[width=0.2\linewidth]{images/adain_wiki_source/056_0024-to-pic_001_no_cartoon.jpg} &
\includegraphics[width=0.2\linewidth]{images/adain_wiki_source/056_0024-to-pic_001_no_sketch.jpg} &
\includegraphics[width=0.2\linewidth]{images/adain_wiki_source/056_0024-to-pic_001_no_photo.jpg}
\end{tabular}

\caption{Example of application of style transfer using AdaIN. The top left image comes from the PACS Photo domain and is used as content while the top center image comes from PACS Art Painting domain and is used as style image. On top right there is the translation performed using AdaIN trained on MS-COCO and WikiArt images. In the second row we see the translations performed using our AdaIN models trained on source data only, respectively when the Art Paintings, Cartoon, Sketch and Photo domains are used as style sources.}
\label{fig:transfer_comparison}\end{figure}
\begin{table}[tb]
    \centering
    \caption{Comparison of AdaIN training strategies}
    \resizebox{0.5\textwidth}{!}{
    \begin{tabular}{c@{~~}|c@{~~}c@{~~}c@{~~}c@{~~}|c}
    \hline
    & Art Painting & Cartoon & Sketch & Photo & Average \\
    \hline
    Stylized Baseline & $71.96$ & $72.47$ & $76.47$ & $88.34$ & $77.31 \pm 1.1$ \\
    MSCOCO-WikiArt  Baseline & $73.00$ & $73.78$ & $76.37$ & $89.04$ & $\textbf{78.05} \pm 0.9$ \\    
    \hline
    \end{tabular}
    }
    \label{tab:adain_training}\end{table}
