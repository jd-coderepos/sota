\documentclass[letterpaper,12pt]{article}
\usepackage{amssymb, amsmath}
\usepackage{amsthm}
\usepackage{algorithm}
\usepackage{algorithmic}
\usepackage{enumerate}
\usepackage[numbers,square,comma,sort&compress]{natbib}
\usepackage{a4}

\newtheorem{theorem}{Theorem}
\newtheorem{corollary}[theorem]{Corollary}
\newtheorem{fact}[theorem]{Fact}
\newtheorem{lemma}[theorem]{Lemma}
\newtheorem{definition}[theorem]{Definition}
\newcommand{\comment}[1]{}
\newcommand\bm[1]{\mbox{\boldmath\unboldmath}} 
\newcommand\bs[1]{\boldsymbol{#1}}
\DeclareMathOperator*{\argmin}{argmin}


\begin{document}
\title{A lower bound for metric -median selection \footnote{A
preliminary version of this paper appears in {\em Proceedings of the 30th
Workshop on Combinatorial Mathematics and Computation Theory}, Hualien,
Taiwan, April 2013, pp.~65--68.}}

\author{
Ching-Lueh Chang \footnote{Department of Computer Science and
Engineering, Yuan Ze University, Taoyuan, Taiwan. Email:
clchang@saturn.yzu.edu.tw}
\footnote{Supported in part by the National Science Council
of Taiwan under
grant
101-2221-E-155-015-MY2.}
}


\maketitle

\begin{abstract}
Consider the problem of finding
a point in
an -point
metric space
with the minimum average distance to
all
points.
We show
that
this problem
has no deterministic
-query -approximation algorithms.
\end{abstract}

\section{Introduction}

Given oracle access to
a
metric
space ,
the {\sc metric -median}
problem asks for a point with the minimum average distance to all points.
Indyk~\cite{Ind99, Ind00} shows that {\sc metric -median} has a
Monte-Carlo -time -approximation algorithm
with an  probability of success.
The more general {\sc metric -median} problem asks for
,
, ,  minimizing
.
Randomized
as well as
evasive
algorithms are well-studied for
{\sc metric -median} and the related -means problem~\cite{GMMMO03,
MP04, AGKMMP04, Che09, KSS10, JKS12},
where  is part of the input rather than a constant.


This paper focuses on {\em deterministic sublinear-query} algorithms for
{\sc metric -median}.
Guha et al.~\cite[Sec.~3.1--3.2]{GMMMO03} prove
that {\sc metric -median} has
a deterministic
-time
-space
-approximation algorithm
that reads distances in a single pass, where .
Chang~\cite{Cha13} presents a deterministic
nonadaptive
-time -approximation algorithm for {\sc metric -median}.
Wu~\cite{Wu14} generalizes Chang's result by showing an
-time
-approximation
algorithm for
any integer .
On the negative side,
Chang~\cite{Cha12}
shows that {\sc metric -median}
has no
deterministic
-query
-approximation algorithms for any constant
~\cite{Cha12}.
This
paper
improves upon his result by showing
that {\sc metric -median} has no deterministic
-query -approximation algorithms for any
constant .

In social network analysis,
the importance of an actor in a network
may
be
quantified by
several
centrality measures, among which the closeness centrality of an actor is
defined to be
its average distance to other actors~\cite{WF94}.
So
{\sc metric -median}
can
be interpreted as the problem of finding the
most important point in a metric space.
Goldreich and Ron~\cite{GR08} and Eppstein and Wang~\cite{EW04} present
randomized algorithms for approximating the closeness centralities of
vertices in undirected graphs.



\section{Definitions}\label{definitionssection}

For ,
denote
.
Trivially, .
An -point metric space  is the set , called the groundset,
endowed with a function
 satisfying
\begin{enumerate}[(1)]
\item\label{nonnegative}
 (non-negativeness),
\item
 if and only if  (identity of indiscernibles),
\item\label{symmetry}
 (symmetry), and
\item
 (triangle inequality)
\end{enumerate}
for all , , .
An
equivalent definition
requires the triangle inequality only for distinct
, , , axioms~(\ref{nonnegative})--(\ref{symmetry}) remaining.




An algorithm with oracle access to a metric space 
is given  and may query
 on
any
 to obtain .
Without loss of generality,
we
forbid
queries for , which trivially return ,
as well as repeated queries, where
a query
for 
is considered to repeat that for
.
For convenience,
denote
an algorithm ALG with oracle access to 
by .

Given oracle access to a finite metric space , the
{\sc metric -median} problem asks for a point
in

with the minimum average distance to
all
points.
An algorithm for
this problem is
-approximate if
it outputs
a point 
satisfying

where
.

The following theorem is due to Chang~\cite{Cha13} and generalized by
Wu~\cite{Wu14}.

\begin{theorem}[{\cite{Cha13, Wu14}}]\label{nonadaptiveupperbound}
{\sc Metric -median} has a deterministic nonadaptive -time
-approximation algorithm.
\end{theorem}

\section{Lower bound}

Fix arbitrarily a
deterministic
-query algorithm

for {\sc metric -median}
and a constant
.
By padding queries, we may assume the existence of a function
 such that  makes exactly
 queries given oracle access to any metric space with groundset
.

We introduce some notations concerning a function

to be determined later.
For , denote the th query of  by
; in other
words, the th query of  asks for .
Note that  depends only on , ,
,  because 
is deterministic and
has been fixed.
For  and ,

following Chang~\cite{Cha12} with a slight change in notation.
Equivalently,  is the degree of  in the undirected graph
with vertex set  and edge set .
As ,
 for .
Note that
 depends only on
, , , .
Denote the
output of 
by .
By adding at most  dummy queries,
we may
assume without loss of generality that

for all .
Consequently,

Fix any set  of size ,
e.g., .

We proceed to
construct  by gradually freezing distances.
For brevity, freezing
the value of

implicitly freezes  to the same value, where , .
Inductively, having
answered the first  queries of  by freezing
, , ,
, where ,
answer the th query by

It is not hard to verify that
the seven cases
in equation~(\ref{distanceassignment})
are
exhaustive and
mutually
exclusive.
We have now
frozen
 for all  and none of the other distances.
As repeated queries are forbidden, equation~(\ref{distanceassignment})
does not freeze one distance twice,
preventing inconsistency.

Set

breaking ties arbitrarily.
For all distinct
, 
with , ,
let

Clearly, the six cases in equation~(\ref{completingthemetric}) are
exhaustive and
mutually
exclusive.
Furthermore, equation~(\ref{completingthemetric}) assigns the same value
to  and .
Finally, for all ,

Equations~(\ref{distanceassignment}),~(\ref{completingthemetric})~and~(\ref{trivialdistance})
complete the construction of  by freezing all distances.


The following lemma is straightforward.

\begin{lemma}\label{distancesarezeroto4}
For all distinct , ,
.
\end{lemma}

Below is an immediate consequence of equation~(\ref{trueoptimal}).

\begin{lemma}\label{optimalisinpreservedregion}
.
\end{lemma}

The following lemma is a consequence of
equations~(\ref{neighborhoodinsubgraph})--(\ref{numberoffrozenincidentdistances}) and
our
forbidding
repeated queries.

\begin{lemma}\label{monotonicity}
For all  and ,

\end{lemma}
\begin{proof}
The case of  is immediate from
equations~(\ref{neighborhoodinsubgraph})--(\ref{numberoffrozenincidentdistances}).
Suppose that .
By symmetry,
we may assume
.
So by equation~(\ref{neighborhoodinsubgraph}),

As
 is the th query
and we
forbid
repeated
queries,

by equation~(\ref{neighborhoodinsubgraph}).\footnote{In detail,
if , then

for some  by
equation~(\ref{neighborhoodinsubgraph}); hence the th query 
repeats the th query, a contradiction.}
Equations~(\ref{numberoffrozenincidentdistances})~and~(\ref{newneighbor})--(\ref{reallynewneighbor})
complete the
proof.
\end{proof}

In short, Lemma~\ref{monotonicity} says that adding the edge 
to an undirected graph without that edge increases the degree of 
by  if and only if .



\begin{lemma}\label{monotonicitysame}
For all  and ,
if , then
.
\end{lemma}
\begin{proof}
By Lemma~\ref{monotonicity}, .
Invoking equation~(\ref{badpoints}) then completes the proof.
\end{proof}

\begin{lemma}\label{sumofdegrees}

\end{lemma}
\begin{proof}
Recall that the left-hand side
is the sum of degrees in the undirected graph with vertex set 
and edge set .
As we
forbid
repeated queries, 
Finally, it is a basic fact in graph
theory that
the sum of degrees in an undirected graph equals
twice the number of edges.
\end{proof}

\begin{lemma}[{Implicit in~\cite[Lemma~13]{Cha12}}]\label{fewbadpoints}
.
\end{lemma}
\begin{proof}
We have

This gives  as  is a constant and .
\end{proof}

\begin{lemma}\label{sparselyaskedpoint}
For all sufficiently large  and all ,

\end{lemma}
\begin{proof}
By Lemma~\ref{fewbadpoints},  and 
being a constant,
 for all sufficiently large .
By equation~(\ref{badpoints}), 
 for some ,
which together with equation~(\ref{trueoptimal})
gives .
Finally, Lemma~\ref{monotonicity} and 
imply inequality~(\ref{sparselyaskedpointequation}) for all .
\comment{Furthermore,

So
,
which together with
Lemma~\ref{monotonicity}
shows

for all sufficiently large 
and all .
}\end{proof}

Henceforth, assume  to be sufficiently large to satisfy
inequality~(\ref{sparselyaskedpointequation}) for all .

\begin{lemma}\label{onlysourceofdistance1}
For all , , if , then
one of the following conditions is true:
\begin{itemize}
\item  and ;
\item  and .
\end{itemize}
\end{lemma}
\begin{proof}
Inspect equation~(\ref{completingthemetric}), which
is the only equation that may set distances to .
\end{proof}

\comment{Below is a consequence of Lemma~\ref{onlysourceofdistance1}.

\begin{lemma}\label{sourceofshortdistances}
For all distinct
, , , if , then  and ,
.
\end{lemma}
}




\begin{lemma}\label{ordinarydistancesare2}
For all distinct , ,
.
\end{lemma}
\begin{proof}
By Lemma~\ref{monotonicitysame},

means , where .
So
only the second-to-last case
in equation~(\ref{distanceassignment}), which sets ,
may be consistent
with , .

By
Lemma~\ref{optimalisinpreservedregion},
.
So
only the last case
in equation~(\ref{completingthemetric}),
which sets ,
may be consistent
with , .
\end{proof}



\comment{\begin{lemma}\label{askeddistancesincidentonoptimalpoint}
For all ,
.
\end{lemma}
\begin{proof}
By Lemma~\ref{sparselyaskedpoint},
.
, which
together with equation~(\ref{badpoints}) and
Lemma~\ref{monotonicity}
completes the proof.
\end{proof}
}
\begin{lemma}\label{optimalpointdistances1or3}
For all ,
.
\end{lemma}
\begin{proof}
By
Lemma~\ref{optimalisinpreservedregion} and
inequality~(\ref{sparselyaskedpointequation}),
only
the first three
cases in
equation~(\ref{distanceassignment}),
which set ,
may be consistent with
 or .

Again by Lemma~\ref{optimalisinpreservedregion},
only the first three cases in equation~(\ref{completingthemetric}),
which set ,
may be consistent with  or .
\end{proof}

\begin{lemma}\label{illegaldistances1}
There do not exist distinct , , 
with  and .
\end{lemma}
\begin{proof}
By Lemma~\ref{onlysourceofdistance1},
 implies .
By symmetry, assume .
Then  by
Lemma~\ref{optimalpointdistances1or3}.
\end{proof}

\begin{lemma}\label{illegaldistances2}
There do not exist distinct , , 
with  and .
\end{lemma}
\begin{proof}
By Lemma~\ref{onlysourceofdistance1},
 implies  and , .
Then  by
Lemma~\ref{ordinarydistancesare2}.
\end{proof}


Lemmas~\ref{illegaldistances1}--\ref{illegaldistances2}
forbid all possible violations of the triangle inequality,
yielding the following lemma.


\begin{lemma}\label{itismetric}
 is a metric space.
\end{lemma}
\begin{proof}
Lemmas~\ref{distancesarezeroto4}~and~\ref{illegaldistances1}--\ref{illegaldistances2}
establish the triangle inequality for .
Furthermore,  is symmetric because (1)~freezing  automatically
freezes  to the same value,
(2)~forbidding
repeated queries
prevents equation~(\ref{distanceassignment})
from assigning inconsistent values to one distance and
(3)~equation~(\ref{completingthemetric}) is symmetric.
All the other axioms for metrics
are easy to verify.
\end{proof}

Recall that  denotes the output of .
We proceed to
compare  with .

\begin{lemma}\label{identifyingjumps}
There exist
, , , 
and distinct , , , 
such that

for all .
\end{lemma}
\begin{proof}
By
Lemma~\ref{monotonicity}, equation~(\ref{outputallasked}) and the easy fact
that ,
there exist distinct , , ,  satisfying
equations~(\ref{beforejumping})--(\ref{afterjumping})
for all .\footnote{Observe that
 must go through all of , , ,  as 
increases from  to .}
Lemma~\ref{monotonicity} and
equations~(\ref{beforejumping})--(\ref{afterjumping}) imply
, establishing the existence of
 satisfying
equation~(\ref{algorithmoutputparticipate}).
If , , ,  are not distinct, then
there are repeated queries by
equation~(\ref{algorithmoutputparticipate}), a contradiction.
\end{proof}

From now on, let
, , , 
and distinct , , , 
satisfy
equations~(\ref{beforejumping})--(\ref{algorithmoutputparticipate})
for all .

\begin{lemma}\label{algorithmoutputtypicaldistances}
For each , if  and ,
then
.
\end{lemma}
\begin{proof}
Assume in equation~(\ref{algorithmoutputparticipate}) that
 and ; the other case will be symmetric.
By equation~(\ref{beforejumping}),

\begin{enumerate}[{Case }1:]
\item .
By equation~(\ref{distanceassignment}),

and ,

\item .
By equation~(\ref{distanceassignment}),

and ,

\end{enumerate}
Equation~(\ref{algorithmoutputverymuchasked})
together with any one of
equations~(\ref{distancesfromalgorithmoutput1})--(\ref{distancesfromalgorithmoutput2})
implies .
Hence .
\end{proof}

We are now able to analyze the quality of  as a solution to
{\sc metric -median}.

\begin{lemma}\label{analyzingalgorithmoutputassuboptimal}

\end{lemma}
\begin{proof}
By the distinctness of
, , ,  in
Lemma~\ref{identifyingjumps},

Write
.
As , , ,  are distinct,

Furthermore,

Equations~(\ref{initialinequalityinanalyzingalgorithmoutput})--(\ref{lastinequalityinanalyzingalgorithmoutput})
and  complete the proof.
\end{proof}

We now analyze the quality of  as a solution to
{\sc metric -median}.
The following lemma is immediate from
equation~(\ref{completingthemetric}).

\begin{lemma}\label{optimalpointhasmanydistancesbeing1}
For all ,
if
 and
, ,
then .
\end{lemma}

\begin{lemma}\label{analyzingoptimalpoint}

\end{lemma}
\begin{proof}
By equation~(\ref{neighborhoodinsubgraph}),

This and
Lemma~\ref{optimalpointhasmanydistancesbeing1} imply
 for all  with
 and
.
Therefore,

Clearly,

Furthermore,

This and Lemma~\ref{fewbadpoints}
imply

as .
To complete the proof, sum
up
inequalities~(\ref{distance1parts})--(\ref{largedistancesparts})
and then use
inequality~(\ref{fewneighborsforoptimalpoint})
in the trivial way.
\end{proof}



Combining
Lemmas~\ref{itismetric},~\ref{analyzingalgorithmoutputassuboptimal}~and~\ref{analyzingoptimalpoint}
yields our main theorem, stated below.

\begin{theorem}\label{maintheorem}
{\sc Metric -median} has no deterministic -query
-approximation algorithm for any constant .
\end{theorem}
\begin{proof}
Lemma~\ref{itismetric} asserts that  is a metric space.
By
Lemmas~\ref{analyzingalgorithmoutputassuboptimal}~and~\ref{analyzingoptimalpoint},

This proves the theorem because
the deterministic -query algorithm 
and
the constant 
are picked arbitrarily (note that  denotes the output of ).
\end{proof}

Theorem~\ref{maintheorem}
complements
Theorem~\ref{nonadaptiveupperbound}.

It is possible to
simplify equation~(\ref{completingthemetric})
at the expensive of an additional assumption.
Without loss of generality, we may assume that 
for all ; this increases the query complexity by a multiplicative
factor of  by equation~(\ref{badpoints}).
Therefore, if  or , then
 will be frozen by equation~(\ref{distanceassignment}).
So the third to fifth
cases
in equation~(\ref{completingthemetric}),
which satisfies  or ,
can be omitted.

\comment{Define

to be the set of
queries of 
treated as unordered
pairs.
Without loss of generality, assume  for all .
Let  be the simple undirected graph with vertex set  and
edge set .
Denote the degree of  in  by
,
and


In the sequel, we will specify  incrementally in several steps.
Note that  and
 are independent of

because of the nonadaptivity of ;
hence
they will remain intact
during our
specification of .

Below is an easy lemma.

\begin{lemma}[{Implicit in~\cite{Cha12}}]\label{numberofbadvertices}
.
\end{lemma}
\begin{proof}
We have

where the last equality follows from the well-known fact that the sum of
degrees in an undirected graph is
twice the
number of edges.
This
completes the proof
because
 is 's query complexity and  is a constant.
\end{proof}

Henceforth we will assume  to be sufficiently large so
that

by Lemma~\ref{numberofbadvertices}.
For all ,

For all  with ,  or ,

Clearly, this does not assign different values to  and .

As
Eq.~(\ref{distancesonquerysetandbadvertices})
specifies
 on a superset of  (which is the set of 's queries)
and  is deterministic,
the output of 
has
now
been
fixed
even though

is not fully specified yet.
Let
 and 
be
such that  contains
the output of .

\begin{lemma}\label{notbadandnotconnectedwithalgorithmoutput}

\end{lemma}
\begin{proof}
Eq.~(\ref{setofbadvertices}) and  imply , i.e., .
\end{proof}

\comment{
}
Take

arbitrarily,
as
can be done
by
Lemma~\ref{notbadandnotconnectedwithalgorithmoutput}
and Eq.~(\ref{numberofremainingpointstobemadegood}).
Trivially, .

We now complete the specification of .
For all  with  and ,\footnote{These are precisely the pairs whose -distances are not
specified by
Eqs.~(\ref{zerodistancestoself})--(\ref{distancesonquerysetandbadvertices}).}

The four cases
in Eq.~(\ref{makingourpointgoodandalgorithmpointbad})
are mutually exclusive because  by
Eq.~(\ref{pickingourpoint}).
Clearly, Eq.~(\ref{makingourpointgoodandalgorithmpointbad}) does not assign
different values to  and .

The following lemma is straightforward from
Eqs.~(\ref{zerodistancestoself})--(\ref{distancesonquerysetandbadvertices})~and~(\ref{makingourpointgoodandalgorithmpointbad}).

\begin{lemma}\label{rangeofdistances}
For all , , .
\end{lemma}


\begin{lemma}\label{speciallydesignedpointisgood}

\end{lemma}
\begin{proof}
By Lemmas~\ref{numberofbadvertices}~and~\ref{rangeofdistances},

Furthermore,

where the last inequality follows from Eq.~(\ref{setofbadvertices})
and .
We have

because all summands are  by
Eq.~(\ref{makingourpointgoodandalgorithmpointbad}) and .
By Lemma~\ref{rangeofdistances},

Summing up
Eqs.~(\ref{distancesfromourpointtobad})--(\ref{thetrivialdistance})
completes the proof.
\end{proof}

\begin{lemma}\label{outputpointisterribleifnotbad}

\end{lemma}
\begin{proof}
Recall that .
We have

\end{proof}

The next lemma is immediate from Eq.~(\ref{distancesonquerysetandbadvertices})
and .

\begin{lemma}\label{outputpointisterribleifbad}

\end{lemma}

We proceed to prove that  is a metric space through a few lemmas.








The following
lemma is
immediate from
Eqs.~(\ref{distancesonquerysetandbadvertices})~and~(\ref{makingourpointgoodandalgorithmpointbad}).

\begin{lemma}\label{onlybadandalgorithmoutputcanhavedistance4}
For all  , if , then .
\end{lemma}


Below is a consequence of  and
Eqs.~(\ref{pickingourpoint})--(\ref{makingourpointgoodandalgorithmpointbad}).

\begin{lemma}\label{distancebetweennonbadoutputandourdesignedpoint}
.
\end{lemma}

\begin{lemma}\label{constructeddistanceismetric}
 is a metric space.
\end{lemma}
\begin{proof}
We only need to prove the triangle inequality for  because all the other
axioms
are easy to verify.
Consider
the following cases for all distinct , , :
\begin{itemize}
\item ,  and .
By Lemma~\ref{sourceofshortdistances},
.
Hence if  (resp., ), then  (resp., )
by
Lemma~\ref{distancebetweennonbadoutputandourdesignedpoint},
a
contradiction.
Therefore, , which together with
Lemma~\ref{onlybadandalgorithmoutputcanhavedistance4} forces .
But
if
 (resp., ), then
 (resp., )
by Eq.~(\ref{distancesonquerysetandbadvertices}), a contradiction.
\item ,  and .
By Lemma~\ref{sourceofshortdistances},
.
On the other hand,  means  by
Eq.~(\ref{makingourpointgoodandalgorithmpointbad}) (which is the only
equation that may set distances to ),
contradicting .
\item ,  and .
By Lemma~\ref{onlybadandalgorithmoutputcanhavedistance4},
.
But if  (resp., ), then  (resp., ) by
Eq.~(\ref{distancesonquerysetandbadvertices}), a contradiction.
Therefore, .
Furthermore,  by Lemma~\ref{onlysourceofdistance1}.
Consequently, 
(note that  by Eq.~(\ref{pickingourpoint})),
implying 
by
Lemma~\ref{distancebetweennonbadoutputandourdesignedpoint},
a contradiction.
\end{itemize}
We have excluded all possibilities of , where
, , .
\end{proof}


Combining
Lemmas~\ref{speciallydesignedpointisgood}--\ref{outputpointisterribleifbad},~\ref{constructeddistanceismetric}
and that
 contains the output of 
yields our main theorem.

\begin{theorem}\label{maintheorem}
{\sc Metric -median} has no deterministic nonadaptive -query
-approximation
algorithms
for any constant .
\end{theorem}



Theorem~\ref{maintheorem}
shows that
the approximation ratio of  in
Theorem~\ref{nonadaptiveupperbound} cannot be improved to any constant
.

\comment{For all  with ,

Clearly, this
and Eq.~(\ref{distancesonquerysetandbadvertices}) uniquely determines
 for all .

To complete specifying ,
for all  with , 

}


For a metric space  and a set 
of unordered pairs,
let  be the
undirected graph with
vertex set  and
edge set .
Assign to each edge
 of 
the
length .
For , , denote by  the
shortest-path distance between  and  in .
Clearly,  for all , .
For a finite set  and a function ,
the  norm of

is .
The following corollary investigates, with respect to the normalized
 norm, the
inapproximability of
metrics
by small sets of distances.

\begin{corollary}
There do not exist sets  of unordered pairs
satisfying

for all metric spaces .
\end{corollary}
\begin{proof}
Suppose for contradiction that  satisfies
Eqs.~(\ref{numberofqueries})--(\ref{recoveryerror}).
Let

So  is
the optimal solution to {\sc Metric -median} with respect to .
Now,

By Eq.~(\ref{optimalsolutiononpseudodistances}),
we may find  with  queries,
which together with
Eqs.~(\ref{qualityofpseudo1median})--(\ref{optimalvalue4times})
contradict Theorem~\ref{maintheorem}.
\end{proof}

\comment{\section{Additional section --- to be modified}

This section modifies XXX slightly to XXX.

\begin{figure}
\begin{algorithmic}[1]
\FOR{each }
	\FOR{each }
		\IF{, }
			\STATE Query for ;
			\STATE Query for ;
			\STATE ;
		\ELSE
			\STATE Query for ;
			\STATE ;
		\ENDIF
	\ENDFOR
\ENDFOR
\STATE ,
breaking ties arbitrarily;
\STATE Output ;
\end{algorithmic}
\caption{Algorithm {\sf Approx-Centroid}.}
\label{deterministic16approximationalgorithm}
\end{figure}


For all ,  and ,
define

The same definition is made by Chang~\cite{Cha13}.

\begin{fact}[{\cite[Lemma~2]{Cha13}}]\label{pseudodistanceupper}
For all ,  and ,

after finishing the loop in lines~1--12 of {\sf Approx-Centroid}.
\end{fact}

\begin{fact}[{\cite[Lemma~4]{Cha13}}]
For all , ,

after finishing the loop in lines~1--12 of {\sf Approx-Centroid}.
\end{fact}

In the following two lemmas,
 and  are
independent and uniformly random elements in .

\begin{lemma}\label{distancessquareexpected}
For all ,

\end{lemma}
\begin{proof}
By Fact~\ref{pseudodistanceupper},

where the second inequality follows from Cauchy's inequality.
\end{proof}

\begin{lemma}\label{fsquarelemma}
For all ,

\end{lemma}
\begin{proof}
By Eq.~(\ref{additionalterm}),
{\small 
}Observe that,
conditional on any realization
of  and  with , ,
both  and  are uniformly distributed over .
Therefore,
{\small 
}This and inequality~(\ref{fsquare})
imply

\end{proof}


Below is a consequence of
Lemmas~\ref{distancessquareexpected}--\ref{fsquarelemma}.

\begin{lemma}\label{therecomestheratioof16}
For all ,

\end{lemma}

\begin{theorem}
{\sc Metric -median} has a deterministic -query
-approximation algorithm.
\end{theorem}
\begin{proof}
By line~13 of {\sf Approx-Centroid},

\comment{Equivalently,

}By Lemma~\ref{therecomestheratioof16},

\end{proof}

\section{Something new}

For
a metric space
,
uniformly random points
, 
in  and
the output  of
Indyk's algorithm given oracle access to ,

That is,
writing
 for all , ,

This and the easily verifiable fact that  is a metric on 
show how to recover  in  time with a bounded  error.
}}
\bibliographystyle{plain}
\bibliography{adaptivemedian}

\noindent

\end{document}
