\documentclass[12pt]{article}

\usepackage{amsmath,amssymb,amsthm}
\usepackage{latexsym,color,graphicx}

\setlength{\textwidth}{6.5in} \setlength{\evensidemargin}{0in}
\setlength{\oddsidemargin}{0in} \setlength{\textheight}{9.0in}
\setlength{\topmargin}{0in} \setlength{\headheight}{0in}
\setlength{\headsep}{0in} \setlength{\parskip}{2mm}
\setlength{\baselineskip}{1.7\baselineskip}

\def\eps{{\varepsilon}}
\def\A{{\cal A}}
\def\E{{\cal E}}
\def\C{{\cal C}}
\def\K{{\cal K}}
\def\T{{\cal T}}
\def\reals{{\mathbb R}}
\def\bd{{\partial}}
\def\SS{{\mathbb S}}

\newtheorem{theorem}{Theorem}[section]
\newtheorem{lemma}[theorem]{Lemma}
\newtheorem{claim}[theorem]{Claim}
\newtheorem{definition}[theorem]{Definition}
\newtheorem{corollary}[theorem]{Corollary}

\def\marrow{{\marginpar[\hfill]{}}}

\def\micha#1{{\sc Micha says: }{\marrow\sf #1}}
\def\roel#1{{\sc Roel says: }{\marrow\sf #1}}

\newcommand\remove[1]{}

\begin{document}

\title{An Improved Bound on the Number of Unit Area Triangles\thanks{Work on this paper was supported by NSF Grants CCF-05-14079 and
  CCF-08-30272, 
  by a grant from the U.S.-Israeli Binational Science Foundation, 
  by grant 155/05 from the Israel Science Fund and by the
  Hermann Minkowski--MINERVA Center for Geometry at Tel Aviv University.} }
\author{
    Roel Apfelbaum\thanks{School of Computer Science, Tel Aviv University,
    Tel Aviv 69978, Israel;
    \texttt{roel6@hotmail.com}.}
    \and
  Micha Sharir\thanks{School of Computer Science, Tel Aviv University, Tel~Aviv 69978,
    Israel; and Courant Institute of Mathematical Sciences, New York
    University, New York, NY~~10012,~USA. E-mail:
    \texttt{michas@post.tau.ac.il}} }

\maketitle

\begin{abstract}
We show that the number of unit-area triangles determined by a set of
 points in the plane is , for any ,
improving the recent bound  of Dumitrescu et al.
\end{abstract}


\section{Introduction}
\label{sec:intro}

In 1967, A.~Oppenheim (see \cite{EP95}) asked the following question:
Given  points in the plane and , how many triangles spanned
by the points can have area ? By applying a scaling transformation,
one may assume  and count the triangles of {\em unit} area.
Erd\H{o}s and Purdy~\cite{EP71} showed that a  section of the integer lattice determines
 triangles of the same area. They also showed
that the maximum number of such triangles is at most . In
1992, Pach and Sharir~\cite{PS92} improved the bound to
, using the Szemer\'edi-Trotter theorem~\cite{ST83} 
on the number of point-line incidences. Recently,
Dumitrescu et al.~\cite{DST08} have further improved the upper bound 
to , by estimating the number of incidences
between the given points and a 4-parameter family of quadratic curves. 

In this paper we further improve the bound to , for
any . Our proof borrows some ideas from \cite{DST08}, but
works them into a different approach, which reduces the problem to 
bounding the number of incidences between points and certain kind 
of surfaces in three dimensions.


\section{Unit-area triangles in the plane}
\label{sec:unit2}

To simplify the notation, we write  for an upper bound of
the form , which holds for any ,
where the constant of proportionality  depends on .

\begin{theorem}\label{thm:unit2}
The number of unit-area triangles spanned by  points in the plane
is .
\end{theorem}
\noindent{\bf Proof.}
We begin by borrowing some notation and preliminary ideas from
\cite{DST08}.
Let  be the given set of  points in the plane. Consider a triangle
 spanned by . We call the three lines containing the
three sides of , {\em base lines} of , and the
three lines parallel to the base lines and incident to the respective
third vertices, {\em top lines} of .

For a parameter , , to be optimized
later, call a line  {\em -rich} (resp., {\em -poor}) if
 contains at least  (resp., fewer than ) points of .
Call a triangle  {\em -rich} if each of its three top
lines is -rich; otherwise  is {\em -poor}. 

We first observe that the number of -poor unit-area triangles
spanned by  is . Indeed, assign a -poor unit-area
triangle  whose top line through , say, is -poor
to the opposite base . Then all the triangles assigned to a base
 are such that their third vertex lies on one of the two lines
parallel to  at distance , where that line contains
fewer than  points of . Hence, a base  can be assigned at
most  triangles, and the bound follows.

So far, the analysis follows that of \cite{DST08}.
We now focus the analysis on the set of -rich unit-area triangles
spanned by , and use a different approach.

Let  denote the set of -rich lines, and let  denote the set 
of all pairs

By the Szemer\'edi-Trotter theorem~\cite{ST83}, we have,
for any , , and
.

A pair ,  of elements of  is said
to {\em match} if the triangle with vertices , ,
 has area ; see Figure~\ref{qmatch}.

\begin{figure}[htb]
\begin{center}
\begin{picture}(0,0)\includegraphics{qmatch.pstex}\end{picture}\setlength{\unitlength}{3947sp}\begingroup\makeatletter\ifx\SetFigFontNFSS\undefined \gdef\SetFigFontNFSS#1#2#3#4#5{\reset@font\fontsize{#1}{#2pt}\fontfamily{#3}\fontseries{#4}\fontshape{#5}\selectfont}\fi\endgroup \begin{picture}(3617,3179)(730,-3238)
\put(4253,-2520){\makebox(0,0)[lb]{\smash{{\SetFigFontNFSS{12}{14.4}{\rmdefault}{\mddefault}{\updefault}}}}}
\put(2303,-218){\makebox(0,0)[lb]{\smash{{\SetFigFontNFSS{12}{14.4}{\rmdefault}{\mddefault}{\updefault}}}}}
\put(3098,-1021){\makebox(0,0)[lb]{\smash{{\SetFigFontNFSS{12}{14.4}{\rmdefault}{\mddefault}{\updefault}}}}}
\put(2333,-2513){\makebox(0,0)[lb]{\smash{{\SetFigFontNFSS{12}{14.4}{\rmdefault}{\mddefault}{\updefault}}}}}
\put(1651,-1036){\makebox(0,0)[lb]{\smash{{\SetFigFontNFSS{12}{14.4}{\rmdefault}{\mddefault}{\updefault}}}}}
\put(1808,-1898){\makebox(0,0)[lb]{\smash{{\SetFigFontNFSS{12}{14.4}{\rmdefault}{\mddefault}{\updefault}{\color[rgb]{0,0,0}}}}}}
\put(2348,-1509){\makebox(0,0)[lb]{\smash{{\SetFigFontNFSS{12}{14.4}{\rmdefault}{\mddefault}{\updefault}{\color[rgb]{0,0,0}}}}}}
\end{picture} \caption{The ordered pair  is a matching
         pair of elements of .}
\label{qmatch}
\end{center}
\end{figure}

To upper bound the number of unit-area triangles, all of whose
three top lines are -rich, it suffices to bound the number of
matching pairs in . Indeed, given such a unit-area triangle 
, let  (resp., ) be the top line of 
 through  (resp., through ). 
Then  and  form a
matching pair in , by definition (again, see Figure \ref{qmatch}).
Conversely, a matching pair
,  determines at most one unit-area
triangle , where  is the intersection point of the line
through  parallel to  and the line through  
parallel to ; we get an actual triangle if and only if 
the point  belongs to .

In other words, our problem is now reduced to that of bounding the 
number of matching pairs in . (Since we do not enforce the condition 
that the third point  of the corresponding triangle belongs to , 
we most likely over-estimate the true bound.) 

Since elements of  have three degrees of freedom, we can represent
them in an appropriate 3-dimensional parametric space. For example,
we can assume that no line in  is vertical, and parametrize an
element  of  by the triple , where
 are the coordinates of , and  is the slope
of . For simplicity of notation, we refer to this 3-dimensional
parametric space as .

So far, the matching relationship is symmetric. To simplify the
analysis, and with no loss of generality, we make it assymmetric, by
requiring that, in an (ordered) matching pair
, ,  lies counterclockwise to
, where . See Figure~\ref{qmatch}.

Let us express the matching condition algebraically.
Let  be the triple representing a pair
, and  be the triple representing
another pair .
Clearly,  in a matching pair.
The lines  and  intersect at a point , for which
there exist real parameters  which satisfy

or

It is now easy to verify that the condition of matching, with
 lying counterclockwise to , is given by

or, alternatively,

Similarly, the condition of ``reverse'' matching, with 
lying clockwise to , is given by


Fix an element  of , and associate with it a surface
, which is the locus of all
pairs  that match  (i.e., 
is an ordered matching pair).
By the preceding analysis,  satisfies (\ref{eq:m1}),
where  is the parametrization of ,
and is thus a 2-dimensional algebraic surface
in  of degree . We thus obtain a system  of 
 2-dimensional algebraic surfaces in , and a set  
of  points in , and our goal is to bound the number 
of incidences between  and . 

The main technical step in the analysis is to rule out the possible
existence of
{\em degeneracies} in the incidence structure, where many points 
are incident to many surfaces; 
this might happen when many points lie on a common intersection
curve of many surfaces (a situation which might 
arise, e.g., in the case of planes and points in ). 
However, for the class of surfaces under consideration, namely, the
surfaces  generated by some line-point incidence pair
, such a degeneracy is impossible, as the
following lemma shows.
\begin{lemma} \label{lem:gamma}
Let  and  be two distinct line-point
incidence pairs,
let  be the
intersection curve of their associated surfaces, and assume that
 is non-empty.
Let  be some incidence pair and assume further that
.
Then either  or .
\end{lemma}
\begin{proof}
We establish the equivalent claim that, given a curve ,
which is the intersection of some unknown pair of surfaces
 and ,
one can reconstruct  and  uniquely
(up to a swap between the two incidence pairs) from .
Morever, it is enough to know the projection
 of  onto the -plane in order to uniquely
reconstruct the incidence pairs  and  that
generate .

We start by computing the algebraic representation of .
Let  and  be the respective
parametrizations of  and .
By (\ref{eq:m1}),  satisfies the equation

Recall the additional requirement in (\ref{eq:m1}), namely that
 and . This requirement is
implicit in (\ref{eq:m1}) and in (\ref{eq:m12}), meaning that
equation (\ref{eq:m12}) is defined only for values of  and 
for which the value of  is not  or .
Consulting (\ref{eq:m1}), this implies that no point  can satisfy
 or .
Put

and write (\ref{eq:m12}) as

or

which we can rewrite as

where

We can further simplify the equation by noting that
 is a linear expression is . That is,

where

We can thus write (\ref{eq:m12}) as

Figure \ref{fig:lines} illustrates the different lines defined by
the linear equations , and their relations with
 and .
The linearity of , and  implies that
the equation (\ref{eq:m14}) of  is cubic.
\begin{figure}[htb]
\begin{center}
\begin{picture}(0,0)\includegraphics{lines6.pstex}\end{picture}\setlength{\unitlength}{4144sp}\begingroup\makeatletter\ifx\SetFigFontNFSS\undefined \gdef\SetFigFontNFSS#1#2#3#4#5{\reset@font\fontsize{#1}{#2pt}\fontfamily{#3}\fontseries{#4}\fontshape{#5}\selectfont}\fi\endgroup \begin{picture}(4073,1984)(170,-1280)
\put(3694,-793){\makebox(0,0)[b]{\smash{{\SetFigFontNFSS{9}{10.8}{\rmdefault}{\mddefault}{\updefault}{\color[rgb]{0,0,0}}}}}}
\put(3475, 73){\makebox(0,0)[b]{\smash{{\SetFigFontNFSS{9}{10.8}{\rmdefault}{\mddefault}{\updefault}{\color[rgb]{0,0,0}}}}}}
\put(3231,-1225){\makebox(0,0)[b]{\smash{{\SetFigFontNFSS{9}{10.8}{\rmdefault}{\mddefault}{\updefault}{\color[rgb]{0,0,0}}}}}}
\put(2476,-241){\rotatebox{14.0}{\makebox(0,0)[lb]{\smash{{\SetFigFontNFSS{9}{10.8}{\rmdefault}{\mddefault}{\updefault}{\color[rgb]{0,0,0}}}}}}}
\put(2521, 74){\rotatebox{357.5}{\makebox(0,0)[lb]{\smash{{\SetFigFontNFSS{9}{10.8}{\rmdefault}{\mddefault}{\updefault}{\color[rgb]{0,0,0}}}}}}}
\put(1171,-511){\rotatebox{38.0}{\makebox(0,0)[lb]{\smash{{\SetFigFontNFSS{9}{10.8}{\rmdefault}{\mddefault}{\updefault}{\color[rgb]{0,0,0}}}}}}}
\put(1801,-601){\rotatebox{357.5}{\makebox(0,0)[lb]{\smash{{\SetFigFontNFSS{9}{10.8}{\rmdefault}{\mddefault}{\updefault}{\color[rgb]{0,0,0}}}}}}}
\put(2670,557){\makebox(0,0)[b]{\smash{{\SetFigFontNFSS{9}{10.8}{\rmdefault}{\mddefault}{\updefault}{\color[rgb]{0,0,0}}}}}}
\put(2881,-466){\rotatebox{38.0}{\makebox(0,0)[lb]{\smash{{\SetFigFontNFSS{9}{10.8}{\rmdefault}{\mddefault}{\updefault}{\color[rgb]{0,0,0}}}}}}}
\put(2656,-961){\makebox(0,0)[b]{\smash{{\SetFigFontNFSS{9}{10.8}{\rmdefault}{\mddefault}{\updefault}{\color[rgb]{0,0,0}}}}}}
\put(1126,-871){\makebox(0,0)[b]{\smash{{\SetFigFontNFSS{9}{10.8}{\rmdefault}{\mddefault}{\updefault}{\color[rgb]{0,0,0}}}}}}
\put(1981,209){\makebox(0,0)[b]{\smash{{\SetFigFontNFSS{9}{10.8}{\rmdefault}{\mddefault}{\updefault}{\color[rgb]{0,0,0}}}}}}
\put(2400,-354){\rotatebox{314.0}{\makebox(0,0)[lb]{\smash{{\SetFigFontNFSS{9}{10.8}{\rmdefault}{\mddefault}{\updefault}{\color[rgb]{0,0,0}}}}}}}
\end{picture} \caption{
	The lines , for , in the general case.
	The line 
	connects  and ,  passes through  and is
	parallel to ,  passes through  and is
	parallel to , and  connects
	 with the intersection point  of
	 and  (and bisects the edge ).
}
\label{fig:lines}
\end{center}
\end{figure}
We have the following two special cases to rule out:
\begin{enumerate}
\item If , that is,  and , then
	  , , and . But then the equation
	  becomes , or , contrary to the
	  assumption that . Hence, the
	  equation has no solutions, meaning that 
	  is empty and the surfaces do not intersect.
\item If  but , that is,
	  , then
	  , , and ,
	  resulting in the equation , which is not allowed in
	  (\ref{eq:m14}). Hence  is not defined in this case either.
\end{enumerate}
We can therefore restrict our attention to the general case.
Consider the cubic part of the equation . In this term,
each factor can be thought of as a line defined by the equation
, for . The lines  and  respectively
are simply  and , whereas  is
the line  passing through  and  (see Figure
\ref{fig:lines}). Note that  may coincide with one of the
other two lines.
Indeed, if  happens to be incident with , then 
coincides with . Similarly, if  then 
coincides with  (these are the only possible coincidences,
since we have ruled out the case ).
These cases will be handled shortly, but for now, we ignore them and
consider the general
case. In this case,  has three distinct asymptotes given by
, , and ; the proof of this fact is given in Lemma
\ref{lem:as1} in the appendix

Using this fact, one can reconstruct the two line-point pairs that
generate  as follows.
Suppose we are given a curve  generated by some unknown
pair of incidence pairs,  and , and
we want to reconstruct these pairs.
 is given as the zero set of some cubic bivariate polynomial
, where  can be written as , but the decomposition of 
into , and  is unknown, and, moreover,
is not known a priori to be unique (a fact which we establish in this
proof).
First, we find its three asymptotes , ,
and , where for each ,
 is linear in  and .
Since, by Lemma \ref{lem:as1}, these asymptotes are , ,
and , we know that each  is equal to some 
multiplied by a constant, but we do not know which is which.
To determine the roles of the asymptotes correctly, observe that
 for some constant
. Thus, there exists some unique constant , such that
 is linear in
 and . The line  is parallel to the line ,
which happens to be the median of the triangle spanned by the three
asymptotes, which emanates from the vertex , and
bisects the edge ; see Figure \ref{fig:lines}.
We thus have enough information to determine which vertex of the
triangle is , and which are  and , and which edges of the
triangle are supported by  and .
This proves the lemma for the general case where all the points
and lines are distinct, and no point lies on both lines .

Finally, consider the case where  (a symmetric argument
applies when ).
In this case, , and , so the
equation of the curve  can be rewritten as

Note that  under the preliminary assumption that there
are no vertical lines in the system, since both 
and  are on .
Note also that , for otherwise,
 and  would have to coincide, a case which we
have ruled out earlier.
Finally, note that  is a nonzero constant.
Hence, the equation of  is, up to a constant multiple,

This equation defines a cubic curve with two asymptotes given by
, and , namely, the lines  and ;
the proof is given in Lemma \ref{lem:as2} in the appendix.
Since , it follows that
 does not intersect , whereas  is
intersected at a single point  for which .
Using this point, one can compute the values of , and
, and hence, reconstruct the line .
The point  is then simply the intersection of the lines 
and . Thus, one can uniquely reconstruct , ,
, and  in this case too.
This completes the proof of Lemma \ref{lem:gamma}.
\end{proof}

\paragraph{Bounding the number of incidences.}
Recall that we need to bound the number of incidences between the set
 of surfaces , for ,
and the set  of points. This is done by following the standard
method of Clarkson et al. \cite{CEGSW}.
The first step in this method is to derive a simple but weaker
bound, usually by extremal graph theory. Then, we strengthen the bound
by {\em cutting} the arrangement of the surfaces into cells, and by
summing the weaker bounds on the number of incidences within each cell,
over all the cells.

\paragraph{The first step: A simple bound.}
Lemma \ref{lem:gamma} implies that the incidence graph between 
and  does not contain  as a subgraph, or,
in other words, no three distinct surfaces of  and ten distinct
points of  can all be incident to one another. Indeed, the
intersection points of three surfaces , for
, are the intersection points of the two curves
, and
. These
intersection points project to (some of) the intersection points of the
projections  and  of 
and , respectively, onto the -plane.
By Lemma \ref{lem:gamma}, these two curves are distinct (or empty).
Since each of them is cubic, and since, as shown in Lemmas \ref{lem:as1}
and \ref{lem:as2} in the appendix, they are the zero sets of irreducible
polynomials,
B\'ezout's theorem \cite{Sha} implies that
they intersect in at most  points.
Hence, the incidence graph between  and  does not
contain , so
by the K\H{o}vari--S\'os--Tur\'an theorem~\cite{KST54}, the
number of incidences between  and  can be bounded by

Since the matching relation is essentially symmetric (up to some
sign changes; see (\ref{eq:m1}) and (\ref{eq:m1'})), we can interchange
the roles of points and surfaces, and conclude that the number of
incidences is also at most


\paragraph{Cutting.}
To improve the bound, we apply the following fairly standard space 
decomposition technique.
Fix a parameter , whose specific value will be chosen later,
and construct a -cutting  of  \cite{Ch05}. 
We use the more simple-minded technique in which we choose a random
sample  of  surfaces of  and construct the
vertical decomposition (see e.g. \cite{SA}) of the arrangement
. We obtain
 relatively open cells of dimensions 0,1,2, and 3, each
of which is crossed by (intersected by, but not contained in) at most
 surfaces; this latter property holds with high
probability, and we simply assume that our sample  does satisfy it.


\paragraph{Summing over all cells.}
Fix a cell  of , and put  and . Let  denote the subset of surfaces of 
which cross , and put .

We now apply the simple bound (\ref{bound:weak}) obtained in the
first step to each cell  of our cutting , handling,
for the time being, only surfaces that {\em cross} . The
overall number of incidences is

which, using the bounds , and , is

To minimize this expression, we choose , making it
.

We also have to take into account incidences between points in a
cell  and surfaces that fully contain . 
This is done separately for cells of dimension , , and  (it
is vacuous for cells of dimension ). Indeed, a 2-dimensional cell
 is contained in exactly one surface, so a point 
takes part in only one such incidence.
Thus, in this case we only need to add , the number of points,
to the above bound.

The same argument applies for points in 1-dimensional cells.
Assuming that the vertical decomposition is performed in a generic
coordinate frame, it suffices to consider only 1-dimensional cells that
are portions of the intersection curves between the surfaces of
. By Lemma \ref{lem:gamma}, each such cell  is contained
in exactly two surfaces of .
Thus, we need to add at most  to the number of incidences to handle
these cells.

Each cell of dimension  is a single point , and, arguing as above,
we may assume it
to be a vertex of the undecomposed arrangement . Any surface
 incident to  has to cross or bound an adjacent
full-dimensional cell , so we charge the incidence of 
with  to the pair , and note that such a pair can
be charged only  times. It follows that the number of incidences
with 0-dimensional cells of  is ,
which, for the chosen value of , is equal to the bound obtained
above for the crossing surfaces.

In conclusion, the overall number of incidences between  and
 is .

Recall now that , and that we also have the bound 
for the number of unit-area triangles with at least one -poor
top line. Thus, the overall bound on the number of unit-area triangles
is

which, if we choose , becomes , as
asserted.


\paragraph{Discussion.}
Theorem \ref{thm:unit2} constitutes a major improvement over previous
bounds, but it still leaves a substantial gap from the near-quadratic
lower bound. One major weakness of our proof is that, in bounding the
number of matching pairs, it ignores the constraint that a matching
pair is relevant only when the (uniquely defined) third vertex  of
the resulting triangle belongs to , and that the (uniquely defined)
top line of this triangle through  is -rich. It is therefore
natural to conjecture that our bound is not tight, and that the true
bound is nearly quadratic, perhaps coinciding with the lower bound of
\cite{EP71}.


\begin{thebibliography}{99}

\bibitem{Ch05}
B.~Chazelle,
Cuttings,
In {\em Handbook of Data Structures and Applications}
(D. Mehta and S. Sahni, editors), chap.~25, Chapman and Hall/CRC Press, 2005.

\bibitem{CEGSW}
K. Clarkson, H. Edelsbrunner, L. Guibas, M. Sharir and E. Welzl,
Combinatorial complexity bounds for arrangements of curves and
spheres,
{\it Discrete Comput. Geom.} 5 (1990), 99--160.

\bibitem{DST08}
A. Dumitrescu, M. Sharir and Cs. D. T\'oth,
Extremal problems on triangle areas in two and three dimensions,
{\it J. Combinat. Theory, Ser.~A}, accepted. Also in
in {\it Proc. 24th ACM Symp. on Computational Geometry} (2008), 208--217.

\bibitem{EP71}
P.~Erd\H{o}s and G.~Purdy,
Some extremal problems in geometry,
{\it J. Combinat. Theory} 10 (1971), 246--252.

\bibitem{EP95}
P.~Erd\H os and G.~Purdy,
Extremal problems in combinatorial geometry.
in {\em Handbook of Combinatorics}
(R. Graham, M. Gr\"otschel and L. Lov\'asz, editors),
Vol. 1, 809--874, Elsevier, Amsterdam, 1995.

\bibitem{KST54}
T.~K\H{o}vari, V.~T.~S\'os, and P.~Tur\'an,
On a problem of K.~Zarankiewicz,
{\it Colloquium Math.} 3 (1954), 50--57.

\bibitem{PS92}
J.~Pach and M.~Sharir,
Repeated angles in the plane and related problems,
{\it J. Combinat. Theory Ser. A} 59 (1992), 12--22.

\bibitem{Sha}
I. R. Shafarevich,
{\it Basic Algebraic Geometry},
Springer-Verlag, 1977.

\bibitem{SA}
M. Sharir and P. K. Agarwal,
{\it Davenport-Schinzel Sequences and Their Geometric Applications},
Cambridge University Press, New York, 1995.

\bibitem{ST83}
E.~Szemer\'edi and W.~T.~Trotter,
Extremal problems in discrete geometry,
{\it Combinatorica} 3 (1983), 381--392.

\end{thebibliography}

\appendix

\section{Asymptotes of cubic curves}
In this appendix, we analyze the class of cubic curves defined by
equations (\ref{eq:m14}) and (\ref{eq:m13}) of Section
\ref{sec:unit2}, derive their asymptotes, and show them to be the
zero sets of irreducible bivariate polynomials.
We start by analyzing a normalized version
of these equations, in which two of the generating lines
(and, as we show henceforth, the asymptotes) are the  and -axes.
We then reduce equation (\ref{eq:m14}) to the normalized case.
Finally, we handle equation (\ref{eq:m13}) in a different and simpler
way.

\begin{lemma} \label{lem:irr1_n}
Let  and  be two distinct lines in
, given by the equations , where
,
and  and  are both nonzero, for .
Let  be the bivariate cubic polynomial

Then  is irreducible.
\end{lemma}
\begin{proof}
Assume, to the contrary, that  is reducible. Then it has a
linear factor . Without loss of generality,
 (a symmetric
argument follows for the case ), so we can assume .
Then , as a polynomial in  with coefficients from ,
has  as root, i.e., if we put

where , for ,
then . But then, the term  can not be
properly cubic, nor quadratic, so, , or  is a
constant, possibly zero. In the former case, , but
 and  by assumption,
a contradiction.
If , then we
must also have , and so both lines  and
 coincide (with the line ), contrary to assumption.
If  is a nonzero constant, then the term  must also
be constant, or else  is a proper quadratic polynomial,
hence . But then, for 
to be constant, we must have , in contradiction.
Either way,  cannot be reducible.
\end{proof}

\begin{lemma} \label{lem:as_n}
Let  and  be two distinct lines in ,
given by the equations , where
, for ,
such that  and  are both nonzero.
Let  be the algebraic cubic curve defined by the equation

Then  is asymptotic to the -axis and to the -axis.
\end{lemma}
\begin{proof}
We only prove in detail that the -axis is an asymptote. Note
that, for any fixed , (\ref{eq:n14}) is a quadratic
equation in , which we rewrite as

or

Hence

We only consider the solution with positive square root, which is

The expression in the square brackets is of the form
. Since ,  tends to 0 as . Using the inequalities
,
for , we obtain, for  sufficiently large,

which tends to 0 as . This shows that the
-axis is indeed an asymptote of  (on both sides).
A symmetric argument shows that the -axis is also an
asymptote.
\end{proof}

We are now ready to prove the more general cases discussed in
Section \ref{sec:unit2}.
\begin{lemma} \label{lem:as1}
Let  be four distinct lines in
, given by the equations , where
, for .
Assume that no pair of  are parallel,
and that  is not parallel to any of  and .
Put

and
let  be the algebraic cubic curve defined by the equation

Then,  is irreducible, and  is asymptotic to the lines
.
\end{lemma}
\begin{proof}
We may assume, by an appropriate change of variables, that one of
, and  is the -axis and another one is
the -axis. For example, put , and , and write
, and
,
for some appropriate coefficients
.
Note that, by the preliminary assumptions on the lines,
 and  are both nonzero, for .
 can then be written as

in the  coordinate system. It then follows, by Lemma
\ref{lem:irr1_n}, that  is irreducible, for otherwise, any
factorization of  could be transformed into a factorization of
, in contradiction.
It also follows, by Lemma \ref{lem:as_n}, that  and
 are asymptotes of .
Note that, for this part of the argument, the choice of 
 and  as axes is arbitrary, and we could just 
as well choose any other pair of lines among .
In more detail, since no pair of these three lines are parallel, we
can make any two of them as the axes of a new -coordinate 
system, and then, in the equation of the third line, both the - and
-coefficients would be nonzero, which is the condition assumed in 
Lemma \ref{lem:as_n}. Hence,  is also an asymptote of .
\end{proof}

\begin{lemma} \label{lem:as2}
Let  and  be two distinct intersecting lines in
, given by the equations , where
, for .
Put , for some constant , and
let  be the algebraic curve defined by the equation

Then  is asymptotic to the lines  and .
Furthermore, if , then  is an irreducible bivariate
polynomial.
\end{lemma}
\begin{proof}
If , then the claim is easy. Indeed, in this case we
have , so  is the union of the line
 and the hyperbola , which is asymptotic to
the lines , and .

If , put , and . Then, in the 
coordinate system,  is defined by the equation

Note that  is clearly irreducible, and so is .
This equation can be rewritten as

Clearly, this function tends to 0 as  tends to , which
means that  is asymptotic to the -axis, i.e.,
to .
Furthermore, the function has a pole at , meaning that
 is asymptotic to the -axis, i.e., to .
\end{proof}
\end{document}
