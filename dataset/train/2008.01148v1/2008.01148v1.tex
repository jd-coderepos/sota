\documentclass[runningheads]{llncs}
\usepackage{graphicx}
\usepackage{comment}
\usepackage{amsmath,amssymb} \usepackage{color}

\usepackage{ruler}
\usepackage[width=122mm,left=12mm,paperwidth=146mm,height=193mm,top=12mm,paperheight=217mm]{geometry}
\usepackage{amsmath} \usepackage{amssymb}  \usepackage{mathtools}
\usepackage{xcolor}

\usepackage{eqnarray}
\usepackage{graphicx}
\graphicspath{{images}}

\newcommand{\pa}{HAMLET}
\newcommand{\hattn}{HAT}
\newcommand{\fpa}{HAMLET: \textbf{H}ierarchic\textbf{A}l \textbf{M}ultimodaL s\textbf{E}lf-a\textbf{T}tention}
\newcommand{\fhattn}{HAT: \textbf{H}ierarchical \textbf{At}tention}

\title{HAMLET: Hierarchical Multimodal Attention for Human Activity Recognition
}
\titlerunning{ECCV-20 submission ID \ECCVSubNumber} 
\authorrunning{ECCV-20 submission ID \ECCVSubNumber} 
\author{Anonymous ECCV submission}
\institute{Paper ID \ECCVSubNumber}


\begin{document}
\pagestyle{headings}
\mainmatter
\def\ECCVSubNumber{100}  


\maketitle
\thispagestyle{empty}
\pagestyle{empty}

\begin{abstract}


\keywords{Multimodal Learning, Attention, Human-Activity Recognition}
\end{abstract}


\section{Introduction}
\label{sec:introduction}


Robots are sharing physical spaces with humans in various collaborative environments, from manufacturing to assisted living to healthcare \cite{Riek2017HealthCare,iqbal2019human,Iqbal2016T-RO}, to improve productivity and to reduce human cognitive and physical workload \cite{andi_iros}. To be effective in close proximity to people, collaborative robotic systems (CRS) need the ability to automatically and accurately recognize human activities \cite{fosapt}. This capability will enable CRS to operate safely and autonomously to work alongside human teammates \cite{iqbal2017coordination}.

To fluently and fluidly collaborate with people, CRS needs to recognize the activities performed by their human teammates robustly \cite{tiqbal_joint_action,Iqbal2015TAC,mit_ucsd}. Although modern robots are equipped with various sensors, robust human activity recognition (HAR) remains a fundamental problem for CRS \cite{iqbal2019human}. This is partly because fusing multimodal sensor data efficiently for HAR is challenging. Therefore, to date, many researchers have focused on recognizing human activities by leveraging on a single modality, such as visual, pose or wearable sensors \cite{andi_iros,new_sk_rep,st_graph_sk,space_time_sk_review,iqbal2016tempo}. However, HAR models reliant on unimodal data often suffer a single point feature representation failure. For example, visual occlusion, poor lighting, shadows, or complex background can adversely affect only visual sensor-based HAR methods. Similarly, noisy data from accelerometer or gyroscope sensors can reduce the performance of HAR methods solely depending on these sensors \cite{mit_ucsd,multimodal_survey}. 
















\begin{figure}[!t]
    \centering
    \includegraphics[width=0.9\columnwidth]{images/attention_example.png}
    \caption{Example of two activities (\textit{Sit-Down} and \textit{Carry}) from the UT-Kinect dataset (the first row). The second row presents the temporal-attention weights on the corresponding RGB frames using {\pa }. For these sequences, {\pa } pays more attention to the third RGB image segment for the \textit{Sit-Down} activity (top) and on the fourth RGB image segment for the \textit{Carry} activity (bottom). Here, a lighter color represents a lower attention.}
    \label{fig:hat}
     \vspace{-0.3in}
\end{figure}

Several approaches have been proposed to overcome the weaknesses of the unimodal methods by fusing multimodal sensor data that can provide complementary strengths to achieve a robust HAR  \cite{multimodal_survey,mit_ucsd,self_attention_iros19,keyless,fusion_approaches,Hasan_2019}. 
Although many of these approaches exhibit robust performances than unimodal HAR approaches, there remain several challenges that prevent these methods from efficiently working on CRSs \cite{multimodal_survey}. 
For example, while fusing data from multiple modalities, these methods rely on a fixed-fusion approach, e.g., concatenate, average, or sum. Although one type of fusion approach works for a specific activity, these approaches can not provide any guaranty that the same performance can be achieved on a different activity class using the same merging method. Moreover, these proposed approaches provide uniform weightage on the data from all modalities. However, depending on the environment, one sensor modality may provide more enhanced information than the other sensor modality. For example, a visual sensor may provide valuable information about a gross human activity than a gyroscope sensor data, which a robot needs to learn from data automatically. Thus, these approaches can not provide robust HAR for CRSs.


To address these challenges, in this work, we introduce a novel multimodal human activity recognition algorithm, called {\fpa } algorithm for CRS. 
{\pa } first extracts the spatio-temporal salient features from the unimodal data for each modality. {\pa } then employs a novel multimodal attention mechanism, called \fhattn, for disentangling and fusing the unimodal features. These fused multimodal features enable {\pa } to achieve higher HAR accuracies (see Sec.~\ref{sec:proposedApproach}).

The modular approach to extract spatial-temporal salient features from unimodal data allows {\pa } to incorporate pre-trained feature encoders for some modalities, such as pre-trained ImageNet models for RGB and depth modalities. This flexibility enables {\pa} to incorporate deep neural network-based transfer learning approaches. Additionally, the proposed novel multimodal fusion approach (MAT) utilizes a multi-head self-attention mechanism, which allows {\pa } to be robust in learning weights of different modalities based on their relative importance in~HAR from data.   


We evaluated {\pa } by assessing its performance on three human activity datasets (UCSD-MIT\cite{mit_ucsd}, UTD-MHAD\cite{utd_mhad} and UT-Kinect \cite{ut_kinect}) compared with several state-of-the-art activity recognition algorithms from prior literature (\cite{keyless,mit_ucsd,posemap,sdd_iccv,mcrl,sos,jdm_cnn,dcnn,dmm_mff,utd_mhad}) and two baseline methods (see  Sec.~\ref{sec:experimenResults}).
In our empirical evaluation, {\pa } outperformed all other evaluated baselines across all datasets and metrics tested, with the highest top-1 accuracy of 95.12\% and 97.45\% on the UTD-MHAD \cite{utd_mhad} and the UT-Kinect\cite{ut_kinect} datasets respectively, and F1-score of 81.52\% on the UCSD-MIT \cite{mit_ucsd} dataset (see Sec.~\ref{sec:results_and_discussion}).
We visualize an attention map representing how the unimodal and the multimodal attention mechanism impacts multimodal feature fusion for HAR (see Sec.~\ref{sec:attention_vis}).









\begin{comment}

\par The main contributions of this work are as follows:
\begin{itemize}
    \item We propose a modular multimodal feature learning architecture, {\pa}, where we first introduced modality-based transfer learning by plugging the state-of-the-art pre-trained unimodal feature encoders.
\item We develop a novel multimodal self-attention method, {\hattn}. To the best of our knowledge, we are the first to employ the multi-head self-attention mechanism to disentangle and fuse the unimodal feature objects hierarchically.
    \item In the extensive experimental analysis, our proposed learning method {\pa } showed competitive performance improvement on three heterogeneous multimodal datasets, UTD-MAD\cite{utd_mhad}, UCSD-MIT\cite{mit_ucsd} and UT-Kinect\cite{ut_kinect}.
    \item We also visualize the unimodal and multimodal attention map for analyzing the impact of multimodal attention approach in HAR, which we discussed in Section~\ref{sec:attention_vis}.
\end{itemize}

\end{comment}



%
 \section{Related Works}
\label{sec:relatedWorks}

\textbf{Unimodal HAR:} Human activity recognition has been extensively studied by analyzing and employing the unimodal sensor data, such as skeleton, wearable sensors, and visual (RGB or depth) modalities \cite{hussein2013human}.  As generating hand-crafted features is found to be a difficult task, and these features are often highly domain-specific, many researchers are now utilizing the deep neural network-based approaches for human activity recognition. 


Deep learning-based feature representation architectures, especially convolutional neural networks (CNNs) and long-short-term memory (LSTM), have been widely adopted to encode the spatio-temporal features from visual (i.e., RGB and depth) \cite{new_sk_rep,co_occurrence,closer_look_sp,sp_temporal_relation,sp_3d_conv,slowfast} and non-visual (i.e., sEMG and IMUs) sensors data \cite{andi_iros,mit_ucsd,totty2017muscle}. For example, Li et al. \cite{co_occurrence} developed a CNN-based learning method to capture the spatio-temporal co-occurrences of skeletal joints. To recognizing human activities from video data, Wang et al. proposed a 3D-CNN and LSTM-based hybrid model to detect compute salient features \cite{sp_3d_conv_lstm}. 
Recently, the graphical convolutional network has been adopted to find spatial-temporal patterns in unimodal data \cite{st_graph_sk}.





Although these deep-learning-based HAR methods have shown promising performances in many cases, these approaches rely significantly on modality-specific feature embeddings. If such an encoder fails to encode the feature properly because of noisy data (e.g., visual occlusion or missing or low-quality sensor data), then these activity recognition methods suffer to perform correctly.







\textbf{Multimodal HAR:} Many researchers have started working on designing multimodal learning methods by utilizing the complementary features from different modalities effectively to overcome the dependencies on a single modality data of modality-specific HAR models \cite{Garcia_2018_ECCV,keyless,self_attention_iros19,joze2019mmtm}. One crucial challenge that remains in developing a multimodal learning model is to fuse the various unimodal features efficiently. 



Several approaches have been proposed to fuse data from similar modalities \cite{two_stream_cnn,sp_two_stream_residual,sp_mul_motion_gating,conv_fusion,zhang2018fusing}. For example, Simonyan et al. proposed a two-stream CNN-based architecture, where they incorporated a spatial CNN network to capture the spatial features, and another CNN-based temporal network to learn the temporal features from visual data \cite{two_stream_cnn}.  As CNN-based two-stream network architecture allows to appropriately combine the spatio-temporal features, it has been studied in several recent works, e.g., residual connection in streams \cite{sp_two_stream_residual}, convolutional fusion \cite{conv_fusion} and slow-fast network~\cite{slowfast}.



Other works have focused on fusing features from various modalities, i.e., fusing features from visual (RGB), pose, and wearable sensor modalities simultaneously \cite{multimodal_survey,mfas,joze2019mmtm}. M\"{u}nzner et al. \cite{fusion_approaches} studied four types of feature fusion approaches: early fusion, sensor and channel-based late fusion, and shared filters hybrid fusion. They found that the late and hybrid fusion outperformed early fusion. Other approaches have focused on fusing modality-specific features at a different level of a neural network architecture \cite{mfas}. For example, Joze et al. \cite{joze2019mmtm} designed an incremental feature fusion method, where the features are merged at different levels of the architecture.
Although these approaches have been proposed in the literature, generating multimodal features by dynamically selecting the unimodal features is still an open challenge.


\textbf{Attention mechanism for HAR:}
Attention mechanism has been adopted in various learning architectures to improve the feature representation as it allows the feature encoder to focus on specific parts of the representation while extracting the salient features \cite{attention,attention_effective_approach,xu2015show,lu2017knowing,keyless,mnih2014recurrent,lu2019vilbert,gao2019multi}. Recently, several multi-head self-attention based methods have been proposed, which permit to disentangle the feature embedding into multiple features (multi-head) and to fuse the salient features to produce a robust feature embedding \cite{transformer}. 

Many researchers have started adopting the attention mechanism in human activity recognition \cite{self_attention_iros19,keyless}. For example, Xiang et al. proposed a multimodal video classification network, where they utilized an attention-based spatio-temporal feature encoder to infer modality-specific feature representation \cite{keyless}. The authors explored the different types of multimodal feature fusion approaches (feature concatenation, LSTM fusion, attention fusion, and probabilistic fusion), and found that the concatenated features showed the best performance among the other fusion methods. To date, most of the HAR approaches have utilized attention-based methods for encoding the unimodal features. However, the attention mechanism has not been used for extracting and fusing salient features from multiple modalities.


To address these challenges, in our  proposed multimodal HAR algorithm (\pa),
we have designed a modular way to encode unimodal spatio-temporal features by adopting a multi-head self-attention approach. Additionally, we have developed a novel multimodal attention mechanism, {\hattn}, for disentangling and fusing the salient unimodal features to compute the multimodal features. 












%
 \section{Proposed Modular Learning Method}
\label{sec:proposedApproach}

In this section, we present our proposed multimodal human-activity recognition method, called \fpa. We present the overall architecture in Fig.~\ref{fig:hat}. In {\pa }, the multimodal features are encoded into two steps, and those features are then used for activity recognition as follows:
\begin{itemize}
    \item At first, the Unimodal Feature Encoder module encodes the spatial-temporal features for each modality by employing a modality-specific feature encoder and a multi-head self-attention mechanism (UAT).
    \item In the second step, the Multimodal Feature Fusion module (MAT) fuses the extracted unimodal features by applying our proposed novel multimodal self-attention method.
    \item These computed multimodal features are then utilized by a fully connected neural network to calculate the probability of each activity class.
\end{itemize}



\begin{figure}[!t]
    \centering
    \includegraphics[width=\columnwidth]{images/hat.png}
    \caption{{\pa }: Hierarchical Multimodal Self-Attention based HAR.}
    \label{fig:hat}
    \vspace{-0.2in}
\end{figure}

\subsection{Unimodal Feature Encoder}
The first step of {\pa} is to compute a feature representation for data from every modality. To achieve that, we have designed modality-specific feature encoders to encode data from different modalities. The main reasoning behind this type of modality-specific modular feature encoder architecture is threefold. First, each of the modalities has different feature distribution and thus needs to have a different feature encoder architecture. For example, the distribution and representation of visual data differ from the skeleton and inertial sensor data. Second, the modular architecture allows incorporating unimodal feature encoders without interrupting the performance of the encoders of other modalities. This capability enables the modality-specific transfer learning. Thus we can employ a pre-trained feature encoder to produce robust feature representation for each modality. Third, the unimodal feature encoders can be trained and executed in parallel, which reduces the computation time during the training and inference phases.





Each of the unimodal feature encoders is divided into three separate sequential sub-modules: spatial feature encoder, temporal feature encoder, and unimodal attention module (UAT). Before applying a spatial feature encoder, at first the whole sequence of data  from modality  is converted into segmented sequence  of size , where  is the batch size,  and  are the number of segments and feature dimension for modality  respectively. In this work, we represent the feature dimension  for RGB and depth modality as , where  is the number of channels in an image. 

\subsubsection{Spatial Feature Encoder}
We used a temporal pooling method to encode segment-level features instead of extracting the frame-level features, similar to \cite{keyless}. We have implemented the temporal pooling for two reasons: first, as the successive frames represent similar features, it is redundant to apply spatial feature encoder on each frame, which increases the training and testing time. By Utilizing the temporal pooling, {\pa} reduces its computational time. Moreover, this polling approach is necessary to implement {\pa} on a real-time robotic system. Second, the application of recurrent neural networks for each frame is computationally expensive for a long sequence of data. We used adaptive temporal max-pool to pool the encoded segment level features. 


As our proposed modular architecture allows modality-specific transfer learning, we have incorporated the available state-of-the-art pre-trained unimodal feature encoders. For example, we have incorporated ResNet50 to encode the RGB modality. We extend the convolutional co-occurrence feature learning method \cite{co_occurrence} to hierarchically encode segmented skeleton and inertial sensor data. In this work, we used two stacked 2D-CNNs architecture to encode co-occurrence features: first 2D-CNN encodes the intra-frame point-level information and second 2D-CNN extract the inter-frame features in a segment. Finally, spatial feature encoder for modality  produces a spatial feature representation  of size  from segmented , where  is the spatial feature embedding dimension. 

\subsubsection{Temporal Feature Encoder}
After encoding the segment level unimodal features, we employ recurrent neural networks, specifically unidirectional LSTM, to extract the temporal feature features  of size  from , where  is the LSTM hidden feature dimension. Our choice of unidirectional LSTM over other recurrent neural network architectures (such as gated recurrent units) was based on the ability of LSTM units to capture long-term temporal relationships among the features. Besides, we need our model to detect human activities in real-time, which motivated our choice of unidirectional LSTMs over bi-directional LSTMs. 





\subsubsection{Unimodal Self-Attention ({\uat }) Mechanism}
\label{sec:msa}
The spatial and temporal feature encoder sequentially encodes the long-range features. However, it cannot extract salient features by employing sparse attention to the different parts of the spatial-temporal feature sequence. Self-attention allows the feature encoder to pay attention to the sequential features sparsely and thus produce a robust unimodal feature encoding. Taking inspiration from the Transformer-based multi-head self-attention methods \cite{transformer}, {\uat } combines the temporal sequential salient features for each modality. As each modality has its unique feature representation, the multi-head self-attention enables the {\uat } to disentangle and attend salient unimodal features. 

To compute the attended modality-specific feature embedding  for modality  using unimodal multi-head self-attention method, at first we need to linearly project the spatial-temporal hidden feature embedding  to create query (), key () and value () for head  in the following way,

Here, each modality  has its own projection parameters, , and , where  and  are projection dimensions, , and  is the total number of heads for modality . After that we used scaled dot-product softmax approach to compute the attention score for head  as:

After that, all the head feature representation is concatenated and projected to produce the attended feature representation,  in the following way, 

Here,  is the projection parameters of size , and the shape of  is , where  is the attended feature embedding size. We used the same feature embedding size  for all modalities to simplify the application of multimodal attention {\hattn } for fusing all the modality-specific feature representation, which is presented in the next section~\ref{sec:mat}. However, our proposed multimodal attention based feature fusion method can handle different unimodal feature dimensions. Finally, we fused the attended segmented sequential feature representation  to produce the local unimodal feature representation  of size . We can use different types of fusion to combine the spatio-temporal segmented feature encodings, such as sum, max, or concatenation. However, the concatenation fusion method is not a suitable approach to fuse large sequences, whereas max fusion may lose the temporal feature embedding information. As the sequential feature representations produced from the same modality, we have used the sum fusion approach to fuse attended unimodal spatial-temporal feature embedding ,


\begin{figure}[!t]
    \centering
    \includegraphics[width=\columnwidth]{images/mat.png}
    \caption{{\hattn :} Multimodal Attention-based Feature Fusion Architecture.}
    \label{fig:mma}
     \vspace{-0.2in}
\end{figure}

\subsection{Multimodal Feature Fusion}
\label{sec:mat}

In this work, we developed a novel multimodal feature fusion architecture based on our proposed multi-head self-attention model, \fhattn, which is depicted in Fig.~\ref{fig:mma}. After encoding the unimodal features using the modular feature encoders, we combine these feature embeddings  in an unordered multimodal feature embedding set  of size , where  is the total number of modalities. After that, we fed the set of unimodal feature representations  into \hattn, which produces the attended fused multimodal feature representation . 


The multimodal multi-head self-attention computation is almost similar to the self-attention method described in Section~\ref{sec:msa}. However, there are two key differences. First, unlike encoding the positional information using LSTM to produce the sequential spatial-temporal feature embedding before applying the multi-head self-attention, in {\hattn}, we combine all the modalities feature embeddings without encoding any positional information. Also, {\hattn } and {\uat } modules have separate multi-head self-attention parameters. Second, after applying the multimodal attention method on the extracted unimodal features, we used two fusion approaches to fused the multimodal features:


\begin{itemize}
    \item MAT-SUM: extracted unimodal features are summed after applying the multimodal attention
    
    \item MAT-CONCAT: in this approach the attended multimodal features are concatenated
    
\end{itemize}

\subsection{Activity Recognition}
Finally, the fused multimodal feature representation  is passed through a couple of fully-connected layers to compute the probability for each activity class. For aiding the learning process, we applied activation, dropout, batch normalization in different parts of the learning architecture (see the section~\ref{sec:implementation_details} for the implementation details).
As all the tasks of human-activity recognition, which we addressed in this work, are multiclass classification, we trained the model using cross-entropy loss function, mini-batch stochastic gradient optimization with weight decay regularization \cite{adamw}.
 \section{Results and Discussion}
\label{sec:results_and_discussion}

\begin{table}[!t]
    \centering
    \caption{Performance comparison (mean top-1 accuracy) of multimodal HAR methods on UT-Kinect dataset \cite{ut_kinect}}
    \label{tab:com_on_ut_kinect}
    \begin{tabular}{llc}
        \toprule
        \multirow{1}{*}{Method}& {Fusion Type} & Top-1 Accuracy (\%) \\
        \hline
        \multirow{2}{*}{NSA} & SUM & 54.34 \\
& CONCAT & 52.31\\
        \hline
        \multirow{2}{*}{USA} & SUM & 55.82 \\
& CONCAT & 54.34 \\
        \hline
        \multirow{1}{*}{{KEYLESS \cite{keyless} (2018)}} & CONCAT & 94.50\\
        \hline
        \multirow{2}{*}{\textbf{\pa}} & MAT-SUM & 95.56\\
& \textbf{MAT-CONCAT} & \textbf{97.45}\\
        \bottomrule
    \end{tabular}
    \vspace{-0.2in}
\end{table}






\subsection{Multimodal Attention-based Fusion Approaches}
\label{sec:attention_fusion_approach}

We first evaluated the accuracy of two multimodal attention-based feature fusion approaches of \pa: MAT-SUM and MAT-CONCAT. We also varied the number of heads used in UAT and MAT steps to determine the optimal configuration of these values.


\par \textbf{Results:} We evaluated UAT and MAT attention methods as well as the fusion approaches (MAT-SUM and MAT-CONCAT) on the UT-Kinect dataset \cite{ut_kinect}, presented in Table~\ref{tab:com_fusion_on_ut_kinect}. We used the RGB and skeleton modalities and reported top-1 accuracy by following the original evaluation scheme. The results suggest that the MAT-CONCAT fusion method showed the highest top-1 accuracy (97.45\%), with one and two heads in UAT and MAT methods, respectively.


\par \textbf{Discussion:} The results suggest the concatenation-based fusion approach (MAT-CONCAT) performed better than the summation-based fusion approach (MAT-SUM). Because the MAT-CONCAT allows {\hattn } to disentangle and apply attention mechanisms on the unimodal features to generate robust multimodal features for activity classification. On the other hand, the sum-based fusion method merged the unimodal features into a single representation, which makes it difficult for {\hattn } to disentangle and apply appropriate attention to unimodal features.

\par The results from Table~\ref{tab:com_fusion_on_ut_kinect} also indicate an improvement in activity recognition accuracy with the increment of the number of heads in the MAT when keeping the number of heads fixed in the UAT. However, this relationship does not hold when the number of heads was changed in the UAT. As a large number of heads reduce the size of feature embedding, increasing the number of heads in the UAT may result in an inadequate feature representation. Thus, based on the size of the features used in this work, the results suggest that one head in the UAT and two heads in the MAT methods display the best accuracy. Thus, we utilized these values for further evaluations. 

\begin{table}[!t]
\centering
\caption{Performance comparison (mean top-1 accuracy) of multimodal fusion methods on UTD-MHAD dataset \cite{utd_mhad}}
    \label{tab:com_utd_mhad}
\begin{tabular}{ccc}
\toprule
    Method & Year & Top-1 Accuracy (\%) \\ \hline
    Kinect \& Inertial \cite{utd_mhad} & 2015 & 79.10 \\
    DMM-MFF \cite{dmm_mff} & 2015 & 88.40 \\
    DCNN \cite{dcnn} & 2016 & 91.2\\
    JDM-CNN \cite{jdm_cnn} & 2017 & 88.10 \\
    SDDI \cite{sdd_iccv} & 2017 & 89.04 \\
    SOS \cite{sos} & 2018 & 86.97 \\
    MCRL \cite{mcrl} & 2018 & 93.02 \\
    PoseMap \cite{posemap} & 2018 & 94.51 \\ 
    \textbf{\pa{ }(MAT-CONCAT)} & - & \textbf{95.12} \\
\bottomrule
\end{tabular}
\label{tab:accuracy33}
    \vspace{-0.2in}
\end{table}


\subsection{Comparison with Multimodal HAR Methods}\label{sec:com_multimodal_har}
As {\pa } takes a multimodal approach, it is reasonable to evaluate the accuracy against the state-of-the-art multimodal approaches. Thus, we compare the performance of {\pa } with two baseline methods (the USA and the NSA, see Sec.~\ref{sec:baselines}) and several state-of-the-art multimodal approaches. We presented the results in Tables~\ref{tab:com_on_ut_kinect} (UT-Kinect),~\ref{tab:com_utd_mhad} (UTD-MHAD) \&~\ref{tab:com_on_mit_ucsd} (UCSD-MIT).




{\textbf{Results:}} In the UT-Kinect dataset, RGB and skeleton modalities have been used to train the learning models. Following the original evaluation scheme, we report the top-1 accuracy in Table~\ref{tab:com_on_ut_kinect}. The results indicate that {\pa } achieved the highest 97.45\% top-1 accuracy across all other methods.

We also evaluate the performance of {\pa } on the UTD-MHAD \cite{utd_mhad} dataset. We train and test {\pa } on RGB and Skeleton data and report the top-1 accuracy while using MAT-CONCAT in Table~\ref{tab:com_utd_mhad}. The results suggest that {\pa } outperformed all the evaluated state-of-the-art baselines and achieved the highest accuracy of 95.12\%. 

\par For the UCSD-MIT dataset, all the learning methods are trained on the skeleton, inertial, and sEMG data. All the training models have been used late or intermediate fusion except for the results presented from \cite{mit_ucsd}, which used an early feature fusion approach. In Table~\ref{tab:com_on_mit_ucsd}, the results suggest that {\pa } with MAT-SUM fusion method outperformed the baselines and state-of-the-art works by achieving the highest 81.52\% F1-score (in \%).  

\begin{table}[!t]
    \centering
    \caption{Performance comparison (mean F1-scores in \%) of multimodal HAR methods on UCSD-MIT dataset \cite{mit_ucsd}}
    \label{tab:com_on_mit_ucsd}
    \begin{tabular}{llc}
        \toprule
        \multirow{1}{*}{Method} & Fusion Type &  F1-Score (\%) \\
        \hline
        \multirow{2}{*}{NSA} & SUM & 59.61\\
& CONCAT & 45.10\\
        \hline
        \multirow{2}{*}{USA} & SUM & 60.78\\
& CONCAT & 69.85\\
        \hline
        \multirow{1}{*}{{KEYLESS \cite{keyless}}} (2018) & CONCAT & 74.40\\
        \hline
        \multirow{1}{*}{{Best of UCSD-MIT\cite{mit_ucsd}}} (2019) & Early Fusion & 59.0 \\
        \hline
        \multirow{2}{*}{\textbf{\pa}} & \textbf{MAT-SUM} & \textbf{81.52}\\
& MAT-CONCAT & 76.86\\
        \bottomrule
    \end{tabular}
     \vspace{-0.2in}
\end{table}



\begin{figure*}[!t]
  \centering
  \begin{tabular}{*{3}{c}}
    \includegraphics[width=.65\columnwidth]{images/utk_cm_utk_hamlet_bl_nsa.png} &
    \includegraphics[width=.65\columnwidth]{images/utk_cm_utk_hamlet_bl.png} &
    \includegraphics[width=.65\columnwidth]{images/utk_cm_utk_hamlet.png} \\
    \footnotesize{(a) Without attention} & \footnotesize{(b) Unimodal attention} & \footnotesize{(c) Unimodal and multimodal attention} \\
  \end{tabular}
  \caption{Comparative impact of multimodal and unimodal attention in {\pa } for different activities on UT-Kinect dataset.}
  \label{fig:com_cm}
\end{figure*}

\begin{figure*}[!t]
  \centering
  \begin{tabular}{*{3}{c}}
    \includegraphics[width=.65\columnwidth]{images/utk_rgb_seq_attn_ts8.png} &
    \includegraphics[width=.65\columnwidth]{images/utk_skeleton_seq_attn_ts8.png} & \includegraphics[width=.65\columnwidth]{images/utk_modality_attn_ts8.png} \\
    \footnotesize{(a) RGB sequence embedding attention} & \footnotesize{(b) Skeleton sequence embedding attention} & \footnotesize{(c) Multimodal fusion attention} \\
  \end{tabular}
  \caption{Multimodal and unimodal attention visualization for different activities on UT-Kinect Dataset.}
  \label{fig:vis_attention}
\end{figure*}




{\textbf{Discussion:}} {\pa } outperformed all other evaluated baselines across all datasets and metrics tested. The results on the UTD-MHAD dataset suggest that {\pa } outperformed all the state-of-the-art multimodal HAR methods. These methods didn't leverage the attention-based approaches to dynamically weighting the unimodal features to generate multimodal features. The results also suggest that, the other attention-based approaches, such as USA and Keyless \cite{keyless}, also showed better performance compared to the non-attention based approaches on UT-Kinect (Table~\ref{tab:com_on_ut_kinect}) and UCSD-MIT (Table~\ref{tab:com_on_ut_kinect}) datasets. The overall results support that our proposed approach is robust in finding appropriate multimodal features, hence it has achieved the highest HAR~accuracies.    


\par The results indicate that the MAT-CONCAT approach achieved higher accuracy on the UT-Kinect dataset; however, the MAT-SUM approach delivered higher accuracy on the UCSD-MIT dataset. One explanation behind this variation is that the modalities (skeleton, sEMG, and IMUs) in the UCSD-MIT dataset represent similar physical body features, thus summing up the feature vectors work well. However, as the UT-Kinect dataset modalities have different characteristics, the visual (RGB) and the physical body (skeleton) features, MAT-CONCAT works better than MAT-SUM.

\par Finally, the overall results suggest that {\pa} achieved the mean F-1 score of 81.52\% on the UCSD-MIT dataset, which is lower compared to the highest accuracy on other datasets (please note that the top-1 accuracies were presented for other datasets). The main reason behind this performance degradation in UCSD-MIT is that this dataset contains missing data, especially sEMG, and IMUs data are missing in many instances. However, in the presence of the missing information, {\pa } showed the best performance compared to all other approaches.


\begin{comment}


\subsubsection{Comparison with Unimodal HAR}
\label{sec:com_unimodal_har}
We evaluated our proposed multimodal HAR method, {\pa}, with the state-of-the-art unimodal HAR approaches on the UCSD-MIT\cite{mit_ucsd} and UT-Kinect\cite{ut_kinect} datasets, which are presented in Table~\ref{tab:com_on_uni_mit_ucsd}. 

\par HAMLET and state-of-the-art multimodal HAR methods, such as Keyless\cite{keyless} and KNN\cite{mit_ucsd}, didn't able to outperform the single modality(skeleton) based approach. The main reason behind this performance degradation of all multimodal HAR methods is that UCSD-MIT contains missing information, especially sEMG, and IMUs data. Due to the relatively less missing feature training samples in skeleton modality, KNN\cite{mit_ucsd} with skeleton data modality showed comparatively better performance. However, the same KNN\cite{mit_ucsd} learning method with multimodal data (skeleton, sEMG, and IMUs) didn't show better performance in comparison with only skeleton-based KNN classifier. If we provide enough non-missing training data samples, then {\pa } outperformed the unimodal based learning approach and achieved the highest accuracy 89.49\%, which is shown in Table~\ref{tab:com_on_uni_mit_ucsd} [{\pa }(FT)]. In this case, We keep the first performer data, which contains relatively less non-missing training sample data across the modalities, and performed the leave-one-actor-out cross-validation on the rest of the performers, which contains missing modality features.

\begin{table}[!h]
    \centering
    \caption{Performance comparison of unimodal HAR methods with {\pa} on UCSD-MIT\cite{mit_ucsd} (S: skeleton, E: sEMG, I: IMUs)}
    \label{tab:com_on_uni_mit_ucsd}
    \begin{tabular}{lcc}
        \toprule
        Method& Modality & F1-Score (\%) \\
        \hline
        KNN \cite{mit_ucsd} & S & 88.0\\
KNN \cite{mit_ucsd} & S+E+I & 59.0\\
KEYLESS\cite{keyless}& S+E+I & 74.40\\
{\pa} & S+E+I & 81.52\\ 
{\pa } (FT) & S+E+I & 89.49\\ 
        \bottomrule
    \end{tabular}
\end{table}

Moreover, we also compared the performance of {\pa } with the state-of-the-art unimodal methods on UT-Kinect datasets. {\pa } showed competitive performance by achieving 97.5\% top-1 accuracy in comparison with skeleton-based GR-GCN methods, which achieved 98.5\% accuracy. As there are some low-quality data in the RGB modality, if we provide some better training samples and conduct the leave-one-actor-out cross-validation, then our proposed method {\pa } outperformed the state-of-the-art works and achieved the highest accuracy 98.89\%.

\begin{table}[!h]
    \centering
    \caption{Performance comparison of unimodal HAR methods with {\pa} on UT-Kinect \cite{ut_kinect} (V:RGB, S: skeleton)}
    \label{tab:com_on_uni_mit_ucsd}
    \begin{tabular}{lcc}
        \toprule
        Method& Modality & F1-Score (\%) \\
        \hline
        Bi-LSTM \cite{bi_lstm}  & S & 96.9\\
ST-LSTM(Tree) + Trust Gate \cite{st_lstm_trust_gate} & S & 97.0\\
GR-GCN \cite{gr_gcn}& S & 98.5\\
{\pa} & V+S & 97.5\\ 
{\pa } (FT) & V+S & 98.89\\ 
        \bottomrule
    \end{tabular}
\end{table}


\end{comment}



\subsection{Combined Impact of Unimodal and Multimodal Attention}
\par We evaluated the comparative importance of unimodal and multimodal attention mechanism (presented in Fig.~\ref{fig:com_cm}). We can observe that the incorporation of unimodal attention (Fig.~\ref{fig:com_cm}-b) can help to reduce the miss-classification error in comparison to the non-attention based feature learning method (Fig.~\ref{fig:com_cm}-a). This is because unimodal attention can able to extract the sparse salient spatio-temporal features. We also can observe an improved accuracy in activity classification when the multimodal attention based unimodal feature fusion approach was incorporated (Fig.~\ref{fig:com_cm}-c vs. a, b). The results indicate that {\pa } can reduce the number of miss-classification, especially in the cases of similar activities, such as sitDown and pickUp, which is depicted in the confusion matrix in Fig.~\ref{fig:com_cm}-c.



\subsection{Visualizing Impact of Multimodal Attention: \hattn}
\label{sec:attention_vis}

We visualize the attention map of the unimodal and multimodal feature encoders to gauge the impact of attention in local (unimodal) and global (multimodal) feature representation in Fig~\ref{fig:vis_attention}. We used the data of the eighth performer from the UT-Kinect dataset \cite{ut_kinect} as a sample data to produce the attention map for different activities, as shown in Fig.~\ref{fig:vis_attention}, where we observe that the unimodal attention is able to detect salient segments of RGB (Fig~\ref{fig:vis_attention}-a) and skeleton (Fig~\ref{fig:vis_attention}-b) modalities. For example, the unimodal attention method focuses on the beginning parts of the \textit{sitDown} and the \textit{pull} activities, as these activities have distinguishable actions in the beginning parts of the activity. On the other hand, the unimodal attention method needs to pay attention to the full sequence to differentiate the \textit{carry} and the \textit{push} activities, as a specific part of these activities are not more informative than the other parts.

\par Moreover, we evaluate the impact of {\hattn } by observing the multimodal attention map in Fig.~\ref{fig:vis_attention}-c, which represents the relative attention given to unimodal features. For example, the \textit{pickUp} and \textit{sitDown} may involve similar skeleton joints movements, and thus if we concentrate only on the skeleton data, it may be challenging to differentiate between these two activities. However, if we incorporate the complementary modalities, such as RGB and skeleton, it may be easier to differentiate between similar activities. Thus, {\hattn} pays equal attention to the RGB and skeleton data while recognizing the \textit{sitDown} activity, whereas solely pay attention to the skeleton data while identifying the \textit{pickUp} activity (Fig.~\ref{fig:vis_attention}-c).  

 \section{Conclusion}
\label{sec:conclusion}



In this paper, we presented {\pa }, a novel multimodal human activity recognition algorithm, for collaborative robotic systems. 
{\pa } first extracts the spatio-temporal salient features from the unimodal data and then employs a novel multimodal attention mechanism for disentangling and fusing the unimodal features for activity recognition. The experimental results suggest that {\pa } outperformed all other evaluated baselines across all datasets and metrics tested for human activity recognition.


\par In the future, we plan to implement {\pa } on a robotic system to enable it to perform collaborative activities in close proximity with people in an industrial environment. We also plan to extend {\pa } so that it can appropriately learn the relationship among the data from the modalities to address the missing data problem.  \cite{keyless}

\bibliographystyle{splncs04}
\bibliography{references}

\end{document}
