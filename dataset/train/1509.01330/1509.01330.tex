\subsection{Problem Statement}
Given a network represented by a graph, where  is the set of all servers across data centers. We assume that data centers are connected through a backbone network. The backbone's links are represented by a set of edges   . Each edge  is characterized by a bandwidth capacity . Let  denote the set of partitions, replicas of which are stored across the servers where   is the size of replica of partition . We define a configuration as a particular placement of the replicas of partitions within servers. Given an initial and a final configuration, denoted respectively by  and , our goal is to find the optimal sequence of replica migrations that minimizes the migration time from  to   while meeting the minimum partition availability threshold  and abiding by edge bandwidth capacities. We model this as an Integer Linear Programming (ILP) problem.
Tables I and~II~show respectively the inputs of the ILP and its variables.
\begin{table} [!htb]
\caption{Problem inputs}
\begin{tabular}{|m{23pt}|m{208pt}|} 
\hline
	\textbf{Input}
 	& 
	\textbf{Definition}
 \\
\hline
	
 	& 
 	Set of servers across data centers, where 
\\ 
 
\hline
	
 	& 
 	Set of edges connecting servers in 
\\

\hline
	
	& 
	Bandwidth capacity 
\\

\hline
 	
	& 
	Set of partitions, where  and  is the size of partition  
 \\

\hline
	
	& 
	 matrix representing an initial configuration, where
	

\\

\hline
	
	&
	 matrix representing a final configuration, where 
\\

\hline
	
 	&  
	Worst-case migration time
 \\

\hline
	
	&
	 matrix representing edges used in a path,  where \\ & 

 \\

\hline

&  
A big constant 
 \\
\hline
\end{tabular}
 \end{table}




\begin{table}[!htb]

\caption{Problem variables}
\begin{tabular}{|p{22pt}|p{208pt}|}
\hline
\parbox{22pt}{\raggedright 
\textbf{Variable}
} & \parbox{208pt}{\raggedright 
\textbf{Definition}
} \\
\hline
\parbox{22pt}{\raggedright 



} & \parbox{208pt}{\raggedright 

 matrix representing migration sequence, where

} \\
\hline
\parbox{22pt}{\raggedright 



} & \parbox{208pt}{\raggedright 
 matrix representing a need for partition migration, where

} \\
\hline
\parbox{22pt}{\raggedright 



} & \parbox{208pt}{\raggedright 

 matrix representing replica placement, where

} \\
\hline
\parbox{22pt}{\raggedright 


} & \parbox{208pt}{\raggedright 

 matrix representing load  on edge  at time ,  where 
} \\
\hline
\parbox{22pt}{\raggedright 



} & \parbox{208pt}{\raggedright 

 matrix, where  represents the bandwidth allocated for migrating replica of  from  ro  at time 
} \\
\hline
\parbox{22pt}{\raggedright 



} & \parbox{208pt}{\raggedright 


 matrix representing the replicas that are to be deleted at time , where 

} \\
\hline
\parbox{22pt}{\raggedright 



} & \parbox{208pt}{\raggedright 


matrix, where represents the capacity of the path between  to
 at time 
} \\
\hline
\parbox{22pt}{\raggedright 



} & \parbox{208pt}{\raggedright 
A vector of size , where 
} \\
\hline
\end{tabular}
\end{table}
\subsection{Constraints}
Given the initial and final configurations, any discrepancy in the configurations necessitates either the migration or the deletion of the partition replicas. Consider that the migration or deletion of the replica is identified by variables
  and , respectively. Then, if a server  has replica of partition  in the initial and final configuration and no action is required, then there should be no migration of replica of partition    from any server  to , that is, ,  and there should be no deletion of replica of partition  from server , that is,  , as in (\ref{eq:1}).  However, if the replica of partition  is present on server  in the initial configuration and not in the final configuration, then the partition should be eventually deleted, hence,  is set to~1, with constraint (\ref{eq:2}). Most importantly, in constraint (\ref{eq:3}) we capture the need for a replica migration, if the server  does not have the replica of partition  in the initial configuration. In this case, there should be some server  that delivers the replica of partition  to server , therefore .



Before we can initiate the migration, we have to identify the servers  that hold the replica of partition  at time , in variable  in constraint (\ref{eq:4}). To begin the migration, only those servers  can participate in the replica migration that hold a replica of partition  in the initial configuration.

The variable  that indicates potential sources of replicas that indicates potential sources of replicas is updated in each time instance , as servers  may begin to hold a copy of the replica of partition , due to migration in earlier time instances, as in constraints (\ref{eq:5}) and (\ref{eq:6}). A server holds a copy of the replica when the sum of all the bandwidth allocated,  to the migration of that replica of partition  from source  to destination , in previous time instances , equals the size of the partition . Then, in following time instances, server  could potentially participate in the replica migration.
\vspace{-5mm}

\vspace{-10mm}

The bandwidth allocated for migration of replica of partition  at time  from source  to destination server , in variable , is dependent on the capacity of the shortest path from  to , in variable , or the remaining size of the partition replica to be migrated in constraint (\ref{eq:7}).
\vspace{-3mm}

The capacity of the shortest path between  and  is inferred from the load on the edges traversed in the path. The edge with the least capacity, bounds the capacity of the path from above. The available path capacity at time  is in constraint (\ref{eq:8}). The edges can be traversed by multiple paths, that is, multiple paths between different source destination pairs can share common edges. Therefore, the load on an edge, not exceeding edge capacity, is conjured as the sum of all partition replicas migrating in the network across all source and destination servers at time , on edge  in the shortest path between  and , by constraints (\ref{eq:9}).
\vspace{-3mm}

\vspace{-10mm}

Once the model has been initialized with potential providers, need for migration and the bandwidth allocated for the migration of partition replicas, we can initiate migration by associating the replica of partition migration indicator variable  with need for migration of replica of partition  from  to , in variable , in constraints (\ref{eq:10}) and (\ref{eq:11}). Consequentially, from (\ref{eq:10}) and (\ref{eq:11}), we ensure only sequential migration of replica of partition, since concurrency is set to~1. Furthermore, in constraint (\ref{eq:12}), we bind the source server , such that, only those servers that hold complete replica of partition  can initiate migration.  



To ensure continuous sequential migration of replica of partition , from same source server  to same destination server  for next time instance , we set the indicator of migration is progress, in variable , to 1, until the entire replica of the partition has been migrated. This is depicted in constraints (\ref{eq:13}) and (\ref{eq:14}). 
\vspace{-5mm}

\vspace{-10mm}

During the migration process the servers must maintain the minimum availability threshold for each partition with constraint (\ref{eq:15}).

The total migration time is extended to include all migrations in progress in constraint (\ref{eq:16}) and stopping the migration process in constraint (\ref{eq:17}).
\vspace{-3mm}


\vspace{-3mm}
\subsection{Objective}
\vspace{-5mm}

We minimize the total migration time in (\ref{eq:18}). As the optimization minimizes migration times, it will select source servers for replica migration that reduce migration time, such that it selects source-destination pairs that have minimum overlapping edges in the shortest path, while ensuring sequential migrations, meeting minimum partition availability threshold and abiding by edge bandwidth capacities.