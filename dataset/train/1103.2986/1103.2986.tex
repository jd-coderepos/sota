\documentclass{llncs}

\usepackage{multicol}
\usepackage{amsmath}
\usepackage{amsfonts}
\usepackage{amssymb}
\usepackage{graphicx}
\usepackage{epic}
\usepackage{eepic}
\usepackage{epsfig,float}
\usepackage{pdfsync}

\DeclareGraphicsRule{.tif}{png}{.png}{`convert #1 `dirname #1`/`basename #1 .tif`.png}

\renewcommand{\le}{\leqslant}
\renewcommand{\ge}{\geqslant}

\newcommand{\ol}{\overline}
\newcommand{\eps}{\varepsilon}
\newcommand{\emp}{\emptyset}
\newcommand{\rhoR}{R}
\newcommand{\Sig}{\Sigma}
\newcommand{\sig}{\sigma}
\newcommand{\noin}{\noindent}
\newcommand{\pf}{prefix-focused}
\newcommand{\ur}{uniquely reachable}
\newcommand{\bi}{\begin{itemize}}
\newcommand{\ei}{\end{itemize}}
\newcommand{\be}{\begin{enumerate}}
\newcommand{\ee}{\end{enumerate}}
\newcommand{\bd}{\begin{description}}
\newcommand{\ed}{\end{description}}
\newcommand{\bq}{\begin{quote}}
\newcommand{\eq}{\end{quote}}
\newcommand{\txt}[1]{\mbox{ #1 }}
\newcommand{\defeq}{\stackrel{\rm def}{=}}
\newcommand{\etc}{\mbox{\it etc.}}
\newcommand{\ie}{\mbox{\it i.e.}}
\newcommand{\eg}{\mbox{\it e.g.}}
\newcommand{\FigureDirectory}{FIGS}

\newcommand{\timg}{\mathop{\mbox{rng}}}
\newcommand{\tdom}{\mathop{\mbox{dom}}}

\newcommand{\inv}[1]{\mbox{}}

\newcommand{\stress}[1]{{\fontfamily{cmtt}\selectfont #1}}

\def\shu{\mathbin{\mathchoice
{\rule{.3pt}{1ex}\rule{.3em}{.3pt}\rule{.3pt}{1ex}
\rule{.3em}{.3pt}\rule{.3pt}{1ex}}
{\rule{.3pt}{1ex}\rule{.3em}{.3pt}\rule{.3pt}{1ex}
\rule{.3em}{.3pt}\rule{.3pt}{1ex}}
{\rule{.2pt}{.7ex}\rule{.2em}{.2pt}\rule{.2pt}{.7ex}
\rule{.2em}{.2pt}\rule{.2pt}{.7ex}}
{\rule{.3pt}{1ex}\rule{.3em}{.3pt}\rule{.3pt}{1ex}
\rule{.3em}{.3pt}\rule{.3pt}{1ex}}\mkern2mu}}

\newcommand{\cA}{{\mathcal A}}
\newcommand{\cB}{{\mathcal B}}
\newcommand{\cC}{{\mathcal C}}
\newcommand{\cD}{{\mathcal D}}
\newcommand{\cI}{{\mathcal I}}
\newcommand{\cL}{{\mathcal L}}
\newcommand{\cM}{{\mathcal M}}
\newcommand{\cN}{{\mathcal N}}
\newcommand{\cS}{{\mathcal S}}
\newcommand{\cT}{{\mathcal T}}
\newcommand{\one}{{\mathbf 1}}
\newcommand{\ra}{{\rightarrow}}
\newcommand{\Lra}{{\hspace{.1cm}\Leftrightarrow\hspace{.1cm}}}
\newcommand{\lra}{{\hspace{.1cm}\leftrightarrow\hspace{.1cm}}}
\newcommand{\la}{{\hspace{.1cm}\leftarrow\hspace{.1cm}}}
\newcommand{\raL}{{\hspace{.1cm}{\sim_L} \hspace{.1cm}}}
\newcommand{\lraL}{{\hspace{.1cm}{\approx_L} \hspace{.1cm}}}
\newcommand{\seq}{{\mathfrak{s}}}
\newcommand{\tpath}{P}


\newcommand{\Bsf}{\mathbf{B}_{\mathrm{sf}}} 
\newcommand{\Bbf}{\mathbf{B}_{\mathrm{bf}}} 
\newcommand{\Bff}{\mathbf{B}_{\mathrm{ff}}}
\newcommand{\Vsf}{\mathbf{W}^{\le 5}_{\mathrm{sf}}} 
\newcommand{\Wsf}{\mathbf{W}^{\ge 6}_{\mathrm{sf}}} 
\newcommand{\Vbf}{\mathbf{W}^{\le 5}_{\mathrm{bf}}} 
\newcommand{\Wbf}{\mathbf{W}^{\ge 6}_{\mathrm{bf}}} 
\newcommand{\Wff}{\mathbf{W}_{\mathrm{ff}}} 
\newcommand{\Gsf}{\mathbf{G}^{\le 5}_{\mathrm{sf}}} 
\newcommand{\Hsf}{\mathbf{G}^{\ge 6}_{\mathrm{sf}}}
\newcommand{\Gbf}{\mathbf{G}^{\le 5}_{\mathrm{bf}}} 
\newcommand{\Hbf}{\mathbf{G}^{\ge 6}_{\mathrm{bf}}}
\newcommand{\Hff}{\mathbf{G}_{\mathrm{ff}}} 
\newcommand{\Uf}{\mathbf{U}}

\newcommand{\bsf}{{\mathrm{b}_{\mathrm{sf}}}} 
\newcommand{\bbf}{{\mathrm{b}_{\mathrm{bf}}}} 
\newcommand{\bff}{{\mathrm{b}_{\mathrm{ff}}}} 
\newcommand{\vsf}{{\mathrm{w}^{\le 5}_{\mathrm{sf}}}} 
\newcommand{\wsf}{{\mathrm{w}^{\ge 6}_{\mathrm{sf}}}}
\newcommand{\vbf}{{\mathrm{w}^{\le 5}_{\mathrm{bf}}}} 
\newcommand{\wbf}{{\mathrm{w}^{\ge 6}_{\mathrm{bf}}}} 
\newcommand{\wff}{{\mathrm{w}_{\mathrm{ff}}}} 

\newcommand{\pc}{{prefix-continuous}}

\newcommand{\qedb}{\hfill} 





\titlerunning{Syntactic Complexity of Prefix-, Suffix-, Bifix-, and Factor-Free Regular Languages}
\authorrunning{J.~Brzozowski, B.~Li, and Y.~Ye}

\title{Syntactic Complexity of Prefix-, Suffix-, Bifix-, and Factor-Free Regular Languages
\thanks{This work was supported by the Natural Sciences and Engineering Research Council of Canada under grant No.~OGP0000871 and  a Postgraduate Scholarship, and by a Graduate Award from the Department of Computer Science, University of Toronto.
}}

\author{Janusz~Brzozowski\inst{1}, Baiyu Li\inst{1}, \and Yuli Ye\inst{2}}

\institute{David R. Cheriton School of Computer Science, University of Waterloo \\
Waterloo, ON, Canada N2L 3G1\\
email: \email{\{brzozo, b5li\}@uwaterloo.ca}
\and
Department of Computer Science, University of Toronto\\
 Toronto, ON,  Canada M5S 3G4\\
email: \email{y3ye@cs.toronto.edu}
}

\begin{document}

\maketitle

\begin{abstract}
The syntactic complexity of a regular language is the cardinality of its syntactic semigroup.
The syntactic complexity of a subclass of the class of regular languages is the maximal syntactic complexity of languages in that class, taken as a function of the state complexity  of these languages.
We study the syntactic complexity of  prefix-, suffix-, bifix-, and factor-free regular languages.
We prove that  is a tight upper bound for prefix-free regular languages. We present properties of the syntactic semigroups of suffix-, bifix-, and factor-free regular languages, conjecture tight upper bounds on their size to be , , and , respectively, and exhibit languages with these syntactic complexities.
\end{abstract}

\noin{\bf keyword}
bifix-free, factor-free, finite automaton, monoid, prefix-free, regular language, reversal, semigroup,  suffix-free, syntactic complexity 

\section{Introduction}

A language is \emph{prefix-free} (respectively, \emph{suffix-free}, \emph{factor-free}) if it does not contain any pair of words such that one is a proper prefix (respectively, suffix, factor) of the other. It is \emph{bifix-free} if it is both prefix- and suffix-free.
We refer to  prefix-, suffix-, bifix-, and factor-free languages as \emph{free} languages.
Nontrivial prefix-, suffix-, bifix-, and factor-free languages are also known as prefix, suffix, bifix, and infix codes~\cite{BPR09,Shy01}, respectively and, have many applications in areas such as cryptography, data compression, and information processing. 


The \emph{state complexity} of a regular language is the  number of states in the minimal deterministic finite automaton (DFA) recognizing that language. 
An equivalent notion is that of \emph{quotient complexity,} which is the number of left quotients of the language.
State complexity of regular operations has been studied quite extensively: for surveys of this topic and lists of references we refer the reader to~\cite{Brz09,Yu01}. 
With regard to free regular languages,  Han, Salomaa and Wood~\cite{HSW09} examined  prefix-free regular languages, and  Han and Salomaa~\cite{HS09} studied suffix-free regular languages. 
Bifix- and factor-free regular languages were studied by Brzozowski, Jir\'askov\'a, Li, and Smith~\cite{BJLS11}. 

The notion of quotient complexity can be derived from the Nerode right congruence~\cite{Ner58}, 
while the Myhill congruence~\cite{Myh57} leads to the syntactic semigroup of a language and to its \emph{syntactic complexity}, which is the cardinality of the syntactic semigroup.
It was pointed out in~\cite{BrYe11} that syntactic complexity can be very different for regular languages with the same quotient complexity.
Thus, for a fixed , languages with quotient complexity  may possibly be distinguished by their syntactic complexities.

In contrast to state complexity, syntactic complexity has not received much attention. In 1970 Maslov~\cite{Mas70} dealt with the problem of generators of the semigroup of all transformations in the setting of finite automata. In 2003--2004, Holzer and K\"onig~\cite{HoKo04}, and independently, Krawetz, Lawrence and Shallit~\cite{KLS03} studied the syntactic complexity of languages with unary and binary alphabets. 
In 2010 Brzozowski and Ye~\cite{BrYe11} examined the syntactic complexity of ideal and closed regular languages, and in 2011 Brzozowski and Li~\cite{BL11} studied the syntactic complexity of star-free languages. 
Here, we deal with the syntactic complexity of  prefix-, suffix-, bifix-, and factor-free regular languages, and their complements.

Basic definitions and facts are stated in Sections~\ref{sec:trans} and~\ref{sec:complexity}. In Section~\ref{sec:pf} we obtain a tight upper bound on the syntactic complexity of prefix-free regular languages. In Sections~\ref{sec:sf}--\ref{sec:ff} we study the syntactic complexity of suffix-, bifix-, and factor-free regular languages, respectively.  We state conjectures about tight upper bounds for these classes, and exhibit languages in these classes that have large syntactic complexities.   In Section~\ref{sec:rev} we show that the upper bounds on the quotient complexity of reversal of prefix-, suffix-, bifix-, and factor-free regular languages can be met by our languages with largest syntactic complexities. Section~\ref{sec:cl} concludes the paper.


\section{Transformations}\label{sec:trans}

A {\em transformation} of a set  is a mapping of  into itself. In this paper we consider only transformations of finite sets, and we assume without loss of generality  that . Let  be a transformation of . If  , then  is the {\it image} of  under .  If  is a subset of , then , and the {\em restriction} of  to , denoted by , is a mapping from  to  such that  for all . The {\em composition} of two transformations  and  of  is a transformation  such that  for all . We usually drop the composition operator ``'' and write  for short. 
An arbitrary transformation can be written in the form

where ,  , and . The {\em domain}  of  is 
The {\em range}  of  under  is the set
 We also use the notation  for the transformation  above. 

A \emph{permutation} of  is a mapping of  \emph{onto} itself. In other words,  a permutation  of  is a transformation where . 
The \emph{identity} transformation maps each element to itself, that is,  for .
A transformation  contains a \emph{cycle} of length  if there exist pairwise different elements  such that
, and .
A cycle is denoted by .
For , a \emph{transposition} is the cycle , and  is the identity.
A \emph{singular} transformation, denoted by , has  and  for all , and  is the identity.
A~\emph{constant} transformation,  denoted by , has  for all .

The set of all transformations of a set , denoted by , is a finite monoid. The set of all permutations of  is a group, denoted by  and called the \emph{symmetric group} of degree . 
It was shown in~\cite{Hoy1895,Pic38} that two generators are sufficient to generate the symmetric group of degree . 
In 1935 Piccard~\cite{Pic35} proved that three transformations of  are sufficient to generate the monoid . In the same year, Eilenberg showed that fewer than three generators are not possible, as reported by Sierpi\'nski~\cite{Sie35}. We refer the reader to the book of Ganyushkin and Mazorchuk~\cite{GaMa09} for a detailed discussion of finite transformation semigroups. The following are well-known facts about generators of  and : 

\begin{theorem}[Permutations,~\cite{Hoy1895,Pic38}]
\label{thm:piccard}
The symmetric group  of size  can be generated by any cyclic
permutation of  elements together with any transposition. In particular,  can be generated by
~and~.
\end{theorem}

\begin{theorem}[Transformations,~\cite{Pic35}]
\label{thm:salomaa}
The complete transformation monoid  of size  can be generated by any cyclic
permutation of  elements together with a transposition and a ``returning'' transformation . In particular,  can be generated by ,   and .
\end{theorem}


\section{Quotient Complexity and Syntactic Complexity}\label{sec:complexity} 

If  is a non-empty finite alphabet, then  is the free monoid generated by , and  is the free semigroup generated by .  A \emph{word} is any element of , and the empty word is . The length of a word  is . A \emph{language} over  is any subset of . 
If  for some , then  is a {\em prefix\/} of ,  is a {\em suffix\/} of , and  is a {\em factor\/} of . Both  and  are also factors of . 
A~{\em proper} prefix (suffix, factor) of  is a prefix (suffix, factor) of  other than~. 


The \emph{left quotient}, or simply \emph{quotient,} of a language  by a word  is  the language . 
For any  , the \emph{Nerode right congruence}~\cite{Ner58}  of  is defined as follows: 

Clearly,  if and only if .
Thus each equivalence class of this right congruence corresponds to a distinct quotient of .


The \emph{Myhill congruence}~\cite{Myh57}  of  is defined as follows:

This congruence is also known as the \emph{syntactic congruence} of .
The quotient set  of equivalence classes of the relation  is a semigroup called the \emph{syntactic semigroup} of , and 
 is the \emph{syntactic monoid} of~. 
The \emph{syntactic complexity}  of  is the cardinality of its syntactic semigroup.
The \emph{monoid complexity}  of  is the cardinality of its syntactic monoid.
If the  equivalence class containing  is a singleton in the syntactic monoid, then ; otherwise, .

A~\emph{deterministic finite automaton} (DFA) is a quintuple , where 
 is a finite, non-empty set of \emph{states},  is a finite non-empty \emph{alphabet},  is the \emph{transition function},  is the \emph{initial state}, and  is the set of \emph{accepting states}. We extend  to  in the usual way.
The DFA  accepts a word  if . 
The set of all words {\it accepted} by  is . 
By the \emph{language of a state}  of  
we mean the language accepted 
by the DFA . 
A state is \emph{empty} if its language is empty.

Let  be a regular language. 
The \emph{quotient DFA} of  is 
, where , , 
,  .
The number   of distinct quotients of  is the \emph{quotient complexity} of . 
The quotient DFA of  is the minimal DFA accepting , and so quotient complexity is the same as  state complexity, but there are advantages to using quotients~\cite{Brz09}.


In terms of automata, each equivalence class  of  is the set of all words  that take the automaton to the same state from the initial state, and each equivalence class  of  is the set of all words that perform the same transformation on the set of states~\cite{McNP71}.
In terms of quotients,  is the set of words  that can be followed by the same quotient . 

Let  be a DFA. For each word , the transition function for  defines a transformation  of  by the word : for all , 
 
The set  of all such transformations by non-empty words forms a subsemigroup of , called the \emph{transition semigroup} of ~\cite{Pin97}. 
Conversely, we can use a set   of transformations to define , and so the DFA . When the context is clear we simply write , where  is a transformation of , to mean that the transformation performed by  is~.

If   is the quotient DFA of , then  is isomorphic to the syntactic semigroup  of ~\cite{McNP71}, and we represent elements of  by transformations in~. 

We attempt to obtain tight upper bounds on the syntactic complexity  of  as a function of the quotient complexity  of .
First we consider the syntactic complexity of regular languages over a unary alphabet, where the  concepts prefix-, suffix-, bifix-, and factor-free, coincide. So we may consider only unary prefix-free regular languages  with quotient complexity . When , the only prefix-free language is  with . For , a prefix-free language  must be a singleton, . The syntactic semigroup  of  consists of  transformations  by words , where . Thus we have 

\begin{proposition}[Unary Free Regular Languages]
If  is a unary free regular language with , then .
\end{proposition}


The tight upper bound for regular unary languages~\cite{HoKo04} is . 

We assume that  in the following sections. 
Since the syntactic semigroup of a language is the same as that of its complement, we deal only with prefix-, suffix-, bifix-, and factor-free languages. All the syntactic complexity results, however, apply also to the complements of these languages.
\goodbreak

\section{Prefix-Free Regular Languages}\label{sec:pf}
To simplify notation we write  for the language . Recall that a regular language  is prefix-free if and only it has exactly one accepting quotient, and that quotient is ~\cite{HSW09}.

\begin{theorem}[Prefix-Free Regular Languages]
\label{thm:prefix-free}
If  is regular and prefix-free with , then . Moreover, this bound is tight 
for   if , for   if , for  if , and for 
if .
\end{theorem}
\begin{proof}
If  is prefix-free, the only accepting quotient of  is . Thus  also has the empty quotient, since  for  . 
Let  be the quotient DFA of , where, without loss of generality,   is the only accepting state, and   is the empty state. For any transformation , . Thus we have .

The only prefix-free regular language for  is  with ; here the bound  does not apply. 
For  and , the language  meets the bound.
For  and ,   meets the bound.
For , let , where , , , and  for .
DFA  is shown in Fig.~\ref{fig:PrFree}, 
where .
For ,  input  coincides with ; hence only  inputs are needed. 

\begin{figure}[hbt]
\begin{center}
\setlength{\unitlength}{0.00056868in}
\begingroup\makeatletter\ifx\SetFigFont\undefined \gdef\SetFigFont#1#2#3#4#5{\reset@font\fontsize{#1}{#2pt}\fontfamily{#3}\fontseries{#4}\fontshape{#5}\selectfont}\fi\endgroup {\renewcommand{\dashlinestretch}{30}
\begin{picture}(4890,2694)(0,-10)
\put(3207,78){\makebox(0,0)[b]{\smash{{\SetFigFont{8}{9.6}{\familydefault}{\mddefault}{\updefault}}}}}
\put(507,1270){\makebox(0,0)[b]{\smash{{\SetFigFont{8}{9.6}{\familydefault}{\mddefault}{\updefault}1}}}}
\put(1866,1338){\ellipse{382}{382}}
\put(1880,1270){\makebox(0,0)[b]{\smash{{\SetFigFont{8}{9.6}{\familydefault}{\mddefault}{\updefault}2}}}}
\put(3296,1343){\ellipse{382}{382}}
\put(3297,1270){\makebox(0,0)[b]{\smash{{\SetFigFont{8}{9.6}{\familydefault}{\mddefault}{\updefault}3}}}}
\put(4691,1343){\ellipse{382}{382}}
\put(4692,1270){\makebox(0,0)[b]{\smash{{\SetFigFont{8}{9.6}{\familydefault}{\mddefault}{\updefault}4}}}}
\put(2564,258){\ellipse{382}{382}}
\put(2564,258){\ellipse{324}{324}}
\put(2564,190){\makebox(0,0)[b]{\smash{{\SetFigFont{8}{9.6}{\familydefault}{\mddefault}{\updefault}5}}}}
\put(3957,259){\ellipse{382}{382}}
\put(3959,190){\makebox(0,0)[b]{\smash{{\SetFigFont{8}{9.6}{\familydefault}{\mddefault}{\updefault}6}}}}
\put(4285.677,317.559){\arc{274.198}{3.3655}{8.5574}}
\blacken\path(4320.693,213.226)(4197.000,213.000)(4306.247,154.991)(4320.693,213.226)
\put(4512,393){\makebox(0,0)[b]{\smash{{\SetFigFont{8}{9.6}{\familydefault}{\mddefault}{\updefault}}}}}
\put(1182.000,66.750){\arc{2812.500}{4.3527}{5.0721}}
\blacken\path(1553.620,1391.800)(1677.000,1383.000)(1572.277,1448.826)(1553.620,1391.800)
\put(1182.000,2609.250){\arc{2812.500}{1.2111}{1.9305}}
\blacken\path(810.380,1284.200)(687.000,1293.000)(791.723,1227.174)(810.380,1284.200)
\put(2622.000,-31.286){\arc{4901.878}{3.8024}{5.6224}}
\blacken\path(739.587,1584.958)(687.000,1473.000)(786.087,1547.039)(739.587,1584.958)
\path(3499,1338)(4480,1338)
\blacken\path(4360.000,1308.000)(4480.000,1338.000)(4360.000,1368.000)(4360.000,1308.000)
\path(2082,1338)(3089,1338)
\blacken\path(2969.000,1308.000)(3089.000,1338.000)(2969.000,1368.000)(2969.000,1308.000)
\path(642,1203)(2397,348)
\blacken\path(2275.982,373.587)(2397.000,348.000)(2302.260,427.526)(2275.982,373.587)
\path(1947,1158)(2442,438)
\blacken\path(2349.295,519.889)(2442.000,438.000)(2398.738,553.881)(2349.295,519.889)
\path(3207,1158)(2712,438)
\blacken\path(2755.262,553.881)(2712.000,438.000)(2804.705,519.889)(2755.262,553.881)
\path(4557,1203)(2757,348)
\blacken\path(2852.522,426.585)(2757.000,348.000)(2878.265,372.388)(2852.522,426.585)
\path(12,1338)(313,1338)
\blacken\path(193.000,1308.000)(313.000,1338.000)(193.000,1368.000)(193.000,1308.000)
\path(2757,258)(3747,258)
\blacken\path(3627.000,228.000)(3747.000,258.000)(3627.000,288.000)(3627.000,228.000)
\path(417,1518)(416,1521)(413,1527)
	(409,1537)(403,1551)(396,1568)
	(389,1587)(382,1607)(376,1628)
	(371,1650)(367,1673)(366,1696)
	(367,1720)(372,1743)(379,1761)
	(387,1777)(395,1789)(402,1798)
	(409,1804)(414,1809)(420,1812)
	(425,1814)(430,1816)(436,1818)
	(443,1820)(451,1823)(462,1826)
	(475,1829)(490,1832)(507,1833)
	(524,1832)(539,1829)(552,1826)
	(563,1823)(571,1820)(578,1818)
	(584,1816)(590,1814)(594,1812)
	(600,1809)(605,1804)(612,1798)
	(619,1789)(627,1777)(635,1761)
	(642,1743)(647,1720)(648,1696)
	(647,1673)(643,1650)(638,1628)
	(632,1607)(625,1587)(618,1568)
	(611,1551)(597,1518)
\blacken\path(616.249,1640.186)(597.000,1518.000)(671.483,1616.753)(616.249,1640.186)
\path(3207,1518)(3206,1521)(3203,1527)
	(3199,1537)(3193,1551)(3186,1568)
	(3179,1587)(3172,1607)(3166,1628)
	(3161,1650)(3157,1673)(3156,1696)
	(3157,1720)(3162,1743)(3169,1761)
	(3177,1777)(3185,1789)(3192,1798)
	(3199,1804)(3204,1809)(3210,1812)
	(3215,1814)(3220,1816)(3226,1818)
	(3233,1820)(3241,1823)(3252,1826)
	(3265,1829)(3280,1832)(3297,1833)
	(3314,1832)(3329,1829)(3342,1826)
	(3353,1823)(3361,1820)(3368,1818)
	(3374,1816)(3380,1814)(3384,1812)
	(3390,1809)(3395,1804)(3402,1798)
	(3409,1789)(3417,1777)(3425,1761)
	(3432,1743)(3437,1720)(3438,1696)
	(3437,1673)(3433,1650)(3428,1628)
	(3422,1607)(3415,1587)(3408,1568)
	(3401,1551)(3387,1518)
\blacken\path(3406.249,1640.186)(3387.000,1518.000)(3461.483,1616.753)(3406.249,1640.186)
\path(1767,1518)(1766,1521)(1763,1527)
	(1759,1537)(1753,1551)(1746,1568)
	(1739,1587)(1732,1607)(1726,1628)
	(1721,1650)(1717,1673)(1716,1696)
	(1717,1720)(1722,1743)(1729,1761)
	(1737,1777)(1745,1789)(1752,1798)
	(1759,1804)(1764,1809)(1770,1812)
	(1775,1814)(1780,1816)(1786,1818)
	(1793,1820)(1801,1823)(1812,1826)
	(1825,1829)(1840,1832)(1857,1833)
	(1874,1832)(1889,1829)(1902,1826)
	(1913,1823)(1921,1820)(1928,1818)
	(1934,1816)(1940,1814)(1944,1812)
	(1950,1809)(1955,1804)(1962,1798)
	(1969,1789)(1977,1777)(1985,1761)
	(1992,1743)(1997,1720)(1998,1696)
	(1997,1673)(1993,1650)(1988,1628)
	(1982,1607)(1975,1587)(1968,1568)
	(1961,1551)(1947,1518)
\blacken\path(1966.249,1640.186)(1947.000,1518.000)(2021.483,1616.753)(1966.249,1640.186)
\path(4602,1518)(4601,1521)(4598,1527)
	(4594,1537)(4588,1551)(4581,1568)
	(4574,1587)(4567,1607)(4561,1628)
	(4556,1650)(4552,1673)(4551,1696)
	(4552,1720)(4557,1743)(4564,1761)
	(4572,1777)(4580,1789)(4587,1798)
	(4594,1804)(4599,1809)(4605,1812)
	(4610,1814)(4615,1816)(4621,1818)
	(4628,1820)(4636,1823)(4647,1826)
	(4660,1829)(4675,1832)(4692,1833)
	(4709,1832)(4724,1829)(4737,1826)
	(4748,1823)(4756,1820)(4763,1818)
	(4769,1816)(4775,1814)(4779,1812)
	(4785,1809)(4790,1804)(4797,1798)
	(4804,1789)(4812,1777)(4820,1761)
	(4827,1743)(4832,1720)(4833,1696)
	(4832,1673)(4828,1650)(4823,1628)
	(4817,1607)(4810,1587)(4803,1568)
	(4796,1551)(4782,1518)
\blacken\path(4801.249,1640.186)(4782.000,1518.000)(4856.483,1616.753)(4801.249,1640.186)
\put(1857,1968){\makebox(0,0)[b]{\smash{{\SetFigFont{8}{9.6}{\familydefault}{\mddefault}{\updefault}}}}}
\put(2622,2508){\makebox(0,0)[b]{\smash{{\SetFigFont{8}{9.6}{\familydefault}{\mddefault}{\updefault}}}}}
\put(1182,1518){\makebox(0,0)[b]{\smash{{\SetFigFont{8}{9.6}{\familydefault}{\mddefault}{\updefault}}}}}
\put(1182,1023){\makebox(0,0)[b]{\smash{{\SetFigFont{8}{9.6}{\familydefault}{\mddefault}{\updefault}}}}}
\put(2532,1428){\makebox(0,0)[b]{\smash{{\SetFigFont{8}{9.6}{\familydefault}{\mddefault}{\updefault}}}}}
\put(3927,1428){\makebox(0,0)[b]{\smash{{\SetFigFont{8}{9.6}{\familydefault}{\mddefault}{\updefault}}}}}
\put(3207,888){\makebox(0,0)[b]{\smash{{\SetFigFont{8}{9.6}{\familydefault}{\mddefault}{\updefault}}}}}
\put(1317,663){\makebox(0,0)[b]{\smash{{\SetFigFont{8}{9.6}{\familydefault}{\mddefault}{\updefault}}}}}
\put(3837,663){\makebox(0,0)[b]{\smash{{\SetFigFont{8}{9.6}{\familydefault}{\mddefault}{\updefault}}}}}
\put(3297,1968){\makebox(0,0)[b]{\smash{{\SetFigFont{8}{9.6}{\familydefault}{\mddefault}{\updefault}}}}}
\put(507,1968){\makebox(0,0)[b]{\smash{{\SetFigFont{8}{9.6}{\familydefault}{\mddefault}{\updefault}}}}}
\put(4737,1968){\makebox(0,0)[b]{\smash{{\SetFigFont{8}{9.6}{\familydefault}{\mddefault}{\updefault}}}}}
\put(2307,888){\makebox(0,0)[b]{\smash{{\SetFigFont{8}{9.6}{\familydefault}{\mddefault}{\updefault}}}}}
\put(507,1338){\ellipse{382}{382}}
\end{picture}
}
 \end{center}
\caption{Quotient DFA   of prefix-free regular language with 1,296 transformations.}
\label{fig:PrFree}
\end{figure}

Any transformation  has the form 

where  for .
There are three cases: 
\be
\item If  for all , , then by Theorem~\ref{thm:salomaa},  can do .\\
\item If  for all , , and there exists some  such that , then there exists some ,  such that  for all , .
For all , define  as follows:  if , and  if . 
Let 

By Case 1 above,   can do .
Since ,   can do  as well.\\
\item Otherwise, there exists some  such that . Then there exists some , , such that  for all , .
For all , define   as follows:  if ,  if , and  otherwise.
Let  be as above but with new .
By Case 2 above,   can do .
Since ,   can do  as well.
\ee

Therefore, the syntactic complexity of  meets the desired bound. \qed
\end{proof}

We conjecture that the alphabet sizes cannot be reduced. As shown in Table~\ref{tab:Summary1}, 
on p.~\pageref{table1}, 
we have verified this conjecture for  by enumerating all prefix-free regular languages with  using \emph{GAP}~\cite{GAP}.
\medskip


\section{Suffix-Free Regular Languages}\label{sec:sf}


For any regular language , a quotient  is \emph{uniquely reachable}~\cite{Brz09} if  implies that . 
It is known from~\cite{HS09} that, if  is a suffix-free regular language, then   is uniquely reachable by , and  has the empty quotient. 
Without loss of generality,  we assume that  is the initial state, and  is the empty state. 
We will show that the cardinality of , defined below, is an upper bound ( for ``bound'') on the syntactic complexity of suffix-free regular languages with quotient complexity . Let
 


\begin{proposition}
\label{prop:sf}
If  is a regular language with quotient DFA  and syntactic semigroup , then the following hold:
\be
\item If  is suffix-free, then  is a subset of .
\item If  has the empty quotient, only one accepting quotient, and , then  is suffix-free.
\ee
\end{proposition}

\begin{proof}
1. Let  be suffix-free, and let  be its quotient DFA. 
Consider an arbitrary . Since the quotient  is uniquely reachable,  for all . Since the quotient corresponding to state  is empty, . 
Since  is suffix-free, for any two quotients  and , where ,  for some , and , we must have , and so . 
This means that, for any  and , if , then  for all , . So , and .


2. Assume that , and let  be the only accepting state. If  is not suffix-free, then there exist non-empty words  and  such that . Let  and  be the transformations by  and , and let ; then . 
Assume without loss the generality that  is the empty state. Then , and we have , which contradicts the fact that . Therefore  is suffix-free. \qed
\end{proof}

Let . We now prove that  is an upper bound  on the syntactic complexity of suffix-free regular languages. 

With each transformation  of , we associate a directed graph , where  is the set of nodes, and  is a directed edge from  to  if . We call such a graph  the {\em transition graph} of . For each node , there is exactly one edge leaving  in . Consider the infinite sequence  for any . Since  is finite, there exists least  such that  for some . Then the finite sequence  contains all the distinct elements of the above infinite sequence, and it induces a directed path  from  to  in . In particular, if , and , then we call  the {\em principal sequence} of , and , the {\em principal path} of . 
\begin{proposition}\label{prop:ppsf} 
There exists a principal sequence for every transformation~ in~. 
\end{proposition}

\begin{proof} 
Suppose  and . 
If  does not have a principal sequence, 
then , and  for some .
Let ; then  and , violating the last property of .
Therefore there is a principal sequence for every . \qed
\end{proof}

Fix a transformation . Let  be such that . If the sequence  does not contain any element of the principal sequence  other than , then we say that  has {\em no principal connection}. Otherwise, there exists least  such that  and  for some , and we say that  has a {\em principal connection} at . If , the  principal connection is {\em short}; otherwise, it is {\em long}. 

\begin{lemma}\label{lem:shortpath} 
For all  and , the sequence  has no long principal connection.
\end{lemma}

\begin{proof} 
Let  be any transformation in . Suppose for some , the sequence  has a long principal connection at , where . Hence , and , which is a contradiction. Therefore, for all ,  has no long principal connection. \qed
\end{proof}


To calculate the cardinality of , we need the following observation. 
\begin{lemma}\label{lem:ptree} 

For all  and , if  has a principal connection, then there is no cycle incident to the path  in the transition graph . 

\end{lemma}

\begin{proof} 
This observation can be derived from Theorem 1.2.9 of~\cite{GaMa09}. However, our proof is shorter. 
Pick any  such that  has a principal connection at  for some  and . Then the sequence  contains , and the path  does not contain any cycle. Suppose  is a cycle which includes node 
. 
Since there is only one outgoing edge for each node in , the cycle  must be oriented and must contain a node  such that  is an edge in .
Then the next node in the cycle must be  since there is only one outgoing edge from . But then   can never be reached from , and so no such cycle can exist. \qed
\end{proof}


By Lemma~\ref{lem:ptree}, for any , where , the union of directed paths from various nodes  to , if  and  has a principal connection at , forms a labeled tree  rooted at . Suppose there are  nodes in  for each , and suppose there are  elements of  that are not in the principal sequence  nor in any tree , for some . Note that,  is the only node in  that is also in the principal sequence . Each tree  has height at most ; otherwise, some  has a long principal connection. In particular, tree  has height 1; so it is trivial with only one node . Then , and we need only consider trees  for . Let  be the number of labeled rooted trees with  nodes and height at most . This number can be found in the paper of Riordan~\cite{Rio60}; the calculation is somewhat complex, and we refer the reader to~\cite{Rio60} for details. For convenience, we include the values of  for small values of  and  in Table~\ref{tab:Smh}, where the row number is  and the column number is . 


\begin{table}[ht]
\caption{The number  of labeled rooted trees with  nodes and height at most~.}
\label{tab:Smh}
\begin{center}

\end{center}
\label{table0}
\end{table}

Since each of the  nodes can be the root, there are  labeled trees rooted at a fixed node and having  nodes and height at most . The following is an example of trees  in transformations . 

\begin{example}\label{ex:ptree} 
Let . Consider any transformation  with principal sequence . There are  elements of  that are not in , and some of them are in the trees  for . Consider the cases where , , , and . Fig.~\ref{fig:ptree} shows one such transformation . 

\begin{figure}[hbt]
\begin{center}
\setlength{\unitlength}{0.00052493in}
\begingroup\makeatletter\ifx\SetFigFont\undefined \gdef\SetFigFont#1#2#3#4#5{\reset@font\fontsize{#1}{#2pt}\fontfamily{#3}\fontseries{#4}\fontshape{#5}\selectfont}\fi\endgroup {\renewcommand{\dashlinestretch}{30}
\begin{picture}(6231,2507)(0,-10)
\put(462,2033){\ellipse{450}{450}}
\put(5856.375,1583.000){\arc{731.250}{5.2449}{7.3215}}
\blacken\path(6114.282,1368.376)(6042.000,1268.000)(6153.014,1322.552)(6114.282,1368.376)
\put(4962.000,2352.500){\arc{261.000}{2.3318}{7.0930}}
\blacken\path(5061.910,2381.296)(5052.000,2258.000)(5118.765,2362.127)(5061.910,2381.296)
\put(2262,2033){\ellipse{450}{450}}
\put(1362,2033){\ellipse{450}{450}}
\put(4062,2033){\ellipse{450}{450}}
\put(3162,2033){\ellipse{450}{450}}
\put(2262,1133){\ellipse{450}{450}}
\put(2712,233){\ellipse{450}{450}}
\put(3612,233){\ellipse{450}{450}}
\put(4062,1133){\ellipse{450}{450}}
\put(4962,2033){\ellipse{450}{450}}
\put(5862,233){\ellipse{450}{450}}
\put(1362,1133){\ellipse{450}{450}}
\put(3162,1133){\ellipse{450}{450}}
\put(5862,2033){\ellipse{450}{450}}
\put(5862,1133){\ellipse{450}{450}}
\path(687,2033)(1137,2033)
\blacken\path(1017.000,2003.000)(1137.000,2033.000)(1017.000,2063.000)(1017.000,2003.000)
\path(1587,2033)(2037,2033)
\blacken\path(1917.000,2003.000)(2037.000,2033.000)(1917.000,2063.000)(1917.000,2003.000)
\path(12,2033)(237,2033)
\blacken\path(117.000,2003.000)(237.000,2033.000)(117.000,2063.000)(117.000,2003.000)
\path(2487,2033)(2937,2033)
\blacken\path(2817.000,2003.000)(2937.000,2033.000)(2817.000,2063.000)(2817.000,2003.000)
\path(3387,2033)(3837,2033)
\blacken\path(3717.000,2003.000)(3837.000,2033.000)(3717.000,2063.000)(3717.000,2003.000)
\path(2262,1358)(2262,1808)
\blacken\path(2292.000,1688.000)(2262.000,1808.000)(2232.000,1688.000)(2292.000,1688.000)
\path(1497,1313)(2082,1853)
\blacken\path(2014.172,1749.562)(2082.000,1853.000)(1973.475,1793.650)(2014.172,1749.562)
\path(3162,1358)(3162,1808)
\blacken\path(3192.000,1688.000)(3162.000,1808.000)(3132.000,1688.000)(3192.000,1688.000)
\path(2847,413)(3117,908)
\blacken\path(3085.875,788.287)(3117.000,908.000)(3033.201,817.018)(3085.875,788.287)
\path(3477,413)(3207,908)
\blacken\path(3290.799,817.018)(3207.000,908.000)(3238.125,788.287)(3290.799,817.018)
\path(4062,1358)(4062,1808)
\blacken\path(4092.000,1688.000)(4062.000,1808.000)(4032.000,1688.000)(4092.000,1688.000)
\path(5862,458)(5862,908)
\blacken\path(5892.000,788.000)(5862.000,908.000)(5832.000,788.000)(5892.000,788.000)
\path(5862,1358)(5862,1808)
\blacken\path(5892.000,1688.000)(5862.000,1808.000)(5832.000,1688.000)(5892.000,1688.000)
\path(4287,2033)(4737,2033)
\blacken\path(4617.000,2003.000)(4737.000,2033.000)(4617.000,2063.000)(4617.000,2003.000)
\put(462,1975){\makebox(0,0)[b]{\smash{{\SetFigFont{7}{8.4}{\familydefault}{\mddefault}{\updefault}}}}}
\put(1362,1975){\makebox(0,0)[b]{\smash{{\SetFigFont{7}{8.4}{\familydefault}{\mddefault}{\updefault}}}}}
\put(2262,1975){\makebox(0,0)[b]{\smash{{\SetFigFont{7}{8.4}{\familydefault}{\mddefault}{\updefault}}}}}
\put(3162,1975){\makebox(0,0)[b]{\smash{{\SetFigFont{7}{8.4}{\familydefault}{\mddefault}{\updefault}}}}}
\put(4062,1975){\makebox(0,0)[b]{\smash{{\SetFigFont{7}{8.4}{\familydefault}{\mddefault}{\updefault}}}}}
\put(1362,1075){\makebox(0,0)[b]{\smash{{\SetFigFont{7}{8.4}{\familydefault}{\mddefault}{\updefault}}}}}
\put(2262,1075){\makebox(0,0)[b]{\smash{{\SetFigFont{7}{8.4}{\familydefault}{\mddefault}{\updefault}}}}}
\put(3162,1075){\makebox(0,0)[b]{\smash{{\SetFigFont{7}{8.4}{\familydefault}{\mddefault}{\updefault}}}}}
\put(2712,175){\makebox(0,0)[b]{\smash{{\SetFigFont{7}{8.4}{\familydefault}{\mddefault}{\updefault}}}}}
\put(3612,175){\makebox(0,0)[b]{\smash{{\SetFigFont{7}{8.4}{\familydefault}{\mddefault}{\updefault}}}}}
\put(4062,1075){\makebox(0,0)[b]{\smash{{\SetFigFont{7}{8.4}{\familydefault}{\mddefault}{\updefault}}}}}
\put(5862,175){\makebox(0,0)[b]{\smash{{\SetFigFont{7}{8.4}{\familydefault}{\mddefault}{\updefault}}}}}
\put(5862,1075){\makebox(0,0)[b]{\smash{{\SetFigFont{7}{8.4}{\familydefault}{\mddefault}{\updefault}}}}}
\put(5862,1975){\makebox(0,0)[b]{\smash{{\SetFigFont{7}{8.4}{\familydefault}{\mddefault}{\updefault}}}}}
\put(4962,1975){\makebox(0,0)[b]{\smash{{\SetFigFont{7}{8.4}{\familydefault}{\mddefault}{\updefault}}}}}
\end{picture}
}
 \end{center}
\caption{Transition graph of some  with principal sequence .}
\label{fig:ptree}
\end{figure}


For , the tree  has height at most , and there are  possible . For , there are  possible , which are of one of the three types shown in Fig.~\ref{fig:Tj}. Among the 10 possible , one is of type (a), three are of type (b), and six are of type~(c). For , there are  possible . 

\begin{figure}[hbt]
\begin{center}
\setlength{\unitlength}{0.00052493in}
\begingroup\makeatletter\ifx\SetFigFont\undefined \gdef\SetFigFont#1#2#3#4#5{\reset@font\fontsize{#1}{#2pt}\fontfamily{#3}\fontseries{#4}\fontshape{#5}\selectfont}\fi\endgroup {\renewcommand{\dashlinestretch}{30}
\begin{picture}(5641,2697)(0,-10)
\put(2933,2449){\ellipse{450}{450}}
\put(908,2449){\ellipse{450}{450}}
\put(4958,2449){\ellipse{450}{450}}
\put(908,1549){\ellipse{450}{450}}
\put(1583,1549){\ellipse{450}{450}}
\put(2933,1549){\ellipse{450}{450}}
\put(2483,649){\ellipse{450}{450}}
\put(4508,1549){\ellipse{450}{450}}
\put(5408,1549){\ellipse{450}{450}}
\put(5408,649){\ellipse{450}{450}}
\put(233,1549){\ellipse{450}{450}}
\put(3383,649){\ellipse{450}{450}}
\path(2933,1774)(2933,2224)
\blacken\path(2963.000,2104.000)(2933.000,2224.000)(2903.000,2104.000)(2963.000,2104.000)
\path(2573,829)(2888,1324)
\blacken\path(2848.885,1206.654)(2888.000,1324.000)(2798.265,1238.867)(2848.885,1206.654)
\path(3293,829)(2978,1324)
\blacken\path(3067.735,1238.867)(2978.000,1324.000)(3017.115,1206.654)(3067.735,1238.867)
\path(908,1774)(908,2224)
\blacken\path(938.000,2104.000)(908.000,2224.000)(878.000,2104.000)(938.000,2104.000)
\path(368,1729)(773,2224)
\blacken\path(720.230,2112.128)(773.000,2224.000)(673.793,2150.122)(720.230,2112.128)
\path(1448,1729)(1043,2224)
\blacken\path(1142.207,2150.122)(1043.000,2224.000)(1095.770,2112.128)(1142.207,2150.122)
\path(4598,1774)(4868,2224)
\blacken\path(4831.985,2105.666)(4868.000,2224.000)(4780.536,2136.536)(4831.985,2105.666)
\path(5318,1774)(5048,2224)
\blacken\path(5135.464,2136.536)(5048.000,2224.000)(5084.015,2105.666)(5135.464,2136.536)
\path(5408,874)(5408,1324)
\blacken\path(5438.000,1204.000)(5408.000,1324.000)(5378.000,1204.000)(5438.000,1204.000)
\put(2933,2391){\makebox(0,0)[b]{\smash{{\SetFigFont{7}{8.4}{\familydefault}{\mddefault}{\updefault}}}}}
\put(908,2391){\makebox(0,0)[b]{\smash{{\SetFigFont{7}{8.4}{\familydefault}{\mddefault}{\updefault}}}}}
\put(4958,2391){\makebox(0,0)[b]{\smash{{\SetFigFont{7}{8.4}{\familydefault}{\mddefault}{\updefault}}}}}
\put(908,64){\makebox(0,0)[b]{\smash{{\SetFigFont{7}{8.4}{\familydefault}{\mddefault}{\updefault}(a)}}}}
\put(2933,64){\makebox(0,0)[b]{\smash{{\SetFigFont{7}{8.4}{\familydefault}{\mddefault}{\updefault}(b)}}}}
\put(4958,64){\makebox(0,0)[b]{\smash{{\SetFigFont{7}{8.4}{\familydefault}{\mddefault}{\updefault}(c)}}}}
\put(233,1491){\makebox(0,0)[b]{\smash{{\SetFigFont{7}{8.4}{\familydefault}{\mddefault}{\updefault}}}}}
\put(908,1491){\makebox(0,0)[b]{\smash{{\SetFigFont{7}{8.4}{\familydefault}{\mddefault}{\updefault}}}}}
\put(1583,1491){\makebox(0,0)[b]{\smash{{\SetFigFont{7}{8.4}{\familydefault}{\mddefault}{\updefault}}}}}
\put(2933,1491){\makebox(0,0)[b]{\smash{{\SetFigFont{7}{8.4}{\familydefault}{\mddefault}{\updefault}}}}}
\put(2483,591){\makebox(0,0)[b]{\smash{{\SetFigFont{7}{8.4}{\familydefault}{\mddefault}{\updefault}}}}}
\put(3383,591){\makebox(0,0)[b]{\smash{{\SetFigFont{7}{8.4}{\familydefault}{\mddefault}{\updefault}}}}}
\put(4508,1491){\makebox(0,0)[b]{\smash{{\SetFigFont{7}{8.4}{\familydefault}{\mddefault}{\updefault}}}}}
\put(5408,1491){\makebox(0,0)[b]{\smash{{\SetFigFont{7}{8.4}{\familydefault}{\mddefault}{\updefault}}}}}
\put(5408,591){\makebox(0,0)[b]{\smash{{\SetFigFont{7}{8.4}{\familydefault}{\mddefault}{\updefault}}}}}
\end{picture}
}
 \end{center}
\caption{Three types of trees of the form , where .}
\label{fig:Tj}
\end{figure}
\end{example}

Let  be the binomial coefficient, and let  be the multinomial coefficient. Then we have
\begin{lemma}\label{lem:Gn} 
For , we have 

\end{lemma}

\begin{proof} 
Let  be any transformation in . Suppose  for some , . There are  different principal sequences . Now, fix . Suppose , where, for , tree  contains  nodes, for some . There are  different tuples . Each tree  has height at most , and it is rooted at . There are  different trees . Let  be the set of the remaining  elements  of  that are not in any tree  nor in the principal sequence . The image  can only be chosen from . There are  different mappings of . Altogether we have the desired formula. \qed
\end{proof}


From Proposition~\ref{prop:sf} and Lemma~\ref{lem:Gn} we have 

\begin{proposition}\label{prop:Gncard} 
For , if  is a suffix-free regular language with quotient complexity , then its syntactic complexity  satisfies that , where  is the cardinality of , and it is given by Equation~(\ref{eq:g}). 
\end{proposition}

Note that  is not a semigroup for  because , , but . Hence, although  is an upper bound on the syntactic complexity of suffix-free regular languages, that bound is not tight. Our objective is to find the largest subset of  that is a semigroup. Let

where  stands for ``witness''.

\begin{proposition}\label{prop:Pncard}
For ,  is a semigroup contained in , and its cardinality is

\end{proposition}

\begin{proof}

We know that any  is in  if and only if the following hold: 
\be
\item  for all , and ; 
\item for all  , such that ,  either  or . 
\ee

Clearly . For any transformations , consider the composition . Since , we have . 
We also have . Pick any  such that .  
Suppose  or . 
If , then  and thus , a contradiction. Hence , and  is a semigroup contained in . 

Let  be any transformation. 
Note that  is fixed. 
Let , and . Suppose  elements in  are mapped to  by , where ; then there are  choices of these elements. For the set  of the remaining  elements, which must be mapped by  to pairwise distinct elements of , there are  choices for the mapping . When , there is no such  since . 
Altogether, the cardinality of  is 
   \qed
\end{proof}


We now construct a generating set  ( for ``generators'') of size  for , which will show that there exist DFA's accepting suffix-free regular languages with quotient complexity  and syntactic complexity .


\begin{proposition}\label{prop:Pgen}
When , the semigroup  is generated by the following set  of transformations of :
, where  and ; , where , , ; and for , , where


\bi
\item ,
\item ,
\item For ,  for , , and  for .
\ei

\end{proposition}


\begin{proof}
First note that  is a subset of , and so , the semigroup generated by , is a subset of . We now show that . 

Pick any  in . Note that  is fixed. 
Let , ,  , and . Then , and , since .  We prove by induction on  that . 

First, note that , the semigroup generated by , is isomorphic to the symmetric group  by Theorem~\ref{thm:piccard}. Consider  for some . Then . Moreover, since , there exists  such that  for all . Then .

Assume that any transformation  with  can be generated by , where . 
Consider  with . 
Suppose . 
Let  be such that . By assumption,  can be generated by . 
Let ; 
then , and  for all . 
Moreover, we have .
Thus, there exists  such that, 
for all , . 
Altogether, for all , we have , for all , , and . Thus , and .


Therefore . \qed
\end{proof}

\begin{theorem}
\label{thm:DFAsf}
For , let  be the DFA with alphabet , where each  defines a transformation as in Proposition~\ref{prop:Pgen}, and . 
Then  has quotient complexity , and syntactic complexity . Moreover,  is suffix-free.
\end{theorem}

\begin{proof}
First we show that all the states of  are reachable:  is the initial state,  state  is reached by , and for , state  is reached by . 
Also, the initial state  accepts  while state  rejects  for all . 
For ,  state  accepts , while state  rejects it, for all . Also  is the empty state. Thus all the states of  are distinct, and .


By Proposition~\ref{prop:Pgen}, the syntactic semigroup of  is . The syntactic complexity of  is . Also, by Proposition~\ref{prop:sf},  is suffix-free. \qed
\end{proof}

As shown in Table~\ref{tab:Summary1} on p.~\pageref{table1}, the size of  cannot be decreased for . 


\begin{theorem}\label{thm:sfsmall} 
For , if a suffix-free regular language  has quotient complexity , then its syntactic complexity satisfies that , and this is a tight upper bound. 
\end{theorem}

\begin{proof} 
By Proposition~\ref{prop:sf}, the syntactic semigroup of a suffix-free regular language  is contained in . 
For , . 
So  is an upper bound, and it is met by the language  for  and by  for . 
For , we have  and . 
Two transformations,  and , in  are such that  conflicts with  (), and  conflicts with  (). 
Thus . 
Let ; then  and . So the bound is tight.

For , we have  and . Let . For each , we enumerated transformations in  using \emph{GAP} and found a unique  such that the semigroup  is not contained in . Thus at most one transformation in each pair  can appear in the syntactic semigroup of . So we reduce the upper bound to~. By Theorem~\ref{thm:DFAsf}, this bound is tight. 
\end{proof}
\smallskip


For , the semigroup  is no longer the largest semigroup contained in . In the following, we define and study another semigroup , which is a larger semigroup contained in . Let 
 
Note that, we are interested only in situations where , although some statements also hold for smaller . 


\begin{proposition}\label{prop:Wsf} 
For , the set  is a semigroup contained in , and its cardinality is 

\end{proposition}

\begin{proof} 
Pick any  in . If , then  and . If , then, for all ,  and ; so  as well. Hence  is a semigroup contained in . 

For any ,  is fixed. There are two possible cases: 
\be
\item : For each ,  can be chosen from . Then there are  different 's in this case. 
\item : Now  can be chosen from . For each ,  is fixed. There are  different 's in this case. 
\ee
Therefore . \qed
\end{proof}



\begin{proposition}\label{prop:Wsfgen} 
For , the semigroup  is generated by the set  of transformations, where 
\be 
\item , , ; 
\item For , ; 
\item . 
\ee 
\end{proposition}

\begin{proof} 
Clearly , and . We show in the following that . 

Let . By Theorem~\ref{thm:salomaa},  and  together generate the semigroup  which is isomorphic to  and is contained in . Next, consider any . We have two cases: 
\be 
\item : Let . Since , . Suppose , for some . Then there exists  such that, for all , . Let . Note that , and, for all , . So . 
\item : If , then . Otherwise, , and we know from the above case that there exists  such that . Then , and , for all . Hence . 
\ee 

Therefore . \qed
\end{proof}

\begin{theorem}\label{thm:wsfaut} 
For , let  be the DFA with alphabet  of size , where each letter defines a transformation as in Proposition~\ref{prop:Wsfgen}, and . Then  has quotient complexity  and syntactic complexity . 
\end{theorem}

\begin{proof}
First we show that . From the initial state, we can reach state  by  and state  by . From state  we can reach state , , by . So all the states in  are reachable. Now, the initial state accepts , but all other states reject it. For , state  accepts , while all other states reject it. State  is the empty state, which rejects all words. Thus all the states in  are distinct. 

By Proposition~\ref{prop:Wsfgen}, the syntactic semigroup of  is , and . Also  is suffix-free by Proposition~\ref{prop:sf}. \qed
\end{proof}


We know that the upper bound on the  syntactic complexity of suffix-free regular languages is achieved by the largest semigroup contained in . We conjecture that  is such a semigroup. 
\medskip

\begin{conjecture}[Suffix-Free Regular Languages]
\label{con:sf}
If  is a suffix-free regular language with , then  and this is a tight bound. 
\end{conjecture}



We prove the conjecture for :

\begin{proof} 
For ,   and . Let . For each , we enumerated transformations in  using \emph{GAP} and found a unique  such that  is not contained in . As in the proof of Theorem~\ref{thm:sfsmall}, for each , at most one transformation in  can appear in the syntactic semigroup of . Then we can reduce the upper bound to . This bound is met by the language  in Theorem~\ref{thm:wsfaut}; so it is tight. \qed
\end{proof}



\section{Bifix-Free Regular Languages}\label{sec:bf}


Let  be a regular bifix-free language with . From Sections~\ref{sec:pf} and~\ref{sec:sf} we have: 
\be
\item  has  as a quotient, and this is the only accepting quotient; 
\item  has  as a quotient; 
\item  as a quotient is uniquely reachable.
\ee

Let  be the quotient DFA of , with  as the set of states. We assume that  is the initial state,  corresponds to the quotient , and  is the empty state. Consider the set

The following is an observation similar to Proposition~\ref{prop:sf}.

\begin{proposition}\label{prop:bf}
If  is a regular language with quotient complexity  and syntactic semigroup , then the following hold: 
\be
\item If  is bifix-free, then  is a subset of . 
\item If  is the only accepting quotient of , and , then  is bifix-free.
\ee
\end{proposition}

\begin{proof}\mbox{}
1. Since  is suffix-free,  . Since  is also prefix-free,  it has  and  as quotients. By assumption,  corresponds to the quotient . Thus for any , , and so . 

2. Since  is the only accepting quotient of ,  is prefix-free, and  has the empty quotient. Since ,   is suffix-free by Proposition~\ref{prop:sf}. Therefore  is bifix-free. \qed
\end{proof}

\begin{lemma}\label{lem:Hn} 
For , we have , where 

\end{lemma}

\begin{proof} 
Let  be any transformation in . Suppose , where . For , suppose tree  contains  nodes, for some ; then there are  different trees . Let  be the set of elements of  that are not in any tree  nor in the principal sequence~. Then there are two cases: 

\be 

\item : Since , we must have , and . So there are  different . Let . Then there are  tuples . For any , its image  can be chosen from . Then the number of transformations  in this case is . 

\item : Then , and there are  different . Note that , and  is fixed. Let . Then there are  tuples . For any ,  can be chosen from . Thus the number of transformations  in this case is . 

\ee

Altogether we have the desired formula. \qed
\end{proof}

Let . From Proposition~\ref{prop:bf} and Lemma~\ref{lem:Hn} we have 
\begin{proposition}\label{prop:Hncard} 
For , if  is a bifix-free regular language with quotient complexity , then its syntactic complexity  satisfies that , where  is the cardinality of  as in Lemma~\ref{lem:Hn}.
\end{proposition}


For , the set  is a semigroup. But for , it is not a semigroup because ,  while . Hence  is not a tight upper bound on the syntactic complexity of bifix-free regular languages in general. We look for a large semigroup contained in  that can be the syntactic semigroup of a bifix-free regular language. Let 

(The reason for using the superscript  will be made clear in Theorem~\ref{thm:bfsmall}.) 


\begin{proposition}\label{prop:Vncard}
For ,  is a semigroup contained in  with cardinality

\end{proposition}


\begin{proof}
First, note that , and that  is a semigroup contained in  by Proposition~\ref{prop:Pncard}. For any , we have , and ; so . Then , and  is a semigroup contained in~. 

Pick any . Note that  and  are fixed, and . Let , , and . Suppose , where ; then there are  choices of . Elements of  are mapped to pairwise different elements of ; then there are  different mappings . Altogether, we have
 \qed
\end{proof}

\begin{proposition}\label{prop:bfgen}
For , let  and . Then the semigroup  is 
generated by

\end{proposition}

\begin{proof}\label{proof:bfgen}
We want to show that . Since , we have . 
Let . By definition, . Let . If , then ; 
otherwise,  there exists  such that . We prove by induction on  that . 


First note that, for all ,  is an injective mapping from  to . Consider   for some . 
Since , . Let  be defined by
\be
\item  for , ,  for , 
\item ,  for ,  for . 
\ee
Then , and .
\goodbreak

Assume that any transformation  with  can be generated by , where . 
Consider  with . 
Suppose , and let , where . 
By assumption, all  with  can be generated by . 
Let  be such that ; then . In addition, let , and let  for . Let  be such that  for , , and  for . Then , and .
Therefore, . \qed
\end{proof}

\begin{theorem}\label{thm:bfaut}
For , let  be the DFA 
with alphabet  of size , where each  defines a distinct transformation , and . 
Then  has quotient complexity , and syntactic complexity . 
Moreover,  is bifix-free.
\end{theorem}

\begin{proof}\label{proof:bfaut}
We first show that all the states of  are reachable. 
Note that there exists  such that . 
State  is the initial state, and
  reaches  state  for . 
Furthermore, for , state  accepts  , while for ,  state  rejects it. Also,  is the empty state. Thus all the states of  are distinct, and . 

By Proposition~\ref{prop:bfgen}, the syntactic semigroup of  is . Hence the syntactic complexity of  is . By Proposition~\ref{prop:bf},  is bifix-free. \qed
\end{proof}

\begin{theorem}\label{thm:bfsmall} 
For , if a bifix-free regular language  has quotient complexity , then , and this bound is tight. 
\end{theorem}

\begin{proof}
We know by Proposition~\ref{prop:bf} that the upper bound on the syntactic complexity of bifix-free regular languages is reached by the largest semigroup contained in . Since  for , , and ,  is an upper bound, and it is tight by Theorem~\ref{thm:bfaut}. 

For , we have , and . Let . We found for each  a unique  such that the semigroup  is not a subset of :


Since , if both  and  are in ,
then , and   is not bifix-free by Proposition~\ref{prop:bf}. Thus, for , at most one of  and  can appear in , and . Since  and  is a semigroup, we have  as the upper bound for . This bound is reached by the DFA  in Theorem~\ref{thm:bfaut}. \qed 
\end{proof}


For , the semigroup  is no longer the largest semigroup contained in . We find another large semigroup  suitable for bifix-free regular languages. Let 

and let . 
When , we have , and these cases were already discussed. So we are only interested in larger values~of~. 

\begin{proposition}\label{prop:Rcard} 
For ,  is a semigroup contained in  with cardinality 

\end{proposition}

\begin{proof} 

First we show that  is a semigroup. For any , since , we have . Next, let  and . If , then  and ; so . If , then  and ; so  as well. Thus  is also a semigroup. For any  and , since  for all , and , we have , and . Also , so . Hence  is a semigroup contained in . 


Note that , , and  are pairwise disjoint. For any , there are three cases: 

\be
\item : For any ,  can be chosen from . Then ; 
\item : For any ,  can be chosen from . Then ; 
\item : Now,  can be chosen from . For any ,  has two choices:  or . Then . 
\ee 
Therefore we have . \qed
\end{proof}

The next proposition describes a generating set of . 
\begin{proposition}\label{prop:Rgen} 
For , the semigroup  is generated by , where  and , and 
\be
\item , , ; 
\item For , ; 
\item Each  defines a distinct transformation in  other than ; 
\item Each  defines a distinct transformation in  other than  for all . 
\ee
\end{proposition}

\begin{proof} 
Since , we have . It remains to be shown that . Let . 
\be 
\item First consider . By Theorem~\ref{thm:salomaa},  and  together generate the semigroup  which is contained in . For any , let ; then . Suppose , where . Then there exists  such that, for all , . Let . Note that , and, for all , . So , and . 

\item Next, the transformations that are in  but not in  are , where . Note that , and, for each , . Then , and . 
\ee 

Therefore . \qed
\end{proof}


\begin{theorem}\label{thm:bfaut1} 
For , let  be the DFA with alphabet  of size , where each letter defines a transformation as in Proposition~\ref{prop:Rgen}, and . Then  has quotient complexity , and syntactic complexity . Moreover,  is bifix-free. 
\end{theorem}


\begin{proof} 
First, for all , there exists  such that , and state  is reachable by . So all the states in  are reachable. Next, there exist  such that  and . The initial state accepts , while all other states reject it. For , state  accepts , while all other states reject it. Also, state  is the only accepting state, and state  is the empty state. Then all the states in  are distinct, and . 

By Proposition~\ref{prop:Rgen}, the syntactic semigroup of  is ; so . By Proposition~\ref{prop:bf},  is bifix-free. \qed
\end{proof}


\smallskip

\begin{conjecture}[Bifix-Free Regular Languages]
\label{con:bf}
If  is a bifix-free regular language with , then  and this is a tight bound. 
\end{conjecture}

\smallskip

The conjecture holds for  as we now show:
\begin{proof} 
When ,  and . There are 126 transformations  in . For each , we enumerated transformations in  using \emph{GAP} and found a unique  such that . Thus, for each , at most one of  and  can appear in the syntactic semigroup  of . So we further lower the bound to . This bound is reached by the DFA  in Theorem~\ref{thm:bfaut1}; so it is a tight upper bound for~. \qed
\end{proof}

\section{Factor-Free Regular Languages}\label{sec:ff}

Let  be a factor-free regular language with . Since factor-free regular languages are also bifix-free,  as a quotient is uniquely reachable,  is the only accepting quotient of , and  also has the empty quotient. As in Section~\ref{sec:bf}, we assume that  is the set of states of quotient DFA of , in which  is the initial state, and states  and  correspond to the quotients ~and~, respectively. Let 

We first have the following observation: 

\begin{proposition}\label{prop:ff}
If  is a regular language with quotient complexity  and syntactic semigroup , then the following hold: 
\be 
\item If  is factor-free, then  is a subset of . 
\item If  is the only accepting quotient of , and , then  is factor-free. 
\ee
\end{proposition}

\begin{proof}
1. Assume  is factor-free. Then  is bifix-free, and  by Proposition~\ref{prop:bf}. For any transformation  performed by some non-empty word , if  for some , then . If we also have  for some , then  as  for all . Thus there exist non-empty words  and  such that state  is reachable by , and state  accepts . So , which is a contradiction. Hence . 

2. Since  is the only accepting state and ,  is bifix-free by Proposition~\ref{prop:bf}. If  is not factor-free, then there exist non-empty words  and  such that . Thus , and . Since  is bifix-free,  and ; thus , which contradicts the assumption that . Therefore  is bifix-free. \qed
\end{proof}

The properties of suffix- and bifix-free regular languages still apply to factor-free regular languages. Moreover, we have 


\begin{lemma}\label{lem:ffseq} 
For all  and , if , then . 
\end{lemma}

\begin{proof} 
Suppose  for some . If , then for all , . In particular, , which contradicts the definition of . Therefore . \qed
\end{proof}



\begin{lemma}\label{lem:Fn} 
For , we have , where 


and  as given in Equation~(\ref{eq:H2}). 
\end{lemma}

\begin{proof} 
Let  be any transformation. Suppose , where . Then there are two cases: 

\be 
\item . Since , we have , and . If , then , and  for all ; such a  is unique. Consider . There are  different . For , suppose there are  nodes in tree ; then there are  such trees. Let  be the set of elements  that are not in any tree  nor in , and let . By Lemma~\ref{lem:ffseq},  for all . Then the union of paths  for all  form a labeled tree rooted at  with height at most , and there are  such trees. Thus the number of transformations in this case is . 

\item . Now, for all , . Then . As in the proof of Lemma~\ref{lem:Hn}, the number of transformations in this case is . 

\ee

Altogether we have the desired formula. \qed
\end{proof}

Let . From Proposition~\ref{prop:ff} and Lemma~\ref{lem:Fn} we have 
\begin{proposition}\label{prop:Fncard} 
For , if  is a factor-free regular language with quotient complexity , then its syntactic complexity  satisfies that , where  is the cardinality of  as in Lemma~\ref{lem:Fn}.
\end{proposition} 

The tight upper bound on the syntactic complexity of factor-free regular languages is reached by the largest semigroup contained in . When ,  is a semigroup. The languages ,  over alphabet , and  have syntactic complexities , , and , respectively. So  is a tight upper bound for . However, the set  is not a semigroup for , because  but . 


Next, we find a large semigroup that can be the syntactic semigroup of a factor-free regular language. 

Let , and let . When , we have . So we are interested in larger values of  in the rest of this section. 

\begin{proposition}\label{prop:Uncard}
For ,  is a semigroup contained in  with cardinality 

\end{proposition}

\begin{proof}

As we have shown in the proof of Proposition~\ref{prop:Rcard},  is a semigroup. For any , since , we have ; so  is also a semigroup. We also know that, for any  and , since ,  for all ; so . If , then  and ; otherwise, , and  or . Hence  is a semigroup. 

For any , since , we have . For any , , and  for all ; then  as well. Clearly . Hence  is contained in . 

We know that  and . Therefore . \qed
\end{proof}

We now describe a generating set of . 
\begin{proposition}\label{prop:Ugen} 
For , the semigroup  is generated by , where , and 
\be
\item , , ; 
\item For , ; 
\item Each  defines a distinct transformation in  other than  for all .
\ee
\end{proposition}

\begin{proof} 
We know from the proof of Proposition~\ref{prop:Rgen} that  is generated by . Also, the transformations that are in  but not in  are , where . Each  is a composition of  and . Therefore . \qed
\end{proof}

\begin{theorem}\label{thm:ffaut} 
For , let  be the DFA with alphabet  of size , where each letter defines a transformation as in Proposition~\ref{prop:Ugen}, and . Then  has quotient complexity , and syntactic complexity . Moreover,  is factor-free. 
\end{theorem}

\begin{proof} 
Since , the DFA  can be obtained from the DFA  of Theorem~\ref{thm:bfaut1} by restricting the alphabet. The words used to show that all the states of  are reachable and distinct still exist in . Then we have . By Proposition~\ref{prop:Ugen}, the syntactic semigroup of  is ; so . By Proposition~\ref{prop:ff},  is factor-free. \qed
\end{proof}

\begin{conjecture}[Factor-Free Regular Languages]\label{con:ff}
If  is a factor-free regular language with , where , then  and this is a tight upper~bound. 
\end{conjecture}

We prove the conjecture for  and . 

\begin{proof} 
For , , and . There are 6 transformations  in . For each , , we found a unique  such that : 
 

For each , at most one of  and  can appear in the syntactic semigroup  of a factor-free regular language . Then . By Theorem~\ref{thm:ffaut}, this upper bound is tight for . 

For , , and . There are 96 transformations  in . For each , , we enumerated the transformations in  using \emph{GAP} and found a unique  such that . Thus  is a tight upper bound for . \qed
\end{proof}



\section{Quotient Complexity of the Reversal of Free Languages}\label{sec:rev}

It has been shown in~\cite{BrYe11} that for certain regular languages with maximal syntactic complexity, the reverse languages have maximal quotient complexity. This is also true for some free languages, as we now show. 

In this section we consider \emph{non-deterministic finite automata} (NFA). A NFA  is a quintuple , where , , and  are as in a DFA,  is the non-deterministic transition function, and  is the set of initial states. For any word , the \emph{reverse} of  is defined inductively as follows:  if , and  if  for some  and . The \emph{reverse} of any language  is the language . For any finite automaton (DFA or NFA) , we denote using  the automaton obtained by reversing  and exchanging the roles of initial states and accepting states, and , the DFA obtained by applying the subset construction to . Then , and . To simplify our proofs, we use an observation from~\cite{Brz62} that, for any NFA  whose states are all reachable, if the automaton  is deterministic, then the DFA  is minimal. 

\begin{theorem}\label{thm:pfrev}
The reverse of the prefix-free regular language accepted by 
the DFA  of Theorem~\ref{thm:prefix-free} restricted to 
has  quotients, which is the maximum possible for a prefix-free regular language. 
\end{theorem}


\begin{proof}\label{proof:pfrev}
Let  be the DFA  restricted to . Since  is prefix-free, so is . We show that  . 

Let  be the NFA obtained by removing unreachable states from the NFA . (See Fig.~\ref{fig:pfrev} for .)  We first prove that the following  sets of states of  are reachable:


\begin{figure}[hbt]
\begin{center}
\setlength{\unitlength}{0.00056868in}
\begingroup\makeatletter\ifx\SetFigFont\undefined \gdef\SetFigFont#1#2#3#4#5{\reset@font\fontsize{#1}{#2pt}\fontfamily{#3}\fontseries{#4}\fontshape{#5}\selectfont}\fi\endgroup {\renewcommand{\dashlinestretch}{30}
\begin{picture}(6256,1578)(0,-10)
\put(5758,161){\makebox(0,0)[b]{\smash{{\SetFigFont{8}{9.6}{\familydefault}{\mddefault}{\updefault}5}}}}
\put(199,131){\makebox(0,0)[b]{\smash{{\SetFigFont{8}{9.6}{\familydefault}{\mddefault}{\updefault}1}}}}
\put(1558,199){\ellipse{382}{382}}
\put(1572,131){\makebox(0,0)[b]{\smash{{\SetFigFont{8}{9.6}{\familydefault}{\mddefault}{\updefault}2}}}}
\put(2988,204){\ellipse{382}{382}}
\put(2989,131){\makebox(0,0)[b]{\smash{{\SetFigFont{8}{9.6}{\familydefault}{\mddefault}{\updefault}3}}}}
\put(4383,204){\ellipse{382}{382}}
\put(4384,131){\makebox(0,0)[b]{\smash{{\SetFigFont{8}{9.6}{\familydefault}{\mddefault}{\updefault}4}}}}
\put(2314.000,-1170.286){\arc{4901.878}{3.8024}{5.6224}}
\blacken\path(4149.913,408.039)(4249.000,334.000)(4196.413,445.958)(4149.913,408.039)
\put(201,194){\ellipse{324}{324}}
\put(5743,222){\ellipse{382}{382}}
\blacken\path(3311.000,229.000)(3191.000,199.000)(3311.000,169.000)(3311.000,229.000)
\path(3191,199)(4172,199)
\blacken\path(1894.000,229.000)(1774.000,199.000)(1894.000,169.000)(1894.000,229.000)
\path(1774,199)(2781,199)
\blacken\path(506.000,229.000)(386.000,199.000)(506.000,169.000)(506.000,229.000)
\path(386,199)(1367,199)
\blacken\path(4698.000,237.000)(4578.000,207.000)(4698.000,177.000)(4698.000,237.000)
\path(4578,207)(5559,207)
\path(6244,207)(5943,207)
\blacken\path(6063.000,237.000)(5943.000,207.000)(6063.000,177.000)(6063.000,237.000)
\path(109,379)(108,382)(105,388)
	(101,398)(95,412)(88,429)
	(81,448)(74,468)(68,489)
	(63,511)(59,534)(58,557)
	(59,581)(64,604)(71,622)
	(79,638)(87,650)(94,659)
	(101,665)(106,670)(112,673)
	(117,675)(122,677)(128,679)
	(135,681)(143,684)(154,687)
	(167,690)(182,693)(199,694)
	(216,693)(231,690)(244,687)
	(255,684)(263,681)(270,679)
	(276,677)(282,675)(286,673)
	(292,670)(297,665)(304,659)
	(311,650)(319,638)(327,622)
	(334,604)(339,581)(340,557)
	(339,534)(335,511)(330,489)
	(324,468)(317,448)(310,429)
	(303,412)(289,379)
\blacken\path(308.249,501.186)(289.000,379.000)(363.483,477.753)(308.249,501.186)
\path(2899,379)(2898,382)(2895,388)
	(2891,398)(2885,412)(2878,429)
	(2871,448)(2864,468)(2858,489)
	(2853,511)(2849,534)(2848,557)
	(2849,581)(2854,604)(2861,622)
	(2869,638)(2877,650)(2884,659)
	(2891,665)(2896,670)(2902,673)
	(2907,675)(2912,677)(2918,679)
	(2925,681)(2933,684)(2944,687)
	(2957,690)(2972,693)(2989,694)
	(3006,693)(3021,690)(3034,687)
	(3045,684)(3053,681)(3060,679)
	(3066,677)(3072,675)(3076,673)
	(3082,670)(3087,665)(3094,659)
	(3101,650)(3109,638)(3117,622)
	(3124,604)(3129,581)(3130,557)
	(3129,534)(3125,511)(3120,489)
	(3114,468)(3107,448)(3100,429)
	(3093,412)(3079,379)
\blacken\path(3098.249,501.186)(3079.000,379.000)(3153.483,477.753)(3098.249,501.186)
\path(1459,379)(1458,382)(1455,388)
	(1451,398)(1445,412)(1438,429)
	(1431,448)(1424,468)(1418,489)
	(1413,511)(1409,534)(1408,557)
	(1409,581)(1414,604)(1421,622)
	(1429,638)(1437,650)(1444,659)
	(1451,665)(1456,670)(1462,673)
	(1467,675)(1472,677)(1478,679)
	(1485,681)(1493,684)(1504,687)
	(1517,690)(1532,693)(1549,694)
	(1566,693)(1581,690)(1594,687)
	(1605,684)(1613,681)(1620,679)
	(1626,677)(1632,675)(1636,673)
	(1642,670)(1647,665)(1654,659)
	(1661,650)(1669,638)(1677,622)
	(1684,604)(1689,581)(1690,557)
	(1689,534)(1685,511)(1680,489)
	(1674,468)(1667,448)(1660,429)
	(1653,412)(1639,379)
\blacken\path(1658.249,501.186)(1639.000,379.000)(1713.483,477.753)(1658.249,501.186)
\put(1549,829){\makebox(0,0)[b]{\smash{{\SetFigFont{8}{9.6}{\familydefault}{\mddefault}{\updefault}}}}}
\put(2989,829){\makebox(0,0)[b]{\smash{{\SetFigFont{8}{9.6}{\familydefault}{\mddefault}{\updefault}}}}}
\put(199,829){\makebox(0,0)[b]{\smash{{\SetFigFont{8}{9.6}{\familydefault}{\mddefault}{\updefault}}}}}
\put(2314,1392){\makebox(0,0)[b]{\smash{{\SetFigFont{8}{9.6}{\familydefault}{\mddefault}{\updefault}}}}}
\put(3702,281){\makebox(0,0)[b]{\smash{{\SetFigFont{8}{9.6}{\familydefault}{\mddefault}{\updefault}}}}}
\put(2374,296){\makebox(0,0)[b]{\smash{{\SetFigFont{8}{9.6}{\familydefault}{\mddefault}{\updefault}}}}}
\put(1001,289){\makebox(0,0)[b]{\smash{{\SetFigFont{8}{9.6}{\familydefault}{\mddefault}{\updefault}}}}}
\put(5119,295){\makebox(0,0)[b]{\smash{{\SetFigFont{8}{9.6}{\familydefault}{\mddefault}{\updefault}}}}}
\put(199,199){\ellipse{382}{382}}
\end{picture}
}
 \end{center}
\caption{NFA  of  with quotient complexity ; empty state omitted.}
\label{fig:pfrev}
\end{figure}

The singleton set  of initial states  of  is reached by . From  we reach the empty set by . 
The set  is reached by  from , and from here,  is reached by . From any set , where , we reach 
 by . Thus we reach  from  by
.
Now assume that any set  of cardinality  can be reached; then we can get a set of cardinality  by deleting an element~ from  by applying 
. Hence all the subsets of   can be reached.

The automaton  is a subset of , and it is deterministic. Then  is minimal. Hence , which is the maximal quotient complexity of reversal of prefix-free languages as shown in~\cite{HSW09}. \qed
\end{proof} 




\medskip
It is interesting that, for suffix-, bifix-, and factor-free regular languages, although we don't have tight upper bounds on their syntactic complexities, some languages in these classes with large syntactic complexities have their reverse languages reaching the upper bounds on the quotient complexities for the reversal operation. 

\begin{theorem}\label{thm:sfrev}
The reverse of the suffix-free regular language accepted by the DFA  of Theorem~\ref{thm:wsfaut} restricted to  has  quotients, which is the maximum possible for a suffix-free regular language. 
\end{theorem}

\begin{proof}
Let  be the DFA  restricted to the alphabet . Since  is suffix-free, so is . Let  be the NFA obtained from  by removing unreachable states. Figure~\ref{fig:sfrev} shows the NFA . 

\begin{figure}[hbt]
\begin{center}
\setlength{\unitlength}{0.00052493in}
\begingroup\makeatletter\ifx\SetFigFont\undefined \gdef\SetFigFont#1#2#3#4#5{\reset@font\fontsize{#1}{#2pt}\fontfamily{#3}\fontseries{#4}\fontshape{#5}\selectfont}\fi\endgroup {\renewcommand{\dashlinestretch}{30}
\begin{picture}(5986,2010)(0,-10)
\put(2318.000,1182.000){\arc{1530.000}{1.0808}{2.0608}}
\blacken\path(2579.885,431.678)(2678.000,507.000)(2555.982,486.711)(2579.885,431.678)
\put(1643.000,968.250){\arc{337.500}{2.4981}{6.9267}}
\blacken\path(1781.575,990.641)(1778.000,867.000)(1839.339,974.413)(1781.575,990.641)
\put(2993.000,968.250){\arc{337.500}{2.4981}{6.9267}}
\blacken\path(3131.575,990.641)(3128.000,867.000)(3189.339,974.413)(3131.575,990.641)
\put(4343.000,968.250){\arc{337.500}{2.4981}{6.9267}}
\blacken\path(4481.575,990.641)(4478.000,867.000)(4539.339,974.413)(4481.575,990.641)
\put(5693.000,968.250){\arc{337.500}{2.4981}{6.9267}}
\blacken\path(5831.575,990.641)(5828.000,867.000)(5889.339,974.413)(5831.575,990.641)
\put(3684.189,-495.622){\arc{4435.362}{3.7528}{5.6470}}
\blacken\path(5370.521,898.143)(5468.000,822.000)(5417.821,935.058)(5370.521,898.143)
\put(2993,597){\ellipse{570}{570}}
\put(293,597){\ellipse{570}{570}}
\put(1643,597){\ellipse{570}{570}}
\put(4343,597){\ellipse{570}{570}}
\put(5693,597){\ellipse{570}{570}}
\put(293,597){\ellipse{450}{450}}
\path(1373,597)(608,597)
\blacken\path(728.000,627.000)(608.000,597.000)(728.000,567.000)(728.000,627.000)
\path(2723,597)(1958,597)
\blacken\path(2078.000,627.000)(1958.000,597.000)(2078.000,567.000)(2078.000,627.000)
\path(1643,12)(1643,282)
\blacken\path(1673.000,162.000)(1643.000,282.000)(1613.000,162.000)(1673.000,162.000)
\path(4073,597)(3308,597)
\blacken\path(3428.000,627.000)(3308.000,597.000)(3428.000,567.000)(3428.000,627.000)
\path(5423,597)(4658,597)
\blacken\path(4778.000,627.000)(4658.000,597.000)(4778.000,567.000)(4778.000,627.000)
\put(293,539){\makebox(0,0)[b]{\smash{{\SetFigFont{7}{8.4}{\familydefault}{\mddefault}{\updefault}1}}}}
\put(1643,539){\makebox(0,0)[b]{\smash{{\SetFigFont{7}{8.4}{\familydefault}{\mddefault}{\updefault}2}}}}
\put(2993,539){\makebox(0,0)[b]{\smash{{\SetFigFont{7}{8.4}{\familydefault}{\mddefault}{\updefault}3}}}}
\put(4343,539){\makebox(0,0)[b]{\smash{{\SetFigFont{7}{8.4}{\familydefault}{\mddefault}{\updefault}4}}}}
\put(5693,539){\makebox(0,0)[b]{\smash{{\SetFigFont{7}{8.4}{\familydefault}{\mddefault}{\updefault}5}}}}
\put(1643,1272){\makebox(0,0)[b]{\smash{{\SetFigFont{9}{10.8}{\familydefault}{\mddefault}{\updefault}}}}}
\put(2318,192){\makebox(0,0)[b]{\smash{{\SetFigFont{9}{10.8}{\familydefault}{\mddefault}{\updefault}}}}}
\put(2993,1272){\makebox(0,0)[b]{\smash{{\SetFigFont{9}{10.8}{\familydefault}{\mddefault}{\updefault}}}}}
\put(4343,1272){\makebox(0,0)[b]{\smash{{\SetFigFont{9}{10.8}{\familydefault}{\mddefault}{\updefault}}}}}
\put(5693,1272){\makebox(0,0)[b]{\smash{{\SetFigFont{9}{10.8}{\familydefault}{\mddefault}{\updefault}}}}}
\put(3668,665){\makebox(0,0)[b]{\smash{{\SetFigFont{9}{10.8}{\familydefault}{\mddefault}{\updefault}}}}}
\put(5018,665){\makebox(0,0)[b]{\smash{{\SetFigFont{9}{10.8}{\familydefault}{\mddefault}{\updefault}}}}}
\put(3668,1812){\makebox(0,0)[b]{\smash{{\SetFigFont{9}{10.8}{\familydefault}{\mddefault}{\updefault}}}}}
\put(2318,665){\makebox(0,0)[b]{\smash{{\SetFigFont{9}{10.8}{\familydefault}{\mddefault}{\updefault}}}}}
\put(968,665){\makebox(0,0)[b]{\smash{{\SetFigFont{9}{10.8}{\familydefault}{\mddefault}{\updefault}}}}}
\end{picture}
}
 \end{center}
\caption{NFA  of  with quotient complexity ; empty state omitted.}
\label{fig:sfrev}
\end{figure}

Apply the subset construction to , we get a DFA . Its initial state is a singleton set . From the initial state, we can reach state  by , where . Then the state  is reached from  by . Assume that any set  of cardinality  can be reached, where . If , then we can reach  from  by . So all the nonempty subsets of  can be reached. We can also reach the singleton set  from  by , and, from there, the empty state by  again. Hence  has  reachable states. 


Since the automaton , the reverse of , is a subset of , it is deterministic; hence  is minimal. Then the quotient complexity of  is , which meets the upper bound for reversal of suffix-free regular languages~\cite{HS09}. \qed
\end{proof}



\begin{theorem}\label{thm:ffrev}
The reverse of the factor-free regular language accepted by the DFA  of Theorem~\ref{thm:ffaut} restricted to the alphabet , where , has  quotients, which is the maximum possible for a bifix- or factor-free regular language. 
\end{theorem}

\begin{proof}
Let  be the DFA  restricted to the alphabet ; then  is factor-free. Let  be the NFA obtained from  by removing unreachable states. An example of  is shown in Figure~\ref{fig:ffrev}. 

\begin{figure}[hbt]
\begin{center}
\setlength{\unitlength}{0.00052493in}
\begingroup\makeatletter\ifx\SetFigFont\undefined \gdef\SetFigFont#1#2#3#4#5{\reset@font\fontsize{#1}{#2pt}\fontfamily{#3}\fontseries{#4}\fontshape{#5}\selectfont}\fi\endgroup {\renewcommand{\dashlinestretch}{30}
\begin{picture}(5986,3135)(0,-10)
\put(293,1722){\ellipse{570}{570}}
\put(293,1722){\ellipse{450}{450}}
\put(1643,597){\ellipse{570}{570}}
\put(2318.000,2307.000){\arc{1530.000}{1.0808}{2.0608}}
\blacken\path(2579.885,1556.678)(2678.000,1632.000)(2555.982,1611.711)(2579.885,1556.678)
\put(1643.000,2093.250){\arc{337.500}{2.4981}{6.9267}}
\blacken\path(1781.575,2115.641)(1778.000,1992.000)(1839.339,2099.413)(1781.575,2115.641)
\put(2993.000,2093.250){\arc{337.500}{2.4981}{6.9267}}
\blacken\path(3131.575,2115.641)(3128.000,1992.000)(3189.339,2099.413)(3131.575,2115.641)
\put(4343.000,2093.250){\arc{337.500}{2.4981}{6.9267}}
\blacken\path(4481.575,2115.641)(4478.000,1992.000)(4539.339,2099.413)(4481.575,2115.641)
\put(5693.000,2093.250){\arc{337.500}{2.4981}{6.9267}}
\blacken\path(5831.575,2115.641)(5828.000,1992.000)(5889.339,2099.413)(5831.575,2115.641)
\put(3684.189,629.378){\arc{4435.362}{3.7528}{5.6470}}
\blacken\path(5370.521,2023.143)(5468.000,1947.000)(5417.821,2060.058)(5370.521,2023.143)
\put(2993,1722){\ellipse{570}{570}}
\put(4343,1722){\ellipse{570}{570}}
\put(5693,1722){\ellipse{570}{570}}
\put(1643,1722){\ellipse{570}{570}}
\path(1373,1722)(608,1722)
\blacken\path(728.000,1752.000)(608.000,1722.000)(728.000,1692.000)(728.000,1752.000)
\path(2723,1722)(1958,1722)
\blacken\path(2078.000,1752.000)(1958.000,1722.000)(2078.000,1692.000)(2078.000,1752.000)
\path(4073,1722)(3308,1722)
\blacken\path(3428.000,1752.000)(3308.000,1722.000)(3428.000,1692.000)(3428.000,1752.000)
\path(5423,1722)(4658,1722)
\blacken\path(4778.000,1752.000)(4658.000,1722.000)(4778.000,1692.000)(4778.000,1752.000)
\path(1643,12)(1643,282)
\blacken\path(1673.000,162.000)(1643.000,282.000)(1613.000,162.000)(1673.000,162.000)
\path(1643,867)(1643,1407)
\blacken\path(1673.000,1287.000)(1643.000,1407.000)(1613.000,1287.000)(1673.000,1287.000)
\put(293,1664){\makebox(0,0)[b]{\smash{{\SetFigFont{7}{8.4}{\familydefault}{\mddefault}{\updefault}1}}}}
\put(1643,539){\makebox(0,0)[b]{\smash{{\SetFigFont{7}{8.4}{\familydefault}{\mddefault}{\updefault}6}}}}
\put(1643,1664){\makebox(0,0)[b]{\smash{{\SetFigFont{7}{8.4}{\familydefault}{\mddefault}{\updefault}2}}}}
\put(2993,1664){\makebox(0,0)[b]{\smash{{\SetFigFont{7}{8.4}{\familydefault}{\mddefault}{\updefault}3}}}}
\put(4343,1664){\makebox(0,0)[b]{\smash{{\SetFigFont{7}{8.4}{\familydefault}{\mddefault}{\updefault}4}}}}
\put(5693,1664){\makebox(0,0)[b]{\smash{{\SetFigFont{7}{8.4}{\familydefault}{\mddefault}{\updefault}5}}}}
\put(1643,2397){\makebox(0,0)[b]{\smash{{\SetFigFont{9}{10.8}{\familydefault}{\mddefault}{\updefault}}}}}
\put(2318,1317){\makebox(0,0)[b]{\smash{{\SetFigFont{9}{10.8}{\familydefault}{\mddefault}{\updefault}}}}}
\put(2993,2397){\makebox(0,0)[b]{\smash{{\SetFigFont{9}{10.8}{\familydefault}{\mddefault}{\updefault}}}}}
\put(4343,2397){\makebox(0,0)[b]{\smash{{\SetFigFont{9}{10.8}{\familydefault}{\mddefault}{\updefault}}}}}
\put(5693,2397){\makebox(0,0)[b]{\smash{{\SetFigFont{9}{10.8}{\familydefault}{\mddefault}{\updefault}}}}}
\put(3668,1790){\makebox(0,0)[b]{\smash{{\SetFigFont{9}{10.8}{\familydefault}{\mddefault}{\updefault}}}}}
\put(5018,1790){\makebox(0,0)[b]{\smash{{\SetFigFont{9}{10.8}{\familydefault}{\mddefault}{\updefault}}}}}
\put(3668,2937){\makebox(0,0)[b]{\smash{{\SetFigFont{9}{10.8}{\familydefault}{\mddefault}{\updefault}}}}}
\put(2318,1790){\makebox(0,0)[b]{\smash{{\SetFigFont{9}{10.8}{\familydefault}{\mddefault}{\updefault}}}}}
\put(968,1790){\makebox(0,0)[b]{\smash{{\SetFigFont{9}{10.8}{\familydefault}{\mddefault}{\updefault}}}}}
\put(1418,1047){\makebox(0,0)[b]{\smash{{\SetFigFont{9}{10.8}{\familydefault}{\mddefault}{\updefault}}}}}
\end{picture}
}
 \end{center}
\caption{NFA  of  with quotient complexity ; empty state omitted.}
\label{fig:ffrev}
\end{figure}

Note that  can be obtained from the NFA  in Theorem~\ref{thm:sfrev} by adding a new state , which is the only initial state in , and the transition from  to  under input . We know that all non-empty subsets of  are reachable from . The accepting state  is also reachable from . From the initial state , we reach the empty state under input . Then  has  reachable states. 

Since  is a subset of  and it is deterministic, the DFA  is minimal. Therefore , and it reaches the upper bound for reversal of both bifix- and factor-free regular languages with quotient complexity ~\cite{BJLS11}. \qed
\end{proof}

\section{Conclusions}\label{sec:cl}

Our results are summarized in Tables~\ref{tab:Summary1} and~\ref{tab:Summary2}. Each cell of Table~\ref{tab:Summary1} shows the syntactic complexity bounds of prefix- and suffix-free regular languages, in that order, with a particular alphabet size. Table~\ref{tab:Summary2} is structured similarly for bifix- and factor-free regular languages. The figures in bold type are tight bounds verified by {\it GAP}. To compute the bounds for suffix-, bifix-, and factor-free languages, we enumerated semigroups generated by elements of , , and  that are contained in , , and , respectively, and recorded the largest ones. By Propositions~\ref{prop:sf},~\ref{prop:bf},~\ref{prop:ff}, we obtained the desired bounds from the enumeration. The asterisk  indicates that the bound is already tight for a smaller alphabet. In Table~\ref{tab:Summary1}, the last four rows include the tight upper bound  for prefix-free languages, , which is a tight upper bound for  for suffix-free languages, conjectured upper bound  for suffix-free languages, and a weaker upper bound  for suffix-free languages. In Table~\ref{tab:Summary2}, the last four rows include , which is a tight upper bound for bifix-free languages for , conjectured upper bounds  for bifix-free languages and  for factor-free languages, and weaker upper bounds  for bifix-free languages and  for factor-free languages. 
\vspace{-.4cm}


\begin{table}[H]
\caption{Syntactic complexities of prefix- and suffix-free regular languages.}
\label{tab:Summary1}
\begin{center}

\end{center}
\label{table1}
\end{table}

\vspace{-1.4cm}

\begin{table}[H]
\caption{Syntactic complexities of bifix- and factor-free regular languages.}
\label{tab:Summary2}
\begin{center}

\end{center}
\label{table2}
\end{table}

\begin{thebibliography}{10}

\bibitem{BPR09}
Berstel, J., Perrin, D., Reutenauer, C.:
\newblock Codes and Automata.
\newblock Cambridge University Press (2009)

\bibitem{Brz09}
Brzozowski, J.:
\newblock Quotient complexity of regular languages.
\newblock J. Autom. Lang. Comb \textbf{15}(1/2) (2010)  71--89

\bibitem{BrYe11}
Brzozowski, J., Ye, Y.:
\newblock Syntactic complexity of ideal and closed languages.
\newblock In Mauri, G., Leporati, A., eds.: 15th International Conference on
  Developments in Language Theory, DLT 2011. Volume 6795 of LNCS, Springer
  Berlin / Heidelberg (2011)  117--128

\bibitem{BJLS11}
Brzozowski, J., Jir{\'a}skov{\'a}, G., Li, B., Smith, J.:
\newblock Quotient complexity of bifix-, factor-, and subword-free regular
  languages.
\newblock In D\"{o}m\"{o}si, P., Iv\'{a}n, S., eds.: Automata and Formal
  Languages – 13th International Conference AFL 2011, College of
  Ny\'{i}regyh\'{a}za, Debrecen, Hungary (2011)  123--137

\bibitem{Brz62}
Brzozowski, J.A.:
\newblock {Canonical regular expressions and minimal state graphs for definite
  events}.
\newblock In: Mathematical theory of Automata. Volume 12 of MRI Symposia
  Series.
\newblock Polytechnic Press, Polytechnic Institute of Brooklyn, N.Y. (1962)
  529--561

\bibitem{BL11}
Brzozowski, J.A., Li, B.:
\newblock Syntactic complexity of star-free languages.
\newblock {\tt http://arxiv.org/abs/1109.3381} (September 2011)

\bibitem{GaMa09}
Ganyushkin, O., Mazorchuk, V.:
\newblock Classical Finite Transformation Semigroups: An Introduction.
\newblock Springer (2009)

\bibitem{GAP}
GAP-Group:
\newblock GAP - Groups, Algorithms, Programming - a System for Computational
  Discrete Algebra.
\newblock (2010) {\tt http://www.gap-system.org/}.

\bibitem{HS09}
Han, Y.S., Salomaa, K.:
\newblock State complexity of basic operations on suffix-free regular
  languages.
\newblock Theoret. Comput. Sci. \textbf{410}(27-29) (2009)  2537--2548

\bibitem{HSW09}
Han, Y.S., Salomaa, K., Wood, D.:
\newblock Operational state complexity of prefix-free regular languages.
\newblock In {\'E}sik, Z., F{\"u}l{\"o}p, Z., eds.: Automata, Formal Languages,
  and Related Topics, Inst. of Informatics, University of Szeged, Hungary
  (2009)  99--115

\bibitem{HoKo04}
Holzer, M., K\"{o}nig, B.:
\newblock On deterministic finite automata and syntactic monoid size.
\newblock Theoret. Comput. Sci. \textbf{327}(3) (2004)  319--347

\bibitem{Hoy1895}
Hoyer, M.:
\newblock Verallgemeinerung zweier s\"atze aus der theorie der
  substitutionengruppen.
\newblock Math. Ann. \textbf{46} (1895)  539--544

\bibitem{KLS03}
Krawetz, B., Lawrence, J., Shallit, J.:
\newblock State complexity and the monoid of transformations of a finite set.
\newblock {\tt http://arxiv.org/abs/math/0306416v1} (2003)

\bibitem{Mas70}
Maslov, A.N.:
\newblock Estimates of the number of states of finite automata.
\newblock Dokl. Akad. Nauk SSSR \textbf{194} (1970)  1266--1268 (Russian)
  {English} translation: Soviet Math. Dokl. 11 (1970), 1373--1375.

\bibitem{McNP71}
McNaughton, R., Papert, S.A.:
\newblock Counter-Free Automata. Volume~65 of M.I.T. Research Monographs.
\newblock The MIT Press (1971)

\bibitem{Myh57}
Myhill, J.:
\newblock Finite automata and representation of events.
\newblock Wright Air Development Center Technical Report \textbf{57--624} 1957.

\bibitem{Ner58}
Nerode, A.:
\newblock Linear automaton transformations.
\newblock Proc. Amer. Math. Soc. \textbf{9} (1958)  541--544

\bibitem{Pic35}
Piccard, S.:
\newblock Sur les fonctions d\'efinies dans les ensembles finis quelconques.
\newblock Fund. Math. \textbf{24} (1935)  298--301

\bibitem{Pic38}
Piccard, S.:
\newblock Sur les bases du groupe sym\'etrique et du groupe alternant.
\newblock Commentarii Mathematici Helvetici \textbf{11} (1938)  1--8

\bibitem{Pin97}
Pin, J.E.:
\newblock Syntactic semigroups.
\newblock In: Handbook of Formal Languages, vol.~1: Word, Language, Grammar.
\newblock Springer-Verlag New York, Inc., New York, NY, USA (1997)  679--746

\bibitem{Rio60}
Riordan, J.:
\newblock The enumeration of trees by height and diameter.
\newblock IBM J. Res. Dev. \textbf{4} (November 1960)  473--478

\bibitem{Shy01}
Shyr, H.J.:
\newblock Free Monoids and Languages. Third edn.
\newblock Hon Min Book Co, Taiwan (2001)

\bibitem{Sie35}
Sierpi\'nski, W.:
\newblock Sur les suites infinies de fonctions d\'efinies dans les ensembles
  quelconques.
\newblock Fund. Math. \textbf{24} (1935)  209--212

\bibitem{Yu01}
Yu, S.:
\newblock State complexity of regular languages.
\newblock J. Autom. Lang. Comb. \textbf{6} (2001)  221--234

\end{thebibliography}

\end{document}
