\documentclass{article}
\PassOptionsToPackage{numbers, compress}{natbib}


\usepackage[preprint]{neurips_2021}
\usepackage{amsmath}







\usepackage[utf8]{inputenc} \usepackage[T1]{fontenc}    \usepackage{hyperref}       \usepackage{url}            \usepackage{booktabs}       \usepackage{amsfonts}       \usepackage{nicefrac}       \usepackage{microtype}      \usepackage{xcolor}         


\usepackage{multirow}
\usepackage{booktabs}
\usepackage{adjustbox}
\usepackage{array}

\usepackage{caption} 
\usepackage{subcaption}
\usepackage{wrapfig}
\captionsetup[table]{skip=3pt}
\usepackage{enumitem}

\usepackage{xspace}
\usepackage{amssymb}
\usepackage{pifont}
\usepackage{wrapfig}

\makeatletter
\DeclareRobustCommand\onedot{\futurelet\@let@token\@onedot}
\def\@onedot{\ifx\@let@token.\else.\null\fi\xspace}

\def\eg{\emph{e.g}\onedot} \def\Eg{\emph{E.g}\onedot}
\def\ie{\emph{i.e}\onedot} \def\Ie{\emph{I.e}\onedot}
\def\cf{\emph{c.f}\onedot} \def\Cf{\emph{C.f}\onedot}
\def\etc{\emph{etc}\onedot} \def\vs{\emph{vs}\onedot}
\def\wrt{w.r.t\onedot} \def\dof{d.o.f\onedot}
\def\etal{\emph{et al}\onedot}
\makeatother

\newcommand{\cmark}{\ding{51}}\newcommand{\xmark}{\ding{55}}

\title{Space-time Mixing Attention for Video Transformer}



\author{Adrian Bulat \\
  Samsung AI Cambridge\\
  \texttt{adrian@adrianbulat.com} \\
\And
   Juan-Manuel Perez-Rua \\
   Samsung AI Cambridge \\
   \texttt{j.perez-rua@samsung.com} \\
   \AND
   Swathikiran Sudhakaran \\
   Samsung AI Cambridge \\
   \texttt{swathikir.s@samsung.com} \\
   \And
   Brais Martinez \\
   Samsung AI Cambridge \\
   \texttt{brais.a@samsung.com} \\
   \And
   Georgios Tzimiropoulos \\
   Samsung AI Cambridge \\
   Queen Mary University of London\\
   \texttt{g.tzimiropoulos@qmul.ac.uk} \\
}

\begin{document}

\maketitle

\begin{abstract}

This paper is on video recognition using Transformers. Very recent attempts in this area have demonstrated promising results in terms of recognition accuracy, yet they have been also shown to induce, in many cases, significant computational overheads due to the additional modelling of the temporal information. In this work, we propose a Video Transformer model the complexity of which scales linearly with the number of frames in the video sequence and hence induces \textit{no overhead} compared to an image-based Transformer model. To achieve this, our model makes two approximations to the full space-time attention used in Video Transformers: (a) It restricts time attention to a local temporal window and capitalizes on the Transformer's depth to obtain full temporal coverage of the video sequence. (b) It uses efficient space-time mixing to attend \textit{jointly} spatial and temporal locations without inducing any additional cost on top of a spatial-only attention model. We also show how to integrate 2 very lightweight mechanisms for global temporal-only attention which provide additional accuracy improvements at minimal computational cost. We demonstrate that our model produces very high recognition accuracy on the most popular video recognition datasets while at the same time being significantly more efficient than other Video Transformer models. Code will be made available.

\end{abstract}

\section{Introduction}

Video recognition -- in analogy to image recognition -- refers to the problem of recognizing events of interest in video sequences such as human activities. Following the tremendous success of Transformers in sequential data, specifically in Natural Language Processing (NLP)~\citep{vaswani2017attention,chen2018best}, Vision Transformers were very recently shown to outperform CNNs for image recognition too~\citep{yuan2021tokens,dosovitskiy2020image,touvron2020training}, signaling a paradigm shift on how visual understanding models should be constructed. In light of this, in this paper, we propose a Video Transformer model as an appealing and promising solution for improving the accuracy of video recognition models. 

A direct, natural extension of Vision Transformers to the spatio-temporal domain is to perform the self-attention \textit{jointly} across all  spatial locations and  temporal locations. Full space-time attention though has complexity  making such a model computationally heavy and, hence, impractical even when compared with the 3D-based convolutional models. As such, our aim is to exploit the temporal information present in video streams while minimizing the computational burden within the Transformer framework for efficient video recognition. 

A baseline solution to this problem is to consider spatial-only attention followed by temporal averaging, which has complexity . Similar attempts to reduce the cost of full space-time attention have been recently proposed in~\citep{bertasius2021space,arnab2021vivit}. These methods have demonstrated promising results in terms of video recognition accuracy, yet they have been also shown to induce, in most of the cases, significant computational overheads compared to the baseline (spatial-only) method due to the additional modelling of the temporal information. 

\textbf{Our main contribution} in this paper is a Video Transformer model that has complexity  and, hence, is as efficient as the baseline model, yet, as our results show, it outperforms  recently/concurrently proposed work ~\citep{bertasius2021space,arnab2021vivit} in terms of efficiency (\ie accuracy/FLOP) by significant margins. To achieve this, our model makes two approximations to the full space-time attention used in Video Transformers: (a) It restricts time attention to a local temporal window and capitalizes on the Transformer's depth to obtain full temporal coverage of the video sequence. (b) It uses efficient space-time mixing to attend \textit{jointly} spatial and temporal locations without inducing any additional cost on top of a spatial-only attention model. Fig.~\ref{fig:overall_idea} shows the proposed approximation to space-time attention. We also show how to integrate two very lightweight mechanisms for global temporal-only attention, which provide additional accuracy improvements at minimal computational cost. We demonstrate that our model is surprisingly effective in terms of capturing long-term dependencies and producing very high recognition accuracy on the most popular video recognition datasets, including Something-Something-v2~\citep{goyal2017something}, Kinetics~\citep{carreira2017quo} and Epic Kitchens~\citep{damen2018scaling}, while at the same time being significantly more efficient than other Video Transformer models.

\begin{figure}
    \centering
    \begin{subfigure}[t]{0.23\textwidth}
        \centering
        \includegraphics[width=3.3cm]{figures/attention_maps/full.png}
        \caption{Full space-time attention: }
    \end{subfigure}
    ~
     \begin{subfigure}[t]{0.23\textwidth}
    \centering
        \includegraphics[width=3.3cm]{figures/attention_maps/space.png}
        \caption{Spatial-only attention: }
    \end{subfigure}
     ~
    \begin{subfigure}[t]{0.23\textwidth}
        \centering
        \includegraphics[width=3.3cm]{figures/attention_maps/timesformer.png}
        \caption{TimeSformer~\citep{bertasius2021space}: }
    \end{subfigure} 
    ~
    \begin{subfigure}[t]{0.23\textwidth}
    \centering
        \includegraphics[width=3.3cm]{figures/attention_maps/ours_2.png}
        \caption{Ours: }
    \end{subfigure}
    \caption{Different approaches to space-time self-attention for video recognition. In all cases, the key locations that the query vector, located at the center of the grid in red, attends are shown in orange. Unlike prior work, our key vector is constructed by mixing information from tokens located at the same spatial location within a local temporal window. Our method then performs self-attention with these tokens. Note that our mechanism allows for an efficient approximation of local space-time attention at no extra cost when compared to a spatial-only attention model.}
    \label{fig:overall_idea}
    \vspace*{-0.4cm}
\end{figure}

\section{Related work}

\textbf{Video recognition:} Standard solutions are based on CNNs and can be broadly classified into two categories: 2D- and 3D-based approaches. 2D-based approaches process each frame independently to extract frame-based features which are then aggregated temporally with some sort of temporal modeling (e.g. temporal averaging) performed at the end of the network ~\citep{wang2018temporal,lin2019tsm,liu2020tam}. The works of~\citep{lin2019tsm,liu2020tam} use the ``shift trick''~\citep{wu2018shift} to have some temporal modeling at a layer level. 3D-based approaches~\citep{carreira2017quo,feichtenhofer2019slowfast,tran2015learning} are considered the current state-of-the-art as they can typically learn stronger temporal models via 3D convolutions. However, they also incur higher computational and memory costs. To alleviate this, a large body of works attempt to improve their efficiency via spatial and/or temporal factorization~\citep{tran2019video,tran2018closer,feichtenhofer2020x3d}.

\textbf{CNN vs ViT:} Historically, video recognition approaches tend to mimic the architectures used for image classification (e.g. from AlexNet~\citep{krizhevsky2012imagenet} to~\citep{karpathy2014large} or from ResNet~\citep{he2016deep} and ResNeXt~\citep{xie2017aggregated} to~\citep{feichtenhofer2019slowfast}). After revolutionizing NLP~\citep{vaswani2017attention,raffel2019exploring}, very recently, Transformer-based architectures showed promising results on large scale image classification too~\citep{dosovitskiy2020image}. While self-attention and attention were previously used in conjunction with CNNs at a layer or block level~\citep{chen20182,zhang2019self,srinivas2021bottleneck}, the Vision Transformer (ViT) of \citet{dosovitskiy2020image} is the first convolution-free, Transformer-based architecture that achieves state-of-the-art results on ImageNet~\citep{deng2009imagenet}.

\textbf{Video Transformer:} Recently/concurrently with our work, vision transformer architectures, derived from~\citep{dosovitskiy2020image}, were used for video recognition~\citep{bertasius2021space,arnab2021vivit}, too. Because performing full space-time attention is computationally prohibitive (\ie ), their main focus is on reducing this via temporal and spatial factorization. In TimeSformer~\citep{bertasius2021space}, the authors propose  applying spatial and temporal attention in an alternating manner reducing the complexity to . In a similar fashion, ViViT~\citep{arnab2021vivit} explores several avenues for space-time factorization. In addition, they also proposed to adapt the patch embedding process from~\citep{dosovitskiy2020image} to 3D (\ie video) data. Our work proposes a completely different approximation to full space-time attention that is also efficient. To this end, we firstly restrict full space-time attention to a local temporal window which is reminiscent of~\citep{beltagy2020longformer} but applied here to space-time attention and video recognition. Secondly, we define a local joint space-time attention which we show that can be implemented efficiently via the ``shift trick''~\citep{wu2018shift}.

\section{Method}\label{sec:method}

\textbf{Video Transformer:} We are given a video clip  (, ). Following ViT~\citep{dosovitskiy2020image}, each frame is divided into  non-overlapping patches which are then mapped into visual tokens using a linear embedding layer .  Since self-attention is permutation invariant, in order to preserve the information regarding the location of each patch within space and time, we also learn two positional embeddings, one for space:  and one for time: . These are then added to the initial visual tokens. Finally, the token sequence is processed by  Transformer layers.

The visual token at layer , spatial location  and temporal location  is denoted as: 

In addition to the  visual tokens extracted from the video, a special classification token  is prepended to the token sequence~\citep{devlin2018bert}.
The th Transformer layer processes the visual tokens  of the previous layer using a series of  Multi-head Self-Attention (MSA), Layer Normalization (LN), and MLP () layers as follows:


The main computation of a single full space-time Self-Attention (SA) head boils down to calculating:

where  are the query, key, and value
vectors computed from  (after LN) using embedding matrices . Finally, the output of the  heads is concatenated and projected using embedding matrix .

The complexity of the full model is:  ( projections)  (MSA for  attention heads)  (multi-head projection)  (MLP). From these terms, our goal is to reduce the cost  (for a single attention head) of the full space-time attention which is the dominant term. For clarity, from now on, we will drop constant terms and  to report complexity unless necessary. Hence, the complexity of the full space-time attention is . 

\textbf{Our baseline} is a model that performs a simple approximation to the full space-time attention by applying, at each Transformer layer, spatial-only attention:

the complexity of which is . Notably, the complexity of the proposed space-time mixing attention is also . Following spatial-only attention, simple temporal averaging is performed on the class tokens  to obtain a single feature that is fed to the linear classifier.  

\textbf{Recent work} by~\citep{bertasius2021space, arnab2021vivit} has focused on reducing the cost  of the full space-time attention of Eq. \ref{eq:SA}. \citet{bertasius2021space} proposed the factorised attention:

where  are new query, key and value vectors calculated from ~\footnote{More precisely, Eq. \ref{eq:fSA} holds for  heads. For , the different heads  are concatenated and projected to produce .}. The above model reduces complexity to . However, temporal attention is performed for a fixed spatial location which is  ineffective when there is camera or object motion and there is spatial misalignment between frames. 

The work of~\citep{arnab2021vivit} is concurrent to ours and proposes the following approximation:  Transformer layers perform spatial-only attention as in Eq.~\ref{eq:sSA} (each with complexity ). Following this, there are  Transformer layers performing temporal-only attention on the class tokens . The complexity of the temporal-only attention is, in general, .

\textbf{Our model} aims to better approximate the full space-time self-attention (SA) of Eq.~\ref{eq:SA} while keeping complexity to , i.e. inducing no further complexity to a spatial-only model. 

To achieve this, we make a first approximation to perform full space-time attention but restricted to a local temporal window :  

where  and  is the vector with the corresponding attention weights. Eq.~\ref{eq:localSA} shows that, for a single Transformer layer,  is a spatio-temporal combination of the visual tokens in the local window . It follows that, after  Transformer layers,  will be a spatio-temporal combination of the visual tokens in the local window  which in turn conveniently allows to perform spatio-temporal attention over the whole clip. For example, for  and , the local window becomes  which spans the whole video clip for the typical case .

The complexity of the local self-attention of Eq.~\ref{eq:localSA} is . To reduce this even further, we make a second approximation on top of the first one as follows: the attention between spatial locations  and  according to the model of Eq.~\ref{eq:localSA} is:

i.e. it requires the calculation of  attentions, one per temporal location over . Instead, we propose to calculate a single attention over  which can be achieved by  attending . Note that to match the dimensions of  and  a further projection of  to  is normally required which has complexity  and hence compromises the goal of an efficient implementation. To alleviate this, we use the ``shift trick''~\cite{wu2018shift, lin2019tsm} which allows to perform both zero-cost dimensionality reduction, space-time mixing and attention (between  and  ) in . In particular, each  is assigned  channels from  (i.e. ). Let  denote the operator for indexing the  channels from . Then, a new key vector is constructed as:

Fig.~\ref{fig:detailed-diff} shows how the key vector  is constructed. In a similar way, we also construct a new value vector . Finally, the proposed approximation to the full space-time attention is given by: 

This has the complexity of a spatial-only attention () and hence it is more efficient than previously proposed video transformers~\cite{bertasius2021space, arnab2021vivit}. Our model also provides a better approximation to the full space-time attention and as shown by our results it significantly outperforms~\cite{bertasius2021space, arnab2021vivit}.   

\begin{figure}
    \centering
    \begin{subfigure}[b]{0.4\textwidth}
        \centering
        \includegraphics[trim={0.5cm 0.7cm 0.5cm 0.7cm}, clip,width=5.5cm]{figures/Full.pdf}
        \caption{Full Spatio-temporal attention.}
    \end{subfigure}
    ~
    \begin{subfigure}[b]{0.58\textwidth}
        \centering
        \includegraphics[trim={1.8cm 0.7cm 0.5cm 0.7cm}, clip,clip,width=6cm]{figures/Ours2-new.pdf}
        \caption{Ours.}
    \end{subfigure}

    \caption{Detailed self-attention computation graph for (a) full space-time attention and (b) the proposed space-time mixing approximation. Notice that in our case only S tokens participate instead of TS. The temporal information is aggregated by indexing channels from adjacent frames. Tokens of identical colors share the same temporal index.}
    \label{fig:detailed-diff}
    \vspace*{-0.5cm}
\end{figure}

\noindent\textbf{Temporal Attention aggregation:} The final set of the class tokens   are used to generate the predictions. To this end, we propose to consider the following options: (a) simple temporal averaging   as in the case of our baseline. (b) An obvious limitation of temporal averaging is that the output is treated purely as an ensemble of per-frame features and, hence, completely ignores the temporal ordering between them. To address this, we propose to use a lightweight Temporal Attention (TA) mechanism that will attend the  classification tokens. In particular, a  token attends the sequence  using a temporal Transformer layer and then fed as input to the classifier. This is akin to the (concurrent) work of~\cite{arnab2021vivit} with the difference being that in our model we found that a single TA layer suffices whereas~\cite{arnab2021vivit} uses . A consequence of this is that the complexity of our layer is  vs  of~\cite{arnab2021vivit}.     

\noindent\textbf{Summary token:} As an alternative to TA, herein, we also propose a simple lightweight mechanism for information exchange between different frames at intermediate layers of the network. Given the set of tokens for each frame ,  (constructed by concatenating all tokens ),  we compute a new set of  tokens  which summarize the frame information and hence are named ``Summary'' tokens. These are, then, appended to the visual tokens of all frames to calculate the keys and values so that the query vectors attend the original keys plus the Summary tokens. 
Herein, we explore the case that  performs simple spatial averaging  over the tokens of each frame ( for this case). Note that, for , the extra cost that the Summary token induces is .

\noindent\textbf{X-ViT:} We call the Video Transformer based on the proposed (a) space-time mixing attention and (b)  lightweight global temporal attention (or summary token) as \textbf{X-ViT}. 

\section{Results}\label{sec:results}

\subsection{Experimental setup}\label{ssec:results-experimental}

\noindent \textbf{Datasets:} We trained and evaluated the proposed models on the following datasets (all datasets are publicly available for research purposes):

\textit{Kinetics-400 and 600}: The Kinetics~\citep{kay2017kinetics} dataset consists of short clips (typically 10 sec long) sampled from YouTube and labeled using 400 and 600 classes, respectively. Due to the removal of certain videos from YouTube, the versions of the datasets used in this paper consist of approximately 261k clips for Kinetics-400 and 457k for Kinetics-600. Note, that these numbers are lower than the original datasets and thus might induce a negative performance bias when compared with prior works.

\textit{Something-Something-v2 (SSv2)}: The SSv2~\citep{goyal2017something} dataset consists of 220,487 short videos, with a length between 2 and 6 seconds that picture humans performing pre-defined basic actions with everyday objects. Since the objects and backgrounds in the videos are consistent across different action classes, this dataset tends to require stronger temporal modeling.  Due to this, we conducted most of our ablation studies on SSv2 to better analyze the importance of the proposed components.

\textit{Epic Kitchens-100 (Epic-100)}: Epic-100 is an egocentric large scale action recognition dataset consisting of more than 90,000 action segments that span across 100 hours of recordings in native environments, capturing daily activities~\citep{damen2020rescaling}. The dataset is labeled using 97 verb and 300 noun classes. The evaluation results are reported using the standard action recognition protocol: the network predicts the ``verb'' and the ``noun'' using two heads. The predictions are then merged to construct an ``action'' which is used to calculate the accuracy.

\noindent \textbf{Network architecture:} The backbone models closely follow the ViT architecture~\citep{dosovitskiy2020image}. Most of the experiments were performed using the ViT-B/16 variant (, , , ), where  represents the number of transformer layers,  the number of heads,  the embedding dimension and  the patch size. We initialized our models from a pretrained ImageNet-21k~\citep{deng2009imagenet} ViT model. The spatial positional encoding  was initialized from the pretrained 2D model and the temporal one, , with zeros so that it does not have a great impact on the tokens early on during training. The models were trained on 8 V100 GPUs using PyTorch~\citep{paszke2019pytorch}.



\textbf{Testing details:} Unless otherwise stated, we used ViT-B/16 and  frames. We mostly used Temporal Attention (TA) for temporal aggregation. We report accuracy results for  views (1 temporal clip and 3 spatial crops) departing from the common approach of using up to  views~\citep{lin2019tsm,feichtenhofer2019slowfast}. The  views setting was also used in~\citet{bertasius2021space}. To measure the variation between runs, we trained one of the 8--frame models 5 times. The results varied by . 

\subsection{Ablation studies}\label{ssec:results-ablation}

Throughout this section we study the effect of varying certain design choices and different components of our method. Because SSv2 tends to require a more fine-grained temporal modeling, unless otherwise specified, all results reported, in this subsection, are on the SSv2. 

\begin{wraptable}[14]{r}{5.5cm}
\vspace{-0.4cm}
    \caption{Effect of local window size. To isolate its effect from that of temporal aggregation, the models were trained using temporal averaging. Note, that \textit{(Bo.)} indicates that only features from the boundaries of the local window were used, ignoring the intermediate ones. }\label{tab:local-window-size}
    
    \centering
    \begin{tabular}{ccc}
        \toprule
      Variant & Top-1 & Top-5  \\
       \midrule
         & 45.2 & 71.4  \\
         & \textbf{62.5} & \textbf{87.8}  \\
         & 60.5 & 86.4  \\
         (Bo.) & 60.4 & 86.2  \\
        \bottomrule
    \end{tabular}
\end{wraptable}

\noindent\textbf{Effect of local window size:} Table~\ref{tab:local-window-size} shows the accuracy of our model by varying the local window size  used in the proposed space-time mixing attention. Firstly, we observe that the proposed model is significantly superior to our baseline () which uses spatial-only attention. Secondly, a window of  produces the best results. This shows that more gradual increase of the effective window size that is attended is more beneficial compared to more aggressive ones, i.e. the case where . A performance degradation for the case  could be attributed to boundary effects (handled by filling with zeros) which are aggravated as  increases. Based on these results, we chose to use  for the models reported hereafter.

\noindent\textbf{Effect of SA position:} We explored which layers should the proposed space-time mixing attention operation be applied to \textit{within the Transformer}. Specifically, we explored the following variants: Applying it to the first  layers, to the last  layers, to every odd indexed layer and finally, to all layers. As the results from Table~\ref{tab:SSv2-mixinglocation} show, the exact layers within the network that self-attention is applied to do not matter; what matters is the number of layers it is applied to. We attribute this result to the increased temporal receptive field and cross-frame interactions.

\noindent\textbf{Effect of temporal aggregation:} Herein, we compare the two methods used for temporal aggregation: simple temporal averaging~\citep{wang2016temporal} and the proposed Temporal Attention (TA) mechanism. Given that our model already incorporates temporal information through the proposed space-time attention, we also explored how many TA layers are needed. As shown in Table~\ref{tab:SSv2-vit-num-layers} replacing temporal averaging with one TA layer improves the Top-1 accuracy from 62.5\% to 64.4\%. Increasing the number of layers further yields no additional benefits. We also report the accuracy of spatial-only attention plus TA aggregation. In the absence of the proposed space-time mixing attention, the TA layer alone is unable to compensate, scoring only 56.6\% as shown in Table~\ref{tab:SSv2-mixingamount}. This highlights the need of having both components in our final model. For the next two ablation studies, we therefore used 1 TA layer.

\begin{table}[t]
    \centering
    \caption{Effect of (a) proposed SA position, (b) temporal aggregation and number of Temporal Attention (TA) layers, (c) space-time mixing  vectors and (d) amount of mixed channels on SSv2.}
    \label{tab:ablations}
    \begin{subtable}[t]{.5\linewidth}
        \caption{Effect of applying the proposed SA to certain layers.}\label{tab:SSv2-mixinglocation}
    \centering
    \begin{tabular}{ccc}
        \toprule
       Transform. layers & Top-1 & Top-5 \\
       \midrule
       1st half & 61.7 & 86.5 \\
       2nd half & 61.6 & 86.3 \\
       Half (odd. pos) & 61.2 & 86.4 \\
       All & \textbf{62.6} & \textbf{87.8} \\
        \bottomrule
    \end{tabular}
        \caption{Effect of number of TA layers. 0 corresponds to temporal averaging.}\label{tab:SSv2-vit-num-layers}
    \centering
    \begin{tabular}{ccc}
        \toprule
        \#. TA layers & Top-1 & Top-5 \\
       \midrule
       0 (temp. avg.) & 62.4 & 87.8 \\
       1 & 64.4 & \textbf{89.3} \\
       2 & \textbf{64.5} & \textbf{89.3} \\
       3 & \textbf{64.5} & \textbf{89.3} \\
        \bottomrule
    \end{tabular}
    \end{subtable}
    \hfill
      \begin{subtable}[t]{.45\linewidth}
        \caption{Effect of space-time mixing. x denotes the input token before  projection. Query produces equivalent results with key and thus omitted.}\label{tab:SSv2-mixingplacemnt}
    \centering
    \begin{tabular*}{\textwidth}{c@{\extracolsep{\fill}}cccc}
        \toprule
       x & key & value & Top-1 & Top-5 \\
       \midrule
       \xmark & \xmark & \xmark & 56.6 & 83.5 \\
       \checkmark & \xmark & \xmark & 63.1 & 88.8 \\
       \xmark & \checkmark & \xmark & 63.1 & 88.8 \\
       \xmark & \xmark & \checkmark & 62.5 & 88.6 \\
       \xmark & \checkmark & \checkmark & \textbf{64.4} & \textbf{89.3} \\
        \bottomrule
    \end{tabular*}
        \caption{Effect of amount of mixed channels. * uses temp. avg. aggregation.}\label{tab:SSv2-mixingamount}
    
    \centering
    \begin{tabular*}{\textwidth}{c@{\extracolsep{\fill}}cccc}
        \toprule
      0\%* & 0\% & 25\% & 50\% & 100\% \\
       \midrule
       45.2 & 56.6 & 64.3  & \textbf{64.4} & 62.5 \\
        \bottomrule
    \end{tabular*}
    \end{subtable}
\vspace*{-0.3cm}
\end{table}

\begin{table}[t]
    \centering
    \caption{Comparison between TA and Summary token on SSv2 (left) and Kinetics-400 (right).}
    \label{tab:ablations-summarization}
    \begin{subtable}[t]{.47\linewidth}
            \centering
            \begin{tabular}{cccc}
                \toprule
               Summary & TA & Top-1 & Top-5 \\
               \midrule
               \xmark & \xmark & 62.4 & 87.8 \\
               \checkmark & \xmark & 63.7 & 88.9 \\
               \checkmark & \checkmark & 63.4 & 88.9 \\
               \xmark & \checkmark & \textbf{64.4} & \textbf{89.3} \\
                \bottomrule
            \end{tabular}
    \end{subtable}
    \hfill
    \begin{subtable}[t]{.47\linewidth}
    \centering
            \begin{tabular}{cccc}
                \toprule
               Summary & TA & Top-1 & Top-5 \\
               \midrule
               \xmark & \xmark & 77.8 & 93.7 \\
               \checkmark & \xmark & \textbf{78.7} & \textbf{93.7} \\
               \checkmark & \checkmark & 78.0 & 93.2 \\
               \xmark & \checkmark & 78.5 & \textbf{93.7} \\
                \bottomrule
            \end{tabular}
    \end{subtable}
\vspace*{-0.4cm}
\end{table}

\begin{table}[t]
\caption{Comparison with state-of-the-art on the Kinetics-600 dataset.  is the number of frames used by our method.}\label{tab:sota-k600}
\centering
    \begin{tabular}{ccccc}
        \toprule
      Method &  Top-1 & Top-5 & Views & FLOPs () \\
      \midrule
      AttentionNAS~\citep{wang2020attentionnas} & 79.8 & 94.4 & - & 1,034 \\
      LGD-3D R101~\citep{qiu2019learning} & 81.5 & 95.6 &  & - \\
      SlowFast R101+NL~\citep{feichtenhofer2019slowfast} & 81.8 & 95.1 &  &  3,480  \\
      X3D-XL~\citep{feichtenhofer2020x3d} & 81.9 & 95.5 &  & 1,452  \\
      TimeSformer-HR~\citep{bertasius2021space} & 82.4 & 96.0 &  & 5,110  \\
      ViViT-L/16x2~\citep{arnab2021vivit} & 82.5 & 95.6 &  & 17,352  \\
      \midrule
      X-ViT (8) (Ours) & 82.5 & 95.4 &  & 425 \\
      X-ViT (16) (Ours) & \textbf{84.5} & \textbf{96.3} &  & 850 \\
        \bottomrule
    \end{tabular}
\end{table}

\begin{wraptable}[9]{r}{5.5cm}
    \vspace{-0.4cm}
    \caption{Effect of number of tokens on SSv2.}\label{tab:SSv2-arch-variations}
    
    \centering
    \begin{tabular}{ccccc}
        \toprule
      Variant & Top-1 & Top-5 &  \begin{tabular}{@{}c@{}}FLOPs \\ ()\end{tabular}  \\
       \midrule
        ViT-B/32 & 60.5 & 87.4 & 95 \\
        ViT-L/32 & 61.8 & 88.3 & 327 \\
       ViT-B/16 & \textbf{64.4} & \textbf{89.3} & 425 \\
        \bottomrule
    \end{tabular}

\end{wraptable}

\noindent\textbf{Effect of space-time mixing  vectors:} Paramount to our work is the proposed space-time mixing attention of Eq.~\ref{eq:oursSA} which is implemented by constructing  and  efficiently via channel indexing (see Eq.~\ref{eq:construct_key}). Space-time mixing though can be applied in several different ways in the model. For completeness, herein, we study the effect of space-time mixing to various combinations for the key, value and to the input token prior to  projection. As shown in Table~\ref{tab:SSv2-mixingplacemnt}, the combination corresponding to our model (\ie space-time mixing applied to the key and value) significantly outperforms all other variants by up to 2\%. This result is important as it confirms that our model, derived from the proposed approximation to the local space-time attention, gives the best results when compared to other non-well motivated variants.

\noindent\textbf{Effect of amount of space-time mixing:} We define as  the total number of channels imported from the adjacent frames in the local temporal window  (\ie ) when constructing  (see  Section~\ref{sec:method}). Herein, we study the effect of  on the model's accuracy. As the results from Table~\ref{tab:SSv2-mixingamount} show, the optimal  is between 25\% and 50\%. Increasing  to 100\% (\ie all channels are coming from adjacent frames) unsurprisingly degrades the performance as it excludes the case  when performing the self-attention.

\noindent\textbf{Effect of Summary token:} Herein, we compare Temporal Attention with Summary token on SSv2 and Kinetics-400. We used both datasets for this case as they require different type of understanding: fine-grained temporal (SSv2) and spatial content (K400). From Table~\ref{tab:ablations-summarization}, we conclude that the Summary token compares favorable on Kinetics-400 but not on SSv2 showing that is more useful in terms of capturing spatial information. Since the improvement is small, we conclude that 1 TA layer is the best global attention-based mechanism for improving the accuracy of our method adding also negligible computational cost. 

\noindent\textbf{Effect of the number of input frames:} Herein, we evaluate the impact of increasing the number of input frames  from 8 to 16 and 32. We note that, for our method, this change results in a linear increase in complexity. As the results from Table~\ref{tab:SSv2-sota} show, increasing the number of frames from 8 to 16 offers a 1.8\% boost in Top-1 accuracy on SSv2. Moreover, increasing the number of frames to 32 improves the performance by a further 0.2\%, offering diminishing returns. Similar behavior can be observed on Kinetics and Epic-100 in Tables~\ref{tab:k400-sota} and~\ref{tab:sota-epic}. 

\noindent\textbf{Effect of number of tokens:} Herein, we vary the number of input tokens by changing the patch size . As the results from Table~\ref{tab:SSv2-arch-variations} show, even when the number of tokens decreases significantly (ViT-B/32) our approach is still able to produce models that achieve satisfactory accuracy. The benefit of that is having a model which is significantly more efficient. 

\noindent \textbf{Effect of the number of crops at test time.} Throughout this work, at test time, we reported results using 1 temporal and 3 spatial crops (\ie ). This is noticeable different from the current practice of using up to  crops~\citep{feichtenhofer2019slowfast,arnab2021vivit}. 

To showcase the behavior of our method, herein, we test the effect of increasing the number of crops on Kinetics-400. As the results from Fig.~\ref{fig:k400-ncrops} show, increasing the number of crops beyond two temporal views (\ie ), yields no additional gains. Our findings align with the ones from the work of~\citet{bertasius2021space} that observes the same properties for the transformer-based architectures.

\begin{figure}[h]
    \centering
    \includegraphics[width=7.5cm]{figures/crops-k400.png}
    \caption{Effect of the number of temporal crops at test time as measured on Kinetics 400 in terms of Top 1 accuracy. For each temporal crop, 3 spatial clips are sampled, for a total of  clips. Notice that beyond  no additional accuracy gains are observed.}
    \label{fig:k400-ncrops}
\end{figure}

\noindent \textbf{Latency and throughput considerations:} While the channel shifting operation used by the proposed space-time mixing attention is zero-FLOP, there is still a small cost associated with memory movement operations. In order to ascertain that the induced cost does not introduce noticeable performance degradation, we benchmarked a Vit-B/16 (8 frames) model using spatial-only attention and the proposed one on 8 V100 GPUs and a batch size of 128. The spatial-only attention model has a throughput of 312 frames/second while our model 304 frames/second.

\begin{table}[ht!]
    \caption{Comparison with state-of-the-art on the Kinetics-400.  is the number of frames used by our method.}\label{tab:k400-sota}
    
    \centering
    \begin{tabular}{ccccc}
        \toprule
       Method &  Top-1 & Top-5 & Views & FLOPs () \\
       \midrule
       bLVNet~\citep{fan2019more} & 73.5 & 91.2 & 3  3 & 840\\
       STM~\citep{jiang2019stm} & 73.7 & 91.6 & - & - \\
       TEA~\citep{li2020tea} & 76.1 & 92.5 & 10  3 & 2,100 \\
       TSM R50~\citep{lin2019tsm} & 74.7 & - & 10  3 & 650 \\
       I3D NL~\citep{wang2018non} & 77.7 & 93.3 & 10  3 &  10,800 \\
       CorrNet-101~\citep{wang2020video} & 79.2 & - & 10  3 &  6,700 \\
       ip-CSN-152~\citep{tran2019video} & 79.2 & 93.8 & 10  3 &  3,270 \\
       LGD-3D R101~\citep{qiu2019learning} & 79.4 & 94.4 & - & - \\
       SlowFast 88 R101+NL~\citep{feichtenhofer2019slowfast} & 78.7 & 93.5 & 10  3 &  3,480 \\
       SlowFast 168 R101+NL~\citep{feichtenhofer2019slowfast} & 79.8 & 93.9 & 10  3 &  7,020 \\
       X3D-XXL~\citep{feichtenhofer2020x3d} & 80.4 & 94.6 & 10  3 &  5,823 \\
       TimeSformer-L~\citep{bertasius2021space} & \textbf{80.7} & 94.7 & 1  3 & 7,140 \\
       ViViT-L/16x2~\citep{bertasius2021space} & 80.6 & 94.7 & 4  3 & 17,352 \\
       \midrule
       X-ViT (8) (Ours) & 78.5 & 93.7 & 1  3 & 425 \\
       X-ViT (16) (Ours) & 80.2 & \textbf{94.7} & 1  3 & 850 \\
        \bottomrule
    \end{tabular}

\end{table}

\begin{table}[ht!]
    \caption{Comparison with state-of-the-art on SSv2.  is the number of frames used by our method. * - denotes models pretrained on Kinetics-600.}\label{tab:SSv2-sota}
    
    \centering
    \begin{tabular}{ccccc}
        \toprule
       Method &  Top-1 & Top-5 & Views & FLOPs () \\
       \midrule
       TRN~\citep{zhou2018temporal} & 48.8 & 77.6 & - & - \\
       SlowFast+multigrid~\citep{wu2020multigrid} & 61.7 & - &   & - \\
       TimeSformer-L~\citep{bertasius2021space} & 62.4 & - & 1  3 & 7,140  \\
       TSM R50~\citep{lin2019tsm} & 63.3 & 88.5 & 2  3 & - \\
       STM~\citep{jiang2019stm} & 64.2 & 89.8 & - & - \\
       MSNet~\citep{kwon2020motionsqueeze} & 64.7 & 89.4 & - & - \\
       TEA~\citep{li2020tea} & 65.1 & - & - & -\\
       ViViT-L/16x2~\citep{bertasius2021space} & 65.4 & 89.8 & 4  3 & 11,892 \\
       \midrule
       X-ViT (8) (Ours) & 64.4 & 89.3 & 1  3 & 425 \\
       X-ViT (16) (Ours) & 66.2 & 90.6 & 1  3 & 850 \\
       X-ViT* (16) (Ours) & \textbf{67.2} & \textbf{90.8} & 1  3 & 850 \\
       X-ViT (32) (Ours) & 66.4 & 90.7 & 1  3 & 1,270 \\
        \bottomrule
    \end{tabular}
\end{table}


\subsection{Comparison to state-of-the-art}\label{ssec:results-sota}

Our best model uses the proposed space-time mixing attention in all the Transformer layers and performs temporal aggregation using a single lightweight temporal transformer layer as described in Section~\ref{sec:method}. Unless otherwise specified, we report the results using the  configuration for the views (1 temporal and 3 spatial) for all datasets.

On \textbf{Kinetics-400},  we match the current state-of-the-art results while being significantly faster than the next two best recently/concurrently proposed methods that also use Transformer-based architectures:  faster than ViVit~\citep{arnab2021vivit} and  than TimeSformer-L~\citep{bertasius2021space}. Note that both models from~\citep{arnab2021vivit,bertasius2021space} and ours were initialized from a ViT model pretrained on ImageNet-21k~\citep{deng2009imagenet} and take as input frames at a resolution of px. Similarly, on Kinetics-600 we set a new state-of-the-art result. See Table~\ref{tab:sota-k600}.

On \textbf{SSv2} we match and surpass the current state-of-the-art, especially in terms of Top-5 accuracy (ours: 90.8\% vs ViViT: 89.8\%~\citep{arnab2021vivit}) using models that are  (16 frames) and  (32 frames) faster.

Finally, we observe similar outcomes on \textbf{Epic-100} where we set a new state-of-the-art, showing particularly large improvements especially for ``Verb'' accuracy, while again being more efficient.


\begin{wraptable}[15]{r}{7cm}
\vspace{-0.5cm}
\centering
\caption{Comparison with state-of-the-art on Epic-100.  is the \#frames used by our method. Results for other methods are taken from~\citep{arnab2021vivit}.}\label{tab:sota-epic}
    \begin{tabular}{cccc}
        \toprule
      Method &  Action & Verb & Noun  \\
      \midrule
      TSN~\citep{wang2016temporal} & 33.2 & 60.2 & 46.0 \\
      TRN~\citep{zhou2018temporal} & 35.3 & 65.9 & 45.4 \\
      TBN~\citep{kazakos2019epic} & 36.7 & 66.0 & 47.2 \\
      TSM~\citep{kazakos2019epic} & 38.3 & 67.9 & 49.0 \\
      SlowFast~\citep{feichtenhofer2019slowfast} & 38.5 & 65.6 & 50.0  \\
      ViViT-L/16x2~\citep{arnab2021vivit} & 44.0 & 66.4 & \textbf{56.8} \\
      \midrule
      X-ViT (8) (Ours) & 41.5 & 66.7 & 53.3  \\
      X-ViT (16) (Ours) & \textbf{44.3} & \textbf{68.7} & 56.4 \\
        \bottomrule
    \end{tabular}
\end{wraptable}

\section{Conclusions}\label{sec:conclusions}

We presented a novel approximation to the full space-time attention that is amenable to an efficient implementation and applied it to video recognition. Our approximation has the same computational cost as spatial-only attention yet the resulting Video Transformer model was shown to be significantly more efficient than recently/concurrently proposed Video Transformers~\cite{bertasius2021space,arnab2021vivit}. By no means this paper proposes a complete solution to video recognition using Video Transformers. Future efforts could include combining our approaches with other architectures than the standard ViT, removing the dependency on pre-trained models and applying the model to other video-related tasks like detection and segmentation. Finally, further research is required for deploying our models on low power devices.



\medskip

{
\small
\bibliographystyle{plainnat}
\bibliography{egbib}
}

\end{document}
