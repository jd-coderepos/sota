\documentclass{article}

\usepackage{amsmath,amssymb,amsbsy,amsgen,amsopn,amsthm}
\usepackage{graphicx}
\usepackage[margin=1in]{geometry}

\title{Approximating the generalized terminal backup problem via
half-integral multiflow relaxation\footnotemark[2]}

\author{Takuro Fukunaga
\thanks{National Institute of Informatics,
2-1-2 Hitotsubashi, Chiyoda-ku, Tokyo, Japan.
JST, ERATO, Kawarabayashi Large Graph Project, Japan.
  {\tt takuro@nii.ac.jp}}}
  \date{}
  
\newtheorem{theorem}{Theorem}
\newtheorem{lemma}{Lemma}
\newtheorem{fact}{Fact}
\newtheorem{corollary}{Corollary}
\newtheorem{definition}{Definition}
\newtheorem{assumption}{Assumption}

\newcommand{\Cfam}{\mathcal{C}}
\newcommand{\Vfam}{\mathcal{V}}
\newcommand{\Lfam}{\mathcal{L}}
\newcommand{\Pfam}{\mathcal{P}}
\newcommand{\Qfam}{\mathcal{Q}}
\newcommand{\Afam}{\mathcal{A}}
\newcommand{\Rfam}{\mathcal{R}}
\newcommand{\Sfam}{\mathcal{S}}
\newcommand{\Zset}{\mathbb{Z}}
\newcommand{\Rset}{\mathbb{Q}}
\newcommand{\Mfam}{\mathcal{M}}
\newcommand{\allone}{\mathbf{1}}
\newcommand{\cut}{P}
\newcommand{\lp}{\mathsf{LP}}
\newcommand{\f}{f^{\kappa}}
\newcommand{\g}{f^{\lambda}}






\begin{document}
\maketitle

\renewcommand{\thefootnote}{\fnsymbol{footnote}}
\footnotetext[2]{An extended abstract of this work appeared in the proceedings of STACS 2015.}


\begin{abstract}
We consider a network design problem called the generalized terminal backup problem.
Whereas earlier work investigated 
the edge-connectivity constraints only,
we consider both edge- and node-connectivity constraints for this problem.
A major contribution of this paper is 
the development of a strongly polynomial-time -approximation algorithm 
for the problem. Specifically,
we show that a linear programming relaxation of the problem is half-integral,
and that the half-integral optimal solution can be rounded to a -approximate solution.
We also prove that the linear programming relaxation of the problem with the edge-connectivity constraints 
is equivalent to minimizing the cost
of half-integral multiflows that satisfy flow demands given from terminals.
This observation implies a strongly polynomial-time algorithm 
for computing a minimum cost half-integral multiflow
under flow demand constraints.
\end{abstract}



\section{Introduction} 
\label{sec:introduction}

\subsection{Generalized terminal backup problem}
The network design problem is the problem of constructing a low cost network that satisfies
given constraints. It includes many fundamental optimization problems, and 
has been extensively studied. In this paper, we consider a network design problem
called the \emph{generalized terminal backup problem}, recently introduced by Bern{\'a}th and
Kobayashi~\cite{Bernath2014}. 

The generalized terminal backup problem is defined as follows. 
Let  and  
denote the sets of non-negative rational numbers and non-negative integers, 
respectively. 
Let  be an undirected graph
with node set  and edge set , 
 be an edge cost function,
and let  be an edge capacity function.
A subset  of  denotes the \emph{terminal} node set in which
each terminal  is associated with a connectivity requirement
.
A solution is a multiple edge set on  containing at most  edges parallel 
to . The objective is to find a solution
 that minimizes  under certain constraints.
In Bern{\'a}th and Kobayashi~\cite{Bernath2014}, 
the subgraph  was required to contain  
edge-disjoint paths that connect each  to other terminals.
In addition to these edge-connectivity constraints, we consider
node-connectivity constraints, under which the paths must be 
inner disjoint (i.e., disjoint in edges and nodes in ) rather than edge-disjoint.
To avoid confusion,
we refer to the problem as \emph{edge-connectivity terminal backup} 
when the edge-connectivity constraints are required, and
as \emph{node-connectivity terminal backup}
when the node-connectivity constraints are imposed.
When , the problem is called the \emph{terminal backup problem}.
Since there is no difference between edge-connectivity and node-connectivity
when , these names make no confusion.

The generalized terminal backup problem 
models a natural data management situation. Suppose that each terminal represents a data storage server
in a network, and  is the amount of data stored in the server at a terminal .
Backup data must be stored in servers different from that storing the original data.
To this end, 
a sub-network that transfers data stored at one terminal to other terminals is required. 
We assume that 
edges can transfer a single unit of data per time unit.
Hence, transferring data from terminal  to other terminals within one time unit
requires  edge-disjoint paths from  
to , which is represented by the edge-connectivity constraints. 
When nodes are also capacitated,  
inner-disjoint paths are required; these requirements are met by the node-connectivity
constraints.

The generalized terminal backup problem is interesting also from theoretical point of view.
Anshelevich and Karagiozova~\cite{AnshelevichK11} demonstrated
that the terminal backup problem is reducible to the simplex
matching problem, which is solvable in polynomial time. 
On the other hand, when , the generalized terminal backup problem is equivalent to the capacitated 
-edge cover problem with degree lower
bound  for . Since the capacitated 
-edge cover problem admits a polynomial-time algorithm, the
generalized terminal backup problem is solvable in polynomial time also when . 
Therefore, we may naturally ask whether the generalized terminal backup
problem is solvable in polynomial time. Bern{\'a}th and Kobayashi~\cite{Bernath2014} proposed a
polynomial-time algorithm for the uncapacitated case (i.e.,  for each )
in the edge-connectivity terminal backup. Their result partially
answers the above question, but their assumptions may be overly stringent in some situations;
that is, their algorithm admits unfavorable solutions that select too many copies of a cheap edge. 
Moreover, their algorithm cannot deal with the node-connectivity constraints.
Unfortunately, 
when the edge-capacities are bounded or node-connectivity constraints
are imposed,
we do not know whether
the generalized terminal backup problem is NP-hard or admits a
polynomial-time algorithm.
Instead, we propose approximation algorithms as follows.

\begin{theorem}\label{thm:main-4/3}
There exist a strongly polynomial-time -approximation algorithm for
the generalized terminal backup problem.
\end{theorem}


The present study contributes two major advances
to the generalized terminal backup problem.
\begin{itemize}
	\item Bern{\'a}th and Kobayashi~\cite{Bernath2014} discussed the 
	generalized terminal backup problem 
	in the uncapacitated setting
	with edge-connectivity constraints, noting that 
	the problem in the capacitated setting is open.
	Here, we discuss the capacitated setting, and 
	introduce the node-connectivity constraints.
	\item The generalized
	terminal backup problem can be formulated as the problem of covering skew
	supermodular biset functions, which is known to admit a 2-approximation algorithm.
	On the other hand, as stated in Theorem~\ref{thm:main-4/3}, we develop 
	-approximation algorithms, that outperform this 2-approximation algorithm.
\end{itemize}

Let us explain the second advance more specifically.
Given an edge set  and a nonempty subset  of , 
let  denote the set of edges in  with one end node
in  and the other in . 
Let  be a function such that 
 if , and  otherwise. 
By the edge-connectivity version of 
Menger's theorem,  satisfies the edge-connectivity constraints if and only if
 for each . 
Bern{\'a}th and
Kobayashi~\cite{Bernath2014} showed that the function  is skew supermodular 
(skew supermodularity is defined in Section~\ref{sec:preliminaries}). 
For any skew supermodular set function ,
Jain~\cite{Jain01} proposed a seminal -approximation algorithm for computing
a minimum-cost edge set 
satisfying , .
Although the node-connectivity constraints cannot be captured by set functions 
as the edge-connectivity constraints,
they can be regarded as a request for covering a skew supermodular
\emph{biset} function, to which the 2-approximation algorithm is extended~\cite{FleischerJW06} 
(see Section~\ref{sec:preliminaries}). Therefore, the
generalized terminal backup problem admits 2-approximation algorithms, 
regardless of the imposed connectivity constraints. 
One of our contributions is to improve these 2-approximations to
-approximations.


Both of the above 2-approximation algorithms involve 
iterative rounding of the linear programming (LP) relaxations.
Primarily, their performance analyses
prove that the value of a variable in each extreme point solution of the LP relaxations
is at least . Once this property of extreme point solutions is proven,
the variables can be repeatedly rounded until 
a 2-approximate solution is obtained. 
Our -approximation algorithms are based on the same LP 
relaxations as the iterative rounding algorithms. We
show that, in the generalized terminal backup problem,
all variables in extreme point solutions of the relaxation take 
half-integral values.
We also prove that the half-integral solution can be rounded into an integer solution
with loss of factor at most .

It may be helpful for understanding our result to see the well-studied special case of  
and  for each  (i.e., feasible solutions are simple -edge covers).
In this case, our LP relaxation minimizes  subject to 
 for each 
and  for each , where  is the set of edges incident to the node .
It has been already known that an extreme point solution of this LP 
is half-integral, and the edges in  form odd cycles.
The half-integral variables of the edges on an odd cycle
can be rounded as follows. Suppose that edges  
appear in the cycle in this order, where  is the cycle length (i.e., odd integer larger than one).
For each ,
we define 

if  and ,
or if  and ,
and 
 otherwise.
See Figure~\ref{fig.oddcicle}.
for an illustration of this definition.
Note that exactly  variables
in  are equal to one,
and the other  variables are equal to zero for each .
This means that 

Let  minimize  in .
Then, since ,
replacing  by 
increases their costs by a factor at most .
We also observe that 
the feasibility of the solution is preserved
even after the replacement.
By applying this rounding for each odd cycle, the half-integral solution 
can be transformed into a -approximate integer solution.

\begin{figure}[h]
\centering
\includegraphics[]{fig-oddcycle.pdf}
\caption{Rounding of half-integral variables corresponding to a cycle of length 5. 
A dotted line represents , and a solid thick line represents .}
\label{fig.oddcicle}
\end{figure}


Our result is obtained by extending 
the characterization of the edge structure whose corresponding variables are
not integers, but the extension is not immediate.
As in the above special case, those edges form cycles
in the generalized terminal backup problem if the solution is a minimal feasible solution to the LP relaxation.
However, the length of a cycle is not necessarily odd, and it is not clear how the half-integral solution should
be rounded; In the above special case, we round up and down variables of edges on a cycle alternatively, but this 
obviously does not preserve the feasibility in the generalized terminal backup problem.
The key ingredient in our result is to characterize the relationship between the cycles and the node sets or bisets
corresponding to linearly independent tight constraints in the LP relaxation.
We show that a cycle crosses maximal tight node set or bisets an odd number of times,
which extends the property that the length of each cycle is odd in the special case.
Our rounding algorithm decides how to round a non-integer variable 
from the direction of the crossing between the corresponding edge and a tight node set or biset.



\subsection{Minimum cost multiflow problem}\label{subsec.intro-multiflow}

Multiflows are
closely related to the generalized terminal backup problem.
Among the many multiflow variants, we focus on the type
sometimes called \emph{free multiflows}.
For ,
 denotes the set of paths that terminate at  and .
Let  denote ,
and  denote .
 and  denote the sets of edges and nodes in , respectively.
We define a multiflow as a function .
In the edge-capacitated setting,
an edge capacity  is given,
and we must satisfy 
 for each .
In the node-capacitated setting, a node capacity  is given
and  is required for each 
.
The multiflow  is called an \emph{integral} multiflow
if  for each ,
and is called a \emph{half-integral} multiflow
if  for each .
Let  denote  for .
The cost of  is given by
.


In the edge-connectivity terminal backup, 
the connectivity requirement from a terminal  equates to requiring that
a flow of amount  can be delivered from  to 
in the graph  with unit edge-capacities if  is a feasible solution.
This condition appears similar to the constraint that 
the graph  with unit edge-capacities
admits a multiflow 
such that .
We note that  with unit edge-capacities admits a multiflow 
if and only if
the number of copies of  in  is at least .
These observations suggest a correspondence between the edge-connectivity terminal backup
and the problem of finding a minimum cost multiflow 
under the constraint that 
 for  in the edge-capacitated setting.
We refer to such a multiflow computation as the \emph{minimum cost multiflow problem}
(in the edge-capacitated setting).
The same correspondence exists between
the node-connectivity terminal backup and the 
node-capacitated setting in the minimum cost multiflow problem.

However,
the generalized terminal backup and the minimum cost multiflow problems are not equivalent.
Especially, the minimum cost multiflow problem can be formulated in LP, whereas 
the generalized terminal backup problem is an integer programming problem.
Even if multiflows are restricted to integral multiflows, 
the two problems are not equivalent.
To observe this, let  be a star with an odd number of leaves. 
We assume that  is the set of leaves,
and each edge incurs one unit of cost.
This star is a feasible solution 
to the terminal backup problem (i.e.,  for ).
In contrast, setting  and 
admits no integral multiflow in the edge-capacitated setting, and
no feasible (fractional) multiflows in the node-capacitated setting.

Nevertheless,
similarities exist between terminal backups and multiflows.
As mentioned above, we will show that an LP relaxation of the generalized terminal backup problem
always admits a half-integral optimal solution.
Similarly, half-integrality results are frequently reported for multiflows.
Lov\'asz~\cite{Lovasz76} and Cherkassky~\cite{Cherkasskky77} 
investigated  in the edge-capacitated setting,
and showed that a half-integral multiflow maximizes  over all 
multiflows . Using an identical objective function to ours,
Karzanov~\cite{Karzanov94,Karzanov79} sought to
minimize the cost of multiflows.
His feasible multiflow solutions are those attaining 
in the edge-capacitated setting with ,
and he showed that the minimum cost is achieved by a half-integral multiflow.
Babenko and Karzanov~\cite{BabenkoK12} and Hirai~\cite{Hirai13} extended 
Karzanov's result to node-cost minimization in the node-capacitated setting.
In this scenario also, the optimal multiflow is half-integral.


In the present paper, we present a useful relationship 
between the generalized terminal backup problem
and
the minimum cost multiflow problem in the edge-capacitated setting.
We prove that the optimal solution of the LP
used to approximate the edge-connectivity terminal backup is a half-integral multiflow,
which also optimizes the minimum cost multiflow problem.
Thereby, we can compute the minimum cost half-integral multiflow by
solving the LP relaxation. This result is summarized in the following theorem.

\begin{theorem} \label{thm:flow}
The minimum cost multiflow problem admits a half-integral optimal solution in the edge-capacitated 
setting, which can be computed in strongly polynomial time.
\end{theorem}

In contrast, we find no useful relationship between the node-connectivity terminal backup
and the node-capacitated setting of the minimum cost multiflow problem.
We can only show that the LP relaxation of the node-connectivity terminal backup also
has an optimal solution which is a half-integral multiflow in the edge-capacitated setting.


Despite its natural formulation,
the minimum cost multiflow problem has not been previously investigated to our knowledge.
We emphasize that Theorem~\ref{thm:flow} cannot be derived from previously known results on multiflows.
The minimum cost multiflow problem
may be solvable 
by reducing it to minimum cost maximum multiflow problems that (as mentioned above)
admit polynomial-time algorithms.
A naive reduction can be implemented as follows.
Let  be a minimum cost multiflow that satisfies the flow demands from terminals, and let

for each . 
For each ,
we add a new node  and connect  and  by a new edge of capacity .
The new terminal set  is defined as .
Now the multiflow  can be extended to the multiflow of maximum flow value for the terminal
set . Applying
the algorithm in~\cite{Karzanov94} to this new instance,
we can solve the original problem.
Moreover, if  is an integer for each ,
this reduction together with the half-integrality result in \cite{Karzanov79,Karzanov94} implies that 
an optimal multiflow in the minimum cost multiflow problem is half-integral.
However, this naive reduction has two limitations. First,  is indeterminable
without computing . We only know that  cannot be smaller than .
Second, we cannot ascertain that  is always an integer for each .
Hence, this naive reduction
seems to yield neither a polynomial-time algorithm nor the half-integrality of optimal
multiflows claimed in Theorem~\ref{thm:flow}.

Applying a structural result in~\cite{Bernath2014} on the generalized terminal backup problem, 
it is easily shown that any integral solution to the edge-connectivity terminal backup
provides a half-integral multiflow at the same cost. 
However, since the way to find an optimal solution for the edge-connectivity terminal backup is
unknown, Theorem~\ref{thm:flow} is not derivable from this relationship.
In proving the half-integrality of the LP
relaxation required for Theorem~\ref{thm:main-4/3},
we immediately imply the quarter-integrality of a minimum cost multiflow (i.e.,
 for each 
). The proof of Theorem~\ref{thm:flow}
requires deeper investigation into the structure of half-integral LP solutions.

\subsection{Structure of this paper}

Section~\ref{sec:preliminaries} introduces notations and essential preliminaries on bisets.
Section~\ref{sec:characterization} proves that an LP relaxation of the generalized terminal backup
problem admits half-integral optimal solutions, and characterizes the edges assigned with half-integral
values.
Section~\ref{sec.algorithm} introduces our -approximation algorithm for the generalized terminal
backup problem, which proves Theorem~\ref{thm:main-4/3}.
Section~\ref{sec:half_integralty_of_minimum_cost_multiflow}
discusses relationship between the generalized terminal backup 
and the minimum cost multiflow problems with a proof of Theorem~\ref{thm:flow}.
Section~\ref{sec.conclusion} concludes the paper.




\section{Preliminaries} 
\label{sec:preliminaries}

\subsection{Bisets}
A biset  is defined as an ordered pair  of node sets  and  with 
. 
The former and latter elements are respectively called the \emph{inner part} and \emph{outer
part} of the biset. 
Throughout the paper, we denote the inner part of a biset  by , and the outer
part by . 
 is called the \emph{neighbor} of
, and is denoted by . 
 is the family of all bisets with nonempty inner parts of . 
For an edge set  and a biset , 
denotes the set of edges in  with one end node in  and the other in . 
We identify a node  with the biset . Thereby  denotes the set
of edges incident to  in .
For simplicity, we
write  as  when the edge set is unambiguously .
If an edge  is in , we say that  is \emph{incident} to .

For two bisets  and , we define  as ,  as , and  as 
. 
If  and , then we write . 
This inclusion relationship defines a partial order on the bisets, from which
we define the maximality and minimality among the bisets.

We say that  and  are 
\emph{strongly disjoint} when .
 If  and  are strongly disjoint,  and 
 .
 and  are called \emph{noncrossing} when strongly disjoint,
, 
or when .
Otherwise,  and  are called \emph{crossing}.
A family of bisets is called \emph{laminar} if each pair of bisets in the family is noncrossing.
The laminarity naturally defines a child-parent relationship among bisets (or a forest structure on
bisets). Let  be a laminar family of bisets in . If  satisfy  and , laminarity implies
that  or . 
Hence, each  admits a unique minimal biset  with 
unless  is maximal in . Such a biset  is defined as the \emph{parent} of 
, and  is a \emph{child} of 
. This child-parent relationship
naturally leads to terminologies such as ``ancestor'' and ``descendant.'' 
For a biset  in a laminar family  and an edge set , 
we let  
and  
 respectively denote 

and 
, where 
denotes the set of children of  in .
If  has no child, 
 and
.


\subsection{Bisets and connectivity of graphs}

For , let 

We denote  by .
For a vector  and , 
let  represent .
We define a biset function  by
 
for each .
According to the node-connectivity version of Menger's theorem, 
the graph  contains  inner-disjoint paths between  and 
if and only if
 
for each .
This condition is equivalent to 
for all .

In Section~\ref{sec:introduction},
we defined the set function  representing 
the edge-connectivity constraints.
For treating both node-connectivity and edge-connectivity simultaneously,
we sometimes extend
 to a biset function by
identifying  with the biset .
Specifically, the biset function  is defined by

for each .

Given a biset function  and 
an edge-capacity function
,
we define  as the set of 
 such that 

and 


Let  be a multiset of edges in , and  denote the characteristic vector of 
(i.e.,  and  contains  copies of  for each ).
Note that  for .
Hence, 
if and only if  is a feasible solution
to the node-connectivity terminal backup.
Similarly, 
if and only if  is a feasible solution 
to the edge-connectivity terminal backup.
These statements imply that the LP 
 
relaxes the node-connectivity and the edge-connectivity terminal backups 
when  and , respectively.


A biset function  is called \emph{{\rm (}positively{\rm )} skew supermodular} when, 
for any  with  and  with , 
 satisfies 

or

For any biset function  and a vector
,
we let  denote the biset function such that
 for each .
The skew supermodularity of  was reported by Bern{\'a}th and
Kobayashi~\cite{Bernath2014}. Here, we prove that  is also skew supermodular.

\begin{theorem}\label{thm:skewsupermodular}
The biset function  is skew supermodular for any .
\end{theorem}
\begin{proof}
Let  and  be two bisets.
 and  are known to always satisfy
,
,
,
and
.
These inequalities can be proven by counting contributions of edges on both sides.

Suppose that  and .
Then .
If  for some , then both  and 
 belong to .
From this statement and the above inequalities, we have 
 in this case.
If  and  for some  with , 
then  and .
In this case, we have 
.
\end{proof}





\section{Structure of extreme point solutions}
\label{sec:characterization}

In this section, we present the properties of the extreme points of  and .
More precisely, we prove that each extreme point of  and  is half-integral,
and that the edges whose corresponding variables are not integers are characteristically structured.
Note that both  and  are integer-valued skew supermodular functions, and 

for any . 
In the following, we denote an integer-valued skew supermodular function by ,
and an extreme point of  by .


\subsection{Half-integrality} 

Given an edge set  on  and , let  denote the characteristic
vector of ,
i.e., an -dimensional vector whose components are set to  if indexed by
an edge in , and  otherwise.
The following lemma has been previously proposed~\cite{Cheriyan2006,FleischerJW06}.

\begin{lemma}\label{lem.terminal-laminar}
Let  be a skew supermodular biset function, 
and  be an extreme point of .
Let ,
, and
. 
Let  be an inclusion-wise maximal laminar subfamily
of 
such that the vectors in 
are linearly independent. Then
,
and  is a unique vector that satisfies
 for each ,
  for each , and  for each .
 Moreover, if some   satisfies
 ,
 then  is represented as a convex combination of vectors 
 ,  .
\end{lemma}

We note that  in Lemma~\ref{lem.terminal-laminar} can be
constructed from the extreme point solution  in a greedy
way;
initialize  to an empty set, and repeatedly add a biset 
such that ,  is linearly independent from the 
vectors defined from the bisets in the current , and adding  to  preserves
laminarity of .
Hereafter, we assume that  is constructed as claimed in Lemma~\ref{lem.terminal-laminar}.
Similarly, 
, , and  are defined from  as in Lemma~\ref{lem.terminal-laminar}.

Let ,
and define a biset function 
 for .
Let  denote the -dimensional all-one vector.
The following lemma relates only to the extreme points of .
In Corollary~\ref{cor.half-integrality}, we will show
that this is sufficient for proving the half-integrality of .
If  holds only for , 
we have . In this case, no biset in  
has more than one child, and  is characterized as follows.

\begin{lemma}\label{lem.terminal-characterization}
Suppose that  is an integer-valued skew supermodular biset function 
such that  only for .
Let ,
and let  be an extreme point of .
Let  denote .
 Then the following conditions hold\/{\rm :}
 \begin{enumerate}
  	\item[\rm (i)]  
	     for each \/{\rm ;}
  	\item[\rm (ii)] If  is incident to a maximal biset in , then it is incident to
  	exactly two maximal bisets in \/{\rm ;}
  	\item[\rm (iii)]  for each .
 \end{enumerate}
\end{lemma}

\begin{proof}
We first prove (i) and (ii) by contradiction. Let us assume that not all of these
conditions hold. For each pair of  and its end node , we distribute a token
to a biset in . The biset that obtains the token corresponding to  is decided
as follows:
\begin{itemize}
 \item If there exist one or more bisets  such that 
	 and , the token is assigned to the minimal of these bisets.
\item Otherwise, the token is assigned to the minimal biset  that 
includes both end nodes of  in its outer part (if such a biset exists).
	Notice that such a minimal biset is unique because  is laminar and  is incident to at
	least one biset in .
\end{itemize}

The total number of tokens is at most . In the following, 
we prove that tokens may be rearranged so that each
biset in  receives at least two tokens and at least one biset receives three tokens. 
This rearrangement implies that the number of tokens exceeds
, contradicting our requirement that .

Recall that .
Let  denote , and
let  be a minimal biset in . 
The minimality of  implies 
and .
Since  and  for each 
, we have 
. 
Since each edge in  allocates one token to , 
obtains at least two tokens. If  violates (i), 
then , and
 obtains at least three tokens.

Next, let  be a biset in  that admits a child .
Since  and  are linearly independent,
.
Therefore,
if , then
 and .
If , then 
 because , 
. Similarly,
if , then 
.
In summary, either case yields
.
Since  receives a token from
each edge in , it obtains
at least two tokens and 
at least three tokens if condition (i) is violated.

Extending the above discussion, each biset in  obtains at least two tokens, implying that the number of tokens is at least . If (i) is violated
for any biset in ,
that biset receives more than two tokens. Now suppose that (ii) is violated. Then 
there exists an edge  incident to exactly one maximal biset  in . 
The relation  indicates that  has an end node ,
and the token corresponding to  is assigned to no biset in .
Therefore, if either (i) or (ii) is violated, the number of tokens exceeds the required .

Let  be the vector with components  for each , and 
for each .
Let , and denote the child of  (if it exists) by .
From the above discussion, we obtain the following statements:
\begin{itemize}
\item  and  if  is minimal;
\item  if
 is not minimal and ;
\item 
,  
and 
if
 is not minimal and ;
\item 
, , 
and 
if
 is not minimal and .
\end{itemize}
Therefore,  satisfies  
for each . 
Since this condition is also uniquely satisfied by vector ,
we have , which proves (iii).
\end{proof}

\begin{corollary}\label{cor.half-integrality}
Suppose that  is a skew supermodular biset function such that  
only if . Let .
Given , we define 
 and  
by  
and , respectively for each .
If  is an extreme point of ,
then  is an extreme point of .
Moreover,  is half-integral if  is integer-valued.
\end{corollary}
\begin{proof}
Note that  for 
and 
for .
Hence, .
In the following, we show that  is an extreme point of 
if  is an extreme point of .
This proves that  is half-integral because
 is half-integral by Lemma~\ref{lem.terminal-characterization}.

If  is not an extreme point of 
, there exist  
and a real number 
such that  and .
Then, .
Note that both of  and  are contained in ,
implying that  is not an extreme point of .
\end{proof}

\subsection{Path decompositions of extreme point solutions} 
\label{subsec.path-decomp}

We denote  by  for each .
Let  with ,
and let  be the maximal biset in . 
We obtain a graph  
from  by shrinking all the nodes in  into a single node .
Removing  from ,
we obtain another graph 
(i.e.,  is the subgraph of  induced by ).
We suppose that each edge  in 
 or in  is capacitated by .
If , 
each node  in  except  and  has
unit capacity. When ,
each node has unbounded capacity.
The capacities of  and  are always unbounded.
Since all capacities are half-integral, the maximum flow 
between  and  in 
can be decomposed into a set of paths 
 each of which accommodates a half unit of flow. 

Let .
Each path between  and  
passes through an edge in  or a node in .
Since , the edges in  
and nodes in  are used to full capacity by the maximum flow,
and each path  includes exactly one edge in 
or one node in .


Suppose that both  and  include a node .
Let  and  be the edges incident to  on , where
 is near to  than .
We define the edges  and  incident to  on , similarly.
We assume that the following fact holds for any such paths  and .

  \begin{assumption}
   \label{assump:stay-half}
   If  is half-integral and  is an integer,
   and if exactly one of  and  is half-integral,
   then  is half-integral.
  \end{assumption}

Indeed, if Assumption~\ref{assump:stay-half} does not hold, then 
exchanging the subpaths between  and  makes them satisfy it.

In the following discussion, we consider a maximum flow between a terminal  and  in , where
 may equal . In such a flow, each edge  is capacitated by , and each node 
is assigned the unit capacity or an unbounded capacity if  or , respectively. The capacities of the
terminals are assumed as unbounded. The flow quantity for each  is at least  if and only if 
satisfies \eqref{eq:t-cut}. Let  be a path decomposition of the
flow between  and , in which each path in 
accommodates a half unit of flow. Let  be the set of paths in  that contain nodes in 
(recall that  is the maximal biset in ). Without loss of generality, we can state the
following fact.

\begin{assumption}\label{assump:path-decomposition}
 Each path in  ends at .
 For a path , let  be the subpath of  between  and the nearest node in
 . 
 Then,  holds.
\end{assumption}

If Assumption~\ref{assump:path-decomposition} is 
not satisfied by , we can modify the flow between 
and 
by replacing the subpaths of those in
 by appropriate paths in , without decreasing the amount of flow.


We say that  is \emph{minimal in }
if  and
no 
exists such that  and  for any .
Let edge  be incident to a node in .
If  is minimal in ,
then ;
Otherwise,
as  is decreased, it would remain in .



\begin{lemma}\label{lem.degree}
Suppose that  or , and 
let  be an extreme minimal point in . Then
 is an integer for each .
\end{lemma}
 \begin{proof}
  Define  and 
from  as in Corollary~\ref{cor.half-integrality},
and define sets  and  
for  and 
as in Lemma~\ref{lem.terminal-laminar}.
In other words,
,
and  is a maximal laminar subfamily of 

(because  for )
such that the vectors in 
are linearly independent.
It suffices to show that  is even for each .

Let  be a node with . 
We first observe that  is included by the outer part of some biset in .
Let .
There exists some  with ; otherwise 
a slight decrease in  retains  in .
Let  be the maximal biset such that .
If , then (ii) of Lemma~\ref{lem.terminal-characterization} implies
the existence of another biset  with ,
where  satisfies .

We now prove that  is even. First, we consider the case of .
The laminarity of  permits two cases: 
(i) the existence of maximal bisets 
with ,
and (ii) 
the existence of exactly one maximal biset  with
.

First, we consider the case (i).
In the following discussion,
we show that an even number of edges in  remains in 
for each .
Each edge  is associated with
exactly one biset  that includes the both end nodes of  in its outer
part. 
 remains in , and does not remain in  
for any  with .
Therefore the claim proves that  is even.
Denote by  the terminal with .
Note that
 is included in exactly two paths in ,
say  and . 
 is adjacent to  in  and . 
For , let  be the edge that joins  to the neighbor opposite  in . 
If , then , and  has no incident edge in  remaining in .
If , then . Among the edges in  remaining in ,
these edges alone are incident to .
Hence, the number of edges in  remaining in 
is zero or two.

We now discuss case (ii).
Let  be the terminal with .
By laminarity of , no biset 
in  includes  in its outer part. Hence, 
it suffices to show that an even number of edges in 
 remains in .
At most two paths in  pass through , but 
if no biset in  includes  in its neighbor,
 may not be used to full capacity.
However, each edge in  is used to full capacity by the minimality of .
If , then , and 
is an integer. If , then , and  is again an integer.
In either case,  is even, which completes the proof for .

The lemma can be similarly proven for .
Case (i) does not occur because 
for each .
 \end{proof}






\section{-approximation algorithm for the generalized terminal backup problem}
\label{sec.algorithm}

In this section, we prove Theorem~\ref{thm:main-4/3} by presenting
a -approximation algorithm for the generalized terminal backup
problem.
We first explain how our algorithm works for the case of  
for smooth understanding.
Then, we present a full proof of Theorem~\ref{thm:main-4/3}.


\subsection{Algorithm for case of }


Our algorithm rounds a half-integral optimal solution to the LP
relaxations into an integer solution.
Let us assume that a minimal half-integral optimal solution  and a
laminar biset family  in Lemma~\ref{lem.terminal-laminar}
are given.
In what follows, we explain how to round .

When , the edge- and node-connectivity are equivalent.
Since the neighbor of each biset in  is empty,
we identify  with a family of subsets of .

Let  denote .
We call the edges in  \emph{half-integral edges}.
 is even for each 
because  is an integer by Lemma~\ref{lem.degree}.
Hence  can be decomposed into an edge-disjoint set of cycles.
Let  be a cycle in the decomposition.

For each ,  contains a node set to which  is
incident.
Let  be the subset of  that consists of the node sets
to which edges in  are incident.
Since ,
 exactly two edges in  are incident to
each node set in .

Let  be the terminals such that  for each .
We can prove that  is an odd number larger than one.
For each ,
let  denote the maximal node set in
,
and let  be the subpath of  comprising of edges incident to node
sets in .
If an edge is incident to both  and , the edge is
shared by  and .

Let  be an edge incident to , where we assume without loss
of generality that  and  .
Consider traversing , starting from  in the direction from
 to . 
We say that  \emph{appears} when
we traverse an edge incident to two node sets  and  with  in the direction from the end node in
 to the one in .
Without loss of generality, we assume that the terminals appear in the increasing order
of subscripts.
Therefore, during the traverse of , we first visit edges in ,
then those in , and so on.
Suppose that  and .
We say that  is \emph{outward} with respect to  if  is
traversed from the end node in  to the other.
Otherwise,  is called \emph{inward}.
This implies that, during the traverse of ,
we first traverse edges inward with respect to , and then those
outward with respect to .

We define  assignments of labels to the edges in , 
where each edge is labeled by either ``'' or ``.''
Let us define the -th assignment.
If , then  is labeled by ``.'' If  for
some , then its label is decided by the following rules.
\begin{itemize}
\item If  is odd and  is outward with respect to ,
 is labeled by ``.''
\item If  is odd and  is inward with respect to ,
 is labeled by``.''
\item If  is even and  is outward with respect to ,
 is labeled by ``.''
\item If  is even and  is inward with respect to , 
 is labeled by``.''
\end{itemize}
If  for some , we assign the opposite label to the
above rules; For example, if
 is odd and  is outward with respect to ,
 is labeled by ``.''

Note that this assignment is consistent;
if
 is included in both  and , then
 is outward with respect to  and inward with respect to
,
and hence  is assigned the same label from  and ;
 is shared by  and , and similarly it is assigned the same label 
because  is odd.
Figure~\ref{fig.cicle2} shows an example of the cycle , and the first
assignment of labels to the edges on .

\begin{figure}
 \centering
 \includegraphics[scale=.8]{fig-cycle2.pdf}
\caption{An example of a cycle of half-integral edges and the first
 assignment of labels to the edges. 
Edges drawn by 
solid and dashed lines are assigned ``'' and ``,'' respectively.
The edges are oriented in the direction of traverse. The areas surrounded by thin solid lines
represent the node sets in .}
\label{fig.cicle2}
\end{figure}


Our algorithm rounds  into  if  is labeled by ``,''
and into  otherwise. 
Since we have  assignments of labels, we have  ways of rounding of
variables corresponding to the edges in .
Our algorithm chooses the most cost-effective one among them.

Let us observe that this algorithm is -approximation. First, we
prove that the above rounding increases the cost by a factor of at most .
Let  be the vector obtained from  by the rounding.

\begin{lemma}\label{lem.cost-terminal}

\end{lemma}
\begin{proof}
 Let  be a cycle of half-integral edges.
 We show that .
 Applying this claim to all cycles in the decomposition of , we
 can prove the lemma.
 We use the notations used in the definition of the rounding.
 
 Let  denote the vector obtained by rounding , 
according to the -th assignment of labels.
We note that 
 

Recall that  is an odd number larger than one.
In the  assignments,  is 
labeled ``'' by the  assignments.
Thus, 

Note that . Therefore,

where the last inequality follows from .
\end{proof}

Next, let us prove the feasibility of .
For a path  and nodes  on , we denote the subpath of  between
 and  by .

\begin{lemma}\label{lem.feasibility-easy}
  is a feasible solution to the terminal backup problem.
\end{lemma}
 \begin{proof}
  Obviously  is an integer vector. Hence, to prove the feasibility
  of , the graph with edge-capacities 
  admits a unit of flow from each terminal  to the other terminals.
  Since  for each ,
  the graph capacitated by  admits such a flow.
  Hence we show that a flow for  can be obtained by
  modifying the flow for .
  In the following, we assume that  is obtained by rounding variables
  corresponding to the half-integral edges in a cycle .
  If required, the modification is repeated for each cycle of half-integral edges.

  Recall the definition of  in Section~\ref{subsec.path-decomp}.
  Since we are considering the case of , we have two paths  and
   for each terminal  with .
  We assume these paths satisfy Assumption~\ref{assump:stay-half}.
  Fix a terminal , and 
  suppose that the flow from  to the other terminals
  with edge-capacities  delivers a half unit of flow along a path ,
  and another half unit along a path .
  We assume that  satisfies Assumption~\ref{assump:path-decomposition}.

  If both  and  contains no half-integral edge (with respect to ) labeled by
  ``,''
  the flow satisfies the capacity constraints defined from .
  Thus, let us consider the case where  includes a half-integral edge labeled
  by ``.''
    Let  be the one nearest to  among such edges, and let  be
  the end node of  near to .

  We first show that there exists  such that  and . For arriving at a contradiction,
  suppose that such  does not exist.
   is incident to at least one node set in .
  In particular, Lemma~\ref{lem.terminal-characterization}(ii)
  implies that there exists a terminal  and node set  such that  and
  .
  However, this means that  and  enters  when traversed from
   to .
  Assumption~\ref{assump:path-decomposition} indicates that
  the subpath of  between  and the end opposite to  is included
  by
   or . Hence,
  the end of  opposite to  is , and  does not include
  , which is a contradiction. Therefore,
  there exists  such that  and .

  This fact indicates that  contains no ``''-labeled half-integral
  edge because of the following reason.
  Let  be the subpath of  that is included by a maximal node set
  in .
  Since , there exists  and .
  By Assumption~\ref{assump:path-decomposition},
   is equal to  or . Without loss of generality, let  be
  equal to .
  Then, Assumption~\ref{assump:stay-half} indicates that 
  all ``''-labeled half-integral edges incident to node sets in
    is included in .
  Since  and  share no half-integral edges,  does not include
  these edges in .
  Hence, if  contains a ``''-labeled half-integral edge, its both
  end node is included by some node sets in .
  However, we can derive a contradiction similarly for the above claim
  with .

  Since ,
  the other edge  incident to  on  is also incident to .
  By the label-assignment rules,  is labeled by ``.''
  Let  denote the subpath of
   consisting of ``''-labeled edges and terminating at .
  Let  be the other end node of ,
  and let  be the edge incident to  on .
  By Lemma~\ref{lem.terminal-characterization}, there exists  with
   and  .
   belongs to  for some .
   is included in a path  or .
  Without loss of generality, we suppose that  includes .
  We replace  by the concatenate of , , and .
  See Figure~\ref{fig.modification} for illustration of this modification.
  
  \begin{figure}
  \centering
   \includegraphics[scale=.8]{fig-modification.pdf}
   \caption{The definitions in the proof of Lemma~\ref{lem.feasibility-easy}}
   \label{fig.modification}
  \end{figure}
  
  Let us observe that this modification preserves the capacity
  constraints.
   was a part of  before the modification.
  The capacity of each edge on  is increased by  when 
  replaces .
  The capacity of each edge in  is integer.
  Hence no capacity constraint is violated.
 \end{proof}


\subsection{Algorithm for the general case}
In this subsection, we present a strongly polynomial-time algorithm for the generalized terminal
backup problem.
In the following discussion,
 denotes a skew supermodular function such that 
only when .

\subsubsection*{Solving the LP relaxation}
We wish to ensure that any optimal solution  to  
is minimal in . Clearly, this condition holds when  for each .
If  for some , the condition is ensured by perturbing .
Since we can restrict our attention to half-integral solutions,
it is sufficient to reset  
to a positive number smaller than  for each  with ,
where  is the maximum denominator of the edge costs.

The number of constraints of  is exponential;
hence, it is unclear how to solve  in polynomial time.
If  or ,
the separation is reducible to
a maximum flow computation, and  can be solved by the ellipsoid method.
Alternatively, the constraints can be written in a compact form by introducing flow variables for
each terminal, as implemented in Jain~\cite{Jain01}.
Hence, if  or , there are two ways of solving
 in polynomial time.
However, Theorem~\ref{thm:main-4/3} claims a strongly polynomial-time algorithm.
All coefficients in the constraints of  are one. Accordingly, 
Tardos' algorithm~\cite{Tardos1986} computes
an optimal solution to  in strongly polynomial time,
but does not guarantee an extreme point solution.

Our algorithm first finds an optimal solution to  by Tardos' algorithm.
The obtained solution is denoted by .
Defining  by  for , we then compute an extreme point optimal solution  to .
 is not necessarily an extreme point of ,
but is a half-integral optimal solution to .
The following lemma shows that 
 can be computed by iterating Tardos' algorithm.

\begin{lemma}\label{lem.lp-stronglypoly}
An extreme point optimal solution to  
can be computed in strongly polynomial time.
\end{lemma}
 \begin{proof}
  As noted above, 
an optimal solution to  
can be computed in strongly polynomial time.
Moreover, 
whether fixing a variable  to a specific value 
increases the optimal value is also testable in strongly polynomial time 
by solving  
with an additional constraint . 
We sequentially test fixing the variables  to  or ,
and if the fix does not increase the optimal value, the variable is set to the fixed value.
If  is not fixed to  or ,
it is set to .

Optimality of the above-constructed solution  
follows from the existence of a half-integral optimal
solution (see Lemma~\ref{lem.terminal-characterization}).
We must now prove that the obtained solution  is an extreme point.
If not,  can be represented by
, where ,
 are extreme points of ,
and  are positive real numbers with
.
Let .
The optimality of  indicates that
 is an optimal solution to 
.
Moreover,   holds if .
Therefore, there exists some  such that  and ,
which contradicts the way of constructing .
 \end{proof}

 Let .
 Our algorithm also requires  defined from  (i.e.,  is
 a maximal laminar subfamily of 
such that the vectors , 
 are linearly independent).
 As stated in the paragraph following Lemma~\ref{lem.terminal-laminar},
  can be constructed by
 repeatedly adding a
 biset  in 
 such that adding  to  preserves the laminarity of
 
 and the linear independence of the vectors 
 , .
 If  is not maximal, such a biset  
 can be found as follows.
 By Lemma~\ref{lem.terminal-characterization}, 
 one of such  satisfies either of the following conditions:
 \begin{itemize}
  \item[(i)]   is minimal in , and ;
  \item[(ii)] There exits  such that
	 and
	.
 \end{itemize}
 The number of bisets satisfying one of these conditions is strongly
 polynomial.
We can decide in strongly polynomial time whether adding
a biset to the current  preserves the conditions of . 
Therefore,  can computed in strongly polynomial time.



\subsubsection*{Rounding half-integral solutions to -approximate solutions}
Our algorithm rounds , the extreme point optimal solution
to ,
to an integer vector 
subject to .
It then outputs .

The rounding procedure
is almost same as the algorithm for .
Let .
By Lemma~\ref{lem.degree},  is even for each  because
 is minimal in .
We can see that  is an even number at most four.

\begin{lemma}
  for each .
 If , there exist 
 such that , ,
 and .
\end{lemma}
\begin{proof}
 Let . Then,
 Lemma~\ref{lem.terminal-characterization} (ii) implies that
  is included in the inner-part of some biset in .
 Let  be the minimal biset in  such that .
 If  is minimal in ,
 then , and
  follows from .
 In the rest of the proof, suppose that
   has the child .
 Then .
 Suppose that , and let  be the minimal biset in
 
 such that  and ,
 where  is
 possibly equal to .
 Let  be the child of .
 Each edge   is incident to  or .
 Notice that  is not incident to  or .
 Hence,
  if  is incident to , 
 and  if  is incident to .
 Thus .
 If  does not exist,
 .
 If , we can similarly show that , and hence
  follows from
 .
\end{proof}

We decompose  into a set of cycles.
We assume without loss of generality that the decomposition satisfies
the following assumption.

 \begin{assumption}
  \label{assump.cycle}
 Let  be a node such that . Let  be the bisets such that , , and
 .
  Then the two edges in 
  {\rm (}resp., {\rm )}
  are included in the same cycle in the decomposition.
   \end{assumption}

  Suppose that  includes an edge incident to a biset in
   and another in  for some terminals
   with .
   Let  be one of such edges.
   We traverse , starting from .
   Suppose that  is traversed from a biset in  to one in
   .
   Let  be the
   sequence of terminals that appear when we traverse  from ,
   where  denotes the terminal that appears immediately after .
   A different fact from the case of  is that
   a terminal can appear more than once during the traverse.
   Thus  and  may stand for the same terminal unless  or .

\begin{figure}
\centering
\includegraphics[scale=.6]{fig-cycle.pdf}
\caption{An example of a cycle of half-integral edges and labels assigned to the edges. 
Edges drawn by 
solid and dashed lines are assigned ``'' and ``,'' respectively.
The edges are oriented in the direction of traverse. The areas surrounded by thin solid lines
represent the outer parts of bisets in , and gray areas indicate their neighbors.
In this figure, neighbors of bisets in  are disjoint for visibility, but neighbors can overlap
in general.}
\label{fig.cicle}
\end{figure}

Let  be the subpath of  that consists of edges
between the appearance of  and , where
 and  share an edge that is incident to both a biset in
 and one in ,
and  and  share .
We also define ``inward'' and ``outward'' edges in  with respect to

as in the case of .
Another different fact in the general case from the case of  is
that the direction of edges on  with respect to  changes more
than once because  may contain more than two edges incident to a biset in .

If all edges on  are incident to only bisets in  for some
terminal , we let , and  for convention.
In the following lemma, we see that  is an odd number larger than one.

\begin{lemma}\label{lem.even}
A cycle such that  is one or an even number does not exist.
\end{lemma}
\begin{proof}
Suppose that  is one or an even number for a cycle .
 Let us assign labels to each edge in  as follows.
 Let .
 If  is odd and  is inward to , or if
  is even and  is outward to , 
 then  is labeled ``.''
 Otherwise,  is labeled ``.''
 We note that, for each ,
exactly half of the edges  in 
are labeled by ``.''

 Let  be a constant. 
 For each edge  in ,
 update the corresponding variable  to 
 if  is labeled by ``'',
 and update to  otherwise.
Let  denote the obtained vector.
The number of labels assigned indicates that 
 
 for each .
 If
  holds for a biset
 ,
 
is implied by the linear dependence of   from
 , , shown in Lemma~\ref{lem.terminal-laminar}.
Therefore, both  and 
 belong to 
 for a sufficiently small positive number
,
contradicting that  is an extreme point of .
\end{proof}

Since  by Lemma~\ref{lem.even}, we can choose  so that
. We assume this condition in the rest of this section.

We define  assignments of labels ``'' and ``''
to the edges on  as in the case of .
 Figure~\ref{fig.cicle} illustrates 
 a cycle of half-integral edges and the first assignment of labels 
 to its edges. In this example, , and  and  indicate the same terminal.


Our algorithm computes an integer vector  from  as follows.
For each cycle  of half-integral edges,
the algorithm selects the most cost-effective choice from  assignments of labels.
Based on the labels,
 is rounded to obtain the vector ;
If an edge  is labeled by ``'',   is defined as .
Otherwise,  is .
Recall that the algorithm outputs .



\subsubsection*{Performance guarantee}

We can prove

similarly for Lemma~\ref{lem.cost-terminal}.
The next lemma proves that  is a feasible solution.
Theorem~\ref{thm:main-4/3} is immediately proven from
these facts and Lemmas~\ref{lem.lp-stronglypoly}.



\begin{lemma}\label{lem.feasibility}
  when  or .
\end{lemma}
\begin{proof}
Consider the case of .
Assume that 
nodes in  have unit capacities
and nodes in  have unbounded capacities.
We also regard 
 and  as edge capacities.
To prove that , 
it suffices to show that, for each ,
the graph capacitated by  admits
a flow of amount 
between  and .

Now consider a maximum flow between  and 
in the graph capacitated by .
Suppose that the maximum flow is decomposed into a set 
 of paths, each running a half unit of flow from  to another terminal.
Since  satisfies 
for each ,
the flow amount is at least  (i.e., ).
Recall that we are assuming Assumption~\ref{assump:path-decomposition}.
We now modify  to satisfy the capacity constraints when 
the capacity of  is changed from  to .
In the following, we assume that  is obtained by rounding variables corresponding to 
the half-integral edges in a cycle . If required, the modification is repeated for each cycle
of half-integral edges.
We define the notations such as  and 
 from  as we defined above.


We traverse  from  to the other end.
When arriving at an edge  labeled by ``,''
we reroute the flow along  as follows.
Let  be the end node of  near to .
By Assumption~\ref{assump:path-decomposition} and the label-assignment rules,
 shares node  with an edge labeled ``'' on .
Let  denote the subpath of  consisting of ``''-labeled edges and terminating at .
We follow  instead of .
Let  be the other end node of ,
and let  be the edge incident to  on .
By Lemma~\ref{lem.terminal-characterization}, there exists  with .

\begin{figure}
\centering
\includegraphics[scale=.7]{fig-flow.pdf}
\caption{Transformation of  in the proof of Lemma~\ref{lem.feasibility}. The left and right
panels illustrate the cases of  and , respectively, with . The paths  and  are represented by dark gray lines;
the black lines represent the paths obtained by modifying  and .}
\label{fig.flow-transformation}
\end{figure}


Suppose that . 
Let  be the minimal biset such that  and , and
let  be the child of .
Then, , and .
Moreover, another half-integral edge , labeled ``,'' is incident to
.
Edge  is included in another path .
Let  be the terminal such that  and .
After reaching , we move to  along the path .
In other words,
path  is replaced by the concatenate of ,
, and .
If  contains a half-integral edge labeled by ``'', we modified it recursively.
These definitions are illustrated in the left panel of Figure~\ref{fig.flow-transformation}.
Let us observe that this modification does not violate 
the capacity constraints when the edges are capacitated by .
 Assumption~\ref{assump.cycle}
 indicates that
 exactly two half-integral edges are incident to each inner node on .
 The capacity of each edge on  increases by  by the modification,
 exactly counterbalancing the unused half capacity
of each inner node on  prior to the modification.
 Even if the capacities of edges and nodes on
  are used before the modification,
 the flow along  is modified so that these capacities are unused.
 Thus the capacity constraints are preserved by the
 modification.

 Next, suppose that  for some  with .
 We first consider the case of .
 Lemma~\ref{lem.terminal-characterization}(ii) indicates that
 we can assume .
 We let  be the minimal among such bisets.
Another half-integral edge , labeled by ``,'' is incident to ,
and is included in a path in .
Without loss of generality, we suppose that that  is such a path.
After arriving at , we reach  along , 
as shown in the right panel of Figure~\ref{fig.flow-transformation}.
Again, this modification preserves the capacity constraints. To see this,
suppose that another path  includes .
Then,
 includes a ``''-labeled edge before reaching  when traversed from  to .
 will be diverted to another route, and 
half of the edge and node capacity on  will be no longer used.
Prior to modification,
half of the inner node capacity of  was unused because
the nodes were incident to exactly two half-integral edges.

 We next discuss the case of .
 In this case, .
Recall that all edges in  are labeled by ``.''
Each of  and  shares exactly one edge with . We let 
 denote the edge shared by  and , and  denote the one
 shared by  and .
 and  are traversed inward and outward with respect to , respectively.
If , we modify each path in 
as when each outward-traversed edge in  is labeled ``,'' whereas other edges are
labeled ``.''
If , we perform the converse operation, implemented when
each outward-traversed edge in  is labeled ``,'' whereas other edges are labeled ``.''
 The modification when  is illustrated in Figure~\ref{fig.flow2}.
 Recall that we chosen  so that .
The capacity constraints are preserved because
no path in  includes  when , and 
no path in  includes  if 
before the modification.

\begin{figure}\centering
\includegraphics{fig-flow2.pdf}
\caption{Modification of  when . Gray thick lines represent paths before the modification,
and black lines represent those after the modification.}\label{fig.flow2}
\end{figure}


These transformations generate a flow of amount  from  to 
in the graph capacitated by .
This indicates that .  
Assigning unbounded capacity to each node in ,
a similar proof can be derived for .
\end{proof}




\section{Relationship between terminal backup and multiflow}
\label{sec:half_integralty_of_minimum_cost_multiflow}

In this section, we limit the constraints on the generalized terminal backup problem
to the edge-connectivity constraints, unless otherwise stated.
Furthermore, our discussion of multiflows assumes
that edges alone are capacitated.
Let  denote the set of paths connecting distinct terminals,
and assume that the capacity constraints and flow demands are satisfied by 
a multiflow ,
i.e.,  for each 
and  for each .
We call a vector (or a function) 
\emph{-fractional} if each entry multiplied by  is an integer.

In this section,
we answer the question: to what extent 
the edge-connectivity terminal backup differs from the 
minimum cost multiflow problem in the edge-capacitated setting?
The differences are small, as demonstrated below.

\begin{lemma}\label{lem.fractionality}
For each -fractional multiflow, 
there exists a -fractional vector of the same cost in .
For each -fractional vector , where
 is minimal in  and 
 is -fractional for each , there exists a -fractional
multiflow 
such that .
\end{lemma}

The former part of Lemma~\ref{lem.fractionality} is straightforward to prove;
if  is a -fractional multiflow,
then  defined by 
is -fractional and belongs to .

To prove the latter part, we use a graph operation called \emph{splitting off}.
Let  and  be two edges incident to the same node .
\emph{Splitting off  and } replaces both  and  by a new edge .
In this section, we regard  as a set function.
To avoid confusion, we denote  defined from  by .
Let  be an edge set on  such that 

We say that a pair of edges in  incident to the same node is \emph{admissible} (with respect
to )
when \eqref{eq:admissible} holds
after splitting off the edges.

\begin{lemma}\label{lem.admissible-splitting}
Let  be an edge set on  that satisfies \eqref{eq:admissible},
and let  be a node in  with .
Then  includes 
an admissible pair with respect to  or 
\eqref{eq:admissible} holds even after an edge is removed from .
\end{lemma}

Lemma~\ref{lem.admissible-splitting} derives from a theorem in~\cite{Nutov2009,Bernath2012},
which gave a condition for admissible pairs in a more general setting.
Bern{\'a}th and Kobayashi~\cite{Bernath2014} 
proved an almost identical claim when discussing the degree-specified version
of the edge-connectivity terminal backup, but did not explicitly specify 
the condition under which admissible pairs can exist.
For completeness, we provide a proof of
Lemma~\ref{lem.admissible-splitting} in Appendix~\ref{sec:splitting}.


{\em Proof of Lemma~\ref{lem.fractionality}.}
The former part of Lemma~\ref{lem.fractionality} has been proven above.
Here, we concentrate on the latter part.
Since  is -fractional,  for each .
Let  be the set of  edges parallel to  for each .
Since  for each ,
 satisfies 

Let . Since  is -fractional,  is an even integer. 
By the minimality of ,
no edge can be removed from  without violating
\eqref{eq:cut-J}.
Hence, by Lemma~\ref{lem.admissible-splitting},  includes an admissible pair
with respect to .
For each ,
we repeatedly split off admissible pairs of edges incident to 
until no edge is incident to .
The graph at the end of this process is denoted by .
In , no edge is incident to nodes in , and 
at least  edges join  to other terminals.
An edge joining terminals  and  in  is generated by
splitting off edges on a path between  and  in .
In other words, edges in  correspond to edge-disjoint -paths in .
By pushing a  unit of flow along each of these -paths in ,
we obtain the required multiflow.
\qquad\endproof

We see that Theorem~\ref{thm:flow} follows from
Lemma~\ref{lem.fractionality} and the properties of  
described in Section~\ref{sec:characterization}.

{\em Proof of Theorem~\ref{thm:flow}.}
The former part of Lemma~\ref{lem.fractionality} 
implies that  relaxes the minimum cost multiflow problem.
As proven in Corollary~\ref{cor.half-integrality},
 admits a half-integral optimal solution . 
This solution can be computed in strongly polynomial time
and is guaranteed minimal in , as shown in Section~\ref{sec.algorithm}.
By Lemma~\ref{lem.degree},  is integer-valued for each .
Hence, the latter part of Lemma~\ref{lem.fractionality} implies that 
there exists a half-integral multiflow 
such that .
Note that , and therefore  minimizes
the cost among all feasible multiflows.

How  should be computed from  in strongly polynomial time is unknown.
However, because , 
 can be computed for each .
Moreover,  is an integer for each .
Therefore, as explained in Section~\ref{subsec.intro-multiflow}, this problem reduces to 
minimizing the cost of maximum multiflow,
for which a strongly polynomial-time algorithm is known~\cite{Karzanov94}.
\qquad\endproof

Each vector  belongs to . Hence,
we can show that each minimal extreme point of 
admits a half-integral multiflow of the same cost which is feasible in the edge-capacitated setting.
However we cannot relate extreme points of  to feasible multiflows 
in the node-capacitated setting as we observed for star graphs in Section~\ref{subsec.intro-multiflow}.



\section{Conclusion}
\label{sec.conclusion}

We have presented -approximation algorithms for the generalized terminal backup problem.
Our result also implies that the integrality gaps of 
the LP relaxations are at most .
These gaps are tight even in 
the edge cover problem (i.e.,  and ): 
Consider an instance in which  is a triangle with unit edge costs;
The half-integral solution  with
 for all  is feasible to the LPs, and its cost is ;
On the other hand, any integer solution chooses at least two edges from the triangle; 
Since the costs of these integer solutions are at least , the integrality gap is not smaller than  in this instance.


An obvious open problem is whether the generalized terminal
backup problem admits polynomial-time exact algorithms or not. 
It seems hard to obtain such an algorithm by rounding
solutions of the LP relaxations because of their integrality gaps.
For the capacitated -edge cover problem,
an LP relaxation of integrality gap one is known~\cite{schrijver-book}. 
For obtaining an LP-based polynomial-time algorithm for the generalized terminal backup problem,
we have to
extend this LP relaxation for the capacitated -edge cover problem.

Another interesting approach is offered by combinatorial 
approximation algorithms
because it is currently a major open problem to find 
a combinatorial constant-factor approximation
algorithm for the survivable network design problem,
for which 
the Jain's iterative rounding algorithm~\cite{Jain01}
achieves 2-approximation.
The survivable network design problem involves
more complicated connectivity constraints than the generalized terminal backup problem.
Hence, study on combinatorial algorithms for the 
latter problem may give useful insights for the 
former problem.
Recently, Hirai~\cite{Hirai14L-extendable} showed that  can
be solved by a combinatorial algorithm. Indeed, he also showed that
his algorithm can be used to implement our -approximation algorithm for the edge-connectivity
terminal backup without generic LP solvers.


Many problems related to multiflows also remain open. We have shown that 
an LP solution provides
a minimum cost half-integral
multiflow that satisfies the flow demand from each terminal
in the edge-capacitated setting. However, how the computation should proceed in 
the node-capacitated setting remains elusive.
Computing a minimum cost integral multiflow under the same constraints 
is yet another problem worth investigating. We note that
Burlet and Karzanov~\cite{BurletK98} solved a similar problem related to integral multiflows
in the edge-capacitated setting.
Their problem differs from ours in the fact that
 is required to 
match the specified value for each terminal .




\section*{Acknowledgements}
This work was partially supported by Japan Society for the Promotion of
Science (JSPS), Grants-in-Aid for Young Scientists (B) 25730008.
The author thanks Hiroshi Hirai for sharing information on multiflows
and his work in~\cite{Hirai14L-extendable}.


\begin{thebibliography}{10}

\bibitem{AnshelevichK11}
E.~Anshelevich and A.~Karagiozova.
\newblock Terminal backup, {3D} matching, and covering cubic graphs.
\newblock {\em {SIAM} Journal on Computing}, 40(3):678--708, 2011.

\bibitem{BabenkoK12}
M.~A. Babenko and A.~V. Karzanov.
\newblock Min-cost multiflows in node-capacitated undirected networks.
\newblock {\em Journal of Combinatorial Optimization}, 24(3):202--228, 2012.

\bibitem{Bernath2012}
A.~Bern{\'a}th and T.~Kir{\'a}ly.
\newblock A unifying approach to splitting-off.
\newblock {\em Combinatorica}, 32:373--401, 2012.

\bibitem{Bernath2014}
A.~Bern{\'a}th and Y.~Kobayashi.
\newblock The generalized terminal backup problem.
\newblock In {\em {SODA}}, pages 1678--1686, 2014.

\bibitem{BurletK98}
M.~Burlet and A.~V. Karzanov.
\newblock Minimum weight {()}-joins and multi-joins.
\newblock {\em Discrete Mathematics}, 181(1-3):65--76, 1998.

\bibitem{Cheriyan2006}
J.~Cheriyan, S.~Vempala, and A.~Vetta.
\newblock Network design via iterative rounding of setpair relaxations.
\newblock {\em Combinatorica}, 26:255--275, 2006.

\bibitem{Cherkasskky77}
B.~V. Cherkassky.
\newblock A solution of a problem on multicommodity flows in a network.
\newblock {\em Ekonomika i Matematicheskie Metody}, 13(1):143--151, 1977.

\bibitem{FleischerJW06}
L.~Fleischer, K.~Jain, and D.~P. Williamson.
\newblock Iterative rounding 2-approximation algorithms for minimum-cost vertex
  connectivity problems.
\newblock {\em Journal of Computer and System Sciences}, 72(5):838--867, 2006.

\bibitem{Hirai13}
H.~Hirai.
\newblock Half-integrality of node-capacitated multiflows and tree-shaped
  facility locations on trees.
\newblock {\em Mathematical Programming}, 137(1-2):503--530, 2013.

\bibitem{Hirai14L-extendable}
H.~Hirai.
\newblock L-extendable functions and a proximity scaling algorithm for minimum
  cost multiflow problem.
\newblock {\em ArXiv e-prints}, Nov. 2014.

\bibitem{Jain01}
K.~Jain.
\newblock A factor 2 approximation algorithm for the generalized {S}teiner
  network problem.
\newblock {\em Combinatorica}, 21(1):39--60, 2001.

\bibitem{Karzanov79}
A.~V. Karzanov.
\newblock A problem on maximum multifow of minimum cost.
\newblock {\em Combinatorial Methods for Flow Problems}, pages 138--156, 1979.
\newblock in Russian.

\bibitem{Karzanov94}
A.~V. Karzanov.
\newblock Minimum cost multifows in undirected networks.
\newblock {\em Mathematical Programming}, 66(3):313--325, 1994.

\bibitem{Lovasz76}
L.~Lov\'asz.
\newblock On some connectivity properties of {E}ulerian graphs.
\newblock {\em Acta Mathematica Hungarica}, 28(1):129--138, 1976.

\bibitem{Nutov2009}
Z.~Nutov.
\newblock Approximating connectivity augmentation problems.
\newblock {\em ACM Transactions on Algorithms}, 6(1):5, 2009.

\bibitem{schrijver-book}
A.~Schrijver.
\newblock {\em Combinatorial Optimization -- Polyhedra and Efficiency}.
\newblock Springer, 2003.

\bibitem{Tardos1986}
E.~Tardos.
\newblock A strongly polynomial algorithm to solve combinatorial linear
  programs.
\newblock {\em Operations Research}, 34(2):250--256, 1986.

\end{thebibliography}

\appendix

\section{Proof of Lemma~\ref{lem.admissible-splitting}}
\label{sec:splitting}
Since Lemma~\ref{lem.admissible-splitting} is trivial when ,
we here suppose that .
Assuming that no edge in  can be removed without violating \eqref{eq:admissible},
we prove that an admissible pair exists in .

We denote 
 by ,
 by ,
and  by .
For each ,
we let  denote ,
and define  as 
.
Note that  is a symmetric skew supermodular function on .
 satisfies \eqref{eq:admissible} if and only if
 for each .
The assumption implies that
each  is incident to some 
such that .
A pair of  is admissible if and only if no  satisfies
 and .
We call  a \emph{dangerous set} when
.


If  is a dangerous set,
then . Since  implies
 or ,
we have  or  for such .
Without loss of generality, we assume that
each  admits  with  and 
(otherwise, it suffices to prove the lemma after removing  from ).
We denote 
 by ,
and the set of  attaining 
by .
Since  for all ,
we have  for each .
Since  satisfies \eqref{eq:admissible},
 for each .

\begin{lemma}\label{lemm.uncrossing-core}
 Let  with .
 \begin{itemize}
  \item[\rm (i)] If , then .
  \item[\rm (ii)] If  and , then
	 and .
  \item[\rm (iii)] If  is minimal in  and ,
	then .
 \end{itemize}
\end{lemma}
\begin{proof}
 It is known that
 
 and
  
 hold for any .
 If , then .
 If  and  with ,
 then  and .
  (i) and (ii) follow from these properties.
 (iii) is indicated by (ii).
\end{proof}

(i) implies that a minimal node set and a maximal node set in 
are unique.
We denote the minimal node set in  by , and the maximal
node set in  by .

In previous work~\cite{Nutov2009,Bernath2012}, it was shown that  includes an admissible pair
if  holds for some .
Hence, in the following discussion, we assume that  for each .
By this assumption, 
 holds if and only if . Moreover,
 is a dangerous set if and only if
, and 
 or  belongs to .

First, let us prove by contradiction that .
For this purpose, we suppose that .
As mentioned above, for each , there exists
 such that , and 
or  holds.
We let  denote one of such .
Because ,
there exist  and distinct edges 
such that  or , and
 or .
If both  and   belong to ,
then . Since this contradicts , 
 or  holds.
Without loss of generality, let
.
Then .
Since ,
 holds for each .
We notice that 
holds for each ,
and  holds for each
 with .
Since these facts imply ,
they contradict the definition of .
Therefore .

Let  with , , and
.
Suppose that the pair of  and  is not admissible.
Then, there exists a dangerous set  with .
  or  for some .
In the former case,
if ,
the existence of 
contradicts ,
and 
if ,
the 
existence of 
contradicts .
Hence,
.
Existence of  and  implies that 
.
If , the minimality of  or  is violated.
Hence, . Now, let ,
and .
Since
, 
we obtain , which also presents a contradiction.



\end{document}
