\documentclass{llncs}
\usepackage{latexsym}
\usepackage{algorithmic}
\usepackage{amsmath}
\usepackage{amssymb}
\usepackage{cite}
\usepackage{graphicx}
\title{Approximate Point-to-Face Shortest Paths in \thanks{This
research was supported in part by NSF award CCF-0635013.}}

\author{Yam Ki Cheung and Ovidiu Daescu}
\institute{Department of Computer Science \\
       University of Texas at Dallas \\
       Richardson, TX 75080, USA \\
       {\tt \{ykcheung,daescu\}@utdallas.edu}}

\begin{document}

\pagestyle{empty}
\maketitle

\begin{abstract}
We address the {\em point-to-face} approximate shortest path problem in :
Given a set of polyhedral obstacles with a total of  vertices, a source point ,
an obstacle face , and a real positive parameter , compute a path from  to 
that avoids the interior of the obstacles and has length at most  times the length
of the shortest obstacle avoiding path from  to .
We present three approximation algorithms that take
 time,
 time, and
 time,
respectively, where  is the precision of the integers used,  is the time
complexity of the point-to-point shortest path algorithm used,  is the ratio of the length
of the longest obstacle edge to the
Euclidean distance between  and , and  is a very slowly-growing function related to
the inverse of the Ackermann's function.
\end{abstract}

\section{Introduction}

The Euclidean shortest path problem among obstacles in the plane or space is one of the oldest and
 well-known problems in computational geometry. It has been intensively studied,
see~\cite{agarwal,chen,Clar87,har-peled,Her99,KMM97, Sar99,Pap85,sharir,Sharir86,GM87,KM88,OW88,Mit96,Mount84, Mit87},
as well as the survey by Mitchell \cite{Mit00}.
In general the problem is stated as: Given a set of obstacles in the d-dimensional
space  with a total of  vertices, find a shortest (Euclidean) path between a source
point  and a target point  while avoiding the interior of the obstacles. There are two
commonly studied versions of this problem. One is the {\em single pair} version, that asks to find a
shortest path between two given query points. The other one is the
single source version, which first constructs a shortest path map with respect to a source
point . After that, for any given query point , a shortest path between  and  can be
found based on the shortest path map.

In , shortest paths are polygonal and turn only at the vertices of the polygonal obstacles.
Sharir and Schorr~\cite{Sharir86} have developed an  time algorithm based on discrete
graph searching and the visibility graph of the obstacles, where  is the number of the obstacle
vertices. Various studies, e.g.~\cite{GM87,KM88,OW88}, improved the time to quadratic in worst case.
Kapoor, Maheshwari, and Mitchell \cite{KMM97} gave an interesting  time and 
space algorithm based on the visibility graph approach, where  is the number of holes (obstacles)
of the given input. This algorithm is the only algorithm known to be linear in  in both time and space.
However, the time dependence on  is quadratic, so the algorithm does not perform well if  is not
relatively small compared to . Avoiding the visibility graph approach, Mitchell~\cite{Mit96} developed
a version of the continuous Dijkstra method and obtained the first subquadratic, 
time algorithm. Subsequently, based on the same technique, this result was improved by Hershberger
and Suri~\cite{Her99} to  time and  space.

In , shortest paths among polyhedral obstacles are polygonal and turn only on obstacle
edges or vertices. However, unlike the case in , shortest paths need not lie on any
discrete graph. Sharir and Schorr~\cite{Sharir86} have shown that shortest paths in 
are {\it geodesic}, i.e. paths must enter and leave an edge at the same angle. Given a distinct sequence
of edges, the local optimal path between two points can be unfolded at each edge to form a straight line,
and the local optimal path can be uniquely identified. Nevertheless, the problem is still significantly
harder than in . For algebraic considerations, Bajaj~\cite{Bajaj85, Bajaj88} has shown
that the algebraic complexity is exponential, since comparing the lengths of two paths may require
exponentially many bits.
Considering the combinatorial aspect of the problem, Canny and Reif~\cite{reif} have shown that the
shortest path problem in  is NP-hard.

An interesting special case of the shortest path problem is that in which the path is
restricted to the surface of a single polytope.
The first significant study of this special case in computational
geometry  is by Sharir and Schorr~\cite{Sharir86}. They gave an  time
algorithm for convex polytopes by exploiting the special structure of geodesic paths along the surface
of a convex polytope. Mount \cite{Mount84} gave an improved algorithm for convex polytopes with running
time . For general nonconvex polytopes, Mitchell, Mount and Papadimitriou~\cite{Mit87}
presented an  algorithm extending the technique of Mount \cite{Mount84}.
See also~\cite{chen,agarwal,har-peled,sharir} for more studies for this topic.

Our real interest is in the general case in .
Several papers~\cite{Pap85,Clar87,Sar99} have presented polynomial time
approximation algorithms for computing an - path, which has length at
most  times the length of the shortest path between two query points,
where  is a real parameter defining the quality of the approximation.
We will discuss those relevant to this paper in Section~\ref{prev}.

Throughout the paper we use the following notations.

\begin{tabbing}
 denotes a set of polyhedral obstacles;\\
 denotes the set of edges of ;\\
 denotes a shortest path from  to  that avoids the interior of the obstacles; \\
 denotes an approximate path of  (see next section); \\
 denotes the length of a given path ; \\
 denotes the line segment between points  and :\\
 denotes the Euclidean distance between  and .\\
\end{tabbing}

\subsection{Our Results}

In this paper, we address the {\em point-to-face} shortest path problem in :
Given a set of polyhedral obstacles with a total of  vertices, a source point , and a destination
face  of some obstacle in the input, compute an - path from  to  that
avoids the interior of the obstacles.
We assume the obstacle surfaces are triangulated, so  is a triangle in .


The problem has multiple applications, including robot navigation, path planning and
structural proteomics.
For example, in structural proteomics, after a protein surface has been segmented and pockets identified,
various descriptors can be associated
with key points of the pocket. Given a key point , one can measure the pocket depth of  by
computing various distance measures from
 to the ``caps'' of the pocket, the most natural of which is the shortest collision free
distance from  to the caps (a cap is a face of the convex hull of the protein).

We present three approximation algorithms for the point-to-face shortest path problem that take
 time,
 time, and
 time,
respectively, where  is the precision of the integers used,  is the time
complexity of the point-to-point shortest path algorithm used,  is the ratio of the length
of the longest obstacle edge to the
Euclidean distance between  and , and  is a very slowly-growing function related to
the inverse of the Ackermann's function. The main contribution of this paper is in the third algorithm.



\subsection{Previous Work in }
\label{prev}


In the approximate shortest path problem, an additional real positive parameter , which defines the quality of the approximation, is given as part of
the input, and the goal is to find a path between two given points  and 
that avoids (the interior of)
the obstacles and has length at most  times the length of the shortest obstacle avoiding
path between those two points. Such an approximate path is referred to as an -
path or an - of the shortest path. In this paper, we assume
.
There is also a more general approach, where given a point , a shortest path map with respect to 
is constructed. In , this approach has been investigated for shortest paths on polytopes,
as well as for shortest paths among obstacles~\cite{Sar99}.

Papadimitriou \cite{Pap85} gave the first fully polynomial time approximation scheme for the general shortest
path problem in .
The time complexity of the algorithm is ,
where  is the complexity of the set of obstacles , i.e. the number of edges, and  is the precision of the
integers used, that is, the number of bits in the largest integer describing the coordinates of
any scene element.

\begin{figure}\begin{center}
    \leavevmode
\includegraphics[height=2in]{jfig1.eps}
    \caption{A shortest path and its approximation.}
    \label{jfig1}
    \end{center}
    \vspace{-0.15in}
\end{figure}

The approach is relatively simple and thus it could be implemented in practice. It discretizes the
problem by breaking every edge into a number of small segments. A visibility graph  is constructed,
in which each node represents a segment. A link between two nodes is created if the two segments
represented are visible to each other. The weight of the link is set to the Euclidean distance between
the mid-points of the segments (see Fig.~\ref{jfig1} for an illustration).



To partition one edge  of , a coordinate system is chosen such that  lies on the x-axis and the
origin is the closest point on  to . A sequence of points is added on  with coordinates
, where  is the Euclidean distance from
 to  and  is a real positive parameter. As a result, the length of each segment is
no more than  times the distance from  to the segment. It is shown in \cite{Pap85}
that applying Dijkstra's algorithm on  gives a path which is at most  times the
length of the shortest path. Taking  gives an - path.



Given a source point , a real positive parameter , and a set of polyhedral obstacles  in , with a total of  vertices,
Har-Peled \cite{Sar99} presented a technique to construct approximate shortest path maps in  time on each face and in  time total in , where  is the time complexity of the point-to-point shortest path
algorithm used. In \cite{Sar99}, the point-to-point shortest path
algorithm used is that of Clarkson\cite{Clar87}, which we will discuss later.
Once the map is constructed, it takes 
time for each query, that is, given a query point , the length of a
- path
from  to  can be reported in  time.

\begin{figure}\begin{center}
    \leavevmode
\includegraphics[height=2in]{jfig2.eps}
    \caption{The distance function  is defined as }
    \label{jfig2}
    \end{center}
\end{figure}

To construct a shortest path map for 
on an obstacle face , Har-Peled's algorithm places a set of
 points on . A weighted Voronoi
diagram is then constructed on those points.
The weight of each point is the length of an - path from  to the point,
obtained by any
existing shortest path algorithm. For a point , the distance function  is defined as
,
where  is the point in the weighted Voronoi diagram closest
to ,  is the
weight of , and  is the Euclidean distance between  and  (see Fig.~\ref{jfig2}).
It was shown in \cite{Sar99} that
if the points are placed ``carefully'',  is at most  times the length of the shortest
path between  and .


Clarkson~\cite{Clar87} gave an

algorithm for computing an - of the shortest obstacle avoiding path
between two given points  and  in
, where  is the ratio of the length of the longest obstacle edge to the
Euclidean distance between  and , and  is a very slowly-growing function related to
the inverse of the Ackermann's function.

Let  be the set of obstacles.
The general idea of Clarkson's approach is to construct a visibility graph .
 contains , , and the points on  that  might pass through. To limit the number of
edges per node in , a cone structure  is applied on every node 
(see Fig.~\ref{jfig3}). In a cone ,
with apex , if  is the closest node visible to  then an edge  between  and  is added to
. The weight of  is . Notice that there could be many nodes visible to  in , but at
most one edge is added. Clarkson showed that to ensure that the path found by running a single
source shortest path algorithm on  is an - path of ,
the size of the cone structure needed is , where  is the dimension of the problem
(i.e.,  in ).
Hence, there are  edges incident to a node in ~\cite{Clar87}.
In the three-dimensional case,  could make a turn at any point on any edge .
Let  and  be the end points of an edge . A point  can be expressed as a function
, where .
For a cone  with apex ,
define the distance function , where
 is the closest node in  visible to .
One important observation Clarkson made is that  is piecewise linear.
Given a predefined cone structure , each  can be divided into segments according to
this piecewise
linearity. The set of line segments formed is called the \emph{combinatorial characterization} of
 \cite{Clar87}, and is denoted as . According to the study of Davenport-Schinzel
sequences \cite{SCK86}, the size of  is , where  is the
number of obstacle edges, and  is a very slowly-growing function (see above).
Hence the connectivity relations of  can be represented by .

\begin{figure}\begin{center}
    \leavevmode
\includegraphics[height=2in]{jfig3.eps}
    \caption{A simple cone structure with 8 cones.}
    \label{jfig3}
    \end{center}
\end{figure}

 contains a set of
carefully selected Steiner points chosen as follows.
All endpoints of segments in  are Steiner
points. Additional Steiner points are added to further divide each edge into segments no longer than
, where  is a lower bound on .

The algorithm for computing an - path consists of two phases.
In the first phase,  is set to
,  is set to 1/2, and a simpler cone structure is used
to compute a - path. This provides a better lower bound on  for the second phase,
which gives the final approximation.


\section{Approximate Point-to-Face Shortest Paths}
In this section we
present three algorithms for finding approximate point-to-face shortest paths among polyhedral obstacles in .

\subsection{A General Approach}
The main idea used
is to place (a grid of) Steiner points on the target face 
such that there is at least one point close enough to  to give us a good approximation.

Let  and  denote a lower bound and an upper bound on the shortest path between  and ,
respectively.

Lemma \ref{2approximation} below is an application of Lemma~2.10 of \cite{Sar99}, which proved a more
general claim.
\begin{lemma}
\label{2approximation}
Let  be any point on the target face . Let  be the point on  closest to the source 
with respect to the Euclidean distance.
We have .
\end{lemma}

\begin{proof}
Refer to Fig. \ref{jfig4}. Let  be the projection of  on the plane supporting .
If , then , otherwise  lies on the boundary of .
It is obvious that . We have .
\hfill 
\end{proof}

\begin{figure}\begin{center}
    \leavevmode
\includegraphics[height=2in]{jfig4.eps}
    \caption{ is no more than .}
    \label{jfig4}
    \end{center}
\end{figure}

Let  be the end point of the shortest path between  and . Hence the shortest path between  and  can also be expressed as . By Lemma~\ref{2approximation}, we have . By computing an - path between  and , i.e. , we can obtain the lower bound and upper bound on  as  and .

\subsection{The First Algorithm}
Our first algorithm for finding approximate point-to-face shortest paths extends Papadimitriou's algorithm for the point-to-point version. We proceed as follows.
The lower bound  and the upper bound  of the shortest path can be computed by identifying the
point  and computing  using Papadimitriou's algorithm.
Obviously , which implies  must be within distance  of  (see Fig.~\ref{jfig5}).

\begin{figure}\begin{center}
    \leavevmode
\includegraphics[height=2in]{jfig5.eps}
    \caption{The point  must be within  distance of .}
    \label{jfig5}
    \end{center}
\end{figure}

We define a sample grid as a
uniform grid of unit length , which is applied on the target face  within distance  of . The total number
of sample points is .

\begin{lemma}
\label{sample}
There exists one sample point  that gives a - path.
\end{lemma}

\noindent \begin{proof}
Let  be the sample point closest to . We have  and

where  is the approximating path and  is the shortest path.
\hfill 
\end{proof}

In Papadimitriou's algorithm, each edge of  is split into at most
 segments \cite{Pap85}.
The number of visibility graph edges resulting from this subdivision of edges of  is
. With  we have
 edges. We also have
 graph edges corresponding to visibility edges between
the segments on edges of  and the sample points on .
The number of vertices of the graph is . Dijkstra's algorithm takes
 time, where  is the number of graph edges and  is the number of graph nodes. With  and , Dijkstra's algorithm takes  time to compute a - path between  and  by Lemma~\ref{sample}.

In order to have an - path between  and , we simply use a new parameter  to construct the visibility graph. The new parameter does not affect the complexity of the algorithm.

\begin{theorem}
Given a point , a face , a set  of obstacles, with a total of  vertices, and a real positive parameter , an - of
the shortest obstacle-avoiding path from  to  can be computed in
 time,
where  is the complexity of  and  is the number of bits in the largest integer describing the coordinates of any scene element.
\end{theorem}

\subsection{The Second Algorithm}
Our second algorithm builds upon Har-Peled's point-to-point approximate shortest path algorithm. Har-Peled' algorithm constructs an approximate shortest path map on a given face  with respect to a source  in  time, where  is the complexity of the point-to-point shortest path algorithm used. As mentioned previously, a full map on  is unnecessary as  must locate within  radius of . Hence only a partial map within  radius of  is sufficient to capture a good approximation. There is no actual query to be performed, since the sample points placed on  to compute the weighted Voronoi diagram already serve the purpose as a sample grid. The number of sample points placed within  radius of  is  \cite{Sar99}. It takes  time to construct the partial map. If Clarkson's algorithm is used for , we have:

\begin{theorem}
Given a query point , a face  and a real positive parameter , an - of
the shortest obstacle-avoiding path from  to  can be computed in  time
, where  is the ratio of the length of the longest obstacle edge to the
Euclidean distance between  and , and  is a very slowly-growing function related to
the inverse of the Ackermann's function.
\end{theorem}

This second algorithm is very similar to the first one. Both algorithms need a sample grid of size  and execute a point-to-point shortest path algorithm on each point, which means we have to run a point-to-point shortest path algorithm  times.
In the next subsection, we show how to obtain a point-to-face shortest path algorithm which has the same asymptotic complexity as Clarkson's point-to-point version.

\subsection{The Third Algorithm}
\begin{theorem}
\label{main}
Given a query point , a face , a set of obstacles  and a real positive parameter , we can find an - path of
the shortest obstacle-avoiding path from  to  in time
, where  is the ratio of the length of the longest obstacle edge to the
Euclidean distance between  and , and  is a very slowly-growing function related to
the inverse of the Ackermann's function.
\end{theorem}
Notice that in Theorem~\ref{main} the point-to-face approximation has the same complexity as Clarkson's
point-to-point approximation~\cite{Clar87}. Indeed, our solution builds upon Clarkson's point-to-point solution.

We will use a different approach to prove Theorem~\ref{main}.
Instead of building a sample grid on  first, we will construct a visibility graph first and then
decide the additional Steiner points needed on  based on the visibility graph constructed.
All additional Steiner points on  can be added directly to the visibility graph.
Hence, we only need to execute Dijkstra's algorithms once for all Steiner points placed on .
We will explain how additional Steiner points and the corresponding visibility graph edges are produced, and give a count on the total number of Steiner points.

\begin{figure}\begin{center}
    \leavevmode
\includegraphics[height=2in]{jfig6.eps}
    \caption{ cannot travel beyond the sphere  centered at  with radius .}
    \label{jfig6}
    \end{center}
\end{figure}

Clarkson's algorithm consists of two phases. In the first phase, a coarse approximation of
the shortest path between the source and destination is computed, which gives a better lower bound
for the second phase. We set , which is the closest point on  to  in Euclidean distance,
as the destination point and find a  of the shortest path between 
and  using
the first phase of Clarkson's approach. Let  be the end point of the shortest path between
 and . Set the length of the  path as the upper bound  of 
and let  as the lower bound, following Lemma.~\ref{2approximation}. Obviously,  cannot
travel beyond the sphere centered at  with radius  (see Fig.~\ref{jfig6}). In the second phase,
following Clarkson's approach, we only need to partition obstacle edges within the sphere into
segments of length  and then apply the cone structure on each Steiner point created.
With this, we have the following lemma.

\begin{lemma}
\label{lemma-edge}
If  is on an edge  then there exists a Steiner point  such that
.
\end{lemma}

\begin{proof}
On , choose the Steiner point closest to  as .
See Fig.~\ref{jfig7} for an illustration.
Since each segment is no longer than
, .
By triangle inequality, we have  and
\\

\hfill 


\begin{figure}\begin{center}
    \leavevmode
\includegraphics[height=2in]{jfig7.eps}
    \caption{The Steiner point  gives a good approximation for .}
    \label{jfig7}
    \end{center}
\end{figure}
\hfill 
\end{proof}

To handle the case when  is in the interior of  we need to add two sets of Steiner points.
We do this as follows.
For the first set, for each obstacle edge , we add point  as a new Steiner point, if  satisfies:\\
1) its projection  in the plane containing  is in the interior of , and \\
2)  is visible from , and\\
3)  is tangent to some obstacle, which does not contain . \\
We apply the cone structure (following Clarkson's algorithm) on  and add the corresponding edges to the visibility graph. Since there are at most  such points and the number of Steiner points in the original visibility graph is , the complexity of the visibility graph is unchanged.

To obtain the second set of Steiner points, we proceed as follows.
For each existing Steiner point  on an obstacle edge, let  be the projection of  in the plane containing . We add  and edge  to the visibility graph if  is in the interior of  and  is visible from . Note that we do not apply the cone structure on . Observe that, we at most double the number of Steiner points
and each additional Steiner point introduces exactly one edge. The complexity of the visibility graph remains the same. We will discuss later how to efficiently determine the visibility between a point and its projection on a plane.

\begin{lemma}
\label{lemma-per}
Let the contact point of  on the last obstacle edge before reaching  be , that is,
 is the last segment of . If  is an interior point of  then
 must be perpendicular to .
\end{lemma}

\begin{proof}
We make the proof by contradiction. Suppose  is in the interior of  and  is
not perpendicular to . We will construct a new path from  to some point 
such that the new path is shorter than .

Let  be the projection of  on the plane  containing . Consider the plane
formed by ,  and . Let . We shift  along the segment  until either  is
reached or the segment  intersects some edge  at a point . We have , a contradiction.

\hfill 
\end{proof}
\begin{figure}\vspace*{-0.35in}
    \begin{center}
    \leavevmode
\includegraphics[height=2in]{fig2.eps}
    \caption{Illustration of (the proof of) Lemma~\ref{lemma-per}.}
    \label{fig2}
    \end{center}
    \vspace*{-0.35in}
\end{figure}


\begin{lemma}
\label{lemma-interior}
If  is an interior point of , in the modified visibility graph there exists at least one Steiner
point  on  and such that .
\end{lemma}

\begin{proof}



By , the last segment  of  is perpendicular to , since  is an interior
point of . Assume  is not a Steiner point, since otherwise we are done.
On the edge containing , let  be one of the two Steiner points neighboring ,
specifically, one visible from its projection on . Such a Steiner point exists from our placement of Steiner points (set one). We have .
Let the projection of  on  be . Use an alternative path from  to  to approximate  (See Fig.~\ref{jfig8} for an illustration). We have




\hfill 
\end{proof}

\begin{figure}\begin{center}
    \leavevmode
\includegraphics[height=2in]{jfig8.eps}
    \caption{Approximation of  by an alternative path.}
    \label{jfig8}
    \end{center}
\end{figure}


Following  and , the new visibility graph gives us a
- path from  to .
To conclude the proof of Theorem~\ref{main} we need to show how to find the additional Steiner points
and visibility edges within the given time bound.

\subsubsection{Visibility Computation}

We now show how to compute the additional Steiner points on obstacle edges as well as on the target obstacle face . Since addition of Steiner points does not change the complexity of the visibility graph, the cost of the shortest path computation is .
We would like to keep the computation of additional Steiner points and edges that capture
the visibility between Steiner points on obstacle edges and their projections on  within this time bound.

A naive approach to compute the visibility between two points  and 
would require  time by checking whether the line segment  intersects any obstacle.
Over all points, this will exceed the time bound above.

We can try to reduce the time to compute the visible pairs by using a ray shooting data structure.
As mentioned in~\cite{Shar03}, the general ray shooting problem in three dimensions is still far
from being fully solved.
However, our problem is a special case of the ray shooting problem. We need to determine the visibility
between a point and its orthogonal projection on the target plane  supporting , instead of the
visibility between any two arbitrary points, i.e. the direction for ray shooting is always perpendicular
to . On each obstacle edge , all Steiner points and their projections on  are coplanar.
Let  be the plane passing through  and perpendicular to . By computing
 for each edge of  (see Fig.~\ref{Fig-2d}), we can transform the three dimensional
visibility problem
into  subproblems in dimension two, where  is the number of obstacle edges.

\begin{figure}\begin{center}
    \leavevmode
\includegraphics[height=2in]{fig4.eps}
    \caption{The sweep line intercepts edges  and }
    \label{Fig-2d}
    \end{center}
\end{figure}

Let edge  be the projection of an edge  onto .
Notice that all line segments between Steiner points on 
and their projections on  (that is, on ) are parallel. We can perform a plane sweeping on 
with a horizontal line and maintain a data structure which stores edges intersecting with the sweep line
at a certain time instance. The data structure is updated when
the sweep line passes a vertex and
each insertion and deletion precess requires  time~\cite{shamos}. During this process,
we could partition
 into  open segments, such that all points on each segment have the same visibility answer
with respect to their corresponding
projections on . Furthermore, the end points of segments that are visible from their projection on  are the new Steiner points to be added on obstacle edges. The total time of adding Steiner points is , where  is the number of Steiner points in the original visibility graph.

We can simplify this process as follows.
Since the order of the edges intersecting with the sweep line is not important, we
can avoid the plane sweep procedure altogether while still partitioning each edge into  segments
in  time. Recall that we assume the obstacles are triangulated. For each triangle , find the
intersection segment . If at least one vertex of  is inside of the quadrilateral
formed by the endpoints of  and , we project  on . The projection of  on  is the
region in which the visibility of Steiner points to their counterparts are blocked by . Hence we can
divide  into  segments in  time. Again, the total time is .

Note that the cost of visibility computation is dominated by the cost of the shortest path computation, i.e. the overall time to approximate the shortest path from  to  is .

This concludes the proof of Theorem~\ref{main}.

\section{Conclusions}
In this paper we discussed three point-to-face approximate shortest path algorithms
in .
It is interesting to notice that the point-to-face shortest
path must end on  within a certain range from , where  is the point on  closest to  in
Euclidean distance. That range can be obtained by applying known point-to-point shortest path algorithms
between  and . After placing a sample grid near , we proved we
can find an - path
between  and  in  time
by extending Papadimitriou's algorithm or in
 time by
extending Har-Peled's algorithm. However, this approach still requires to execute a point-to-point
shortest path algorithm  times.
Finally we showed that Clarkson's point-to-point shortest path approach can be extended to solve the
problem by adding additional Steiner points directly to the visibility graph, without changing the
asymptotic complexity of the algorithm, resulting in an
 time algorithm for finding
an - path between a source point  and an obstacle face .

\baselineskip=12pt
\begin{thebibliography}{}

\bibitem{agarwal}
P.K. Agarwal, S. Har-Peled, M. Sharir, and K.R. Varadarajan.
Approximate shortest paths on a convex polytope in three dimensions.
{\it J. Assoc. Comput. Mach.}, 44:567-584, 1997.

\bibitem{Bajaj85}
C. Bajaj.
The algebraic complexity of shortest paths in polyhedral spaces.
tech. rept. CSD-TR-523, Computer Science Dept., Purdue University, June 1985.


\bibitem{Bajaj88}
C. Bajaj.
The algebraic degree of geometric optimization problems.
{\it Discrete Comput. Geom.}, 3:177-191, 1988.


\bibitem{reif}
J. Canny, and R.H. Reif.
New lower bound techniques for robot motion planning problems.
{\it Proc. 28th IEEE Sympos. Found. Comput. Sci.}, pp. 49-60, 1987.

\bibitem{chen}
J. Chen, and Y. Han.
Shortest paths on a polyhedron; Part I: computing shortest paths.
{\it Internat. J. Comput. Geom. Appl.}, 6: 127-144, 1996.

\bibitem{Clar87}
K.L. Clarkson.
Approximation algorithms for shortest path motion planning.
{\it Proc. 19th ACM Sympos. Theory Comput.}, New York, pp.56-65, 1987.

\bibitem {GM87}
S.K. Ghosh, and D.M. Mount.
An output sensitive algorithm for computing visibility graphs.
{\it Proc. 28th Annual IEEE Sympos. on Foundations of Computer Science}, pp. 11-19, 1987.

\bibitem{har-peled}
S. Har-Peled.
Approximate shortest paths and geodesic diameters on convex polytopes in three dimensions.
\emph{Discrete \& Computational Geometry}, 21(2): 217-231, 1999.

\bibitem{KM88}
S. Kapoor and S.N. Maheshwari.
Efficient algorithms for Euclidean shortest path and visibility
problems with polygonal obstacles.
{\it Proc. 4th Annual ACM Sympos. on Computational Geometry}, pp. 172-182, 1988.

\bibitem{KMM97}
S. Kapoor, S. N. Maheshwari, and J. S. B. Mitchell.
An efficient algorithm for Euclidean shortest paths among polygonal obstacles in the plane.
{\it Discrete Comput. Geom.}, 18:377-383, 1997.

\bibitem{Sar99}
S. Har-Peled.
Constructing Approximate Shortest Path Maps in Three Dimensions.
{\it SIAM J. Comput.}, 28(4): 1182-1197, 1999.

\bibitem{Her99}
J. Hershberger, and S. Suri.
An optimal algorithm for Euclidean shortest paths in the plane.
{\it SIAM J. Comput.}, 28(6):2215-2256, 1999.

\bibitem{Mit96}
J. S. B. Mitchell.
Shortest paths among obstacles in the plane.
{\it Internat. .L Comput. Geom. Appl.}, 6:309- 332, 1996

\bibitem{Mit00}
J. S. B. Mitchell.
Geometric shortest paths and network optimization.
In J.-R. Sack and J. Urrutia, editors, Handbook of Computational Geometry.
pp. 633-701. Elsevier Publishers B.V. North-Holland, Amsterdam, 2000.

\bibitem{Mount84}
D.M. Mount.
On finding shortest paths on convex polyhedra.
Tech. Rept., Computer Sience Dept., Univ. Maryland, College Park, Octobor, 1984.

\bibitem{Mit87}
J.S.B. Mitchell, D.M. Mount, and C.H. Papadimitriou.
The discret geodesic problem.
{\it SIAM J. Comput.}, 16:647-668, 1987

\bibitem{OW88}
M.H. Overmars and E. Welzl.
New methods for computing visibility graphs.
{\it Proc. 4th Annual ACM Sympos. on Computational Geometry}, pp. 164-171, 1988.

\bibitem{Pap85}
C.H. Papadimitrou.
An algorithm for shortest-path motion in three dimensions.
{\it Information Processing Letters}, 20:259-263, 1985.

\bibitem{shamos}
F.P. Preparata and M.I. Shamos.
Computational Geometry: An Introduction, Springer-Verlag, New York, 1985.

\bibitem{sharir}
Y. Schreiber and M. Sharir.
An optimal-time algorithm for shortest paths on a convex polytope in three dimensions.
\emph{Discrete \& Computational Geometry (DCG)}, 39(1-3):500-579, 2008.

\bibitem{SCK86}
M. Sharir, R. Cole, K. Kedem, D. Leven, R. Pollack, and S. Sifrony.
Geometric application of Davenport-Schinzel sequences.
{\it Proc. 27th IEEE FOCS}, pp. 77-86, 1986.

\bibitem{Sharir86}
M. Sharir and A. Schorr.
On Shortest paths in polyhedral spaces.
{\it SIAM J. Comput.,} 15:193-215, 1986.

\bibitem {Shar03}
M. Sharir, and H. Shaul,
Ray shooting and stone throwing with near-linear storage.
{\it Comput. Geom.}, 30(3): 239-252, 2005.


\end{thebibliography}
\end{document}  