\documentclass{article}

\usepackage{xcolor}
\usepackage{natbib}
\usepackage[hang]{subfigure}
\usepackage[utf8]{inputenc} \usepackage[T1]{fontenc}    \usepackage{hyperref}       \usepackage{url}            \usepackage{booktabs}       \usepackage{amsfonts}       \usepackage{nicefrac}       \usepackage{microtype}      \usepackage{amsmath,amsfonts,mathtools}
\usepackage{amsthm}
\usepackage{afterpage}      \usepackage{cleveref }
\usepackage{algorithm}
\usepackage{algorithmic}
\usepackage{authblk}
\usepackage{subfiles}
\usepackage{adjustbox}

\theoremstyle{plain}
\newtheorem{thrm}{Theorem}
\newtheorem{lem}{Lemma}
\newtheorem{prop}{Proposition}
\newtheorem*{cor}{Corollary}

\theoremstyle{definition}
\newtheorem{defn}{Definition} 
\newtheorem{conj}{Conjecture}
\newtheorem{exmp}{Example}

\theoremstyle{remark}
\newtheorem*{rem}{Remark}
\newtheorem*{note}{Note}

\newcommand\numberthis{\addtocounter{equation}{1}\tag{\theequation}}

\def\*#1{\boldsymbol{#1}}

\newcommand{\jaakko}[1]{\textcolor{red}{#1}}
\newcommand{\jon}[1]{\textcolor{orange}{#1}}

\title{Scalable Probabilistic Matrix Factorization with Graph-Based Priors}



\author[1]{Jonathan Strahl}
\author[2]{Jaakko Peltonen}
\author[1,3]{Hiroshi Mamitsuka}
\author[1]{Samuel Kaski}\affil[1]{Helsinki Institute for Information Technology HIIT,
\protect\\ Department of Computer Science, Aalto University}
\affil[2]{Faculty of Information Technology and Communication Sciences, Tampere University, Finland}
\affil[3]{Bioinformatics Center, Institute for Chemical Research, Kyoto University, Japan}



\begin{document}

\maketitle



\begin{abstract}
In matrix factorization, available graph side-information may not be well suited for the matrix completion problem, having edges that disagree with the latent-feature relations  learnt from the incomplete data matrix. We show that removing these \textit{contested} edges improves prediction accuracy and scalability. We identify the contested edges through a highly-efficient graphical lasso approximation. The identification and removal of contested edges adds no computational complexity to state-of-the-art graph-regularized matrix factorization, remaining linear with respect to the number of non-zeros. Computational load even decreases proportional to the number of edges removed. Formulating a probabilistic generative model and using expectation maximization to extend graph-regularised alternating least squares (GRALS) guarantees convergence. Rich simulated experiments illustrate the desired properties of the resulting algorithm. On real data experiments we demonstrate improved prediction accuracy with fewer graph edges (empirical evidence that graph side-information is often inaccurate). A 300 thousand dimensional graph with three million edges (Yahoo music side-information) can be analyzed in under ten minutes on a standard laptop computer demonstrating the efficiency of our graph update.
\end{abstract}

\section{Introduction}\label{sec:intro}

Matrix factorization (MF) is popular in a number of domains including recommender systems \cite{koren2009matrix,mehta2017review}, bioinformatics \cite{brunet2004metagenes,jacoby2018future,stein2018enter,zakeri2018gene,zheng2013collaborative}, image restoration \cite{xue2017depth} and many more \cite{davenport2016overview}. Much of the data is of a very large scale and sparse, and additional (side-)information is usually available. Therefore, many methods focus on scalability \cite{davenport2016overview,mnih2008probabilistic,sardianos2019optimizing} and the addition of side information (SI) \cite{chiang2015matrix,Chiang2018NoisySI,gonen2013kernelized,ma2011recommender,zakeri2018gene,zhou2012kernelized,zhao2015expert}, and more recently scalable methods with SI \cite{monti2017geometric,rao2015collaborative,yao2018convolutional}.

Empirical evidence shows that prediction accuracy is significantly improved by graph SI, where edges in the graph represent similarity between connected nodes \cite{cai2011graph,ma2011recommender,monti2017geometric,rao2015collaborative,yao2018convolutional,zhou2012kernelized,zhao2015expert}. MF (or low-rank matrix completion) has theoretical guarantees for exact completion without and with noise \cite{candes2010matrix,candes2009exact}. 
Introducting noisy SI is shown to reduce sample-complexity, and is reduced even further handling the noise \cite{chiang2015matrix}. Reduction in sample complexity through the introduction of graph SI has also been shown \cite{ahn2018binary,rao2015collaborative}, as a function of graph quality.  However, to the best of our knowledge there is no work on scalable methods to handle the noise in the graph SI.

Mnih and Salakhutdinov  \cite{mnih2008probabilistic} introduced probabilistic matrix factorisation (PMF), which is equivalent to -regularised (alternating least squares) MF. Probabilistic interpretations for MF with graph SI are kernelized PMF (KPMF  \cite{zhou2012kernelized}) and kernelized Bayesian MF (KBMF \cite{gonen2013kernelized}): placing priors over the columns of the latent feature matrices. This type of prior models the pairwise relation between rows, where these rows correspond to rows or columns of the incomplete data matrix. KPMF and KBMF showed good results on moderate-sized data but failed to scale to large data. 

To address scalability, graph-regularised least squares (GRALS \cite{rao2015collaborative}) was  proposed, with conjugate gradient descent exploiting the sparsity in the data matrix and the graphs, resulting in linear computational complexity and fast convergence. Recently there has been progress on applying deep learning to matrix completion, with and without side information, with good accuracy and showing potential for scalability \cite{berg2017graph,hartford2018deep,monti2017geometric,yao2018convolutional}.

All of the non-Bayesian or scalable methods incorporating graph SI \cite{cai2011graph,ma2011recommender,monti2017geometric,rao2015collaborative,zhou2012kernelized} fix the edges in the graph, considering them as true.  However, these graphs are known to be uncertain \cite{adar2007managing,asthana2004predicting}, and furthermore, the similarities they represent (e.g. homophily \cite{mcpherson2001birds}) are rarely specific to the matrix factorization task leaving no guarantee that correlations correspond \cite{ma2011recommender,singla2008yes}; graphs are often formed for other purposes, and hence their usefulness for MF is uncertain. This leaves room for improving the quality of the graph, leading to a significant reduction in sample complexity \cite{ahn2018binary}. In this work we will introduce a solution based on contested edges, defined later in the paper.

\paragraph{Example of Graph Side-Information and Contested Edges}
To better understand how graph similarities are not task-specific (are non-specific) to MF, take a common example of a movie-recommendation problem with social network (SN) SI (\citet{ma2011recommender} and in our experiments on Douban data). Connected users in the SN do not connect based on their similar preference of movies, instead they connect on the basis of a broader social context.
Similarly, the demographic information in MovieLens\footnote{https://grouplens.org/datasets/movielens/}, used to form a user-similarity graph, is only very indirectly related to the movie preferences \cite{mcpherson2001birds}. Nevertheless, more general similarity has been shown to often work well in practice, but some parts of it may turn out to be detrimental as we illustrate below.
 
\Cref{fig:CombinedIllustration} (top) 
shows a small movie-recommendation data matrix with SN SI (bottom-left).  Without SI, if row/column observations in the data matrix are similar, latent features will be similar. This can be inaccurate, e.g. users 2 and 3 would be considered similar based on the observations, and thus predictions for user 2 would be similar to ratings of user 3, whereas actually user 2 is similar to user 1. Graph information can help by encouraging latent features of connected users, like user 1 and user 2 here, to be similar, even when there is no observed data in the matrix to indicate they should be. However, for other users such as 4 and 5 the graph may mismatch with the data, indicating similarity whereas 4 and 5 are actually negatively correlated (as seen in their ratings of movies 5 and 6), and using the graph would thus worsen their predictions. We propose using this discrepancy to \textit{contest} the graph edge between users 4 and 5; removing this edge as in \Cref{fig:CombinedIllustration} (bottom-right) would improve predictions for users 4 and 5 to be consistent with their observed negative correlation, while the beneficial edge between users 1 and 2 will still remain. In real cases, mismatch between the data matrix and the SI would be detected based on much more data than in this illustration.



\begin{figure}[t]
\begin{tabular}{ccc}
  \begin{minipage}[c]{0.6\textwidth}
  \vspace{0pt}
  \begin{tabular}{llllllll}
    \toprule
    \multicolumn{1}{c}{} & \multicolumn{3}{l}{Movie}                   \\
    \cmidrule(r){2-8}
     User     &  &  &  &  &  &  &  \\
    \midrule
      &  5 &   &  1   &&&&\\
          & \textcolor{lightgray}{5} &  4 & \textcolor{lightgray}{1} &&&&\\
       &  1 & 4 & 5  &&&& \\
      &&&&  5 & 4 & 2 &  \textcolor{lightgray}{1}   \\
     &&&& \textcolor{lightgray}{1} &  2 & 4 & 5 \\
    \bottomrule
  \end{tabular}
  \end{minipage}
  & 
  \begin{minipage}{0.35\textwidth}
  \flushleft
  {\vspace{0.01cm}\includegraphics[width=0.7\textwidth]{User1User2EdgeGraphSmoothBenefitContested}}
\flushright
   {\vspace{0.01cm}\includegraphics[width=0.7\textwidth]{User1User2EdgeGraphSmoothBenefitPruned}}
\end{minipage}
\end{tabular}
\caption{An illustrative movie recommendation problem.
\emph{Left:} data matrix where entries are user-ratings for movies: observations in black, unseen entries are blank and unseen entries to be predicted are in grey.
\emph{Middle:} Social Network SI; connected users assumed to have similar ratings. The edge shown in red is contested due to negative correlation of  and  in the data matrix. \emph{Right:} a graph update with removal of the contested edge to improve prediction accuracy.
}
\label{fig:CombinedIllustration}
\end{figure}



We do not propose to identify contested edges directly from the observed data but from correlations between the latent features. We introduce a probabilistic generative model that we call graph-based prior PMF (GPMF). Using the expectation-maximization (EM, \cite{bishop2006PRML}) algorithm we find a maximum a posteriori (MAP) estimate for the latent features and a maximum likelihood estimate (MLE) for the correlations of the latent features. We show in \Cref{sec:Mstep} how using GLASSO approximation we can remove contested edges by simply thresholding a constrained sample covariance matrix (SCM).

There exist a number of approaches to reduce the edges in a labelled graph, graph summarization, \citet{liu2018graph} for example. Most of these approaches do not use node attributes (labels) and to the best of our knowledge none use latent features for edge pruning. There are link prediction models that are probabilistic and use node attributes \cite{haghani2017systemic} but none of them can (yet) scale to large data \cite{li2014lrbm,nguyen2012latent,zhao2017leveraging}.

This paper introduces GPMF: the generative model in \Cref{sec:GenarativeModel}, the scalable constrained EM algorithm in \Cref{sec:TheEMAlgo}, experiments in \Cref{sec:experiments} and a conclusion in \Cref{sec:conclusion}.

\section{GPMF Generative Model and Relations to the Graph Side-Information} \label{sec:GenarativeModel}

We are provided with a partially observed data matrix  with  rows and  columns.  is approximated as the product of two low-rank matrices,  and . The number of latent features  is fixed;  and  have  columns, each row is a latent feature vector for each row / column of  respectively. We use an index set  where  is one if the element in row  and column  of  is observed, and zero otherwise. The goal is to learn latent-feature matrices  and  that most accurately represent the full matrix .

-regularized MF has a scalable probabilistic interpretation: PMF. Each observed entry  is assumed to have Gaussian noise ; each row of  and  has a zero-mean spherical Gaussian prior.  Similar to KPMF \cite{zhou2012kernelized}, our model replaces the spherical Gaussian prior with a full-covariance Gaussian over the columns of the latent features (introducing row-wise dependencies):

Graph SI constrains the structure of the precision matrices ( or ) of \eqref{eq:GraphBasedPriorU} and \eqref{eq:GraphBasedPriorV},  discussed next. 
\subsection{Gaussian Markov Random Field (GMRF) relation to Precision matrix} An undirected graph  with a set of nodes , representing a set of random variables , and a set of edges , defines the conditional independence of the random variables, where the absence of an edge  implies that the two random variables are conditionally independent  given the remaining random variables \cite{bishop2006PRML,hastie2009elements,lauritzen1996graphical,rue2005gaussian}: . In the remainder of the paper we refer to the adjacency matrix of  : a symmetric matrix where  is one if an edge exists between nodes  and  and zero otherwise. We can summarize the GMRF relation as .
\subsection{Laplacian Matrix relation to Precision Matrix}  \label{sec:RegLaplaceMat} The Laplacian matrix of a graph is , where  is a diagonal degree matrix, and is positive-semi-definite by definition.  The regularised Laplacian  is a positive-definite matrix; a valid precision matrix retaining the GMRF property \cite{dong2016learning,egilmez2016graph,egilmez2017graph,hastie2009elements,liu2014bayesian}: .
\begin{lem} \label{lem:PosteriorEquivGRALS}
If the precision matrix in \eqref{eq:GraphBasedPriorU} and  \eqref{eq:GraphBasedPriorV} is the regularised Laplacian matrix , then the MAP estimator of our model has the same objective function as GRALS \cite{rao2015collaborative}. Our GPMF model therefore gives a generalization of the GRALS objective function.
\end{lem}
\begin{proof}[Proof of Lemma \ref{lem:PosteriorEquivGRALS}.]
Our generative model is biconvex, and hence it suffices to prove for 
 that the posterior is equivalent to the GRALS objective.
Holding  fixed and finding the log posterior of :

where  is row  of matrix  and  is column  and noting that . \Cref{eq:MAP_GRALS_Equiv} is the GRALS objective function \cite{rao2015collaborative}. Derivations in the supplementary material.
\end{proof}

\section{GRAEM: Scalable EM for GPMF} \label{sec:TheEMAlgo}

We naturally extend each least-squares sub-problem of GRALS \cite{rao2015collaborative} with graph-regularised alternating EM (GRAEM), having the same global convergence guarantees as GRALS \cite{xu2013block}.  We work through optimising  with  fixed, solving for  has the same form.

\subsection{The EM Formulation}
We have an incomplete data matrix , fixed matrix , latent variable matrix  and graph SI. From the graph we derive  (see \Cref{sec:RegLaplaceMat}), then set the precision matrix , which we consider our model parameters. We want to maximize the expectation of the joint density of the data and the latent variables, with  as our unknowns and  as our input parameters:

\subsection{E-step: Expected Value of the Latent Variables} \label{sec:EStepFullU} The expected value of our latent variables has a Gaussian posterior distribution (see supplementary material), we can therefore use the MAP, which is equivalent to the GRALS objective function as shown in \Cref{lem:PosteriorEquivGRALS}: .

\subsection{M-step: Removing Contested Edges}\label{sec:Mstep}

We can remove edges in the graph that correspond to negative correlations between the latent features by simply removing negative covariances from an SCM; this relationship holds for large scale and sparse problems; details follow.
\subsubsection{The MLE of the parameters and GLASSO} To find the MLE we maximise the  function in Equation \eqref{eq:EMQFunc} with respect to . The maximum can be found in closed form by taking the derivative with respect to the parameter  and setting to zero:

\Cref{eq:MaxOfQForPrecU} is the inverse of an SCM, where each sample is one of the columns of . Values for  are unknown, so we use the MAP given the previous estimate of the parameters (.  The solution (if any) is almost surely not sparse. Graphical lasso (GLASSO \cite{mazumder2012graphical}) finds a sparse solution for the MLE of the precision matrix, where samples are assumed to be normally distributed, in line with our model assumptions in \Cref{sec:GenarativeModel}. We therefore propose solving \eqref{eq:MaxOfQForPrecU} with GLASSO. 

\subsubsection{Constrained GLASSO and Highly Efficient Approximation} GLASSO finds the MLE of the precision matrix under an  penalty, given an SCM . \citet{grechkin2015pathway} showed that the problem space can be reduced with prior knowledge on which pairwise relationships do not exist, forcing them to be zero in the solution:

\citet{zhang2018largescaleprec} uses a relation between the sparsity structure of the -thresholded SCM and the GLASSO solution;  for large-scale problems, when the solution is very sparse, the connected components are equivalent \cite{mazumder2012graphical}, given further assumptions the complete sparsity structure is equivalent \cite{fattahi2019graphical,sojoudi2016equivalence,sojoudi2016graphical}. 
However, this solution will locate correlations, positive and negative, with a strong magnitude, greater than . Next we detail how to identify edges that correspond to only negative correlations.

\subsubsection{Removing a Contested Edge}
The sparsity structure of the SCM and the (GLASSO) solution are equivalent under mild assumptions that are found to be true for sufficiently large , that result in  non-zeros in the solution \cite{fattahi2017graphical,fattahi2019graphical}. One of these assumptions is sign-consistency where each non-zero element of the solution has the opposite sign in the SCM. Assuming sign-consistency we can identify all graph edges that correspond to negative correlations in the latent features, with  from \Cref{eq:MaxOfQForPrecU} as our SCM:




where  is the updated adjacency matrix, the threshold parameter  is set to zero (or can be increased for a sparser solution) and  is the adjacency matrix of the graph SI; CE is a contested edge and con-E is a constrained edge. To solve \Cref{eq:posCovThesh} we need to compute , we can decompose the problem:


The remaining task is to efficiently approximate the posterior covariance  for each column, , of , which we discuss next.





\subsubsection{Posterior Covariance Approximation} \label{sec:SparsePrecEstMStep:ApproxPostCov}

The posterior of our GPMF model, in \Cref{sec:GenarativeModel}, is a joint Gaussian distribution, where the likelihood in \Cref{eq:PmfLikelhood} introduces relations between the columns of the latent features and the prior in  \Cref{eq:GraphBasedPriorU} introduces relations between the rows. This results in a posterior covariance matrix with an inverse Kronecker sum structure \cite{kalaitzis2013bigraphical,schacke2004kronecker}:  where  is the Kronecker product operator and 





\paragraph{Column-wise independence assumption.} We simplify the Kronecker sum with a column-wise independence assumption, setting all off-diagonals of  to zero:
 
where  is the inverse of the observation noise in \eqref{eq:PmfLikelhood}, diag takes a vector to create a diagonal matrix and blkdiag takes a sequence of matrices to construct a block-diagonal matrix.

\paragraph{Sparse Cholesky factorisation:} 

Each  is still too large to invert. Assuming the high-dimensional matrix is sparse, as in \citet{zhang2018largescaleprec}, its Cholesky factorisation is computable in  time \cite{davis2004column}. We compute  samples as an unbiased estimate for the approximate posterior covariance:



\begin{figure*}[t!]
\centering
\subfigure[Decreasing contested edges]{
  \includegraphics[width=0.45\columnwidth]{GraphFidelityAnalysisPlot_AAAI}
}
\subfigure[Increasing observation noise]{
    \includegraphics[width=0.45\columnwidth]{ObsNoise_AAAI}
}

\subfigure[Increasing observed entries]{
    \includegraphics[width=0.45\columnwidth]{PropOfObs_AAAI}
}
\subfigure[Increasing dimensionality]{
    \includegraphics[width=0.45\columnwidth]{LatentFeatureDimensionComparisonPlot_AAAI}
}
\caption{Synthetic data experiments} \label{fig:SynthDataExps}
\end{figure*}



\subsection{The Algorithm}
The EM algorithm iterates between E-step and M-step until convergence.  We initialize the latent feature matrices () by finding the MAP with no graph SI using PMF, to learn latent features that reflect the observed entries of the data matrix. In practise any method to learn the latent features with no SI can be used. The M step uses the relations between the latent features to identify negative correlations and remove them from the graph SI. The E-step then finds the MAP of the latent features given the updated graph. In theory the E and M step could be continued until some convergence criterion was met, but this would be less efficient and we get good results with just one step. So the three steps of our algorithm are lines 1,3 and 4:
\begin{algorithm}
\caption{Graph-regularised alternating EM (GRAEM)}
\hspace*{\algorithmicindent} \textbf{Input: } \\
 \hspace*{\algorithmicindent} \textbf{Output}: 
 \begin{algorithmic}[1]
\STATE 
\WHILE{not converged}
\STATE    \text{Run M-step \Cref{eq:posCovThesh} with }  and  as structural constraints
\STATE   \text{Run E-step with regularized Laplacians} \text{given } 
\ENDWHILE
\end{algorithmic} \label{algo:GPMF}
\end{algorithm}

\subsection{Scalability: Computational Complexity}

The algorithm has three steps: lines 1,3,4 in \Cref{algo:GPMF}. Line 1 is linear in the number of non-zeros  in the data matrix  per conjugate gradient (CG) iteration. Line 3 comprises sparse Cholesky factorisation, linear in time with respect to the dimension size , constrained SCM computation and thresholding,  both converge in one time step. Line 4 uses GRALS with the sparsified graphs:  per CG iteration. Line 4 is initialised with  values from the PMF run, largely reducing the number of iterations required. Our algorithm remains linear with respect to the number of non-zeros. The additional M-step is a trivial additional cost, and if  are much sparser, reducing iteration costs in Line 3, the overall computational load can be less than GRALS using the original graphs.



\section{Experiments} \label{sec:experiments}
We compare our algorithm to a baseline with no graph SI (PMF, \cite{mnih2008probabilistic}), the current most scalable method, GRALS \cite{rao2015collaborative}, and for accuracy less scalable methods KPMF \cite{zhou2012kernelized} and sRMGCNN \cite{monti2017geometric}. For sRMGCNN we used their published code, ran it on a (NVIDIA Tesla P100) GPU and used cross validation to find a good T value; note that this model took several orders of magnitude more time than the other methods: on Flixster data GPMF and GRALS converged in 20 seconds, PMF in 0.2 seconds, sRMGCNN took 30 minutes. We also ran KBMF \cite{gonen2013kernelized} but with an extremely long computational time on even the smallest dataset, and a large number of parameters, we failed to achieve reasonable results.

\subsection{Experiments on Synthetic Data} \label{sec:synthetic_experiments}
To analyze the behaviour of our algorithm we generate a data matrix with a known underlying graph.
Therefore we can replace true edges in the graph with \textit{corrupted edges} (CEs) that contest the true underlying structure, controlling the accuracy of the graph SI. We use a block-diagonal regularised-Laplacian precision matrix. 
We generate a  data matrix by Equations \eqref{eq:PmfLikelhood}-\eqref{eq:GraphBasedPriorV}, with proportion of corrupted edges 0.3, observation noise 0.01, 7\% observed values, and 40 latent dimensions;
we vary these settings in the experiments below.
See supplementary material for further 
details. 




\iffalse
\begin{figure*}[h]
\centering
    \begin{minipage}{.38\linewidth}
  \centering
  \includegraphics[width=0.8\columnwidth]{GraphFidelityAnalysisPlot}
  \caption{Decreasing contested edges}
  \label{fig:graphCorruptionCompare}
  \end{minipage}\hfill
  \begin{minipage}{.38\linewidth}
\centering
\includegraphics[width=0.8\columnwidth]{ObsNoise}
\caption{Increasing observation noise}
\label{fig:ObsNoise}
    \end{minipage}\hfill
      \begin{minipage}{.24\linewidth}
    \end{minipage}
\centering
    \begin{minipage}{.38\linewidth}
\centering
\includegraphics[width=0.8\columnwidth]{PropOfObs}
\caption{Increasing observed entries}
\label{fig:obscompare}
  \end{minipage}\hfill
  \begin{minipage}{.38\linewidth}
  \centering
\includegraphics[width=0.8\columnwidth]{LatentFeatureDimensionComparisonPlot}
\caption{Increasing dimensionality}
\label{fig:DimSizeComparePlot}
    \end{minipage}\hfill
      \begin{minipage}{.24\linewidth}
    \end{minipage}
\end{figure*}
\fi

\textbf{Graph Fidelity.} 
In \Cref{fig:SynthDataExps} (a) we vary the number of CEs. A graph with no CEs has fidelity one (), with all CEs . GPMF consistently improves prediction accuracy over methods with graph SI for , and performance is equal for . PMF with no graph performs better below , showing that a graph of low quality can make prediction accuracy worse.


\begin{figure*}[t!]
\begin{adjustbox}{max width=1.5\textwidth,center}
\centering
\subfigure[Epinions (45\% of edges)]{\includegraphics[width=0.49\textwidth]{EpinionsLargeDataRMSEVsTime}}
\subfigure[Yahoo Music (80\% edges)]{\includegraphics[width=0.49\textwidth]{RMSE_YahooMusic}}
\subfigure[MovieLens20M (65\% edges)]{\includegraphics[width=0.49\textwidth]{RMSE_MovieLens20M}}
\end{adjustbox}
\caption{Convergence time; vertical lines show start and end of M-step. c) 40NN graph.}
\label{fig:LargeDataExperiments}

\end{figure*}

\textbf{Observation Noise.} 
\Cref{fig:SynthDataExps} (b)  shows the benefit of GPMF diminishes as noise increases; learning negative correlations requires learning from the observations. However, at worst GPMF is only as bad as using the original corrupted graph.

\textbf{Proportion of Observations.} 
In \Cref{fig:SynthDataExps} (c)  with just 10\% of observed entries our algorithm can almost attain the same prediction accuracy as using the true graph. GRALS requires 30\% to achieve a similar accuracy. At 40\% of observed entries the graph is no longer beneficial. Note that most large scale matrix completion problems have fewer than 10\% observed entries.

\textbf{Model Capacity.} \Cref{fig:SynthDataExps} (d)  shows that with too few latent features all models are negatively effected, but overall GPMF attains the best prediction accuracy.





\textbf{GLASSO accuracy} We analyse the accuracy of removing CEs over several simulations. With 7\% of observed entries, 31.7\% of CEs are correctly removed and 19\% of true edges (TEs) are wrongly removed; increasing observed entries to 40\%, 44.3\% of CEs are removed and 0.3\% of TEs. Fixing observed entries at 20\%, with noise , 39\% of CEs and 2.7\% of TEs are removed, and with , 34.3\% CEs and 42.7\% TEs are removed. We see clearly that observation noise strongly effects the ability to identify contested edges, as shown in \Cref{fig:SynthDataExps} (b).  Accuracy improves with more observed entries, but even with low levels of noise and a reasonable amount of observations successful removal of CEs is only moderate. Regardless of this moderate accuracy, experiments show this is enough to attain significant improvements in prediciton accuracy.






\subsection{Experiments on Real Data}
\label{sec:real_data_experiments}

In \Cref{tab:RMSEsComputationalComplexity} GPMF using GRAEM (our method) gives improved accuracy over GRALS on all small datasets: 3000 (3k) by 3k subsets of Flixster and Douban  \cite{monti2017geometric} (full datasets not attainable) and MovieLens100k \cite{harper2015movielens}); the bottom rows of the table show the size and number of observations for each data matrix and the number of edges in each side-information graph.   In \Cref{fig:LargeDataExperiments} our method is shown to add no computational cost on large data: MovieLens 20 million
\cite{harper2015movielens}, Epinions \cite{tang2012mtrust} and Yahoo Music \cite{rao2015collaborative,dror2011yahoo}), note that proportion of edges used by GPMF is reported in figure title. \Cref{fig:LargeDataExperiments} (a) is an example of poor quality graph side-information, we see this as PMF outperforms GRALS with the side-information; our method (GPMF using GRAEM) estimates over half the edges as contested, removing them seem to improve the quality. We believe that there were no gains in \Cref{fig:LargeDataExperiments} (b) as the graph is extremely sparse and removing some edges has little effect. We test this hypothesis with MovieLens 20M in \Cref{tab:RMSEsComputationalComplexity} by increasing the number of nearest neighbours from 10 to 40, we see that GRALS with the original graph decreases in performance while our algorithm continues to improve, we plot the best results in \Cref{fig:LargeDataExperiments} (c); the computational time to estimate contested edges (between the two vertical broken lines) is a fraction of the running time of the algorithm.

We also tested general usefulness of the updated graph: We get a small improvement for Douban with KPMF using with 77 \% of edges, we also get the same accuracy for Flixster with almost helf the edges.





\begin{table*}[h]

\caption{Result summary on real datasets (RMSE),  is the graph updated with GRAEM (our method). Bold = best result.}
\label{tab:RMSEsComputationalComplexity}
\begin{footnotesize}
\begin{sc}
\hskip-3.2cm\begin{tabular}{lrrrrrr}
\toprule
&  Flixster &Douban&MovieLens& Epinions & Yahoo & MovieLens 20M \\
Algo.  &  (3k) &  (3k) & 100k &  &  Music &  (10-/20-/40-NN)\\
\midrule
PMF & 0.9809 & 0.7492  & 0.9728 & 0.31& 22.991 & 0.7980 / 0.7980 / 0.7980\\
GRALS & 0.9152 & 0.7504  & 0.9178 & 0.32& \textbf{22.760} & 0.7898 / 0.7925 / 0.7922\\
GPMF / GRAEM (\scriptsize{ours}) & \textbf{0.8857} & 0.7497 & \textbf{0.9174} & \textbf{0.28} & 22.795 & 0.7894 / 0.7895 / \textbf{0.7887}\\
KPMF & 0.9212 & 0.7324 & 0.9336 & - & - & - \\
KPMF ()  & 0.9212 & \textbf{0.7323} & 0.9374 & - & - & - \\
sRMGCNN & 0.9108 & 0.7915  & 0.9263  & - & - & - \\
\midrule
Data dims. & \scriptsize{3k x 3k} & \scriptsize{3k x 3k}  & \scriptsize{1k x 1.5k}  & \scriptsize{22k x 296k} & \scriptsize{250k x 300k} & \scriptsize{138k x 27k} \\
num. of obs. &  2.6k & 137k &  100k & 824k & 6M & 20M \\
\midrule
 edges (/) & 59k / 51k & 2.7k / 0 & 12.6k / 29k  & 574k / 0 & 0 / 3M
 &  0 / 493k - 0 / 963k - 0 / 1.9M \\  
 \midrule
 prop. (/) & 0.57 / 0.63 & 0.77 / 0 & 0.63 / 0.61 & 0.45 / 0 & 0 / 0.8
 & 0 / 0.88 - 0 / 0.71 - 0 / 0.65 \\  
\bottomrule
\end{tabular}
\end{sc}
\end{footnotesize}
\end{table*}

\section{Conclusion} \label{sec:conclusion}
We present a highly efficient method to improve the quality of graph side-information for matrix factorisation. Of the three steps in the algorithm, the initialisation of the latent features and the estimation of the latent features with the updated graph (the E-step) can be performed with any method for matrix completion without SI and with graph SI respectively. With such a small computational cost a graph update (the M-step) to improve quality seems like a valuable step when including graph side-information into matrix factorisation.  Furthermore, we demonstrated the added robustness using our algorithm on real graph side-information. By increasing the number of nearest neighbours for generating graphs from feature side-information our algorithm, GRAEM, improved while GRALS worsened. Our graph update step allows for more noisy graphs to improve the matrix completion accuracy.


Future work on improving the graph update could further improve this method; we showed with simulated data the GLASSO approximation is only moderately successful. 

\section{Acknowledgements}
The research was partly funded by the Academy of Finland grant 313748 and Business Finland grant 211548, computational resources provided by the Aalto Science-IT project.

\newpage



\bibliography{main}
\bibliographystyle{plainnat}

\newpage



\subfile{supplement}

\end{document}
