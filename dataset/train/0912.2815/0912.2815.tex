\documentclass[proceedings]{stacs}
\stacsheading{2010}{609-620}{Nancy, France}
\firstpageno{609}



\theoremstyle{plain}\newtheorem{satz}[thm]{Satz}
\theoremstyle{definition}\newtheorem{crucial}[thm]{Crucial Definition}
\def\eg{{\em e.g.}}
\def\cf{{\em cf.}}

\usepackage{latexsym}
\usepackage{amsmath}
\usepackage{amssymb}

\newcommand{\ct}{\!\cdot\!}
\newcommand{\Ot}{\tilde{O}}
\newcommand{\hd}{\hat{\delta}}
\newcommand{\etal}{{\em et al.\ }}
\newcommand{\del}{\delta}
\newcommand{\eps}{\epsilon}
\newcommand{\eq}{\;=\;}
\newcommand{\inde}{}

\newcommand{\reals}{{\mathbb R}}
\newcommand{\ALGORITHM}{{\tt Algorithm}}
\newcommand{\SPAN}{\mbox{\bf disk-spanner}}
\newcommand{\PATH}{\mbox{\bf path}}
\newcommand{\NN}{\mbox{\sf NN}}

\newcommand{\LSPAN}{\mbox{\bf local-disk-spanner}}
\newcommand{\USPAN}{\mbox{\bf unit-disk-spanner}}
\newcommand{\GSPAN}{\mbox{\bf geom-spanner}}

\def\deg{\mbox{deg}}
\def\SP{\mbox{\tiny SP}}
\def\UDG{\mbox{\tiny UDG}}
\def\DIR{\mbox{\tiny DIR}}

\def\E{\mbox{\tiny E}}
\def\NE{\mbox{\tiny NE}}
\def\NW{\mbox{\tiny NW}}
\def\SE{\mbox{\tiny SE}}
\def\SW{\mbox{\tiny SW}}
\def\W{\mbox{\tiny W}}

\begin{document}

\title[Relaxed spanners for directed disk graphs]{Relaxed spanners for directed disk graphs}

\author[lab1]{D. Peleg}{David Peleg}
\address[lab1]{Department of Computer Science and Applied Mathematics,
  \newline The Weizmann Institute of Science, Rehovot 76100, Israel}  \email{david.peleg@weizmann.ac.il}  


\author[lab2]{L. Roditty}{Liam Roditty}
\address[lab2]{Department of Computer Science, Bar-Ilan University,
\newline Ramat-Gan 52900, Israel}   \email{liamr@macs.biu.ac.il}  


\thanks{Thanks: Supported in part by grants from
the Minerva Foundation and the Israel Ministry of Science} 

\keywords{Spanners, Directed graphs} \subjclass{F.2 ANALYSIS OF
ALGORITHMS AND PROBLEM COMPLEXITY, F.2.0 General }




\begin{abstract}
Let  be a finite metric space, where  is a set of
 points and  is a distance function defined for these
points. Assume that  has a constant doubling dimension
 and assume that each point  has a disk of radius
 around it. The disk graph that corresponds to  and
 is a \emph{directed} graph , whose vertices
are the points of  and whose edge set includes a directed edge
from  to  if . In~\cite{PeRo08} we
presented an algorithm for constructing a -spanner of
size , where  is the maximal radius .
The current paper presents two results. The first shows that the
spanner of~\cite{PeRo08} is essentially optimal, i.e., for metrics
of constant doubling dimension it is not possible to guarantee a
spanner whose size is independent of . The second result shows
that by slightly relaxing the requirements and allowing a small
perturbation of the radius assignment, considerably better
spanners can be constructed. In particular, we show that if it is
allowed to use edges of the disk graph , where
 for every , then it
is possible to get a -spanner of size  for
. Our algorithm is simple and can be implemented
efficiently.
\end{abstract}

\maketitle


\section*{Introduction}\label{S:one}

This paper concerns efficient constructions of spanners for disk
graphs, an important family of directed graphs. A {\em spanner} is
essentially a skeleton of the graph, namely, a sparse spanning
subgraph that faithfully represents distances. Formally, a
subgraph  of a graph  is a -spanner of  if
 for every two nodes 
and , where  denotes the distance between 
and  in . We refer to  as the {\em stretch factor} of
the spanner. Graph spanners have received considerable attention
over the last two decades, and were used implicitly or explicitly
as key ingredients of various distributed applications. It is
known how to efficiently construct a -spanner of size
 for every weighted undirected graph, and this
size-stretch tradeoff is conjectured to be tight. Baswana and
Sen~\cite{BaSe07} presented a linear time randomized algorithm for
computing such a spanner. In directed graphs, however, the
situation is different. No such general size-stretch tradeoff can
exist, as indicated by considering the example of a directed
bipartite graph  in which all the edges are directed from one
side to the other; clearly, the only spanner of  is  itself,
as any spanner for  must contain every edge.

The main difference between undirected and directed graphs is that
in undirected graphs the distances are symmetric, that is, a path
of a certain length from  to  can be used also from  to
. In directed graphs, however, the existence of a path from 
to  does not imply anything on the distance in the opposite
direction from  to . Hence, in order to obtain a spanner for
a directed graph one must impose some restriction either on the
graph or on its distances. In order to bypass the problem of
asymmetric distances of directed graphs, Cowen and
Wagner~\cite{CoWa99} introduced the notion of {\em roundtrip
distances} in which the distance between  and  is composed
of the shortest path from  to  plus the shortest path from
 to . It is easy to see that under this definition distances
are symmetric also in directed graphs. It is shown by Cowen and
Wagner~\cite{CoWa99} and later by Roditty, Thorup and
Zwick~\cite{RoThZw08} that methods of path approximations from
undirected graphs can work using more ideas also in directed
graphs when roundtrip distances are considered. Bollob{\'a}s,
Coppersmith and Elkin~\cite{BoCoEl05} introduced the notion of
{\em distance preservers} and showed that they exist also in
directed graphs.

In~\cite{PeRo08} we presented a spanner construction for directed
graphs without symmetric distances. The restriction that we
imposed on the graph was that it must be a disk graph. More
formally, let  be a finite metric space of constant
doubling dimension , where  is a set of  points and
 is a distance function defined for these points. A metric
is said to be of {\em constant doubling dimension} if a ball with
radius  can be covered by at most a constant number of balls of
radius . Every point  is assigned with a radius
. The disk graph that corresponds to  and  is a
directed graph , whose vertices are the points of 
and whose edge set includes a directed edge from  to  if 
is inside the disk of , that is, .
In~\cite{PeRo08} we presented an algorithm for constructing a
-spanner with size , where  is
the maximal radius. In the case that we remove the radius
restriction the resulted graph is the complete undirected graph
where the weight of every edge is the distance between its
endpoint. In such a case it is possible to create
-spanners of size ,
see~\cite{HaMe06},~\cite{GaoGuiNgu04} and \cite{Ro07b} for more
details. Moreover, when the radii are all the same and the graph
is the unit disk graph then it is also possible to create
-spanners of size ,
see~\cite{gao05geometric}, \cite{PeRo08}.

As a result of that, a natural question is whether a spanner size
of  in the case of directed disk graph is
indeed the best possible or maybe it is possible to get a spanner
of size  as in the cases of the complete graph and
the unit disk graph. For the case of the Euclidean metric space,
the answer turns out to be positive; a simple modification of the
Yao graph construction~\cite{Yao82} to fit the directed case
yields a directed spanner of size . However, the
question remains for more general metric spaces, and in particular
for the important family of metric spaces of bounded doubling
dimension.

In this paper we provide an answer for this question. We show that
our construction from~\cite{PeRo08} is essentially optimal by
providing a metric space with a constant doubling dimension and a
radius assignment whose corresponding disk graph has 
edges and none of its edges can be removed. (This does not
contradict our spanner construction from~\cite{PeRo08} as the
maximal radius in that case is  and hence .)

This (essentially negative) optimality result motivates our main
interest in the current paper, which focuses on attempts to
slightly relax the assumptions of the model, in order to obtain
sparser spanner constructions. Indeed, it turns out that such
sparser spanner constructions are feasible under a suitably
relaxed model. Specifically, we demonstrate the fact that if a
small perturbation of the radius assignment is allowed, then a
-spanner of size  is attainable. More
formally, we show that if we are allowed to use edges of the disk
graph , where  for every , then it is possible to get a
-spanner of size  for the original disk
graph . This approach is similar in its nature to the
notation of \textit{emulators} introduced by Dor, Halperin and
Zwick~\cite{DoHaZw00}. An emulator of a graph may use any edge
that does not exist in the graph in order to approximate its
distances. It was used in the context of spanners with an additive
stretch.

The main application of disk graph spanners is for topology
control in the wireless ad hoc network model. In this model the
power required for transmitting from  to  is commonly taken
to be , where  denotes the
distance between  and  and  is a constant typically
assumed to be between  and . Most of the ad hoc network
literature makes the assumption that the transmission range of all
nodes is identical, and consequently represents the network by a
\emph{unit disk graph} (UDG), namely, a graph in which two nodes
 are adjacent if their distance satisfies . A unit disk graph can have as many as  edges.

There is an extensive body of literature on spanners of unit disk
graphs. Gao et al.~\cite{gao05geometric}, Wang and
Yang-Li~\cite{WaLi06a} and Yang-Li et al.~\cite{Yang2003}
considered the restricted Delaunay graph, whose worst-case stretch
is constant (larger than ). In~\cite{PeRo08} we showed
that any -geometric spanner can be turned into a
-UDG spanner.

Disk graphs are a natural generalization of unit disk graphs, that
provide an intermediate model between the complete graph and the
unit disk graph. Our size efficient spanner construction for disk
graphs whose radii are allowed to be slightly larger falls exactly
into the model of networks in which the stations can change their
transmission power. In particular our constriction implies that if
any station increases its transmission power by a small fraction
then a considerably improved topology can be built for the
network.

Our result has both practical and theoretical implications. From a
practical point of view it shows that, in certain scenarios,
extending the transmission radii even by a small factor can
significantly improve the overall quality of the network topology.
The result is also very intriguing from a theoretical standpoint,
as to the best of our knowledge, our relaxed spanner is the first
example of a spanner construction for directed graphs that enjoys
the same properties as the best constructions for undirected
graphs. (As mentioned above, it is easy to see that for general
directed graphs, it is not possible to have an algorithm that
given any directed graph produces a sparse spanner for it.) In
that sense, our result can be viewed as a significant step towards
gaining a better understanding for some of the fundamental
differences between directed and undirected graphs. Our result
also opens several new research directions in the relaxed model of
disk graphs. The most obvious research questions that arise are
whether it is possible to obtain other objects that are known to
exist in undirected graphs, such as compact routing schemes and
distance oracles, for disk graphs as well.

The rest of this paper is organized as follows. In the next
section
we present a metric space of constant doubling dimension in which
no edge can be removed from its corresponding disk graph. Section
\ref{s:DG-3} first describes a simple variant of our construction
from~\cite{PeRo08}, and then uses it together with new ideas in
order to obtain our new relaxed construction. Finally, in Section
\ref{s:con} we present some concluding remarks and open problems.

\section{Optimality of the spanner construction}
\label{s:Bad-Example}


In this section we build a disk graph  with  vertices and
 edges that is non-sparsifiable, namely, whose only
spanner is  itself. In this graph  hence our
spanner construction from~\cite{PeRo08} has a size of
 and is essentially optimal.

Given a set of points, we present a distance function such that
for a given assignment of radii for the points any spanner of the
resulting disk graph must have  edges. We then prove
that the underlying metric space has a constant doubling
dimension.

We partition the points into two types, 
and . We now define the distance function
 and the radii assignment . The
main idea is to create a bipartite graph  in which every
point of  is connected by a directed edge to all the points of
.

The distance between any two points  and  is at least
 for some small  and the radius assignment of
every point  is exactly . Thus, there are no edges between
the points of .

We now define the distances between the points of  and the
points of . We start with the point . Let
 for every  and let .
Place the points of  on the boundary of a ball of radius 
centered at  such that the distance between any two
consecutive points  and  is exactly . This
is depicted in Figure~\ref{F-badexample}(a).

\begin{figure}[!t]
\begin{center}
\begin{picture}(0,0)\special{psfile=example-combined-tiny.pstex}\end{picture}\setlength{\unitlength}{3947sp}\begingroup\makeatletter\ifx\SetFigFont\undefined
\def\x#1#2#3#4#5#6#7\relax{\def\x{#1#2#3#4#5#6}}\expandafter\x\fmtname xxxxxx\relax \def\y{splain}\ifx\x\y   \gdef\SetFigFont#1#2#3{\ifnum #1<17\tiny\else \ifnum #1<20\small\else
  \ifnum #1<24\normalsize\else \ifnum #1<29\large\else
  \ifnum #1<34\Large\else \ifnum #1<41\LARGE\else
     \huge\fi\fi\fi\fi\fi\fi
  \csname #3\endcsname}\else
\gdef\SetFigFont#1#2#3{\begingroup
  \count@#1\relax \ifnum 25<\count@\count@25\fi
  \def\x{\endgroup\@setsize\SetFigFont{#2pt}}\expandafter\x
    \csname \romannumeral\the\count@ pt\expandafter\endcsname
    \csname @\romannumeral\the\count@ pt\endcsname
  \csname #3\endcsname}\fi
\fi\endgroup
\begin{picture}(4577,2152)(226,-1577)
\put(1151,-100){\makebox(0,0)[lb]{\smash{\SetFigFont{14}{16.8}{rm}}}}
\put(2147,-280){\makebox(0,0)[lb]{\smash{\SetFigFont{14}{16.8}{rm}}}}
\put(150,-267){\makebox(0,0)[lb]{\smash{\SetFigFont{14}{16.8}{rm}}}}
\put(150,-472){\makebox(0,0)[lb]{\smash{\SetFigFont{14}{16.8}{rm}}}}
\put(1054,-1250){\makebox(0,0)[lb]{\smash{\SetFigFont{14}{16.8}{rm}}}}
\put(3099, 79){\makebox(0,0)[lb]{\smash{\SetFigFont{14}{16.8}{rm}}}}
\put(3293,-1139){\makebox(0,0)[lb]{\smash{\SetFigFont{14}{16.8}{rm}}}}
\put(4073,-311){\makebox(0,0)[lb]{\smash{\SetFigFont{14}{16.8}{rm}}}}
\put(4608,-701){\makebox(0,0)[lb]{\smash{\SetFigFont{14}{16.8}{rm}}}}
\put(4803,-944){\makebox(0,0)[lb]{\smash{\SetFigFont{14}{16.8}{rm}}}}
\end{picture}
 \end{center}
\caption{(a) First step in constructing the non-sparsifiable disk
graph . (b) The non-sparsifiable disk graph .}
\label{F-badexample}
\end{figure}

Turning to the point , let  for every
, , and . Hence
there is an edge from  to all the points of , but no edge
connects  and .

We now turn to define the general case. Consider . Let
 and  for every . Let , and in general, for
every  we have

implying that

It is easy to verify that  has outgoing edges to the points
of  (and to them only) and it does not have any incoming edges.
See Figure \ref{F-badexample}(b).

The resulting disk graph  has  vertices and 
edges. Clearly, removing any edge from  will increase the
distance between its head and its tail to infinity, and thus the
only spanner of  is  itself.

It is left to show that the metric space defined above for  has
a constant doubling dimension. Given a metric space ,
its {\em doubling dimension} is defined to be the minimal value
 such that every ball  of radius  in the metric space can
be covered by  balls of radius . In the next Theorem
we prove that for the metric space described above,  is
constant.

\begin{theorem}
The metric space  defined for  has a constant
doubling dimension.
\end{theorem}
\begin{proof}
Let  be a ball with an arbitrary radius . We show that it is
possible to cover all the points of  within  using a
constant number of balls whose radius is . The proof is
divided into two cases.

{\bf Case a:} There is some  within the ball . (If
there is more than one such point, then let  be the point
whose index is maximal.) Let  be a ball of radius 
centered at . Clearly , so  contains all the
points of . In what follows we show that all the points of
 within  can be covered by a constant number of balls
of radius . Let  be the point within  whose index is
maximal.  We have to consider two possible scenarios. The first is
that . This implies that , hence . We now show that it is
possible to cover  by a constant number of balls of radius
. If , then only  is within  and it is
covered by a ball of radius  centered at itself. If , then  contains all the points of 
and . From packing arguments it follows that it is possible
to cover all the points of  by a constant number of balls of
radius , hence also by a constant number of balls of radius
. The point  itself is covered by a ball centered at
it. Finally, if , then  is
at least .
A ball centered at  of radius  covers every 
within , where , as . Hence, we cover  by balls of radius 
whose centers are , ,  and . We
cover  as before. This completes the first scenario,
where . Assume now that . This implies
that  and that ,
where the first inequality follows from the fact that 
and the second inequality follows from the fact that  is the
point with maximal index inside , hence, .
As , we get that
. Also, by (\ref{eq:dist2}),
. We conclude that  and that . A ball centered at
 of radius  covers every  within , where
, as . Hence, we can
cover  by balls of radius  whose centers are ,
,  and . We cover  as before.
This completes the first case.

{\bf Case b:} The ball  does not contain any point from .
The points of  are spread as appears in
Figure~\ref{F-badexample}(a), thus by standard packing arguments,
any ball that contains only points from  is covered by a
constant number of balls of half the radius.
\end{proof}

\section{Improved spanner in the relaxed disk graph model}
\label{s:DG-3}


The (negative) optimality result from the previous section
motivates us to look for a slightly relaxed definition of disk
graphs in which it will still be possible to create a spanner of
size .

Let  be a metric space of constant doubling dimension
 with a radius assignment  for its points and let
 be its corresponding disk graph. Assume that we
multiply the radius assignment of every point by a factor of
, for some , and let  be the
corresponding disk graph. It is easy to see that .
In this section we show that it is possible to create a
-spanner of size  if we are allowed to use
edges of . As a first step we present a simple variant of our
-spanner construction of size 
from~\cite{PeRo08}. This variation is needed in order to obtain
the efficient construction in the relaxed model which is presented
right afterwards.

\subsection{Spanners for general disk graphs} \label{s:DG}


Let  be a metric space of constant doubling dimension
and assume that any point  is the center of a ball of
radius , where  is taken from the range . In
this section we describe a simple variant of our construction
from~\cite{PeRo08}, which computes a -spanner with
 edges for a given disk graph. We then
use this variant, together with new ideas, in order to obtain (in
the next section) our main result, namely, a spanner with only
 edges.

The spanner construction algorithm receives as input a directed
graph  and an arbitrarily small (constant) approximation
factor , and constructs a set of spanner edges
, returning the spanner subgraph
. The construction of the spanner is
based on a hierarchical partition of the points of  that takes
into account the different radius of each point. The construction
operates as follows. Let  and  be two small
constants depending on , to be fixed later on. Assume
that the ball radii are scaled so that the smallest edge in the
disk graph is of weight . Let  be an integer from the range
 and let . The edges of  are partitioned into
classes by length, letting . Let  be the level of the
edge , that is,  such that . Let  be a point whose ball is of radius
. It follows that level  is the first
level in which  can have outgoing edges. We denote this level
by .

For every , starting
from , the edges of the class  are considered
by the algorithm in a non-decreasing order. (Assume that in each
class the edges are sorted by their weight.) In each stage of the
construction we maintain a set of pivots . Let  and
let  be the nearest neighbor of  among the points
of . For a pivot , define , namely, all the
points that have a directed edge to  and  is their nearest
neighbor from . We refer to  as the {\em close
neighborhood} of .

The algorithm is given in Figure~\ref{F-App-Alg}. Let  be
an edge considered by the algorithm in the th iteration. The
algorithm first checks whether  or  or both should be added
to the pivots set . The main change with respect
to~\cite{PeRo08} is that if  is assigned with a large enough
radius it might become a pivot when the edge  is examined.
When considering the edge , the algorithm acts according to
the following rule: If the distance from  to its nearest
neighbor in  is greater than  then  is
added to . If the distance from  to its nearest neighbor
in  is greater than  and the radius of  is
at least  then  is added to . To decide whether
the edge  is added to the spanner, the following two cases
are considered. The first case is when . In
this case, if there is no edge from the close neighborhood of 
to the close neighborhood of  then  is added to the
spanner. The second case is when . In this case,
if there is no edge from the close neighborhood of  to  then
 is added to the spanner. When  reaches , the algorithm handles all the edges that
belong to . This includes also edges whose weight
is , the minimal possible weight. The algorithm returns the
directed graph .

In what follows we prove that for suitably chosen  and
,  is a -spanner
with  edges of the directed graph
.

\begin{figure}[t]
\begin{center}
\framebox{\parbox[t]{12.0in}{
\begin{tabbing}
\ALGORITHM \SPAN5pt]
\\
\\
for \= to \\
\> for \= each  do\\
\>\> if\=\  then
 \\
\>\> if\=\  then
 \\
\>\> if\=\ \\
\>\>\> if\=\  s.t. \=
 \\
\>\>\>\> then \\
\>\> if\=\ \\
\>\>\> if\=\ 
s.t.  \\
\>\>\>\> then \\
\> \\
return 
\end{tabbing}
}}
\end{center}
\caption{ A high level implementation of the spanner construction
algorithm for \emph{general} disk graphs }\label{F-App-Alg}
\end{figure}

\begin{lemma}[Stretch]
Let , set  and let
 be the graph returned by Algorithm
. If  then .
\end{lemma}
\begin{proof}
Recall that the radii are scaled so that the shortest edge is of
weight . We prove that every directed edge of an arbitrary node
 is approximated with  stretch. Let ]. The proof is by induction on
. For a given node , the base of the induction is the
maximal value of  in which  has an edge in .
Let  be this value for , that is, the set 
contains the shortest edge that touches . Every other node is
at distance at least  away from , hence  is a pivot
at this stage and every edge that touches  from the set
 is added to .

Let  for some  and let
. Assume that  and let
. It follows from definition that  and .

If the edge  is not in the spanner, then there must be an
edge , where  and . The crucial observation
is that the radius of  and  is at least . By
the choice of , it follows that 
and . Thus, there is a (directed)
path from  to  of the form  whose length is .
However, only its middle edge, , is in
. The length of this edge is bounded by the length
of the edge  since the algorithm picked the minimal edge
that connects between the neighborhoods. This implies that the
length of  is at most .

By the inductive hypothesis, the edges  and
 whose weight is at most  are
approximated with  stretch. Thus, there is a path in
the spanner from  to  whose length is at most

and this can be bounded by
 As the edge  it follows that . It
remains to prove that , which follows directly from the choice of 
and .

If  then there must be an edge , where . Following similar
arguments to those used above it can be shown that there is a path
in the spanner from  to  of length at most
 and hence bounded by
.
\end{proof}

\paragraph{The size of the spanner.}
We now prove that the size of the spanner
 is . As a
first step, we state the following well-known lemma,
cf.~\cite{GaoGuiNgu04}.

\begin{lemma}\label{L-pack}[Packing Lemma]
If all points in a set  are at least  apart
from each other, then there are at most  points in
 within any ball  of radius .
\end{lemma}

The next lemma establishes a bound on the number of incoming
spanner edges that a point may be assigned on stage  of the algorithm.

\begin{lemma}\label{L-level-size}
Let  and let .
The total number of incoming edges of  that were added to the
spanner on stage  is .
\end{lemma}
\begin{proof}
Let  be a spanner edge and let . We
associate  to . From the spanner construction algorithm
it follows that this is the only incoming edge of  whose source
is in . Thus, this is the only incoming edge of 
which is associated to . Now consider all the incoming edges of
 on stage . The source of each of these edges is associated
to a unique pivot within distance of at most 
away from  and any two pivots are  apart from
each other. Using Lemma~\ref{L-pack}, we get that the number of
edges entering  is .
\end{proof}

It follows from the above lemma that the total number of edges
that were added to  in the main loop is
. The total cost of the construction
algorithm is . For more details on the construction
time see~\cite{PeRo08}.

\subsection{Spanner for relaxed disk graphs}

Let  be a metric space of constant doubling dimension
 with a radius assignment  for its points and let
 be its corresponding disk graph. Assume that we
multiply the radius assignment of every point by a factor of
, for some , and let  be the
corresponding disk graph. In this section we show that it is
possible to create a -spanner of  of size
 if we are allowed to use edges of .

Our construction consists of two stages: a building stage and a
pruning stage. The building stage creates two spanners,  and
, using the algorithm of Section~\ref{s:DG}, where  is the
spanner of  and  is the spanner of . In the pruning
stage we prune the union of these two spanners. Throughout the
pruning stage we use the radius assignment of each point before
the increase. Let  and let  be the first level in
which  can have outgoing edges, that is,  (recall that as the levels get larger
the edges get shorter). In the pruning stage we only prune
incoming edges of  whose level is below . In other
words, we do not touch the incoming edges of  that are shorter
than the radius of . The pruning is done as follow. Let . We keep in the spanner the incoming
edges of  that come from the first  different levels
below .

Let  be the resulting spanner and let  be the
remaining set of edges after the pruning step. In the remainder of
this section we show that the size of  is
 and its stretch with respect to the distances in
 is . We start by showing that the size of
 is . Notice that the first part of the
proof below is possible only due to the change we have done in the
previous section to our spanner construction from~\cite{PeRo08}.
Roughly speaking, given an edge  that is shorter than
 we use pivot selection also on 's side (and not only on
's) to sparisify the graph. This allows us to deal separately
with edges of  of length larger than  and those of length
smaller than .

\begin{lemma}\label{L-smaller}
.
\end{lemma}
\begin{proof}
Let  be a spanner edge that survived the pruning step.
There are two possible cases to consider.

The first case is that . Let 
and let  and . By packing
considerations similar to Lemma~\ref{L-level-size} it follows that
the total number of edges at level  that connects between two
pivots as the edge  that are associated with  (and with
) is . The distance between  and  is at
most , therefore at level  either
 or  are no longer pivots.

Let  and , that is,  is the first
pivot set that contains . Then we charge  with every
(incoming and outgoing) edge of this type from levels
 that is incident to . Now given such an edge
 whose level is , either  or  are not pivots in
level , which means that either  or  has been
charged for this edge, since one of them first becomes a pivot
between levels  and .

The second case is that . In this case, it
must be that level   is among the  first
different levels below  from which an incoming edge is
allowed to enter . Subsequently, we associate the edge 
with , as the total number of such edges that  can have is
.
\end{proof}


We now turn to prove that the stretch of the spanner 
with respect to the disk graph  is .

\begin{lemma}\label{L-Stretch1}
Let  be an edge of the spanner  that was pruned. We show
that there is a path in  whose length is at most
.
\end{lemma}
\begin{proof}
The proof is by induction on the lengths of the pruned edges. For
the induction base let  be the shortest edge that was
pruned. For every , let  be the head of an edge
whose level is the -th level below  from which
 has an incoming edge. Let  be a
sequence of points, where   and . As
, it follows that . Combining this with the fact that
 we get that
. Therefore,
.

The analysis distinguishes between two cases.

{\bf Case a:} There is a point  such that . This situation is depicted in
Figure~\ref{F-proofexample}. (If there is more than one point that
satisfies this requirement, take the one whose index is minimal.)

\par\bigskip\noindent {\bf Claim:}
.

\begin{proof}
For the sake of contradiction, assume that
. This implies
that

where the last inequality follows from the triangle inequality as
the distance between  and  is at most .
For every  we have , which implies that
.
Combined with (\ref{eq:qtqt-1}), we get
 If  we have  and this yields a
contradiction.
\end{proof}

We now focus our attention on the point .  The minimality
of  implies that .
By combining it with the triangle inequality we get that
. Therefore,
in the graph  there must be an edge from  to .

\begin{figure}[!t]
\begin{center}
\begin{picture}(0,0)\special{psfile=proof-small.pstex}\end{picture}\setlength{\unitlength}{3947sp}\begingroup\makeatletter\ifx\SetFigFont\undefined
\def\x#1#2#3#4#5#6#7\relax{\def\x{#1#2#3#4#5#6}}\expandafter\x\fmtname xxxxxx\relax \def\y{splain}\ifx\x\y   \gdef\SetFigFont#1#2#3{\ifnum #1<17\tiny\else \ifnum #1<20\small\else
  \ifnum #1<24\normalsize\else \ifnum #1<29\large\else
  \ifnum #1<34\Large\else \ifnum #1<41\LARGE\else
     \huge\fi\fi\fi\fi\fi\fi
  \csname #3\endcsname}\else
\gdef\SetFigFont#1#2#3{\begingroup
  \count@#1\relax \ifnum 25<\count@\count@25\fi
  \def\x{\endgroup\@setsize\SetFigFont{#2pt}}\expandafter\x
    \csname \romannumeral\the\count@ pt\expandafter\endcsname
    \csname @\romannumeral\the\count@ pt\endcsname
  \csname #3\endcsname}\fi
\fi\endgroup
\begin{picture}(3872,3484)(467,-2675)
\put(350,-871){\makebox(0,0)[lb]{\smash{\SetFigFont{14}{16.8}{rm}}}}
\put(3500,-871){\makebox(0,0)[lb]{\smash{\SetFigFont{14}{16.8}{rm}}}}
\put(3409,-1311){\makebox(0,0)[lb]{\smash{\SetFigFont{14}{16.8}{rm}}}}
\put(4030,-126){\makebox(0,0)[lb]{\smash{\SetFigFont{14}{16.8}{rm}}}}
\put(3831,-583){\makebox(0,0)[lb]{\smash{\SetFigFont{14}{16.8}{rm}}}}
\put(3831,300){\makebox(0,0)[lb]{\smash{\SetFigFont{14}{16.8}{rm}}}}
\end{picture}
 \end{center}
\caption{The case in which  exists} \label{F-proofexample}
\end{figure}


Let . There are two possible scenarios for the
spanner . The first scenario is when .
In this case, there is an edge in  from some  to , whose length is at most
.

There are  different levels below  from
which edges that belong to the spanners  and  are not being
pruned and survived to the spanner . We know that the
edge  is such an edge from the -th
non-empty level below . We also know that
. Therefore, as
the length of the edge  is at most
 it is within the 
non-empty levels below  and it is not pruned. We
can now build a path from  to  by concatenating three
segments as follows: A path from  to , the edge
 and a path from  to . The point  is
at most  away from  and
for the right choice of  it is less than
, hence the weight of every edge on the path
that approximates the distance between  and  in 
is less than , the shortest pruned edge, and the
entire path survived the punning stage. Similarly, the point
 is at most  away from  and again
for the right choice of  every edge on the path that
approximates the distance between  and  survived the
punning stage. Thus, we get that there is a path whose length is
at most

which is less than  for .

The second scenario is when . In this
case, there is an edge in  from some  to some 
whose length is at most , which is
not being pruned. We can build a path from  to  by
concatenating three segments as follows: A path from  to ,
the edge  and a path from  to . As before, for the
right choice of  the paths from  to  and from  to
 are composed from edges that are shorter from ,
the length of the shortest pruned edge, hence, from the minimality
 every edge on these paths survived the punning
stage. We get that there is a path whose length is at most

which is less than  for . This
completes the proof for case a.

{\bf Case b:} There is no point  such that .
In this case, let  be the last point in the sequence of
points , where  and . Similarly to before, there are two possible scenarios for
the spanner . Let . The first scenario is
when . In this case, there is an edge in
 from some  to  whose
length is at most . This edge could
not be pruned, since if it was pruned then  could not
have been the last point in the sequence. Hence we can construct a
path from  to  exactly as we have done in the first scenario
of case a, described above. The second scenario is when
. In this case, we can construct a path
from  to  exactly as we have done in the second scenario of
case a, described above.

This completes the proof of the induction base. The proof of the
general inductive step is almost identical. The only difference is
that when a path is constructed from  to , its portions from
 to  and from  to  in the first scenario and
from  to  and from  to  in the second scenario exist
in  by the induction hypothesis and not by the minimality
of .
\end{proof}

We end this section by stating its main Theorem. The proof of this
Theorem stems from Lemma~\ref{L-smaller} and
Lemma~\ref{L-Stretch1}.

\begin{theorem}
Let  be a metric space of constant doubling dimension
with a radius assignment  for its points and let
 be its corresponding disk graph. Let
 be the corresponding disk graph in the
relaxed model. It is possible to create a -spanner of
size  for  using edges of .
\end{theorem}

\section{Concluding remarks and open problems}
\label{s:con}


This paper presents a spanner construction for disk graphs in a
slightly relaxed model that is as good as spanners for complete
graphs and unit disk graphs. This result opens many other research
directions for disk graphs. We list here two questions that we
find particularly intriguing: Is it possible to design an
efficient compact routing scheme for disk graphs? And is it
possible to build an efficient distance oracle for disk graphs?

\bibliographystyle{plain}
\bibliography{peleg}

\end{document}
