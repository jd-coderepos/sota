\documentclass{sig-alternate}
\usepackage{jeffstyle}
\usepackage{epstopdf}
\newlength{\figsize} \setlength{\figsize}{0.22\textwidth}
\newlength{\widthfigsize} \setlength{\widthfigsize}{0.32\textwidth}
\usepackage[numbers]{natbib}
\begin{document}
\conferenceinfo{KDD}{2015 Sydney, Australia}

\title{Binary Coding in Stream}

\numberofauthors{2} 
\author{
\alignauthor
Mina Ghashami
       \\
       \affaddr{University of Utah}\\
       \affaddr{Utah, USA}\\
       \email{ghashami@cs.utah.edu}
\alignauthor
Amirali Abdullah
\\
       \affaddr{University of Utah}\\
       \affaddr{Utah, USA}\\
       \email{amirali@cs.utah.edu}
}

\date{20 Feb 2015}

\maketitle
\begin{abstract}
Big data is becoming ever more ubiquitous, ranging over massive video repositories, document corpuses, image sets and Internet routing history. Proximity search and clustering are two algorithmic primitives fundamental to data analysis, but suffer from the ``curse of dimensionality'' on these gigantic datasets. 
A popular attack for this problem is to convert object representations into short binary codewords, while approximately preserving near neighbor structure. However, there has been limited research on constructing codewords in the ``streaming" or ``online" settings often applicable to this scale of data, where one may only make a single pass over data too massive to fit in local memory. 

In this paper, we apply recent advances in matrix sketching techniques to construct binary codewords in both streaming and online setting. 
Our experimental results compete outperform several of the most popularly used algorithms, and we prove theoretical guarantees on performance in the streaming setting under mild assumptions on the data and randomness of the training set.
\end{abstract}

\section{Introduction}
Due to overwhelming increase in sheer volume of data being generated every day, fundamental algorithmic primitives of data analysis are being run on ever larger data sets. These primitives include approximating nearest neighbour search~\cite{indyk1998approximate, binarylsh}, clustering~\cite{ball1967clustering,kanungo2002efficient}, low dimensional embeddings~\cite{niyogi2004locality,belkin2001laplacian}, or learning distributions from a limited number of samples~\cite{learnability} etc.

A prominent approach for handling gigantic datasets is to convert object representations to short \textit{binary codewords} such that similar objects map to similar binary codes.
Binary representation is widely used in data analysis tasks, for example Song et.al~\cite{videosearch} gave an algorithm for converting a large video dataset into a set of binary hashes. Seo~\cite{seo} proposed a binary hashing scheme for music retrieval. Fergus, Weiss and Torralba~\cite{fergus2009semi} employed a spectral hashing scheme for labeling gigantic image datasets in semi-supervised setting. Julie and Triggs~\cite{recognition} used binary feature vectors for visual recognition of objects inside images. Guruswami and Sahai~\cite{ecc} give an embedding into Hamming space that reduces multi-class learning to an easier binary classification problem.

Codewords as succinct representation of data serve multiple purposes: 1) They can be used for dimensionality reduction, 2) They can emphasize user-desired distance thresholds, i.e. to encode data points such that near neighbors become much closer in Hamming space, rather than a simple proportionate embedding of distances, and 3) They allow the use of efficient tree based search data structures and enable the use of nearest neighbor techniques in Hamming space. (For more on how to conduct such searches quickly in Hamming space, see for instance the work by Norouzi, Punjani and Fleet~\cite{fasthamming} or by 
Esmaeili, Ward and Fatourechi~\cite{2012fast}.)



Sometimes these codes may be found trivially, e.g. if a dataset is already described by binary features, or is partitioned in a locality preserving and hierarchical manner. 
However where we are not so fortunate, we need to \emph{learn} them by seeking help of a constructive similarity function.
For instance,
unsupervised methods derive codewords from feature vectors in Euclidean space, or construct them from a data independent affinity matrix. 
On the opposite side, supervised methods \cite{torralba2008small,semantic} take additional contextual information into account and use a similarity notion that is semantically meaningful for codewords, e.g. two documents are similar if they are about the same topic or two images are similar if they contain same objects and colors.

On a meta-level, any binary coding scheme should satisfy three following properties to be considered effective: 
\begin{enumerate}
\item{The codes should be short so that we can store large datasets in memory.}
\item{Codes should be similarity-preserving; i.e., similar data points should map to similar binary codes while far data points should not collapse to small neighborhoods.}
\item{The learning algorithm should efficiently compute codes for newly inserted points without having to recompute the entire codebook.}
\end{enumerate}
The need to simultaneously satisfy all three constraints above makes learning binary codes a challenging problem.
Broadly speaking, binary coding techniques fall into two categories: first class is the family of techniques, referred to as \textit{symmetric}, which binarize both datapoints of a dataset or database and query points, usually according to the same hashing scheme. This class includes locality sensitive hashing(LSH)~\cite{indyk1998approximate}, spectral hashing~\cite{weiss2009spectral}, 
locality sensitive binary codes\cite{raginsky2009locality}, Iterative Quantization(ITQ)~\cite{gong2011iterative} or semi-supervised hashing\cite{wang2012semi} techniques.
In contrast, the second class of methods, namely \emph{asymmetric} algorithms, binarize only data points and not query points, e.g. ~\cite{jegou2011product,dong2008asymmetric,gordo2014asymmetric,jegou2010aggregating}. These methods achieve higher accuracy due to greater precision in the query description, yet still have the storage and efficiency gains from binarizing the ground dataset.

\section{Background and Notation}
First, we briefly review some notation. We use lower case letters to denote functions, e.g.  and upper case letters to represent matrices, e.g. . An  matrix  can be written as a set of  rows as  where each row  is a datapoint of length . Equivalently, this matrix can be written as a set of  columns as . The element at row  and column  of matrix  is denoted by .
The Frobenius norm of a matrix  is defined  where  is Euclidean norm of .  
Let  refer to the best rank  approximation of , specifically .  
The singular value decomposition of , written , produces three matrices  so that .  Matrices  and  are orthogonal and their columns are the left singular vectors and right singular vectors, respectively. Matrix  is all s except for the diagonal entries , the \emph{singular values}, where  is the rank.  Note that  for all , spectral norm of a matrix is , and  describes the norm along direction .  
Numeric rank of a matrix  is defined as  and trace of a square matrix  is 
For square matrix , eigen decomposition of  is  where  contains eigen vectors as columns, and  is a diagonal matrix containing eigen values  in non-increasing order. 
Finally, expected value of a matrix is defined as the matrix of expected values, i.e. 

\subsection{Related Works}
One of the basic and most popular binary encoding schemes is ``Locality Sensitive Hashing'' (LSH)\cite{datar2004locality} which uses random projections to embed data into lower dimensional space. This is done by employing a class of functions called locality-sensitive hash functions under which similar points collide with high probability.
A family of hash functions  is called -sensitive if for any two points  and any hash function , the following two properties hold:
\begin{enumerate}
\item{If  then  and 2) if  then , where  denotes the probability of an event under family of hash functions , and  is the hashed value of point  under hash function . Note in this definition  is a threshold on distance and  is an approximation ratio, and in order for LSH family to be useful it should be that .}
\item{LSH is a data independent method and can be done in streaming setting as it does not need to store data points and hashing or projection can be done on the fly. It is folklore that for random datasets LSH is near optimal, but in practice is generally outperformed by methods that use spectrum of data. To simplify somewhat,  bit binary codewords of LSH can be assigned to a point in  by taking dot product with a collection of  random vectors, and assigning each bit as  or  according to the sign of the value obtained~\cite{binarylsh}.}
\end{enumerate} 

One of the most famous binary encoding schemes is ``Spectral Hashing''(SH) \cite{weiss2009spectral}. If  is similarity matrix and  is the binary coding matrix for  being the length of codewords, then this method 
formulates the problem as minimizing  with subject to , balance constraint, i.e.  for each binary codeword , and evenly distributed constraint that enforce each bit be evenly distributed on  and  over the dataset.
It's not too hard to show that this optimization is equivalent to minimizing  where  is the matrix containing codewords,  is the degree matrix with . 
However, due to the binary constraint , this probelm is NP hard ,so instead authors threshold a spectral
relaxation whose solution is the bottom  eigenvectors of graph Laplacian matrix . 
This however provides a solution to only training datapoints. In order to extend it to out-of-samples, they assume datapoints are sampled from a separable probability distribution ; using the fact that graph Laplacian eigenvectors converge to the Laplace-Beltrami eigenfunctions of manifolds, they set thresholded eigen functions as codewords. However they only examine the simple case of a \textit{multidimensional uniform distribution} or box shaped data, as these eigen functions are well-studied. 

In \cite{fergus2009semi}, Fergus \etal extended their previous work\cite{weiss2009spectral} to \textit{any} separable distribution, i.e. any distribution  with a product form.
They consider semi-supervised learning in a graph setting, where a labeled dataset of input-output pairs 
 is given, and they need to label a larger set  of unlabelled points.
Authors form the graph of all datapoints , where vertices represent datapoints and edges are weighted with a Gaussian function . 
The goal is to find functions  which agree with labeled data but are also smooth with respect to the graph, therefore they formulate the problem as minimizing the error function , where  is the embedding of -th point and  is a diagonal matrix whose diagonal elements are
 if  is a labeled point and  otherwise.
Note that  is the smoothness operator defined on the entire graph Laplacian as , and  represents the loss on the labeled data.
Similar to their previous work\cite{weiss2009spectral} authors approximate the eigen vectors of  by eigen functions
of laplace-beltrami operator defined on probability distribution .

Finally, in the most recent work of this series, ``Multidimensional Spectral Hashing''(MDSH) \cite{weiss2012multidimensional}, Weiss \etal introduced a new formulation for learning binary codes; unlike other methods that minimize Hamming distance , MDSH  approximates original affinity  with weighted Hamming affinity , where  gives a weight to each bit.
The authors show the best binary codes are obtainable via performing binary matrix factorization of affinity matrix, with the optimal weights given by the singular values. 

SSH and MDSH can be adapted to the streaming setting, but have the unsatisfactory elements that neither addresses approximating the initial matrix optimization directly. Moreover the hashing functions (eigenfunctions) they learn are wholly determined by the initial training set and do not adapt as more points are streamed in.

In another line of works, authors formulate the problem as an iterative optimization. 
In \cite{gong2011iterative}, Gong and Lazebnik suggest ``Iterative Quantization''(ITQ) algorithm which is an iterative approach based on alternate minimization scheme that first projects datapoints onto top  right singular vectors of data matrix, and then takes the sign of projected vectors to produce binary codes. Authors show that if we consider projected datapoints as vectors in a -dimensional binary hypercube , then sign of vector entries in each dimension is determined by the closest vertex of hypercube along that dimension. As rotating this hypercube does not change the codes, they alternatively minimize the quantization loss  by fixing one of two variables  the binary codes, or  the rotation matrix, and solving for the other. They show that in practice repeating this for at most  iterations beats some well-known methods including \cite{fergus2009semi,raginsky2009locality} and spectral hashing\cite{weiss2009spectral}.

Heo \etal ~cite{spherical} present an iterative scheme that partitions points using hyperspheres. Specifically, the algorithm places  balls, such that the th bit a point  is 1 if it is contained in the -th ball and  otherwise. At each step of the process if the intersection of any two balls contains too many points, a repulsive force is applied between them, whereas if the intersection contains too few an attractive force is applied. This minimization continues until a reasonably balanced number of the points are contained in each hypersphere.  Both these iterative algorithms seem difficult to adapt to a streaming setting, in the sense that the hash functions are expensive to learn on a training set and not easily updated.

In the supervised setting, Quadrianto \etal~\cite{quadrianto} present a probabilistic model for learning and extending binary hash codes. They assume the input dataset follows certain simple and well studied probability distributions, and that supervision is provided in terms of labels indicating which points are neighbors and which are far. Under these constraints, they may train a latent feature model to extend binary hash codes in the streaming setting as new data points are provided.

In the broader context, the most comparable line of works with our problem is matrix sketching in the stream.
Although there has been a flurry of results in this direction\cite{numerical,ghashami1,Lib12}, we mention those which are most related to our current work.
In ~\cite{drineas2005nystrom}, Drineas and Mahoney approximate a gram matrix  by sampling  columns (datapoints) of an input matrix  proportional to the squared norm of columns. 
They approximate  with , where  is the gram matrix between  datapoints and  sampled points,  is the best rank  to  where  is the gram matrix between sampled points. 
They need to sample  columns to achieve Frobenius error bound , and need to sample  columns to get spectral error bound . Their algorithm needs  space and has running time of  on training set. The update time for any future datapoint (or test point) is .

The state-of-the-art matrix sketching technique is \FD algorithm first introduced by Liberty~\cite{Lib12} and then reanalyzed by Ghashami and Phillips~\cite{ghashami1}. \FD  maintains a deterministic, small space sketch for an input matrix and can be easily incrementally updated in the stream. In fact, for any input matrix , \FD maintains a sketch  with  rows, achieves error bound  and runs in time . It is shown by Woodruff that the approximation quality is optimal~\cite{lowerbound}. 

\subsection{Our Result}
 In this paper, we focus on finding codewords for a dataset  given in a stream. We consider an unsupervised setting where mutual similarity between datapoints is induced by Gaussian kernel function  rather than any contextual information. 
We develop a reasonable model of data holding two assumptions:
\begin{enumerate}
\item \textit{sparsity},  that enforces data similarity not being dominated by ``near-duplicates''.
\item \textit{bounded doubling dimension}, that is data has a low-dimensional structure. This assumption is widely used  as ``effective low-dimension'' in Euclidean near neighbor search problems \cite{KRu,KLee,BKL06,CG06,IN07,HPK13-lowd,abdullah}, and corresponds well with existence of a good binary codebook \footnote{A good binary codebook is roughly equivalent to a low distortion embedding into a low-dimensional Hamming space.}.
\end{enumerate}
Under this model, we propose the ``Streaming Spectral Binary Coding'' (SSBC) algorithm that builds off of \fd and shows that if training set is a ``good representor'' of the stream, i.e. that the stream is in random order, then one can accurately update important directions (eigen vectors) of the weight matrix in a stream. These vectors are then used to construct the desired codewords.

In fact, as we show in section \ref{sec:exp} our technique works in both streaming and online settings, achieves  space in former setting and  space in latter setting. Note both bounds are much smaller than  which is the required space for storing similarity matrix. 

Our starting matrix optimization formulation is closely adapted from those posed in this line of work by Fergus, Weiss and Torralba. 
However, our approach to \emph{solving} the problem and out-of-sample extension differs fundamentally from previous tactics of using functional approximation methods and learned eigenfunctions. We maintain a sketch of the weight matrix instead, and adjust it during the course of the stream. 
While known functional analysis techniques rely on assumptions on the data distribution (in particular that it is drawn from a separable distribution) we argue that solving for the matrix approximation directly addresses the original optimization problem without such restrictions, thereby achieving the superior accuracy our experiments demonstrate.

\section{Setup and Algorithm}\label{sec:algorithm}
In this section, we first set up matrix optimization problem that is the starting point of the work by Weiss, Fergus and Torralba~\cite{weiss2012multidimensional}, then we describe our algorithm ``Streaming Spectral Binary Coding'' (SSBC) for approximating binary codewords in a stream or online setting.
\subsection{Model and Setup}\label{sec:model}  
We denote input dataset as  containing  datapoints in  space and represent binary codes as , where  is a parameter specifying length of codewords.

We define affinity or similarity between datapoints  and  as  where  is a parameter set by user,  corresponding to a threshold between ``near" and ``far" distances.
Since codewords are vectors with  entries, one can write , and match \textit{Hamming affinity}  with   instead of minimizing Hamming distance. 
Similar to~\cite{weiss2012multidimensional}, we define a diagonal weight matrix  to give an importance weight  to -th bit of codewords.
Therefore we formulate the problem as:


This optimization problem is solvable by a binary matrix factorization of the affinity matrix, . As discussed in \cite{srebro2003weighted,weiss2012multidimensional}, the  binary constraint makes this problem computationally intractable, but a relaxation to real numbers results in a standard matrix factorization problem that is easily solvable. 
If  is eigen decomposition of , then -th row of  provides a codeword of length  for -th datapoint, which can be easily translated into a binary codeword by taking sign of entries.
The result binary codeword will be an approximation to the solution of binary matrix factorization. 

We consider solving binary encoding problem in two settings ``streaming'' and ``online'', where in both model one datapoint arrives at a time, is processed quickly and not read again. In the streaming setting, we output all binary codewords at the end of stream, while in the online setting, we are obliged to output binary codeword of current datapoint before seeing next datapoint.
Space usage is highly constrained in both models, so we cannot store the entire weight matrix  (of size ) nor even the dataset itself (of size ).   

Below, we specify assumptions we make in our data model for the purposes of theoretical analysis. However, we note that our experiments show strong results without enforcing any restrictions on the datasets we consider.

\begin{enumerate}
\item{Our first assumption is ``sparsity", namely that no two points  and  are asymptotically close to each other. Specifically, that , for all , where  is the threshold distance parameter of our Gaussian kernel. When our data is being analyzed for clustering/near neighbor purposes, this condition implies that identical points have either been removed or combined into a single representative point.}
\item{Our second assumption is that the data has bounded doubling dimension . Namely that a ball  of radius  contains at most  points spaced at distance at least . This is a standard model in the algorithms community for modeling data drawn from a low dimensional manifold. It is also intuitively compatible with the existence of a good representation of our data by -bit codewords for bounded , as binary encoding is simply an embedding into -dimensional Hamming space.}
\end{enumerate}


\subsection{Streaming Binary Coding Algorithm}
Our method, which we refer to as ``SSBC'' is described in algorithm \ref{alg:ssh}.
SSBC takes three input values ,  and  where  is a small training set sampled uniformly at random from the underlying distribution of data, e.g. . We denote size of  by , and we assume . For ease of analysis, wherever we come across some  to a constant exponent, we assume the term to be smaller than .
On the other hand,  is a potentially unbounded set of data points coming from same distribution . Even though  can be unbounded, for the sake of analysis, we denote total number of datapoints in union of both sets as .
Value  is the length of the codewords we seek.

The algorithm maintains a small sketch  with only  rows. For each datapoint , SSBC computes its (Gaussian) affinity with all points in , outputs an  dimensional vector  as the result, and inserts it into . Once  is full, SSBC takes \svd of  (), subtracts off smallest singular value squared, i.e. , from squared of all singular values, and reconstruct  as . This results in zeroing out last row of , and making space for processing next upcoming train datapoint. Note after processing , matrix  contains an -dimensional approximation to similarity structure of train set. 
As we observe, SSBC employs \FD algorithm \cite{Lib12} to process affinity vectors in streaming manner; instead of referring to \FD, we included its pseudocode completely in algorithm \ref{alg:ssh}. 
As many similarity measures can be used to capture the affinity between datapoints, SSBC uses \textit{Gaussian affinity} , where  is a parameter denoting the average near neighbor distance we care about. 
This function is called in subroutine \ref{alg:ha} to measure the affinity between any test point and all train datapoints.


At any point in time, we can get binary codeword of any datapoint  by first computing its affinity with , getting vector  as output and multiplying it by right singular vectors. More specifically if  denotes binary codeword of , then  gives a -length codeword. To get a codeword of length , we truncate  to its first  columns, . 


\begin{algorithm}
\caption{\label{alg:ssh} Streaming Spectral Binary Coding (SSBC)}
\begin{algorithmic}
\STATE \textbf{Input:} ,  as length of codeword
\STATE Define , , and 
\STATE Set  as sketch size
\STATE Set  to full zero matrix
\FOR {}
  \STATE )
  \STATE Insert  into a full zero row of 
  \IF { has no full zero rows}
    \STATE 
    \STATE 
    \STATE 
  \ENDIF
\ENDFOR
\STATE \textbf{Return} 
\end{algorithmic}
\end{algorithm}

A notable point about SSBC is that it can construct binary codewords on the fly in an online manner, i.e. using current iteration's matrix  to generate the binary codeword for current datapoint. 
As we show in section \ref{sec:err} this leads to the small space usage of . Clearly, SSBC can generate all codewords at the end of stream too (streaming setting); in that case it needs to store all  vectors and uses final matrix  to construct codewords. Space usage in streaming setting is .
The update time (or test time) in both models is . 

\begin{algorithm}
\caption{\label{alg:ha} Gaussian Affinity}
\begin{algorithmic}
\STATE \textbf{Input:}  as a test point, 
\STATE Define  to similarity threshold between points in 
\STATE Set  to zero vector, where 
\STATE Set 
\FOR { in }
  \STATE 
  \STATE 
\ENDFOR
\STATE \textbf{Return} 
\end{algorithmic}
\end{algorithm}

To explain good performance of SSBC, we argue that under the data model described in Section~\ref{sec:model}, squared norms of the columns of  are within a  factor of each other. Using this fact, we show a uniform sample of the columns of  is a good approximation to . In what follows, let ,  and  denote squared norm of -th column of , maximum and minimum squared norm of any column of  respectively.

\begin{lemma}
Under ``sparsity'' and ``bounded doubling dimension'' assumptions:

\end{lemma}
\begin{proof}
First note that it is trivially true that , since . We now upper bound . Let  denote squared norm of an arbitrary column of , so that upper bounding  would also bound . Let  be the corresponding datapoint associated with column .
We proceed by partitioning points of  close to (similar) and far (dissimilar) from  as  and , respectively.
Define , s.t.  and , s.t. . Note that the contribution of  to  is at most , and contribution of  to  is at most 
 . So we bound the size of . First we upper bound distance of any point  to point  as following:

Therefore .

Now considering the sparsity condition, we have that the number of points  within  distance of  is at most 
.
\end{proof}
We immediately get the following corollary as a consequence:
\begin{corollary}\label{cor:first}
It holds ,  that .
\end{corollary}
\begin{proof}
For the upper bound, we have , or . But for  arbitrary , we have  and hence . The lower bound on  follows similarly using .
\end{proof}

\section{Error Analysis}\label{sec:err}
In this section, we prove our main result. 
Let  be the exact affinity matrix of  datapoints in , where . 
Let  be size of training set and  be the rescaled affinity matrix between all points in  and . 
Under the assumption that  is drawn at random, we can imagine  is a column sample drawn uniformly at random from .
In the general case, column samples are only good matrix approximations to  if each column is drawn proportional to its norm, which is not known in advance in streaming setting. However we show that under our data model assumptions of Section \ref{sec:model}, a uniform sample suffices.
Define  to be the column of  that gets sampled for -th column of . (This corresponds to a choice of  as the -th point in ). Now define the scaling factor of  as . 
Define  as approximated affinity that could be constructed from  and the output of SSBC, i.e. . 
\footnote{Our algorithm does not actually construct  and . Rather we use them as existential objects for our theoretical analysis.}

 We show that for  and , then  with probability at least .
In our proof we use the Bernstein inequality on sum of zero-mean random matrices, which is stated below.

\paragraph{Matrix Bernstein Inequality}
Let  be independent random matrices such that for all ,  and  for a fixed constant . If we define variance parameter as 
Then for all :

Lemma below bounds spectral error between  and .
\begin{lemma}
\label{lem:cs_bernstein}
If  is the exact affinity matrix of points  and  is the affinity matrix between points in  and , then for 


holds with probability at least .
\end{lemma}
\begin{proof}
Consider  independent random variables . We can show  as follows

Note that last equality is correct because  is a symmetric matrix, and therefore .
We can now bound . Using this result we bound  as follows

Where the fourth line is achieved using Jensen's inequality on expected values, which states  for any random variable  and the third last line by Corollary \ref{cor:first}. 
Therefore  for all s.

In order to bound variance parameter , first note due to symmetry of matrices , its definition reduces to 

Where the last step follows since all the  are identical random variables. We already have an upper bound on the value  may achieve, and hence the square of this upper bounds . We bound  as follows:



Setting  and using Bernstein inequality with  we obtain 

Taking natural logarithm from both sides and inverse ratios, we get:

Considering that , we seek to bound:

Solving for  we obtain that for , the bound holds with probability at least .\end{proof}

Hence  has a similar spectrum to . In the lemma below, we argue that spectrum of  can be captured well by sketch . To this end we define  and show  is again similar to  
; intuitively  approximates projection of  onto the right singular vectors of .
\begin{lemma}
\label{lem:FD}
Let  be the affinity matrix between datapoints in  and . Then for  and 

\end{lemma}
\begin{proof}
We can bound  as following:

And we can also bound :

Putting the two bounds together we get:

Since we already showed  in lemma \ref{lem:cs_bernstein}, setting  suffices to complete the proof.
\end{proof}

\begin{theorem}\label{thm:eq}
Let  be similarity matrix of , and  be the weight matrix constructed by  and , where  is the set of columns sampled with replacement from , and  is the output of algorithm \ref{alg:ssh}. Then for  and :

holds with probability .
\end{theorem}
\begin{proof}
Having results of Lemmas \ref{lem:cs_bernstein} and \ref{lem:FD}, assuming  to be sufficiently large to meet conditions of both Lemmas and using triangle inequality and rescaling  to  proves the result.
\end{proof}

We explain some informal intuition of what Theorem \ref{thm:eq} implies. First we could infer  
is small.  Now writing the SVD decomposition of 
 as , we get . The intuition then is that if  is similar to , then , which is just the projection of  onto the right singular space of . 
This suggests that the right singular space of  captures most of the spectrum of , in the sense that a column sample of  projected on the right singular space of  and scaled appropriately recovers  closely. 

\section{Experiments}
\label{sec:exp}
Herein we describe an extensive set of experiments on a wide variety of large input data sets.
We ran all algorithms under a common implementation framework using Matlab to have a fair basis for comparision.

We compared efficiency and accuracy of our algorithm (SSBC) versus well-known streaming binary encoding techniques, including ``Multidimensional Spectral Hashing''(MDSH)\cite{weiss2012multidimensional},``Locality Sensitive Hashing'' (LSH)\cite{datar2004locality} and ``Spectral Hashing'' (SH)\cite{weiss2009spectral}.

We also compare accuracy of these algorithms against exact solution for binary coding problem when posed as a matrix optimization. As the exact solution, we compute the affinity matrix of whole dataset , and take the eigen decomposition of that. Let  denote the affinity matrix for . If  is the eigen decomposition of , then -row of  matrix provides a binary code of length  to -th datapoint in . In our experiments, we considered two types of thresholding on exact solution, namely deterministic rounding and randomized rounding. The deterministic rounding version is called ``Exact-D'' in the plots, and it basically takes the sign of  only. The randomized rounding one is called ``Exact-R'' in the plots, and what it does is that after computing  it multiplies it by a random rotation matrix , and then takes the sign of entries. 

\paragraph{Datasets}
We compare performance of our algorithm on both synthetic and real datasets.
Each data set is divided into two subsets,  and , with same number of dimensions and different number of datapoints. 
Table \ref{tbl:datasets} lists all datasets along with some statistics about them.
We refer to each set as an  matrix , with  datapoints and  dimensions. Training Set is taken small in size so that it easily fits into memory, while  is a large stream of data whose datapoints are processed one-by-one by our algorithm.    

As synthetic dataset we used multidimensional uniform distribution with  dimensions in which -th dimension  has a uniform distribution in range . In spectral hashing algorithm\cite{weiss2009spectral}, authors argue their learned eigenfunctions converge most sharply for rectangle distribution and include experimental results on uniform distributions demonstrating this efficacy. We added such dataset here so as to evaluate SSBC for a dataset model well suited to their algorithm.

\begin{table}[t!!!!]
\begin{center}
\begin{tabular}{|c||c|c|c|c|c|}
\hline
\textbf{DataSet} & \textbf{\# Train} & \textbf{\# Test} & \textbf{Dimension} & \textbf{Rank} \\
\hline
\hline
\textsf{PAMAP} & 100 & 21000 & 44 & 44 \\
\hline
\textsf{CBM}& 200 & 11000 & 18 & 16  \\  
\hline   
\textsf{Uniform} & 500 & 10000 & 50 & 50 \\
\hline 
\textsf{Covtype} & 500 & 20000 & 54 & 53 \\  
\hline 
\end{tabular} 
\end{center}
\vspace{-2mm}
\caption{\label{tbl:datasets} Datasets Statistics.}
\end{table}

We used three real-world datasets in our experiments. In each dataset, we uniformly sampled a small subset of data at random and considered it as , and used a subset of remaining part as . Information about size of training set and test set is provided in table \ref{tbl:datasets}.
First real-world dataset was the famous \s{Covtype}\cite{Covtype} that contains information about predicting forest cover type from cartographic variables.
Second one was \s{CBM} or ``Condition Based Maintenance of Naval Propulsion Plants''\cite{Coraddu2013Machine} which is a dataset generated from simulator of a gas turbine propulsion plant. It contains  datapoints in  dimensional space.

The \s{PAMAP}\cite{reiss2012introducing} dataset is a Physical Activity Monitoring dataset that contains data of  different physical activities (such as walking, cycling, playing soccer, etc.), performed by  subjects wearing  inertial measurement units and a heart rate monitor. The dataset contains  columns including a timestamp, an activity label (the ground truth) and  attributes of raw sensory data. In our experiments, we removed columns containing missing values and used a subset with  columns.

\paragraph{Metrics}
We use three following metrics to compare accuracy of discussed algorithms:
\begin{itemize}
\item \textit{Precision}: The number of true similar datapoints returned by an algorithm over total number of datapoints returned by the algorithm.
\item \textit{Recall}: The number of true similar datapoints returned by an algorithm over correct number of similar datapoints.
\item \textit{Mean Average Precision (MAP)}: The mean of the average precision scores for each test point.
\end{itemize}

We have used the Guassian function  to compute affinity between any two datapoints  and . We set  in each dataset to the average distance of all train datapoints to their -th nearest neighbour, and set this threshold in both Hamming and Euclidean space to designate whether two points are similar. We refer to this parameter as . We have used  in all the experiments involving ``precision" and ``recall" metrics. 
For Mean Average Precision(MAP) metric, we consider  different similarity levels comprising , the average of all pairs distance in training set () , and . In all cases, we set the choice of the  parameter in our Gaussian weight kernel equal to our similarity threshold for classifying points as near. The number of bits we use ranges from  to  with increments of .



\begin{figure}[t!]
\begin{centering}
\includegraphics[width=\figsize]{precision_100_21000_44_30_pamap.eps}
\includegraphics[width=\figsize]{recall_100_21000_44_30_pamap.eps}
\includegraphics[width=\figsize]{map_100_21000_44_30_pamap.eps}
\includegraphics[width=\figsize]{precVrec_100_21000_44_30_pamap.eps}
\vspace{-3mm}
\caption{\label{fig:pamap}
Results on \s{PAMAP} dataset.}  
\vspace{-2mm}
\end{centering}
\end{figure}


\begin{figure}[t!]
\begin{centering}
\includegraphics[width=\figsize]{precision_200_11000_18_30_CBM.eps}
\includegraphics[width=\figsize]{recall_200_11000_18_30_CBM.eps}
\includegraphics[width=\figsize]{map_200_11000_18_30_CBM.eps}
\includegraphics[width=\figsize]{precVrec_200_11000_18_30_CBM.eps}
\vspace{-3mm}
\caption{\label{fig:cbm}
Results on \s{CBM} dataset.}  
\vspace{-2mm}
\end{centering}
\end{figure}


\begin{figure}[t!]
\begin{centering}
\includegraphics[width=\figsize]{precision_500_10000_50_30_uniform.eps}
\includegraphics[width=\figsize]{recall_500_10000_50_30_uniform.eps}
\includegraphics[width=\figsize]{map_500_10000_50_30_uniform.eps}
\includegraphics[width=\figsize]{precVrec_500_10000_50_30_uniform.eps}
\vspace{-3mm}
\caption{\label{fig:uniform}
Results on \s{Uniform} dataset.}  
\vspace{-2mm}
\end{centering}
\end{figure}


\begin{figure}[t!]
\begin{centering}
\includegraphics[width=\figsize]{precision_500_20000_55_30_covtype.eps}
\includegraphics[width=\figsize]{recall_500_20000_55_30_covtype.eps}
\includegraphics[width=\figsize]{map_500_20000_55_30_covtype.eps}
\includegraphics[width=\figsize]{precVrec_500_20000_55_30_covtype.eps}
vspace{-3mm}
\caption{\label{fig:covtype}
Results on \s{Covtype} dataset.}  
\vspace{-2mm}
\end{centering}
\end{figure}


\begin{figure}[t!]
\begin{centering}
\includegraphics[width=\figsize]{map_200_1000_18_30_CBM.eps}
\includegraphics[width=\figsize]{recall_200_1000_18_30_CBM.eps}
\includegraphics[width=\figsize]{map_100_3000_44_30_pamap.eps}
\includegraphics[width=\figsize]{recall_100_3000_44_30_pamap.eps}
\vspace{-3mm}
\caption{\label{fig:exact_cbm_pamap}
Comparing algorithms with Exact methods on \s{CBM} (first row) and \s{PAMAP} (second row) datasets. Training set for each dataset was of size  and  and test set was of size  and , respectively.}  
\vspace{-2mm}
\end{centering}
\end{figure}

As we observe in precision and recall plots of figures \ref{fig:uniform},\ref{fig:covtype},\ref{fig:cbm} and \ref{fig:pamap}, SSBC performs exceptionally well on precision, providing very few ``false positives" compared to the other algorithms and consistently providing the highest precision of the methods evaluated. 
On recall metric also SSBC provides the best results over all the approaches evaluated. In both cases, this edge in performance is maintained over all tested ranges of length  of codewords. Combining these two plots we get precision-recall comparison (last plot in all above mentioned figures) which shows that SSBC forms an almost 45-degree line in all figures, i.e. basically its mistake rate does not increase by returning more candidates for nearest neighbours (having high recall).

In a separate set of experiments, we compared accuracy of all algorithms with exact methods. This time in order to allow exact algorithms to load the whole  by  weight matrix in RAM, we used a much smaller test set. Size of test set and training set for these experiments are mentioned in caption of plot \ref{fig:exact_cbm_pamap}.
As we see in this plot, SSBC secures higher mean average precision and recall than the exact methods, ``exact-D'' and ``exact-R'' which solve the matrix optimization by applying an SVD over enitre dataset. This is likely because maintaining a column sample of the weight matrix through a training set helps prevent overfitting errors.

\bibliographystyle{abbrv}
\bibliography{hashing}  

\end{document}
