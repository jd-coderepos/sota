\documentclass[submission]{eptcs}
\providecommand{\event}{WORDS 2011} \usepackage{amssymb}
\usepackage{amsthm}
\usepackage{amsmath}

    \newtheorem{theorem}{Theorem}[section]
    \newtheorem{lemma}[theorem]{Lemma}
    \newtheorem{corollary}[theorem]{Corollary}
    \theoremstyle{definition}
    \newtheorem{example}[theorem]{Example}

    \newcommand{\eps}{\varepsilon}
    \newcommand{\dd}{\cdots}
\newcommand{\Z}{\mathbb Z}
    \newcommand{\Q}{\mathbb Q}
    \newcommand{\No}{\mathbb N_0}
    \newcommand{\Ni}{\mathbb N_1}
    \newcommand{\set}[2]{ \left\{ #1 \mid #2 \right\} }
    \newcommand{\px}[1]{ P_{#1} }
\newcommand{\qx}[3]{ Q_{#1,#2,#3} }
    \newcommand{\sx}[2]{ S_{#1,#2} }
    \newcommand{\len}[1]{\mathrm{len}_{#1}}
    \newcommand{\lhp}{\mathcal M}

\begin{document}

\title{Systems of Word Equations and Polynomials: \\ a New Approach
    \thanks{Supported by the Academy of Finland under grant 121419}
}
\author{Aleksi Saarela \institute{
    Turku Centre for Computer Science TUCS and Department of Mathematics \\
    University of Turku, FI-20014 Turku, FINLAND
    \email{amsaar@utu.fi}
} }
\def\titlerunning{Systems of Word Equations and Polynomials: a New Approach}
\def\authorrunning{Aleksi Saarela}

\maketitle

\begin{abstract}
We develop new polynomial methods for studying systems of word
equations. We use them to improve some earlier results and to
analyze how sizes of systems of word equations satisfying certain
independence properties depend on the lengths of the equations.
These methods give the first nontrivial upper bounds for the sizes
of the systems.
\end{abstract}

\section{Introduction}

Word equations are a fundamental part of combinatorics on words, see
e.g. \cite{Lo83} or \cite{ChKa97} for a general reference on these
subjects. One of the basic results in the theory of word equations
is that a nontrivial equation causes a defect effect. In other
words, if  words satisfy a nontrivial relation, then they can be
represented as products of  words. Not much is known about the
additional restrictions caused by several independent relations
\cite{HaKa04}.

In fact, even the following simple question, formulated already in
\cite{CuKa83}, is still unanswered: how large can an independent
system of word equations on three unknowns be? The largest known
examples consist of three equations. The only known upper bound
comes from the Ehrenfeucht Compactness Property, proved in
\cite{AlLa85} and independently in \cite{Gu86}: an independent
system cannot be infinite. This question can be obviously asked also
in the case of  unknowns. Then there are independent systems
of size  \cite{KaPl96}. Some results concerning
independent systems on three unknowns can be found in \cite{HaNo03},
\cite{CzKa07} and \cite{CzPl09}, but the open problem seems to be
very difficult to approach with current techniques.

There are many variations of the above question: we may study it in
the free semigroup, i.e. require that  for every
solution  and unknown , or examine only the systems having a
solution of rank , or study chains of solution sets instead of
independent systems. See e.g. \cite{HaKaPl02}, \cite{HaKa04},
\cite{Cz08} and \cite{KaSa11}.

In this article we will try to use polynomials to study some
questions related to systems of word equations. Algebraic techniques
have been used before, most notably in the proof of Ehrenfeucht's
conjecture, which is based on Hilbert's Basis Theorem. However, the
way in which we use polynomials is quite different and allows us to
apply linear algebra to the problems.

One of the main contributions of this article is the development of
new methods for attacking problems on word equations. This is done
in Sections \ref{sect:fixedlength} and \ref{sect:solsets}. Other
contributions include simplified proofs and generalizations for old
results in Sect. \ref{sect:appl} and in the end of Sect.
\ref{sect:solsets}, and studying maximal sizes of independent
systems of equations in Sect. \ref{sect:indsyst}. Thus the
connection between word equations and linear algebra is not only
theoretically interesting, but is also shown to be very useful at
establishing simple-looking results that have been previously
unknown, or that have had only very complicated proofs. In addition
to the results of the paper, we believe that the techniques may be
useful in further analysis of word equations.

Now we give a brief overview of the paper. First, in Sect.
\ref{sect:basic} we define a way to transform words into polynomials
and prove some basic results using these polynomials.

In Sect. \ref{sect:fixedlength} we prove that if the lengths of the
unknowns are fixed, then there is a connection between the ranks of
solutions of a system of equations and the rank of a certain
polynomial matrix. This theorem is very important for all the later
results.

Section \ref{sect:appl} contains small generalizations of two
earlier results. These are nice examples of the methods developed in
Sect. \ref{sect:fixedlength} and have independent interest, but they
are not important for the later sections.

In Sect. \ref{sect:solsets} we analyze the results of Sect.
\ref{sect:fixedlength}, when the lengths of the unknowns are not
fixed. For every solution these lengths form an -dimensional
vector, called the \emph{length type} of the solution. We prove that
the length types of all solutions of rank  of a pair of
equations are covered by a finite union of -dimensional
subspaces, if the equations are not equivalent on solutions of rank
. This means that the solution sets of pairs of equations are
in some sense more structured than the solution sets of single
equations. This theorem is the key to proving the remaining results.
We conclude Sect. \ref{sect:solsets} by proving a theorem about
unbalanced equations. This gives a considerably simpler reproof and
a generalization of a result in \cite{HaNo03}

Finally, in Sect. \ref{sect:indsyst} we return to the question about
sizes of independent systems. There is a trivial bound for the size
of a system depending on the length of the longest equation, because
there are only exponentially many equations of a fixed length. We
prove that if the system is independent even when considering only
solutions of rank , then there is an upper bound for the size
of the system depending quadratically on the length of the shortest
equation. Even though it does not give a fixed bound even in the
case of three unknowns, it is a first result of its type -- hence
opening, we hope, a new avenue for future research.

\section{Basic Theorems} \label{sect:basic}

Let  be the length of a word  and  be the number of
occurrences of a letter  in . We use the notation ,
if  is a prefix of . We denote the set of nonnegative integers
by  and the set of positive integers by . The empty word
is denoted by .

In this section we give proofs for some well-known results. These
serve as examples of the polynomial methods used. Even though the
standard proofs of these are simple, we hope that the proofs given
here illustrate how properties of words can be formulated and proved
in terms of polynomials.

Let  be an alphabet of numbers. For a word  we define a polynomial

Now  is an injective mapping from words to
polynomials (here we need the assumption ). If
, then

If  and , then


The polynomial  can be viewed as a characteristic polynomial
of the word . We could also replace  with a suitable number
 and get a number whose reverse -ary representation is . Or
we could let the coefficients of  be from some other
commutative ring than . Similar ideas have been used to analyze
words in many places, see e.g. \cite{Ku97}, \cite{Sa85} and
\cite{HoKo09}.

\begin{example}
If , then

\end{example}

A word  is \emph{primitive}, if it is not of the form  for
any . If  and  is primitive, then  is a
\emph{primitive root} of .

\begin{lemma} \label{lem:primdiv}
If  is primitive, then  is not divisible by any
polynomial of the form

where  is a divisor of .
\end{lemma}
\begin{proof}
If  is divisible by , then there
are numbers  such that

so .
\end{proof}

The next two theorems are among the most basic and well-known
results in combinatorics on words (except for item \eqref{item:r} of
Theorem \ref{thm:commutation}).

\begin{theorem}
Every nonempty word has a unique primitive root.
\end{theorem}
\begin{proof}
Let , where  and  are primitive. We need to show
that . We have

Because , we get

If , then .
Thus  must be divisible by  and
 must be divisible by . By Lemma
\ref{lem:primdiv}, both  and  can be primitive only if .
\end{proof}

The primitive root of a word  is denoted by
.

\begin{theorem} \label{thm:commutation}
For , the following are equivalent:
\begin{enumerate}
\item , \label{item:rho}
\item if  and , then , \label{item:all}
\item  and  satisfy a nontrivial relation, \label{item:exists}
\item . \label{item:r}
\end{enumerate}
\end{theorem}
\begin{proof}
(\ref{item:rho})  (\ref{item:all}):


(\ref{item:all})  (\ref{item:exists}): Clear.

(\ref{item:exists})  (\ref{item:r}): Let

where . Now

for some polynomial . If  or  for some ,
then , and thus .

(\ref{item:r})  (\ref{item:rho}): We have

so  and

\end{proof}

Similarly, polynomials can be used to give a simple proof for the
theorem of Fine and Wilf. In fact, one of the original proofs in
\cite{FiWi65} uses power series. Algebraic techniques have also been
used to prove variations of this theorem \cite{MiShWa01}.

\begin{theorem}[Fine and Wilf] \label{thm:finewilf}
If  and  have a common prefix of length

then .
\end{theorem}

\section{Solutions of Fixed Length} \label{sect:fixedlength}

In this section we apply polynomial techniques to word equations.
From now on, we will assume that the unknowns are ordered as  and that  is the set of these unknowns.

A (coefficient-free) \emph{word equation}  on  unknowns
consists of two words . A \emph{solution} of this
equation is any morphism  such that . The equation is \emph{trivial}, if  and  are the same
word.

The (combinatorial) \emph{rank} of a morphism  is the smallest
number  for which there is a set  of  words such that  for every unknown . A morphism of rank at most one is
\emph{periodic}.

Let  be a morphism. The \emph{length type} of
 is the vector

This length type  determines a morphism .

For a word equation , where , a variable  and a length type , let


\begin{theorem} \label{thm:weqpeq}
A morphism  of length type  is a solution
of an equation  if and only if

\end{theorem}
\begin{proof}
Now  if and only if , and the
polynomial  can be written as  by \eqref{eq:prodp}.
\end{proof}

\begin{example}
Let ,  and . Now

If  is the morphism defined by ,  and , then  is a solution of  and

\end{example}

A morphism  is an \emph{elementary
transformation}, if there are  so that  and  for .
If , then  is \emph{regular}, and if , then  is \emph{singular}. The next lemma follows
immediately from results in \cite{Lo83}.

\begin{lemma} \label{lem:elemtrans}
Every solution  of an equation  has a factorization

where  for all ,

every  is an elementary transformation and  is a solution of . If  for  unknowns
 and  of the  are singular, then the rank of  is .
\end{lemma}

\begin{lemma} \label{lem:rdim}
Let  be an equation on  unknowns. Let  be a solution of length type  that has rank . There
is an -dimensional subspace  of  such that  but
those length types of the solutions of  of rank  that are in
 are not covered by any finite union of -dimensional
spaces.
\end{lemma}
\begin{proof}
Let

as in Lemma \ref{lem:elemtrans}. Let . Now  is a solution of  for
every morphism . The length type of  is

To prove the theorem, we need to show that at least  of the
vectors in this sum are linearly independent.

Let  be the  matrix

If there are  unknowns  such that , then the
rank of  is . If  is regular, then the matrix
 is obtained from  by adding one of the columns to
another column, so the ranks of these matrices are equal. If
 is singular, then  is obtained from  by
adding one of the columns to another column and setting some column
to zero, so the rank of the matrix is decreased by at most one. If
 of the  are singular, then the rank of  is at least
. The rank of  is , so  and at
least  of the columns of  are linearly independent.
\end{proof}

\begin{lemma} \label{lem:rdim2}
Let  be an equation and  be a
solution of length type  that has rank . There are morphisms
 and  and polynomials
 such that the following conditions hold:
\begin{enumerate}
\item ,
\item  is a solution of ,
\item  for all
    , if  is a morphism of the same
    length type as ,
\item  of the vectors , where
    , are linearly independent.
\end{enumerate}
\end{lemma}
\begin{proof}
Let  be as in the proof of Lemma \ref{lem:rdim} and let 
be such that . For every , there are
polynomials  so that

for all  ( ``encodes'' the positions
of the word  in ). Let  be the 
matrix

The matrix  is obtained from  by adding one of the
columns to another column, and multiplying some column with a
polynomial. Like in Lemma \ref{lem:rdim}, we conclude that at least
 of the columns of  are linearly independent and . If we let , then the four conditions hold.
\end{proof}

With the help of these lemmas, we are going to analyze solutions of
some fixed length type. Fundamental solutions (which were implicitly
present in the previous lemmas, see \cite{Lo83}) have been used in
connection with fixed lengths also in \cite{Ho01} and \cite{Ho00}.


\begin{theorem} \label{thm:rank}
Let  be a system of equations on  unknowns and
let . Let

If the system has a solution of length type  that has rank ,
then the rank of the  matrix  is at most
. If the rank of the matrix is 1, at most one component of 
is zero and the equations are nontrivial, then they have the same
solutions of length type .
\end{theorem}
\begin{proof}
Let  be a solution of length type  that has rank . If
, the first claim follows from Theorem \ref{thm:weqpeq}, so
assume that . Let  be an equation that has the same
nonperiodic solutions as the system. We will use Lemma
\ref{lem:rdim2} for this equation. Fix  and let  be the morphism determined by

and

for all  (we assumed earlier that , but it
does not matter here). Then  is a solution of every
,

and

for all  by Theorem \ref{thm:weqpeq}. Thus the vectors  are solutions of the linear system of equations
determined by the matrix . Because at least  of these
vectors are linearly independent, the rank of the matrix is at most
.

If at most one component of  is zero and the equations are
nontrivial, then all rows of the matrix are nonzero. If also the
rank of the matrix is 1, then all rows are multiples of each other
and the second claim follows by Theorem \ref{thm:weqpeq}.
\end{proof}

\section{Applications} \label{sect:appl}

The \emph{graph} of a system of word equations is the graph, where
 is the set of vertices and there is an edge between  and
, if one of the equations in the system is of the form . The following well-known theorem can be proved with the help
of Theorem \ref{thm:rank}.

\begin{theorem}[Graph Lemma] \label{thm:graph}
Consider a system of equations whose graph has  connected
components. If  is a solution of this system and  for all , then  has rank at most .
\end{theorem}
\begin{proof}
We can assume that the connected components are

and the equations are

where 
and . Let  be as in Theorem \ref{thm:rank}. If we
remove the columns  from the 
matrix , we obtain a square matrix , where the diagonal
elements are not divisible by , but all elements above the
diagonal are divisible by . This means that  is not
divisible by , so . Thus the rank of the matrix
 is  and  has rank at most  by Theorem
\ref{thm:rank}.
\end{proof}

The next theorem generalizes a result from \cite{CzKa07} for more
than three unknowns.

\begin{theorem}
If a pair of nontrivial equations on  unknowns has a solution 
of rank , where no two of the unknowns commute, then there is a
number  such that the equations are of the form

\end{theorem}
\begin{proof}
By Theorem \ref{thm:graph}, the equations must be of the form . Let them be

where  and . We can assume that  and

If it would be , then  and 
would commute, so . If  would contain
, then  and  would commute by Theorem
\ref{thm:finewilf}, so  for some .

Let  be the length type of  and let  be as in Theorem
\ref{thm:rank}. By Theorem \ref{thm:rank}, the rank of the matrix
 must be 1 and thus

The term of

of the lowest degree is . The same must hold for

and thus the term of  of the lowest degree must be
. This means that  and  . As above, we conclude that ,  cannot contain  and .
\end{proof}

It was proved in \cite{Ko98} that if

holds for  consecutive values of , then it holds for all
. By using similar ideas as in Theorem \ref{thm:rank}, we improve
this bound to  and prove that the values do not need to be
consecutive. In \cite{Ko98} it was also stated that the
arithmetization and matrix techniques in \cite{Tu87} would give a
simpler proof of a weaker result. Similar questions have been
studied in \cite{HoKo07} and there are relations to independent
systems \cite{Pl03}.

\begin{theorem}
Let ,  and . Let

and

If  holds for  values of , then it holds for all
.
\end{theorem}
\begin{proof}
The equation  is equivalent with .
This equation can be written as

where  are some polynomials, which do not depend on ,
and  is the set of those  for which  is not any of the numbers  (). If  and , then

Thus  and the size of  is at
most . If \eqref{eq:1} holds for  values
of , it can be viewed as a system of equations, where 
are unknowns. The coefficients of this system form a generalized
Vandermonde matrix, whose determinant is nonzero, so the system has
a unique solution  for all , \eqref{eq:1} holds
for all  and  for all .
\end{proof}

\section{Sets of Solutions} \label{sect:solsets}

Now we analyze how the polynomials

behave when  is not fixed. Let

be the additive monoid of linear homogeneous polynomials with
nonnegative integer coefficients on the variables .
The \emph{monoid ring} of  over  is the ring formed by
expressions of the form

where  and , and the addition and
multiplication of these generalized polynomials is defined in a
natural way. This ring is denoted by . If ,
then the value of a polynomial  at the point  is denoted by , and the polynomial we get by
making this substitution in  is denoted by .

The ring  is isomorphic to the ring  of polynomials on  variables. The isomorphism is given by
. However, the generalized polynomials, where
the exponents are in , are suitable for our purposes.

If  for , then we use the notation

If  and , then  for all
.

For an equation 
we define

Now . Theorem \ref{thm:weqpeq} can be
formulated in terms of the generalized polynomials .

\begin{theorem}
A morphism  of length type  is a solution
of an equation  if and only if

\end{theorem}

\begin{example}
Let . Now

\end{example}

The \emph{length} of an equation  is .


\begin{theorem} \label{thm:cover}
Let  be a pair of nontrivial equations on  unknowns
that don't have the same sets of solutions of rank . The length
types of solutions of the pair of rank  are covered by a union
of  -dimensional subspaces of . If  is a minimal such cover and  for some ,
then  and  have the same solutions of length type  and
rank .
\end{theorem}
\begin{proof}
Let

for  and . If all  minors of
the  matrix  are zero, then for all length
types  of solutions of rank  the rank of the matrix
 in Theorem \ref{thm:rank} is 1 and  and  are
equivalent, which is a contradiction. Thus there are  such that
. The generalized
polynomial  can be written as

where  and  for all . If  is
a length type of a solution of rank , then  and  must
be a solution of the system of equations

for some permutation . For every  the equations
determine an at most -dimensional space.

Let

where , , and so on. The
polynomials  form a subset of the polynomials ,
,  and  (the reason that they
form just a subset is that we assumed  for all ).
For any , let  be the smallest index  such that  for some . Now for every  such that  we have . We can do a similar
thing for the polynomials  and  and . In this way we obtain at most  polynomials  such
that for any  the value of one of these polynomials is minimal
among the values . Similarly we obtain at most 
``minimal'' polynomials . It is sufficient to consider only
those systems \eqref{eq:ssyst}, where one of the equations is formed
by these ``minimal'' polynomials . There are at most
 possible pairs of such polynomials, and each of them
determines an -dimensional space.

Consider the second claim. Because the cover is minimal, there is a
solution of rank  whose length type is in , but not in any
other . By Lemma \ref{lem:rdim}, the length types of solutions
of rank  in this space cannot be covered by a finite union of
-dimensional spaces. Thus one of the systems \eqref{eq:ssyst}
must determine the space . The same holds for systems coming
from all other nonzero  minors of the matrix ,
so  and  have the same solutions of rank  and length
type  for all  by Theorem \ref{thm:rank}.
\end{proof}

The following example illustrates the proof of Theorem
\ref{thm:cover}. It gives a pair of equations on three unknowns,
where the required number of subspaces is two. We do not know any
example, where more spaces would be necessary.

\begin{example}
Consider the equations

and

and the generalized polynomial

If  is a length type of a nontrivial solution of the pair , then . If , then  must satisfy an
equation , where

and

The possible relations are

If  satisfies one of the first three, then . If 
satisfies the last one, then , except if . So if
 is a nonperiodic solution, then

There are no nonperiodic solutions with , but every
 with  or  is a solution.
\end{example}

An equation  is \emph{balanced}, if  for every
unknown . In \cite{HaNo03} it was proved that if an independent
pair of equations on three unknowns has a nonperiodic solution, then
the equations must be balanced. With the help of Theorem
\ref{thm:cover} we get a significantly simpler proof and a
generalization for this result.

\begin{theorem} \label{thm:balance}
Let  be a pair of equations on  unknowns having a
solution of rank . If  is not balanced, then every
solution of  of rank  is a solution of .
\end{theorem}
\begin{proof}
The length types of solutions of  are covered by a single
-dimensional space . Because the pair  has a
solution of rank ,  is a minimal cover for the length types
of the solutions of the pair of rank . By Theorem
\ref{thm:cover},  and  have the same solutions of length
type  and rank  for all .
\end{proof}

Another way to think of this result is that if  is not balanced
but has a solution of rank  that is not a solution of ,
then the pair  causes a larger than minimal defect effect.


\section{Independent Systems} \label{sect:indsyst}

A system of word equations  is \emph{independent},
if for every  there is a morphism that is not a solution of
, but is a solution of all the other equations.

A sequence of equations  is a \emph{chain}, if for
every  there is a morphism that is not a solution of , but
is a solution of all the preceding equations.

The question of the maximal size of an independent system is open.
Only things that are known are that independent systems cannot be
infinite and there are systems of size , where  is
the number of unknowns. For a survey on these topics, see
\cite{KaSa11}.

We study the following variation of the above question: how long can
a sequence of equations  be, if for every  there
is a morphism of rank  that is not a solution of , but is
a solution of all the preceding equation? We prove an upper bound
depending quadratically on the length of the first equation. For
three unknowns we get a similar bound for the size of independent
systems and chains.

\begin{theorem} \label{thm:chain}
Let  be nontrivial equations on  unknowns having
a common solution of rank . For every , assume that there is a solution of the system  of rank  that is not a solution of . If the
length types of solutions of the pair  of rank  are
covered by a union of  -dimensional subspaces, then . In general, .
\end{theorem}
\begin{proof}
We can assume that  is equivalent with the system  for all . Let the length types of
solutions of  of rank  be covered by the
-dimensional spaces . Some subset of these
spaces forms a minimal cover for the length types of solutions of
 of rank . If this minimal cover would be the whole set,
then  and  would have the same solutions of rank  by
the second part of Theorem \ref{thm:cover}. Thus the length types of
solutions of  of rank  are covered by some  of these
spaces. We conclude inductively that the length types of solutions
of  of rank  are covered by some  of these spaces
for all . It must be , so . By the first part of Theorem \ref{thm:cover}, .
\end{proof}

In Theorem \ref{thm:chain} it is not enough to assume that the
equations are independent and have a common solution of rank .
If the number of unknowns is not fixed, then there are arbitrarily
large such systems, where the length of every equation is 10, see
e.g. \cite{HaKaPl02}.

In the case of three unknowns, Theorem \ref{thm:chain} gives an
upper bound depending on the length of the shortest equation for the
size of an independent system of equations, or an upper bound
depending on the length of the first equation for the size of a
chain of equations. A better bound in Theorem \ref{thm:cover} would
immediately give a better bound in the following corollary.

\begin{corollary}
If  is an independent system on three unknowns
having a nonperiodic solution, then . If  is a chain of equations on three unknowns, then .
\end{corollary}

\bibliographystyle{eptcs}
\bibliography{ref_poly}

\end{document}
