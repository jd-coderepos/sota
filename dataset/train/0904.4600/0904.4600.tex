\documentclass[11pt,a4paper]{article}
\usepackage{times}


\usepackage{a4wide}

\usepackage{latexsym}
\usepackage{amssymb}
\usepackage{epsfig}
\usepackage{amsfonts}
\usepackage{color}
\usepackage{amsmath}
\usepackage{fleqn}
\usepackage[T1]{fontenc}
\usepackage{ifthen}
\usepackage{amsthm}
\usepackage{tkz-berge}



\newboolean{commentson} \setboolean{commentson}{true}


\newcommand{\comment}[1]
{\ifthenelse{\boolean{commentson}}
   {{\par\noindent\mbox{}{\small[ *** #1 ]\par}\noindent\par}}{}}


\begin{document}



\newcommand{\bull}{\rule{.85ex}{1ex} \par \bigskip}
\newenvironment{sketch}{\noindent {\bf Proof (sketch):\ }}{\hfill \bull}


\newtheorem{theorem}{Theorem}[section]
\newtheorem{definition}[theorem]{Definition}
\newtheorem{proposition}[theorem]{Proposition}
\newtheorem{lemma}[theorem]{Lemma}
\newtheorem{corollary}[theorem]{Corollary}
\newtheorem{conjecture}[theorem]{Conjecture}
\newtheorem{exmp}{Example}
\newtheorem{notation}[theorem]{Notation}
\newtheorem{problem}{Problem}
\newtheorem{remark}[theorem]{Remark}
\newtheorem{observation}[theorem]{Observation}


\newcommand{\QCSP}[1]{\mbox{\rm QCSP}}
\newcommand{\CSP}[1]{\mbox{\rm CSP}}
\newcommand{\MCSP}[1]{\mbox{{\sc Max CSP}}}
\newcommand{\wMCSP}[1]{\mbox{\rm weighted Max CSP}}
\newcommand{\cMCSP}[1]{\mbox{\rm cw-Max CSP}}
\newcommand{\tMCSP}[1]{\mbox{\rm tw-Max CSP}}
\renewcommand{\P}{\mbox{\bf P}}
\newcommand{\G}[1]{\mbox{\rm I}}
\newcommand{\NE}[1]{\mbox{}}

\newcommand{\MCol}[1]{\mbox{\sc Max -Col}}

\newcommand{\NP}{\mbox{\bf NP}}
\newcommand{\NL}{\mbox{\bf NL}}
\newcommand{\PO}{\mbox{\bf PO}}
\newcommand{\NPO}{\mbox{\bf NPO}}
\newcommand{\APX}{\mbox{\bf APX}}
\newcommand{\Aut}{\mbox{\rm Aut}^*}
\newcommand{\bound}{\mbox{\rm -}}

\newcommand{\GIF}[3]{\ensuremath{h_\{{#2},{#3}\}^{#1}}}

\newcommand{\Spmod}{\mbox{\rm Spmod}}
\newcommand{\Sbmod}{\mbox{\rm Sbmod}}

\newcommand{\Inv}[1]{\mbox{\rm Inv()}}
\newcommand{\Pol}[1]{\mbox{\rm Pol()}}
\newcommand{\sPol}[1]{\mbox{\rm s-Pol()}}

\newcommand{\un}{\underline}
\newcommand{\ov}{\overline}
\def\ar{\hbox{ar}}
\def\vect#1#2{#1 _1\zdots #1 _{#2}}
\def\zd{,\ldots,}
\let\sse=\subseteq
\let\la=\langle
\def\lla{\langle\langle}
\let\ra=\rangle
\def\rra{\rangle\rangle}
\let\vr=\varrho
\def\vct#1#2{#1 _1\zd #1 _{#2}}
\newcommand{\va}{{\bf a}}
\newcommand{\vb}{{\bf b}}
\newcommand{\vc}{{\bf c}}
\newcommand{\bx}{{\bf x}}
\newcommand{\by}{{\bf y}}
\def\Z{{\bur Z^+}}
\def\R{{\bur R}}
\def\D{{\cal D}}
\def\F{{\cal F}}
\def\I{{\cal I}}
\def\C{{\cal C}}
\def\U{{\cal U}}
\def\K{{\cal K}}
\def\Lat{{\cal L}}

\def\2mat#1#2#3#4#5#6#7#8{
\begin{array}{c|cc}
 & #3 & #4\\
\hline
#1 & #5& #6\\
#2 & #7 & #8 \end{array}}




\renewcommand{\phi}{\varphi}
\renewcommand{\epsilon}{\varepsilon}

\def\tup#1{\mathchoice{\mbox{\boldmath}}
{\mbox{\boldmath}}
{\mbox{\boldmath}}
{\mbox{\boldmath}}}
\newcommand{\draft}{\begin{center}\huge Draft!!! \end{center}}
\newcommand{\void}{\makebox[0mm]{}}     



\renewcommand{\text}[1]{\mbox{\rm \,#1\,}}        





\renewcommand{\emptyset}{\varnothing}  \newcommand{\union}{\cup}               \newcommand{\intersect}{\cap}           \newcommand{\setdiff}{-}                \newcommand{\compl}[1]{\overline{#1}}   \newcommand{\card}[1]{{|#1|}}           \newcommand{\set}[1]{\{{#1}\}} \newcommand{\st}{\ |\ }                 \newcommand{\suchthat}{\st}             \newcommand{\cprod}{\times}             \newcommand{\powerset}[1]{{\bf 2}^{#1}} 

\newcommand{\tuple}[1]{\langle{#1}\rangle}  \newcommand{\seq}[1]{\langle #1 \rangle}
\newcommand{\emptyseq}{\seq{}}
\newcommand{\floor}[1]{\left\lfloor{#1}\right\rfloor}
\newcommand{\ceiling}[1]{\left\lceil{#1}\right\rceil}

\newcommand{\map}{\rightarrow}
\newcommand{\fncomp}{\!\circ\!}         


\newcommand{\transclos}[1]{#1^+}
\newcommand{\reduction}[1]{#1^-}        










\newcommand{\perfimp}{\stackrel{p}{\Longrightarrow}}

\newcommand{\ie}{{\em ie.}}                \newcommand{\eg}{{\em eg.}}
\newcommand{\paper}{paper}                

\newcommand{\emdef}{\em}                   \newcommand{\rinterpretation}{-interpretation}
\newcommand{\rmodel}{-model}
\newcommand{\transp}{^{\rm T}}

\newcommand{\unprint}[1]{}
\newcommand{\blankline}{}

\newcommand{\prob}[1]{{\sc #1}}

\newcommand{\Solv}{{\it TSolve}}
\newcommand{\Neg}{{\it Neg}}
\newcommand{\logname}{XX}

\newcommand{\props}{{\it props}}
\newcommand{\rels}{{\it rels}}
\newcommand{\deduce}{\vdash_p}

\newcommand{\pform}{{\rm Pr}}
\newcommand{\axform}{{\rm AX}}
\newcommand{\axset}{{\bf AX}}
\newcommand{\resdeduce}{\vdash_{\rm R}}
\newcommand{\resaxdeduce}{\vdash_{\rm R,A}}

\newcommand{\cmis}{{\em \#mis}}
\newcommand{\combine}{{\em comb}}

\newcommand{\xcsp}{{\sc X-Csp}}
\newcommand{\csp}{{\sc Csp}}

\newcommand{\cc}[1]{\textnormal{\textbf{#1}}} \newcommand{\opt}[0]{\textrm{{\sc opt}}}


\newcommand{\MC}{mc}
\newcommand{\Ha}{\textrm{\textit{H\aa}}}
\renewcommand{\atop}[2]{\genfrac{}{}{0pt}{}{#1}{#2}}
\newcommand{\GHAT}{{\cal G_\equiv}}
\newcommand{\HOMEQ}{\equiv}

\pagestyle{plain}

\author{Robert Engstr\"om\footnote{\tt{engro910@student.liu.se}}, Tommy F\"arnqvist\footnote{\tt{\{tomfa, petej, johth\}@ida.liu.se }}, Peter Jonsson\footnotemark[2],  and Johan Thapper\footnotemark[2]\\
\\
\small
Department of Computer and Information Science\\
\small
Link\"{o}pings universitet\\
\small
SE-581 83 Link\"{o}ping, Sweden\\}


\title{Graph Homomorphisms, Circular Colouring, and \\ Fractional Covering by -cuts}

\date{}
\maketitle
\bibliographystyle{abbrv}
   

\begin{abstract}
A graph homomorphism is a vertex map which carries edges from a source 
graph to edges in a
target graph. The instances of the \emph{Weighted Maximum 
-Colourable Subgraph} problem () are
edge-weighted graphs  and
the objective is to find a subgraph of  that has maximal total edge
weight, under 
the condition that the subgraph has a homomorphism to ; note that for
 this 
problem is equivalent to {\sc Max -cut}. F\"arnqvist et al.\ have
introduced a parameter on the space of graphs that allows close study of
the approximability properties of . Specifically, it can be
used to extend previously known (in)approximability results to larger
classes of graphs. Here, we investigate the properties of this parameter
on circular complete graphs , where . The
results are extended to -minor-free graphs and graphs with bounded
maximum average degree. We also consider connections with
\v{S}\'{a}mal's work on fractional covering by cuts: we address,
and decide, two conjectures concerning cubical chromatic numbers.

\noindent
{\bf Keywords}: graph -colouring, circular colouring, fractional
colouring, combinatorial optimisation
\end{abstract}

\section{Introduction}
Denote by  the set of all simple, undirected and
finite graphs. 
A \emph{graph homomorphism} from  to  is a vertex map which
carries the edges in  to edges in .
The existence of such a map will be denoted by .
For a graph , let
 be the set of \emph{weight functions}
 assigning weights
to edges of .
Now,
  {\em Weighted Maximum -Colourable Subgraph} (\MCol{H}) is the
maximisation problem with
  \begin{description}
  \item[Instance:] An edge-weighted graph , where  and
    .
  \item[Solution:] A subgraph  of  such that .
  \item[Measure:] The weight of  with respect to .
  \end{description}

\noindent
Given an edge-weighted graph , denote by  the measure
of the optimal solution to the problem \MCol{H}.
Denote by  the
(weighted) size of a largest -cut in .
This notation is justified by the fact that
.
In this sense, \MCol{H} generalises {\sc Max -cut} which is a
well-known and well-studied problem that is computationally hard
when .
Since \MCol{H} is a hard problem to solve exactly, efforts have been
made to find suitable approximation algorithms.
F\"arnqvist et al.~\cite{farnqvist:etal:09} introduce a
method that can be used to extend previously known
(in)approximability bounds on \MCol{H} to new and larger classes of
graphs. For example, they 
present concrete approximation ratios for certain graphs (such as the odd cycles) 
and
near-optimal asymptotic results for large graph classes.
The fundament of this promising technique is the ability to compute 
(or closely approximate) a function
 defined as follows: 


It is not surprising that estimating  
is, in many cases,
non-trivial. One way is to solve a certain linear program that
we present in Section~\ref{sec:linprog}: the program
can be tedious to write down since it is based on the structure
of 's automorphism group, and can be prohibitively large.
Another way is to use the following lemma:

\begin{lemma}[\cite{farnqvist:etal:09}]
\label{lem:sandwich}
Let . Then,  and
.
\end{lemma}

It is apparent that in order to use this result effectively, we need a large
selection of graphs  that are known to be close to each other with respect
to . For the moment, the set of such examples is quite meagre.
Hence, we set out to investigate how the function  behaves on
certain classes of graphs. In Section~\ref{sec:meas}, we will take a careful
look at 3-colourable circular complete graphs and, amongst other things,
find that  is constant between a large number of these graphs.
Moreover, we will extend bounds on  to other classes of graphs using
known results about homomorphisms to circular complete graphs; 
examples include -minor-free graphs and graphs with bounded
maximum average degree. 

Yet another way of estimating the function  is to relate it
to other graph parameters. In this vein, Section~\ref{sec:cut} is dedicated
to generalising the work of \v{S}\'{a}mal~\cite{samal:05,samal:06}
on fractional covering by cuts to obtain a new family of `chromatic
numbers'. This reveals that  and the new chromatic numbers
 are closely related quantities, which provides us with an
alternative way of computing .
We also use our knowledge about the behaviour of  to
disprove a conjecture by \v{S}\'{a}mal concerning the cubical chromatic
number and, finally, we decide in the positive another conjecture by
\v{S}\'{a}mal concerning the same parameter.
We conclude the paper, in Section~\ref{sec:open}, by discussing
open problems and directions for future research.
To improve readability some proofs are deferred to the appendices.





\section{A Linear Program for }
\label{sec:linprog}

F\"arnqvist et al.~\cite{farnqvist:etal:09} have identified 
an alternative expression for 
which depends on the automorphism group of .
Let  and  be graphs and let  be the
(edge) automorphism group of , i.e.,  acts on 
by permuting the edges.
Let  be the set of
weight functions  which satisfy 

and for which  for all  and
. That is, the weight functions in 
are constant over the edges belonging to each orbit of .
\begin{lemma}[\cite{farnqvist:etal:09}]
\label{lem:auto}
   Let .
    Then,
    .
    In particular, when  is edge-transitive,
    .
\end{lemma}
Lemma~\ref{lem:auto} shows that in order to determine , it is
sufficient to minimise  over , and it follows
that  can be computed by solving a linear program.
For , let  be the orbits
of  and, for , define

That is,  is the number of edges in  which are mapped to an
edge in  by . The measure of a solution  when  is equal to 
where  is the weight of an edge in . Given an ,
the measure of a solution  depends only on the vector
. We call this vector the {\em
signature} of . When there is no risk of confusion,
we will let  denote the signature as well.
Since we have seen that the measure of a solution only
depends on its signature the solution space is taken to be the set of
possible signatures  

The variables of the linear program are  and
, where  represents the weight of each element in the orbit
 and  is an upper bound on the signatures measure.

Given a solution  to this program, 
when  is a weight function which minimises .
The value of this solution is . 






\section{Solutions to (\ref{lp}) for Circular Complete Graphs} \label{sec:meas}

A circular complete graph  is a graph with vertex set
 and edge set . This can be seen as placing the vertices on a circle and
connecting two vertices by an edge if they are at a distance at least
 from each other. A fundamental property of these graphs is that
 iff .
Due to this fact, when we write , we will assume that  and 
are relatively prime.
We will denote the orbits of the action of  by
p,
for .
We finally note that a
homomorphism from a graph  to  is called a (circular)
-colouring of . 
More information on this topic can be gained
from the book by Hell and Ne\v{s}et\v{r}il~\cite{HN04} and from the
survey by Zhu~\cite{zhu:survey}.

In this section we start out by investigating 
for rational numbers .
In Section~\ref{ssec:k2}, we fix  and choose  so that
 has few orbits.
We find some interesting properties of these numbers which lead
us look at the case  in Section~\ref{ssec:odd}.
Our approach is based on relaxing the linear program (\ref{lp})
that was presented in Section~\ref{sec:linprog}, combined with
arguments that our chosen relaxations in fact find the optimum
in the original program.


\subsection{Maps to } \label{ssec:k2}

We consider  for
 with , where  and  are integers.
The number of orbits of  then equals .
We choose to begin our study of  using small values of .
When ,  is isomorphic to the cycle .
The value of , for  
was obtained in~\cite{farnqvist:etal:09}.
Combined with the following result, where we set , 
this has an immediate and perhaps surprising consequence.
\begin{proposition}
\label{prop:4k+4}
Let  be an integer, then .
\end{proposition}
\begin{proof}
Let  and .
We will present two maps .
 sends a vertex  to  if  and to  if
.
It is not hard to see that .
The map  sends  to  if  is even and to  if  is odd.
Then,  maps all of  to  but none of the edges in ,
so .
It remains to argue that these two solutions suffice to determine .
But we see that any map  with  must cut at least two edges in
the even cycle , leading to , thus ,
componentwise.
The proposition now follows by solving the relaxation of (\ref{lp})
using only the two inequalities obtained from  and .
\end{proof}

\begin{corollary}
\label{cor:intervals}
Let  and .
Then, .
\end{corollary}
\begin{proof}
  Note that we have the chain of homomorphisms
  .
  By Lemma~\ref{lem:sandwich}, we get . But since , and  is edge-transitive with  edges,  and therefore . 
  Again by Lemma~\ref{lem:sandwich}, we have
  
\end{proof}

We find that there are intervals  where  is constant.
In Figure~\ref{fig:interval} these intervals are shown for the first few values of . The intervals  form an infinite sequence with endpoints tending to . 
Similar intervals appear throughout the space of circular complete graphs.
More specifically,  F\"{a}rnqvist et al.~\cite{farnqvist:etal:09} have shown
that  for arbitrary integers .
Furthermore, it can be proved that  for .
Two applications of Lemma~\ref{lem:sandwich} now shows that  is 
constant on the regions , where
  .

\begin{figure}[h]
\centering
\begin{tikzpicture}
\draw[dashed] (0,0) -- (1.875,0);
\draw[very thick] (1.875,0) -- (2.1428,0);
\draw[thin] (2.1428,0) -- (2.5,0);
\draw[very thick] (2.5,0) -- (3,0);
\draw[thin] (3,0) -- (3.75,0);
\draw[very thick] (3.75,0) -- (5,0);
\draw[thin] (5,0) -- (7.5,0);
\draw[thick](0,-0.2) -- (0,0.2);
\draw[thick](1.875,-0.2) -- (1.875,0.2);
\draw[thick](2.1428,-0.2) -- (2.1428,0.2);
\draw[thick](2.5,-0.2) -- (2.5,0.2);
\draw[thick](3,-0.2) -- (3,0.2);
\draw[thick](3.75,-0.2) -- (3.75,0.2);
\draw[thick](5,-0.2) -- (5,0.2);
\draw[thick](7.5,-0.2) -- (7.5,0.2);
\node at (0,0.5) {};
\node at (1.875,0.5) {};
\node at (2.1428,0.5) {};
\node at (2.5,0.5) {};
\node at (3,0.5) {};
\node at (3.75,0.5) {};
\node at (5,0.5) {};
\node at (7.5,0.5) {};
\node at (0,-0.5) {};
\node at (2,-0.5) {};
\node at (2.75,-0.5) {};
\node at (4.375,-0.5) {};
\node at (-0.45,0.5) {};
\end{tikzpicture}
\caption{The space between  and  with the intervals  marked for .}\label{fig:interval}
\end{figure}

As we proceed with determining  we can now, 
thanks to Corollary~\ref{cor:intervals}, disregard those  which fall
inside these constant intervals.
For , we see that if
 (mod 3), then  is an odd cycle,
and if  (mod 3), then .
Therefore, we assume that  is of the form 
 for an integer .


\begin{proposition} \label{prop:3k+1}
Let  be an integer. Then, .
\end{proposition}








For , we find that we only need to consider the case when .
We then have graphs  with  for integers .




\begin{proposition} \label{prop:4k+1}
Let  be an integer. Then, .
\end{proposition}




The expressions for  in Proposition~\ref{prop:3k+1}~and~\ref{prop:4k+1}
have some interesting similarities, but for  it becomes harder to
pick out a suitable set of solutions which guarantee that the relaxation has
the same optimum as (\ref{lp}) itself.
Using computer calculations, we have however determined the first
two values () for the case 
and the first value () for the case .


\subsection{Maps to Odd Cycles} \label{ssec:odd}

It was seen in Corollary~\ref{cor:intervals} that  is constant
on the region .
In this section, we will study what happens when  remains in , but
 is set to .
A first observation is that the absolute jump of the function
 when  goes from being less than 
to   must be largest for .
Let  and
.
The map  with the indices of  taken modulo  has the
signature .
Since the subgraph induced by the orbit  is isomorphic to ,
any map to an odd cycle must exclude at least one edge from .
It follows that  alone determines , 
and we can solve (\ref{lp}) to obtain .
Thus, for , we have

Smaller  can be expressed as , 
where .
We will write  for positive integers  and  which 
implies the form , with .
For , it turns out to be sufficient to keep two inequalities 
from (\ref{lp}) to get an optimal value of . 
From this we get the following result:





\begin{proposition} \label{prop:m1}
Let  be integers. Then, .
\end{proposition}
There is still a non-zero jump of  when we move from 
 to , but it is obviously smaller than that
of (\ref{eq:jump}) and tends to 0 as  increases.
For , we have  and  relatively prime only when 
is odd.
In this case, it turns out that we need to include an increasing number of
inequalities to obtain a good relaxation.
Furthermore, we are not able to ensure that the obtained value is the
optimum of the original (\ref{lp}).
We will therefore have to settle for a lower bound for .
Explicit calculations have shown that, for small values of  and ,
equality holds in Proposition~\ref{th:qisfunny}.
We conjecture this to be true in general.

\begin{proposition} \label{th:qisfunny}
Let  be an integer and  be an odd integer. Then,

where

and  are the reciprocals of the roots of
.
\end{proposition}






\subsection{Extending the Results} \label{sec:apply}

We will now take a look at one possible way of extending the results in
the previous sections. To do this, we need to find graphs or classes of
graphs we can homomorphically sandwich between graphs with known
 value. Clearly,  has a homomorphism to all non-empty
graphs, and that if a graph  has circular chromatic number  it has a homomorphism to . These facts, together with
Lemma~\ref{lem:sandwich}, combine into the following easily proved
lemma:

\begin{lemma}
\label{lem:circlesandwich}
Let  be a non-empty graph with . Then, .
If, additionally,  has odd girth no greater than , 
then .
\end{lemma}



We can now make use of known results about bounds on the circular
chromatic number for certain classes of graphs. Much of the extensive
study conducted in this direction was instigated by the restriction of a
conjecture by Jaeger~\cite{jaeger:88} to planar graphs, which is
equivalent to the claim that every planar graph of girth at least 
has a circular chromatic number at most , for . The
case  is Gr\"{o}tzsch's theorem; that every triangle-free planar
graph is 3-colourable. Currently, the best proven girth for when the
circular chromatic number of a planar graph is guaranteed to be at most
 is  and due to Borodin et
al.~\cite{Borodin:etal:jctb2004}. This result was used by F\"{a}rnqvist
et al.\ to achieve the bound  for planar
graphs  of girth at least . Here, we significantly improve
this bound by considering -minor-free graphs, for which Pan and
Zhu~\cite{pan:zhu:02} have shown how their circular chromatic number is
upper-bounded by their odd girth.
\begin{proposition} \label{thmI}
Let  be a -minor-free graph, and  an integer. If 
has an odd girth of at least , then . If  has an odd girth of at least , then
.
\end{proposition}


Of course, it is a big limitation to only consider -minor-free
graphs. Almost all work on the circular chromatic number for planar
graphs have focused on finding limits when ,
that is, when there exists a homomorphism to the odd cycle .
However, Corollary~\ref{cor:intervals} implies that for two graphs
 and , if  and  then
, so for our purposes it would be interesting to have
more results when . 
For general graphs, we can use results from Raspaud and
Roussel~\cite{raspaud:rousell:07} relating the circular chromatic number
of graphs to their maximum average degree. Specifically, they show that
for a general graph  of girth at least 12, 11, or 10, its circular
chromatic number is bounded from above by , , and ,
respectively, which translates into corresponding upper bounds ,
, and  on  (using Propositions \ref{prop:4k+4},
\ref{prop:3k+1}, \ref{prop:4k+1} and Lemma~\ref{lem:circlesandwich}).

\section{Fractional Covering by -cuts} \label{sec:cut}

In the following, we slightly generalise the work of 
\v{S}\'{a}mal~\cite{samal:05,samal:06} on fractional
covering by cuts to obtain a complete correspondence between  
and a family of `chromatic numbers'  which generalise 
\v{S}\'{a}mal's cubical chromatic number . 
The latter corresponds to the case when .
First, we recall the notion of a {\em fractional colouring} of a
(hyper-) graph.
Let  be a (hyper-) graph with vertex set  and edge set 
.
A subset  of  is called independent in  if no edge 
 is a subset of .
Let  denote the set of all independent sets of 
and for a vertex , let  
denote all independent sets which contain .
Then, the fractional chromatic number  of  is given by the
linear program:



The definition of fractional covering by cuts mimics fractional colouring,
but replaces vertices with edges and independent sets with certain cut
sets of the edges.
Let  and  be undirected simple graphs and  be an arbitrary
vertex map from  to .
The map  induces a partial map 
from  to  and we will call the preimage of this map an {\em -cut} in .
When  is a complete graph , this is precisely the notion of a
{\em -cut}.
Let  denote the set of -cuts in  and for an edge 
, let  denote all -cuts which
contain .
The following definition is the generalisation of
{\em cut -covers}~\cite{samal:06} to arbitrary -cuts:
\begin{definition}
  An {\em -cut -cover} of  is a collection  of 
  -cuts in  such that every edge of  is in at least  of them.
  The graph parameter  is defined as:
  
\end{definition}

By reasoning analogous to that of \v{S}\'{a}mal~\cite{samal:06} Lemma~5.1.3,  is also
given by the following linear program:


For , an alternative definition of  was
obtained in~\cite{samal:06} by taking the infimum 
(actually minimum due to the formulation in (\ref{coverprimal}))
over  for  and  such that .
Here,  is the graph on vertex set  with
an edge  if , where  denotes the
Hamming distance.
We generalise this family as well to produce a scale for each .
Namely, let  be the graph on vertex set  and an edge
between  and  when
.
A moments thought shows that we can express  as:

\v{S}\'{a}mal also notes that  is given by the fractional chromatic number
of a certain hypergraph associated to .
For the general case, let  be the hypergraph obtained from  by taking
 and letting  be the set of minimal subgraphs 
 such that .
A short argument shows that indeed .

Finally, we can work out the correspondence to .
Consider the dual program of (\ref{coverprimal}):

Let  and make the substitution  in 
(\ref{coverdual}).
Comparing with (\ref{lp}), we have




We now move on to address two conjectures by \v{S}\'{a}mal~\cite{samal:06} on the cubical
chromatic number .
In Section~\ref{sec:neg} we discuss an upper bound on  which relates to
the first conjecture, Conjecture~5.5.3~\cite{samal:06}.
This is the suspicion
that  can be determined by measuring the maximum cut over
all subgraphs of .
We show that this is false by providing a counterexample from
Section~\ref{ssec:k2}.
We then consider Conjecture~5.4.2~\cite{samal:06}, 
concerning ``measuring the scale'', i.e.,
determining  for the graphs  themselves.
We prove that this conjecture is true, 
and state it as Proposition~\ref{prop:samal} in Section~\ref{sec:pos}.


\subsection{An Upper Bound on } \label{sec:neg}

In Section~\ref{sec:meas} we obtained lower bounds on  by relaxing the
linear program (\ref{lp}).
In most cases, the corresponding solution was proven feasible for the
original (\ref{lp}), and hence optimal.
Now, we take a look at the only known source of upper bounds for .

Let , with  and take an arbitrary
 such that .
Then, applying Lemma~\ref{lem:sandwich} followed by Lemma~\ref{lem:auto}
gives

When  it follows that

where  denotes the bipartite density of .
\v{S}\'{a}mal~\cite{samal:06} conjectured that
this inequality, expressed on the form
,
can be replaced by an equality.
We answer this in the negative, using  as our counterexample.
Lemma~\ref{prop:3k+1} with  gives .
If  for some  it means
that  must have at least  edges. 
Since  has exactly  edges, then .
However, a cut in a cycle must contain an even number of edges.
Since the edges of  can be partitioned into two cycles,
we have that the maximum cut in  must be of even size,
hence .
This is a contradiction.

\subsection{Confirmation of a Scale} \label{sec:pos}

As a part of his investigation of , \v{S}\'{a}mal~\cite{samal:06}
set out to determine the value of .
We complete the proof of his Conjecture~5.4.2~\cite{samal:06}
to obtain the following result.
\begin{proposition} \label{prop:samal}
  Let  be integers such that .
  Then, 
 if  is even and  if  is odd. \end{proposition}
\v{S}\'{a}mal provides the upper bound and an approach to the lower bound 
using the largest eigenvalue of the Laplacian of a subgraph of .
The computation of this eigenvalue
boils down to an inequality (Conjecture~5.4.6~\cite{samal:06})
involving some binomial coefficients.
We first introduce the necessary notation and then prove the remaining
inequality in Lemma~\ref{lem:ineq}, whose second part, for odd ,
corresponds to one of the formulations of the conjecture.
Proposition~\ref{prop:samal} then follows from
Theorem~5.4.7~\cite{samal:06} conditioned on the result of this lemma.

  Let  be positive integers such that , and let
   be an integer such that .
  For , let  denote the set of all -subsets of  that have an odd number of elements in common with the set .
  Define  analogously as the -subsets with an even number of common elements.
  Let  and . Then, 


  When  is odd, the function 
  given by the complement 
  is a bijection.
  Since , we have
  
  

  \begin{lemma} \label{lem:help}
    Let  with  odd. Then,
     and .
  \end{lemma}

  \begin{proof}
    First, partition  into  and .
    Similarly, partition  into  and .
    Note that .
    We argue that .
    To prove this, define the function  by

i.e.,  acts on  by ignoring the element  and renumbering
    subsequent elements so that the image is a subset of .
    Note that  and .
    Since  is odd, it follows from (\ref{eqn:1}) that
    .
    The first part of the lemma now follows from the injectivity of
    the restrictions  and .
    The second equality is proved similarly.
\end{proof}

  \begin{lemma} \label{lem:ineq}
    Choose  and  so that  and .
    For odd ,
    
\end{lemma}
  
  \begin{proof}
    We will proceed by induction over  and .
    The base cases are given by , , and .
    For ,
 and ,
where the inequality holds for all .
    For  and odd , we have , and for
    even , we have .
    For ,
 if  is even and 0 otherwise.
Let  and consider  for odd  and .
    Partition the sets  into those for which 
     on the one hand and those for which
     on the other hand.
    These parts contain  and  sets,
    respectively.
    Since  is even, and since  when , 
    it follows from the induction hypothesis that

The case for  and even  is treated identically.

    Finally, let .
    If  is odd, then Lemma~\ref{lem:help} is applicable, 
    so we can assume that  is even.
Now, as before

where the first term is evaluated using (\ref{eqn:1}).
    The same inequality can be shown for 
    and even ,
    which completes the proof.
\end{proof}

\section{Conclusions and Open Problems}
\label{sec:open}

We have seen that for all integers ,  is constant on .
It follows that our sandwich approach using Lemma~\ref{lem:sandwich} with 
and  can not distinguish between the class
of graphs with circular chromatic number  and the (larger) class with
circular chromatic number .
As previously noted, Jaeger's conjecture and subsequent research
has provided partial information on the members of the former class.
We remark that Jaeger's conjecture implies a weaker statement in our
setting. Namely, if  is a planar graph with girth greater than ,
then  implies . Deciding this to be true would certainly provide support for
the original conjecture, and would be an interesting result in its
own right.
Our starting observation shows that the slightly weaker condition
 implies the same result.

When it comes to completely understanding how  behaves on circular complete graphs, even
restricted to those between  and , there is still work to be done.
For edge-transitive graphs , in our case the cycles and the complete graphs, 
it is not surprising
that the expression  assumes a finite number of values seen as a function of .
Indeed, Lemma~\ref{lem:auto} says that  which
leaves at most  values for .
This produces a number of constant intervals which are partly 
responsible for the constant regions of Corollary~\ref{cor:intervals} and the discussion
following it.
More surprising are the constant intervals that arise from
.
They give some hope that the behaviour of  is possible to characterise more generally.
One direction could be to identify additional constant regions,
perhaps showing that they completely tile the entire space?




In Section~\ref{sec:cut} we generalised the notion of covering by cuts
due to \v{S}\'{a}mal.
By doing this, we have found a different interpretation of the -numbers
as an entire family of `chromatic numbers'.
It is our belief that these alternate viewpoints can benefit from each other.
The refuted conjecture in Section~\ref{sec:neg} is an immediate example of
this.
On the other hand, it would be interesting to determine
when the generalised upper bound in (\ref{eq:ub}) is tight.
For , the proof of Proposition~\ref{prop:samal} is precisely such a result
for the graphs ,
which is evident from studying the proof of Theorem 5.4.7~\cite{samal:06}.
Following this, a natural step would be to calculate  for
more general graphs , starting with .



It is fairly obvious that  is a special case of the 
{\em maximum constraint satisfaction} ({\sc Max CSP}) problem;
in this problem, one is given a finite collection of constraints on overlapping
sets of variables, and the goal is to assign values from a given domain to the
variables so as to maximise the number of satisfied constraints.
By letting  be a finite set of relations, we can
parameterise {\sc Max CSP} with  ({\sc Max CSP}) so that
the only allowed constraints are those constructed from the relations in .
By viewing a graph  as a binary relation, the problems {\sc Max CSP}
and  are virtually identical. 
Raghavendra~\cite{raghavendra:08} has presented
an algorithm for {\sc Max CSP} based on semi-definite programming.
Under the so-called {\em unique games conjecture}, this algorithm
optimally approximates {\sc Max CSP} in polynomial-time, i.e. no
other polynomial-time algorithm can approximate the problem substantially better.
However, it is notoriously difficult to find out exactly how well the
algorithm approximates {\sc Max CSP} for a given .
It seems plausible that
the function  can be extended into a function  from pairs of sets
of relations to , and that  can be used for studying
the approximability of {\sc Max CSP} by extending the approach in
F\"arnqvist~et al.~\cite{farnqvist:etal:09}. This would constitute a novel method for
studying the approximability of {\sc Max CSP} --- a method that, hopefully, may
cast some new light on the performance of Raghavendra's algorithm.


\bibliography{wg}

\newpage

\appendix
\begin{center}
  {\bf APPENDIX}
\end{center}




\noindent
Let   be positive integers. We often assign names to the vertices, so that . Then, we have . Note that  does not have any edges unless , since the circular distance between two vertices is as most . 
For a fixed , let  (mod ).  is then the directed circular distance (in positive direction) between  and . Furthermore let . This is then the undirected circular distance. We do index arithmetics for circular complete graphs modulo , e.g.
. Even though  is isomorphic to , we distinguish them by letting  be an edge in  if , while  is an edge in  if .

Let  and  be graphs and
let  be a set of signatures to  of .
If  is a subset for which the relaxation of (\ref{lp}) 
has the same optimal solution as the original program,
we will call  a \emph{complete} set of signatures with respect to
 of .

\section{Proofs of Results from Section~\ref{ssec:k2}}














































\subsection*{Proposition~\ref{prop:3k+1}}
\begin{proof}
Let  and . 
Let  be the solution with  if  and 
if .
From  only the edges ,  and  are mapped to a single vertex in , so . From  only the edge  is mapped to a single vertex in , so .
Thus,  has the signature .



Note that since  and  are relatively prime, the edges of , 
as well as , form cylces of length .
Therefore, any solution which maps more than  edges from  to  
must map exactly .
Let  be such a solution. We will show that .
We may assume that  is the edge in  which is not
mapped to  by .
Note that, if , then  and 
are both mapped to  by  which implies that .
Now, let  be an edge in  and let
.
Then, this edge is mapped to  by , i.e.,  
if and only if
.
Since  and  are adjacent in , they can not
both be in .
Therefore, there are  edges that are mapped
to  by , so .
We conclude that solving (\ref{lp}) with the inequalities obtained from
 and  yields the correct value of .
\end{proof}

\subsection*{Proposition~\ref{prop:4k+1}}
\begin{proof}
Let  and . 
Define  by  if  and  if .
Here, the edges  and 
 in  are mapped to a single vertex in  by 
From ,  maps edges  and  to a single 
vertex in .
Finally,  maps all edges in  to the edge in .
The signature of this solution is .

Let  be defined by

and 

From  only the edges  and  are mapped to
a single vertex in . From  we partition the edges which are mapped to
the edge in  by  into four sets, with  edges in each set. These are




Finally, for ,  maps the  edges 
 as well as the  edges 
 to the edge in .
In summary, .

The relaxation of (\ref{lp}) corresponding to the two solutions  and  has
the following solution:

We will now show that  and  is feasible in
the original program.
We will show that for all solutions , we must have .
We will also show that if  is such that , then
.
Finally, we will show that if , then  must be 0.
In the final case, we note that .

The edges of  connects vertices at a distance of .
Since we have a common factor  in  and ,
the edges of  consists of two odd cycles, each of length .
Since a cut of a cycle must include an even number of edges, 
we can then at most have a solution that maps  edges to .

For the second case, 
note that . 
This means that the shortest path between  and  in 
is of length . The edge  is mapped to  
if and only if at least one edge in each of the paths from  to  in
  is not mapped to , since they are both of even length.
If a solution  has , only two edges from  are
 not mapped to .
Therefore no more than  paths of length  can include 
at least one of these two edges, hence .

Finally, if a solution  includes an edge from  it means that 
 for some .
But since both paths from  to  in  are of even length,
not all edges from  can be mapped to . 
So if , then .

\end{proof}

\section{Proof of Proposition~\ref{prop:m1}} \label{app:propm1proof}

The proof of Proposition~\ref{prop:m1} follows from Lemma~\ref{lem:solalpha} and~\ref{lem:solbeta} introduced and proved in this section.

\subsection*{Proposition~\ref{prop:m1}}
\begin{proof}
  Let .
  From Lemma~\ref{lem:solalpha}, we get a solution , with
  
  where .
  From Lemma~\ref{lem:solbeta}, we get another solution , with
  
  where .
  The last constraint in (\ref{lp}) can be written as
  
  We now insert (\ref{eq:isol1}) into the inequalities obtained from  and
   to get the following relaxation of (\ref{lp}):
  
  The solution to this is ,
  which yields the -value in the proposition.

  To show that this is optimal for the original program,
  let us consider the restriction of (\ref{lp}) in which we force
   for .
  Due to the second part of Lemma~\ref{lem:solbeta}, it suffices to
  keep the two inequalities from  and  in the program.
  The equality constraint can now be written as
  
  By inserting (\ref{eq:isol2}) into the two remaining
  inequalities we again obtain (\ref{eq:prog}).
  Thus, the solution to the relaxation gives the right value for .
\end{proof}

\begin{lemma}
\label{lem:solalpha}
Let  be integers with  and , . Then, there exist a solution  to  of MAX -COL with signature . 
\end{lemma}

\begin{proof}
Let  and , , . The construction of  will depend on whether  or .
When  we define  as follows.

,

,

 

,

,



,

,



,

,



.

\noindent
Note, in particular, that

When , we define  as follows:

, 

,

 

,

,



,

,



,

,



.

\noindent
In this case,


Now,
consider a vertex  with . Take one edge . Then,

Let . That is,  is the vertex with lowest index which is mapped to . Furthermore let . We then have

We now want to show that . It will then follow that , i.e., all edges outside of  are mapped to an edge in .
To do this, we will show that .

First, we bound  from above by taking a walk along the vertices between  and . We need to pass at most  vertices to enter the set . We then continue until  or  depending on the parity of . Our walk continues from  or  up until we come to the last vertex in . Finally we take one last step into  and reach . We have then passed

vertices. There are  sets among . When  each set has either  or  vertices. However, at most  of them can contain  vertices. Thus,

In the case of , each set has either  or  vertices but at most  of them can contain  vertices. Thus,

When bounding  from below, we take a similar walk, but now we want to determine the fewest possible vertices we will pass. Therefore, we assume that we immediately move into the set  and will go all the way to the last vertex in . We have then passed a total of

vertices. There are  sets among .  When  at least  of the sets has  vertices. Thus,

In the case of , at least  has  vertices. Thus,

Combining the lower and upper bounds with (\ref{eq:deltail}), we find that

hence .
Since  was an arbitrary edge in , this implies that  for .

It remains to determine .
Recall that .
As before, we want to check if  to determine if .
This means that  is a non-edge in  
if and only if .
This, in turn, can only happen if the walk from  to  passes all  of the vertices from  (excluding ).
Thus,  has to be the vertex with the lowest index in .
In total there are  such vertices, one for each vertex in .
Furthermore, it must be the case that the walk fully passes the  sets

with  vertices in the case when  and the  sets

with  vertices when .
When  this happens precisely when  is odd and 
, i.e.  times.
When  it happens precisely when  is odd and 
 which is also  times.
In all cases, there will be  edges in  which are not mapped to edges in
 so  which concludes the proof.
\end{proof}

Let  and  be relatively prime and let . Define a function  by letting  if  and  (mod ). Note that  is a bijection on . We will think of  as indicating the length of a path (in the positive direction) from  to  in the cycle .
We will denote the length from  to  in  by  taken modulo . 
Note that  for all integers .
Closed and half-open intervals are defined by  and , respectively.


Let .
Given a subset , we will now describe a general construction
of a solution  to an instance  of {\sc Max -COL}.
The idea is to map the nodes  in order of increasing  starting by . We then map  to a node adjacent to , picking one of the two possibilities depending on whether  or not.
To give the formal definition, it will be convenient to introduce the rotation
 on  defined as .
We then have,

Note that the last vertex to be mapped is .
If the created solution has  or , then
, otherwise .
In the latter case, it does not matter whether  or not and we
can assume that .
However, to maintain consistency in the case of ,
we want to have  if  and
 otherwise.
Therefore,  if and only if .


\begin{exmp}
The solution  with , , , , , , ,  looks as follows.

Note that the  are mapped in the order .
 is given in the order in which the vertices appear along the .
To start, we let . Neither of  or  appear in
, so these are mapped consecutively.
Then, we get to  which is in .
Since  we let . 
Finally,  so the signature of  has .   
\end{exmp}

We will now give some basic properties of the solutions created using this
construction for the case when  and .
We will from now on assume that . 
This occurs when the construction has an equal number of applications of
 and  modulo . That is, when  (mod ).
Solving for  we get:

Assume that .
Then, the index  is determined by  and
 as follows:







\noindent
Relation (\ref{eq:index}) implies the following useful lemma:

\begin{lemma} \label{lem:usefulcong}
   iff
   (mod ).
\end{lemma}










\begin{lemma}
\label{lem:solbeta}
Let  be integers.
There exists a solution  to  of {\sc Max -COL} with , and

Furthermore, for any other solution , if , then ,
componentwise.
\end{lemma}

\begin{proof}
Let , and .
The desired solution  is obtained from the construction 
with .
As required by (\ref{eq:fulla1}), we have  so that
.
It remains to determine  for .

Let  be an edge.
In order to count the edges only once, we will assume that  (mod ). To be able to use the condition in Lemma~\ref{lem:usefulcong} we need to
determine .
But,  (mod ), where
, the inverse of  modulo .
We then obtain  by reducing  modulo .

Assuming , we have two cases in Lemma~\ref{lem:usefulcong}.
We conclude that 
 if and only if
either 

In both cases the condition is equivalent to .
Therefore, the edges  which are not mapped to an edge in  by  are the ones with an endpoint in . (There are no edges with both endpoints in this set.) 
When  is even and , this number equals .
In all other cases, there are  such edges.
The first part of the lemma follows.

For the second part, we pick an arbitrary solution  and show that we can
find at least  edges in  which can not be mapped to ,
provided that .
It is easy to see that, up to rotational symmetry, a  with  
must be constructible by  for some . 
We already know that such an  must satisfy  (mod ).
This implies .
From  (mod ), we also see that we must have
|.
As argued before, an edge from  is mapped to  if and only if one of the
congruences in (\ref{eq:recycle}) holds.
Since  (mod ), we can equivalently write this as
 if and only if
either 

Hence, either the intersection of  with  is empty or
the latter is a subset of the former.
As the two cases can be treated identically, we assume, 
without loss of generality, that the intersection is empty.
Note that 
We will now determine  edges which can not be mapped to edges
in .
Let  be the first vertex in  encountered following  from  in
the positive direction.
Similarly, let  be the first vertex in  encountered following  
from  in the negative direction.
Then, ,
for ,
but  and  by construction.
Thus, from (\ref{eq:recycle2}), the edges  
can not be mapped to .
In the other direction, we have ,
for ,
but  and  by construction.
From this we get another  edges which can not be mapped to .
Finally, we note that since  and
, the edges
 and
 are distinct.
This proves that .
\end{proof}


\begin{exmp}
With  and  the solution  to of MAX -COL created as in Lemma~\ref{lem:solbeta} with , , , , ,  looks like:

\end{exmp}









\section{Proof of Proposition~\ref{th:qisfunny}}

The proof of Proposition~\ref{th:qisfunny} follows from a series of lemmas.
The function  and how it is used for constructing solutions is
presented in Appendix~\ref{app:propm1proof}.

\begin{lemma}
\label{lem:splitend}
Let  be an integer, and  be an odd integer. Then, there exists a solution  to  of MAX -COL with the following signature:

\end{lemma}
\begin{proof}
Let  ,  and . A solution  with this signature is obtained with

We then have , so  by (\ref{eq:fulla1}).

An orbit  includes edges which connects vertices at a distance . . 
Lemma~\ref{lem:usefulcong} then says that 
if and only if  or  (mod ).
That is,  must be 0.
This is the case only when

which implies

which holds for exactly  vertices .

An orbit  includes edges which connects vertices at a distance . . Applying Lemma~\ref{lem:usefulcong} again asserts that  must be empty. Thus,

which implies

which holds for exactly  vertices .
\end{proof} 

\begin{exmp}
For . The solution  as in Lemma~\ref{lem:splitend} with  and  has , , , ,  and looks like:

\end{exmp}

The following technical lemma will prove useful in analysing the solutions
in Lemma~\ref{lem:corw}.
Some cases of the defined (partial) function  which are not needed 
for this analysis have been left out.


\begin{lemma}
\label{lem:gamma}
Let  be positive integers so that  and . Now consider  elements equidistantly placed on a circle, and select two sequences  and , each containing  consecutive elements, with  elements between them on one side and  on the other side. Let  be the number of ways to select  consecutive elements on the circle with exactly  elements from . Then,
when :

when :

and when :

\end{lemma}
\begin{proof}
Call the elements . Suppose  and . 
For . Then the sequences that starts with  as well as  are the only ones that do not contain any element from  and that is  and  elements, and in total . To get  elements we have the sequences that starts with  and  as the only options, and that is  in both cases so  in total. Also if  clearly there is no sequence of length  that includes element from both  and .

For , the ones starting with  are the only ones that do not contains any element from  and that is  elements. To get  elements we have that for any sequence of length  starting with an element  contains  number of elements from  so all sequences starting with  contains  elements from , and that is  in total. This fact also makes it clear that no sequence can contain more than p elements from .

Finally when , again sequences starting with  are the only ones to not contain any element from  or , and for sequences that include  element, they must start with  or . That is both with  for a total of . Also the only ones to contain  elements are the ones starting with  and , and since  there are of course no sequence to contain more than that.
\end{proof}

 
\begin{lemma}
\label{lem:corw}
Let  be a solution to  of {\sc MAX -COL} with  and where , and , where

and 

such that 
.
Let  be the orbit consisting of edges  with . Then,

\end{lemma}
\begin{proof}
We can apply Lemma~\ref{lem:gamma}, since we according to Lemma~\ref{lem:usefulcong} must have  or . So for Lemma~\ref{lem:gamma} we have , ,  and . We see that  when  and when  then  and when  then . So all we have to do is for each case count .
\end{proof}

Now it is possible to construct a series of signatures with solutions  where  will have the properties sought after by Lemma~\ref{lem:corw}.

\begin{lemma}
\label{lem:splitmiddle}
There exists a set of solutions  to  of the problem  with signatures:

\end{lemma}
\begin{proof}
Let  with ,
where

and
.
We have  so , implying  due to (\ref{eq:fulla1}).

The orbits  include edges which connects vertices at a distance  and
. We now have the situation in Lemma~\ref{lem:corw} with  and . When  then . When  then  and when  then .

The orbits  include edges which connects vertices at a distance  and 
. Since we have  and , then  for all  and . So for Lemma~\ref{lem:corw} only the third case applies with  and . Thus, we have .

Let  with 
, where

and
.
Again, we have  so  and .

For the orbits  we have that  for all  and  so only case 1 in Lemma~\ref{lem:corw} applies and .

For the orbits  we have exactly the same situation as in Lemma~\ref{lem:corw} with  and . We notice that when  then  and  , when  then  and  and when  then  and .
\end{proof}

One important thing to notice here is that  with  is the continuation of  with . Since . Another observation is that signature  from Lemma~\ref{lem:splitmiddle} can always be removed from a complete set of signatures and it will still remain complete since the signature from Lemma~\ref{lem:splitend} is better or equal for all orbits. 

\begin{exmp}
With  and  the solution  to  of {\sc MAX -COL} from Lemma~\ref{lem:splitmiddle} has

and looks like:

\end{exmp}







\noindent
We will now prove Proposition~\ref{th:qisfunny} using the solutions from
Lemma~\ref{lem:solalpha}, Lemma~\ref{lem:splitend} and Lemma~\ref{lem:splitmiddle}.

\subsection*{Proposition~\ref{th:qisfunny}}
\begin{proof}
We get  inequalities from Lemma~\ref{lem:solalpha},~\ref{lem:splitend}, 
and~\ref{lem:splitmiddle}, where as noted above, we have removed the inequality
generated by .
As variables we have  and , for .
To solve the relaxation of (\ref{lp}), we solve the corresponding system with
equalities.
A similar treatment of the dual confirms that the obtained solution is
indeed the optimum.



We start by reducing our  system to
a  system.
However we need to rearrange the orbits to conveniently describe how they depend on each other. Let , where  and . Furthermore introduce new solutions  so that  denotes the solution that maximises . This rearrangement makes sense, as it puts the orbits and solution in such an order that for all solutions  we have .

Now we compare the equations in (\ref{lp}) from the signatures  and .  Note that these are the signatures  and  from Lemma~\ref{lem:splitend}. We see then that we have 
 
since we assume , we get .
For the general case we have

for .
Since again we assume , we get

for . For  this means . For all other , we use the fact that (\ref{eq:omega}) also holds for  and thus have:

for .
From (\ref{eq:omega+1}) we get,

We then insert (\ref{eq:sum}) into (\ref{eq:omega}) to express  in terms of  and  only:

for .
We now define  with

Thus, we can express  in terms of . However, to proceed we need to express the coefficients  in terms of  and . Define . We do not have to worry about the upper limit, since as far as we are concerned the recursion could go on towards infinity, without affecting the values we are interested in. After multiplying with  and summing up from  we get

which we identify as

Solving for  gives

The denominator has two distinct roots whose reciprocals are:

Hence, we can express the th coefficient of  as

We can now write down the smaller  system of equations. Let . From the equations of the signatures  (from Lemma~\ref{lem:solalpha}),  (from Lemma~\ref{lem:splitend}), and  ( from Lemma~\ref{lem:splitmiddle}), we get

Solving this gives

where

\end{proof}






























\section{Proofs of Results from Section~\ref{sec:apply}}

\subsection*{Lemma~\ref{lem:circlesandwich}}
\begin{proof}
Since  means there exist one  such that , and since  we have  and  has a homomorphism to every graph that contains at least one edge so  and we can apply Lemma~\ref{lem:sandwich}. We also have that  has a homomorphism to each graph which contains an odd cycle with length at most . It is obvious that  has an homomorphism into a graph containing a cycle of length exactly . But we also know that  if  so if  contains an odd cycle of length at most  then we have . 
\end{proof}

\subsection*{Proposition~\ref{thmI}}
\begin{proof}
By Pan and Zhu we know the following for graphs  that are
-minor-free and integers :
\begin{itemize}
\item If G has odd girth at least  then ;
\item If G has odd girth at least  then ;
\item If G has odd girth at least  then .
\end{itemize}
The above, combined with Proposition~\ref{prop:4k+4}, can be used to
specify values on . We get that when the odd girth is at least
 then  and when the odd girth is at
least  then . For graphs with odd
girth  the result of Pan and Zhu give no other guarantee than that
a homomorphism exists to the cycle , which gives us no better
bound than for graphs with girth .
\end{proof}




\end{document}
