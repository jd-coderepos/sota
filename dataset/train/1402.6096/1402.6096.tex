\documentclass[11pt]{article}


\usepackage{cite}
\usepackage{graphicx}
\usepackage{amsmath}
\usepackage{amsthm}
\usepackage{amsfonts}
\usepackage{amssymb}
\usepackage{fullpage}
\usepackage{color}
\usepackage{latexsym}
\usepackage{algorithm}
\usepackage[noend]{algorithmic}
\usepackage{subfigure}
\usepackage{soul}
\usepackage{fixltx2e}


\newtheorem{theorem}{Theorem}[section]
\newtheorem{corollary}[theorem]{Corollary}
\newtheorem{lemma}[theorem]{Lemma}
\newtheorem{claim}[theorem]{Claim}
\newtheorem{observation}[theorem]{Observation}
\newtheorem{definition}[theorem]{Definition}
\newtheorem{conjecture}[theorem]{Conjecture}
\newtheorem{proposition}[theorem]{Proposition}
\newtheorem{problem}[theorem]{Problem}



\def\marrow{\marginpar[\hfill]{}}
\def\todo#1{\textsc{(TODO: \marrow\textsf{#1})}}
\def\rom#1{\textsc{(Rom says: \marrow\textsf{#1})}}
\def\matya#1{\textsc{(Matya says: \marrow\textsf{#1})}}


\renewcommand{\algorithmicrequire}{\textbf{Input:}}
\renewcommand{\algorithmicensure}{\textbf{Output:}}

\newcommand{\old}[1]{{{}}}

\def\bd{{\partial}}
\def\segment#1{{\overline{#1}}}
\def\wedge#1{{\textsc{w}_{#1}}}
\def\orientation#1{{\theta(#1)}}

\def\halfplane#1{{h_{{#1}}}}


\def\leftray#1{{\mathrel{\vbox{\offinterlineskip\ialign{\hfil##\hfil\cr
    \cr
    \cr}}}}}

    

\def\rightray#1{\mathrel{\vbox{\offinterlineskip\ialign{\hfil##\hfil\cr
    \cr
\cr}}}}


\MakeRobust{\leftray}
\MakeRobust{\rightray}

\def\thirdray#1{\widetilde{\textsc{w}}_{#1}}

\def\connected#1#2{\{{#1}\} \leftrightarrow \{{#2}\}}
\def\notconnected#1#2{\{{#1}\} \not\leftrightarrow \{{#2}\}}
\def\bisector#1{bis(\wedge{#1})}


\def\leftline#1{{l(\leftray{#1})}}
\def\rightline#1{{l(\rightray{#1})}}
\def\topregion#1{{{R}^{top}_{{#1}}}}
\def\bottomregion#1{{{R}^{bot}_{{#1}}}}


\def\ra{R_1}
\def\rb{R_2}
\def\rc{R_3}
\def\rd{R_4}
\def\re{R_5}
\def\rf{R_6}

\newcommand{\newtext}[1]{\hl{#1}}
\newcommand{\oldtext}[1]{\st{#1}}


\def\C{{\cal{C}}}


\def\grid{{\cal{G}}}
\def\graph{{{G}}}
\def\UDG{{\textsc{udg\/(\textit{\small P})}}}
\def\nf{{n\!f}}


\begin{document}

\title{Bounded-Angle Spanning Tree: Modeling Networks with Angular Constraints\thanks{Work by R. Aschner was partially supported by the Lynn and William Frankel Center for Computer Sciences. Work by M. Katz was partially supported by grant 1045/10 from the Israel Science Foundation. Work by M. Katz and R. Aschner was partially supported by grant 2010074 from the United States -- Israel Binational Science Foundation.}
}


\author{Rom Aschner \ \ \  Matthew J. Katz
\\
\\
{\small Department of Computer Science, Ben-Gurion University, Israel} \\
{\small {\tt romas,matya@cs.bgu.ac.il}}
}



\maketitle


\begin{abstract} 
We introduce a new structure for a set of points in the plane and an angle , which is similar in flavor to a bounded-degree MST. We name this structure -MST. 
Let  be a set of points in the plane and let  be an angle. An -ST of  is a spanning tree of the complete Euclidean graph induced by , with the additional property that for each point , the smallest angle around  containing all the edges adjacent to  is at most . An -MST of  is then an -ST of  of minimum weight.
For , an -ST does not always exist, and, for , it always exists~\cite{AGP13,AHHHPSSV13,CKLR11}. In this paper, we study the problem of computing an -MST for several common values of .


Motivated by wireless networks, 
we formulate the problem in terms of directional antennas. With each point , we associate a wedge  of angle  and apex .
The goal is to assign an orientation and a radius  to each wedge , such that the resulting graph is connected and its MST is an -MST. (We draw an edge between  and  if , , and .)
Unsurprisingly, the problem of computing an -MST is NP-hard, at least for  and . We present constant-factor approximation algorithms for .

One of our major results is a surprising theorem for , which, besides being interesting from a geometric point of view, has important applications. For example, the theorem guarantees that 
given {\em any} set  of  points in the plane and {\em any} partitioning of the points into  triplets, one can orient the wedges of each triplet {\em independently}, such that the graph induced by  is connected. We apply the theorem to the {\em antenna conversion} problem.
\end{abstract}

\section {Introduction}

Let  be a set of points in the plane and let  be an angle. An -ST of  is a spanning tree of the complete Euclidean graph induced by , with the additional property that for each point , the smallest angle around  containing all the edges adjacent to  is at most . An -MST of  is then an -ST of  of minimum weight.
 
In this paper, we study the problem of computing an -MST for several common values of . For , an -ST does not always exist (consider, e.g., an equilateral triangle). Moreover, it is well known that there always exists a Euclidean MST of degree at most 5.
Therefore, it is interesting to focus on the range .

Carmi et al.~\cite{CKLR11} showed that, for , an -ST always exists. A somewhat simpler construction was subsequently proposed by Ackerman et al.~\cite{AGP13}. Aichholzer et al.~\cite{AHHHPSSV13} have also obtained this result (together with additional related results),   independently. However, in all these papers, the goal is to construct an -ST (for ) and not an -MST. 

The problem of computing an -MST is similar in flavor to the problem of computing a Euclidean minimum weight degree- spanning tree, which has been studied extensively (see, e.g.,~\cite{A98, C04, JR09, KRY96, M99}). A minimum weight degree- spanning tree is a minimum weight spanning tree, such that the degree of each point is at most , where the interesting values of  are 2,3, and 4. Notice that for  we get the Euclidean traveling salesman path problem. 

The problem of computing an -ST is closely related to problems in which one needs to compute a Hamiltonian path or cycle, with some restrictions on the angles. Fekete and Woeginger~\cite{FW97} showed that every set of points has a Hamiltonian {\em path}, such that all its angles are bounded by . An alternative construction was given later in~\cite{CKLR11}. Fekete and Woeginger also conjectured that for every set of  points there exists a Hamiltonian {\em cycle}, such that all its angles are bounded by . Recently, Dumitrescu et al.~\cite{DPT12} showed how to construct a Hamiltonian cycle whose angles are bounded by .
As for lower bound, in~\cite{CKLR11} and, independently, in~\cite{DPT12} it is shown that, for any , there exists a set of points, for which any Hamiltonian path has an angle greater than .
The problem of finding Hamiltonian paths with large angles was also considered in~\cite{FW97}, where it is conjectured that every point set admits a Hamiltonian path, whose angles are at least ; B{\'a}r{\'a}ny et al.~\cite{BPV09} showed how to construct a path, whose angles are at least . 

Unsurprisingly, the problem of computing an -MST is NP-hard, at least for  and . 
For , one can show this by a reduction from the problem of finding a Hamiltonian path in grid graphs of degree at most , which is known to be NP-hard~\cite{IPS82}. The reduction is similar to the one described for the problem of computing a minimum weight degree-3 spanning tree~\cite{PV84}, with a few simple adaptations. 
For , one can show this by a straight-forward reduction from Hamiltonian path in hexagonal grid graphs. Arkin et al.~\cite{AFIMMRPRX09} showed that the problem of finding a Hamiltonian cycle in hexagonal grid graphs is NP-hard. However, with not too much effort, one can prove that finding a Hamiltonian path in hexagonal grid graphs is NP-hard as well. 

Motivated by wireless networks, we formulate the problem of computing an -MST in terms of directional antennas. 
In the last few years, directional antennas have received considerable attention (see, e.g.,~\cite{KKM,BCDFKM11,CKKKW08}), as they have some noticeable advantages over omni-directional antennas. In particular, they require less energy to reach a receiver at a given distance, and
when broadcasting to this receiver the affected region is much smaller, reducing the probability of causing interference at friendly receivers or being subject to eves dropping by hostile receivers.
With each point , we associate a wedge  of angle  and apex . The goal now is to assign an orientation and a radius  to each wedge , such that the resulting graph is connected and its MST is an -MST. (We draw an edge between  and  if , , and .) 

An interesting related problem is the {\em antenna conversion} problem. The {\em unit disk graph} of , denoted , is the graph in which there is an edge between  and  if . This is the communication graph induced by , where each point in  represents a transceiver equipped with an omni-directional antenna of radius 1. We assume that  is connected. Suppose that one wishes to replace the omni-directional antennas with directional antennas of angle . The goal now is to assign an orientation to each of the wedges  and to fix a common range , such that the resulting (symmetric) communication graph is a -hop-spanner of , where . Moreover,  and  should be small constants.
Aschner et al.~\cite{AKM13} considered this problem for . Here we solve it for , using significantly smaller constants. 



\paragraph{Our results.}
In Section~\ref{sec:gadget} we focus on the case . We begin by describing a simple gadget: Given any set  of three points in the plane, we show how to orient the wedges associated with the points of , such that , the graph induced by , is connected, and, moreover, the union of the wedges of  covers the plane. We then prove a surprising theorem, which, besides being interesting from a geometric point of view, has far-reaching applications, such as the one mentioned in the abstract. Informally, the theorem states that any two such gadgets are connected. That is, let  and  be two triplets of points in the plane, and assume that the wedges (associated with the points) of  and, independently, of  are oriented according to the gadget construction instructions, then the graph induced by  is connected.
Proving this theorem turned out to be a very challenging task, due to the huge number of possible configurations that must be considered, and only after arriving at the current three-stage proof structure (see Section~\ref{sec:main_theorem}), were we able to complete the proof.


In Section~\ref{sec:tsp_apx}, we present constant-factor approximation algorithms for computing an -MST. In particular, we compute a -approximation for a -MST, a -approximation for a -MST, and a -approximation for a -MST. These approximations are actually with respect to a Euclidean MST, which is a lower bound for an -MST, for any .
In Section~\ref{sec:boundedrange}, we present a solution to the antenna conversion problem for , based on the theorem above. Specifically, we construct, in  time, a 6-hop-spanner of , in which each edge is of length at most 7.
Finally, NP-hardness proofs for the problem of computing an -MST, for  and , can be found in Section~\ref{sec:np_hardness}.

\section{}\label{sec:gadget}

{\bf Notation.}
Let  be a point and let  be an angle. We denote the wedge of angle  and apex  by . 
The left ray bounding  (when looking from  into ) is denoted by  and the right ray by . The bisector of  is denoted by .  
The orientations of , , and  are denoted by , , and , respectively.
The {\em orientation} of  is the orientation of its bisector and is denoted by
. We denote the ray emanating from  of orientation  by ; its orientation is denoted by . 

Let  be a set of points, where each point  is associated with a wedge  of some orientation.
The graph induced by , denoted , is the graph in which there is an edge between  if and only if  and . If there is an edge between  and , we say that  and  are {\em connected} and denote this by . Similarly, if  and  are two such sets of points, and there exist a point  in  and a point  in  such that  and  are connected, then we say that  and  are {\em connected} and denote this by . The notation  means that  and  are not connected, and, similarly,  means that there does not exist a point in  and a point in  such that these points are connected.

\subsection{The basic gadget}\label{sec:orient}

\begin{claim}\label{lem:three_pts}
Let  be a set of three points in the plane, and set . Then, one can orient the wedges
of , such that , the induced graph of , contains a -ST of , and the wedges of  cover
the plane.
\end{claim}

\begin{proof}
Consider , and
assume w.l.o.g. that . Then,  and . Draw , such that  is horizontal (with  to the left of ) and  is not below the line containing .
Orient the wedges of  as follows (see Figure~\ref{fig:three_pts_fig}(a)): 
,
,
. 

It is easy to see that the non-directed edges  and  are in the induced graph .
Thus,  contains a -ST. As for the second requirement, notice that 
 contains the wedge  of orientation  and apex , and
 contains the wedge  of orientation  and apex . But, clearly,
.  
\end{proof}



\begin{figure}[htb]
\centering
  \subfigure[]{
   \centering
       \includegraphics[width=0.35\textwidth,page=1]{fig/three_pts_regions}
  }
  \subfigure[]{
   \centering
       \includegraphics[width=0.35\textwidth,page=2]{fig/three_pts_regions}
  }
  \subfigure[]{
   \centering
       \includegraphics[height=0.19\textwidth,page=1]{fig/cones}
  }
 	\caption{(a) The basic gadget of Claim~\ref{lem:three_pts}. , , and . A point  is in region  if and only if  and , i.e., . Regions  and  are defined analogously. (b) , , and . (c) The six ranges .}	\label{fig:three_pts_fig}	
\end{figure} 


The gadget of Claim~\ref{lem:three_pts} has some noticeable properties:\\
{\bf Property 1.} For any , the orientations of the wedges of  are  and .\\ 
{\bf Property 2.} For any , the orientations of the rays bounding the wedges of  are  and . Moreover, each of these three orientations appears exactly twice, once as the orientation of a left ray bounding some wedge and once as the orientation of a right ray bounding some other wedge (see Figure~\ref{fig:three_pts_fig}(b)).\\
{\bf Property 3.} Consider any two wedges  and  and the four rays defining them. Then, by Property~2, exactly two of these rays,  from  and  from , have the same orientation. Let  be a line intersecting both  and  and perpendicular to  (and to ). Then,  covers the halfplane defined by  that does not include the points  and .

Finally, let  denote the range , for  (see Figure~\ref{fig:three_pts_fig}(c)).

  

\subsection{The induced graph of  is connected}
\label{sec:main_theorem}

In this section, we prove the following surprising theorem (Theorem~\ref{thm:no_cliques}), which, as mentioned, has far-reaching applications.
Let  and  be two triplets of points in the plane, and assume that the wedges (associated with the points) of  and, independently, of  are oriented according to the proof of Claim~\ref{lem:three_pts}. Then, the induced graph of  is connected. 

In order to cope with the huge number of cases, we prove Theorem~\ref{thm:no_cliques} in three stages.
In the first stage (Lemma~\ref{lem:two_cliques}), we prove the statement assuming that both induced graphs of  and of  are cliques. In the second stage (Lemma~\ref{lem:one_clique}), we prove the statement assuming only one of the induced graphs is a clique, using, of course, Lemma~\ref{lem:two_cliques}. Finally, in the third stage (Theorem~\ref{thm:no_cliques}), we prove the statement without any additional assumptions, using Lemma~\ref{lem:one_clique}.

Throughout this section, we assume (as in the proof of Claim~\ref{lem:three_pts}) that, in , ,  is horizontal, with  to the left of , and  is not below the line  containing  (see Figure~\ref{fig:three_pts_fig}(a)).     

\begin{lemma}[Two cliques]\label{lem:two_cliques}
Let  and  be two triplets of points in the plane and let .
Assume that the wedges (associated with the points) of  and, independently, of  are oriented according to the proof of Claim~\ref{lem:three_pts}, and that both induced graphs,  and , are cliques.
Then, the induced graph  is connected.
\end{lemma}

\old{
\begin{proof}
The wedges of  cover the plane, in particular they cover all points of . Therefore, we distinguish between three (not necessarily disjoint) cases: 
(i) there exists a point  such that  covers all points of , 
(ii) there exists a point  such that  covers exactly two points of , and 
(iii) the wedge of each point in  covers exactly one point of .

{\bf Case (i):} There exists a point  such that  covers all points of . Since
the wedges of  cover the plane, at least one of them must cover , and therefore
.

{\bf Case (ii):} There exists a point  such that  covers exactly two points of . We divide this
case into three sub-cases, according to which two points of  are covered by .

\begin{figure}[htb]
 \centering
 \subfigure[]{
   \centering
       \includegraphics[width=0.45\textwidth]{fig/proof_case_2_b_2}
 }
 \subfigure[]{
  \centering
       \includegraphics[width=0.45\textwidth]{fig/proof_case_2_b_1}
 }
 \caption{Proof of Lemma~\ref{lem:two_cliques}, Case (ii)(1).}	\label{fig:case2b}	
\end{figure} 

(1)  covers  and  and does not cover . Assume  (since otherwise
we are done), then  and one of the rays of  intersects  and
. Notice that this ray must be  and that  also intersects  (see~Figure~\ref{fig:case2b}).
Since  lies below ,  intersects , and , we have that . 
It follows that , , and .
Therefore,  (whose orientation is ) does not intersect .
Let  be the point of  such that  and 
. Since , we have that
 and  lies to the right of . Notice that  contains the (imaginary) wedge of orientation  and apex . If  (see Figure~\ref{fig:case2b}(a)), then , since  covers . 
Otherwise,  and in particular  lies to the right of  (see Figure~\ref{fig:case2b}(b)). In this case we show that . Indeed,  intersects  to the right of , since , and, since  is parallel to  and below it, we have that  intersects  to the left of . We conclude that  and .

\begin{figure}[htb]
 \centering
       \includegraphics[width=0.35\textwidth]{fig/proof_case_2_c}
 \caption{Proof of Lemma~\ref{lem:two_cliques}, Case (ii)(2).}	\label{fig:case2c}	
\end{figure} 
  
(2)  covers  and  and does not cover . This case is similar to Case~(ii)(1). Assume  (since otherwise
we are done), then 
and one of the rays of  intersects  and . Notice that this ray must be  and that  also intersects  (see Figure~\ref{fig:case2c}).
Since  lies above ,  intersects , and , we have that . 
It follows that , , and . Therefore,  does not intersect .
Let  be the point of  such that  and
. 
Since , we have that  and  lies to the right of . Notice that  contains the (imaginary) wedge of orientation  and apex . If , then , since  covers . 
Otherwise,  and in particular  lies to the left of . In this case we show that . Indeed,  does not intersect , since  lies below  and , and, since  is parallel to  and above it, we have that  intersects . We conclude that  and .

\begin{figure}[htb]
 \centering 
  \subfigure[ intersects  and ]{
    \centering
        \includegraphics[width=0.45\textwidth]{fig/proof_case_2_d}
	\label{fig:case2d}
 }
  \subfigure[ intersects  and ]{
    \centering
        \includegraphics[width=0.45\textwidth]{fig/proof_case_2_e}
	\label{fig:case2e}
 }
 \caption{Proof of Lemma~\ref{lem:two_cliques}, Case (ii)(3).}	\label{fig:caseb}
\end{figure}
 
(3)  covers  and  and does not cover . Assume  (since otherwise
we are done), then ,
and one of the rays of  intersects  and . Notice that this ray 
can be either  or . 

If it is  (see Figure~\ref{fig:case2d}), then
the orientations associated with  are:
, , and .
Notice that  does not intersect .
Let  be the point of  such that  and . Since , we have that  and  lies to the right of . 
Notice that  contains the (imaginary) wedge of orientation  and apex . If , then , since  covers . 
Otherwise, . Since  passes to the right of  (or through ) and , we have that  and therefore .

If the ray intersecting  and  is  (see Figure~\ref{fig:case2e}), then
the orientations associated with  are:
, , and .
Notice that  does not intersect .
Let  be the point of  such that  and . 
Since , we have that  and  lies to the left of . 
Notice that  contains the (imaginary) wedge of orientation  and apex . If , then , since  covers . 
Otherwise, . Since  passes to the right of  (or through ) and , we have that  and therefore .

\begin{figure}[htb]
 \centering 
    \includegraphics[width=0.5\textwidth]{fig/proof_case_3_a}
 \label{fig:case3a}
 \caption{Proof of Lemma~\ref{lem:two_cliques}, Case (iii).}	\label{fig:case3}
\end{figure}

{\bf Case (iii):} The wedge of each point in  covers exactly one point of . We may assume that this condition also holds for the wedges of ; that is, the wedge of each point in  covers exactly one point of . Since, otherwise, we can simply interchange the set names. 
It follows that each point of  lies in its own private region among the regions , , and . 

Let  be the point that lies in . We claim that .
Assume that . We show that there exists a point  that covers two points of .
If  covers  (see Figure~\ref{fig:case3}), then , which implies that . Let  be the point of  such that  and . Since , we have that  and  lies to the left of , but then  must cover  and  -- contradiction.
If  covers , then , which implies that . Let  be the point of  such that  and . Since , we have that  and  lies to the right of , but then  must cover  and  -- contradiction.

\end{proof}
}

\begin{proof}
The wedges of  cover the plane, in particular they cover all points of . Therefore, we distinguish between three (not necessarily disjoint) cases: 
(i) there exists a point  such that  covers all points of , 
(ii) there exists a point  such that  covers exactly two points of , and 
(iii) the wedge of each point in  covers exactly one point of .

{\bf Case (i):} There exists a point  such that  covers all points of . Since
the wedges of  cover the plane, at least one of them must cover , and therefore
.

{\bf Case (ii):} There exists a point  such that  covers exactly two points of . We divide this
case into three sub-cases, according to which two points of  are covered by .

\begin{figure}[htb]
 \centering
 \subfigure[]{
   \centering
       \includegraphics[width=0.45\textwidth]{fig/proof_case_2_b_2}
 }
 \subfigure[]{
  \centering
       \includegraphics[width=0.45\textwidth]{fig/proof_case_2_b_1}
 }
 \caption{Proof of Lemma~\ref{lem:two_cliques}, Case (ii)(1).}	\label{fig:case2b}	
\end{figure} 

(1)  covers  and  and does not cover . Assume  (since otherwise
we are done), then  and one of the rays of  intersects  and
. Notice that this ray must be  and that  also intersects  (see~Figure~\ref{fig:case2b}).
Since  lies below ,  intersects , and , we have that . 
It follows that , , and .
Therefore,  (whose orientation is ) does not intersect .
Let  be the point of  such that  and 
. Since , we have that
 and  lies to the right of . Notice that  contains the (imaginary) wedge of orientation  and apex . If  (see Figure~\ref{fig:case2b}(a)), then , since  covers . 
Otherwise,  and in particular  lies to the right of  (see Figure~\ref{fig:case2b}(b)). In this case we show that . Indeed,  intersects  to the right of , since , and, since  is parallel to  and below it, we have that  intersects  to the left of . We conclude that  and .

\begin{figure}[htb]
 \centering
       \includegraphics[width=0.35\textwidth]{fig/proof_case_2_c}
 \caption{Proof of Lemma~\ref{lem:two_cliques}, Case (ii)(2).}	\label{fig:case2c}	
\end{figure} 
  
(2)  covers  and  and does not cover . Assume  (since otherwise
we are done), then 
and one of the rays of  intersects  and . Notice that this ray must be  and that  also intersects  (see Figure~\ref{fig:case2c}).
Since  lies above ,  intersects , and , we have that . 
It follows that , , and . The rest of the proof for this case is very similar to the proof of Case~(ii)(1), thus we omit further details.

\begin{figure}[htb]
 \centering 
  \subfigure[ intersects  and ]{
    \centering
        \includegraphics[width=0.45\textwidth]{fig/proof_case_2_d}
	\label{fig:case2d}
 }
  \subfigure[ intersects  and ]{
    \centering
        \includegraphics[width=0.45\textwidth]{fig/proof_case_2_e}
	\label{fig:case2e}
 }
 \caption{Proof of Lemma~\ref{lem:two_cliques}, Case (ii)(3).}	\label{fig:caseb}
\end{figure}

 
(3)  covers  and  and does not cover . Assume  (since otherwise
we are done), then ,
and one of the rays of  intersects  and . Notice that this ray 
can be either  or . 

If it is  (see Figure~\ref{fig:case2d}), then
the orientations associated with  are:
, , and .
The rest of the proof for this branch is very similar to the proof of Case (ii)(1), thus we omit further details.

If the ray intersecting  and  is  (see Figure~\ref{fig:case2e}), then
the orientations associated with  are:
, , and .
Again, the rest of the proof for this branch is very similar to the proof of Case (ii)(1), thus we omit further details.

\begin{figure}[htb]
 \centering 
    \includegraphics[width=0.5\textwidth]{fig/proof_case_3_a}
 \label{fig:case3a}
 \caption{Proof of Lemma~\ref{lem:two_cliques}, Case (iii).}	\label{fig:case3}
\end{figure}

{\bf Case (iii):} The wedge of each point in  covers exactly one point of . We may assume that this condition also holds for the wedges of ; that is, the wedge of each point in  covers exactly one point of . Since, otherwise, we can simply interchange the set names. 
It follows that each point of  lies in its own private region among the regions , , and . 

Let  be the point that lies in . We claim that .
Assume that . We show that there exists a point  that covers two points of .
If  covers  (see Figure~\ref{fig:case3}), then , which implies that . Let  be the point of  such that  and . Since , we have that  and  lies to the left of , but then  must cover  and  -- contradiction.
If  covers , then , which implies that . Let  be the point of  such that  and . Since , we have that  and  lies to the right of , but then  must cover  and  -- contradiction.

\end{proof}











\begin{lemma}[One clique]\label{lem:one_clique}
Let  and  be two triplets of points in the plane and let .
Assume that the wedges of  and, independently, of  are oriented according to the proof of 
Claim~\ref{lem:three_pts}, and that the induced graph  is a clique.
Then, the induced graph  is connected.
\end{lemma}

\begin{proof}
If the induced graph  is also a clique, then, by Lemma~\ref{lem:two_cliques}, we are done.
Assume therefore that  is not a clique. 
Let  be the intersection point of  and  (see Figure~\ref{fig:oneclique}), and consider the wedge  of orientation  and apex .
The graph induced by  is a clique, and therefore, by Lemma~\ref{lem:two_cliques}, .
If , then we are done, so assume that .
Let  be a point of  such that , and assume that  does not cover  (if it does, then , since ). Then,  lies above  and  intersects .
Below we consider the three cases: (i)  intersects , (ii)  intersects  to the left of , and (iii)  does not intersect . However, in the first case (i.e., Case~(i)) and in sub-cases (1) and (2) of the second case (i.e., Case~(ii)(1) and Case~(ii)(2)) we refrain from using the assumption that  is a clique. This is because these cases appear again later in the proof of Theorem\mbox{~\ref{thm:no_cliques}}, where we may not assume that  is a clique.  


\begin{figure}[htb]
 \centering 
 \subfigure[Case (i)]{
    \centering
        \includegraphics[width=0.30\textwidth]{fig/one_clique_1a}
	     \label{fig:oneclique_case_1_a}}
 \subfigure[Case (ii)]{
    \centering
        \includegraphics[width=0.30\textwidth]{fig/one_clique_2a}
	     \label{fig:oneclique_case_1_b}}
 \subfigure[Case (iii)]{
    \centering
        \includegraphics[width=0.30\textwidth]{fig/one_clique_3a}
	     \label{fig:oneclique_case_1_c}}
	\caption{Proof of Lemma~\ref{lem:one_clique}.}	\label{fig:oneclique}
\end{figure}



{\bf Case (i):}  intersects  (see Figure~\ref{fig:oneclique_case_1_a}). Notice that in this case  does not cover points  and . Since  and , we get that , , and
.
Between the two points in , let  be the one whose wedge covers more points of ; in case of tie, let  be any one of them. We know that one of 's rays has orientation in .
There are five sub-cases:

(1)  covers all points of . There must exist a point in  that covers , so we are done. 

(2)  covers  and  and does not cover . If , then we are done. Otherwise, . Now, since  and  must cover  and  and avoid , we get that . But, this is impossible, since  is not among the three relevant ranges mentioned above. 

(3)  covers  and  and does not cover . If , then we are done. Otherwise, . We show that this is impossible.
If , then , and . And, if , then , and  must also cover .

(4)  covers  and  and does not cover . This case is analogous to the previous one.


\old{
If , then we are done. Otherwise, .
We show that this is impossible.
If  , then , and . And, if , then , and  must also cover .
}

(5)  covers exactly one point of . Therefore, the wedge of each point in  covers exactly one point of . Since  does not cover points  and , it must cover .
Assume, w.l.o.g., that  covers  and , the wedge of the remaining point, covers . Next, we show that this is impossible.
Indeed, if  and , then  must also cover  and . 
And, if  and , then both  and  must lie below . (Since, if  is above , then , and, if  is above , then ). Therefore,  covers the halfplane above  (see Property~3), and, in particular, at least one of the two wedges covers .



{\bf Case (ii):}  intersects  to the left of  (see Figure~\ref{fig:oneclique_case_1_b}). In this case, as in Case (i), , , and
. Notice that in this case , so we assume that
, since otherwise . Let  be a point of  whose wedge covers
. We distinguish between three sub-cases:

(1)  covers all points of . There must exist a point in  that covers , so we are done.

(2)  covers exactly two points of . If  covers  and  and , then  and either  or .
However, in both cases,  must also cover  -- contradiction. (Since, in the former case,  does not intersect , and in the latter case,  does not intersect .) If  covers  and  and , then  and . However, in the case,  must also cover  -- contradiction. (Since  passes above  and is directed upwards, and  passes below  and is directed downward.)

(3)  covers exactly one point of , namely, . We know that either  or . In the latter case,  must also cover , which is impossible. In the former case, if  is above , then , so  is necessarily below .
Let  be the remaining point. Then, . We show below that . 
Notice first that  separates between  and  and between  and , since  and  covers only . Since  is a clique, we know that , and therefore  lies to the right of . Clearly,  and  lie to the left of  (whose orientation is in ), and to the right of  (whose orientation is in ). In other words,  covers both  and . Notice also that , since  (whose orientation is in ) intersects  to the right of , and  lies to the right of . Therefore, either  or  (or both) covers . We conclude that .  
 

{\bf Case (iii):}  does not intersect , i.e.,  (see Figure~\ref{fig:oneclique_case_1_c}). Since  covers , we may assume that . Therefore, . We thus have that  and . Notice that  (whose orientation is in )
intersects  to the right of . Moreover,  necessarily covers , since  and  intersects  between  and . Let  be the point of  such that .
Since  is a clique, we know that , and therefore  lies to the right of . If  is above , then . Otherwise,  is below  and in  (since it is to the left of ). But then , since  passes above  and  is directed downwards.
\end{proof}

\begin{theorem}\label{thm:no_cliques}
Let  and  be two triplets of points in the plane and let .
Assume that the wedges of  and, independently, of  are oriented according to the proof of 
Claim~\ref{lem:three_pts}.
Then, the induced graph  is connected.
\end{theorem}

\begin{proof}
If one (or both) of the induced graphs ,  is a clique, then, by Lemma~\ref{lem:one_clique}, we are done.
Assume therefore that none of them is a clique. 
Let  be the intersection point of  and , and consider the wedge  of orientation  and apex .
The graph induced by  is a clique, and therefore, by Lemma~\ref{lem:one_clique}, .
If , then we are done, so assume that .
Let  be a point of  such that , and assume that  (otherwise
we are done). Then,  lies above  and  intersects .
We distinguish between three cases, as in the proof of Lemma~\ref{lem:one_clique}:
(i)  intersects , (ii)  intersects  to the left of , and (iii)  does not intersect . 
As mentioned in the proof of Lemma~\ref{lem:one_clique}, our arguments there for Case~(i) and Cases~(ii)(1) and (ii)(2) do not use the extra assumption that  is a clique. Therefore, we can reuse them here. It remains to show that  in Cases~(ii)(3) and~(iii).  

{\bf Case~(ii)(3):}  covers exactly one point of , namely, . We know that either  or . In the latter case,  must also cover , which is impossible. In the former case, if  is above , then , so  is necessarily below .
Let  be the remaining point. Then, . 
At this point, we would like to show, as in the proof of Lemma~\ref{lem:one_clique}, that . However, we cannot assume now that . So, we first prove that , by proving that , and then we proceed as in the proof of Lemma~\ref{lem:one_clique}.



Thus, our goal now is to prove that . Let  be the midpoint of , and let  be the projection of  onto . According to the construction in the proof of Claim~\ref{lem:three_pts},  lies somewhere between  and  (not including ). Let  be the intersection point of  and . We know that  is somewhere between  and  (not including ). Finally, let  be the intersection point of  and  (see Figure~\ref{fig:no_clique}). We show that  lies to the left of  and therefore also to the left of . If  is to the left of  (or ), then this is clear. Assume therefore that  is to the right of , and consider the two triangles  and . Recall first that  is above  and notice that it is below  (since, if  were above , then ). Therefore  and the projection of  onto  lies to the left of . Now, in ,  and , and therefore . And, in ,  and , and therefore . Together, we get that , so  lies to the left of  and therefore to the left of .

Since the projection of  onto  lies to the left of  and so does , we have that  lies to the left of . Now, if , then  must lie to the right of  and therefore cover , which is impossible. We conclude that , and therefore  (and ). 

From this point, we continue as in the proof of Lemma~\ref{lem:one_clique}.  
Notice that  separates between  and  and between  and , since  and  covers only . Since , we know that  lies to the right of . Clearly,  and  lie to the left of  (whose orientation is in ), and to the right of  (whose orientation is in ). In other words,  covers both  and . Notice also that , since  (whose orientation is in ) intersects  to the right of , and  lies to the right of . Therefore, either  or  (or both) covers . We conclude that .  


{\bf Case~(iii):}  does not intersect , implying that .
Notice that in this case , so we assume that
, implying that . 
It follows that , , , and . 
Notice also that , whose orientation is in , intersects  to the left of .

Let  be the point of  such that  and , and let  be the point of  such that  and . 
Notice that for  to intersect  to the right of ,  must lie above , and, therefore,  covers .

We first show that if , then .
Indeed, if , then  must lie to the right of . 
If  is above , then . Assume, therefore, that  is below . Notice that  intersects  at a point above , implying that  passes above . Moreover,  passes below , since it is directed downwards. It follows that  covers . But, , since  lies to the right of , which intersects  to the left of . We conclude that .

Next, we address the most difficult case, in which .
If , then necessarily  is connected to both  and . 
Notice that  must lie below . Also, if it is above , then . Assume, therefore, that  is below .
Since 's rays are directed upwards and , we know that  is below  and therefore also below . 
According to the construction in the proof of Claim~\ref{lem:three_pts}, either  or  lies on , and the angle at this point in  does not exceed the angle at the other point. It follows that the point that lies on  is necessarily . Since, if it were , then , as it contains .


\begin{figure}[htp]
\centering

  \subfigure[]{
   \centering
       \includegraphics[width=0.31\textwidth]{fig/no_clique}   \label{fig:no_clique}
	}
  \subfigure[]{
   \centering
       \includegraphics[width=0.31\textwidth]{fig/proof_case_3}  \label{fig:case3abc}
	}
	\subfigure[]{
		 \centering
     \includegraphics[width=0.31\textwidth]{fig/example_no_edge}  \label{fig:no_edge}
   }
   \caption{(a)  lies to the left of . (b)  lies on . (c)
Each of the triplets induces a connected graph and covers the plane, but the graph of their union is not connected.}
\end{figure}

The case where  lies on  is also impossible, as we show below (see Figure~\ref{fig:case3abc}).
If , then , since  is below  and  is above . Assume, therefore, that  but .
Let  be the intersection point of  and . Then,  is above  (since otherwise ). 
Notice that  is equilateral, and consider the bisector of .
Let  be the intersection point of this bisector and side . Then,  is the perpendicular bisector of .

Next, we show that  lies above .
Let  be the intersection point of  and , and let  the intersection point of
 and .
We show that , implying that  is somewhere between  and  and thus above .
Consider . Since , we know that . But , so we get that . Now, consider .
 and , and therefore .
It follows that
.

Since all its corners lie above ,  is above . Since  and  is below , we have that 
, and therefore  is closer to  than to  -- contradiction the construction of Claim~\ref{lem:three_pts}. 
\end{proof}






\paragraph{Remark.} 
Theorem~\ref{thm:no_cliques} above proves that when the wedges of each of the triplets are oriented, independently, according to the construction of Claim~\ref{lem:three_pts}, then there is always an edge between the two triplets. This is not necessarily true for other constructions with similar properties. For example, the wedges of each of the triplets in Figure~\ref{fig:no_edge} form a connected graph and cover the plane, but there is no edge between the triplets. 

\section{Approximating the -MST} \label{sec:tsp_apx}
Let  be a set of  points in the plane.
In this section we consider the problem of computing an -MST of , for . For each of these angles, we devise a constant-factor approximation algorithm. The approximation ratios are actually with respect to the weight of a Euclidean MST, which is a lower bound for the weight of an -MST, for any .
 
Consider the TSP tour  obtained by applying the standard 2-approximation algorithm for metric TSP. This algorithm first duplicates the edges of a MST to obtain an Eulerian tour, and then transforms the Eulerian tour into a TSP tour by introducing shortcuts. Thus, .  
Each of our approximation algorithms below begins by constructing . It then constructs, using , a connected -graph, i.e., a graph in which, for each node , the angle spanned by the edges adjacent to  is at most . Finally, it construct an -ST from the -graph, whose weight is bounded by , for some constant , and thus is a -approximation of an -MST.

\paragraph{.}
Observe that any graph of maximum degree two is a -graph. In particular,  is a -graph, and, by removing an arbitrary edge, we obtain a -ST of weight at most .

\paragraph{.}
Assume, for convenience, that , for some integer .
We partition  into  triplets, by traversing  from an arbitrary point .
That is, each of the triplets consists of three consecutive points along .
Orient the wedges of each triplet, independently, according to Claim~\ref{lem:three_pts}.
By Theorem~\ref{thm:no_cliques}, the graph induced by , denoted here  (instead of ), is connected. In particular, for any two consecutive triplets  along , there exists an edge of the graph between a point of  and a point of .

Next, we construct a -ST, , and show that .
Initially,  has no edges. For each of the  triplets , add to  any two edges (of the at least two edges) of  connecting between pairs of points of . We call these edges {\em inner-edges}. Next, for each of the  pairs of consecutive triplets  along  (except for the pair consisting of the `last' triplet and the `first' triplet), add to  any edge (of the at least one edge) of  connecting between a point of  and a point of . We call these edges {\em connecting-edges}.
 is connected and has  inner-edges and  connecting-edges, thus the total number of edges is , and  is a -ST. 

We now bound the weight of . By the triangle inequality, the weight of an edge  of  does not exceed the weight of the shorter path (in terms of number of edges) in  between  and . We charge the weight of this path for the edge .
Each edge of  between two points of the same triplet  is charged at most four times. Twice for the two inner-edges chosen for , and twice for the two connecting-edges that connect  to its two adjacent triplets along .
Each edge of  between two consecutive triplets  (except for the edge between the last and first) is charged only once for the corresponding connecting-edge of . 
Thus, each edge of  is charged at most four times, and .

Next, we improve the approximation ratio. Observe, that there are three possible ways to partition  into  triplets.
In other words, the set of edges of  connecting between the triplets can be either , or , where
, for . By the pigeon hole principle, the weight of one of these sets, say , is at least . We partition  into triplets, such that the set of edges connecting between the triples is . Now, each of the edges of  (except ) is charged exactly once, and each of the edges of  is charged at most four times. Thus, 
.

\paragraph{.}
Assume, for convenience, that , for some integer .
Our construction for  is similar to the one for , but slightly more complicated. It is based on a basic gadget described by Aschner et al.~\cite{AKM13} for a set  of four points, indicating the locations of four -wedges. This gadget is presented as the proof for the claim that one can orient the wedges of , such that the induced graph is connected, and the wedges of  cover the plane.
Unfortunately, we cannot claim that two quadruplets, whose wedges are oriented independently, are connected. However, if they are separable by a line, then they are connected, see~\cite{AKM13}. 

We use this latter claim in our construction.
We partition the tour  into  sections, each consisting of 8 consecutive points along . Then, we partition each of the sections into two quadruplets, a left quadruplet consisting of the 4 leftmost points of the section and a right quadruplet consisting of the 4 rightmost points. (Notice that the points of a quadruplet are not necessarily consecutive along .) Thus, in each section, the two quadruplets are separable by a (vertical) line. Now, orient the wedges of each quadruplet, independently, such that their induced graph is connected and the wedges cover the plane. Let  be the graph induced by . Observe that  is connected, since, for any two consecutive sections, there exists two quadruplets, one from each section, that are separable by a (vertical) line and thus connected.

Next, we construct the tree  from . We distinguish between three types of edges. The first type are the {\em inner-edges}, which connect between points of the same quadruplet. For each quadruplet, we pick three such edges that make the quadruplet connected. The second type are the {\em q-connecting-edges}, which connect between quadruplets of the same section. For each section, we pick one such edge. The third type are the {\em s-connecting-edges}, which connect between consecutive sections along . For each pair of consecutive sections along  (except for the pair consisting of the last and first sections), we pick one such edge. Notice that  is a -ST, since it is connected and it has  edges, i.e.,  inner-edges,  q-connecting-edges, and  s-connecting-edges.

We compute the approximation ratio by charging the edges of . Each edge of  either connects between points of the same section, or between points of consecutive sections. An edge of the former kind is charged at most nine times. Since for a section, we have six inner-edges, one q-connecting-edge, and two s-connecting-edges.
An edge of the latter kind is charged only once. 

As for , we can choose the subset of edges of  that connect between consecutive sections, so that its weight is at least . Let  denote this subset. Then, 
.
 
The following theorem summarizes the results of this section.
\begin{theorem}
Let  be a set of points in the plane. Then, one can construct (i) a -ST
of weight at most , (ii) a -ST of weight at most , and
(iii) a -ST of weight at most .
\end{theorem}

\paragraph{Remark.} As mentioned, the approximation ratios above are with respect to , which is a lower bound for . It is possible that by comparing the weight of the constructed -ST with that of an -MST, one can get better ratios, but it is not clear how to do so.
Moreover, it is easy to see that, for , 2 is a lower bound on the ratio with respect to a MST, e.g., consider  points on a line. And, for ,  is a lower bound on the ratio, e.g., consider 3 points at the corners of an equilateral triangle and a fourth point at the center of the circle passing through them. 
Finally, for , it is easy to give an example where . Therefore, any construction algorithm for an angle  in this range, should be analyzed with respect to .   
\old{
\paragraph{Remark.} As mentioned, the approximation ratios above are with respect to the weight of a MST, which is a lower bound for the weight of an -MST. It is possible that by comparing the weight of the constructed -ST with that of an -MST, one can get better ratios, but it is not clear how to do so.
Moreover, it is easy to see that, for , 2 is a lower bound on the ratio with respect to a MST, e.g., consider  points on a line. And, for ,  is a lower bound on the ratio, e.g., consider 3 points at the corners of an equilateral triangle and a fourth point at the center of the circle passing through them. 
}





\section{Constant range hop-spanner for } \label{sec:boundedrange}
In this section we apply Theorem~\ref{thm:no_cliques} to obtain a solution to a problem that arises in wireless communication networks.
Let  be a set of  points in the plane, where each point in  represents a transceiver equipped with an omni-directional antenna. The coverage region of 's antenna is modeled by a disk centered at , and assume that all disks are of radius 1. Then, the resulting communication graph is the {\em unit disk graph} of , denoted . (I.e, there is an edge between points  and  if the distance between them is at most 1.)
As mentioned in the introduction, directional antennas have some advantages over omni-directional antennas and are gaining popularity. The coverage region of a directional antenna of angle  is modeled by a circular sector of angle . 

Assume that  is connected.
Before stating our problem, we need the following definition. 
A graph  is a {\em -hop-spanner} of , for some constant , if for any two points , the minimum number of hops between  and  in  is at most  times this number in . That is, for each edge  in , there exists a path in  between  and  consisting of at most  edges. Assume now that one replaces each of the omni-directional antennas by a directional antenna of angle . We address the following {\em Antenna Conversion} problem:
Orient the directional antennas and fix a range , such that the resulting (symmetric) communication graph is a -hop-spanner of , for some constant . I.e., construct a -graph, such that the length of its edges is bounded by  and it is a -hop-spanner of~.

We show how to construct such a graph with  and , in  time.
We first partition the points of  into connected components (of ) of size at most three. This is done greedily. Set . As long as , perform the following step, which finds the next component . Pick any point , add it to  (which is initially empty), and remove it from . Now, if  and there exists a point in  whose distance from  is at most 1, then pick any such point , add it to , and remove it from . Finally, if  and there exists a point in  whose distance to either  or  (or both) is at most 1, then pick any such point , add it to , and remove it from . 

\begin{claim}
\label{clm:comp}
Let  be a connected component of size one or two. Then, each of the neighbors of  in  belongs to a component of size three.
\end{claim}
\begin{proof}
Assume that one of the neighbors of  belongs to a component  of size one or two, i.e., there exists an edge of  between a point in  and a point in . Moreover, assume, e.g., that  was found before . Then, in the iteration in which  was found, we would have found a larger component, i.e., with at least one additional point.
\end{proof}

Now, consider the connected components that were found. We first orient the wedges of each connected component of size exactly three, independently, according to the proof of Claim~\ref{lem:three_pts}. Next, for each connected component  of size one or two, let  be any connected component of size exactly three, such that  has a neighbor in . Recall that the wedges of  cover the plane. We orient each of the wedges of  (alternatively, the single wedge of ) towards the wedge of  that covers it. 
Observe that if the length of the edges is not limited, then the -graph, , that is induced by the wedges of  is connected. Moreover, it is easy to verify that  is a -hop-spanner, for . However, our goal is to limit the length of the edges without increasing  by much. 

Let  be a component of size one or two.
Then, the edge of  connecting between  and , where  is the component of size three to which  was connected, is of length at most . Moreover, consider any two components of size three  and , such that  has a neighbor both in  and in . Then, the edge of  connecting between  and  is of length at most 7. Finally, the edge of  connecting between two neighboring components of size three is of length at most 5. 
Therefore, one can drop all edges of length greater than 7 from , without disconnecting it. 

Next, we show that the resulting graph  is a -hop spanner.
Let  be an edge of . We show that the number of hops between  and  in  is at most . 
If  are in the same component of size three, then, clearly, the path between them consists of at most  edges. 
If  are in the same component of size two, then, the path between them passes through a single component of size three, and thus consists of at most  edges. 
If  are in different components, then, by Claim~\ref{clm:comp}, at least one of them, say , is in a component of size three. If  is also in a component of size three, then the path from  to  consists of at most  edges. Otherwise, the path between them goes from  to some component  of size three (which is not necessarily 's component) and from there to 's component, and thus consists of at most 6 edges. 


\old{
The following claim shows that the final graph  is a -hop-spanner, for . 
\begin{claim}\label{clm:hop-spanner}
 is a -hop spanner.
\end{claim}
\begin{proof}
Let  be an edge of . We show that the number of hops between  and  in  is at most . 
If  are in the same component of size three, then, clearly, the path between them consists of at most  edges. 
If  are in the same component of size two, then, the path between them passes through a single component of size three, and thus consists of at most  edges. 
If  are in different components, then, by Claim~\ref{clm:comp}, at least one of them, say , is in a component of size three. If  is also in a component of size three, then the path from  to  consists of at most  edges. Otherwise, the path between them goes from  to some component  of size three (which is not necessarily 's component) and from there to 's component, and thus consists of at most 6 edges.
\end{proof}
}

The following theorem summarizes the result of this section. \begin{theorem}
\label{thm:hop-spanner}
Let  be a set of points in the plane and assume that  is connected. Let . Then, one can construct, in  time, a 6-hop-spanner of , in which each edge is of length at most 7.
\end{theorem}


\paragraph{Running time.}
It is possible to implement the algorithm described above in  time, using a data structure presented by Efrat et al.~\cite{EIK01}. This data structure is designed for a given set  of  points in the plane. It supports queries of the following form: Given a query point , return a point  whose distance from  is at most 1, and also delete it from  (if requested to do so). The data structure can be constructed in  time, and a query (including deletion if needed) can be answered in amortized  time. We use it in both phases of the algorithm. For the first phase, in which  is partitioned into connected components of  of size at most three, we construct the data structure for the set . Now, finding a single component requires at most three queries (plus deletions), and, since there are  components, the total running time of this phase is . In the second phase, we orient the wedges of each component. Orienting the wedges of the components of size three can be done in  time per component. For the components of size one or two, we construct the data structure for the subset of  consisting of all points belonging to components of size three. Now, by Claim~\ref{clm:comp}, orienting the wedge of a component of size one (alternatively, the wedges of a component of size two) can be done in amortized  time. For a component of size one, we perform a single query (without deletion) in the data structure, and, for a component of size two, we perform one or two queries (without deletion), depending on whether the query with the first point is successful or not.  
We conclude that the overall running time of the algorithm for constructing  is .



\old{
\paragraph{Remark.} If our goal is solely to limit one of the measures (i.e., either range or hop distance), then better constants can be easily obtained. E.g., a -hop spanner or an -graph with maximum length .
}




\section{NP-hardness}\label{sec:np_hardness}

We prove that the problem of computing an -MST, for  and , is NP-hard.

\subsection{}

\begin{figure}[htb]
\centering
  \subfigure[]{
   \centering
       \includegraphics[width=0.2\textwidth,page=1]{fig/grid_graph_reduction}\label{fig:grid_red_a}
  }
  \hspace{.75cm}
  \subfigure[]{
   \centering
       \includegraphics[width=0.2\textwidth,page=2]{fig/grid_graph_reduction}\label{fig:grid_red_b}
  }
  \hspace{.75cm}
  \subfigure[]{
   \centering
       \includegraphics[width=0.2\textwidth,page=3]{fig/grid_graph_reduction}\label{fig:grid_red_c}
  }
 	\caption{(a) A square grid graph  consisting of  vertices and a 2-coloring of the graph. (b) For each point , we add a corresponding point  (denoted by a square). (c) A Hamiltonian path of . The semi-circles indicate the orientation of the wedges at the points of .}
 	\label{fig:reduction180}	
\end{figure} 

We describe a reduction from the problem of finding a Hamiltonian path in square grid graphs of degree at most 3. This problem was shown to be NP-hard by Itai et al.~\cite{IPS82}.
A {\em square grid graph} is a graph whose vertices correspond to points in the plane with integer coordinates, and there is an edge between two vertices if the distance between their corresponding points is 1. 
Let , , be a square grid graph of degree at most 3. Our reduction is very similar to the one in~\cite{PV84}. Since every square grid graph is bipartite, one can color  with two colors, say, black and white (see Figure~\ref{fig:grid_red_a}). Let  denote 's color in some 2-coloring of , and let  and , , be the number of black and white points (i.e., vertices), respectively. 
We add a set  of  points as follows. For each point , consider any edge  of the {\em complete} grid graph that is adjacent to  and is missing in . We place a point  on , such that , if  is black, and , if  is white (see Figure~\ref{fig:grid_red_b}). Notice that the distance from  to any point in  is greater than . 
Therefore, any MST of the point set  contains the  edges , , and  edges from . Its weight is .


\begin{lemma}
\label{lem:hardness1}
G has a Hamiltonian path if and only if  has a -MST of weight . 
\end{lemma}
\begin{proof}
Assume  has a Hamiltonian path, then its weight is , since it consists of  edges and the weight of each edge in  is 1.
By adding an edge between each point  and its corresponding point , we obtain a tree, , of weight . Notice that the degree of each point in  is at most 3. Moreover,  is a -ST of , since, at each vertex of , one can place a -wedge that covers all its neighbors (see Figure~\ref{fig:grid_red_c}). Finally, the weight of  is , and, therefore,  is a -MST of .

Assume now that  has a -MST, , of weight . Then,  must contain the  edges , , plus  edges from . Moreover, the maximum degree in  is at most 3 (since, a -wedge can cover at most 3 orthogonal directions). We conclude that  has a spanning tree of maximum degree at most 2 of weight . That is,  has a Hamiltonian path.
\end{proof}


\subsection{}

We describe a reduction from the problem of finding a Hamiltonian {\em path} in hexagonal grid graphs. 
Consider a tiling of the plane with regular hexagons of side length 1. The vertex set of an {\em hexagonal grid graph} is a subset of the vertices of the tiling, and there is an edge between two vertices of the graph if the distance between them is 1 (see Figure~\ref{fig:reduction120}). 
The problem of finding a Hamiltonian {\em cycle} in such graphs was shown to be NP-hard by Arkin et al.~\cite{AFIMMRPRX09}.
We first show that the path version is also NP-hard.

\begin{figure}[htb]
\centering
  \subfigure[]{
   \centering
       \includegraphics[width=0.3\textwidth,page=2]{fig/hex_grid_reduction3}\label{fig:reduction_a}	
  }
	\hspace{.75cm}
  \subfigure[]{
   \centering
       \includegraphics[width=0.3\textwidth,page=2]{fig/hex_grid_reduction2}\label{fig:reduction_b}	
  }
 	\caption{(a) If , then we add three points. (b) If , then we add two points.} \label{fig:reduction120}	
\end{figure} 
 
Let  be an hexagonal grid graph, and let  be the highest point in . (If there are several highest points, then pick the leftmost among them.). Notice that since  is the highest point, its degree cannot be 3, i.e., . 
Moreover, if , then one of the edges adjacent to  must be horizontal.
We construct another hexagonal grid graph, , by adding at most three points to , depending on 's degree in .
If , then .
If , then we add the points, , , and  to , as in Figure~\ref{fig:reduction_a}. The only edges that are formed due to this addition are , , and .
Finally, if , then  we add the points  and  to , as in Figure~\ref{fig:reduction_b}. The only edges that are formed due to this addition are  and , where  is the horizontal neighbor of .

\begin{lemma}
 contains a Hamiltonian cycle if and only if  contains a Hamiltonian path.
\end{lemma}
\begin{proof}
Assume that  contains a Hamiltonian cycle. Then, , and  is an edge of this cycle. By dropping the edge , we obtain a Hamiltonian path between  and  in . And, by adding the edges  and  to this path, we obtain a Hamiltonian path between  and  in . 

Assume now that  contains a Hamiltonian path. We claim that this is possible only if  was obtained by adding two points to . Indeed, if , then , and  cannot contain a Hamiltonian path. And, if  was obtained by adding three points to , then, since  and  and  have a common neighbor, namely, ,  cannot contain a Hamiltonian path. 
So, consider the graph  that is obtained by adding the points  and  to  (see Figure~\ref{fig:reduction_b}), and recall that we are assuming that  contains a Hamiltonian path. Since , the endpoints of this path are necessarily  and . By dropping the edges  and  from this path, we get a Hamiltonian path  in , between  and . Notice that the edge  is not in , and by adding it to , we get a Hamiltonian cycle in . 
\end{proof}

We are now ready to show that the problem of computing an -MST, for , is NP-hard.
Let  be an hexagonal grid graph, where . Since the distance between any two points in  is at least 1, the weight of a MST of  is at least .
\begin{lemma}
 has a Hamiltonian path if and only if  has a -MST of weight . 
\end{lemma}
\begin{proof}
Assume first that  has a Hamiltonian path. Then, its weight is . Moreover, the angle between any two consecutive edges along the path is . Therefore, the path is also a -ST, and, since its weight is , it is a -MST.

Assume now that  has a -MST, , of weight .
Then, all the edges of  are of length exactly 1, and therefore belong also to . It follows that the degree of any point in  is at most 2. Therefore,  is a Hamiltonian path of . 
\end{proof}


\begin{thebibliography}{1}

\bibitem{AGP13}
E. Ackerman, T. Gelander, and R. Pinchasi.
\newblock Ice-creams and wedge graphs.
\newblock {\em Comput. Geom.: Theory \& Applications}, 46(3):213--218, 2013. 

\bibitem{AHHHPSSV13}
O. Aichholzer, T. Hackl, M. Hoffmann, C. Huemer, A. P{\'o}r,
F. Santos, B. Speckmann, and B. Vogtenhuber.
\newblock Maximizing maximal angles for plane straight-line graphs.
\newblock {\em Comput. Geom.: Theory \& Applications}, 46(1):17--28, 2013. 

\bibitem{AFIMMRPRX09}
E. M. Arkin, S. P. Fekete, K. Islam, H. Meijer, J. S. B. Mitchell, Y. N. Rodr\'{\i}guez, V. Polishchuk,  D. Rappaport, and H. Xiao.
\newblock Not being (super) thin or solid is hard: A study of grid Hamiltonicity.
\newblock {\em Comput. Geom.: Theory \& Applications}, 42(6-7):582--605, 2009.

\bibitem{A98}
S. Arora.
\newblock Polynomial time approximation schemes for Euclidean traveling salesman and other geometric problems.
\newblock {\em J. ACM}, 45(5):753--782, 1998.
  
\bibitem{AKM13}
R. Aschner, M. J. Katz, and G. Morgenstern.
\newblock Symmetric connectivity with directional antennas.
\newblock {\em Comput. Geom.: Theory \& Applications}, 46(9):1017--1026, 2013.

\bibitem{BPV09}
I. B{\'a}r{\'a}ny, A. P{\'o}r, and P. Valtr.
\newblock Paths with no small angles.
\newblock {\em SIAM Journal Discrete Mathematics}, 23(4):1655--1666, 2009.

\bibitem{BCDFKM11}
P. Bose, P. Carmi, M. Damian, R. Flatland, M. J. Katz, and A. Maheshwari.
\newblock Switching to directional antennas with constant increase in radius and hop distance.
\newblock In {\em  Proc. 12th Algorithms and Data Structures Sympos.}, pages 134--146, 2011.

\bibitem{CKKKW08}
I. Caragiannis, C. Kaklamanis, E. Kranakis, D. Krizanc, and A. Wiese.
\newblock Communication in wireless networks with directional antennas.
\newblock In {\em 20th ACM Sympos. on Parallelism in Algorithms and Architectures}, pages 344--351, 2008.

  
\bibitem{CKLR11}
P. Carmi, M. J. Katz, Z. Lotker, and A. Ros\'{e}n.
\newblock Connectivity guarantees for wireless networks with directional antennas.
\newblock {\em Comput. Geom.: Theory \& Applications}, 44(9):477--485, 2011.

\bibitem{C04}
T. M. Chan.
\newblock Euclidean bounded-degree spanning tree ratios.
\newblock {\em Discrete {\&} Computational Geometry}, 32(2):177--194, 2004. 

\bibitem{DPT12}
A. Dumitrescu, J. Pach, and G. T{\'o}th.
Drawing Hamiltonian cycles with no large angles.
\newblock {\em Electronic Journal of Combinatorics}, 19(2):P31, 2012.

\bibitem{EIK01}
A. Efrat, A. Itai, and M. J. Katz.
\newblock Geometry helps in bottleneck matching and related problems.
\newblock {\em Algorithmica}, 31(1):1--28, 2001.

\bibitem{FW97}
S. P. Fekete and G. J. Woeginger.
\newblock Angle-restricted tours in the plane.
\newblock {\em Comput. Geom.: Theory \& Applications}, 8:195--218, 1997.




\bibitem{IPS82}
A. Itai, C. H. Papadimitriou, and J. L. Szwarcfiter.
\newblock Hamilton paths in grid graphs.
\newblock {\em SIAM Journal on Computing}, 11(4):676--686, 1982.

\bibitem{JR09}
R. Jothi and B. Raghavachari.
\newblock Degree-bounded minimum spanning trees.
\newblock {\em Discrete Applied Mathematics}, 157(5):960--970, 2009.


\bibitem{KRY96}
S. Khuller, B. Raghavachari, and N. E. Young.
\newblock Low-degree spanning trees of small weight.
\newblock {\em SIAM Journal on Computing}, 25(2):355--368, 1996.

\bibitem{KKM}
E. Kranakis, D. Krizanc, and O. Morales.
\newblock Maintaining connectivity in sensor networks using directional antennae.
In Theoretical Aspects of Distributed Computing in Sensor Networks, Chapter 3, S. Nikoletseas and J. D. P. Rolim (Eds.), Springer.
   
\bibitem{M99}
J. S. B. Mitchell.
\newblock Guillotine subdivisions approximate polygonal subdivisions: a simple polynomial-time approximation scheme for geometric TSP, k-MST, and related problems.
\newblock {\em SIAM Journal on Computing}, 28(4):1298--1309, 1999.

\bibitem{PV84}
C. H. Papadimitriou and U. V. Vazirani.
\newblock On two geometric problems related to the travelling salesman problem.
\newblock {\em Journal of Algorithms}, 5(2):231--246, 1984.

\end{thebibliography}
\old{
\newpage

\appendix

\section{Proofs of Lemma~\ref{lem:two_cliques} and Lemma~\ref{lem:one_clique}}

\setcounter{lemma}{0}
\begin{lemma}[Two cliques]
Let  and  be two triplets of points and let .
Assume that the wedges (associated with the points) of  and, independently, of  are oriented according to the proof of Claim~\ref{lem:three_pts}, and that both induced graphs,  and , are cliques.
Then, the induced graph  is connected.
\end{lemma}


\begin{proof}
The wedges of  cover the plane, in particular they cover all points of . Therefore, we distinguish between three (not necessarily disjoint) cases: 
(i) there exists a point  such that  covers all points of , 
(ii) there exists a point  such that  covers exactly two points of , and 
(iii) the wedge of each point in  covers exactly one point of .

{\bf Case (i):} There exists a point  such that  covers all points of . Since
the wedges of  cover the plane, at least one of them must cover , and therefore
.

{\bf Case (ii):} There exists a point  such that  covers exactly two points of . We divide this
case into three sub-cases, according to which two points of  are covered by .

\begin{figure}[htb]
 \centering
 \subfigure[]{
   \centering
       \includegraphics[width=0.45\textwidth]{fig/proof_case_2_b_2}
 }
 \subfigure[]{
  \centering
       \includegraphics[width=0.45\textwidth]{fig/proof_case_2_b_1}
 }
 \caption{Proof of Lemma~\ref{lem:two_cliques}, Case (ii)(1).}	\label{fig:case2b}	
\end{figure} 

(1)  covers  and  and does not cover . Assume  (since otherwise
we are done), then  and one of the rays of  intersects  and
. Notice that this ray must be  and that  also intersects  (see~Figure~\ref{fig:case2b}).
Since  lies below ,  intersects , and , we have that . 
It follows that , , and .
Therefore,  (whose orientation is ) does not intersect .
Let  be the point of  such that  and 
. Since , we have that
 and  lies to the right of . Notice that  contains the (imaginary) wedge of orientation  and apex . If  (see Figure~\ref{fig:case2b}(a)), then , since  covers . 
Otherwise,  and in particular  lies to the right of  (see Figure~\ref{fig:case2b}(b)). In this case we show that . Indeed,  intersects  to the right of , since , and, since  is parallel to  and below it, we have that  intersects  to the left of . We conclude that  and .

\begin{figure}[htb]
 \centering
       \includegraphics[width=0.35\textwidth]{fig/proof_case_2_c}
 \caption{Proof of Lemma~\ref{lem:two_cliques}, Case (ii)(2).}	\label{fig:case2c}	
\end{figure} 
  
(2)  covers  and  and does not cover . Assume  (since otherwise
we are done), then 
and one of the rays of  intersects  and . Notice that this ray must be  and that  also intersects  (see Figure~\ref{fig:case2c}).
Since  lies above ,  intersects , and , we have that . 
It follows that , , and . The rest of the proof for this case is very similar to the proof of Case~(ii)(1), thus we omit further details.

\begin{figure}[htb]
 \centering 
  \subfigure[ intersects  and ]{
    \centering
        \includegraphics[width=0.45\textwidth]{fig/proof_case_2_d}
	\label{fig:case2d}
 }
  \subfigure[ intersects  and ]{
    \centering
        \includegraphics[width=0.45\textwidth]{fig/proof_case_2_e}
	\label{fig:case2e}
 }
 \caption{Proof of Lemma~\ref{lem:two_cliques}, Case (ii)(3).}	\label{fig:caseb}
\end{figure}

 
(3)  covers  and  and does not cover . Assume  (since otherwise
we are done), then ,
and one of the rays of  intersects  and . Notice that this ray 
can be either  or . 

If it is  (see Figure~\ref{fig:case2d}), then
the orientations associated with  are:
, , and .
The rest of the proof for this branch is very similar to the proof of Case (ii)(1), thus we omit further details.

If the ray intersecting  and  is  (see Figure~\ref{fig:case2e}), then
the orientations associated with  are:
, , and .
Again, the rest of the proof for this branch is very similar to the proof of Case (ii)(1), thus we omit further details.

\begin{figure}[htb]
 \centering 
    \includegraphics[width=0.5\textwidth]{fig/proof_case_3_a}
 \label{fig:case3a}
 \caption{Proof of Lemma~\ref{lem:two_cliques}, Case (iii).}	\label{fig:case3}
\end{figure}

{\bf Case (iii):} The wedge of each point in  covers exactly one point of . We may assume that this condition also holds for the wedges of ; that is, the wedge of each point in  covers exactly one point of . Since, otherwise, we can simply interchange the set names. 
It follows that each point of  lies in its own private region among the regions , , and . 

Let  be the point that lies in . We claim that .
Assume that . We show that there exists a point  that covers two points of .
If  covers  (see Figure~\ref{fig:case3}), then , which implies that . Let  be the point of  such that  and . Since , we have that  and  lies to the left of , but then  must cover  and  -- contradiction.
If  covers , then , which implies that . Let  be the point of  such that  and . Since , we have that  and  lies to the right of , but then  must cover  and  -- contradiction.

\end{proof}

\begin{lemma}[One clique]
Let  and  be two triplets of points and let .
Assume that the wedges of  and, independently, of  are oriented according to the proof of 
Claim~\ref{lem:three_pts}, and that the induced graph  is a clique.
Then, the induced graph  is connected.
\end{lemma}

\begin{proof}
If the induced graph  is also a clique, then, by Lemma~\ref{lem:two_cliques}, we are done.
Assume therefore that  is not a clique. 
Let  be the intersection point of  and  (see Figure~\ref{fig:oneclique}), and consider the wedge  of orientation  and apex .
The graph induced by  is a clique, and therefore, by Lemma~\ref{lem:two_cliques}, .
If , then we are done, so assume that .
Let  be a point of  such that , and assume that  does not cover  (if it does, then , since ). Then,  lies above  and  intersects .
Below we consider the three cases: (i)  intersects , (ii)  intersects  to the left of , and (iii)  does not intersect . However, in the first case (i.e., Case~(i)) and in sub-cases (1) and (2) of the second case (i.e., Case~(ii)(1) and Case~(ii)(2)) we refrain from using the assumption that  is a clique. This is because these cases appear again later in the proof of Theorem\mbox{~\ref{thm:no_cliques}}, where we may not assume that  is a clique.  


\begin{figure}[htb]
 \centering 
 \subfigure[Case (i)]{
    \centering
        \includegraphics[width=0.30\textwidth]{fig/one_clique_1a}
	     \label{fig:oneclique_case_1_a}}
 \subfigure[Case (ii)]{
    \centering
        \includegraphics[width=0.30\textwidth]{fig/one_clique_2a}
	     \label{fig:oneclique_case_1_b}}
 \subfigure[Case (iii)]{
    \centering
        \includegraphics[width=0.30\textwidth]{fig/one_clique_3a}
	     \label{fig:oneclique_case_1_c}}
	\caption{Proof of Lemma~\ref{lem:one_clique}.}	\label{fig:oneclique}
\end{figure}



{\bf Case (i):}  intersects  (see Figure~\ref{fig:oneclique_case_1_a}). Notice that in this case  does not cover points  and . Since  and , we get that , , and
.
Between the two points in , let  be the one whose wedge covers more points of ; in case of tie, let  be any one of them. We know that one of 's rays has orientation in .
There are five sub-cases:

(1)  covers all points of . There must exist a point in  that covers , so we are done. 

(2)  covers  and  and does not cover . If , then we are done. Otherwise, . Now, since  and  must cover  and  and avoid , we get that . But, this is impossible, since  is not among the three relevant ranges mentioned above. 

(3)  covers  and  and does not cover . If , then we are done. Otherwise, . We show that this is impossible.
If , then , and . And, if , then , and  must also cover .

(4)  covers  and  and does not cover . This case is analogous to the previous one.
\old{FULL VERSION
If , then we are done. Otherwise, .
We show that this is impossible.
If  , then , and . And, if , then , and  must also cover .
}

(5)  covers exactly one point of . Therefore, the wedge of each point in  covers exactly one point of . Since  does not cover points  and , it must cover .
Assume, w.l.o.g., that  covers  and , the wedge of the remaining point, covers . Next, we show that this is impossible.
Indeed, if  and , then  must also cover  and . 
And, if  and , then both  and  must lie below . (Since, if  is above , then , and, if  is above , then ). Therefore,  covers the halfplane above  (see Property~3), and, in particular, at least one of the two wedges covers .



{\bf Case (ii):}  intersects  to the left of  (see Figure~\ref{fig:oneclique_case_1_b}). In this case, as in Case (i), , , and
. Notice that in this case , so we assume that
, since otherwise . Let  be a point of  whose wedge covers
. We distinguish between three sub-cases:

(1)  covers all points of . There must exist a point in  that covers , so we are done.

(2)  covers exactly two points of . If  covers  and  and , then  and either  or .
However, in both cases,  must also cover  -- contradiction. (Since, in the former case,  does not intersect , and in the latter case,  does not intersect .) If  covers  and  and , then  and . However, in the case,  must also cover  -- contradiction. (Since  passes above  and is directed upwards, and  passes below  and is directed downward.)

(3)  covers exactly one point of , namely, . We know that either  or . In the latter case,  must also cover , which is impossible. In the former case, if  is above , then , so  is necessarily below .
Let  be the remaining point. Then, . We show below that . 
Notice first that  separates between  and  and between  and , since  and  covers only . Since  is a clique, we know that , and therefore  lies to the right of . Clearly,  and  lie to the left of  (whose orientation is in ), and to the right of  (whose orientation is in ). In other words,  covers both  and . Notice also that , since  (whose orientation is in ) intersects  to the right of , and  lies to the right of . Therefore, either  or  (or both) covers . We conclude that .  
 

{\bf Case (iii):}  does not intersect , i.e.,  (see Figure~\ref{fig:oneclique_case_1_c}). Since  covers , we may assume that . Therefore, . We thus have that  and . Notice that  (whose orientation is in )
intersects  to the right of . Moreover,  necessarily covers , since  and  intersects  between  and . Let  be the point of  such that .
Since  is a clique, we know that , and therefore  lies to the right of . If  is above , then . Otherwise,  is below  and in  (since it is to the left of ). But then , since  passes above  and  is directed downwards.
\end{proof}


\vspace{-7mm}
\section{Proof of Theorem~\ref{thm:hop-spanner} --- Run-time analysis}

It is possible to implement the algorithm described in Section~\ref{sec:boundedrange} in  time, using a data structure presented by Efrat et al.~\cite{EIK01}. This data structure is designed for a given set  of  points in the plane. It supports queries of the following form: Given a query point , return a point  whose distance from  is at most 1, and also delete it from  (if requested to do so). The data structure can be constructed in  time, and a query (including deletion if needed) can be answered in amortized  time. We use it in both phases of the algorithm.

For the first phase, in which  is partitioned into connected components of  of size at most three, we construct the data structure for the set . Now, finding a single component requires at most three queries (plus deletions), and, since there are  components, the total running time of this phase is .

In the second phase, we orient the wedges of each component. Orienting the wedges of the components of size three can be done in  time per component. For the components of size one or two, we construct the data structure for the subset of  consisting of all points belonging to components of size three. Now, by Claim~\ref{clm:comp}, orienting the wedge of a component of size one (alternatively, the wedges of a component of size two) can be done in amortized  time. For a component of size one, we perform a single query (without deletion) in the data structure, and, for a component of size two, we perform one or two queries (without deletion), depending on whether the query with the first point is successful or not.  
We conclude that the overall running time of the algorithm for constructing  is .





\vspace{-3mm}
\section{NP-hardness}

We prove that the problem of computing an -MST, for  and , is NP-hard.

\vspace{-3mm}
\subsection{}

\begin{figure}[htb]
\centering
  \subfigure[]{
   \centering
       \includegraphics[width=0.2\textwidth,page=1]{fig/grid_graph_reduction}\label{fig:grid_red_a}
  }
  \hspace{.75cm}
  \subfigure[]{
   \centering
       \includegraphics[width=0.2\textwidth,page=2]{fig/grid_graph_reduction}\label{fig:grid_red_b}
  }
  \hspace{.75cm}
  \subfigure[]{
   \centering
       \includegraphics[width=0.2\textwidth,page=3]{fig/grid_graph_reduction}\label{fig:grid_red_c}
  }
 	\caption{(a) A square grid graph  consisting of  vertices and a 2-coloring of the graph. (b) For each point , we add a corresponding point  (denoted by a square). (c) A Hamiltonian path of . The semi-circles indicate the orientation of the wedges at the points of .}
 	\label{fig:reduction180}	
\end{figure} 

We describe a reduction from the problem of finding a Hamiltonian path in square grid graphs of degree at most 3. This problem was shown to be NP-hard by Itai et al.~\cite{IPS82}.
A {\em square grid graph} is a graph whose vertices correspond to points in the plane with integer coordinates, and there is an edge between two vertices if the distance between their corresponding points is 1. 
Let , , be a square grid graph of degree at most 3. Our reduction is very similar to the one in~\cite{PV84}. Since every square grid graph is bipartite, one can color  with two colors, say, black and white (see Figure~\ref{fig:grid_red_a}). Let  denote 's color in some 2-coloring of , and let  and , , be the number of black and white points (i.e., vertices), respectively. 
We add a set  of  points as follows. For each point , consider any edge  of the {\em complete} grid graph that is adjacent to  and is missing in . We place a point  on , such that , if  is black, and , if  is white (see Figure~\ref{fig:grid_red_b}). Notice that the distance from  to any point in  is greater than . 
Therefore, any MST of the point set  contains the  edges , , and  edges from . Its weight is .


\begin{lemma}
\label{lem:hardness1}
G has a Hamiltonian path if and only if  has a -MST of weight . 
\end{lemma}
\begin{proof}
Assume  has a Hamiltonian path, then its weight is , since it consists of  edges and the weight of each edge in  is 1.
By adding an edge between each point  and its corresponding point , we obtain a tree, , of weight . Notice that the degree of each point in  is at most 3. Moreover,  is a -ST of , since, at each vertex of , one can place a -wedge that covers all its neighbors (see Figure~\ref{fig:grid_red_c}). Finally, the weight of  is , and, therefore,  is a -MST of .

Assume now that  has a -MST, , of weight . Then,  must contain the  edges , , plus  edges from . Moreover, the maximum degree in  is at most 3 (since, a -wedge can cover at most 3 orthogonal directions). We conclude that  has a spanning tree of maximum degree at most 2 of weight . That is,  has a Hamiltonian path.
\end{proof}


\vspace{-3mm}
\subsection{}

We describe a reduction from the problem of finding a Hamiltonian {\em path} in hexagonal grid graphs. 
Consider a tiling of the plane with regular hexagons of side length 1. The vertex set of an {\em hexagonal grid graph} is a subset of the vertices of the tiling, and there is an edge between two vertices of the graph if the distance between them is 1 (see Figure~\ref{fig:reduction120}). 
The problem of finding a Hamiltonian {\em cycle} in such graphs was shown to be NP-hard by Arkin et al.~\cite{AFIMMRPRX09}.
We first show that the path version is also NP-hard.

\begin{figure}[htb]
\centering
  \subfigure[]{
   \centering
       \includegraphics[width=0.3\textwidth,page=2]{fig/hex_grid_reduction3}\label{fig:reduction_a}	
  }
	\hspace{.75cm}
  \subfigure[]{
   \centering
       \includegraphics[width=0.3\textwidth,page=2]{fig/hex_grid_reduction2}\label{fig:reduction_b}	
  }
 	\caption{(a) If , then we add three points. (b) If , then we add two points.} \label{fig:reduction120}	
\end{figure} 
 
Let  be an hexagonal grid graph, and let  be the highest point in . (If there are several highest points, then pick the leftmost among them.). Notice that since  is the highest point, its degree cannot be 3, i.e., . 
Moreover, if , then one of the edges adjacent to  must be horizontal.
We construct another hexagonal grid graph, , by adding at most three points to , depending on 's degree in .
If , then .
If , then we add the points, , , and  to , as in Figure~\ref{fig:reduction_a}. The only edges that are formed due to this addition are , , and .
Finally, if , then  we add the points  and  to , as in Figure~\ref{fig:reduction_b}. The only edges that are formed due to this addition are  and , where  is the horizontal neighbor of .

\begin{lemma}
 contains a Hamiltonian cycle if and only if  contains a Hamiltonian path.
\end{lemma}
\begin{proof}
Assume that  contains a Hamiltonian cycle. Then, , and  is an edge of this cycle. By dropping the edge , we obtain a Hamiltonian path between  and  in . And, by adding the edges  and  to this path, we obtain a Hamiltonian path between  and  in . 

Assume now that  contains a Hamiltonian path. We claim that this is possible only if  was obtained by adding two points to . Indeed, if , then , and  cannot contain a Hamiltonian path. And, if  was obtained by adding three points to , then, since  and  and  have a common neighbor, namely, ,  cannot contain a Hamiltonian path. 
So, consider the graph  that is obtained by adding the points  and  to  (see Figure~\ref{fig:reduction_b}), and recall that we are assuming that  contains a Hamiltonian path. Since , the endpoints of this path are necessarily  and . By dropping the edges  and  from this path, we get a Hamiltonian path  in , between  and . Notice that the edge  is not in , and by adding it to , we get a Hamiltonian cycle in . 
\end{proof}

We are now ready to show that the problem of computing an -MST, for , is NP-hard.
Let  be an hexagonal grid graph, where . Since the distance between any two points in  is at least 1, the weight of a MST of  is at least .
\begin{lemma}
 has a Hamiltonian path if and only if  has a -MST of weight . 
\end{lemma}
\begin{proof}
Assume first that  has a Hamiltonian path. Then, its weight is . Moreover, the angle between any two consecutive edges along the path is . Therefore, the path is also a -ST, and, since its weight is , it is a -MST.

Assume now that  has a -MST, , of weight .
Then, all the edges of  are of length exactly 1, and therefore belong also to . It follows that the degree of any point in  is at most 2. Therefore,  is a Hamiltonian path of . 
\end{proof}
}

\end{document}
