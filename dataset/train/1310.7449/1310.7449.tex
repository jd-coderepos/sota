

This section develops the technical details for proving Theorem~\ref{thm.invariance}.

\begin{definition}
A history $\alpha$ is 
\begin{description}
\item[$\M$-\emph{yielding}] 
\ \\
if
$\alpha = \alpha_1^{h_1}\beta_1 \cdots 
\alpha_n^{h_n} \beta_n$ such that, for every $i$, 
$\alpha_i$ is a recursive history, $\beta_i \preceq \alpha_i$, 
and $\alpha = \alpha' \M_i$ implies the program has the definition
$\M_i(\wt{x}_i) = \CP[\M(\wt{u}) ]$, for some $\wt{u}$.
The \emph{kernel} of $\alpha$, denoted $[\alpha]$, is 
$\alpha_1^{h_1'}\beta_1 \cdots \alpha_{n}^{h_n'}\beta_{n}$,
where $h_i' = {\tt min}(h_i, 1)$.
\end{description}
\end{definition}


By definition, if $\alpha$ is $\M$-saturating then it is also 
$\M$-yielding. In this case, the kernel $[\alpha]$ has a suffix that
 is $\M$-complete. 
In the $\M\Q\PP$-program,  $\ordermut{\M}=4$,
$\ordermut{\Q}=3$, and $\ordermut{\PP}=1$, and
the recursive histories of $\M$, $\Q$ and $\PP$ are equal to $\M\Q$, to 
$\Q\M$ and to $\PP$, respectively.
Then $\alpha= (\M\Q)^{4}$ is the $\M$-complete history and 
$\alpha'= \PP^2\M$ is $\Q$-yielding, with $[\alpha']=\PP\M$.

We notice that every history of an informative
lam (obtained by evaluating $\ilam{\por{I}_{\P},\,\varnothing ,\, 
\addh{\varepsilon}{\P}}{}$) is a yielding sequence.
We also notice that, for every $\M$, $\varepsilon$ is $\M$-yielding.
In fact, $\varepsilon$ is the history of every function invocation in the
initial lam, which may concern every function name of the program.
As regards the kernel, 
in Lemma~\ref{lem.flashback}, we 
demonstrate that, 
if $\alpha = \alpha_1^{h_1}\beta_1 \cdots \alpha_n^{h_n} \beta_n$ is a $\M$-yielding history such that every $h_i \geq 2$,
then every term 
$\suhs{\alpha}{\M(\wt{u})}$ may be mapped by a flashback $\rho$ to a term
$\suhs{[\alpha]}{\M(\rho(\wt{u}))}$; similarly for dependencies. This is the
basic property that allows us to map circularities to past circularities
 (see Theorem~\ref{thm.invariance}). 

Next we introduce an ordering relation over renamings, (in particular, flashbacks) 
and the operation of renaming composition. The definitions are almost standard:
\begin{itemize}
\item 
$\rho \lessfb \rho'$ if, for every $x \in \dom{\rho}$, $\rho(x) = \rho'(x)$.
\item
$\rho  {\scriptstyle \, \circ \,} \rho'$ be defined as follows: 

\smallskip

$\qquad(\rho {\scriptstyle \, \circ \,} \rho')(x) \eqdef 
\left\{ \begin{array}{l@{\qquad}l}
		\rho'(x) & {\it if} \;  \rho'(x) \notin \dom{\rho}
		\\ 
		\rho(\rho'(x)) & {\it otherwise} 
		\end{array} \right.$
\end{itemize}
We notice that, if both
\begin{enumerate}
\item
 $\rho$ and $\rho'$ are flashbacks and 
\item
for every $x \in \dom{\rho}$,  
$\rho'(x) = x$
\end{enumerate}
then $\rho \lessfb \rho {\scriptstyle \, \circ \,} \rho'$ holds.
In the following, lams $\flatt{\P}$ and $\flatt{\mathbb{L}}$, being $\seq$ of
terms that are dependencies composed with $\sparop$, will be written 
$\T_1 \seq \cdots \seq \T_m$ and
$\suh{\T_1} \seq \cdots \seq \suh{\T_m}$, for some $m$, 
respectively, where $\T_i$ and $\suh{\T_i}$ contain dependencies $(x,y)$ and 
$\suhs{\alpha}{(x,y)}$. Let also $\rho(\prod_{i \in I} (x_i,y_i)) = 
\prod_{i \in I}(\rho(x_i), \rho(y_i))$.

With an abuse of notation, we will use the set operation ``$\in$''
for $\P$ and $\suh{\P}$. For instance, we will write $\P' \in \P$  when there is
$\CP[~]$ such that $\P = 
\CP[\P']$. Similarly, we will write $\T \in \T_1 \seq \cdots \seq \T_n$ when there
is $\T_i$ such that $\T \in \T_i$.

A consequence of the
axiom $\T \sparop (\P' \seq \P'')  = \T\sparop \P' \seq \T \sparop \P''$ is 
the following property of the informative operational semantics.

\begin{proposition}
\label{prop.storie-insieme}
Let $\ilam{\por{V}_1,\suh{\Mplus}, \suh{\CP}_0[\suhs{\alpha}{\M_1(\wt{u_1})]}}{}$
be a state of an informative operational semantics. For every $1 \leq i 
\leq n$, let $\M_i(\wt{u_i}) = \P_i'$ and $\addh{\alpha\M_0 \cdots \M_i}{\P_i'}$ be 
$\suh{\CP}_i[\suhs{\alpha\M_1 \cdots \M_i}{\M_{i+1}(\wt{u_{i+1}})}]$. 
Finally, let

\smallskip

$\begin{array}{rl}
\flatt{\suh{\CP}_1[\cdots \suh{\CP}_n[\suhs{\alpha\M_1 \cdots \M_n}{\M_{n+1}(\wt{u_{n+1}})}] \cdots]} = &
\suh{\T}_1 \seq \cdots \seq \suh{\T}_r
\\
\flatt{\suh{\CP}_n[\suhs{\alpha\M_1 \cdots \M_n}{\M_{n+1}(\wt{u_{n+1}})}]} = 
& \suh{\T}_1' \seq \cdots \seq \suh{\T}_{r'}'\; .
\end{array}$

\smallskip

If  \ $\suhs{\alpha\M_1 \cdots \M_n}{(x,y)} \sparop 
\addh{\alpha'}{\T} \in \suh{\T}_1 \seq \cdots \seq \suh{\T}_r$ 
then, for every $1 \leq j \leq r'$, 
$\suh{\T}_{j}' \sparop \addh{\alpha'}{\T} \in \suh{\T}_1 \seq \cdots \seq \suh{\T}_r$.
\end{proposition}

The next lemma allows us to map, through a flashback, terms in a saturated 
state to terms that have been produced in the past. The correspondence is defined
by means of the (regular) structure of histories. 

\begin{lemma}
\label{lem.flashback}
Let $\ilam{\por{I}_{\P},\,\varnothing ,\, \addh{\varepsilon}{\P}
}{} \lred{}^* \ilam{\mathbb{V},\,\suh{\Mplus}, \, \mathbb{L}}{}$ 
and $\ilam{\mathbb{V}, \,\suh{\Mplus} ,\, \mathbb{L}}{}$ 
be saturated and 
$\flatt{\mathbb{L}}=\suh{\T}_1\seq\cdots\seq\suh{\T}_m$. Then
\begin{enumerate}
\item
if $\suhs{\beta\alpha^{n+2}\beta'}{\M(\wt{u})}\in\mathbb{L}$, where 
$\beta\alpha^{n+2}\beta'$ is $\M$-yielding, then there
are $n+1$ $\mathbb{V}$-flashbacks $\rho_{\beta,\alpha,\beta'}^{(2)}, \cdots  
, \rho_{\beta,\alpha,\beta'}^{(n+2)}$ such that:
\begin{enumerate}
\item
$\suhs{\beta \alpha^{n+1}\beta'}{\M(\rho_{\beta,\alpha,\beta'}^{(n+2)}(\wt{u}))} \in 
\suh{\Mplus}$;
\item
$\prod_{j\in J}\addh{\beta \alpha^{k+1}\beta_j}{\T_j'} \in \suh{\T}_1\seq\cdots\seq
\suh{\T}_m$ where, 
for every $j$, $\beta_j \preceq \alpha$, implies 
$\prod_{j\in J}\addh{\beta \alpha^{k}\beta_j}{\rho_{\beta,\alpha,\beta'}^{(k+1)}(\T_j')} 
\in \suh{\T}_1\seq\cdots\seq\suh{\T}_m$;
\item
$\suhs{\beta \alpha^{k+1}\beta'}{\M(\wt{u})} \in \suh{\Mplus}$ 
 implies 
$\suhs{\beta \alpha^{k}\beta'}{\M(\rho_{\beta,\alpha,\beta'}^{(k+1)}(\wt{u}))} \in 
\suh{\Mplus}$.
\end{enumerate}

\item 
if $\alpha_1, \cdots, \alpha_k$ are  $\M_1$-yielding, $\cdots$, $\M_k$-yielding, respectively,
  then there are flashbacks $\rho_{\alpha_1}, \cdots, \rho_{\alpha_k}$ 
such that
\begin{enumerate}
\item
if $\suhs{\alpha_1}{\M_1(\wt{u})} \in \mathbb{L}$ or 
$\suhs{\alpha_1}{\M_1(\wt{u})} \in  \suh{\Mplus}$ 
then $\suhs{[\alpha_1]}{\M(\rho_{\alpha_1}(\wt{u}))} \in \suh{\Mplus}$;
\item
if $\prod_{1 \leq j \leq k}\addh{\alpha_j}{\T_j} \in \suh{\T}_1\seq\cdots\seq\suh{\T}_m$ 
then \\
$\prod_{1 \leq j \leq k}\addh{[\alpha_j]}{\rho_{\alpha_j}(\T)} \in \suh{\T}_1\seq\cdots\seq\suh{\T}_m$;

\item
if $\alpha_1 \preceq \alpha_2$ then $\rho_{\alpha_1} \lessfb \rho_{\alpha_2}$.

(In particular, if $\alpha_1 = \beta \alpha^{n+2}\beta'$, with $\beta' \preceq 
\alpha$, and $\alpha_2 = \beta \alpha^{n+3}$ then
$\rho_{\alpha_1} \lessfb \rho_{\alpha_2}$).
\end{enumerate}
\end{enumerate}
\end{lemma}

\begin{proof} (Sketch)
As regards item 1, let  $\alpha = \beta'\beta''$ and let $\beta'' \beta' = \M \M_1 \cdots \M_m$ (therefore the length of $\alpha$ is $m+1$).
The evaluation $\ilam{\por{I}_{\P},\,\varnothing ,\, \addh{\varepsilon}{\P}
}{} \lred{}^* \ilam{\mathbb{V},\,\suh{\Mplus}, \, \mathbb{L}}{}$ may be decomposed as follows
\[
\begin{array}{rl}
\ilam{\por{I}_{\P},\,\varnothing ,\, \addh{\varepsilon}{\P}
}{} \lred{}^* &
\ilam{\mathbb{V}', \, \suh{\Mplus}', \, 
\suh{\CP}[\suhs{\beta\alpha^{n+1}\beta'}{\M(\wt{u'})}]}{}
\\
\qquad \lred{}^{*} &
\ilam{\mathbb{V}, \, \suh{\Mplus}, \, \mathbb{L}}{}
\end{array}
\] 
By definition of the operational semantics there is the \emph{alternative} evaluation
\[
\begin{array}{@{\!\!\!}l}
\ilam{\mathbb{V}', \, \suh{\Mplus}', \, 
\suh{\CP}[\suhs{\beta\alpha^{n+1}\beta'}{\M(\wt{u'})}]}{}
\\
\lred{} \ilam{\mathbb{V}'', \, \suh{\Mplus}'', \, 
\suh{\CP}[\suh{\CP}'[\suhs{\beta\alpha^{n+1}\beta'\M}{\M_1(\wt{u_{1}})}] ]}{}
 \\
\! \! \lred{}^{*}
\ilam{\mathbb{V}''', \, \suh{\Mplus}''', \, 
\suh{\CP}[ \suh{\CP}'[\suh{\CP}_1[ \cdots \suh{\CP}_m[
\suhs{\beta\alpha^{n+1}\beta'\M\M_1 \cdots \M_m}{\M(\wt{u''})}] \cdots]]]}{}
\end{array}
\] 
[notice that $\beta\alpha^{n+1}\beta'\M\M_1 \cdots \M_m = \beta\alpha^{n+2}\beta'$].
Property (1.a) is an immediate consequence of Proposition~\ref{prop.flashback-order};
let $\varrho_{\beta,\alpha,\beta'}^{(n+2)}$ be the flashback 
for the last state. The property (1.b), when $k=n$, is also an immediate consequence of 
Propositions~\ref{prop.flashback-order}
and of~\ref{prop.storie-insieme}. In the general case, we need to iterate 
the arguments on shorter histories and the arguments are similar for (1.c).
In order to conclude the proof of item 1, we need an additional argument. By Proposition~\ref{prop.diamond}, there exists an evaluation
\[
\begin{array}{l}
\ilam{\mathbb{V}''', \, \suh{\Mplus}''', \, 
\suh{\CP}[\suh{\CP}'[\suh{\CP}_1[ \cdots \suh{\CP}_m[
\suhs{\beta\alpha^{n+1}\beta'\M\M_1 \cdots \M_m}{\M(\wt{u''})}] \cdots]]]}{}
\\
\qquad \lred{}^{*} \ilam{\mathbb{V}^\sharp, \, \suh{\Mplus}^\sharp, \, 
\mathbb{L}^\sharp}{}
\end{array}
\]
such that $\ilam{\mathbb{V}^\sharp, \, \suh{\Mplus}^\sharp, \, 
\mathbb{L}^\sharp}{}$ and $\ilam{\mathbb{V}, \, \suh{\Mplus}, \, 
\mathbb{L}}{}$ are identified by a bijective renaming, let it be $\jmath$. We define
the $\rho_{\beta,\alpha,\beta'}^{(n+2)}$
corresponding to the evaluation
$\ilam{\por{I}_{\P},\,\varnothing ,\, \addh{\varepsilon}{\P}
}{} \lred{}^* \ilam{\mathbb{V},\,\suh{\Mplus}, \, \mathbb{L}}{}$
as $\rho_{\beta,\alpha,\beta'}^{(n+2)} \eqdef \jmath \circ 
\varrho_{\beta,\alpha,\beta'}^{(n+2)} \circ \jmath^{-1}$. Similarly for the other
$\rho_{\beta,\alpha,\beta'}^{(k+1)}$. The properties of item 1 for 
$\ilam{\mathbb{V}, \, \suh{\Mplus}, \, 
\mathbb{L}}{}$ follow
by the corresponding ones for
\[
\ilam{\mathbb{V}''', \, \suh{\Mplus}''', \, 
\suh{\CP}[\suh{\CP}'[\suh{\CP}_1[ \cdots \suh{\CP}_m[
\suhs{\beta\alpha^{n+1}\beta'\M\M_1 \cdots \M_m}{\M(\wt{u''})}] \cdots]]]}{} \; .
\]

We prove item 2. We observe that a term with history
$\beta_0(\alpha_1')^{h_1}$ $\beta_1 \cdots \beta_{n-1}
(\alpha_n')^{h_n} \beta_n$ in $\suh{\Mplus}$ or in $\mathbb{L}$ may have no
corresponding term (by a flashback) with history
$\beta_0(\alpha_1')^{h_1-1}\beta_1$ $(\alpha_2')^{h_2} \cdots \beta_{n-1}$ 
$(\alpha_n')^{h_n} \beta_n$. This is because the evaluation to the saturated state
may have not expanded some invocations. It is however true that terms with 
histories $[\beta_0(\alpha_1')^{h_1}\beta_1 \cdots 
\beta_{n-1}(\alpha_n')^{h_n} \beta_n]$ (kernels) are either in $\suh{\Mplus}$ or in 
$\mathbb{L}$ and the item 2 is demonstrated by proving that a flashback to terms
with histories that are kernels does exist. 

Let $\alpha_1 = \beta_0(\alpha_1')^{h_1}\beta_1 \cdots 
\beta_{n-1}(\alpha_n')^{h_n}\beta_n$ be a $\M$-yielding sequence. We proceed by induction on $n$.
When $n = 1$ there are two cases: $h_1 \leq 1$ and $h_1\geq 2$. In the first
case there is nothing to prove because $[\alpha] = \alpha$. When $h_1\geq 2$,
since $\alpha$ fits with the hypotheses of Item 1, there exist
$\rho_{\beta_0,\alpha_1',\beta_1}^{(2)}, \cdots ,\rho_{\beta_0,\alpha_1',\beta_1}^{(h_1)}$.  
Let $\delta_{\beta_0,\alpha_1',\beta_1}^{(2)} = \rho_{\beta_0,\alpha_1',\beta_1}^{(2)}$ and 
$\delta_{\beta_0,\alpha_1',\beta_1}^{(i+1)} = \rho_{\beta_0,\alpha_1',\beta_1}^{(i+1)}[x \mapsto x \; | \; 
x \in \dom{\delta_{\beta_0,\alpha_1',\beta_1}^{(i)}}]$.
We also let $\rho_{\alpha_1} =\delta_{\beta_0,\alpha_1',\beta_1}^{(2)} {\scriptstyle \, \circ \,} \cdots {\scriptstyle \, \circ \,} \delta_{\beta_0,\alpha_1',\beta_1}^{(h_1)}$ and we observe that, by definition of renaming composition, 
if $\alpha_1 \preceq \alpha_2$ then
$\rho_{\alpha_1} \lessfb \rho_{\alpha_2}$. 
In this case, the items 2.a and 2.b follow by item 1, 
Proposition~\ref{prop.storie-insieme} and the diamond property of Proposition~\ref{prop.diamond}.  


We assume the statement holds for a generic $n$ and we prove the case $n+1$.
Let $\alpha_1 = \beta 
\beta_{n}(\alpha_{n+1}')^{h_{n+1}}\beta_{n+1}$ and $h_{n+1} >0$ (because
$[\beta_{n}(\alpha_{n+1}')^1\beta_{n+1}] = \beta_{n}\alpha_{n+1}'\beta_{n+1}$). We consider the map
\[
\rho_{\alpha_1} \eqdef \rho_{\beta} {\scriptstyle \, \circ \,} \delta_{\beta_n,\alpha_{n+1}',\beta_{n+1}}^{(2)} {\scriptstyle \, \circ \,}
\cdots {\scriptstyle \, \circ \,} \delta_{\beta_n,\alpha_{n+1}',\beta_{n+1}}^{(h_{n+1})}\]
where $\delta_{\beta_n,\alpha_{n+1}',\beta_{n+1}}^{(i)}$, $2 \leq i \leq h_{n+1}$
are defined as above.
As before, the items 2.a and 2.b follow by item 1 for 
$\delta_{\beta_n,\alpha_{n+1}',\beta_{n+1}}^{(2)} {\scriptstyle \, \circ \,}
\cdots$ ${\scriptstyle \, \circ \,} \delta_{\beta_n,\alpha_{n+1}',\beta_{n+1}}^{(h_{n+1})}$ and by
Proposition~\ref{prop.storie-insieme} and the diamond property of Proposition~\ref{prop.diamond}.  Then we apply the inductive hypothesis for $\rho_{\beta}$.
The property (2.c) $\alpha_1 \preceq \alpha_2$ implies
 $\rho_{\alpha_1} \lessfb \rho_{\alpha_2}$ is an immediate consequence of the definition.
\end{proof}


Every preliminary statement is in place for our key theorem that details the mapping
of circularities created by transitions of saturated states to past circularities.
For readability sake, we restate the theorem.

\bigskip

\noindent
{\bf Theorem~\ref{thm.invariance}.} {\em
Let $\ilam{\por{I}_{\P}, \, \varnothing, \,  \addh{\varepsilon}{\P}}{} \lred{}^* \ilam{\mathbb{V}, \, \suh{\Mplus}, \, \mathbb{L}}{}$ 
and
$\ilam{\mathbb{V}, \,  \suh{\Mplus}, \,\mathbb{L}}{}$ be a saturated state.
If
$\ilam{\mathbb{V},  \suh{\Mplus}, \,\mathbb{L}}{} \lred{}
\ilam{\mathbb{V}', \,  \suh{\Mplus}', \,\mathbb{L}'}{}$ then
\begin{enumerate}
\item
$\ilam{\mathbb{V}', \,  \suh{\Mplus}', \,\mathbb{L}'}{}$ is saturated;
\item
if $\mathbb{L}'$ has a circularity then $\mathbb{L}$ has already 
a circularity.
\end{enumerate}
}


\begin{proof}
The item 1. is an immediate consequence of Proposition~\ref{prop.flashback-order}.
We prove 2.  
Let
\begin{itemize}
\item[--]
$\mathbb{L} =  \suh{\CP}[\suhs{\alpha}{\M(\wt{u})} ]$;
 
\item[--]
$\M(\wt{u}) = \P'$

\item[--]
$\mathbb{L}' = \suh{\CP}[\addh{\alpha\M}{\P'}]$; 

\item[--]
$\flatt{\mathbb{L}}=\flatt{\suh{\CP}[\suhs{\alpha}{\M(\wt{u})} ]}=\suh{\T}_1\seq\cdots\seq\suh{\T}_p$;

\item[--]
$\flatt{\P'} = \T'_1\seq \cdots \seq \T_{p'}'$;

\item[--]
$\flatt{\mathbb{L}'}
= \suh{\T}_1''\seq\cdots\seq\suh{\T}_q''$;

\item[--]
$\suhs{\alpha_0}{(x_0,x_1)} \sparop \cdots \sparop
\suhs{\alpha_n}{(x_n,x_0)} \in \suh{\T}_1''\seq\cdots\seq\suh{\T}_q''$ 
 (it is a circularity).
\end{itemize}
Without loss of generality, we may reduce to the following case (the general case
is demonstrated by iterating the arguments below).

Let $\alpha \M = \beta (\alpha')^{m+2}\beta'$ and let 
\[
\begin{array}{@{\!}rl}
\suhs{\alpha_0}{(x_0,x_1)} \sparop \cdots \sparop
\suhs{\alpha_n}{(x_n,x_0)} =  &
\prod_{0 \leq j \leq n'} \suhs{\beta(\alpha')^{m+1}\beta'\beta_j}{(x_j, x_{j+1})}
\\
& \sparop \suhs{\alpha_{n'+1}}{(x_{n'+1}, x_{n'+2})}
\\
& \sparop \cdots \sparop
\suhs{\alpha_n}{(x_n,x_0)}
\end{array}
\] 
with $\varepsilon \precneq \beta_j \preceq \beta''\beta'$, where $\beta'\beta'' = \alpha'$, and $n'<n$
(otherwise 2 is straightforward because the circularity may be mapped
to a previous circularity by $\rho_{\beta,\alpha',\beta'}^{(m+2)}$, see Lemma~\ref{lem.flashback}(1.b), or it
is already contained in $\mathbb{L}$). This is the case of crossover circularities, as discussed in Section~\ref{sec.introduction}.

By Lemma~\ref{lem.flashback}, 
\begin{eqnarray}
\label{eq.lemma}
\begin{array}{l}
\suhs{\beta (\alpha')^{m}\beta'\beta_0}{(\rho_{\beta,\alpha,\beta'}^{(m+2)}(x_0), 
\rho_{\beta,\alpha,\beta'}^{(m+2)}(x_{1}))} \sparop  \cdots  
\\
\sparop \;
\suhs{\beta (\alpha')^{m+1}\beta'\beta_{n'}}{
(\rho_{\beta,\alpha,\beta'}^{(m+2)}(x_{n'}), 
\rho_{\beta,\alpha,\beta'}^{(m+2)}(x_{n'+1}))}
\end{array}
\end{eqnarray}
is in some $\suh{\T}_{i}''$. There are two cases.

\emph{Case 1}: for every $n'+1 \leq i \leq n$, 
$\alpha_{i} \precneq \beta (\alpha')^{m+1}\beta'$. Then, by 
Lemma~\ref{lem.flashback}(1), we have $\rho_{\beta,\alpha,\beta'}^{(m+2)}(x_{0}) = 
\rho_{\beta,\alpha,\beta'}^{(m+1)}(x_{0})$ and 
$\rho_{\beta,\alpha,\beta'}^{(m+2)}(x_{n'+1}) = 
\rho_{\beta,\alpha,\beta'}^{(m+1)}(x_{n'+1})$. Therefore, by 
Lemma~\ref{lem.flashback}(2), 
\[
\begin{array}{rl}
(\ref{eq.lemma})
\sparop &
\suhs{\alpha_{n'+1}'}{(\rho_{\beta,\alpha,\beta'}^{(m+1)}(x_{n'+1}),\rho_{\beta,\alpha,\beta'}^{(m+1)}(x_{n'+2}))}
\\
\sparop & \cdots \sparop \suhs{\alpha_{n}'}{(
\rho_{\beta,\alpha,\beta'}^{(m+1)}(x_{n}),\rho_{\beta,\alpha,\beta'}^{(m+1)}(x_{0}))}
\end{array}
\] 
with suitable $\alpha_{n'+1}', \cdots , \alpha_n'$, 
is a circularity in $ \suh{\T}_1''\seq\cdots\seq\suh{\T}_q''$. In particular, 
whenever, for every $n'+1 \leq i \leq n$, $\alpha_i = \beta (\alpha')^{m}\beta'\beta_i$ 
with $\varepsilon \precneq \beta_{i} \preceq \beta''\beta'$, the flashback 
$\rho_{\beta,\alpha,\beta'}^{(m+1)}$ maps dependencies $\suhs{\alpha_i}{(x_i,x_{i+1})}$ to 
dependencies 
\[
\suhs{\beta (\alpha')^{m-1}\beta'\beta_i}{(\rho_{\beta,\alpha,\beta'}^{(m+1)}(x_i),\rho_{\beta,\alpha,\beta'}^{(m+1)}(x_{i+1}))}
\]
 if $m >0$. It is the identity,
if $m=0$.

\emph{Case 2}: there is $n'+1 \leq i \leq n$ such that $\alpha_i \not \preceq \beta (\alpha')^{m+2}\beta'$. Let this $i$ be $n'+1$. For instance, 
$\beta = \beta_1' (\alpha'')^{m'} \beta_1''$ and $\alpha_{n'+1} = \beta_1'(\alpha'')^{m'+1} \beta_1''(\alpha''')^{m''} \beta_1'''$ with $m' \geq 2$ and $m'' \geq 2$.
In this case it is possible that there is no pair $\suhs{\gamma}{(y, 
y')}$, with $\gamma \succeq  \beta_1' (\alpha'')^{m'}$, to which map 
$\suhs{\alpha_{n'+1}}{(x_{n'+1}, x_{n'+2})}$ by means of a flashback.
To overcome 
this issue, we consider the flashbacks
$\rho_{\alpha_0}, \cdots , \rho_{\alpha_{n'}}, \rho_{\alpha_{n'+1}}$ and we observe 
that
\begin{eqnarray}
\label{eq.lemma2}
\begin{array}{l}
\suhs{[\alpha_0]}{(\rho_{\alpha_0}(x_0), 
\rho_{\alpha_0}(x_{1}))} \sparop  \cdots  
\sparop \;
\suhs{[\alpha_{n'}]}{
(\rho_{\alpha_{n'}}(x_{n'}), 
\rho_{\alpha_{n'}}(x_{n'+1}))} \hspace{-.6cm}
\\
\sparop \; \suhs{[\alpha_{n'+1}]}{
(\rho_{\alpha_{n'+1}}(x_{n'+1}), 
\rho_{\alpha_{n'+1}}(x_{n'+2}))} \sparop \cdots 
\\
\sparop \; \suhs{[\alpha_{n}]}{(\rho_{\alpha_{n}}(x_{n}), 
\rho_{\alpha_{n}}(x_{1}))}
\end{array}
\end{eqnarray}
verifies
\begin{enumerate}
\item[(a)]
for every $0 \leq i < n$, $\rho_{\alpha_i}(x_{i+1}) = 
\rho_{\alpha_{i+1}}(x_{i+1})$ and $\rho_{\alpha_n}(x_{0}) = 
\rho_{\alpha_{0}}(x_{0})$;
\item[(b)]
the term (\ref{eq.lemma2}) is a subterm of $ \suh{\T}_1''\seq\cdots\seq\suh{\T}_q''$.
\end{enumerate}
As regards (a), the property derives by definition of the flashbacks $\rho_{\alpha_i}$
and $\rho_{\alpha_{i+1}}$ in Lemma~\ref{lem.flashback}.
As regards (b), it follows by Lemma~\ref{lem.flashback}(2.b) because 
$\suhs{\alpha_0}{(x_0,x_1)} \sparop \cdots \sparop$
$\suhs{\alpha_{n}}{(x_{n},x_{1})} \in \suh{\T}_1''$ $\seq\cdots\seq\suh{\T}_q''$.
\end{proof}

