

This section develops the technical details for proving Theorem~\ref{thm.invariance}.

\begin{definition}
A history  is 
\begin{description}
\item[-\emph{yielding}] 
\ \\
if
 such that, for every , 
 is a recursive history, , 
and  implies the program has the definition
, for some .
The \emph{kernel} of , denoted , is 
,
where .
\end{description}
\end{definition}


By definition, if  is -saturating then it is also 
-yielding. In this case, the kernel  has a suffix that
 is -complete. 
In the -program,  ,
, and , and
the recursive histories of ,  and  are equal to , to 
 and to , respectively.
Then  is the -complete history and 
 is -yielding, with .

We notice that every history of an informative
lam (obtained by evaluating ) is a yielding sequence.
We also notice that, for every ,  is -yielding.
In fact,  is the history of every function invocation in the
initial lam, which may concern every function name of the program.
As regards the kernel, 
in Lemma~\ref{lem.flashback}, we 
demonstrate that, 
if  is a -yielding history such that every ,
then every term 
 may be mapped by a flashback  to a term
; similarly for dependencies. This is the
basic property that allows us to map circularities to past circularities
 (see Theorem~\ref{thm.invariance}). 

Next we introduce an ordering relation over renamings, (in particular, flashbacks) 
and the operation of renaming composition. The definitions are almost standard:
\begin{itemize}
\item 
 if, for every , .
\item
 be defined as follows: 

\smallskip


\end{itemize}
We notice that, if both
\begin{enumerate}
\item
  and  are flashbacks and 
\item
for every ,  

\end{enumerate}
then  holds.
In the following, lams  and , being  of
terms that are dependencies composed with , will be written 
 and
, for some , 
respectively, where  and  contain dependencies  and 
. Let also .

With an abuse of notation, we will use the set operation ``''
for  and . For instance, we will write   when there is
 such that . Similarly, we will write  when there
is  such that .

A consequence of the
axiom  is 
the following property of the informative operational semantics.

\begin{proposition}
\label{prop.storie-insieme}
Let 
be a state of an informative operational semantics. For every , let  and  be 
. 
Finally, let

\smallskip



\smallskip

If  \  
then, for every , 
.
\end{proposition}

The next lemma allows us to map, through a flashback, terms in a saturated 
state to terms that have been produced in the past. The correspondence is defined
by means of the (regular) structure of histories. 

\begin{lemma}
\label{lem.flashback}
Let  
and  
be saturated and 
. Then
\begin{enumerate}
\item
if , where 
 is -yielding, then there
are  -flashbacks  such that:
\begin{enumerate}
\item
;
\item
 where, 
for every , , implies 
;
\item
 
 implies 
.
\end{enumerate}

\item 
if  are  -yielding, , -yielding, respectively,
  then there are flashbacks  
such that
\begin{enumerate}
\item
if  or 
 
then ;
\item
if  
then \\
;

\item
if  then .

(In particular, if , with , and  then
).
\end{enumerate}
\end{enumerate}
\end{lemma}

\begin{proof} (Sketch)
As regards item 1, let   and let  (therefore the length of  is ).
The evaluation  may be decomposed as follows
 
By definition of the operational semantics there is the \emph{alternative} evaluation
 
[notice that ].
Property (1.a) is an immediate consequence of Proposition~\ref{prop.flashback-order};
let  be the flashback 
for the last state. The property (1.b), when , is also an immediate consequence of 
Propositions~\ref{prop.flashback-order}
and of~\ref{prop.storie-insieme}. In the general case, we need to iterate 
the arguments on shorter histories and the arguments are similar for (1.c).
In order to conclude the proof of item 1, we need an additional argument. By Proposition~\ref{prop.diamond}, there exists an evaluation

such that  and  are identified by a bijective renaming, let it be . We define
the 
corresponding to the evaluation

as . Similarly for the other
. The properties of item 1 for 
 follow
by the corresponding ones for


We prove item 2. We observe that a term with history
  in  or in  may have no
corresponding term (by a flashback) with history
  
. This is because the evaluation to the saturated state
may have not expanded some invocations. It is however true that terms with 
histories  (kernels) are either in  or in 
 and the item 2 is demonstrated by proving that a flashback to terms
with histories that are kernels does exist. 

Let  be a -yielding sequence. We proceed by induction on .
When  there are two cases:  and . In the first
case there is nothing to prove because . When ,
since  fits with the hypotheses of Item 1, there exist
.  
Let  and 
.
We also let  and we observe that, by definition of renaming composition, 
if  then
. 
In this case, the items 2.a and 2.b follow by item 1, 
Proposition~\ref{prop.storie-insieme} and the diamond property of Proposition~\ref{prop.diamond}.  


We assume the statement holds for a generic  and we prove the case .
Let  and  (because
). We consider the map

where , 
are defined as above.
As before, the items 2.a and 2.b follow by item 1 for 
  and by
Proposition~\ref{prop.storie-insieme} and the diamond property of Proposition~\ref{prop.diamond}.  Then we apply the inductive hypothesis for .
The property (2.c)  implies
  is an immediate consequence of the definition.
\end{proof}


Every preliminary statement is in place for our key theorem that details the mapping
of circularities created by transitions of saturated states to past circularities.
For readability sake, we restate the theorem.

\bigskip

\noindent
{\bf Theorem~\ref{thm.invariance}.} {\em
Let  
and
 be a saturated state.
If
 then
\begin{enumerate}
\item
 is saturated;
\item
if  has a circularity then  has already 
a circularity.
\end{enumerate}
}


\begin{proof}
The item 1. is an immediate consequence of Proposition~\ref{prop.flashback-order}.
We prove 2.  
Let
\begin{itemize}
\item[--]
;
 
\item[--]


\item[--]
; 

\item[--]
;

\item[--]
;

\item[--]
;

\item[--]
 
 (it is a circularity).
\end{itemize}
Without loss of generality, we may reduce to the following case (the general case
is demonstrated by iterating the arguments below).

Let  and let 
 
with , where , and 
(otherwise 2 is straightforward because the circularity may be mapped
to a previous circularity by , see Lemma~\ref{lem.flashback}(1.b), or it
is already contained in ). This is the case of crossover circularities, as discussed in Section~\ref{sec.introduction}.

By Lemma~\ref{lem.flashback}, 

is in some . There are two cases.

\emph{Case 1}: for every , 
. Then, by 
Lemma~\ref{lem.flashback}(1), we have  and 
. Therefore, by 
Lemma~\ref{lem.flashback}(2), 
 
with suitable , 
is a circularity in . In particular, 
whenever, for every ,  
with , the flashback 
 maps dependencies  to 
dependencies 

 if . It is the identity,
if .

\emph{Case 2}: there is  such that . Let this  be . For instance, 
 and  with  and .
In this case it is possible that there is no pair , with , to which map 
 by means of a flashback.
To overcome 
this issue, we consider the flashbacks
 and we observe 
that

verifies
\begin{enumerate}
\item[(a)]
for every ,  and ;
\item[(b)]
the term (\ref{eq.lemma2}) is a subterm of .
\end{enumerate}
As regards (a), the property derives by definition of the flashbacks 
and  in Lemma~\ref{lem.flashback}.
As regards (b), it follows by Lemma~\ref{lem.flashback}(2.b) because 

 .
\end{proof}

