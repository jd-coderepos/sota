\documentclass{llncs}

\usepackage{multicol}

\usepackage{amsmath}
\usepackage{amsfonts}
\usepackage{amssymb}
\usepackage{graphicx}
\usepackage{epic}
\usepackage{eepic}
\usepackage{epsfig,float}
\usepackage{pdfsync}
\pagestyle{plain}

\DeclareGraphicsRule{.tif}{png}{.png}{`convert #1 `dirname #1`/`basename #1 .tif`.png}
\renewcommand{\le}{\leqslant}
\renewcommand{\ge}{\geqslant}

\newcommand{\ol}{\overline}
\newcommand{\eps}{\varepsilon}
\newcommand{\emp}{\emptyset}
\newcommand{\rhoR}{R}
\newcommand{\Sig}{\Sigma}
\newcommand{\sig}{\sigma}
\newcommand{\noin}{\noindent}
\newcommand{\bi}{\begin{itemize}}
\newcommand{\ei}{\end{itemize}}
\newcommand{\be}{\begin{enumerate}}
\newcommand{\ee}{\end{enumerate}}
\newcommand{\bd}{\begin{description}}
\newcommand{\ed}{\end{description}}
\newcommand{\bq}{\begin{quote}}
\newcommand{\eq}{\end{quote}}
\newcommand{\txt}[1]{\mbox{ #1 }}
\newcommand{\defeq}{\stackrel{\rm def}{=}}
\newcommand{\etc}{\mbox{\it etc.}}
\newcommand{\ie}{\mbox{\it i.e.}}
\newcommand{\eg}{\mbox{\it e.g.}}



\newcommand{\trng}{\mathrm{rng}}
\newcommand{\tdom}{\mathrm{dom}}
\newcommand{\trank}{\mathrm{rank}}

\newcommand{\tid}{\mbox{{\bf 1}}}
\newcommand{\inv}[1]{\mbox{}}
\newcommand{\stress}[1]{{\fontfamily{cmtt}\selectfont #1}}

\def\shu{\mathbin{\mathchoice
{\rule{.3pt}{1ex}\rule{.3em}{.3pt}\rule{.3pt}{1ex}
\rule{.3em}{.3pt}\rule{.3pt}{1ex}}
{\rule{.3pt}{1ex}\rule{.3em}{.3pt}\rule{.3pt}{1ex}
\rule{.3em}{.3pt}\rule{.3pt}{1ex}}
{\rule{.2pt}{.7ex}\rule{.2em}{.2pt}\rule{.2pt}{.7ex}
\rule{.2em}{.2pt}\rule{.2pt}{.7ex}}
{\rule{.3pt}{1ex}\rule{.3em}{.3pt}\rule{.3pt}{1ex}
\rule{.3em}{.3pt}\rule{.3pt}{1ex}}\mkern2mu}}


\newcommand{\cA}{{\mathcal A}}
\newcommand{\cB}{{\mathcal B}}
\newcommand{\cC}{{\mathcal C}}
\newcommand{\cD}{{\mathcal D}}
\newcommand{\cE}{{\mathcal E}}
\newcommand{\cF}{{\mathcal F}}
\newcommand{\cG}{{\mathcal G}}
\newcommand{\cL}{{\mathcal L}}
\newcommand{\cP}{{\mathcal P}}
\newcommand{\cM}{{\mathcal M}}
\newcommand{\cN}{{\mathcal N}}
\newcommand{\cT}{{\mathcal T}}
\newcommand{\cS}{{\mathcal S}}
\newcommand{\gL}{{\mathcal L}}
\newcommand{\gR}{{\mathcal R}}
\newcommand{\gJ}{{\mathcal J}}
\newcommand{\gH}{{\mathcal H}}
\newcommand{\gD}{{\mathcal D}}
\newcommand{\rmPi}{{\mathrm \Pi}} 
\newcommand{\Lra}{{\Leftrightarrow}}
\newcommand{\lra}{{\leftrightarrow}}
\newcommand{\ra}{{\rightarrow}}
\newcommand{\raL}{{\sim_L}}
\newcommand{\lraL}{{\mathbin{\approx_L}}}

\newcommand{\meet}{{\mathbin{\wedge}}}
\newcommand{\join}{{\mathbin{\vee}}}

\newcommand{\sn}{{semiautomaton}}
\newcommand{\sa}{{semiautomata}}
\newcommand{\Sn}{{Semiautomaton}}
\newcommand{\Sa}{{Semiautomata}}
\newcommand{\se}{{settable}}
\newcommand{\Se}{{Settable}}
\newcommand{\qedb}{\hfill} 

\newcommand{\com}{\mathbb{C}}
\newcommand{\rev}{\mathbb{R}}
\newcommand{\deter}{\mathbb{D}}
\newcommand{\mini}{\mathbb{M}}
\newcommand{\trim}{\mathbb{T}}

\DeclareMathOperator{\Fix}{Fix}
\DeclareMathOperator{\Orbit}{\Omega}
\DeclareMathOperator{\Max}{Max} 

\title{Syntactic Complexity of - and -Trivial Regular Languages\thanks{This work was supported by the Natural Sciences and Engineering Research Council of Canada under grant No.~OGP0000871 and  a Postgraduate Scholarship.
}
}

\author{Janusz~Brzozowski, Baiyu Li\thanks{Present address: 
Optumsoft, Inc.,
275 Middlefield Rd, Suite 210, Menlo Park, CA 94025, USA}
}
\authorrunning{Brzozowski, Li}   

\institute{David R. Cheriton School of Computer Science, University of Waterloo \\
Waterloo, ON, Canada N2L 3G1\\
\{{\tt \{brzozo, b5li\}@uwaterloo.ca} \}
}

\begin{document}

\maketitle
\begin{abstract}
The syntactic complexity of a subclass of the class of regular languages is the maximal cardinality of syntactic semigroups of languages in that class, taken as a function of the state complexity  of these languages.
We prove that  and  are tight upper bounds for 
the syntactic complexity of - and -trivial regular languages, respectively. 
We also prove that  is the tight upper bound on the state complexity of reversal of -trivial regular languages. 
\bigskip

\noin
{\bf Keywords:}
finite automaton, -trivial, monoid, regular language, reversal, -trivial, semigroup, syntactic complexity 
\end{abstract}


\section{Introduction}\label{sec:intro}

The \emph{state complexity} of a regular language  is the number of states in the minimal deterministic finite automaton (DFA) accepting . An equivalent notion is \emph{quotient complexity}, which is the number of distinct left quotients of . 
The \emph{syntactic complexity} of  is the cardinality of the syntactic semigroup of . 
Since the syntactic semigroup of  is isomorphic to the semigroup of transformations performed by the minimal DFA of , it is natural to consider the relation between syntactic complexity and state complexity. The \emph{syntactic complexity of a subclass of regular languages} is the maximal syntactic complexity of languages in that class, taken as a function of the state complexity of these languages.

Here we consider the classes of languages defined using the well-known Green equivalence relations on semigroups~\cite{Pin97}. Let  be a monoid, that is, a semigroup with an identity, and let  be any two elements of . The Green relations on , denoted by  and , are defined as follows: 

If  is an equivalence relation on , then  is \emph{-trivial} if and only if  implies  for all . A language is \emph{-trivial} if and only if its syntactic monoid is -trivial. 
In this paper we consider only regular -trivial languages. 
-trivial regular languages are exactly the star-free languages~\cite{Pin97}, and -, -, and -trivial regular languages are all subclasses of star-free languages. 
The class of -trivial languages is the intersection of - and -trivial classes. 

A language  is \emph{piecewise-testable} if it is a finite boolean combination of languages of the form , where . Simon~\cite{Sim72,Sim75} proved in 1972 that a language is piecewise-testable if and only if it is -trivial. A \emph{biautomaton} is a finite automaton which can read the input word alternatively from left and  right. In 2011 Kl\'{\i}ma and Pol{\'a}k~\cite{KP11} showed that a language is piecewise-testable if and only if it is accepted by an acyclic biautomaton; here self-loops are allowed, as they are not considered cycles. 

In 1979 Brzozowski and Fich~\cite{BrFi80} proved that a regular language is -trivial if and only if its minimal DFA is \emph{partially ordered}, that is, it is acyclic as above. They also showed that -trivial regular languages are  finite boolean combinations of languages , where  and . Recently Jir{\'a}skov{\'a} and Masopust proved a tight upper bound on the state complexity of reversal of -trivial languages~\cite{JiMa12}. 

The syntactic complexity of the following subclasses of regular languages was considered: In 1970 Maslov~\cite{Mas70} noted that  was a tight upper bound on the number of transformations performed by a DFA of  states. In 2003--2004, Holzer and K\"onig~\cite{HoKo04}, and Krawetz, Lawrence and Shallit~\cite{KLS03} studied unary and binary languages. In 2010 Brzozowski and Ye~\cite{BrYe11} examined ideal and closed regular languages. In 2012 Brzozowski, Li and Ye studied prefix-, suffix-, bifix-, and factor-free regular languages~\cite{BLY12}, Brzozowski and Li~\cite{BL12a} considered the class of star-free languages and three of its subclasses, and Brzozowski and Liu~\cite{BrLiu12} studied finite/cofinite, definite, and reverse definite languages, where 
 is \emph{definite} (\emph{reverse-definite}) if it can be decided whether a word  belongs to  by examining the suffix (prefix) of  of some fixed length.

We state basic definitions and facts in Section~\ref{sec:pre}. In Sections~\ref{sec:Rtrivial} and~\ref{sec:Jtrivial} we prove tight upper bounds on the syntactic complexities of - and -trivial regular languages, respectively. 
In Section~\ref{sec:rev} we prove the tight upper bound on the quotient complexity of reversal of -trivial regular languages, and we show that this bound can be met by our languages with maximal syntactic complexities. 
Section~\ref{sec:con} concludes the paper. 


\section{Preliminaries}\label{sec:pre}

Let  be a non-empty finite set with  elements, and assume without loss of generality that . There is a linear order on , namely the natural order  on integers. If  is a non-empty subset of , then the maximal element in  is denoted by . A \emph{partition}  of  is a collection  of non-empty subsets of  such that  , and
 for all .
We call each subset  a \emph{block} in . For any partition  of , let . The set of all partitions of  is denoted by . We define a partial order  on  such that, for any ,  if and only if each block of  is contained in some block of . We say  \emph{refines}  if . 
The poset  is a finite lattice: For any , the \emph{meet}  is the -largest partition that refines both  and , and the \emph{join}  is the -smallest partition that is refined by both  and . From now on, we refer to the lattice  simply  as . 


A {\em transformation} of a set  is a mapping of  into itself. We consider only transformations  of a finite set . If  , then  is the {\it image} of  under .  If  is a subset of , then , and the {\em restriction} of  to , denoted by , is a mapping from  to  such that  for all . The {\em composition} of  transformations  and  of  is a transformation  such that  for all . We~usually drop the  operator ``'' and write  for short. 
An arbitrary transformation can be written in the form

where ,  , and . We also use the notation  for  above. The {\em domain}  of  is 
The {\em range}  of  is the set  The \emph{rank}  of  is the cardinality of , \ie, . The binary relation  on  is defined as follows: For any ,  if and only if  for some . This is an equivalence relation, and each equivalence class is called an \emph{orbit} of . For any , the orbit of  containing  is denoted by . The~set of all orbits of  is denoted by . Clearly,  is a partition of . 

A \emph{permutation} of  is a mapping of  \emph{onto} itself, so here . 
The \emph{identity} transformation  maps each element to itself. 
A transformation  is a \emph{cycle} of length , where , if there exist pairwise different elements  such that
, and , and the remaining elements are mapped to themselves.
A~cycle is denoted by .
For , a \emph{transposition} is the cycle .
A~\emph{singular} transformation, denoted by , has  and  for all .
A~\emph{constant} transformation,  denoted by , has  for all .
A transformation   is an \emph{idempotent} if .
The set  of all transformations of  is a finite semigroup, in fact, a monoid. We refer the reader to the book of Ganyushkin and Mazorchuk~\cite{GaMa09} for a detailed discussion of finite transformation semigroups. 

\medskip

For background about regular languages, we refer the reader to~\cite{Yu97}. Let  be a non-empty finite alphabet. Then  is the free monoid generated by , and  is the free semigroup generated by . A \emph{word} is any element of , and the empty word is . The length of a word  is . A \emph{language} over  is any subset of . The \emph{reverse of a word}  is denoted by . For a language , its \emph{reverse} is . The \emph{left quotient}, or simply \emph{quotient}, of a language  by a word  is   . 

The \emph{Myhill congruence}~\cite{Myh57}  of any language  is defined as follows:

This congruence is also known as the \emph{syntactic congruence} of . The quotient set  of equivalence classes of the relation  is a semigroup called the \emph{syntactic semigroup} of , and  is the \emph{syntactic monoid} of~. 
The \emph{syntactic complexity}  of  is the cardinality of its syntactic semigroup.
A language is regular if and only if its syntactic semigroup is finite. We consider only regular languages, and so assume that all syntactic semigroups and  monoids are finite.

A DFA is denoted by , as usual. The DFA  accepts a word  if . The language accepted by  is denoted by . If  is a state of , then the language  of  is the language accepted by the DFA . Two states  and  of  are \emph{equivalent} if . If  is a regular language, then its \emph{quotient DFA} is , where , , ,  . The \emph{quotient complexity}  of  is the number of distinct quotients of . The quotient DFA of  is the minimal DFA accepting , and so quotient complexity is the same as state complexity. 


If  is a DFA, then its \emph{transition semigroup}~\cite{Pin97}, denoted by , consists of all transformations  on  performed by non-empty words  such that  for all . The syntactic semigroup  of a regular language  is isomorphic to the transition semigroup of the quotient DFA  of ~\cite{McNP71}, and we represent elements of  by transformations in . 
Given a set  of transformations of , we can define the transition function  of some DFA  such that  for all . The transition semigroup of such a DFA is the semigroup generated by . When the context is clear, we write ,  to mean that the transformation performed by  is~.

\section{-Trivial Regular Languages}\label{sec:Rtrivial} 

Given DFA , we  define the \emph{reachability relation}  as follows. For all ,  if and only if  for some . We~say that  is \emph{partially ordered}~\cite{BrFi80} if the relation  is a partial order on . 

Consider the natural order  on . A transformation  of  is \emph{non-decreasing} if  for all . The set  of all non-decreasing transformations of  is a semigroup, since the composition of two non-decreasing transformations is again non-decreasing. It was shown in~\cite{BrFi80} that a language  is -trivial if and only if its quotient DFA is partially ordered. Equivalently,  is an -trivial language if and only if its syntactic semigroup contains only non-decreasing transformations. 


 It is known~\cite{GaMa09} that   is generated by the following set 

For any transformation  of , let . Then 


\begin{lemma}\label{lem:fixrank} 
For any , . 
\end{lemma}

\begin{proof} 
Pick arbitrary . The claim holds trivially for . Assume . Clearly . Suppose there exists  but . Then  for some , and . However, since ,  is not an idempotent, which is a contradiction. Therefore . \qed
\end{proof}

If , then  contains only the identity transformation , and . So . If , then we have

\begin{lemma}\label{lem:carGF}
For , . 
\end{lemma}

\begin{proof} 
Pick  such that . Then , and, by Lemma~\ref{lem:fixrank}, . There is only one element , and . Note that  is fully determined by the pair . Hence there are  different . Together with the identity , the cardinality of  is . \qed
\end{proof}

\begin{lemma}\label{lem:minGF} 
If  and  generates , then . 
\end{lemma}

\begin{proof}
Suppose there exists  such that . Since  generates ,  can be written as  for some , where . Then . Note that  is the only element in  with range ; so if , then , a~contradiction. 

Assume , and  for all . Then , and . Since each  is non-decreasing, for all , we must have  as well; so . Moreover, since  and  by Lemma~\ref{lem:fixrank}, . Now, let  be the unique element in . Then , and . So~. However, since ,  and . Hence , and we get a contradiction again. Therefore . \qed
\end{proof}

Consequently,  is the unique minimal generator of . So we obtain  

\begin{theorem}\label{thm:Rtrivial}
If  is an -trivial regular language of quotient complexity , then its syntactic complexity  satisfies , and this bound is tight if  for  and  for .  
\end{theorem}


\begin{proof}
Let  be the quotient DFA of , and let  be its syntactic semigroup. Then  is a subset of . Pick an arbitrary . For each , since ,  can be chosen from . Hence there are exactly  transformations in , and . 

When , the only regular languages are  or , and they are both  -trivial. To see the bound is tight for , let  be the DFA with alphabet  of size  and set of states , where each  defines a distinct transformation in . For each , let . Since  generates  and ,  for some , where  depends on . Then there exist  such that each  performs  and state  is reached by . Moreover,  is the only final state of~. Consider any non-final state . Since , there exist  such that the word  performs . State  can be distinguished from other non-final states by the word . Hence  has quotient complexity . The syntactic monoid of  is , and so . By Lemma~\ref{lem:minGF}, the alphabet of  is minimal. \qed
\end{proof}


\begin{example}\label{ex:Rtrivial}
When , there are  non-decreasing transformations of . Among them, there are 11 transformations with rank . The following 6 transformations from the 11 are idempotents: 

Together with the identity transformation , we have the generating set  for  with 7 transformations. We can then define the DFA  with 7 inputs as in the proof of Theorem~\ref{thm:Rtrivial};  is shown in Fig.~\ref{fig:RTDFA}. The quotient complexity of  is , and the syntactic complexity of  is . \qedb
\end{example}


\begin{figure}[hbt]
\begin{center}
\setlength{\unitlength}{0.00065617in}
\begingroup\makeatletter\ifx\SetFigFont\undefined \gdef\SetFigFont#1#2#3#4#5{\reset@font\fontsize{#1}{#2pt}\fontfamily{#3}\fontseries{#4}\fontshape{#5}\selectfont}\fi\endgroup {\renewcommand{\dashlinestretch}{30}
\begin{picture}(4070,2116)(0,-10)
\put(1587.000,1110.250){\arc{337.500}{2.4981}{6.9267}}
\blacken\path(1725.575,1132.641)(1722.000,1009.000)(1783.339,1116.413)(1725.575,1132.641)
\put(1587,829){\ellipse{450}{450}}
\put(2712.000,1110.250){\arc{337.500}{2.4981}{6.9267}}
\blacken\path(2850.575,1132.641)(2847.000,1009.000)(2908.339,1116.413)(2850.575,1132.641)
\put(2712,829){\ellipse{450}{450}}
\put(3837.000,1110.250){\arc{337.500}{2.4981}{6.9267}}
\blacken\path(3975.575,1132.641)(3972.000,1009.000)(4033.339,1116.413)(3975.575,1132.641)
\put(3837,829){\ellipse{450}{450}}
\put(462.000,1110.250){\arc{337.500}{2.4981}{6.9267}}
\blacken\path(600.575,1132.641)(597.000,1009.000)(658.339,1116.413)(600.575,1132.641)
\put(2149.500,232.750){\arc{3270.249}{3.6052}{5.8195}}
\blacken\path(3527.789,1054.601)(3612.000,964.000)(3580.332,1083.571)(3527.789,1054.601)
\put(462,829){\ellipse{450}{450}}
\put(3837,829){\ellipse{372}{372}}
\path(687,829)(1362,829)
\blacken\path(1242.000,799.000)(1362.000,829.000)(1242.000,859.000)(1242.000,799.000)
\path(1812,829)(2487,829)
\blacken\path(2367.000,799.000)(2487.000,829.000)(2367.000,859.000)(2367.000,799.000)
\path(2937,829)(3612,829)
\blacken\path(3492.000,799.000)(3612.000,829.000)(3492.000,859.000)(3492.000,799.000)
\path(462,604)(462,379)(2712,379)(2712,604)
\blacken\path(2742.000,484.000)(2712.000,604.000)(2682.000,484.000)(2742.000,484.000)
\path(1587,604)(1587,244)(3837,244)(3837,604)
\blacken\path(3867.000,484.000)(3837.000,604.000)(3807.000,484.000)(3867.000,484.000)
\path(12,829)(237,829)
\blacken\path(117.000,799.000)(237.000,829.000)(117.000,859.000)(117.000,799.000)
\put(462,762){\makebox(0,0)[b]{\smash{{\SetFigFont{9}{10.8}{\familydefault}{\mddefault}{\updefault}}}}}
\put(1587,762){\makebox(0,0)[b]{\smash{{\SetFigFont{9}{10.8}{\familydefault}{\mddefault}{\updefault}}}}}
\put(2712,762){\makebox(0,0)[b]{\smash{{\SetFigFont{9}{10.8}{\familydefault}{\mddefault}{\updefault}}}}}
\put(3837,762){\makebox(0,0)[b]{\smash{{\SetFigFont{9}{10.8}{\familydefault}{\mddefault}{\updefault}}}}}
\put(2712,1369){\makebox(0,0)[b]{\smash{{\SetFigFont{8}{9.6}{\familydefault}{\mddefault}{\updefault}}}}}
\put(3837,1369){\makebox(0,0)[b]{\smash{{\SetFigFont{8}{9.6}{\familydefault}{\mddefault}{\updefault}}}}}
\put(462,1369){\makebox(0,0)[b]{\smash{{\SetFigFont{8}{9.6}{\familydefault}{\mddefault}{\updefault}}}}}
\put(1587,1369){\makebox(0,0)[b]{\smash{{\SetFigFont{8}{9.6}{\familydefault}{\mddefault}{\updefault}}}}}
\put(2262,1954){\makebox(0,0)[b]{\smash{{\SetFigFont{8}{9.6}{\familydefault}{\mddefault}{\updefault}}}}}
\put(2712,64){\makebox(0,0)[b]{\smash{{\SetFigFont{8}{9.6}{\familydefault}{\mddefault}{\updefault}}}}}
\put(1002,199){\makebox(0,0)[b]{\smash{{\SetFigFont{8}{9.6}{\familydefault}{\mddefault}{\updefault}}}}}
\put(1025,874){\makebox(0,0)[b]{\smash{{\SetFigFont{8}{9.6}{\familydefault}{\mddefault}{\updefault}}}}}
\put(2172,874){\makebox(0,0)[b]{\smash{{\SetFigFont{8}{9.6}{\familydefault}{\mddefault}{\updefault}}}}}
\put(3297,874){\makebox(0,0)[b]{\smash{{\SetFigFont{8}{9.6}{\familydefault}{\mddefault}{\updefault}}}}}
\end{picture}
}
 \end{center}
\caption[DFA  with  and ]{DFA  with  and ; the input performing the identity transformation is not shown.}
\label{fig:RTDFA}
\end{figure}


\section{-Trivial Regular Languages}\label{sec:Jtrivial}

For any , we define an equivalence relation  on  as follows. 
For any ,  if any only if for every  with , 
 is a subword of  if and only if  is a subword of .
Let  be any language over . Then  is \emph{piecewise-testable} if there exists  such that, for every ,  implies that . Let  be a DFA. If  is a subset of , a \emph{component} of  restricted to  is a minimal subset  of  such that, for all  and ,  if and only of . A state q of  is \emph{maximal} if  for all . Simon~\cite{Sim75} proved the following characterization of piecewise-testable languages. 

\begin{theorem}[Simon]\label{thm:simon}
Let  be a regular language over , let  be its quotient DFA, and let  be its syntactic monoid. Then the following are equivalent:
\be
\item 
 is piecewise-testable.
\item 
 is partially ordered, and for every non-empty subset  of , each component of  restricted to  has exactly one maximal state.
\item 
 is -trivial. 
\ee
\end{theorem}

Consequently, a regular language is piecewise-testable if and only if it is -trivial. The following characterization of -trivial monoids is due to Saito~\cite{Sai98}. 

\begin{theorem}[Saito]\label{thm:saito}
Let  be a monoid of transformations of . Then the following are equivalent:
\be
\item 
 is -trivial.
\item 
 is a subset of  and  for all . 
\ee
\end{theorem}


\vspace{12pt}
Let  be a  -trivial language with quotient DFA  and syntactic monoid . Since , an upper bound on the cardinality of -trivial submonoids of~ is an upper bound on the syntactic complexity of~. 

\begin{lemma}\label{lem:fixmax} 
If , then
\be
\item 
.
\item 
 implies , where  if and only if .
\ee 
\end{lemma}

\begin{proof} 
1. First, for each , since , we have , and . So . On the other hand, if there exists , then , and . Let ; then  and, for any , . So , which is a contradiction. Hence . 

2. Assume . By definition, we have . Then, by~1, . Furthermore,  if and only if , and if and only if . \qed
\end{proof}



\begin{example}\label{ex:fixmax}
Consider non-decreasing , as shown in Fig.~\ref{fig:fixmax}~(a). The orbit set  has three blocks: , , and . Note that , as expected. 

Let  be another non-decreasing transformation, as shown in Fig.~\ref{fig:fixmax}~(b). The orbit set  has two blocks:  and . Note that  and . \qedb


\begin{figure}[hbt]
\begin{center}
\setlength{\unitlength}{0.00065617in}
\begingroup\makeatletter\ifx\SetFigFont\undefined \gdef\SetFigFont#1#2#3#4#5{\reset@font\fontsize{#1}{#2pt}\fontfamily{#3}\fontseries{#4}\fontshape{#5}\selectfont}\fi\endgroup {\renewcommand{\dashlinestretch}{30}
\begin{picture}(4113,1339)(0,-10)
\put(1577.500,2011.500){\arc{4005.253}{1.0921}{2.0495}}
\blacken\path(2404.842,154.974)(2500.000,234.000)(2378.849,209.052)(2404.842,154.974)
\put(1915.000,508.500){\arc{261.000}{2.3318}{7.0930}}
\blacken\path(2028.393,486.820)(2005.000,414.000)(2054.601,472.221)(2028.393,486.820)
\put(3962.500,508.300){\arc{261.402}{2.3356}{7.0891}}
\blacken\path(4076.335,486.839)(4053.000,414.000)(4102.555,472.261)(4076.335,486.839)
\put(565,324){\ellipse{226}{226}}
\put(1229,330){\ellipse{226}{226}}
\put(1915,324){\ellipse{226}{226}}
\put(2612,324){\ellipse{226}{226}}
\put(3287,324){\ellipse{226}{226}}
\put(3963,324){\ellipse{226}{226}}
\path(1330,324)(1780,324)
\blacken\path(1705.000,309.000)(1780.000,324.000)(1705.000,339.000)(1705.000,309.000)
\path(2725,324)(3175,324)
\blacken\path(3100.000,309.000)(3175.000,324.000)(3100.000,339.000)(3100.000,309.000)
\path(3400,324)(3850,324)
\blacken\path(3775.000,309.000)(3850.000,324.000)(3775.000,339.000)(3775.000,309.000)
\put(1915.000,1183.500){\arc{261.000}{2.3318}{7.0930}}
\blacken\path(2028.393,1161.820)(2005.000,1089.000)(2054.601,1147.221)(2028.393,1161.820)
\put(3962.500,1183.300){\arc{261.402}{2.3356}{7.0891}}
\blacken\path(4076.335,1161.839)(4053.000,1089.000)(4102.555,1147.261)(4076.335,1161.839)
\put(565.000,1183.500){\arc{261.000}{2.3318}{7.0930}}
\blacken\path(678.393,1161.820)(655.000,1089.000)(704.601,1147.221)(678.393,1161.820)
\put(3265.000,2101.500){\arc{2656.525}{1.1147}{2.0269}}
\blacken\path(3788.058,864.131)(3850.000,909.000)(3775.566,891.406)(3788.058,864.131)
\put(565,999){\ellipse{226}{226}}
\put(1229,1005){\ellipse{226}{226}}
\put(1915,999){\ellipse{226}{226}}
\put(2612,999){\ellipse{226}{226}}
\put(3287,999){\ellipse{226}{226}}
\put(3963,999){\ellipse{226}{226}}
\path(1330,999)(1780,999)
\blacken\path(1705.000,984.000)(1780.000,999.000)(1705.000,1014.000)(1705.000,984.000)
\path(3400,999)(3850,999)
\blacken\path(3775.000,984.000)(3850.000,999.000)(3775.000,1014.000)(3775.000,984.000)
\put(565,279){\makebox(0,0)[b]{\smash{{\SetFigFont{7}{8.4}{\familydefault}{\mddefault}{\updefault}}}}}
\put(1227,279){\makebox(0,0)[b]{\smash{{\SetFigFont{7}{8.4}{\familydefault}{\mddefault}{\updefault}}}}}
\put(1915,279){\makebox(0,0)[b]{\smash{{\SetFigFont{7}{8.4}{\familydefault}{\mddefault}{\updefault}}}}}
\put(2612,279){\makebox(0,0)[b]{\smash{{\SetFigFont{7}{8.4}{\familydefault}{\mddefault}{\updefault}}}}}
\put(3287,279){\makebox(0,0)[b]{\smash{{\SetFigFont{7}{8.4}{\familydefault}{\mddefault}{\updefault}}}}}
\put(3963,279){\makebox(0,0)[b]{\smash{{\SetFigFont{7}{8.4}{\familydefault}{\mddefault}{\updefault}}}}}
\put(115,279){\makebox(0,0)[b]{\smash{{\SetFigFont{7}{8.4}{\familydefault}{\mddefault}{\updefault}(b)}}}}
\put(115,954){\makebox(0,0)[b]{\smash{{\SetFigFont{7}{8.4}{\familydefault}{\mddefault}{\updefault}(a)}}}}
\put(565,954){\makebox(0,0)[b]{\smash{{\SetFigFont{7}{8.4}{\familydefault}{\mddefault}{\updefault}}}}}
\put(1227,954){\makebox(0,0)[b]{\smash{{\SetFigFont{7}{8.4}{\familydefault}{\mddefault}{\updefault}}}}}
\put(1915,954){\makebox(0,0)[b]{\smash{{\SetFigFont{7}{8.4}{\familydefault}{\mddefault}{\updefault}}}}}
\put(2612,954){\makebox(0,0)[b]{\smash{{\SetFigFont{7}{8.4}{\familydefault}{\mddefault}{\updefault}}}}}
\put(3287,954){\makebox(0,0)[b]{\smash{{\SetFigFont{7}{8.4}{\familydefault}{\mddefault}{\updefault}}}}}
\put(3963,954){\makebox(0,0)[b]{\smash{{\SetFigFont{7}{8.4}{\familydefault}{\mddefault}{\updefault}}}}}
\end{picture}
}
 \end{center}
\caption{Nondecreasing transformations  and .}
\label{fig:fixmax}
\end{figure}
\end{example}

\newcommand{\trmax}{{t_{\mathrm{max}}}} 

Define the transformation . The subscript ``'' is chosen because  is the maximal element in the lattice . Clearly  and . For any submonoid  of , let  be the smallest monoid containing  and all elements of . 

\begin{lemma}\label{lem:trmax} 
Let  be a -trivial submonoid of . Then
\be 
\item 
 is -trivial.
\item 
Let  be the DFA in which each  defines a distinct transformation in . Then  is minimal. 
\ee 
\end{lemma}

\begin{proof}
1. By Theorem~\ref{thm:saito}, it is enough to prove that for any ,  and . Note that ; so we have . On the other hand, since  and , both  and  are non-decreasing as well. Suppose ; then . Since  is non-decreasing, ; and since  is also non-decreasing, . Hence , and , which implies that  and . Then  and . Similarly,  and . Therefore  is also -trivial. 


2. Suppose  performs the transformation . Each state  can be reached from the initial state 1 by the word , and  accepts the word , while all other states reject . So  is minimal. \qed
\end{proof}


For any -trivial submonoid  of , we denote by  the DFA in Lemma~\ref{lem:trmax}. Then  is the quotient DFA of some -trivial regular language . Next, we have

\begin{lemma}\label{lem:orbitfix} 
Let  be a -trivial submonoid of . For any , 
if , then . 
\end{lemma}

\begin{proof}
Pick any  such that . If , then it is trivial that . Assume , and . By Part~2 of Lemma~\ref{lem:fixmax}, we have  and . Then there exists  such that . Let . We define  as follows. If , then let ; so . Otherwise , and there exists  such that ; let . Now , and since , we have  as well. Note that  in both cases. Consider the DFA  with alphabet , and suppose that  performs  and  performs~. Let  be the DFA  restricted to . Since  and ,  are in the same component  of . However,  and  are two distinct maximal states in , which contradicts Theorem~\ref{thm:simon}. Therefore . \qed
\end{proof}



\begin{example}\label{ex:orbitfix}
To illustrate one usage of Lemma~\ref{lem:orbitfix}, we consider two non-decreasing transformations  and . They have the same set of fixed points . However,  and . By Lemma~\ref{lem:orbitfix},  and  cannot appear together in a -trivial monoid. Indeed, consider any minimal DFA  having at least two inputs  such that  performs  and  performs~. The DFA  of  restricted to the alphabet  is shown in Fig.~\ref{fig:orbitfix}. There is only one component in , but there are two maximal states  and . By Theorem~\ref{thm:simon}, the syntactic monoid of  is not -trivial. \qedb
\end{example}


\begin{figure}[hbt]
\begin{center}
\setlength{\unitlength}{0.00065617in}
\begingroup\makeatletter\ifx\SetFigFont\undefined \gdef\SetFigFont#1#2#3#4#5{\reset@font\fontsize{#1}{#2pt}\fontfamily{#3}\fontseries{#4}\fontshape{#5}\selectfont}\fi\endgroup {\renewcommand{\dashlinestretch}{30}
\begin{picture}(4070,1408)(0,-10)
\put(1587.000,1011.250){\arc{337.500}{2.4981}{6.9267}}
\blacken\path(1725.575,1033.641)(1722.000,910.000)(1783.339,1017.413)(1725.575,1033.641)
\put(1587,730){\ellipse{450}{450}}
\put(3837.000,1011.250){\arc{337.500}{2.4981}{6.9267}}
\blacken\path(3975.575,1033.641)(3972.000,910.000)(4033.339,1017.413)(3975.575,1033.641)
\put(3837,730){\ellipse{450}{450}}
\put(1587.000,1855.000){\arc{3150.000}{0.9273}{2.2143}}
\blacken\path(2450.498,501.955)(2532.000,595.000)(2416.300,551.256)(2450.498,501.955)
\put(462,730){\ellipse{450}{450}}
\put(2712,730){\ellipse{450}{450}}
\path(2937,730)(3612,730)
\blacken\path(3492.000,700.000)(3612.000,730.000)(3492.000,760.000)(3492.000,700.000)
\path(12,730)(237,730)
\blacken\path(117.000,700.000)(237.000,730.000)(117.000,760.000)(117.000,700.000)
\path(687,730)(1362,730)
\blacken\path(1242.000,700.000)(1362.000,730.000)(1242.000,760.000)(1242.000,700.000)
\put(462,663){\makebox(0,0)[b]{\smash{{\SetFigFont{9}{10.8}{\familydefault}{\mddefault}{\updefault}}}}}
\put(1587,663){\makebox(0,0)[b]{\smash{{\SetFigFont{9}{10.8}{\familydefault}{\mddefault}{\updefault}}}}}
\put(2712,663){\makebox(0,0)[b]{\smash{{\SetFigFont{9}{10.8}{\familydefault}{\mddefault}{\updefault}}}}}
\put(3837,663){\makebox(0,0)[b]{\smash{{\SetFigFont{9}{10.8}{\familydefault}{\mddefault}{\updefault}}}}}
\put(1025,775){\makebox(0,0)[b]{\smash{{\SetFigFont{8}{9.6}{\familydefault}{\mddefault}{\updefault}}}}}
\put(1587,55){\makebox(0,0)[b]{\smash{{\SetFigFont{8}{9.6}{\familydefault}{\mddefault}{\updefault}}}}}
\put(1587,1270){\makebox(0,0)[b]{\smash{{\SetFigFont{8}{9.6}{\familydefault}{\mddefault}{\updefault}}}}}
\put(3837,1270){\makebox(0,0)[b]{\smash{{\SetFigFont{8}{9.6}{\familydefault}{\mddefault}{\updefault}}}}}
\put(3297,775){\makebox(0,0)[b]{\smash{{\SetFigFont{8}{9.6}{\familydefault}{\mddefault}{\updefault}}}}}
\end{picture}
}
 \end{center}
\caption{DFA  with two inputs  and , where  and .}
\label{fig:orbitfix}
\end{figure}


Let  be any partition of . A block  of  is \emph{trivial} if it contains only one element of ; otherwise it is \emph{non-trivial}. We define the set . Then 

\begin{lemma}\label{lem:Ecard} 
If  is a partition of  with  blocks, where , then . Moreover, the equality holds if and only if  has exactly one non-trivial block.
\end{lemma}

\begin{proof}
Suppose , and  for each , . Without loss of generality, we can rearrange blocks  so that . Let  be any transformation. Then , and hence . Consider each block , and suppose  with . Since , we have  and . On the other hand, if , then , and since , we have ; since , . So there are  different , and there are  different transformations  in . 

Clearly, if , then  and . Assume . Note that  for all , , and . If , then , and . Otherwise, let  be the smallest index such that . For all , , since , we have . Then

Therefore the lemma holds. \qed
\end{proof}

\begin{example}\label{ex:Ecard}
Suppose , , and consider the partition , where , , and . Then , , and . Let  be an arbitrary transformation; then . For any , if , then  could be  or ; otherwise  or , and  must be . So there are  different . Similarly, there are  different  and  different . So . 

Consider another partition  with three blocks, where , , and . Now , , and . We have . Then, for any ,  as well. Since , both  and  are unique. There are  different . Together we have , which is the upper bound in Lemma~\ref{lem:Ecard} for  and . \qedb
\end{example}

Note that, for any , we have . Let  be the set of all subsets  of  such that . Then we obtain the following upper bound. 

\begin{proposition}\label{prop:Jbound} 
For , if  is a -trivial submonoid of , then  
\end{proposition}


\begin{proof} 
Assume  is a -trivial submonoid of . For any , let . Then , and for any  with , . 


Pick any . By Lemma~\ref{lem:orbitfix}, for any , since , we have  for some partition . Then . Suppose . By Lemma~\ref{lem:Ecard}, . Since , ; and since there are  different , we have 
 
The last equality is a well-known  identity in combinatorics. \qed
\end{proof}

The above upper bound is met by the following monoid . For any , suppose  such that  for some ; then we define partition  if , and  otherwise. Let 


\begin{example}\label{ex:cS} 
Suppose ; then . First consider . By definition, . There is only one transformation  in . If , then . There are two transformations  and  in . Table~\ref{tab:cS} summarizes the number of transformations in  for each . Note that the set  contains  transformations in total. 
\qedb


\begin{table}[hbt]
\caption{Number of transformations in  for each .}
\label{tab:cS}
\begin{center}

\end{center}
\end{table}
\end{example}





\begin{proposition}\label{prop:cS}
For , the set  is a -trivial submonoid of  with cardinality 

\end{proposition}




\begin{proof}
First we prove the following claim: 

\medskip

{\bf Claim:} For any ,  for some . 

Let  and  for some . Suppose  for any . Then there exists a block  such that  and . Suppose  with . We must have  or ; otherwise  and , which implies . However, in either case, there exists large  such that  or , respectively. Then , a contradiction. So the claim holds. \qedb 

\medskip

By the claim, for any , since  for some , . Hence  is a submonoid of . 

Next we show that  is -trivial. Pick any , and suppose  and  for some . Suppose , for some . Then we have , where  and . On the other hand, by the claim,  for some , where  and . Note that, since , . Then  and . Hence . By Theorem~\ref{thm:saito},  is -trivial. 

For any  with , where , we have  with  for , and . By Lemma~\ref{lem:Ecard}, . Moreover, if , then . Since  is fixed, there are  different . Therefore . \qed
\end{proof}

We now define a generating set of the monoid . 
Suppose . For any , if , then let . Otherwise, let , and let  be a transformation of  defined by: For all , 

Let . 




\begin{example}\label{ex:genJT}
Suppose . As the first example, consider . Then , and . If , then  and . If , then  and . The set  contains the following  transformations: 
 
\qedb
\end{example}


\begin{proposition}\label{prop:genJT}
For , the monoid  can be generated by the set  of  transformations of . 
\end{proposition}


\begin{proof}
First, for any , where , we have ; hence . So  and . 

Fix arbitrary , and suppose . If , then  and . Assume  in the following. Let  be the only non-trivial block in . Note that  and . For any , since  and , ; and since  is an orbit of , . We prove by induction on  that . 
\be
\item If , then . So , and . 

\item Otherwise  for some  and . Assume that, for any  with , we have . Then , and . For any , since  is an orbit of  and ,  must have the form  where  for . 
Let  such that  if and only if . Suppose  and , where , ,  and . Let , , and . Note that  for , and  for . Also note that , and since  for all , we have  for . By assumption, . Now

Thus  and . 
\ee
By induction, . Therefore . Since there are  different , there are  transformations in~. \qed
\end{proof}


\begin{example}\label{ex:GS}
Suppose . The list of all transformations in  is shown in Example~\ref{ex:genJT}. 
Consider , and . The transition graph of  is shown in Fig.~\ref{fig:genJT}~(a). As in Proposition~\ref{prop:genJT}, we have  and . To show that , we find  and . Then let , , and . We assume that ; in fact,  for  in this example. The transition graphs of , , and  are shown in Fig.~\ref{fig:genJT}~(b), (c), and~(d), respectively. One can verify that , and hence . 
\qedb

\begin{figure}[hbt]
\begin{center}
\setlength{\unitlength}{0.00043745in}
\begingroup\makeatletter\ifx\SetFigFont\undefined \gdef\SetFigFont#1#2#3#4#5{\reset@font\fontsize{#1}{#2pt}\fontfamily{#3}\fontseries{#4}\fontshape{#5}\selectfont}\fi\endgroup {\renewcommand{\dashlinestretch}{30}
\begin{picture}(5378,4355)(0,-10)
\put(645,3881){\ellipse{450}{450}}
\put(645,2756){\ellipse{450}{450}}
\put(645,1631){\ellipse{450}{450}}
\put(645,506){\ellipse{450}{450}}
\put(2895.000,4162.250){\arc{337.500}{2.4981}{6.9267}}
\blacken\path(3033.575,4184.641)(3030.000,4061.000)(3091.339,4168.413)(3033.575,4184.641)
\put(5145.000,4162.250){\arc{337.500}{2.4981}{6.9267}}
\blacken\path(5283.575,4184.641)(5280.000,4061.000)(5341.339,4168.413)(5283.575,4184.641)
\put(2895.000,3037.250){\arc{337.500}{2.4981}{6.9267}}
\blacken\path(3033.575,3059.641)(3030.000,2936.000)(3091.339,3043.413)(3033.575,3059.641)
\put(5145.000,3037.250){\arc{337.500}{2.4981}{6.9267}}
\blacken\path(5283.575,3059.641)(5280.000,2936.000)(5341.339,3043.413)(5283.575,3059.641)
\put(2895.000,1912.250){\arc{337.500}{2.4981}{6.9267}}
\blacken\path(3033.575,1934.641)(3030.000,1811.000)(3091.339,1918.413)(3033.575,1934.641)
\put(5145.000,1912.250){\arc{337.500}{2.4981}{6.9267}}
\blacken\path(5283.575,1934.641)(5280.000,1811.000)(5341.339,1918.413)(5283.575,1934.641)
\put(645.000,1912.250){\arc{337.500}{2.4981}{6.9267}}
\blacken\path(783.575,1934.641)(780.000,1811.000)(841.339,1918.413)(783.575,1934.641)
\put(2895.000,5766.500){\arc{4581.000}{1.1238}{2.0177}}
\blacken\path(3787.784,3624.521)(3885.000,3701.000)(3763.231,3679.267)(3787.784,3624.521)
\put(2895.000,3516.500){\arc{4581.000}{1.1238}{2.0177}}
\blacken\path(3787.784,1374.521)(3885.000,1451.000)(3763.231,1429.267)(3787.784,1374.521)
\put(1770.000,3037.250){\arc{337.500}{2.4981}{6.9267}}
\blacken\path(1908.575,3059.641)(1905.000,2936.000)(1966.339,3043.413)(1908.575,3059.641)
\put(2895.000,787.250){\arc{337.500}{2.4981}{6.9267}}
\blacken\path(3033.575,809.641)(3030.000,686.000)(3091.339,793.413)(3033.575,809.641)
\put(5145.000,787.250){\arc{337.500}{2.4981}{6.9267}}
\blacken\path(5283.575,809.641)(5280.000,686.000)(5341.339,793.413)(5283.575,809.641)
\put(4020.000,787.250){\arc{337.500}{2.4981}{6.9267}}
\blacken\path(4158.575,809.641)(4155.000,686.000)(4216.339,793.413)(4158.575,809.641)
\put(3435.000,3228.500){\arc{6440.662}{1.0915}{2.0501}}
\blacken\path(4825.906,290.710)(4920.000,371.000)(4799.193,344.436)(4825.906,290.710)
\put(645.000,3037.250){\arc{337.500}{2.4981}{6.9267}}
\blacken\path(783.575,3059.641)(780.000,2936.000)(841.339,3043.413)(783.575,3059.641)
\put(2895,3881){\ellipse{450}{450}}
\put(5145,3881){\ellipse{450}{450}}
\put(1770,3881){\ellipse{450}{450}}
\put(4020,3881){\ellipse{450}{450}}
\put(2895,2756){\ellipse{450}{450}}
\put(5145,2756){\ellipse{450}{450}}
\put(1770,2756){\ellipse{450}{450}}
\put(4020,2756){\ellipse{450}{450}}
\put(2895,1631){\ellipse{450}{450}}
\put(5145,1631){\ellipse{450}{450}}
\put(1770,1631){\ellipse{450}{450}}
\put(4020,1631){\ellipse{450}{450}}
\put(2895,506){\ellipse{450}{450}}
\put(5145,506){\ellipse{450}{450}}
\put(1770,506){\ellipse{450}{450}}
\put(4020,506){\ellipse{450}{450}}
\path(870,3881)(1545,3881)
\blacken\path(1425.000,3851.000)(1545.000,3881.000)(1425.000,3911.000)(1425.000,3851.000)
\path(4245,3881)(4920,3881)
\blacken\path(4800.000,3851.000)(4920.000,3881.000)(4800.000,3911.000)(4800.000,3851.000)
\path(4245,2756)(4920,2756)
\blacken\path(4800.000,2726.000)(4920.000,2756.000)(4800.000,2786.000)(4800.000,2726.000)
\path(4245,1631)(4920,1631)
\blacken\path(4800.000,1601.000)(4920.000,1631.000)(4800.000,1661.000)(4800.000,1601.000)
\path(870,506)(1545,506)
\blacken\path(1425.000,476.000)(1545.000,506.000)(1425.000,536.000)(1425.000,476.000)
\put(645,3814){\makebox(0,0)[b]{\smash{{\SetFigFont{6}{7.2}{\familydefault}{\mddefault}{\updefault}}}}}
\put(645,2689){\makebox(0,0)[b]{\smash{{\SetFigFont{6}{7.2}{\familydefault}{\mddefault}{\updefault}}}}}
\put(645,1564){\makebox(0,0)[b]{\smash{{\SetFigFont{6}{7.2}{\familydefault}{\mddefault}{\updefault}}}}}
\put(645,439){\makebox(0,0)[b]{\smash{{\SetFigFont{6}{7.2}{\familydefault}{\mddefault}{\updefault}}}}}
\put(1770,3814){\makebox(0,0)[b]{\smash{{\SetFigFont{6}{7.2}{\familydefault}{\mddefault}{\updefault}}}}}
\put(2895,3814){\makebox(0,0)[b]{\smash{{\SetFigFont{6}{7.2}{\familydefault}{\mddefault}{\updefault}}}}}
\put(4020,3814){\makebox(0,0)[b]{\smash{{\SetFigFont{6}{7.2}{\familydefault}{\mddefault}{\updefault}}}}}
\put(5145,3814){\makebox(0,0)[b]{\smash{{\SetFigFont{6}{7.2}{\familydefault}{\mddefault}{\updefault}}}}}
\put(15,3836){\makebox(0,0)[lb]{\smash{{\SetFigFont{6}{7.2}{\rmdefault}{\mddefault}{\updefault}(a)}}}}
\put(1770,2689){\makebox(0,0)[b]{\smash{{\SetFigFont{6}{7.2}{\familydefault}{\mddefault}{\updefault}}}}}
\put(2895,2689){\makebox(0,0)[b]{\smash{{\SetFigFont{6}{7.2}{\familydefault}{\mddefault}{\updefault}}}}}
\put(4020,2689){\makebox(0,0)[b]{\smash{{\SetFigFont{6}{7.2}{\familydefault}{\mddefault}{\updefault}}}}}
\put(5145,2689){\makebox(0,0)[b]{\smash{{\SetFigFont{6}{7.2}{\familydefault}{\mddefault}{\updefault}}}}}
\put(1770,1564){\makebox(0,0)[b]{\smash{{\SetFigFont{6}{7.2}{\familydefault}{\mddefault}{\updefault}}}}}
\put(2895,1564){\makebox(0,0)[b]{\smash{{\SetFigFont{6}{7.2}{\familydefault}{\mddefault}{\updefault}}}}}
\put(4020,1564){\makebox(0,0)[b]{\smash{{\SetFigFont{6}{7.2}{\familydefault}{\mddefault}{\updefault}}}}}
\put(5145,1564){\makebox(0,0)[b]{\smash{{\SetFigFont{6}{7.2}{\familydefault}{\mddefault}{\updefault}}}}}
\put(15,2711){\makebox(0,0)[lb]{\smash{{\SetFigFont{6}{7.2}{\rmdefault}{\mddefault}{\updefault}(b)}}}}
\put(15,1586){\makebox(0,0)[lb]{\smash{{\SetFigFont{6}{7.2}{\rmdefault}{\mddefault}{\updefault}(c)}}}}
\put(1770,439){\makebox(0,0)[b]{\smash{{\SetFigFont{6}{7.2}{\familydefault}{\mddefault}{\updefault}}}}}
\put(2895,439){\makebox(0,0)[b]{\smash{{\SetFigFont{6}{7.2}{\familydefault}{\mddefault}{\updefault}}}}}
\put(4020,439){\makebox(0,0)[b]{\smash{{\SetFigFont{6}{7.2}{\familydefault}{\mddefault}{\updefault}}}}}
\put(5145,439){\makebox(0,0)[b]{\smash{{\SetFigFont{6}{7.2}{\familydefault}{\mddefault}{\updefault}}}}}
\put(15,461){\makebox(0,0)[lb]{\smash{{\SetFigFont{6}{7.2}{\rmdefault}{\mddefault}{\updefault}(d)}}}}
\end{picture}
}
 \end{center}
\caption[Transition graphs of , , and .]{Transition graphs of , , and .}
\label{fig:genJT}
\end{figure}
\end{example}


Now, by Propositions~\ref{prop:Jbound},~\ref{prop:cS}, and~\ref{prop:genJT}, we have

\begin{theorem}\label{thm:Jtrivial} 
Let  be a -trivial regular language with quotient complexity . Then its syntactic complexity  satisfies , and this bound is tight if . 
\end{theorem}

\begin{remark}
It was shown by Saito~\cite{Sai98} that, if  is a -trivial submonoid of~, then  forms a -semilattice, called a \emph{--semilattice}, such that . Let  be the set of all --semilattices that are subsets of . A maximal -trivial submonoid  of  corresponds to a maximal element  in , with respect to set inclusion, such that .  is called \emph{full} if , which is an maximal element in  with respect to set inclusion. The monoid  then corresponds to a full --semilattice, and hence it is maximal. Saito described all maximal -trivial submonoid of  and those corresponding to full --semilattices. However, here we consider the -trivial submonoid of  with maximum cardinality. 
\end{remark}

\begin{remark}
The number  also appears in the paper of Brzozowski and Liu~\cite{BrLiu12} as a lower bound and the conjectured upper bound for the syntactic complexity of definite languages. However, the semigroup  with this cardinality in~\cite{BrLiu12} for definite languages is not isomorphic to~, since  is not -trivial. 
\end{remark}

\section{Quotient Complexity of the Reversal of - and -Trivial Regular Languages}\label{sec:rev}

In this section we consider \emph{nondeterministic finite automata} (NFA's). An NFA  is a quintuple , where , , and  are as in a DFA,  is the nondeterministic transition function, and  is the set of initial states. For any word , the \emph{reverse} of  is defined inductively as follows:  if , and  if  for some  and . The \emph{reverse} of any language  is the language . For any finite automaton (DFA or NFA) , we let  denote the NFA obtained by reversing all the transitions of  and exchanging the roles of initial and final states, and by , the DFA obtained by applying the subset construction to~  keeping only the reachable subsets. Then , and . To simplify our proofs, we use an observation from~\cite{Brz62} that, for any NFA  without empty states, if the automaton  is deterministic, then the DFA  is minimal. 

In~2004, Salomaa, Wood, and Yu~\cite{SWY04} showed that if a regular language  has quotient complexity  and syntactic complexity , then its reverse language  has quotient complexity~, which is maximal for regular languages. As shown in~\cite{BrYe11} and~\cite{BLY12}, for certain regular languages with maximal syntactic complexity in their subclasses, the reverse languages have maximal quotient complexity. This also holds for - and -trivial regular languages. 

It was proved by Jir{\'a}skov{\'a} and Masopust~\cite{JiMa12} that, if  is an -trivial language with  quotients, then  is a tight upper bound on the quotient complexity of , and this bound can be met if  is a ternary language. Note that the syntactic semigroup of any -trivial language is a subset of  for some set . Hence the upper bound  on  can also be reached if  has  quotients with maximal syntactic complexity . 


For -trivial languages , it was conjectured by Masopust\footnote{Personal communication} that, if  has  quotients, then the upper bound  on the quotient complexity of  can be reached using  letters. This conjecture holds:


\begin{theorem}\label{thm:Jrev}
For , if  is a regular -trivial language with quotient complexity , then . Moreover, this bound can be met by a language  over an alphabet of size .
\end{theorem}

\begin{proof}
Since any -trivial regular language is also -trivial, the upper bound  also holds for -trivial regular languages. 

To see that the bound is tight, consider the DFA  such that , , where each  defines the following transformation of : 

The DFA  is minimal since, for each , state  can be reached by , and the word  is only accepted by state . Let . Then . 

Let  be an NFA accepting , which contains no unreachable states. The NFA  is shown in Fig.~\ref{fig:JTrev}. Let  be any subset of  containing~. If , then it is the initial set of states of~. Otherwise, suppose , where  and . Let  be a transformation of . Then, for any ,  if and only if . Since , there exists a word  such that  performs the transformation~, \ie, . This means that, for any ,  if and only if . Hence we can reach the set  of states of  from the initial set of states by the word . Since there are  distinct subsets  of  containing , there are  reachable states in . 

\begin{figure}[hbt]
\begin{center}
\setlength{\unitlength}{0.00087489in}
\begingroup\makeatletter\ifx\SetFigFont\undefined \gdef\SetFigFont#1#2#3#4#5{\reset@font\fontsize{#1}{#2pt}\fontfamily{#3}\fontseries{#4}\fontshape{#5}\selectfont}\fi\endgroup {\renewcommand{\dashlinestretch}{30}
\begin{picture}(4582,2391)(0,-10)
\put(1558,1587){\ellipse{382}{382}}
\put(2988,1592){\ellipse{382}{382}}
\put(4383,1592){\ellipse{382}{382}}
\put(2256,507){\ellipse{382}{382}}
\put(199,1587){\ellipse{382}{382}}
\put(199,1587){\ellipse{314}{314}}
\blacken\path(499.000,1617.000)(379.000,1587.000)(499.000,1557.000)(499.000,1617.000)
\path(379,1587)(1369,1587)
\blacken\path(1894.000,1617.000)(1774.000,1587.000)(1894.000,1557.000)(1894.000,1617.000)
\path(1774,1587)(2781,1587)
\blacken\path(3311.000,1617.000)(3191.000,1587.000)(3311.000,1557.000)(3311.000,1617.000)
\path(3191,1587)(4172,1587)
\blacken\path(455.018,1426.413)(334.000,1452.000)(428.740,1372.474)(455.018,1426.413)
\path(334,1452)(2089,597)
\blacken\path(1731.705,1325.111)(1639.000,1407.000)(1682.262,1291.119)(1731.705,1325.111)
\path(1639,1407)(2134,687)
\blacken\path(2855.738,1291.119)(2899.000,1407.000)(2806.295,1325.111)(2855.738,1291.119)
\path(2899,1407)(2404,687)
\blacken\path(4153.478,1373.415)(4249.000,1452.000)(4127.735,1427.612)(4153.478,1373.415)
\path(4249,1452)(2449,597)
\path(1774,462)(2075,462)
\blacken\path(1955.000,432.000)(2075.000,462.000)(1955.000,492.000)(1955.000,432.000)
\path(2899,1767)(2898,1770)(2895,1776)
	(2891,1786)(2885,1800)(2878,1817)
	(2871,1836)(2864,1856)(2858,1877)
	(2853,1899)(2849,1922)(2848,1945)
	(2849,1969)(2854,1992)(2861,2010)
	(2869,2026)(2877,2038)(2884,2047)
	(2891,2053)(2896,2058)(2902,2061)
	(2907,2063)(2912,2065)(2918,2067)
	(2925,2069)(2933,2072)(2944,2075)
	(2957,2078)(2972,2081)(2989,2082)
	(3006,2081)(3021,2078)(3034,2075)
	(3045,2072)(3053,2069)(3060,2067)
	(3066,2065)(3072,2063)(3076,2061)
	(3082,2058)(3087,2053)(3094,2047)
	(3101,2038)(3109,2026)(3117,2010)
	(3124,1992)(3129,1969)(3130,1945)
	(3129,1922)(3125,1899)(3120,1877)
	(3114,1856)(3107,1836)(3100,1817)
	(3093,1800)(3079,1767)
\blacken\path(3098.249,1889.186)(3079.000,1767.000)(3153.483,1865.753)(3098.249,1889.186)
\path(1459,1767)(1458,1770)(1455,1776)
	(1451,1786)(1445,1800)(1438,1817)
	(1431,1836)(1424,1856)(1418,1877)
	(1413,1899)(1409,1922)(1408,1945)
	(1409,1969)(1414,1992)(1421,2010)
	(1429,2026)(1437,2038)(1444,2047)
	(1451,2053)(1456,2058)(1462,2061)
	(1467,2063)(1472,2065)(1478,2067)
	(1485,2069)(1493,2072)(1504,2075)
	(1517,2078)(1532,2081)(1549,2082)
	(1566,2081)(1581,2078)(1594,2075)
	(1605,2072)(1613,2069)(1620,2067)
	(1626,2065)(1632,2063)(1636,2061)
	(1642,2058)(1647,2053)(1654,2047)
	(1661,2038)(1669,2026)(1677,2010)
	(1684,1992)(1689,1969)(1690,1945)
	(1689,1922)(1685,1899)(1680,1877)
	(1674,1856)(1667,1836)(1660,1817)
	(1653,1800)(1639,1767)
\blacken\path(1658.249,1889.186)(1639.000,1767.000)(1713.483,1865.753)(1658.249,1889.186)
\path(4294,1767)(4293,1770)(4290,1776)
	(4286,1786)(4280,1800)(4273,1817)
	(4266,1836)(4259,1856)(4253,1877)
	(4248,1899)(4244,1922)(4243,1945)
	(4244,1969)(4249,1992)(4256,2010)
	(4264,2026)(4272,2038)(4279,2047)
	(4286,2053)(4291,2058)(4297,2061)
	(4302,2063)(4307,2065)(4313,2067)
	(4320,2069)(4328,2072)(4339,2075)
	(4352,2078)(4367,2081)(4384,2082)
	(4401,2081)(4416,2078)(4429,2075)
	(4440,2072)(4448,2069)(4455,2067)
	(4461,2065)(4467,2063)(4471,2061)
	(4477,2058)(4482,2053)(4489,2047)
	(4496,2038)(4504,2026)(4512,2010)
	(4519,1992)(4524,1969)(4525,1945)
	(4524,1922)(4520,1899)(4515,1877)
	(4509,1856)(4502,1836)(4495,1817)
	(4488,1800)(4474,1767)
\blacken\path(4493.249,1889.186)(4474.000,1767.000)(4548.483,1865.753)(4493.249,1889.186)
\path(2179,327)(2178,324)(2175,318)
	(2171,308)(2165,294)(2158,277)
	(2151,258)(2144,238)(2138,217)
	(2133,195)(2129,172)(2128,149)
	(2129,125)(2134,102)(2141,84)
	(2149,68)(2157,56)(2164,47)
	(2171,41)(2176,36)(2182,33)
	(2187,31)(2192,29)(2198,27)
	(2205,25)(2213,22)(2224,19)
	(2237,16)(2252,13)(2269,12)
	(2286,13)(2301,16)(2314,19)
	(2325,22)(2333,25)(2340,27)
	(2346,29)(2352,31)(2356,33)
	(2362,36)(2367,41)(2374,47)
	(2381,56)(2389,68)(2397,84)
	(2404,102)(2409,125)(2410,149)
	(2409,172)(2405,195)(2400,217)
	(2394,238)(2387,258)(2380,277)
	(2373,294)(2359,327)
\blacken\path(2433.483,228.247)(2359.000,327.000)(2378.249,204.814)(2433.483,228.247)
\put(1572,1519){\makebox(0,0)[b]{\smash{{\SetFigFont{12}{14.4}{\familydefault}{\mddefault}{\updefault}2}}}}
\put(2989,1519){\makebox(0,0)[b]{\smash{{\SetFigFont{12}{14.4}{\familydefault}{\mddefault}{\updefault}3}}}}
\put(4384,1519){\makebox(0,0)[b]{\smash{{\SetFigFont{12}{14.4}{\familydefault}{\mddefault}{\updefault}4}}}}
\put(1009,912){\makebox(0,0)[b]{\smash{{\SetFigFont{12}{14.4}{\familydefault}{\mddefault}{\updefault}}}}}
\put(1999,1137){\makebox(0,0)[b]{\smash{{\SetFigFont{12}{14.4}{\familydefault}{\mddefault}{\updefault}}}}}
\put(2899,1137){\makebox(0,0)[b]{\smash{{\SetFigFont{12}{14.4}{\familydefault}{\mddefault}{\updefault}}}}}
\put(3529,912){\makebox(0,0)[b]{\smash{{\SetFigFont{12}{14.4}{\familydefault}{\mddefault}{\updefault}}}}}
\put(2256,439){\makebox(0,0)[b]{\smash{{\SetFigFont{12}{14.4}{\familydefault}{\mddefault}{\updefault}5}}}}
\put(199,1519){\makebox(0,0)[b]{\smash{{\SetFigFont{12}{14.4}{\familydefault}{\mddefault}{\updefault}1}}}}
\put(1999,102){\makebox(0,0)[b]{\smash{{\SetFigFont{12}{14.4}{\familydefault}{\mddefault}{\updefault}}}}}
\put(1549,2217){\makebox(0,0)[b]{\smash{{\SetFigFont{12}{14.4}{\familydefault}{\mddefault}{\updefault}}}}}
\put(2989,2217){\makebox(0,0)[b]{\smash{{\SetFigFont{12}{14.4}{\familydefault}{\mddefault}{\updefault}}}}}
\put(4429,2217){\makebox(0,0)[b]{\smash{{\SetFigFont{12}{14.4}{\familydefault}{\mddefault}{\updefault}}}}}
\put(874,1677){\makebox(0,0)[b]{\smash{{\SetFigFont{12}{14.4}{\familydefault}{\mddefault}{\updefault}}}}}
\put(2224,1677){\makebox(0,0)[b]{\smash{{\SetFigFont{12}{14.4}{\familydefault}{\mddefault}{\updefault}}}}}
\put(3619,1677){\makebox(0,0)[b]{\smash{{\SetFigFont{12}{14.4}{\familydefault}{\mddefault}{\updefault}}}}}
\end{picture}
}
 \end{center}
\caption{NFA  for  accepting .}
\label{fig:JTrev}
\end{figure}

Note that there is no unreachable state in . Then the DFA  is minimal, and . This shows that the upper bound  is tight for reversal of -trivial regular languages. \qed
\end{proof}


Consider again the above DFA . The orbit of each transformation  is ; this is exactly the partition  for . So  by definition. Then the transition semigroup of  is a subsemigroup of . It follows that, if a -trivial language  has  quotients and syntactic semigroup , then its reverse  has the maximal quotient complexity. 


\section{Conclusion}\label{sec:con}

We proved that  and  are the tight upper bounds on the syntactic complexities of - and -trivial languages with  quotients, respectively. For , the upper bound for -trivial languages can be met using  letters, and the upper bound for -trivial languages, using  letters. It remains open whether the upper bound for -trivial languages can be met with fewer than  letters. The syntactic complexity of -trivial languages is also open. 
We also observed that, if - and -trivial languages have maximal syntactic complexities, their reverses have maximal quotient complexities. 
The proof of Theorem~\ref{thm:Jrev} can be extended to the following template for languages  in some subclass  of regular languages: Suppose  is the minimal DFA of . To prove , where  is an upper bound on  for , one can show that there are at least  distinct subsets  of  such that  can perform a transformation  of  with  if and only if . 

\newpage



\providecommand{\noopsort}[1]{}
\begin{thebibliography}{10}

\bibitem{Brz62}
Brzozowski, J.:
\newblock {Canonical regular expressions and minimal state graphs for definite
  events}.
\newblock In: Mathematical theory of Automata. Volume 12 of MRI Symposia
  Series.
\newblock Polytechnic Press, Polytechnic Institute of Brooklyn, N.Y. (1962)
  529--561

\bibitem{BrFi80}
Brzozowski, J., Fich, F.E.:
\newblock Languages of {}-trivial monoids.
\newblock J. Comput. System Sci. \textbf{20}(1) (1980)  32--49

\bibitem{BL12a}
Brzozowski, J., Li, B.:
\newblock Syntactic complexities of some classes of star-free languages.
\newblock In Kutrib, M., Moreira, N., Reis, R., eds.: DCFS 2012. Volume 7386 of
  LNCS, Springer (2012)  117--129

\bibitem{BLY12}
Brzozowski, J., Li, B., Ye, Y.:
\newblock Syntactic complexity of prefix-, suffix-, bifix-, and factor-free
  regular languages.
\newblock Theoret. Comput. Sci. \textbf{449} (2012)  37--53

\bibitem{BrLiu12}
Brzozowski, J., Liu, D.:
\newblock Syntactic complexity of finite/cofinite, definite, and reverse
  definite languages.
\newblock {\tt http://arxiv.org/abs/1103.2986} (March 2012)

\bibitem{BrYe11}
Brzozowski, J., Ye, Y.:
\newblock Syntactic complexity of ideal and closed languages.
\newblock In Mauri, G., Leporati, A., eds.: DLT 2011. Volume 6795 of LNCS,
  Springer Berlin / Heidelberg (2011)  117--128

\bibitem{GaMa09}
Ganyushkin, O., Mazorchuk, V.:
\newblock Classical Finite Transformation Semigroups: An Introduction.
\newblock Springer (2009)

\bibitem{HoKo04}
Holzer, M., K\"{o}nig, B.:
\newblock On deterministic finite automata and syntactic monoid size.
\newblock Theoret. Comput. Sci. \textbf{327}(3) (2004)  319--347

\bibitem{JiMa12}
Jir{\'a}skov{\'a}, G., Masopust, T.:
\newblock On the state and computational complexity of the reverse of acyclic
  minimal dfas.
\newblock In Moreira, N., Reis, R., eds.: CIAA 2012. Volume 7381 of LNCS,
  Springer (2012)  229--239

\bibitem{KP11}
Kl\'{\i}ma, O., Pol{\'a}k, L.:
\newblock On biautomata.
\newblock In Freund, R., Holzer, M., Mereghetti, C., Otto, F., Palano, B.,
  eds.: Third Workshop on Non-Classical Models for Automata and Applications -
  NCMA 2011, Milan, Italy, July 18--July 19, 2011. Proceedings. Volume 282,
  Austrian Computer Society (2011)  153--164

\bibitem{KLS03}
Krawetz, B., Lawrence, J., Shallit, J.:
\newblock State complexity and the monoid of transformations of a finite set.
\newblock {\tt http://arxiv.org/abs/math/0306416v1} (2003)

\bibitem{Mas70}
Maslov, A.N.:
\newblock Estimates of the number of states of finite automata.
\newblock Dokl. Akad. Nauk SSSR \textbf{194} (1970)  1266--1268 (Russian)
  {English} translation: Soviet Math. Dokl. 11 (1970), 1373--1375.

\bibitem{McNP71}
McNaughton, R., Papert, S.A.:
\newblock Counter-Free Automata. Volume~65 of M.I.T. Research Monographs.
\newblock The MIT Press (1971)

\bibitem{Myh57}
Myhill, J.:
\newblock Finite automata and the representation of events.
\newblock Wright Air Development Center Technical Report \textbf{57--624}
  (1957)

\bibitem{Pin97}
Pin, J.E.:
\newblock Syntactic semigroups.
\newblock In Rozenberg, G., Salomaa, A., eds.: Handbook of Formal Languages,
  vol.~1: Word, Language, Grammar.
\newblock Springer (1997)  679--746

\bibitem{Sai98}
Saito, T.:
\newblock {}-trivial subsemigroups of finite full
  transformation semigroups.
\newblock Semigroup Forum \textbf{57} (1998)  60--68

\bibitem{SWY04}
Salomaa, A., Wood, D., S.Yu:
\newblock On the state complexity of reversals of regular languages.
\newblock Theoret. Comput. Sci. \textbf{320}(2–3) (2004)  315--329

\bibitem{Sim72}
Simon, I.:
\newblock Hierarchies of Events With Dot-Depth One.
\newblock PhD thesis, Dept. of Applied Analysis \& Computer Science, University
  of Waterloo, Waterloo, Ont., Canada (1972)

\bibitem{Sim75}
Simon, I.:
\newblock Piecewise testable events.
\newblock In: Proceedings of the 2nd GI Conference on Automata Theory and
  Formal Languages, London, UK, Springer-Verlag (1975)  214--222

\bibitem{Yu97}
Yu, S.:
\newblock Regular languages.
\newblock In Rozenberg, G., Salomaa, A., eds.: Handbook of Formal Languages,
  vol.~1: Word, Language, Grammar.
\newblock Springer (1997)  41--110

\end{thebibliography}

\end{document}
