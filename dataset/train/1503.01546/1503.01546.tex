\documentclass{sig-alternate-2013}

\usepackage{authblk}

\usepackage{subfig}
\def\etal{{\it et al.}}

\usepackage[hyphens]{url}

\usepackage{graphicx}
\usepackage{epstopdf}
\usepackage{color}
\usepackage{booktabs}
\newcommand\note[2]{{\color{#1}\bf #2}}

\newcommand\HH[1]{{\note{red}{HH: #1}}}
\newcommand\Tassos[1]{{\note{blue}{IW: #1}}}

\newfont{\mycrnotice}{ptmr8t at 7pt}
\newfont{\myconfname}{ptmri8t at 7pt}
\let\crnotice\mycrnotice \let\confname\myconfname 

\permission{Permission to make digital or hard copies of all or part of this work for personal or classroom use is granted without fee provided that copies are not made or distributed for profit or commercial advantage and that copies bear this notice and the full citation on the first page. Copyrights for components of this work owned by others than ACM must be honored. Abstracting with credit is permitted. To copy otherwise, or republish, to post on servers or to redistribute to lists, requires prior specific permission and/or a fee. Request permissions from permissions@acm.org.}
\conferenceinfo{DH'15,}{May 18--20, 2015, Florence, Italy.\\
{\mycrnotice{Copyright is held by the owner/author(s). Publication rights licensed to ACM.}}}
\copyrightetc{ACM \the\acmcopyr}
\crdata{978-1-4503-3492-1/15/05\ ...\55\rho\rhoNN_{pic}N_{pic}p>=100>=500\mu_{likes}N_{pic}^2$ of 0.013, illustrating that without the deeper demographic knowledge, the picture metadata was not sufficient to predict county-wide obesity. 

A distinction -- not present in the previous studies -- we examine here, is one between local and chain restaurants, as empirically determined using the aggregate data. We find that users are almost twice as likely to post a picture at a local restaurant than a chain, and keep posting (which may indicate more frequent visits). We also find hashtags associated with small business like \texttt{\#eatlocal} and \texttt{\#smallbiz} to be associated with lower-obesity areas. The importance of these small restaurants is further underscored by the fact that, unlike large chains, it may be easier to work with them as a community to promote healthier menus, such as \emph{Eat Well! El Paso} program in Texas\footnote{\url{http://www.kvia.com/news/healthier-options-for-children-at-local-restaurants/26415336}} or \emph{Choose Health} in Los Angeles\footnote{\url{http://laist.com/2013/09/13/city_partners_with_local_restaurant.php}}. We also find that the chains which may not be considered as fast food could be popular for social gatherings, including \emph{The Cheesecake Factory}, \emph{Yard House}, and \emph{Dallas BBQ}. Thus, beyond fast food, such chains provide a social setting in which to share food, which may not be the healthiest choice. 

Our foray into the perception of food shows an alarming, overwhelming approval of addictive foods high in sugar and fat. Even though the users associate \texttt{\#foodporn} with potentially healthier cuisines (mainly Asian), the social approval of the images emphasizes the values social media reinforces. More work needs to be done to reveal the motivations behind users' interactions on social media, in order to harness them for the benefit of their dietary health.


\section{Conclusion}
\label{sec:conclusion}

In this paper we have taken a data-driven approach in analyzing food consumption on massive scale using Instagram and Foursquare. We used millions of posts from individuals in restaurants across the United States and correlated these with national statistics. Our analyses revealed a relationship between small businesses and local foods with obesity, with these restaurants getting more attention on these social media. However, the social approval, manifesting itself in the form of likes and comments, often favors the unhealthy dietary choices, favoring donuts, cupcakes, and other sweets.

In our future efforts we plan to use sentiment analysis techniques on picture comments in order to assess the level of approval and disapproval of the individuals' social network. We are also working on image recognition techniques using machine learning methods to enable inference of the content of the pictures and potential translation to caloric values. This will enable an increase in the confidence of our analysis for all locations and improve our ability to understand the actual food consumed.

 	
\section*{Acknowledgments}
We appreciate support from Haewoon Kwak in data collection.

\begin{thebibliography}{10}

\bibitem{abbar2014you}
S.~Abbar, Y.~Mejova, and I.~Weber.
\newblock You tweet what you eat: Studying food consumption through twitter.
\newblock In {\em Proceedings of the SIGCHI Conference on Human Factors in
  Computing Systems}, CHI '15, 2015.

\bibitem{bodicoat2014number}
D.~H. Bodicoat, P.~Carter, A.~Comber, C.~Edwardson, L.~J. Gray, S.~Hill,
  D.~Webb, T.~Yates, M.~J. Davies, and K.~Khunti.
\newblock Is the number of fast-food outlets in the neighbourhood related to
  screen-detected type 2 diabetes mellitus and associated risk factors?
\newblock {\em Public health nutrition}, pages 1--8, 2014.

\bibitem{Culotta:2014:ECH:2556288.2557139}
A.~Culotta.
\newblock Estimating county health statistics with twitter.
\newblock In {\em Proceedings of the SIGCHI Conference on Human Factors in
  Computing Systems}, CHI '14, pages 1335--1344, New York, NY, USA, 2014. ACM.

\bibitem{currie2009effect}
J.~Currie, S.~DellaVigna, E.~Moretti, and V.~Pathania.
\newblock The effect of fast food restaurants on obesity and weight gain.
\newblock Technical report, National Bureau of Economic Research, 2009.

\bibitem{drewnowski1992taste}
A.~Drewnowski, D.~D. Krahn, M.~A. Demitrack, K.~Nairn, and B.~A. Gosnell.
\newblock Taste responses and preferences for sweet high-fat foods: evidence
  for opioid involvement.
\newblock {\em Physiology \& Behavior}, 51(2):371--379, 1992.

\bibitem{InnovativeHealth}
E.~Eggleston and E.~Weitzman.
\newblock Innovative uses of electronic health records and social media for
  public health surveillance.
\newblock {\em Current Diabetes Reports}, 14(3), 2014.

\bibitem{DBLP:journals/corr/FriedSKHB14}
D.~Fried, M.~Surdeanu, S.~G. Kobourov, M.~Hingle, and D.~Bell.
\newblock Analyzing the language of food on social media.
\newblock {\em International Conference on Big Data}, 2014.

\bibitem{Linehan:2010:TST:1753846.1753980}
C.~Linehan, M.~Doughty, S.~Lawson, B.~Kirman, P.~Olivier, and P.~Moynihan.
\newblock Tagliatelle: Social tagging to encourage healthier eating.
\newblock In {\em CHI '10 Extended Abstracts on Human Factors in Computing
  Systems}, CHI EA '10, pages 3331--3336, New York, NY, USA, 2010. ACM.

\bibitem{newman2014implications}
C.~L. Newman, E.~Howlett, and S.~Burton.
\newblock Implications of fast food restaurant concentration for preschool-aged
  childhood obesity.
\newblock {\em Journal of Business Research}, 67(8):1573--1580, 2014.

\bibitem{conf/icwsm/PaulD11}
M.~J. Paul and M.~Dredze.
\newblock You are what you tweet: Analyzing twitter for public health.
\newblock In {\em ICWSM}, 2011.

\bibitem{foodporn}
S.~Rousseau.
\newblock {Food Porn in Media}.
\newblock In P.~B. Thompson and D.~M. Kaplan, editors, {\em Encyclopedia of
  Food and Agricultural Ethics}, pages 1--8. Springer, 2014.

\bibitem{schlosser2012fast}
E.~Schlosser.
\newblock {\em Fast food nation: The dark side of the all-American meal}.
\newblock Houghton Mifflin Harcourt, 2012.

\bibitem{Silva14:youare}
T.~Silva, P.~{Vaz De Melo}, J.~Almeida, M.~Musolesi, and A.~Louriero.
\newblock {You are What you Eat (and Drink): Identifying Cultural Boundaries by
  Analyzing Food \& Drink Habits in Foursquare}.
\newblock In {\em Proceedings of the 8th AAAI International Conference on
  Weblogs and Social Media (ICWSM'14)}, 2014.

\bibitem{Adolescents}
C.~Stevenson, G.~Doherty, J.~Barnett, O.~Muldoon, and K.~Trew.
\newblock Adolescents' views of food and eating: identifying barriers to
  healthy eating.
\newblock {\em Journal of Adolescence}, 30 (3)(3):417--434, 2007.

\bibitem{Takeuchi:2014:USM:2638728.2641330}
T.~Takeuchi, T.~Fujii, K.~Ogawa, T.~Narumi, T.~Tanikawa, and M.~Hirose.
\newblock Using social media to change eating habits without conscious effort.
\newblock In {\em Proceedings of the 2014 ACM International Joint Conference on
  Pervasive and Ubiquitous Computing: Adjunct Publication}, UbiComp '14
  Adjunct, pages 527--535, New York, NY, USA, 2014. ACM.

\bibitem{FoodFetish}
V.~H. Taylor.
\newblock {Food Fetish: Society's Complicated Relationship with Food}.
\newblock {\em Canadian Journal of Diabetes}, 37(2), April 2013.

\bibitem{ventura2014neurobiologic}
T.~Ventura, J.~Santander, R.~Torres, and A.~M. Contreras.
\newblock Neurobiologic basis of craving for carbohydrates.
\newblock {\em Nutrition}, 30(3):252--256, 2014.

\bibitem{wang2007obesity}
Y.~Wang, H.~Liang, L.~Tussing, C.~Braunschweig, B.~Caballero, and B.~Flay.
\newblock Obesity and related risk factors among low socio-economic status
  minority students in chicago.
\newblock {\em Public health nutrition}, 10(09):927--938, 2007.

\end{thebibliography}


\balancecolumns
\end{document}
