






\documentclass[11pt]{article}\usepackage{amsmath}
\usepackage{graphics}
\usepackage{color}
\usepackage{epsfig}
\usepackage{graphicx}\usepackage{amsfonts}\usepackage{amssymb}
\usepackage{amsthm}


\usepackage{setspace}
\usepackage[margin=0.8in]{geometry}
\usepackage{comment}
\setlength{\textheight}{8.75in}
\setlength{\columnsep}{2.0pc}
\setlength{\textwidth}{6.0in}
\setlength{\topmargin}{0.25in}
\setlength{\headheight}{0.0in}
\setlength{\headsep}{0.0in}
\setlength{\oddsidemargin}{0.0in}
\setlength{\evensidemargin}{0.0in}
\setlength{\parindent}{1pc}
\setlength{\parskip}{1ex plus 0.5ex minus 0.2ex}
\newtheorem{theorem}{Theorem}[section]
\newtheorem{conjecture}{Conjecture}[section]
\newtheorem{definition}{Definition}[section]
\newtheorem{claim}{Claim}[section]
\newtheorem{lemma}{Lemma}[section]
\newtheorem{fact}{Fact}[section]
\newtheorem{corollary}{Corollary}[section]
\newtheorem{note}{Note}[section]
\newtheorem{observation}{Observation}[section]
\newtheorem{remark}{Remark}[section]
\newtheorem{proposition}{Proposition}[section]
\newtheorem{example}{Example}[section]


\newenvironment{oneshot}[1]{\@beginlemma{#1}{\unskip}}{\@endlemma}


\newenvironment{claimproof}{{\bf Proof of claim.}}{  \rule{1mm}{3mm}}
\newenvironment{sketchproof}{\vspace*{-.1in}\noindent{\bf Proof sketch.}}{  \rule{2mm}{3mm}}
\newcommand{\esect}[2]{\bigskip \centerline{#1 {\sc #2}}}
\newcommand{\esubsect}[2]{\bigskip \\ #1 {\em #2}}
\newcommand{\set}[1]{ \{ #1 \} }
\newcommand{\A}{ }
\newcommand{\D}{\Delta}
\newcommand{\C}{{\cal C}}
\renewcommand{\P}{{\cal P}}
\renewcommand{\L}{\Lambda}
\renewcommand{\l}{\lambda} 
\newcommand{\eps}{\varepsilon}
\newcommand{\bx}{\bar{x}}
\newcommand{\precs}{ }
\newcommand{\tu}{{\cal T}}
\newcommand{\lbfl}{{\sc Lbfl}}
\newcommand{\cfl}{{\sc Cfl}}
\newcommand{\ufl}{{\sc Ufl}}




\newcommand*{\mylabel}[1]{\refstepcounter{equation}\label{#1}\textrm{(\theequation)}}\newcommand*{\myref}[1]{\textrm{(\ref{#1})}}



\date{}

\allowdisplaybreaks



\setcounter{secnumdepth}{4}




\begin{document}

\title{
Integrality gaps for strengthened LP relaxations of Capacitated
and Lower-Bounded Facility Location\thanks{
This research has been co-financed by the European Union (European
Social Fund -- ESF) and Greek national funds through the Operational
Program ``Education and Lifelong Learning'' of the National Strategic
Reference Framework (NSRF) - Research Funding Program:
``Thalis. Investing in knowledge society through the European Social Fund''.}
}

\author {Stavros G. Kolliopoulos\thanks{Department of Informatics and
Telecommunications, National and Kapodistrian 
University of Athens, Panepistimiopolis Ilissia, Athens
157 84, Greece; (\texttt{www.di.uoa.gr/}\~{\tt sgk}). Part of this
work conducted while visiting the 
  IEOR Department, Columbia University, New York, NY 10027.}   
\and Yannis Moysoglou\thanks{ Department of Informatics and
Telecommunications, National and Kapodistrian 
University of Athens, Panepistimiopolis Ilissia, Athens
157 84, Greece; (\texttt{gmoys@di.uoa.gr}). Partially supported by an
NKUA-ELKE graduate fellowship.} }


\date{July 8, 2013}

\maketitle


\thispagestyle{empty}



\iffalse ------------------------------------------------
\begin{abstract}
The metric uncapacitated facility location problem (\ufl)  enjoys a
special stature in approximation algorithms 
as  a testbed for various techniques, among which  LP-based
methods  have been especially prominent and successful.  
Two generalizations of \ufl\  are  {\em capacitated facility
location (\cfl\/)} and 
{\em lower-bounded facility location (\lbfl\/).}
In the former, every facility has a capacity which is the maximum demand that can be assigned
to it, while in the latter, every open facility is required to serve a given minimum amount of
demand.
Both \cfl\ and \lbfl\ are 
approximable within a constant factor but  
their respective  natural LP relaxations
have an unbounded  integrality gap. 
One could hope that different, less natural relaxations might provide 
better lower bounds.  According to Shmoys and Williamson, the existence of a
relaxation-based algorithm for \cfl\ is one of the top 10  open
problems  in approximation algorithms. 

In this paper we give the first results on this problem. We provide
substantial evidence against the existence of a good LP relaxation for
\cfl\  by showing unbounded
integrality gaps for two families of strengthened formulations. 


The first family we consider  is the 
hierarchy of LPs resulting from 
repeated applications of the lift-and-project Lov\'{a}sz-Schrijver
procedure starting from the standard relaxation. We show that
 the LP
relaxation for \cfl\ resulting after  rounds, where  is
the number of facilities in the instance, has  unbounded integrality
gap. Note that  the Lov\'{a}sz-Schrijver
procedure is known to yield an exact formulation for \cfl\ in at most  rounds. 


We also introduce the family of {\em proper} relaxations 
which generalizes to its logical extreme  the  classic
star relaxation, an equivalent form of the natural LP. In the star relaxation  
each variable corresponds to  a {\em star:} a facility   and a set of
clients  assigned to  
 In a proper LP,   every variable corresponds to a {\em class:} a set  with
an arbitrary number of facilities and
clients and an assignment of each client to a facility in the
set. 
The {\em complexity} of a proper relaxation is the maximum  fraction of the
number of facilities in a feasible integral solution that appear in a class.
We prove the  dichotomy result 
that every proper LP relaxation of \cfl\ 
whose complexity is below   has unbounded integrality gap, while
there exist proper relaxations of complexity  that express the
integral \cfl\  polytope. We show the same  negative result for proper
relaxations of \lbfl. 
\end{abstract}

\fi 





\begin{abstract}
The metric uncapacitated facility location problem (\ufl)  enjoys a
special stature in approximation algorithms 
as  a testbed for various techniques, among which  LP-based
methods  have been especially prominent and successful.  
Two generalizations of \ufl\  are  {\em capacitated facility
location (\cfl\/)} and 
{\em lower-bounded facility location (\lbfl\/).}
In the former, every facility has a capacity which is the maximum demand that can be assigned
to it, while in the latter, every open facility is required to serve a given minimum amount of
demand.
Both \cfl\ and \lbfl\ are 
approximable within a constant factor but  
their respective  natural LP relaxations
have an unbounded  integrality gap. 
One could hope that different, less natural relaxations might provide 
better lower bounds.  According to Shmoys and Williamson, the existence of a
relaxation-based algorithm for \cfl\ is one of the top 10  open
problems  in approximation algorithms. 

In this paper we give the first results on this problem and they are
negative in nature.  We  show  unbounded
integrality gaps for two substantial families of strengthened formulations. 


The first family we consider  is the 
hierarchy of LPs resulting from 
repeated applications of the lift-and-project Lov\'{a}sz-Schrijver
procedure starting from the standard relaxation. We show that
 the LP
relaxation for \cfl\ resulting after  rounds, where  is
the number of facilities in the instance, has  unbounded integrality
gap. Note that  the Lov\'{a}sz-Schrijver
procedure is known to yield an exact formulation for \cfl\ in at most  rounds. 


We also introduce the family of {\em proper} relaxations 
which generalizes to its logical extreme  the  classic
star relaxation, an equivalent form of the natural LP. 
We characterize the behavior  of proper relaxations for both
\lbfl\  and \cfl\ through  a sharp
threshold phenomenon under which the integrality gap drops from unbounded to 1.
\end{abstract}






\clearpage
\setcounter{page}{1}


\section{Introduction}
\label{intro}

Facility location problems have been studied extensively in 
 operations research,
mathematical programming, and theoretical computer science. 
The 
{\em uncapacitated facility location (\ufl)} problem is defined as follows. 
A set  of 
facilities  and a set  of  clients  are given. 
Every client has to be assigned to an opened facility. 
Opening a facility
 incurs a nonnegative cost  while assigning a client  to facility
 incurs a nonnegative connection cost    The goal is to open a subset 
 of the facilities and  assign each
client to an open facility so that  the total cost is minimized. 
Hochbaum gave a greedy  -approximation algorithm
\cite{Hochbaum82}. By a straightforward reduction from Set Cover this
cannot be improved, unless {\sf P = NP} \cite{RazS97}. 

In the {\em metric} \ufl\ 
the connection  costs satisfy the following variant of the triangle inequality:
 for any  and 
The first constant-factor approximation of  was given by Shmoys, 
Tardos and Ardaal \cite{ShmoysTA97}. Over the years,
\ufl\ has served as a prime testbed for several techniques in
the design of approximation algorithms (see, e.g., 
\cite{ShmoysWbook}). Among those techniques LP-based
methods, such as filtering, randomized rounding and the primal-dual method,
 have been particularly prominent and have yielded several improved bounds.
After a long series of papers 
the currently best approximation ratio
for metric \ufl\ is  \cite{Li11}. 
Guha and Khuller \cite{GuhaK99} proved that there is 
no -approximation algorithm for metric \ufl\ with 
 unless  using Feige's hardness result for Set
Cover \cite{Feige98}.
Sviridenko (see \cite{Vygen05}) showed that the lower bound holds
unless {\sf P = NP.}
In this paper we focus on two generalizations of the metric \ufl:  the 
 {\em capacitated facility location (\cfl\/)} and
 the {\em lower-bounded facility location (\lbfl\/)} problems.
To our knowledge  the  lower bound  is the only inapproximability result known
for these two as well. 



\cfl\/ is the  generalization of metric \ufl\ where every facility  has a
capacity   that specifies  the maximum number of clients that may
be
assigned to  In {\em uniform} \cfl\ all facilities have
the same capacity   
  Finding an approximation algorithm for  \cfl\/ that uses a linear programming
lower bound, or even proving a constant integrality gap for an
efficient  LP
relaxation, are notorious open problems. Intriguingly, 
the  natural LP relaxations  have 
an unbounded integrality gap and
the only known  -approximation algorithms are
based on local search.  The currently best ratios 
for the non-uniform  and the uniform case are  \cite{BansalGG12} and   
\cite{AggarwalLBGGJ12} respectively. 
\iffalse ----------------
Approximation algorithms that use LP
lower bounds are known for 
 the {\em soft capacitated} \cfl\  where multiple copies of a facility
may be opened \cite{ShmoysTA97}  
and   the special case 
where all  facilities have  equal opening cost \cite{LeviSS12}. 
-------- \fi 
Compared to local search, relaxations have 
the distinct advantage that they provide, on an instance-by-instance basis,
a  concrete  lower
bound on the optimum. A small  gap between the
optimal integer and fractional solutions  could be exploited
to speed up an exact  computation. From the viewpoint of
approximation, 
comparing the LP optimum against the  solution  output by an
LP-based algorithm establishes 
a  guarantee than is at least as strong as the one established  a priori by  worst-case
analysis.
In contrast, when a  local search algorithm terminates, 
it is not at all clear what the lower
bound is. 
According to Shmoys and Williamson \cite{ShmoysWbook} devising 
a relaxation-based algorithm for \cfl\  is one of
the top  open problems in approximation algorithms. 



 \lbfl\ is in a sense the opposite problem to \cfl\ and 
was introduced independently by Karger and Minkoff \cite{KargerM00} and Guha et
al. \cite{GuhaMM00} in the context of network design problems with buy-at-bulk
features.  In an instance of \lbfl\ every facility  comes with a  lower
bound  
which is the minimum number of clients that must be assigned
 to   if we open it. In {\em uniform} \lbfl\ all the lower bounds
have the same value   \lbfl\ is even less well-understood than \cfl. 
The first true approximation algorithm for the uniform case 
was given in \cite{Svitkina08} with a performance guarantee of
 which has been recently improved  to  \cite{AhmadianS12}. 
Interestingly, the \lbfl\ algorithms
from \cite{Svitkina08,AhmadianS12}  both use a \cfl\ algorithm on a
suitable instance as a subroutine. 






Studying the limits of linear programming  relaxations for 
intractable problems is
an active area of research. 
The inherent challenge in this work is  to characterize collections
of LPs 
for which no explicit description is known. 
The
main  direction is to lower bound 
the size of extended formulations
that express optimal or near-optimal solutions,
or to determine
the  integrality gap of  comprehensive families  of
valid LP relaxations. 
  Yannakakis \cite{Yannakakis91}  proved early on that 
any 
symmetric linear relaxation that expresses the Traveling Salesman
polytope must have exponential size. 
 Recent results lift the symmetry assumption 
\cite{FioriniMPTD12}   and  characterize  the size of LPs that
express approximate solutions  to Clique \cite{BraunFPS12,BravermanM13}. 


A lot of effort has been  devoted to 
understanding the quality of relaxations obtained by an iterative 
lift-and-project procedure. Such procedures define hierarchies of 
successively  stronger relaxations, where  valid inequalities are added at 
each level. After at most  rounds, where  is the number of 
variables, all 
valid inequalities have been  added and thus the integer polytope is
expressed. 
Relevant methods  include those  developed  by Balas et
al. \cite{BalasCC93},  Lov\'{a}sz and Schrijver \cite{LovaszS91} (for
linear and semidefinite programs, denoted respectively LS and LS), 
Sherali and Adams \cite{SheraliA90} (denoted SA),    Lassere  \cite{Lasserre01}
(for semidefinite programs). 
See
\cite{laurent} for a comparative discussion.  
Exploring the
structure of the successive relaxations  in a hierarchy 
is of intrinsic interest in polyhedral
combinatorics. The seminal work of Arora et al.
\cite{AroraBL02,AroraBLT06} introduced the use of 
hierarchies as a model of computation for obtaining  hardness of
approximation results. 
Proving that the integrality 
gap for a problem remains large  after many 
rounds   is an unconditional  guarantee against the  class of
sophisticated relaxations
obtained through the specific procedure. 
Despite the amount of effort, the effect on approximation  of the
 different hierarchies is not well-understood. Vertex Cover is a
  prominent case among the problems studied early on. 
Arora et al. \cite{AroraBLT06} showed that after
 rounds of the LS procedure the
integrality gap for Vertex Cover remains  Schoenebeck at
al. \cite{SchoenebeckTT07} proved that the  gap survives
for  rounds of  LS. The body of 
work on   hierarchies keeps growing, see, e.g., 
\cite{GeorgiouMPT10,FernandezdlVKM07,SchoenebeckTT07b,CharikarMM09,MathieuS09,Tulsiani09,BhaskaraCVGZ12}. Some
of
those results examine  semidefinite relaxations, a direction we do not
 pursue here. 

Investigating the strength  of linear relaxations  is driven by the 
perception of LP-based algorithms as a  powerful paradigm for designing
approximation algorithms. 
Recent work inspired from 
\cite{Raghavendra08} 
explores a complementary direction: how to translate integrality gaps for LPs into {\sf UGC}-based
hardness of approximation results \cite{KumarMTV11}. 


In  recent work,  improved  approximations  were given  for
-median   \cite{LiS13}      and   capacitated   -center
\cite{CyganHK12,AnBS13},  problems closely  related to
facility location.  For both, the improvements are obtained 
by LP-based techniques
that  include preprocessing  of the  instance in  order to  defeat the
known integrality gap. For -median, the authors of \cite{LiS13}
state that their 
-approximation algorithm   can
be converted  to a  rounding algorithm on an 
 -level LP  in the SA hierarchy.  In \cite{AnBS13}
the authors raise as an important question  to understand  the
power  of lift-and-project methods  for capacitated  location
problems, including  whether they 
automatically capture such preprocessing steps. 




\subsection{Our results}
In this paper we give the first characterization  of 
 the    integrality gap  for families of 
 linear relaxations  for metric \cfl\ and \lbfl\  and thus provide the
 first results on the open problem of \cite{ShmoysWbook}. 
We study two substantial families of
strengthened LPs. 
Our derivations make  no time-complexity assumptions 
and are thus unconditional.  We also partially  answer 
the question of \cite{AnBS13} for \cfl:  if there is an efficient relaxation, 
it is not captured even after a linear number of rounds in
the LS hierarchy.  



We first introduce    
the family of  proper relaxations
which are ``configuration''-like linear programs.
The so-called Configuration LP was  used by 
Bansal and Sviridenko 
\cite{BansalS06} for the Santa Claus problem and has yielded valuable insights
and improved results, mostly
for resource allocation  and scheduling problems
(e.g., \cite{Svensson11, AsadpourFS11,
HaeuplerSS11, SviridenkoW13}).
A configuration in a scheduling setting usually refers to a set of
jobs  that
can be feasibly assigned to a given  machine  while meeting some load
constraint. A typical Configuration LP has therefore    an exponential number of
variables.    
The analogue of the Configuration
LP for facility location already exists (see, e.g., \cite{JainMMSV03}): it is the well-known 
{\em star relaxation,} in which every  variable corresponds to a {\em star,} i.e., a
facility  and a set
of clients assigned to  
The natural star relaxation 
for   \cfl\ and \lbfl\  is  equivalent to the standard LPs 
so it has an unbounded integrality gap.  
We take the idea of a star  to its logical  extreme by 
introducing  classes. 
A {\em class} consists of a set with an arbitrary number of facilities and clients
together with an 
assignment of each client to a facility in the set. 
The definition
of a class can thus vary from simple, ``local'' 
assignments of  some clients to  a  single facility, to  
 ``global'' snapshots  of the instance that 
express  the assignment  of many
clients to a large set of  facilities.  
A {\em proper relaxation} for an instance is defined by a collection
 of classes and a decision variable for every class. 
We allow great freedom in 
defining   
the only requirement   is that the resulting
formulation is symmetric and valid. 
The {\em complexity } of a proper relaxation is the maximum fraction
 of the 
available facilities that is contained in a class of 
Proper LPs are stronger than the standard relaxation. 
One can
construct infinite 
families  of instances 
where, by  increasing the complexity in a proper relaxation, one cuts off  more
and more fractional solutions.  
In this sense, all proper LPs for an instance can
be  thought of as forming  a (non-strict)    hierarchy, with the star
relaxation at the lowest level. 
We characterize their behavior  through a threshold result: 
anything less than maximum complexity results in unboundedness of
the integrality gap, while there are
proper relaxations of maximum complexity with an integrality gap of
.  In the latter,  corresponds simply to 
the set of all integer feasible solutions. 
Our precise results 
are the following theorems. Their proofs rely 
on the symmetry of the formulations. 

\begin{theorem}
\label{theorem:proper}
Every proper relaxation for uniform \lbfl\ with complexity  has an
unbounded integrality gap of  where  is the number of
facilities. There exist proper relaxations of complexity  that have an
integrality gap of  
\end{theorem}

\begin{theorem}
\label{theorem:proper2}
Every proper relaxation for uniform \cfl\ with complexity  has an
unbounded integrality gap of  where  is the number of
facilities. There exist proper relaxations of complexity  that have an
integrality gap of  
\end{theorem}


The second family we investigate consists of  linear  relaxations resulting from
repeated applications of the 
LS procedure starting from the natural LP relaxation for  \cfl\/.
We show that a specific bad solution with unbounded integrality gap 
 survives  rounds of LS.
The solution is defined on an instance  
 with  facilities and  
clients.    



It is  well-known that the LS procedure 
 extends to mixed 0-1  programs \cite{LovaszS91,BalasCC93} such as  \cfl
 \ with general client demands.  In that case the convex hull of the
 mixed-integer feasible set is known to be obtained the latest at the th
 level of the LS hierarchy, where  is the number of binary variables
 (\cite{LovaszS91},  \cite[Theorem~2.6]{BalasCC93}). For  \cfl,   
 equals the number  of facilities. 
In our instance 
 the clients have unit demands and as such the integer and
mixed-integer versions of the problem are 
equivalent. 
 In the lifting procedure, we treat  both the facility opening and the assignment
 variables as binary. 
It is easy to see that in  every round we obtain a polytope which is at least as tight as the one obtained 
when only the facility-opening variables are binary. 
 Therefore  our lower bound of   applies also 
 to the mixed-integer LS lifting procedure and is linear in the parameter  
Our proof is via protection matrices \cite{LovaszS91}. 
Using a simple reformulation of LS we give an explicit, fully
constructive definition of the matrices generated at each level that
witness the survival of the bad  fractional solution. 
The result is the following. 

\begin{theorem}
\label{theorem:ls-cfl}
For every sufficiently large   
there is an instance of uniform \cfl \   with  facilities and  clients 
so that the integrality gap 
 after  rounds of the LS procedure is    
\end{theorem}

 






\subsection{Other related work}
Koropulu et al. \cite{KoropuluPR00} gave the first constant-factor approximation
algorithm for uniform \cfl. Chudak and Williamson
\cite{ChudakW05} obtained a ratio of   subsequently  improved to 
\cite{CharikarG99}.
 P\'{a}l et al. \cite{PalTW01} gave
the first constant-factor approximation for non-uniform \cfl. This was improved
by Mahdian and P\'{a}l \cite{MahdianP03} and Zhang et al. \cite{ZhangCY04} to a
-approximation algorithm. As mentioned, the currently best guarantee is 5,
due  to Bansal et al. \cite{BansalGG12}. All these approaches use local search. 

Levi et al.  \cite{LeviSS12} gave a  5-approximation 
algorithm, based on the standard LP, for the special case of \cfl\ where
all facilities have the same opening cost. 
In the {\em soft-capacitated} facility location problem one is allowed to open
multiple copies of the same facility. 
Work on this problem 
includes \cite{ShmoysTA97,ChudakS99, ChudakW05, JainV01}.  
As observed in \cite{JainMMSV03}  a -approximation for  \ufl\ yields a
-approximation for the case with soft capacities. Mahdian, Ye and Zhang
\cite{MahdianYZ03} noticed a sharper tradeoff and obtained a
-approximation. A tradeoff between the blowup  of capacities and
the cost approximation for \cfl\ was studied in
\cite{AbramsMMP02}. Bicriteria approximations for \lbfl\  appeared in
\cite{KargerM00,GuhaMM00}. 

For hard capacities and general demands the feasiblity of the
{\em unsplittable} case, where the demand of each client has to be assigned to a
single facility, is NP-complete, as {\sc Partition} reduces to it. 
Bateni and Hajiaghayi \cite{BateniH12} considered the unsplittable problem 
with an  violation of the capacities and obtained an
-approximation. 


\bigskip

The outline of this paper is as follows. In Section~\ref{sec:prel} we give
preliminary definitions and in Section~\ref{sec:firstfamily} we introduce the proper
relaxations. The proofs of Theorems \ref{theorem:proper}, \ref{theorem:proper2} are  in Sections
\ref{sec:proof_theorem_p1} and \ref{sec:proof_theorem_p2} respectively. In Section~\ref{sec:ls} we present the
necessary background for the Lov\'{a}sz-Schrijver procedure. The proof of
Theorem~\ref{theorem:ls-cfl} is in Section~\ref{sec:ls_cfl}. We conclude with a 
discussion of our results in Section~\ref{sec:open}. 






\iffalse 


\section{Preliminaries}
\label{sec:prel}

\iffalse -------- removed, they are already in the intro --------
In an  instance  of the  general lower bounded-capacitated facility  location problem
(\lbfl\/, \cfl\/)   we  are  given a  set of  facilities
 and a set of clients  in some metric space.
Each  facility  has  an opening  cost   and a capacity  (\cfl\/) or a
bound  (\lbfl\/). We seek to open some facilities of 
and assign the demand  of each client 
to some opened facilities (we allow splittable demands).  
The cost for serving a unit of demand of client  by an open facility  is
, 
where    is  the distance  between    and
. As mentioned, distances  satisfy the triangle inequality. Moreover each
open facility  must serve at most
 in a \cfl\ instance or at least  total demand in a \lbfl\/.Our
objective  is to  minimize the  total facility  opening cost  plus the
total service cost. We use  to denote  and 
respectively.
-------------- \fi


Given an instance  of \cfl\ or \lbfl, we use  to denote  and 
respectively.
We will show our negative results for uniform, integer, capacities and lower
bounds. Each client can be thought of as representing one unit of demand.
 It  is  well-known  that in such a setting  the  splittable  and
unsplittable versions  of the problem are equivalent. 
The following 0-1  IP is the standard  valid formulation of uncapacitated 
facility location with unsplittable unit demands.

We give two well-known relaxations  for the problem.  The first one is the
natural  LP resulting from the above IP by replacing constraints
\eqref{eq:intx},\eqref{eq:inty} with \eqref{1>x>0},\eqref{eq:nni}: 
 
To model instances of uniform \cfl\ and \lbfl\ the following constraints are added
respectively:
 
The standard LP relaxation for \cfl\ (\lbfl) consists of \eqref{obj},  \eqref{x<y}, \eqref{eq},
 \eqref{1>x>0}, \eqref{eq:nni}, \eqref{sat1}  (resp. \eqref{obj}, \eqref{x<y}, \eqref{eq},
 \eqref{1>x>0}, \eqref{eq:nni}, \eqref{sat2}). 
In the rest of the paper we slightly abuse terminology by 
using  the term {\em (LP-classic)} for both. It will be clear from the context to
which problem we refer. 

The second well-known LP  is the star relaxation. 
A {\em star} is a set consisting of some  clients and
one facility. Let  be a set of stars. For a star   let   be an indicator variable denoting
whether   is picked.   The cost   of star    is equal
to the opening cost of the corresponding facility plus the cost of
connecting the star's clients to it. 
 
Defining  as the set of all stars  where  the total number of the clients in  
is at most the
capacity  (at least the bound ),  we get corresponding relaxations for
  \cfl\ (\lbfl).
In the rest of the paper we  slightly abuse terminology by 
using    {\em (LP-star)}  to refer to the
star relaxation for the problem we examine each time (\cfl\  or \lbfl).

It is well known  that for both \cfl\ and \lbfl, (LP-classic) and (LP-star) are
equivalent, therefore (LP-star) can be solved in  polynomial time.





\fi 





\section{Preliminaries}
\label{sec:prel}


Given an instance  of \cfl\ or \lbfl, we use  to denote  and 
respectively.
We will show our negative results for uniform, integer, capacities and lower
bounds. Each client can be thought of as representing one unit of demand.
 It  is  well-known  that in such a setting  the  splittable  and
unsplittable versions  of the problem are equivalent. 
The following 0-1  IP is the standard  valid formulation of uncapacitated 
facility location with unsplittable unit demands.

\iffalse 

\fi




\iffalse 

\fi 


We give two well-known relaxations  for the problem.  The first one is the
natural  LP resulting from the above IP by replacing the integrality constraints
 with: 

 

\iffalse

\fi
To obtain the standard LP relaxations for 
uniform \cfl\ and \lbfl\ the following constraints are added
respectively:

 

\iffalse

\fi

In the rest of the paper we slightly abuse terminology by 
using  the term {\em (LP-classic)} for both LPs. It will be clear from the context to
which problem we refer (\cfl\  or \lbfl). 

The second well-known LP  is the star relaxation.
A {\em star} is a set consisting of some  clients and
one facility. Let  be a set of stars. For a star   let   be an indicator variable denoting
whether   is picked.   The cost   of star    is equal
to the opening cost of the corresponding facility plus the cost of
connecting the star's clients to it. 

 

\iffalse    

 \fi

Defining  as the set of all stars  where  the total number of the clients in  
is at most the
capacity  (at least the bound ),  we get corresponding relaxations for
  \cfl\ (\lbfl).
In the rest of the paper we  slightly abuse terminology by 
using    {\em (LP-star)}  to refer to the
star relaxation for the problem we examine each time (\cfl\  or \lbfl).

It is well known  that for both \cfl\ and \lbfl, (LP-classic) and (LP-star) are
equivalent, therefore (LP-star) can be solved in  polynomial time.
 For
the sake of completion we include the relevant 
Lemma~\ref{lemma:ap-classic} in the Appendix. 








\section{Proper Relaxations}     \label{sec:firstfamily}

In this section  we introduce  the family of  proper relaxations.

Consider a   -  vector on the set of
 variables  of the  classic  relaxation (LP-classic)
such that  for all   The meaning  of
  is the usual one that
 we open facility   Likewise, the meaning of  is
 that we assign client  to facility . We call such a vector a 
 \emph{class}. Note that  the definition is quite general  and a class
 can be defined  from any such , which may or  may not have a
 relationship  to a  feasible  integer solution.  
Classes generalize the notion of a star. 
We  denote the  vector
 corresponding  to a  class   as .  We  associate with
 class     the  {\em cost   of  the  class}   . Let the {\em assignments of class}  be defined as 
  in .
We say that  {\em contains}  facility  if the corresponding entry
  in the vector  equals   
The set of facilities contained in  is denoted by 
  


\iffalse ------------

\begin{definition}  {\bf (Constellation LPs)} \label{def:constell}
Let  be a set of classes defined for an instance 
of
\lbfl. Let   be a variable associated with class 
The {\em  constellation LP with class set}
   is defined as 

 

\end{definition}
In what follows we will refer simply to a constellation LP when
 is implied from the context.  
------------------ \fi 

\begin{definition}  {\bf (Constellation LPs)} \label{def:constell}
Let  be a set of classes defined for an instance 
of
\lbfl. Let   be a variable associated with class 
The {\em  constellation LP with class set}
   denoted LP(),  is defined as 


 

\iffalse

\fi

\end{definition}
In what follows we will refer simply to a constellation LP when
 is implied from the context.  
We define the \emph{projection}  of solution  
of a  constellation LP to the classic facility opening
and assignment variables  as  and 
.


We will restrict our attention to  constellation LPs that satisfy a  natural property:   the LP is symmetric
 with respect to the clients and  the  facilities. 
The fact  that all facilities have the  same capacity / lower bound and all
clients have unit demand makes  this  property quite sound. 
For a class    and
 a permutation of the facilities, we denote by 
the class resulting by exchanging  for all  the values  of the   and
 coordinates 
of   with   the value of  the   and  coordinates of
. Similarly,  for  a permutation of the clients, we denote  by  the class resulting  by exchanging for
every   the value of  the  coordinate of   with
 the value of the  coordinate of .


\begin{definition} {\bf (: Symmetry)} \label{def:symmetry}
We  say  that property    holds for the constellation linear program LP()   if  the
following is  true: let  be any  permutation of   and  any permutation of .
 Then, for every  class    
and  are also in 
\end{definition}

The second property we require is the obvious one that the 
relaxation  is {\em valid,} i.e., the projection of its  feasible region to 
 contains all the characteristic vectors of the feasible integer solutions of the instance.



\begin{definition}  {\bf (Proper Relaxations)}  \label{def:proper} 
We call {\em proper relaxation}  for \cfl\ (\lbfl\/)  a constellation LP
 that is valid and satisfies property  
\end{definition}


Relaxation (LP-star) is
obviously a proper relaxation, while (LP-classic) is equivalent to
(LP-star). Therefore proper relaxations generalize the known natural
relaxations for \cfl\ and \lbfl. 









\subsection{Complexity of proper relaxations}
\label{proper proof}

Our main result on proper relaxations is that     proper LPs that  are not
``complex'' enough have an unbounded integrality gap while those  that
are sufficiently ``complex'' have an integrality gap of   To that end, we
define  the complexity  of  a  proper LP 
\iffalse
Furthermore, for each  such facility   we  denote by
 the  set of clients   for which  there is a facility   so
that  in .
\fi 

\begin{definition}   \label{def:complexity}
Given an instance  of \cfl\ (\lbfl\/)
let  be  a 
maximum-cardinality set  of open facilities in an integral feasible
solution. The {\em complexity } of a  proper
relaxation  for  is
defined as the  
\end{definition}

Note that for \lbfl\ it is possible  to have a proper relaxation with complexity greater
than . 
The  complexity of  a  proper LP  represents the  maximum
fraction of the  total number of feasibly openable  facilities that is
allowed in a single class. 
For   a  proper relaxation   the  complexity
describes to what extent 
 classes in  consider the instance locally. 
A complexity of nearly 
means that there are classes that take into consideration almost the whole instance
at once, while a low complexity means that all classes consider
the assignments of a small fraction of the instance at a time.  
\iffalse ===========================
We remark
that    the   proper    LP    with   an    integral   polytope    from
Theorem~\ref{thm:gap1} has
a complexity of   since every class corresponds  by construction to
a feasible integral solution. 
(Clearly not every LP with complexity  has an integrality gap of 
since it might contain weak classes together with the strong
ones.) 
============ \fi 
By increasing the complexity of a proper LP  for a given instance 
 we can produce strictly stronger 
proper relaxations. A simple example is given below. 


\begin{example}\label{proper_str}
An increased complexity allows strictly stronger proper relaxations.
\end{example}

First we show how one can construct any integer solution using classes that open the
same number of facilities.
Consider an integer solution  with opened facilities . We will use the following classes 
in which exactly  facilities are opened:
For any set of   consecutive classes in a cyclic ordering, namely , define a class that opens those facilities and makes the same assignments to them 
as . Then the integer solution is obtained  if for every  we set .
Observe that the latter solution is feasible for the proper relaxation.

We give a toy example showing that by increasing the complexity, we can
get strictly stronger relaxations. Consider an \lbfl\ instance with  facilities  sets 
of  clients each and 2 sets  of  clients each and .  For the star relaxation
(complexity  for this instance)
there is a feasible solution  whose projection to 
 is the following : for facility   and is assigned  integrally, for facility   and is assigned  integrally, for facility   and is assigned each client of  with a fraction of  and each of  with , and similarly for facility   and is assigned
each client of  with a fraction of  and each of  with . Actually
a direct consequence of Theorem \ref{theorem:proper} is that for any proper relaxation of the same complexity as the star relaxation, the above solution is feasible.

Now consider the following proper relaxation: all characteristic vectors 
of integer solutions with at most
 facilities are classes plus all the 
vectors of solutions with  facilities restricted in any  facilities ( parts of integer solutions that open all four facilities).
It is symmetric and valid by the previous discussion and has complexity . 
In any assignment of values to the class variables  that projects to  the following are true:
since classes with less than  facilities are integer solutions, they contain
assignments for all the clients and thus if we were 
to use a non-zero measure of such classes we would make non-zero assignment 
that does not exist in the support of .
 If we use
classes with exactly  facilities, then exactly one of facilities  must be present, 
since no integer solution opens them both with just the clients in . 
So we have to use at least  measure of such classes. 
So each one of facilities 
must be present in more than a unit of classes, which would make the solution infeasible.


\iffalse   ---- contained in the main body of the paper 
\begin{theorem}  \label{thm:gap1}
There is a proper LP relaxation for \cfl\ (\lbfl\/)  whose projection to  expresses the
integral polytope. 
\end{theorem}
\fi 
\medskip


If we allow the complexity to be , then one can find
proper relaxations that have integrality gap equal to  



\begin{theorem}  \label{thm:gap1}
There is a proper LP relaxation for \cfl\ (\lbfl\/)  that has complexity  
and whose projection to  expresses the
integral polytope. 
\end{theorem}

{\noindent{\em Proof of Theorem \ref{thm:gap1}.}}
For a given instance let  consist of a class for each
distinct integral solution. The resulting  is clearly
proper. Let  be any feasible solution of  and let
 be the support of  the solution. For every  and 
for every client  there is an  such that  Therefore 

This implies that  is a convex combination of integral
solutions. By the boundedness of the feasible region of
 the  corresponding polytope is integral.  
\mbox{} \hfill \mathqed  

Clearly not every LP with complexity  has an integrality gap of 
since it might contain weak classes together with  strong
ones.


We proceed  to show that a complexity of   is necessary 
in order to avoid a dramatic drop in solution quality as stated in 
Theorems \ref{theorem:proper} and \ref{theorem:proper2}.







\section{Proof of Theorem \ref{theorem:proper}}
\label{sec:proof_theorem_p1}

Our proof includes the following steps. We define an instance  
and consider any proper relaxation  for  that has complexity
 
Given  we use   the validity  and symmetry properties to show the existence of
a specific set of classes in . Then we use these classes to construct a
desired feasible fractional solution, relying again on symmetry. 
In the last step  we specify  the distances between the clients and  the facilities, so
that the instance is metric and the constructed solution proves unbounded integrality
gap.



\subsection{Existence of a certain type of classes}

Let us fix for the remainder of the section 
an instance  with  facilities, where  is
sufficiently large to ensure  that   where
  is a
constant greater than or equal to  . Let the bound , and let
the number of  clients be . Notice that  there are enough clients
to open  facilities, with  exactly  clients assigned  to each
one that is opened. The  facility costs  and the assignment  costs will  be defined
later.  Recall  that the  space  of  feasible  solutions of  a  proper
relaxation is independent of the costs.

 We  assume that  the  facilities are  numbered
. 
For a solution  we  denote by 
the set  of clients  that are assigned  to facility   in solution
, and  likewise for a  class  we denote  by 
the set of clients that are assigned to facility .  
Consider  an  integral  solution    to  the  instance  where 
facilities   are opened. 
Since our proper  relaxation is valid, it must have   a feasible 
solution   whose projection to  gives the characteristic
vector of .  We prove the existence of a  class  with some desirable
properties, in the support of  

By Definition~\ref{def:constell},
 can only be obtained as a positive combination of classes  such that for
every    facility       we   have    , Otherwise,  if the variables  of a  class 
with  have  non-zero value,
then in  there will be  some client assigned to
some facility with a positive fraction, while the projection of  namely 
does  not include  the
particular  assignment.  
Moreover,  since exactly  clients are  assigned to each
facility in  ,   for every facility 
that   is  contained   in  such   a  class     . To see why this  is true, 
since in   we have  for all    it follows  that for every facility ,
 .  
But  then  we have  that
.  We have already established   that . Then  is  a convex combination
of quantities less than or equal to , so for all such classes 
we have .


Therefore
in the class set of any proper relaxation for  there is 
a class  that assigns exactly  clients to each of  the
facilities in  By the value of  
  The following
lemma has been proved. 


\iffalse     --- old version 
\begin{lemma}
There is a class  that is contained in the class set of the proper
relaxation, that assigns  clients to each of  for some facilities.
\end{lemma}
\fi 
\begin{lemma}   \label{lemma:existence} 
Given the specific instance  any proper relaxation  of complexity
 for  contains in its class set a class 
 that assigns  clients to each of  facilities, for some
integer   
\end{lemma}


\subsection{Construction of a bad  solution}  
\label{subsec:badlbfl}

In the present section we will use the class  along with the
symmetric classes to construct a solution to the proper LP with 
the following
property: there are some   facilities   that
are almost integrally opened while the number of distinct  clients assigned to them will be less than . 

Recall that by property  every class that is isomorphic to  is
also a class of our proper relaxation. This means that
every set  of  facilities and every  set of  clients
assigned to those facilities so that each facility is assigned exactly
 clients, defines a class, called {\em admissible,} that belongs to the set of classes
defined of a  proper relaxation for the instance .

Let  us  turn  again  to  the solution    to  provide  some  more
definitions.  For   every  facility     ,  we  choose
arbitrarily a client  assigned to  it by . For each such facility
  we   denote  by     the  set  of   clients   i.e., the set of clients assigned to
 by  after we discard  (we will also call them the
{\em exclusive clients of }). For facilities   the sets
  are identical  and defined to be equal to 
 the union of  with all
the  discarded clients from  the other  facilities. In  the fractional
solution that we will construct below, the clients in 
will be almost integrally assigned to  for .

We  are   ready  to  describe  the  construction   of  the  fractional
solution. We will use a subset  of admissible classes that 
do not contain both  and  .  contains all such classes   
  that assign to each facility   in the class  the set  of clients   plus  one more
client selected  from the sets   for those facilities
 that do not belong  to  (there are at least 
of them). As for facility  (resp. ), if it  is contained in  then
it is  assigned some set  of  clients  out of the total   in
 (resp. ).  
All classes not in   will get a value
of zero in our solution.
We
will distinguish the classes in  into two types: the classes
of {\em type } that contain facility   or  but not both, and classes
of {\em type } that
 contain neither  nor .

We consider  first classes of type  . We give  to each such class   a
very small  quantity of  measure . Let   be  the total
amount of measure used. We call this step .  The
following  lemma shows  that after  , the  partial fractional
solution  induced  by  the  classes  has a  convenient  and  symmetric
structure:

\begin{lemma}  \label{lemma:roundA}
After  ,  each client      is
assigned to   with a fraction of  and is
assigned to each other facility    with a
fraction of . Each client  ()   is   assigned   to    and to  
 with   a   fraction   of .
\end{lemma}

\begin{proof}
Consider a facility  . Since exactly one of facilities   is present in
all the classes of type   and each class contains  facilities,
 is  present in the  classes of    of
the time due to symmetry of the classes. Each time  is present in
a class   that  class  assigns  all  to
.  So  client    is  assigned  to   with  a  fraction  of
. When   is not  present in  class ,
which happens   of the time, then  its exclusive clients
along with  the exclusive  clients of all  the other   facilities
that  are also  not  present in    are used  to  help the  
facilities   reach the bound  of clients (recall
that the number of exclusive clients of each such facility is equal to
).  Each time  this happens, the  facilities  in  need
  additional clients, while  the exclusive  clients of  the 
facilities that are  not present in  are   in total. Due
to symmetry  once again, a  specific client  is
assigned to  one of those   facilities 
of the  time of those cases.  So in total  this happens  of  the
time, so it follows that client  is assigned to a specific facility
    of the
time. The fraction with which   is assigned to  after 
is .

For  the proof  of the  second part  of the  lemma,  consider facilities
. Each one of those is present in the classes of type  an equal 
fraction  of the time. The
only clients that  are assigned to them are  their exclusive clients. Each
class  assigns exactly  out of those  clients. So,
due to symmetry, each client  is present in
   of the time,  so  is assigned to  and 
with a fraction of  to each.
\end{proof}

Note that  after  each  facility  has  a total
amount   of clients (since it is present
in a class   of the time and  when this happens
it is  given  clients).  Similarly, facilities   after 
have a total amount  each.

Now we can explain the underlying intuition for distinguishing between
the two
types of classes.  The feasible fractional solution 
we  intend to  construct is  the following:  for each
facility  its exclusive clients are assigned to it with
a fraction of  each, while they are assigned with a
fraction of  to  each other facility . As  for facilities  , all  of their exclusive  clients are
assigned with a fraction of   to each.  If  we  project  the solution  to  
 , the  variables will be forced 
to  take   the  values
 for  and . Observe as we give some  amount of  measure to  ,
 the  variables  concerning the
assignments to facilities  tend to their intended values in the
solution we want to construct ``faster'' than the variables concerning the
assignments to the other facilities. This is because, by Lemma~\ref{lemma:roundA}
after  each exclusive client  of  is assigned to each of them with
a fraction of  which is  of  the intended value. At the  same time, every
exclusive  client of  each other  facility is  assigned to  it  with a
fraction of  which is   of the  intended value.  For sufficiently
large  instance  ,  as    tends  to  infinity,  the  assignments
to  and  will reach their intended values while there will
be   some    fraction   of   every    other   client   left    to   be
assigned. Subsequently we have to use classes of type , 
to achieve the opposite effect: the
variables  concerning the  assignments of  the first   facilities
should tend
to their intended values ``faster''  than those of  and  (since
 and   are 
not  present in  any of  the classes  of type  ,  the corresponding
speed will actually be zero).

We  proceed  with  giving  the  details  of  the  usage  of  type  
classes. As before,  we give to each such class  a very small quantity
of  measure .  Let    be the  total  amount of  measure
used. We call this step .

\begin{lemma}   \label{lemma:roundB}
After  ,  each client      is
assigned  to   with a  fraction of    and is
assigned to each other  facility    with a
fraction of .
\end{lemma}


\begin{proof}
The proof  follows closely that of  Lemma~\ref{lemma:roundA}. A  facility   is present in  a class of  type    of the
time (since   this  fraction is less  than ).  Each such
time, every  client  is  assigned to  it (again
this  is due  to the  definition  of classes  of type  ). So  after
,      is  assigned   to      with   a  fraction   of
.  
Also, when   is  present  in a  class, it is assigned exactly one client
which is exclusive to a facility
not in the class. Since in total there are  such candidate clients,
and by symmetry, after round  
each one of them is picked an equal fraction of the time to
be assigned to , we have that
each client  is assigned to a facility for which  is not
exclusive with  a fraction
.
\end{proof}


\noindent
To  construct  the   aforementioned  fractional  solution ,  set      and     , and  add the  fractional assignments of  the two
rounds. 

It is easy to check that the facility and assignment variables of facilities 
take the value they have in . Same is true for the facility variables for 
and the assignment variables of the clients to the facilities they are exclusive. 
To see that the same goes for the non-exclusive assignments, observe that since
every class assign exactly  clients to its facilities we have that .
So each  takes exactly  demand from non-exclusive clients which  are
 in total. Thus, by symmetry of the construction, each one them is assigned to 
with a fraction of 

\subsection{Proof of unbounded integrality gap of the constructed solution}

In the  present subsection, we  manipulate the costs of  instance ,
which we left undefined, so as to create a large integrality gap while
ensuring that the distances form a metric.

Set each facility opening cost to zero. As for the connection costs (distances)
consider the -dimensional Euclidean space . Put
every facility    together with its  exclusive clients on a
distinct vertex of an -dimensional regular simplex with edge length
. Put facilities  together with their exclusive clients to a point
far away  from the simplex, so  the minimum distance from  a vertex is
 Setting  is enough.

Since the distance between a facility and one of its exclusive clients
is  ,  the  cost  of  the fractional  solution  we  constructed  is
. This cost  is due to the assignments  of exclusive clients of
facility   to facilities  with   
As  for the cost  of an arbitrary integral  solution, observe
that since the  exclusive  clients of  are very far from
the  rest of  the facilities,  using   of them  to  satisfy some
demand of  those facilities and help  to open all of  them, incurs a
cost of  On the other hand, if we do not open all of the 
facilities on  the vertices of the  simplex (since they  have in total
  exclusive clients  which is  not enough  to open  all of
them), there  must be  at least  one such facility  not opened  in the
solution, thus its  exclusive clients must be assigned elsewhere,
incurring a cost of  

This concludes the proof of Theorem~\ref{theorem:proper}. 




\section{Proof of Theorem ~\ref{theorem:proper2}} 
\label{sec:proof_theorem_p2}

The proof is similar to that for \lbfl. 
We prove that the relaxation must use 
a specific set of classes and then we use these classes to construct a
desired feasible solution. In the last step we 
 define appropriately  the costs of the instance. 

\subsection{Existence of a specific type of classes}

Consider
an instance  with  facilities, where  is
sufficiently large to ensure  that   where
  is a
constant greater than or equal to  . Let the capacity be , and let
the number of  clients be . Notice that in every integer solution of the instance
 we must open  at least  facilities. The  facility costs  and the assignment  costs will  be defined
later.  

 We  assume, like before, that  the  facilities are  numbered
. 
Consider  an  integral  solution    for    where  all the
facilities  are opened, and furthermore 
facilities   are assigned  clients each 
and facility  is assigned one client. 
Since our proper  relaxation is valid, there must be a solution  in the  space of
feasible solutions of the proper relaxation whose  projection is the characteristic 
vector of .  
By Definition~\ref{def:constell},
it is easy to see that  can only be obtained as a 
positive combination of classes  such that for
every    facility       we   have    .  Recall  that  since  the  complexity  of  our
relaxation is , the classes in the support of any solution 
have at most 
facilities. 

Now consider the support  of . We will distinguish the classes  for
which variable  is in the support of  into 2 sets. The first set consists 
of the classes that assign exactly one client to facility ; call them \emph{type A} classes.
The second set  consists  of the classes that do not assign any client to facility 
; call those \emph{type B} classes. By the discussion above those sets form a
partition of the classes in the support of , and moreover they are both non-empty: this is
 by the fact that at most   facilities are in any class, and by the fact
 that in  all  
facilities are opened integrally. Notice also that no  class
of type B can contain facility  even though the definition of a class does not
exclude the possibility that a class contains a facility to which no clients are
assigned. 

We call \emph{density} of  a class  the ratio 
. By the discussion 
above we have that  for all  in the support of . The following holds:

\begin{lemma}
All classes in the support of  have density 
\end{lemma}

\begin{proof}
The amount of demand that a class  contributes to the demand assigned to the set
of the first  facilities by  is 
 We have .
 Observe
that by the projection of  on  and by the fact that for ,
  in , we have . 
Setting  we have from
the above  and . 
The latter together with the fact that  we have that  for all classes
 in the support of .
\end{proof}

The following corollary is immediate from the above:

\begin{corollary}
There is a type  class in the support of  that has density 
\end{corollary}

So far we have proved that 
in the class set of any proper relaxation for  there is 
a class  of type  with density .
 Let  


\subsection{Construction of a bad solution}

Consider the symmetric classes of  for all permutations of the  facilities
and for all permutations of the clients. Those classes are not necessarily in the support of . Take a quantity of measure  and distribute it equally among all 
those classes. Since class  has density  all those symmetric classes
assign on average  clients to each of their facilities. 
Due to symmetry, each facility is in a class  of the time and is assigned  demand. Each client is assigned to
each facility  of the time. We call that step of our construction \emph{round }.

Now consider the symmetric classes of  for all permutations of the first  facilities
and for all permutations of the clients (those classes are well defined since ).
Again distribute a quantity of measure  equally among all 
those classes. Similarly to the previous, each facility is in a class  of the time and is assigned  demand. Each client is assigned to
each facility  of the time. 
We call that step of our construction \emph{round }.

Spending  measure in round  and  
measure in round  we construct a solution  whose projection to  is the 
following :
 for , , and for every client   and 
 for  It is easy to see that  is
a feasible solution for our proper relaxation.

Now simply set all distances to , and define the facility opening costs as 
 and  for  It is easy to see
that the integrality gap of the proper relaxation is . 
In Section \ref{sec:ls_cfl}, where we prove unbounded integrality gap for the
Lov\'{a}sz-Schrijver procedure, 
 we will have to use a somewhat more general "bad" solution on an instance with 
 many costly facilities.




\section{The LS Hierarchy}
\label{sec:ls}

The Lov\'{a}sz-Schrijver hierarchy was defined in \cite{LovaszS91}. For a 
comprehensive presentation and 
various reformulations see \cite{Tulsiani11-chapter}. In this section
we give the necessary definitions and the reformulation we are
employing in our proof. 

In \cite{LovaszS91} an operator  was defined which refines a convex set 
, when applied to  it.
After  applications the resulting convex set is the integer hull of . 
Starting with a polytope  we define
.
The following Lemma characterizes the vectors  of  
that survive  after  iterations.

\begin{lemma} [\cite{LovaszS91}]\label{lemmals}
 If  is a cone in , then  iff there is an  symmetric
matrix  satisfying
\begin{itemize}
\item[1.] 
\item[2.] For , both  and  are in 
\end{itemize}
\end{lemma}

In such a case,  is called \emph{the protection matrix} of . Since we are interested in
the projection of the cones on the hyperplane  which contains our
original polytope, we restate the conditions of survival of  as the following
corollary which is immediate from Lemma \ref{lemmals}.


\begin{corollary} [\cite{AroraBL02}]\label{corls}
 Let  be a cone in  and suppose  
where  Then  iff there
is an  symmetric matrix  satisfying

\begin{itemize}
\item[1.]  
\item[2.]  For : If  then ; 
If  then ; Otherwise, , 
both lie in the projection of  onto the hyperplane 
\end{itemize}
\end{corollary}


Let  denote the vector  i.e., the th row of  
Corollary  \ref{corls} makes  it  convenient to  work with  individual
vector solutions that can be  combined as rows to build the protection
matrix. 
 We focus now on the survival of
a vector   for one  round and state  some simple properties of  


Given a protection matrix  of  
we define a set of at most  {\em
  witnesses} of vector  For each variable ,  there are at most
 such witnesses: the one that equals 
(if  ),   which we  call \emph{type  1 witness  of 
  corresponding to variable } and the vector 
(if ), which we call \emph{type 2 witness of 
  corresponding to  variable } For the validity  of the upcoming
observation recall  that if   and hence the  type  witness
corresponding to  is undefined,  

\begin{observation}\label{obs_diag}
The condition that 's main diagonal is equal to the vector  is
equivalent to the following: the variable  of the type  witness
 corresponding to variable  is equal to 
\end{observation}


The rows of  that correspond to zero variables in  are filled 
with zeros and called {\em
  special.}  
Moreover if   
To account for these requirements it is not enough that the integer values
in  appear on the main diagonal. The following claim states that
they are replicated across  all witnesses. 

\begin{claim}\label{simplefact1}
Let   be a witness  of  .  If  for some , , then 
\end{claim}


To enforce symmetry  for a special row  that corresponds
to a  variable  it must be  the case that the  th column is
set to  zero as well.  This is ensured by  Claim~\ref{simplefact1} for
all  entries   of  the column  for  which   (The
remaining entries  of the column belong  to special zero  rows and are
equal to zero anyway). 
For  the remaining rows, it will be convenient to 
  express  each variable of a type 1   child of  corresponding
to some variable  , by defining the factors by which  the variables of 
differ from the  corresponding variables of . Then  the symmetry condition of
 is satisfied by ensuring that  the condition of the following claim on those
factors holds. 


\begin{claim}\label{simplefact2}
Let indices  take values in  The symmetry condition of the
protection  matrix  of    holds  iff  Claim~\ref{simplefact1} holds
 and for  each type  1  witness     of  
corresponding to variable , for which 
,  
then, for  the type 1 witness   of   corresponding to variable
, we have .
\end{claim}



Observe that  when we construct a  type  witness   corresponding to ,
 the type   witness   corresponding  to  is  automatically defined. We
 say that  is the \emph{twin} of  .
 
\begin{claim}\label{simplefact3}
Let indices  take values in  If the protection matrix of 
exists, 
the following must hold. 
If  in the type  witness  corresponding to
  then
 where  is the  type  twin of 
\end{claim}


To prove the existence of a  protection matrix  for a vector  we
will proceed as follows. We will define a set  of witness
vectors that  will contain a type   and a type   witness for every
non-integer variable and  one of the appropriate type  for each 
integer  variable. We  will ensure  that the  vectors in    meet the
conditions of Observation~\ref{obs_diag}, Claims~\ref{simplefact1},
\ref{simplefact2}  and \ref{simplefact3}. In  addition  we will prove 
that the vectors
meet the  feasibility constraints for  Then 
we will have shown how to  construct  a protection matrix  of  whose rows
consist of: 
the type  1 vectors  from  scaled each  by the  corresponding variable
 together with  one special {\bf  0} vector  for each zero variable in 



To prove  the survival of a vector  for many rounds we  just embed the
strategy above in an inductive argument. 
The following fact  is immediate from Corollary
\ref{corls}:   if,   for   all      the   vectors      and
 (that are  defined) witnessing the survival of
our initial vector , survive  themselves  rounds of LS, then 
survives actually  rounds of LS. 


We
define a tree structure which we 
call the  \emph{evolution tree  of } Every node in  is {\em
  associated} with a vector. The tree is defined 
recursively: vector  is associated with the root of the tree, and if  is a node
of  associated with vector   then  the vectors  witnessing the
survival of  
are associated in one-to-one manner with the children 
of  If  there are no such witnesses,  the fractional solution   does
not survive one round of LS, and  we call  a \emph{terminal} node. 
The number  of rounds  that our
initial vector  survives, is the length of the shortest path from the root of
the evolution tree   to a terminal node.


Given a root vector, we will show that as long as we have walked down the
evolution tree at depth , where   is the target number of rounds, then the
protection matrices of the root and all its descendants are well-defined. 
The inductive step shows how to define all children of a node  and
therefore increase the depth of the tree by one. 
\iffalse ================ redundant
 For any child 
of    there is some  so that 
 corresponds to 
the th row of the protection matrix  of  and thus   to the
variable  of    

For each variable 
there  are    corresponding 
children  nodes of  the  one  that is associated with   the
witness    which we call the \emph{type 1 child of
   corresponding  to variable } and  the one which is
associated with the witness  
which we call the \emph{type 2 child of  corresponding to
  variable }
============ \fi 
From now on  we  refer
interchangeably to a node and its  associated solution
vector. Accordingly, if  is a child of node   () is 
associated with  (resp. ) and 
and   is a type  () witness  of  corresponding to variable
 
we will refer to node  as 
a {\em type  (\mbox{resp.} )  child of node-solution 
  corresponding to variable } 



Finally, the following fact will be useful for the feasibility proof. 


\begin{lemma}\label{eqconlemma}
Given a solution  in  the evolution tree that satisfies an equality
constraint ,  and given a child of   that is a
type   solution   corresponding  to some    that satisfies
, then the twin  type  solution  of 
also satisfies .
\end{lemma}
\begin{proof}
Let  .   From    and
 we get . Then by
Claim~\ref{simplefact3},          
\end{proof}








\section{Proof of Theorem~\ref{theorem:ls-cfl}}
\label{sec:ls_cfl}

In  this  section we  show  that the  integrality  gap  on a  suitable
instance of \cfl\ remains unbounded even after applying a large number
of iterations of the LS procedure.  \iffalse More precisely, we reduce
the existence of protection matrices conditions to the construction of
the nodes  of an evolution  tree and we  prove that the nodes  of that
tree up to a desired depth (rounds of LS) are well defined.  \fi

The instance  is the following.  Consider a set  of 
facilities  which have   opening cost.  We call  that set
\emph{Cheap}. Moreover, 
consider a  set of  facilities  that have an opening cost  of  each.
Call that set  \emph{Costly}. Think of  as being ; we will later
prove that the number of rounds  of survival are maximized for
. The set of facilities  is   Let all the facilities have the same capacity
, and let there be a  total of  clients in the set   All clients and facilities
are at a  distance of  of  each other. Clearly all integral  solutions to the
instance have a cost of at least .

Consider  the following solution   to  (LP-classic): For  each
facility      ,    and   for   each   client      set
,  .  For  each  facility    
, for  some sufficiently large 
constant  ( is enough), and  for each
client    set  .  The  constructed  solution  incurs  a  cost  of
 which is  if  

It is well-known that some simple valid 
inequalities are not produced early in the LS procedure.
For example, in the case of \cfl\ 
our proof implies that  rounds
are required to obtain the simple inequality  which is facet-inducing for our instance. 
This inequality is not critical however for our proof. 
It is easy to modify the input by adding one  facility and
one client at a large distance from the rest of the instance, 
so that Theorem~\ref{theorem:ls-cfl} continues to hold 
while the inequality   above is  satisfied by a bad fractional solution.
Given an analogous  fixed set of inequalities, an adversary can modify the
instance in a similar manner. 


Solution   cannot  survive   rounds of  application of  the  LS procedure:
consider the path  from the root where we  descend each time via a  type 2 child
corresponding to  a   variable of a  costly facility (a  different facility
each time).  After  such steps, assuming  of course that the  nodes are defined,
one can show  that all facilities in  will be  closed and the facilities
in   have to absorb  all the  demand in the  instance, which leads  to an
infeasible solution. 


\begin{observation}
\label{observation:l-rounds}
Solution  survives less than  rounds of the LS procedure.
\end{observation}

For the proof of Observation \ref{observation:l-rounds}
we will actually prove the following stronger Lemma:


\iffalse ----------------- old version with typos --------------
\begin{lemma}
Let  be a set of variables  of vector . If 
 survive  rounds of LS and , then
 there is a path  of length  starting from the root  of the evolution tree 
that ends with a node-solution  such that in 
  for all .
or\\
b)  does not survive  rounds.
\end{lemma}

\begin{proof}
The proof is by induction on .
Suppose that  survive  rounds. Let  be the variable in   with
the highest value. Consider the type 2  child of  corresponding to .
Then  otherwise by Claim   in the twin type  solution
 of '. So we have \\
\\
\\
.\\

Thus we set  and we have , so, by inductive hypothesis
 we have a path  starting at . By adding   before the first node  of 
we get the desired path .
\end{proof}

--------------------------- \fi 

\begin{lemma}
Let  be a set of variables  of vector  s.t. 
 If   survives  rounds of LS, then 
 there is a path  of length  starting from the root  of the evolution tree 
that ends with a node-solution  such that in  for all   
\end{lemma}

\begin{proof}
The proof is by induction on .
Suppose that  survive  rounds. Let  be the variable in   with
the highest value. Consider the type 2  child of  corresponding to .
Then  otherwise by Claim
 
 in the twin type  solution
 of  So we have \\
\\
\\
.\\

\noindent
Setting  we have  By the  inductive hypothesis
the evolution tree contains a  path  starting at  that has 
length   By appending   before the first node  of 
we obtain the desired path .
\end{proof}



To prove Observation \ref{observation:l-rounds}, assume that  survives  rounds.
Then there is a path of length  starting from the root of the evolution tree,
such that in  the last
node solution  of the path all the facilities in 
 are closed. Clearly this cannot
be a feasible solution.


\noindent 
We are ready to state the main theorem of this section 
which implies that the solution   survives  
 rounds of LS. We  do not make any attempt to optimize the   constant.
At every level of the induction the new witness solutions cannot
differ drastically from  their parent node. 
We identify a  set of invariants that express this controlled
evolution of the values. 


\begin{theorem}\label{smalltheorem}
Let  and  be a constant of value  
 We can construct an evolution tree  with root 
such that
any  node   of  at depth  
is associated with a  feasible solution  that satisfies the following invariants:\\
\vspace*{-0.8cm}
\begin{itemize}
\item[1] For variable  , 
 and 
\item[2] 
\begin{itemize}
\item[(a)] For variable , 
  
\item[(b)] For variable ,  and , .
\item[(c)]  For variable ,  and
  , .
\end{itemize}
\item[3] For , .
\item[4] For ,
\begin{itemize}
\item[(a)]  if ,.
\item[(b)]  if ,.
\end{itemize}

\end{itemize}

\end{theorem}




Setting in our instance  by Theorem~\ref{smalltheorem} we obtain that the solution
 survives  rounds. 
Thus we have proved Theorem \ref{theorem:ls-cfl}.













\subsection{Proof of Theorem ~\ref{smalltheorem}}
\label{subsec:invproof}





The  proof is  by induction  on the  depth of  node .  More  specifically, by
assuming that the invariants hold for an arbitrary node  at depth
less than , we
show  how  to construct  all  the  children nodes  of    so   that they  are
well-defined and 
the invariants are met.  

In the proof, whenever we give the  construction of a type  or type  child
of    corresponding  to  some  variable  , we  refer  to    as  the
\emph{touched  variable} --  we also  say that   is
\emph{touched}  as type 1 or type 2 in the
current step.  We will consider cases according to which variable is touched and
whether it  is touched  as type   or as  type . When  we touch  a variable
  as type  ,    always takes  the  value   so  by
Observation \ref{obs_diag}  we satisfy  the condition that  the diagonal  of the
underlying protection matrix is equal to the th row.  Note that we will not  give
the construction  for the case in  which    is touched, since
 is  always  and  the construction is  trivial in those cases.  The same
applies  to  the  cases of  all  variables  that  have  integral values  in  the
node-solution  of the inductive hypothesis, as we simply 
enforce Claim~\ref{simplefact1}.
 
Another   feature of   our construction is  the following: when  a fractional
variable  is touched as type , it is set to   and for all
  
becomes . If  is touched as type , it is set to   and
in order to maintain feasibility 
its previous value is
distributed  among the  other assignments  of client  . Thus  for  every ,
either there is  some  such that   and for all other  
 (e.g.,  when cases ,   below have happened for  an ancestor of  ), or
there are at most  facilities to which the assignment of  is  (if there
are type  nodes,  through cases , , , along the  path of the tree
that leads  to ). 
In fact, as far as assignments to cheap facilities are concerned,
the upper bound of  holds cumulatively  across all clients,
since 
no more than   assignment variables can be touched as Type 2 along  a
path of length  
Specifically, 
let  be the set of clients  for which, for all   
We will use the fact that  



Note  that  the  invariants of    Theorem~\ref{smalltheorem}  imply the  satisfaction  of
constraints  \eqref{x<y},\eqref{1>x>0},\eqref{eq:nni} and  \eqref{sat1}  for the
number  of  rounds  we consider.  Thus,  when  proving  the feasibility  of  the
constructed solution each time, we only have to ensure that \eqref{eq} holds.




\iffalse ---------------- One shot TRICK BELOW Didn't work
\begin{oneshot}{Lemma~\ref{easyconstraints2}}
Let    be a node-solution defined  at   depth   of the evolution tree   If  satisfies
Invariants 1--4, then  meets 
constraints \eqref{x<y},\eqref{1>x>0},\eqref{eq:nni} and \eqref{sat1}.
\end{oneshot}


\newtheorem*{thm:associativity}{Theorem \ref{thm:associativity}}
\begin{thm:associativity}
Lorem ipsum ...
\end{thm:associativity}

---------------- \fi  





 
\begin{lemma}\label{easyconstraints2}
Let    be a node-solution defined  at   depth   of the evolution tree   If  satisfies
Invariants 1--4, then  meets 
constraints \eqref{x<y},\eqref{1>x>0},\eqref{eq:nni} and \eqref{sat1}.
\end{lemma}





We now explain the  inductive step that constructs the children of
node  where  is at depth  We  distinguish  cases according to the
variable that is touched. 

\medskip
\noindent
{\bf Case 1: type  children}

\noindent 
{\bf \underline{subcase :} touched variable is  , }

{\sc Algorithm}

Consider the type  child  of  corresponding to variable . Variables  for all  are multiplied by a factor of  and so  .
Note that since we only consider cases where  is fractional, by 
the inductive hypothesis we have that for all variables , Invariant 2.b holds.
The variables involving facilities , namely  for all , remain the same.  For all  and for all  such that   we have , where  is
the number of facilities in  for which  is assigned with a non-zero fraction (so ).


{\sc Feasibility}

Constraint \eqref{eq} is satisfied by construction:\\

\noindent
\\
\\

{\sc Invariants}

{\sf Invariant }

For ,  remain unchanged so Invariant  holds by
the inductive hypothesis (from now abbreviated as i.h).

{\sf Invariant }

For  we have :

\noindent
 \hfill  (by Invariants ,  of i.h. and being  generous)\\
\\
\\

For ,  holds since variables  were not changed. For :

\noindent
 \hfill (by Invariants , )\\


{\sf Invariant }

Observe than the total demand assigned to each facility in  was decreased so Invariant  holds by the 
inductive hypothesis.

{\sf Invariant }

For  Invariant 4 holds by inductive hypothesis. For  we have :

\noindent
 \hfill  (by the invariants of i.h.)\\
\\






\medskip
\noindent
  {\bf \underline{subcase :} touched variable is  , }

{\sc Algorithm}

Consider the type  children  of  corresponding to variable . Variable  is multiplied by a factor of  and so   (and  of course , and  for ). Every other variable remains the same.


{\sc Feasibility}
The feasibility of this case is trivial.

{\sc Invariants}
The Invariants    in this case are satisfied trivially. For  we have for facility :

\iffalse --------- previous derivation 
\noindent
\hfill (variable  becomes )\\
\hfill (by  of i.h.)\\
\hfill (being very generous)\\

-------------------- \fi 
\noindent
\hfill (variable  becomes )\\
\hfill (by  of i.h.)\\
\hfill  if   or \\
\hfill if \\ 

In either of the two cases Invariant  holds for the new value 



\medskip
\noindent
{\bf \underline{subcase  :} touched variable is  , }

{\sc Algorithm}

Consider the type  children  of  corresponding to variable . Variables  with  are multiplied by a factor of , where  is again the number of
facilities in  for which  is assigned with a non zero fraction (so ).
Of course , and  for  as usual. Every other variable remains the same.


{\sc Feasibility}
Obviously \eqref{eq} is satisfied. All other constraints are satisfied by 
Lemma \ref{easyconstraints2}.

{\sc Invariants}

{\sf Invariant }

For each  such that  we have:

\noindent
\hfill (by Invariant  of i.h.)\\
\hfill (by Invariants
,  of i.h.)\\


{\sf Invariant }

Variables  remain unchanged for . For ,  while for 
we have , so  is trivially satisfied.

{\sf Invariant }

For  the total demand is decreased (because of ). For :

\noindent
\hfill (by  of i.h.)\\
\\

{\sf Invariant }

The demand assigned to facilities in  is decreased  (because of ) 
so   trivially hold. 





\medskip \medskip
\noindent
{\bf Case 2: type  children}

\noindent 
{\bf \underline{subcase :} touched variable is  , }

{\sc Algorithm}

Consider the type  children  of  corresponding to variable .
Let . Solution  is dictated by its twin type  solution (case 1a):
 variables  for all  are multiplied by a factor of  and so   and , that is facility  closes. The variables involving facilities , namely  for all , remain the same. 
For all  and for  such that  we have , where  is again the number of
facilities in  for which  is assigned with a non zero fraction (so ).


{\sc Feasibility}
Constraint \eqref{eq} is satisfied by Lemma \ref{eqconlemma}.

{\sc Invariants}

{\sf Invariant }

For ,  remain unchanged so Invariant  holds by inductive hypothesis.

{\sf Invariant }

For  we have :\\

\noindent
 \hfill (by Invariants ,  of i.h.)\\
 \hfill (being very generous)\\
  


For ,  holds since variables  were
not changed. \\


{\sf Invariant }

For  we have:\\

\noindent
 \hfill (by Invariants   of i.h.)\\
\\
\\

The  above is due to the fact that at most  assignment variables
 for some cheap facilities may have been touched as type  and are  
For  those same clients the assignment to  is fractional,
 so the demand
corresponding to them  that was assigned to    must be 
distributed among the, at least , available
cheap facilities. That additional demand is 
at most 

{\sf Invariant }

For  Invariant  holds by inductive hypothesis. For  we have .\\





\medskip
\noindent
{\bf \underline{subcase :} touched variable is  , }

{\sc Algorithm}

Consider the type  children  of  corresponding to variable . 
Let . Solution  is dictated by its twin type  node-solution (case ):
variable  is multiplied by a factor of  and  for ,  and
  . Every other variable remains the same.


{\sc Feasibility}
The feasibility of this case is trivial by Lemma \ref{eqconlemma}.

{\sc Invariants}

{\sf Invariant }

For facilities  the proof is trivial (no change).
For  we have:

\noindent
\\
 \hfill (by Invariants   of i.h.) \\
\\
\\

{\sf Invariant }

For client  and facility  we have :

\noindent
 \hfill (by Invariant  of i.h.)\\
\\
\\

For client  and facility  we have :

\noindent

\\


Case  similarly.

{\sf Invariant }

For :

\noindent
 \hfill (by  of i.h.)\\
\\

{\sf Invariant }

For  the total demand is decreased while for :

\noindent
 \hfill (by  of i.h.)\\
\hfill  if   or \\
\hfill if \\ 






\medskip 

\noindent
{\bf \underline{subcase :} touched variable is  , }

{\sc Algorithm}

Consider the type  children  of  corresponding to variable .
Let . Solution  is dictated by its twin type  node-solution (case ):
variables  are multiplied by a
factor of  where 
  is again the number of
facilities in  for which  is assigned with a non zero fraction (so ).
 For ,  while . Every other variable remains the same.


{\sc Feasibility}
The satisfaction of \eqref{eq} is ensured by Lemma \ref{eqconlemma}.

{\sc Invariants}

{\sf Invariant }

For  facility  such that  we have:

 \noindent
\\
\hfill (by Invariants   of i.h.) \\
\\
\\

{\sf Invariant }

For client  and facility  we have :

\noindent
 \hfill (by Invariant 2 of i.h.)\\
\\
\\

For client  and facility  we have :

\noindent
\hfill (by Invariant  of i.h.)\\
\\
\\

{\sf Invariant }

The demand assigned to  is decreased. For :

\noindent
\hfill (by  of i.h.)\\
\\

{\sf Invariant }

For :

\noindent
\hfill (by  of i.h.)\\
\hfill if  or \\
\hfill if \\ 


\medskip \medskip 
The case analysis is complete. 
It remains to show that the witness vectors we constructed for node
 satisfy the symmetry requirements. 




\begin{lemma}\label{symmetry}
The symmetry condition, as stated in Claim \ref{simplefact2},
 is satisfied  for  the children of node-solution 
\end{lemma}

\begin{proof}
By construction we never alter integer values of variables, therefore
the condition of Claim~\ref{simplefact1} holds. 

When a variable  ,  is touched then  for the symmetry between
 and each other variable we have:

 For all , variables  are multiplied by  (case ), and when
 some  is touched, variable  is multiplied by  (case ).

  For  all  ,   variables  ,     are  multiplied  by
   (case ), and when some  is
  touched, variable  is multiplied by 
  (case ).

 For all , variables , ,  are
 multiplied  by   (case  ),  and when    or  some   is
 touched, variable  is multiplied by  (cases , ). \\\\


When  a variable    is  touched then  for the  symmetry
between  and each other variable we have:

For all    and all  , variables   are multiplied  by 
(case  ),  and  when  some    is  touched,  variable    is
multiplied by  (cases , ).

  For , variables   are multiplied by   (case ), and
  when some  is touched,  variable  is multiplied by  (cases
  , ). \\\\



Finally, when  variable    is  touched then  for the  symmetry
between  and each other variable, the remaining cases that have not been
covered above are: 

For all  and all   variables  are multiplied
by    (case  1c), and  when    is  touched, variable    is
multiplied by  (case ). 

For all   variables  are multiplied by  (case
),  and when   is touched,  variable   is multiplied  by 
(case ). 

 

\end{proof}


The proof of Theorem~\ref{smalltheorem} is now complete.






The proof yields a tradeoff between 
the number of rounds as a function of the dimension of the instance  
and the integrality gap, 
which can be obtained by toying with the quantities , , and  that are
left as parameters. One can obtain a higher gap that survives for a
smaller number of rounds. 







\iffalse ---------- moved before the proof ----------

It is well-known that some simple inequalities are not produced early in the LS procedure.
For example, in the case of \cfl\ 
our proof implies that  rounds
are required to obtain the simple inequality  which is facet-inducing for our instance. 
It is easy to modify the input by adding appropriate clusters of facilities and
clients at a large distance from each other, so that our proof continues to hold 
while inequalities of this flavor  are  satisfied by the bad fractional solution.
This would mean  that such inequalities alone
fail to capture the "core" of the difficulty
of the instance.


\fi 






\section{Discussion}\label{disc}
\label{sec:open}



It is not hard to see that our proof of Theorem~\ref{theorem:ls-cfl} 
also yields the same lower bound 
for the mixed LS \cite{Cornuejols08}
procedure:  simply restrict  the constructed  protection  matrices  to the  
variables. The  resulting matrices  are of the  form    which    are   well-known   to    be   positive   semidefinite
(see, e.g., \cite{GoemansT01}). 




The \cfl\ instance for which  the LS procedure fails is essentially a
 Minimum Knapsack instance which can be approximated within a constant
 factor by adding the, exponentially many, knapsack-cover inequalities
 \cite{CarrFLP00}. Note that such  an instance might be a sub-instance
 of a  larger \cfl\ instance  with positive connection costs.   To add
 constraints  of  the knapsack-cover  flavor  would  at least  require
 preprocessing to recognize sub-instances  that are similar to the one
 in our  proof, assuming that  such a task  can be done  in polynomial
 time.  ``Similar''  would mean clusters of closely  located cheap and
 costly facilities  and clients,  where the definition  of "closely'',
 ``cheap'' and ``costly"  would depend somehow on the  actual costs in
 the instance. We would like to emphasize that our proof on the number
 of  rounds  in  Theorem~\ref{smalltheorem}  is  robust  since  it  is
 completely  independent of the  cost structure  of the  instance. One
 could modify all the facility opening and connection costs,
the survival  of the
  fractional solution  is guaranteed.


 Theorem~\ref{theorem:ls-cfl} implies that the LS lift-and-project method
 fails to capture an efficient strong formulation for \cfl, 
including any useful preprocessing steps as sought by \cite{AnBS13}. 
It would be interesting to complement our  result with a similar one
on the SA hierarchy.

\iffalse ---------------- Ruling out exponential-sized LPs that can be
solved efficiently  or can be  used for analysis without  solving them
(e.g.,  through the  primal-dual method)  is a  much  more challenging
task.  Towards that direction ---- \fi 

Theorems~\ref{theorem:proper}
and \ref{theorem:proper2} on proper relaxations rule out a constant
integrality gap for ``configuration''-type symmetric LPs of
superpolynomial size, without any  assumptions on the time required to
solve them.  Obtaining a non-symmetric proper LP with a small
gap, if one  exists, seems to require looking  into the cost structure
of the instance, which would entail again some sort of preprocessing.
Of course, one should be careful about calling an algorithm with 
 drastic preprocessing    relaxation-based. 


Finally, we conjecture that there is a bad fractional solution for
\lbfl\  that  survives   rounds  of  the  LS
procedure.









\iffalse ----------------

[Here we clarify what type of relaxation based algorithm consist an acceptable answer to the question of \cite{ShmoysWbook}. In the quest of finding lps with constant gaps, one could argue that we could use a known constant factor -approximation algorithm for CFL as a subroutine to find the value  of an approximate solution and then just enhance the classic relaxation with a simple inequality that forces the total cost being at least . The generated relaxation is valid, polynomial sized, and has integrality gap of .  Although we can generate such a relaxation in polynomial time, it is not an acceptable answer since it solves the problem with another technique. We make the notion of acceptable answer to the problem strict by the following definition of a compact LP:

Let  a polynomial oracle  for CFL take as input the number of facilities  and the number of clients  and then generate a polynomially sized relaxation treating facility and connection costs as parameters. We define as compact relaxation the output of such an oracle. 

Thus we can see the procedure of producing a compact relaxation as the following two player game: the first player chooses the instance and provides the second player (the oracle ) with . The second player produce the inequalities and then the first player specifies facility and connection costs.]

---------------- \fi 




\bibliographystyle{plain}

\bibliography{bibliography-ver1}

 

\appendix

\section{Appendix to Section~\ref{sec:prel}}




\begin{lemma} {(\bf Folklore)}  \label{lemma:ap-classic}
Let  be an instance of \lbfl\ (\cfl\/) and  the
corresponding optimal values of 
relaxations (LP-classic) and (LP-star). Then 
\end{lemma}
\begin{proof}
It is easy to see that for any feasible solution to (LP-star) that
satisfies   for all  we can 
construct a solution to (LP-classic) of the same cost. We set   and  


For the converse, we are given a feasible solution  to
(LP-classic) and we wish to produce a solution  to (LP-star) of
the same cost.  We proceed to define the stars in the support of  

Fix a facility  with  Consider  a rectangle  of height 
 and width  By the
feasibility of    We consider  the quantity
 as fractional weight that we will pack 
within  
We divide the rectangle  into  vertical strips of width 
and height  that are initially
empty.  
We
start packing from height  Let  be the
current strip position. For the current client
 we pack weight within the current strip   starting from the current height
 and we
update  to   If 
 this means that we can pack  no more weight   at the
current position  we set  and pack the remaining quantity  in the next strip at position  Because  every client  will be fully packed by using at most 
two 
consecutive strips. By  the definition of  we have enough
area to pack all of  within  

For every value of  that was used by the packing algorithm   draw a horizontal
line that stabs  at this height. These lines partition   into
regions that are rectangles of width  
 Each of them intersects at least  non-empty vertical strips. Because for every   no two of 
these non-empty strips contain fractional weight corresponding to the  same  The clients
corresponding to  those strips, together with  form a star  We
set  equal to the height of the horizontal region. 
We repeat the process above for every  
It is easy
to see that in this way we have produced a solution  that is
feasible  for (LP-star) and has the  same cost as  
In the case of \cfl\ the proof is similar.
\end{proof} 









\end{document}
