

\documentclass[11pt,a4paper,colorlinks=true,urlcolor=blue,citecolor=red]{article}

\usepackage[T1]{fontenc}

\usepackage[boldsans]{concmath} \usepackage{euler} 

\usepackage[boxruled,vlined]{algorithm2e}
\usepackage{amsmath,amsfonts,amssymb,amsthm,dsfont,bm}\usepackage{enumitem,marvosym,graphicx}
\usepackage[justification=centering]{subfig}
\captionsetup[subfloat]{justification=centering}
\usepackage[margin=1in]{geometry}
\usepackage{thmtools}\usepackage{thm-restate}
\usepackage{hyperref}
\graphicspath{{figures/}}

\usepackage{color,soul} \newcommand{\xxx}[1]{}

\usepackage{refcount}
\newcommand{\pagedifference}[2]{\number\numexpr\getpagerefnumber{#2}-\getpagerefnumber{#1}\relax}

\newcommand{\figid}[1]{\marginpar{\small Figure ID: \##1}} \newcommand{\tr}{\ensuremath{\mathrm t}} \newcommand{\rootof}{\ensuremath{\mathop{\uparrow}}} \newcommand{\p}{\ensuremath{\mathrm p}} \newcommand{\treeat}[1]{B\langle{#1}\rangle} \newcommand{\treeatt}[2]{B{#1}\langle{#2}\rangle} \usepackage[nofancy]{svninfo} 

\svnInfo 

\theoremstyle{plain}
\newtheorem{theorem}{Theorem}
\newtheorem{lemma}[theorem]{Lemma}
\newtheorem{proposition}[theorem]{Proposition}
\newtheorem{corollary}[theorem]{Corollary}
\newtheorem{claim}[theorem]{Claim}
\newtheorem{observation}[theorem]{Observation}
\newcommand{\HRule}{\rule{\linewidth}{0.5mm}}
\newenvironment{proofof}[1]{\par\medskip\noindent\textbf{\sffamily Proof of #1.}~}{\qed\par\medskip}

\newcommand{\subsubparagraph}[1]{\paragraph{#1}}
\newcommand{\case}[1]{\par\vspace{.5\baselineskip}\noindent\textbf{\sffamily Case~#1}}
\newcommand{\EB}{\mathrm{E}(B)}

\title{The Planar Tree Packing Theorem ({R\svnInfoRevision})}

\begin{document}


\begin{titlepage}
  \begin{center}
    \HRule \0cm]
    \HRule \0.4cm]
        \large
        \textsc{Michael Kaufmann}\\
        \small
Universit\"{a}t T\"{u}bingen, Germany\\
        \verb|mk@informatik.uni-tuebingen.de|\0.4cm]
        \large
        \textsc{Vincent Kusters}\\
        \small ETH Z\"{u}rich, Switzerland\\
        \verb|vincent.kusters@inf.ethz.ch|\0.6cm]
    \begin{minipage}{0.45\textwidth}
      \begin{center} \large
        \textsc{Csaba D. T\'oth}\\
        \small \small California State University Northridge\\ Los
        Angeles, CA, USA\\
        \verb|cdtoth@acm.org|
      \end{center}
    \end{minipage}
  \end{center}

  \vspace{\baselineskip}

  \begin{center}
    {\large \svnToday}
  \end{center}

  \vspace{\baselineskip}

  \begin{abstract}
    Packing graphs is a combinatorial problem where several given graphs
    are being mapped into a common host graph such that every edge is
    used at most once. In the planar tree packing problem we are given
    two trees  and  on  vertices and have to find a planar
    graph on  vertices that is the edge-disjoint union of  and
    . A clear exception that must be made is the star which cannot
    be packed together with any other tree. But according to a
    conjecture of Garc\'ia et al.\ from 1997 this is the only exception,
    and all other pairs of trees admit a planar packing. Previous
    results addressed various special cases, such as a tree and a spider
    tree, a tree and a caterpillar, two trees of diameter four, two
    isomorphic trees, and trees of maximum degree three. Here we settle
    the conjecture in the affirmative and prove its general form, thus
    making it the planar tree packing theorem. The proof is constructive
    and provides a polynomial time algorithm to obtain a packing for two
    given nonstar trees.
\end{abstract}

  \vfill

  \begin{center}
    \includegraphics{tree_packing_example}
  \end{center}

  \vfill

\noindent
 \scriptsize Supported by the ESF EUROCORES programme EuroGIGA, CRP
 GraDR and the Swiss National Science Foundation, SNF
 Project 20GG21-134306.\\
 \scriptsize Supported by the NSF awards CCF-1422311 and
 CCF-1423615.


\end{titlepage}

\section{Introduction}\label{sec:introduction}
The \emph{packing problem} is to find a graph  on  vertices that
contains a given collection  of graphs on  vertices
each as edge-disjoint subgraphs. This problem has been studied in a wide
variety of scenarios (see, e.g., \cite{AkiyamaC90,CaroY97,FrSz}). Much
attention has been devoted to the packing of trees (e.g., tree packing
conjectures by Gy\'arfas~\cite{gl-ptdok-78} and by Erd\H{o}s and
S\'os~\cite{e-epgt-65}). Hedetniemi~\cite{MR629868} proved that any two
nonstar trees can be packed into . Teo and Yap~\cite{ty-ptgo-90}
showed, extending an earlier result by Bollob\'as and
Eldridge~\cite{be-pgacc-78}, that \emph{any} two graphs of maximum
degree at most  with a total of at most  edges pack into
 unless they are one of thirteen specified pairs of graphs. Maheo
et al{.}~\cite{msw-1996} characterized triples of trees that can be
packed into .

In the \emph{planar packing} problem the graph  is required to be
planar. Garc\'ia et al.~\cite{ghhnt-2002} conjectured in~1997 that there
exists a planar packing for any two nonstar trees, that is, for any two
trees with diameter greater than two. The assumption that none of the
trees is a star is necessary, since a star uses all edges incident to
one vertex and so there is no edge left to connect that vertex in the
other tree. Garc\'ia et al{.} proved their conjecture when one
of the trees is a path and when the two trees are isomorphic. Oda and
Ota~\cite{oo-2006} addressed the case that one of the trees is a
caterpillar or that one of the trees is a spider of diameter at most
four. A \emph{caterpillar} is a tree that becomes a path when all leaves
are deleted and a \emph{spider} is a tree with at most one vertex of
degree greater than two. Frati~et~al.~\cite{j-fgk-pptst-08} gave an
algorithm to construct a planar packing of any spider with any
tree. Frati~\cite{f-ppdft-09} proved the conjecture for the case that
both trees have diameter at most four. Finally,
Geyer~et~al.~\cite{gkh-ppbt-13} proved the conjecture for binary trees
(maximum degree three). In this paper we settle the general conjecture
in the affirmative:
\begin{theorem}\label{thm:planar_packing}
  Every two nonstar trees of the same size admit a planar packing.
\end{theorem}

\subsubparagraph{Related work.} Finding subgraphs with specific
properties within a given graph or more generally determining
relationships between a graph and its subgraphs is one of the most
studied topics in graph theory. The \emph{subgraph isomorphism}
problem~\cite{Epp-JGAA-99,GareyJ79,Ullmann76} asks to find a subgraph
 in a graph . The \emph{graph thickness} problem~\cite{mutzel}
asks for the minimum number of planar subgraphs which the edges of a
graph can be partitioned into. The \emph{arboricity}
problem~\cite{Epp-IPL-94} asks to determine the minimum number of
forests which a graph can be partitioned into. Another related classical
combinatorial problem is the  edge-disjoint spanning trees problem
which dates back at least to Tutte~\cite{t-pdgncf-61} and
Nash-Williams~\cite{nw-edstfg-61}, who gave necessary and sufficient
conditions for the existence of  edge-disjoint spanning trees in a
graph. The interior edges of every maximal planar graph can be
partitioned into three edge-disjoint trees, known as a \emph{Schnyder
  wood}~\cite{s-pgpd-89}. Gon\c{c}alves~\cite{1060666} proved that every
planar graph can be partitioned in two edge-disjoint outerplanar graphs.

The study of relationships between a graph and its subgraphs can also be
done the other way round. Instead of decomposing a graph, one can ask for a graph  that encompasses a given set of
graphs  and satisfies some additional properties. This
topic occurs with different flavors in the computational geometry and
graph drawing literature. It is motivated by applications in
visualization, such as the display of networks evolving over time and
the simultaneous visualization of relationships involving the same
entities. In the \emph{simultaneous embedding}
problem~\cite{BrassCDEEIKLM07}
the graph  is given and the goal is to draw it so that
the drawing of each  is plane. The \emph{simultaneous embedding
  without mapping} problem~\cite{BrassCDEEIKLM07} is to find a graph 
on  vertices such that: (i)  contains all 's as subgraphs,
and (ii)  can be drawn with straight-line edges so that the drawing
of each  is plane.

\section{Notation and Overview}\label{sec:def_overview}

A \emph{rooted tree} is a directed tree  with exactly one vertex of
outdegree zero: its root, denoted . Every vertex  has
exactly one outgoing edge . The target  is the
\emph{parent} of  in , and conversely  is a \emph{child} of
. In figures we denote the root of a tree by an outgoing
vertical arrow. For a vertex  of a rooted tree , denote by
 the \emph{subtree rooted at }, that is, the subtree of 
induced by the vertices from which  can be reached on a directed
path. The subscript is sometimes omitted if  is clear from the
context. A \emph{subtree of (or below) } is a tree , for a
child  of  in . For a tree , denote by  the \emph{size}
(number of vertices) of . We denote by  the degree
(indegree plus outdegree) of  in . For a graph  we denote by
 the edge set of . A \emph{star} is a tree on 
vertices that contains at least one vertex of degree . Such a
vertex is a \emph{center} of the star. A star on  vertices has a
unique center. For a star on two vertices, both vertices act as a
center. When considered as a rooted tree, there are two different rooted
stars on  vertices. A star rooted at a center is called
\emph{central-star}, whereas a star rooted at a leaf
that is not a center is called a \emph{dangling star}. In particular,
every star on one or two vertices is a central-star. A \emph{nonstar} is
a graph that is not a star. A \emph{substar} of a graph is a subgraph
that is a star.
A \emph{one-page book embedding} of a graph  is an embedding of 
into a closed halfplane such that all vertices are placed on the
bounding line. This line is called the \emph{spine} of the book
embedding.

We embed vertices equidistantly along the positive -axis and refer to
them by their -coordinate, that is, . An
\emph{interval}  in  is a sequence of the form
, for , or , for
. Observe that we consider an interval  as
oriented and so we can have . Denote the \emph{length} of an
interval  by .  A \emph{suffix} of an interval
 is an interval , for some . To avoid
notational clutter we often identify points from  with vertices
embedded at them.



\subsubparagraph{Overview.} We construct a plane drawing of two -vertex trees  and  on the point set
. We call  the \emph{blue tree}; its edges are shown as
solid blue arcs in figures. The tree  is called the \emph{red
  tree}; its edges are shown as dotted red arcs. The algorithm first
computes a preliminary one-page book embedding of  onto  (the
\emph{blue embedding}) in Section~\ref{sec:emb_t1}. In the second step
we recursively construct an embedding for the red tree to pair up with
the blue embedding. In principle we follow a similar strategy as in the
first step, but we take the constraints imposed by the blue embedding
into account. During this process we may reconsider and change the blue
embedding locally. For instance, we may \emph{flip} the embedding of
some subtree of  on an interval , that is, reflect the
embedded tree at the vertical line  through the
midpoint of . In some cases we also perform more drastic changes
to the blue embedding. In particular, the blue embedding may not be a
one-page book embedding in the final packing. Although neither of the
two trees  and  we start with is a star, it is possible---in
fact, unavoidable---that stars appear as subtrees during the recursion.
We have to deal with stars explicitly whenever they arise, because the
general recursive step works for nonstars only. We introduce the
necessary concepts and techniques in Section~\ref{sec:preT2} and give
the actual proof in Section~\ref{sec:embedding_the_red_tree}.




\section{A preliminary blue embedding}\label{sec:emb_t1}
We begin by defining a preliminary one-page book embedding
 for a tree  rooted at .
In every recursive step, we are given a tree  rooted at a vertex 
and an interval  of length . Recall that we may have 
or . We place  at position  and recursively embed the
subtrees of  on pairwise disjoint subintervals of
. The embedding is guided by two rules illustrated
in \figurename~\ref{fig:emb_T_1}.
\begin{itemize}
\item The \emph{larger-subtree-first rule} (LSFR) dictates that for any
  two subtrees of , the larger of the subtrees must be embedded on an
  interval closer to . Ties are broken arbitrarily.
\item The \emph{one-side rule} (1SR) dictates that for every vertex
all neighbors are mapped to the same side. That is, if  denotes the set of neighbors of  in 
  (including its parent), then either  for all
   or  for all .
\end{itemize}
These rules imply that every subtree  is embedded onto
an interval  so that  is an edge of 
and either  or  is the root of . Together with ,
these rules define the embedding (up to tiebreaking). An explicit
formulation of the algorithm can be found as
Algorithm~\ref{alg:embed_t1} below and an example is depicted in
\figurename~\ref{fig:emb_T_1:3}.
\begin{figure}[htbp]
  \centering \subfloat[LSFR]{\includegraphics{lsfr}\label{fig:emb_T_1:1}}\hfil \subfloat[1SR]{\includegraphics{1sr}\label{fig:emb_T_1:2}}\hfil \subfloat[A preliminary blue embedding.]{\includegraphics{title_blue}\hspace{2em}\includegraphics{title_blue_embed}\label{fig:emb_T_1:3}}\caption{Illustrations for the two rules and an example embedding.\label{fig:emb_T_1}}
\end{figure}


\begin{algorithm}[H] \label{alg:embed_t1}\newcommand{\alet}{\leftarrow}\newcommand{\id}[1]{\mathit{#1}}\DontPrintSemicolon \KwIn{A rooted tree  and a directed interval  with .}\KwOut{A map .}

  Let  be the root of  and let .\;\;\If{}{Let  be the children of  in  such that
    .\;\;\For{}{.\;}
    \eIf{}{\For{}{\;}
    }{\For{}{\;}
    }
  }
  \caption{.}
\end{algorithm}

\section{A red tree and a blue forest}\label{sec:preT2}

As common with inductive proofs, we prove a stronger statement than
necessary. This stronger statement does not hold
unconditionally but we need to impose some restrictions on the input.
The goal of this section is to derive this more general
statement---formulated as Theorem~\ref{thm:main}---from which
Theorem~\ref{thm:planar_packing} follows easily.

Our algorithm receives as input a nonstar subtree  of the red tree
and an interval  of size  along with a blue graph 
embedded on . Without loss of generality we assume . In the
initial call  is a tree, but in a general recursive call  is a
\emph{blue forest} that may consist of several components. For
 let  denote the component of  that contains
. For  let  denote the subgraph of 
induced by the vertices in , and for  let  denote the
component of  that contains .



In general the algorithm sees only a small part of the overall picture
because it has access to the vertices in  only. However, blue
vertices in  may have edges to vertices outside of  and also
vertices of  may have neighbors outside of . We have to ensure
that such \emph{outside edges} are used by one tree only and can be
routed without crossings.
In order to control the effect of outside edges, we allow only one
vertex in each component---that is, the root of  and the root of each
component of ---to have neighbors outside of . Whenever we change
the blue embedding we need to maintain the relative order of these roots
so as to avoid crossings among outside edges.

\subsubparagraph{Conflicts.} Typically  has at least one
neighbor outside of : its parent . But  may also have
children in . We assume that all neighbors---parent and
children---of  in  are already embedded outside of
 when the algorithm is called for . There are two principal
obstructions for mapping  to a point :
\begin{itemize}
\item A vertex  is in \emph{edge-conflict} with , if
   for some neighbor  of  in
  . Mapping  to  would make  an edge of
  both  and 
  (\figurename~\ref{fig:conflicts_1}--\ref{fig:conflicts_2}). In figures
  we mark vertices in edge-conflict with  by a lightning symbol
  \Lightning.
\item A vertex  is in \emph{degree-conflict} with  on  if
  . If we map  to , then no child of
   in  can be mapped to the same vertex as a child of  in
  . With only  vertices available there is not enough room for
  both groups (\figurename~\ref{fig:conflicts_3}).
\end{itemize}
\begin{figure}[htbp]
  \centering \subfloat[]{\includegraphics{edge_conflict}\label{fig:conflicts_1}}\hfil
  \subfloat[]{\includegraphics{edge_conflict_1}\label{fig:conflicts_2}}\hfil
  \subfloat[]{\includegraphics{degree_conflict}\label{fig:conflicts_3}}\hfil
  \caption{An interval  on which a tree  is to be
    embedded. Two neighbors  and  of  in  are
    already embedded (a). Then the situation on  presents itself
    as in (b), where the three central vertices are in edge-conflict
    with  due to blue outside edges to  or .  In (c) the vertex
     is in degree-conflict with  because
    . We cannot map  to the blue
    vertex at  because there is not enough room for the neighbors of
    both in .\label{fig:conflicts}}
\end{figure}

We cannot hope to avoid conflicts entirely and we do not need to. It
turns out that is sufficient to avoid a very specific type of conflict
involving stars.
\begin{itemize}
\item An interval  is in \emph{edge-conflict}
  (\emph{degree-conflict}) with  if  is a
  central-star and the root of  is in edge-conflict
  (degree-conflict) with  (\figurename~\ref{fig:starconflicts}).
\item An interval  is in \emph{conflict} with  if  is in
  edge-conflict or degree-conflict with  (or both).
\end{itemize}

\begin{figure}[htbp]
  \centering \subfloat[]{\includegraphics{edge_star_conflict-0}\label{fig:starconflicts_1}}\hfil
  \subfloat[]{\includegraphics{edge_star_conflict}\label{fig:starconflicts_1a}}\hfil
  \subfloat[]{\includegraphics{edge_star_conflict-1}\label{fig:starconflicts_2}}\hfil
  \subfloat[]{\includegraphics{edge_star_conflict-2}\label{fig:starconflicts_3}}\hfil
  \subfloat[]{\includegraphics{edge_star_conflict-3}\label{fig:starconflicts_4}}\hfil
  \caption{An interval  in edge-conflict (a)--(b), and examples
    where  is not in edge-conflict (c)--(e). In (c) the center of
     is not in edge-conflict; it may be in degree-conflict, though,
    if . In both (d) and (e) the tree  is
    not a central-star.\label{fig:starconflicts}}
\end{figure}



\noindent
The following lemma shows that a degree-conflict cannot be caused by a
very small star.
\begin{restatable}{lemma}{degconthree}\label{lem:degcon3}
  If an interval  is in degree-conflict with a nonstar subtree
   of , then  is a central-star on at least three
  vertices.
\end{restatable}
\begin{proof}
  By the definition of degree-conflict for ,  is
  a central-star. Let  denote its root.  Then a degree-conflict
  implies . As  is not a star, we have
  . Therefore , that is,
  .
\end{proof}

We claim that  can be packed with  onto  unless  is in
conflict with . The following theorem presents a precise formulation
of this claim. Only  and the graph  determine whether or
not an interval  is in conflict with . Therefore we can phrase
the statement without referring to an embedding of  but just
regarding it as a sequence of trees. The set  represents the set of
roots from  that are in edge-conflict with .
\begin{theorem}\label{thm:main}
  Let  be a nonstar tree with  and let  be a nonstar
  forest with , together with an ordering  of
  the  roots of  and a set
  . Suppose (i)  is not a central-star or (ii)  and
  . Then there is a plane packing  of 
  and  onto any interval  with  such that
\begin{itemize}
  \item  and
  \item we can access  in this order from the outer
    face of , that is, we can add a new vertex  in the outer
    face of  and route an edge to each of  such
    that the resulting multigraph is plane and the circular order of
    neighbors around  is . (If , for some
    , then two distinct edges must be routed from
     to  so that the result is a non-simple plane multigraph.)
  \end{itemize}
  Such a packing  we call an \emph{ordered plane packing} of 
  and  onto .
\end{theorem}

\noindent
Theorem~\ref{thm:main} is a strengthening of
Theorem~\ref{thm:planar_packing} and so we obtain
Theorem~\ref{thm:planar_packing} as an easy corollary.
\begin{proofof}{Theorem~\ref{thm:planar_packing} from
    Theorem~\ref{thm:main}}
  Select roots arbitrarily so that  and .
Then use Theorem~\ref{thm:main} with , , ,
  , and . By assumption  is not a star and so
  (i) holds. Therefore we can apply Theorem~\ref{thm:main} and obtain
  the desired plane packing of  and .
\end{proofof}

It is not hard to see that forbidding conflicts in
Theorem~\ref{thm:main} is necessary: The example families depicted in
\figurename~\ref{fig:confness} do not admit an ordered plane packing.
\begin{figure}[htbp]
  \centering \subfloat[]{\includegraphics{understanding_ec_k}\label{fig:confness:1}}\hfil \subfloat[]{\hspace{1cm}\includegraphics{understanding_dc_k}\hspace{1cm}\label{fig:confness:2}}\hfil \caption{The statement of Theorem~\ref{thm:main} does not hold without
    (i) or (ii). In the examples the trees of  are ordered from left
    to right so that  is a central-star. Vertices in  are
    labeled with \Lightning.\label{fig:confness}}
\end{figure}


\paragraph{Runtime analysis.} The algorithm is parameterized with a
subtree  of  and an interval , which  is to
be packed onto together with an already embedded subforest of . If
we represent  as an adjacency matrix and the embeddings as arrays,
then after an  time initialization we can test in constant time
for the presence of an edge between . To represent  we
use an adjacency list where the children are sorted by the size of their
subtrees, which can be precomputed in  time. Then at each
step, the algorithm spends  time and makes at most two recursive
calls with disjoint sub-intervals of , which yields  time
overall.

\section{Embedding the red tree: fundamentals}\label{sec:embedding_the_red_tree}

In this section we discuss some fundamental tools for our recursive
embedding algorithm to prove Theorem~\ref{thm:main}. First we formulate
four invariants that hold for every recursive call of the
algorithm. Next we present three tools that are specific types of
embeddings to handle a ``large'' substar of  or . All of these
embeddings rearrange the given embedding of  to make room for the
center of the star. Finally, we conclude with an outline of the
algorithm.

\subsection{Invariants}

In the algorithm we are given a red tree , a blue forest 
with roots , an interval  with
, and a set  that we consider to be the vertices from
 in edge-conflict with . As a first step, we embed  onto  by
embedding  in this order from left to right,
each time using the algorithm from Section~\ref{sec:emb_t1}.
\begin{observation}\label{obs:invariants}
  We may assume that ,  and  satisfy the following
  invariants:\normalfont
\begin{enumerate}[leftmargin=*,label={(I\arabic*)}]\setlength{\itemindent}{\labelsep}
  \item\label{inv:starconflict}  is not in conflict with .
    \emph{(peace invariant)}
  \item\label{inv:bluelocal} Every component of  satisfies LSFR and
    1SR. All edges of  are drawn in the upper halfplane (above the
    -axis).
All roots of  are visible from above (that is, a vertical ray
    going up from  does not intersect any edge of
    ). \emph{(blue-local invariant)}
  \item\label{inv:placement}  is not in edge-conflict with
    . \emph{(placement invariant)}
  \end{enumerate}
\end{observation}
\begin{proof}
  \ref{inv:starconflict} follows from the assumption (i) or (ii) in
  Theorem~\ref{thm:main}. \ref{inv:bluelocal} is achieved by using
  the embedding from Section~\ref{sec:emb_t1}. If  is in conflict
  with , then \ref{inv:starconflict} implies that  is not
  a singleton (which would be a central-star). Therefore flipping
   establishes \ref{inv:placement} without affecting
  \ref{inv:starconflict} or \ref{inv:bluelocal}.
\end{proof}

Theorem~\ref{thm:main} ensures that all roots of  along with 
appear on the outer face in the specified order. We cannot assume that
we can draw an edge to any other vertex of  or  without crossing
edges of the embedding given by Theorem~\ref{thm:main}. Therefore it is
important that whenever the algorithm is called recursively,
\begin{enumerate}[leftmargin=*,label={(I\arabic*)},start=4]\setlength{\itemindent}{\labelsep}
\item\label{inv:rootsonly} only the roots  and  have
  edges to the outside of .
\end{enumerate}
Assuming 1SR for  helps when splitting intervals for recursive
treatment.
\begin{observation}\label{obs:bluelocal}
  If  satisfies \ref{inv:bluelocal} and \ref{inv:rootsonly} on an
  interval , then both invariants also hold for  on ,
  for every subinterval .
\end{observation}
In the remainder of the proof we will ensure and assume that invariants
\ref{inv:starconflict}--\ref{inv:rootsonly} hold for every call of the
algorithm. For the initial instance of packing  and , we know
that \ref{inv:starconflict}--\ref{inv:placement} hold by
Observation~\ref{obs:invariants} and \ref{inv:rootsonly} holds trivially
because there are no vertices outside of .


\subsection{Blue-star embedding}
\label{sec:greedy_grab_embedding}
The blue-star embedding is useful to handle the center  of a
substar  of . It explicitly embeds a subtree  of  onto a
part of  that includes . It may use some of the leaves of
. After taking care of , any unused leaf of  appears
as a locally isolated vertex in the remaining interval of vertices.

The blue-star embedding consists of several steps:
It rearranges some vertices of , moves some edges of  below the
-axis, and introduces edges that straddle both halfplanes above and
below the -axis (\figurename~\ref{fig:greedygrabex}).
\begin{figure}[htbp]
  \centering\hfil \subfloat[]{\includegraphics{greedygrab_2}\label{fig:greedygrabex:1}}\hfil
  \subfloat[]{\includegraphics{greedygrab_0}\label{fig:greedygrabex:2}}\hfil
  \subfloat[Result]{\includegraphics{greedygrab_5}\label{fig:greedygrabex:3}}\hfil
  \caption{blue-star embedding  onto a part of
    .\label{fig:greedygrabex}}
\end{figure}


Suppose that  is a subtree of  with  (possibly
) and  is the center of a star .
Either  is the root of  or
. Denote by  the subgraph of 
induced by  and all its neighbors (parent and children). Note
that either  or . Put  and
let  be a sequence of elements from
. Furthermore, suppose the following four conditions
hold:
\begin{enumerate}[label={(BS\arabic*)}]\setlength{\itemindent}{4\labelsep}
  \item\label{gg:ec}  is not in edge-conflict with ,
  \item\label{gg:dc}  and
    ,
  \item\label{gg:int} at least one of  or
     forms an interval, and
  \item\label{gg:cs} if  does not form an
    interval, then  is not a central-star,  and
    .
  \end{enumerate}
Note that \ref{gg:cs} is a trivial consequence of \ref{gg:int} in case
. Furthermore, \ref{gg:ec} is trivially satisfied if no
neighbor of  in  has been embedded yet.

Let  denote the children of  in  such that
. Partition the leaves of 
into  groups  such that ,
for , and . We
intend to embed the vertices of  on the
leaves in . Note that some (possibly all) of the sets  may be
empty. Also note that
, where the
 accounts for . Therefore
 is nonnegative by
\ref{gg:dc} and so our assignment is well-defined.

If  does not form an interval, then by
\ref{gg:cs}  is not a central-star and so . In this case,
we move one leaf from  to  and add  to 
instead.

The \emph{blue-star embedding of  from  with }
proceeds in four steps, as detailed below. The first two steps rearrange
the embedding of  to make room for the embedding of  in the third
step. The fourth step ensures that the remaining unused vertices appear
in a form that allows to further process them.

\subsubparagraph{Step~1 (Flip)} We draw all edges of  below the
spine. All edges of  not inside  remain above the spine
(\figurename~\ref{fig:greedygrab_1}).

\subsubparagraph{Step~2 (Mix)} Leaving  where it is, we
distribute the leaves of  among the vertices in  as
follows: for , move the vertices of  so that
they appear as a contiguous subsequence immediately to the right of
 (\figurename~\ref{fig:greedygrab_3}). If
 does not form an interval, then we have
 in . As  is not a leaf of , we cannot move it
around so easily. Fortunately, no relocation is necessary because by
\ref{gg:cs}  appears right next to  in . Any
remaining vertices in  are placed between  and .
\begin{figure}[htbp]
  \centering\hfil \subfloat[flip]{\includegraphics{greedygrab_1}\label{fig:greedygrab_1}}\hfil
  \subfloat[mix]{\includegraphics{greedygrab_3}\label{fig:greedygrab_3}}\hfil
  \subfloat[complete]{\includegraphics{greedygrab_4}\label{fig:greedygrab_4}}\hfil
  \subfloat[cleanup]{\includegraphics{greedygrab_5}\label{fig:greedygrab_5}}\hfil
  \caption{The example from \figurename~\ref{fig:greedygrabex} in
    detail. We blue-star embed  from  where 
    takes the vertices of  from right to
    left.\label{fig:greedygrab}}
\end{figure}


\subsubparagraph{Step~3 (Complete)} Embed  by first mapping  to
, which is possible by \ref{gg:ec}. Next map  to , for
, drawing the edge to  below the spine. Then
embed each subtree  explicitly (using
Algorithm~\ref{alg:embed_t1} and drawing all edges above the spine) on
the interval of  locally isolated vertices immediately to
the right of  (\figurename~\ref{fig:greedygrab_4}). Note that
 is locally isolated even if
 does not form an interval because by
\ref{gg:cs} we have .

It remains to describe the embedding for . Before we do this,
let us consider the properties that we want the embedding to fulfill.
Note that the blue-star embedding---as far as described---does not use
any of the invariants \ref{inv:starconflict}--\ref{inv:bluelocal} other
than that we start from a one-page book embedding. However, if
\ref{inv:starconflict}--\ref{inv:bluelocal} hold for , then we would
like to maintain these invariants also for the part
 of 
that is not yet used by  after the blue-star embedding. A necessary
prerequisite is that  forms an interval, that is, the vertices of
 appear as a contiguous subsequence of . Given that we are
still free to place the vertices in , it is enough that the
vertices in  form a subinterval of  that is
reachable from  (without crossing edges).

\subsubparagraph{Step~4 (Cleanup)} Suppose without loss of generality
that  is to the right of . (If  is
to the left of , replace all occurrences of
``right'' by ``left'' in the following paragraph.)

Move the vertices of  so that they appear as a contiguous
subsequence immediately to the right of the rightmost vertex  of
. In order to establish that all edges are drawn
above the spine, we cannot draw the edges between  and 
in the same way as we did for  above. Instead we route
all edges between  and  as parallel biarcs (curves that
cross the spine once) that leave  below the spine, then cross
the spine just to the right of the rightmost vertex of , and
finally enter their destination from above
(\figurename~\ref{fig:greedygrab_5}). As a result, for the purpose of
embedding some part of  onto , the vertices of 
become isolated roots; each is connected with a single edge to the
outside that is (locally) routed in the upper halfplane.

This completes the description of the blue-star embedding. Below is a
formal statement summarizing the pre- and postconditions.
\begin{proposition}\label{p:greedygrab}
  Let  be a subtree of , let  be the
  center of a star , and let  be a sequence
  of  pairwise distinct vertices from , where
   denotes the subgraph of  induced by  and all its
  neighbors. If  and  fulfill \ref{gg:ec}--\ref{gg:cs}, then
  the blue-star embedding of  from  with  provides
  an ordered plane packing of  onto , for some
  subinterval .

  Furthermore,  after the blue-star
  embedding, where , if , and , if
  . Put , if
   is an interval, and
  , otherwise. Then  is the union
  of  with some (possibly empty) sequence of isolated vertices on the
  side of  opposite from .

  Finally, if the embedding of  on  initially satisfies
  \ref{inv:bluelocal}, then after the blue-star embedding the modified
  embedding of  on  satisfies \ref{inv:bluelocal}.
\end{proposition}
\begin{proof}
  The packing for  is immediate by construction. Let us first argue
  that after the blue-star embedding an interval 
  remains. We distinguish two cases.

  If  is an interval, then the embedding
  uses exactly the vertices of , and
  the vertices of  are placed so that they extend the interval
  .

  Otherwise,  does not form an
  interval. Then the embedding uses exactly the vertices of
   (where one vertex originally in
   is moved to ). By \ref{gg:int} we know that
   forms an interval and the vertices of
   are placed so that they extend this interval.

  Next we argue that . By \ref{gg:dc}
  we have . As  consists of
   vertices, at least one vertex in  is not in
  . Due to the way we run the cleanup step, it follows that the
  vertex of  furthest from  is in 
  (whereas the closest vertex may be in , which is adjacent to
  ). By construction no vertex of  is adjacent
  to  in . The description of  holds by
  construction.

  It remains to argue that if  satisfies \ref{inv:bluelocal}, then so
  does . The blue-star embedding does not change the order
  of the vertices in  and the vertices of 
  become isolated roots. Given the way the edges incident to 
  have been drawn, they do not affect the visibility of the roots in
  .  Therefore, \ref{inv:bluelocal} holds for
  . The validity of \ref{inv:bluelocal} for the
  vertices in  follows from the discussion in Step~4 above.
\end{proof}

\subsection{Red-star embedding}
There is a natural counterpart to the blue-star embedding that we call
\emph{red-star embedding}. It embeds a red central-star onto a blue
tree.

Consider an interval  on which we wish to embed a subtree 
of  that is a central-star with . Consider some 
such that  is the root of .  Let
 and let  denote the children of
 in , such that  is the subtree closest to
.  Choose any interval  such
that  is either completely inside or completely outside
, for every . See
\figurename~\ref{fig:sgg_setup}. We require that
\begin{enumerate}[label={(RS\arabic*)}]\setlength{\itemindent}{3em}
\item\label{sgg:ec}  is not in edge-conflict with  and
\item\label{sgg:dc} .
\end{enumerate}
Note that~\ref{sgg:ec} and~\ref{sgg:dc} are analogous to~\ref{gg:ec} and
\ref{gg:dc}, but only one inequality is needed in~\ref{sgg:dc}. In the
blue-star embedding, we need~\ref{gg:int} and~\ref{gg:cs} to handle
central-stars whose parent is also present in the interval under
consideration. In the red-star embedding, we have no requirements on 
other than \ref{sgg:ec} and \ref{sgg:dc}.

\subsubparagraph{Step 1 (Embed)} First embed  at . This works
by~\ref{sgg:ec}. Let  and let  denote
the children of  in . By~\ref{sgg:dc} the interval  contains
enough vertices not adjacent to  in order to embed
. Let  be the set of the  closest non-neighbors of
 in . Embed  onto . We next describe how
to draw the red edges from  to . Consider a vertex
 and let  be the vertex of the blue forest we embedded 
onto.  Refer to \figurename~\ref{fig:sgg_embed}. If ,
then draw  as a semi-circle in the lower halfplane. If
 with  then draw  as a biarc that
is in the upper halfplane near , in the lower halfplane near ,
and crosses the spine between  and .  Finally, if
, then draw  as a biarc that is in the
upper halfplane near , in the lower halfplane near , and crosses
the spine right after . Afterwards, the vertices of
, i.e. the blue vertices that are not mapped to any
, are visible from below.

\begin{figure}
  \centering\hfil \subfloat[Setup.]{\includegraphics{sgg_setup}\label{fig:sgg_setup}}\hfil \subfloat[Embed.]{\includegraphics{sgg_embed}\label{fig:sgg_embed}}\hfil \subfloat[Cleanup.]{\includegraphics{sgg_cleanup}\label{fig:sgg_cleanup}}\hfil \caption{Using the red-star embedding to embed  with
    .}
  \label{fig:sgg}
\end{figure}


\subsubparagraph{Step 2 (Cleanup)} In general, the vertices of
 do not form an interval. Assume without loss of
generality that  is the rightmost vertex of . Let
. We rearrange the vertices on : from left to
right, we first place all vertices of  (maintaining
their relative order) and then all vertices of  (maintaining their
relative order). Refer to \figurename~\ref{fig:sgg_cleanup}. In
particular,  is still at the rightmost position after this
rearrangement. The edges of  are drawn as before, as
are the edges of . We must redraw the edges that have one end
vertex in  and one in . The edges
 are drawn as triarcs: the edge is in the upper
halfplane near  and . Its first spine intersection is to
the right of the rightmost vertex of . Its second
spine intersection is such that it maintains the cyclic order of edges
leaving  (as before the rearrangement). The other edges are
drawn similarly.

The pre- and postconditions of the red-star embedding are
summarized by the following proposition.

\begin{proposition}\label{prop:stargreedygrab}
  Let  be an interval for which  satisfies
  \ref{inv:bluelocal}. Let  be a subtree of  that is a
  central-star. Consider some  such that  is the
  root of . Let  and denote the
  children of  in  by . Let
   be any interval such that
   is either completely inside or completely outside ,
  for every .

  If  and  and  fulfill~\ref{sgg:ec}
  and~\ref{sgg:dc}, then the red-star embedding of  from 
  on  provides an ordered plane packing of  onto a subinterval
   of . The set  forms an interval that satisfies
  \ref{inv:bluelocal} and consists of  followed by some
  vertices (possibly zero) originally in .
\end{proposition}
\begin{proof}
  As argued above, Step~1 produces a plane packing of  and 
  by~\ref{sgg:ec} and~\ref{sgg:dc}. Any remaining vertices of
   remain visible from below. Furthermore, if a
  subtree of  is embedded onto a (directed)
  interval  with the root at , then Step~1 embeds children of
   on a (possibly empty) suffix of . Since Step~2 does not
  change the relative position of the remaining vertices of
   nor the relative position of the other
  vertices in , the set  satisfies
  \ref{inv:bluelocal} after Step~2.
\end{proof}

\subsection{Leaf-isolation shuffle}
While we are at discussing how to deal with red stars, let us introduce
another basic operation that will turn out useful in this context.

Suppose we need to embed a substar  onto a subinterval
. Then we need to pair the center of  with an
isolated vertex in . If there is no such vertex, we occasionally
embed  onto  after a rearrangement of  that ensures
that  has a suitable isolated vertex. The goal of such a
\emph{leaf-isolation shuffle} is to modify  so that a leaf of
 is at  and its parent is at
. \figurename~\ref{fig:leafshuffle_2} shows the result of performing
a left-isolation shuffle on \figurename~\ref{fig:leafshuffle_0} with
. The idea is then to take the parent out of the interval
by embedding  onto  instead and mapping the center of
 to , which is locally isolated on . The
proposition below guarantees that such a leaf-isolation shuffle is
always possible. Note that we do not care about the invariant
\ref{inv:bluelocal} in this scenario because we cannot use a recursive
embedding for a star anyway. There is one part of the invariant that we
need to maintain, though, which is the visibility of the blue root from
above.
\begin{figure}[htbp]
  \centering\hfil \subfloat[]{\includegraphics{leafshuffle_0}\label{fig:leafshuffle_0}}\hfil
  \subfloat[]{\includegraphics{leafshuffle_1}\label{fig:leafshuffle_1}}\hfil
  \subfloat[]{\includegraphics{leafshuffle_2}\label{fig:leafshuffle_2}}\hfil
  \caption{A leaf is shuffled into position , with its parent at
    .\label{fig:leafshuffle}}
\end{figure}

Below is a formal statement summarizing the conditions and properties of
the leaf-isolation shuffle.
\begin{proposition}\label{prop:leafshuffle}
  Every rooted tree  on  vertices admits a one-page book
  embedding onto  such that  is visible from
  above,  is a leaf  of , and  is the parent of
  . Moreover, if  is a central-star, then ; otherwise,
  .
\end{proposition}
\begin{proof}
  We use induction on . Clearly the statement holds for .
  For  we start by constructing a one-page book embedding for
   with a modified version of Algorithm~\ref{alg:embed_t1} where we
  invert the order of subtrees, that is, we use a ``smaller subtree
  first rule'' (SSFR). By starting from  and placing it at  we
  ensure that it is visible from above. As  is a tree, the embedding
  uses the edge . If  is a leaf of , then  is its
  parent and by SSFR  is a central-star. Therefore, flipping 
  yields the desired embedding. Otherwise, let 
  denote the smallest (index) neighbor of  and obtain the desired
  embedding inductively for  (whose root is ). The root of
  this subtree  ends up at either  or , both of which are
  visible from above. Therefore, we can complete the embedding by
  routing all edges from  or  to the existing forest on
  . \figurename~\ref{fig:leafshuffle} illustrates
  the execution of the leaf-isolation shuffle on an example. The root
   is at  if and only if  is a central-star; otherwise, it
  remains at .
\end{proof}

\subsection{Algorithm outline}\label{proofstart}

Recall that we are given a red tree , a blue forest  with
roots , an interval  with
, and a set  of vertices in edge-conflict with
.

Let  denote a child of  that minimizes  among all
children  of  in . Denote  and .
If , then  cannot be a central-star: if it were, then
 and  would be a star. Another easy consequence of the choice
of  is the following.
\begin{restatable}{lemma}{degr}\label{lem:degr}
  If , then .
\end{restatable}
\begin{proof}
  Set  and suppose to the contrary that
  . Adding  on both sides of the
  inequality yields . By the minimality of  we have
  . Solving for  and combining with the previous
  inequality yields
  
  which is impossible, given that .
\end{proof}


Ideally, we can recursively embed  onto  and  onto
 (\figurename~\ref{fig:ideal:1}).  But in general the
invariants may not hold for the recursive subproblems. For instance,
some of the subgraphs could be stars, or if , then
placing  at  may put  in edge-conflict with
. Therefore, we explore a number of alternative strategies, depending
on which---if any---of the four forests , ,  and
 in our decomposition is a star.
\begin{figure}[htbp]
  \centering \subfloat[]{\includegraphics{ideal}\label{fig:ideal:1}}\hfil
  \subfloat[]{\includegraphics{ideal2}\label{fig:ideal:2}}\hfil
  \caption{Our recursive strategy in an ideal world.\label{fig:ideal}}
\end{figure}


To complete the proof of Theorem~\ref{thm:main} we distinguish seven
cases. In each of these seven cases, we follow the notation of and
assume the preconditions discussed above. First, in
Section~\ref{subsec:rec_general} we discuss the general case, where none
of the four forests is a star. Then, in Section~\ref{subsec:rec_unary}
and Section~\ref{subsec:rec_singleton} we handle the special cases
 and , respectively. The final four sections each
correspond to one of the four forests being a star. Capturing the
general intuition we refer to  as ``large'' and to  as
``small'', although they may have almost the same size and---in special
cases, like --- may actually be larger than .


\section{Embedding the red tree: the general case}
\label{subsec:rec_general}
In the general case, we suppose that none of the subtrees in our current
decomposition is a star.
\begin{lemma}\label{lem:rec_general}
  If none of , , , and  is a star,
  then there is an ordered plane packing of  and  onto .
\end{lemma}
\begin{proof}
  As  is a minimum size subtree of  in , and neither  nor
   is a star, we know that  has at least one more subtree other
  than  and every subtree of  in  has size at least four. (All
  trees on three or less vertices are stars.) It follows that



  The general plan is to use one of the following two options. In both
  cases we first embed  recursively onto . Then we
  conclude as follows.
  \begin{description}
  \item[Option 1:] Embed  recursively onto 
    (\figurename~\ref{fig:ideal:1}).
  \item[Option 2:] Embed  recursively onto 
    (\figurename~\ref{fig:ideal:2}).
  \end{description}
  In some cases neither of these two options works and so we have to use
  a different embedding.

  As we embed  after , the (final) mapping for  is not known
  when embedding . However, we need to know the position of  in
  order to determine the conflicts for embedding . Therefore,
  before embedding  we \emph{provisionally} embed  at
   (Option 1) or
   (Option 2).  That
  is, for the recursive embedding of  we pretend that some neighbor
  of  is embedded at . In this way we ensure that  is not
  in edge-conflict with the interval in its recursive embedding.
The final placement for  is then determined by the recursive
  embedding of , knowing the definite position of its parent .

  For the recursive embeddings to work, we need to show that the
  invariants \ref{inv:starconflict}, \ref{inv:bluelocal} and
  \ref{inv:rootsonly} hold (\ref{inv:placement} then follows as in
  Observation~\ref{obs:invariants}). For \ref{inv:bluelocal} and
  \ref{inv:rootsonly} this is obvious by construction and
  Observation~\ref{obs:bluelocal}, as long as we do not change the
  embedding of . As we do not change the embedding in Option~1 and 2,
  it remains to ensure \ref{inv:starconflict}. So suppose that for both
  options, \ref{inv:starconflict} does not hold for at least one of the
  two recursive embeddings. There are two possible obstructions for
  \ref{inv:starconflict}: edge-conflicts and degree-conflicts. We
  discuss both types of conflicts, starting with edge-conflicts.

  \case{1}  is not in degree-conflict with  and
   is not in degree-conflict with . Then Option~1 works,
  unless  is in edge-conflict with . Recall that
   is not in edge-conflict with  after embedding 
  onto  due to the provisional placement of .


  We claim that an edge-conflict between  and  implies
  . To prove this claim, suppose that  is in
  edge-conflict with . Then  is a
  central-star whose root  is in edge-conflict with . If ,
  then by \ref{inv:placement} there was no such conflict initially (for
   and ). So, as claimed, the conflict can only come from a
  blue edge to  (provisionally placed) at . Otherwise,  and
  by 1SR there is no edge in  from  to any point in . It
  follows that . The conflict between
   and  does not come from the edge to  but from an edge to a
  vertex outside of . This contradicts \ref{inv:starconflict} for
   and , which proves the claim.

  The presence of the edge  implies that  is a tree and by
  \ref{inv:rootsonly} only (the root)  or  may have edges out of
  . Consider Option~2, which embeds  onto ,
  provisionally placing  at
  .  There are two possible
  obstructions: an edge-conflict for  or a degree-conflict for
  . In both cases we face a central-star  with
  center . Due to 1SR and , we know
  that . We distinguish three
  cases.

  \case{1.1} . Then we consider a third option:
  provisionally place  at , embed  recursively onto  and then  onto  (\figurename~\ref{fig:general2_1}). The edge
   of  prevents any edge-conflict between  and
   (and, as before, for ). Given that we assume in Case~1 that
   is not in degree-conflict with , we are left with
   being in degree-conflict with  as a last possible
  obstruction.
\begin{figure}[htbp]
    \centering\hfil \subfloat[]{\includegraphics{general_4}\label{fig:general2_1}}\hfil
    \subfloat[]{\includegraphics{general_5}\label{fig:general2_2}}\hfil
    \caption{A third embedding when the first two options
      fail.\label{fig:general2}}
  \end{figure}


\noindent
  Then the tree  is a central-star 
  with root  such that



  \noindent
  Combining Lemma~\ref{lem:degr} with \eqref{eq:degcon5} we get
. Note that  can be huge, but
  we know that it does not include  (because  is not a
  star). We also know that :
If , then by 1SR we have
, in contradiction to
  . Therefore 
  and by 1SR its parent is to the right. Due to  and
  since  is a tree rooted at , we have . As
   is a subtree of  in  on at least five vertices, by LSFR
   cannot have a leaf at . Therefore, the star
   consists of a single vertex only,
  that is,  (\figurename~\ref{fig:general2_2}). We consider
  two subcases.
In both the packing is eventually completed by recursively embedding  onto
  .

  \case{1.1.1} , for some 
  (\figurename~\ref{fig:general3_1}). Select  to be maximal with this
  property. Then we exchange the order of the two subtrees  and
   of  (\figurename~\ref{fig:general3_2}). This may violate LSFR
  for  at , but \ref{inv:bluelocal} holds for both 
  and .
Clearly there is still no edge-conflict for  with 
  after this change. We claim that there is no degree-conflict anymore,
  either.

  \begin{figure}[htbp]
\subfloat[]{\includegraphics{general_1}\label{fig:general3_1}}\hfill
    \subfloat[]{\includegraphics{general_2}\label{fig:general3_2}}\hfill
    \subfloat[]{\includegraphics{general_6}\label{fig:general4}}
    \caption{Swapping two subtrees of  in Case~1.1.1 and an explicit
      embedding for Case~1.1.2.\label{fig:general3}}
  \end{figure}

  To prove the claim, note that by LSFR at  we have
  . As the size of both subtrees combined is at most
  , we have .  Then, using~\eqref{eq:degr},
  .
  Therefore after the exchange  is not in degree-conflict
  with , which proves the claim and concludes this case.

  \case{1.1.2}  and  are the only neighbors of 
  in . We claim that in this case  extends all the way up to
  , that is, .
  To prove this claim, suppose to the contrary that . Then
  there is another subtree of  to the left of  and, in particular,
  . By LSFR this closer subtree is at least as large as . Using \eqref{eq:degr} and
  \eqref{eq:degcon5} we get
  ,
  in contradiction to . Therefore , as
  claimed (\figurename~\ref{fig:general4}).

  The vertex  has high degree in  but it is not adjacent to
  . Therefore, we can embed  as follows: put  at  and
  embed an arbitrary subtree  of  onto  recursively
  or, if it is a star, explicitly, using the locally isolated vertex at
   for the center (and  for the root in case of a dangling
  star). As  is isolated on  there is no conflict between 
  and . As , the remaining graph 
  consists of isolated vertices only, on which we can explicitly embed
  any remaining subtrees of  using the algorithm from
  Section~\ref{sec:emb_t1}.


  \case{1.2}  and . Then  is a
  locally isolated vertex in , whose only neighbor in  is
  at . Therefore, we can provisionally
  place  at  so that  is not in conflict with . By the assumption
  of Case~1  is not in degree-conflict with . Therefore, we obtain
  the claimed packing by first embedding  onto 
  recursively and then  onto .

  \case{1.3}  and . As
   and  is provisionally placed at
  , the interval 
  is not in edge-conflict with .
  Thus, Option~2 (\figurename~\ref{fig:ideal:2}) succeeds unless
   is in degree-conflict with . Hence suppose



  By Lemma~\ref{lem:degcon3} we have . As , by LSFR  has exactly one neighbor in 
  outside of : its parent 
  (\figurename~\ref{fig:general_inline2}). Let
  . We blue-star embed  starting from 
  with  so that  takes the
  vertices of  from right to left. Let us argue that the
  conditions for the blue-star embedding hold.

  \ref{gg:ec} holds due to  and
  . For the first inequality of
  \ref{gg:dc} we have to show , which is
  immediate from \eqref{eq:cconf}. For the second inequality of
  \ref{gg:dc} we have to show . This follows
  from
  . Regarding
  \ref{gg:int} note that in  we take the vertices of
   from right to left. If  reaches beyond
  , then  forms an interval
  (\figurename~\ref{fig:general_inline2_gg5}); otherwise,
   forms an interval
  (\figurename~\ref{fig:general_inline2_gg2}).  Conversely, if
   does not form an interval, then 
  reaches beyond . In particular, in that case 
  includes  and we may simply move  to the front
  of , establishing the second condition in
  \ref{gg:cs}. Regarding the remaining two conditions it suffices to
  note that  is not a star by assumption and that  is not
  a neighbor of  in  because  is the only neighbor of 
  outside of .

  Therefore, we can blue-star embed  as claimed. By construction and
  Proposition~\ref{p:greedygrab} that leaves us with an interval
  , where . This ``new'' interval is obtained from the
  interval  before the blue-star embedding by replacing some suffix of vertices by a corresponding number of
  locally isolated vertices. In particular,  is a
  subtree of  and
  .

  We complete the packing by recursively embedding  onto
  . This interval is not in edge-conflict with  by
  \ref{inv:placement},  and
  . We claim that it is not in
  degree-conflict with , either. Suppose towards a contradiction
  that  is in degree-conflict with . Then
   is a central-star and so by LSFR also
   is a central-star on at least this many
  vertices before the blue-star embedding. This contradicts the
  assumption of Case~1 that  is not in degree-conflict with
  . Therefore,  is not in degree-conflict with  and
  we can complete the packing as described. This completes the proof for
  Case~1.

  \begin{figure}[thbp]
    \centering \subfloat[]{\includegraphics{general_inline2_gg4}\label{fig:general_inline2_gg4}}\hfil \subfloat[]{\includegraphics{general_inline2_gg5}\label{fig:general_inline2_gg5}}\hfil \subfloat[]{\includegraphics{general_inline2_gg6}\label{fig:general_inline2_gg6}}\\
    \subfloat[]{\includegraphics{general_inline2_gg1}\label{fig:general_inline2_gg1}}\hfil \subfloat[]{\includegraphics{general_inline2_gg2}\label{fig:general_inline2_gg2}}\hfil \subfloat[]{\includegraphics{general_inline2_gg3}\label{fig:general_inline2_gg3}}
    \caption{Explicit embedding of  in Case~1.3. The edge
       need not be present in
      .\label{fig:general_inline2}}
  \end{figure}

  \case{2}  is in degree-conflict with . Then
   is a central-star 

and  by Lemma~\ref{lem:degcon3}. We distinguish
  two cases.

  \case{2.1} . Then . If
  necessary, flip  to put its center at . If 
  is a central-star on  vertices, then---if necessary---flip
   to put its root at . We use a blue-star embedding
  for  starting from  with . As
   consists of  vertices, we have
  . If
   is a central-star on  vertices, then
  use  instead (and note that
  ).

  In the notation of the blue-star embedding we have
  . We need to show that the conditions for this
  embedding hold. \ref{gg:ec} holds by \ref{inv:starconflict} (for
  embedding  onto ). For \ref{gg:dc} we have to show
  . The first inequality holds by
  \eqref{eq:degconf} and . The second inequality holds due
  to \ref{inv:starconflict} (for embedding  onto ), which
  implies . As  and
  , \ref{gg:dc} follows. \ref{gg:int} is
  obvious by the choice of  and \ref{gg:cs} is trivial for
   due to \ref{gg:int}. That leaves us with an interval , where .  We
  claim that  is not in conflict with .

\begin{figure}[htbp]
    \centering\hfil \subfloat[]{\includegraphics{greedygrab_m2}\label{fig:greedygrab_m2}}\hfil
    \subfloat[]{\includegraphics{greedygrab_m0}\label{fig:greedygrab_m0}}\hfil
    \subfloat[blue-star]{\includegraphics{greedygrab_m1}\label{fig:greedygrab_m1}}\hfil
    \caption{Handling a degree-conflict for  in
      Case~2.1.\label{fig:modgreedygrab}}
  \end{figure}


  To prove the claim we consider two cases. If , then initally
   was a central-star on  vertices rooted
  at . By the choice of  a leaf of this star is at 
  whose only neighbor in  is at . Therefore  is
  not in edge-conflict with .  As  is an
  isolated vertex, by Lemma~\ref{lem:degcon3} there is no
  degree-conflict between  and , either, which proves the
  claim.

  Otherwise,  and 
  is not a central-star on  vertices. Therefore by
  Lemma~\ref{lem:degcon3} there is no degree-conflict between 
  and . In order to show that there is no edge-conflict, either, it
  is enough to show that  is not
  adjacent to  in . If 
  this follows from 1SR. Otherwise
  , and  because
   is a star but  is not. Therefore the claim holds and
  we can complete the packing by recursively embedding  onto
  .

  \case{2.2} . By 1SR this means that 
  and  has at least one more neighbor in . Since by assumption  is not a star, we have . Since  is a central-star and , by LSFR
  for  the only neighbor of  in  outside of  is its
  parent .
We claim that such a configuration is impossible. To prove the claim,
  note that  has at least two children in  because
   and . By LSFR, the corresponding
  subtrees have size at least , and so
,
where the last inequality uses \eqref{eq:degconf}. Rewriting and using
  \eqref{eq:degr} yields

It follows that  and hence that
  . Since  is a smallest subtree of  in ,
  this means that  is binary in  and thus unary in . This,
  finally, contradicts the degree-conflict for  with 
  because  and hence .

  \case{3}  is in degree-conflict with  and 
  is not in degree-conflict with . Then 
  is a central-star  with  by
  Lemma~\ref{lem:degcon3} and



  \case{3.1} . Then we claim that we may
  assume  and .

  Let us prove this claim. If , then by 1SR it does not have
  any neighbor in . Flipping  establishes
  the claim. Otherwise, . Suppose that  has a neighbor
  . As  is the root of
  , it does not have a neighbor in
   and therefore . By LSFR and because
   is not a star, . In particular, since
  , LSFR for  implies . It follows that
  after flipping  the resulting subtree
   is not a central-star anymore and so there
  is no conflict for embedding  onto  anymore. Therefore
  we can proceed as above in Case~1 (the conflict situation for 
  did not change because  remains unchanged). Hence we may
  suppose that there is no such neighbor  of , which establishes
  the claim.

  We blue-star embed  starting from  with
  . In the terminology of the
  blue-star embedding we have . Let us argue that the
  conditions for the embedding hold. \ref{gg:ec} is trivial because no
  neighbor of  is embedded yet. For \ref{gg:dc} we have to show
  . The first inequality holds by
  \eqref{eq:degcz} and the second by
  , which implies
  . \ref{gg:int} is obvious by the
  choice of  and given , \ref{gg:cs} is trivial.
That leaves us with an interval , where .

  The plan is to recursively embed  onto . This works fine,
  unless  and  are in conflict. So suppose that they are in
  conflict. Then there is a central-star
  . Considering how  consumes the
  vertices in  from right to left,  appears as a part of some
  component of , that is, , for some .

  We claim that . To prove the claim, suppose to the
  contrary that . Then by 1SR  \emph{is} a
  component of . Thus, a degree-conflict contradicts the assumption
  of Case~3 that  is not in degree-conflict with , and
  an edge-conflict contradicts \ref{inv:starconflict} for embedding 
  onto  together with the fact that by 1SR  is not adjacent to
  any vertex outside of  in ---in particular not to , where 
  was placed. This proves the claim and, furthermore, that
   and  appears in .

  By \ref{inv:placement} for embedding  onto  and
   we know that  and  are not in
  edge-conflict and so they are in degree-conflict.  In particular,
  .

  Undo the blue-star embedding. We claim . To prove the claim, suppose to the contrary that
  . Then  because in  the
  vertex  lies below the edge . By LSFR the subtree
  of  rooted at  is at least as large as . Therefore,
  
where the last inequality uses \eqref{eq:degr}. This is in
  contradiction to , which implies
  . Therefore, the claim holds and
  .

  Flip  and perform the blue-star embedding again. Although
  1SR may be violated at , this is of no consequence for the
  blue-star embedding. As , we know that
   appears in  and so the offending vertex is not part of
   after the blue-star embedding. Furthermore, in this way we
  also get rid of the high-degree vertex of  that was at 
  initially so that the vertices in  are isolated.
  In particular,  is isolated in  and its only neighbor in
   is at . Therefore,  and  are not in
  conflict, unless  initially and  is in
  edge-conflict with .

  In other words, it remains to consider the case  is a dangling star whose root  (at
   before flipping) is in edge-conflict with 
  (\figurename~\ref{fig:31:1}).
Then  is an independent set in  that consists of leaves of
  the two stars  and  plus the isolated vertex at . Yet we
  cannot simply embed  using the algorithm from
  Section~\ref{sec:emb_t1} because  is and  may be in
  edge-conflict with . Given that  and  gets to
   only, at least two leaves of  remain in  and so, in
  particular,  is not in conflict with . We explicitly embed
   as follows (\figurename~\ref{fig:31:2}): place  at  and
  a child  of  in  at . Then collect  leaves
  from  and/or  and put them right in between  and .
  First---from left to right---the leaves of  whose blue edges leave
  them upwards to bend down and cross the spine immediately to the right
  of the vertices of the red subtree rooted at  (the leftmost subtree of ) and then
  reach  from below. Next come the leaves of  whose blue edges to
   are drawn as arcs in the upper halfplane. In order to make room
  for those leaves, the blue edge  is re-routed to leave 
  downwards to bend up and cross the spine immediately to the left of
   in order to reach  from above. Using the algorithm from
  Section~\ref{sec:emb_t1} we can now embed  onto these leaves
  and any remaining subtrees of  can be embedded explicitly
  on the vertices  (ignoring the change of numbering caused
  by the just discussed repositioning of leaves).
\begin{figure}[htbp]
    \centering \subfloat[]{\includegraphics{c31-0}\label{fig:31:1}}\hfil
    \subfloat[]{\includegraphics{c31-1}\label{fig:31:2}}\hfil
    \subfloat[]{\includegraphics{c3221}\label{fig:gen_322}}\hfil
    \caption{(a)--(b): Relocating some leaves of the stars
       and  in Case~3.1. One subtree of
       of  is embedded at  and the leaves to the right
      of ; all other subtrees of  are embedded to the right of
      . Both from  and from  we can route as many blue edges
      as desired to either of these ``pockets''. (c): Evading a
      degree-conflict for  in Case~3.2.1.\label{fig:31}}
  \end{figure}


  \case{3.2} . Then  because  is enclosed
  by  and therefore cannot be the root of . Moreover,
   by LSFR and since  is not a star. By LSFR  does
  not have any child in  and as  is not a
  star, . In particular,  is not adjacent to
  any vertex in . We provisionally place  at any
  vertex in , say, at . Then  is
  not in edge-conflict with . We claim that it is not in
  degree-conflict, either. As  is a star on  vertices,
  by Lemma~\ref{lem:degr} we have
   and the
  claim follows. We recursively embed  onto ,
  treating all local roots of  other than  as in conflict with
  . It remains to recursively embed  onto .

  Suppose towards a contradiction that  is in conflict with
  . Then there is a central-star .
  Due to  and 1SR we have  and .
  Together with LSFR for  it follows that
  . Due to the conflict setting for
  embedding ,  is the only vertex in  that may be
  in edge-conflict with . As  is a central-star and
  , we cannot have  because then  would be a star.
  It follows that  and so  is not in edge-conflict
  with . Therefore  and  are in degree-conflict.
  Then  by Lemma~\ref{lem:degcon3} and

Depending on  we consider two final subcases.

  \case{3.2.1} 
  (\figurename~\ref{fig:gen_322}). Then the edge 
  encloses  so that, in particular,
  . We provisionally place  at
   and claim that 
  and  are not in conflict.

  To prove the claim, consider 
  and suppose it is a central-star. (If it is not, then we are done.) If
  , then by 1SR and  we
  have , in contradiction to LSFR for .
  Therefore . In order for  to be the
  local root for  in the presence of , it
  follows that  and so by 1SR . Therefore by
  Lemma~\ref{lem:degcon3} there is no degree-conflict between
   and . As  prevents any
  connection in  from  to  and to vertices outside of
  , there is no edge-conflict between  and ,
  either. This proves the claim. Recursively embed  onto
  . Recall that . There is no conflict for
  embedding  onto  since  and
   is not a central-star of size at least 2 by
  LSFR at . Finish the packing by recursively embedding  onto
  .

  \case{3.2.2} . Then by 1SR  is the only
  neighbor of  outside of  in . We provisionally place  at
   and employ a blue-star embedding for , starting from
   with , that is,  takes
  vertices from left to right, skipping over . Let us argue
  that the conditions for the blue-star embedding hold.

  In the terminology of the blue-star embedding we have  and
  . \ref{gg:ec} holds because
  . For the first inequality of \ref{gg:dc} we have to
  show , which is immediate from
  \eqref{eq:degcon4}. For the second inequality of \ref{gg:dc} we have
  to show . This follows from
  . Regarding
  \ref{gg:int} note that in  we take the vertices of
   from left to right. As there are not enough vertices
  in  to embed the neighbors of  (which causes the
  degree-conflict),  reaches beyond  and so
   forms an interval. In particular,
   includes  and we may simply move  to
  the front of , establishing the second condition in
  \ref{gg:cs}. Regarding the remaining two conditions in
  \ref{gg:cs} note that  is not a star by assumption and that  is not
  a neighbor of  in  because  is the only neighbor of 
  outside of .

  Therefore, we can blue-star embed  as claimed, which
leaves us with an interval , where . As
   and  is not the local root of  ( is),
  there is no edge-conflict between  and . As there is no
  degree-conflict between  and  and the number of
  neighbors of  in  can only decrease compared to
   (if they appear in ), there is no
  degree-conflict between  and , either. Therefore, we can
  complete the packing by embedding  onto  recursively.
\end{proof}
















\section{Embedding the red tree: a unary root}\label{subsec:rec_unary}

In this section we handle all cases where the root  of  is unary.

\begin{proposition}\label{prop:rec_unary_star}
  If  and  is a star, then there is an ordered plane
  packing of  and  onto .
\end{proposition}
\begin{proof}
  Since  and  is not a star by assumption,  must be a
  dangling star. Thus, we know exactly what  looks like: it is rooted
  at , which has a single child , which has a single child ,
  which finally has zero or more leaf children.

  \case{1} . We consider three cases.

  \begin{figure}[b]
    \centering
    \subfloat[Case~1.1]{\includegraphics{unary_star_ij_isolated}\label{fig:unary_star_ij_isolated}}\hfil \subfloat[Case~1.2]{\includegraphics{unary_star_i_not_isolated}\label{fig:unary_star_i_not_isolated}}\hfil \subfloat[Case~1.3]{\includegraphics{unary_star_j_not_isolated_1}\label{fig:unary_star_j_not_isolated_1}}\\
    \subfloat[Case~1.3]{\includegraphics{unary_star_j_not_isolated_2}\label{fig:unary_star_j_not_isolated_2}}\hfil \subfloat[Case~2.1]{\includegraphics{unary_star_ij_used_1}\label{fig:unary_star_ij_used_1}}\hfil \subfloat[Case~2.2]{\includegraphics{unary_star_ij_used_2}\label{fig:unary_star_ij_used_2}}
    \caption{The case analysis in the proof of
      Proposition~\ref{prop:rec_unary_star}.}
    \label{fig:unary_star}
  \end{figure}

  \case{1.1}  and  are both isolated in . Embed  to , 
  to ,  to , and the children of  onto . See
  \figurename~\ref{fig:unary_star_ij_isolated}. Note that  is not in
  edge-conflict due to the placement invariant. Every red edge is
  incident to  or  and hence does not occur in  by assumption.

  \case{1.2}  is not isolated in . If  is a
  central-star, flip it if necessary to put its root (which is not in
  edge-conflict by the peace invariant) at . Otherwise, use
  the leaf-isolation shuffle to put a leaf at  and its parent at
  . Since  is not a central-star, by
  Proposition~\ref{prop:leafshuffle}, this will place the root of
   at some position . In both cases, embed  onto
  ,  onto ,  onto , and the children of  onto
  . See \figurename~\ref{fig:unary_star_i_not_isolated}. The
  edge  is not used by  since  is not used by
  assumption (and the leaf-isolation shuffle cannot change that). The
  red edges incident to  are not used since the only neighbor of
   in  is .

  \case{1.3}  is isolated and  is not isolated in . Flip
   if its root is currently at . Note that  is not
  in degree-conflict for embedding : this would imply that
   is a star since . If  is not in
  conflict, then the invariants hold for  and we can apply
  Case~1.2 by embedding  on  instead of . Otherwise,
   is a central-star on at least two vertices.

  If , then embed  onto ,  onto ,  onto
  , and the children of  onto . See
  \figurename~\ref{fig:unary_star_j_not_isolated_1}. If
  , then embed  onto ,  onto ,  onto
  , and the children of  onto . See
  \figurename~\ref{fig:unary_star_j_not_isolated_2}. This works because
  the root of  is not at  (so  is not in
  edge-conflict), the size of the star  is at least three
  (so  is not used), and  is isolated in  (so the
  red edges incident to  are not used by ).

  \case{2} . Let  be the root of .
  We claim that (1) some vertex of  has distance at least three
  to  or (2) . To prove the claim, suppose
  that all vertices in  have distance at most two to  and
  that  is unary. Then the child of  has distance one to all other
  vertices of : hence  is a star centered at the child
  of , a contradiction. We perform a case analysis on whether (1) or
  (2) holds.

  \case{2.1} Some vertex  of  has distance at least three to
  . Let  be the child of  that contains  in its subtree
  . Let  be the size of . We re-embed  as follows.
   is not a central-star by choice of . Hence, by
  Proposition~\ref{prop:leafshuffle}, we can use the leaf-isolation
  shuffle to embed  on , placing a leaf at , its
  parent at , and the root  at some position in .
  Complete this embedding of  to any one-page book embedding of
  . Note that this embedding does not use the edge
  . Embed  at ,  at ,  at , and the
  children of  at . See
  \figurename~\ref{fig:unary_star_ij_used_1}. This works because  is
  not at  (so  is not in edge-conflict),  does not use the
  edge  (so  is not used by ), and  is
  isolated in  (so the red edges incident to  are not used
  by ).

  \case{2.2} . Since  is not a star,
  some vertex  has distance at least two to  in . Let 
  be the child of  that contains  in its subtree . Let  be
  the size of . We re-embed  as follows. Use the
  leaf-isolation shuffle to embed  together with  on ,
  placing a leaf at , its parent at , and  at . Complete
  this embedding to any one-page book embedding of . Note that
  this embedding does not use the edge . Finish by using the
  same embedding for  as in Case~2.1. See
  \figurename~\ref{fig:unary_star_ij_used_2}.
\end{proof}

\begin{proposition}\label{prop:rec_unary_regular_ij_used}
  If ,  is not a star, and ,
  then there is an ordered plane packing of  and  onto .
\end{proposition}
\begin{proof}
  Flip  if necessary to put its root at . The general plan is
  to embed  onto  and  recursively onto . This works
  unless (1)  is a star, (2) , or (3) there
  is a conflict for embedding  onto . Below, we find an ordered
  plane packing under a weaker condition than (1) to allow for reuse in
  cases (2) and (3).
  In case (2), by LSFR,  is a central-star on at
  least two vertices. In case (3),  is a
  central-star. We deal with these cases below.

  \case{1}  is a star or  is a star. If 
  is a star, then we flip  to reduce to the case that
   is a star. Thus, in the following, assume that 
  is a star and that the root of  may be either at  or at
  . We know exactly what  looks like: since  is not a
  star, the star  must be centered at  and rooted at .
  Flip the blue embedding at : this puts the star-center at
  . Note that . Embed  onto . The
  interval  is in edge-conflict with  if the root of
   is now at . Hence, we embed  explicitly. Embed 
  onto . Since  is not a star, it must have a subtree of size
  . Embed this subtree explicitly at . Embed the
  other subtrees of  explicitly on the remainder. See
  \figurename~\ref{fig:unary_regular_ij_used_1}.

  \begin{figure}[t]
    \centering
    \subfloat[Case~1]{\includegraphics{unary_regular_ij_used_1}\label{fig:unary_regular_ij_used_1}}\hfil
    \subfloat[Case~2]{\includegraphics{unary_regular_ij_used_2}\label{fig:unary_regular_ij_used_2}}\hfil
    \subfloat[Case~3]{\includegraphics{unary_regular_ij_used_3}\label{fig:unary_regular_ij_used_3}}\hfil
    \caption{The case analysis in the proof of
      Proposition~\ref{prop:rec_unary_regular_ij_used}.}
    \label{fig:unary_regular}
  \end{figure}

  \case{2}  is a central-star on at least two
  vertices. Let  be such that . By Case~1
  we may assume that . By LSFR at  and by choice of ,
   is a tree. Flip . Since , the root of
   is no longer at . Embed  onto  and  recursively onto
  . See \figurename~\ref{fig:unary_regular_ij_used_2}. Since
   after flipping and  is isolated in
  , this works unless  is in conflict for . Then
   is a central-star that is rooted at the root of
  . But this contradicts LSFR at  before flipping: a
  contradiction. Hence, there is no conflict for .

  \case{3}  is a central-star. Let  be such
  that . Since ,
   is rooted and centered at  and the parent of  is at
  . Hence  is a dangling star. By Case~1 we may assume that
  . Flip . Embed  at  and  recursively at
  . See \figurename~\ref{fig:unary_regular_ij_used_3}. Since
   after flipping and  is isolated in
  , this works unless  is in conflict for . Then
   is a central-star that is rooted at the root of
  . But this contradicts LSFR at  before flipping: a
  contradiction. Hence, there is no conflict for .
\end{proof}

\begin{proposition}\label{prop:rec_unary_regular_ij_not_used}
  If ,  is not a star, and
  , then there is an ordered plane packing
  of  and  onto .
\end{proposition}
\begin{proof}
  The general plan is to embed  onto  and  recursively onto
  . Since  and  is not a star,
  this works unless (1)  is a star or (2) there is a
  conflict for embedding  onto . In case (2), the star
   is either in edge-conflict or in
  degree-conflict for embedding . If it is in edge-conflict, then
  there must be an edge from the root of  to . By 1SR, the root
  of  must be at . But that means that
  , a contradiction. Thus, in case (2), there
  is a degree-conflict for embedding  onto . We deal with
  these cases below.

  \case{1}  is a star. Since , vertex 
  is isolated in . Flip  if necessary to put the
  center of  at . If the root of  is at ,
  then embed  onto  and  recursively onto the independent set
  . Since  is isolated in the blue embedding,  is
  not in conflict for . If the root of  is at , then
  flip the blue embedding at . This places the root at 
  and a leaf of the star at . After flipping, the interval 
  still satisfies the invariants. Embed  onto  (which is not in
  edge-conflict) and  recursively onto . See
  \figurename~\ref{fig:unary_regular_ij_not_used_star}. Since  is
  isolated in the blue embedding,  is not in conflict for .

  \begin{figure}[b]
    \centering
    \subfloat[Case~1]{\includegraphics{unary_regular_ij_not_used_star}\label{fig:unary_regular_ij_not_used_star}}\hfil
    \subfloat[Case~2.1]{\includegraphics{unary_regular_ij_not_used_no_conflict}\label{fig:unary_regular_ij_not_used_no_conflict}}\hfil
    \subfloat[Case~2.2:]{\includegraphics{unary_regular_ij_not_used_conflict_k2}\label{fig:unary_regular_ij_not_used_conflict_k2}}\hfil
    \subfloat[Case~2.2:
    ]{\includegraphics{unary_regular_ij_not_used_conflict_no_k2}\label{fig:unary_regular_ij_not_used_conflict_no_k2}}\hfil
    \caption{The case analysis in the proof of
      Proposition~\ref{prop:rec_unary_regular_ij_not_used}.}
    \label{fig:unary_regular_ij_not_uses}
  \end{figure}

  \case{2} There is a degree-conflict for embedding  onto .
  Let  be such that . Due to the
  degree-conflict,  is a central-star on at least three
  vertices. Since , the root of  is
  not adjacent to any vertex in . By 1SR, if it were adjacent
  to , then the root of  must be at : this however,
  violates the assumption that . Hence,
  . Since  is a tree and ,
   is not a star. We distinguish two cases.

  \case{2.1} There is no conflict for embedding  onto .
  Flip  if necessary to put its root at . This preserves
  all invariants on . Embed  onto  (which is not the root
  of ) and  recursively onto . See
  \figurename~\ref{fig:unary_regular_ij_not_used_no_conflict}. This
  works by the assumption that there is no conflict for embedding
   onto  before flipping  and since
   due to .

  \case{2.2} There is a conflict for embedding  onto .  By
  the 1SR and the fact that , there is a
  degree-conflict for embedding  onto . Let  be such that
  . By the same argumentation that proved
   we have . Thus, we can divide
   into three disjoint parts:  (a central-star),
   (about which we know nothing), and  (a
  central-star). For notational convenience, let ,
  , and  be the corresponding sizes. Let
   and let  be the children of , ordered
  by increasing size of their subtrees ( is the largest).
  Since  is not a star . Let  be the
  number of leaf children of . Then  if and only if
  .

  Flip  if necessary to put the root (and center) at  and
  flip  if necessary to put the root (and center) at . We
  first explain how to embed  and then prove that it always works.
  Refer to \figurename~\ref{fig:unary_regular_ij_not_used_conflict_k2}
  for the case  and
  \figurename~\ref{fig:unary_regular_ij_not_used_conflict_no_k2} for the
  case . Embed  onto  and  onto . This works so far:
  by the peace invariant the root of  is not
  in conflict and . Next, embed 
  recursively onto . Since
   and  is isolated in
  , this works provided  fits inside
  , i.e. provided .
  Next, embed a leaf child of  on each vertex in  (this
  interval may be empty). This embeds the children 
  and works provided that . This leaves two disjoint
  intervals to embed the remaining subtrees
   of :
   and . Thus, it
  remains to prove that (i) , and (ii) , and that (iii) we can distribute the remaining subtrees over
   and .

  We begin by showing that  must be large. Since there is a
  degree-conflict for embedding  onto  we have
  , and since there is a degree-conflict for embedding 
  onto  we have :

Recall that . Adding~\eqref{eq:unary_2dc_left}
  and~\eqref{eq:unary_2dc_right} yields  and so

\subsubparagraph{Proof of (i)} We must show that .
  Using~\eqref{eq:unary_2dc_left} we get
  . Since the
  total size of the subtrees at the children of  is  we have
  ,
  which completes the proof of (i). 

  \subsubparagraph{Proof of (ii)} We must show that .
  Since  for all , ,
  we have  and so


  \subsubparagraph{Proof of (iii)} It remains to prove that we can
  distribute  over the disjoint
  intervals  and . We
  use the following observation on partitioning natural numbers.
\begin{observation}
    \label{obs:unary_2dc_partition}
    Let  and  be positive integers with  and let  be positive integers
    with . Then for all  there exists
    a set  such that .
  \end{observation}
  \begin{proof}
    We prove the statement by induction on . The statement is true
    for : in this case we must have  and , and so
     and  work. Suppose that the statement
    holds for all positive integers smaller than . It suffices to
    prove the statement for  since we can
    choose  for . If
     then  and we choose . Otherwise,
    by the assumption on  we have  and hence .
By the assumption on  and since  we have . Hence, by the
    induction hypothesis, there exists a set  with . Choose
     to complete the proof.
  \end{proof}
The total size of the remaining subtrees is
   since .
  Then

where the last step uses that  for  and  for .
Hence,  and  satisfy the precondition of
  Observation~\ref{obs:unary_2dc_partition}. We apply the observation
  with . This gives us a set  such that  and .

  Since  and  have no internal edges and no edges to the
  position of  at , we can embed the subtrees  with
   explicitly from left to right on  and the remaining
  subtrees explicitly from left to right on . This completes the
  proof.
\end{proof}

Proposition~\ref{prop:rec_unary_star},
Proposition~\ref{prop:rec_unary_regular_ij_used}, and
Proposition~\ref{prop:rec_unary_regular_ij_not_used} together prove the
following.

\begin{lemma}
  \label{lem:rec_unary}
  If , then there is an ordered plane packing of  and
   onto .
\end{lemma}

\section{Embedding the red tree: a singleton subtree}
\label{subsec:rec_singleton}
Here we completely handle the case .

\begin{lemma}\label{lem:rec_singleton}
  If , then  and  admit an ordered plane packing onto
  .
\end{lemma}
\begin{proof}
  We distinguish two cases.

  \case{1}  is not a star. We first describe an embedding that
  works whenever  is a star. Flip  if necessary to
  put its center at . In addition to the star at , the
  blue embedding may use the edge . Note that it cannot use
  , as this would imply that  is a star. Thus,  is
  isolated in . Embed  onto  and  onto . Let 
  be a largest subtree of  in . Since  is not a star,
  . Embed  recursively onto . Since  is
  locally isolated in  and  is not adjacent to 
  (which is where we embedded ), this always works. Embed the
  remaining subtrees of  in  explicitly on .

  Assume now that  is not a star.
  If , then we embed  at , and recursively
  embed  onto .
 has no edge-conflict with  by the peace invariant.
  It also has no degree-conflict with :
  otherwise  would already have had a degree-conflict with .

  So assume that . Flip  if necessary to put
  its root at . If  is a star now, then use the embedding
  described in the first paragraph to find an ordered plane packing.
  Otherwise,  is not in edge-conflict with any vertex in .
  The general plan is to embed  at  and  recursively onto
  . Since  is not a star and  is rooted at , the edge
   is not used. Hence, this works unless there is a
  conflict for embedding  onto . This means in
  particular that  is a central-star
  . See
  \figurename~\ref{fig:singleton_rm_no_star_default}. By assumption,
  . Due to the presence of the edge  and
  since , the root (and hence also the center) of  must
  be at .

  \begin{figure}[b]
    \centering\hfil \subfloat[Case~1]{\includegraphics{singleton_rm_no_star_default}\label{fig:singleton_rm_no_star_default}}\hfil \subfloat[Case~1.1]{\includegraphics{singleton_rm_no_star_x_large_1}\label{fig:singleton_rm_no_star_x_large_1}}\hfil \subfloat[Case~1.1]{\includegraphics{singleton_rm_no_star_x_large_2}\label{fig:singleton_rm_no_star_x_large_2}}\hfil \subfloat[Case~1.2]{\includegraphics{singleton_rm_no_star_x_small}\label{fig:singleton_rm_no_star_x_small}}\hfil \caption{Case~1 in the proof of Lemma~\ref{lem:rec_singleton}.}
  \end{figure}

  \case{1.1} . Flip . Note that afterwards
   and  satisfies 1SR and LSFR.
  Embed  onto  and  recursively onto . See
  \figurename~\ref{fig:singleton_rm_no_star_x_large_1}. Since , the interval  contains at least one leaf of  and so
   is not a star. Hence, this works unless there is a
  conflict for embedding  onto . In that case, note
  that  is now formed by the root of  and its
  subtrees other than . Since  is a
  central-star, it follows that the subtrees of the root  of 
  other than  are all leaves. Flip  again to restore the
  original embedding. Embed  onto  and  recursively onto
  . See \figurename~\ref{fig:singleton_rm_no_star_x_large_2}.
  Since ,  is a tree that is not a star and
  . Hence, the peace invariant holds for
  .

  \case{1.2} . Flip . Embed  onto  and  onto
  . Embed the remaining subtrees of  in  explicitly onto the
  independent set , putting the largest one (which has size
  at least two) next to . See
  \figurename~\ref{fig:singleton_rm_no_star_x_small}.

  \case{2}  is a star. Then  and the child  of 
  in  is the root and center of a star .

  \case{2.1} . Let  be the root of . If
  , then flip  if necessary to put its root at
  . Then  is isolated in  and 
  since  is not a star and by LSFR. Embed  onto ,  onto
  ,  onto , and the children of  onto . See
  \figurename~\ref{fig:singleton_rm_star_ij_1}.

  \begin{figure}[t]
    \centering\hfil \subfloat[Case~2.1]{\includegraphics{singleton_rm_star_ij_1}\label{fig:singleton_rm_star_ij_1}}\hfil \subfloat[Case~2.1]{\includegraphics{singleton_rm_star_ij_2}\label{fig:singleton_rm_star_ij_2}}\hfil \subfloat[Case~2.2]{\includegraphics{singleton_rm_star_no_ij_1}\label{fig:singleton_rm_star_no_ij_1}}\hfil \subfloat[Case~2.2]{\includegraphics{singleton_rm_star_no_ij_2}\label{fig:singleton_rm_star_no_ij_2}}\hfil \subfloat[Case~2.2]{\includegraphics{singleton_rm_star_no_ij_3}\label{fig:singleton_rm_star_no_ij_3}}\hfil \caption{Case~2 in the proof of Lemma~\ref{lem:rec_singleton}.}
  \end{figure}

  If , then flip  if necessary to put its root
  at . Let  be such that , which is
  a smallest subtree of . Since  is not a star, 
  is not a central-star. Flip . This
  puts the root  at . Use a leaf-isolation-shuffle on 
  to embed a leaf at , its parent of , and the root at .
  This works by Proposition~\ref{prop:leafshuffle}. Embed  onto ,
   onto ,  onto  and the children of  onto .
  See \figurename~\ref{fig:singleton_rm_star_ij_2}.

  \case{2.2} . Then . If
  , then perform a leaf-isolation-shuffle to put a
  leaf at  and its parent at . Since ,
  this does not touch the blue vertex at . Embed  onto , 
  onto ,  onto , and the children of  onto .
  See \figurename~\ref{fig:singleton_rm_star_no_ij_1}.

  If  and  is not a central-star, then
  flip  if necessary to put its root at . Since it is
  not a central-star, . Embed  onto , 
  onto ,  onto , and the children of  onto . See
  \figurename~\ref{fig:singleton_rm_star_no_ij_2}.

  Finally, if  and  is a central-star, then
  let  such that . We have  by the
  peace invariant. Flip  if necessary to put its root
  at . By the peace invariant,  is not in edge-conflict
  with . Embed  onto ,  onto ,  onto , and the
  children of  onto  and . See
  \figurename~\ref{fig:singleton_rm_star_no_ij_3}.
\end{proof}

\section{Embedding the red tree: a large blue star}
\label{subsec:rec_large_blue_star}
In this and the following section we handle the case that 
is a star. The graphs , , and  may or not be
stars. The case that we actually handle is more general, as specified in
the following
\begin{lemma}\label{lem:rec_large_blue_star}
  If  is a star, for , then  and  admit
  an ordered plane packing onto .
\end{lemma}
\begin{proof}
  By Lemma~\ref{lem:rec_unary} and Lemma~\ref{lem:rec_singleton}, we may
  assume  and . The following observation does
  not depend on the context of this proof.
\begin{observation}\label{obs:rmsize}
    .
  \end{observation}
  \begin{proof}
    If , then by the minimality of  we have . It
    follows that  and so  is a star, contrary to our assumption.
  \end{proof}
By Observation~\ref{obs:rmsize}, . Select  maximally so
  that  is a star, and let . Note that
  . We distinguish two cases.

  \case{1}  is a central-star. Then by LSFR we have
  . If necessary, flip  to put its root and center
  at . We will use a blue-star embedding to embed  from
   with . Let us first check the
  conditions for the blue-star embedding. \ref{gg:ec} holds by
  \ref{inv:starconflict} for embedding  onto . For \ref{gg:dc}
  we must show  and . We wish to
  argue that at least one leaf of  remains after the blue-star
  embedding, and thus we show . This inequality holds
  since  and . For the second inequality, by
  \ref{inv:starconflict}, we have  and so
  . \ref{gg:int} and \ref{gg:cs}
  hold since  forms an interval. Hence, by
  Proposition~\ref{p:greedygrab}, the blue-star embedding succeeds and
  leaves an interval  such that  is not in
  edge-conflict for embedding  and a non-empty prefix of 
  consists of isolated vertices that are in edge-conflict for embedding
   (these are leaves of ). Recursively embed  onto .
  This works unless  is a star or  is in conflict
  (which must be a degree-conflict) for embedding .


  \case{1.1}  is a star. If  is a dangling star then embed 
  onto , the child  of  onto  (which is locally isolated),
  and the children of  onto . Otherwise,  is a
  central-star. If there is a locally isolated vertex in  that
  is not in edge-conflict, then use this vertex to embed  and embed
  the children of  on the remainder. Otherwise, undo the blue-star
  embedding. Consider the blue vertex at . It does not get consumed
  by the blue-star embedding. Since it was not isolated after the
  blue-star embedding, it is not isolated now. By choice of , we
  have . Perform a
  leaf-isolation-shuffle on  to place a leaf  at 
  and its parent at . Perform the original blue-star embedding, but
  now with  if  and  if
  . The conditions of the blue-star embedding still hold. The
  resulting interval  contains the now isolated vertex 
  and we embed  by placing  onto  and embedding the children
  of  on the remainder.

  \case{1.2}  is a central-star that raises a
  degree-conflict. Note that  is composed of some locally
  isolated vertices plus some a suffix of the interval  before
  the blue-star embedding. Undo the blue-star embedding. Now
   is a central-star for some minimal . We claim that we may
  assume that  is rooted at . Indeed, if  is rooted
  at , then by 1SR we have  and we can flip
   to establish the claim. Perform the original blue-star
  embedding for , but now with  if  and  if . In the remaining interval , the
  vertex  is a leaf of what was the central-star  before the
  blue-star embedding. Hence,  is locally isolated and not in
  edge-conflict with . Recursively embed  onto  to
  complete the embedding.

  \case{2}  is a dangling star. In this case \ref{inv:starconflict}
  does not tell us anything about the size of  (because it applies
  to central-stars only). If the root of  is at , then its
  center is at  and by 1SR  is the only neighbor of  in .
  Hence by flipping  we may suppose that the root of  is at
  . Note that  may have more neighbors, in addition to the center
  of  at . Also note that  may be in conflict with , in
  case we flipped  (the original vertex at  cannot be in
  conflict by \ref{inv:placement}). We distinguish two cases.

  \case{2.1} . In this case we know almost completely what 
  looks like:  is a star rooted at  and centered at 
  and the edge  may or may not be used. We embed  explicitly
  as follows. Since , there is a subtree  of
   different from . Embed  onto  and embed 
  explicitly onto the independent set at . Since
  , we know that , and hence  is not
  embedded at the center of the star . If  is not a star,
  embed it recursively onto . This works because  is
  locally isolated in  and  is not adjacent to 
  (which is where we embedded ). If  is a central-star, embed 
  onto  and its children onto . If  is a dangling
  star, embed  onto , the child  of  onto , and
  the children of  onto . Embed the remaining
  subtrees (if any) of  on the remaining interval ,
  which forms a locally independent set, none of whose vertices are
  adjacent to .

  \case{2.2}  and  is a central-star
   on  (in particular, this holds if
   has a degree-conflict for embedding on ). We
  distinguish two subcases.

  \case{2.2.1} . Then the root and center  of
   must be at  and cannot be the root of  because then
  LSFR would imply that  is a star. Therefore  is the root of 
  (\figurename~\ref{fig:large_blue_star_221_ij_1}) and it is not in
  conflict with  due to \ref{inv:placement}. We modify the embedding
  of  as follows: Move  to  and all leaves of  (as
  , there is at least one) in sequence
  immediately to the right of , at position  and onward,
  shifting all vertices between there and  to the right accordingly.
  Draw the edge  below the spine to avoid crossings, and all
  other edges incident to  above the spine
  (\figurename~\ref{fig:large_blue_star_221_ij_2}).
\begin{figure}[htbp]
    \centering\hfil \subfloat[]{\includegraphics{lbs_221_ij_0}\label{fig:large_blue_star_221_ij_1}}\hfil
    \subfloat[]{\includegraphics{lbs_221_ij}\label{fig:large_blue_star_221_ij_2}}\hfil
    \caption{
      (Case~2.2.1).\label{fig:large_blue_star_221_ij}}
  \end{figure}


  We place  at  and explicitly embed  onto , which
  in  consists of a single edge  with isolated vertices
  (at least one because ) in between. Recall that  is
  not a central-star and so we can embed it as described. It remains to
  embed  onto . As  is a leaf of ,
  which is isolated on , there is no conflict for this
  embedding and  is not a star. Therefore, if  is not a
  star, then we can complete the packing recursively by embedding 
  onto .

  It remains to consider the case that  is a star. If  is a
  central-star, then we can put this center at the locally isolated
  vertex . Otherwise,  is a dangling star with .
  As , we have at least two locally
  isolated vertices (leaves of ) at  and . We
  put the root of  at  and the center at  to
  complete the packing.

  \case{2.2.2} . Let us consider the
  central-star . We claim that
  . Indeed, if the root  of  is on
  the left, then by 1SR . Otherwise,  is at
  . By definition of ,  has no neighbors in
  . Since  is a star and
  , the only remaining possible neighbor of  would
  be , but this is excluded by the assumption. We conclude that
  . If necessary, flip  to put its root (and
  center) at .

  \case{2.2.2.1} 
  (\figurename~\ref{fig:large_blue_star_221_1}). Then we change the
  embedding of  by moving one leaf  of  all the way to
  the left at . As a leaf of  it is not in conflict with ,
  and so we can map  to  and embed  explicitly onto the
  locally independent set . If  is not a star, then we
  recursively embed  onto 
  (\figurename~\ref{fig:large_blue_star_221_2}). Note that  is
  the center of , which is isolated in  and not
  adjacent to the leaf of  at . Therefore,  is
  not a star and there is no degree-conflict and no edge-conflict for
  embedding  onto .
\begin{figure}[htbp]
    \centering\hfil \subfloat[]{\includegraphics{lbs_221_0}\label{fig:large_blue_star_221_1}}\hfil
    \subfloat[]{\includegraphics{lbs_221}\label{fig:large_blue_star_221_2}}\hfil
    \caption{ (Case~2.2.2.1).\label{fig:large_blue_star_221}}
  \end{figure}


  It remains to consider the case that  is a star. If  is a
  central-star, then the center can be embedded on the isolated vertex
  at . Otherwise,  is a dangling star with . Then
  at least one more leaf of  remains at , where we can
  embed the root of . The center of  is again mapped to the
  isolated vertex  and the edge  is drawn as a
  biarc, crossing the spine between  and .

  \case{2.2.2.2} 
  (\figurename~\ref{fig:large_blue_star_223_1}). Then we change the
  embedding of  by simultaneously moving the root of  to  and
  and moving all other vertices of  to the left by one
  (\figurename~\ref{fig:large_blue_star_223_2}). Embed  at . We
  will use a blue-star embedding to embed  on  from
   where  consists of the rightmost 
  non-neighbors of  in  from right to left. Note that
   and hence  in the terminology
  of the blue-star embedding. Let us check the conditions for the
  blue-star embedding. \ref{gg:ec} holds because  and so
   is a leaf of  that is adjacent to  only.
  \ref{gg:int} and \ref{gg:cs} hold because
   forms an interval. For \ref{gg:dc}
  we must show  and
  . The first inequality follows from the
  assumption of Case~2.2. For the second inequality, we have
   and , and so
  , since . Hence, the
  conditions for the blue-star embedding are satisfied.

  Since , we have , and hence the blue-star embeding embeds
  a child of  onto the root of , which was embedded at ,
  and the center of , which was embedded at . Therefore,
  the remaining vertices not used for the embedding of  form an
  independent set in  and we can explicitly embed  on them.
\begin{figure}[htbp]
    \centering\hfil \subfloat[]{\includegraphics{lbs_223_0}\label{fig:large_blue_star_223_1}}\hfil
    \subfloat[]{\includegraphics{lbs_223}\label{fig:large_blue_star_223_2}}\hfil
    \caption{
      (Case~2.2.2.2).\label{fig:large_blue_star_223}}
  \end{figure}


  \case{2.3}  () and
   is \textbf{not} a central-star  on
   vertices. We first prove a claim and then
  distinguish two subcases.

  \emph{Claim:} We may suppose that  is a star or . To prove
  the claim, suppose that . Then  is a leaf of
   and we can explicitly embed  onto the independent set
  . As  is the only neighbor of  in , we
  have  and so there is no edge-conflict for
  embedding  onto . By assumption there is no
  degree-conflict for this embedding, either, and  is not
  a star because the root of  is part of it but . The
  only remaining obstruction for the recursive embedding of  onto
   is  being a star. This proves the claim.

  \case{2.3.1} . Then we rearrange the
  embedding of  as follows: move the center  of  to 
  and the vertex  at  (the leftmost vertex not in ) to
  . In order to avoid crossings with the edge(s) incident to
  , draw all edges between  and its neighbors in 
  below the spine, whereas edges to neighbors in  remain
  above the spine (\figurename~\ref{fig:large_blue_star}). After this
  transformation  is an independent set, on which we can
  embed  explicitly. However, we have to take care because of the
  blue edges drawn below the spine and the (possibly) conflicting root
  . Without loss of generality suppose that the root of  at 
  is in conflict with .
\begin{figure}[htbp]
    \centering\hfil \subfloat[]{\includegraphics{lbs_3}\label{fig:large_blue_star_1}}\hfil
    \subfloat[]{\includegraphics{lbs_3a}\label{fig:large_blue_star_2}}\hfil
    \caption{ and 
      (Case~2.3).\label{fig:large_blue_star}}
  \end{figure}


  Recall that  is not a central-star and so there is at least one
  non-leaf child  of  in . Denote  and map both 
  and the conflicting root of  to  (by exchanging the
  order of leaves of  in ). As , we have
   and so the local order for the roots of
  subtrees from  is maintained. On the other hand, we have
   and , which imply
  . Therefore (the leaf now
  at)  is not in conflict with .

  As  initially, after the transformation
  we have  and so
   is an independent set in . Therefore, we can
  embed  onto  explicitly, drawing all edges above the
  spine, and complete the embedding of  by embedding  onto  explicitly, again drawing all edges above the
  spine. After these changes to the embedding of , the only neighbor
  of  in  is . Together with  it follows
  that there is no edge-conflict for recursively embedding  onto
  . We also know that  is not a star because
  it contains part of  (at least the center at ) and at
  least one more vertex not connected to that part of : the vertex
  at . (As , there were at least two such vertices
  initially, but one, the vertex , has been moved and used for
  embedding .) Two possible obstructions for the recursive
  embedding of  onto  remain: a degree-conflict or 
  is a star. We conclude by considering both cases.

  \case{2.3.1.1}  is a star. Undo the rearrangement of . We will
  redo the rearrangement, but first do some other modifications.

  Suppose first that  or
  . In the former case, use a leaf-isolation
  shuffle on  to place a leaf at  and its parent at
  . We can apply the shuffle because  and therefore
  . After the modification (as described in the first
  paragraph of Case~2.3.1) we proceed as follows. If  is a
  central-star, then  can be placed at , which is adjacent to
   only and therefore locally isolated in .
  Otherwise,  is a dangling star. Then either there is a (non-root)
  leaf of  in  or the center  of  is isolated
  in . In either case, we put the center of  on .
  In the former case, we put the root of  on  (the leftmost
  leaf of  in , and draw the edge  as a
  biarc that crosses the spine between  and . In the
  latter case, we put the root of  on . Either way, we can
  complete the star  and the embedding of  works just as before.

  Otherwise,  and
  . Since  by the
  assumption of Case~2.3.1, we know that
   and hence also
  . It follows that  and
  hence . Perform a leaf-isolation-shuffle on 
  to place a leaf at  and its parent at . Rearrange the
  embedding of  as follows: move the center  of  to
  , the vertex  at  to , and the vertex 
  at  to . Draw the blue edges as explained in the first
  paragraph of Case~2.3.1. We embed  analogously to the previous
  paragraph, using  as the location for the star-center. To embed
  , let us consider the embedding . It is again an
  independent set. As opposed Case~2.3.1, however, we have local roots
  at  and at . Fortunately, since  and  is
  a smallest subtree, also , and hence , as required. Hence, we can embed  explicitly,
  analogously to the second paragraph of Case~2.3.1.

  \case{2.3.1.2} There is a degree-conflict for embedding  onto
  . Then this conflict must have been created by our
  transformation of the embedding of . Before this transformation
  there was no degree-conflict by assumption (Case~2.2 handles this
  scenario). In other words,  is the root of a star  in
  the initial embedding whose center is at . After moving  out of
  the interval ,  became the local root, which raised
  the degree-conflict. By our claim and the preceding Case~2.3.1.1 we
  may suppose that 
  (\figurename~\ref{fig:large_blue_star_2312_1}). We use a different,
  explicit embedding as follows: flip the star  so that its
  root is at  and the center is at  and draw all edges below the
  spine. Next move the center at  to  instead, shifting all
  vertices in between to the right by one. Then put  at  (not
  being the root of  it is not in conflict), and explicitly
  embed  onto the (now) independent set . Finally,
  explicitly embed  onto the (now) independent set 
  (\figurename~\ref{fig:large_blue_star_2312_2}). Note that  might be
  a tree (in which case the two roots in the figure are actually
  connected), but the embedding works also in this case.
\begin{figure}[htbp]
    \centering\hfil \subfloat[]{\includegraphics{lbs_2312_0}\label{fig:large_blue_star_2312_1}}\hfil
    \subfloat[]{\includegraphics{lbs_2312}\label{fig:large_blue_star_2312_2}}\hfil
    \caption{A new degree-conflict for  on 
      (Case~2.3.1.2).\label{fig:large_blue_star_2312}}
  \end{figure}


  \case{2.3.2}  and . Then by
  our claim we may suppose that  is a star. We embed  explicitly
  onto , noting that  is a non-root leaf of  and,
  therefore, not in conflict with . If  is a central-star, then we
  put the center at , which is connected to  only and therefore
  locally isolated on . Otherwise,  and  is a
  dangling star. Given that , we have  and therefore
  can put the root of  on  and the center on .

  \case{2.3.3}  and . We distinguish
  two final subcases.

  \case{2.3.3.1} The root of  is at . Then we change the
  embedding of  by moving the vertex at  to  and shifting the
  vertices in between to the right by one. We explicitly embed 
  onto . This is possible because  is an
  independent set except for the single edge  and  is
  not a central-star. Then if  is a central-star, we embed
  the center at , which is an isolated vertex in .
  If  is a dangling star, then we embed the root at  and
  the center at . Otherwise,  is not a star and we
  recursively embed  onto . Recall that  is a
  locally isolated vertex and 
  (because  is a leaf whose only neighbor is at ).
  Therefore, there is no conflict for the recursive embedding and
   is not a star.

  \case{2.3.3.2} The root of  is at  (and possibly in
  conflict with ; \figurename~\ref{fig:large_blue_star_2331_1}).

  If  is a central-star, then we change the embedding of  as
  follows: First flip  so that its center is at  and then
  exchange the vertices at  (center  of ) and  (a leaf
  of ). Put  at , which is a leaf of  and therefore not
  in conflict. Then put  at , whose only neighbor is (now) at
   (originally at ), drawing the edge  above the spine.
  Next put a leaf of  on  (center  of ), again drawing
  the edge  above the spine. Put the remaining leaves of  on
  the vertices , drawing the edges to  below the spine.
  This leaves us with a set of isolated vertices, all accessible from
  below the spine, on which we can complete an explicit embedding of
   (\figurename~\ref{fig:large_blue_star_2331_2}).
\begin{figure}[htbp]
    \centering\hfil \subfloat[]{\includegraphics{lbs_2331_0}\label{fig:large_blue_star_2331_1}}\hfil
    \subfloat[]{\includegraphics{lbs_2331}\label{fig:large_blue_star_2331_2}}\hfil
    \subfloat[]{\includegraphics{lbs_2331_1}\label{fig:large_blue_star_2331_3}}\hfil
    \caption{,  and  is a
      central-star (Case~2.3.3.1).\label{fig:large_blue_star_2331}}
  \end{figure}


  A similar embedding also works for a dangling star : Put  at
   (which is another leaf of  because ), put 
  at  and the center of  at .

  Otherwise,  is not a star. Then we modify the embedding of  by
  drawing all edges of  below the spine and exchanging  and
  . Explicitly embed  onto . This is possible
  because  is an independent set except for the edge
   and  is not a central-star. Recursively embed 
  onto  (\figurename~\ref{fig:large_blue_star_2331_3}). As
   is a locally isolated vertex in , we know that
   is not a star. There is no degree-conflict by
  assumption (Case~2.2 handles this scenario) and---as opposed to
  Case~2.3.1.2---we do not change  here. Again by assumption
   and so there is no edge-conflict for the
  recursive embedding of , either.
\end{proof}

\section{Embedding the red tree: a large red star}
\label{subsec:rec_large_red_star}
In this section we handle the case where  is a star. If  is a
star, then it must be a dangling star: otherwise, by the choice of 
as a smallest subtree,  would be a star. Let  be the child of 
in  and let . Then  is a central-star. Our default
approach in this case is to explicitly embed  and recursively embed
. Note that . Consequently, when we
try to recursively embed  onto some interval , there can be
a degree-conflict only if  is a star: a case we must handle
separately anyway. Hence, for a recursive embedding of  it suffices
to check that we are not embedding against a star, to establish the
placement invariant, and to check that there is no edge-conflict.

\begin{proposition}\label{prop:rec_large_red_star_ij_not_used_sp_not_star}
  If  is a star,  is not a star, and
  , then  and  admit an ordered plane
  packing onto .
\end{proposition}
\begin{proof}
  Let . We have  since  is not a star.
  Hence, . Flip  if necessary to put its
  root at . We first try the following. Use the red-star
  embedding to embed  at  and the children of  on the
   rightmost non-neighbors of  in . Let  be
  the interval of remaining vertices. Embed  recursively onto .
  See \figurename~\ref{fig:large_red_ij_not_used_default}.

  \begin{figure}[b]
    \centering\hfil \subfloat[Default]{\includegraphics{large_red_ij_not_used_default}\label{fig:large_red_ij_not_used_default}}\hfil \subfloat[Case~1.1]{\includegraphics{large_red_ij_not_used_ix_star_large}\label{fig:large_red_ij_not_used_ix_star_large}}\hfil \subfloat[Case~1.2]{\includegraphics{large_red_ij_not_used_ix_star_equal}\label{fig:large_red_ij_not_used_ix_star_equal}}\hfil \subfloat[Case~2]{\includegraphics{large_red_ij_not_used_dc_default}\label{fig:large_red_ij_not_used_dc_default}}\hfil \caption{The case analysis in the proof of
      Proposition~\ref{prop:rec_large_red_star_ij_not_used_sp_not_star}~(Part~1/6).}
  \end{figure}

  Let us first consider the conditions under which the embedding of
   works.
The embedding fails if  is a star, which happens only if
  (Case~1)  is a star with  and
  . Otherwise, suppose there is an edge-conflict for
  embedding  onto . Then  is a
  central-star rooted at . By choice of , we have
  . By~\ref{inv:starconflict},  has no edge to the
  parent of . If it had an edge to  (which is where we embedded
  ), then by 1SR  is embedded at . But then ,
  a contradiction. As argued at the start of this section, there can be
  no degree-conflict for . Hence, \ref{inv:starconflict} holds.

  The embedding of  works unless there is a degree-conflict for
  placing  onto  and embedding the children of  onto
  , that is unless (Case~2)
  . We deal with both cases below.

  \case{1}  is a star with  and
  . Let  be such that .

  \case{1.1} . Flip  if necessary to put its
  center at . Embed  onto  and the children of  explicitly
  onto . See
  \figurename~\ref{fig:large_red_ij_not_used_ix_star_large}. This works
  since . Embed  recursively onto .
  This works since  and hence  is an
  independent set and  is not the root of .

  \case{1.2} . Flip  if necessary to put its center
  at . By the peace invariant, this star-center is not in
  edge-conflict with . Since , the blue vertex at 
  is a leaf in  that is not the root of . We change the blue
  embedding as follows. Simultaneously move  to 
  and  to . The edge  is drawn in the lower
  halfplane. Note that  is now an isolated vertex.
  Embed  onto  and the children of  onto  and
  . This works because  is isolated. Embed 
  at . Embed  recursively onto  if  is not a star. See
  \figurename~\ref{fig:large_red_ij_not_used_ix_star_equal}. Otherwise,
   is a dangling star. In this case, embed  at , the child 
  of  onto  and the children of  onto . This
  works because  is isolated in  (and so  is not
  used) and  is isolated in  (and so  is not used
  and there is no conflict for embedding  onto ).

  \case{2} . Then
   and so . Let
   and  such that  and . Since
   we have . Flip  to put its root at
  . The proof of this case will not rely on the peace invariant,
  except in Case~2.3.3.

  We first try the following. If  has a subtree that is not a
  central-star on at least  vertices, then rearrange  to
  put a smallest such subtree at . Embed  at  and the
  children of  at . Embed  recursively at . See
  \figurename~\ref{fig:large_red_ij_not_used_dc_default}. This fails
  immediately if (1)  is a star, in which case every subtree of
   is a central-star on at least  vertices or a dangling star
  on exactly  vertices. Otherwise, suppose there is a
  edge-conflict for embedding . Then 
  is a star rooted at a center . Since  is the only vertex on
   with edges to the outside of ,  must be adjacent to
   (which is where we placed ). By 1SR, we have . Hence,
  we must handle the case (2)  separately.
  This covers the possible issues with the recursive embedding of .
  The embedding of  works unless (3)  is not isolated in
  . We deal with these three cases next.

  \case{2.1}  is a star. Then by the rearrangement of 
  performed above, every subtree of  is a central-star on at least
   vertices or a dangling star on exactly  vertices.

  \case{2.1.1} Some subtree  of  is a dangling star on exactly
   vertices. Re-embed , placing the root at  and 
  as the closest subtree. Afterwards, , the center of 
  is at , and  is isolated in . Embed  at , 
  recursively onto ,  onto , and the children of  onto
  . The embedding of  works because 
  and  is isolated in . The embedding of  works because
   is a leaf of  and hence adjacent only to  in .

  \case{2.1.2} Every subtree of  is a central-star on at least
   vertices. Flip  to put its root at . Embed 
  onto  and  onto . We embed  explicitly, as follows. Let
   be the children of  such that  is a
  largest subtree. Since  is not a central-star,
  . Let  be the children of  ordered
  by proximity of  ( is the closest). By the assumption of
  Case~2 we have  and hence . We embed
   as follows. Reroute the edges of  to its
   rightmost neighbors via the lower halfplane. Embed
   explicitly on these  rightmost neighbors of 
  and on  in the upper halfplane. For , we embed 
  as follows. Reroute the edges of  to its  rightmost
  neighbors via the lower halfplane and embed  explicitly on
  these vertices in the upper halfplane. Since we embedded a vertex of
   on ,  is isolated on the remainder. Embed  onto
   and the children of  onto the remainder. See
  \figurename~\ref{fig:large_red_ij_not_used_dc_star}.

  \begin{figure}[t]
    \centering\hfil \subfloat[Case~2.1.2]{\includegraphics{large_red_ij_not_used_dc_star}\label{fig:large_red_ij_not_used_dc_star}}\hfil \subfloat[Case~2.2.1]{\includegraphics{large_red_ij_not_used_dc_not_star_not_yy1_default}\label{fig:large_red_ij_not_used_dc_not_star_not_yy1_default}}\hfil \subfloat[Case~2.2.1.1]{\includegraphics{large_red_ij_not_used_dc_not_star_not_yy1_not_star}\label{fig:large_red_ij_not_used_dc_not_star_not_yy1_not_star}}\hfil \caption{The case analysis in the proof of
      Proposition~\ref{prop:rec_large_red_star_ij_not_used_sp_not_star}~(Part~2/6).}
  \end{figure}

  \case{2.2}  is not a star and . We
  consider two cases.

  \case{2.2.1} . Embed  onto  and
  the children of  onto . This works by 1SR and
  . Embed  recursively onto . See
  \figurename~\ref{fig:large_red_ij_not_used_dc_not_star_not_yy1_default}.
  Since  is not a star by assumption and since  is
  rooted at , the only possible issue is an edge-conflict. In
  that case, let  such that . The root
   of  is in edge-conflict with . Due to the edge
   that is used by the blue embedding, the edge-conflict can be
  caused only by an edge from  to  (which is where we embedded
  ). By 1SR, . Since  and  is not
  a star, we have . We consider two cases.

  \begin{figure}[b]
    \centering\hfil \subfloat[Case~2.2.1.2]{\includegraphics{large_red_ij_not_used_dc_not_star_not_yy1_star_1}\label{fig:large_red_ij_not_used_dc_not_star_not_yy1_star_1}}\hfil \subfloat[Case~2.2.1.2]{\includegraphics{large_red_ij_not_used_dc_not_star_not_yy1_star_2}\label{fig:large_red_ij_not_used_dc_not_star_not_yy1_star_2}}\hfil \subfloat[Case~2.2.2.1]{\includegraphics{large_red_ij_not_used_dc_not_star_yy1_not_star_1}\label{fig:large_red_ij_not_used_dc_not_star_yy1_not_star_1}}\hfil \subfloat[Case~2.2.2.1]{\includegraphics{large_red_ij_not_used_dc_not_star_yy1_not_star_2}\label{fig:large_red_ij_not_used_dc_not_star_yy1_not_star_2}}\hfil \caption{The case analysis in the proof of
      Proposition~\ref{prop:rec_large_red_star_ij_not_used_sp_not_star}~(Part~3/6).}
  \end{figure}

  \case{2.2.1.1}  is not a star. Embed  onto ,  onto ,
  the children of  onto , and  recursively onto
  . See
  \figurename~\ref{fig:large_red_ij_not_used_dc_not_star_not_yy1_not_star}.
  Since  is a star and , by 1SR  is isolated on
  : hence the embedding of  works and the edges  and
   incident to  are not used by the blue embedding. Hence,
  the only possible issues are caused by recursively embedding .
  Suppose there is a conflict for embedding  onto .
  Then  is a central-star. Since
   and  it follows that the root (and hence the
  center) of  is at . But since  is a
  subtree of  on more than one vertex, this violates LSFR at .
  Hence, there is no conflict for embedding , which concludes
  this case.

  \case{2.2.1.2}  is a star. Since  is not a star,  is a
  dangling star. Let  be the child of . We distinguish two cases.
  If , then embed  onto ,  onto , the children of
   onto ,  onto ,  onto , and the children
  of  onto . See
  \figurename~\ref{fig:large_red_ij_not_used_dc_not_star_not_yy1_star_1}.
  Since  is not the root of ,  has no edges to the outside
  of the interval and hence it is safe to embed  there. Since
   is a star centered at ,  is adjacent only to  in
  the blue embedding and hence  and
  . By 1SR,  is isolated in  and
  hence we can embed  as described and similarly  is isolated in
   and hence we can embed  as described.

  If , then flip the blue embedding at . This
  places the center of the star  at  and its root at .
  Since , the vertices  and  are adjacent only to
   now. Embed  onto  (which is not the root of ), 
  onto ,  onto , the children of  onto , 
  onto  (the edge  is no longer used after flipping),
  and the children of  onto . See
  \figurename~\ref{fig:large_red_ij_not_used_dc_not_star_not_yy1_star_2}.
  After flipping,  is isolated in  and hence the
  embedding of  works.

  \case{2.2.2} .

  \case{2.2.2.1}  is not a star centered at . Let  be
  such that . Since 
  and  (Case~2.2),  is a central-star. Since
   is not a star (Case~2.2), . By LSFR at 
  we know that . Since  we know that
  .

  Suppose first that  is a dangling star and let  be the child of
   in . Then embed  onto ,  onto , the children of
   onto , and  onto . Flip  into the lower
  halfplane. Embed  onto , drawing the edge  in the
  upper halfplane. Embed the children of  onto . The
  edges between  and its children embedded at  are drawn
  as biarcs. See
  \figurename~\ref{fig:large_red_ij_not_used_dc_not_star_yy1_not_star_1}.

  Otherwise,  is not a star since  is not a star. Flip
  . Embed  onto  and the children of  onto
  . Embed  recursively onto . See
  \figurename~\ref{fig:large_red_ij_not_used_dc_not_star_yy1_not_star_2}.
  Since  is isolated in ,  is not a star. If there is
  a conflict for the embedding of  onto , then
   must be a central-star rooted and centered at
  . But this violates the LSFR at  before flipping. Hence, this
  embedding works.



  \case{2.2.2.2}  is a star centered at . Let
  . We reembed  as follows. Use the normal
  embedding algorithm for blue trees to embed  onto 
  (placing the root at ), but embed  as the closest subtree,
  i.e., embed  at . This embeds the center of  at
  .

  Embed  onto ,  onto , and the children of  onto
  . This works so far:  is adjacent only to  in the
  blue embedding. If  is a star (it is not rooted at a center) then
  embed  onto , the child  of  onto , and the children
  of  onto . See
  \figurename~\ref{fig:large_red_ij_not_used_dc_not_star_yy1_star_1}.
  This works because  is isolated on . If  is not a
  star, embed  recursively onto . See
  \figurename~\ref{fig:large_red_ij_not_used_dc_not_star_yy1_star_2}.
  Since  is isolated in ,  is not a star and
  there is no conflict for embedding  onto .


  \begin{figure}
    \centering\hfil \subfloat[Case~2.2.2.2]{\includegraphics{large_red_ij_not_used_dc_not_star_yy1_star_1}\label{fig:large_red_ij_not_used_dc_not_star_yy1_star_1}}\hfil \subfloat[Case~2.2.2.2]{\includegraphics{large_red_ij_not_used_dc_not_star_yy1_star_2}\label{fig:large_red_ij_not_used_dc_not_star_yy1_star_2}}\hfil \subfloat[Case~2.3.1.1]{\includegraphics{large_red_ij_not_used_dc_not_isolated_1}\label{fig:large_red_ij_not_used_dc_not_isolated_1}}\hfil \caption{The case analysis in the proof of
      Proposition~\ref{prop:rec_large_red_star_ij_not_used_sp_not_star}~(Part~4/6).}
  \end{figure}


  \case{2.3}  is not a star and  is not isolated in
  . We distinguish three cases.

  \case{2.3.1}  is not a star and  is not isolated in
  . By 1SR at , the edge  is not used. Let 
  be the rightmost neighbor of  on . Then  and
   is a tree on at least three vertices.

  If  is a central-star, then its root and center is at . Use
  the leaf-isolation-shuffle on  to place a leaf at  and
  its parent at . By Proposition~\ref{prop:leafshuffle}, if 
  is not a central-star, this places the root of  at  and
  preserves the 1SR at . If  is a central-star, then let 
  be the largest index such that  is a central-star. Since
   is not a star, we have  and since  we
  have . The leaf-isolation-shuffle places the root of
   at . Note that all vertices in  are now also
  adjacent to .

  \case{2.3.1.1}  is not a central-star or  is a
  central-star but . In the latter case, the edge  is
  not used, as this would imply that  is used by LSFR at ,
  contradicting the choice of . Embed  onto  and the children
  of  onto . This works because  is adjacent only to
   in . Embed  recursively onto . See
  \figurename~\ref{fig:large_red_ij_not_used_dc_not_isolated_1}. Since
  the root of  is at , any edge-conflicts must be caused by
  edges to  (which is where we embedded ). However, only  is
  adjacent to  in  and  by 1SR on
   or by . Hence, there is no conflict for embedding 
  onto .

  \case{2.3.1.2}  is a central-star with . Then  is
  a dangling star centered at . Since , we can proceed as in
  Case~2.2.2.2 (the argument still works for the larger star we have in
  this case).

  \case{2.3.2}  is not a star and  is isolated in .
  Since , it follows that  is not isolated in
  . We first try the following. Embed  onto ,  onto
  , the children of  onto , and  recursively onto
  . See
  \figurename~\ref{fig:large_red_ij_not_used_dc_not_isolated_2}. The
  embedding of  works because  is isolated in . The
  embedding of  fails if (1)  is a star. In addition, the
  embedding could fail if  is a star or if there is a conflict
  for embedding  onto , in which case 
  is a central-star. We cover these cases with (2)  is a
  dangling star and (3)  is in conflict for embedding .

  \case{2.3.2.1}  is a star. Since  is not a star,  is a
  dangling star centered at . Let  be such that
  . Suppose first that  is not a
  central-star. Use a leaf-isolation shuffle on  to put a leaf
  at , its parent at , and the root at . This works by
  Proposition~\ref{prop:leafshuffle}. Embed  onto ,  onto ,
  and the children of  onto . This works so far, since the
  leaf-isolation shuffle preserves the 1SR at . Embed  onto ,
   onto , and the children of  onto . This works
  because  and because  is isolated in
  . See
  \figurename~\ref{fig:large_red_ij_not_used_dc_not_isolated_3}.

  \begin{figure}
    \centering\hfil \subfloat[Case~2.3.2]{\includegraphics{large_red_ij_not_used_dc_not_isolated_2}\label{fig:large_red_ij_not_used_dc_not_isolated_2}}\hfil \subfloat[Case~2.3.2.1]{\includegraphics{large_red_ij_not_used_dc_not_isolated_3}\label{fig:large_red_ij_not_used_dc_not_isolated_3}}\hfil \subfloat[Case~2.3.2.1]{\includegraphics{large_red_ij_not_used_dc_not_isolated_4}\label{fig:large_red_ij_not_used_dc_not_isolated_4}}\hfil \subfloat[Case~2.3.2.2]{\includegraphics{large_red_ij_not_used_dc_not_isolated_5}\label{fig:large_red_ij_not_used_dc_not_isolated_5}}\hfil \caption{The case analysis in the proof of
      Proposition~\ref{prop:rec_large_red_star_ij_not_used_sp_not_star}~(Part~5/6).}
  \end{figure}

  Otherwise,  is a central-star. Then it must be rooted and
  centered at . By the assumption of Case~2.3, we must have .
  Embed  onto ,  onto , and  onto . This works so far
  since  and
  . Embed a child of  on
  every vertex in . Exactly  children of  remain
  to be embedded. Since  is not isolated in  (assumption
  of Case~2.3),  and  must have some common parent  with
  . By LSFR at , we have
  , and so  is large enough to
  accomodate all remaining children of . This is true even if 
  (for which we modified the order of the subtrees), since  is not
  the last subtree. Thus, embed the remaining children of  onto
  . Embed  onto the leaf  of  and the
  children of  onto the remainder. See
  \figurename~\ref{fig:large_red_ij_not_used_dc_not_isolated_4}.

  \case{2.3.2.2}  is a dangling star. Since  is not
  isolated in , we must have . Simultaneously
  shift  to  and  to . Embed  onto ,
   onto , the children of  onto , and 
  recursively onto . Since  is isolated in , the
  recursive embedding of  always works. See
  \figurename~\ref{fig:large_red_ij_not_used_dc_not_isolated_5}.

  \case{2.3.2.3}  is in conflict for embedding . Let
   be such that . Then  is a
  central-star rooted at . If , then  must be connected to
   and so  is a star. This contradicts our assumption of
  Case~2.3 and hence . The root  of  cannot be in
  edge-conflict with  because . Thus, it is
  in degree-conflict and we have .
  Recall from the start of Case~2 that the root of  has degree
  at least . Since  is not an isolated vertex, all other
  subtrees of  must have size at least two. Thus,
  . Embed  onto . Use a
  blue-star embedding to embed  onto . Note that
   and that \ref{gg:dc} is satisfied by the
  discussion above. Embed  onto  and the children of  onto
   to complete the embedding.

  \case{2.3.3}  is a star. Recall that  is rooted at
  . Since ,  is a central-star.
  Then the rearrangement of  at the start of Case~2 did not
  change anything, and hence  satisfies the invariants. We replay the
  case analysis, starting from the very start of this proof, but now we
  embed on  (i.e. we embed from the other side). Note
  that  may not satisfy the peace invariant, but it satisfies
  the other invariants. Consider the initial embedding in the proof,
  which performs a red-star embedding of  from  and then
  embed  on the left of . The embedding of  always
  works:  is a star of size larger than  which appears on
  the left of the interval , and hence the first 
  elements of  form an independent set. Thus, if the initial
  embedding fails, we must land in Case~2. Since we have not yet used
  the peace invariant in Case~2 so far, we can simply execute the case
  analysis of Case~2 until we get an embedding or we arrive at this case
  (Case~2.3.3).

  It remains to consider the event that the embedding procedure also
  reaches this case (Case~2.3.3) for embedding  onto . Refer
  to Figure~\ref{fig:large_red_ij_not_used_dc_two_stars}. Then
   and  are both central-stars of size larger
  than . Flip  if necessary to put its root at 
  and flip  if necessary to put its root at . Let  be
  such that . By the peace invariant for embedding
   on , the root of  at  is not in
  edge-conflict with . Embed  at  and  at , drawing the
  edge  as a biarc that is in the upper halfplane near  and
  crosses the spine between  and . Embed a child of  on
  every vertex in . Using that
  , this works because
  .
  Embed  onto . The remaining blue vertices in  form an
  independent set on which we can easily embed the remaining children of
  s  and  explicitly.

  \begin{figure}
    \centering \subfloat[Case~2.3.3]{\includegraphics{large_red_ij_not_used_dc_two_stars}\label{fig:large_red_ij_not_used_dc_two_stars}}\caption{The case analysis in the proof of
      Proposition~\ref{prop:rec_large_red_star_ij_not_used_sp_not_star}~(Part~6/6).}
  \end{figure}
\end{proof}


\begin{proposition}\label{prop:rec_large_red_star_ij_not_used_sp_star}
  If  and  are both stars and ,
  then  and  admit an ordered plane packing onto .
\end{proposition}
\begin{proof}
  Let  be the child of  in  and let . Then  is a
  star centered at  and  is a star centered at . The case
   is handled by Lemma~\ref{lem:rec_singleton}. In the remainder
  we assume .  We deal with two red stars here, so we
  frequently use the red-star embedding. Since all embeddings in this
  proof are explicit (we cannot recursively embed stars, after all), we
  only perform Step~1 (Embed) of the red-star embedding for ease of
  explanation.

  Let  such that . Re-embed  by putting
  its root at  and embedding its subtrees according to the
  smaller-subtree-first rule (SSFR) and the 1SR. By assumption,
   and hence these modifications do not touch
  . Our general plan is the following. Embed  at .
  This works by the placement invariant. Perform a red-star
  embedding to embed  onto  and the children of  onto the
  rightmost  non-neighbors of  in . Since
  ,  is not in edge-conflict with  and
  hence~\ref{sgg:ec} holds. Hence, this works unless~\ref{sgg:dc} fails,
  i.e., unless (1) . We
  embed  onto the rightmost child  of . This works unless 
  has no children, i.e., unless (2) . We finally embed the
  children of  onto the remaining vertices. See
  \figurename~\ref{fig:two_red_no_ij_default}. Since the
  red-star embedding ensures that all remaining vertices are
  visible from below, this is possible unless  has an edge to a
  remaining vertex. Note that all edges of  are in , and we
  embedded  onto . Hence, it suffices to handle the case where the
  red-star embedding did not embed a child of  onto every
  vertex of , i.e., the case that (3)
  . We deal with these remaining cases
  below. We first state a useful observation.

  \begin{figure}[b]
    \centering\hfil \subfloat[Default]{\includegraphics{two_red_no_ij_default}\label{fig:two_red_no_ij_default}}\hfil \subfloat[Case~1]{\includegraphics{two_red_no_ij_dc}\label{fig:two_red_no_ij_dc}}\hfil \subfloat[Case~2]{\includegraphics{two_red_no_ij_deg0_default}\label{fig:two_red_no_ij_deg0_default}}\\
    \subfloat[Case~2.1]{\includegraphics{two_red_no_ij_deg0_y1_not_iso_1}\label{fig:two_red_no_ij_deg0_y1_not_iso_1}}\hfil \subfloat[Case~2.1]{\includegraphics{two_red_no_ij_deg0_y1_not_iso_2}\label{fig:two_red_no_ij_deg0_y1_not_iso_2}}\hfil \label{fig:two_red_no_ij_1}
    \caption{The case analysis in the proof of
      Proposition~\ref{prop:rec_large_red_star_ij_not_used_sp_star}
      (Part~1/4).}
  \end{figure}


  \begin{observation}\label{obs:two_stars_sdc}
    Let  be the roots of two different trees of the
    forest . Suppose that . Then
    \begin{enumerate}[label={(P\arabic*)}]\setlength{\itemindent}{3\labelsep}
    \item\label{obs:two_stars_sdc_large} ;
    \item\label{obs:two_stars_sdc_leaves} at least three children of 
      are leaves; and
    \item\label{obs:two_stars_sdc_nodc} 
    \end{enumerate}
  \end{observation}
  \begin{proof}
    Since  has two subtrees and  is the smaller one, we have
     and hence . By
    the assumption, we have , as claimed
    in~\ref{obs:two_stars_sdc_large}. Let  be the number of
    leaf subtrees of . The other subtrees of  have size at least
    two and the total size of  is at most , since 
    and  are the roots of different trees. Hence,
    , and so
    . Then,
    by~\ref{obs:two_stars_sdc_large}, ,
    which proves~\ref{obs:two_stars_sdc_leaves}.

    Since  has two subtrees and  is the larger one, we have
     and hence . The
    first inequality of~\ref{obs:two_stars_sdc_nodc} follows from
    . Suppose towards a contradiction that the second
    inequality of~\ref{obs:two_stars_sdc_nodc} is false, that is,
    . Adding this equation to the
    assumption, we obtain . Since , it follows that
    , which contradicts .
    Claim~\ref{obs:two_stars_sdc_nodc} follows.
  \end{proof}

  \case{1} . Then
  Observation~\ref{obs:two_stars_sdc} applies with . Re-embed
   by placing its root at  and embedding its subtrees with
  LSFR and 1SR. Embed  onto  and  onto . This works
  because  and  are leaves by~\ref{obs:two_stars_sdc_leaves} and
  LSFR. If necessary, flip  to put its root at . Use
  the red-star embedding to embed  onto  and the children
  of  onto the leftmost  non-neighbors of  in
  . \ref{sgg:ec} holds since .
  \ref{sgg:dc} holds since  and
  by~\ref{obs:two_stars_sdc_nodc} with  and . Let  be
  the largest index on which a child of  was embedded. Then
  . Since 
  by~\ref{obs:two_stars_sdc_large} we have . Then
  . It follows that
  , and so the red-star embedding embedded a child of
   onto . Since this is the only vertex in  adjacent to
   (which is where we embedded ), we can embed the children of
   on the remainder. See \figurename~\ref{fig:two_red_no_ij_dc}.

  \case{2} . Let  such that . If  is
  isolated in  then embed  onto ,  onto , the
  children of  onto ,  onto , and the children of
   onto . See
  \figurename~\ref{fig:two_red_no_ij_deg0_default}. This works due to
  the placement invariant and the fact that  is isolated in 
  and  is isolated in . Otherwise,  is not isolated in
  . We distinguish two cases.

  \case{2.1}  is not isolated in . Let  be the
  rightmost neighbor of  in . We have . If
  , then perform a leaf-isolation-shuffle on  to put a
  leaf at  and its parent at . Embed  onto . Since ,
  the blue vertex at  was not changed and hence this works by the
  placement invariant. Embed  onto  and the children of  onto
  . This works since  is adjacent only to  in
  . Finally, embed  onto  and the children of  onto
  . This works because  is isolated in . See
  \figurename~\ref{fig:two_red_no_ij_deg0_y1_not_iso_1}.

  Otherwise, . Since  was chosen as the rightmost vertex of
   in , we have  and hence
  . Flip . After flipping,
  . Embed  onto  and and  onto . Flip
   into the lower halfplane and embed the children of 
  onto . This works because after flipping,  is adjacent
  only to  in . Finally, embed  onto  and the
  children of  onto . This works because  is isolated
  in . See
  \figurename~\ref{fig:two_red_no_ij_deg0_y1_not_iso_2}.

  \case{2.2}  is isolated in . In other words, all
  (possibly zero) edges incident to  leave  to the right. We
  distinguish two cases.

  \case{2.2.1}  is a central-star. If
   then use Lemma~\ref{lem:rec_large_blue_star}
  to compute an ordered plane packing. Otherwise
  . Flip  if necessary to put
  its root at .

  If  then flip  if necessary to put the
  root away from  (recall that  is not isolated in ).
  Embed  onto ,  onto , the children of  onto
  ,  onto  and the children of  onto .
  See \figurename~\ref{fig:two_red_no_ij_deg0_y1_iso_star_1}. This works
  because  and  are both isolated in .

  \begin{figure}
    \centering\hfil \subfloat[Case~2.2.1]{\includegraphics{two_red_no_ij_deg0_y1_iso_star_1}\label{fig:two_red_no_ij_deg0_y1_iso_star_1}}\hfil \subfloat[Case~2.2.1]{\includegraphics{two_red_no_ij_deg0_y1_iso_star_2}\label{fig:two_red_no_ij_deg0_y1_iso_star_2}}\hfil \label{fig:two_red_no_ij_2}
    \caption{The case analysis in the proof of
      Proposition~\ref{prop:rec_large_red_star_ij_not_used_sp_star}
      (Part~2/4).}
  \end{figure}

  If , then we change the blue embedding as follows.
  Simultaneously shift  to  and  to . The
  new edge  is drawn in the lower halfplane. Afterwards,  is
  isolated in  and  is isolated in . Embed  onto
  . By the peace invariant,  is not in edge-conflict with
  . Embed  onto  and the children of  onto .
  Embed  onto  and the children of  onto  and .
  See \figurename~\ref{fig:two_red_no_ij_deg0_y1_iso_star_2}.

  \begin{figure}
    \centering\hfil \subfloat[Case~2.2.2]{\includegraphics{two_red_no_ij_deg0_y1_iso_no_star_1}\label{fig:two_red_no_ij_deg0_y1_iso_no_star_1}}\hfil \subfloat[Case~2.2.2]{\includegraphics{two_red_no_ij_deg0_y1_iso_no_star_2}\label{fig:two_red_no_ij_deg0_y1_iso_no_star_2}}\hfil \subfloat[Case~2.2.2]{\includegraphics{two_red_no_ij_deg0_y1_iso_no_star_3}\label{fig:two_red_no_ij_deg0_y1_iso_no_star_3}}\hfil \subfloat[Case~2.2.2]{\includegraphics{two_red_no_ij_deg0_y1_iso_no_star_4}\label{fig:two_red_no_ij_deg0_y1_iso_no_star_4}}\hfil \label{fig:two_red_no_ij_3}
    \caption{The case analysis in the proof of
      Proposition~\ref{prop:rec_large_red_star_ij_not_used_sp_star}
      (Part~3/4).}
  \end{figure}


  \case{2.2.2}  is not a central-star. Then
  . Flip  if necessary to put its root
  at . Let  such that  and let  be the
  leftmost neighbor of . Then .

  If  then embed  onto , 
  onto , the children of  onto ,  onto ,
  and the children of  onto . See
  \figurename~\ref{fig:two_red_no_ij_deg0_y1_iso_no_star_1}.

  Otherwise, . Embed  onto  and 
  onto . Embed children of  onto . Use the
  red-star embedding to embed the remaining children of  onto
  the  leftmost non-neighbors of  in
  . If~\ref{sgg:dc} is not violated, we complete the
  embedding by placing  at  and embedding the children of  on
  the remainder. See
  \figurename~\ref{fig:two_red_no_ij_deg0_y1_iso_no_star_2}.
  If~\ref{sgg:dc} is violated, then
  . Equivalently,
  . It follows that
  . Since
   we have . Hence . Instead of performing the red-star embedding on
  , we now perform it on . If~\ref{sgg:dc} is not
  violated, then since our first red-star embedding failed, the
  remaining vertices are exactly the neighbors of  in , which
  form an independent set. Complete the embedding by placing  at the
  rightmost neighbor of  (which is not adjacent to ) and the
  children of  on the remainder. See
  \figurename~\ref{fig:two_red_no_ij_deg0_y1_iso_no_star_3}.

  It remains to consider the case where~\ref{sgg:dc} is again violated.
  In this case we have . Due to the
  degree-conflict and the fact that  we have . Flip  to put its root at .
  Observation~\ref{obs:two_stars_sdc} applies with . By LSFR
  and~\ref{obs:two_stars_sdc_leaves},  is a leaf of . We want
  to apply Observation~\ref{obs:unary_2dc_partition} on . We
  first argue that the preconditions are satisfied. Let
   and . Then  and by~\ref{obs:two_stars_sdc_large} , as required. Apply
  Observation~\ref{obs:unary_2dc_partition} with  and
  rearrange  to put the corresponding subtrees at .
  Afterwards, all edges adjacent to  leave  to the right. Embed
   onto ,  onto , the children of  onto , 
  onto , and the children of  onto . See
  \figurename~\ref{fig:two_red_no_ij_deg0_y1_iso_no_star_4}. This works
  since  is isolated in  and  is isolated in .

  \case{3} . Recall that we re-embedded
   by placing the root at  and embedding the subtrees of 
  according to the SSFR and 1SR. We defined  as the rightmost child
  of . We have . Hence, all
  subtrees of  have size at least . We distinguish two
  cases.

  \case{3.1} .

  \case{3.1.1} All subtrees of  are central-stars. Then we flip
  , placing its root at . Embed  onto ,  onto ,
  and the children of  onto the rightmost  non-neighbors
  of  in . Each edge is drawn with a biarc that is in the
  upper halfplane close to . This works because  (by
  our assumption and after flipping ) and since all subtrees of
   have size at least . Since  is adjacent only to 
  (which is where we embedded ), we can safely place  on  and
  the children of  on the remainder. See
  \figurename~\ref{fig:two_red_no_ij_smalls_largej_1}.

  \case{3.1.2} Some subtree  of  is not a central-star. Re-embed
  , putting the root at  and embedding the subtrees of  in
  any order that places  leftmost. Let  and  with  such
  that . By Proposition~\ref{prop:leafshuffle} and since 
  is not a central-star, we can use the leaf-isolation shuffle to place
  a leaf of  at , its parent at , and the root at .
  Embed  onto . This works because of the placement invariant.
  Embed  onto  and  onto . This works since  is
  incident only to  in  and . Embed a
  child of  onto , drawing the edge in the upper halfplane. This
  works because  and  is the leftmost subtree of
  . Embed the remaining children of  onto the rightmost vertices
  of . This works because all subtrees of  have size at
  least . Finally, note that we already embedded a vertex on the
  only blue vertex incident to , and hence we can embed the
  children of  onto the remainder. See
  \figurename~\ref{fig:two_red_no_ij_smalls_largej_2}.

  \begin{figure}
    \centering\hfil \subfloat[Case~3.1.1]{\includegraphics{two_red_no_ij_smalls_largej_1}\label{fig:two_red_no_ij_smalls_largej_1}}\hfil \subfloat[Case~3.1.2]{\includegraphics{two_red_no_ij_smalls_largej_2}\label{fig:two_red_no_ij_smalls_largej_2}}\hfil \subfloat[Case~3.2]{\includegraphics{two_red_no_ij_smalls_smallj_1}\label{fig:two_red_no_ij_smalls_smallj_1}}\hfil \subfloat[Case~3.2]{\includegraphics{two_red_no_ij_smalls_smallj_2}\label{fig:two_red_no_ij_smalls_smallj_2}}\hfil \label{fig:two_red_no_ij_4}
    \caption{The case analysis in the proof of
      Proposition~\ref{prop:rec_large_red_star_ij_not_used_sp_star}
      (Part~4/4).}
  \end{figure}


  \case{3.2} . We first try the following. Embed  onto
  . Use the red-star embedding to embed  onto  and the
  children of  onto the rightmost  non-neighbors of  in
  . \ref{sgg:ec} holds since . Since
   we have  and hence
  . This
  establishes~\ref{sgg:dc}. Embed  onto  and the children of 
  onto the remainder. See
  \figurename~\ref{fig:two_red_no_ij_smalls_smallj_1}. This works unless
  some remaining vertex is adjacent to .

  Since all neighbors of  are in , this implies , or equivalently, . Embed  onto
  . By , this is not  and hence there
  is no edge-conflict. Embed  onto  and  onto . This works
  because all neighbors of  (which is where we placed )
  in the blue embedding are in  since . Embed the
  children of  onto . This works because
  . Embed the children of  onto
   and  (with biarcs). This works
  because the only neighbor of  is at . See
  \figurename~\ref{fig:two_red_no_ij_smalls_smallj_2}.
\end{proof}


\begin{proposition}\label{prop:rec_large_red_star_ij_used}
  If  is a star and , then  and 
  admit an ordered plane packing onto .
\end{proposition}
\begin{proof}
  The presence of edge  implies that  is a tree. In
  this case, we discard the initial embedding of . Instead, we embed
   using Algorithm~\ref{alg:embed_t1}, and then \emph{re-embed} 
  using the additional information that  is a star.

  To simplify notation, we exchange the roles of  and . Refer to
  \figurename~\ref{fig:color_exchange_1}--\ref{fig:color_exchange_2}.
  That is, we assume that  has been embedded using
  Algorithm~\ref{alg:embed_t1}, its root is at , and it is composed
  of two trees  and  (corresponding to  and ,
  respectively):  is a tree of size  rooted at
  , and  is a star of size 
  centered at  and rooted at . We do not make any assumption
  about , apart from that it fulfills
  invariants~\ref{inv:starconflict} and \ref{inv:bluelocal}. It remains
  to embed  onto . Let  be a smallest subtree of  and
  let .

\begin{figure}[htbp]
  \centering\hfil \subfloat[]{\includegraphics{color_exchange_1}\label{fig:color_exchange_1}}\hfil \subfloat[]{\includegraphics{color_exchange_2}\label{fig:color_exchange_2}}\hfil \subfloat[]{\includegraphics{color_exchange_3}\label{fig:color_exchange_3}}\hfil \caption{When the default embedding of both  and  contains edge
    , we exchange the roles of  and  to simplify
    notation.\label{fig:color_exchange}}
\end{figure}

If , then Lemma~\ref{lem:rec_unary} completes the proof. If
, then Lemma~\ref{lem:rec_singleton} completes the proof. Hence
we may assume  and . Since  is a smallest
of two or more subtrees of , we have  and
. Let  be such that .
Since , it follows that  is a leaf of the
star , and  consists of isolated vertices.

Our first option to embed  is the following. Embed  onto 
using Algorithm~\ref{alg:embed_t1}, and then embed  recursively
onto ; see \figurename~\ref{fig:color_exchange_3}. This works
unless  is in degree-conflict with , or  is a star. We
consider these two possibilities separately.

\case{1}  is in degree-conflict with , but  is not a
star. In this case,  is a central-star
rooted at  and . Note that
 consists of two central-stars,  (rooted at )
and  (rooted at ). Since  and
 we have .

\case{1.1} . Embed  explicitly onto  and 
recursively onto . Since , the blue vertex at  is a
leaf that is adjacent only to . Hence, there is no edge-conflict
for the recursive embedding of . Since  is a
central-star, there could be a degree-conflict. In this case we have
. Adding this equation to the
equation for the degree-conflict at , we obtain
.
It follows that . Since each subtree of 
in  has size at least , we get
, a
contradiction. Hence, there is no degree-conflict and the recursive
embedding of  always works.

\case{1.2} . In this case  and .
Hence,  is binary in . Let  be the subtree of  in
. Embed  onto ,  explicitly onto the independent set at
, and  explicitly onto the independent set at .
Since the blue embedding uses neither  nor , this
always works.

\case{2}  is a star. Since  and  is a smallest subtree
of ,  is a dangling star, that is, it is centered at the unique
child  of  in . In this case,  and  have even more similarities:
their roots each have two children, and both  and  are dangling stars.
See \figurename~\ref{fig:color_exchange_4}.

\begin{figure}[htbp]
  \centering
  \subfloat[]{\includegraphics{color_exchange_4}\label{fig:color_exchange_4}}\hfil \subfloat[]{\includegraphics{color_exchange_6}\label{fig:color_exchange_6}}\hfil \caption{(a) In Case~2, the default embeddings of  and  share
    several edges. (b) We embed  and  explicitly
    (right).\label{fig:color_exchange_again} }
\end{figure}

We embed  and  simultaneously such that the root of  and 
are mapped to the same point, and all other vertices of  are 
are disjoint. This is possible since  and  each have size at
most . Refer to \figurename~\ref{fig:color_exchange_6}. Embed
the star  on  such
that its root (which is the root of ) is embedded at  and its center at .
The edges between the center  and other vertices of  are
semicircles \emph{below} the spine; the edge  is also a semicircle below the spine;
and the edges between  and  are biarcs that start from  below the
spine and cross the spine right after .
Embed the subtree  onto 
using Algorithm~\ref{alg:embed_t1}, with semicircles above the spine.
Embed the tree  on  such that its root (which is the root of )
is at , and its center is at , using semicircles below the spine.
If  is not a star, then finish by embedding  onto 
recursively, such that the edge  and all edges of  are
semicircles \emph{above} the spine.
If  is a central-star, embed  explicitly onto  above
the spine. If  is a dangling star, then . Flip the blue
embedding at , placing the star-center of  at
. Since , . Embed  onto
, the child  of  onto , and the children of  onto
 (all above the spine).

It remains to show that we can recursively embed  as described above
when  is not a star.
Note that  consists of an isolated vertex at  (the
root of ), and a star  centered and rooted at
. Hence \ref{inv:starconflict} and \ref{inv:bluelocal} follow.
\end{proof}

Proposition~\ref{prop:rec_large_red_star_ij_not_used_sp_not_star},
Proposition~\ref{prop:rec_large_red_star_ij_not_used_sp_star}, and
Proposition~\ref{prop:rec_large_red_star_ij_used} together prove the
following.

\begin{lemma}
  \label{lem:rec_large_red_star}
  If  is a star, then  and  admit an ordered plane packing
  onto .
\end{lemma}

\section{Embedding the red tree: a small blue star}
\label{subsec:rec_small_blue_star}
In this section, we consider the case that  is a star, but
 is not a star. The size of the star is . Due to
Lemma~\ref{lem:rec_singleton}, we may assume . Note, however,
that  may be part of a larger star within . Let
 be the maximal star in  that contains . Note
that the tree  may be larger than . Clearly, we have
. Due to 1SR, the center and the root
of  are each located at either  or the leftmost vertex of
, which may be outside of the interval . We
distinguish two cases: either 
(Proposition~\ref{prop:rec_small_blue_star_larger}) or 
(Proposition~\ref{prop:rec_small_blue_star_equal_ij_not_used},
Proposition~\ref{prop:rec_small_blue_star_equal_ij_used_no_red_star},
and Proposition~\ref{prop:rec_small_red_star_ij_used}). These cases are
tackled below.

\begin{proposition}\label{prop:rec_small_blue_star_larger}
  If  is a star and , then  and 
  admit an ordered plane packing onto .
\end{proposition}
\begin{proof}
  By Lemma~\ref{lem:rec_large_blue_star}, we may assume that
   is not a star. By Lemma~\ref{lem:rec_large_red_star}, we
  may assume that  is not a star. Recall that the center of
   is , and its root is either  or the leftmost vertex of
  . We start by rearranging the tree  such that its
  center moves to ; see \figurename~\ref{fig:move_center}. If
   is rooted at its center, then the root automatically moves to
  , as well. Otherwise  is rooted at a leaf, which is the
  leftmost vertex of  and the root of the entire tree
   due to 1SR, and then we move the root of  to .
  In both cases,  consists of  isolated vertices, and
   continues to fulfill invariant~\ref{inv:bluelocal}.

\begin{figure}
  \centering\hfil \subfloat[]{\includegraphics{move_center_1}\label{fig:move_center_1}}\hfil \subfloat[]{\includegraphics{move_center_2}\label{fig:move_center_2}}\hfil \subfloat[]{\includegraphics{move_center_3}\label{fig:move_center_3}}\hfil \subfloat[]{\includegraphics{move_center_4}\label{fig:move_center_4}}\hfil \caption{Moving the center of star  from  to :
  when  is rooted at its center (a--b), and when it is rooted at a leaf (c--d).
  \label{fig:move_center}
}
\end{figure}

\case{1}  is not a central-star of size at least
. In this case, we embed  explicitly onto
 and then  recursively onto . Since
 consists of isolated vertices and  is not
adjacent to , the embedding of  always works. We can embed 
on  because it fulfills invariants \ref{inv:starconflict} and
\ref{inv:bluelocal}. Invariant~\ref{inv:bluelocal} holds by
construction. If  is not a central-star, then
\ref{inv:starconflict} is immediate; otherwise 
is a central-star of size at most . Hence,  has
no degree-conflict with , and \ref{inv:starconflict} follows.

\case{2}  is a central-star of size at least
. Let . We claim
that . Suppose that 
for the sake of contradiction. Then before rearranging , we had
. By LSFR and since  is not a star, the root of 
was not at . Again by LSFR, the root could have been at  only if
 is a dangling star. But then, since ,
 was not a star to begin with: a contradiction. The claim
follows. Since , we have
 and so .

Since  has a degree-conflict with , we follow a
different strategy. We first blue-star embed  from  with
, and then embed  on
. The conditions for the blue-star embedding are met:
\ref{gg:ec} holds by \ref{inv:placement} for embedding  onto ;
for \ref{gg:dc} on the one hand
 and on the
other hand, by \ref{inv:starconflict}, we have 
and so . As , the vertices
in  form an interval and both \ref{gg:int}
and \ref{gg:cs} hold.

By Proposition~\ref{p:greedygrab} we are left with an interval
 that satisfies \ref{inv:bluelocal}. Note that 
includes the center  of the star , but does not include
. Consequently,  consists of isolated vertices after the
blue-star embedding, and  is not in edge-conflict with . Hence, we
can embed  explicitly onto .
\end{proof}

\begin{proposition}\label{prop:rec_small_blue_star_equal_ij_not_used}
  If  is a star, , and
  , then  and  admit an ordered plane
  packing onto .
\end{proposition}
\begin{proof}
  By Lemma~\ref{lem:rec_large_blue_star}, we may assume that
   is not a star. By Lemma~\ref{lem:rec_unary} and
  Lemma~\ref{lem:rec_large_red_star}, we may assume that
   and  is not a star. By
  Lemma~\ref{lem:rec_singleton}, we may assume that . Due to
  LSFR, the center and the root of  are each located at either
   or , but  may be either a central-star or a
  dangling star. We distinguish two cases.

  \case{1}  is not a central-star or  is a central-star. If
  necessary, flip  to put its center at . We will later perform a
  blue-star embedding of  from  with .

  Let us first check the conditions for the blue-star embedding.
  \ref{gg:ec} follows from the condition that .
  For the other conditions, consider first the case that  is a
  central-star. If the parent  of  is in  then since
   is maximal,  must have a subtree other than . By
  LSFR and , we have that . Hence, regardless
  of whether  is in , we know that
   forms an interval. \ref{gg:int} and
  \ref{gg:cs} follow. For \ref{gg:dc}, on the one hand we have
  . On the other hand we have
  .

  Otherwise,  is a dangling star. Then \ref{gg:cs} is satisfied
  (with ) and \ref{gg:int} is satisfied by the assumption
  of Case~1 and the choice of . For \ref{gg:dc}, on the one hand
  we have  since .
  On the other hand we have .

  Before performing this blue-star embedding, we modify the embedding
  of . Since  this does not affect
  the validity of the preconditions of the blue-star embedding. We
  want to ensure the following: if  is in degree-conflict with
   after the blue-star embedding, then  is a star
  before the blue-star embedding. We proceed as follows.

  The interval that the blue-star embedding will leave for 
  consists of , followed by  isolated vertices (all
  in edge-conflict). Suppose that this interval would be in
  degree-conflict for embedding . Then
   is a central-star. If
   then we do nothing. Otherwise,  is rooted
  at . Let  be the parent of  in . By 1SR we have
  . If  then  is a dangling
  star and we do nothing. Otherwise, . We claim that
  then . To prove the claim, suppose to the contrary that
  . Since  and
   we know that . By LSFR
  and 1SR and ,  has at least one subtree 
  besides  in  with size
  . Since , we know that the
  blue-star embedding consumes  and all except at most one vertex of
  . Hence, . By the
  degree-conflict, we know that .
  Since every subtree of  in  has size at least , we get

a contradiction. This proves our claim that . Now let
   be the subtrees of  from left to right. Note that
  . We select a parameter  as follows.
  If  then let . If  coincides with the root of a subtree
  of , then let  be such that  coincides with the root of
  . Otherwise, let  be such that  is contained in
  . Then  since . Modify the embedding of
   as follows. Flip the embedding of each subtree
   individually. Simultaneously shift each subtree
   one position to the right and shift  to the
  position before . In this modified embedding, 
  satisfies LSFR and 1SR and  is not a
  central-star, as intended.

  Now perform the blue-star embedding of  with the parameters listed
  above. Recursively embed  onto . This works unless
  there is a conflict. There can be no edge-conflict since
   and by \ref{inv:starconflict}. If there is
  a degree-conflict, then  is a central-star
  centered at  and . By
  the modification of the embedding described above, we know that
   was a (possibly larger) star before the blue-star
  embedding. Undo the blue-star embedding. We have
  . We distinguish two subcases.

  \case{1.1} . Flip  to put
  its center at . If necessary, flip  to put its
  center at . First blue-star embed  from  with 
  as the  leftmost vertices following ; and
  then embed  onto  using Algorithm~\ref{alg:embed_t1}.
  The conditions for the blue-star embedding are met: \ref{gg:ec}
  follows from \ref{inv:starconflict} and \ref{gg:dc} follows from
   and the assumption of
  Case~1.1. \ref{gg:int} and \ref{gg:cs} hold by choice of . The
  blue-star embedding replaces the center of  at 
  with an isolated vertex, but it does not affect . Consequently,
  after the blue-star embedding  consists of 
  isolated vertices, where  is not in edge-conflict with . Thus,
  we can embed  onto .

  \case{1.2} . By
  \ref{inv:starconflict},  is not a central-star and must
  hence be a dangling star . Since  is a
  central-star,  is centered at  and . Flip
   to place the center at . Perform the original
  blue-star embedding of  from  again. Let us consider the
  interval  that remains for . Since ,
   is an independent set. At  we have the original root
  of , which may be in edge-conflict with . Each of the
   rightmost vertices of  is in edge-conflict
  with . Since  and
   we have
  . Every subtree of  in
   has size at least , and hence we can embed one subtree
  explicitly on a prefix of  (which takes care of the vertex
   which is potentially in edge-conflict) and one subtree explicitly on
  a suffix of  (which takes care of all  vertices
  which are in edge-conflict). The remaining vertices are not in
  edge-conflict, and so we can explicitly complete this partial
  embedding of  to a complete embedding of .

  \case{2}  is a central-star and  is a dangling star.
  Flip  if necessary to put its root at . This preserves
  1SR on . We distinguish two cases.

  \case{2.1} Every vertex in  is a neighbor of .
  Since  the blue embedding is completely
  determined. Flip the blue embedding at . Embed  onto
   and the children of  onto . 
  consists of an isolated vertex at  (which is not in edge-conflict
  with  by \ref{inv:placement}) and a central-star .
  Hence, we can embed  recursively onto .

  \case{2.2} Some vertex in  is not a neighbor of .
  We first try the following. Use the red-star embedding to embed  to
   and the children of  onto the rightmost 
  non-neighbors of  in . \ref{sgg:ec} holds due to
  . For \ref{sgg:dc} we have to show that there are
  at least  non-neighbors of  in . This is the
  case because  already contains 
  non-neighbors of , and the last vertex is provided by the
  assumption of this case. Embed  recursively onto .

  This works unless there is a conflict, in which case
   is a central-star. As usual, this
  central-star cannot be in edge-conflict for  and is hence in
  degree-conflict. Since  both before
  and after the red-star embedding and since the
  red-star embedding either leaves  untouched or
  replaces \emph{only} its rightmost vertex by a vertex that is isolated
  in , we know that  was a (dangling or
  central-)star before the blue-star embedding. Undo the
  red-star embedding. By the degree-conflict, we have
  . We proceed analogously to
  Case~1.1 and Case~1.2.

  \case{2.2.1} . Recall that the
  center of  is at . If necessary, flip  to
  put its center at . First blue-star embed  from  with
   as the  leftmost vertices following
  ; and then embed  onto  using
  Algorithm~\ref{alg:embed_t1}. The conditions for the blue-star
  embedding are met: \ref{gg:ec} follows from \ref{inv:starconflict} and
  \ref{gg:dc} follows from  and
  the assumption of Case~2.2.1. \ref{gg:int} and \ref{gg:cs} hold by
  choice of  and since  is not a star. The blue-star
  embedding replaces the center of  at  with an
  isolated vertex, but it does not affect . Consequently, after the
  blue-star embedding  consists of  isolated
  vertices, where  is not in edge-conflict with . Thus, we can
  embed  onto .

  \case{2.2.2} . By
  \ref{inv:starconflict},  is not a central-star and must
  hence be a dangling star . Since the red-star
  embedding of  used only one vertex of  and since
   was a central-star after the red-star embedding,
   must be rooted at  and centered at . Flip
   to place the center at . Perform the original
  red-star embedding of  from  again. Let us consider the
  interval  that remains for .  is an
  independent set. At  we have the original root of ,
  which may be in edge-conflict with . The rightmost vertex of
   is in edge-conflict with . Since
   and  we have
  . Every subtree of  in
   has size at least , and hence we can embed one subtree
  explicitly on a prefix of  (which takes care of the vertex
   which is potentially in edge-conflict) and one subtree explicitly
  on a suffix of  (which takes cares of the vertex 
  which is in edge-conflict). The remaining vertices are not
  in edge-conflict, and so we can explicitly complete this partial
  embedding of  to a complete embedding of .
\end{proof}

\begin{proposition}\label{prop:rec_small_blue_star_equal_ij_used_no_red_star}
  If  is a star, ,
  , and  is not a star, then  and 
  admit an ordered plane packing onto .
\end{proposition}
\begin{proof}
  The presence of edge  means that  is a tree, rooted
  at  or . We distinguish two cases based on the root of . By
  Lemma~\ref{lem:rec_large_red_star}, we may assume that  is not a star.

  \case{1}  is a tree rooted at . We shall flip , and show that
   is no longer a star after the flip. By LSFR, 
  is a smallest subtree of . The largest subtree of  has size at
  least , and so its root is outside of .
  Therefore,  is an isolated vertex.
  Consequently, after flipping ,  is
  an isolated vertex, and  cannot be a star. If
   is a star now, use Lemma~\ref{lem:rec_large_blue_star} to
  find an ordered plane packing. Otherwise, none of , ,
   and  are stars, and we can use
  Lemma~\ref{lem:rec_general} to find an ordered plane packing.

  \begin{figure}[htbp]
    \centering\hfil \subfloat[]{\includegraphics{shuffle_subtrees_1}\label{fig:shuffle_subtrees_1}}\hfil \subfloat[]{\includegraphics{shuffle_subtrees_2}\label{fig:shuffle_subtrees_2}}\hfil \subfloat[]{\includegraphics{shuffle_subtrees_3}\label{fig:shuffle_subtrees_3}}\hfil \caption{(a) , , and  is rooted at .
      (b) When two subtrees of  are central-stars each with at least 2 vertices.
      (c) When  has a unique maximal subtree, and all other subtrees are singletons or not central-stars.
}
    \label{fig:shuffle_subtrees}
  \end{figure}

  \case{2}  is a tree rooted at . See
  \figurename~\ref{fig:shuffle_subtrees_1}. Since  is not a star,
  LSFR implies that  is a dangling star rooted at . That is,
   is a central-star, and by LSFR it is a largest
  subtree of . Because  is a smallest subtree of , we have
  , and so every subtree of  has size at most
  . Consequently,  has at least 3 subtrees in
  . We distinguish subcases based on the subtrees of .

  Recall that  has a maximal subtree that is a central-star
  (). If  has another maximal subtree, then either
  this is a central-star (Case~2.2) or not (Case~2.1). Otherwise,
   is the unique maximal subtree of  and either there
  exists another subtree of  that is a central-star on 
  vertices (Case~2.2) or every other subtree of  is a singleton or
  not a central-star (Case~2.3).

  \case{2.1}  has two or more maximal subtrees, but not all of them
  are central-stars. Re-embed  using Algorithm~\ref{alg:embed_t1}
  such that the subtree closest to  is \emph{not} a central-star (we
  only change a tie-breaking rule in Algorithm~\ref{alg:embed_t1}). Then
   is no longer a star, and  does not become a
  star. Use Lemma~\ref{lem:rec_general} to find an ordered plane
  packing.

  \case{2.2} Two or more subtrees of  are central-stars each with at
  least 2 vertices. Let , which is central-star
  subtree of  with at least 2 vertices. Let  be another subtree
  of  that is a central-star and has minimal size (possibly 1). We
  re-embed  as follows (see
  \figurename~\ref{fig:shuffle_subtrees_2}). Embed the root of  at
  . Embed  onto  and  onto
   each respecting 1SR. Embed all
  remaining subtrees onto  each respecting 1SR.
  Note that  does not obey 1SR because its root has subtrees on both
  sides. However,  and  each satisfy both LSFR
  and 1SR. Furthermore, neither  nor  is a
  star (since  has at least 3 subtrees); and
   is an isolated vertex.

  Provisionally place  at . We embed  recursively onto
  . There is no conflict for this embedding since 
  is isolated in  and not adjacent to . Embed 
  recursively onto . This works because
   is a singleton or not a central-star. Indeed,
  suppose to the contrary that  is a
  central-star. By construction,
   and contains the root of
  . Hence, apart from  and , all subtrees of the root of
   are singletons. By the choice of , however,  is also a
  singleton. Therefore the root of  has only one subtree with at
  least 2 vertices, contradicting our assumption.

  \case{2.3}  has a unique maximal subtree, which is a central-star,
  and every other subtree is either a singleton or not a central-star.
  Recall that  has at least 3 subtrees. Re-embed  such that its
  root is at , an arbitrary smallest subtree is embedded closest to
  , and all other subtrees are embedded according to LSFR (all
  subtrees are embedded recursively by Algorithm~\ref{alg:embed_t1}). In
  particular,  is now the second subtree of , counting from
  the right. See \figurename~\ref{fig:shuffle_subtrees_3}. As a result,
   is no longer a star, and  does not become a
  star. Note also that  becomes an
  isolated vertex (it is a leaf of the dangling star ); and
   is either an isolated vertex or not a
  central-star.

  Provisionally place  at . Embed  recursively onto
  . This works because  is
  locally isolated and not adjacent to . Embed  recursively onto
  . The recursive embedding of  works because
   is either an isolated vertex (which is not
  adjacent to the blue vertex on which  was embedded) or not a
  central-star.
\end{proof}

It remains to consider the case where  is a star, , , and  is a star. We deal with this
case by handling the case where  is a star and  in
full generality.

\begin{proposition}\label{prop:rec_small_red_star_ij_used}
  If  is a star and , then  and  admit
  an ordered plane packing onto .
\end{proposition}
\begin{proof}
  Since ,  is a tree, rooted at  or ,
  and we can use symmetry by exchanging the roles of  and 
  (\figurename~\ref{fig:smallred_ij_1}). Remove the embedding of .
  Embed  using Algorithm~\ref{alg:embed_t1}, placing its root at .
  Rename  to  and  to . Define  to be a smallest subtree
  of . Since  is rooted at  and  is not a star, there is no
  conflict for embedding  onto .

  \xxx{MK: This is very concise, maybe elaborate on what covers what?}
  Embedding  onto  is handled by Lemma~\ref{lem:rec_general},
  Lemma~\ref{lem:rec_unary}, Lemma~\ref{lem:rec_singleton},
  Lemma~\ref{lem:rec_large_blue_star},
  Lemma~\ref{lem:rec_large_red_star}, or
  Proposition~\ref{prop:rec_small_blue_star_larger} unless the situation
  after the color exchange is as follows: , ,
   is not a star,  is not a star, and (i)  is a star
  with  or (ii)  is a star and the maximal star
  that contains  has size exactly . If  is not a
  star then (ii) holds and we can use
  Proposition~\ref{prop:rec_small_blue_star_equal_ij_used_no_red_star}
  to find an ordered plane packing.

  Otherwise, we are in Case~(i) and  is a star. This means that the
  smallest subtree of both  and  is a star on at least two
  vertices and both  and  have at least two subtrees each. Denote
  by  a smallest subtree of . By symmetry (possibly exchanging
  roles again), we may assume .

  \begin{figure}[htbp]
    \centering\hfil \subfloat[]{\includegraphics{smallred_ij_1}\label{fig:smallred_ij_1}}\hfil \subfloat[]{\includegraphics{smallred_ij_2}\label{fig:smallred_ij_2}}\hfil \caption{When  and  play symmetric
      roles.\label{fig:smallred_ij}}
  \end{figure}

  We proceed as follows (\figurename~\ref{fig:smallred_ij_2}). Re-embed
   in the upper halfplane, placing  at ,  as the closest
  subtree, and the remaining subtrees according to LSFR.

  We first explain how to embed . We will do this in such a way that
   is embedded on a vertex of  at .
If  is a central-star, this re-embedding places its root and
  center at . Since , now  is an
  independent set. Embed  explicitly onto .
If  is a dangling star, the re-embedding places its root at
   and its center at . If , then embed  onto
   and its child onto .
If  and  is a central-star, flip  to
  put the root of  at  and the center at  and embed
   onto  and the children of  onto .
If  and  is a dangling star, flip  to
  put the center of  at  and embed  onto ,
  its child  onto , and the children of  onto
  .

  Next, embed  recursively onto . Since  was not
  embedded at , the only obstacle for this recursive embedding is a
  possible conflict, in which case  is a
  central-star. Since  (which is where we embedded ) is
  adjacent only to vertices of  and possibly , and since none of
  these vertices are part of , the conflict must be a
  degree-conflict. Then . As the root  of  is not in
  , we can reorder the subtrees of 
  arbitrarily without having to worry about LSFR on .
  Therefore, we may suppose that \emph{all} subtrees of  are
  central-stars on  vertices and each of them leads to a
  degree-conflict when taking the role of  above. Given
  that there are at least two such substars, we may as well choose a
  smallest one,  to have its center at . Any other substar can
  take the role intended for  in
  \figurename~\ref{fig:smallred_ij_2} originally, its leaves being
  paired up with .

  We claim that then there is no degree-conflict for embedding 
  onto  recursively. For such a degree-conflict to occur
  we need . So let us argue that this
  does not happen.

  By the choice of  as a minimal size subtree of , we have
  . As  is a smallest of at least two
  subtrees of , we have . Together this
  yields
  
  We want to show . So consider the
  expression
  
  which is non-negative because . This proves our
  claim and shows that there is no degree-conflict for embedding 
  onto  recursively. Therefore at least one of the two
  options provides an ordered plane packing as claimed.
\end{proof}

Lemma~\ref{lem:rec_singleton},
Proposition~\ref{prop:rec_small_blue_star_larger},
Proposition~\ref{prop:rec_small_blue_star_equal_ij_not_used},
Proposition~\ref{prop:rec_small_blue_star_equal_ij_used_no_red_star},
and Proposition~\ref{prop:rec_small_red_star_ij_used} together prove the
following.

\begin{lemma}
  \label{lem:rec_small_blue_star}
  If  is a star, then  and  admit an ordered plane
  packing onto .
\end{lemma}

\section{Embedding the red tree: a small red star}
\label{subsec:small_red_star}
Next, we handle the case where  is a star. We may assume that
 and  are not stars. The graph  is also
not a star and .

\begin{proposition}\label{prop:rec_small_red_star_ij_not_used}
  If  is a star and , then  and 
  admit an ordered plane packing onto .
\end{proposition}
\begin{proof}
  We may assume  by Lemma~\ref{lem:rec_unary}.  can
  be a central-star or a dangling star. We handle these cases
  separately. By Lemma~\ref{lem:rec_singleton}, we may assume that
  . Let  be such that . Flip  if
  necessary to put the root at . We use the following observation
  several times.

  \begin{observation}
    \label{obs:rec_small_red_star_sc}
    Suppose that we embedded  on a vertex of  and that at
    most  rightmost vertices of  have been replaced
    by locally isolated vertices. Then  is not in
    edge-conflict for embedding  onto .
  \end{observation}
  \begin{proof}
    Suppose to the contrary that  is in edge-conflict for
    embedding . Let  such that
    . Then the root of  is in
    edge-conflict with . It cannot be due to an edge to  since
    . Hence, it must have an edge to the
    outside of . By 1SR and LSFR,  must then also be
    a (possibly larger) central-star whose root is in edge-conflict with
    . This contradicts the peace invariant for embedding  onto 
    and thus concludes the proof.
  \end{proof}

  \case{1}  is a central-star. Since  we have
   and hence this does not change the blue
  vertex at . Use the red-star embedding to embed  onto
   and the children of  onto the rightmost 
  non-neighbors of  in . If  is a star now, then
  it was also a star before the red-star embedding (which may
  have modified ), and we can find an ordered plane packing with
  Lemma~\ref{lem:rec_large_blue_star}. Otherwise, recursively embed
   onto . By the placement invariant and since
  , the placement invariant for the recursive
  embedding of  holds. Hence, the embedding of  fails only if
  (1) there is a conflict for embedding  onto . For the
  embedding of , \ref{sgg:ec} holds and so the embedding works unless
  \ref{sgg:dc} fails, i.e. unless (2) .
  We deal with (1)-(2) next.

  \case{1.1} There is a conflict for embedding  onto .  Let
   such that . Then  is a
  central-star rooted at a vertex . By
  Observation~\ref{obs:rec_small_red_star_sc}, the conflict for
  embedding  onto  is a degree-conflict. In other words,
  . Consequently,
  . Additionally, 
  since  is not a star and  by
  Lemma~\ref{lem:degcon3}. Revert to the original blue embedding.  See
  \figurename~\ref{fig:small_red_no_ij_cstar_sc_1}. Note that 
  is still a central-star.  We distinguish two cases.

  \begin{figure}[b]
    \centering\hfil \subfloat[Case~1.1]{\includegraphics{small_red_no_ij_cstar_sc_1}\label{fig:small_red_no_ij_cstar_sc_1}}\hfil \subfloat[Case~1.1.2.1]{\includegraphics{small_red_no_ij_cstar_sc_2}\label{fig:small_red_no_ij_cstar_sc_2}}\hfil \subfloat[Case~1.1.2.1]{\includegraphics{small_red_no_ij_cstar_sc_3}\label{fig:small_red_no_ij_cstar_sc_3}}\hfil \label{fig:small_red_no_ij_1}
    \caption{The case analysis in the proof of
      Proposition~\ref{prop:rec_small_red_star_ij_not_used} (Part~1/3).}
  \end{figure}

  \case{1.1.1}  is not a central-star. Then in particular
  . Since  and
   we get by Lemma~\ref{lem:degr} that
  . If  is rooted at  then
   by 1SR and we can flip  to put the
  root (and center) at .  is now a small blue star.
  Flip  to put its root at the left and embed  onto
   with Lemma~\ref{lem:rec_small_blue_star}. This works because
   is not in edge-conflict with  and  is not a
  central-star.

  \case{1.1.2}  is a central-star. Flip  if
  necessary to put its root (and center) at . If
   then use Lemma~\ref{lem:rec_small_blue_star} to
  find an ordered plane packing. Otherwise . We
  distinguish two cases.

  \case{1.1.2.1}  is a central-star. Let  such that
   and note that . If necessary, flip
   to put its root at . By the peace invariant,  is not in
  edge-conflict with . Since  is in degree-conflict with  for
  embedding  onto  we have .

  In our first attempt at embedding , we embedded  from  using
  a red-star embedding and tried to embed  onto .  Since
  , the red-star embedding moved all (possibly zero) children
  of  in  to a suffix of . Since there was a
  degree-conflict for the embedding of  onto , it follows
  that . Let  such that
  . See
  \figurename~\ref{fig:small_red_no_ij_cstar_sc_2}.

  We know  by the peace invariant. It
  follows that .  Combining this
  with the degree-conflict at , we obtain
  .
  Hence, \ref{gg:dc} is satisfied and we can perform a blue-star
  embedding to embed  onto  (which will not embed any vertex
  onto ). Before doing so, modify the blue embedding by
  simultaneously shifting  to  (redrawing the
  edges to  with biarcs) and  to . See
  \figurename~\ref{fig:small_red_no_ij_cstar_sc_3}. Since
  , the blue-star embedding will embed a
  vertex on every child of . Complete the embedding by
  placing  at  and the children of  onto the remainder.

  \case{1.1.2.2}  is not a central-star. Let  such that
  . Since  is a central-star, by 1SR 
  must be rooted at  and  must be rooted at . See
  \figurename~\ref{fig:small_red_no_ij_cstar_sc_4}.

  We claim that . Towards a contradiction, suppose that
  . Recall that .
  Since  is a smallest subtree of  in , we have
  . Hence, . By LSFR,  and hence
  . Since
   and , we obtain
  , a
  contradiction. The claim follows.

  Since  is a central-star rooted at , by LSFR
   is a star centered at  and rooted at . If
   then we can use Lemma~\ref{lem:rec_large_blue_star} to find
  an ordered plane packing. Otherwise, .

  \begin{figure}[thbp]
    \centering \subfloat[Case~1.1.2.2\label{fig:small_red_no_ij_cstar_sc_4}]{\includegraphics{small_red_no_ij_cstar_sc_4}}\hfil \subfloat[Case~1.1.2.2\label{fig:small_red_no_ij_cstar_sc_5}]{\includegraphics{small_red_no_ij_cstar_sc_5}}\hfil \subfloat[Case~1.2]{\includegraphics{small_red_no_ij_cstar_dc}\label{fig:small_red_no_ij_cstar_dc}}\hfil \caption{The case analysis in the proof of
      Proposition~\ref{prop:rec_small_red_star_ij_not_used} (Part~2/3).}
  \end{figure}

  Since there was a conflict for the original embedding, the
  red-star embedding of  from  embeds a child of  onto
  all blue vertices originally at . Flip  to put its root
  at  and center at . See
  \figurename~\ref{fig:small_red_no_ij_cstar_sc_5}. Execute the
  red-star embedding of  from  again. This embeds a child
  of  onto the center of  at  and hence the remaining
  vertices of  form an independent set. Consider the
  now-modified blue embedding at . The leftmost vertex of
   is the original root of  and may be in
  edge-conflict with . The suffix of  of size
   is formed by blue vertices adjacent to  (which
  is where we embedded ) that were placed there by the
  red-star embedding of  from . All of these blue vertices
  are in edge-conflict with . However, by the original
  degree-conflict, we know that  and hence we can
  find an explicit embedding of  onto  that avoids placing
  the root at  or at the suffix of size . This uses that all
  subtrees of  in  have size at least .

  \case{1.2} . Then
   and hence
   has a leaf. Let  such that . Then
   and so . If  is a star, then we flip
   if necessary to put its center at  and use
  Lemma~\ref{lem:rec_small_blue_star} to find an ordered plane packing.
  Otherwise,  is not a star. We claim that then . Indeed,
  if , then  and so
  , a contradiction.
  The claim follows. Flip  to put the root on the left. This
  places a leaf at . Embed  onto  and the children of  onto
  . Embed  recursively onto . See
  \figurename~\ref{fig:small_red_no_ij_cstar_dc}. The placement
  invariant holds since  and  is the only vertex incident to 
  (which is where we embedded ).  By LSFR and since  is not a
  star,  is not a central-star. Hence the peace
  invariant holds and we can complete the packing.

  \case{2}  is a dangling star. Then it is rooted at the child  of
  . Let . We will embed  similarly to Case~1. Let 
  such that . We distinguish two cases.

  \case{2.1} Suppose that  is not a central-star. Then in
  particular . Let  be the rightmost neighbor of 
  in . If , then embed  onto ,  onto
  , and the children of  onto . See
  \figurename~\ref{fig:small_red_no_ij_dstar_bj_no_star}. Otherwise,
  embed  onto ,  onto , and embed a child of  onto
  every blue vertex of . Use the red-star embedding
  to embed the remaining vertices onto the rightmost
   non-neighbors of  of . In either
  case, embed  recursively onto . The embedding of 
  works unless (1) there is a conflict for embedding  onto
  . The embedding of  works unless~\ref{sgg:dc} fails,
  i.e. unless (2) .

  \begin{figure}[b]
    \centering\hfil \subfloat[Case~2.1]{\includegraphics{small_red_no_ij_dstar_bj_no_star}\label{fig:small_red_no_ij_dstar_bj_no_star}}\hfil \subfloat[Case~2.2]{\includegraphics{small_red_no_ij_dstar_bj_star_1}\label{fig:small_red_no_ij_dstar_bj_star_1}}\hfil \subfloat[Case~2.2.2]{\includegraphics{small_red_no_ij_dstar_bj_star_2}\label{fig:small_red_no_ij_dstar_bj_star_2}}\hfil \subfloat[Case~2.2.2]{\includegraphics{small_red_no_ij_dstar_bj_star_3}\label{fig:small_red_no_ij_dstar_bj_star_3}}\hfil \label{fig:small_red_no_ij_2}
    \caption{The case analysis in the proof of
      Proposition~\ref{prop:rec_small_red_star_ij_not_used} (Part~3/3).}
  \end{figure}


  \case{2.1.1} There is a conflict for embedding  onto . Let
   such that . Then  is a
  central-star. By Observation~\ref{obs:rec_small_red_star_sc}, the
  conflict for embedding  onto  is a degree-conflict, and
  hence . Following the reasoning in
  Case~1.1.1, we see that  and hence  is
  a small blue star after flipping  if necessary. Flip 
  to put its root at  and use Lemma~\ref{lem:rec_small_blue_star} to
  embed  onto . This works because  is not in edge-conflict
  with  and  is not a central-star.

  \case{2.1.2} . This case is similar to
  Case~1.2. Since  we have
  . Then
  . Since
   is not a central-star, we get  as in Case~1.2. Let
   be the number of leaf children of . Then
  . Since
   it follows that 
  and hence . Flip  to put its root at .  Since
   now has  leaf children in , in particular 
  and  are leaves. Embed  onto ,  onto , and the
  children of  onto . Embed  recursively onto
  . Since  is not a star by assumption and by LSFR,
   is not a central-star on at least two vertices.
  Hence the peace invariant holds.

  \case{2.2} Suppose that  is a central-star. If 
  then  is a star and we can find an ordered plane packing by
  Lemma~\ref{lem:rec_small_blue_star}. Otherwise . Flip
   if necessary to put its root at . Embed  onto
  ,  onto , and a child of  on every vertex in .
  Use the red-star embedding to embed the remaining children of
   onto the rightmost  non-neighbors of  in
  . Embed  recursively onto . See
  \figurename~\ref{fig:small_red_no_ij_dstar_bj_star_1} for the
  situation before the cleanup step of the red-star
  embedding. The embedding of  works unless (1) there is a conflict
  for embedding  onto . The embedding of  works
  unless~\ref{sgg:dc} fails, i.e. unless (2)
  .

  \case{2.2.1} There is a conflict for embedding  onto . Let
   such that . Then  is a
  central-star. By Observation~\ref{obs:rec_small_red_star_sc}, the
  conflict is a degree-conflict. Revert to the original blue embedding
  (before the red-star embedding in Case~2.2) and note that
   is still a central-star. We proceed similarly to Case~1.1.2.

  \case{2.2.1.1}  is a central-star. Let  such that
   and note that . If necessary, flip
   to put its root at . By the peace invariant,  is not in
  edge-conflict with . Since  is in degree-conflict with  for
  embedding  onto  we have .

  We blue-star embed  starting from  with
  . Let us argue that the conditions for the
  blue-star embedding hold. The peace invariant guarantees \ref{gg:ec}
  and . It follows that
  , which is the second inequality
  of \ref{gg:dc}. The first inequality of \ref{gg:dc} holds by the
  degree-conflict condition. \ref{gg:int} holds by construction, making
  \ref{gg:cs} trivial.  Hence, the conditions are satisfied and we can
  perform the blue-star embedding as described.

  Since we attain the first inequality in~\ref{gg:dc} strictly, the
  blue-star embedding does not exhaust all vertices in . Indeed,
  , while the blue-star embedding
  embeds only  vertices on the neighbors of .
  Perform the blue-star embedding of  onto . This leaves an
  interval containing  (since the blue-star-embedding always leaves
  at least one vertex) and at least one locally isolated vertex
  (originating from ). Embed  onto ,  onto this
  locally isolated vertex, and the children of  onto the remainder to
  complete the embedding.

  \case{2.2.1.2}  is not a central-star. We proceed
  similarly to Case~1.1.2.2. Let  such that . The
  exact same argument as in Case~1.1.2.2 shows that  is a star
  rooted at  and centered at . If  then we can use
  Lemma~\ref{lem:rec_large_blue_star} to find an ordered plane packing.
  Otherwise, .

  Since there was a conflict for the original embedding, the red-star
  embedding of (the remainder of)  from  embeds a child of 
  onto all blue vertices originally at . Flip  to put its
  root at  and center at . Embed  onto ,  onto , and
  a child of  onto all vertices in .  Execute the red-star
  embedding of the remainder of  from  onto 
  again. This embeds a child of  onto the center of  and
  hence the remaining vertices form an independent set. Consider the
  now-modified blue embedding at .  The leftmost vertex of
   is the original root of  and may be in
  edge-conflict with . We embedded a child of  onto all neighbors
  of  (which is where we embedded ), and hence there are no
  further edge-conflicts. Hence, we can embed  explicitly onto
  .

  \case{2.2.2} . Let  such that
  . It is possible that  and
  . Analogously to Case~2.1.2 we get
   and that  has at least  leaf
  children. It follows that . Recall that . Flip
   to put its root at . If  and  is now a
  star, use Lemma~\ref{lem:rec_large_blue_star} to find an ordered plane
  packing. Otherwise, flipping  placed a leaf child of  at
  . Embed  onto ,  onto , and the children of 
  onto  and . This works because  and .

  We first try to embed  recursively onto . See
  \figurename~\ref{fig:small_red_no_ij_dstar_bj_star_2}. Since ,
  this works unless  is a central-star, which
  implies that  is a central-star by LSFR. In this scenario we
  already handled the case  and so we may assume
  . Embed  recursively onto . See
  \figurename~\ref{fig:small_red_no_ij_dstar_bj_star_3}. By the
  placement invariant, this works unless there is a conflict for
  embedding  onto .

  So suppose there is a conflict for embedding  onto .
  Since  and , we have
   and hence  is a
  central-star. By the peace invariant, the root of  is not
  in edge-conflict with . Flip  if necessary to put its
  root at . Then  is in degree-conflict with  and hence
  . Adding this inequality to the
  inequality in the assumption (replacing  by  due to our
  flipping), we get
  . Since
  , , and  are all different we have
  . Hence
  . Since
   we get . Since  is
  a smallest subtree of , we have 
  and hence . It follows that
  , which has no solution for
   and . We conclude that there is no
  conflict for embedding  onto , as desired.
\end{proof}

\noindent
Propositions~\ref{prop:rec_small_red_star_ij_not_used} and
\ref{prop:rec_small_red_star_ij_used} together prove the following.
\begin{lemma}
  \label{lem:rec_small_red_star}
  If  is a star, then  and  admit an ordered plane packing
  onto .
\end{lemma}

\noindent
Finally, Lemmata~\ref{lem:rec_general}, \ref{lem:rec_large_blue_star},
\ref{lem:rec_large_red_star}, \ref{lem:rec_small_blue_star}, and
\ref{lem:rec_small_red_star} together prove
Theorem~\ref{thm:main}. \label{proofend}

\bibliographystyle{mh-url}\bibliography{bibliography}

\end{document}
