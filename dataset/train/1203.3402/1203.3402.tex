\documentclass{llncs}
\usepackage{makeidx}  \begin{document}
\frontmatter          \pagestyle{headings}  \addtocmark{Hamiltonian Mechanics} \chapter*{Preface}
This textbook is intended for use by students of physics, physical
chemistry, and theoretical chemistry. The reader is presumed to have a
basic knowledge of atomic and quantum physics at the level provided, for
example, by the first few chapters in our book {\it The Physics of Atoms
and Quanta}. The student of physics will find here material which should
be included in the basic education of every physicist. This book should
furthermore allow students to acquire an appreciation of the breadth and
variety within the field of molecular physics and its future as a
fascinating area of research.

For the student of chemistry, the concepts introduced in this book will
provide a theoretical framework for that entire field of study. With the
help of these concepts, it is at least in principle possible to reduce
the enormous body of empirical chemical knowledge to a few basic
principles: those of quantum mechanics. In addition, modern physical
methods whose fundamentals are introduced here are becoming increasingly
important in chemistry and now represent indispensable tools for the
chemist. As examples, we might mention the structural analysis of
complex organic compounds, spectroscopic investigation of very rapid
reaction processes or, as a practical application, the remote detection
of pollutants in the air.

\vspace{1cm}
\begin{flushright}\noindent
April 1995\hfill Walter Olthoff\\
Program Chair\\
ECOOP'95
\end{flushright}
\chapter*{Organization}
ECOOP'95 is organized by the department of Computer Science, Univeristy
of \AA rhus and AITO (association Internationa pour les Technologie
Object) in cooperation with ACM/SIGPLAN.
\section*{Executive Commitee}
\begin{tabular}{@{}p{5cm}@{}p{7.2cm}@{}}
Conference Chair:&Ole Lehrmann Madsen (\AA rhus University, DK)\\
Program Chair:   &Walter Olthoff (DFKI GmbH, Germany)\\
Organizing Chair:&J\o rgen Lindskov Knudsen (\AA rhus University, DK)\\
Tutorials:&Birger M\o ller-Pedersen\hfil\break
(Norwegian Computing Center, Norway)\\
Workshops:&Eric Jul (University of Kopenhagen, Denmark)\\
Panels:&Boris Magnusson (Lund University, Sweden)\\
Exhibition:&Elmer Sandvad (\AA rhus University, DK)\\
Demonstrations:&Kurt N\o rdmark (\AA rhus University, DK)
\end{tabular}
\section*{Program Commitee}
\begin{tabular}{@{}p{5cm}@{}p{7.2cm}@{}}
Conference Chair:&Ole Lehrmann Madsen (\AA rhus University, DK)\\
Program Chair:   &Walter Olthoff (DFKI GmbH, Germany)\\
Organizing Chair:&J\o rgen Lindskov Knudsen (\AA rhus University, DK)\\
Tutorials:&Birger M\o ller-Pedersen\hfil\break
(Norwegian Computing Center, Norway)\\
Workshops:&Eric Jul (University of Kopenhagen, Denmark)\\
Panels:&Boris Magnusson (Lund University, Sweden)\\
Exhibition:&Elmer Sandvad (\AA rhus University, DK)\\
Demonstrations:&Kurt N\o rdmark (\AA rhus University, DK)
\end{tabular}
\begin{multicols}{3}[\section*{Referees}]
V.~Andreev\\
B\"arwolff\\
E.~Barrelet\\
H.P.~Beck\\
G.~Bernardi\\
E.~Binder\\
P.C.~Bosetti\\
Braunschweig\\
F.W.~B\"usser\\
T.~Carli\\
A.B.~Clegg\\
G.~Cozzika\\
S.~Dagoret\\
Del~Buono\\
P.~Dingus\\
H.~Duhm\\
J.~Ebert\\
S.~Eichenberger\\
R.J.~Ellison\\
Feltesse\\
W.~Flauger\\
A.~Fomenko\\
G.~Franke\\
J.~Garvey\\
M.~Gennis\\
L.~Goerlich\\
P.~Goritchev\\
H.~Greif\\
E.M.~Hanlon\\
R.~Haydar\\
R.C.W.~Henderso\\
P.~Hill\\
H.~Hufnagel\\
A.~Jacholkowska\\
Johannsen\\
S.~Kasarian\\
I.R.~Kenyon\\
C.~Kleinwort\\
T.~K\"ohler\\
S.D.~Kolya\\
P.~Kostka\\
U.~Kr\"uger\\
J.~Kurzh\"ofer\\
M.P.J.~Landon\\
A.~Lebedev\\
Ch.~Ley\\
F.~Linsel\\
H.~Lohmand\\
Martin\\
S.~Masson\\
K.~Meier\\
C.A.~Meyer\\
S.~Mikocki\\
J.V.~Morris\\
B.~Naroska\\
Nguyen\\
U.~Obrock\\
G.D.~Patel\\
Ch.~Pichler\\
S.~Prell\\
F.~Raupach\\
V.~Riech\\
P.~Robmann\\
N.~Sahlmann\\
P.~Schleper\\
Sch\"oning\\
B.~Schwab\\
A.~Semenov\\
G.~Siegmon\\
J.R.~Smith\\
M.~Steenbock\\
U.~Straumann\\
C.~Thiebaux\\
P.~Van~Esch\\
from Yerevan Ph\\
L.R.~West\\
G.-G.~Winter\\
T.P.~Yiou\\
M.~Zimmer\end{multicols}
\section*{Sponsoring Institutions}
Bernauer-Budiman Inc., Reading, Mass.\\
The Hofmann-International Company, San Louis Obispo, Cal.\\
Kramer Industries, Heidelberg, Germany
\tableofcontents
\mainmatter              \title{Hamiltonian Mechanics unter besonderer Ber\"ucksichtigung der
h\"ohreren Lehranstalten}
\titlerunning{Hamiltonian Mechanics}  \author{Ivar Ekeland\inst{1} \and Roger Temam\inst{2}
Jeffrey Dean \and David Grove \and Craig Chambers \and Kim~B.~Bruce \and
Elsa Bertino}
\authorrunning{Ivar Ekeland et al.} \tocauthor{Ivar Ekeland, Roger Temam, Jeffrey Dean, David Grove,
Craig Chambers, Kim B. Bruce, and Elisa Bertino}
\institute{Princeton University, Princeton NJ 08544, USA,\\
\email{I.Ekeland@princeton.edu},\\ WWW home page:
\texttt{http://users/\homedir iekeland/web/welcome.html}
\and
Universit\'{e} de Paris-Sud,
Laboratoire d'Analyse Num\'{e}rique, B\^{a}timent 425,\\
F-91405 Orsay Cedex, France}

\maketitle              
\begin{abstract}
The abstract should summarize the contents of the paper
using at least 70 and at most 150 words. It will be set in 9-point
font size and be inset 1.0 cm from the right and left margins.
There will be two blank lines before and after the Abstract. \dots
\keywords{computational geometry, graph theory, Hamilton cycles}
\end{abstract}
\section{Fixed-Period Problems: The Sublinear Case}
With this chapter, the preliminaries are over, and we begin the search
for periodic solutions to Hamiltonian systems. All this will be done in
the convex case; that is, we shall study the boundary-value problem

with  a convex function of , going to  when
.

\subsection{Autonomous Systems}
In this section, we will consider the case when the Hamiltonian 
is autonomous. For the sake of simplicity, we shall also assume that it
is .

We shall first consider the question of nontriviality, within the
general framework of
-subquadratic Hamiltonians. In
the second subsection, we shall look into the special case when  is
-subquadratic,
and we shall try to derive additional information.
\subsubsection{The General Case: Nontriviality.}
We assume that  is
-sub\-qua\-dra\-tic at infinity,
for some constant symmetric matrices  and ,
with  positive definite. Set:


Theorem~\ref{ghou:pre} tells us that if , the
boundary-value problem:

has at least one solution
, which is found by minimizing the dual
action functional:

on the range of , which is a subspace 
with finite codimension. Here

is a convex function, and


\begin{proposition}
Assume  and . Set:


If ,
the solution  is non-zero:

\end{proposition}
\begin{proof}
Condition (\ref{eq:one}) means that, for every
, there is some  such that


It is an exercise in convex analysis, into which we shall not go, to
show that this implies that there is an  such that


\begin{figure}
\vspace{2.5cm}
\caption{This is the caption of the figure displaying a white eagle and
a white horse on a snow field}
\end{figure}

Since  is a smooth function, we will have

for  small enough, and inequality (\ref{eq:two}) will hold,
yielding thereby:


If we choose  close enough to , the quantity

will be negative, and we end up with


On the other hand, we check directly that . This shows
that 0 cannot be a minimizer of , not even a local one.
So  and
. \qed
\end{proof}
\begin{corollary}
Assume  is  and
-subquadratic at infinity. Let
  be the
equilibria, that is, the solutions of .
Denote by 
the smallest eigenvalue of , and set:

If:

then minimization of  yields a non-constant -periodic solution
.
\end{corollary}

We recall once more that by the integer part  of
, we mean the 
such that . For instance,
if we take , Corollary 2 tells
us that  exists and is
non-constant provided that:


or


\begin{proof}
The spectrum of  is . The
largest negative eigenvalue  is given by
,
where

Hence:


The condition  now becomes:

which is precisely condition (\ref{eq:three}).\qed
\end{proof}

\begin{lemma}
Assume that  is  on  and
that  is non-de\-gen\-er\-ate for any . Then any local
minimizer  of  has minimal period .
\end{lemma}
\begin{proof}
We know that , or
 for some constant , is a -periodic solution of the Hamiltonian system:


There is no loss of generality in taking . So

for all  in some neighbourhood of  in
.

But this index is precisely the index
 of the -periodic
solution  over the interval
, as defined in Sect.~2.6. So


Now if  has a lower period,  say,
we would have, by Corollary 31:


This would contradict (\ref{eq:five}), and thus cannot happen.\qed
\end{proof}
\paragraph{Notes and Comments.}
The results in this section are a
refined version of \cite{clar:eke};
the minimality result of Proposition
14 was the first of its kind.

To understand the nontriviality conditions, such as the one in formula
(\ref{eq:four}), one may think of a one-parameter family
, 
of periodic solutions, ,
with  going away to infinity when ,
which is the period of the linearized system at 0.

\begin{table}
\caption{This is the example table taken out of {\it The
\TeX{}book,} p.\,246}
\begin{center}
\begin{tabular}{r@{\quad}rl}
\hline
\multicolumn{1}{l}{\rule{0pt}{12pt}
                   Year}&\multicolumn{2}{l}{World population}\2pt]
\hline
\end{tabular}
\end{center}
\end{table}
\begin{theorem} [Ghoussoub-Preiss]\label{ghou:pre}
Assume  is
-subquadratic at
infinity for all , and -periodic in 





Assume also that  is , and  is positive definite
everywhere. Then there is a sequence , , of
-periodic solutions of the system

such that, for every , there is some  with:

\qed
\end{theorem}
\begin{example} [{{\rm External forcing}}]
Consider the system:

where the Hamiltonian  is
-subquadratic, and the
forcing term is a distribution on the circle:

where . For instance,

where  is the Dirac mass at  and
 is a
constant, fits the prescription. This means that the system
 is being excited by a
series of identical shocks at interval .
\end{example}
\begin{definition}
Let  and  be symmetric
operators in , depending continuously on
, such that
 for all .

A Borelian function

is called
-{\it subquadratic at infinity}
if there exists a function  such that:





If  and
, with
,
we shall say that  is
-subquadratic
at infinity. As an example, the function
, with
, is -subquadratic at infinity
for every . Similarly, the Hamiltonian

is -subquadratic for every .
Note that, if , it is not convex.
\end{definition}

\paragraph{Notes and Comments.}
The first results on subharmonics were
obtained by Rabinowitz in \cite{rab}, who showed the existence of
infinitely many subharmonics both in the subquadratic and superquadratic
case, with suitable growth conditions on . Again the duality
approach enabled Clarke and Ekeland in \cite{clar:eke:2} to treat the
same problem in the convex-subquadratic case, with growth conditions on
 only.

Recently, Michalek and Tarantello (see \cite{mich:tar} and \cite{tar})
have obtained lower bound on the number of subharmonics of period ,
based on symmetry considerations and on pinching estimates, as in
Sect.~5.2 of this article.

\begin{thebibliography}{5}
\bibitem {clar:eke}
Clarke, F., Ekeland, I.:
Nonlinear oscillations and
boundary-value problems for Hamiltonian systems.
Arch. Rat. Mech. Anal. 78, 315--333 (1982)

\bibitem {clar:eke:2}
Clarke, F., Ekeland, I.:
Solutions p\'{e}riodiques, du
p\'{e}riode donn\'{e}e, des \'{e}quations hamiltoniennes.
Note CRAS Paris 287, 1013--1015 (1978)

\bibitem {mich:tar}
Michalek, R., Tarantello, G.:
Subharmonic solutions with prescribed minimal
period for nonautonomous Hamiltonian systems.
J. Diff. Eq. 72, 28--55 (1988)

\bibitem {tar}
Tarantello, G.:
Subharmonic solutions for Hamiltonian
systems via a  pseudoindex theory.
Annali di Matematica Pura (to appear)

\bibitem {rab}
Rabinowitz, P.:
On subharmonic solutions of a Hamiltonian system.
Comm. Pure Appl. Math. 33, 609--633 (1980)

\end{thebibliography}

\title{Hamiltonian Mechanics2}

\author{Ivar Ekeland\inst{1} \and Roger Temam\inst{2}}

\institute{Princeton University, Princeton NJ 08544, USA
\and
Universit\'{e} de Paris-Sud,
Laboratoire d'Analyse Num\'{e}rique, B\^{a}timent 425,\\
F-91405 Orsay Cedex, France}

\maketitle
\makeatletter
\renewenvironment{thebibliography}[1]
     {\section*{\refname}
      \small
      \list{}{\settowidth\labelwidth{}\leftmargin\parindent
            \itemindent=-\parindent
            \labelsep=\z@
            \if@openbib
              \advance\leftmargin\bibindent
              \itemindent -\bibindent
              \listparindent \itemindent
              \parsep \z@
            \fi
            \usecounter{enumiv}\let\p@enumiv\@empty
            \renewcommand\theenumiv{}}\if@openbib
        \renewcommand\newblock{\par}\else
        \renewcommand\newblock{\hskip .11em \@plus.33em \@minus.07em}\fi
      \sloppy\clubpenalty4000\widowpenalty4000\sfcode`\.=\@m}
     {\def\@noitemerr
       {\@latex@warning{Empty `thebibliography' environment}}\endlist}
      \def\@cite#1{#1}\def\@lbibitem[#1]#2{\item[]\if@filesw
        {\def\protect##1{\string ##1\space}\immediate
      \write\@auxout{\string\bibcite{#2}{#1}}}\fi\ignorespaces}
\makeatother
\begin{abstract}
The abstract should summarize the contents of the paper
using at least 70 and at most 150 words. It will be set in 9-point
font size and be inset 1.0 cm from the right and left margins.
There will be two blank lines before and after the Abstract. \dots
\keywords{graph transformations, convex geometry, lattice computations,
convex polygons, triangulations, discrete geometry}
\end{abstract}
\section{Fixed-Period Problems: The Sublinear Case}
With this chapter, the preliminaries are over, and we begin the search
for periodic solutions to Hamiltonian systems. All this will be done in
the convex case; that is, we shall study the boundary-value problem

with  a convex function of , going to  when
.

\subsection{Autonomous Systems}
In this section, we will consider the case when the Hamiltonian 
is autonomous. For the sake of simplicity, we shall also assume that it
is .

We shall first consider the question of nontriviality, within the
general framework of
-subquadratic Hamiltonians. In
the second subsection, we shall look into the special case when  is
-subquadratic,
and we shall try to derive additional information.
\subsubsection{The General Case: Nontriviality.}
We assume that  is
-sub\-qua\-dra\-tic at infinity,
for some constant symmetric matrices  and ,
with  positive definite. Set:


Theorem 21 tells us that if , the boundary-value
problem:

has at least one solution
, which is found by minimizing the dual
action functional:

on the range of , which is a subspace 
with finite codimension. Here

is a convex function, and


\begin{proposition}
Assume  and . Set:


If ,
the solution  is non-zero:

\end{proposition}
\begin{proof}
Condition (\ref{2eq:one}) means that, for every
, there is some  such that


It is an exercise in convex analysis, into which we shall not go, to
show that this implies that there is an  such that


\begin{figure}
\vspace{2.5cm}
\caption{This is the caption of the figure displaying a white eagle and
a white horse on a snow field}
\end{figure}

Since  is a smooth function, we will have

for  small enough, and inequality (\ref{2eq:two}) will hold,
yielding thereby:


If we choose  close enough to , the quantity

will be negative, and we end up with


On the other hand, we check directly that . This shows
that 0 cannot be a minimizer of , not even a local one.
So  and
. \qed
\end{proof}
\begin{corollary}
Assume  is  and
-subquadratic at infinity. Let
  be the
equilibria, that is, the solutions of .
Denote by 
the smallest eigenvalue of , and set:

If:

then minimization of  yields a non-constant -periodic solution
.
\end{corollary}

We recall once more that by the integer part  of
, we mean the 
such that . For instance,
if we take , Corollary 2 tells
us that  exists and is
non-constant provided that:


or


\begin{proof}
The spectrum of  is . The
largest negative eigenvalue  is given by
,
where

Hence:


The condition  now becomes:

which is precisely condition (\ref{2eq:three}).\qed
\end{proof}

\begin{lemma}
Assume that  is  on  and
that  is non-de\-gen\-er\-ate for any . Then any local
minimizer  of  has minimal period .
\end{lemma}
\begin{proof}
We know that , or
 for some constant , is a -periodic solution of the Hamiltonian system:


There is no loss of generality in taking . So

for all  in some neighbourhood of  in
.

But this index is precisely the index
 of the -periodic
solution  over the interval
, as defined in Sect.~2.6. So


Now if  has a lower period,  say,
we would have, by Corollary 31:


This would contradict (\ref{2eq:five}), and thus cannot happen.\qed
\end{proof}
\paragraph{Notes and Comments.}
The results in this section are a
refined version of \cite{2clar:eke};
the minimality result of Proposition
14 was the first of its kind.

To understand the nontriviality conditions, such as the one in formula
(\ref{2eq:four}), one may think of a one-parameter family
, 
of periodic solutions, ,
with  going away to infinity when ,
which is the period of the linearized system at 0.

\begin{table}
\caption{This is the example table taken out of {\it The
\TeX{}book,} p.\,246}
\begin{center}
\begin{tabular}{r@{\quad}rl}
\hline
\multicolumn{1}{l}{\rule{0pt}{12pt}
                   Year}&\multicolumn{2}{l}{World population}\2pt]
\hline
\end{tabular}
\end{center}
\end{table}
\begin{theorem} [Ghoussoub-Preiss]
Assume  is
-subquadratic at
infinity for all , and -periodic in 





Assume also that  is , and  is positive definite
everywhere. Then there is a sequence , , of
-periodic solutions of the system

such that, for every , there is some  with:

\qed
\end{theorem}
\begin{example} [{{\rm External forcing}}]
Consider the system:

where the Hamiltonian  is
-subquadratic, and the
forcing term is a distribution on the circle:

where . For instance,

where  is the Dirac mass at  and
 is a
constant, fits the prescription. This means that the system
 is being excited by a
series of identical shocks at interval .
\end{example}
\begin{definition}
Let  and  be symmetric
operators in , depending continuously on
, such that
 for all .

A Borelian function

is called
-{\it subquadratic at infinity}
if there exists a function  such that:





If  and
, with
,
we shall say that  is
-subquadratic
at infinity. As an example, the function
, with
, is -subquadratic at infinity
for every . Similarly, the Hamiltonian

is -subquadratic for every .
Note that, if , it is not convex.
\end{definition}

\paragraph{Notes and Comments.}
The first results on subharmonics were
obtained by Rabinowitz in \cite{2rab}, who showed the existence of
infinitely many subharmonics both in the subquadratic and superquadratic
case, with suitable growth conditions on . Again the duality
approach enabled Clarke and Ekeland in \cite{2clar:eke:2} to treat the
same problem in the convex-subquadratic case, with growth conditions on
 only.

Recently, Michalek and Tarantello (see Michalek, R., Tarantello, G.
\cite{2mich:tar} and Tarantello, G. \cite{2tar}) have obtained lower
bound on the number of subharmonics of period , based on symmetry
considerations and on pinching estimates, as in Sect.~5.2 of this
article.

\begin{thebibliography}{}
\bibitem[1980]{2clar:eke}
Clarke, F., Ekeland, I.:
Nonlinear oscillations and
boundary-value problems for Hamiltonian systems.
Arch. Rat. Mech. Anal. 78, 315--333 (1982)

\bibitem[1981]{2clar:eke:2}
Clarke, F., Ekeland, I.:
Solutions p\'{e}riodiques, du
p\'{e}riode donn\'{e}e, des \'{e}quations hamiltoniennes.
Note CRAS Paris 287, 1013--1015 (1978)

\bibitem[1982]{2mich:tar}
Michalek, R., Tarantello, G.:
Subharmonic solutions with prescribed minimal
period for nonautonomous Hamiltonian systems.
J. Diff. Eq. 72, 28--55 (1988)

\bibitem[1983]{2tar}
Tarantello, G.:
Subharmonic solutions for Hamiltonian
systems via a  pseudoindex theory.
Annali di Matematica Pura (to appear)

\bibitem[1985]{2rab}
Rabinowitz, P.:
On subharmonic solutions of a Hamiltonian system.
Comm. Pure Appl. Math. 33, 609--633 (1980)

\end{thebibliography}
\clearpage
\addtocmark[2]{Author Index} \renewcommand{\indexname}{Author Index}
\printindex
\clearpage
\addtocmark[2]{Subject Index} \markboth{Subject Index}{Subject Index}
\renewcommand{\indexname}{Subject Index}
\input{subjidx.ind}
\end{document}
