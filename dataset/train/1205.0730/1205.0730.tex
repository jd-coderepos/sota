\documentclass{amsart}
\usepackage{xspace}
\usepackage[usenames,dvipsnames]{pstricks}
\usepackage{epsfig}
\usepackage{graphicx}
\usepackage[normalem]{ulem}\vfuzz2pt \hfuzz2pt \newtheorem{thm}{Theorem}
\newtheorem{cor}[thm]{Corollary}
\newtheorem{conj}[thm]{Conjecture}
\newtheorem{lem}[thm]{Lemma}
\newtheorem{prop}[thm]{Proposition}
\newtheorem{Clm}{Claim}[thm]
\theoremstyle{definition}
\newtheorem{defn}[thm]{Definition}
\theoremstyle{remark}
\newtheorem{rem}[thm]{Remark}
\newtheorem{Cas}{\bf Case}[thm]
\newenvironment{prf}{{\bf \noindent Proof. } }{\hfill\\}
\newenvironment{PrfClaim}{{\bf Proof. }}{{\hfill\tiny{\\}}}


\renewcommand{\theCas}{\arabic{Cas}}
\renewcommand{\theClm}{\arabic{Clm}}
\newcommand{\norm}[1]{\left\Vert#1\right\Vert}
\newcommand{\abs}[1]{\left\vert#1\right\vert}
\newcommand{\set}[1]{\left\{#1\right\}}
\newcommand{\Real}{\mathbb R}
\newcommand{\eps}{\varepsilon}
\newcommand{\To}{\longrightarrow}
\newcommand{\BX}{\mathbf{B}(X)}
\newcommand{\A}{\mathcal{A}}

\newcommand{\psp}{()-sparse graph }
\newcommand{\psps}{()-sparse graphs }
\newcommand{\gfrees}{-free graphs\xspace}
\newcommand{\gfree}{-free graph\xspace}
\newcommand{\ignore}[1]{}
\newcommand{\Wh}{well-hooped\xspace}
\newcommand{\wh}{well-hooped\xspace}
\newcommand{\ExtB}{Buoy\xspace}
\newcommand{\ExtBs}{Buoys\xspace}
\newcommand{\extB}{buoy\xspace}
\newcommand{\extBs}{buoys\xspace}

\newcommand{\Ajoute}[1]{{{\uwave{#1}}} }\newcommand{\Enleve}[1]{{\xout{#1}} }\begin{document}

\title[]{Reed's conjecture on some special classes of graphs}
\author{J.L. Fouquet, J.M. Vanherpe }
\address{L.I.F.O., Facult\'e des Sciences, B.P. 6759 \\
Universit\'e d'Orl\'eans, 45067 Orl\'eans Cedex 2, FR}
\subjclass{035 C} \keywords{Vertex coloring, Chromatic number, Clique number, Maximum degree}

\begin{abstract}

Reed  conjectured that for any graph , , 
where , , and  respectively denote the chromatic number, the clique number and the maximum degree of . 
In this paper, we verify this conjecture for some special classes of graphs, in particular for subclasses of -free graphs or -free graphs.
\end{abstract}

\maketitle
\begin{center}
\today
\end{center}

\section{Introduction}
We consider here simple and undirected graphs. For terms which are not defined we refer to Bondy and Murty \cite{BonMur08}.


In 1998, Reed proposed the following Conjecture which gives, for any graph , an upper bound of the chromatic number  in terms of the clique number  and the maximum degree .



\begin{conj}[Reed's Conjecture \cite{Ree1998}] \label{Conjecture:Reed1998} For any graph , .
\end{conj}



In \cite{AraKarSub2011}, Aravind et al. considered Conjecture \ref{Conjecture:Reed1998} for some graph classes defined by forbidden configurations.
In particular, when ,  and  respectively denote a chordless path, a chordless cycle and a complete graph on  vertices while {\em Chair},
 {\em House}, {\em Bull}, {\em Dart} and {\em Kite} are the graphs depicted in 
Figure \ref{fig:ChairHouseBullDartKite}, Aravind et al. have shown that Conjecture \ref{Conjecture:Reed1998} holds for~:
\begin{itemize}
 \item ()-free graphs,
 \item ()-free graphs,
 \item ()-free graphs,
 \item ()-free graphs,
 \item ()-free graphs.
\end{itemize}
\begin{figure}[t]
\scalebox{1} {
\begin{pspicture}(0,-0.83640623)(9.785937,0.79640627)
\psdots[dotsize=0.1](0.386875,0.7264063)
\psdots[dotsize=0.1](0.386875,0.22640625)
\psdots[dotsize=0.1](0.386875,-0.27359375)
\psdots[dotsize=0.1](0.886875,0.22640625)
\psdots[dotsize=0.1](0.886875,-0.27359375)
\psline[linewidth=0.03cm](0.886875,0.22640625)(0.886875,-0.27359375)
\psline[linewidth=0.03cm](0.386875,0.22640625)(0.886875,0.22640625)
\psline[linewidth=0.03cm](0.386875,0.7264063)(0.386875,-0.27359375)
\psdots[dotsize=0.1](1.986875,-0.27359375)
\psdots[dotsize=0.1](1.986875,0.22640625)
\psdots[dotsize=0.1](2.586875,-0.27359375)
\psdots[dotsize=0.1](2.586875,0.22640625)
\psdots[dotsize=0.1](2.286875,0.7264063)
\psline[linewidth=0.03cm](2.286875,0.7264063)(1.986875,0.22640625)
\psline[linewidth=0.03cm](2.586875,0.22640625)(2.586875,-0.27359375)
\psline[linewidth=0.03cm](1.986875,-0.27359375)(1.986875,0.22640625)
\psline[linewidth=0.03cm](1.986875,-0.27359375)(2.586875,-0.27359375)
\psline[linewidth=0.03cm](2.586875,0.22640625)(2.286875,0.7264063)
\psline[linewidth=0.03cm](1.986875,0.22640625)(2.586875,0.22640625)
\usefont{T1}{ptm}{b}{n}
\rput(0.57984376,-0.66859376){\small }
\psdots[dotsize=0.1](3.786875,0.22640625)
\psdots[dotsize=0.1](4.086875,-0.27359375)
\psdots[dotsize=0.1](4.386875,0.22640625)
\psdots[dotsize=0.1](3.786875,0.7264063)
\psdots[dotsize=0.1](4.386875,0.7264063)
\psline[linewidth=0.03cm](3.786875,0.7264063)(3.786875,0.22640625)
\psline[linewidth=0.03cm](3.786875,0.22640625)(4.086875,-0.27359375)
\psline[linewidth=0.03cm](4.086875,-0.27359375)(4.386875,0.22640625)
\psline[linewidth=0.03cm](4.386875,0.22640625)(3.786875,0.22640625)
\psline[linewidth=0.03cm](4.386875,0.22640625)(4.386875,0.7264063)
\psdots[dotsize=0.1](5.886875,0.7264063)
\psdots[dotsize=0.1](5.586875,0.42640626)
\psdots[dotsize=0.1](6.186875,0.42640626)
\psdots[dotsize=0.1](5.886875,0.12640625)
\psdots[dotsize=0.1](5.886875,-0.27359375)
\psline[linewidth=0.03cm](5.886875,0.7264063)(6.186875,0.42640626)
\psline[linewidth=0.03cm](6.186875,0.42640626)(5.886875,0.12640625)
\psline[linewidth=0.03cm](5.886875,0.12640625)(5.886875,0.7264063)
\psline[linewidth=0.03cm](5.886875,0.7264063)(5.586875,0.42640626)
\psline[linewidth=0.03cm](5.586875,0.42640626)(5.886875,0.12640625)
\psline[linewidth=0.03cm](5.886875,0.12640625)(5.886875,-0.27359375)
\psdots[dotsize=0.1](7.586875,0.7264063)
\psdots[dotsize=0.1](7.286875,0.42640626)
\psdots[dotsize=0.1](7.886875,0.42640626)
\psdots[dotsize=0.1](7.586875,0.12640625)
\psdots[dotsize=0.1](7.586875,-0.27359375)
\psline[linewidth=0.03cm](7.586875,0.7264063)(7.886875,0.42640626)
\psline[linewidth=0.03cm](7.886875,0.42640626)(7.586875,0.12640625)
\psline[linewidth=0.03cm](7.586875,0.7264063)(7.286875,0.42640626)
\psline[linewidth=0.03cm](7.286875,0.42640626)(7.586875,0.12640625)
\psline[linewidth=0.03cm](7.586875,0.12640625)(7.586875,-0.27359375)
\psline[linewidth=0.03cm](7.286875,0.42640626)(7.886875,0.42640626)
\usefont{T1}{ptm}{b}{n}
\rput(2.3098438,-0.66859376){\small }
\usefont{T1}{ptm}{b}{n}
\rput(3.9798439,-0.66859376){\small }
\usefont{T1}{ptm}{b}{n}
\rput(5.7898436,-0.66859376){\small }
\usefont{T1}{ptm}{b}{n}
\rput(7.469844,-0.66859376){\small }
\psdots[dotsize=0.1](8.686875,0.32640624)
\psdots[dotsize=0.1](9.186875,0.5264062)
\psdots[dotsize=0.1](8.986875,0.02640625)
\psdots[dotsize=0.1](9.686875,0.32640624)
\psdots[dotsize=0.1](9.386875,0.02640625)
\psline[linewidth=0.03cm](9.186875,0.5264062)(9.686875,0.32640624)
\psline[linewidth=0.03cm](9.686875,0.32640624)(9.386875,0.02640625)
\psline[linewidth=0.03cm](9.186875,0.5264062)(8.986875,0.02640625)
\psline[linewidth=0.03cm](9.186875,0.5264062)(9.386875,0.02640625)
\psline[linewidth=0.03cm](9.386875,0.02640625)(8.986875,0.02640625)
\psline[linewidth=0.03cm](8.686875,0.32640624)(9.186875,0.5264062)
\psline[linewidth=0.03cm](8.686875,0.32640624)(8.986875,0.02640625)
\usefont{T1}{ptm}{b}{n}
\rput(9.209844,-0.66859376){\small }
\end{pspicture} 
}
\label{fig:ChairHouseBullDartKite}
\caption{Configurations Chair, House, Bull, Dart,Kite}
\end{figure}

 


This paper proves that Reed's Conjecture holds for some classes of graphs. Our results extend those given in \cite{AraKarSub2011} on subclasses of -free or -free graphs.
\section{Notations and preliminary results}


\subsection{Odd hole expansions}

Given a graph  on  vertices  and a family of graphs , an {\em expansion} of  (or {\em expansion}), denoted  is obtained from  by replacing
each vertex  of  with  for  and joining a vertex  in  to a vertex  of , () if and only if  and  are adjacent in .
The graph ,  is said to be the {\em component} of the expansion associated to .
For an expansion   of some graph , we will assume in the following that the vertices of  are weighted with the chromatic number of their associated component
while an edge of  is weighted with the sum of the weights of its endpoints.


 When  is an odd hole, that is a chordless odd cycle of length at least , we shall say that  is an {\em odd hole expansion}.




Conjecture \ref{Conjecture:Reed1998} was  studied by Rabern \cite{Rab2008}.

\begin{thm}\cite{Rab2008} \label{Theorem:Complement}If  is disconnected then
.
\end{thm}

 Moreover~:



In \cite{AraKarSub2011} Aravind et al observed that the so-called {\em complete} expansion (every component of the expansion induces a complete graph) and {\em independent} expansion 
(every component of the expansion induces a stable) of an odd hole satisfy  Conjecture \ref{Conjecture:Reed1998}.
In \cite{FouVan2011} we have shown:

\begin{thm} \cite{FouVan2011} \label{Theorem:ReedsForBipartiteExpansion}
Any expansion of a bipartite graph satisfies  Conjecture \ref{Conjecture:Reed1998}.
\end{thm}


\begin{thm}\cite{FouVan2011} \label{Theorem:ChromaticNumberExpansionOddHole}
Let  be an expansion of an odd hole  of length  with and such that the edge  has maximum weigth in . For , 
let  be the chromatic number of .
Let  be an index such that


Then
\begin{itemize}
 \item If  then 
 \item else .
\end{itemize}
\end{thm}

\begin{cor}\cite{FouVan2011} \label{Corollary:ReedsConjecturePourSpecialOddHole}
 Conjecture \ref{Conjecture:Reed1998} holds for an odd hole expansion when,
 in the conditions of Theorem \ref{Theorem:ChromaticNumberExpansionOddHole}, we have  for .
\end{cor}

\begin{thm} \cite{FouVan2011} \label{Theorem:C5_Expansion} If  is a -expansion then  satisfies  Conjecture \ref{Conjecture:Reed1998}.
\end{thm}

\subsection{Notations and definitions}
\\ 
Let ,  will denote the set of vertices in  adjacent to at least one vertex in  while  will denote the subgraph of  induced by . 
If  we write  instead of .
A vertex in  is said to be {\em partial} for  if it is adjacent to some (but not all) vertex of . 
As usual, given a graph , ,  and  denote respectively the maximum number of vertices in a clique of , the chromatic number and the maximum degree.
In addition, for a vertex ,  denotes the size of a maximum clique containing , and  is the degree of .



In \cite{FouGiaMaiThu1995}, a {\em buoy} was defined as a special case of -expansion, that is an expansion of the odd hole .
We extend here this notion to odd holes of length at least . We shall say that an induced subgraph of a graph  is an {\em \extB of length , ()} 
whenever we can find a partition of its vertex set into  subsets 
(considered as organized in a cyclic order) such that any two consecutive sets  in the list are joined by every
possible edge, while no edges are allowed between two non consecutive sets, and such that these sets are maximal for these properties. 

Observe that a \extB, as defined above is merely an odd hole expansion 
and that an \extB of length  is precisely as defined in \cite{FouGiaMaiThu1995}. Moreover, a \extB as well as its complement are connected graphs.

A graph  will be said a {\em minimal counter example to Conjecture \ref{Conjecture:Reed1998}} whenever  
and when Conjecture \ref{Conjecture:Reed1998} holds for any subgraph of .


\subsection{Technical lemmas}



\begin{lem}\label{Lemma:No_X_Y_minimum_cexemple}
Let  be a minimal counter example to Conjecture \ref{Conjecture:Reed1998} (if any). Then there are no two disjoint subsets  and   such that  
and .
\end{lem}


\begin{prf}
Let  be the subgraph obtained from   by deleting . Since  satisfies Conjecture \ref{Conjecture:Reed1998} by hypothesis, we have
. We can then color the vertices of  by using the colors appearing in  since .
Since  and , we have \\
, a contradiction.
\end{prf}

\begin{lem}\label{Lemma:Connected_component_expansion_minimum_cexemple}
Let  be an  expansion that is a minimal counter-example to Conjecture \ref{Conjecture:Reed1998} (if any). Then each component  () is connected.
\end{lem}
\begin{prf} Without loss of generality assume that the subgraph induced by  is not connected. Let  and  be two subset of  inducing a connected component and suppose that . We get immediately a contradiction with Lemma \ref{Lemma:No_X_Y_minimum_cexemple} since it can be easily checked that .
\end{prf}


\begin{lem}\label{Lemma:ReedPourSousGrapheQuiAtteintLeChromaticNumber}
 Let  be an induced subgraph of some graph  such that  .
If  then .
\end{lem}

In \cite{AraKarSub2011} Aravind et al consider -critical graphs in order to prove that every vertex in a minimum counter example to Conjecture \ref{Conjecture:Reed1998} belongs to an odd hole.

A graph  is said to be {\em k-critical} if  and  for all .

\begin{thm}\cite{AraKarSub2011}\label{thm:Odd_Hole_k-critical}
If  is k-critical and , for , then  must belong to some odd hole in .
\end{thm}

We can extend the result of \cite{AraKarSub2011} to minimal counter examples to Conjecture \ref{Conjecture:Reed1998}.
\begin{lem} \label{Lemme:Hole} 
If  is a minimal counter example to Conjecture \ref{Conjecture:Reed1998} then any vertex is contained in an odd hole.
\end{lem}
\begin{prf}
 Since  is a minimal counter example to Conjecture \ref{Conjecture:Reed1998}, it follows that  is k-critical. Then, for every , , 
and hence by Theorem \ref{thm:Odd_Hole_k-critical},  is part of some odd hole in .
\end{prf}


\section{On \Wh graphs.}


A hole in a graph  will be said {\em \Wh}, if any vertex of  which is partial to  is connected to precisely three consecutive vertices of  or to  precisely two vertices at distance two on . The graph  itself 
will be said {\em \Wh} when all odd holes of  are \Wh.

Observe that the vertices of a \wh cycle  together the vertices which are partial to  induce a \extB which, by construction, is not distinguished from the outside.



Lemma \ref{lem:P5etLeurComplémentsSontStroumpfs} below comes from a result already stated in \cite{FouGiaMaiThu1995}.
\begin{lem}\label{lem:P5etLeurComplémentsSontStroumpfs}
If  is a ()-free graph then  is \wh.
\end{lem}
\begin{prf}
Let  be some odd hole in  and  be a vertex partial to . Since  is -free,  has length .

The neighbours of  in  are either two independant vertices or three consecutive vertices, otherwise the vertices of  together with  would contain an induced  or , a contradiction.
\end{prf}

\begin{thm} \label{Theorem:Structure_Extended_bouées} 
Let  be a \wh graph. Any two distinct \extBs are vertex disjoint or one is contained in the other.
\end{thm}
\begin{prf}
Let  and  be two distinct \extBs of  such that \\
,  and . 

Observe that the vertices of  
as well as the vertices of  are not partial with respect to .

There is a vertex, say , in  that is connected to some vertex of , otherwise  would be disconnected, a contradiction.
By the definition of a\Enleve{n} \extB,  must be adjacent to all vertices of .
Consequently, there must be a vertex in , say , which is adjacent to .

Let  be a vertex not connected to some vertex of  , then  cannot be connected to  since . 
But now,  is connected to  and not to , another contradiction.

Consequently, all vertices in  are adjacent to all vertices in , in other words  is not connected, a final contradiction.
\end{prf}

By Lemma \ref{Lemme:Hole}, every vertex in a mimimal counter example to Conjecture \ref{Conjecture:Reed1998} belongs to an odd hole, consequently :
\begin{cor}\label{cor:PartitionStroumpfGraphEnExtendedBuoys}
 Let  be a \wh graph  which is a minimal counter example to Conjecture \ref{Conjecture:Reed1998}. There is a partition of the vertices of  in \extBs.
\end{cor}


In \cite{FouGiaMaiThu1995} the following theorem was proved for the -free graphs. This result can be easily extended to \wh graphs. We give here the proof for sake of completeness.
\begin{thm}\label{Theorem:Transversal_C5} Let  be a \wh graph. If  is a minimum transversal of the odd cycles of  then .
\end{thm}
\begin{prf}
 For every vertex  of , there exists an odd hole, denoted , such that . Since  is a minimal transversal of the odd holes of , we call  the private odd hole of .

We have . Assume that  and let  be a maximum clique of .

Let  be a vertex of  such that the \extB which contains , say  is minimal among all \extBs generated by private odd holes of vertices of , that is  
does not contain as a proper subset
any other  with .

Assume that  has length  (). We write  since  is an odd hole expansion of length  and we suppose that .

If  meets neither  nor  then . Let  be a vertex of ,  is a clique of , a contradiction.

We suppose now, without loss of generality, that  meets . Let . By minimality of   and by Theorem \ref{Theorem:Structure_Extended_bouées}, . 
Moreover, by the definition of 
a \extB, we have .

We have  since every odd hole obtained from  by substituting another vertex of  to  must intersect . But  must instersect  and , a contradiction.
\end{prf}


Using the Strong Perfect Graph Theorem \cite{ChuRobSeyTho2006}, this result leads to

 \begin{thm} \label{Theorem:Gfree_ChiBound}
If  is a -free \wh graph then .
 \end{thm}
\begin{prf}
Since  is ()-free, the odd holes of  have length . If we remove a transversal  of the 's, we obtain a perfect graph . The perfection of  implies that  
and by Theorem \ref{Theorem:Transversal_C5}, .

Applying recursively this observation we get .
\end{prf}





It follows from a result of King \cite{Kin2010} that if  is a minimum counter-example to Conjecture \ref{Conjecture:Reed1998}
then . Hence, if we restrict ourself to \wh graphs which are ()-free,
a minimum counter-example to this conjecture is such that .


An {\em independent buoy} is a buoy such that any set of the associated partition is a stable set.



\begin{thm} \label{Theorem:Independent_Buoy} If  is a ()-free \wh graph where each buoy of  is independent then  satisfies Conjecture \ref{Conjecture:Reed1998}.
\end{thm}
\begin{prf}
By Corollary \ref{cor:PartitionStroumpfGraphEnExtendedBuoys}, there is a partition of the vertex set into \extBs.

Let  be a minimum transversal of the odd holes.
We get immediately .
Moreover, since   and  does not contain any odd hole nor the complement of an odd hole, these graphs are perfect (\cite{ChuRobSeyTho2006}).



Let  be the simple graph obtained from  by shrinking each \extB of the partition of  and deleting multiple edges.
It is an easy task to see that . Hence we have .


Let  be a vertex contained in a maximum clique of . Then  and .

We have thus
 as soon as , a contradiction.


\end{prf}

An {\em full \extB} is a \extB such that any set of the associated partition is a clique. We have immediately by Corollary \ref{Corollary:ReedsConjecturePourSpecialOddHole} 
that a full \extB satisfies Conjecture \ref{Conjecture:Reed1998}.




\begin{thm} \label{Theorem:Full_Buoy} If  is a -free \wh graph where each \extB is full then .
\end{thm}

\begin{prf}
Since the buoys of  are full, a buoy cannot be contained into another, thus  by Theorem \ref{Theorem:Structure_Extended_bouées} the buoys of  are pairwise disjoint. 
Let  be the set of buoys of . Assume that the buoy , , has length , we write . 
Without loss of generality we can consider that  is a maximum clique of  and set  . Hence, we certainly have  or .
For , let  be a set of minimum size and let .

Since  and  do not contain any odd hole nor its complement ( is -free), theses graphs are perfect
and . Without loss of generality we can write a maximum clique of  as the set  for some . By Theorem \ref{Theorem:Structure_Extended_bouées} 
this maximum clique of  leads to a clique of  which is . Hence .

That is .


\end{prf}


\section{Applications}

We do not know in general whether a \wh graph  satisfies Conjecture \ref{Conjecture:Reed1998}. 
We are concerned here with various families of \wh graphs.






\begin{thm} \label{Theorem:general} If  is a -free \wh graph then  satisfies Conjecture \ref{Conjecture:Reed1998} 
or  contains a subgraph isomorphic to a  and a subgraph isomorphic to .
\end{thm}
\begin{prf}



Suppose that  is  a -free \wh graph being a minimal counter example to Conjecture \ref{Conjecture:Reed1998}.
We can consider that  is connected. Since the graph is -free, the odd holes of  have length . 

By Corollary \ref{cor:PartitionStroumpfGraphEnExtendedBuoys}, there is a partition of the vertex set of  into \extBs. 

Let  be the graph obtained from  by shrinking each buoy of the above partition in a single vertex. Observe that  is -free.

If  has only one vertex then  is a  expansion and the result follows from Theorem \ref{Theorem:C5_Expansion}.

Assume that  contains an induced path on four vertices . Since each \extB of  contains an induced , this  leads to a subgraph isomorphic to the expansion
 as a subgraph of . 

If  is -free and contains at least two vertices, it is well known (see Seinsche \cite{Sei1974}) that its complement
is not connected.  Henceforth,  itself is not connected and  satisfies Conjecture \ref{Conjecture:Reed1998} by Theorem \ref{Theorem:Complement}.

Moreover, by Theorem \ref{Theorem:ReedsForBipartiteExpansion} we can suppose that  is not bipartite. Consequently  contains a triangle, that means that  contains a subgraph isomorphic to .
\end{prf}



Theorem \ref{Theorem:general} above implies that any -free \wh graph   not containing some fixed subgraph of the expansion 
 nor some subgraph of the expansion of  satisfies Conjecture \ref{Conjecture:Reed1998}.

For example, Conjecture \ref{Conjecture:Reed1998} holds for -free \wh graphs of  with no induced , since  is a subgraph of .

Moreover, by this way we get shorter proofs of results given in \cite{AraKarSub2011}.

\begin{cor} \cite{AraKarSub2011} Any -free graph satisfies Conjecture \ref{Conjecture:Reed1998}.
\end{cor}
\begin{prf} Let  be a ()-free graph. Since  is -free, the odd holes of  have length . It is not difficult to check that a vertex partial to some odd hole of , say , 
is precisely connected to  consecutive vertices of .
 By definition, a -free graph is  a 
-free \wh graph. Since a   contains a , the result follows from Theorem \ref{Theorem:general}
\end{prf}

\begin{cor}  Any -free graph satisfies Conjecture \ref{Conjecture:Reed1998}
\end{cor}
\begin{prf} By Lemma \ref{lem:P5etLeurComplémentsSontStroumpfs}, a  -free graph is \wh. 
Moreover, it is obviously a -free graph. Since a   contains a , the result follows from Theorem \ref{Theorem:general}.
\end{prf}

\begin{cor} \cite{AraKarSub2011} Any -free graph satisfies Conjecture \ref{Conjecture:Reed1998}
\end{cor}



\begin{cor}  Any -free graph satisfies Conjecture \ref{Conjecture:Reed1998}
\end{cor}

\begin{prf} 
Let  be a ()-free graph. Since  is -free, the odd holes of  have length . It is not difficult to check that a vertex partial to some odd hole of , say , 
is precisely connected to  vertices at distance  on .
By definition  is  \wh. Moreover  is -free. Since a   contains a , 
the result follows from Theorem \ref{Theorem:general}
\end{prf}

\begin{cor}  \cite{AraKarSub2011} Any -free graph satisfies Conjecture \ref{Conjecture:Reed1998}
\end{cor}

\subsection{-free graphs}
\\



\begin{lem}\label{lemma:Chair-bull-free}
Let  be a ()-free graph  and  () be an odd hole of . Let  be a vertex of  partial to .\\
One of the following holds~:
\begin{enumerate}
\item  is adjacent to precisely  consecutive vertices of ,
\item   and  is  adjacent to precisely four vertices of .
\end{enumerate}
\end{lem}
\begin{prf}
Let us write . Without loss of generality we can assume that  is adjacent to  and not adjacent to .

The vertex  must have at least one neighbour in  otherwise the set  would induce a , a contradiction.
If  is connected to  and not to , the set  would induce a , a contradiction.

If  is connected to  but not to , the vertex  must be adjacent  or the vertices , , ,  and  would induce a , a contradiction. 
Consequently, , otherwise the vertices  and  are distinct and independent and  induces a  when  is adjacent to  and a  otherwise. 
But now the vertex  together with , ,  and  would induce a , a contradiction.

It follows that  is adjacent to  and .

If  has another neighbour on , say , we have , otherwise the vertices , , , ,  iduce a , a contradiction. Once again, we have , or the vertices  
 and  being distinct and independent, the set  would contain an induced  when  and  are not adjacent and an induced  otherwise, a contradiction.

Hence,  and  is adjacent to precisely four vertices of the cycle .
\end{prf}


Let us denote  the following set of graphs  (see Figure~\ref{fig:ChairHouseBullDartKite}).

 \begin{thm} \label{Theorem:Bull_House_Chair}  If  is a -free graph with , then  satisfies Conjecture \ref{Conjecture:Reed1998}.
\end{thm}
\begin{prf}
Let  be a ()-free graph. Assume that  is a minimal counter example to Conjecture \ref{Conjecture:Reed1998}.
We can consider that  is connected. 
Since  is -free, by Lemma \ref{lemma:Chair-bull-free},  is a \wh graph. Since  is -free, it is not difficult to check that the \extBs are full.
By Corollary \ref{cor:PartitionStroumpfGraphEnExtendedBuoys} there is a partition of the vertex set of  into \extBs.

Let  be the graph obtained from  by shrinking each \extB of the above partition in a single vertex. Observe that  is  odd hole free.

If  has only one vertex then  itself is a full odd hole expansion. By Corollary \ref{Corollary:ReedsConjecturePourSpecialOddHole}, Conjecture \ref{Conjecture:Reed1998} holds for .

In addition,  is -free. As a matter of fact, since each vertex of  represents an odd hole, such a  in  would represent a subgraph of  which is not -free, a contradiction.

Consequently, if  contains at least two vertices, it is well known (see Seinsche \cite{Sei1974}) that its complement
is not connected.  Henceforth,  itself is not connected and  satisfies Conjecture \ref{Conjecture:Reed1998} by Theorem \ref{Theorem:Complement}.
\end{prf}

\begin{cor}  \cite{AraKarSub2011} Any -free graph satisfies Conjecture \ref{Conjecture:Reed1998}
\end{cor}



\bibliographystyle{plain}
\bibliography{BibliographieReed}


\end{document}
