\documentclass{llncs}
\usepackage{amssymb,amsmath}
\usepackage{graphicx}

\usepackage{multirow}

\title{Algorithms parameterized by vertex cover and modular width, through potential maximal cliques}
\author{Fedor V. Fomin\inst{1} \and Mathieu Liedloff\inst{2} \and Pedro Montealegre\inst{2} \and Ioan Todinca\inst{2}}

\institute{
Department of Informatics, University of Bergen, N-5020 Bergen, Norway,\\ \texttt{fedor.fomin@ii.uib.no}
\and
Univ. Orl\'{e}ans, INSA Centre Val de Loire, LIFO EA 4022, BP 6759, F-45067 Orl{\'e}ans Cedex 2, France,\\ \texttt{(mathieu.liedloff  ioan.todinca  pedro.montealegre)@univ-orleans.fr}
}

\date{\today}
\pagestyle{plain}

\newcommand{\cP}{\mathcal{P}}
\newcommand{\cF}{\mathcal{F}}
\newcommand{\cO}{\mathcal{O}}
\newcommand{\vc}{\operatorname{vc}}
\newcommand{\tw}{\operatorname{tw}}
\newcommand{\cw}{\operatorname{cw}}
\newcommand{\mw}{\operatorname{mw}}
\newcommand{\poly}{\operatorname{poly}}
\newcommand{\spmc}{\#\operatorname{pmc}}
\newcommand{\sm}{\setminus}
\newcommand{\pmc}{potential maximal clique}
\newcommand{\cmsot}{\operatorname{CMSO}_2}
\newcommand{\cmsoo}{\operatorname{CMSO}_1}
\newcommand{\msot}{\operatorname{MSO}_2}
\newcommand{\msoo}{\operatorname{MSO}_1}
\newcommand{\goldratio}{1.6181}
\newcommand{\pmcb}{1.7347}

\newcommand{\defproblem}[3]{
  \vspace{1mm}
\noindent\fbox{
  \begin{minipage}{0.96\textwidth}
  \begin{tabular*}{\textwidth}{@{\extracolsep{\fill}}lr} #1   \\ \end{tabular*}
  {\bf{Input:}} #2  \\
  {\bf{Task:}} #3
  \end{minipage}
  }
  \vspace{1mm}
}

\newcommand{\mmct}{\textsc{Minimum -Cycle Transversal}}
\newcommand{\mims}{\textsc{Maximum Induced -Cycle-free Subgraph}}
\newcommand{\mifd}{{\sc Maximum Induced -free Subgraph}}
\newcommand{\cmifd}{{\sc Connected Maximum Induced -free Subgraph}}
\newcommand{\afd}{{\sc Minimum Deletion to  at most   copies of }}
\newcommand{\maifd}{{\sc Maximum Induced Subgraph with  copies of }}
\newcommand{\imf}{{\sc Maximum Induced -packing}}
\newcommand{\amims}{\textsc{Maximum Induced Subgraph with  copies of -cycles}}
\newcommand{\lcd}{{\sc Minimum Deletion to  at most  disjoint  -cycles}}
\newcommand{\mfvs}{{\sc Minimum Feedback Vertex Set}}
\newcommand{\mislkc}{\textsc{Maximum Induced Subgraph with  copies of -cycles}}
\newcommand{\mdlk}{{\sc Minimum Deletion to  at most  disjoint  -cycles}}
\newcommand{\mislf}{{\sc Maximum Ind. Subgraph with    copies of Minor Models from }}
\newcommand{\midcls}{{\sc Minimum Induced Disjoint Connected -Subgraphs}}
\newcommand{\mcls}{{\sc Minimum Connected -Subgraphs}}
\newcommand{\kip}{{\sc -in-a-Path}}
\newcommand{\kit}{{\sc -in-a-Tree}}
\newcommand{\kic}{{\sc -in-a-Cycle}}
\newcommand{\kig}{{\sc -in-a-Graph From }}
\newcommand{\idclp}{{\sc  Minimum Induced Disjoint Connected -Subgraphs from }}
\newcommand{\migp}{{\sc  Maximum Induced -Packing}}
\newcommand{\igp}{{\sc  Independent -Packing}}
\newcommand{\mim}{{\sc Maximum Induced Matching}}
\newcommand{\fd}{{\sc Minimum -Deletion}}

\makeatletter
\def\imod#1{\allowbreak\mkern10mu({\operator@font mod}\,\,#1)}
\makeatother

\newcommand{\msphit}{\textsc{Max\- Induced\- Subgraph\- of\- \- satisfiying\- }}

\newcommand{\npmc}{4} 
\newcommand{\Ioan}[1]{\marginpar{\scriptsize \textbf{Comment IT:} #1}}

\newtheorem{observation}{Observation}

\begin{document}
\maketitle

\begin{abstract}
In this paper we give upper bounds on the number of \emph{minimal separators} and \emph{potential maximal cliques of graphs} w.r.t. two graph parameters, namely \emph{vertex cover} () and \emph{modular width} (). We prove that for any graph, the number of minimal separators is  and , and the number of \pmc s is  and , and these objects can be listed within the same running times. (The  notation suppresses polynomial factors in the size of the input.) Combined with known results~\cite{BoTo01,FoToVi14}, we deduce that a large family of problems, e.g., \textsc{Treewidth}, \textsc{Minimum Fill-in}, \textsc{Longest Induced Path}, \textsc{Feedback vertex set} and many others, can be solved in time  or . \end{abstract}

\section{Introduction}

The \emph{vertex cover} of a graph , denoted by , is the minimum number of vertices that cover all edges of the graph.
The \emph{modular width}  can be defined as the maximum degree of a prime node in the modular decomposition of  (see~\cite{TCHP08}  and Section~\ref{se:mw} for definitions).
The main results of this paper are of combinatorial nature: we show that the number of \emph{minimal separators} and the number of \emph{potential maximal cliques} of a graph are upper bounded by a function in each of these parameters. More specifically, we prove the number of minimal separators is at most  and , and the number of potential maximal cliques is  and , and these objects can be listed within the same running time bounds. Recall that the  notation suppresses polynomial factors in the size of the input, i.e.,  should be read as  where  is the number of vertices of the input graph. Minimal separators and potential maximal cliques have been used for solving several classical optimization problems, e.g., \textsc{Treewidth}, \textsc{Minimum Fill-In}~\cite{FKTV08}, \textsc{Longest Induced Path}, \textsc{Feedback Vertex Set} or \textsc{Independent Cycle Packing}~\cite{FoToVi14}. Pipelined with our combinatorial bounds, we obtain a series of algorithmic consequences in the area of FPT algorithms parameterized by the vertex cover and the modular  width of the input graph. In particular, the problems mentioned above can be solved in time  and . These results are complementary in the sense that graphs with small vertex cover are sparse, while graphs with small modular width may be dense.


Vertex cover and modular width are strongly related to treewidth () and cliquewidth () parameters, since for any graph  we have  and . The celebrated theorem of Courcelle'~\cite{Courcelle90} states that all problems expressible in Counting Monadic Second Order Logic () can be solved in time  for some function  depending on the problem. A similar result for cliquewidth~\cite{CMR00} shows that all  problems can be solved in time , if the clique-decomposition is also given as part of the input. (See the Appendix~\ref{ap:logic} for definitions of different types of logic. Informally,  allows logic formulae with quantifiers over vertices, edges, edge sets and vertex sets, and counting modulo constants. The  formulae are more restricted, we are not allowed to quantify over edge sets.) 


Typically function  is a tower of exponentials, and the height of the tower depends on the formula. Moreover Frick and Grohe~\cite{FrGr04} proved that this dependency on treewidth or cliquewidth cannot be significantly improved in general. 
Lampis~\cite{Lampis12} shows that the running time for  problems can be improved  when parametrized by vertex cover, but he also shows that this cannot be improved to  (under the exponential time hypothesis). We are not aware of similar improvements for parameter modular width, but we refer to~\cite{GLO13} for discussions on problems parameterized by modular width. 

Most of our algorithmic applications concern a restricted, though still large subset of  problems, but we guarantee algorithms that are single exponential in the vertex cover:  and in the modular width: . We point out that our result for modular width extends the result of~\cite{FoVi10,FoToVi14}, who show a similar bound of  for the number of \pmc s and for the running times for these problems, but parameterized by the number of vertices of the input graph. 


We use the following  generic problem proposed by~\cite{FoToVi14}, that encompasses many classical optimization problems.
Fix an integer  and a  formula . 
Consider the problem of finding, in the input graph , an induced subgraph  together with a vertex subset , such that the treewidth of  is at most , the graph  together with the vertex subset  satisfy formula , and  is of maximum size under this conditions. This optimization problem is called \msphit:


\vspace{-0.2cm}

\vspace{-0.2cm}

Note that our formula  has a free variable corresponding to the vertex subset . 
 For several examples, in  formula  the vertex set  is actually equal to . E.g., even when  only states that , for  we obtain the \textsc{Maximum Independent set problem}, and for  we obtain the \textsc{Maximum Induced Forest}. If  and   states that  and  is a path we obtain the \textsc{Longest Induced Path} problem. Still under the assumption that , we can express the problem of finding the largest induced subgraph  excluding a fixed planar graph  as a minor, or the largest induced subgraph with no cycles of length . But  can correspond to other parameters, e.g. we can choose the formula  such that  is the number of connected components of . Based on this we can express problems like \textsc{Independent Cycle Packing}, where the goal is to find an induced subgraph with a maximum number of components, and such that each component induces a cycle. 


The result of~\cite{FoToVi14} states that problem \msphit\ can be solved in a running time of the type  where  is the number of potential maximal cliques of the graph, assuming that the set of all potential maximal cliques is also part of the input. 
Thanks to our combinatorial bounds we deduce that the problem \msphit\  can be solved in time  and  , for some small constant .

There are several other graph parameters that can be computed in time  if the input graph is given together with the set of its potential maximal cliques. E.g.,\textsc{Treewidth}, \textsc{Minimum Fill-in}~\cite{FKTV08}, their weighted versions~\cite{BoFo05,Gysel13} and several problems related to phylogeny~\cite{Gysel13},  or  \textsc{Treelength}~\cite{Lokshtanov10}. Pipelined with our main combinatorial result, we deduce that all these problems can be solved in time  or . Recently Chapelle et al.~\cite{CLTV13} provided an algorithm solving \textsc{Treewidth} and \textsc{Pathwidth} in  , but those completely different techniques do not seem to work for \textsc{Minimum Fill-in} or \textsc{Treelength}. The interested reader may also refer., e.g., to~\cite{CLP+14,FLM+08} for more (layout) problems parameterized by vertex cover.



\section{Minimal separators and \pmc s}\label{se:prelim}


Let  be an undirected, simple graph. We denote by  its number of vertices and by  its number of edges. The \emph{neighborhood} of a vertex  is
. We say that a vertex  \emph{sees} a vertex subset  (or vice-versa) if  intersects .
For a vertex set  we denote by  the set . We write  (resp. ) for  (resp. ). 
Also  denotes the subgraph of  induced by , and  is the graph . 




A \emph{connected component} of graph  is the vertex set of a maximal induced connected subgraph of . Consider a vertex subset  of graph . Given two vertices  and , we say that  is a -separator if  and  are in different connected components of . Moreover, if  is inclusion-minimal among all -separators, we say that  is a \emph{minimal -separator}. A vertex subset  is called a \emph{minimal separator} of  if  is a -minimal separator for some pair of vertices  and .

 Let  be a component of . If , we say that  is a \emph{full component} associated to . 

\begin{proposition}[folklore]\label{pr:full}
A vertex subset  of  is a minimal separator if  has at least two full components associated to . Moreover,  is a minimal minimal -separator if and only if  and  are in different full components associated to . 
\end{proposition}



A graph  is \emph{chordal} or \emph{triangulated} if every cycle with four or more vertices has a chord, i.e., an edge between two non-consecutive vertices of the cycle.
A {\em triangulation} of
a graph  is a chordal graph  such that . Graph  is a {\em minimal triangulation} of  if
 for every
edge set  with , the
graph  is not chordal.





A set of vertices  of a graph  is called a
{\em potential maximal clique} if there is a minimal triangulation
 of  such that  is a maximal clique of . 



The following statement due to Bouchitt{\'e} and Todinca~\cite{BoTo01} provides a characterization of potential maximal cliques, and in particular allows to test in polynomial time if a vertex subset  is a potential maximal clique of :


\begin{proposition}[\cite{BoTo01}]\label{pr:pmc_sep}
Let  be a set of vertices of the graph  and
   be the set of 
connected components of . We denote  , where  for all . Then
 is a potential maximal clique of  if and only if
\begin{enumerate}
\item each  is strictly contained in ;
\item the graph on the vertex set  obtained from  by
completing each  into a clique is a
complete graph.
\end{enumerate}
Moreover, if  is a potential maximal clique, then
 is  the set of minimal separators of  contained
in .
\end{proposition}

Another way of stating the second condition is that for any pair of vertices , if they are not adjacent in  then 
there is a component  of  seeing both  and . 

\begin{figure}[h]
\label{fi:cubewaterm}
\begin{center}
\includegraphics[scale=0.35]{fig1_1}
\includegraphics[scale=0.4]{fig1_2}
\end{center}
\caption{Cube graph (left) and watermelon graph (right).}
\end{figure}


To illustrate Proposition~\ref{pr:pmc_sep}, consider, e.g., the cube graph depicted in Figure~\ref{fi:cubewaterm}. The set  is a potential maximal clique and the minimal separators contained in  are  and . Another \pmc\ of the cube graph is  containing the minimal separators , ,  and . 
 


 
Based on Propositions~\ref{pr:full} and~\ref{pr:pmc_sep}, one can easily deduce:
\begin{corollary}[see e.g., \cite{BoTo01}]\label{co:pmc_rec}\label{co:sep_rec}
There is an  time algorithm testing if a given vertex subset  is a minimal separator of , and  time algorithm testing if a given vertex subset  is a potential maximal clique of . 
\end{corollary}

We also need the following observation.
\begin{proposition}[\cite{BoTo01}]\label{pr:pmc_comp}
Let  be a \pmc\ of  and let  be a minimal separator. Then  is contained into a unique component  of , and moreover 
 is a full component associated to .
\end{proposition}


\section{Relations to vertex cover}\label{se:vc}
A vertex subset  is a \emph{vertex cover} of  if each edge has at least one endpoint in . Note that if  is a vertex cover, that  induces an \emph{independent set} in , i.e.  contains no edges. We denote by  the size of a minimum vertex cover of . The parameter ) is called the \emph{vertex cover number} or simply (by a slight abuse of language) the \emph{vertex cover} of .

\begin{proposition}[folklore]\label{pr:vc_FPT}
There is an algorithm computing the vertex cover of the input graph in time .
\end{proposition}

Let us show that any graph  has at most  minimal separators.

\begin{lemma}\label{le:sep3part}
Let  be a graph,  be a vertex cover and  be a minimal separator of . Consider a three-partition  of  such that 
both  and  are formed by a union of components of , and both  and  contain some full component associated to .
Denote  and . 

Then .
\end{lemma}
\begin{proof}
Let  and  be two full components associated to .
Let . Vertex  must have neighbors both in  and , hence both in  and . Since  and  is a vertex cover, we have . Consequently  has neighbors both in   and .

Conversely, let  s.t.  intersects both  and . We prove that . By contradiction, assume that , thus  is in some component  of . Suppose w.l.o.g. that . Since , we must have . Thus  cannot intersect ---a contradiction.
\qed
\end{proof}

\begin{theorem}\label{th:sep_vc}
Any graph  has at most  minimal separators. Moreover the set of its minimal separators can be listed in  time.
\end{theorem}
\begin{proof}
Let  be a minimum size vertex cover of . For each three-partition  of , let . According to Lemma~\ref{le:sep3part}, each minimal separator of  will be generated this way, by an appropriate partition  of . Thus the number of minimal separators is at most , the number of three-partitions of .

These arguments can be easily turned into an enumeration algorithm, we simply need to compute an optimum vertex cover then test, for each set  generated from a three-partition, if  is indeed a minimal separator. The former part takes  time by Proposition~\ref{pr:vc_FPT}, and the latter takes polynomial time for each set  using Corollary~\ref{co:sep_rec}.
\qed
\end{proof}
 
Observe that the bound of Theorem~\ref{th:sep_vc} is tight up to a constant factor. Indeed consider the watermelon graph   formed by  disjoint paths of three vertices plus two vertices  and  adjacent to the left, respectively right ends of the paths (see Figure~\ref{fi:cubewaterm}). Note that this graph has vertex cover  (the minimum vertex cover contains the middle of each path and vertices  and ) and it also has  minimal -separators, obtained by choosing arbitrarily one of the three vertices on each of the  paths. 


 


We now extend Theorem~\ref{th:sep_vc} to a similar result on \pmc s. Let us distinguish a particular family of potential maximal cliques, which have \emph{active} separators. They have a particular structure which makes them easier to handle.

\begin{definition}[\cite{BoTo02}]\label{de:active}
Let  be a potential maximal clique of graph , let 
   be the set of 
connected components of  and let , for .

Consider now the graph  obtained from  by completing into a clique all minimal separators , , such that .

We say that  is an \emph{active separator} for  if   is not a clique in this graph . A pair of vertices  that are not adjacent in  is called an \emph{active pair}. Note that, by Proposition~\ref{pr:pmc_sep}, we must have . 
\end{definition}

The following statement characterizes potential maximal cliques with active separators.

\begin{proposition}\label{pr:pmc_active}
Let  be a potential maximal clique having an active separator , with an active pair . Denote by  the unique component of  containing . Then  is a minimal -separator in the graph . 
\end{proposition}

Again on the cube graph of Figure~\ref{fi:cubewaterm}, for the \pmc\ , both minimal separators are active. E.g., for the minimal separator  the pair  is active. Not all potential maximal cliques have active separators, as illustrated by the potential maximal clique  of the same graph.



Let us first focus on potential maximal cliques having an active separator. 
We give a result similar to Lemma~\ref{le:sep3part}, showing that such a potential maximal clique can be determined by a certain partition of the vertex cover  of .
 
 \begin{lemma}\label{le:pmc4part}
Let  be a graph and  be a vertex cover of . Consider a \pmc\  of  having an active separator  and an active pair .
Let  be the unique connected component of  intersecting  and let  be the union of all other connected components of . 
Denote by   the union of components of  contained in , seeing , by  the union of components of  contained in  not seeing . 



Now let ,  and .  

Then one of the following holds:
\begin{enumerate}
\item\label{it:1} There is a vertex  such that .
\item There is a vertex  such that .
\item A vertex  is in  if and only if
\begin{enumerate}
\item\label{it:SW}  sees  and , or
\item\label{it:TW}  does not see  but is sees  ,  and .
\end{enumerate}
\end{enumerate}
\end{lemma} 
\begin{proof}


Note that  and  form a partition of the vertex set .

We first prove that any vertex  satisfying conditions~\ref{it:SW} or~\ref{it:TW} must be in . 

Consider first the case~\ref{it:SW} when  sees  and . So  sees  and ; we can apply Lemma~\ref{le:sep3part} to partition  
thus .
Consider now the case~\ref{it:TW} when  sees ,  and  but not . Again by Lemma~\ref{le:sep3part} applied to partition , vertex  cannot be in . Since  has a neighbor in , we have . Let  and  (thus we also have ). Recall that  is an -minimal separator in  by Proposition~\ref{pr:pmc_active}. By definition of set , we have that  is exactly the component of  containing . Note that  is the union of the component of   containing  and of all other components of  (that no not see  nor ). By applying Lemma~\ref{le:sep3part} on graph , with vertex cover  and with partition  we deduce that . 

Conversely, let . We must prove that either  satisfies conditions~\ref{it:SW} or~\ref{it:TW}, or we are in one of the first two cases of the Lemma. 
We distinguish the cases  and . When , by Lemma~\ref{le:sep3part} applied to partition ,  must see  and . If  sees some vertex in , we are done because  sees   so we are in case~\ref{it:SW}. Assume now that , we prove that actually , so we are in case~\ref{it:1}. Assume there is . By Proposition~\ref{pr:pmc_sep}, there must be a connected component  of  such that . Since , this component  must be a subset of , so . Together with , this contradicts the assumption . 

It remains to treat the case . Clearly  cannot see  because  separates  from . 
We again take graph , with vertex cover , and apply Lemma~\ref{le:sep3part} with partition . We deduce that  sees both   and
. Assume that  does not see . So  thus . 
If  contains some vertex , no component of  can see both  and  (because ), contradicting Proposition~\ref{pr:pmc_sep}. We conclude that either  sees  (so satisfies condition~\ref{it:TW}) or  (thus we are in the second case of the Lemma).
\qed
\end{proof}

\begin{theorem}\label{th:pmca_vc}
Any graph  has  \pmc s with active separators. Moreover the set of its \pmc s with active separators can be listed in  time.
\end{theorem}
\begin{proof}
The number of \pmc s with active separators satisfying the second condition of Lemma~\ref{le:pmc4part} is at most , and they can all be listed in polynomial time by checking, for each vertex , if  is a \pmc.

For enumerating the \pmc s with active separators satisfying the first condition of Lemma~\ref{le:pmc4part}, we enumerate all minimal separators  using Theorem~\ref{th:sep_vc}, then 
for each  and each of the at most  components  of  we check if  is a \pmc. Recall that testing if a vertex set is a \pmc\ can be done in polynomial 
time by Corollary~\ref{co:pmc_rec}. Thus the whole process takes  time, and this is also an upper bound on the number of listed objects.

It remains to enumerate the \pmc s with active separators satisfying the third condition of Lemma~\ref{le:pmc4part}. For this purpose, we ``guess'' the sets  ,  as in the Lemma and then we compute . More formally, for each four-partition  of , we let  be the set of vertices  satisfying conditions~\ref{it:SW} or~\ref{it:TW} of Lemma~\ref{le:pmc4part}, and we test using Corollary~\ref{co:pmc_rec} if  is indeed a \pmc. By Lemma~\ref{le:pmc4part}, this enumerates in  all \pmc s of this type. 
\qed
\end{proof}


For counting and enumerating all \pmc s of graph , including the ones with no active separators, we apply the same ideas as in~\cite{BoTo02}, based on the following statement.


\begin{proposition}[\cite{BoTo02}]\label{pr:pmc_a}
Let  be a graph, let  be an arbitrary vertex of  and  be a potential maximal clique of . Denote by  the graph . Then one of the following holds.
\begin{enumerate}
\item\label{it:active}  has an active minimal separator .
\item\label{it:witha}  is a potential maximal clique of .
\item\label{it:minusa}  is a potential maximal clique of .
\item\label{it:minsep}  is a minimal separator of .
\end{enumerate}
\end{proposition}

\begin{theorem}\label{th:pmc_vc}
Any graph  has  \pmc s. Moreover the set of its \pmc s can be listed in  time.
\end{theorem}
\begin{proof}
Let  be an arbitrary ordering of the vertices of . Denote by  the graph  induced by the first  
vertices, for all . Let . Note that for all  we have . Actually, if  is a vertex cover of , then  is a vertex 
cover of . In particular, by Theorems~\ref{th:sep_vc} and~\ref{th:pmca_vc}, each  has at most  minimal separators and  \pmc s with active separators.

For , graph  has a unique \pmc\ equal to .

For each  from  to , in increasing order, we compute the \pmc s of  from those of  using Proposition~\ref{pr:pmc_a}. 
Observe that . We initialize the set of \pmc s of  with the ones having active separators. This can be done in  time by Theorem~\ref{th:pmca_vc}.
Then for each minimal separator  of  we check if  is a \pmc\ of  and if so we add it to the set. 
This takes  time by Theorem~\ref{th:sep_vc} and Corollary~\ref{co:pmc_rec}.
Eventually, for each \pmc\  of , we test using Corollary~\ref{co:pmc_rec} if  (resp. ) is a \pmc\ of . If so, we add it to the set of \pmc s of . The running time of this last part is the number of \pmc s of  times . Altogether, it takes  time.

By Proposition~\ref{pr:pmc_a}, this algorithm  covers alls cases and thus lists all \pmc s of . Hence for  we obtain all \pmc s of , and they have been enumerated in  time.
\qed
\end{proof}

\section{Relations to modular width}\label{se:mw}

A \emph{module} of graph  is a set of vertices  such that, for any vertex , either  or  does not intersect . For the reader familiar with the modular decompositions of graphs, the modular width  of a graph  is the maximum size of a prime node in the modular decomposition tree. Equivalently,
graph  is of modular width at most  if:
\begin{enumerate}
\item  has at most one vertex (the base case).
\item  is a disjoint union of graphs of modular width at most .
\item  is a \emph{join} of graphs of modular width at most . I.e.,  is obtained from a family of disjoint graphs of modular width at most  by taking the disjoint union and then adding all possible edges between these graphs.
\item the vertex set of  can be partitioned into  modules  such that  is of modular width at most , for all .
\end{enumerate}
The modular width of a graph can be computed in linear time, using e.g.~\cite{TCHP08}. Moreover, this algorithm outputs the algebraic expression of  corresponding to this grammar. 

Let  be a graph with vertex set  and let  be a family of pairwise disjoint graphs, for all , . Denote by  the graph obtained from  by replacing each vertex  by the module . I.e., . We say that graph  has been obtained from  by \emph{expanding} each vertex  by the module .

A vertex subset  of  is an \emph{expansion} of vertex subset  of  if . Given a vertex subset  of , the \emph{contraction} of  is . 

\begin{lemma}\label{le:sepmod1}
Let  be a minimal -separator of , for . Then  is a minimal separator of  and . 
\end{lemma}
\begin{proof}
Note that all vertices of  are in , by construction of graph  and the fact that  and  are in the same module induced by . Therefore  must be contained in . Let . Since , we have that  separates  and  in graph . Assume that  is not a minimal -separator of , so let  be a minimal -separator in graph . We claim that  is a -separator in . Indeed each -path of  is either contained in  (in which case it intersects ) or intersects . In both cases, it passes through , which proves the claim. Since  is a subset of  and  is a -minimal separator of , the only possibility is that . This proves that  is a minimal separator of 
 and .
\qed
\end{proof}

\begin{lemma}\label{le:sepmod}
Let  be a minimal separator of . Assume that some  intersects , but is not contained in .
Then  intersects all full components of  associated to . In particular  is a minimal separator in  and 
.
\end{lemma}
\begin{proof}
Let  and . 
By Proposition~\ref{pr:full}, there are at least two full  components of , associated to . Let  be one of them, not containing . Let  be a neighbor of  in , we prove that . If , then , and since  is a module in  we also have . This contradicts the fact that  and  are in different components of . It remains that  . By applying the same argument for  instead of , it follows that  intersects each full component  of  and moreover  has a neighbor in . 

By Proposition~\ref{pr:full},  is a minimal -separator in , for some . The rest follows by Lemma~\ref{le:sepmod1}.
\qed
\end{proof}

\begin{lemma}\label{le:sepexp}
Let  be a minimal separator of . One of the following holds~:
\begin{enumerate}
\item  is the expansion of a minimal separator  of . 
\item There is  such that  is a minimal separator of  and .
\end{enumerate}
\end{lemma}
\begin{proof}
Assume there is a set  intersecting  but not contained in it. By Lemma~\ref{le:sepmod},  is a minimal separator of  and . Hence we are in the second case of the Lemma.


Otherwise, for any  intersecting , we have . Thus  is the expansion of a vertex subset  of , formed exactly by the vertices  of  such that  intersects .  Let  and  be two full components of  associated to  and let , . Recall that, by Proposition~\ref{pr:full},  is a minimal -separator of . Let  be the set containing  and  the set containing . Consider first the possibility that . Then, by Lemma~\ref{le:sepmod1},  satisfies the second condition of this lemma, for . (This case may occur when  is disconnected and .)


It remains the case . We prove that  is a minimal -separator of . Consider a  path of . If this path does not intersect  in , then there is a path from  to  in , obtained by replacing each vertex  of the path by some vertex of  ( and  are replaced by  and  respectively). This would contradict the fact that  separates  and  in . Therefore  is indeed a -separator in . Assume that  is not minimal among the -separators of , and let  such that  separates  and  in . We claim that  also separates  from  in . By contradiction, assume there is a path from  to  in , avoiding . By contracting, on this path, all vertices belonging to a same  into vertex , we obtain a path (or a connected subgraph) joining  to  in . This contradicts the fact that all such paths should intersect . Therefore  is a minimal separator of . 
\qed
\end{proof}

Lemma~\ref{le:sepexp} provides an injective mapping from the set of minimal separators of  to the union of the sets of minimal separators of  and of the graphs . Therefore we have:
\begin{corollary}\label{co:sepexp}
The number of minimal separators of  is at most the number of minimal separators of  plus the number of minimal separators of each . 
\end{corollary}


We now aim to prove a statement equivalent of Corollary~\ref{co:sepexp}, for potential maximal cliques instead of minimal separators. 

\begin{lemma}\label{le:pmcexp}
Let  be a \pmc\ of , and let . Assume that  is the expansion of , i.e. . Then  is a \pmc\ of .
\end{lemma} 
\begin{proof}
We prove that  satisfies, in graph , the conditions of Proposition~\ref{pr:pmc_sep}. For the first condition, let  be a component of  and let . Assume that  is not strictly contained in , hence . Let  be the expansion of  in  and note that  is the expansion of , thus . If  is formed by at least two vertices, since  is connected then so is .  Therefore, in graph , we have  and  is a component of . But this contradicts the first condition of Proposition~\ref{pr:pmc_sep} applied to the \pmc\  of . In the case that  is formed by a unique vertex , its expansion  might not induce a connected subset in  (if  is disconnected). But it is sufficient to consider a connected component  of , and again this is also a component of  with the property that its neighborhood in  is the whole set , contradicting Proposition~\ref{pr:pmc_sep} applied to .

For the second condition of Proposition~\ref{pr:pmc_sep}, let  such that  is not an edge of . Let  and . These vertices are non-adjacent in , so by Proposition~\ref{pr:pmc_sep} applied to the \pmc\  of  there must be a component  of  seeing both  and . Consider an -path in . The contraction of this path contains a -path in , whose internal vertices are not in . This proves that  and  are in the neighborhood of a same component of , thus  satisfies the second condition of Proposition~\ref{pr:pmc_sep}.
\qed
\end{proof}

\begin{lemma}\label{le:pmcmod}
Let  be a \pmc\ of , and assume that there is some set  that intersects  but is not contained in . Then  is a \pmc\ of  and .
\end{lemma}
\begin{proof}
Let  be a vertex set that intersects , but is not contained in . 

We claim that  contains a minimal -separator of , for some pair of vertices  and . If  intersects some minimal separator  contained in , then by Lemma~\ref{le:sepmod},  is a minimal separator of  and  intersects all full components of  associated to , which proves our claim. Consider the case when  does not intersect any minimal separator of  contained in . Let  and note that  separates, in graph , vertex  from all other vertices (because by Proposition~\ref{pr:pmc_sep},  has no neighbors in ). Recall that , thus there is some , then  contains some minimal -separator  in graph .


By Lemma~\ref{le:sepmod1},  is a minimal separator of  and  intersects all full components of  associated to . Let  be the unique component of  intersecting ; recall that it exists and moreover it is full w.r.t. , by Proposition~\ref{pr:pmc_comp}. Then, by Lemma~\ref{le:sepmod},  also intersects . Also by Lemma~\ref{le:sepmod},  and  is a minimal separator of . We claim that actually  and  is also a full component of . Recall that  separates in graph  the vertices of  from the rest of the graph. Since  intersects ,  is connected and  separates  from all other vertices, we must have . Since  is connected, so is , thus  is contained in some component of . But each such component is also a component of , hence  is both a component of  and of . In particular .

It remains to prove that   is a \pmc\ of . By the above observations, we also have . We show that  satisfies, in graph , the conditions of Proposition~\ref{pr:pmc_sep}. Let  be a component of . Observe that  is also a component of  and let . Either  is a component of  disjoint from , or it is contained in . In the former case,  is a subset of , hence  is a strict subset of  (since  is itself a strict subset of  by Proposition~\ref{pr:pmc_sep} applied to \pmc\  of ). In the later case, if , note that  because  is also contained in the neighborhood of  in . This contradicts Proposition~\ref{pr:pmc_sep} applied to \pmc\  of .

For the second condition, let , non-adjacent in . Then there is a component  of  seeing, in graph , both  and  (by Proposition~\ref{pr:pmc_sep} applied to ). Since this component sees , it must be contained in . So  is also a component of  seeing both  and  in , which concludes our proof.
\qed
\end{proof}

From Lemmata~\ref{le:pmcexp} and~\ref{le:pmcmod}, we directly deduce~:
\begin{lemma}\label{le:pmcmw}
Let  be a \pmc\ of . One of the following holds~:
\begin{enumerate}
\item  is the expansion of a \pmc\  of . 
\item There is some  such that  is a \pmc\ of  and .
\end{enumerate}
\end{lemma}

The previous lemma provides an injective mapping from the set of \pmc s of  to the union of the sets of \pmc s of  and of the graphs . Therefore we have:
\begin{corollary}\label{co:pmcexp}
The number of \pmc s of  is at most the number of \pmc s of  plus the number of \pmc s of each . 
\end{corollary}

The following proposition bounds the number of minimal separators and \pmc s of arbitrary graphs with respect to .
\begin{proposition}[\cite{FoVi10,FoVi12}]\label{pr:nbsep}\label{pr:nbpmc}
Every -vertex graph has  minimal separators and  \pmc s. Moreover, these objects can be enumerated within the same running times. 
\end{proposition}

We can now prove the main result of this section.
\begin{theorem}\label{th:sepmw}\label{th:pmcmw}
For any graph , the number of its minimal separators is  and the number of its potential maximal cliques is . Moreover, the minimal separators and the potential maximal cliques can be enumerated in time  and  time respectively.
\end{theorem}
\begin{proof}
Let . By definition of modular width, there is a decomposition tree of graph , each node corresponding to a leaf, a disjoint union, a join or a decomposition into at most  modules. The leaves of the decomposition tree are disjoint graphs with at single vertex, thus these vertices form a partition of . In particular, there are at most  leaves and, since each internal node is of degree at least two, there are  nodes in the decomposition tree. For each node , let  be the graph associated to the subtree rooted in . We prove that   has  minimal separators and  \pmc, where  is the number of nodes of the subtree rooted in . We proceed by induction from bottom to top. The statement is clear when  is a leaf.

Let  be an internal node  be its sons in the tree. 
Graph  is the expansion of some graph  by replacing the -th vertex with module . If  is a \emph{join} node, then  is a clique. When  is a \emph{disjoint union} node, graph  is an independent set, and in the last case  is a graph of at most  vertices. In all cases, by Proposition~\ref{pr:nbsep} graph  has   minimal separators. Thus  has at most  more minimal separators than all its sons taken together, which completes our proof for minimal separators. 

Concerning potential maximal cliques, when  is a clique it has exactly one potential maximal clique, and when  is of size at most  is has  \pmc s. We must be more careful in the case when  is an independent set (i.e.,  is a disjoint union node), since in this case it has  \pmc s, one for each vertex, and  can be as large as . Consider a \pmc\  of  corresponding to an expansion of vertices of  (see Lemma~\ref{le:pmcmw}).  It follows that this \pmc\ is exactly the vertex set of some , for a child  of . By construction this vertex set is disconnected from the rest of , and by Proposition~\ref{pr:pmc_sep} the only possibility is that this vertex set induces a clique in . But in this case  is also a \pmc\ of . This proves that, when  is of type disjoint union,  has no more \pmc s than the sum of the numbers of \pmc s of all its sons. We conclude that the whole graph  has   \pmc s. All our arguments are constructive and can be turned directly into enumeration algorithms for these objects.
\qed
\end{proof}



\section{Applications}\label{se:appli}

The \emph{treewidth} of graph , denoted , is the minimum number  such that  has a triangulation  of clique size at most . The \emph{minimum fill in} of  is the minimum size of , over all (minimal) triangulations  of . The \emph{treelength} of  is the minimum  such that there exists a minimal triangulation , with the property that any two vertices adjacent in  are at distance at most  in graph . 

\begin{proposition}
Let  denote the set of \pmc s of graph . The following problems are solvable in  time, when  is given in the input~: \textsc{(Weighted) Treewidth}~{\cite{FKTV08,BoRo03}}, \textsc{(Weighted) Minimum Fill-In}~{\cite{FKTV08,Gysel13}},
\textsc{Treelength}~\cite{Lokshtanov10}.
\end{proposition}


Let us also recall the \msphit\ problem where, for a fixed integer  and a fixed  formula , the goal is to find a pair of vertex subsets  such that ,  models  and  is of maximum size. 

\begin{proposition}[\cite{FoToVi14}]\label{pr:msphit}
For any fixed integer  and any fixed  formula , problem  \msphit\ is solvable in  time, when  is given in the input.
\end{proposition}



Pipelined with Theorems~\ref{th:pmc_vc} and~\ref{th:pmcmw}, we deduce:
\begin{theorem}\label{th:appli}
Problems  \msphit\ \textsc{(Weighted) Treewidth}, \textsc{(Weighted) Minimum Fill-In},  \textsc{Treelength} can be solved in time  and in time .\\
\end{theorem}
We re-emphasis that problem \msphit\ generalizes many classical problems, e.g.,  \textsc{Maximum Independent Set},  \textsc{Maximum Induced Forest},  \textsc{Longest Induced Path}, \textsc{Maximum Induced Matching}, \textsc{Independent Cycle Packing},  \kip,\ \kit, \textsc{Maximum Induced Subgraph With a Forbidden Planar Minor}. More examples of particular cases are given in Appendix~\ref{ap:appli} (see also~\cite{FoToVi14}). 

The polynomial factors hidden by the  notation depend on the problem and on the parameter, they are typically between  to . 





\section{Conclusion}

We have provided single exponential upper bounds for the number of minimal separators and the number of potential maximal cliques of graphs, with respect to parameters vertex cover and modular width. 

A natural question is whether these results can be extended to other natural graph parameters. We point out that for parameters like clique-width or maximum leaf spanning tree, one cannot obtain upper bounds of type  for any function . A counterexample is  provided by the graph , formed by  disjoint paths of   vertices plus two vertices  and  seeing the left, respectively right ends of the paths (similar to the watermelon graph of Figure~\ref{fi:cubewaterm}). Indeed this graph has a maximum leaf spanning tree of  vertices and a clique width of no more than , but it has roughly  minimal -separators. 

Finally, we point out that our bounds on the number of \pmc s w.r.t. vertex cover and to modular width do not seem to be tight. Any improvement on these bounds will immediately provide improved algorithms for the problems mentioned in Section~\ref{se:appli}.
\bibliographystyle{siam}
\bibliography{VC_PMC.bib}
 
\appendix

\section{More applications}\label{ap:appli}
We give in this Appendix several problems that are all known to be particular cases of \msphit (see~\cite{FoToVi14} proofs and more applications). Proposition~\ref{pr:msphit} also extends to the weighted version and the annotated version of the problems (in the annotated version, a fixed vertex subset must be part of the solution ).

  Let     be the set of cycles of length  .
   Let   be an  integer.  Our first example is the following  problem.
   
   \medskip
\defproblem{\amims{}}{A graph .}{Find a  set  of maximum size such that  
    contains at most  vertex-disjoint cycles  from .}
 \medskip 
  
   \amims{} encompasses several interesting problems.
   For example, when , the problem is to find a maximum induced subgraph without cycles divisible by . For  and  this is \textsc{Maximum Induced Forest}. 


   For  integers  and , the problem related to \amims{} is the   following.
   
   \medskip
\defproblem{\mislkc{}}{A graph .}{Find a  set  of maximum size such that  
    contains at most  vertex-disjoint cycles  of length at least .
   }
   
   \medskip 
   
Next example concerns properties described by forbidden minors. 
    Graph  is a \emph{minor} of graph    if  can be obtained from a
subgraph of  by a (possibly empty) sequence of edge contractions.  A \emph{model}  of minor  in  is a minimal subgraph of , where the edge set  is partitioned into \emph{c-edges (contraction edges)} and \emph{m-edges (minor edges)} such that the graph resulting from contracting all c-edges is isomorphic
 to .
Thus,  is isomorphic to a minor of  if and only if there exists a model
of  in . For  an integer  a finite set of graphs , containing a planar graph we define he following generic problem.
 
\medskip





 \defproblem{\mislf{} }{A graph .}{Find a set  
  of maximum size such that   contains at most  vertex disjoint  minor models of  graphs  from
     }


Even the special case with , this problem and its complementary version called the \fd, encompass many different problems.


 



  \medskip

   
   Let  be an integer and  be a CMSO-formula. Let  be a class of connected graphs of treewidth at most  and with property expressible by .
Our next example is  the following problem.






 
 
\defproblem{\igp{}}{A graph .}{Find a  set  with  maximum number of connected components such that  
   each connected component of  is in .}

As natural sub cases studied in the literature we can cite~\textsc{Independent Triangle Packing} or~\textsc{Independent Cycle Packing}.
\medskip

  



The next problem is an example of \emph{annotated version} of optimization problem \msphit.



\defproblem{\kig{}}{A graph ,  with  terminal vertices.}{Find   an induced graph from    containing all  terminal vertices.}

\medskip
Many variants of \kig{} can be found in the literature, like  \kip,\ \kit, \kic.




\section{Monadic Second-Order Logic}\label{ap:logic}


We use Counting Monadic Second Order Logic (), an extension of , as a basic tool to express properties of vertex/edge sets in graphs. 
 \smallskip

The syntax of Monadic Second Order Logic () of graphs includes the logical connectives    
   variables for 
vertices, edges, sets of vertices, and sets of edges, the quantifiers   that can be applied 
to these variables, and the following five binary relations: 
\begin{enumerate}


\item 
 where  is a vertex variable 
and  is a vertex set variable; 
\item 
  where  is an edge variable and  is an edge 
set variable;
\item 
  where  is an edge variable,   is a vertex variable, and the interpretation 
is that the edge  is incident with the vertex ; 
\item 
  where   and  are 
vertex variables  and the interpretation is that  and  are adjacent; 

\item 
 equality of variables representing vertices, edges, sets of vertices, and sets of edges.
\end{enumerate}


The  is a restriction of  in which one cannot use edge set variables (in particular the incidence relation becomes unnecessary). For example \textsc{Hamiltonicity} is expressible in  but not in . 

 In addition to the usual features of monadic second-order logic, if we have atomic sentences testing whether the cardinality of a set is equal 
to  modulo  where  and  are integers such that  and  then 
this extension of the  (resp. ) is called the {\em counting monadic second-order logic}  (resp. ). So essentially  (resp. )
is  (resp. ) with the following atomic sentence for a set : 
\begin{quote}
 if and only if  
\end{quote}


We refer to~\cite{ArnborgLS91,Courcelle90} and the book of Courcelle and Engelfriet~\cite{CoEn12} for a detailed introduction on different types of logic.
 \end{document}
