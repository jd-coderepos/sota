\pdfoutput=1  \documentclass[]{article}  
\usepackage{url,float}
\usepackage{graphicx}
\usepackage{amsfonts}
\usepackage{amssymb}
\usepackage{latexsym}




\newcommand{\hide}[1]{}

\newcommand{\ABox}{
\raisebox{3pt}{\framebox[6pt]{\rule{6pt}{0pt}}}
}
\newenvironment{proof}{{\bf Proof:}}{\hfill\ABox}


\newtheorem{theorem}{{\bf Theorem}}
\newtheorem{corollary}[theorem]{Corollary}
\newtheorem{lemma}[theorem]{Lemma}
\newtheorem{conjecture}[theorem]{Conjecture}
\newtheorem{definition}[theorem]{Definition}

\newcommand{\lemlab}[1]{\label{lemma:#1}}
\newcommand{\thmlab}[1]{\label{thm:#1}}
\newcommand{\eqlab}[1]{\label{eq:#1}}
\newcommand{\corlab}[1]{\label{cor:#1}}
\newcommand{\deflab}[1]{\label{def:#1}}
\newcommand{\tablab}[1]{\label{tab:#1}}
\newcommand{\figlab}[1]{\label{fig:#1}}
\newcommand{\seclab}[1]{\label{sec:#1}}
\newcommand{\chaplab}[1]{\label{chap:#1}}

\newcommand{\lemref}[1]{\ref{lemma:#1}}
\newcommand{\thmref}[1]{\ref{thm:#1}}
\newcommand{\corref}[1]{\ref{cor:#1}}
\newcommand{\defref}[1]{\ref{def:#1}}
\newcommand{\chapref}[1]{\ref{chap:#1}}
\newcommand{\secref}[1]{\ref{sec:#1}}
\newcommand{\eqref}[1]{\ref{eq:#1}}
\newcommand{\figref}[1]{\ref{fig:#1}}
\newcommand{\tabref}[1]{\ref{tab:#1}}




{\makeatletter
 \gdef\xxxmark{\expandafter\ifx\csname @mpargs\endcsname\relax \expandafter\ifx\csname @captype\endcsname\relax \marginpar{xxx}\else
       xxx \fi
   \else
     xxx \fi}
 \gdef\xxx{\@ifnextchar[\xxx@lab\xxx@nolab}
 \long\gdef\xxx@lab[#1]#2{{\bf [\xxxmark #2 ---{\sc #1}]}}
 \long\gdef\xxx@nolab#1{{\bf [\xxxmark #1]}}
\gdef\turnoffxxx{\long\gdef\xxx@lab[##1]##2{}\long\gdef\xxx@nolab##1{}}}


\def\P{{\mathcal P}}
\def\Q{{\mathcal Q}}
\def\D{{\mathcal D}}
\def\G{{\Gamma}}
\def\g{{\gamma}}
\def\l{{\lambda}}
\def\k{{\kappa}}
\def\o{{\omega}}
\def\r{{\rho}}
\def\s{{\sigma}}
\def\d{{\delta}}
\def\a{{\alpha}}
\def\b{{\beta}}
\def\sp{\mathop{\rm sp}\nolimits}
\def\bP{{\partial P}}


\newcommand{\squeezelist}{\setlength{\itemsep}{0pt}}



\title{
Unfolding Convex Polyhedra \\ 
via
Quasigeodesic Star Unfoldings\footnote{
   A preliminary version of this work appeared
   in~\cite{iov-ucpq-07,iov-ucpq-07a}.
}
}


\author{Jin-ichi Itoh\thanks{Dept. Math.,
	Faculty Educ., Kumamoto Univ.,
	Kumamoto 860-8555, Japan.
    \protect\url{j-itoh@kumamoto-u.ac.jp}}
\and
Joseph O'Rourke\thanks{Dept. Comput. Sci., Smith College, Northampton, MA
      01063, USA.
      \protect\url{orourke@cs.smith.edu}.}
\and
Costin V\^{i}lcu\thanks{Inst. Math. `Simion Stoilow' Romanian Acad.,
P.O. Box 1-764,
RO-014700 Bucharest, Romania.
    \protect\url{Costin.Vilcu@imar.ro}.}
}

\begin{document}
\maketitle

\begin{abstract}
We extend the notion of a star unfolding to be based on
a simple quasigeodesic loop  rather than on a point.
This gives a new general method to unfold the surface of any convex polyhedron
 to a simple, planar polygon: shortest paths
from all vertices of  to  are cut, and all but one
segment of  is cut.
\end{abstract}

\section{Introduction}

There are two general methods known to unfold the surface  
of any convex polyhedron
to a simple polygon in the plane:
the source unfolding and the star unfolding.
Both unfoldings are with respect to a point .
Here we define a third general method:
the star unfolding
with respect to a
simple closed ``quasigeodesic loop''  on .
In a companion paper~\cite{iov-ucpqsu-08b}, we 
will extend the analysis
to the source unfolding with respect to a wider class of curves .


The \emph{point source unfolding} cuts the \emph{cut locus} of
the point :
the closure of set of all those points  to which there is
more than one shortest path on  from .
Alternatively, the cut locus is the set of all extremities
(different from ) of maximal (with respect to inclusion) shortest paths
starting at .
The notion of cut locus was introduced by 
Poincar\'e~\cite{p-lgsc-1905}
in 1905, and since then has gained an
important place in global Riemannian geometry; see, e.g., 
\cite{k-ccl-67} or~\cite{s-rg-96}.
The point source unfolding has been studied
for polyhedral surfaces since~\cite{ss-spps-86} (where the cut locus is called the ``ridge tree'').
The \emph{point star unfolding} cuts the shortest paths from  to every
vertex of .
The idea goes back to Alexandrov~\cite[p.~181]{a-kp-48};\footnote{
   It is called the ``Alexandrov unfolding'' in~\cite{mp-mccpc-05}.
}
that it unfolds  to a simple polygon was established
in~\cite{ao-nsu-92}.

In this paper we extend the star unfolding to be based on a
simple closed polygonal curve  with particular properties,
rather than based a single point.
This unfolds any convex polyhedron to a simple polygon,
answering a question
raised in~\cite[p.~307]{do-gfalop-07}.

The curves  for which our star unfolding works are
quasigeodesics, which we now define.



\paragraph{Geodesics \& Quasigeodesics.}
A \emph{geodesic} is a locally shortest path on a smooth surface.
A quasigeodesic is a generalization that extends the notion to 
nondifferentiable, and in particular, to
polyhedral surfaces.
Let  be any directed curve on a convex surface , 
and  be any point in the relative interior of , i.e., not
an endpoint.
Let  be the total face angle incident to the left side of ,
and  the angle to the right side.
If  is a geodesic, then .
A \emph{quasigeodesic}  loosens this condition to  and ,
again for all  interior to ~\cite[p.~16]{az-igs-67}~\cite[p.~28]{p-egcs-73}.
So a quasigeodesic  has  total face angle incident to each
side at all nonvertex points (just like a geodesic),
and has  angle to each side where  passes through a
polyhedron vertex.
(Geodesics can never pass through vertices.)
A \emph{simple closed geodesic} is non-self-intersecting (\emph{simple}) closed
curve that is a geodesic, and
a \emph{simple closed quasigeodesic} is a simple closed curve
on  that is quasigeodesic throughout its length.
As all curves we consider must be simple, we will henceforth
drop that prefix.
Pogorelov showed that any convex polyhedron  has at least three closed
quasigeodesics~\cite{p-qglcs-49}, extending the celebrated earlier result of
Lyusternik-Schnirelmann showing that the same holds for geodesics on 
differentiable convex surfaces.

A \emph{geodesic loop} is a closed curve that is geodesic
everywhere except possibly at one point,
and similarly a \emph{quasigeodesic loop} is quasigeodesic except
possibly at one point , the \emph{loop point}, at which the angle conditions
on  and  may be violated---one may be .
Quasigeodesic loops encompass
closed geodesics and quasigeodesics,
as well as geodesic loops.

Although it is known that every  must have at least
three closed quasigeodesics, 
there is no algorithm known that will find a simple closed quasigeodesic
in polynomial time:
Open Problem~24.2~\cite[p.~374]{do-gfalop-07}.
Fortunately it is in general
easy to find quasigeodesic loops on a given :
start at any nonvertex point , and extend a geodesic from
 in opposite directions, following each branch until they meet
at .  If no vertices are encountered, we have a geodesic loop;
if vertices are encountered, maintaining the angle conditions through the
vertices (which is always possible, e.g., by bisecting the surface angle) will
result in a quasigeodesic loop.

An exception to this ease of finding a geodesic loop could
occur on an \emph{isosceles tetrahedron}: a tetrahedron
whose four faces are congruent triangles, or, equivalently,
one at which the total face angle incident to each vertex is .
It is proved in~\cite{iv-gcit-08} that a convex surface possesses
a \emph{simple quasigeodesic line}---a
non-self-intersecting quasigeodesic infinite in both directions---if
and only if the surface is an isosceles tetrahedron.
So, excepting this case, the procedure described above will produce a
quasigeodesic loop.





\paragraph{Discrete Curvature.}
The discrete \emph{curvature}  at any point ,
is the \emph{angle deficit}
or \emph{gap}:  minus the sum of the face angles incident to .
The curvature is only nonzero at vertices of ;
at each vertex it is positive because  is convex.
By the Gauss-Bonnet theorem, a closed geodesic partitions the
curvature into  in each ``hemisphere'' of .
For quasigeodesics that pass through vertices, the curvature in each
half is .
The curvature in each half defined by a quasigeodesic loop depends on the
angle at the exceptional loop point.


\paragraph{Some Notation.}
For a quasigeodesic loop  on ,
 separates  into two ``halves''
 and .
As our main focus is usually on one such half, 
to ease notation we sometimes use  without a subscript
to represent either of  or 
when the distinction between them does not matter.
Unless otherwise stated,
vertices of  are labeled  in arbitrary order.
Other notation will be introduced as needed.
A glossary of all symbols defined (chronologically) 
throughout the paper is provided in Appendix~1.



\section{Example of Star Unfolding}
\seclab{Example}
We start with an example.
Figure~\figref{GeodesicLoopCube}(a) shows a geodesic loop
 on the surface  of a cube.
 at every point  of  except at
, where  and  .
Note that three cube vertices, , are to the left
of , and the other five to the right.
This is consistent with the Gauss-Bonnet theorem, because
 has a total turn of ,
so turn plus enclosed curvature is .

For each vertex , we select a shortest path
 to : a geodesic from  to a point 
whose length is minimal among all geodesics to .
In general there could be several shortest paths from  to ;
we use  to represent an arbitrarily selected one.
The point  is called
a \emph{projection} of  onto ,
and the path  is called a \emph{segment} on .
In the example, all the shortest path segments  are unique,
which is the generic situation.

\begin{figure}[htbp]
\centering
\includegraphics[width=\linewidth]{Figures/GeodesicLoopCube}
\caption{
(a)~Geodesic loop  on cube.
Shortest paths  are shown.  Faces are labeled
.
(b)~Star unfolding with respect to ,
joined at 
.
}
\figlab{GeodesicLoopCube}
\end{figure}



\paragraph{Algorithm.}
If we view the star unfolding as an algorithm
with input  and , it consists of three
main steps:
\begin{enumerate}
\squeezelist
\item Select shortest paths  from each  to .
\item Cut along  and flatten each half.
\item Cut along , joining the two halves at an uncut segment .
\end{enumerate}

After cutting along , we conceptually insert an isoceles
triangle with apex angle  at each , which flattens
each half.  One half (in our example, the left half), is convex,
while the other half has several points of nonconvexity,
at the images of .
(In our example, only  is nonconvex, when the inserted
``curvature triangles'' are included.)
In the third and final step of the procedure,
we select a segment  of  whose interior contains
neither a vertex  nor any vertex projection , 
such that the extension of  is a supporting line of each half,
and cut all of  except for .
In our example, we choose 

(many choices for  work in this example),
which leads to non-overlap of the two halves.


We now proceed to detail the three steps of the procedure,
this time with proofs.  We use a different example
to illustrate the main ideas.

\section{Shortest Path Cuts}
We again use
a cube as an illustrative example, but this time with a 
closed quasigeodesic , not a loop: 
; see
Figure~\figref{cube_geodesic_hemis}(a).
There is  angle incident to the right at , 
and  incident to
the left; and similarly at  and .  At all other
points , .  Thus  is indeed a quasigeodesic.
We will call the left half (including ) , and the right half 
(including ) .
In Figure~\figref{cube_geodesic_hemis}(a), the paths from 
are uniquely shortest.
From  there are three paths tied for shortest, and from
 also three are tied.
\begin{figure}[htbp]
\centering
\includegraphics[height=0.95\textheight]{Figures/cube_geodesic_hemis}
\caption{
(a)~Cube and quasigeodesic . 
Shortest paths  as indicated.
(b,c) Flattening the left half  by insertion of curvature triangles
along the shortest paths .
(d,e) Flattening the right half.
(f)~Two halves joined at .
}
\figlab{cube_geodesic_hemis}
\end{figure}


A central fact that enables our construction is this key lemma from
\cite[Cor.~1]{iiv-qfpcs-07}, slightly modified for our circumstances:

\begin{lemma}                                                                   
Let  be a quasigeodesic on a convex surface , and  any point of       
 not on .                                                                
Then for any choice of , this is the unique shortest path from      
 to  and it is orthogonal to  if  is in the relative              
interior of .                                                                
\lemlab{IIV}                                                                    
\end{lemma}


In our situation, the orthogonality condition is only guaranteed
to hold when  is not the exceptional loop point  of a quasigeodesic
loop.  In the example of Figure~\figref{cube_geodesic_hemis}(a),
there is no exceptional point, so all projections are orthogonal
to .
Note that this lemma does
not say that the shortest path from  to  is unique---which we
know is not always true---but that,
among those that are tied for shortest, each is the unique shortest path
between its two endpoints.

A second fact we need concerning these shortest paths is that they are disjoint,
excepting those arriving at the exceptional loop point,
in which case they share precisely that point.
The reader who accepts this basic fact is invited to skip beyond
the proof.
\begin{lemma}
Any two shortest paths 
 and ,
not incident to the loop point ,
are disjoint,
for distinct vertices .
\lemlab{sp.disjoint}
\end{lemma}
\begin{proof}
Suppose for contradiction that at least one point  is shared:
.
We consider four cases:
one shortest path is a subset of the other,
the shortest paths cross, the shortest paths touch at an interior point
but do not cross, or their endpoints coincide.
\begin{enumerate}
\item .
Then  contains a vertex  in its interior, which violates
a property of shortest paths~\cite[Lem.~4.1]{ss-spps-86}.
\item  and  cross properly at .
It must be that , otherwise both paths
would follow whichever tail is shorter.
But now it is possible to shortcut the path in the vicinity of  via
 as shown in 
Figure~\figref{sp_lemma}(a), and the path  is shorter than .
\item  and  touch at  but do not cross properly there.
Then there is a shortcut  to one side (the side with angle ),
as shown in 
Figure~\figref{sp_lemma}(b).
\item .
Then from Lemma~\lemref{IIV}, we know the two paths are orthogonal to the
quasigeodesic .  If we are not in the previous case, then it
must be that there is an angle  separating the paths in a neighborhood
of the common endpoint; see Figure~\figref{sp_lemma}(c).
Then  has more than  angle to one side at
this point, violating the definition of a quasigeodesic.
Note that it is here we use the assumption that the paths
are not incident to the loop point .
\end{enumerate}
\end{proof}


\begin{figure}[htbp]
\centering
\includegraphics[width=\linewidth]{Figures/sp_lemma}
\caption{Lemma~\protect\lemref{sp.disjoint}:
(a)~paths cross;
(b)~paths touch at an interior point;
(c)~paths meet at endpoint.}
\figlab{sp_lemma}
\end{figure}


This lemma ensures that the cuttings along  do not interfere
with one another.



\section{Flattening the Halves.}
The next step is to flatten each chosen
half  and  (independently) 
by suturing in ``curvature triangles''
along each  path.
Let  be one of  or .
The basic idea goes back to Alexandrov~\cite{a-cp-05}[p.~241, Fig.~103],
and was used also in~\cite{iv-cfpcs-08}.
Let  be the length  of a shortest path,
and let  be the curvature at .
We glue into  the isosceles \emph{curvature triangle}
 with apex angle  gluing
to ,
and incident sides of length  gluing along the cut . This is illustrated in
Figure~\figref{cube_geodesic_hemis}, where we show the faces
incident to  in a planar development in~(b) and~(d),
and after gluing in the triangles in~(c) and~(e).
We display this in the plane for convenience of presentation;
the triangle insertion should be viewed as operations on the manifolds
 and , each independently.


This procedure only works if , for  becomes the apex of the
inserted triangle .
If , we glue in two triangles of apex angle , both with their
apexes at .\footnote{
   One can view this as having two vertices with half
   the curvature collocated at .
}
Slightly abusing notation, we use  to represent
these two triangles together.  
In fact we must have  for any vertex 
(else there would be no face angle at ),
so  and this insertion is indeed well defined.

We should remark that an alternative method of handling 
would be to simply not glue in anything to the 
vertex  with , in which
case we still obtain the lemma below leading to the exact same unfolding.

Now, because  is the curvature (angle deficit) at , gluing in 
there flattens  to have total incident angle .
Thus  disappears as a vertex from  (and two new vertices are created along ).

Let  be
the new manifold with boundary after insertion of all 
curvature triangles into .
We want to claim that a planar development of  does not
overlap.  This is straightforward for a closed quasigeodesic,
but requires some argument for a quasigeodesic loop.
\begin{lemma}
For each half  of ,
 is a planar, simple (non-overlapping) polygon.
\lemlab{convex.polygon}
\end{lemma}
\begin{proof}
 is clearly a topological disk:  is, and the insertions of 's
maintains it a disk.
At every interior point of , the curvature is zero by construction.
So the interior is flat.

Let  be the total curvature enclosed within  on ,
and  the total turn of , i.e., the turn of .
The Gauss-Bonnet Theorem yields .
This is precisely the total turn of , because
that boundary turns , plus a total of  for all the
inserted curvature triangles.
So indeed the boundary of  turns just as much as it should
if it is a planar polygon.  It remains to establish
that it is a simple polygon.

There are two cases to consider,
depending on whether  is a closed quasigeodesic, or a closed
quasigeodesic loop.
\begin{enumerate}
\item  is a closed quasigeodesic.
In this case we show that the boundary 

is convex.
This follows from the orthogonality of  guaranteed by Lemma~\lemref{IIV},
as the base angle of the inserted triangle(s) is
 for ,
or  for 
(see Figure~\figref{Q_convex}; ),
so the new angle is smaller than  by  or .
\begin{figure}[htbp]
\centering
\includegraphics[width=0.75\linewidth]{Figures/Q_convex}
\caption{Lemma~\protect\lemref{convex.polygon}, Case~1: 
(a)~; (b)~.}
\figlab{Q_convex}
\end{figure}
Thus  is a planar convex polygon, and therefore simple.
See Figure~\figref{cube_geodesic_hemis}(c,e) for examples.

\hide{
Note that, when the total curvature in  is 
then the straight development
of  is turned  by the  insertions,
as in~(b) of the figure.
When the total curvature in  is ,
the development of  is not straight, but the  insertions
turn it exactly the additional amount needed to close it to ,
as in~(d) of the figure.
}


\item  is a closed quasigeodesic loop, with loop point .
If  is the half of  in which the angle at  is ,
then the argument above applies.
So assume  is the half in which the angle  at  exceeds .
We consider two subcases.

\begin{enumerate}
\item No vertex of  projects to .

Then after insertion of the curvature triangles,  is a topological
disk whose boundary is locally convex at all points except at ,
whose internal angle is .
We now argue that a planar development of such a domain is non-overlapping.
Let  and  be rays from  along the two edges of 
incident to ;
see Figure~\figref{xNeighborhoodNov}.
\begin{figure}[htbp]
\centering
\includegraphics[width=0.5\linewidth]{Figures/xNeighborhoodNov}
\caption{Lemma~\protect\lemref{convex.polygon}, Case~2(a): 
 has one point  of local nonconvexity.
Here .
The spiral depicted encloses a point  of winding number 2.
}
\figlab{xNeighborhoodNov}
\end{figure}
The boundary  of  must be exterior to the cone
delimited by  and  in a neighborhood of those rays,
because the boundary turns convexly at each boundary vertex.
So now we have a convex curve, a subset of ,
leaving  and returning to .
For the purposes of contradiction, assume this curve self-intersects.
Then it must ``spiral,'' enclosing a point  of winding number .
We noted above that the total turn of  is .
Thus the total turn of
the convex portion of , i.e.,
,
exceeds  by the amount  needed
to close the shape at .  
But , so
the convex curve
turns at most .
However, the point  must ``see'' a turn of  to have
winding number .
Therefore,   does not selt-intersect,
and   is a simple polygon.

\item One or more vertices of  project to .

\hide{
\begin{figure}[htbp]
\centering
\includegraphics[width=\linewidth]{Figures/quasicones}
\caption{Lemma~\protect\lemref{convex.polygon}, Case~2(b): 
(a)~Several vertices project to . (b,c,d)~Insertion of curvature triangles. Here
 for .
}
\figlab{quasicones}
\end{figure}
}\begin{figure}[htbp]
\centering
\includegraphics[width=0.75\linewidth]{Figures/NonconvexChain}
\caption{Lemma~\protect\lemref{convex.polygon}, Case~2(b): 
(a)~Several vertices project to .
(b)~After insertion of curvature triangles, with  held fixed.  is the
planar image of .
}
\figlab{NonconvexChain}
\end{figure}


Let  be the vertices that project to 
in circular order,
as illustrated in Figure~\figref{NonconvexChain}(a).
\hide{
Insert the curvature triangles one by one, for .
The first rotates the ``elbow''
 by  about  to an elbow with ray ,
the second insertion rotates the second elbow by  about ,
and so on, as illustrated in~(b,c,d) of the figure.
Let  be the elbow angle at , the angle between  and  on .
It must be that ,
because  is a shortest path.
This angle condition ensures that 
successive elbows do not intersect, and so they 
bound between them ``quasicones'' with disjoint interiors, 
and disjoint
from the original cone delimited by  and ,
which remains after the final rotation~(d).
}Let  and  be the extreme images of  in a planar
development of 
after insertion of all curvature triangles, i.e.,
incident to the planar image of  and of  respectively;
see Figure~\figref{NonconvexChain}(b).
View the curve  as composed of two pieces:
, the curve counterclockwise from  to ,
and , the complementary curve counterclockwise from 
to ; .
 is locally convex everywhere, but
 may be nonconvex, as illustrated in Figure~\figref{NonconvexChain}(b).
Now we partition the remaining argument into three parts.

\begin{enumerate}
\item  does not self-intersect.
The convex portion does not self-intersect for the same
reason we just established in the case above:
the curve would have to spiral and turn , but
that total turn angle is not available.

\item  does not self-intersect.
It clearly cannot if , so we
henceforth assume .
\begin{figure}[htbp]
\centering
\includegraphics[width=0.5\linewidth]{Figures/x2x1}
\caption{Lemma~\protect\lemref{convex.polygon}, Case~2(b)ii: 
The curve  is formed from the bases
of curvature triangles. It needs to turn  to self-intersect.}
\figlab{x2x1}
\end{figure}


This portion of  is composed entirely of bases of
curvature triangles.  For this portion to self-intersect, it must
turn at least .  We now compute the total turn  and show this
leads to a contradiction.
Let  be the angle  and  the angle .
It must be that  because  is a shortest path.
Let  be the angle at  on  between  and .
See Figure~\figref{x2x1}.
Thus,
because ,
, 
and because
 
we get
.
We henceforth drop the limits on the sums, which all run
the full appropriate range.
The base angle for the curvature triangle incident to 
is .
Thus the turn of the curve at the angle  is 

The total turn is therefore


To insist that  is to say that

a contradiction to .
Thus .
It is also worth noting that the same turn-angle bound holds
for any subchain of  (just by narrowing the sum limits),
a fact we will use below.

\item
 and  do not intersect.
For , a potentially nonconvex curve, to intersect ,
it would have to form a path from , crossing 
(see Figure~\figref{NonconvexChain}(b)), and returning to 
(or symmetrically, from  cross  and return to ).
But this requires some subchain of  to turn ,
contradicting the turn-angle conclusion above.
Therefore,  does not self-cross,
and  is indeed a simple polygon.
\end{enumerate}

\end{enumerate}
\end{enumerate}
\end{proof}

\noindent
The above argument would be simpler if it were established that
the cone  in  Figure~\figref{NonconvexChain}(b)
is empty in the planar development.
We leave this for future work.


\section{Joining the Halves}
The third and final step of the unfolding procedure
selects a \emph{supporting segment} 
whose relative interior does not contain
a projection  of a vertex.
All of  will be cut except for .
Our choice of  depends on whether  is a closed quasigeodesic
or a quasigeodesic loop:
\begin{enumerate}
\squeezelist
\item  is a closed quasigeodesic.
Then any  generates a supporting line to a planar development of ,
,
because  is a convex domain.
Then joining planar developments of  and  along 
places them on opposite sides of the line through ,
thus guaranteeing non-overlap.
See Figure~\figref{cube_geodesic_hemis}(f),
where .
\item  is a quasigeodesic loop.
Let  be the half of  that contains the angle  at .
Thus  is potentially nonconvex at
points along the chain  from  to .
\hide{
Nevertheless, some  might still be a supporting line of
.
This is the case in Figure~\figref{GeodesicLoopCube}, where several
supporting  options exist.

Although we do not have an example where no such supporting
 exists, neither do we have a method to guarantee that one
does exist.
So we follow a different strategy.

The argument using disjoint quasicone interiors 
in Lemma~\lemref{convex.polygon}
established that the complementary cone at

of angle , cone  in 
Figure~\figref{quasicones}(d), is empty in the planar
development of .
Thus this represents a \emph{supporting cone} to .
Now  is a convex planar domain, whose angle
at  is .  Thus the planar development of
 may be placed so that  of  and  of  coincide,
and  is the segment incident to .
In Figure~\figref{GeodesicLoopCube}(b),
.
Note that, due to symmetry, the segment  incident to 
is an equally valid choice.
}Let  and  be the rays from  and  respectively, tangent
to .
\begin{figure}[htbp]
\centering
\includegraphics[width=0.8\linewidth]{Figures/SupportingLine}
\caption{
 cannot cross both extensions of 
edges incident to .
Here .
}
\figlab{SupportingLine}
\end{figure}
Let  and  be lines parallel to  and 
tangent to  at  and  respectively.
Let  be any vertex between  and ;
 may be  or , as in
Figure~\figref{SupportingLine}.
Now we claim that one of the two edges of 

incident to  can serve as .
For both these edges to fail to extend to supporting
lines,  would have to cross both edge extensions.
But, the angle at  is 
(because  is convex),
so crossing both edge extensions would require
 
to turn more than , which we established in
Lemma~\lemref{convex.polygon} is impossible.

Case~2 is illustrated in Figure~\figref{GeodesicLoopCube}(b),
where this reasoning leads to .
\end{enumerate}

\noindent
In either case  extends to a supporting line
of both halves, and thus we obtain
a non-overlapping placement of the
planar developments of  and .



It should be clear now that this procedure works for any convex polyhedron:
\begin{theorem}
Let  be a quasigeodesic loop on a convex polyhedral surface .
Cutting shortest paths from every vertex to , and cutting all but
a supporting segment  of  as designated above,
unfolds  to a simple planar polygon.
\thmlab{main}
\end{theorem}

\begin{figure}[htbp]
\centering
\includegraphics[width=0.8\linewidth]{Figures/DodecahedronUnfolding}
\caption{(a)~ here is a geodesic; it includes no vertices, as is
evident in the layout~(b).
The region isometric to a right
circular cylinder is highlighted.
The convex domain 
from Lemma~\protect\lemref{convex.polygon}
is shown in~(c),
and one possible unfolding in~(d).
}
\figlab{DodecahedronUnfolding}
\end{figure}



Figure~\figref{DodecahedronUnfolding} shows another
example, a closed geodesic on a dodecahedron, this time a pure geodesic.
The unfolding following the above construction is shown
in Figure~\figref{DodecahedronUnfolding}(c,d).
In this case when
 is a pure, closed geodesic, there is additional structure that
can be used for an alternative unfolding.
For now 
lives on a region isometric to a right
circular cylinder. 
Figure~\figref{DodecahedronUnfolding}(b)
illustrates that
the upper and lower rims of the cylinder are loops
parallel to  through the vertices of  at minimum distance
to  (at least one vertex on each side.) 
In the figure, these
shortest distances to the upper rim are the short vertical paths from 
to the five pentagon vertices.
Those rim loops are themselves
closed quasigeodesics.
An alternative unfolding keeps the cylinder between the rim loops intact
and attaches the two reduced halves to either side.
See Figure~\figref{DodecaCostin}.
\begin{figure}[htbp]
\centering
\includegraphics[width=0.75\linewidth]{Figures/DodecaCostin}
\caption{Alternative unfolding of the example in
Figure~\protect\figref{DodecahedronUnfolding}.
Various construction lines are shaded lightly.
}
\figlab{DodecaCostin}
\end{figure}


\section{Future Work}
We have focused on establishing Theorem~\thmref{main} rather than the algorithmic
aspects.  Here we sketch preliminary thoughts on computational complexity.
Let  be the number of vertices of , and
let  be the number of faces crossed by the
geodesic loop .
In general  cannot be bound as a function of .
Finally, let , the total combinatorial complexity
of the ``input'' to the algorithm.
Constructing  from a given point and direction will
take  time.
Identifying a supporting segment , and laying out the final
unfolding, is proportional to .
The most interesting algorithmic challenge is to find the shortest
paths from each vertex  to .
It appears that this can be accomplished efficiently,
in  time, by first computing the cut locus of .
We expect to address this computation in~\cite{iov-ucpqsu-08b}.

We do not believe that quasigeodesic loops constitutes the widest class of
curves for which the star unfolding leads to non-overlap.
In particular, we believe we can extend
Theorem~\thmref{main} to
quasigeodesics with two exceptional points, one with angle 
to one side, and the other with angle  to the other side.
But whether this extension constitutes the widest class of curves for which
Theorem~\thmref{main} holds remains unclear. 

If one fixes a nonvertex point  and a surface direction  
at , a quasigeodesic loop 
can be generated to have direction  at .
It might be interesting to study the continuum of star unfoldings 
generated by spinning  around .


\section*{Appendix~1: Symbol Glossary}
\begin{tabular}{l l}
 
   & convex polyhedron \\
 
   & the two ``halves''  \\

   & one half, either  or  \\
 
   & vertex of  \\
 
   & a quasigeodesic loop \\

   & the exceptional loop point of  \\

   & angle incident to left/right side of  on curve \\

   & directed curve \\

   & one selected shortest path from  to  \\

   & the projection of  onto :  \\

   & the curvature at ,  minus the incident face angles \\

   & a curvature triangle \\

   &  \\

   & the boundary of  \\

   & the manifold  after insertion of all curvature triangles  \\

   & total curvature inside  on  \\

   & total turn of  \\

   & the boundary of   \\

   & angle  at the loop point  \\

   & extreme images of  in planar development of  \\

   & rays along edges incident to \\

   &  is angle  and  is angle \\

   & angle at  on  between  and  \\

   &  counterclockwise from  to  \\

   &  counterclockwise from  to \\

   & total turn of curve  from  to \\

   & lines parallel to  tangent to \\

   & tangency points of \\

   & a vertex between  and \\

   & supporting segment\\

   & number of vertices of  \\

   & combinatorial complexity of \\

   & \\
 
   & direction vector through 
\end{tabular}





\bibliographystyle{alpha}
\bibliography{/home/orourke/bib/geom/geom}
\end{document}
