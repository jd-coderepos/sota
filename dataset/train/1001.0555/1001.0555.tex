\documentclass[a4paper,10pt]{llncs}



\usepackage{epsfig}
\usepackage{times}
\usepackage{graphicx}
\usepackage{amsmath}
\usepackage{amssymb}
\usepackage{gensymb}
\usepackage[english]{babel}
\usepackage{multirow}

\usepackage[T1]{fontenc}


\pagenumbering{arabic} \pagestyle{plain}



\newenvironment{sketchofproof}{\noindent\textit{Sketch of proof.}}{\mbox{}\hfill\qed\par\medskip}
\newenvironment{propostion}{proposition}{}
\newcounter{prop}
\setcounter{prop}{0}


\renewenvironment{proof}
{{\bf Proof:}}{\hspace*{\fill}\par\vspace{2mm}}

\newcommand{\rephrase}[3]{\noindent\textbf{#1 #2}.~\emph{#3}}

\newcommand{\T}{\mbox{ }}
\renewcommand{\P}{\mbox{ }}
\newcommand{\remove}[1]{}

\newcommand{\ETN}{\texttt{ETN}}
\newcommand{\Bundles}{\texttt{BND}}
\newcommand{\Inorder}{E_{in}}
\newcommand{\Outorder}{E_{out}}
\newcommand{\EN}{\texttt{EN}}
\newcommand{\eat}[1] {{}}

\begin{document}

\title{On a Tree and a Path with no Geometric Simultaneous Embedding}

\author{P. Angelini , M. Geyer ,  M. Kaufmann  and D. Neuwirth }

\tocauthor{Patrizio Angelini, Markus Geyer, Michael Kaufmann, and Daniel Neuwirth}

\institute{Dipartimento di Informatica e Automazione -- Universit\`a  Roma Tre, Italy\\
        \email{angelini@dia.uniroma3.it}\\
    Wilhelm-Schickard-Institut f\"{u}r Informatik -- Universit\"{a}t T\"{u}bingen, Germany\\
    \email{geyer/mk/neuwirth@informatik.uni-tuebingen.de}
}


\maketitle              






\begin{abstract}
Two graphs  and  admit a \emph{geometric simultaneous embedding} if there exists a set of points  and a bijection  that induce planar straight-line embeddings both for  and for . While it is known that two caterpillars always admit a geometric simultaneous embedding and that two trees not always admit one, the question about a tree and a path is still open and is often regarded as the most prominent open problem in this area. We answer this question in the negative by providing a counterexample. Additionally, since the counterexample uses disjoint edge sets for the two graphs, we also negatively answer another open question, that is, whether it is possible to simultaneously embed two edge-disjoint trees. As a final result, we study the same problem when some constraints on the tree are imposed. Namely, we show that a tree of depth  and a path always admit a geometric simultaneous embedding. In fact, such a strong constraint is not so far from closing the gap with the instances not admitting any solution, as the tree used in our counterexample has depth .
\end{abstract}



\section{Introduction}

Embedding planar graphs is a well-established field in graph theory and algorithms with a great variety of applications. Keystones in this field are the works of Thomassen~\cite{t-eg-94}, of Tutte~\cite{t-hdg-63}, and of Pach and Wenger~\cite{pw-epgfvl-01}, dealing with planar and convex representations of graphs in the plane.

Since recently, motivated by the need of contemporarily represent several different relationships among the same set of elements, a major focus in the research lies on \emph{simultaneous graph embedding}. In this setting, given a set of graphs with the same vertex-set, the goal is to find a set of points in the plane and a mapping between these points and the vertices of the graphs such that placing each vertex on the point it is mapped to yields a planar embedding for each of the graphs, if they are displayed separately. Problems of this kind frequently arise when dealing with the visualization of evolving networks and with the visualization of huge and complex relationships, as in the case of the graph of the Web.

Among the many variants of this problem, the most important and natural one is the \emph{geometric simultaneous embedding}. Given two graphs  and , the task is to find a set of points  and a bijection  that induce planar straight-line embeddings for both  and .

In the seminal paper on this topic~\cite{J-bcdeeiklm-spge-07}, Brass \emph{et al.} proved that geometric simultaneous embeddings of pairs of paths, pairs of cycles, and pairs of caterpillars always exist. A \emph{caterpillar} is a tree such that deleting all its leaves yields a path. On the other hand, many negative results have been shown. Brass \emph{et al.}~\cite{J-bcdeeiklm-spge-07} presented a pair of outerplanar graphs not admitting any simultaneous embedding and provided negative results for three paths, as well. Erten and Kobourov~\cite{ek-sepgfb-04} found a planar graph and a path not allowing any simultaneous embedding. Geyer \emph{et al.}~\cite{gkv-ttsids-09} proved that there exist two trees that do not admit any geometric simultaneous embedding. However, the two trees used in the counterexample have common edges, and so the problem is still open for edge-disjoint trees.

The most important open problem in this area is the question whether a tree and a path always admit a geometric simultaneous embedding or not. In this paper we answer this question in the negative.

Many variants of the problem, where some constraints are relaxed, have been studied in the literature. If the edges do not need to be straight-line segments, a famous result of Pach and Wenger~\cite{pw-epgfvl-01} shows that any number of planar graphs admit a simultaneous embedding, since it states that any planar graph can be planarly embedded on any given set of points in the plane. However, the same result does not hold if the edges that are shared by two graphs have to be represented by the same Jordan curve. In this setting the problem is called {\it simultaneous embedding with fixed edges}~\cite{f-egsfe-06,gjpss-sgefe-06,fjks-crppgasefe-08}.

The research on this problem opened a new exciting field of problems and techniques, like ULP trees and graphs \cite{efk-culpt-06,fk-culpg-07,fk-mlnpt-07}, colored simultaneous embedding~\cite{g-csge-07}, near-simultaneous embedding \cite{fkk-csnse-07}, and matched drawings \cite{gdkls-mdpg-07}, deeply
related to the general fundamental question of point-set embeddability.

In this paper we study the geometric simultaneous embedding problem of a tree and a path. We answer the question in the negative by providing a counterexample, that is, a tree and a path not admitting any geometric simultaneous embedding. Moreover, since the tree and the path used in our counterexample do not share any edge, we also negatively answer the question on two edge-disjoint trees.

The main idea behind our counterexample is to use the path to enforce a part of the tree to be in a certain configuration which cannot be drawn planar. Namely, we make use of level nonplanar trees~\cite{efk-culpt-06,fk-mlnpt-07}, that is, trees not admitting any planar embedding if their vertices have to be placed inside certain regions according to a particular leveling. The tree of the counterexample contains many copies of such trees, while the path is used to create the regions. To prove that at least one copy has to be in the particular leveling that determines a crossing, we need a quite huge number of vertices. However, such a huge number is often needed just to ensure the existence of particular structures playing a role in our proof. A much smaller counterexample could likely be constructed with the same techniques, but we decided to prefer the simplicity of the argumentations rather than the search for the minimum size.

The paper is organized as follows. In Sect.~\ref{se:preliminaries} we give preliminary definitions and we introduce the concept of level nonplanar trees. In Sect.~\ref{se:tree-path} we describe the tree \T and the path \P used in the counterexample. In Sect.~\ref{se:overview} we give an overview of the proof that \T and \P do not admit any geometric simultaneous embedding, while in Sect.~\ref{se:proofs} we give the details of such a proof. In Sect.~\ref{se:depth2} we present an algorithm for the simultaneous embedding of a tree of depth  and a path, and in Sect.~\ref{se:conclusions} we make some final remarks.



\section{Preliminaries}\label{se:preliminaries}

A (undirected) \emph{-level tree}  on  vertices is
a tree , called the \emph{underlying tree} of , together with a
leveling of its vertices given by a function , such that for every edge , it holds  (See~\cite{efk-culpt-06,fk-mlnpt-07}). A drawing of  is a
\emph{level drawing} if each vertex  such that  is
placed on a horizontal line . A
level drawing of  is \emph{planar} if no two edges intersect
except, possibly, at common end-points. A tree  is \emph{level nonplanar} if it does not admit any planar level drawing.

We extend this concept to the one of \emph{region-level drawing} by enforcing the vertices of each level to lie inside a certain region rather than on a horizontal line. Let  be  pairwise non-crossing straight lines and let  be the regions of the plane such that any straight-line segment connecting a point in  and a point in , with , cuts all and only the lines , in this order. A drawing of a -level tree  is called \emph{region-level drawing} if each vertex  such that  is placed inside region . A region-level drawing of  is \emph{planar} if no two edges intersect except, possibly, at common end-points. A tree  is \emph{region-level nonplanar} if it does not admit any planar region-level drawing.

\begin{figure}[tb]
\begin{center}
\begin{tabular}{c c c}
\mbox{\includegraphics[height=2.7cm]{./pictures/ulp-tree.eps}} \hspace{0.5cm} &
\mbox{\includegraphics[height=3.3cm]{./pictures/ulp.eps}} \hspace{0.5cm} &
\mbox{\includegraphics[height=3.3cm]{./pictures/uap.eps}} \hspace{0.5cm} \\
(a) \hspace{0.5cm} & (b)  \hspace{0.5cm} & (c)
\end{tabular}
\caption{(a) A tree . (b) A level nonplanar tree  whose
underlying tree is . (c) A region-level nonplanar tree  whose
underlying tree is .}\label{fig:T_level_nonplanar}
\end{center}
\end{figure}

The -level tree  whose underlying tree is shown in
Fig.~\ref{fig:T_level_nonplanar}(a) has been shown to be level
nonplanar~\cite{fk-mlnpt-07} (see
Fig.~\ref{fig:T_level_nonplanar}(b)). In the next lemma we show that  is also region-level nonplanar (see Fig.~\ref{fig:T_level_nonplanar}(c)).

\begin{lemma}\label{lem:uap-tree}
The -level tree  whose underlying tree is shown in Fig.~\ref{fig:T_level_nonplanar}(a) is region-level nonplanar.
\end{lemma}

\begin{proof}
Refer to Fig.~\ref{fig:T_level_nonplanar}(c). First observe that, in any possible region-level planar drawing of , the paths  and  define a polygon  (a polygon ) inside region  (region ). We have that  is inside , as otherwise one of edges  or  would cross one of  or . Hence, vertex  has to be inside , as otherwise edge  would cross one of  or . However, in this case, there is no placement for vertices  and  that avoids a crossing between one of edges  or  and one of the already drawn edges.
\end{proof}

Lemma~\ref{lem:uap-tree} will be vital for proving that there exist a tree \T and a path \P not admitting any geometric simultaneous embedding. In fact, \T contains many copies of the underlying tree of , while \P connects vertices of \T in such a way to create the regions satisfying the above conditions and to enforce at least one of such copies to lie inside these regions according to the leveling making it nonplanar.

\section{The Counterexample}\label{se:tree-path}

In this section we describe a tree  and a path  not admitting any geometric simultaneous embedding.

\subsection{Tree \T}
The tree \T contains a root  and  \remove{:= {535\choose 7}} vertices  at distance  from , called {\it joints}. Each joint , with , is connected to  copies  of a subtree, called {\it branch}, and to  vertices of degree 1, called {\it stabilizers}. See Fig.~\ref{fig:complete_tree}(a).
Each branch  consists of a root ,  vertices of degree  adjacent to , and  leaves at distance  from . Vertices belonging to a branch  are called -\emph{vertices} and denoted by , , or vertices, according to their distance from their joint. Fig.~\ref{fig:complete_tree}(b) displays , , and vertices of a branch .

\begin{figure}[tb]
\begin{center}
\begin{tabular}{c c}
\mbox{\includegraphics[height=3.6cm]{./pictures/complete_tree_m.eps}} \hspace{1.5cm} &
\mbox{\includegraphics[height=3.3cm]{./pictures/subtree_m.eps}} \\
(a) \hspace{1.5cm} & (b)
\end{tabular}
\caption{(a) A schematization of the complete tree \T. Joints and stabilizers are small circles, branches are solid triangles, while complete subtrees connected to a joint are dashed triangles. (b) A schematization of a branch .}\label{fig:complete_tree}
\end{center}
\end{figure}

Because of the huge number of vertices, in the rest of the paper, for the sake of readability, we use variables , , and  as parameters describing the size of certain configurations. Such parameters will be given a value when the technical details of the argumentations are described. At this stage we just claim that a total number  of vertices (see Lemmata~\ref{lemma:2_channels} and~\ref{lemma:k_cluster_passage}) suffices for the counterexample.

As a first observation we note that, despite the oversized number of vertices, tree \T has limited \emph{depth}, that is, every vertex is at distance from the root at most . This leads to the following property.

\begin{property}\label{prop:three_bends}
Any path of tree edges starting at the root has at most  bends.
\end{property}

\subsection{Path \P}
Path \P is given by describing some basic and recurring subpaths on the vertices of \T and how such subpaths are connected to each other. The idea is to partition the set of branches  adjacent to each joint  into subsets of  branches each and to connect their vertices with path edges, according to some features of the tree structure, so defining the first building block, called {\it cell}.
Then, cells belonging to different branches are connected to each other, hence creating structures, called \emph{formations}, for which we can ensure certain properties regarding the intersection between tree and path edges. Further, different formations are connected to each other by path edges in such a way to create bigger structures, called \emph{extended formations}, which are, in their turn, connected to create a \emph{sequence of extended formations}.

All of these structures are constructed in such a way that there exists a set of cells such that any four of its cells, connected to the same joint and being part of the same formation or extended formation, contain a region-level nonplanar tree for any possible leveling, where the levels correspond to cells. Hence, proving that four of such cells lie in different regions satisfying the properties of separation described above is equivalent to proving the existence of a crossing in the tree. This allows us to consider only the bigger structures instead of dealing with single copies of the region-level nonplanar tree.

In the following we define such structures more formally and state their properties.

{\bf Cell:}
The most basic structure defined by \P is defined by looking at how it connects vertices of some branches  connected to the same joint  of \T. Consider a set of  branches , , connected to . Assume the vertices of a level inside each tree to be arbitrarily ordered. For each , define a \emph{cell}  to be composed of its \emph{head}, its \emph{tail}, and a number  of
\remove{ } stabilizers of .

\begin{figure}[tb]
  \centering{
\begin{tabular}{c}
\includegraphics[height=3cm]{./pictures/cell_m2.eps}
\end{tabular}}
\caption{A cell. -vertices of the head are depicted by large white circles,  -vertices of the tail are large grey circles, -vertices not part of the cell (showing the tree structure) are small grey circles and stabilizers are small white cirlces. Tree edges are grey and path edges are black.}\label{fig:cell}
\end{figure}

The {\it head} of  consists of the unique 1-vertex of , the first three 2-vertices of each branch , with  and , that are not already used in a cell , with , and, for each 2-vertex not in  and not in , the first 3-vertices not already used in a cell , with .

The \emph{tail} of  consists of a set of  branches  adjacent to . This set is partitioned into  subsets of  subtrees each. The vertices of each of the subsets are distributed between the cells in the same way as for the vertices of the head.

This implies that each cell contains one 1-vertex,  2-vertices, and  3-vertices of the head, an additional  1-vertices,  2-vertices, and  3-vertices of the tail, plus  stabilizers.

Path \P inside cell  visits the vertices in the following order: It starts at the unique 1-vertex of the head, then it reaches all the 2-vertices of the head, then all the 3-vertices of the head, then all the 2-vertices of the tail, and finally all the 3-vertices of the tail, visiting each set in arbitrary order.
After each occurrence of a 2- or 3-vertex of the head, \P visits a 1-vertex of the tail, and after each occurrence of a 2- or a 3-vertex of the tail, it visits a stabilizer of joint  (see Fig.~\ref{fig:cell}).

Note that, by this construction, for each joint there exists a set of cells such that each subset of size four contains region-level nonplanar trees with all possible levelings, where the levels correspond to the membership of the vertices to a cell. We now define two bigger structures describing how cells of this set are connected to cells of sets connected to other joints.

{\bf Formation:}
In the definition of a cell we described how the path traverses through one set of branches connected to the same joint. Now we describe how cells from four different sets are connected.

A {\it formation}  consists of 592 cells, namely of 148 cells  from the set of cells constructed above for each . Path \P connects these cells in the order , that is, \P repeats four times the following sequence: It connects  to , then to , then to , and so on till , from which it then connects to , to , and so on till  (see Fig.~\ref{fig:formation}(a)). A connection between two consecutive cells  and  is done with an edge connecting the end vertices of the parts \P and \P of \P restricted to the vertices of  and , respectively. Namely, the unique vertex in  having degree 1 both in \P and in \T is connected to the unique vertex in  having degree 1 in \P but not in \T. The following property holds:

\noindent
\begin{property}\label{prop:four_sep_areas_PS}
For any formation  and any joint , with , if four cells  are pairwise separated by straight lines, then there exists a crossing in .
\end{property}

\begin{figure}[tb]
  \centering{
\begin{tabular}{c c}
\includegraphics[height=3.5cm]{./pictures/formation_m2.eps} \hspace{1cm} &
\includegraphics[height=3.7cm]{./pictures/ef_m2.eps} \\
(a) \hspace{1cm} & (b)
\end{tabular}}
\caption{(a) A formation. Tree edges are depicted by grey and path edges by black lines. Please note in this figure also the bundle of tree edges connecting the different cells belonging to the same branch. (b) A subsequence  of an extended formation. Formations are inside a table to represent the -tuple they belong to and to emphasize that in each repetition (a row of the table) a formation at a certain -tuple is missing.}\label{fig:formation}
\end{figure}

{\bf Extended Formation:}
Formations are connected by the path in a special sequence, defined as {\it extended formation} and denoted by , where   is a tuple of tuples of disjoint indexes of joints (see Fig.~\ref{fig:formation}(b)). Let  be  formations  not belonging to any other extended formation and composed of cells of the same set . These formations are connected in the order , but in each of these  repetitions one  is missing. Namely, in the -th repetition the path does not reach any formation at , with . We say that the -th repetition has a {\it defect} at . We call a subsequence  a \emph{full repetition} inside . A full repetition has exactly one defect at each tuple.

Note that the size of  can now be fixed as the number of formations creating repetitions inside one extended formation times the number of cells inside each of these formations, that is . We claim that  and  is sufficient throughout the proofs. However, for readability reasons, we will keep on using variables  and  in the remainder of the paper.

{\bf Sequence of Extended Formation:}
Extended formations are connected by the path in a special sequence, called {\it sequence of extended formations} and denoted by , where  is a tuple of tuples of tuples. For each tuple , where , consider  extended formations , with , not already belonging to any other sequence of extended formations.
These extended formations are connected inside  in the order . There exist two types of sequences of extended formations. Namely, in the first type there is one extended formation missing in each subsequence , that we call \emph{defect}, as for the extended formations. In the second type, two consecutive extended formations are missing. Namely, in the -th repetition the path skips the extended formations connecting at  and at , with . In this case, we say that the repetition has a \emph{double defect}.

Since, for each set of  joints,  different disjoint sequences of extended formations exist, we just consider the sequences where the order defined by the tuple is the order of the joints around the root.

\section{Overview}\label{se:overview}

In this section we present the main argumentations leading to the final conclusion that the tree \T and the path \P described in Sect.~\ref{se:tree-path} do not admit any geometric simultaneous embedding. The main idea in this proof scheme is to use the structures given by the path to fix a part of the tree in a specific shape creating specific restrictions for the placement of the further substructures of \T and of \P attached to it.

We first give some further definitions and basic topological properties on the interaction among cells that are enforced by the preliminary arguments about region-level planar drawings and by the order in which the subtrees are connected inside one formation.

{\bf Passage:} Consider two cells  that can not be separated by a straight line and a cell , with . We say that there exists a {\it passage}  between , , and  if the polyline given by the path of  separates vertices of  from vertices of  (see Fig.~\ref{fig:passage}(a)). Since the polyline can not be straight, there is a vertex of  lying inside the convex hull of the vertices of , which implies the following.

\begin{figure}[tb]
  \centering{
\begin{tabular}{c c}
\includegraphics[height=3.4cm]{./pictures/passage_m2.eps} \hspace{1cm} &
\includegraphics[height=3.1cm]{./pictures/interconnected_m.eps} \\
(a) \hspace{1cm} & (b)
\end{tabular}}
\caption{(a) A passage between cells , , and . (b) Two interconnected passages.}\label{fig:passage}
\end{figure}

\begin{property}\label{prop:passage_2_edges}
In a passage between cells , , and  there exist at least two path-edges  of  such that both  and  are intersected by tree-edges connecting vertices of  to vertices of .
\end{property}

For two passages  between , , and , and  between  , , and  (w.l.o.g., we assume , , and ), we distinguish three different configurations: (i) If ,  and  are {\it independent}; (ii) if ,  is {\it nested} into ; and (iii) if ,  and  are {\it interconnected} (see Fig.~\ref{fig:passage}(b)).

{\bf Doors:}
Let , and  be three cells creating a passage. Consider any triangle given by a vertex  of  inside the convex hull of  and by any two vertices of .
This triangle is a \emph{door} if it encloses neither any other vertex of  nor any vertex of  that is closer than  to  in \T. A door is \emph{open} if no tree edge incident to  crosses the opposite side of the triangle, that is, the side between the vertices of  and  (see Fig.~\ref{fig:door}(a)), otherwise it is \emph{closed} (see Fig.~\ref{fig:door}(b)).

\begin{figure}[hb]
  \centering{
\begin{tabular}{c c}
\includegraphics[height=3.5cm]{./pictures/open_door_m.eps} \hspace{1cm} &
\includegraphics[height=3.5cm]{./pictures/closed_door_m.eps} \\
(a) \hspace{1cm} & (b)
\end{tabular}}
\caption{(a) An open door. (B) A closed door.}\label{fig:door}
\end{figure}

Consider two joints  and , with  appearing in this circular order around the root. Any polyline connecting the root to , then to , and again to the root, without crossing tree edges, must traverse each door by crossing both the sides adjacent to . If a door is closed, such a polyline has to bend after crossing one side adjacent to  and before crossing the other one.
Also, if two passages  and  are interconnected, either all the closed doors of  are traversed by a path of tree-edges belonging to  or all the closed doors of  are traversed by a path of tree-edges belonging to  (see Fig.~\ref{fig:passage}(b)).

In the rest of the argumentation we will exploit the fact that the closed door of a passage requests a bend in the tree to obtain the claimed property that a large part of \T has to follow the same shape. In view of this, we state the following lemmata relating the concepts of doors, passages, and formations.

\begin{lemma}\label{lemma:PS_passage}
For each formation , with , there exists a passage between some cells , with .
\end{lemma}

\begin{lemma}\label{lemma:closed_door_in_each_passage}
Each passage contains at least one closed door.
\end{lemma}

From the previous lemmata we conclude that each formation contains at least one closed door. To prove that the effects of closed doors belonging to different formations can be combined to obtain more restrictions on the way in which the tree has to bend, we exploit a combinatorial argument based on the Ramsey Theorem~\cite{grs-rt-90} and state that there exists a set of joints pairwise creating passages.

\begin{lemma}\label{lemma:k_cluster_passage}
Given a set of joints , with , there exists a subset , with , such that for each pair of joints  there exist two cells  creating a passage with a cell .
\end{lemma}

Now we formally define the claimed property that part of the tree has to follow a fixed shape by considering how the drawing of the subtrees attached to two different joints force the drawing of the subtrees attached to the joints between them in the order around the root.

{\bf Enclosing bendpoints:}
Consider two paths  and . The bendpoint  of  \emph{encloses} the bendpoint  of  if  is internal to triangle . See Fig.~\ref{fig:bendpoint}(a).

{\bf Channels:}
Consider a set of joints  in clockwise order around the root. The \emph{channel}  of a joint , with , is the region given by the pair of paths, one path of  and one path of , with the maximum number of enclosing bendpoints with each other. We say that  is an \emph{x-channel} if the number of enclosing bendpoints is . Observe that, by Property~\ref{prop:three_bends}, . A -channel is depicted in Fig.~\ref{fig:bendpoint}(b). Note that, given an -channel  of , all the vertices of the subtree rooted at  that are at distance at most  from the root lie inside .

\begin{figure}[tb]
  \centering{
\begin{tabular}{c c}
\includegraphics[height=2.8cm]{./pictures/bendpoints.eps} \hspace{1cm} &
\includegraphics[height=3cm]{./pictures/channels.eps} \\
(a) \hspace{1cm} & (b)
\end{tabular}}
\caption{(a) An enclosing bendpoint. (b) A -channel and its channel segments.}\label{fig:bendpoint}
\end{figure}

{\bf Channel segments:}
An -channel  is composed of  parts called \emph{channel segments} (see Fig.~\ref{fig:bendpoint}(b)). The first channel segment  is the part of  that is visible from the root. The -th channel segment  is the region of  disjoint from  that is bounded by the elongations of the paths of  and  after the -th bend.

Observe that, since the channels are created by tree-edges, any tree-edge connecting vertices in the channel has to be drawn inside the channel, while path-edges can cross other channels. In the following we study the relationships between path-edges and channels. The following property descends from the fact that every second vertex reached by \P in a cell is either a 1-vertex or a stabilizer.

\begin{property}\label{prop:CS_1_2}
For any path edge , at least one of  and  lie inside either  or .
\end{property}

{\bf Blocking cuts:}
A \emph{blocking cut} is a path edge connecting two consecutive channel segments by cutting some of the other channels twice. See Fig.~\ref{fig:blocking}.

\begin{property}\label{prop:blocking-cut}
Let  be a channel that is cut twice by a blocking cut. If  has vertices in both the channel segments cut by the path edge, then it has some vertices in a different channel segment.
\end{property}
\begin{proof}
Consider the vertices lying in the two channel segments of . In order to connect them in \T, a vertex  is needed in the bendpoint area of . However, in order to have path connectivity between  and the vertices in the two channel segments, some vertices in a different channel segment are needed.
\end{proof}

\begin{figure}[hb]
  \centering{
\begin{tabular}{c}
\includegraphics[height=2.9cm]{./pictures/blocking_m2.eps}
\end{tabular}}
\caption{A blocking cut.}
\label{fig:blocking}
\end{figure}

In the following lemma we show that in a set of joints as in Lemma~\ref{lemma:k_cluster_passage} it is possible to find a suitable subset such that each pair of paths of tree-edges starting from the root and containing such joints has at least two common enclosing bendpoints, which implies that most of them create -channels.

\begin{lemma}\label{lemma:2_channels}
Consider a set of joints  such that there exists a passage between each pair , with . Let , for  and  and  be two sets of passages between pairs of joints in  (see Fig.~\ref{fig:lemma5}). Then, for at least  of the joints of one set of passages, say , there exist paths in \T, starting at the root and containing these joints, which traverse all the doors of  with at least 2 and at most 3 bends. Also, at least half of these joints create an -channel, with .
\end{lemma}

\begin{figure}[tb]
  \centering{
\begin{tabular}{c}
\includegraphics[height=3.3cm]{./pictures/lemma5.eps}
\end{tabular}}
\caption{Two sets of passages  and  as described in Lemma~\ref{lemma:2_channels}.}
  \label{fig:lemma5}
\end{figure}

By Lemma~\ref{lemma:2_channels}, any formation attached to a certain subset of joints must use at least three different channel segments. In the remainder of the argumentation we focus on this subset of joints and give some properties holding for it, in terms of interaction between different formations with respect to channels. Since we need a full sequence of extended formations attached to these joints,  has to be at least eight times the number of channels inside a sequence of extended formations, that is, .

First, we give some further definitions.

{\bf Nested formations}
A formation  is {\it nested} in a formation  if there exist two edges  and two edges  cutting a border  of a channel  such that all the vertices of the path in  between  and  lie inside the region delimited by  and by the path in  between  and  (see Fig.~\ref{fig:nested_formations}(a)).

A series of pairwise nested formations  is {\it -nested} if there exist  formations , with , belonging to the same channel and such that, for each pair , there exists at least one formation , , belonging to another channel and such that  is nested in  and  is nested in  (see Fig.~\ref{fig:nested_formations}(b)).

\begin{figure}[htb]
  \centering{
\begin{tabular}{c c}
\includegraphics[height=3.3cm]{./pictures/nested.eps} \hspace{1cm} &
\includegraphics[height=3.3cm]{./pictures/r-nested.eps} \\
(a) \hspace{1cm} & (b)
\end{tabular}}
\caption{(a) A formation  nested in a formation . (b) A series of -nested formations.}
  \label{fig:nested_formations}
\end{figure}

{\bf Independent sets of formations}
Let  be sets of formations of one extended formation such that each set  contains formations  on the set of -tuples , where the joints of  are between the joints of  and of  in the order around the root. Further, let  be not nested in , for each  and . If for each pair of sets  there exist two lines  separating the vertices of  and  inside channel segment  and , respectively, the sets are \emph{independent} (see Fig.~\ref{fig:independent}).

\begin{figure}[htb]
  \centering{
\begin{tabular}{c}
\includegraphics[height=3.8cm]{./pictures/independent.eps}
\end{tabular}}
\caption{Two independent sets  and .}
  \label{fig:independent}
\end{figure}

In the following lemmata we prove that in any extended formation there exists a nesting of a certain depth (Lemma~\ref{lemma:nest_independent}). This important property will be the starting point for the final argumentation and will be deeply exploited in the rest of the paper. We get to this conclusion by first proving that in an extended formation the number of independent sets of formations is limited (Lemma~\ref{n-independent nestings}) and then by showing that, although there exist formations that are neither nested nor independent, in any extended formation there exists a certain number of pairs of formation that have to be either independent or nested (Lemma~\ref{lem:nest_sequence}).

\begin{lemma}\label{n-independent nestings}
There exist no  independent sets of formations  inside any extended formation,
where each  contains formations of a fixed set of channels of size .
\end{lemma}

\begin{lemma}\label{lem:nest_sequence}
Consider four subsequences , where , of an extended formation , each consisting of a whole repetition of . Then, there exists either a pair of nested subsequences or a pair of independent subsequences.
\end{lemma}

\begin{lemma}\label{lemma:nest_independent}
Consider an extended formation .
Then, there exists a -nesting, where , among the formations of .
\end{lemma}

Once the existence of -channels and of a nesting of a certain depth in each extended formation has been shown, we turn our attention to study how such a deep nesting can be performed inside the channels.

Let  and , with , be two channel segments. If the elongation of  intersects , then it is possible to connect from  to  by cutting both the sides of . In this case,  and  have a \emph{side connection} (see Fig.~\ref{fig:12-side-connection}(b)). On the contrary, if the elongation of  does not intersect , only one side of  can be used. In this case,  and  have a \emph{side connection} (see Fig.~\ref{fig:12-side-connection}(a)).

\begin{figure}[htb]
\begin{center}
\begin{tabular}{c c}
\mbox{\includegraphics[height=2.7cm]{./pictures/one-side.eps}} \hspace{1cm} &
\mbox{\includegraphics[height=2.7cm]{./pictures/two-side.eps}} \\
(a) & (b)\\
\end{tabular}
\caption{(a) A side connection. (b) A side connection.}
\label{fig:12-side-connection}
\end{center}
\end{figure}

Based on these different ways of connecting distinct channel segments, we split our proof into three parts, the first one dealing with the setting in which only -side connections are allowed, the second one allowing one single -side connection, and the last one tackling the general case.

\begin{proposition}\label{prop:only-1-side}
If there exist only side connections, then \T and \P do not admit any geometric simultaneous embedding.
\end{proposition}

We prove this proposition by showing that, in this configuration, the existence of a deep nesting in a single extended formation, proved in Lemma~\ref{lemma:nest_independent}, results in a crossing in either \T or \P.

\begin{lemma}\label{lemma:k-nesting}
If an extended formation lies in a part of the channel that contains only side connections, then \T and \P do not admit any geometric simultaneous embedding.
\end{lemma}

Next, we study the case in which there exist -side connections. We distinguish two types of -side connections, based on the fact that the elongation of channel segment  intersecting channel segment  starts at the bendpoint that is closer to the root, or not. In the first case we have a \emph{low Intersection} (see Fig.~\ref{fig:intersection}(a)), denoted by , and in the second case we have a \emph{high Intersection} (see Fig.~\ref{fig:intersection}(b)), denoted by , where . We use the notation  to describe both  and . We say that two intersections  and  are \emph{disjoint} if  and . For example,  and  are disjoint, while  and  are not.

\begin{figure}[htb]
\begin{center}
\begin{tabular}{c c}
\mbox{\includegraphics[height=2.5cm]{./pictures/2side-low.eps}} \hspace{2cm} &
\mbox{\includegraphics[height=2.5cm]{./pictures/2side-high.eps}} \\
(a) \hspace{2cm} & (b)\\
\end{tabular}
\caption{(a) A low Intersection. (b) A high Intersection.}
\label{fig:intersection}
\end{center}
\end{figure}

Since consecutive channel segments can not create any -side connection, in order to explore all the possible shapes we consider all the combinations of low and high intersections created by channel segments  and  with channel segments  and .
With the intent of proving that intersections of different channels have to maintain certain consistencies, we state the following lemma.

\begin{lemma}\label{lem:2-different-shapes}
Consider two channels  with the same intersections. Then, none of channels , where , have an intersection that is disjoint with the intersections of  and of .
\end{lemma}

As for Proposition~\ref{prop:only-1-side}, in order to prove that -side connections are not sufficient to obtain a simultaneous embedding of \T and \P, we exploit the existence of the deep nesting shown in Lemma~\ref{lemma:nest_independent}. First, we analyze some properties relating such nesting to channel segments and bending areas. A \emph{bending area}  is the region between  and  where bendpoints can be placed. We first observe that all the extended formations have to place vertices inside the bending area of the channel segment where the nesting takes place, and then prove that not many of the formations involved in the nesting can use the part of the path that creates the nesting to place vertices in such a bending area, which implies that the extended formations have to reach the bending area in a different way.

\begin{lemma}\label{lem:nesting-bending-area}
Consider an -nesting of a sequence of extended formations on an intersection , with .
Then, there exists a triangle  in the nesting that separates some of the triangles nesting with  from the bending area  (or ).
\end{lemma}

Then, we study some of the cases involving -side connections and we show that the connections between the bending area and the "endpoints" of the nesting create a further nesting of depth greater than . Hence, if no further side connection is available, this second nesting is not drawable.

\begin{proposition}\label{prop:triangle}
Let  be a triangle open on a side splitting a channel segment  into two parts such that every extended formation  has vertices in both parts. If the only possibility to connect vertices in different parts of  is with a -side connection and if any such connection creates a triangle open on a side that is nested with , then \T and \P do not admit any geometric simultaneous embedding.
\end{proposition}

\begin{figure}[htb]
\begin{center}
\begin{tabular}{c}
\mbox{\includegraphics[height=4cm]{./pictures/turning-vertex_m2.eps}} \hspace{0.2cm}
\end{tabular}
\caption{A situation as in Proposition~\ref{prop:triangle}. The chosen turning vertex is represented by a big black circle and is in configuration . The inner and the outer areas are represented by a light grey and a dark grey region, respectively.}
\label{fig:turning-vertex}
\end{center}
\end{figure}

Refer to Fig.~\ref{fig:turning-vertex}. Consider the two path-edges  creating  such that the common point  is in the channel segment  that is split into two parts, that we call {\it inner area} and {\it outer area}, respectively. We assume that  do not cut any channel segment  completely, since such a cut would create more restrictions than placing  or  inside . Consider the path in an extended formation  connecting the inner and the outer area through a -side connection at . As a generalization, consider for such a path of  only a vertex, called \emph{turning vertex}, which is placed in  and for which no other path in  exists that connects the inner and the outer area by using a channel segment  such that the subpath to  intersects either  or its elongation. If there exist more than one of such vertices, then arbitrarily choose one of them. Observe that the path connecting from the inner area to the outer area through the turning vertex encloses exactly one of  and . If it encloses , it is in configuration , otherwise it is in configuration . If there exist both paths in  and paths in  configuration, then we arbitrarily consider one of them. Finally, consider the connections between different extended formations inside a sequence of extended formations. Consider a turning vertex  in a channel segment  of a channel  such that the edges incident to  cut a channel . Then, any connection of an extended formation of  from the inner to the outer area in the same configuration as  and with its turning vertex  in  is such that  lies inside the convex hull of the two edges incident to .

In the following two lemmata we show that in the setting described in Proposition~\ref{prop:triangle} there exists a crossing either in \T or in \P.

\begin{lemma}
\label{lem:one_channel_segment}
In a situation as described in Proposition~\ref{prop:triangle}, not all the extended formations in a sequence of extended formations can place turning vertices in the same channel segment.
\end{lemma}

\begin{lemma}\label{lem:prop3-nodrawing}
In a situation as described in Proposition~\ref{prop:triangle}, \T and \P do not admit any geometric simultaneous embedding.
\end{lemma}

Based on the property given by Proposition~\ref{prop:triangle}, we present the second part of the proof, in which we show that having two intersections  and  does not help if  and  are not disjoint.

\begin{proposition}\label{prop:non-disjoint-intersections}
If there exists no pair of disjoint -side connections, then \T and \P do not admit any geometric simultaneous embedding.
\end{proposition}

Observe that, in this setting, it is sufficient to restrict the analysis to cases  and , since the cases involving  and  can be reduced to them.

\begin{lemma}\label{lem:intersects_one_three}
If a shape contains an intersection  and does not contain any other intersection that is disjoint with , then \T and \P do not admit any geometric simultaneous embedding.
\end{lemma}

\begin{lemma}\label{lem:intersects_three_one}
If there exists a sequence of extended formation in any shape containing an intersection , then \T and \P do not admit any geometric simultaneous embedding.
\end{lemma}

Observe that, in the latter lemma, we proved a property that is stronger than the one stated in Proposition~\ref{prop:non-disjoint-intersections}. In fact, we proved that a simultaneous embedding cannot be obtained in any shape containing an intersection , even if a second intersection that is disjoint with  is present.

Finally, in the third part of the proof, we tackle the general case where two disjoint intersections exist.

\begin{proposition}\label{prop:disjoint}
If there exists two disjoint -side connections, then \T and \P do not admit any geometric simultaneous embedding.
\end{proposition}

Since the cases involving intersection  were already considered in Lemma~\ref{lem:intersects_three_one}, we only have to consider the eight different configurations where one intersection is  and the other is one of . In the next three lemmata we cover the cases involving  and in Lemma~\ref{lem:cs_two_convex_hull} the ones involving .

Consider two consecutive channel segments  and  of a channel  and let  be a path-edge crossing the border of one of  and , say . We say that  creates a \emph{double cut} at  if the elongation of  cuts  in . A double cut is \emph{simple} if  does not cross  (see Fig.~\ref{fig:double-cut}(a)) and \emph{non-simple} otherwise (see Fig.~\ref{fig:double-cut}(b)). Also, a double cut of an extended formation  is \emph{extremal} with respect to a bending area  if there exists no double cut of  that is closer than it to .

\begin{figure}[htb]
\begin{center}
\begin{tabular}{c c}
\mbox{\includegraphics[height=3.3cm]{./pictures/double-cut-simple_m2.eps}} \hspace{1.5cm} &
\mbox{\includegraphics[height=3.3cm]{./pictures/double-cut_m2.eps}} \\
(a) \hspace{1.5cm} & (b)\\
\end{tabular}
\caption{(a) A simple double cut. (b) A non-simple double cut.}
\label{fig:double-cut}
\end{center}
\end{figure}

\begin{property}\label{prop:double-cut}
Any edge  creating a double cut at a channel  in channel segment  blocks visibility to the bending area  for a part of  in each channel  with  (with ).
\end{property}

In the following lemma we show that a particular ordering of extremal double cuts in two consecutive channel segments leads to a non-planarity in \T or \P. Note that, any order of extremal double cuts corresponds to an order of the connections of a subset of extended formations to the bending area.

\begin{lemma}\label{lemma:no-ordered-double-cuts}
Let  and  be two consecutive channel segments. If there exists an ordered set  of extremal double cuts cutting  and  such that the order of the intersections of the double cuts with  (with ) is coherent with the order of , then \T and \P do not admit any geometric simultaneous embedding.
\end{lemma}

Then, we show that shape   induces this order. To prove this, we first state the existence of double cuts in shape  . The existence of double cuts in shape   can be easily seen.

\begin{lemma}\label{lem:double_cuts_13}
Each extended formation in shape   creates double cuts in at least one bending area.
\end{lemma}

\begin{lemma}\label{lem:ordered-set-of-double-cuts-exists}
Every sequence of extending formations in shape   contains an ordered set  of extremal double cuts with respect to bending area either  or .
\end{lemma}

Finally, we consider the configurations where one intersection is  and the other one is one of . Observe that, in both cases, channel segment  is on the convex hull.

\begin{lemma}\label{lem:cs_two_convex_hull}
If channel segment  is part of the convex hull, then \T and \P do not admit any geometric simultaneous embedding.
\end{lemma}

Based on the above discussion, we state the following theorem.

\begin{theorem}
There exist a tree and a path that do not admit any geometric simultaneous embedding.
\end{theorem}
\begin{proof}
Let  and  be the tree and the path described in Sect.~\ref{se:tree-path}. Then, by Lemma~\ref{lemma:2_channels}, Lemma~\ref{lem:2-different-shapes}, and Property~\ref{prop:three_bends}, a part of \T has to be drawn inside channels having at most four channel segments. Also, by Lemma~\ref{lemma:nest_independent}, there exists a nesting of depth at least  inside each extended formation.

By Proposition~\ref{prop:only-1-side}, if there exist only -side connections, then \T and \P do not admit any simultaneous embedding. By Proposition~\ref{prop:non-disjoint-intersections}, if there exists either one -side connections or a pair of non-disjoint intersections, then \T and \P do not admit any simultaneous embedding. By Proposition~\ref{prop:disjoint}, even if there exist two disjoint -side intersections, then \T and \P do not admit any simultaneous embedding. Since it is not possible to have more than two disjoint -side intersections, the statement follows.
\end{proof}

\section{Detailed Proofs}\label{se:proofs}

\rephrase{Lemma}{\ref{lemma:PS_passage}}{
For each formation , with , there exists a passage between some cells , with .
}

\begin{proof}
Suppose, for a contradiction, that there exists no passage inside . First observe that, if two cells  are separated by a polyline given by the path passing through , then either they are separable by a straight line or such a polyline is composed of edges belonging to a cell  of the same joint . Since, by Property~\ref{prop:four_sep_areas_PS}, there exists no set of four cells of a given joint inside  that are separable by a straight line, it follows that all the cells of  of a given joint can be grouped into at most  different sets , , and  such that cells from different sets can be separated by straight lines, but cells from the same set can not. Therefore, the cells inside one of these sets can only be separated by other cells of the same set.

\begin{figure}[ht]
\begin{center}
\includegraphics[width=8cm]{./pictures/regions.eps}
\caption{The five path edges  connecting five cells of set  with five cells of set .}
\label{fig:regions}
\end{center}
\end{figure}

Consider the connections of the path through  with regard to this notion of sets of cells.
Let , with  and , be the set of cells belonging to set  and attached to joint . Hence, for any two cells  there are nine possible ways to connect between some  and .
Since the part of \P through  visits 37 times cells from , in this order, there exist five path edges  connecting five cells of set  with five cells of set , where  (see Fig.\ref{fig:regions}). Without loss of generality, we assume that edges  appear in this order in the part of \P through . Observe that , together with the five cells of  and the five cells of  they connect, subdivide the plane into five regions. Since the path is continuous in , it connects from the end of  (a cell of joint ) to the beginning of  (a cell of joint ), from the end of  to the beginning of , and so on. If in the region between edges  and , with , there exists no cell of joint , then the path through  will not traverse the region between these edges in the opposite direction, since the path contains no edges going from a cell of  to a cell of  and since the start- (and end-) cells of these edges cannot be separated by straight lines.
Furthermore, note that, in this case, the path-connection from  to  does not traverse the region between the edges, therefore forming a spiral shape, in the sense that the part of the path following  is separated from the part of the path prior to .
Since we have five edges between  and  but only 3 possible sets of cells on joint , at least one pair of edges exists creating an empty region and therefore a spiral separating the path.

By this argument, it follows that cells attached to joint  in different repetitions of the subsequence  in  are separated by path edges of the spirals formed by the repeated subsequence of visited cells of the joints . Since four repetitions create four of such separated cells on , by Property~\ref{prop:four_sep_areas_PS} there exists a pair of cells that are not separable by a straight line but are separated by the path. Since the path of the spiral separating them consists only of cells on different joints, any possible separating polyline leads to a contradiction to the non-existence of a passage inside .
\end{proof}

\rephrase{Lemma}{\ref{lemma:closed_door_in_each_passage}}{
Each passage contains at least one closed door.
}

\begin{figure}[ht]
\begin{center}
\includegraphics[height=3.5cm]{./pictures/closed-in-passage.eps}
\caption{There exists a closed door in each passage.}
\label{fig:closed_door_in_passage}
\end{center}
\end{figure}
\begin{proof}
Refer to Fig.~\ref{fig:closed_door_in_passage}. Let  be a passage between , , and . Consider any vertex  of  inside the convex hull of . Further, consider all the triangles  created by  with any two vertices  such that  does not enclose any other vertex of . The path of tree edges connecting  to  enters one of the triangles. Then, either it leaves the triangle on the opposite side, thereby creating a closed door, or it encounters a vertex  of . Since at least one vertex of  lies outside the convex hull of , otherwise they would not be separated by , it is possible to repeat the argument on triangle  until a closed door is found.
\end{proof}

\rephrase{Lemma}{\ref{lemma:k_cluster_passage}}{
Given a set of joints , with , there exists a subset , with , such that for each pair of joints  there exist two cells  creating a passage with a cell .
}

\begin{proof}
By construction of the tree, for each set of four joints, there are formations that visit only cells of these joints. By Lemma~\ref{lemma:PS_passage}, there exists a passage inside each of these formations, which implies that for each set of four joints there exists a subset of two joints creating a passage.
The actual number of joints needed to ensure the existence of a subset of joints of size  such that passages exist between each pair of joints is given by the Ramsey Number . This number is defined as the minimal number of vertices of a graph  such that  either has a complete subgraph of size  or an independent set of size . Since in our case we can never have an independent set of size , we conclude that a subset of size  exists with the claimed property. The Ramsey number  is not exactly known, but we can use the upper bound directly extracted from the proof of the Ramsey theorem to arrive at the bound stated above. \cite{grs-rt-90}
\end{proof}

\rephrase{Lemma}{\ref{lemma:2_channels}}{
Consider a set of joints  such that there exists a passage between each pair , with . Let , for  and  and  be two sets of passages between pairs of joints in  (see Fig.~\ref{fig:lemma5}). Then, for at least  of the joints of one set of passages, say , there exist paths in \T, starting at the root and containing these joints, which traverse all the doors of  with at least 2 and at most 3 bends. Also, at least half of these joints create an -channel, with .
}

\begin{figure}[ht]
\begin{center}
\includegraphics[height=4cm]{./pictures/lemma5.eps}
\caption{The two sets of passages  and  described in Lemma~\ref{lemma:2_channels}.}
\label{fig:lemma5}
\end{center}
\end{figure}

\begin{proof}
Observe first that each passage of  is interconnected with each passage of  and that all the passages of  and all the passages of  are nested.

By Lemma~\ref{lemma:closed_door_in_each_passage} and Property~\ref{prop:three_bends}, for one of  and , say , either for every joint of  between the joints of  in the order around the root or for every joint of  not between the joints of , there exists a path  in \T, starting at the root and containing these joints, which has to traverse all the doors of  by making at least  and at most  bends. Also, paths  can be ordered in such a way that a bendpoint of  encloses a bendpoint of  for each .
It follows that there exist -channels with  for each joint.
Consider now the set of joints  visited by these paths.
We assume the joints of  to be in this order around the root.

\begin{figure}[htb]
  \centering{
\begin{tabular}{c c}
\includegraphics[height=4.5cm]{./pictures/lemma5-case1_m2.eps} &
\includegraphics[height=4.5cm]{./pictures/lemma5-case2_m2.eps} \\
(a) & (b)
\end{tabular}}
\caption{(a) The separating cell  is in the outermost channel. (b) The separating cell  is in the innermost channel.}
  \label{fig:lemma5-cases}
\end{figure}

Consider the path  whose bendpoint encloses the bendpoint of each of all the other paths and the path  whose bendpoint encloses the bendpoint of none of the other paths (see Figs.~\ref{fig:lemma5-cases}(a) and~\ref{fig:lemma5-cases}(b)).
Please note that either  visits  and  visits  or vice versa, say  visits .
By construction, there exists a passage between cells from  and cells from . In this passage there exist either two path-edges  of a cell  separating two cells , thereby crossing the channel of , or two edges of a cell  separating two cells , thereby crossing the channel of . We show that 1-channels are not sufficient to draw these passages.

In the first case (see Fig.~\ref{fig:lemma5-cases}(a)), both separating edges  cross the path  before and after the bend, thereby creating blocking cuts separating vertices of the same cell, say . Since they are connected by the path, by Property~\ref{prop:blocking-cut}, an additional bend is needed.
In the other case (see Fig.~\ref{fig:lemma5-cases}(b)), any edge connecting vertices of  is not even crossing any edge of  and therefore at least another bend is needed in the channel.
So at least one of the joints needs an additional bend. Since there are passages between each pair of joints in , all but one joint  have a path that has to bend an additional time. We note that the additional bendpoint of each path  aside from , , and  has to enclose all the additional bendpoints either of  or of . It follows that, for at least half of the joints, there exist -channels where .
\end{proof}

\rephrase{Lemma}{\ref{n-independent nestings}}{
There exist no  independent sets of formations  inside any extended formation,
where each  contains formations of a fixed set of channels of size .
}

\begin{proof}
Assume for a contradiction, that such independent sets  exist.
By Lemma~\ref{lemma:PS_passage}, each formation in each set will contain a passage and thereby an edge cutting the channel border.
By Property~\ref{prop:CS_1_2} each formation in each set will place an edge to either channel segment  or . As can be easily seen, there exists a set  of size  of sets of formations that will have at least one common connection for a fixed formation  in each set , where .
By repeating the argument we can find a subset  of size  such that these sets will have at least two common connections for formations  in each set . By continuing this procedure we arrive at a subset  of size  that will have at least  common connections. Since all these common connections have to connect to either  or , we have identified a set  of size  of sets of formations of size at least  that has all the connections to one specific channel segment .

We now consider the cutting edges for each of the formations of  in . Since any of those can intersect the channel border on two different sides, at least half of the connections for a fixed formation  in all the sets will intersect with one side of the channel border, thereby crossing either all the channels  or all the channels , assume the first. Consider now the formations  in each of the sets.
These formations of the sets  will be separated on  by the edges of the formations  of the sets . To avoid a monotonic ordering of the separated formations and thereby the existence of an region-level nonplanar tree these formations  have to place vertices in an adjacent channel segment . This will create blocking cuts for either all the channels  or all the channels , assume the first.
Consider now the formations  in each of the sets. These formations of the sets  will be separated on  by the edges of the formations  of the sets .
By the same argument as above also these formations have to place vertices in an adjacent channel segment that are visible from some of the separated areas of . Since the connection from the formations  are blocking for the connection to , the formations  have to use the remaining adjacent channel segment , thereby blocking all the channels . We finally consider the formations   of the sets . These formations are now separated in  by the blocking edges to  of the formations  and by the blocking edges to  of the formations . Therefore, these formations cannot use part of any channel segment (tree-)visible to the separated areas in . So, by Property~\ref{prop:four_sep_areas_PS}, we identified a region-level nonplanar tree, in contradiction to the assumption.
\end{proof}

\rephrase{Lemma}{\ref{lem:nest_sequence}}{
Consider four subsequences , where , of an extended formation , each consisting of a whole repetition of . Then, there exists either a pair of nested subsequences or a pair of independent subsequences.
}

\begin{proof}
Assume that no pair of nested subsequences exists. We show that a pair of independent subsequences exists.

First, we consider how  use the first two channel segments  and . Each of these subsequences uses either only , only , or both to place its formations. Observe that, if a subsequence uses only  and another one uses only , then such subsequences are clearly independent. So we can assume that all of  use a common channel segment, say .

\begin{figure}[ht]
\begin{center}
\includegraphics[height=4cm]{./pictures/nested-or-independent_m2.eps}
\caption{If three subsequences use the same channel segment , then at least two of them are either nesting or separated in .}
\label{fig:nested-or-independent}
\end{center}
\end{figure}

Then we show that, if three subsequences use the same channel segment , then at least two of them are separated in . In fact, if two subsequences using  are not independent, then they contain formations on the same channel  that intersect with different channel borders of . However, a third subsequence containing a formation that intersects a channel border of  is such that there exists either a nesting or a clear separation between this subsequence and the other subsequence intersecting the same channel border of  (see Fig.~\ref{fig:nested-or-independent}). This fact implies that if three subsequences use only , then at least two of them are independent. From this and from the fact that there are four subsequences using , we derive that two subsequences, say , are separated in  and are not separated in . Then, the third subsequence  can be placed in such a way that it is not separated from  and  in . However, this implies that  is separated in  from two of  and in  from two of , which implies that  is separated in both channel segments from one of .
\end{proof}

\rephrase{Lemma}{\ref{lemma:nest_independent}}{
Consider an extended formation .
Then, there exists a -nesting, where , among the formations of .
}

\begin{proof}
Assume, for a contradiction, that there is no -nesting among the sequence of formations in . We claim that, under this assumption, there exist more than  sequences of independent formations in  from the same set of channels , where  and . By Lemma~\ref{n-independent nestings}, such a claim clearly implies the statement.

Consider sequences that use some common channels in channel segments  and . Then, their separation in  has the opposite ordering with respect to their separation in .

Observe that, by Lemma~\ref{lem:nest_sequence}, there exist at most  different nestings of subsequences such that there are less than  independent sets of subsequences.
Also note that, if some formations belonging to two different subsequences are nesting, then all the formations of these subsequences have to be part of some nesting. However, this does not necessarily mean for all the formations to nest with each other and to build a single nesting.

Since the number of channels used inside  is greater than , where , we have a nesting consisting of subsequences with at least  different defects.

Let the nesting consist of subsequences , where  denotes the -th occurrence of a subsequence of  with a defect at channel . Further, let the path connect them in the order .
We show that there exists a pair of independent subsequences within this nesting.

\begin{figure}[ht]
\begin{center}
\begin{tabular}{c c c}
\mbox{\includegraphics[height=2.7cm]{./pictures/2repetitions_in.eps}} &
\mbox{\includegraphics[height=2.7cm]{./pictures/2repetitions_out.eps}} &
\mbox{\includegraphics[height=2.7cm]{./pictures/repetitions_out.eps}} \\
(a) & (b) & (c)\\
\end{tabular}
\caption{(a) and (b) Possible configurations for , , and . (c) The repetitions follow the outward orientation.}
\label{fig:2repetitions}
\end{center}
\end{figure}


\begin{figure}[ht]
\begin{center}
\begin{tabular}{c c}
\mbox{\includegraphics[height=3cm]{./pictures/defect-a.eps}} \hspace{0.2cm} &
\mbox{\includegraphics[height=3cm]{./pictures/defect-b.eps}} \\
(a) & (b)\\
\end{tabular}
\caption{The connection between channels  and  blocks visibility for the following repetitions to the part of the channel segment where vertices of channel  were placed till that repetition.}
\label{fig:defects}
\end{center}
\end{figure}


\begin{figure}[ht]
\begin{center}
\begin{tabular}{c}
\mbox{\includegraphics[height=4cm]{./pictures/defect-c.eps}} \\
(a)\\
\end{tabular}
\caption{All the channels  are shifted and the next repetition starts in a completely different region.}
\label{fig:global-shift}
\end{center}
\end{figure}

Consider now the first two nesting repetitions of sequence , that is,  and . Let the nesting consist of a formation  from  nesting in a formation  from .
Consider the edges  and  that are responsible for the nesting.
Without loss of generality we assume the path  that connects  and  not to contain edges .
Consider the two parts  of the channel border of , where  is between  and  and  is between   and . Consider now the closed region delimited by the path through , the path , the path through , and . Such a region is split into two closed regions  and  by  (see Fig.~\ref{fig:nesting_formations}).

\begin{figure}[htb]
  \centering{
\begin{tabular}{c}
\includegraphics[height=3.5cm]{./pictures/nesting-regions.eps}
\end{tabular}}
\caption{Regions  and .}\label{fig:nesting_formations}
\end{figure}

Observe that, in order to reach from  to the outer region, any path has to cross both  and .
We note that the part of \P starting at  and not containing  is either completely contained in the outer region or has to cross over between  and the outer region by traversing . Similarly, the
the part of \P starting at  and not containing  either does not reach the outer region or has to cross over between  and the outer region by traversing . Furthermore, any formation  on such a path is also either crossing over and thereby cutting both  and , or not. In the first case  is nested in  and  is nested in .

Consider now the third nesting repetition  of sequence  (see Figs.~\ref{fig:2repetitions}(a) and~\ref{fig:2repetitions}(b)).
It is easy to see that if  is nested between  and , then there exists a nesting of depth  because  contains a defect at a different channel. So we have to consider the cases when the repetitions create the nesting by strictly going either outward or inward. By this we mean that the -th repetition  has to be placed such that either  is nested inside  (inward) or vice versa (outward).
Without loss of generality, we assume the latter (see Fig.~\ref{fig:2repetitions}(c)).

Consider now a defect in a channel , with , at a certain repetition . Since the path is moving outward, the connection between channels  and  blocks visibility for the following repetitions to the part of the channel segment where vertices of channel  were placed till that repetition (see Fig.~\ref{fig:defects}(a) for an example with ).

A possible placement for the vertices of  in the following repetitions that does not increase the depth of the nesting could be in the same part of the channel segment where vertices of a channel , with , were placed till that repetition. We call \emph{shift} such a move. However, in order to place vertices of  and of  in the same zone, all the vertices of  belonging to the current cell have to be placed there (see dashed lines in Fig.~\ref{fig:defects}(b), where ), which implies that a further defect in channel  at one of the following repetitions encloses all the vertices of each of the previously drawn cells, hence separating them with a straight line from the following cells. Hence, also the vertices of  have to perform a shift to a channel , with . Again, if the vertices of  and of  lie in the same zone, we have two cells that are separated by a straight line and hence also the vertices of  have to perform a shift. By repeating such an argument we conclude that the only possibility for not having vertices of different channels lying in the same zone is to shift all the channels  and to go back to channel  for starting the following repetition in a completely different region (see Fig.~\ref{fig:global-shift}, where the following repetition is performed completely below the previous one). However, this implies that there exist two repetitions in one configuration that have to be separated by a straight line and therefore are independent, in contradiction to our assumption.
Therefore, we can assume that, after  repetitions, we arrive at a nesting of depth 1. By repeating this argument we arrive after  repetitions at the nesting of depth  claimed in the lemma.
\end{proof}

\rephrase{Lemma}{\ref{lemma:k-nesting}}{
If an extended formation lies in a part of the channel that contains only side connections, then \T and \P do not admit any geometric simultaneous embedding.
}

\begin{proof}
First observe that, by Lemma~\ref{lemma:nest_independent}, there exists a -nesting with  in any extended formation .

\begin{figure}[ht]
\begin{center}
\includegraphics[height=4cm]{./pictures/1-side-nested_m.eps}
\caption{Illustration for the case in which only -side connections are possible.}
\label{fig:1-side-nested}
\end{center}
\end{figure}

Consider two nested formations  belonging to the -nesting. Such formations, by definition, belong to the same channel. Consider now the formation  belonging to a different channel such that  is nested in  and  is nested in . Since each pair of channel segments have a side connection, we have that  blocks visibility for  on the channel segment used by  for the nesting (see Fig.~\ref{fig:1-side-nested}). Hence,  has to use a different channel segment to perform its nesting, which increases by one the number of used channel segments for each level of nesting. Since the tree supports at most  channel segments, the statement follows.
\end{proof}

\rephrase{Lemma}{\ref{lem:2-different-shapes}}{
Consider two channels  with the same intersections. Then, none of channels , where , have an intersection that is disjoint with the intersections of  and of .
}

\begin{proof}
The statement follows from the fact that the channel borders of  and  delimit the channel for all joints between  and . So, if any channel , with , had an intersection different from the one of  and , it would either intersect with one of the channel borders of  or  or it would have to bend around one of the channel borders, hence crossing a straight line twice.
\end{proof}

\rephrase{Lemma}{\ref{lem:nesting-bending-area}}{
Consider an -nesting of a sequence of extended formations on an intersection , with .
Then, there exists a triangle  in the nesting that separates some of the triangles nesting with  from the bending area  (or ).
}

\begin{proof}
Consider three extended formations  lying in a channel  and two extended formations  lying in a channel  such that all the channels of the sequence of extended formations are between  and  and there is no formation  nesting between  and . Suppose, without loss of generality, that the bending point of  is enclosed into the bending point of .

Consider a formation  nesting with a formation . We have that the connections from  to channel segment  and back has to go around the vertex placed by  on channel segment . Therefore, at least one of the connections of  cuts all the channels between  and , that is, all the channels where the sequence of extended formations is placed. Such a connection separates the vertices of  from the vertices of a formation  on channel segment . Therefore, at least one of the connections of  to channel segment  cuts either all the channels in channel segment  or all the channels in channel segment  (or ), hence becoming a blocking cut for such channels. It follows that all the formations nesting inside   on such channels can not place vertices in the bending area  (or ) outside .
\end{proof}

\rephrase{Lemma}{\ref{lem:one_channel_segment}}{
In a situation as described in Proposition~\ref{prop:triangle}, not all the extended formations in a sequence of extended formations can place turning vertices in the same channel segment.
}

\begin{proof}
Assume, for a contradiction, that all the turning vertices are in the same channel segment. Consider a sequence of extended formations  and the extended formations in  using one of the sets of channels .

We first show that in  there exist some extended formations using connections in  configuration and some using connections in  configuration on the channels . Consider the continuous subsequence of extended formations   in . Assume that all the turning vertices of these extended formations are in  configuration. Consider a further subsequence of  on the same set of channels with a defect at . Then, the connection between  and  crosses , thereby blocking any further  from being in  configuration. Therefore, when considering another subsequence of  on the same set of channels which does not contain defects at , either the extended formation  is in  configuration or it uses another channel segment to place the turning vertex, as stated in the lemma.

So, consider two channels  such that there exists an extended formation  in  configuration and an extended formation  in  configuration. Since all the extended formations contain a triangle open on one side that is nested with triangle , we consider five of such triangles, one for each set of channels  and two for set , such that four of the considered extended formations   are continuous in  and the other one  is the first extended formation on the set of channels  following  in .

\begin{figure}[ht]
\begin{center}
\begin{tabular}{c c}
\mbox{\includegraphics[height=5cm]{./pictures/nested-triangles-prop-3_m2.eps}} \hspace{0.1cm} &
\mbox{\includegraphics[height=4.5cm]{./pictures/nested-triangles-defect_m2.eps}} \\
(a) & (b)\\
\end{tabular}
\caption{(a) Two triangles from the same channel have to use different channel segments if a triangle of another channel is between them. Turning vertices are represented by black circles. (b) When a defect at  in encountered, the connection between  and  does not permit the following  to respect the ordering of triangles.}
\label{fig:nested-triangles-prop-3}
\end{center}
\end{figure}

Notice that, if a triangle of an extended formation  is nested in a triangle of an extended formation  and the triangle of  is nested inside a triangle of an extended formation , with , then  has to use a different channel segment to place its turning vertex (see Fig.~\ref{fig:nested-triangles-prop-3}(a)). Hence, the triangles have to be ordered according to the order of the used channels. Also, if the continuous path connecting two triangles  of consecutive extended formations  connects vertex  to vertex  (or  to ) via the outer area, then a triangle of  that occurs prior to  and a triangle of  that occurs after  are nested with the triangle given by the connection of  and  in an ordering that is different from the order of the channels.

Consider now the following subsequence of  having a defect at . The connection of  to  in this subsequence blocks access for the following  to the area where it would have to place vertices in order to respect the ordering of triangles (see Fig.~\ref{fig:nested-triangles-prop-3}(b)).
Therefore, after 3 full repetitions of the sequence in , at least one extended formation has to use a different channel segment to place its turning vertex.
\end{proof}

\rephrase{Lemma}{\ref{lem:prop3-nodrawing}}{
In a situation as described in Proposition~\ref{prop:triangle}, \T and \P do not admit any geometric simultaneous embedding.
}

\begin{proof}
Consider two extended formations  that are consecutive in .
First note that the connection between  and  cuts all channels  in either channel segment  or .
Since both of these extended formations are also connected to the bending area between channel segments  and , it is not possible for an extended formation , with , to connect from vertices above the connection between  and  to vertices below it by following a path to the bending area.
Note, further, that if all the extended formations , with , are in the channel
segment that is not cut by the connection between  and , then a connection is needed from  to  in channel . However, by Lemma~\ref{lem:one_channel_segment}, after three defects in the subsequence of  it is no longer possible for some extended formation , with , to place its turning vertex in the same channel segment. Therefore, different channel segments have to be used by the extended formation , with . However, since the path is continuous and since the connection between  and  is repeated after a certain number of steps, we can follow that the path creates a spiral. Also, we note that, in order to respect the order of the sequence, it will be impossible for the path to reverse the direction of the spiral. Hence, once a direction of the spiral has been chosen, either inward or outward, all the connections in the remaining part of the sequence have to follow the same.
This implies that, if a connection between  and  changes channel segment, that is, it is performed in a different channel segment than the one between  and , then all the connections of this type have to change. However, when a defect at channel  is encountered, also the connection between  and  has to change channel segment, thereby making impossible for any future connection between  to  to change channel segment. Therefore, after a whole repetition of the sequence of  containing defects at each channel, all the extended formations have to place their turning vertices in the same channel segment, which is not possible, by Lemma~\ref{lem:one_channel_segment}. This concludes the proof that no valid drawing can be achieved in this configuration.
\end{proof}

\rephrase{Lemma}{\ref{lem:intersects_one_three}}{
If a shape contains an intersection  and does not contain any other intersection that is disjoint with , then \T and \P do not admit any geometric simultaneous embedding.
}

\begin{proof}
First observe that only the intersections  and  are not disjoint with  and could occur at the same time as . By Lemma~\ref{lemma:nest_independent}, there exists at least a nesting greater than, or equal to, 6. Each of such nestings has to take place either at intersections ,  or at . Remind that, by Property~\ref{prop:CS_1_2}, -vertices can only be placed in  or . Also, the sorting of head vertices to avoid a region-level nonplanar trees can only be done by placing vertices into  or . This implies that the stabilizers have to be placed in  or . Note that the stabilizers also work as -vertices in the tails of other cells. This means that if there exist seven sets of tails that can be separated by straight lines, then there exist a region-level nonplanar tree, by Lemma~\ref{n-independent nestings}.
Observe that, by nesting them according to the sequence, the previous condition would be fulfilled.
This means that we have either a sorting or other nestings. We first show that there exist at most two -nestings with . Every -nesting has to take place at either ,  or . We assume, w.l.o.g., to have to deal with the greatest possible number of intersections.

Consider the case  (see Fig~\ref{fig:1324}(a)). Observe that intersections  and  are either both high or both low and use channel segment . Also, every connection from  to  cuts either  or  and, if one of these connections cuts , then every nesting cutting  closer to  has to cut . Hence, we can consider all the connections to  as connections to  or . Also, since any connection cutting a channel segment is more restrictive than a connection inside the same channel segment, such two nestings can be considered as one. Finally, since such a nesting connects to , it is not possible to have at the same time a nesting taking place at . Hence, we conclude that only one nesting is possible in this case.

\begin{figure}[ht]
\begin{center}
\begin{tabular}{c c}
\mbox{\includegraphics[height=4.5cm]{./pictures/1324h.eps}} \hspace{0.1cm} &
\mbox{\includegraphics[height=3cm]{./pictures/1324l.eps}} \\
(a) & (b)\\
\end{tabular}
\caption{(a) Case  . (b) Case  .}
\label{fig:1324}
\end{center}
\end{figure}

Consider the case  (see Fig~\ref{fig:1324}(b)). Observe that -vertices can be placed at most in  and -vertices can be placed at most in . This means that the extended formations in every nesting have to visit these vertices. Therefore, if there exists both a nesting at  and at , then the connections to the 1- and 2-vertices in the bending areas  and  are such that every EF nesting at  makes a nesting with the extended formations nesting at . Hence, also in this case only one nesting is possible.

So we consider the unique nesting of depth  and we show that any way of sorting the nesting formations in the channels will cause separated cells, hence proving the existence of a nonplanar region-level tree.
Consider four consecutive repetitions of the sequence of formations. It is clear that these formations are visiting areas of  and are separated by previously placed formations from other formations on the same channels. This will result in some cells to become separated in . Since, by Property~\ref{prop:four_sep_areas_PS}, the number of monotonically separated cells in  cannot be larger than , for any set of four such separated formations there exists a pair of formations  that change their order in . These connections have to be made on either side of the nesting. If between this pair of formations there is a formation of a different channel, then this formation has to choose the other side to reorder with a formation outside . We further note that, if there are two such connections  and  on the same side that are connecting formations of one channel, nested in the order , and another connection on the same side between  such that  is nested between  and  between , then this creates a 1-nesting. In the following we show that a nesting of depth at least 6 is reached.

Assume the repetitions of formations in the extended formation to be placed in the order .
If this order is not coherent with the order in which the channels appear in the sequence of formations inside the , then we have already some connections that are closing either side of the nesting for some formations. So we assume them to be in the order given by the sequence. Then, consider a repetition of formations with a defect at some channel . We have that there exists a connection closing off at one side all the previously placed formations of . However, there are sequences with defects also at channels  and , which can not be realized on the same side as the defects at . We generalize this to the fact that all the defects at odd channels are to one side, while the defects at even channels are to the other side.
Since the path is continuous and has to reach from the last formation in a sequence again to the first one, the continuation of the path can only use either the odd or the even defects. This implies that, when considering three further repetitions of formations, the first and the third having a defect at a channel  and the second having no defect at , there will be a nesting of depth one between these three formations. Since, by Lemma~\ref{lemma:k-nesting}, there cannot be a nesting of depth greater than 5 at this place, we conclude that after 6 repetitions of such a triple of formations there will be at least two formations that are separated from each other. By repeating this argument we arrive after  repetitions at either the existence of 7 formations that are separated on  and  or at the existence of a nesting of depth 6, both of which will not be drawable without the aid of another intersection that is able to support the second nesting of depth greater than 5.
\end{proof}

\rephrase{Lemma}{\ref{lem:intersects_three_one}}{
If there exists a sequence of extended formation in any shape containing an intersection , then \T and \P do not admit any geometric simultaneous embedding.
}

\begin{proof}
Consider a sequence of extended formation in a shape containing an intersection . We show that \T and \P do not admit any geometric simultaneous embedding. Observe that there exist several possibilities for channel segment  to be placed. Either there exists no intersection of an elongation of one channel segment with another channel segment or there exists at least one of the intersections , ,  or .
If there are more than one of such intersections, then it is possible to have several nestings of depth .
We note that, if there exists the intersection , then at least one of , , and  are part of the convex hull (see Fig.~\ref{fig:non-convex}).

\begin{figure}[ht]
\begin{center}
\begin{tabular}{c c}
\mbox{\includegraphics[height=3cm]{./pictures/4-non-convex-a.eps}} \hspace{0.1cm} &
\mbox{\includegraphics[height=3cm]{./pictures/4-non-convex-b.eps}} \\
(a) & (b)\\
\end{tabular}
\caption{If channel segment four is not part of the convex hull then either  or  is part of the convex hull. (a) Case . (b) Case .}
\label{fig:non-convex}
\end{center}
\end{figure}

First, we show that there exists a nesting at .

Consider case . We have that  is on the convex hull restricted to the first three channel segments and  can force at most one of  or  out of the convex hull. Hence, one of them is part of the convex hull. We distinguish the two cases.

Suppose  to be part of the convex hull. Assume there exists a nesting at . From  the only possible connection without a -side connection is the one to , which, however, is on the convex hull. Hence, an argument analogous to the one used in Lemma~\ref{lem:intersects_one_three} proves that the nesting at  has size smaller than , which implies that the rest of the nesting has to take place at .

Suppose  to be part of the convex hull. Assume that there exists a nesting at . Every connection from  has to be either to  or to , by Property~\ref{prop:CS_1_2}. Since  is already part of the nesting, we have connections to . However,  is on the convex hull, hence allowing only -side connections. Therefore, an argument analogous to the one used in Lemma~\ref{lem:intersects_one_three} proves that the nesting at  has size smaller than , which implies the rest of the nesting has to take place at .

Consider case . Since  is not part of the convex hull, either  or  are. If  is on the convex hull, then the same argument as before holds, while if  is on the convex hull, then no reordering is possible.

Clearly, if there is no intersection other than , a nesting in the intersection  has to be performed.

Hence, we conclude that a nesting with a depth of  in every extended formation has to take place at  (or at , which can be considered as the same case).

By Lemma~\ref{lem:nesting-bending-area}, the nesting in the bending area is limited. Every extended formation  which has at least one vertex either in  or in  has a vertex in the bending area. Consider a sequence of extended formations  which uses only channels in this particular shape. It's obvious that all of these  in  have to do a nesting at . Observe that there exist two consecutive edges which are forming a triangle with , , and  by simply placing vertices inside the channel segments. Since every EF creates such triangles, there exists a triangle which is not in the bending area and such that there exists no other triangle
between the bending area and this triangle. This triangle is separating the nesting area from the bending area in all but  extended formations. However, since every EF has to use both of such areas, the inner area of  (or ) has to connect to the outer area of  (or ). If  is on the convex hull, then there exist only -sided connections, which implies the statement, by Lemma~\ref{lem:prop3-nodrawing}.
On the other hand, if  is not on the convex hull, then there exists  and  can be also used to perform connections from the inner to the outer area. However, since  is on the convex hull, such connections are only -side. Hence, by Lemma~\ref{lem:prop3-nodrawing}, the statement follows.
\end{proof}

\rephrase{Lemma}{\ref{lemma:no-ordered-double-cuts}}{
Let  and  be two consecutive channel segments. If there exists an ordered set  of extremal double cuts cutting  and  such that the order of the intersections of the double cuts with  (with ) is coherent with the order of , then \T and \P do not admit any geometric simultaneous embedding.
}

\begin{proof}
Suppose, for a contradiction, that such a set  exists. Assume first that  and  are such that the bendpoint of channel  encloses the bendpoint of all the other channels. Hence, any edge creating a double cut at a channel  has to cut all the channels  with , either in  or in . Refer to Fig.~\ref{fig:no-ordered-double-cuts}.

Consider the first repetition . Let  be an edge creating a double cut at channel . Assume, without loss of generality, that  cuts channel segment . Observe that, for channel , the visibility constraints determined in channels  in  and in  by the double cut created by  do not depend on whether it is simple or non-simple. Indeed, by Property~\ref{prop:double-cut}, edge  blocks visibility to  for the part of  where edges creating double cuts at channels  following  in  have to place their end-vertices.

\begin{figure}[ht]
\begin{center}
\begin{tabular}{cc}
\mbox{\includegraphics[height=4cm]{./pictures/no-ordered-double-cuts-2_m2.eps}} \hspace{0.1cm} &
\mbox{\includegraphics[height=4cm]{./pictures/no-ordered-double-cuts_m2.eps}} \\
(a) & (b)\\
\end{tabular}
\caption{Proof of Lemma~\ref{lemma:no-ordered-double-cuts}. (a)  cuts . (b)  cuts .}\label{fig:no-ordered-double-cuts}
\end{center}
\end{figure}

Then, consider an edge  creating a double cut at channel  in the first repetition of .

If  cuts  (see Fig.~\ref{fig:no-ordered-double-cuts}(a)), then it has to create either a non-simple double cut or a simple one. However, in the latter case, an edge  between  and  in channel , which creates a blocking cut in channel , is needed. Hence, in both cases, channel  is cut both in  and in , either by  or by . It follows that an edge  creating a double cut at channel  in the second repetition of  has to cut , hence blocking visibility to  for the part of  where edges creating double cuts at channels  following it in  have to place their end-vertices, by Property~\ref{prop:double-cut}. Further, consider an edge  creating a double cut at channel  in the second repetition of . Since visibility to  is blocked by  and  in  and by  in ,  has to create a non-simple double cut (or a simple one plus a blocking cut), hence cutting channel  both in  and in . It follows that, by Property~\ref{prop:blocking-cut}, an edge  creating a double cut at channel  in the third repetition of  can place its end-vertex neither in  nor in .

If  cuts  (see Fig.~\ref{fig:no-ordered-double-cuts}(b)), then it has to create a simple double cut. Again, by Property~\ref{prop:double-cut}, edge  blocks visibility to  for the part of  where edges creating double cuts following  in  have to place their end-vertices. Hence, an edge  creating a double cut at channel  in the first repetition of  cannot create a simple double cut, since its visibility to  is blocked by  in  and by  in . This implies that  creates a non-simple double cut (or a simple one plus a blocking cut) at channel , cutting either  or , hence cutting channel  both in  and in . It follows that, by Property~\ref{prop:blocking-cut}, an edge  creating a double cut at channel  in the second repetition of  can place its end-vertex neither in  nor in .

The case in which  and  are such that the bendpoint of  encloses the bendpoint of all the other channels can be proved analogously. Namely, the same argumentation holds with channel  playing the role of channel , channel  playing the role of channel , channel  having the same role as before, channel  playing the role of channel , and channel  playing the role of channel . Observe that, in order to obtain the needed ordering in this setting,  repetitions of  are needed. In fact, we consider channel  in the first repetition, channels  and  in the second one, and channels  and  in the third one.
\end{proof}

\rephrase{Lemma}{\ref{lem:double_cuts_13}}{
Each extended formation in shape   creates double cuts in at least one bending area.
}

\begin{proof}
Refer to Fig.~\ref{fig:13high-41high}(a). Assume, without loss of generality, that the first bendpoint of channel  encloses the first bendpoint of all the other channels. This implies that the second and the third bendpoints of channel  are enclosed by the second and the third bendpoints of all the other channels, respectively.

\begin{figure}[ht]
\begin{center}
\begin{tabular}{cc}
\mbox{\includegraphics[height=4cm]{./pictures/13h41h_m2.eps}} \hspace{0.1cm} &
\mbox{\includegraphics[height=4cm]{./pictures/13h42hl_m2.eps}} \\
(a) & (b)\\
\end{tabular}
\caption{(a) Shape   has to connect at least one bend with double cuts. (b) Shape   has to connect bend  with double cuts.}\label{fig:13high-41high}
\end{center}
\end{figure}

Suppose, for a contradiction, that there exists no double cut in  and in . Hence, any edge  connecting to  (to ) is such that  and its elongation cut each channel once. Consider an edge connecting to  in a channel . Such an edge creates a triangle together with channel segments  and  of channel  which encloses the bending areas  of all the the channels  with  by cutting such channels twice. Hence, a connection to such a bending area in one of these channels has to be performed from outside the triangle. However, since in shape   both the bending areas  and  are on the convex hull, this is only possible with a double cut, which contradicts the hypothesis.
\end{proof}

\rephrase{Lemma}{\ref{lem:ordered-set-of-double-cuts-exists}}{
Every sequence of extending formations in shape   contains an ordered set  of extremal double cuts with respect to bending area either  or .
}

\begin{proof}
Shape   is similar to shape  , depicted in Fig.~\ref{fig:13high-41high}(a), with the only difference on the slope of channel segment , which is such that its elongation crosses channel segment  and not channel segment . Shape   is depicted in Fig.~\ref{fig:13high-41high}(b).

Assume, without loss of generality, that the first bendpoint of channel  is enclosed by the first bendpoint of all the other channels. This implies that the second bendpoint of channel  encloses the second bendpoint of all the other channels.

First observe that bending area  is on the convex hull, both in shape   and in shape  .

Also, observe that all the extended formations have some vertices in  and in , and hence all the extended formations have to reach such vertices with path-edges.

In shape  , by Lemma~\ref{lem:double_cuts_13}, there exist double cuts either in  or in , while in shape   there exist double cuts in , since the only possible connections to  are from channel segments  and , which are both creating double cuts (see Fig.~\ref{fig:13high-41high}(b)). Hence, we consider the extremal double cuts of each extended formation with respect to one of  or , say .

\begin{figure}[htb]
\begin{center}
\begin{tabular}{cc}
\mbox{\includegraphics[height=5.5cm]{./pictures/1342-corner-12345.eps}} \hspace{0.1cm} &
\mbox{\includegraphics[height=5.5cm]{./pictures/1342-corner-cs3.eps}} \\
(a) & (b)\\
\mbox{\includegraphics[height=5.5cm]{./pictures/1342-corner-cs4.eps}} \hspace{0.1cm} &
\mbox{\includegraphics[height=5.5cm]{./pictures/1342-corner-spiral.eps}} \\
(c) & (d)\\
\end{tabular}
\caption{(a) The ordering of the extremal double cuts is . (b) and (c) When a double defect is encountered, the connection between channels  and  cannot be performed in the same area as the connection between channels  and  and between channels  and  was performed in the previous repetition: (b) The connection is performed in the same area as the connection between channels  and  was performed. (c) The connection is performed in channel segment . (d) If channel segment four is used to spiral, the considered double cut was not extremal.}\label{fig:bend23-no-ordered}
\end{center}
\end{figure}

\begin{figure}[htb]
\begin{center}
\begin{tabular}{cc}
\mbox{\includegraphics[height=5.5cm]{./pictures/1342-repetitions.eps}} \hspace{0.1cm} &
\mbox{\includegraphics[height=5.5cm]{./pictures/1342-repetitions-2.eps}} \\
(a) & (b)\\
\mbox{\includegraphics[height=5.5cm]{./pictures/1342-repetitions-3.eps}} \hspace{0.1cm} &
\mbox{\includegraphics[height=5.5cm]{./pictures/1342-repetitions-4.eps}} \\
(c) & (d)\\
\end{tabular}
\caption{(a) A repetition with a double defect in channel  is considered. (b) A repetition with a double defect in channel  is considered. (c) A repetition without any double defect in channels  is considered. (d) An ordered set  is obtained.}\label{fig:1342-repetitions}
\end{center}
\end{figure}

Consider two sets of extended formations creating double cuts in  at channels , respectively. Observe that the extended formations in these two sets could be placed in such a way that the ordering of their extremal double cuts is . The same holds for the following occurrences of extended formations creating double cuts in  at channels , respectively. Clearly, in this way an ordering  could be achieved and hence an ordered set  of double cuts would be never obtained (see Fig.~\ref{fig:bend23-no-ordered}(a)).

However, every repetition of extended formations inside a sequence of extended formations contains a double defect at some channel. We show, with an argument similar to the one used in Lemma~\ref{lemma:nest_independent}, that the presence of such double defects determines an ordering  of extremal double cuts after a certain number of repetitions of extended formations inside a sequence of extended formations. Namely, consider a double defect at channel  in a certain repetition. The connection between channels  and  cannot be performed in the same area as the connection between channels  and  and between channels  and  was performed in the previous repetition. Hence, such a connection has to be performed either in the same area as the connection between channels  and  was performed (see Fig.~\ref{fig:bend23-no-ordered}(b)), or in channel segment  (this is only possible in shape  , see Fig.~\ref{fig:bend23-no-ordered}(c)).
Observe that, going to channel segment  to make the connection, then to channel segment , and finally back to , hence creating a spiral, implies that the considered double cut is not extremal (see Fig.~\ref{fig:bend23-no-ordered}(d)). Therefore, the only possibility to consider when channel segment  is used is to make the connection between channels  and  there and then to come back to  with a double cut. Hence, independently on whether channel segment  is used or not, the connection between channels  and  blocks visibility for the following repetitions to the areas where the connections between some channels were performed in the previous repetition. This implies that the ordering  of extremal double cuts cannot be respected in the following repetitions. In fact, a partial order  is obtained in a repetition of formations creating extremal double cuts at channels . Also, when two different double defects having a channel in common are considered, the effect of such defects is combined. Namely, consider a double defect at channel  in a certain repetition. The connection between channels  and  blocks visibility to the areas where the connection between  and  and between  and  were performed at the previous repetitions (see Fig.~\ref{fig:1342-repetitions}(a)). Then, consider a double defect at channel  in a following repetition. We have that the connection between channels  and  can not be performed where the connection between  and  was performed in the previous repetitions, since such an area is blocked by the presence of the connection between channels  and . Hence, a double cut at channel  has to be placed after the double cut at channel  created in the previous repetition (see Fig.~\ref{fig:1342-repetitions}(b)). Consider now a further repetition with a defect not involving any of channels . We have that the area where the connection from  to  was performed in the previous repetitions is blocked by the connection between  and  and hence a double cut at channel  has to be placed after the double cut at channel  created in the previous repetition, which, in its turn, was created after the double cut at channel  (see Fig.~\ref{fig:1342-repetitions}(c)).

\clearpage

Also, all the double cuts at channels  have to be placed after the double cut at , and hence a shift of the whole sequence  after the double cut at  is performed and an ordered set  is obtained (see Fig.~\ref{fig:1342-repetitions}(d)). Observe that at most two sets of repetitions of extended formation inside a sequence of extended formations such that each set contains a double defect at each channel are needed to obtain such a shift. By repeating such an argument we obtain another shifting of the whole sequence , which results in the desired ordered set . We have that a set of repetitions of extended formation containing a double defect at each channel is needed to obtain the first sequence , then two of such sets are needed to get to , and two more are needed to get to , which proves the statement.

Observe that, if it were possible to partition the defects into two sets such that there exists no pair of defects involving a common channel inside the same set, then such sets could be independently drawn inside two different areas and the effects of the defects could not be combined to obtain . However, since each double defect involves two consecutive channels, at least three sets are needed to obtain a partition with such a property. In that case, however, an ordered set  could be obtained by simply considering a repetition of  in each of the sets.
\end{proof}

\rephrase{Lemma}{\ref{lem:cs_two_convex_hull}}{
If channel segment  is part of the convex hull, then \T and \P do not admit any geometric simultaneous embedding.
}

\begin{proof}
First observe that, with an argument analogous to the one used in Lemma~\ref{lem:intersects_one_three}, it is possible to show that there exists a nesting at intersection . Then, by Property~\ref{prop:CS_1_2}, every vertex that is placed in  is connected to two vertices that are placed either in  or in . Hence, the continuous path connecting to a vertex placed in  creates a triangle, having one corner in  and two corners either in  or in its elongation, which cuts  into two parts, the inner and the outer area.

By Lemma~\ref{lem:nesting-bending-area}, not all of these triangles can be placed in the bending area . Hence, every extended formation, starting from the second of the sequence, have to place their vertices in both the inner and the outer area of the triangle created by the first one.

Observe that, in order to connect the inner to the outer area, the extended formations can only use -side connections. Namely,  creates a -side connection. Channel segment  is on the convex hull. Since, by Property~\ref{prop:CS_1_2}, every vertex that is placed in  is connected to two vertices that are placed either in  or in , also  creates a -side connection.

From this we conclude that in this configuration the preconditions of Proposition~\ref{prop:triangle} are satisfied, and hence the statement follows.
\end{proof}

\section{An Algorithm for the Geometric Simultaneous Embedding of a Tree of Depth  and a Path}\label{se:depth2}

In this section we describe an algorithm for constructing a geometric simultaneous embedding of any tree \T of depth  and any path \P. Refer to Fig.~\ref{fig:pseudo}.

Start by drawing the root  of  on the origin in a coordinate system. Choose a ray  emanating from the origin and entering the first quadrant, and a ray  emanating from the origin and entering the fourth quadrant.
Consider the wedge  delimited by  and  and containing the positive -axis. Split  into  wedges , in this clockwise order around the origin, where  is the number of vertices adjacent to  in , by emanating  equispaced rays from the origin.

Then, consider the two subpaths  and  of  starting at . Assign an orientation to  and  such that the two edges  and  incident to  in  are exiting .

Finally, consider the  subtrees  of  rooted at a node adjacent to , such that  and .

\begin{figure}[h]
\begin{center}
\includegraphics[height=7.5cm]{./pictures/pseudomonoton_m.eps}
\caption{A tree with depth two and a path always admit a geometric simultaneous embedding.}
\label{fig:pseudo}
\end{center}
\end{figure}

The vertices of each subtree  are drawn inside wedge , in such a way that:
\begin{enumerate}
\item vertex  is the vertex with the lowest -coordinate in the drawing, except for ;
\item vertices belonging to  are placed in increasing order of -coordinate according to the orientation of ;
\item vertex  is the vertex with the highest -coordinate in the drawing;
\item vertices belonging to  are placed in decreasing order of -coordinate according to the orientation of , in such a way that the leftmost vertex of  is to the right of the rightmost vertex of ; and
\item no vertex is placed below segment .
\end{enumerate}

Since  has depth , each subtree , with , is a star. Hence, it can be drawn inside its own wedge  without creating any intersection among tree-edges. Observe that the same holds even for subtree , where the wedge to consider is the part of  above segment .

Since  and  are drawn in monotonic order of -coordinate and are separated from each other, and edge  connecting such two paths is on the convex hull of the point-set, no intersection among path-edges is created.

From the discussion above, we have the following theorem.

\begin{theorem}\label{th:depth2}
A tree of depth  and a path always admit a geometric simultaneous embedding.
\end{theorem}

\section{Conclusions}\label{se:conclusions}

In this paper we have shown that there exist a tree \T and a path \P on the same set of vertices that do not admit any geometric simultaneous embedding, which means that there exists no set of points in the plane allowing a planar embedding of both \T and \P. We obtained this result by extending the concept of level nonplanar trees~\cite{fk-mlnpt-07} to the one of region-level nonplanar trees. Namely, we showed that there exist trees that do not admit any planar embedding if the vertices are forced to lie inside particularly defined regions. Then, we constructed \T and \P so that the path creates these particular regions and at least one of the many region-level nonplanar trees composing  has its vertices forced to lie inside them in the desired order. Observe that our result also implies that there exist two edge-disjoint trees that do not admit any geometric simultaneous embedding, which answers an open question posed in~\cite{gkv-ttsids-09}, where the case of two non-edge-disjoint trees was solved.

It is important to note that, even if our counterexample consists of a huge number of vertices, it can also be considered as ``simple'', in the sense that the depth of the tree is just . In this direction, we proved that, if the tree has depth , then it admits a geometric simultaneous embedding with any path. This gives raise to an intriguing open question about whether a tree of depth  and a path always admit a geometric simultaneous embedding or not.

\bibliography{SimTreePathArXiv}
\bibliographystyle{plain}

\end{document} 