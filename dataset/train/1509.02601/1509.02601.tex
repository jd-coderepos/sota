\documentclass[11pt,letterpaper,english]{article}

\usepackage[utf8]{inputenc}
\usepackage{amsmath}
\usepackage{amssymb}
\usepackage{amsthm}
\usepackage{amsfonts}
\usepackage{color}
\usepackage{graphicx}
\usepackage{url}
\usepackage{xargs}
\usepackage[numbers,comma]{natbib}
\usepackage{paralist}
\usepackage{float}
\usepackage{lineno}
\usepackage{pgf}

\usepackage[labelformat=simple]{subcaption} \usepackage[font=small,width=0.95\textwidth]{caption}
\usepackage[pdftex]{hyperref}
\usepackage{kvoptions-patch}
\usepackage{cleveref}

\textwidth=15.1cm \textheight=21.8cm \oddsidemargin=0cm
\evensidemargin=0cm \topmargin=0cm \headheight=0cm \headsep=0cm

\graphicspath{{fig/}{./fig/}}

\AtBeginDocument{\maketitle
  \hypersetup{pdftitle = {On the -hull of a planar point set},
    pdfauthor = {Carlos Alegr\'{i}a-Galicia, David Orden, Carlos Seara, Jorge Urrutia}
  }
}

\renewcommand\thesubfigure{(\alph{subfigure})}

\newtheorem{theorem}{Theorem}
\newtheorem{lemma}{Lemma}
\newtheorem{obs}[theorem]{Observation}


\theoremstyle{definition}
\newtheorem{problem}{Problem}

\newcommandx*{\bset}[1][usedefault, 1=\beta]{\mathcal{O}_{#1}}
\newcommandx*{\bhull}[1][usedefault, 1=\beta]{\mathcal{O}_{#1}\text{-hull}}
\newcommandx*{\bhullp}[2][usedefault, 1=P, 2=\beta]{\mathcal{O}_{#2}\mathcal{H}({#1})}

\newcommandx*{\cset}[2][usedefault, 1=k, 2=\beta]{\left( {#1},{#2} \right)}
\newcommandx*{\csetc}[3][usedefault, 1=k, 2=\beta, 3=\theta]{\mathcal{C}_{{#1},{#2}}({#3})}

\newcommand{\perim}{\operatorname{perim}}
\newcommand{\area}{\operatorname{area}}
\newcommandx*{\polygon}[1][usedefault, 1=\beta]{\mathcal{P}(#1)}
\newcommandx*{\triangles}[2][usedefault, 1=\beta,2=i]{\triangle_{#2}(#1)}
\newcommand*{\parallelogram}[1][]{\pgfpicture\pgfsetroundjoin
    \pgftransformxslant{.6}\pgfpathrectangle{\pgfpointorigin}{\pgfpoint{.60em}{.65em}}\pgfusepath{stroke,#1}\endpgfpicture}
\newcommandx*{\parallelograms}[2][usedefault, 1=\beta,2=j]{\parallelogram_{#2}(#1)}

\newcommand{\jorge}[2][says]{** \textsc{jorge #1:} \textcolor{blue}{\textsl{#2}} **}
\let\Jorge\jorge
\newcommand{\seara}[2][says]{** \textsc{seara #1:} \textcolor{blue}{\textsl{#2}} **}
\let\Seara\seara
\newcommand{\orden}[2][says]{** \textsc{orden #1:} \textcolor{blue}{\textsl{#2}} **}
\let\Orden\orden
\newcommand{\alegria}[2][says]{** \textsc{alegria #1:} \textcolor{blue}{\textsl{#2}} **}
\let\Alegria\alegria

\title{On the  of a planar point set\footnote{In memorial of professor Ferran Hurtado, inspirational friend and colleague, acknowledging his key contribution to the development of Computational Geometry.}}

\author{
  Carlos Alegr\'{i}a-Galicia
  \thanks{Posgrado en Ciencia e Ingenier\'{i}a de la Computaci\'on,
    Universidad Nacional Aut\'onoma de M\'exico, {\tt
      alegria\_c@uxmcc2.iimas.unam.mx}. Research supported by H2020-MSCA-RISE project 73499 - CONNECT.}
  \and
  David Orden
  \thanks{Departamento de F\'{i}sica y Matem\'aticas, Universidad de
    Alcal\'a, Spain, {\tt david.orden@uah.es}. Research supported by MINECO Projects
    MTM2014-54207 and TIN2014-61627-EXP, TIGRE5-CM Comunidad de Madrid Project S2013/ICE-2919, and H2020-MSCA-RISE project 73499 - CONNECT.}
  \and
  Carlos Seara
  \thanks{Departament de Matem\`atiques, Universitat
    Polit\`ecnica de Catalunya, Spain, {\tt
      carlos.seara@upc.edu}. Research supported by projects Gen. Cat. DGR 2014SGR46,
      MINECO MTM2015-63791-R, and H2020-MSCA-RISE project 73499 - CONNECT.}
  \and
  Jorge Urrutia
  \thanks{Instituto de Matem\'aticas, Universidad Nacional Aut\'onoma
    de M\'exico, {\tt urrutia@matem.unam.mx}. Research supported by SEP-CONACYT 80268, PAPPIIT IN102117 Programa de Apoyo a la Investigaci\'on e Innovaci\'on Tecnol\'ogica UNAM, and H2020-MSCA-RISE project 73499 - CONNECT.}}

\date{}



\begin{document}

\begin{abstract}
We study the  of a planar point set, a generalization of the
Orthogonal Convex Hull where the coordinate axes form an angle~.
Given a set  of  points in the plane, we show how to maintain
the  of~ while  runs from  to~ in
 time and  space.
With the same complexity, we also find the values of~
that maximize the area and the perimeter of the  and,
furthermore, we find the value of~ achieving the best fitting
of the point set  with a two-joint chain of alternate interior angle~.
\end{abstract}

\section{Introduction}\label{sec:intro}

Let  be a set of two lines with slopes  and ,
where . A region in the plane is said to be
\emph{-convex}, if its intersections with all translations of
any line in  are either empty or connected. An
\emph{-quadrant} is a translation of one of the
(-convex) open regions that result from subtracting the lines
in  from the plane. We call the quadrants of 
as \emph{top-right}, \emph{top left}, \emph{bottom-right}, and \emph{bottom-left}
according to their position with respect to the elements of ,
see Figure~\ref{intro:fig:bhull}(a).
Let  be a set of  points, and
 the set of all -quadrants that are
\emph{-free}; i.e., that contain no elements of . The
\emph{} of  is the set


of points in the plane belonging to all connected
supersets of  which are
-convex~\cite{alegria_2014,ottmann_1984}. See
\Cref{intro:fig:bhull}(b).

\begin{figure}[ht]
  \centering
  \subcaptionbox{\label{intro:fig:bhull:1}}
  {\includegraphics[scale=1.1]{2-conv-1}}
  \hspace{2cm}
  \subcaptionbox{\label{intro:fig:bhull:2}}
  {\includegraphics[scale=0.5]{2-conv-2-v2}}
  \caption{(a) A set , the top-right, top-left, bottom right, and bottom left quadrants. (b) The corresponding  of a point set.}
  \label{intro:fig:bhull}
\end{figure}

The concept of -convexity stemmed from the notion of
\emph{restricted orientations}~\cite{guting_thesis_1983}, where
geometric objects comply with a property (or a set of properties)
related to some fixed set of lines. Researchers have extensively
studied this notion by considering restricted-oriented
polygons~\cite{guting_thesis_1983}, proximity~\cite{widmayer_1987},
visibility~\cite{schuierer_thesis_1991}, and both restrictions and
generalizations of -convexity. The particular case of
\emph{orthogonal convexity}~\cite{rawlins_1988} considers  to
be fixed at . In the more general
\emph{-convexity}~\cite{rawlins_1987,rawlins_1988},
 is replaced by a (possibly infinite) set of lines with
arbitrary orientations. Other restricted-oriented notions of convexity
include \emph{-convexity}~\cite{franek_2009} and
\emph{-convexity}~\cite{rawlins_thesis_1987}. The former
is based in a functional (rather than set-theoretical) definition,
while the latter (unlike -convexity) always leads to connected
sets. For a comprehensive compilation of studies on the area please
refer to~\citet{fink_2004}. Some recent computational results can be
found in~\cite{minimum-area_2012,alegria_2013,alegria_2014,pelaez_2013}.

In this paper, we solve the problem of maintaining the combinatorial structure
of  while  goes from  to , and apply
this result to some optimization problems. Following the lines
of~\citet{bae_2009}, we find the values of  that maximize the
area and the perimeter of . In addition, we include an appendix extending the results
from~\citet{fitting_2011} to fit a two-joint not-necessarily orthogonal polygonal
chain to a point set. See \Cref{intro:fig:optim}.

\begin{figure}[ht]
  \centering
  \subcaptionbox{\label{bhull:fig:optim:1}}
  {\includegraphics[scale=0.85]{bhull_b2}}
\hspace{-0.2cm}
  \subcaptionbox{\label{bhull:fig:optim:2}}
  {\includegraphics[scale=0.85]{bhull_b3}}
\hspace{-0.4cm}
  \subcaptionbox{\label{bhull:fig:optim:3}}
  {\includegraphics[scale=0.85]{fitting_1}}
  \caption{\subref{bhull:fig:optim:1}
    . \subref{bhull:fig:optim:2}
    , where .
    \subref{bhull:fig:optim:3} A two-joint non-orthogonal polygonal
    chain fitting a point set.}
  \label{intro:fig:optim}
\end{figure}

In all cases, our general approach is to perform an angular sweep. We
first discretize the set  into a linear
sequence of increasing angles
. While  runs from
 to , each  corresponds to an angle where there is a
change in the combinatorial structure of . We then solve the
particular problem for any  in
 time, and show how to update our solution in logarithmic time in the
subsequent intervals . All our algorithms run
in  time and  space.

\paragraph{Outline of the paper.}

In~\Cref{sec:bhull} we show how to maintain the  of  while
 goes from  to . In~\Cref{sec:applications} we extend
this result to solve the optimization problems we mentioned above. We
end in~\Cref{sec:conclusions} with our concluding remarks.

\section{The  of }\label{sec:bhull}

In this section we introduce definitions that are central to our
results. We also show how to compute  for a fixed value of
, and how to maintain its combinatorial structure while  runs
from  to .

\subsection{Preliminaries}\label{sec:bhull:preliminaries}

For the sake of simplicity, we
will assume  to have no three colinear points, and no pair of
points on a horizontal line.
Consider the region  obtained by removing from the plane
all top-right -quadrants free of elements of . The
\emph{top-right -staircase} of  is the directed polygonal
chain formed by the segment of the boundary of  that
starts at the rightmost and ends at the topmost vertex (element of 
that lies over the boundary) of , with respect to the
coordinate system defined by the lines in .~We further define
the \emph{top-left}, \emph{bottom-left}, and \emph{bottom-right}
-staircases in a similar way. See \Cref{bhull:fig:staircases}.

\begin{figure}[ht]
  \centering
  \subcaptionbox{\label{bhull:fig:staircases:1}}
  {\includegraphics{bhull_staircases_1}}
  \hspace{1.5cm}
  \subcaptionbox{\label{bhull:fig:staircases:2}}
  {\includegraphics{bhull_staircases_2}}
  \caption{\subref{bhull:fig:staircases:1} Construction of the
    top-right -staircase. \subref{bhull:fig:staircases:2} The
    four -staircases of .}
  \label{bhull:fig:staircases}
\end{figure}

\begin{obs}\label{bhull:obs:maximal}
  A point in  is a vertex of  if, and only if, it is the
  apex of at least one -free -quadrant free of elements of
  . Conversely, a point in the plane lies in the interior of
   if, and only if, every -quadrant with apex on it
  contains at least one point in .
\end{obs}

We say that an -quadrant is \emph{maximal} if its boundary
joins two consecutive elements in the sequence of vertices found
while traversing an -staircase in its corresponding
direction. Two -quadrants are \emph{opposite} to each other if,
after placing their apices over a common point, their rays bound
opposite angles. Similarly, we say that two -staircases are
opposite to each other, if they were constructed using opposite
-quadrants. It is easy to see that  is disconnected
when the intersection of two opposite maximal -quadrants is not
empty. In such case we say that both -quadrants \emph{overlap},
and refer to their intersection as an \emph{overlapping region}. See
the regions bounded by dashed lines in
\Cref{intro:fig:bhull:2,bhull:fig:optim:2}.

\begin{obs}\label{bhull:obs:staircases}
  Non-opposite -staircases cannot generate overlapping
  regions. Moreover, only one pair of -staircases can intersect
  at the same time.
\end{obs}

We will specify  in terms of its vertices and its overlapping
regions. From \Cref{bhull:obs:maximal}, the set of vertices of
 is the set of maximal elements of  under vector
dominance~\cite{theta-maxima_1999}. Thus they can be computed for a
fixed value of  in  time and 
space~\cite{kung_1975,preparata_1985}. Note that -staircases
are monotone with respect to both lines in  (they could not
bound -convex regions otherwise), so any pair of them intersect
with each other at most a linear number of times. From
\Cref{bhull:obs:staircases}, in a fixed value of~ there is at
most a linear number of overlapping regions. Thus, if the vertices of
 are sorted according to either the - or the -axis, we
can compute from them the set of overlapping regions in linear
time. We get then the following theorem where the 
time lower bound comes from the fact that from  we can compute
the convex hull of  in linear time.

\begin{theorem}\label{intro:thm:fixed_computation}
  For a fixed value of , the sets of vertices and overlapping
  regions of  can be computed in  time and
   space.
\end{theorem}

\subsection{The angular sweep}\label{sec:bhull:sweep}

The  of  is shown in \Cref{bhull:fig:initial_config} at the
\emph{initial increasing configuration}, that is, where  is
equal to an angle  for a small enough
. Note that every point in  is the apex of a -free
-quadrant, and is thus contained in at least one
-staircase: both top-right and bottom-left -staircases
contain the whole set , and the top-left and bottom-right
-staircases are formed respectively, by the topmost and
bottom-most points in . Also, the intersection between the
top-right and bottom-left -staircases generate a linear number
of overlapping regions.

\begin{figure}[ht]
  \centering
  \begin{minipage}{0.9\textwidth}
    \centering
    \includegraphics[clip=true,trim=3cm 0 3cm 0]{initial_config}
    \caption{The initial increasing configuration.}
    \label{bhull:fig:initial_config}
  \end{minipage}
\end{figure}

By performing an \emph{increasing sweep} (where  goes from 
to ), the initial increasing configuration is gradually
transformed to the \emph{initial decreasing configuration}, where
 is equal to a value  for a small
enough  (see \Cref{bhull:fig:final_config}). At this
configuration, the top-left and bottom-right -staircases
contain  and generate a linear number of overlapping regions, and
the top-right and bottom-left -staircases contain respectively,
the topmost and bottom-most points in . Clearly, the converse of
the above discussion holds: from the initial decreasing configuration,
a \emph{decreasing sweep} (where  goes from  to ) will
gradually transform  into .

\begin{figure}[ht]
  \centering
  \begin{minipage}{0.9\textwidth}
    \centering
    \includegraphics[clip=true,trim=4cm 0 4cm 0]{final_config}
    \caption{The initial decreasing configuration.}
    \label{bhull:fig:final_config}
  \end{minipage}
\end{figure}

During the transition between initial configurations, we recognize
four types of events that modify the set of vertices and overlapping
regions of . An \emph{insertion} (resp. \emph{deletion})
event occurs when a vertex joins (resp. leaves) a
-staircase. At \emph{overlap} (resp. \emph{release}) events, an
overlapping region is created (resp. destroyed).

Note that a vertex leaves (resp. joins) the same -staircase at
most once, and thus, there is in total a linear number of insertion
(resp. deletion) events.~From Observation \ref{bhull:obs:staircases},
between  and  there is always an interval
 such that, for any
, the  of  contains no overlapping
regions. Let us consider the angular intervals
 and .
An angular sweep in  results in a linear number of releasing
events caused by the deletion of all overlapping regions present at
the initial increasing configuration. As any vertex supports at most
two maximal -quadrants, an additional linear number of region
events are generated by vertex events and, therefore, region events in
 add up to . Using the same argument on , we can
count a linear number of these events during an angular sweep.

\begin{lemma}\label{bhull:lemma:linear_events}
  There are  events during an angular sweep.
\end{lemma}

We now show how to compute the sequence of increasing angles that mark
vertex and overlapping events during an angular sweep.

\paragraph{Insertion and deletion events.}

The set of vertices of  on the top-right -staircase
has a total ordering that, at any value of  is given by
traversing the staircase along its direction. At the
initial configuration, the order is also given by the sequence
 of points in  labeled in ascending vertical
order.

Let us consider the set
 where for each
, the slope of the line through  and  equals
. In an increasing sweep, the first point leaving
the top-right -staircase is .
Indeed, for any , a
top-right -quadrant with apex over  is not -free. This
is not the case for points corresponding to any  such that
 and . See \Cref{bhull:fig:events}.

\begin{figure}[ht]
  \centering
  \begin{minipage}{0.9\textwidth}
    \centering
    \includegraphics[clip=true,trim=0 0.4cm 0 0, scale=1.2]{vertex_events_first}
\caption{Insertion and deletion events for the top-right
      -staircase.}
    \label{bhull:fig:events}
  \end{minipage}
\end{figure}

To compute the next value of  where a point will leave the
top-right -staircase, we must remove  from
, update  to the angle where the slope of the
line through  and  equals ,
and compute the new smallest element of . A recursive
repetition of this computation allows us to obtain all deletion events
corresponding to the top-right -staircase.

\begin{lemma}\label{bhull:lemma:point_events}
  All insertion and deletion events can be computed in 
  time and  space.
\end{lemma}
\begin{proof}
  Store the points in  in a balanced search tree ordered according
  to the -axis, and the set  in a priority queue. From
  \Cref{bhull:lemma:linear_events}, the algorithm described above
  requires  time and  space to compute the sets of
  insertion and deletion events, associated with the top-right
  -staircase.~Considering the angles shown in
  \Cref{bhull:fig:point_events}, a similar algorithm can be used to
  obtain the corresponding events for the remaining -staircases
  in the same time and space complexity.
\end{proof}

\begin{figure}[ht]
  \centering
  \begin{minipage}{0.9\textwidth}
    \centering
    \includegraphics[clip=true,trim=0 0.4cm 0 0, scale=1.2]{vertex_events_all}
\caption{\Cref{bhull:lemma:point_events}.}
    \label{bhull:fig:point_events}
  \end{minipage}
\end{figure}

\paragraph{Overlap and release events.}

Let  and  be respectively, a pair of overlapping
top-right and bottom-left maximal -quadrants. Consider that
 is supported by the vertices , and  by the
vertices . Also, assume the supporting points are labeled
according to the total ordering of their corresponding staircases
(see Figure~\ref{bhull:fig:eventsw}).

\begin{figure}[ht]
  \centering
  \begin{minipage}{0.9\textwidth}
    \centering
\includegraphics[clip=true,trim=0 0 0 0, scale=1.2]{bhull_overlaps}
    \caption{An overlapping region (bounded by dashed lines) generated
      by the intersection between a top-right and a bottom-left
      maximal -quadrants.}
    \label{bhull:fig:eventsw}
  \end{minipage}
\end{figure}

The \emph{full overlap event} for the overlapping region defined by
 and  is the angle  for which the slope of the
line through  and  equals . If
the supporting points do not leave their corresponding staircases,
this event marks the value of  where the overlapping region
disappears.

Let  be the set of full overlap events for all the
overlapping regions at the initial increasing configuration, and
 the set of all deletion events corresponding to the
vertices over the top-right and bottom-left -staircases. Let
 and  be the smallest values in  and
, respectively. Performing an increasing sweep, to obtain
the first release event, we need to deal with the following cases:
\begin{enumerate}
\item \label{bhull:step_1}  corresponds to a supporting
  point, and .\, In this case,
   needs to be processed and  needs to be
  updated. By removing a supporting point, at most two overlapping
  regions are terminated (two release events are added to
  ), and at most one new overlapping region is generated
  (one overlapping event and one full overlap event are added to
  ). After updating ,  and
   are recomputed and the test is repeated.
\item  does not correspond to a supporting point. In this
  case,  is the first release event.
\end{enumerate}

To compute the next release event, we must remove the current release
event from , and recompute  as described above. A
recursive repetition of these steps allow us to obtain all release
events caused by intersections between the top-right and bottom-left
-staircases.

\begin{lemma}\label{bhull:lemma:overlap_events}
  All overlap and release events can be computed in  time
  and  space.
\end{lemma}
\begin{proof}
  Store the points in  in a balanced search tree ordered according
  to the -axis, and the sets  in priority
  queues. From \Cref{bhull:lemma:linear_events}, the algorithm
  described above requires  time and  space to
  compute the sets of overlap and release events associated with the
  top-right and bottom-left -staircases.~A similar algorithm
  can be used to obtain the events associated to the top-right and
  bottom-left -staircases, with the same time and
  space upper bounds.
\end{proof}

\paragraph{Maintaining .}

Considering the previous results, the maintenance of  is
straightforward:
\begin{enumerate}
\item \label{bhull:maintain:step_1} Compute all vertex and overlap
  events, and store them in a list sorted by appearance during an
  increasing sweep.
\item \label{bhull:maintain:step_2} Compute
  . Store in height balanced trees the total
  orders of the sets of vertices lying over the four
  -staircases. Store the set of overlapping regions in any
  constant-time access data structure (such as a hash table).
\item \label{bhull:maintain:step_3} Simulate the angular sweep by
  traversing the list of events.~At each insertion and deletion event,
  update the corresponding set of vertices.~At each overlap and
  release event, update the set of overlapping regions.
\end{enumerate}

From \Cref{bhull:lemma:point_events,bhull:lemma:overlap_events}, to
compute the sets of vertex and overlap events, we require
 time and  space.
As we have a linear number of elements on each set,
we can merge them into a single ordered set using
 time. Thus,
\cref{bhull:maintain:step_1} requires  time and 
space.

From \Cref{intro:thm:fixed_computation}, computing  for any
fixed value of  takes  time and 
space. Every -staircase contains at most  elements and
therefore, to store their total order in a height balanced tree we
require  time.
Using a hash table, we can initialize the set of overlapping regions
in  time. Therefore,
\cref{bhull:maintain:step_2} requires  time and 
space.

At each insertion and deletion event, updating the corresponding set
of -maximal elements requires  time per
operation. Updates on the set of overlapping regions takes constant
time, so \cref{bhull:maintain:step_3} takes  time.~From
this analysis we get that, in total, we can compute and maintain
 through an angular sweep in  time and 
space.  From \Cref{intro:thm:fixed_computation}, this time complexity
is optimal.

\begin{theorem}
  Computing and maintaining  through an angular sweep
  requires  time and  space.
\end{theorem}

\section{Application problems}\label{sec:applications}

In this section we extend the results from Section \ref{sec:bhull} to
the solution of related optimization problems. We deal with the
problem of maximizing the area and the perimeter of  (Sections~\ref{sec:apps:area} and~\ref{sec:apps:perimeter}, respectively). As an extra application, in~\ref{sec:apps:fitting} we deal with the
problem of fitting a two-joint polygonal chain to a point set.

\subsection{Area optimization.}\label{sec:apps:area}

In this section we solve the following problem:

\begin{problem}[Maximum area]
  Given a set  of  points in the plane, compute the value of
   for which  has maximum area.
\end{problem}

Let  be the sequence of (vertex
and overlapping) events, ordered by appearance during an increasing
sweep. Following the lines of~\citet{bae_2009} (see also \Cref{apps:area:fig:area}), we express the area of
 for any  as

where
  denotes the (simple) polygon having the same vertices as  and an edge connecting two vertices if they are
consecutive in a -staircase.~The term  is the -th triangle defined by two consecutive vertices in a -staircase, and  is the -th overlapping region defined by the intersection of two opposite -staircases.


\begin{figure}[ht]
  \centering
  \begin{minipage}{0.9\textwidth}
    \centering
    \includegraphics[scale=1.5]{bhull_area}
    \caption{The area of . The polygon  is bounded
      by dotted lines. A triangle  and two parallelograms
       are filled in blue.}
    \label{apps:area:fig:area}
  \end{minipage}
\end{figure}

Our general approach is to maintain the terms of
\Cref{apps:area:eqn:area} during a complete angular sweep. We first
compute the optimal value of  for . We then
traverse the event sequence, updating the affected terms in
\Cref{apps:area:eqn:area} at each event. At the same time, we compute
the local angle of maximum area for each . With
any new computation, we keep the local optimal angle only if the
previous maximum area is improved.

\paragraph{The polygon .}

At any fixed value of , the polygon can be constructed from the
vertices of  in linear time. Once constructed, it takes a
second linear run to compute its area. During an interval between
events the area does not change. As  only depends on the
vertices of , it is only modified by insertion and deletion
events. Each event can be handled in constant time: the area of a
triangle needs to be added (deletion event) or subtracted (insertion
event) from the previous value of the area of . See
\Cref{apps:area:fig:polygon}.

\begin{figure}[ht]
  \centering
  \subcaptionbox{\label{apps:area:fig:polygon:1}}
  {\includegraphics[scale=1.2]{bhull_polygon_1}}
  \hspace{1.5cm}
  \subcaptionbox{\label{apps:area:fig:polygon:2}}
  {\includegraphics[scale=1.2]{bhull_polygon_2}}
  \caption{Updating
    . \subref{apps:area:fig:polygon:1} The vertex 
    will leave the top-right -staircase in an increasing sweep.
    \subref{apps:area:fig:polygon:2} The area of a triangle needs to
    be added after the deletion event from , once  is no longer a
    vertex.}
  \label{apps:area:fig:polygon}
\end{figure}

\paragraph{The triangles .}

A triangle is defined by a pair of consecutive vertices of . If we
consider a top-right -staircase, the area of  is
bounded by a line through  and , an horizontal line
through , and a line with slope  through
. In this context, the area of  is given by


with  constants, where  and 
are respectively, the coordinates of the points  and .

\begin{figure}[ht]
  \centering
  \subcaptionbox{\label{apps:area:fig:triangle:1}}
  {\includegraphics[scale=1.2]{bhull_triangles_1}}
  \hspace{1cm}
  \subcaptionbox{\label{apps:area:fig:triangle:2}}
  {\includegraphics[scale=1.2]{bhull_triangles_2}}
  \caption{Updating the term .
    \subref{apps:area:fig:triangle:1} The point  will leave the
    top-right -staircase during an increasing
    sweep. \subref{apps:area:fig:polygon:2} When  is no longer a
    vertex, two triangles are deleted, and a new triangle is created.}
  \label{apps:area:fig:triangle}
\end{figure}

The term  is impacted by insertion and
deletion events and, at each event, it needs to be modified a constant
number of times. As any vertex of  supports at most two
maximal -quadrants, at a deletion event two triangles are
removed and one triangle is added. The converse occurs for insertion
events. See \Cref{apps:area:fig:triangle}.

\paragraph{The overlapping regions .}

An overlapping region is defined by two pairs of consecutive vertices
of  belonging to opposite -staircases. Overlapping
regions are bounded by parallelograms with sides parallel to the lines
in . If we consider top-right and bottom-left
-staircases intersecting as shown in
\Cref{apps:area:fig:parallelogram}, the area of a parallelogram is
given by

with  constants, where  and
 are respectively, the
supporting vertices of the overlapping maximal opposite
-quadrants.

\begin{figure}[ht]
  \centering
  \subcaptionbox{\label{apps:area:fig:parallelogram:1}}
  {\includegraphics[scale=1.2]{bhull_overlaps_1}}
  \hspace{1.5cm}
  \subcaptionbox{\label{apps:area:fig:parallelogram:2}}
  {\includegraphics[scale=1.2]{bhull_overlaps_2}}
  \caption{An overlapping region destroyed because of the vertex
     leaving the top-right -staircase, during an
    increasing sweep.}
  \label{apps:area:fig:parallelogram}
\end{figure}

The term  is impacted by all types of
events. Overlap and release events require a single overlapping
region to be added or deleted. For insertion and deletion events, at
most two new overlaps are created, or destroyed.

\paragraph{Characterization.}

Before describing our algorithm, in the following lemmas we answer
some basic questions about the behavior of
. \Cref{apps:area:lemma:max_angle,apps:area:lemma:bimodal}
imply that it seems not possible to restrict the number of
candidate angles of maximum area. On the other hand, Lemma
\ref{apps:area:lemma:events} shows that the angle of maximum area is
actually located at an event.

\begin{lemma}\label{apps:area:lemma:max_angle}
  For any   there exists a point set  such that 
\end{lemma}
\begin{proof}
  Consider the coordinate system formed by . Place one
  point over the -, -, and -semiaxes, and a point over
  the second quadrant (see \Cref{apps:area:fig:max_angle:1}). From
  this position, note that  for any
   (\Cref{apps:area:fig:max_angle:2}), and there
  exists at least one  such that
  
  (\Cref{apps:area:fig:max_angle:3}). Hence  cannot be the
  angle of maximum area.
\end{proof}

\begin{figure}[]ht
  \centering
  \subcaptionbox{\label{apps:area:fig:max_angle:1}}
  {\includegraphics{area_thm_11}}
  \hspace{.5cm}
  \subcaptionbox{\label{apps:area:fig:max_angle:2}}
  {\includegraphics{area_thm_12}}
  \hspace{.5cm}
  \subcaptionbox{\label{apps:area:fig:max_angle:3}}
  {\includegraphics{area_thm_13}}
  \caption{Lemma
    \ref{apps:area:lemma:max_angle}. \subref{apps:area:fig:max_angle:1}
    The set of points. \subref{apps:area:fig:max_angle:2}
     for .
    \subref{apps:area:fig:max_angle:3}  for
    some .}
  \label{apps:area:fig:max_angle}
\end{figure}

\begin{lemma}\label{apps:area:lemma:bimodal}
  For any , there exists a point set  for which  has local maxima in  and .
\end{lemma}

\begin{proof}
  Let  be a line with slope ,  a line
  with slope , and without loss of generality, let us
  assume that . We define  and 
  to be the points located respectively, at the left corner, right
  corner, top corner, and the interior of the triangle bounded by the
  -axis, , and . See
  \Cref{apps:area:fig:bimodal_1}.

  \begin{figure}[ht]
    \centering
    \begin{minipage}{0.9\textwidth}
      \centering
      {\includegraphics[scale=1.2]{area_thm_21}}
      \caption{The points configuration.}
      \label{apps:area:fig:bimodal_1}
    \end{minipage}
  \end{figure}

  Consider the angles , and  as in
  \Cref{apps:area:fig:bimodal_1}. Note that
  .
  Using an increasing sweep from the initial increasing configuration the
  first release event is . From there, the area of
   is given by a parallelogram  of
  constant height, so both the base of  and the
  area of  increase or decrease together as 
  changes. As  goes from  to , the base
  of  increases up to , there
  exist a local maximum. The base of  then
  decreases from  to , to increase again from
   to . At  there is a second local
  maximum, as the base of  starts decreasing
  again after  up to the last construction event at
  , where the area of  is zero. See
  \Cref{apps:area:fig:bimodal_2}.
\end{proof}

 \begin{figure}[ht]
    \centering
    \subcaptionbox{\label{apps:area:fig:bimodal_2:1}}
    {\includegraphics[scale=0.65]{area_thm_22}}
\hspace{-0.2cm}
    \subcaptionbox{\label{apps:area:fig:bimodal_2:2}}
    {\includegraphics[scale=0.65]{area_thm_23}}
\hspace{-0.2cm}
    \subcaptionbox{\label{apps:area:fig:bimodal_2:3}}
    {\includegraphics[scale=0.65]{area_thm_24}}
\hspace{-0.2cm}
    \subcaptionbox{\label{apps:area:fig:bimodal_2:4}}
    {\includegraphics[scale=0.65]{area_thm_25}}
\hspace{-0.2cm}
    \subcaptionbox{\label{apps:area:fig:bimodal_2:5}}
    {\includegraphics[scale=0.65]{area_thm_26}}
\hspace{-0.2cm}
    \subcaptionbox{\label{apps:area:fig:bimodal_2:6}}
    {\includegraphics[scale=0.65]{area_thm_27}}
\hspace{-0.2cm}
    \subcaptionbox{\label{apps:area:fig:bimodal_2:7}}
    {\includegraphics[scale=0.7]{area_thm_28}}
    \caption{Increasing sweep over the point set of
      \Cref{apps:area:fig:bimodal_1}. \subref{apps:area:fig:bimodal_2:1}
      . \subref{apps:area:fig:bimodal_2:2} A
      local maximum on
      . \subref{apps:area:fig:bimodal_2:3}
      . \subref{apps:area:fig:bimodal_2:4}
      A local minimum on
      . \subref{apps:area:fig:bimodal_2:5}
      . \subref{apps:area:fig:bimodal_2:6}
      A second local maximum on
      . \subref{apps:area:fig:bimodal_2:7}
      .}
    \label{apps:area:fig:bimodal_2}
  \end{figure}




\begin{lemma}\label{apps:area:lemma:events}
  The area of  reaches its maximum at values of 
  belonging to the sequence of events.
\end{lemma}

\begin{proof}
  Let us consider the area of  given by
  \Cref{apps:area:eqn:area}. From
  \Cref{apps:area:eqn:triangle_area,apps:area:eqn:or_area}, the area
  of  can be rewritten as

  
  If we consider the different point configurations that define a
  triangle (see \Cref{apps:area:fig:events}), we can express
   as  or
  , according to the specific
  configuration. Thus, we have
    

  \begin{figure}[ht]
    \centering
    \subcaptionbox{\label{apps:area:fig:events:1}}
    {\includegraphics{area_thm_31}}
    \hspace{1.5cm}
    \subcaptionbox{\label{apps:area:fig:events:2}}
    {\includegraphics{area_thm_32}}
    \caption{Relative positions between the vertices of the triangle
      .}
    \label{apps:area:fig:events}
  \end{figure}

  It is possible to make a similar case-by-case analysis for the
  overlapping regions, to obtain from \Cref{apps:area:eqn:or_area} an
  expression with the form . Within an interval
  between events  does not change, and its area remains
  constant. Therefore, in an interval  we can rewrite:

  
  where  and  contain the sum of all constants from the terms in
  \Cref{apps:area:eqn:event_2}. Note that \Cref{apps:area:eqn:event_3}
  is monotone at any interval , as it is
  monotone in . Depending on the particular values of  and
  ,  might be non-decreasing or
  non-increasing. Thus, the local maximum is given either by 
  or .
\end{proof}

\paragraph{The search algorithm.}

The algorithm to compute the angle of optimum area is outlined as
follows.
\begin{enumerate}
\item \label{apps:area:step_1}Traverse the sequence of events to
  identify the first release event , and the last
  overlap event .
  Restrict the sequence to start with  and finish with , so that  has at least one connected component in every interval. Ignored events have no effect in the
  result, as they belong to an initial (increasing or decreasing) configuration, where .
\item \label{apps:area:step_2} At the first interval, compute
 and using \Cref{apps:area:eqn:area} compute
, keeping the angle
 of maximum area.
\item \label{apps:area:step_3}Traverse the sequence of events. At each
  event:
  \begin{enumerate}
  \item \label{apps:area:step_3_1}Update the set of vertices and
    overlapping regions of .
  \item \label{apps:area:step_3_2}Handle each event updating
    \Cref{apps:area:eqn:area} as explained above.
  \item \label{apps:area:step_3_3}Compute the local angle of maximum
    area. Replace  only if the area of  is improved.
  \end{enumerate}
\end{enumerate}

There is a linear number of events in total, so
step~\ref{apps:area:step_1} requires 
time. \Cref{apps:area:eqn:area} contains at most a linear number of terms, as
there is at most a linear number of vertices and overlapping
regions. Thus, from \Cref{intro:thm:fixed_computation} and previous
discussions, step~\ref{apps:area:step_2} requires  time
and  space.

From Section~\ref{sec:bhull:sweep}, the updates on step~\ref{apps:area:step_3_1}
require logarithmic time. Every event results in a constant number of modifications
to~\Cref{apps:area:eqn:area}, as we described previously in this
section. From~\Cref{apps:area:lemma:events} we can obtain the angle of
maximum area in constant time. As there is a linear number of events,
step~\ref{apps:area:step_3} requires a total of  time. From
this analysis we obtain the following Theorem, where the lower bound comes
from the maintenance of .

\begin{theorem}
  Computing the value(s) of  for which 
  has maximum area, requires  time and  space.
\end{theorem}

\subsection{Perimeter optimization.}\label{sec:apps:perimeter}

In this section we solve the following problem:

\begin{problem}[Maximum perimeter]
  Given a set  of  points in the plane, compute the value of
   for which  has maximum perimeter.
\end{problem}

\noindent The perimeter of  is given by

where the  and the  denote the \emph{steps}
 and parallelograms, respectively, defined
by the staircases, and  denotes one of the (at most four)
\emph{antennas} of , that is, a segment of an
-staircase bounding a zero-area region of .
See again~\Cref{apps:area:fig:area}.

The same approach, and most of the arguments we used to maximize the
area can be applied here. Following the same ideas, we will first
analyze the computation and maintenance of~\Cref{apps:perim:eqn:perim},
we then present adaptations of~\crefrange{apps:area:lemma:max_angle}{apps:area:lemma:events}, and
finalize outlining the search algorithm.

\paragraph{The steps .}

Considering a top-right -staircase (see again
\Cref{apps:area:fig:triangle}), the perimeter of  is given
by \Cref{apps:perim:eqn:step}, where  and
 are the points supporting the -th
step. Vertices over the staircase have non-decreasing  coordinates,
so  is always positive. Event handling is done in the same way as
we did with triangles in the previous section.



\paragraph{The overlapping regions .}

If we consider top-right and bottom-left -stair-cases
intersecting as shown in \Cref{apps:area:fig:parallelogram}, the
perimeter of an overlapping region is given by
\Cref{apps:perim:eqn:or}. The constants  and  are always
positive. Event handling is done in the same way as we handled
overlapping regions to optimize the area of .



\paragraph{The antennas .}

An antenna is a semistep at one of the extremes of an
-staircase. Just as steps and triangles, an antenna is defined
by two consecutive -maximal points. If we consider a top-right
-staircase, the perimeter of an antenna is given by
\Cref{apps:perim:eqn:antenna:1} if it is the first semistep of the
staircase, and by \Cref{apps:perim:eqn:antenna:2} if it is the last one
(see \Cref{apps:perim:fig:antennas}). In both equations we consider
 to be the point supporting the corresponding
semistep. The constant  is always positive.

1em]
  \label{apps:perim:eqn:antenna:2}
  \perim_l(\diagdown_k) &= \left( y_{i+1}-y_i \right)\csc(\beta) \nonumber \\
                     &= f_k \cot(\beta)

  \label{apps:perim:eqn:antenna:3}
  \sum_i \perim(\angle_i(\beta)) &= \sum_i \left|
                                   a_i \cot(\beta) + a_i \csc(\beta)
                                   \pm  b_i
                                   \right| \nonumber \\
                                 &= \sum_{i_0}
                                   a_{i_0} \cot(\beta) + a_{i_0} \csc(\beta) + b_{i_0}
                                   +  \sum_{i_1}
                                   a_{i_1} \cot(\beta) + a_{i_1} \csc(\beta) - b_{i_1} \nonumber \\
                                 &= a \cot(\beta) + a \csc(\beta) + b

  \label{apps:perim:eqn:antenna:4}
  \sum_j \perim(\parallelograms) &= \sum_j \left|
                                   c_j \cot(\beta)
                                   + d_j \csc(\beta)
                                   \pm e_j \right| \nonumber \\
                                 &= \sum_{j_0}
                                   c_{j_0} \cot(\beta) + d_{j_0} \csc(\beta) + e_{j_0}
                                   +  \sum_{j_1}
                                   c_{j_1} \cot(\beta) + d_{j_1} \csc(\beta) - e_{j_1} \nonumber \\
                                 &= c \cot(\beta) + d \csc(\beta) + e

  \label{apps:perim:eqn:antenna:5}
  \sum_k \perim_l(\diagdown_k) &= \left|
                                 f_k \cot(\beta) \pm g_k
                                 \right| \nonumber \\
                               &= \sum_{k_0}
                                 f_{k_0} \cot(\beta) + g_{k_0} + \sum_{k_1}
                                 f_{k_1} \cot(\beta) - g_{k_1} \nonumber \\
                               &= f \cot(\beta) + g,
 
   \label{apps:perim:eqn:antenna:6}
   \perim(\bhullp) &= \sum_i \perim(\angle_i(\beta))
                     - \sum_j \perim(\parallelograms)
                     - \sum_k \perim(\diagdown_k(\beta)) \nonumber \\
                   &= \left( a + c + f \right) \cot(\beta)
                     + \left( a + d \right) \csc(\beta)
                     + \left( b + e + g \right) \nonumber \\
                   &= A \cot(\beta) + B \csc(\beta) + C.
 \max_{\beta} \perim(\bhullp)\neq\perim(\bhullp[P][\beta_0]).
    \perim(\bhullp) = A \cot(\beta) + B \csc(\beta) + C,
  \label{apps:perim:eqn:derivative}
    \cos (\beta) = - \frac{A}{B},
  
d_f(p_i,\csetc) = \underset{p \in \ell_{i,\beta}(\theta) \cap
  \csetc}{\text{min}} d(p_i,p),

\mu(\csetc,P) = \underset{p_i \in P} \max \hspace{0.3cm} d_f(p_i,\csetc).

The -fitting problem for  with the Min-Max criterion,
consists on finding a polygonal chain  with minimum error
tolerance . See Figure~\ref{apps:fitting:fig:fitting}.

\begin{theorem}[\cite{fitting_2011}]
  \label{apps:fitting:thm:2-fitting}
  The -fitting problem can be solved in
   time and  space.
\end{theorem}

We consider here the case where  has a constant value, namely
, and we want to find the chain
 of optimal error
tolerance.~More formally, we solve the following problem.

\begin{problem}[Oriented -fitting]
  Given a set  of  points in the plane, compute a polygonal
  chain  such that
   has minimum value.
\end{problem}

Consider the algorithm used in~\cite{fitting_2011} to obtain the
 time bound for the -fitting
problem used to prove \Cref{apps:fitting:thm:2-fitting}.~The
 of  is used as a tool to solve the problem
in  time for a fixed value of  in a closed
orientation interval .~An event sequence of a
linear number of orientation intervals is created to maintain
 as  grows from  to .

To solve Problem~3 we can follow exactly the same techniques.~We refer
the reader to reference~\cite{fitting_2011} just to see the evident
changes coming from the use of a different structure.~More concretely,
the structure  is replaced by 
which needs also a linear number of \emph{interval events}
 to be maintained, and where the angular sweep
is performed over~.~Thus, Lemmas 3 and 4
in~\cite{fitting_2011} can be now stated as follows:\\

(i) \emph{Given a value , an optimal
  solution of the -fitting problem for  is defined by
  a line  with slope
   passing through a point  of  which gives the bipartition of }.\\

(ii) \emph{The optimal solution of the -fitting problem for
  an interval event  occurs either at an
  endpoint of the interval, i.e., at  or , or at
  a value 
  when the left and right error tolerance are equal}.\\

Using the properties (i),(ii) and following the maintenance of
, the problem is solved as follows:
\begin{enumerate}
\item Compute  and the optimal error tolerance for the first
  interval between events.
\item Traverse the event sequence, obtaining the optimal error
  tolerance at each interval between events.
\item Update the previous solution only when it is improved.
\end{enumerate}

Thus, the approach and arguments used in
Theorem~\ref{apps:fitting:thm:2-fitting} hold in the case of the
-fitting problem See
Figure~\ref{apps:fitting:fig:fitting}.~As a consequence, we get the
following theorem.

\begin{theorem}
  \label{apps:fitting:thm:fitting}
  The -fitting problem can be solved in 
  time and  space.
\end{theorem}





\begin{figure}[H]
  \centering
  \begin{minipage}{0.9\textwidth}
    \centering
    {\includegraphics[scale=0.9]{fitting_2}}
    \caption{The polygonal chain  and the
       of .}
    \label{apps:fitting:fig:fitting}
  \end{minipage}
\end{figure}

\end{document}
