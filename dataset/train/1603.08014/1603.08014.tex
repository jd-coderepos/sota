











\documentclass[journal,draftcls,onecolumn,12pt,twoside, narroweqnarray]{IEEEtran}
\usepackage{color}
\usepackage{algorithm}
\usepackage{algorithmic}
\usepackage{graphicx}
\usepackage{amsmath,amssymb,amsthm}
\usepackage{latexsym}
\usepackage{cite}
\newtheorem{prop}{Proposition}
\newtheorem{Def}{Definition}
\newtheorem{Lemma}{Lemma}
\newcounter{MYtempeqncnt}

















\ifCLASSINFOpdf
\graphicspath{{}{../pdf/}{../jpeg/}{../eps/}}
\DeclareGraphicsExtensions{.jpg,.pdf,.jpeg,.png,.eps}
\else
\fi






























\ifCLASSOPTIONcompsoc
\usepackage[caption=false,font=normalsize,labelfont=sf,textfont=sf]{subfig}
\else
\usepackage[caption=false,font=footnotesize]{subfig}
\fi

















\hyphenation{op-tical net-works semi-conduc-tor}
\usepackage{setspace}

\setlength{\textwidth}{6.5in}
\doublespacing

\def \qed {\hfill \vrule height6pt width 6pt depth 0pt}
\begin{document}
\title{Sensor Deployment with Limited Communication Range in Homogeneous and Heterogeneous Wireless Sensor Networks }
\author{Jun~Guo,~\IEEEmembership{Student Member,~IEEE,}
        and~Hamid~Jafarkhani,~\IEEEmembership{Fellow,~IEEE,}
\thanks{The authors are with Center for Pervasive Communications and Computing, University of California, Irvine
(e-mail: guoj4@uci.edu; hamidj@uci.edu).}}

















\maketitle
\begin{abstract}
We study the heterogeneous wireless sensor networks (WSNs) and propose the necessary condition of the optimal sensor deployment. Similar to that in homogeneous WSNs, the necessary condition implies that every sensor node location should coincide with the centroid of its own optimal sensing region. Moreover, we discuss the dynamic sensor deployment in both homogeneous and heterogeneous WSNs with limited communication range for the sensor nodes. The purpose of sensor deployment is to improve sensing performance, reflected by distortion and coverage. We model the sensor deployment problem as a source coding problem with distortion reflecting sensing accuracy. Traditionally, coverage is the area covered by the sensor nodes. However, when the communication range is limited, a WSN may be divided into several disconnected sub-graphs. Under such a scenario, neither the conventional distortion nor the coverage represents the sensing performance as the collected data in disconnected sub-graphs cannot be communicated with the access point. By defining an appropriate distortion measure, we propose a Restrained Lloyd (RL) algorithm and a Deterministic Annealing (DA) algorithm to optimize sensor deployment in both homogeneous and heterogeneous WSNs. Our simulation results show that both DA and RL algorithms outperform the existing Lloyd algorithm when communication range is limited.
\end{abstract}

\begin{IEEEkeywords}
Sensor deployment, homogeneous, heterogeneous, source coding, coverage.
\end{IEEEkeywords}






\IEEEpeerreviewmaketitle



\section{Introduction}
\IEEEPARstart{A}{s} a bridge between the physical world and the virtual information word, wireless sensor networks (WSNs) collect data from the physical world and communicate it with the virtual information world, such as computers. Proper sensor deployment improves monitoring and controlling the physical environment.
To accomplish their tasks, WSNs should address two needs: (i) Sensing in the target area and (ii) Communication between the sensor nodes.
WSNs are utilized to collect physical information, such as temperature, humidity, voice and so on. But, the collected data is useless if it cannot be transmitted to the access point (AP) and to the outside information world through the AP node. When sensors are connected by wire lines, the  connectivity is provided automatically.
On the other hand, the connectivity of WSNs is not guaranteed. In this paper, we consider the sensing and connectivity together and redefine the goal of WSN design accordingly.

A huge body of literature exists on the topic of sensor deployment. The sensor coverage range model assumes that sensors can only monitor the points within a range of . The range  is called the sensing range and the coverage area is the area covered by at least one sensor node \cite{SD}. Three different connectivity criteria are proposed in \cite{CC}. Three movement-assisted protocols, the VECtor-based algorithm, the VORonoi-based algorithm and the Minimax algorithm, are designed to maximize coverage area in \cite{wang1}. Lloyd algorithm has also been used as a tool to deploy sensors in homogeneous WSNs \cite{SD}. The convergence of the Lloyd algorithm has been studied in \cite{kiefferconvergence,wuconvergence,qdu1}. The analysis in [4]-[6] can be applied to the sensor deployment methods in [1].
However, \cite{SD} assumes an infinite communication range and ignores the connectivity limitation. When an infinite communication range is assumed, all the nodes in the network are connected to each other. In reality, each node has a limited communication range that will affect the connectivity of the network \cite{hu1,yousefizadeh1}. A geometric analysis of the relationship between the sensing coverage and the connectivity is proposed in \cite{ICC}. In \cite{DCC}, the authors have come up with some deployment patterns to achieve both sensing coverage and full connectivity.

Unfortunately, given a fixed number of sensor nodes and a finite communication range, connectivity is not guaranteed. When sensor nodes are divided into several disconnected sub-graphs, the conventional Lloyd algorithm cannot converge to a proper deployment. The Critical Sensor Density (CSD) in \cite{CSurvey} is the number of nodes per unit area, required to provide full sensing coverage when the communication range is limited.
When the sensor density is smaller than CSD, we cannot achieve the full sensing coverage. Under such a scenario, coverage may not be the right cost function to optimize. One needs to define an appropriate distortion measure to reflect the sensing accuracy. Distortion, as an important parameter in source coding can also be used to evaluate the WSN performance. Therefore, one can minimize distortion in WSNs through vector quantization techniques in \cite{gray1} and \cite{gershograybook}. The best possible distortion for a given number of sensors, i.e., the minimum distortion for a given rate, can be analyzed through the rate-distortion theory \cite{information theory}. Even if the sensor density is larger than CSD, there is no existing sensor deployment algorithm designed to achieve the full connectivity and minimize the distortion at the same time. In this paper, when the communication range is limited, we take both distortion and connectivity into consideration. The existing coverage area model is a special case of our distortion measure. We propose a method, named Restrained Lloyd (RL) Algorithm, to distribute sensor nodes and to minimize distortion with full connectivity. Then, a more complex approach, named Deterministic Annealing (DA) Algorithm, is designed to avoid sub-optimal solutions.

In many practical situations, different sensors in the WSN have different characteristics such as computational power, sensing range, and sensing accuracy.
The deployment and topology control of such heterogeneous WSNs that include the sensor nodes with different communication or sensing ranges, have been studied in \cite{mahboubi1} and \cite{HWS}. Similar to \cite{wang1}, three movement-assisted protocols in \cite{mahboubi1} are designed to avoid coverage hole in heterogeneous WSNs.
However, \cite{HWS} deploys sensor nodes one-by-one and to deploy a new sensor uses the location information of all previously deployed nodes. Also, \cite{HWS} assumes that sensor can monitor events within a circle with the radius equal to the sensing range. We generalize this model to a sensing accuracy which depends on the distance between the sensor and the event. Our distortion model will include the sensing range model in \cite{HWS} as a special case in which the distortion is a step function.
In such heterogeneous WSNs, weighted Voronoi diagrams \cite{VD} rather than conventional Voronoi diagrams \cite{voronoisurvey} will provide the best regions as we will discuss in this paper. An algorithm to construct weighted Voronoi diagrams for a different application has been suggested in \cite{aurenhammerwvd}. Based on the geometry of the optimal cell partitioning, we will analyze the objective functions in our model and propose the necessary condition for the optimal sensor deployment in heterogeneous WSNs with different sensing abilities.

In the rest of this paper, we first introduce the system model for both homogeneous and heterogeneous WSNs and formulate the problems of sensing and connectivity in Section \ref{sec:model}. Section \uppercase\expandafter{\romannumeral3} analyzes the optimal deployment in heterogeneous sensor networks without communication constraint. Section \uppercase\expandafter{\romannumeral4} proposes RL and DA algorithms to improve distortion and maintain connectivity. Section \uppercase\expandafter{\romannumeral5} presents simulation results and
Section \uppercase\expandafter{\romannumeral6} provides the conclusions.





\section{System Model and General Problems}\label{sec:model}
Let  be a simple convex polygon in  including its interior. Given  sensors in the target area , sensor deployment is defined by , where  is Sensor 's location. For any point ,  is the probability density function of an event at point . A cell partition  of  is a collection of disjoint subsets of  whose union is . Let  be a disk centered at  with radius  in two-dimensional space. For two points  and , let equation , where  is a  matrix and  is a constant, define the perpendicular bisector hyperplane between the two points. Then, the equations  and  define two half spaces. we denote the half space that contains point  by .

As mentioned before, we define the AP as the sensor node that can communicate with the outside information world.
Let  be the set of sensor nodes that can communicate with the AP when the sensor deployment is . Note that in general not all nodes can communicate with the AP and  , where  is the number of elements in set . We define a new sensor deployment, which is a subset of the all sensor locations,  as the vector of sensor locations for the  sensor nodes connected to the AP. When  includes all sensor nodes, we have  and . Let  be the set of sensor deployments that provide full connectivity, i.e., . In our model, two sensor nodes can communicate with each other within one hop if and only if the distance between the two is smaller than , where  is referred to as the communication range.
A sensor node can transfer data outside if and only if there exists a path from the sensor to the AP. The path consists of a sequence of sensor nodes where each hop distance is smaller than the communication range .
Sensor nodes that are connected to the AP construct the backbone network. If all sensors are included in the backbone network, we call the network fully connected. Otherwise, the network is divided into several disconnected sub-graphs.










Another important factor in analyzing the performance of a WSN is its sensing accuracy. Ideally, we would like to sense all events in the covered area. However, the sensing accuracy of a sensor node usually depends on the distance between the sensor and the event to be sensed.
In other words, the accuracy of the gathered data from an event at point  by its associated sensor node  is a non-increasing function of the distance between  and .
Therefore, to represent the average sensing accuracy in the target area, we define the following  general distortion:

where  is the cost function associated with sensing.
The partition  in the above definition include all sensor nodes. However, as explained before, when the communication range is limited, some sensor nodes cannot transfer their data back. As a result, only the sensor nodes in the backbone network can contribute to the sensing and therefore the distortion should be revised as

Note that to derive Eq. (\ref{D_S_P}) from Eq. (\ref{D_P}), one has to replace  with , i.e., one has to consider only the sensor nodes that are in the backbone network.
We reiterate that in the case of a fully connected network,  and Eq. (\ref{D_P}) and Eq. (\ref{D_S_P}) are identical.


Obviously, choosing different cost functions in Eq. (\ref{D_S_P}) results in different problem formulations. One natural choice for the cost function is a continuous function defined by

where the cost parameter  is a constant that depends on the sensor characteristics. In homogeneous WSNs, every sensor node has the same sensing ability and the same cost parameter . Therefore, the cost parameter can be ignored in homogeneous WSNs. However, different sensors with different complexity, power and sensing ability are used in heterogeneous WSNs. Obviously, the cost parameters  reflect the quality of sensor nodes. The smaller the cost parameter, the stronger the sensing ability.


The distortion definitions in Eqs. (\ref{D_P}) and (\ref{D_S_P}) can represent the sensor coverage area model \cite{SD} as well. In such a model, the sensors can monitor events within a circle with a fixed radius called the sensing range. Consider a step function defined by

Adopting the above sensor coverage area model, choosing  in Eq. (\ref{D_S_P}) converts the distortion to the area covered by the sensors. In such a model, sensors can monitor events within a circle with the radius . Obviously, the coverage area should be maximized while the distortion should be minimized. Since by definition  should be a non-decreasing function, to have the sensor coverage area model as a special case of our model, we should choose .

Our main goal is to minimize the distortion defined in Eq. (\ref{D_S_P}). It is easy to show that a necessary condition for such an optimal sensor deployment is to have a fully connected network. Moreover, the distortion is determined by both sensor deployment and cell partitioning. This is the topic of the  discussion in the next section.

\section{Optimal Deployment in Heterogeneous WSNs without Communication Constraint}
In this section, we assume an infinite communication range which results in a connected network for any sensor deployment. Given a fully connected network, the distortion  is determined by the sensor deployment  and the cell partition . In homogeneous WSNs, given sensors' locations, Voronoi partitions provide the smallest distortion. The Voronoi region (partition) for Sensor , denoted by , is the intersection of half spaces . In other words, the Voronoi region for Sensor  is the set of all points that are closer to Sensor  than any other sensor. The Voronoi partition of  generated by  with respect to the Euclidean norm is the collection of sets  defined by

where  is the Euclidean norm. This is because all sensors in homogeneous WSNs have the same cost function  which is only determined by the Euclidean distance . Since an event in  is monitored by Sensor , each event is sensed by the nearest sensor and therefore makes the smallest contribution to the global distortion.


However, in a heterogeneous WSNs, the cost function  is also affected by the cost parameter , which reflects the sensing ability of Sensor . Sensor nodes in heterogeneous WSNs can be classified according to their cost parameters . Let us assume there are  different sensor types with  different cost parameters. Given sensor nodes' locations, an event at point  should be sensed by the sensor with the smallest cost such that its contribution to the global distortion is the lowest possible. Such an optimal partitioning  is refer to as the weighted Voronoi partitioning. The weighted Voronoi partition of  generated by  is the collection of sets  defined by

Note that both Voronoi regions  and weighted Voronoi regions  are functions of . Since the Voronoi partitioning can be considered as a special case of the weighted Voronoi partitioning, in which , we simply use  to represent both. Using the result of weighted Voronoi partitioning as the partition in Eq. (\ref{D_P}), the global distortion can be rewritten as

Our goal is to find the sensor deployment that minimizes the global distortion. In what follows, we generate the machinery to find the necessary condition of the optimal sensor deployment.

\begin{prop}
\label{Proposition.1}
Consider a two-dimensional heterogeneous sensor network, in which the cost function between Sensor  and point  is , the optimal partition for a given deployment  is

where  and .
\end{prop}
\begin{IEEEproof}
The proof is provided in Appendix A.
\end{IEEEproof}
Before we discuss the optimal sensor deployment in heterogeneous WSNs, we need to present the following definitions and lemmas.
\begin{Def}
A set  is called star-shaped if and only if there exists a point  such that for all  and all , one has , where  is the interior of  and  is the boundary of . The point  is the reference point.
\end{Def}
\begin{Def}
A set  is called a convex region if and only if for every pair of points  and all , one has .
\end{Def}
\begin{Lemma}
\label{a}
If a set  is convex, then  is star-shaped.
\end{Lemma}
\begin{IEEEproof}
For any convex region , pick a point . For any point  and all , one has . When , . Therefore, the set  is star-shaped.
\end{IEEEproof}
We will use the fact that the intersection of any collection of convex sets is convex \cite{Convexity,ACA} and as a result star-shaped according to Lemma \ref{a}.
\begin{Lemma}
\label{b}
The union of star-shaped sets that are associated with the same reference point  is star-shaped.
\end{Lemma}
\begin{IEEEproof}
Given  star-shaped sets  with the same reference point , the corresponding union is . Since point  is the reference point of star-shaped sets , where , we have  and therefore . The boundary of  comes from the boundaries of  star-shaped sets , where . Thus, for any point , we have . Because of  star-shaped sets, for all  and all , we will have . Thus, for all  and for all , one can find a subset  such that  and so .
\end{IEEEproof}
\begin{Lemma}
\label{Lamma2}
Let  be a star-shaped set that consists of  disjoint sub-sets , where . We then have

where  is a continuous function of , and  is the unit outward normal to  at .
\end{Lemma}
\begin{IEEEproof}

For any  and  such that , the corresponding curve integral  is 0. On the other hand, for any  and  such that , the corresponding curve integral  because of opposite unit outward normal. Therefore, we have

\end{IEEEproof}
\begin{Lemma}
\label{Lamma3}
Let  be a star-shaped set that consists of  disjoint subsets , . Then for any point  we have

where  and  are, respectively, the mass and the center of of mass with respect to the probability density function  of the set S.
\end{Lemma}
\begin{IEEEproof}
We rewrite the left side of Eq. (\ref{pCM}) to derive

\end{IEEEproof}
Now, we have enough tools to derive the main results in this section.
Proposition A.1. in \cite{SD} presents how to calculate the gradient of the distortion when sensing regions are star-shaped. Unfortunately, it is possible that the sensing regions in heterogeneous WSNs are not star-shaped. In what follows, we show how to calculate the gradient of the distortion in heterogeneous WSNs.
\begin{prop}
\label{prop2}
In a heterogeneous sensor network including  kinds of sensors, let  be the sensor deployment, and  be an arbitrary convex set. Let a series of functions , where , be continuous on , continuously differentiable with respect to its second argument for all , where , and almost all , and such that for each , the maps  and  are measurable, and integrable on . Then the function

is continuously differentiable and

\end{prop}
\begin{IEEEproof}
The proof is provided in Appendix B.
\end{IEEEproof}
Note that the sensing cell  in Proposition \ref{prop2} is a weighted Voronoi region and different from the Voronoi region in \cite{SD}. The weighted Voronoi region  can be a non-star-shaped region and therefore we need to use Proposition \ref{prop2}.


Next, we derive the necessary condition for the optimal deployment in heterogeneous WSNs when the communication range is infinite. The format of the result, as proved in the next proposition, is similar to that of homogeneous WSNs.
\begin{prop}
When the communication range is infinite, the necessary condition for the optimal deployment in heterogeneous WSNs is

where  is the optimal position for sensor node ,  and  are,respectively, the mass and the center of weighted Voronoi cell  with respect to the probability density function  in target region .
\end{prop}
\begin{IEEEproof}
Let  and , where  is the cost function, we can use Proposition \ref{prop2} to calculate the partial derivative of the local distortion as follows:

where  if and only if  is on the boundary of .

Note that .
  Thus, in Eq. (\ref{partial}) only the curve integral for , i.e.,  on , needs to be taken into account. For each curve integral in the second case (), one can find a curve integral with the opposite unit outward normal in the first case (). As a result, the curve integrals in the partial derivative of the global distortion are canceled with each other. Therefore,

Both  and  are functions of . The optimal deployment  will have a zero gradient. We define the cost parameters to be positive. Thus, when the communication range is infinite, the necessary condition for optimal deployment is the same as Eq. (\ref{critical deployment}).
\end{IEEEproof}
\section{Restraint Lloyd Algorithm and Deterministic Annealing Algorithm}
In this section, we design algorithms to minimize the distortion when the communication range is limited.
First, we quickly review the conventional Lloyd algorithm. Lloyd Algorithm has two basic steps in each iteration: (1) Sensor nodes move to their centroid; (2) Partitioning is done by assigning the corresponding Voronoi region to each sensor node. Lloyd Algorithm provides good performance and is simple enough to be implemented distributively. It converges to a minimum distortion when the communication range is infinite \cite{SD}. Unfortunately, it also has three shortcomings. First, since minimizing distortion is a non-convex optimization problem, Lloyd Algorithm may end at a large local minimum point rather than the optimal global minimum. Second, Lloyd Algorithm may not result in a connected network. Third, when WSNs are divided into several disconnected sub-graphs, Lloyd Algorithm is not feasible. In other words, since there is no global information available about the sensor locations, each sub-graph will run the algorithm independently. To deploy a network with full connectivity and lower distortion, we add some restraints on sensors' movements. We design a class of algorithms based on the Lloyd algorithm, referred to as RL Algorithm.

\subsection{Restrained Lloyd Algorithm}
Before we introduce the details of our RL Algorithm, we introduce the concept of a desired region. Let us assume we are trying to move Sensor  at a given step.
Our goal is to keep the connectivity of the backbone network after moving Sensor . Therefore, we define the areas in which Sensor  will be connected to the backbone network as its desired region, denoted by . Note that this region may not be a star-shaped set. In our RL Algorithm, if Sensor  is in the backbone network, we will restrain its movement within its desired region. To achieve this goal, we need to find the desired region .
Given a deployment , if Sensor  from the backbone network is removed, the rest of the sensor nodes in the backbone network will be divided into  components: , where  is a set of sensors included in the th component. Note that  may be equal to one. Then, we can calculate the desired  region as

Since the desired region is primarily influenced by the neighboring sensor nodes, we can approximate it by

where  consists of Sensor 's neighbors when the deployment is . Note that the approximation in (\ref{LiP}) can be calculated locally, but to calculate the exact desired region, one needs global information. Also, according to Lemma \ref{b}, the approximate desired region  is a star-shaped set.

Now, we provide the details of our RL Algorithm. The algorithm iterates between two steps: \\
(1) Sensors in the backbone network move one by one. Every sensor in the backbone network calculates its own approximate desired region  and moves to a location which is the closest point to its centroid  within . Sensors outside the backbone network move randomly and check if there is a path to the AP. Unlike the conventional Lloyd algorithm, these new locations may not be the centroid of the partition regions;
\\
(2) The target area, , is partitioned to weighted Voronoi regions for sensors in the backbone network, .


The main difference between RL Algorithm and the Lloyd algorithm is in the first step.
In what follows, we show that Step (1) in RL Algorithm will not increase the global distortion.





The global distortion is the sum of local distortions defined by

where  is the number of sensors in the backbone network.
The local distortion, whether in homogeneous WSNs or heterogeneous WSNs, is a convex function. According to the parallel axis theorem, the local distortions can be rewritten as

where  is the centroid of the partition region  with respect to the probability density function. Both  and  are constants when the integral area  is fixed. In other words, the local distortion is directly proportional to , i.e., the  sensor's distance to its centroid. Therefore, the movements in Step (1) minimize the local distortions. As the sum of local distortions, the global distortion will not increase. Since the sequence of the global distortion values is a non-increasing sequence with a lower bound of zero, it will converge.





We also show that our RL Algorithm guarantees the connectivity of the network with high probability after enough number of iterations.
Note that once a sensor node finds a path to the AP, our RL Algorithm will keep it in the backbone network.
Intuitively, as we have more iterations, the sensors outside the backbone network will move randomly and eventually connect to the AP as well.
Quantitatively, for the deployment after  iterations, the area in which a sensor can communicate with the backbone network can be calculated by  . Then, the probability that a sensor outside the backbone network is not connected to the AP in its next move is .
After  iterations, the probability that a sensor is still out of the backbone network can be calculated by  and then  because of . In other words, as long as the number of iterations is large enough, almost all sensor nodes will be included in the backbone network, indicating full connectivity, with high probability.
















\subsection{Deterministic Annealing Algorithm}
Like any other steepest-descent algorithm, RL Algorithm may converge to a local minimum a large distortion. One approach to improve the sub-optimal solution or find the global optimal solution, is to use annealing methods. Simulated Annealing (SA) \cite{SA,SR} is a method in which a candidate sensor movement is generated randomly. However, SA ignores the characteristics of the objective function and requires burdensome computations. In this paper, we design a DA algorithm which combines RA with annealing to minimize the distortion. Unlike SA, the proposed DA generates two new sensor positions deterministically at each iteration; however, choose one of the two options randomly.
Like RL Algorithm, our DA Algorithm iterates between two steps. The second step is identical to that of RL Algorithm. In the first step, the algorithm creates two candidate locations for each node in the backbone network. One candidate is the RL Algorithm's candidate that minimizes the local distortion. On the other hand, the second candidate increases the local distortion. It is easy to show that to maximize the local distortion for Sensor  in the backbone network, one should move it to the point  on the boundary of the desired region  that has the largest distance to the centroid . But the goal of the second candidate is to increase the distortion and not necessarily maximizing it. Moreover, the distortion is more sensitive to the sensors with smaller cost parameters. In order to avoid increasing distortion too fast, Sensor  moves to the point . The algorithm will choose the first candidate with probability  and increases  from 0 to 1. Otherwise, the algorithm will choose the second candidate. In our algorithm, the probability  is increased in proportion to , where  is the iteration count and for the last  iterations we force the probability .
Like RL Algorithm, DA Algorithm guarantees connectivity and convergence. The proof is similar to that of RL Algorithm and is omitted.




\section{Performance Evaluation}
We compare the performance of RL Algorithm, DA Algorithm and Lloyd Algorithm in sensor networks. We provide simulations in three sensor networks: (1) WSN1: A homogeneous WSN in which all sensors have the same cost parameter ; (2) WSN2: A heterogeneous WSN including 2 kinds of sensors: four strong sensors with  and twelve weak sensors with ; (3) WSN3: A heterogeneous WSN including three kinds of sensors: two strong sensors with , four medium sensors with  and ten weak sensors with . Sixteen sensors are provided in each sensor network.  The AP is chosen from the sixteen sensors randomly. However, when we report the distortion or coverage area for the Lloyd algorithm, we report that of the largest connected subgraph which may not be connected to the AP. Obviously, this will be advantageous for the Lloyd algorithm, but our proposed algorithms still outperform the Lloyd algorithm. We use ten random initial deployments for each algorithm. To have a fair comparison, we consider the same target domain  as in \cite{SD}.  is determined by the polygon vertices
{(0,0), (2.125,0), (2.9325,1.5), (2.975,1.6), (2.9325,1.7), (2.295,2.1), (0.85,2.3), (0.17,1.2)}. The distribution of the events is also the same as \cite{SD}. The probability density function is the sum of five Gaussian functions of the form . The centers  are (2,0.25), (1,2.25), (1.9,1.9), (2.35,1.25) and (0.1,0.1). We use 0.5 as the communication range . Also, when reporting the coverage area using (\ref{step_cost}), we use 0.25 as the sensing range . In DA Algorithm, the first candidate is accepted at the th iteration by a probability of , where  is the number of regular iterations. Additional  iterations are used in DA Algorithm to avoid ending with a process that increases the local distortions.
\begin{figure*}[!t]
\centering
\subfloat[]{\includegraphics[width=2.7in]{WSN1a.jpg}
\label{Lloyd initial deployment}}
\hfil
\subfloat[]{\includegraphics[width=2.7in]{WSN1b.jpg}
\label{Lloyd final deployment}}
\hfil
\subfloat[]{\includegraphics[width=2.7in]{WSN1c.jpg}
\label{RL final deployment}}
\hfil
\subfloat[]{\includegraphics[width=2.7in]{WSN1d.jpg}
\label{DA final deployment}}
\caption{Sensor deployments in WSN1. (a) The initial sensor deployment and the corresponding Voronoi regions. (b) The final deployment of Lloyd Algorithm after 500 iterations. (c) The final deployment of RL Algorithm after 500 iterations. (d) The final deployment of DA Algorithm after 500 iterations. Sensors in the backbone network are marked by circles. Sensors disconnected from the backbone network are denoted by dots. Voronoi region centroids are marked by stars. The radius of each gray circle is .}
\label{Deployment In WSN1}
\end{figure*}






Figs. \ref{Lloyd initial deployment} and \ref{Lloyd final deployment} show one example of the initial and the finial deployments of Lloyd Algorithm in WSN1. Lloyd Algorithm assumes an  infinite communication range and requires the global knowledge of the sensor locations. Otherwise, disconnected sub-graphs run Lloyd Algorithm independently and there is no guarantee for convergence. Nonetheless, the calculation of the final distortion only considers sensors in the backbone network. In the final deployment of the example in Fig. \ref{Lloyd final deployment}, there are four sensors disconnected from the backbone network, resulting in a large distortion .
Fig. \ref{RL final deployment} shows the outcome of RL Algorithm in WSN1. After 500 iterations, the distortion is decreased from 11.30 to 0.60. Simultaneously, the coverage area is increased from 0.15 to 6.26 and the final deployment is connected.
Fig. \ref{DA final deployment} shows the final deployment of DA Algorithm in WSN1. After 500 iterations, the distortion is decreased from 11.30 to 0.32, which is better than that of RL Algorithm. Simultaneously, the coverage area is increased from 0.15 to 6.99 and full connectivity is provided. Unlike Lloyd Algorithm, both RL Algorithm and DA Algorithm guarantee connectivity.

\begin{figure}[!t]
\centering
\includegraphics[width=4.5in]{DistortionWSN1.jpg}
\caption{Comparison of distortion for different algorithms in WSN1.}
\label{distortion in homogeneous sensor network}
\end{figure}
\begin{figure}[!t]
\centering
\includegraphics[width=4.5in]{CoverageWSN1.jpg}
\caption{Comparison of coverage for different algorithms in WSN1.}
\label{coverage in homogeneous sensor network}
\end{figure}


\begin{figure}[!t]
\centering
\includegraphics[width=4.5in]{Rc.jpg}
\setlength{\belowcaptionskip}{-10pt} \caption{Relationship between performance and Rc in WSN1.}
\label{relationship between distortion, coverage and Rc}
\end{figure}


Fig. \ref{distortion in homogeneous sensor network} illustrates the performance of the above algorithms for 10 random initial deployments. As can be seen from the figure, unlike other algorithms, the performance of DA Algorithm is not sensitive to the initial deployment. In other words, DA Algorithm avoids most poor local minimum solutions. Fig. \ref{distortion in homogeneous sensor network} shows that DA Algorithm has the best performance among the three algorithms.
\begin{figure*}[!t]
\centering
\subfloat[]{\includegraphics[width=2.7in]{WSN2a.jpg}
\label{Lloyd initial deployment in WSN2}}
\hfil
\subfloat[]{\includegraphics[width=2.7in]{WSN2b.jpg}
\label{Lloyd final deployment in WSN2}}
\hfil
\subfloat[]{\includegraphics[width=2.7in]{WSN2c.jpg}
\label{RL final deployment in WSN2}}
\hfil
\subfloat[]{\includegraphics[width=2.7in]{WSN2d.jpg}
\label{DA final deployment in WSN2}}
\caption{Sensor deployments in WSN2. (a) The initial sensor deployment and the corresponding weighted Voronoi regions. (b) The final deployment of Lloyd Algorithm after 500 iterations. (c) The final deployment of RL Algorithm after 500 iterations. (d) The final deployment of DA Algorithm after 500 iterations. Strong sensors and weak sensors in the backbone network are, respectively, denoted by hollow circles and squares. Sensors out of the backbone network are denoted by dots. The corresponding centroid for strong sensors and weak sensors are, respectively, denoted by stars and crosses. The radius of each gray circle is .}
\label{DA Deployment in WSN2}
\end{figure*}
\begin{figure*}[!t]
\centering
\subfloat[]{\includegraphics[width=2.7in]{WSN3a.jpg}
\label{Lloyd initial deployment in WSN3}}
\hfil
\subfloat[]{\includegraphics[width=2.7in]{WSN3b.jpg}
\label{Lloyd final deployment in WSN3}}
\hfil
\subfloat[]{\includegraphics[width=2.7in]{WSN3c.jpg}
\label{RL final deployment in WSN3}}
\hfil
\subfloat[]{\includegraphics[width=2.7in]{WSN3d.jpg}
\label{DA final deployment in WSN3}}
\caption{Sensor deployments in WSN3. Figure (a) The initial deployment and corresponding weighted Voronoi regions. (b) The final deployment of Lloyd Algorithm after 500 iterations. (c) The final deployment of RL Algorithm after 500 iterations. (d) The final deployment of DA Algorithm after 500 iterations. Strong sensors, medium and weak sensors in the backbone network are, respectively, denoted by hollow circles, hollow squares and hollow diamonds. Sensors out of the backbone network are denoted by dots. The corresponding centroid for sensors in the backbone network are denoted by stars. The radius of each gray circle is .}
\label{DA Deployment in WSN3}
\end{figure*}



Fig. \ref{coverage in homogeneous sensor network} compares the final coverage area of RL Algorithm and DA Algorithm with that of Lloyd Algorithm. In most cases, decreasing the distortion results in increasing the coverage area as well. Intuitively, this behavior can be explained by considering coverage area as a hard-decision version of distortion. Next, the relationship between performance (distortion and coverage area) and communication range  in homogeneous WSN1 using DA Algorithm is depicted in Fig. \ref{relationship between distortion, coverage and Rc}.

Figs. \ref{Lloyd initial deployment in WSN2} and \ref{Lloyd final deployment in WSN2} show one example of the initial and the finial deployments of Lloyd Algorithm in WSN2.
As usual, the final distortion only considers sensors in the backbone network.
Initially, two strong sensors and four weak sensors are consisted in the backbone network shown in Fig. \ref{Lloyd initial deployment in WSN2}. In Fig. \ref{Lloyd final deployment in WSN2}, only one strong sensor and four weak sensors are included in the backbone network, resulting in a large distortion  which is only 0.48 smaller than the initial distortion.
Fig. \ref{RL final deployment in WSN2} shows the outcome of RL Algorithm in WSN2. After 500 iterations, the distortion is decreased from 13.15 to 5.52. Simultaneously, the coverage area is increased from 0.08 to 1.46 and the final deployment is connected.
Fig. \ref{DA final deployment in WSN2} shows the final deployment of DA Algorithm in WSN2. After 500 iterations, the distortion is decreased from 13.15 to 1.10, which is better than that of RL Algorithm. Simultaneously, the coverage area is increased from 0.08 to 2.48 and full connectivity is provided.

Figs. \ref{Lloyd initial deployment in WSN3} and \ref{Lloyd final deployment in WSN3} show one example of the initial and the finial deployments of Lloyd Algorithm in WSN3.
Initially, one strong sensor, two medium sensors and four weak sensors are consisted in the backbone network shown in Fig. \ref{Lloyd initial deployment in WSN3}. In Fig. \ref{Lloyd final deployment in WSN3}, two strong sensors, one medium sensors and one weak sensors are disconnected from the backbone network, resulting in a large distortion .
Fig. \ref{RL final deployment in WSN3} shows the outcome of RL Algorithm in WSN3. After 500 iterations, the distortion is decreased from 10.96 to 3.22. Simultaneously, the coverage area is increased from 0.08 to 1.51 and the final deployment is connected.
Fig. \ref{DA final deployment in WSN3} shows the final deployment of DA Algorithm in WSN3. After 500 iterations, the distortion is decreased from 10.96 to 1.35, which is better than that of RL Algorithm. Simultaneously, the coverage area is increased from 0.03 to 2.07 and full connectivity is provided.




\begin{figure}[!t]
\centering
\includegraphics[width=4.5in]{DistortionWSN2.jpg}
\setlength{\belowcaptionskip}{-10pt} \caption{Comparison of distortion with different algorithms in WSN2.}
\label{distortion in heterogeneous sensor network2}
\end{figure}
\begin{figure}[!t]
\centering
\includegraphics[width=4.5in]{CoverageWSN2.jpg}
\setlength{\belowcaptionskip}{-10pt} \caption{Comparison of coverage area with different algorithms in WSN2.}
\label{coverage in heterogeneous sensor network2}
\end{figure}

\begin{figure}[!t]
\centering
\includegraphics[width=4.5in]{DistortionWSN3.jpg}
\setlength{\belowcaptionskip}{-10pt} \caption{Comparison of distortion with different algorithms in WSN3.}
\label{distortion in heterogeneous sensor network3}
\end{figure}
\begin{figure}[!t]
\centering
\includegraphics[width=4.5in]{CoverageWSN3.jpg}
\setlength{\belowcaptionskip}{-10pt} \caption{Comparison of coverage area with different algorithms in WSN3.}
\label{coverage in heterogeneous sensor network3}
\end{figure}

Figs. \ref{distortion in heterogeneous sensor network2} and \ref{coverage in heterogeneous sensor network2} illustrate the performance of the above algorithms for 10 random initial deployments in WSN2. Figs. \ref{distortion in heterogeneous sensor network3} and \ref{coverage in heterogeneous sensor network3} show similar performances in WSN3. The trends in  heterogeneous WSNs 2 and 3 are similar to those in homogeneous WSN1.


















\section{Conclusions}
We studied the deployment of sensors in heterogeneous wireless sensor networks.
Similar to homogeneous WSNs, the
necessary condition for optimal deployment implies that every sensor node location
should coincide with the centroid of its own optimal sensing
region. Moreover, we considered a limited
communication range for the sensor nodes and modeled the sensor deployment
problem as a source coding problem with distortion reflecting
sensing accuracy. By defining an appropriate distortion measure, we
proposed a Restrained Lloyd algorithm and a Deterministic
Annealing algorithm to optimize sensor deployment in both
homogeneous and heterogeneous WSNs. Our simulation results
show that both DA and RL algorithms outperform the existing
Lloyd algorithm when communication range is limited and provide a fully connected network. The DA is not sensitive to initial conditions.

\appendices
\section{Proof of Proposition 1}
\begin{IEEEproof}
Let  be the pairwise weighted Voronoi region of Sensor  when we only consider Sensors  and . Then, the exact weighted Voronoi region of Sensor  is the intersection of these pairwise weighted Voronoi regions, i.e., .
We define the coordinates of ,  and  and define . Then, expanding the hyperplane equation

results in

When , the hyperplane equation is

This hyperplane is the boundary of the half space .
When ,   is defined by

When ,   is defined by

where , .
Therefore, we have

where  denotes the complementary set of .
\end{IEEEproof}

\section{Proof of Proposition 2}
\begin{IEEEproof}
Let  be the weighted Voronoi region of Sensor  when we ignore sensors with larger cost parameters. Accordingly,  is defined by

Review the definition of  in Eq. (\ref{df weighted Voronoi partition}), we have

Obviously,  is the intersection of convex regions and therefore star-shaped. The relationship between  and  is

where .
For any point  such that  and , we have . So  can be rewritten as

Due to the definitions of  and , we have

Replacing  from Eq. (\ref{18}) in Eq. (\ref{16}), we get the final result

The elements in the right side are disjoint subsets.
Without loss of generality, let us assume that the  disjoint cost parameters are ordered such that the -level cost parameter is larger than the -level cost parameter. We also call the set including the indices of all sensors with a -level cost parameter . Then:  \\
(1) For any level-1 sensor i,  is a convex set. Accordingly, the intersection  is a convex set and therefore Eq. (\ref{EQ1}) holds due to Proposition A.1 in \cite{SD}.\\
(2) Assume the equation holds for any sensor whose level is smaller than or equal to . Consider a sensor  whose level is , by using the relationship between  and  in Eq. (\ref{20}), we rewrite the objective function as

Since  is a convex set and therefore star-shaped, the partial derivative of the first term can be solved by proposition A.1 in \cite{SD}. Sensors' levels in the second term are smaller than  and thus the partial derivative of the second term can be solved by our assumption in Step (2). Therefore, the partial derivative becomes

Note that
,
where ,
is a star-shaped set consisting of several disjoint subsets.
By using Lemma \ref{Lamma2}, we have

Also, we have

Eq. (\ref{EQ1}) is derived by replacing Eqs. (\ref{part1_19}) and (\ref{part2_19}) in Eq. (\ref{EQ28}).
In other words, Eq. (\ref{EQ1}) is correct for sensors whose level is smaller than or equal to .
In summary, Eq. (\ref{EQ1}) is correct for sensors in all levels of heterogeneous WSNs.
\end{IEEEproof}












\section*{Acknowledgment}


The authors would like to thank Dr. Erdem Koyuncu for helpful discussions.


\ifCLASSOPTIONcaptionsoff
  \newpage
\fi







\begin{thebibliography}{1}
\bibitem{SD}
J. Cortes, S. Martinez and F. Bullo, ``Spatially-Distributed Coverage Optimization and Control with Limited-Range Interactions,'' {\it ESAIM: COCV} , vol. 11 , pp. 691-719, Oct. 2005.
\bibitem{CC}
X. Liu, ``Coverage with Connectivity in Wireless Sensor Networks,'' {\it Broadband Communications, Networks and Systems}, pp. 1-8, Oct. 2006.
\bibitem{wang1}
G. Wang, G. Cao, and T. F. La Porta, ``Movement-assisted sensor deployment,'' \emph{IEEE Trans. Mob. Comput.}, vol. 5, no. 6, pp. 640--652, June 2006.
\bibitem{kiefferconvergence}
J. C. Kieffer, ``Exponential rate of convergence for Lloyd's method I,'' \emph{IEEE Trans. Inf. Theory}, vol. 28, no. 2, pp. 205--210, Mar. 1982.
\bibitem{wuconvergence}
X. Wu, ``On convergence of Lloyd's method I,'' \emph{IEEE Trans. Inf. Theory}, vol. 38, no. 1, pp. 171--174, Jan. 1992.
\bibitem{qdu1}
Q. Du, M. Emelianenko, and L. Ju, ``Convergence of the Lloyd algorithm for computing centroidal Voronoi tessellations,'' \emph{SIAM J. Numer. Anal.}, vol. 44, no. 1, pp. 102--119, Feb. 2006.
\bibitem{hu1}
C. Hu, X. Wang, Z. Yang, J. Zhang, Y. Xu, and X. Gao, ``A geometry study on the capacity of wireless networks via percolation,'' \emph{IEEE Trans. Commun.}, vol. 58, no. 10, pp. 2916--2925, Oct. 2010.
\bibitem{yousefizadeh1}
H. Yousefi'zadeh, H. Jafarkhani, and J. Kazemitabar, ``A study of connectivity in MIMO fading ad-hoc networks,'' \emph{IEEE/KICS J. Commun. Networks (JCN)}, vol. 11, no. 1, pp. 47--56, Feb. 2009.
\bibitem{ICC}
G. Xing, X. Wang, Y. Zhang, C. Lu, R. Pless, and C. Gill, ``Integrated Coverage and Connectivity Configuration for Energy Conservation in Sensor Networks,'' {\it ACM Transactions on Sensor Networks}, vol. 1, pp. 36-72, Aug. 2005.
\bibitem{DCC}
X. Bai, S. Kumar, D. Xuan, Z. Yun and T. H. Lai, ``Deploying Wireless Sensors to Achieve Both Coverage and Connectivity,'' {\it MobiHoc}, pp. 131-142, May 2006.
\bibitem{CSurvey}
B. Wang ``Coverage Problems in Sensor Networks: A Survey,'' {\it ACM Computing Surveys}, vol. 43, Article 32, Oct. 2011.
\bibitem{gray1}
R. M. Gray and D. L. Neuhoff, ``Quantization,'' \emph{IEEE Trans. Inf. Theory}, vol. 44, no. 6, pp. 2325--2383, Oct. 1998.
\bibitem{gershograybook}
A. Gersho and R. M. Gray, \emph{Vector Quantization and Signal Compression}. Boston, MA: Kluwer, 1992.
\bibitem{information theory}
T. M. Cover and J. A. Thomas, \emph{Elements of Information Theory}. John Wiley and Sons, Inc., New York, 1991.
\bibitem{mahboubi1}
H. Mahboubi, K. Moezzi, A. G. Aghdam, K. Sayrafian-Pour, and V. Marbukh, ``Self-deployment algorithms for coverage problem in a network of mobile sensors with unidentical sensing ranges,'' \emph{IEEE Global Telecommun. Conf. (GLOBECOM)}, Dec. 2010.
\bibitem{HWS}
C. Wu and Y. Chung, ``Heterogeneous Wireless Sensor Network Deployment and Topology Control Based on Irregular Sensor Model,'' {\it Advances in Grid and Pervasive Computing}, vol. 4459 of the series Lecture Notes in Computer Science, pp. 78-88, 2007.
\bibitem{VD}
A. Okabe, B. Boots, K. Sugihara, and S. N. Chiu, {\it Spatial Tessellations: Concepts and Applications of Voronoi
Diagrams}, 2nd ed., Wiley Series in Probability and Statistics. New York, NY: John Wiley \& Sons, 2000.
\bibitem{voronoisurvey}
F. Aurenhammer, ``Voronoi diagrams -- a survey of fundamental geometric data structure,'' \emph{ACM Computing Surveys}, vol. 23, no. 3, pp. 345--405, Sept. 1991.
\bibitem{aurenhammerwvd}
F. Aurenhammer and H. Edelsbrunner, ``An optimal algorithm for constructing the weighted Voronoi diagram in the plane,'' \emph{Pattern Recognition}, vol. 17, no. 2, pp. 251--257, 1984.
\bibitem{Convexity}
S. Valeriu, ``Introduction to the Axiomatic Theory of Convexity,'' Russian, 1984.
\bibitem{ACA}
W. P. Soltan, ``Abstract Convex Analysis,'' {\it Canadian Mathematical Society series of monographs and advanced texts}, John Wiley and Sons, Inc., New York,  1997.
\bibitem{SA}
P.J.M van Laarhoven and E.H.L Aarts. ``Simulated Annealing: Theory and Applications,'' D. Reidel Publishing Company, Dordrecht, Holland, 1987.
\bibitem{SR}
S. Geman and D. Geman. ``Stochastic Relaxation, Gibbs Distortion and the Bayesian Restoration of Images,'' {\it IEEE Trans. Pattern Anal. and Mach. Int.}, 11(6). pp. 689-691, 1984.



\end{thebibliography}



















\end{document}
