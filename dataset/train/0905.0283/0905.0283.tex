\documentclass{llncs}

\usepackage{times}
\usepackage{mathptm}

\usepackage{algorithmic}
\usepackage[ruled]{algorithm}
\usepackage{cite}
\usepackage{listings}
\usepackage{graphicx}
\usepackage{url}

\lstset{mathescape=true}



\long\def\omit#1{\relax}    

\def \figurescale {.85} 

\begin{document}

\lstset{language=Python}

\title{Optimal Embedding Into Star Metrics}

\author{David Eppstein \and Kevin A. Wortman}

\institute{Department of Computer Science \\
Universitiy of California, Irvine \\
\email{\{eppstein, kwortman\}@ics.uci.edu}
}



\maketitle

\begin{abstract}
We  present an -time algorithm for  the following problem: given a finite metric space , create a star-topology network with the points of  as its leaves, such that the distances in the star are at least as large as in , with minimum dilation. As part of our algorithm, we solve in the same time bound the \emph{parametric negative cycle detection problem}: given a directed graph with edge weights that are increasing linear functions of a parameter , find the smallest value of  such that the graph contains no negative-weight cycles.
\end{abstract}

\section{Introduction}
\label{section:introduction}
A \emph{metric space} is a set of sites separated by symmetric positive distances that obey the triangle inequality. If  and  are metric spaces and  does not decrease the distance between any two points, the \emph{dilation} or \emph{stretch factor} of  is

We define a \emph{star metric} to be a metric space in which there exists a \emph{hub}  such that, for all  and , . Given the distance matrix of an -point metric space , we would like to construct a function  that maps  into a star metric , that does not decrease distances, and that has as small a dilation as possible. In this paper we describe an algorithm that finds the optimal  in time .
Our problem may be seen as lying at the confluence of three major areas of algorithmic research:

\smallskip\noindent
{\bf Spanner construction.} A \emph{spanner} for a metric space  is a graph  with the points of  as its vertices and weights (lengths) on its edges, such that path lengths in  equal or exceed those in ; the dilation of  is measured as above as the maximum ratio between path length and distance in . The construction of sparse spanners with low dilation has been extensively studied~\cite{Epp-HCG-00} but most papers in this area limit themselves to bounding the dilation of the spanners they construct rather than constructing spanners of optimal dilation.  Very few optimal spanner construction problems are known to be solvable in polynomial time; indeed, some are known to be NP-complete \cite{klein_kutz} and others NP-hard \cite{1273697,edmonds_2008}. Our problem can be viewed as constructing a spanner in the form of a star (a tree with one non-leaf node) that has optimal dilation.

\smallskip\noindent
{\bf Metric embedding.} There has been a large amount of work within the algorithms community on \emph{metric embedding} problems, in which an input metric space is to be embedded into a simpler target space with minimal distortion~\cite{Lin-ICM-02}; typical target spaces for results of this type include spaces with  norms and convex combinations of tree metrics. As with spanners, there are few results of this type in which the minimum dilation embedding can be found efficiently; instead, research has concentrated on proving bounds for the achievable dilation. Our result provides an example of a simple class of metrics, the star metrics, for which optimal embeddings may be found efficiently. As with embeddings into low-dimensional  spaces, our technique allows an input metric with a quadratic number of distance relationships to be represented approximately using only a linear amount of information.

\smallskip\noindent
{\bf Facility location.} In many applications one is given a collection of \emph{demand points} in some space and must select one or more \emph{supply points} that maximize some objective function. For instance, the 1-median (minimize the sum of all distances from demand points to a single supply point) and 1-center (minimize the greatest distance between any destination point and a single supply point) can be applied to operational challenges such as deciding where to build a radio transmitter or railroad hub so as to maximize its utility \cite{drezner_2002}.  In a similar vein the problem discussed in this paper may be seen as selecting a single supply point to serve as the hub of a star-topology network.  In this context dilation corresponds to the worst multiplicative cost penalty imposed on travel between any pair of input points due to the requirement that all travel is routed through the hub (center) point.  Superficially, our problem differs somewhat from typical facility location problems in that the star we construct has a hub that is not given as part of the input. However, it is possible to show that the hub we find belongs to the \emph{tight span} of the input metric space~\cite{DreHubMou-DM-01}, a larger metric space that has properties similar to those of  spaces. Viewing our problem as one of selecting the optimal hub point from the tight span gives it the format of a facility location problem.

Previously~\cite{Eppstein200727} we considered similar minimum dilation star problems in which the input and output were both confined to low-dimensional Euclidean spaces. As we showed, the minimum-dilation star with unrestricted hub location may be found in  expected time in any bounded dimension, and for  the optimal hub among the input points may be selected in expected time , where  is the inverse Ackermann function. For the general metric spaces considered here, the difficulty of the problems is reversed: it is trivial to select an input point as hub in time , while our results show that an arbitrary hub may be found in time .

As we discuss in Section \ref{section:linear_program}, the minimum dilation star problem can be represented as a linear program; however solving this program directly would give a running time that is a relatively high order polynomial in  and in the number of bits of precision of the input matrix.  In this paper we seek a faster, purely combinatorial algorithm whose running time is strongly polynomial in~.  Our approach is to first calculate the dilation  of the optimal star.  We do this by forming a \emph{-graph} : a directed graph with weights in the form  for parameters  and  determined from the input metric.   has the property that it contains no negative weight cycles if and only if there exists a star with dilation .  Next we calculate , the smallest value such that  contains no negative-weight cycles, which is also the dilation of the star we will eventually create.  Finally we use  and  to compute the lengths of the edges from the star's center to each site, and output the resulting star.

Our algorithm for computing , the smallest parameter value admitting no negative cycles in a parametrically weighted graph, warrants independent discussion.  To our knowledge no known strongly polynomial algorithm solves this problem in full generality.  Karp and Orlin \cite{ko80} gave an  time algorithm for a problem in which the edge weights have the same form  as ours, but where each  is restricted to the set .  If all , the problem is equivalent to finding the minimum mean cycle in a directed graph \cite{karp78}, for which several algorithms run in  time \cite{dasdan98experimental}.  In our problem, each  may be any nonnegative real number;  it is not apparent how to adapt the algorithm of Karp and Orlin to our problem. Gusfield provided an upper bound \cite{gusfield80} on the number of breakpoints of the function describing the shortest path length between two nodes in a -graph, and Carstensen provided a lower bound \cite{carstensen84} for the same quantity; both bounds have the form .  Hence any algorithm that constructs a piecewise linear function that fully describes path lengths for the entire range of  values takes at least  time.   In Section \ref{section:cycle_detection} we describe our algorithm, which is based on a dynamic programming solution to the all pairs shortest paths problem.  Our algorithm maintains a compact piecewise linear function representing the shortest path length for each pair of vertices over a limited range of  values, and iteratively contracts the range until a unique value  can be calculated.  Thus it avoids Carstensen's lower bound by finding only the optimal , and not the other breakpoints of the path length function, allowing it to run in  time.

\begin{figure}
\centering
\scalebox{\figurescale}{\includegraphics{example.eps}}
\caption{Example of a metric space and its optimal star, which has dilation .}
\label{figure:example}
\end{figure}

\section{Linear Programming Formulation}
\label{section:linear_program}

In this section we formally define the overall minimum dilation star problem and describe how to solve it directly using linear programming.  Our eventual algorithm never solves nor even constructs this linear program directly; however stating the underlying linear program and its related terminology will aid our later exposition.

The input to our algorithm is a finite \emph{metric space}.  Formally, a metric space  is a tuple , where  is a set of sites and the function  maps any pair of sites to the nonnegative, real distance between them.  The following \emph{metric conditions} also hold for any :
\begin{enumerate}
\item  if and only if  (positivity);
\item  (symmetry); and
\item  (the triangle inequality).
\end{enumerate}
\noindent The input to our algorithm is a finite metric space ; we assume that the distance  between any  may be reported in constant time, for instance by a lookup matrix.

A \emph{star} is a connected graph with one \emph{center} vertex.  A star contains an edge between the center and every other vertex, but no other edges.  Hence any star is a tree of depth 1, and every vertex except the center is a leaf.  Our algorithm must output a weighted star  whose leaves are the elements  from the input.  The edge weights in  must be at least as large as the distances in , and must obey reflexivity and the triangle inequality.  In other words, if  is the length of a shortest path from  to  in , then , , and  for any vertices  in .

We also ensure that the \emph{dilation} of  is minimized.  For any two vertices  in some weighted graph  whose vertices are points in a metric space, the dilation between  and  is

\noindent The dilation of the entire graph  is the largest dilation between any two vertices, i.e.

Our output graph  is a star; hence every path between two leaves has two edges, so if we apply the definition of dilation to , we obtain

\noindent where  is the weight of the edge connecting  and  in .  Hence the dilation of  may be computed by

\noindent This equation lays the foundation for our formulation of the minimum dilation star problem as a linear program.

\begin{definition}
\label{definition:L}
Let  be the following linear program, defined over the variables  and  for every :

\begin{center}
\noindent Minimize 
\end{center}
\noindent such that for any ,

\noindent and for any ,

\noindent Let  be the value assigned to  in the optimal solution to .  In other words,  is the smallest dilation admitted by any set of distances satisfying all the constraints of .
\end{definition}

\noindent  is clearly feasible.  For example, if , then the solution  and  is a feasible, though poor, solution.

\begin{lemma}
For any optimal solution of , the value of  gives the minimum dilation of any star network spanning , and the  values give the edge lengths of an optimal star network spanning .
\end{lemma}

\begin{proof}
Each variable  corresponds to the weight  of the edge between  and  in .  Inequality \ref{equation:distance_nonzero} ensures that the distances are nonnegative, Inequality
\ref{equation:distance_triangle} ensures that they obey the triangle inequality, and Inequality
\ref{equation:distance_dilation} dictates that  is a largest dilation among any pair of sites from .  The value of  is optimal since  is defined to minimize .
\end{proof}

Unfortunately  contains  variables and  constraints.  Such a program could be solved using general purpose techniques in a number of steps that is a high-order polynomial in  and the number of bits of precision used, but our objective is to obtain a fast algorithm whose running time is strongly polynomial in .  Megiddo showed \cite{megiddo:347} that linear programs with at most two variables per inequality may be solved in strongly polynomial time; however our type (\ref{equation:distance_dilation}) inequalities have three variables, so those results cannot be applied to our problem.

\section{Reduction to Parameteric Negative Weight Cycle Detection}
\label{section:reduction}

In this section we describe a subroutine that maps the set of sites  to a directed, parametrically-weighted -graph .  Every edge of  is weighted according to a nondecreasing linear function of a single graph-global variable .  An important property of  is that the set of values of  that cause  to contain a negative weight cycle is identical to the set of values of  that cause the linear program  to be infeasible.  Thus any assignment of  for which  contains no negative weight cycles may be used in a feasible solution to .

\begin{definition}
\label{definition:lambda_graph}
A \emph{-graph} is a connected, weighted, directed graph, where the weight  of any edge  is defined by a linear function in the form

\noindent where  and  are real numbers and .
\end{definition}

\begin{definition}
\label{definition:G}
Let  be the -graph corresponding to a particular set of input sites .   has vertices  and  for each .  For ,  has an edge of length  from  to , and for ,  has an edge of length  from  to .
\end{definition}

\noindent Note that an edge from  to  has weight  when .  An example -graph  for  is shown in Figure~\ref{figure:G}.

\begin{figure}
\centering
\scalebox{\figurescale}{\includegraphics{G.eps}}
\caption{The graph  for .  The weights of grayed edges are omitted.}
\label{figure:G}
\end{figure}

\begin{lemma}
\label{lemma:make_graph_basics}
 may be constructed in  time.
\end{lemma}

\begin{proof}
 has  vertices and  edges, each of which may be initialized in constant time.
\end{proof}

\omit{
We now prove that feasibility of  is equivalent to the absence of negative weight cycles in .  We begin by showing that feasibility implies the absence of negative weight cycles.
}

\begin{lemma}
\label{lemma:nnwc_implies_feasible}
If  is assigned such that  has a feasible solution, then  contains no negative weight cycle.
\end{lemma}

\begin{proof}
Since  is bipartite, any sequence of edges  traversed by a cycle in  has even length.  Depending on which partition  begins with, the sequence either takes the form

\noindent or

\noindent where  are vertices from .  In either case, the cycle has weight

\noindent by the commutativity of addition.  Since  is feasible, there exists some set of distances  satisfying the constraints of , i.e.

\noindent and

\noindent Substituting (\ref{ineq1}) and (\ref{ineq2}) into (\ref{cycle_weight}), we obtain

\end{proof}

\omit{
We now summarize this section's results as a single theorem.
}

\begin{theorem}
\label{theorem:create_graph}
Any set  of  sites from a metric space may be mapped to a -graph  with  vertices, such that for any ,  contains a negative weight cycle if and only if  is infeasible for that value of .  The mapping may be accomplished in  time.
\end{theorem}

\begin{proof}
By Lemma \ref{lemma:make_graph_basics},  may be created in  time, and by Lemma \ref{lemma:nnwc_implies_feasible}, feasibility of  implies an absence of negative cycles in .  Section \ref{section:extracting_weights} describes an algorithm that, given a value  for which  has no negative cycle, generates an edge length  for every  that obeys the constraints of .  Thus, by the correctness of that algorithm, an absence of negative cycles in  implies feasibility of .
\end{proof}

\section{Searching for }
\label{section:cycle_detection}

We now turn to the problem of computing the quantity .  This problem is an example of \emph{parametric negative weight cycle detection}: given a -graph , find , the smallest value such that  contains no cycles of negative weight.  Our algorithm functions by maintaining a range  which is known to contain .  Initially the range is ; over  iterations, the range is narrowed until it is small enough that  may be calculated easily.  This approach is similar in spirit to Megiddo's general parametric search framework \cite{804326,322410}, which, in loose terms, searches for the solution to an optimization problem by simulating the execution of a parallel algorithm for the corresponding decision problem.

Our algorithm is presented in Listing \ref{algorithm:computing_lambda_star}.  It is an adaptation of a parallel all pairs shortest paths algorithm based on matrix squaring \cite{savage77}.  The original algorithm uses a matrix , which stores the weight of the shortest path from  to  among paths with at most  edges.  Each  may be defined as the smallest sum of two cells of , and  defines the shortest paths in the graph.  In the context of that original algorithm, edges and paths had real-number lengths, so it was sufficient to store real numbers in .  In the context of this paper, an edge's weight is a linear function of a variable ; hence the weight of a path is a linear function of .  Unfortunately the minimum-cost path between  and  may be different for varying values of , so the weight of the shortest path from  to  is defined by the minima of one or more linear functions of .  Such a lower envelope of linear functions may be represented by a piecewise linear function; hence each element of  must store a piecewise linear function.  Without further attention the number of breakpoints in these piecewise linear functions would grow at every iteration, and eventually operating on them would dominate our algorithm's running time.  To address this, at every iteration we choose a new interval  that contains no breakpoints, so that every  may be compacted down to a single linear function.

\floatname{algorithm}{Listing}
\begin{algorithm}[p]
\caption{Computing the quantity .}
\label{algorithm:computing_lambda_star}
\begin{algorithmic}[1]
\STATE {\bf INPUT:} A -graph  with  vertices .
\STATE {\bf OUTPUT:} , the smallest value of  such that  has no negative-weight cycles.
\STATE Let  and .
\STATE {\bf INVARIANT:} 
\STATE {\bf INVARIANT:}  contains a linear function that represents the length of the shortest path from  to  among the subset of paths that use at most  edges, as a function of , for any 
\STATE Let  be an  matrix of piecewise linear functions.
\STATE Initialize u=vG(\lambda)euv
\FOR{  }
  \FOR{  }
    \STATE  \label{line:define_D}
  \ENDFOR

  \STATE Let  be the set of breakpoints of the piecewise linear functions stored in the entries of .

  \STATE Perform a binary search among the values in , seeking an interval bounded by two consecutive breakpoints that contains . At each step, the test value of the binary search is less than  if and only if setting  equal to the test value causes  to contain a negative cycle; use the Bellman--Ford shortest paths algorithm to determine whether this is the case.

  \STATE Set  and  to the endpoints of the interval found in the previous step.

  \FOR{  }
    \STATE Replace the piecewise linear function  with the equivalent linear function over the range . \label{line:simplify_D}
  \ENDFOR  
\ENDFOR
\STATE Compute , the smallest value in the range , such that  for every .
\STATE \textbf{Return} .
\end{algorithmic}
\end{algorithm}

\begin{lemma}
\label{lemma:lambda_star_correct}
For any , the function  as computed in the listing evaluates to the weight of the shortest path from  to  among paths with at most  edges, or  if no such path exists.
\end{lemma}

\begin{proof}
We argue by induction on .  In the base case ,  must represent the weight of shortest path from  to  that includes up to  edges.  The only such paths are trivial paths, for which  and , and single edge paths, for which the path length equals the edge length.

For , each  is first defined as the lower envelope of two entries of  in line \ref{line:define_D}, then redefined as a strictly linear function over the new smaller range  in line \ref{line:simplify_D}, so we argue that the lemma holds after each assignment.  In the first assignment,  is defined to be the lower envelope of  for all ; in other words, every  is considered as a potential ``layover'' vertex, and  is defined as a piecewise linear function that may be defined by differing layover vertices throughout the range .  By the inductive hypothesis, the  values represent weights of minimum cost paths with at most  edges; hence the resulting  values represent weights of minimum cost paths with at most  edges.

When  is reassigned in line \ref{line:simplify_D}, the range endpoints  and  have been contracted such that no entry of  contains breakpoints in the range .  Hence any individual  has no breakpoints in that range, and is replaced by a simple linear function.  This transformation preserves the condition that  represents the weight of the shortest path from  to  for any .
\end{proof}

\begin{lemma}
\label{lemma:binary_search_decider}
Given two values  and  such that , it is possible to decide whether , , or , in  time.
\end{lemma}

\begin{proof}
By Lemma \ref{lemma:nnwc_implies_feasible}, for any value , if  contains a negative cycle when , then .  So we can determine the ordering of , and  using the Bellman--Ford shortest paths algorithm \cite{bellman1958,ford1962} to detect negative cycles, as follows.  First run Bellman--Ford, substituting  to evaluate edge weights.  If we find a negative cycle, then report that .  Otherwise run Bellman--Ford for ; if we find a negative cycle, then  must be in the range .  If not, then .  This decision process invokes the Bellman--Ford algorithm once or twice, and hence takes  time.
\end{proof}

\begin{lemma}
\label{lemma:lambda_star_time}
The algorithm presented in Listing \ref{algorithm:computing_lambda_star} runs in  time.
\end{lemma}

\begin{proof}
Each  is a linear function, so each  is a linear function as well.   is defined as the lower envelope of  such linear functions, which may be computed in  time \cite{textbook}.  So each  may be computed is  time, and all  iterations of the first inner for loop take  total time.  Each  represents the lower envelope of  lines, and hence has  breakpoints.  So the entries of  contain a total of  breakpoints, and they may all be collected and sorted into  in  time.  Once sorted, any duplicate elements may be removed from  in  time.

Next our algorithm searches for a new, smaller  range that contains .  Recall that  is the value of  for which  contains no negative weight cycle, and every entry of  is a piecewise linear function comprised of non-decreasing linear segments; so it is sufficient to search for the segment that intersects the  line.  We find this segment using a binary search in .  At every step in the search, we decide which direction to seek using the decision process described in Lemma \ref{lemma:binary_search_decider}.  Each decision takes  time, and a binary search through the  elements of  makes  decisions, so the entire binary search takes  time.

Replacing an entry of  with a (non-piecewise) linear function may be done naively in  time by scanning the envelope for the piece that defines the function in the range .  So the second inner for loop takes  total time, and the outer for loop takes a total of  time.

The initialization before the outer for loop takes  time.  The last step of the algorithm is to compute , the smallest value in the range  such that  for every .  At this point each  is a non-piecewise increasing linear function, so this may be done by examining each of the  linear functions , solving for its -intercept, and setting  to be the largest intercept.  This entire process takes  time, so the entire algorithm takes  time.
\end{proof}

\begin{theorem}
\label{theorem:lambda_star_algorithm}
The algorithm presented in Listing \ref{algorithm:computing_lambda_star} calculates  in  time.
\end{theorem}

\omit{
\begin{proof}
The theorem follows from Lemmas \ref{lemma:lambda_star_correct} and \ref{lemma:lambda_star_time}.
\end{proof}
}

\section{Extracting the Edge Weights}
\label{section:extracting_weights}

Once  has been calculated, all that remains is to calculate the weight of every edge in the output star.  Our approach is to create a new graph , which is a copy of  with the addition of a new source node  with an outgoing weight 0 edge to every  (see Figure \ref{figure:G_prime}).  We then compute the single source shortest paths of  starting at , and define each  to be a function of the shortest path lengths to  and .  This process is a straightforward application of the Bellman--Ford algorithm, and hence takes  time.  The remainder of this section is dedicated to proving the correctness of this approach.

\begin{figure}
\centering
\scalebox{\figurescale}{\includegraphics{G_prime.eps}}
\caption{The graph  for .  The weights of grayed edges are omitted.}
\label{figure:G_prime}
\end{figure}


\begin{definition}
Let  be a copy of the graph  described in Definition~\ref{definition:G}, with all edge weights evaluated to real numbers for , and the addition of a \emph{source vertex}  with an outgoing 0-weight edge to every .  Let  be a shortest path from  to  for any vertex , and let  be the total weight of any such .  The operation  yields the path formed by appending the edge  to .
\end{definition}

\begin{definition}
Define 
\end{definition}

We now show that our choice of  satisfies all three metric space properties.

\begin{lemma}
Every  satisfies .
\end{lemma}

\omit{
\begin{proof}
Suppose  visits vertices .  By assumption all cycles have nonnegative weight, so the cycle  has nonnegative weight.  Hence .  By definition , so

\noindent By construction  contains an edge from  to  with weight 0; hence the weight of the shortest path to  must obey .  By substitution,

\end{proof}
}

\begin{proof}
For each vertex  there exists an edge from  to  with weight 0.
\end{proof}

\begin{lemma}
Every distinct  and  satisfy 
\end{lemma}
\begin{proof}
By the definition of shortest paths, we have

\noindent and by symmetric arguments,

\noindent Adding these inequalities, we obtain

\end{proof}

\begin{lemma}
Every distinct  and  satisfy 
\end{lemma}

\begin{proof}
Observe that the path  is a path to  with weight , and that the path  is a path to  with weight .  By definition  is a shortest path to , and similarly  is a shortest path to , so we have

\noindent and

\noindent Adding these inequalities, we obtain

\noindent By assumption , so

\end{proof}

\omit{
We now summarize this section's result in a theorem.
}

\begin{theorem}
\label{theorem:find_lambda}
Given  and the corresponding  and , a set  of edge lengths  for each ,  such that for every 

\noindent and for every distinct 


\noindent may be computed in  time.
\end{theorem}

Theorem \ref{theorem:find_lambda} establishes that for any  there exists a set  of valid edge lengths.  This completes the proof of Theorem \ref{theorem:create_graph}.

\section{Conclusion}
\label{section:conclusion}

Finally we codify the main result of the paper as a theorem.

\begin{theorem}
Given a set  of  sites from a metric space , it is possible to generate a weighted star  such that the distances between vertices of  obey the triangle inequality, and such that  has the smallest possible dilation among any such star, in  time.
\end{theorem}

\omit{
\begin{proof}
The result follows from Theorems \ref{theorem:create_graph}, \ref{theorem:lambda_star_algorithm}, and \ref{theorem:find_lambda}.
\end{proof}
}

\subsubsection*{Acknowledgements}

This work was supported in part by NSF grant
0830403 and by the Office of Naval Research under grant
N00014-08-1-1015.

\small\raggedright
\bibliographystyle{abbrv} \bibliography{star_metrics}

\end{document}
