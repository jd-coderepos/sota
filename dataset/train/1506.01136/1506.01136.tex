









\documentclass[journal]{IEEEtran}

\usepackage{booktabs,threeparttable}
\usepackage{graphics} \usepackage{epsfig} \usepackage{mathptmx} \usepackage{times} \usepackage[fleqn]{amsmath} \usepackage{amssymb}  \usepackage{amsthm}
\usepackage{fancyhdr}
\usepackage{subfigure}
\usepackage{graphicx}
\usepackage{algorithmic}
\usepackage{algorithm}
\usepackage{mathrsfs}
\usepackage[]{units}
\newcommand{\makenumbered}[2]{
\newcounter{#1}[section]
\newenvironment{#1}{\begin{list}{}{
  \setlength{\itemsep}{0in}\setlength{\labelwidth}{0in}\setlength{\labelsep}{0in}\setlength{\rightmargin}{0in}\setlength{\leftmargin}{0in}\setlength{\parsep}{0in}\setlength{\listparindent}{\parindent}}\refstepcounter{#1}\item[] {\small\bf #2} }{\end{list}}}
\makenumbered{theoremn}{Theorem \thetheoremn:}\def\thetheoremn{\thesection.\arabic{theoremn}}
\makenumbered{lemman}{Lemma \thelemman:}\def\thelemman{\thesection.\arabic{lemman}}
\newtheorem{definition}{\hspace{1em}Definition}


\begin{document}

\title{Efficient Channel-Hopping Rendezvous Algorithm Based on Available Channel Set}


\author{Lu Yu, Hai Liu,~\IEEEmembership{Member,~IEEE}, Yiu-Wing Leung,~\IEEEmembership{Senior Member,~IEEE}, \\Xiaowen Chu,~\IEEEmembership{Senior Member,~IEEE}, and~Zhiyong~Lin
\IEEEcompsocitemizethanks{\IEEEcompsocthanksitem L. Yu, H. Liu, Y.-W. Leung, and X. Chu are with the Department of Computer Science, Hong Kong Baptist University, Kowloon Tong, Hong Kong SAR.\protect\\
E-mail: \{lyu, hliu, ywleung, chxw\}@comp.hkbu.edu.hk.
\IEEEcompsocthanksitem Z. Lin is with the Department of Computer Science, GuangDong Polytechnic Normal University, GuangZhou, China.\protect\\ Email: sophyca.lin@gmail.com.
\IEEEcompsocthanksitem A preliminary version of this paper was presented at ICC2014.}\thanks{}}


\maketitle


\begin{abstract}
In cognitive radio networks, rendezvous is a fundamental operation by which two cognitive users establish a communication link on a commonly-available channel for communications. Some existing rendezvous algorithms can guarantee that rendezvous can be completed within finite time and they generate channel-hopping (CH) sequences based on the whole channel set. However, some channels may not be available (e.g., they are being used by the licensed users) and these existing algorithms would randomly replace the unavailable channels in the CH sequence. This random replacement is not effective, especially when the number of unavailable channels is large. In this paper, we design a new rendezvous algorithm that attempts rendezvous on the available channels only for faster rendezvous. This new algorithm, called \emph{Interleaved Sequences based on Available Channel set} (ISAC), constructs an odd subsequence and an even subsequence and interleaves these two subsequences to compose a CH sequence. We prove that ISAC provides guaranteed rendezvous (i.e., rendezvous can be achieved within finite time). We derive the upper bound on the maximum time-to-rendezvous (MTTR) to be   under the symmetric model and   under the asymmetric model, where  and  are the number of available channels of two users and   is the total number of channels (i.e., all potentially available channels). We conduct extensive computer simulation to demonstrate that ISAC gives significantly smaller MTTR than the existing algorithms.
\end{abstract}

\begin{IEEEkeywords}
cognitive radio networks; rendezvous; channel hopping
\end{IEEEkeywords}



\section{Introduction}
\IEEEPARstart{I}{n} conventional wireless networks, a significant portion of the licensed spectrum is under-utilized while the unlicensed spectrum is over-crowded [1]. Cognitive radio networks (CRNs) can flexibly exploit the spectrum to improve its utilization. In CRNs, each unlicensed user (or cognitive user, secondary user, SU) uses a cognitive radio to: i) identify the vacant portions of the licensed spectrum by applying a spectrum sensing method [2], and ii) use a vacant portion for communications by adjusting its transmission parameters. For ease of operation, the spectrum is typically divided into channels. If two or more cognitive users want to communicate with each other, they must select a channel which is available to all of them and establish a communications link on this channel. This operation is known as \emph{rendezvous}. For convenience, we refer to ``cognitive users" as ``users" in this paper.
\par Channel hopping (CH) is one of the most representative approaches to rendezvous. Using this approach, each cognitive user follows a certain CH sequence to hop among the channels. When the cognitive users hop on the same channel at the same time, they can exchange handshaking messages to establish a communication link on this channel. Several rendezvous algorithms have been proposed in the literature for generating CH sequences (a detailed survey can be found in [4] and a brief review is given in Section II). They can be divided into two types: i) guaranteed rendezvous (i.e., rendezvous can be completed within finite time) [5, 8, 23], and ii) unguaranteed rendezvous (i.e., rendezvous may take infinite time) [9, 10]. The former type is desirable because it can provide better perception to the users.


\par For the existing algorithms that provide guaranteed rendezvous [5, 8, 23], they generate the CH sequences based on the whole channel set (i.e., set of all channels). In practice, some channels may not be available because: i) these channels are being used by the licensed users, or ii) the channels are not being used but the cognitive users cannot identify these idle channels because of the practical limitations of the spectrum sensing methods [3]. In fact, the available channel set is usually a small subset of the whole channel set. For example, work in [3] optimizes the detection (sensing) time for channel efficiency. It shows that when the SNR (signal-to-noise ratio) is dB, the SUs can only detect one idle channel in 15\% of the total idle period. In other words, on average, at most 15\% of channels can be correctly detected as ``available" for data transmission of SUs. If a cognitive user follows a CH sequence which includes both the available and the unavailable channels, this user would waste time in attempting rendezvous on the unavailable channels. This would increase the time to complete rendezvous, especially when the available channel set is a small subset of the whole channel set. Table I summarizes the characteristics of existing representative CH algorithms.
\begin{table*}
\caption{Summary of Representative CH Algorithms}
\centering
\newcommand{\tabincell}[2]{\begin{tabular}{@{}#1@{}}#2\end{tabular}}
\begin{tabular}{|c|c|c|c|c|c|}
\hline
Algorithms&\tabincell{c}{Channel Set based on\
&&i+2k=a+bm_p\\
&&j+k=a+cn

&&k=bm_p-cn+j-i

&&h-1\equiv t+t_0^1-1~mod~(m_p)\\
&&g-1\equiv t+t_0^2-1~mod~(n)

&&t\equiv (h-t_0^1)~mod~(m_p)\\
&&t\equiv (g-t_0^2)~mod~(n)

t_2=t_1+\mu\times m_p

t_2=t_1+a\times(n-1)+b\times n

&&\mu\times m_p=a\times (\delta\times m_p-1)+b\times\delta\times m_p\\
&&a=((a+b)\times\delta-\mu)\times m_p

\par That is,  should be divisible by , which contradicts the requirement . So any pair of channels  and  in any consecutive time-span of  time slots appears once and only once.
\par This result implies the guaranteed rendezvous of the two users. Notice that there are  combinations of  and  values (i.e.,  channel pairs of  and ) in at most  time slots. Hence, the combinations in  time slots are different to each other. We further derive an upper bound of TTR as follows. If there are  commonly-available channels of the two users, there are  pairs of commonly-available channels among all the  pairs of  and . The worst case is that the whole  times rendezvous occur in the last  time slots of channel hopping sequence. Therefore, rendezvous will occur in at most  time slots in the Even Subsequence and the corresponded time slot in the whole CH sequence is . In other words, .
\par Then we illustrate the case when  and . If , the only channel of sender must be a commonly-available channel of the two users. The sender stay on the only channel while the receiver will hop on this channel in at most  time slots. If , all channels in the odd time slots of sender are one channel and all channels in the even time slots are another channel. No matter which one is a commonly-available channel, the receiver will hop on this channel in at most  time slots in its OS or ES while the corresponded time slot in the whole CH sequence is .
\par To sum up, two users performing the ISAC algorithm can achieve rendezvous in at most  times lots, where  and  are the numbers of channels of the two users,  is the smallest prime number which is not smaller than ,  is the number of commonly-available channels of the two users.
\end{proof}
\section{Simulation}
We built a simulator in Visual Studio 2010 to evaluate the performance of our proposed ISAC algorithm. We consider the following algorithms for comparison: i) MMC [9] (MMC generates CH sequences based on available channel set), ii) Jump-Stay [5] [6] and iii) CRSEQ [8] (Among existing algorithms, Jump-Stay and CRSEQ have been shown to have good performance [7]). We remind that Jump-Stay and CRSEQ apply a random replacement operation to randomly replace the unavailable channels in the CH sequences by the available ones (see discussion in Section I). We introduce a parameter  to control the ratio of the number of available channels to the total number of channels . Available channels are randomly selected from the whole channel set such that the average number of available channels is equal to . Besides , we are concerned with parameter , i.e., the number of commonly-available channels of the two users involved in the rendezvous. For each , we let  properly vary in . For each combination of parameter values, we perform 500,000 independent runs and compute the average TTR, the maximum TTR (MTTR) and the variance of TTR accordingly. The variance of TTR can reveal the stability of performance of rendezvous algorithms. Notice that,  is the number of available channels of receiver, and  is the smallest prime number which is not smaller than the number of available channels of sender.
\subsection{Effectiveness of ISAC}
In this subsection, we demonstrate that  ISAC can effectively improve the rendezvous performance. We consider three scenarios with different number of available channels: 1)  is small and equal to 0.1 (i.e., ratio of the number of available channels to the total number of channels is 0.1), ii)  is moderate and equal to 0.4, and iii)  is large and equal to 0.8.
\subsubsection{Under the Symmetric Model}
\begin{itemize}
\item\emph{Small }: Firstly we study the scenario when  is small. We set  and . Fig. 7(a) shows the average TTRs and Fig. 7(b) shows the maximum TTRs of different algorithms against the total number of channels . Since the maximum TTR of MMC is much bigger than that of other algorithms, we draw one more graph to compare the maximum TTRs of other three algorithms (i.e., Fig. 7(c)). Fig. 7(d) shows the variance of different algorithms. There is no large gap between different algorithms in terms of average TTR. However, the ISAC has significant improvement on the maximum TTR and the variance of TTR. For example, we suppose that each time slot has duration of 20ms [5]. When there are 50 channels, the average TTRs of ISAC, MMC, Jump-Stay and CRSEQ are 4.20 slots (0.08s), 6.85 slots (0.14s), 5.00 slots (0.10s) and 4.99 slots (0.10s) while the maximum TTRs of ISAC, MMC, Jump-Stay and CRSEQ are 8 slots (0.16s), 393 slots (7.86s), 63 slots (1.26s) and 72 slots (1.44s), respectively. In this example, . According to Theorem IV.1, . This result shows that the upper bound given by Theorem IV.1 is tight. The variances of TTR of ISAC, MMC, Jump-Stay and CRSEQ in this case are 6.57, 183.77, 20.03 and 24.02, respectively. The performance of ISAC is the most stable one.
\begin{figure}
\centering
\subfigure[Average TTR VS. ]{
\label{fig:subfig:a} \includegraphics[scale=0.5]{F7a.pdf}}
\hspace{1in}
\subfigure[Maximum TTR VS. ]{
\label{fig:subfig:b} \includegraphics[scale=0.5]{F7b.pdf}}
\hspace{1in}
\subfigure[Maximum TTR VS.  (without MMC)]{
\label{fig:subfig:b} \includegraphics[scale=0.5]{F7c.pdf}}
\hspace{1in}
\subfigure[Variance of TTR VS. ]{
\label{fig:subfig:b} \includegraphics[scale=0.5]{F7d.pdf}}
\caption{Comparison of different algorithms under the symmetric model when .}
\end{figure}
\item \emph{Moderate }: Then we study the scenario when  is moderate. We set  and . Fig. 8 shows the average TTRs, the maximum TTRs and the variances of TTR of different algorithms against . We find that ISAC has about the same average TTR as the existing algorithms, but its maximum TTR is much smaller than that of the existing algorithms. For example, when there are 50 channels, the maximum TTRs of ISAC, MMC, Jump-Stay and CRSEQ are 45, 870, 239 and 310, respectively. In this example, . According to Theorem IV.1, . Again, this result shows that the upper bound given by Theorem IV.1 is tight. The variances of TTR of ISAC, MMC, Jump-Stay and CRSEQ in this case are 208.99, 539.73, 351.34 and 526.13, respectively. The performance of ISAC is the most stable one.
\begin{figure}
\centering
\subfigure[Average TTR VS. ]{
\label{fig:subfig:a} \includegraphics[scale=0.5]{F8a.pdf}}
\hspace{1in}
\subfigure[Maximum TTR VS. ]{
\label{fig:subfig:b} \includegraphics[scale=0.5]{F8b.pdf}}
\hspace{1in}
\subfigure[Maximum TTR VS.  (without MMC)]{
\label{fig:subfig:b} \includegraphics[scale=0.5]{F8c.pdf}}
\hspace{1in}
\subfigure[Variance of TTR VS. ]{
\label{fig:subfig:b} \includegraphics[scale=0.5]{F8d.pdf}}
\caption{Comparison of different algorithms under the symmetric model when .}
\end{figure}
\item \emph{Large }: Then we study the scenario when  is moderate. We set , . Fig. 9 shows the results. We see that ISAC is particularly advantageous in terms of the maximum TTR. For example, when there are 50 channels, the average TTRs of ISAC, MMC, Jump-Stay and CRSEQ are 39.92, 39.42, 33.29 and 58.65 while the maximum TTRs are 80, 3431, 489 and 1267, respectively. The maximum TTR of ISAC is almost one tenth of others. In this example, , . According to Theorem IV.1, . Again, this result shows that the upper bound given by Theorem IV.1 is tight. The variances of TTR of ISAC, MMC, Jump-Stay and CRSEQ in this case are 564.05, 2495.31, 1098.35 and 4933.48, respectively. The performance of ISAC is the most stable one.
\begin{figure}
\centering
\subfigure[Average TTR VS. ]{
\label{fig:subfig:a} \includegraphics[scale=0.5]{F9a.pdf}}
\hspace{1in}
\subfigure[Maximum TTR VS. ]{
\label{fig:subfig:b} \includegraphics[scale=0.5]{F9b.pdf}}
\hspace{1in}
\subfigure[Maximum TTR VS.  (without MMC)]{
\label{fig:subfig:b} \includegraphics[scale=0.5]{F9c.pdf}}
\hspace{1in}
\subfigure[Variance of TTR VS. ]{
\label{fig:subfig:b} \includegraphics[scale=0.5]{F9d.pdf}}
\caption{Comparison of different algorithms under the symmetric model when .}
\end{figure}
\end{itemize}
\subsubsection{Under the Asymmetric Model}
\begin{itemize}
\item \emph{Small }: Firstly we study the scenario when  is small, that is, only a small part of channels are available to users. We set , . Fig. 10 shows the average TTRs, the maximum TTRs and the variances of TTR of different algorithms against . Since there is only one commonly-available channel (), such case is believed to be hard for quick rendezvous. According to Fig. 10, our ISAC algorithm significantly outperforms other algorithms in terms of both average TTR and maximum TTR. For example, when there are 50 channels, the average TTRs of ISAC, MMC, Jump-Stay and CRSEQ are 14.65, 27.00, 21.26 and 21.20 while the maximum TTRs are 98, 1023, 562 and 530, respectively. We find that ISAC has significant improvement on the maximum TTR. Its maximum TTR is almost one tenth of others. In this example, . According to Theorem IV.2, . This result shows that the upper bound given by Theorem IV.2 is tight when . The variances of TTR of ISAC, MMC, Jump-Stay and CRSEQ in this case are 255.06, 2080.64, 738.08 and 526.13, respectively. The performance of ISAC is the most stable one.
\begin{figure}
\centering
\subfigure[Average TTR VS. ]{
\label{fig:subfig:a} \includegraphics[scale=0.5]{F10a.pdf}}
\hspace{1in}
\subfigure[Maximum TTR VS. ]{
\label{fig:subfig:b} \includegraphics[scale=0.5]{F10b.pdf}}
\hspace{1in}
\subfigure[Maximum TTR VS.  (without MMC)]{
\label{fig:subfig:b} \includegraphics[scale=0.5]{F10c.pdf}}
\hspace{1in}
\subfigure[Variance of TTR VS. ]{
\label{fig:subfig:b} \includegraphics[scale=0.5]{F10d.pdf}}
\caption{Comparison of different algorithms under the asymmetric model when  and .}
\end{figure}
\item \emph{Moderate }: Then we study the scenario when  is moderate, that is, there are a moderate part of channels are available to users. We set , . That is, there are 40\% channels are available to the users and 25\% among them are commonly-available to both of the users. Fig. 11 shows the average TTRs, the maximum TTRs and the variances of TTR of different algorithms against .  There is no large gap between different algorithms in terms of average TTR. For example, when there are 50 channels, the average TTRs of ISAC, MMC, Jump-Stay and CRSEQ are 75.94, 90.37, 83.09 and 91.20, respectively. However, ISAC significantly outperforms all the existing algorithms in terms of the maximum TTR. When there are 50 channels, the maximum TTRs are 758, 3482, 1121 and 1203, respectively. Its maximum TTR is almost one fifth of MMC. In this example, , . According to Theorem IV.2, . This result shows that the upper bound given by Theorem IV.2 is not tight when  is large under the asymmetric model. The variances of TTR of ISAC, MMC, Jump-Stay and CRSEQ in this case are 5175.16, 16475.76, 7156.27 and 9463.17, respectively. The performance of ISAC is the most stable one.
\begin{figure}
\centering
\subfigure[Average TTR VS. ]{
\label{fig:subfig:a} \includegraphics[scale=0.5]{F11a.pdf}}
\hspace{1in}
\subfigure[Maximum TTR VS. ]{
\label{fig:subfig:b} \includegraphics[scale=0.5]{F11b.pdf}}
\hspace{1in}
\subfigure[Maximum TTR VS.  (without MMC)]{
\label{fig:subfig:b} \includegraphics[scale=0.5]{F11c.pdf}}
\hspace{1in}
\subfigure[Variance of TTR VS. ]{
\label{fig:subfig:b} \includegraphics[scale=0.5]{F11d.pdf}}
\caption{Comparison of different algorithms under the asymmetric model when  and .}
\end{figure}
\item \emph{Large }: We now study the scenario when  is large, that is, most channels are available to users. We set , . That is, there are 80\% channels are available to the users and 75\% of them are commonly-available to both of the users. Fig. 12 shows the average TTRs, the maximum TTRs and the variances of TTR of different algorithms against .  The performance is similar to the scenario when . For example, when there are 50 channels, the average TTRs of ISAC, MMC, Jump-Stay and CRSEQ are 59.53, 57.46, 53.76 and 87.60 while the maximum TTRs are 654, 10190, 993 and 1747, respectively. Maximum TTR of ISAC is almost one twentieth of MMC. The variances of TTR of ISAC, MMC, Jump-Stay and CRSEQ in this case are 2723.60, 11594.98, 2682.32 and 7363.49, respectively. The performance of ISAC and Jump-Stay are more stable than other two algorithms.
\begin{figure}
\centering
\subfigure[Average TTR VS. ]{
\label{fig:subfig:a} \includegraphics[scale=0.5]{F12a.pdf}}
\hspace{1in}
\subfigure[Maximum TTR VS. ]{
\label{fig:subfig:b} \includegraphics[scale=0.5]{F12b.pdf}}
\hspace{1in}
\subfigure[Maximum TTR VS.  (without MMC)]{
\label{fig:subfig:b} \includegraphics[scale=0.5]{F12c.pdf}}
\hspace{1in}
\subfigure[Variance of TTR VS. ]{
\label{fig:subfig:b} \includegraphics[scale=0.5]{F12d.pdf}}
\caption{Comparison of different algorithms under the asymmetric model when  and .}
\end{figure}
\end{itemize}
\subsection{Influence of  or  for fixed }
Jump-Stay and CRSEQ generate CH sequence based on the whole channel set while Random and ISAC generate CH sequence based on the available channel set only. When we fix the total number of channels , we will get the same average TTR and maximum TTR for different  if we apply Jump-Stay or CRSEQ. However, we will get different results if we apply ISAC or MMC. Fig. 13 shows both the average TTRs and maximum TTRs of different algorithms against  (or ) when we fix  and . We adopt eight values which are . According to Fig. 13, all algorithms get very different results when the number of available channels increases but the number of all channels is fixed. This is because there is a random-replacement in Jump-Stay and CRSEQ. When the channel in CH sequence is not available to the user, it will randomly select an available channel to replace. The chance to achieve rendezvous successfully in such time slots will increase significantly when the number of available channels is large. In terms of our ISAC algorithm, the eight simulated results are 12, 24, 36, 57, 60, 72, 84 and 92, respectively. According to Theorem IV.1, the eight theoretical results are 13, 25, 37, 57, 61, 73, 85 and 105, respectively. These results show that the upper bound given by Theorem IV.1 is tight.
\begin{figure}
\centering
\subfigure[Average TTR VS. ]{
\label{fig:subfig:a} \includegraphics[scale=0.5]{F13a.pdf}}
\hspace{1in}
\subfigure[Maximum TTR VS. ]{
\label{fig:subfig:b} \includegraphics[scale=0.5]{F13b.pdf}}
\hspace{1in}
\subfigure[Maximum TTR VS.  (without MMC)]{
\label{fig:subfig:b} \includegraphics[scale=0.5]{F13c.pdf}}
\hspace{1in}
\subfigure[Variance of TTR VS. ]{
\label{fig:subfig:b} \includegraphics[scale=0.5]{F13d.pdf}}
\caption{Influence of  on rendezvous performance when .}
\end{figure}
\subsection{Influence of  for fixed }
In this subsection, we study another basic property of the proposed approach: how does the number of commonly-available channels of the two users affect the rendezvous performance? We let  and take five values of  which are , , , , . Fig. 14 shows the results when we apply the ISAC algorithm. When there are 60 channels, the average TTRs of five scenarios are 157.79, 84.76, 58.03, 42.82 and 29.91 while the maximum TTRs are 1747, 891, 659, 415 and 60, respectively. In this example, , , . According to Theorem IV.1 and Theorem IV.2, however, the upper bound of MTTR in five scenarios are 2428, 2416, 2404, 2392 and 73, respectively. We note that our upper bound under the symmetric model is tight while the upper bound under the asymmetric model is less tight especially when  is large.
\begin{figure}
\centering
\subfigure[Average TTR VS. ]{
\label{fig:subfig:a} \includegraphics[scale=0.5]{F14a.pdf}}
\hspace{1in}
\subfigure[Maximum TTR VS. ]{
\label{fig:subfig:b} \includegraphics[scale=0.5]{F14b.pdf}}
\caption{Influence of  on rendezvous performance when .}
\end{figure}
\section{Conclusions}
We designed a new rendezvous algorithm, called ISAC, for cognitive radio networks. ISAC is the first one in the literature that generates CH sequences based on the available channel set instead of the whole channel set (the available channel set is subset of the whole channel set) while providing guaranteed rendezvous. We proved that ISAC provides guaranteed rendezvous and derived upper bounds on the maximum TTR (MTTR) under both the symmetric model and the asymmetric model. These upper bounds are expressed in terms of the number of available channels instead of the total number of channels ( under the symmetric model and  under the asymmetric model ), and  they are tighter than the upper bounds of the existing rendezvous algorithms. We conducted extensive computer simulation for performance evaluation and observed the following properties:
\begin{itemize}
\item In terms of the maximum TTR, ISAC significantly outperforms other algorithms in almost all scenarios. For example, ISAC reduces the maximum TTR up to 97\% when  and  in Fig. 7(a). ISAC is suitable to many QoS-concern applications of CRNs which desire small MTTR.
\item In terms of the average TTR, performance of ISAC is better than those of other algorithms when the available channel set is a small subset of the whole channel set (e.g., it reduces the average TTR up to 45.8\% in Fig. 10(a)).
\item In terms of the variance of TTR, ISAC is the most stable algorithm among all the rendezvous algorithms in our simulation.
\item The upper bound on MTTR under the symmetric model is tight while the upper bound under the asymmetric model is less tight especially when  is large (e.g., greater than  in Fig. 14).
\end{itemize}
\begin{thebibliography}{1}

\bibitem{IEEEhowto:kopka}
I. Akyildiz, W. Lee, M. Vuran, and S. Mohanty, ``NeXt Generation Dynamic Spectrum Access Cognitive Radio Wireless Networks: A Survey," \emph{Computer Networks}, vol. 50, no. 13, pp. 2127-2159, 2006.
\bibitem{IEEEhowto:kopka}
T. Ycek and H. Arslan, ``A Survey of Spectrum Sensing Algorithms for Cognitive Radio Applications," \emph{IEEE Communications Surveys \& Tutorials}, vol. 11, no. 1, pp. 116-130, 2009.
\bibitem{IEEEhowto:kopka}
P. Wang, L. Xiao, S. Zhou, and J. Wang, ``Optimization of Detection Time for Channel Efficiency in Cognitive Radio Systems," \emph{Proc. of IEEE WCNC}, pp.111-115, Mar. 2007.
\bibitem{IEEEhowto:kopka}
H. Liu, Z. Lin, X. Chu, and Y. W. Leung, ``Taxonomy and Challenges of Rendezvous Algorithms in Cognitive Radio Networks," invited position paper, \emph{Proc. of IEEE ICNC}, pp. 645-649, Hawaii, 2012.
\bibitem{IEEEhowto:kopka}
H. Liu,  Z. Lin,  X. Chu, and Y. W. Leung, ``Jump-Stay Rendezvous Algorithm for Cognitive Radio Networks," \emph{IEEE Trans. Parallel and Distributed  Systems}, vol. 23, no. 10, pp. 1867-1881, 2012.
\bibitem{IEEEhowto:kopka}
Z. Lin, H. Liu, X. Chu, and Y.-W. Leung, ``Jump-stay Based Channel-hopping Algorithm With Guaranteed Rendezvous for Cognitive Radio Networks," \emph{Proc. of IEEE INFOCOM}, pp. 2444-2452, Apr. 2011.
\bibitem{IEEEhowto:kopka}
Z. Lin,  H. Liu,  X. Chu, and Y. W. Leung, ``Enhanced Jump-Stay Rendezvous Algorithm for Cognitive Radio Networks," \emph{IEEE Communications Letters}, vol. 17, no. 9, pp. 1742-1745, 2013.
\bibitem{IEEEhowto:kopka}
J. Shin, D. Yang, and C. Kim, ``A Channel Rendezvous Scheme for Cognitive Radio Networks," \emph{IEEE Communications Letters}, vol. 14, no. 10, pp. 954-956, 2010.
\bibitem{IEEEhowto:kopka}
N. C. Theis, R. W. Thomas, and L. A. DaSilva, ``Rendezvous for cognitive radios," \emph{IEEE Transactions on Mobile Computing}, vol. 10, no. 2, pp. 216-227, 2010.
\bibitem{IEEEhowto:kopka}
K. Bian, J.-M. Park, and R. Chen, ``Control Channel Establishment in Cognitive Radio Networks Using Channel Hopping," \emph{IEEE Journal on Selected Areas in Communications}, vol. 29, no. 4, pp. 689-703, 2011.
\bibitem{IEEEhowto:kopka}
M. M. Buddhikot, P. Kolodzy, S. Miller, K. Ryan, and J. Evans, ``DIMSUMnet: New Directions in Wireless Networking Using Coordinated Dynamic Spectrum," \emph{Proc. of IEEE WoWMoM}, pp. 78-85, Jun. 2005.
\bibitem{IEEEhowto:kopka}
B. Horine and D. Turgut, ``Performance Analysis of Link Rendezvous Protocol for Cognitive Radio Networks," \emph{Proc. of CROWNCOM}, pp. 503-507, Aug. 2007.
\bibitem{IEEEhowto:kopka}
P. Sutton, K. Nolan, and L. Doyle, ``Cyclostationary Signatures in Practical Cognitive Radio Applications," \emph{IEEE Journal on Selected Areas in Communications}, vol. 26, no. 1, pp. 13-24, Jan. 2008.
\bibitem{IEEEhowto:kopka}
V. Brik, E. Rozner, S. Banerjee, and P. Bahl, ``DSAP: A Protocol for Coordinated Spectrum Access," \emph{Proc. of IEEE DySPAN}, pp. 611-614, Nov. 2005.
\bibitem{IEEEhowto:kopka}
J. Jia, Q. Zhang, and X. Shen, ``HC-MAC: A Hardware-constrained Cognitive MAC for Efficient Spectrum Management," \emph{IEEE Journal on Selected Areas in Communications}, vol. 26, no. 1, pp. 106-117, 2008.
\bibitem{IEEEhowto:kopka}
L. Ma, X. Han, and C.-C. Shen, ``Dynamic Open Spectrum Sharing for Wireless Ad Hoc Networks," \emph{Proc. of IEEE DySPAN}, pp. 203-213, Nov. 2005.
\bibitem{IEEEhowto:kopka}
J. Zhao, H. Zheng, and G.-H. Yang, ``Distributed Coordination in Dynamic Spectrum Allocation Networks," \emph{Proc. of IEEE DySPAN}, pp. 259-268, Nov. 2005.
\bibitem{IEEEhowto:kopka}
L. Lazos, S. Liu, and M. Krunz, ``Spectrum Opportunity-based Control Channel Assignment in Cognitive Radio Networks," \emph{Proc. of IEEE SECON}, pp. 1-9, Jun. 2009.
\bibitem{IEEEhowto:kopka}
K. Bian, J.-M. Park, and R. Chen, ``A Quorum-based Framework for Establishing Control Channels in Dynamic Spectrum Access Networks," \emph{Proc. of ACM MobiCom}, pp. 25-36, Sept. 2009.
\bibitem{IEEEhowto:kopka}
Y. Zhang et al., ``ETCH: Efficient Channel Hopping for Communication Rendezvous in Dynamic Spectrum Access Networks," \emph{Proc. of IEEE INFOCOM}, pp. 2471-2479, Apr. 2011.
\bibitem{IEEEhowto:kopka}
P. Bahl, R. Chandra, and J. Dunagan, ``SSCH: Slotted Seeded Channel Hopping for Capacity Improvement in IEEE 802.11 Ad-hoc Wireless Networks," \emph{Proc. of ACM MobiCom}, pp. 216-230, Sep. 2004.
\bibitem{IEEEhowto:kopka}
S. Krishnamurthy, M. Thoppian, S. Kuppa, R. Chandrasekaran, N. Mittal, S. Venkatesan, and R. Prakash, ``Time-efficient Distributed Layer-2 Auto-configuration for Cognitive Radio Networks," \emph{Computer Networks}, vol. 52, no. 4, pp. 831-849, 2008.
\bibitem{IEEEhowto:kopka}
H. Liu,  Z. Lin,  X. Chu, and Y. W. Leung, ``Ring-Walk Based Channel-Hopping Algorithm with Guaranteed Rendezvous for Cognitive Radio Networks," \emph{Proc. of IEEE/ACM International Conference on Green Computing and Communications}, pp. 755-760, Dec. 2010.
\bibitem{IEEEhowto:kopka}
D. Yang, J. Shin, and C. Kim, ``Deterministic Rendezvous Scheme in Multichannel Access Networks," \emph{Electronics Letters}, vol. 46, no. 20, pp. 1402-1404, 2010.
\bibitem{IEEEhowto:kopka}
K. Bian and J.-M. Park, ``Maximizing Rendezvous Diversity in Rendezvous Protocols for Decentralized Cognitive Radio Networks," \emph{IEEE Transactions on Mobile Computing}, vol. 12, no. 7, pp. 1294-1307, 2013.
\bibitem{IEEEhowto:kopka}
Z. Gu, Q.-S. Hua, Y. Wang, F. Lau, ``Nearly Optimal Asynchronous Blind Rendezvous Algorithm for Cognitive Radio Networks," \emph{Proc. of IEEE SECON}, pp. 371-379, Jun. 2013.
\bibitem{IEEEhowto:kopka}
C. Cormio and K. R. Chowdhury, ``Common Control Channel Design for Cognitive Radio Wireless Ad Hoc Networks Using Adaptive Frequency Hopping," \emph{Ad Hoc Networks}, vol. 8, pp. 430-438, 2010.
\bibitem{IEEEhowto:kopka}
C. Cordeiro and K. Challapali, ``C-MAC: A Cognitive MAC Protocol for Multi-Channel Wireless Networks," \emph{Proc. of IEEE DySPAN}, pp. 147-157, Apr. 2007.
\bibitem{IEEEhowto:kopka}
S. Romaszko, ``A Rendezvous Protocol with the Heterogeneous Spectrum Availability Analysis for Cognitive Radio Ad Hoc Networks," \emph{Journal of Electrical and Computer Engineering}, Article ID 715816. 2013.
\bibitem{IEEEhowto:kopka}
D. E. Knuth, \emph{The Art of Computer Programming}, vol. 1: Fundamental Algorithms, 3rd ed., Addison Wesley, 1997.






\end{thebibliography}

\end{document}
