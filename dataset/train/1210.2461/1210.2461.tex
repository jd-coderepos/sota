\documentclass[submission,copyright,creativecommons]{eptcs}
\providecommand{\event}{Gandalf 2012} \usepackage{breakurl}             \usepackage{mls}
\usepackage{amssymb}
\usepackage{amsfonts}
\usepackage{amsmath}
\usepackage{amsthm}
\usepackage{enumitem}
\usepackage{color}
\usepackage{setspace}
\usepackage{color}

\newtheorem{theorem}{Theorem}
\newtheorem{lemma}{Lemma}
\newtheorem{corollary}{Corollary}
\newtheorem{definition}{Definition}

\newcommand{\Lang}{\ensuremath{\mathbf{\forall}^{\pi}_{0,2}}\xspace}
\newcommand{\LangBounded}[1]{\ensuremath{(\Lang)^{\leq #1}}\xspace}

\newcommand{\univ}[1]{\Delta^{#1}}
\newcommand{\assignment}[1]{M^{#1}}
\newcommand{\pairf}[1]{\pi^{#1}}

\newcommand{\inter}{I}
\newcommand{\iuniv}{\univ{\inter}}
\newcommand{\ipairf}{\pairf{\inter}}
\newcommand{\iassignment}{\assignment{\inter}}

\newcommand{\interp}{J}
\newcommand{\iunivp}{\univ{\interp}}
\newcommand{\ipairfp}{\pairf{\interp}}
\newcommand{\iassignmentp}{\assignment{\interp}}

\newcommand{\Conjs}{\mathsf{Conjs}}
\newcommand{\DomVars}{\mathit{DVars}}





\newcommand{\pairfree}{\mathsf{pair\_free}}
\newcommand{\myTimes}{\mathrel{\times_{\pi}}}
\newcommand{\pairs}[2]{\mathrm{Pairs}_{\pi^{#1}}(#2)}
\newcommand{\Langrange}{\ensuremath{\mathbf{\forall}^{\pi, \range}_{\Delta}}\xspace}
\newcommand{\Langdom}{\ensuremath{\mathbf{\forall}^{\pi +2 \dom}_{0,2}}\xspace}
\newcommand{\myDelta}{D_{\varphi}} 
\newcommand{\D}{\corr{\mathsf{U}}}
\newcommand{\xvar}{\xi}

\newcommand{\nat}{\mathbb{N}}
\newcommand{\dominoSys}{\mathbb{D}}

\newcommand{\peanSys}{\mathbb{S}}
\newcommand{\peanN}{\mathcal{N}}
\newcommand{\peanZ}{\mathcal{Z}}
\newcommand{\peanS}{\mathcal{S}}
\newcommand{\isPeanoSys}{\mathtt{is\_Peano}}

\newcommand{\Partition}{P}
\newcommand{\isPartition}{\mathsf{partition}}
\newcommand{\partn}{A}
\newcommand{\mN}{\mathsf{N}}
\newcommand{\mQ}{\mathsf{Q}}
\newcommand{\mS}{\mathsf{S}}
\newcommand{\mZ}{\mathsf{Z}}
\newcommand{\horcom}{\mathsf{hor}}
\newcommand{\vercom}{\mathsf{ver}}

\newcommand{\vx}{x}
\newcommand{\vy}{y}
\newcommand{\vz}{z}
\newcommand{\svx}{x}
\newcommand{\svy}{y}
\newcommand{\svz}{z}
\newcommand{\mvx}{f}
\newcommand{\mvy}{g}
\newcommand{\mvz}{h}

\newcommand{\sx}{u}
\newcommand{\sy}{v}


\newcommand{\corr}[1]{#1}

\title{A decidable quantified fragment of set theory with ordered
pairs and some undecidable extensions\thanks{Work partially supported
by the INdAM/GNCS 2012 project
\emph{``Specifiche insiemistiche ese\-guibili e loro verifica 
formale''} and by Network Consulting Engineering Srl.}} 

\author{Domenico Cantone
\institute{Department of Mathematics and Computer Science\\
University of Catania, Italy}
\email{cantone@dmi.unict.it}
\and
Cristiano Longo
\institute{Network Consulting Engineering\\
Valverde, Catania, Italy} \email{cristiano.longo@nce.eu} }
\def\titlerunning{A decidable quantified fragment of set theory with
ordered pairs and some undecidable extensions} 
\def\authorrunning{D. Cantone \& C. Longo}
\begin{document}



\maketitle

\begin{abstract}
In this paper we address the decision problem for a fragment of set
theory with restricted quantification which extends the language
studied in \cite{BreFerOmoSch1981} with pair related quantifiers and
constructs, in view of possible applications in the field of
\emph{knowledge representation}.
We will also show that the decision problem for our language has a
non-deterministic exponential time complexity.  However, for the
restricted case of formulae whose quantifier prefixes have length
bounded by a constant, the decision problem becomes
\textsc{NP}-complete.  We also observe that in spite of such
restriction, several useful set-theoretic constructs, mostly related
to maps, are expressible.  Finally, we present some undecidable
extensions of our language, involving any of the operators domain,
range, image, and map composition.
\end{abstract}



\section{Introduction}\label{INTRO}

The intuitive formalism of set theory has helped providing solid and
unifying foundations to such diverse areas of mathematics as geometry,
arithmetic, analysis, and so on.
Hence, positive solutions to the decision problem for fragments of
set theory can have considerable applications to the automation of
mathematical reasoning and therefore in any area which can take
advantage of automated deduction capabilities.

The decision problem in set theory has been intensively studied
in the context of \emph{Computable Set Theory} (see
\cite{CanFerOmo89a,CanOmoPol01,SchCanOmo11}), and decision
procedures or undecidability results have been provided for several
sublanguages of set theory.
\emph{Multi-Level Syllogistic} (in short \mls, cf.\
\cite{FerOmoSch1980}) was the first unquantified sublanguage of set
theory that has been shown to have a solvable satisfiability problem.
We recall that \mls is the Boolean combinations of atomic formulae
involving the set predicates , , , and the Boolean
set operators , , .  Numerous
extensions of \mls with various combinations of operators (such as
singleton, powerset, unionset, etc.)  and predicates (on finiteness,
transitivity, etc.)  have been proved to be decidable.  Sublanguages
of set theory admitting explicit quantification (see for example
\cite{BreFerOmoSch1981, OmoPol2010, OmoPol2012, CanLonNic2011}) are of
particular interest, since, as reported in \cite{BreFerOmoSch1981},
they allow one to express several set-theoretical constructs using
only the basic predicates of membership and equality among sets.

Applications of Computable Set Theory to \emph{knowledge
representation} have been recently investigated in
\cite{CanLonPis2010, CanLonNic2011}, where some interrelationships
between (decidable) fragments of set theory and description logics
have been exploited.\footnote{We recall that description logics are a
well-established framework for knowledge representation; see
\cite{DLHANDBOOK2} for an introduction.}
As knowledge representation mainly focuses on representing
relationships among items of a particular domain, any set-theoretical
language of interest to knowledge representation should include a
suitable collection of operators on \emph{multi-valued
maps}.
\footnote{According to \cite{SchDewSchDub1986}, we use the term
`maps' to denote sets of ordered pairs.}

Non-deterministic exponential time decision procedures for two
unquantified fragments of set theory involving map related constructs
have been provided in \cite{FOS80, CanSch91}.  As in both
cases the map domain operator is allowed together with all the
constructs of \mls, it turns out that both fragments have an
\textsc{ExpTime}-hard decision problem (cf.\ \cite{CanLonNic2010}).
On the other hand, the somewhat less expressive fragment \mlsscart has
been shown to have an \textsc{NP}-complete decision problem in
\cite{CanLonNic2010}, where \mlsscart is a two-sorted
language with set and map variables, which involves various map
constructs like Cartesian product, map restrictions, map inverse, and
Boolean operators among maps, and predicates for single-valuedness,
injectivity, and bijectivity of maps.

In \cite{BreFerOmoSch1981}, an extension of the quantified fragment
 (studied in the same paper---here the subscript `' 
denotes that quantification is restricted) with \emph{single-valued}
maps, the map domain operator, and terms of the form , with 
a function-free term, was considered.  We recall that
-formulae are propositional combinations of restricted
quantified prenex formulae
\,,
where  is a Boolean combination of atoms of the types ,
, \corr{and \emph{quantified variables nesting} is not allowed, in the 
sense that any quantified variable  can not occur at the
right-hand side of a membership symbol  in the same quantifier prefix
(roughly speaking, no  can be a )}. 
More recently, a decision procedure for a new fragment of set theory,
called , has been presented in \cite{CanLonNic2011}.
\corr{The superscript ``'' denotes the presence of operators
related to ordered pairs.} Formulae of the fragment ,
to be reviewed in Section \ref{DECPROC}, involve the operator
, which intuitively represents the collection of the
non-pair members of its argument, and terms of the form , for
ordered pairs.  The predicates  and  allowed in it can occur
only within atoms of the forms , , and ; quantifiers in \Forallpizero-formulae are restricted to the
forms  and , and,
much as in the case of the fragment , quantified
variables nesting is not allowed.

In this paper we solve the decision problem for the
extension \Lang of the fragment  with ordered pairs
and prove that, under particular conditions, our decision procedure
runs in non-deterministic polynomial time.
\Lang is a two-sorted \corr{(as indicated by the second subscript ``'')} quantified fragment of set theory which allows
restricted quantifiers of the forms ,
, ,
, and literals of the forms , , , , where
,  are set variables and ,  are map variables.
Considerably many set-theoretic constructs are expressible in it, as
shown in Table \ref{SETCONS}.  In fact, the language \Lang is also an
extension of \mlsscart.  However, as will be shown in Section
\ref{UNDEC}, it is not strong enough to express inclusions like , , , and , but only those in which the
operators domain, range, \corr{(multi-)}image, and map composition are allowed to
appear on the left-hand side of the inclusion operator .






The paper is organized as follows.  Section \ref{PREL} provides some
preliminary notions and definitions.  In Section \ref{LANG} we give
the precise syntax and semantics of the language \Lang.  Decidability
and complexity of reasoning in the language \Lang are addressed
in Section \ref{DECPROC}.  Some undecidable extensions of \Lang are
then presented in Section \ref{UNDEC}.  Finally, in Section \ref{CONC}
we draw our conclusions and provide some hints for future works.

\begin{table}[h]
\begin{center}
\begin{small}
\begin{tabular}{|c|l|}
\hline
&\0.0cm]

 & 
\0.0cm]

 & \0.0cm]

 & \0.0cm]

 &  \0.0cm]

 & \0.0cm]

 & \0.0cm]

 & \0.0cm] 

 & \0.0cm]

 & \0.0cm]

 & \0.0cm]

 & \0.0cm]



 & \0.0cm]

 & \0.0cm]

 & \.1cm]
\hline
\end{tabular}
\end{small}
\end{center}
\caption{Set-theoretic constructs expressible in \Lang.}\label{SETCONS}
\end{table}

\section{Preliminaries}\label{PREL}

We briefly review basic notions from set theory and introduce
also some definitions which will be used throughout the paper.

Let  and
 be two infinite disjoint
collections of \emph{set} and \emph{map variables},
respectively.  As we will see, map variables will be interpreted as
maps (i.e., sets of ordered pairs).  We put .
For a formula , we write  for the collection
of variables occurring free (i.e., not bound by any quantifier) in
, and put  and .

Semantics of most of the languages studied in the context of
Computable Set Theory are based on the \emph{von Neumann standard
cumulative hierarchy of sets} , which is the class containing
all the \emph{pure} sets (i.e., all sets whose members are recursively
based on the empty set ).  The von Neumann hierarchy 
is defined as follows:

where  is the powerset operator and
 denotes the class of all ordinals. 
The \emph{rank}
 of a set  is defined as the least ordinal
 such that .
We will refer to mappings from 
 to  as \emph{assignments}.

Next we introduce some notions related to pairing functions and
ordered pairs.
Let  be a binary operation over the universe \VNU.
The \emph{Cartesian product}  of two sets , relative to , is defined as
.
When it is clear from the context, for the sake of conciseness we
will omit to specify the binary operation  and simply write 
`' in place of `'.
A binary operation  over sets in \VNU  is said to be a 
\emph{pairing function} if
\begin{itemize}
    \item[(i)] 
    , and

    \item[(ii)] the Cartesian product  (relative to
    ) is a set of \VNU, for all .
\end{itemize}
In view of the replacement axiom, condition (ii) is obvioulsy met 
when  is expressible by a set-theoretic term. This, for 
instance, is the case for Kuratowski's ordered pairs, defined by 
,
for all .
Given a pairing function  and a set , we denote with
 the collection of the \emph{pairs} in  (with respect
to ), namely
.


A \emph{pair-aware interpretation} 
consists of a pairing function  and an assignment
 such that
 
holds for every map variable  (i.e., map variables
can only be assigned sets of ordered pairs, or the empty set).  For
conciseness, in the rest of the paper we will refer to \emph{pair-aware}
interpretations just as interpretations.
An interpretation  associates sets to
variables and pair terms, respectively, as follows:

for all . 
Let  be a finite collection of variables, and let
 be two assignments.  We say that  is a \emph{-variant}
of  if  for all .  For two
interpretations  and , we say that  is a -variant of
 if  is a -variant of  and
.

In the next section we introduce the precise syntax and semantics of 
the language \Lang.

\section{The language \Lang}\label{LANG}

The language \Lang consists of the denumerable infinity of variables
, the binary \emph{pairing} operator
, the predicate symbols , the Boolean
connectives of propositional logic , , ,
, , parentheses, and \emph{restricted}
quantifiers of the forms , ,
, and .
\emph{Atomic} \Lang\emph{-formulae} are expressions of
the following four types

with  and .
\emph{Quantifier-free} \Lang\emph{-formulae} are propositional
combinations of atomic \Lang-formulae. 
\emph{Prenex} \Lang\emph{-formulae} are expressions of the 
following two forms

where , ,
and  is a quantifier-free \Lang-formula.  We will refer to the
variables  as the \emph{domain variables} of
the formulae (\ref{UNIVFORM}) and (\ref{EXFORM}).  Notice that
quantifier-free \Lang-formulae can also be regarded as prenex
\Lang-formulae with an empty quantifier prefix.
A prenex \Lang-formula is said to be \emph{simple} if nesting among
quantified variables is not allowed, i.e., if no quantified variable
can occur also as a domain variable.
Finally, \Lang\emph{-formulae} are Boolean combinations of
simple-prenex \Lang-formulae.



Semantics of \Lang-formulae is given in terms of interpretations.  An
interpretation  \emph{evaluates} a
\Lang-formula  into a truth value  in the following recursive manner.
First of all, interpretation of quantifier-free \Lang-formulae is
carried out following the rules of propositional logic, where atomic
formulae (\ref{atomic}) are interpreted according to the standard
meaning of the predicates  and  in set theory and the pair
operator  is interpreted as in (\ref{interpVarsPairs}).
Thus, for instance, , provided that either  or .
Then, evaluation of simple-prenex \Lang-formulae is defined
recursively as follows:
\begin{itemize}
  \item , provided that
  , for every -variant  of
   such that ;
  
  \item ,
  provided that , for every
  -variant  of  such that ;  
  
  \item , provided that
  ; and 
  
  \item ,
  provided that .
\end{itemize}
Finally, evaluation of \Lang-formulae is carried out following the
rules of propositional logic.  

If an interpretation  evaluates a \Lang-formula to  we
say that  is a \emph{model} for  (and write ).  
A \Lang-formula  is said to be \emph{satisfiable} if and only
if it admits a model.  
Two \Lang-formulae are said to be
\emph{equivalent} if they have exactly the same models.
Two \Lang-formulae  and  are
said to be \emph{equisatisfiable} provided that  is
satisfiable if and only if so is .  
The \emph{satisfiability
problem} (s.p., for short) for the theory \Lang is the problem of
establishing algorithmically whether any given \Lang-formula is
satisfiable or not.

By way of a simple normalization procedure based on disjunctive normal
form, the s.p.\ for \Lang-formulae can be reduced to
that for \emph{conjunctions} of simple-prenex \Lang-formulae of the
types (\ref{UNIVFORM}) and (\ref{EXFORM}).  Moreover, since any such
conjunction of the form

is equisatisfiable with , where
 is obtained from the quantifier-free formula

by a suitable
renaming of the (quantified) variables
, 
it turns out that the s.p.\ for \Lang-formulae can be
reduced to the s.p.\ for \emph{conjunctions} of
simple-prenex \Lang-formulae of the type (\ref{UNIVFORM}) only, which
we call \emph{normalized \Lang-conjunctions}.

Satisfiability of normalized \Lang-conjunctions does not depend
strictly on the pairing function of the interpretation, provided that
suitable conditions hold, as proved in the following technical lemma.

\begin{lemma}\label{PFISO}
Let  be a normalized \Lang-conjunction, and let  and
 be two interpretations such that
\begin{enumerate}[label=(\alph*)]
 \item\label{PFISO_a} , for all ,

 \item\label{PFISO_b} , for all  and .
\end{enumerate}
Then .
\end{lemma}
\begin{proof}
It is enough to prove that

holds, for every (universal) simple-prenex conjunct  occurring
in .  We shall proceed by induction on the length of the
quantifier prefix of .  We begin with observing that, by
\ref{PFISO_a},  and  evaluate to the same truth
values all atomic formulae of the types  and
, for all .  Likewise,

follow directly from \ref{PFISO_a} and \ref{PFISO_b}.  Thus
(\ref{PFISO1}) follows easily when  is quantifier-free, i.e., 
when the length of its quantifier prefix is .

Next, let , for some , where  is a universally quantified
simple-prenex \Lang-formula with one less quantifier than  and
containing no quantified occurrence of .  We prove that
 is a model for  if and only if so is
, for every , where
 and  denote, respectively, the -variants of
 and  such that .  But, for each ,
 and  satisfy conditions \ref{PFISO_a} and
\ref{PFISO_b} of the lemma, so that, by inductive hypothesis, we have
.  Hence
.

The case in which , with
, , and  a
universally quantified simple-prenex \Lang-formula containing no
quantified occurrence of  and , can be dealt with much in
the same manner, thus concluding the proof of the lemma.
\end{proof}

In the following section we show that the s.p.\ for normalized
\Lang-conjunctions is solvable. 

\section{A decision procedure for \Lang}\label{DECPROC}

We solve the s.p.\ for \Lang-formulae by reducing the s.p.\ for
normalized \Lang-conjunctions to the s.p.\ for the fragment of set
theory \Forallpizero, studied in \cite{CanLonNic2011}.
Following \cite{CanLonNic2011},
\Forallpizero-formulae are finite conjunctions of \emph{simple-prenex
\Forallpizero-formulae}, namely expressions of the form

where , for , no domain
variable  can occur quantified, and  is a
quantifier-free Boolean combination of atomic formulae of the types
, , , 
with .\footnote{Thus, normalization is already
built-in into \Forallpizero-formulae, and we could have called them
\emph{normalized \Forallpizero-conjunctions}.}
Intuitively, a term of the form  represents the set of 
the \emph{non-pair} members of .
Notice that \Forallpizero-formulae involve only set variables.

Semantics for \Forallpizero-formulae is given by extending
interpretations also to terms of the form  as indicated
below:

where .
Evaluation of \Forallpizero-formulae is carried out much in the same
way as for \Lang-formulae.  In particular, we also put
, provided
  that , for every -variant 
  of  such that . 

We recall that satisfiability of \Forallpizero-formulae can be tested
in non-deterministic exponential time.  Additionally, the s.p.\ for
\Forallpizero-formulae with quantifier prefixes of length at most ,
for any fixed constant , is \textsc{NP}-complete (cf.\
\cite{CanLonNic2011}).

The s.p.\ for normalized \Lang-conjunctions can be reduced to the
s.p.\ for \Forallpizero-formulae.  To begin with, we define a
syntactic transformation  on normalized
\Lang-conjunctions.  More specifically,  is obtained
from a given normalized \Lang-conjunction  by replacing
\begin{itemize}
    \item each restricted universal quantifier  in
     \corr{by} the quantifier ,
    
    \item each atomic formula  in  \corr{by} the literal
    , and
    
    \item each map variable  occurring in  by a fresh
    set variable , thus identifying an application  from  into , 
    which will be referred to as \emph{\corr{map-variable renaming} for 
    }.
\end{itemize}    
Thus, for instance, if

then

where  is a set variable distinct from , , 
and .

The following lemma
provides a useful semantic relation between universal
simple-prenex \Lang-formulae and their corresponding
\Forallpizero-formula via .



\begin{lemma}\label{SAT0}
Let  be a universal simple-prenex \Lang-formula and let 

be an interpretation such that
\begin{enumerate}[label=(\roman{*}), ref=(\roman{*})]
  \item\label{SAT01}  (i.e.,  is not a pair, for
  any free variable  of ), and

  \item\label{SAT02} , for
  every domain variable  of .
\end{enumerate}
Then  
if and only if .
\end{lemma}
\begin{proof}
We proceed by induction on the quantifier prefix length 
of the formula .  To begin with, we observe that in force of
\ref{SAT01} we have  if and only if
, for any two free variables
 and  of , so that, given any atomic formula 
involving only variables in , 
if and only if .  Hence the lemma follows
directly from propositional logic if  is quantifier-free, i.e.,
.

Next, let , where  is a
universal simple-prenex \Lang-formula with  quantifiers,
 are set variables occurring neither as domain nor as
bound variables in .  Observe that, by
\ref{SAT02}, , since  is a
domain variable of .  Thus
it will be enough to prove that 

holds for every -variant  of  such
that , with .  But
 can not be a pair (with respect to the pairing
function ), as it is a member of  and  is
a domain variable of .  Thus (\ref{SAT03}) follows by
applying the inductive hypothesis to  and to every
interpretation  such that .

Finally, the case in which , where  is a universal simple-prenex
\Lang-formula,  are set variables not occurring as domain
variables in , and  is a map variable, can be dealt with
much in the same way as the previous case, and is left to the
reader.
\end{proof}

In the following theorem we use the transformation  to
reduce the s.p.\ for normalized \Lang-conjunctions to the s.p.\ for
\Forallpizero-formulae.

\begin{theorem}\label{DEC}
The s.p.\ for normalized \Lang-conjunctions can be reduced in linear
time to the s.p.\ for \Forallpizero-formulae, and therefore it is in
\textsc{NExpTime}.
\end{theorem}
\begin{proof}
We prove the theorem by showing that, given any normalized
\Lang-conjunction , we can construct in linear time a
corresponding \Forallpizero-formula  which is
equisatisfiable with .

So, let  be a normalized \Lang-conjunction and let  be the map-variable renaming for .
We define the corresponding \Forallpizero-formula  as
follows:

where  is a fresh set variable. Plainly, the size of  
is linear in the size of .

Let us first assume that  admits a model .  For each  we
have , as , for .  Likewise, for each 
we have , as
, for .  Finally, for each , we have , so that .
We define  as the -variant of 
such that , for .  Plainly,  so
that, by Lemma \ref{SAT0},  as well.

For the converse direction, let
 be a model for .  We shall
exhibit an interpretation  which satisfies .  To
begin with, we define a new pairing function  by putting

for every , where  is the
Kuratowski's pairing function and .  Then we define  as the
-variant of the assignment  such that
, for each
.  From Lemma~\ref{PFISO}, it follows that
the interpretation  satisfies
.  
Moreover, we have

for each .  
Indeed, if for some 
and  we had , then

contradicting the regularity axiom of set theory. 
Next, let  and let  be the -variant of , where
, for , and  .  In view of (\ref{PairsEq}), it is an easy
matter to verify that 

From (\ref{PairsEq}), we have immediately that
, so that

Likewise, by reasoning much in
the same manner as for the proof of (\ref{PairsEq}), one can prove
that

From (\ref{secondEq}), (\ref{firstEq}), and (\ref{thirdEq}), it 
follows at once that , completing the 
proof that  and  are equisatisfiable.

Since the s.p.\ for \Forallpizero-formulae is in \textsc{NExpTime}, 
as was shown in \cite[Section~3.1]{CanLonNic2011}, it readily follows 
that the s.p.\ for normalized \Lang-conjunctions is in 
\textsc{NExpTime} as well.
\end{proof}

\begin{corollary}
    \label{corollaryLangFormulae}
    The s.p.\ for \Lang-formulae is in \textsc{NExpTime}.
\end{corollary}
\begin{proof}
Let  be a satisfiable \Lang-formula.  We may assume without
loss of generality that all existential simple-prenex \Lang-formulae
of the form (\ref{EXFORM}) have already been rewritten in terms of
equivalent universal simple-prenex \Lang-formulae of the form
(\ref{UNIVFORM}), so that  is a propositional combination of
universal simple-prenex \Lang-formulae.  In addition, by suitably
renaming variables, we may assume that all quantified variables in
 are pairwise distinct and that they are also distinct from
free variables.

Let  be the collection of
the universal simple-prenex \Lang-formulae occurring in .  By
traversing the syntax tree of , one can find in linear time
the propositional skeleton  of  and a
substitution  from the propositional variables
 of  into
, such that , where
 is the result of substituting each propositional
variable  in  by the universal
simple-prenex \Lang-formula .  Then to check
the satisfiability of  one can perform the following
non-deterministic procedure:
\begin{itemize}
    \item guess a Boolean valuation  of the propositional 
    variables  of 
     such that ;
    
    \item form the \Lang-conjunction
    
    
    \item transform each conjunct 
    
    of the form  in
    (\ref{FORALLPIZERONU}), where , into
    the equisatisfiable formula
    
Let  be the
    normalized \Lang-conjunction so obtained. Plainly,  is satisfied by any interpretation.
        


    \item Check that  is satisfiable by a 
    \textsc{NExpTime} procedure for normalized \Lang-conjunctions 
    (cf.\ Theorem \ref{DEC}).
\end{itemize}
Since  can be constructed in non-deterministic linear time, 
the corollary follows.
\end{proof}


Next we consider \LangBounded{h}-formulae, namely \Lang-formulae whose
simple-prenex subformulae have \corr{quantifier-prefix lengths} bounded by
the constant .
By reasoning much as in the proofs of Theorem~\ref{DEC} and Corollary
\ref{corollaryLangFormulae}, it is immediate to check that the s.p.\
for \LangBounded{h}-formulae can be reduced in non-deterministic
linear time to the s.p.\ of -formulae, and
thus, by \cite[Corollary~4]{CanLonNic2011}, it can be decided in
non-deterministic polynomial time.
On the other hand, it is an easy matter to show that the s.p.\ for
\LangBounded{h}-formulae is \textsc{NP}-hard.  Indeed, given a
propositional formula , consider the \LangBounded{0}-formula
, obtained from  by replacing each propositional variable
 in  with the atomic \Lang-formula , where  and the 's are distinct set variables.
Plainly,  is propositionally satisfiable if and only if the
\Lang-formula  is satisfiable.
The following lemma summarizes the above considerations.

\begin{lemma}\label{NP}
For any integer constant , the s.p.\ for
\LangBounded{h}-formulae is \textsc{NP}-complete.  \qed
\end{lemma}

It is noticeable that, despite of the large collection of
set-theoretic constructs which are expressible by \Lang-formulae (see
Table \ref{SETCONS}), some very common map-related operators like
domain, range, and map image can not be expressed by \Lang-formulae in
full generality, but only in restricted contexts.  In the next section
we prove that dropping any of such restrictions triggers
undecidability.

\section{Some undecidable extensions of \Lang}\label{UNDEC}

In this section we prove the undecidability of any extension of \Lang
which allows one to express literals of the form .  As we will see, analogous undecidability results hold also
for similar extensions of \Lang in the case of other map related
constructs such as range, map image, and map composition.
Our proof will be carried out via a reduction of the \emph{Domino
Problem}, a well-known undecidable problem studied in \cite{Ber1966}
\corr{(see also \cite{BorGraGur1997})},
which asks for a tiling of the quadrant  subject to
a finite set of constraints.

\begin{definition}[Domino problem]\label{DOMINO}
A \emph{domino system} is a triple , where
 is a finite nonempty set of \emph{domino
types}, and  and , respectively the \emph{horizontal} and
\emph{vertical compatibility conditions}, are two functions
which associate to each domino type  a subset of ,
respectively  and .

A \emph{tiling}  for a domino system 
is any mapping which associates a domino type in  to each 
ordered pair of natural numbers in .
A tiling  is said to be \emph{compatible} 
if and only if   and 
 for all .
The \emph{domino problem} consists in 
determining whether a domino system admits a compatible tiling.
\qed
\end{definition}

In order to reformulate the domino problem in set-theoretic terms, we
make use of the following set-theoretic variant of Peano systems (see, 
for instance, \cite{Mos2005}).


\begin{definition}[Peano systems]\label{PEANO}
Let  be a pairing-function and let  be
three sets in the von Neumann hierarchy of sets.  The tuple
 is said to be a Peano system
if it satisfies the following conditions:

\begin{enumerate}[label=\textbf{(P\arabic*)}]
 \item\label{P1}  is a set to which  belongs;

 \item\label{P2}  is a
 total function over , i.e., a single-valued map with domain
 ;


 \item\label{P3}  is injective;
 
 \item\label{P4}  is not in the range of ;
 
 \item\label{P5} for each  the following holds:

\end{enumerate}
\end{definition}
\noindent The first Peano system was devised by G.\ Peano himself.  It
can be \corr{characterized} as , where  is the minimal set
containing the empty set  and satisfying , and  is the relation
over  such that  if and only if .\footnote{In the original
definition the pairing function was not specified.}

The domino problem can be easily reformulated in pure set-theoretic
terms.  To this purpose, we observe that any tiling  for a domino
system induces a partitioning of the integer plane ,
as it associates exactly one domino type to each pair .  Hence, given a domino system
, the domino problem for
 can be expressed in set-theoretic terms as the problem of
deciding whether there exists a partitioning  of , for some fixed
Peano system , such that for
all , and for all  such
that  and ,
\begin{enumerate}[label=\textbf{(D\arabic*)}]
\item\label{D1} if  (i.e.,  is the successor
of ) and  then , and

\item\label{D2} if  (i.e.,  is the successor
of ) and  then .
\end{enumerate}

Notice that from the properties of Peano systems it follows that if a
domino system  admits a compatible tiling  then we can
construct a partitioning of the integer plane which satisfies \ref{D1}
and \ref{D2} however the Peano system is chosen.



All instances of the domino problem can be formalized with normalized
\Lang-conjunctions extended with two positive literals of the form , with  and ,
where the obvious semantics for the operator  is 

for any interpretation .  In view of the undecidability of the
domino problem, this yields the undecidability of the s.p.\ for the
class  of normalized \Lang-conjunctions extended with two
positive literals of the form , proved in the
following theorem.




\begin{theorem}\label{PEANUNDEC}
The s.p.\ for , namely the class of normalized
\Lang-conjunctions extended with two positive literals of the form , is undecidable.
\end{theorem}
\begin{proof}
Let , with , be a
domino system.  We will show how to construct in polynomial time a
formula  of  which is satisfiable if and only
if there exists a partitioning of the integer plane which satisfies
conditions \ref{D1} and \ref{D2}, so that the undecidability of the
s.p.\ for  will follow directly from the undecidability
of the domino problem.

Let ,  be two distinct set variables, and let  be a map
variable.  In addition, let  be pairwise
distinct map variables, which are also distinct from .  These are
intended to represent the blocks of the partition of the integer plane
induced by a tiling.
To enhance the readability of the formula  we are about to
construct, we introduce some abbreviations which will also make use 
of some map constructs defined in Table~\ref{SETCONS}. 
To begin with, we put

Plainly, for every interpretation \inter, we have  if and only if
 partitions .
Next we define the formulae  and , for , which will encode respectively the horizontal and the
vertical compatibility constraints:

Finally, we denote with  the following formula:

Notice that  is equivalent to
.  In addition, a literal of
the form  can obviously be expressed by the conjunction



Next we show that the formula  is
satisfiable and correctly characterizes Peano systems, in the sense
that if  for an interpretation
, then  is a
Peano system.
Given any interpretation  such that ,
, , and ,  follows
from the very definition of , so that
 is satisfiable.
In addition, if  for an
interpretation , it can easily be proved that  is a Peano system.  Indeed \ref{P1},
\ref{P2}, \ref{P3}, and \ref{P4} follow readily from the first four
conjuncts of .  
Concerning \ref{P5}, we proceed by contradiction.
Thus, let us assume that there exists a proper subset  of  such that the following holds

and let  be a set in  with minimal rank.
We must have , in force of the first conjunct of
(\ref{UNDECEQ1}), and thus  must hold, as
we assumed that  correctly models the conjunct  of the formula .
Hence, there must exist a set  such that .  Since ,  must have rank strictly less than , so that
 must hold, as by assumption  has minimal rank in
.  But (\ref{UNDECEQ1}) would yield ,
which contradicts our initial assumption .



We are now ready to define the formula  of
 intended to express that the domino system
 admits a compatible tiling. This is:

Observe that  can be expanded so as to involve 
only two literals of the form .

We show next that  is satisfiable if and only if
the domino system  admits a compatible tiling.
Let us first assume that  is satisfiable, and
let  be a model for .  Plainly,  is a Peano system, as
 is a conjunct of .
In addition,  partitions
, since .  It remains to
prove that the partition  is
induced by a compatible tiling of the domino system ,
i.e., that properties \ref{D1} and \ref{D2} hold.  Thus let  such that ,
, and ,
for some .  Plainly , so that from  it
follows , proving \ref{D1}.  Likewise, let  be such that ,
, and ,
for some .  Thus , so that from  we obtain
, proving \ref{D2}.

Conversely, let us suppose that  admits a compatible
tiling and let  be the induced
partitioning of  which
satisfies \ref{D1} and \ref{D2}, relative to the Peano system
.
We prove that  is satisfied
be any interpretation  such that

Plainly,  models correctly .  In
addition, , as we assumed that  is a partitioning of
.
Next we prove that  models correctly the conjuncts 
of , for .  To this purpose,
let ,  be any two sets such that , for some .  Then, there
must exist a set  such that , and
.  Hence 
must belong to some , for , such that , proving .
Analogously, one can show that , for , thus proving that  and in turn concluding the proof of the theorem.
\end{proof}

Because of the large number of set-theoretic constructs expressible in
\Lang, the undecidability of various other extensions of normalized
\Lang-conjunctions easily follows from Theorem \ref{PEANUNDEC}.

\begin{corollary}\label{OTHERUNDEC}
The class of normalized \Lang-conjunctions extended with two
literals of any of the following types is undecidable:

where  and .
\end{corollary}
\begin{proof}
In view of Theorem \ref{PEANUNDEC}, it is enough to show that any
literal of the form  can be expressed with
normalized \Lang-conjunctions extended with \emph{one} literal of any
of the types (\ref{typesLiterals}).  Concerning the case of literals
of the types
, 
it suffices to observe that  is equivalent
to each of the two formulae
 and 
,
and that map identity  and map inverse  
are expressible by \Lang-formulae, as shown in Table~\ref{SETCONS}.

Finally, concerning literals of the form ,
it is enough to observe that for every set variable 
distinct from  we have
\begin{itemize}
    \item , for every interpretation ;
    
    \item if , for some
    interpretation , then , where  is the
    -variant of  such that .
\end{itemize}
Therefore, a -formula , where 
is a normalized \Lang-conjunction, is equisatisfiable with , where  and  are two fresh 
distinct set variables not occurring in .
\end{proof}

In the proof of Theorem \ref{DEC} we provided a reduction of the s.p.\
for normalized \Lang-conjunctions to the s.p.\ for
\Forallpizero-formulae.  Therefore, the undecidability results of 
Theorem~\ref{PEANUNDEC} and Corollary~\ref{OTHERUNDEC} hold also for 
the corresponding extensions of \Forallpizero-formulae.

\section{Conclusions and plans for future works}
\label{CONC}

In this paper we presented a quantified sublanguage of set theory,
called \Lang, which extends the language  studied in
\cite{BreFerOmoSch1981} with quantifiers involving ordered pairs.  We
reduced its satisfiability problem to the same problem for formulae of
the fragment studied in \cite{CanLonNic2011}.  The resulting decision
procedure runs in non-deterministic exponential time.  However, if one
restricts to formulae with quantifier prefixes of length bounded by a
constant, the decision procedure runs in non-deterministic polynomial
time.  It turns out that such restricted formulae still allow one to
express a large number of useful set-theoretic constructs, as reported
in Table \ref{SETCONS}.  
Finally, we also proved that by slightly extending \Lang-formulae 
with few literals (at least two) of any of the types , , , and , one runs into 
undecidability.

Other extensions of \Lang are to be investigated, in particular those
involving the transitive closure of maps.  Also, the effects of
allowing nesting of quantifiers should be further studied, extending 
the recent results \cite{OmoPol2010, OmoPol2012} to our context.

In contrast with description logics, the semantics of our language is
\emph{multi-level}, as most of the languages studied in the context of
Computable Set Theory. This characteristic may play
a central role when applying set-theoretic languages to knowledge
representation, with particular reference to the \emph{metamodeling} issue
(see \cite{WelFer1994, Mot2007}), which affects the description logics
framework.
However, the multi-level feature is limited in \Lang-formulae, since
clauses like , , 
with  a set variable and , , and  map
variables, are not expressible in it.
In light of this, we intend to investigate extensions of the theory
\Lang which also admit constructs of these forms, and study
applications of these in the field of knowledge representation.

Finally, we intend to study correlations between our language \Lang
and \emph{Disjunctive Datalog} (cf.\ \cite{EitGotMan1997}) in order to
use some of the machinery already available for the latter to
simplify the implementation of an optimized satisfiability test for
the whole fragment \Lang, or just for a \emph{Horn-like} restriction
of it.

\section*{Acknowledgments}
The authors would like to thank the reviewers for their valuable 
comments and suggestions.

\bibliographystyle{eptcs}

\begin{thebibliography}{10}
\providecommand{\bibitemdeclare}[2]{}
\providecommand{\surnamestart}{}
\providecommand{\surnameend}{}
\providecommand{\urlprefix}{Available at }
\providecommand{\url}[1]{\texttt{#1}}
\providecommand{\href}[2]{\texttt{#2}}
\providecommand{\urlalt}[2]{\href{#1}{#2}}
\providecommand{\doi}[1]{doi:\urlalt{http://dx.doi.org/#1}{#1}}
\providecommand{\bibinfo}[2]{#2}

\bibitemdeclare{book}{DLHANDBOOK2}
\bibitem{DLHANDBOOK2}
\bibinfo{editor}{Franz \surnamestart Baader\surnameend}, \bibinfo{editor}{Diego
  \surnamestart Calvanese\surnameend}, \bibinfo{editor}{Deborah~L.
  \surnamestart McGuinness\surnameend}, \bibinfo{editor}{Daniele \surnamestart
  Nardi\surnameend} \& \bibinfo{editor}{Peter~F. \surnamestart
  Patel-Schneider\surnameend}, editors (\bibinfo{year}{2007}):
  \emph{\bibinfo{title}{The Description Logic Handbook: Theory, Implementation,
  and Applications}}, \bibinfo{edition}{second} edition.
\newblock \bibinfo{publisher}{Cambridge University Press},
  \doi{10.1017/CBO9780511711787}.

\bibitemdeclare{inbook}{Ber1966}
\bibitem{Ber1966}
\bibinfo{author}{R.~\surnamestart Berger\surnameend} (\bibinfo{year}{1966}):
  \emph{\bibinfo{title}{The undecidability of the domino problem}}, pp.
  \bibinfo{pages}{1--72}.
\newblock {\sl \bibinfo{series}{Mem. Amer. Math. Soc.}}~\bibinfo{volume}{66},
  \bibinfo{publisher}{American Mathematical Society},
  \doi{10.1007/978-3-540-69407-6\_51}.

\bibitemdeclare{book}{BorGraGur1997}
\bibitem{BorGraGur1997}
\bibinfo{author}{Egon \surnamestart B{\"o}rger\surnameend},
  \bibinfo{author}{Erich \surnamestart Gr{\"a}del\surnameend} \&
  \bibinfo{author}{Yuri \surnamestart Gurevich\surnameend}
  (\bibinfo{year}{1997}): \emph{\bibinfo{title}{The Classical Decision
  Problem}}.
\newblock \bibinfo{series}{Perspectives in Mathematical Logic},
  \bibinfo{publisher}{Springer}.

\bibitemdeclare{article}{BreFerOmoSch1981}
\bibitem{BreFerOmoSch1981}
\bibinfo{author}{Michael \surnamestart Breban\surnameend},
  \bibinfo{author}{Alfredo \surnamestart Ferro\surnameend},
  \bibinfo{author}{Eugenio~G. \surnamestart Omodeo\surnameend} \&
  \bibinfo{author}{Jacob~T. \surnamestart Schwartz\surnameend}
  (\bibinfo{year}{1981}): \emph{\bibinfo{title}{Decision procedures for
  elementary sublanguages of set theory. {II}. {F}ormulas involving restricted
  quantifiers, together with ordinal, integer, map, and domain notions.}}
\newblock {\sl \bibinfo{journal}{Communications on Pure and Applied
  Mathematics}} \bibinfo{volume}{34}, pp. \bibinfo{pages}{177--195},
  \doi{10.1002/cpa.3160340203}.

\bibitemdeclare{book}{CanFerOmo89a}
\bibitem{CanFerOmo89a}
\bibinfo{author}{Domenico \surnamestart Cantone\surnameend},
  \bibinfo{author}{Alfredo \surnamestart Ferro\surnameend} \&
  \bibinfo{author}{Eugenio \surnamestart Omodeo\surnameend}
  (\bibinfo{year}{1989}): \emph{\bibinfo{title}{Computable set theory}}.
\newblock {\sl \bibinfo{series}{International Series of Monographs on Computer
  Science}}~\bibinfo{volume}{6}, \bibinfo{publisher}{Oxford Science
  Publications. Clarendon Press}, \bibinfo{address}{Oxford, {UK}}.

\bibitemdeclare{inproceedings}{CanLonNic2011}
\bibitem{CanLonNic2011}
\bibinfo{author}{Domenico \surnamestart Cantone\surnameend},
  \bibinfo{author}{Cristiano \surnamestart Longo\surnameend} \&
  \bibinfo{author}{Marianna~Nicolosi \surnamestart Asmundo\surnameend}
  (\bibinfo{year}{2011}): \emph{\bibinfo{title}{A {D}ecidable {Q}uantified
  {F}ragment of {S}et {T}heory {I}nvolving {O}rdered {P}airs with
  {A}pplications to {D}escription {L}ogics}}.
\newblock In \bibinfo{editor}{Marc \surnamestart Bezem\surnameend}, editor:
  {\sl \bibinfo{booktitle}{CSL 2011}}, {\sl
  \bibinfo{series}{LIPIcs}}~\bibinfo{volume}{12}, \bibinfo{publisher}{Schloss
  Dagstuhl - Leibniz-Zentrum fuer Informatik}, pp. \bibinfo{pages}{129--143},
  \doi{10.4230/LIPIcs.CSL.2011.129}.

\bibitemdeclare{inproceedings}{CanLonNic2010}
\bibitem{CanLonNic2010}
\bibinfo{author}{Domenico \surnamestart Cantone\surnameend},
  \bibinfo{author}{Cristiano \surnamestart Longo\surnameend} \&
  \bibinfo{author}{Marianna \surnamestart {Nicolosi Asmundo}\surnameend}
  (\bibinfo{year}{2010}): \emph{\bibinfo{title}{A {D}ecision {P}rocedure for a
  {T}wo-sorted {E}xtension of {M}ulti-{L}evel {S}yllogistic with the
  {C}artesian {P}roduct and {S}ome {M}ap {C}onstructs}}.
\newblock In \bibinfo{editor}{Wolfgang \surnamestart Faber\surnameend} \&
  \bibinfo{editor}{Nicola \surnamestart Leone\surnameend}, editors: {\sl
  \bibinfo{booktitle}{CILC2010 : 25th Italian Conference on Computational
  Logic}}.

\bibitemdeclare{inproceedings}{CanLonPis2010}
\bibitem{CanLonPis2010}
\bibinfo{author}{Domenico \surnamestart Cantone\surnameend},
  \bibinfo{author}{Cristiano \surnamestart Longo\surnameend} \&
  \bibinfo{author}{Antonio \surnamestart Pisasale\surnameend}
  (\bibinfo{year}{2010}): \emph{\bibinfo{title}{{C}omparing {D}escription
  {L}ogics with {M}ulti-Level {S}yllogistics: the {D}escription {L}ogic
  \dlmlsscart}}.
\newblock In: {\sl \bibinfo{booktitle}{6th Workshop on Semantic Web
  Applications and Perspectives (SWAP)}}.

\bibitemdeclare{book}{CanOmoPol01}
\bibitem{CanOmoPol01}
\bibinfo{author}{Domenico \surnamestart Cantone\surnameend},
  \bibinfo{author}{Eugenio \surnamestart Omodeo\surnameend} \&
  \bibinfo{author}{Alberto \surnamestart Policriti\surnameend}
  (\bibinfo{year}{2001}): \emph{\bibinfo{title}{Set theory for computing:
  {F}rom decision procedures to declarative programming with sets}}.
\newblock \bibinfo{series}{Monographs in Computer Science},
  \bibinfo{publisher}{Springer-Verlag}, \bibinfo{address}{New York, NY, USA}.

\bibitemdeclare{article}{CanSch91}
\bibitem{CanSch91}
\bibinfo{author}{Domenico \surnamestart Cantone\surnameend} \&
  \bibinfo{author}{Jacob~T. \surnamestart Schwartz\surnameend}
  (\bibinfo{year}{1991}): \emph{\bibinfo{title}{{D}ecision {P}rocedures for
  {E}lementary {S}ublanguages of {S}et {T}heory: {XI}. {M}ultilevel
  {S}yllogistic {E}xtended by {S}ome {E}lementary {M}ap {C}onstructs}}.
\newblock {\sl \bibinfo{journal}{J. Autom. Reasoning}}
  \bibinfo{volume}{7}(\bibinfo{number}{2}), pp. \bibinfo{pages}{231--256},
  \doi{10.1007/BF00243808}.

\bibitemdeclare{article}{EitGotMan1997}
\bibitem{EitGotMan1997}
\bibinfo{author}{Thomas \surnamestart Eiter\surnameend}, \bibinfo{author}{Georg
  \surnamestart Gottlob\surnameend} \& \bibinfo{author}{Heikki \surnamestart
  Mannila\surnameend} (\bibinfo{year}{1997}):
  \emph{\bibinfo{title}{{D}isjunctive {D}atalog}}.
\newblock {\sl \bibinfo{journal}{ACM Trans. Database Syst.}}
  \bibinfo{volume}{22}(\bibinfo{number}{3}), pp. \bibinfo{pages}{364--418},
  \doi{10.1145/261124.261126}.

\bibitemdeclare{article}{FerOmoSch1980}
\bibitem{FerOmoSch1980}
\bibinfo{author}{Alfredo \surnamestart Ferro\surnameend},
  \bibinfo{author}{Eugenio~G. \surnamestart Omodeo\surnameend} \&
  \bibinfo{author}{Jacob~T. \surnamestart Schwartz\surnameend}
  (\bibinfo{year}{1980}): \emph{\bibinfo{title}{Decision Procedures for
  Elementary Sublanguages of Set Theory. I. Multi-level syllogistic and some
  extensions.}}
\newblock {\sl \bibinfo{journal}{Comm. Pure Appl. Math.}}
  \bibinfo{volume}{XXXIII}(\bibinfo{number}{5}), pp. \bibinfo{pages}{599--608},
  \doi{10.1002/cpa.3160330503}.

\bibitemdeclare{inproceedings}{FOS80}
\bibitem{FOS80}
\bibinfo{author}{Alfredo \surnamestart Ferro\surnameend},
  \bibinfo{author}{Eugenio~G. \surnamestart Omodeo\surnameend} \&
  \bibinfo{author}{Jacob~T. \surnamestart Schwartz\surnameend}
  (\bibinfo{year}{1980}): \emph{\bibinfo{title}{Decision Procedures for Some
  Fragments of Set Theory}}.
\newblock In \bibinfo{editor}{Wolfgang \surnamestart Bibel\surnameend} \&
  \bibinfo{editor}{Robert~A. \surnamestart Kowalski\surnameend}, editors: {\sl
  \bibinfo{booktitle}{CADE}}, {\sl \bibinfo{series}{Lecture Notes in Computer
  Science}}~\bibinfo{volume}{87}, \bibinfo{publisher}{Springer}, pp.
  \bibinfo{pages}{88--96}, \doi{10.1007/3-540-10009-1\_8}.

\bibitemdeclare{book}{Mos2005}
\bibitem{Mos2005}
\bibinfo{author}{Yiannis \surnamestart Moschovakis\surnameend}
  (\bibinfo{year}{2005}): \emph{\bibinfo{title}{Notes on Set Theory}},
  \bibinfo{edition}{second} edition.
\newblock \bibinfo{publisher}{Springer}.

\bibitemdeclare{article}{Mot2007}
\bibitem{Mot2007}
\bibinfo{author}{Boris \surnamestart Motik\surnameend} (\bibinfo{year}{2007}):
  \emph{\bibinfo{title}{On the {P}roperties of {M}etamodeling in {OWL}}}.
\newblock {\sl \bibinfo{journal}{J. Log. Comput.}}
  \bibinfo{volume}{17}(\bibinfo{number}{4}), pp. \bibinfo{pages}{617--637},
  \doi{10.1093/logcom/exm027}.

\bibitemdeclare{article}{OmoPol2010}
\bibitem{OmoPol2010}
\bibinfo{author}{Eugenio \surnamestart Omodeo\surnameend} \&
  \bibinfo{author}{Alberto \surnamestart Policriti\surnameend}
  (\bibinfo{year}{2010}): \emph{\bibinfo{title}{The
  {B}ernays-{S}ch{\"o}nfinkel-{R}amsey {C}lass for {S}et {T}heory:
  {S}emidecidability}}.
\newblock {\sl \bibinfo{journal}{Journal of Symbolic Logic}}
  \bibinfo{volume}{75}(\bibinfo{number}{2}), pp. \bibinfo{pages}{459--480},
  \doi{10.2178/jsl/1268917490}.

\bibitemdeclare{article}{OmoPol2012}
\bibitem{OmoPol2012}
\bibinfo{author}{Eugenio \surnamestart Omodeo\surnameend} \&
  \bibinfo{author}{Alberto \surnamestart Policriti\surnameend}
  (\bibinfo{year}{2012}): \emph{\bibinfo{title}{The
  {B}ernays-{S}ch{\"o}nfinkel-{R}amsey {C}lass for {S}et {T}heory:
  {D}ecidability}}.
\newblock {\sl \bibinfo{journal}{Journal of Symbolic Logic, to appear}}.

\bibitemdeclare{book}{SchCanOmo11}
\bibitem{SchCanOmo11}
\bibinfo{author}{Jacob~T. \surnamestart Schwartz\surnameend},
  \bibinfo{author}{Domenico \surnamestart Cantone\surnameend} \&
  \bibinfo{author}{Eugenio~G. \surnamestart Omodeo\surnameend}
  (\bibinfo{year}{2011}): \emph{\bibinfo{title}{Computational logic and set
  theory: Applying formalized logic to analysis}}.
\newblock \bibinfo{publisher}{Springer-Verlag},
  \doi{10.1007/978-0-85729-808-9}.
\newblock \bibinfo{note}{Foreword by M. Davis}.

\bibitemdeclare{book}{SchDewSchDub1986}
\bibitem{SchDewSchDub1986}
\bibinfo{author}{Jacob~T. \surnamestart Schwartz\surnameend},
  \bibinfo{author}{Robert B.~K. \surnamestart Dewar\surnameend},
  \bibinfo{author}{Edmond \surnamestart Schonberg\surnameend} \&
  \bibinfo{author}{E~\surnamestart Dubinsky\surnameend} (\bibinfo{year}{1986}):
  \emph{\bibinfo{title}{Programming with sets; an introduction to SETL}}.
\newblock \bibinfo{publisher}{Springer-Verlag New York, Inc.},
  \bibinfo{address}{New York, NY, USA}.

\bibitemdeclare{techreport}{WelFer1994}
\bibitem{WelFer1994}
\bibinfo{author}{Christopher~A. \surnamestart Welty\surnameend} \&
  \bibinfo{author}{David~A. \surnamestart Ferrucci\surnameend}
  (\bibinfo{year}{1994}): \emph{\bibinfo{title}{What's in an instance?}}
\newblock \bibinfo{type}{Technical Report}, \bibinfo{institution}{RPI Computer
  Science}.

\end{thebibliography}

\end{document}