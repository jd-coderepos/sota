\documentclass[preprint,10pt]{elsarticle}
\usepackage{amssymb,amsmath}
\usepackage{amsthm}
\usepackage{graphicx}
\usepackage{color}
\usepackage{epstopdf}
\usepackage{subfigure}
\DeclareGraphicsRule{.pdftex}{pdf}{.pdftex}{}
\newcommand{\hide}[1]{}
\newcommand{\QED}{\hfill}
\pagestyle{myheadings}
\newtheorem{theorem}{Theorem}
\newtheorem{lemma}{Lemma}
\newtheorem{corollary}{Corollary}
 \newtheorem{comment}{Comment}
  \newtheorem{example}{Example}
 \newtheorem{algorithm}{Algorithm}
 \newtheorem{procedure}{Procedure}
  \newtheorem{proposition}{Proposition}
 \usepackage{xcolor}
\usepackage{lineno}
\def\linenumberfont{\normalfont\small}
\usepackage{xcolor}
\def\linenumberfont{\normalfont\small}
\newcommand{\red}[1]{\textcolor{red}{#1}}
\newcommand{\blue}[1]{\textcolor{blue}{#1}}
\def\TKmarrow{{\marginpar[\hfill\textcolor{red}{TK}~~~~~~~]{\textcolor{red}{TK}}}}
\def\tiko#1{\textcolor{red}{{\sf Tiko: }{\TKmarrow\sf #1}}}
\def\BBmarrow{{\marginpar[\hfill\textcolor{blue}{BB}~~~~~~~]{\textcolor{blue}{BB}}}}
\def\binay#1{\textcolor{blue}{{\sf Binay: }{\BBmarrow\sf #1}}}


\journal{Discrete Applied Mathematics}
\begin{document}
\begin{frontmatter}

\title{Linear-Time Fitting of a -Step Function}\author{Binay Bhattacharya }
\address{School of Computing Science, Simon Fraser University,
Burnaby, Canada}
\ead{binay@sfu.ca}
\author{Sandip Das}
\ead{sandip.das.69@gmail.com}
\address{Advanced Computing and Microelectronics Unit, Indian Stat. Inst., Kolkata, India}
\author{Tsunehiko Kameda}
\ead{tikokameda@gmail.com}
\address{School of Computing Science, Simon Fraser University,
Burnaby, Canada}



\begin{abstract}
Given a set of  weighted points on the - plane,
we want to find a step function consisting of  horizontal steps
such that the maximum weighted vertical distance from any point to a step is minimized.
Using the prune-and-search technique,
we solve this problem in  time when  is a constant.
Our approach can be applied directly or with small modifications
to solve other similar problems,
such as the maximum error histogram problem
and the line-constrained -center problem,
in  time when  is a constant.
\end{abstract}
\begin{keyword}
linear-time algorithm\sep step function fitting\sep weighted points\sep prune and search\sep
maximum error histogram
\end{keyword}
\end{frontmatter}


\section{Introduction}\label{sec:intro}
Given an integer  and a set  of  weighted points in the plane,
our objective is to fit a -step function to them so that the maximum weighted
vertical distance of the points to the step function is minimized.
We call this problem the {\em -step function problem}.
It has applications in areas such as geographic information systems, 
digital image analysis, data mining, facility locations, and
data representation (histogram), etc.

In the unweighted case, if the points are presorted,
Fournier and Vigneron \cite{fournier2011} showed that
the problem can be solved in linear time using the results of 
\cite{frederickson1991a,frederickson1984,gabow1984}.
Later they showed that the weighted version of the problem can also be solved
in  time~\cite{fournier2013},
using Megiddo's parametric search technique \cite{megiddo1983a}.
Prior to these results, the problem had been discussed by several researchers
\cite{chen2009,diazbanez2001,liu2010,lopez2008,wang2002}. 

Guha and Shim \cite{guha2007} considered this problem in the context of {\em histogram construction}.
In database research, it is known as the {\em maximum error histogram} problem.
In the weighted case,
 this problem is to partition the given points into  buckets based on their -coordinates,
such that the maximum -spread in each bucket is minimized.
This problem is of interest to the data mining community as well (see \cite{guha2007} for references).
Guha and Shim \cite{guha2007} computed the optimum histogram of size ,
minimizing the maximum error.
They present algorithms which run in linear time when the points are unweighted,
and in  time and  space when the points are weighted.

Our objective is to improve the above result to  time when  is a constant.
We show that we can optimally fit a -step function to unsorted weighted points in linear time.
We earlier suggested a possible approach to this problem at an OR workshop~\cite{bhattacharya2013b}.
Here we flesh it out, presenting a complete and rigorous algorithm and proofs.
Our algorithm exploits the well-known properties of prune-and-search along the lines in \cite{bhattacharya2007}.


This paper is organized as follows.
Section \ref{sec:prelim} introduces the notations used in the rest of this paper.
It also briefly discusses how the prune-and-search technique can be used
to optimally fit a -step function (one horizontal line) to a given set of weighted points.
We then consider in Section~\ref{sec:cond2step} a variant of the 2-step function problem,
called the anchored 2-step function problem.
We discuss a ``big partition'' in the context of the -partition of a point set
corresponding to a -step function in Section \ref{sec:kstep}.
Section \ref{sec:algorithm} presents our algorithm for the optimal -step function problem.
Section \ref{sec:conclusion} concludes the paper,
mentioning some applications of our results.



\section{Preliminaries}\label{sec:prelim}
\subsection{Model}\label{sec:model}
Let  be a set of  weighted points in the plane.
For  let  (resp. ) denote the -coordinate (resp. -coordinate)
of point , and let  denote its weight.
The points in  are not sorted,
except that  holds for any .\footnote{For the sake
of simplicity we assume that no two points have the same  or  coordinate.
But the results are valid if this assumption is removed.
}
Let  denote a generic -step function,
whose  segment (=step) is denoted by .
For , segment  represents a half-open horizontal interval 
between two points  and .
The last segment  represents a closed horizontal interval .
Note that ,
which we denote by .
We assume that for any -step function ,
 segments  and  satisfy  and ,
respectively.
Segment  is said to {\em span} a set of points ,
if  holds for each .
A -step function  gives rise to a {\em -partition} of , 
,
 such that segment  spans .
 It satisfies the {\em contiguity condition} in the sense that for each partition ,
if  for ,
then every point  with  also belongs to .
In the rest of this paper, we
consider only partitions that satisfy the contiguity condition.
Fig.~\ref{fig:4steps} shows an example of fitting a 4-step function .
\begin{figure}[ht]
\centering
\includegraphics[height=2.8cm]{figs/4steps.pdf}
\hspace{2mm}
\caption{Fitting a 4-step function.
}
\label{fig:4steps}
\end{figure}

Given a step function , 
defined over an -range that contains ,
let  denote the vertical distance of  from .
We define the {\em cost} of  with respect to 
by the weighted distance

We generalize the cost definition for a set  of points by

Point  is said to be {\em critical} with respect to  if

Note that there can be more than one critical point with respect to a given step function.
We similarly define a critical point with respect to a single segment of 
as a point that has the maximum vertical weighted distance to it.

For a set of weighted points in the plane or on a line,
the point that minimizes the maximum weighted distance to
them is called the {\em weighted 1-center}~\cite{bhattacharya2007}.
Given a {\em -partition} of , 
,
it is clear that  such that .
We call any such partition a {\em big partition}.
A big partition spanned by a segment in an optimal solution plays an important role.
(See Procedure {\tt Big} in Sec.~\ref{sec:findBig}.)


\subsection{Bisector}\label{sec:bisector}
If we map each point  onto the -axis,
the {\em cost} of (or the weighted distance from) 
grows linearly from 0 at  in each direction as a function of .
Consider arbitrary two points  and .
Their costs intersect at either one or two points,
one of which always lies between  and .
If there are two intersections,
the other intersection lies outside interval  on the -axis.
If  and  hold
then there is only one intersection.\footnote{If ,
we can ignore one of the points with the smaller weight.
}
Let  (resp. ) be the -coordinate of the upper (resp. lower) intersection point,
where .
We call the horizontal line  (resp. )
the {\em upper} (resp. {\em lower}) {\em bisector} of  and .
If there are only one intersection,
we pretend that there were two at , which lies between  and .
(Note that the -axis is shown horizontally in Figs.~\ref{fig:2points1} and \ref{fig:2points3} below,
where  increases to the right.)

Let  denote the  pairs,
formed by pairing up the points in  arbitrarily.
The cost lines of the points in each pair  intersect
as shown in Figs.~\ref{fig:2points1}.
\begin{figure}[h]
\centering
\subfigure[]{\includegraphics[height=1.5cm]{figs/2points1.pdf}}
\hspace{2mm}
\subfigure[]{\includegraphics[height=1.5cm]{figs/2points2.pdf}}
\caption{1/3 of upper intersections are at :
(a)  can be ignored at ; (b)  can be ignored at .
}
\label{fig:2points1}
\end{figure}
Let  be the line at or above which at least 2/3 of the upper bisectors lie,
and at or below which at least 1/3 of the upper bisectors lie.
We use  and  to name the subsets of
  that have these two sets of bisectors, respectively.
 Note that  and .
Similarly, let   be the line at or below which at least 2/3 of the  
lower bisectors lie,
and  at or above which at least 1/3 of the bisectors lie.\footnote{{We define  and  this way,
because many points could lie on them.}}
We use  and  to name the subsets of
 that have these two sets of bisectors, respectively.
 Note that  and .
\begin{lemma} \label{lem:one6th}
We can identify  points that can be removed without affecting the weighted 1-center
 for the values
of their -coordinates.
\end{lemma}
\begin{proof}
Consider the following three possibilities.
\begin{enumerate}
\item[(i)]
The weighted 1-center lies above .
\item[(ii)]
The weighted 1-center lies below .
\item[(iii)]
The weighted 1-center lies between  and ,
including  and .
\end{enumerate}

In case (i), there are two subcases,
which are shown in Fig.~\ref{fig:2points1}(a) and (b), respectively.
Since the center lies above , 
we are interested in the upper envelope of the costs in the 
-region given by .
In the case shown in Fig.~\ref{fig:2points1}(a),
the costs of points  and  satisfy  for .
Thus we can ignore .
In the case shown in Fig.~\ref{fig:2points1}(b),
 the costs of points  and  satisfy
 for .
Thus we can ignore .
Since ,
in either case, one point from each such pair can be ignored,
i.e., 1/6 of the points in  can be eliminated, because it cannot affect the weighted 1-center.
In case (ii) a symmetric argument proves that 1/6 of the points in  can be discarded.

\begin{figure}[ht]
\centering
\subfigure[]{\includegraphics[height=1.5cm]{figs/2points3.pdf}}
\hspace{2mm}
\subfigure[]{\includegraphics[height=1.5cm]{figs/2points4.pdf}}
\caption{2/3 of upper bisectors are at .}
\label{fig:2points3}
\end{figure}

In case (iii) see Fig.~\ref{fig:2points3}.
The costs of each pair in  (of the  pairs) as functions of  intersect at most once at .
The cost functions of each pair in  ( pairs) intersect at most once at .
Therefore,   ( pairs must be common to both,)
i.e., both intersections of each such pair occur outside of the -interval .
.
This implies that their cost functions do not intersect within in ,
i.e., one of each pair lies above that of the other in ,
and can be discarded.
\end{proof}


\subsection{Optimal 1-step function}\label{sec:1step}
This problem is equivalent to finding the weighted center for  points on a line.
We pretend that all the points had the same -coordinate.
Then the problem becomes that of finding a weighted 1-center on a line,
i.e., on the -axis.
This can be solved in linear time using Megiddo's {\em prune-and-search} method \cite{bhattacharya2007,chen2015a,megiddo1983a}.
In \cite{megiddo1983b} Megiddo presents a linear time algorithm in the case where the
points are unweighted. 
For the weighted case we now present a more technical algorithm that we can apply
later to solve other related problems.
The following algorithm uses a parameter  which is a small integer constant.

\begin{algorithm}{\rm :} {\tt 1-Step}
\begin{enumerate}
\item
Pair up the points of  arbitrarily.
\item
For each such pair  determine their horizontal bisector lines. 
\item
Determine a horizontal line,  such that 
and  hold.
\item
Determine a horizontal line,  such that   and
 hold.
\item
Determine the critical points for  and .
\item
If there exist critical points for  on both sides of (above and below) , 
then  defines an optimal 1-step function, ; Stop. 
Otherwise, let  (higher or lower than ) be the side of  on which the critical point
lies.
\item
If there exist critical points for  on both sides of ,
 defines ; Stop.
Otherwise, let  (higher or lower than ) be the side of  on which the critical point
lies.
\item
Based on  and , discard 1/6 of the points from ,
based on Lemma~\ref{lem:one6th}.
\item
If the size of the reduced set  is greater than constant , 
repeat this algorithm from the beginning with the reduced set .
Otherwise, determine  using any known method
(which runs in constant time).
\end{enumerate}
\end{algorithm}

\begin{lemma}\label{lem:1step}
An optimal 1-step function  can be found in linear time.
\end{lemma}
\begin{proof}
The recurrence relation for the running time  of {\tt 1-Step} for general  is 
,
which yields .
\end{proof}

\section{Anchored -step function problem}\label{sec:cond2step}
In general, we denote an optimal -step function by 
and its  segment by . 
Later, we need to constrain the first and/or the last step of a step function to be
at a specified height.
A -step function is said to be {\em left-anchored} (resp. {\em right-anchored}),
if  (resp.  ) is assigned a specified value,
and is denoted by  (resp. ).
The {\em anchored -step function} problem is defined as follows.
Given a set  of points and two -values  and ,
determine the optimal -step function   (resp. )
that is left-anchored (resp. right-anchored) at 
 (resp. ) such that cost  (resp. )
 is the smallest possible.
If a -step function is both left- and right-anchored, 
it is said to be {\em doubly anchored} and is
denoted by .


\subsection{Doubly anchored 2-step function}
Suppose that segment  (resp. ) is anchored at  (resp. ).
See Fig.~\ref{fig:anchored2}(a).
\begin{figure}[ht]
\centering
\subfigure[]{\includegraphics[height=3cm]{figs/conditional2.pdf}}
\hspace{4mm}
\subfigure[]{\includegraphics[height=3cm]{figs/goftNhoft.pdf}}
\caption{(a)  and ;
(b) Monotone functions  (in blue) and  (in red).
}
\label{fig:anchored2}
\end{figure}
Let us define two functions  and  by

where  for  and  for .
Intuitively, if we divide the points of  at  into two partitions  and ,
then  (resp. ) gives the cost of partition  (resp. ).
See Fig.~\ref{fig:anchored2}(b).
Clearly the global cost for the entire  is minimized for any 
at the lowest point in the upper envelope of  and ,
which is named .
Since the points in  are not sorted,
 and  are not available explicitly,
but we can compute  in linear time using the {\em prune-and-search} method,
taking advantage of the fact that  is unimodal.

\begin{algorithm}{\rm :} {\tt Doubly-Anch-2-Step}\label{alg:double}
\begin{enumerate}
\item
Initialize .
\item
Find the point in  that has the median -coordinate, .
\item
Evaluate  (resp. ) using (\ref{eqn:g}) (resp. (\ref{eqn:h})).
\item
If  then . Stop.
\item
If  (resp. ), 
i.e.,  (resp. ),
prune all the points  with  (resp. ),
 from ,
remembering just the maximum cost.
\item
Stop when , and find the lowest point .
Otherwise, go to Step~2.
\end{enumerate}
\end{algorithm}

We have the following lemma.
\begin{lemma}\label{lem:doublyAnchored}
An optimal doubly anchored 2-step function
can be found in linear time.
\end{lemma}
\begin{proof}
Steps~2 and 3 of Algorithm {\tt Doubly-Anch-2-Step} can be carried out in linear time.
Since Step~4 cuts the size of  in half every time, Step~2 is entered  times.
Therefore the total time is .
\end{proof}

\subsection{Left- or right-anchored 2-step function}
Without loss of generality, we discuss only a left-anchored 2-step function. 
Given an anchor value ,
we want to determine the optimal 2-step function with the constraint
that , denoted by .
See Fig.~\ref{fig:anchored2}(a).
In this case,  in (\ref{eqn:h}) is not given; 
we need to find the optimal value for it.
But assume for now that  is also given,
and execute {\tt Doubly-Anch-2-Step}.
From the solution that it yields,
can we find the direction in which to move  to find the optimal
left-anchored 2-step function?
\begin{lemma}
Let  (resp. ) be the left (resp right) partition of  generated by {\tt Doubly-Anch-2-Step}
such that   (resp. ), where  without loss of generality.
Assume that  is maximal in the sense that the boundary between  and 
cannot be moved to the right without increasing the cost of the solution.
\begin{enumerate}
\item[(a)]
If  then the optimal right-anchor cannot lie above .
\item[(b)]
If  and there is a critical point in  for  above ,
then the optimal right-anchor cannot lie below .
\item[(c)]
If  and there is a critical point in  for  below ,
then the optimal right-anchor cannot lie above .
\end{enumerate}
\end{lemma}
\begin{proof}
(a) Assume first that the critical point  for  lies below  ().
Then we cannot reduce the cost by changing the value of .
Therefore, assume that  lies above .
By the definition of , the leftmost point in  lies above ,
and moving the boundary between  and  to the right increases the cost,
which is due to the weighted distance between  and the new point in ,
and this increase is independent of the value of .
Moving this boundary to the left cannot decrease the cost,
until  becomes a part of ,
and even then a decrease is not possible unless  is made smaller.
Otherwise,  wouldn't be optimal with the current .

(b) We know that moving the boundary between  and  to the right increases the cost
 if  is kept at the same value.
The cost increases if  is made smaller.

(c) Symmetric to Case (b).
\end{proof}

\begin{figure}[ht]
\centering
\includegraphics[height=4cm]{figs/leftAnc1.pdf}
\hspace{2mm}
\caption{An example for {\tt -Anch-2-Step}.
}
\label{fig:leftAnc}
\end{figure}
\begin{example}
In Fig.~\ref{fig:leftAnc},
assume that  is slightly larger than .
We have a doubly anchored solution with the minimum cost (weighted distance)
equal to .
When the boundary between  and  is at ,
we can reduce the cost of the optimal solution by moving  up.
We cannot do so if the boundary is at ,
because  would increase.
This is why we maximize  in Step 5(b) of Algorithm~\ref{alg:cond2}
presented below.
\QED
\end{example}



To make use of the prune-and-search method,
we want to find the big partition (defined in Sec.~\ref{sec:model}),
 or , that is spanned by one segment of .
\hide{\begin{procedure} {\tt Big2--Anc}\label{proc:big2}
\begin{enumerate}
\item
Divide  into left partition  and right partition ,
whose sizes differ by at most one.\footnote{As before,
we assume that the points have different -coordinates.
}
\item
Let  be the segment with  spanning ,
and let  be the 1-step (optimal) solution for .
\item
If  (resp. )
then  (resp. ) is the big partition.
\end{enumerate}
\end{procedure}
}If  is the big partition, we can eliminate all the points belonging to it,
without affecting  that we will find.
See Step~4 of the Algorithm~\ref{alg:cond2} given below.
We then repeat the process with the reduced set .
If  is the big partition, on the other hand, we need to do more work,
similar to what we did to find an optimal 1-step function.
Namely,
we determine values  and  for  by executing Algorithm {\tt 1-Step}.
We then find a doubly anchored 2-step solution for  with left anchor 
and right anchor .


\begin{algorithm}{\rm :} {\tt -Anch-2-Step}\label{alg:cond2}
\begin{enumerate}
\item
Divide  into left partition  and right partition ,
whose sizes differ by at most one.\footnote{As before,
we assume that the points have different -coordinates.
}
\item
Let  be the segment with  spanning ,
and let  be the 1-step (optimal) solution for .\footnote{
Segment  can be found in  time by Lemma~\ref{lem:1step}.}
\item
If  then
output , which defines .
Stop.
\item
If , remove from  the points of ,
except the critical point for .
Go to Step~6.
\item
If  then carry out the following steps.
\begin{enumerate}
\item
Determine points  and  for  as described in Algorithm {\tt 1-Step}.
\item
Execute {\tt Doubly-Anch-2-Step},
and find the solution whose left partition is maximal.
Repeat it with right anchor .
\item
Eliminate 1/6 of the points of  from , based on the two solutions
(as in Steps 6--8 of Algorithm {\tt 1-Step}.)
\end{enumerate}
\item
If  (a small constant),
repeat Steps~1 to 4.
Otherwise, optimally solve the problem in constant time, using a known method.
\end{enumerate}
\end{algorithm}
In the example in Fig.~\ref{fig:leftAnc}, 
assume that  is not given,
and  is determined by Step~2.
Then we have ,
and Step~5 applies.
According to Step~5(a), we determine .
We then find the doubly anchored solution with the right anchor set to .


\begin{lemma}
Algorithm {\tt -Anch-2-Step} computes  correctly,
and runs in linear time.
\end{lemma}
\begin{proof}
Step~3 is obviously correct.
If  holds in Step~4,
then the first partition of  contains .
We need to keep the critical point for ,
but all other points of  can be ignored from now on
because  will expand.
If  holds in Step~5,
then the first partition of  is contained in .

Each iteration of Steps 3 and 4 will eliminate at least  of the points of .
Such an iteration takes linear time in the input size. 
The total time needed for all the iterations is therefore linear.
\end{proof}

\section{-step function}\label{sec:kstep}
\subsection{Approach}
To design a recursive algorithm, assume that for any set of points ,
we can find the optimal -step function and the optimal left- and right-anchored -step function 
for any  in  time,
where  is a constant .
We have shown that this is true for  in the previous two sections.
So the basis of induction holds.

Given an optimal -step function , for each ,
let  be the set of points vertically closest to segment .
By definition, the partition
 satisfies the contiguity condition.
It is easy to see that
for each segment , there are (local) critical points with respect to ,
lying on the opposite sides of .

In finding an optimal -step function,
we first identify a big partition that will be spanned by a segment in
an optimal solution.
By Lemma~\ref{lem:big},
such a big partition always exists.
Our objective is to eliminate a constant fraction of the points in a big partition.
This will guarantee that a constant fraction of the input set is eliminated when  is a fixed constant.
The points in the big partition other than two critical points are ``useless''
and can be eliminated from further considerations.\footnote{Note that there may
be more than two critical points in which case all but two are ``useless.''}
This elimination process is repeated until the problem size gets small enough
to be solved by an exhaustive method in constant time.

\subsection{Feasibility test}\label{sec:feasibility}
Given a weighted distance (=cost) ,
a point set  is said to be -{\em feasible} if there exists a -step function
 such that . 
To test -feasibility
we first try to identify the first segment  of a possible -step function .
To this end we compute the median  of  in  time,
and divide  into two parts  and ,
which also takes  time.
Note that  and  hold.
We then find the intersection  of the -intervals in .
Assuming that  is -feasible,
then we have two cases.

Case (a): []  ends at some point .
Throw away all the points in  and look for the longest  limited by cost ,
considering only the points in  from the left.



Case (b): []  may end at some point .
Throw away all the points in 
and look for the longest , using  and the points in  from the left.


Clearly,
we can find the longest  in  time.
Remove the points spanned by  from ,
and find  in  time, and so on.
Since we are done after finding  steps ,
it takes  time.

\begin{lemma}\label{lem:feasibility}
We can test -feasibility in  time.
\QED
\end{lemma}

\subsection{Identifying a big partition}\label{sec:findBig}
\begin{lemma}\label{lem:big}
Let  be any -partition of ,
satisfying the contiguity condition,
such that the sizes of the partitions differ by no more than 1,
and let  be an optimal -partition.
Then there exists an index  such that  is a big partition spanned by .
\end{lemma}
\begin{proof} 
Let  be the smallest index such that .
Such an index must exists, because if 
for all  then .
We clearly have ,
which implies that  spans .
\end{proof}

Given a point set  in the - plane,
let  be any -partition of ,
satisfying the contiguity condition,
such that the sizes of the partitions differ by no more than 1.
The following procedure returns a big partition  spanned by ,
whose existence was proved by Lemma~\ref{lem:big}.
Since ,
 is implicit in the input to the next procedure.

\begin{procedure}{\rm :} {\tt Big}\label{proc:bigk}

\noindent
\begin{enumerate}
\item
Using Algorithm~{\tt 1-Step}, compute the optimal 1-step function for 
and let  be its cost for .
If  is not -feasible (i.e., ),
then return  and stop.\footnote{There exists an optimal solution for 
 in which  spans .}
\item
Using Algorithm~{\tt 1-Step}, compute the optimal 1-step function for 
and let  be its cost for .
If  is not -feasible (i.e., ),
then return  and stop.
\item
Find an index  such that for 
 is -feasible, 
and for   is not -feasible.\footnote{This means
that  and ,
where  is the cost of the optimal solution for .
Unless  for all , such an index  always exists.
[We should indicate why.]}
Return  and stop.
\end{enumerate}
\end{procedure}

\begin{lemma}\label{lem:Bigiscorrect}
Procedure {\tt Big} is correct.
\end{lemma}
\begin{proof}
It is clear that Steps~1 and 2 are correct.
To show that Step~3 is also correct, 
we {\em stretch} a step  of an optimal step function
by making it as long as possible as follows.
Move  (resp. ) to the left (resp. right) as far as possible without changing the cost
of the step function.
The step that has been stretched is called a {\em stretched step.} 
Let us assume without loss of generality that  corresponding to  returned by Step~3 is stretched.
Since ,
we must have .

The optimal solution  for  has cost ,
which is too small for  to be -feasible.
Regarding the remaining points ,
let  denote the optimal -step function for this point set.
If ,
the  would be -feasible.
Since it is not,
  would be stretched to the right under the optimal solution ,
i.e., .
Together with ,
it follows that  is spanned by .
\end{proof}

\begin{lemma}\label{lem:Bigislinear}
Procedure {\tt Big} runs in linear time in .
\end{lemma}
\begin{proof}
In Step~1, the optimal 1-step function for  can be found in  time by Lemma~\ref{lem:1step},
and it takes  time to test if  is not -feasible by Lemma~\ref{lem:feasibility}.
Similarly, Step~2 can be carried out in  time.
To carry out Step~3,
we compute, using binary search,  values out of ,
which takes  time for some function ,
under the assumption that any -step function problem, ,
is solvable in time linear in the size of the input point set.
\end{proof}


\section{Algorithm}\label{sec:algorithm}
\subsection{Optimal -step function}
In this section we are assuming that we can solve any -step and anchored
-step function problems for any .
We have shown that this is true for  in the previous section.
So the basis of recursion holds. 

Let us find an optimal doubly anchored -step function, ,
which consists of  horizontal segments,
,
satisfying ,  ,   ,  and ,
where  and  are given constants.
Let   {be}  the set of points of  vertically closest to . For each segment , there are critical points with respect to , lying on the opposite sides of .
In order to find , 
we first execute {\tt Big}
and identify a big partition containing at least  points,
which are vertically closest to the same segment in some optimal solution.


Once a big partition, say , is identified,
We first determine  and  for  as described in Algorithm {\tt 1-Step}.
To illustrate the idea,
let us consider a special case where  and .
We execute {\tt -Anch-2-Step} and  {\tt -Anch-2-Step}
and determine  and  as in  Algorithm {\tt 1-Step}.
We can thus eliminated 1/6 of the points in .
We repeat this with the reduced .
It may turn out that the right partition is the big partition in the next round.
Then we can repeat the above process symmetrically.
Eventually, the size of  gets small enough,
so that we can find the solution using an exhaustive method.

For a general  and ,
we need to find the left- and right-anchored solution for  and ,
and prune  of the points in  using {\tt Prune-Big}, given below,
which is very similar to Algorithm {\tt 1-Step}.
Let  be a big partition spanned by ,
which is an input to the following procedure.
\begin{procedure}{\rm :} {\tt Prune-Big}

\noindent
{\bf Output:} 1/6 of points in  removed.


\begin{enumerate}
\item
Determine  and  for  as in Algorithm {\tt 1-Step}. 
\item 
If , find two right-anchored -step functions  for ,
one anchored by  and the other anchored by .
\item
If , find two left-anchored -step functions
  for ,
one anchored by  and the other anchored by .
\item
Identify 1/6 of the points in  with respect to  and ,
which are ``useless''\footnote{See Step~8 of Algorithm~{\tt 1-Step}.}
based on 
and  found above,
and remove them from . 
\end{enumerate}
\end{procedure}



\begin{lemma}\label{lem:anchored}
{\tt Prune-Big} runs in linear time when  is a constant..
\QED
\end{lemma}

We can now describe our algorithm formally as follows.

\begin{algorithm}{\rm :} {\tt -Step}. 

\noindent
{\bf Output:}  Optimal -step function 
\begin{enumerate}
\item 
Divide  into partitions ,
satisfying the contiguous condition,
such that their sizes differ by no more than one.
\item 
Execute Procedure {\tt Big} to find a big partition 
spanned by . 
\item
Execute Procedure {\tt Prune-Big}.
\item
If  for some fixed , 
repeat Steps~1 to 3 with the reduced .
\end{enumerate}
\end{algorithm}


\subsection{Analysis of algorithm}
To carry out Step 1 of Algorithm {\tt -Step}, 
we first find the  smallest among ,
for .
We then place each point in  into  partitions delineated by these
 values.
It is clear that this can be done in  time.\footnote{This could be done in  time.}
As for Step~2,
we showed in Sec.~\ref{sec:findBig} that finding a big partition spanned by an optimal step
 takes  time, since  is a constant.
Step~3 also runs in  time by Lemma~\ref{lem:anchored}.
Since Steps 1 to 3 are repeated  times,
each time with a point set whose size is at most a constant fraction of the size of the previous set,
the total time is also , when  is a constant.
By solving a recurrence relation for the running time of Algorithm~{\tt -Step},
we can show that it runs in  time.

\begin{theorem}
Given a set of  points in the plane ,
we can find the optimal -step function that minimizes the maximum distance
to the  points in  time.
\QED
\end{theorem}
Thus the algorithm is optimal for a fixed .



\section{Conclusion and Discussion}\label{sec:conclusion}
We have presented a linear time algorithm to solve the optimal -step function problem,
when  a constant.
Most of the effort is spent on identifying a ``big partition.''
It is desirable to reduce the constant of proportionality. 

The {\em size- histogram construction problem}~\cite{guha2007},
where the points are not weighted, 
is similar to the problem we addressed in this paper.
Its generalized version,
where the points are weighted,
is equivalent to our problem, and thus can be solved in optimal linear time when  is a constant.
The {\em line-constrained  center problem} is defined by:
Given a set  of weighted points in the plane and a horizontal line ,
determine  centers on 
such that the maximum weighted distance of the points to their closest centers is minimized.
This problem was solved in optimal  time for arbitrary  even if the points
are sorted~\cite{karmakar2013,wang2014a}.
Our algorithm presented here can be applied to solve this problem 
in  time if  is a constant.

A possible extension of our work reported here is to use a cost other than the weighted vertical distance.
There is a nice discussion in \cite{guha2007} on the various measures one can use. 
Our complexity results are valid
if the cost is more general than (\ref{eqn:pointCost}),
in particular, ,
which is often used as an error measure.
\section*{Acknowledgement}\label{sec:ack}
This work was supported in part by Discovery Grant \#13883 from
the Natural Science and Engineering Research Council (NSERC) of Canada and in part by MITACS,
both awarded to Bhattacharya.


\section*{Reference}
\begin{thebibliography}{10}
\providecommand{\url}[1]{\texttt{#1}}
\providecommand{\urlprefix}{URL }

\bibitem{ajtai1983}
Ajtai, M., Koml\'{o}s, J., Szemer\'{e}di, E.: An  sorting
  network. In: Proc. 15th Annual ACM Symp. Theory of Computing (STOC). pp. 1--9
  (1983)

\bibitem{bhattacharya2007}
Bhattacharya, B., Shi, Q.: Optimal algorithms for the weighted -center
  problems on the real line for small . In: Dehne, F., Sack, J.-R., Zeh, N. (eds.)
  WADS 2007. LNCS, vol. 4619, pp. 529--540. Springer, Heidelberg (2007)

\bibitem{bhattacharya2013b}
Bhattacharya, B., Das, S.: Prune-and-search technique in facility location. In:
  Proc. 55th Conf. Canadian Operational Research Society (CORS). p.~76 (May
  2013)

\bibitem{chen2015a}
Chen, D.Z., Li, J., Wang, H.: Efficient algorithms for the one-dimensional
  -center problem. Theoretical Comp. Sci.  592,  135--142 (August 2015)

\bibitem{chen2009}
Chen, D.Z., Wang, H.: Approximating points by a piecewise linear function: I.
In: Dong, Y., Du, D.-Z., Ibarra, O. (eds.) ISAAC 2009. LNCS, vol. 5878,
pp. 224--233. Springer, Heidelberg (2009)

\bibitem{cole1987}
Cole, R.: Slowing down sorting networks to obtain faster sorting algorithms. J.
  ACM  34,  200--208 (1987)

\bibitem{diazbanez2001}
{D\'{i}az-B\'{a}\~{n}ez}, J., Mesa, J.: Fitting rectilinear polygonal curves to
  a set of points in the plane. European J. Operations Research  130,  214--222
  (2001)

\bibitem{fournier2011}
Fournier, H., Vigneron, A.: Fitting a step function to a point set.
  Algorithmica  60,  95--101 (2011)

\bibitem{fournier2013}
Fournier, H., Vigneron, A.: A deterministic algorithm for fitting a step
  function to a weighted point-set. Information Processing Letters  113,
  51--54 (2013)

\bibitem{frederickson1991a}
Frederickson, G.: Optimal algorithms for tree partitioning. In: Proc. 2nd
  ACM-SIAM Symp. Discrete Algorithms. pp. 168--177 (1991)

\bibitem{frederickson1984}
Frederickson, G., Johnson, D.: Generalized selection and ranking. SIAM J.
  Computing  13(1),  14--30 (1984)

\bibitem{gabow1984}
Gabow, H., Bentley, J., Tarjan, R.: Scaling and related techniques for geometry
  problems. In: Proc. 16th Annual ACM Symp. Theory of Computing (STOC). pp.
  135--143 (1984)

\bibitem{guha2007}
Guha, S., Shim, K.: A note on linear time algorithms for maximum error
  histograms. IEEE Trans. Knowl. Data Eng.  19,  993--997 (2007)

\bibitem{karmakar2013}
Karmakar, A., Das, S., Nandy, S.C., Bhattacharya, B.: Some variations on
  constrained minimum enclosing circle problem. J. Comb. Opt.  25(2),  176--190
  (2013)

\bibitem{liu2010}
Liu, J.Y.: A randomized algorithm for weighted approximation of points by a
  step function. In: Proc. 4th Ann. Int. Conf. Combinatorial Optimization and
  Applications (COCOA), Springer-Verlag. vol. LNCS 6509, pp. 300--308 (2010)

\bibitem{lopez2008}
Lopez, M., Mayster, Y.: Weighted rectilinear approximation of points in the
  plane. In: Laber, E.S., Bornstein, C., Nogueira, L.T., Faria, L. (eds.) LATIN 2008.
LNCS, vol. 4957, pp. 642--653. Springer, Heidelberg (2008)

\bibitem{megiddo1983a}
Megiddo, N.: Applying parallel computation algorithms in the design of serial
  algorithms. J. ACM  30,  852--865 (1983)

\bibitem{megiddo1983b}
Megiddo, N.: Linear-time algorithms for linear-programming in  and
  related problems. SIAM J. Computing  12,  759--776 (1983)

\bibitem{wang2002}
Wang, D.: A new algorithm for fitting a rectilinear -monotone curve to a set
  of points in the plane. Pattern Recognition Letters  23,  329--334 (2002)

\bibitem{wang2014a}
Wang, H., Zhang, J.: Line-constrained -median, -means and -center
  problems in the plane. In: Ahn, H.-K., Shin, C.-S. (eds.) ISAAC 2014. LNCS,
  vol. 8889, pp. 3--14. Springer, Heidelberg (2014)

\end{thebibliography}



\end{document}
