\documentclass[11pt,twocolumn]{IEEEtran}
\usepackage{caption2}
\captionstyle{hang}

\usepackage{stfloats}

\usepackage{amsmath,mathrsfs,amssymb,epsfig,psfrag,amsthm,bm,multirow, graphicx,color}
\usepackage{ mathrsfs,dsfont,yfonts }
\usepackage{bbold}
\usepackage{textcomp}
\DeclareMathOperator*{\argmax}{arg\,max}
\DeclareMathOperator*{\argmin}{arg\,min}
\IEEEoverridecommandlockouts


\addtolength{\oddsidemargin}{.07in}
\addtolength{\evensidemargin}{.07in}
\addtolength{\textwidth}{-0.14in}
\addtolength{\topmargin}{0in}
\addtolength{\textheight}{-0.1in}

\newtheorem{proposition}{Proposition}
\newtheorem{corollary}{Corollary}
\newtheorem{conjecture}{Conjecture}
\newtheorem{lemma}{Lemma}
\newtheorem{definition}{Definition}
\newtheorem{claim}{Claim}

\begin{document}

\title{Downlink Scheduling over Markovian Fading Channels}

\author{\emph{Wenzhuo Ouyang, Atilla Eryilmaz, and Ness B.
Shroff}
\vspace{-9pt}

\vspace{-5pt}
\thanks{Wenzhuo Ouyang and Atilla Eryilmaz are with the Department of ECE, The Ohio State University (e-mails: ouyangw@ece.osu.edu, eryilmaz@ece.osu.edu).
Ness B. Shroff holds a joint appointment in both the Department of ECE and the Department of CSE at The Ohio State University (e-mail: shroff@ece.osu.edu). }
\thanks{A preliminary version of this paper appeared in INFOCOM 2012.}
\thanks{This work was supported in part by NSF grants CAREER-CNS-0953515, CCF-0916664, CNS-1012700, DTRA grant HDTRA 1-08-1-0016, Qatar National Research Fund (QNRF) under the National Priorities Research Program (NPRP) grant NPRP 09-1168-2-455, and ARO MURI award  W911NF-08-1-0238.}
}
\maketitle

\begin{abstract}
We consider the scheduling problem in downlink wireless
networks with heterogeneous, Markov-modulated, ON/OFF channels. It
is well-known that the performance of scheduling over fading
channels relies heavily on the accuracy of the available Channel
State Information (CSI), which is costly to acquire. Thus, we
consider the CSI acquisition via a practical ARQ-based feedback
mechanism whereby channel states are revealed at the end of only scheduled
users' transmissions. In the assumed presence of
temporally-correlated channel evolutions, the desired scheduler must
optimally balance the \emph{exploitation-exploration trade-off},
whereby it schedules transmissions both to exploit those channels
with up-to-date CSI and to explore the current state of those with
outdated CSI.

In earlier works, Whittle's Index Policy had been suggested as a
low-complexity and high-performance solution to this problem.
However, analyzing its performance in the typical scenario of
statistically heterogeneous channel state processes has remained
elusive and challenging, mainly because of the highly-coupled and
complex dynamics it possesses. In this work, we overcome these
difficulties to rigorously establish the asymptotic optimality
properties of Whittle's Index Policy in the limiting regime of
many users. More specifically: (1) we prove the \emph{local
optimality} of Whittle's Index Policy, provided that the
initial state of the system is within a certain neighborhood of a
carefully selected state; (2) we then establish the \emph{global
optimality} of Whittle's Index Policy under a recurrence assumption
that is verified numerically for our problem. These results
establish that Whittle's Index Policy possesses analytically provable optimality
characteristics for scheduling over heterogeneous and
temporally-correlated channels.
\end{abstract}

\section{Introduction}

Channel fluctuation is an intrinsic characteristic of wireless communications. Such a variation calls for allocation of the wireless resources in a dynamic manner, leading to the classic \emph{opportunistic scheduling principle} (e.g., \cite{Knopp}\cite{JSAC_Liu}). Under the assumption that the instantaneous channel state information (CSI) is fully available to the scheduler, many efficient opportunistic scheduling algorithms (e.g., \cite{tassiulas}-\cite{Atilla}) have been proposed and extensively studied.

More recent works have focused on designing scheduling algorithms under imperfect CSI, where the channel state is modeled as independent and identically distributed (\textit{i.i.d.}) processes across time (e.g., \cite{2stage}-\cite{Allerton}). On the other hand, although the \textit{i.i.d.} channel model brings ease of analysis, it fails to capture the time-correlation of the fading channels \cite{Tse}. Specifically, it fails to exploit the channel memory, which is a critical resource for making scheduling decisions. However, designing efficient scheduling schemes under time-correlated channels with imperfect CSI is a very challenging problem. The challenge is mainly because of the difficulty in making the classic `exploitation versus exploration' trade-off  (e.g., \cite{clinical,reinforce}), in which a scheduler needs to strike a balance between selecting the channels with up-to-date channel memory that guarantees high immediate gains, or to explore the channels with outdated CSI for more informed decisions and associated future throughput gains.



We consider the downlink scheduling problem where a base station transmits to the users within its transmission range, subject to scheduling constraints. To model the time correlations present over fading channels, we assume that wireless channels evolve as Markov-modulated ON/OFF processes. The channel state information is obtained from ARQ-based feedback, only \emph{after} each scheduled transmission. Nevertheless, due to time correlation, the memory of the past channel state can be used to predict the current channel state \emph{prior to} scheduling decision.
Hence, channel memory should be intelligently exploited by the scheduler in order to achieve high throughput performance.

In a related work \cite{YingShakkottai}, a similar problem is
considered under delayed CSI, where it is assumed that
perfect CSI is available within a maximum delay, which is in turn smaller than the delay experienced by the ARQ feedback used for collision detection. These assumptions allow  the scheduling decisions to be
decoupled from CSI acquisition, which leads to the
development of centralized as well as distributed
schedulers. However, this approach does not use ARQ as a means of acquiring improved channel quality information. In contrast, in our setup the nature of ARQ feedback creates an implicit impact of scheduling
decisions on the CSI feedback, which completely transforms the nature of
the optimal scheduler design, and therefore requires a different
approach. Under the scenario where all the channels have
\emph{identical Markov statistics},
round-robin-based algorithms (e.g., \cite{Liu}-\cite{Neely_utility}) have
been shown to possess optimality properties in throughput performance.
However, the round-robin-based algorithms are no longer optimal in
\emph{asymmetric scenarios}, e.g., when different channels have different
Markov transition statistics, as is naturally the case in typical
heterogeneous conditions.



Under the asymmetric scenarios, our downlink scheduling problem is an example of the classic Restless Multiarmed Bandit Problem
(RMBP) \cite{Whittle}. Low-complexity Whittle's Index Policies\hspace{3pt}\cite{Whittle}\hspace{3pt}for the downlink scheduling problem have been
proposed in \cite{Zhao_index}\cite{Infocom11} based on RMBP theory.
However, although Whittle's Index Policy can bring significant
throughput gains by exploiting the channel memory \cite{Infocom11},
the analytical characterization of its performance under asymmetric
scenarios is very challenging and prohibitively technical. This is because asymmetry leads to a sophisticated interplay of memory evolution among channels with heterogeneous characteristics, which brings a significant challenge to the analysis of Whittle's Index Policy not present in the perfectly symmetric scenario. 



For RMBP problems under general scenarios, Whittle's Index Policy has been proven in \cite{Weber} to be asymptotically optimal
as the number of users grows, provided a non-trivial condition,
known as Weber's condition, holds. Nonetheless, Weber's condition
concerns the global convergence of a non-linear differential
equation, which is extremely difficult to verify even
numerically in our downlink scheduling scenario. In \cite{ApproxRMBP}, optimality properties of general RMBP is studied, where a sub-optimal BALANCED-INDEX policy, as well as a THRESHOLD-WHITTLE policy, are proved to provide approximation performance, i.e., achieves at least half of the optimal reward. Our work takes a different approach than \cite{ApproxRMBP} to specifically study the per-user throughput performance of the Whittle's Index Policy for downlink scheduling, and consider the strict optimality metric in the asymptotic regime when the number of users scales. 

In this paper, we take significant steps in analyzing the optimality
properties of Whittle's Index Policy for the downlink scheduling
problem in the presence of channel heterogeneity. Specifically, our contributions are as follows.

\begin{itemize}
\item We apply the Whittle's index framework to our downlink scheduling problem and identify the optimal policy for the problem with a relaxed constraint in Section~\ref{sec:bound}. This policy, with carefully selected randomization, provides a performance upper bound to Whittle's Index Policy.

\item We establish the local optimality of Whittle's Index Policy in the asymptotic regime when the number of users scales in Section~\ref{sec:local}. Specifically, we show that the performance of the index policy can get arbitrarily close to that of the relaxed-constraint optimal policy, provided that the initial state of the system is within a certain neighborhood of a carefully selected state.

\item Based on the local optimality result, under a numerically verifiable recurrence assumption, we then establish the global optimality of Whittle's Index Policy in the limiting regime of many users in Section~\ref{sec:global}.
\end{itemize}

\section{System Model and Problem Formulation}
\label{sec:model}

\subsection{Downlink Wireless Channel Model}

We consider a time-slotted, wireless downlink system with one base
station and  users. The wireless channel  between base
station and user  remains static within each time slot  and
evolves stochastically across time slots, independently across
users. We adopt the simplest non-trivial model of time-correlated
fading channels by considering two-state ON/OFF channels, where the
state space of channel  is , with
the value of each state representing the transmission rate a channel
can support at the state.
\begin{figure}
\centering
\includegraphics[width=3in]{chain.eps}
\caption{Two state Markov chain model for channels in class .}
\label{fig:chain}
\end{figure}

One important component of our model is the inclusion of channel heterogeneity that the users will typically experience in real systems. Such asymmetry creates a significant challenge to the design and analysis of optimal scheduling schemes compared to perfectly symmetric channels. To avoid cumbersome notation and unessential technical complications, in this work we model channel asymmetry by considering only \emph{two classes} of channel statistics. Specifically, for all the channels in class , , their states evolve according to the same Markov statistics. However, these characteristics differ between classes. The state transition of channels in class  is depicted in Fig.~\ref{fig:chain}, represented by a  probability transition matrix,
\vspace{-4pt}


\vspace{-7pt}
\noindent where

\vspace{-18pt}


\vspace{-4pt}
\noindent for channel  in class . The number of class  channels is ,  with  being the \emph{proportion} of channels in class  with respective to the total number  of channels.

We study the scenario where all the Markovian channels are positively correlated, i.e.,  for . This assumption, which is commonly made in this domain (e.g., \cite{Neely_capacity, Neely_utility, sugu_aslm}), means that the channel evolution has a positive auto-correlation. Hence, roughly speaking, the channel has a stronger potential to stay in its previous state than jumping to another, which is typical especially in slow fading environment. For ease of exposition, we shall exclude the trivial case when  or , . \vspace{-5pt}

\subsection{Scheduling Model -- Belief Value Evolution}

We assume that the base station can simultaneously transmit to at most  users in a time slot without interference, where  stands for the maximum \emph{fraction} of users that can be activated. For example, in a multi-channel communication model,  would correspond to the fraction of all users that can be simultaneously serviced in unit time. However, the scheduler does not know the exact channel state in the current slot when the scheduling decision is made. Instead, the scheduler maintains a \emph{belief value}  for each channel , which is defined as the probability of channel  being in the ON state at the beginning of slot . The accurate channel state is revealed via ACK/NACK feedback from the scheduled users, only at the end of each time slot after the data is transmitted. This accurate channel state feedback is in turn used by the scheduler to update the belief values.

For user  in class , , let  indicate whether the user is selected for transmission in slot . Then, from the definition the belief values,  evolves as follows,



In our setup, belief values are known to be sufficient statistics to represent the past scheduling decisions and feedback (e.g., \cite{Javidi,Sondik_thesis}). In the meanwhile, in our ON/OFF channel model,  also equals to the expected throughput contributed by channel  if it is scheduled in time slot .

For a user in class , , we use  to denote its
belief value when the most recent observed channel was
, and is  slots in the past. From the belief
update rule (\ref{eq:evolve}),  can be calculated as a
function of  as,


Fig.~\ref{fig:Qupdate} illustrates the belief value update when a channel stays idle
(i.e., ). It is clear
that if the scheduler is never updated of the state of channel 
(in class ), the belief value will converge to its stationary
probability of being ON, denoted by the stationary belief value
.

The vector  denotes the belief values of all channels at the beginning of slot . We use  to represent the set of the belief values for class  channels, where . We assume that the system starts to operate from slot . At the beginning of slot , for each channel the scheduler has either observed its channel state before, or has never been updated of its channel state, i.e., with belief value . It is then clear that, based on the belief update rule (\ref{eq:evolve}),  for all , i.e., each belief value  evolves over countably many states.


In the rest of the paper, we shall use `belief value' and `belief state' interchangeably.

\subsection{Downlink Scheduling Problem -- POMDP Formulation}
We consider the broad class  of (possibly non-stationary) scheduling policies that makes a scheduling decision based on the history of observed channel states and scheduling actions.
The downlink scheduling problem is then to identify a policy in  that maximizes the infinite horizon, \emph{time average expected throughput}, subject to the constraint on the number of users selected at each time slot. Given the initial state , the problem is formulated as,

where the belief value  evolves according to rule (\ref{eq:evolve}) based on the scheduling decision  under policy . Such an objective is standard in literature for Markov Decision Processes under the long term average reward criteria (e.g., \cite{Eitan}). Noting that since the scheduling decisions are made based on incomplete knowledge of channel states, this problem is a Partially Observable Markov Decision Process \cite{Sondik_thesis}.

\begin{figure}
\centering
\includegraphics[width=3.3in]{belief_evol.eps}
\caption{Belief values update when staying idle, , , .}
\label{fig:Qupdate}
\end{figure}

This problem is in fact an example of Restless Multiarmed Bandit
Problem (RMBP) \cite{Whittle}. For a general RMBP, finding an
optimal solution is PSPACE-hard \cite{Tsitsiklis}. However, for the
downlink scheduling problem at hand, a low-complexity Whittle's Index Policy was proposed in \cite{Zhao_index}\cite{Infocom11} based on the RMBP
theory that inherently exploits the channel memory when making scheduling
decisions. For detailed descriptions of general RMBP
and Whittle's Index Policy for downlink scheduling, please refer to
\cite{Whittle}-\cite{Infocom11}.

For the downlink scheduling problem, we note that there is only limited analytical characterization of Whittle's Index Policy, which is restricted in perfectly symmetric scenarios where Whittle's Index Policy takes a special round-robin form \cite{Zhao_index}. In asymmetric cases, however, the scheduling decision no longer takes the form of round-robin, bringing sophisticated complications in belief value evolutions that are tightly coupled among channels, which significantly complicates the analysis. The main focus of this paper is to analytically characterize the performance of Whittle's Index Policy in the asymmetric case with two classes of channels.

\section{Upper Bound on Achievable Throughput}
\label{sec:bound}

We begin our analysis by characterizing an upper bound to the throughput performance of all feasible downlink scheduling policies that satisfies the constraint (\ref{eq:strgt}). The upper bound is obtained from a fictitious policy which is optimal for the downlink scheduling problem under a \emph{relaxed constraint}.

Note here that such relaxation is also a crucial step in the study
of the general RMBP problem. Yet, our analysis, being specific to
the downlink scheduling problem, has its novelties, as we shall
remark on later.

\subsection{Average-Constrained Relaxed Scheduling Problem}

We consider an associated relaxed problem of (2)-(3) that only requires an \emph{average number} of users to be activated in the long run, defined as follows




Note that, contrary to the stringent constraint (\ref{eq:strgt}), the relaxed constraint (\ref{eq:relaxed}) allows the activation of more than  fraction of users in each time slot, provided the long term average fraction does not exceed . Hence the optimal policy under this relaxed constraint, which we shall identify next, provides a throughput upper bound to any policy that satisfies the stringent constraint.

\subsection{Optimal Policy for the Relaxed Problem}

We remark that the relaxed problem is also an important component of Whittle's analysis of general RMBPs \cite{Whittle}, in which an optimal policy for the relaxed problem is developed based on the \emph{Whittle's index values}. Following the approach of classic RMBP framework \cite{Whittle}, in our downlink scenario, we identify an optimal policy for the relaxed problem based on Whittle's indices.

Specifically, for channels in class , the Whittle's index value
 is assigned to each belief state .
These index values intuitively capture the exploitation and
exploration value to be gained from scheduling the associated
channel when its belief value is .  This characteristic of
 is also illustrated in Section~\ref{sec:num:trade-off} via
numerical investigations. The index value function is expressed in closed form as 


Note that the above expression is a modified version of the expression in \cite{Zhao_index}. Details of the derivation can be found in \cite{Wenzhuo_infocom12}.

The following two characteristics
they possess are primarily significant for our analysis:
\begin{itemize}
\item  monotonically increases with .
\item  for all .
\end{itemize}

The next lemma identifies an index-based policy with \emph{appropriate randomization} that is optimal for the relaxed constraint problem. This policy schedules each user based on its own belief value, independently from other users. The proof of the Lemma can be found in \cite{Zhao_index}.

\begin{lemma}
\label{lemma:thres_relax}
For the problem under relaxed constraint, there exists an optimal stationary policy , parameterized by the threshold  and a randomization parameter , such that
\vspace{3pt}

\noindent(i) Channel  in class  is scheduled if
, and stays idle
if \hspace{3pt}.
If , it is
scheduled with probability .

\vspace{3pt}

\noindent(ii) The parameters  and  are such that,
under policy , the relaxed constraint (\ref{eq:relaxed}) is
strictly satisfied with equality.

\end{lemma}

From now on, we shall denote  as the `\emph{Optimal Relaxed
Policy}'. {For technical purposes, we henceforth assume  is such that }. Since each  value maps to a unique  pair \cite{Wenzhuo_wiopt}, only countably many  values correspond to , i.e., achieved by deterministic policies. Therefore, the set of  for which  has Lebesgue measure one.

\subsection{Steady State Distribution of Belief Values}

We next present the transition structure of the belief values under
Optimal Relaxed Policy, captured in the following lemma. The
structure will be critical in the development of our subsequent
main results.

\vspace{2pt}

\begin{lemma}
\label{lemma:pos_rec} For each channel in class , under the
Optimal Relaxed Policy, the structure of belief value evolution
depends on the threshold  of policy. \vspace{2pt}

\noindent(i) If , then the belief value evolution of each class  channels is positive recurrent with a finite recurrent class.
\vspace{2pt}

\noindent(ii) If , the belief value evolution is transient. With probability , ultimately no channel in class  will transmit.
\end{lemma}

\noindent \textbf{Proof:} The proof of this lemma follows from the
monotonic structure of belief evolution, as shown in
Fig.~\ref{fig:Qupdate}. Details are included in Appendix
\ref{appen:recur}.  \vspace{3pt}

Thus, if , the above analysis reveals that ultimately no user
transits, corresponding to the trivial case of . Also, if  is between
 and , the class with the smaller
 will eventually transit into a passive mode, hence
reducing the system to a well-understood scenario with a single class of channels \cite{Liu}\cite{Javidi}. Thus, here we focus on the
heterogeneous case of , where the steady-state belief value distribution
exists for both classes under the Optimal Relaxed Policy.

\subsection{Upper bound on achievable throughput}
The throughput performance of Optimal Relaxed Policy provides an throughput
upper bound for all policies under the stringent constraint. The
value of such an upper bound clearly depends on the number of users
in each class , , as well as the fraction
 of users allowed for activation. Denoting , we represent the time average expected throughput of the Optimal Relaxed Policy as
. The following lemma states
that, as long as  and  are given, the \emph{per-user} throughput (i.e., ) is independent of .

\begin{lemma}
\label{lemma:parameter}
Given  and ,  is independent of , denoted henceforth as .
\end{lemma}

\noindent \textbf{Proof:} The proof follows from showing that, when the number of users  grows, as long as the proportion of each class of channels stays the same and the fraction  of users activated does not change, the form of Optimal Relaxed Policy does not change. Since each user is scheduled independently, the throughput  is proportional to , establishing the lemma. Details are provided in Appendix~\ref{appen:peruser}. 

\vspace{3pt}

We hence refer to the  pair as `\emph{system parameters}'. Therefore  provides a throughput upper bound to any policy in the same system under the stringent constraint (\ref{eq:strgt}). Equivalently,  provides a per-user throughput performance upper bound to all policies that satisfies the stringent constraint.

We next describe Whittle's Index Policy for the strictly-constrained problem (\ref{eq:strgt_obj})-(\ref{eq:strgt}), and later study the closeness of its performance to the upper bound established here.



\section{Whittle's Index Policy Description}
\label{sec:num:index}

In this section we formally introduce Whittle's Index Policy for
solving the stringently-constrained downlink scheduling problem
(\ref{eq:strgt_obj})-(\ref{eq:strgt}).



\subsection{Whittle's Index Policy}
The Optimal Relaxed Policy, along with the Whittle's index values,
gives consistent ordering of belief values with respective to the
indices. For instance, under the Optimal Relaxed Policy, if it is
optimal to schedule one channel, it is then optimal to transmit to
other channels with higher index values. So the Whittle's index
value gives an intuitive order of how attractive the channel is for
scheduling. This intuition leads to Whittle's Index Policy
\cite{Zhao_index} under the stringent constraint on the maximum number of
channels that can be scheduled.

\vspace{3pt}
\noindent\textbf{Whittle's Index Policy:} \emph{At the beginning of each time slot, the channel  in class  is scheduled if its Whittle's index value  is within the top  index values of all channels in that slot, with arbitrary tie-breaking while assuring a total  channels being scheduled.}
\vspace{2pt}

Whittle's Index Policy is attractive because it has very low
complexity, and it was observed via numerical investigations to
yield significant throughput performance gains over the scheduling
strategies that does not utilize channel memory \cite{Infocom11}.
The main focus of our work is to analytically understand the approximate or asymptotic optimality of Whittle's Index Policy in asymmetric scenarios.

\subsection{Whittle's Index Policy over Truncated State Space}

Recall from Section~\ref{sec:model} that the belief values evolve over a countable
state space, also note that if a channel is not scheduled for a long time, its
belief value will get arbitrarily close to its stationary belief
value. This motivates us to consider a truncated version of the
belief value evolution whereby the belief value is set to its steady
state if the corresponding channel is not scheduled for a large
number, say , slots. This mild assumption facilitates more
tractable performance analysis of the policy.  Thus,
if a class  user is not scheduled for  time slots, its channel state
history is entirely forgotten and its belief value will transit to
the stationary belief value , where the truncation  is assumed
to be very large.

Whittle's Index Policy is then implemented over the truncated
belief state, which differs from the non-truncated case merely in
the truncated belief value evolution. We believe that, the truncated
scenario can provide arbitrarily close approximation to the original
system when  is large. More importantly, as we shall see in
the following two sections, Whittle's Index Policy, implemented
over the truncated belief state space, achieve asymptotically
optimal performance as long as the truncation is sufficiently large.

\section{Local Optimality of Whittle's Index policy}
\label{sec:local}

In this section, we study the optimality properties of Whittle's
Index Policy for downlink scheduling, over a large truncated belief
space. This result forms the basis for the subsequent global
optimality result in Section~\ref{sec:global}. We start by introducing a state
space over which the local optimality will be established.

\subsection{System State Vector}
\label{sec:local:Z}

We define the \emph{system state}  as a vector
that represents the proportion of channels in each belief value, over
the truncated space when the total number of users is , i.e.,
,
with

where  and  respectively denote the
\emph{proportion} of channels in the corresponding belief state
 and , with respect to the total number of users
. Hence, each element of  is a multiple of 
so that  takes values in a lattice with mesh size
. Noting that the total number of users in each class does not
change over time, for any  the system state  where


The system state vector  does not distinguish
users with the same belief state, thus its dimension will not scale
with . Therefore, compared with , it provides a
more convenient representation of the system belief state.
Furthermore,  fully determines the instantaneous
throughput gain in slot  under both Whittle's Index Policy
and the Optimal Relaxed Policy (introduced in
Lemma~\ref{lemma:thres_relax}), because the instantaneous
throughput gains under both policies are only determined by the
distribution of the channels with different belief values, not their
identities.



From Lemma~\ref{lemma:pos_rec} and the subsequent remarks,
under the operation of the Optimal Relaxed Policy, the belief state
evolution of each channel is positive recurrent with a steady-state
distribution. The following lemma also establishes the independence
of this steady-state distribution from , and defines a useful
parameter for future use.







\begin{lemma}\label{lemma:zeta_invar}
Given the system parameters , the system state
vector  under the Optimal Relaxed Policy converges
in distribution to a random vector, denoted as  The mean of  is
independent of  and is denoted as

\vspace{-18pt}

\end{lemma}

\noindent{\textbf{Proof:}} This lemma follows from a similar
principle to the one we established in Lemma~\ref{lemma:parameter}.
For details, please refer Appendix~\ref{appen:zeta_invar}.
\hspace{2in} 

It is easy to see that  and the form of 
fully determines the time average throughput of the Optimal Relaxed
Policy. Therefore, the vector  provides an important benchmark for our asymptotic
analysis. If, in the long run under Whittle's Index Policy, the system state  stays close to
, it indicates that Whittle's Index Policy will have throughput performance close to
that of the Optimal Relaxed Policy -- the throughput upper bound. To
capture the closeness, we define the  neighborhood of
 as

for , where  stands for Euclidean distance. We
are now ready to state and prove our first main result regarding a
form of local optimality of Whittle's Index Policy.



\subsection{Local Optimality of Whittle's Index Policy}
Under the system parameters , we let
 represent the time average
throughput obtained over the time duration 
under Whittle's Index Policy, conditioned on the initial system
state , i.e.,

where  denotes the scheduling decision vector made
by Whittle's Index Policy at time 

Recall from Lemma~\ref{lemma:parameter} that 
denotes the per-user throughput under the Optimal Relaxed Policy,
which serves as an upper bound on Whittle's Index Policy
performance. The next proposition characterizes the local
convergence property of Whittle's Index Policy performance to .

\begin{proposition}
\label{prop:local_conv} Under the system parameters , there exists a  neighborhood
 of
 such that, if the initial
system state  is within  , then

where  is any increasing sequence of positive integers
with , , for  and
all .
\end{proposition}



\noindent \textbf{Proof Outline:} Here, we give a high level
description of the proof for an intuitive understanding, and refer the reader to
Appendix~\ref{appen:local} for the rigorous
derivation. \vspace{1pt}

 We start by defining a fluid approximation, whereby the
discrete-time evolution of  under Whittle's Index
Policy is modeled as a deterministic vector  that evolves {in discrete time} over  and is independent of  Under this fluid approximation, the users are no longer
unsplittable entities so that the state space of  is no
longer restricted to a lattice as it was for . Also,
the fluid approximation  evolves in a deterministic
manner, in contrast to the stochastic transition of .
The evolution of  is defined by a {difference} equation
as a function of the \emph{expected} state change of  under Whittle's Index Policy as follows

where  is any integer for which  is a feasible state.


 We then establish local convergence of the fluid
approximation model when  is within a small enough
 neighborhood  of . We show the
convergence by first noting that the differential equation
(\ref{eq:fluid_sketch}) is linear within a wider convex region than
. Within
this region, we obtain a closed form expression of the right hand
side of (\ref{eq:fluid_sketch}), which enables us to investigate the
eigenvalue structure of the linear differential equation. We show
{that each eigenvalue  satisfies } and apply standard linear
system theory to establish the local convergence.

 We then connect the fluid approximation model 
to the discrete-time stochastic system state  by using a discrete-time extension of
Kurtz's Theorem, which can be interpreted as an
extension of the strong law of large numbers to random processes
(see \cite{Weiss_LD}). Essentially, it states that, over any finite
time duration , the actual
system evolution  can be made arbitrarily close
to the above fluid approximation  by increasing the
number of channels  sufficiently, {with exponential convergence rate}.

 The previous convergence result, together with the local convergence
result of the fluid evolution  to , enables us to establish the local
convergence of the system state  to  as the number of users  grows,
provided that the initial state .
Hence the system state under Whittle's Index Policy will stay close
(in a probabilistic sense) to the expectation  of the system state under the Optimal
Relaxed Policy, which, in turn, indicates that the throughput
performance of Whittle's Index Policy will approach the throughput
upper bound , as expressed in the
proposition.

We again emphasize that the technical details of the outlined steps
are fairly intricate and are moved to Appendix~\ref{appen:local}.  \vspace{5pt}



Proposition~\ref{prop:local_conv} illustrates an interesting local
optimality property of Whittle's Index Policy as the number of users 
and the time horizon  increases while the system parameters
 stay the same. It indicates that, under
Whittle's Index Policy, as long as the initial state 
is close enough to , the average
per-user throughput over any finite time duration will get
arbitrarily close to the Optimal Relaxed Policy performance as the
number of users scales. \vspace{3pt}



\noindent \textbf{Remark: } We note that the sequence  is
used to guarantee that the number of channels in each class, as well
as the number of scheduled users, take integer values. In fact, our
result can be generalized to all  by slightly perturbing  and  as a function of  but assuring their limits
are well-defined.

\section{Global Optimality of Whittle's Index Policy}
\label{sec:global}
The above local optimality result heavily relies on the initial
state  being close to , which is difficult to guarantee. In this section, we study
the global optimality of the infinite horizon throughput performance
of Whittle's Index Policy starting from any initial state. We begin
our analysis by presenting the recurrence structure of the system state.

\begin{lemma}
\label{lemma:recur}
Under system parameters , for any , if the number of users  is large enough, 

\noindent(i) The system state  evolves as an aperiodic
Markov chain, in a state space that contains only one recurrent
class.

\noindent(ii) There exists at least one recurrent state within the
 neighborhood  of .
\end{lemma}

\noindent \textbf{Proof:}  We prove this lemma by constructing probability paths from any
state to the neighborhood . Details of the proof are included in Appendix~\ref{appen:recur}. 
\vspace{4pt}



This lemma states that  will ultimately enter any
small neighborhood of  when
 is large enough. Together with
Proposition~\ref{prop:local_conv}, this result shows promise for
establishing the global asymptotic optimality of Whittle's Index
Policy. This is plausible because once  enters
, the
performance of Whittle's Index Policy \emph{afterwards} can
get very close to its upper bound as  scales, as established in
Proposition~\ref{prop:local_conv}. However, since we consider the infinite horizon time average throughput, this argument would break down if the time it takes for  to enter  also scales
up with . This observation
motivates us to introduce a useful assumption, which will later be justified
(in Section~\ref{sec:num:just}) via numerical studies.

\vspace{5pt}



\noindent \textbf{Assumption }: For each , let
 represent the first time of reaching
 starting
from , i.e.,

Then we assume that, the expected time of reaching
 is
bounded by a constant , i.e.,

for all  and large enough .

\vspace{7pt}

Since for each ,  under Whittle's Index Policy is
recurrent and aperiodic with a finite state space, there exists a
steady-state distribution associated with . As before,
we use  to denote the associated limiting random
vector. The next lemma establishes that, under Assumption , the
distribution of  approaches a point-mass at
 as  scales. Here, again,
the sequence  is defined in the same way as in
Proposition \ref{prop:local_conv}.

\vspace{-2pt}
\begin{lemma}
\label{lemma:steady_dist} Under Assumption  and system
parameters , for any , the steady
state probability of  under Whittle's Index Policy
satisfies
\vspace{-3pt}

\end{lemma}

\vspace{-2pt}
\noindent \textbf{Proof:} The proof utilizes Theorem  from
\cite{Weiss_LD}, which builds on the following arguments.

Note that  can be selected to be small enough for the
following argument. As depicted in Fig.~\ref{fig:invar_meas}, we let
 be a random variable denoting, in steady state, the
time duration between \emph{consecutive} hitting times into the
neighborhood  from outside of the neighborhood. Let 
denote the time duration from the time  enters the
neighborhood  from outside until the time it leaves. Hence, the
expected proportion of time that  stays outside this
neighborhood is
].

We know that the numerator  is uniformly
bounded for all  due to Assumption . However, as 
increases, it is more likely for  to stay within the
neighborhood for a long time before exiting it (based on the convergence of fluid approximation model and Kurtz's Theorem in the proof of Proposition~\ref{prop:local_conv}). Thus,
 and hence the denominator , grow to infinity
as  scales. Therefore, the expected proportion of time spent
outside 
vanishes as  scales up, which leads to the statement of the
lemma. Details of the proof can be found in Appendix~\ref{appen:Invar_meas}. 

\begin{figure}
\centering
\includegraphics[width=3.3in]{invar_meas.eps}
\vspace{-8pt}
\caption{Transition behavior of  in steady state.}
\label{fig:invar_meas}
\vspace{-12pt}
\end{figure}

\vspace{3pt} Under Whittle's Index Policy with system parameters
, we let  be
the achieved infinite horizon, time average throughput, conditioned
on the initial system state ,
i.e.,


\vspace{-5pt}From Lemma~\ref{lemma:steady_dist} we know that, in steady-state,
the system state  is increasingly
concentrated around  as 
increases, regardless of the initial state  We build on
this to establish the global asymptotical optimality of Whittle's
Index Policy.



\begin{proposition}
\label{prop:asymp} Under Assumption , for any initial
system state , we have

\vspace{-12pt}
Since  is an upper bound on the maximum
achievable per-user throughput by any policy, this implies that
Whittle's Index Policy is optimal in the many user regime.
\end{proposition}

\noindent \textbf{Proof:} We prove this result by decomposing  as a summation of the expected throughput conditioned on whether the system state is within or outside an arbitrarily small  neighborhood of . Since the latter has diminishing probability according to Lemma~\ref{lemma:steady_dist}, the expected throughput of Whittle's Index Policy can get arbitrarily close to that of Optimal Relaxed Policy. Details of the proof are provided in Appendix~\ref{appen:global}. \vspace{3pt}

\begin{figure}
\centering
\includegraphics[width=3.8in]{recur_r.eps}
\includegraphics[width=3.8in]{recur_bs.eps}
\caption{Average time of hitting . (a) ; (b) . }
\vspace{-8pt}
\label{fig:recur_time}
\end{figure}

\noindent \textbf{Remarks: }



1) We would like to emphasize that the global optimality result is
not a straight-forward extension of the local convergence result by
contrasting Proposition~\ref{prop:local_conv} and
Proposition~\ref{prop:asymp}. Note that in
Proposition~\ref{prop:local_conv}, the time limit is outside the
limit of the number of users , where each convergence (with )
is with respective to a \emph{fixed time duration}. However, the
order of limit is switched in the global optimality result of
Proposition~\ref{prop:asymp}, as it states the convergence with 
\emph{the infinite horizon} average throughput, which is much
stronger and hence is much more challenging to prove.

2) We would like to contrast Assumption  with Weber's
condition \cite{Weber}. For general RMBP problem, Weber's condition
leads to the same global asymptotic optimality result. While
confirming Weber's condition may be possible in very low-dimensional
problems, in our downlink scheduling problem, this requires one to
rule out the existence of both closed orbits and chaotic behavior of
a high-dimensional non-linear differential equation, which is
extremely difficult to check - even numerically. Assumption ,
on the other hand, takes a much simpler form, as it is defined over
the actual stochastic system and is amenable to easy numerical
verification, as is performed in Section~\ref{sec:num:just}.



\begin{table*}
\vspace{3pt}
\begin{center}
\renewcommand{\tabcolsep}{.1cm}
\renewcommand{\arraystretch}{1}
\begin{tabular}{|c|c|c|c|c||c|c|c|c|c|}
\hline
\multicolumn{5}{|c||}{} & \multicolumn{5}{|c|}{}\\
\hline
 &  &  &  &  &  &  &  &  & \\
\hline
0.4360 & (0.2242,0.1379) & (0.6742,0.1376) & [0.6680,0.3320] & 24.8 & 0.1202 & (0.6598,0.0091) & (0.5881,0.1337)  &  [0.3534,0.6466]  & 50\\
\hline
0.0529 & (0.7209,0.2958) & (0.2393,0.0947) & [0.8772,0.1228] & 52.4 & 0.3857 & (0.5024,0.1382) & (0.1818,0.1442)  &  [0.8627,0.1373]  & 51\\
\hline
0.1368 & (0.6402,0.0611) & (0.9357,0.6544) & [0.9446,0.0554] & 20.8 & 0.8013 & (0.8335,0.2617) & (0.8046,0.1486)  &  [0.5621,0.4379]  & 9.8\\
\hline
0.6664 & (0.6016,0.0809) & (0.9163,0.2221) & [0.2571,0.7429] & 19.8 & 0.1410 & (0.5727,0.1403) & (0.0743,0.0418)  &  [0.4514,0.5486]  & 50\\
\hline
0.4558 & (0.8767,0.6747) & (0.8080,0.6483) & [0.6475,0.3525] & 5 & 0.6782 & (0.8871,0.0472) & (0.5157,0.0643) &  [0.2971,0.7029]  & 67.2\\
\hline
0.4606 & (0.9192,0.7814) & (0.2898,0.1686) & [0.9971	0.0029] & 15.8 & 0.0418 & (0.8311,0.0482) & (0.1699,0.0728)  &  [0.8828,0.1172]  & 60.6\\
\hline
0.1367 & (0.6401,0.0611) & (0.9357,0.6543) & [0.9446,0.0554] & 20.8 & 0.5858 & (0.4808,0.1552) & (0.8344,0.5340)  &  [0.4662,0.5338]  & 13 \\
\hline
0.6664 & (0.6016,0.0809) & (0.9163,0.2220) & [0.2571,0.7429] & 19.8 & 0.5271 & (0.7086,0.2569) & (0.8684,0.6064)  &  [0.7992,0.2008]  & 7.6\\
\hline
0.6018 & (0.2008,0.1861) & (0.2826,0.1992) & [0.7762,0.2238] & 3 & 0.8393 & (0.5426,0.1789) & (0.7747,0.4538)  &  [0.2453,0.7547]  & 5\\
\hline
0.1781 & (0.4421,0.0513) & (0.9150,0.4430) & [0.3696,0.6304] & 29 & 0.7498 & (0.5219,0.3849) & (0.6668,0.2956)  &  [0.9673,0.0327]  & 5.8\\
\hline
\end{tabular}
\end{center}
\caption{Evaluation of average time of hitting  under a wide range of parameters.}
\label{tab:tab4}
\vspace{-10pt}
\end{table*}

\section{Numerical Results}

\subsection{Verification and Interpretation of Assumption }
\label{sec:num:just}

We start by numerically verifying Assumption . We consider
the asymmetric scenario with two classes of channels with system
parameters ,
, with , , , .

We next examine the change of the average hitting time , while maintaining  and .

We let  be initial values of  that are selected to be two extreme points in the state space to exhibit the uniformity of  to the initial state. Specifically, state  corresponds to the
case when all the users have just observed their channels to be in
OFF state, i.e., with belief value , . And 
corresponds to the case when all users have no initial observation
of their channels state history, i.e., with belief value ,
.

We examine the average value of hitting time  and  with a very small
neighborhood , when the number of users  grows from  to . As indicated in Fig.~\ref{fig:recur_time}, for both cases, the average time of hitting the  neighborhood first decreases with , and then \emph{converges} and stays almost the same as  scales up. This is especially
intriguing. The rationale behind this phenomenon is as follows. Under Whittle's Index Policy, a total number of  users are activated at each time slot. Therefore, for relatively small number of users, the amount of probabilistic belief state transitions, as well as the amount of system states in the neighborhood, increases with , leading to a higher chance of hitting the desired neighborhood
 and smaller value of hitting time. However, the belief update of each user
contributes to the  change of the system state , which
decreases with . Therefore, as  further increases, the \emph{total amount of transitions} of the system state  due to channel state feedback is roughly , which is invariant of . Table~\ref{tab:tab4} illustrates the average value of hitting time  and  under a variety of randomly generated system parameters when  convergence is reached as  scales. These result shows that the hitting time is bounded and hence of verifies Assumption .

\begin{figure}
\centering
\includegraphics[width=3.1in]{case2.eps}
\includegraphics[width=3.1in]{case2b.eps}
\vspace{-3pt} \caption{The evolution of belief value and Whittle's
index value. (a)
Belief value evolution (b) Whittle's index value evolution. }
\vspace{-15pt}
\label{fig:beliefindex}
\end{figure}







\subsection{`Exploitation versus Exploration' Trade-off}
\label{sec:num:trade-off}

\vspace{-3pt}In this section, we demonstrate how the Whittle's index value captures
the `exploitation versus exploration' trade-off for our \emph{asymmetric
downlink scheduling problem}.

Consider two classes of ON/OFF fading channels with belief value
evolutions plotted in Fig.~\ref{fig:beliefindex}(a). Note that
both classes have the same stationary distribution ,
 of being at ON state, but channels in class  has a higher
degree of time correlation, i.e., fades slower, than channels in
class  since  and . The corresponding Whittle
index values of the two classes of channels are depicted in
Fig.~\ref{fig:beliefindex}(b) as functions of the updated belief
value starting from different initial states.

To understand the nature of Whittle's index value, we first consider
the case when the channels in both classes are observed to be ON at
time  and stay passive since then. As indicated in
Fig.~\ref{fig:beliefindex}(a) the class  channel has a higher
belief value than the class  channel, hence scheduling the
class  channel gives a higher immediate throughput than
scheduling the class  channel. Moreover, once a class  channel
is scheduled, it is more likely to stay in ON state again,
bringing high future gains. Accordingly, the index values in
Fig.~\ref{fig:beliefindex}(b) when both state evolutions start from
ON states capture that it is more attractive to schedule the
class  channel because of the advantage in both exploitation and
exploration.

On the other hand, when the scheduler has observed channels in both
classes to be OFF at time , Fig.~\ref{fig:beliefindex}(a) shows
that the class  channel has a higher belief value than the class
 channel. However, although the Whittle's index value in
Fig.~\ref{fig:beliefindex}(b) of class  channel is initially
smaller than that of class  channel, after a certain amount of
delay (around  slots in the figure) this order is switched, which
is interpreted as follows: initially, since the class 
channel has smaller belief value than that of the class  channel,
it is more attractive to exploit the immediate gain brought by the
class  channel. However, as the passive time grows, as indicated
in Fig.~\ref{fig:beliefindex}(a), the difference between immediate
gain of both classes diminishes. Then, it becomes more attractive to
explore the class  channel because its longer memory can
bring higher future gains if it turns out to be in ON state.

This investigation reveals the intricate nature of Whittle's index
value in capturing the fundamental `exploration versus exploitation'
trade-off. In our scheduling problem with asymmetric channel
statistics, such a property of Whittle's Index Policy turns out to
be crucial in \emph{achieving asymptotically optimal performance}.

\begin{figure}
\centering
\includegraphics[width=3.7in]{Reward_b_new.eps}
\includegraphics[width=3.7in]{Reward_bs_new.eps}
\vspace{-3pt} \caption{Performance evaluation and comparison of per-user throughput of Whittle's Index Policy. (a) ; (b) .}
\vspace{-15pt}
\label{fig:realistic}
\end{figure}

\subsection{Performance Evaluation and Comparison}

Note that our results focus on asymptotic regime when the number of users scales up. We next numerically evaluate the performance of the Whittle's Index Policy under finite number of users. We next consider a system where , ,  and , and evaluate the value  when  increases as multiples of , i.e., . Fig.~\ref{fig:realistic}(a) and (b) respectively correspond to the aforementioned extreme points. As observed in Fig.~\ref{fig:realistic}, the per-user throughput value  of Whittle's Index Policy quickly converges to the upper bound value . This result indicates that, in realistic scenarios with finite , the global convergence result in Proposition~\ref{prop:asymp} holds under moderate number of users (under  as shown in Fig.~\ref{fig:realistic}). 

Fig.~\ref{fig:realistic} also plots the per-user throughput performance of the BALANCEDINDEX  policy, which is proposed in \cite{ApproxRMBP} and proved to achieve throughput half of the optimal throughput, i.e., 2-approximation performance. As observed in Fig.~\ref{fig:realistic}, the asymptotic per-user throughput performance of BALANCEDINDEX is strictly lower than the Whittle's Index Policy. This is because although BALANCEDINDEX policy guarantees 2-approximation to the optimal throughput performance, it does not provide strictly optimal per-user throughput performance in the asymptotic regime of large number of users, as compared with Whittle's Index Policy. Fig.~\ref{fig:realistic} also evaluates the performance of a slight modification Whittle's Index Policy, namely the THRESHOLD-WHITTLE policy, proposed in \cite{ApproxRMBP} by slightly adjusting the Whittles index value at belief values . It can be observed from the figure that the per-user throughput performance of THRESHOLD-WHITTLE policy is very close to that of the Whittle's Index Policy, indicating that the modification of the Whittle's indices in THRESHOLD-WHITTLE policy does not bring significantly change the throughput performance for the plotted example. It was proven in \cite{ApproxRMBP} that the THRESHOLD-WHITTLE policy achieves at least half of the optimal  throughput. However, analytically proving the asymptotic optimality of THRESHOLD-WHITTLE policy remains an open question.

\subsection{Evaluation of Fairness among Users}

In this section, we evaluate the fairness performance of Whittle's Index Policy. We exam the throughput difference between the two types of users, under different set of Markov transition statistics. To facilitate better evaluation, we define the throughput  to be the per-user throughput \emph{within each class  of users}, i.e.,

where  represents the set of users in class . We consider the scenario where  and  with . Therefore, the channels in class  have a much higher degree of correlation than the channels in class , i.e., it is more likely for the channels in class  to stay in its previous-slot state than change to a different state compared with channels in class . However, channels in both classes have the same steady state probability in state `1', i.e., . Fig.~\ref{fig:Fairness} plots the per-user throughput within each class under Whittle's Index Policy. It can be observed that users in class  achieves higher throughput than users in class . The higher throughput gain of class  is brought by the higher degree of temporal correlation and also the aforementioned `Exploitation versus Exploration' trade-off. Since the class- channels have higher degree of time-correlation, if a class- channel is previously observed in state , the scheduler tends to continue to serve it for longer time to obtain high immediate gains. It is also more attractive to explore a channel in class  because, as previously discussed, higher future gains can be obtained if it turns out to be in state `1'. Therefore, channels in class  have higher overall throughput than channels in class , resulting in the big gap in throughput between the two classes of users in Fig.~\ref{fig:Fairness}.

\begin{figure}
\centering
\includegraphics[width=3.2in]{Sum_throughput_1.eps}
\includegraphics[width=3.2in]{Fairness_1.eps}
\vspace{-3pt} \caption{Evaluation of  with . (a) Whittle's Index Policy; (b) Policy .}
\vspace{-15pt}
\label{fig:Fairness}
\end{figure}


To facilitate better performance in terms of fairness, we evaluate the performance of the following heuristic policy  based on the Whittle's index values. In policy , instead of directly using Whittle's index values, the algorithm schedules the  users with the largest

at slot , where  is user 's achieved throughput up to slot , i.e., . Hence a user's priority for scheduling is determined by its Whittle's index value relative to its own actual achieved throughput. Therefore policy  mimics the proportional fair scheduling algorithms (e.g.,  \cite{Tse}) commonly used in communication networks. Fig.~\ref{fig:Fairness}(b) evaluates the performance of policy . As we can see, under the algorithm , the throughput gap between the two classes of channels is closer than Whittle's index policy, indicating improved fairness performance. Finally, we believe that combining Whittle's index and the frame-based scheduling \cite{Neely_utility} can lead to  low-complexity algorithms that optimally meet the fairness constraints among different users.





\section{Conclusion}
In this paper, we studied the problem of downlink scheduling over ON/OFF Markovian fading channels in the presence of channel heterogeneity. We consider the scenario where instantaneous channel state information is not perfectly known at the scheduler, but is acquired via a practical ARQ-styled feedback after each scheduled transmission. We analytically characterized the performance of Whittle's Index Policy for downlink scheduling, and proved its local and global asymptotic optimality properties as the number of users scales. Specifically, provided that the initial system state is within a certain region, we established the local optimality of Whittle's Index Policy by investigating the evolution of the system belief state with a fluid approximation. We then established the global asymptotic optimality of Whittle's Index Policy under a recurrence condition, which is suitable for numerical verification. Our results establish that Whittle's Index Policy, which is attractive due to its low-complexity operation, also processes strong asymptotic optimality properties for scheduling over heterogeneous Markovian fading channels. Future research directions includes design of scheduling algorithms that not only maximizes the sum throughput, but also provides fairness among heterogeneous users using Whittle's index.
\vspace{6pt}
\begin{thebibliography}{1}

\vspace{3pt}
\bibitem{Knopp}
R. Knopp, P. A. Humblet, ``Information capacity and power control
in single cell multiuser communications,'' in \emph{IEEE ICC}, 1995.

\bibitem{JSAC_Liu}
X. Liu, E. K. P. Chong, N. B. Shroff, ``Opportunistic transmission scheduling with resource-sharing constraints in wireless networks,'' \emph{IEEE JSAC}, 2001.

\bibitem{Tse}
D. Tse, P. Viswanath, \emph{``Fundamentals of wireless
communication,''} Cambridge University Press, 2005.

\bibitem{tassiulas}
L. Tassiulas, ``Scheduling and performance limits of networks with constantly changing topology,'' \emph{IEEE Transactions on Information Theory,} vol. 43, no. 3, pp. 1067-1073, 1997.

\bibitem{Xiaojun06}
X. Lin, N. B. Shroff, ``The impact of imperfect scheduling on cross-Layer congestion control in wireless networks,'' \emph{IEEE/ACM Transactions on Networking}, vol. 14, no. 2, pp. 302-315, 2006

\bibitem{Atilla}
A. Eryilmaz, R. Srikant, ``Fair resource allocation in wireless networks using queue-length based scheduling and congestion control,'' \emph{IEEE/ACM Transactions on Networking,} vol. 15, no. 6, pp. 1333-1344, 2007.

\bibitem{clinical}
C. Safran, C. G. Chute, ``Exploration and exploitation of clinical databases'', \emph{International Journal of Bio-Medical Computing,} vol. 39, pp. 151--156, 1995.

\bibitem{reinforce}
L.P. Kaelbling, M.L. Littman, A.W. Moore, ``Reinforcement
learning: a survey,''  \emph{Journal of Artificial Intelligence Research,} vol. cs.AI/9605, pp. 237--285, 1996.

\bibitem{2stage}
M. J. Neely, S. T. Rager, and T. F. La Porta, ``Max weight learning algorithms for
scheduling in unknown environments,'' \emph{IEEE Transactions on Automatic Control,} vol. 57, no. 5, pp. 1179-1191, May 2012.

\bibitem{HARQ}
J. Huang, R. A. Berry, and M. L. Honig, ``Wireless scheduling with hybrid ARQ'', \emph{IEEE Transactins on Wireless Communications}, vol. 4, no. 6, 2005.

\bibitem{rohit}
R. Aggarwal, M. Assaad, C. E. Koksal, and P. Schniter,`` Joint scheduling and resource allocation in the ofdma downlink: utility maximization under imperfect channel-state information,'' \emph{IEEE Transactions on Signal Processing,} vol. 59, no. 11, pp. 5589-5604, 2011.

\bibitem{Junshan2}
C. Thejaswi, J. Zhang, S. Pun, V. H. Poor, ``Distributed
opportunistic scheduling with two-level channel probing,'' \emph{IEEE/ACM Transactions on Networking,} vol. 18, pp.1464--1477, 2009.

\bibitem{Allerton}
W. Ouyang, S. Murugesan, A. Eryilmaz, N. B. Shroff, ``Scheduling with Rate Adaptation under Incomplete Knowledge of Channel/Estimator Statistics,'' \emph{Allerton Conference,} 2010.

\bibitem{YingShakkottai}
L. Ying, S. Shakkottai, ``On throughput optimality with delayed network-state information,'' \emph{IEEE Transactions on Information Theory,} vol. 57, no. 8, pp. 5116-5132, 2011.

\bibitem{Liu}
Q. Zhao, B. Krishnamachari, K. Liu, ``On myopic sensing for multichannel opportunistic access: Structure, optimality, and performance,''
\emph{IEEE Transactions on Wireless Communications,} vol. 7, no. 12, pp. 5431-5440, 2008.

\bibitem{Javidi}
S.H. Ahmad, M. Liu, T. Javidi, Q. Zhao, B. Krishnamachari,
``Optimality of myopic sensing in multi-Channel opportunistic access,'' \emph{IEEE Transactions on Information Theory,} vol. 55, no. 9, pp. 4040-4050, 2009.

\bibitem{Neely_capacity}
C. Li, M. J. Neely, ``Exploiting channel memory for multi-user wireless scheduling without channel measurement: capacity regions and algorithms,'' \emph{Elsevier Performance Evaluation,} vol. 68, no. 8, pp. 631-657, 2011.

\bibitem{Neely_utility}
C. Li and M. J. Neely, ``Network utility maximization over partially observable Markovian channels,'' \emph{IEEE WiOpt,} May 2011.

\bibitem{Whittle}
P. Whittle, ``Restless Bandits: Activity allocation in a
changing world,`` \emph{Journal of Applied Probability,} 1988.

\bibitem{Zhao_index}
K. Liu, Q. Zhao, ``Indexability of restless bandit problems
and optimality of Whittle's index for dynamic multichannel access,''
\emph{IEEE Transactions on Information Theory,} vol. 56, no. 11, pp. 5547-5567, 2010.

\bibitem{Infocom11}
W. Ouyang, S. Murugesan, A. Eryilmaz, N. Shroff, ``Exploiting channel memory for joint estimation and scheduling in downlink networks,'' in \emph{IEEE INFOCOM},  2011.

\bibitem{Weber}
R. Weber and G. Weiss, ``On an Index Policy for Restless Bandits,''
 \emph{Journal of Applied Probability,} vol. 27, no. 3, 1990.

\bibitem{ApproxRMBP}
S. Guha, K. Munagala, and P. Shi, ``Approximation algorithms for restless bandit problems.'' \emph{Journal of the ACM,} vol. 58, no. 1, 2010.

\bibitem{sugu_aslm}
S. Murugesan, P. Schniter, N. B. Shroff, ``Opportunistic scheduling using ARQ feedback in Multi-Cell Downlink,'' in \emph{Asilomar }2010.

\bibitem{Sondik_thesis}
E. J. Sondik, \emph{``The optimal control of partially observable Markov Decision
Processes,''} Ph.D. thesis, Stanford University, 1971.

\bibitem{Eitan}
Eitan Altman, \emph{``Constrained Markov Decision Processes,''} Chapman \& Hall, 1999.

\bibitem{Tsitsiklis}
C. Papadimitriou, J.N. Tsitsiklis `` The complexity of
optimal queueing network control,'' \emph{Mathematics of Operation
Research,} 1999.

\bibitem{Wenzhuo_infocom12}
W. Ouyang, A. Erilmaz, N. B. Shroff, ``Asymptotically optimal downlink scheduling over markovian fading channels,'' \emph{IEEE INFOCOM 2012}, Orlando, Frorida.

\bibitem{Wenzhuo_wiopt}
W. Ouyang, A. Eryilmaz, N.B. Shroff, ``Low-complexity Optimal Scheduling Over Correlated Fading Channels with ARQ Feedback,''  \emph{IEEE WiOpt 2012}, Paderborn, Germany.

\bibitem{Weiss_LD}
A. Shwartz, A. Weiss, \emph{``Large deviation for performance analysis,''} Chapman \& Hall, 1994.

\bibitem{Dutta}
P. K. Dutta, ``What do discounted optima converge to? A theory of discount rate asymptotics in economic models,`` \emph{Journal of Economic Theory}, vol. 55, pp. 64-94, 1991.

\bibitem{Bert_opt}
D. P. Bertsekas, \emph{``Nonlinear programming, 2nd edition``}, Belmont: Athena Scientific.

\bibitem{Kurtz}
T. G. Kurtz, ``Strong approximation theorems for density dependent Markov chains'', \emph{Stochastic Processes and their Applications,} vol. 6, no. 3, pp. 223-240, 1978.

\bibitem{Roger}
R. A. Horn, \emph{`` Matrix analysis, ''} Cambridge University Press, 1999.

\bibitem{Khalil}
W. J. Rugh, \emph{``Linear system theory, ''} Prentice Hall, 1996


\end{thebibliography} 
\appendices

\section{Proof of Lemma~\ref{lemma:pos_rec}}
\label{appen:recur}

\begin{figure}
\centering
\includegraphics[width=3.3in]{stationary.eps}
\vspace{-4pt}
\caption{Belief value transition in steady state when }
\vspace{-3pt}
\label{fig:belief_pos}
\end{figure}

(i) First consider the scenario where  and suppose  for the belief state . If the belief value of a channel is above  at the beginning of a slot, the channel will be activated. According to the belief value evolution rule~(\ref{eq:evolve}), in the next slot its belief value will either be  or , depending on the underlying channel state revealed at the end of a slot. Clearly, the belief evolution in this case is positive recurrent within a finite state space, i.e., the belief state can only take the values . On the other hand, if the belief value is below , the channel remains idle and will activate once its belief value exceeds . Fig.~\ref{fig:belief_pos} illustrates the belief evolution in steady state under this scenario.
\vspace{4pt}

(ii) Consider the scenario where . In this case, a channel is activated if its index value is above . After transmission, if the channel is observed to be in OFF state, its belief value will transit to  and stays idle until its index value crosses . Since , it is clear from the belief value evolution (see Fig.~\ref{fig:Qupdate}) that, starting from , the belief value will always be smaller than . Hence the channel will stay idle at all times. On the other hand, if the channel is observed to be in ON state after transmission, the belief value will transit to  and the channel will keep on transmitting until the underlying channel turns out to be in OFF state. Since we assumed , the channel will ultimately be in OFF state and its belief value will transit to  and stays in idle mode ever since. Therefore eventually no channel in class  will be scheduled and the belief values will keep transit toward, but never reach, the steady state belief value .

\section{Proof of Lemma~\ref{lemma:parameter}}
\label{appen:peruser}

Consider two systems with different total number of users but identical  and . Suppose the first system has  total number of users while the second system has  number of users. For the first system with  total number of users, suppose the policy , specified in Lemma~\ref{lemma:thres_relax}, is optimal for the relaxed-constraint problem. For each channel  in class , we let  denote the expected fraction of time of activation, i.e.,



Then, according to Lemma~\ref{lemma:thres_relax}(ii), the expected number of activated users satisfies


Now apply the same policy  when the total number of users is . Since  schedules each channel independently,  and  does not change in this scenario. Therefore, the expected number of activated users is expressed as

hence the complementary slackness condition (i.e., Lemma~\ref{lemma:thres_relax}(ii)) for the relaxed-constraint problem is also satisfied under , when the total number of users is . Hence the policy  satisfies both Lemma~\ref{lemma:thres_relax}(i) and (ii) under the total number of users , and is an optimal policy for that scenario.

Therefore, fixing system parameters , for different number  of users, the policy  is always optimal. Since the policy  schedules each channel independently, we let  denote the expected reward contributed by each channel in class . Hence we have


Therefore the per-user throughput is

which is independent of . Hence the lemma is proven.

\section{Proof of Lemma~\ref{lemma:zeta_invar}}
\label{appen:zeta_invar}

Given system parameters , we know from the proof of Lemma~\ref{lemma:parameter} that the form of the Optimal Relaxed Policy, denoted by , does not change with the number  of users. Since  schedules each channel independently, we let vector  denote the steady state distribution of the belief value of a user in class  under , with . Therefore,


Since  is independent of ,  is independent of  for . Therefore   is independent of the user number , which proves the lemma.

\section{Proof of Proposition~\ref{prop:local_conv}}
\label{appen:local}

\subsection{Notations}

We shall denote the  element of  as ,  and let  denote the corresponding belief value. The index value corresponding to  is denoted as . In this proof, since we are fixing the system parameters , we shall drop the suffixes  and  to denote  as .

For ease of exposition, in this proof we assume . Hence, in the Optimal Relaxed Policy, channels in class  are activated when their belief values are above  and stay idle if their belief values are below , and activates with probability  at .  For channels in class , they are activated when their belief values no smaller than  and stay idle otherwise.

\subsection{Transition properties of the system state}


We first investigate the belief transition structure of the system state  under the Whittle's Index Policy. It is clear that  evolves as a Markov Chain. We define the \emph{expected drift}  associated with the transition of  as follows,



For a channel with belief value , we let  and  be the probability that its belief state changes to state  under the idle and transmission actions, respectively. For example, if  corresponds to belief value , then  if the channel stays idle, otherwise  and , which corresponds to the probability of observed channel being  or , respectively. Under the Whittle's Index Policy, we let  be the fraction of users in belief value  that are activated,

where .  We use  to denote the probability that the belief value of a channel transit from  to  under system state . Then

with



We shall let , and let  be a vector that has  at the  element,  at the  element, and  at all other elements. Hence if a user changes its belief state from  to , the corresponding change of the system state  is in the direction of  with scale . Therefore,  is a composition of expected changes in each direction . Suppose , since the expected amount of change of  in direction  is , the expected drift  can then be written as,

where the  element of matrix  is


Note that, although the system state  can only take values on a lattice that depends on N, the matrix function  is defined over more general space . Based on this, we proceed to define a fluid approximation model.

\subsection{Fluid Approximation Model}

We consider a fluid approximation model , which is defined by the following difference equation


Note that the right-hand-side is completely determined by equation (\ref{eq:ran_factor})-(\ref{eq:matrixQ}), as a function of  and is independent of . We denote  as the `fluid approximation model' because  is no longer restricted to take values on the lattice as with the case of the original system state , and  evolves in the direction of the \emph{expected change} of the system state \footnote{Note that by `fluid' we mean fluid in users/channels instead of fluid with respective to time.}. Recall that the set  is defined in equation (\ref{eq:beta}), we proceed with the following lemma.

\begin{lemma}
\label{lemma:WitinZ}
If , then  for all .
\end{lemma}

\noindent \textbf{Proof:}
Since from (\ref{eq:dir_comp}) we have


Note that the belief values of a channel can only evolve within the belief states of class of the channel, hence for class ,

where  is a vector with  in each element. Similar result holds for class . Since , we have


Also equation (\ref{eq:dir_comp})-(\ref{eq:fluid}) indicates that  for all  if . Therefore  for all , establishing the lemma. 
\vspace{6pt}

\begin{lemma}
\label{lemma:alpha_achieve}
Given , there exists a unique parameter pair  for the optimal policy .
\end{lemma}

\noindent \textbf{Proof:}
For a single channel  in class , consider the policy where the channel activates if its belief value , stays idle when , and activates with probability  when , for some belief value . From the belief value evolution we can calculate the expected time of activion, denoted by ,


It is clear from its expression that, given ,  is continuous with . Also we have . In addition, some simple algebra reveals that, given ,  strictly increases with . Therefore, since , given   monotonically decreases with .

Also, one can observe from the expression that, given ,  and . Hence by appropriately choosing  and ,  can achieve any value within .
\vspace{3pt}

Note that the index value  monotonically increases with , .  It follows from the above analysis that, as  increases, under policy , the fraction of activation time for each user strictly decreases from  to . Therefore, there exists an unique  pair, such that the policy  strictly satisfies activation constraint~(\ref{eq:relaxed}). 


\begin{lemma}
\label{lemma:fixpt}
The vector  is the unique fixed point of the fluid approximation model, i.e., for all ,
 if and only if .
\end{lemma}

\noindent \textbf{Proof:} The proof follows from a similar line of \cite{Weber}. Note that, under the Optimal Relaxed Policy,  and  fraction of channels are activated on average. Therefore, in the fluid approximation model, we have , i.e., .

Now suppose there exists another fixed point  such that  and . Then  corresponds to the stationary distribution of the system state under another policy  with threshold parameter  and randomization factor . Furthermore, under , the expected fraction of activated channels equals to . However, this contradicts with Lemma~\ref{lemma:alpha_achieve}, which states that  is the unique parameter pairs that strictly satisfies the average constraint of activation. Therefore, the fixed point  is unique.
 \hfill 

\subsection{Convergence of the Fluid Limit Model}

Define the region  as the set of  such that, under the Whittle's Index Policy defined in Section~\ref{sec:num:index}, the channel is activated if and only if its index value is no smaller than , which is the threshold for the Optimal Relaxed Policy defined in Lemma~\ref{lemma:thres_relax}. This means that, at system state , all channels with index value higher than  are scheduled, and the channels with index value smaller than  stay idle, while the channels at index value  are scheduled with certain randomization. Specifically, .

The following lemma characterizes the linearity property of the fluid approximation model in .
\vspace{3pt}

\begin{lemma}\label{lemma:piecewiseL}

\noindent(i) The vector  .

\noindent(ii) The fluid difference equation (\ref{eq:fluid}) is linear within the region , i.e., there exist matrix  and vector  such that

\end{lemma}

\noindent \textbf{Proof:}
(i) The vector   because, if , we have , where  as defined in~(\ref{eq:ran_factor}).
\vspace{3pt}

(ii) Recall that, at the beginning of the section, we have assumed  for the belief value  of class- channel. The difference equation (\ref{eq:fluid}) becomes,

where the second equality is from (\ref{eq:qz}).

Since the total fraction of users activated is , we have


Substituting the expression (\ref{eq:linear_alpha}) back in (\ref{eq:diff_linear}), and noting that  stays constant for  (since the threshold  for activation does not change for ), the linearity property holds. \hfill 

\vspace{10pt}



From Lemma~\ref{lemma:WitinZ} we know that  for all , i.e.,


Taking note of Lemma~\ref{lemma:WitinZ}, instead of using a  dimensional vector , it suffices to represent the system state by a  dimension vector , i.e.,

in which elements  and  are eliminated from . The transition of , when , is obtained by substituting the relationship (\ref{eq:beta2}) in the difference equation (\ref{eq:diff_linear}) and eliminate the elements  and , i.e.,

where the matrix  and vector  are obtained after the substitution.
The next key lemma captures the eigen structure of matrix .

\begin{lemma}
\label{lemma:EigVal}
Each eigen value  of  satisfies .
\end{lemma}

\begin{proof}
The proof is based on explicit study of matrix  and is given in Appendix~\ref{appen:EigVal}.
\end{proof}
\vspace{4pt}

This lemma leads to the local convergence of .

\begin{lemma}
\label{prof:fluid_conv}
There exists a positive constant  such that, if the initial state  of the fluid approximation model is within the  neighborhood  of , where , then
\vspace{4pt}

(i)  for all ;

(ii)  as .
\end{lemma}

\noindent \textbf{Proof:} Similar to  that corresponds to , we let vector  represent the stationary expectation of vector . Therefore, from Lemma~\ref{lemma:fixpt},


Substituting (\ref{eq:Ustation}) in equation~(\ref{eq:Q_tilde}), we have


Since we have assumed that , there exists a  neighborhood  with . Correspondingly, there is a neighborhood of  for which  evolution is linear and is described by~(\ref{eq:Ztilde}). From Lemma~\ref{prof:fluid_conv}, each eigen value  of  satisfies . According to the stability theory of linear systems \cite{Khalil},  converges to  if the initial state is close enough to .

Therefore, there exists a  neighborhood of  for which if the initial state ,  and  as . 

\subsection{Convergence of the system state}

The fluid approximation model provides a good estimate for the system state evolution when the number of users is large, captured in the following proposition, which can be viewed as a \emph{discrete-time version} of Kurtz theorem \cite{Kurtz} applied to our problem. The proof is given in Appendix~\ref{appen:discrete_kurtz}.

\begin{proposition}
\label{prop:discrete_kurtz}
There exists a neighborhood  of  such that if , then for any  and finite time horizon  there exists positive constants  and  such that

where , and  denotes the probability conditioned on the initial state . Furthermore,  and  are independent of  and .
\end{proposition}

According to Proposition~\ref{prop:discrete_kurtz}, the system state  behaves very close to the fluid approximation model  when the number of users  is large. Since we have shown the convergence of  to  within  in Lemma~\ref{prof:fluid_conv}, we are ready to establish the local convergence of the system state  to .

\begin{lemma}
\label{lemma:Kurtz_Index}
If , then for any  there exists a time  such that for each , there exist positive constants  and  with,

\end{lemma}

\noindent \textbf{Proof:}
We let . Noting that , from Lemma~\ref{prof:fluid_conv} we have, given , there exists  such that for all .


From Proposition~\ref{prop:discrete_kurtz} we know that there exist positive constants  and  such that,


Hence the lemma holds. 
\vspace{3pt}

The previous lemma allows us to establish the local convergence result. Let  be a mapping such that  represents the per-user average throughput under system state . Therefore,  is the immediate reward at time  and we also have .

For , we let  be such that for any , if , then


Note that the per-user instantaneous throughput  and  is defined in Lemma~\ref{lemma:Kurtz_Index}. Therefore,


Letting  be the event , we proceed to bound the second term in~(\ref{eq:Tbound}),

where the inequality if from the fact  and the relation~(\ref{eq:ell_ineq}).

According to Lemma~\ref{lemma:Kurtz_Index}, when , we have , therefore,


Since  can be arbitrarily small, we have


Hence, taking limit with  in both sides,


We have thus proved Proposition~\ref{prop:local_conv}.

\section{Proof of Lemma~\ref{lemma:recur}}
\label{appen:recur}

(i) Here we prove the Markov chain has one unique class by stating that, starting from any state, there exists a possibility to reach a particular state, and hence there is only one class of recurrent state. Without loss of generality, we assume .

Case (1). Suppose .  Starting from any initial state , the following transition can occur: whenever the channels in class  are activated, their states are observed to be in ON state, and whenever channels in class  are activated, they are revealed to be in OFF state. Then after a long enough time duration ,  fraction of channels, which are in class , will be in belief value , and other channels will have stationary belief value . Hence the system state will be  (defined in Section \ref{sec:local:Z}) with , , , and with  in all other positions.

Case (2). Suppose . Starting from any initial state , consider the following transition path. Within the first period of time slots, , whenever users in class  are activated, they turn out to be in state , and whenever users in class  are activated, they turn out to be in state . Then if  is long enough,  is such that , with zero in all other elements. In the second period, , whenever users in class  are activated, it will remain in state , and whenever users in class  are activated, it turns out to be in state  as well. Then after long enough time until ,    with , ,  and , with zero in all other elements.

Since the state space of the Markov Chain  is finite, there is at least one recurrent class. As we have seen in the above cases that, starting from all states,  can reach a particular state. Therefore there can only be one recurrent state. We shall henceforth denote this particular state as . It is also clear from the proof that the Markov chain is aperiodic because of the possible self-transition in state .

(ii) Similar to the proof of Proposition~\ref{prop:local_conv}, in this part, we drop the suffix  and  in the notation , and we assume, with no loss of generality, . Recall that from the expression~(\ref{eq:indices}) of Whittle's index value that  for , , . We first characterize the structure of . From the description in Lemma~\ref{lemma:pos_rec} we know that the non-zero elements of  are


We shall proceed to construct a path from the state  to an arbitrary neighborhood of  . For ease of exposition, in the proof we no longer consider the channels as unsplittable entities. Instead, the transition in the each stages (in the following proof) deals with belief state evolution of certain \emph{fraction} of users. As we shall see, under this assumption, we can construct a transition path of  under the Whittle's Index Policy, that transits from  to the \emph{exact} value . Although the identified path may not be feasible in reality for small value of , but as the number of users  increases, we can find a transition path, which operates each user as unsplittable entities, that is arbitrarily close to this identified path, and thus can ultimately get arbitrarily close to any neighborhood of .

Note that when , , where


In the following construction we shall assume that belief values are updated at the end of each slot when the actual channel states are revealed.
\vspace{6pt}

\noindent \textbf{Case (1). }Suppose  and . We shall denote . In this case,
the path is constructed with the stages below, starting from state .
\vspace{7pt}

\noindent\textbf{Stage 1.1.} In the first slot, among the  fraction activated channels,  amount remains in ON state, and  amount turn out in OFF state and are in class . Hence the end of this slot,  has the following non-zero elements


\noindent\textbf{Stage 1.2.} In each of the next  slots,  amount in the activated channels turn out in ON state, and  amount of them turn out to be in OFF state and are in class . So at the end of the last slot of this stage, the non-zero elements of the system state  satisfies


\noindent\textbf{Stage 2.} In the next few slots, all activated channels turn out to be in state . This stage goes on for  slots, until those channels that reach belief state  at the end of stage 1.1 are in belief state . Then by the end of the last slot of this stage, the non-zero elements of the system state  satisfies


\noindent\textbf{Stage 3.} In each of the following slots, among all channel activated, only those in belief state   turn out to be in OFF state. This stage goes on until those channels that transit to belief state  in stage 2 reaches belief state . Hence by the end of the final slot of this stage,


\noindent\textbf{Stage 4.} In each of the next  slots, among all users activated, those in belief state   turn out to be in OFF state, and  amount of activated channels in class  turn out in OFF state.
Then by the end of the final slot in this stage, the system state will be , i.e.,

\vspace{2pt}

\noindent \textbf{Case (2).} Suppose  and . We shall let  and . Starting from state , the path is constructed with the stages below, where stage 1.1 and 1.2 are the same with the previous case.
\hspace{10pt}

\noindent\textbf{Stage 1.1.} In the first slot, among the  fraction of activated channels, only  amount turn out in OFF state and they are in class . Therefore at the end of this slot,  with non-zero elements being


\noindent\textbf{Stage 1.2.} In each of the next  slots,  amount of activated channels are in state `1', and  amount are in OFF state and are in class . Hence at the end of the last slot of this stage, the non-zero elements of  satisfies


Letting  be the slot right after stage 1.2, the path proceeds as follows.
\vspace{7pt}

\noindent\textbf{Stage 2.}

\noindent(1) From slot  to slot , all activated channels in class  turn out to be in state . Hence at the end of slot , the channels that reach belief state  at the end of stage 1.2 are in belief state . Next, from slot  to slot , among the activated channels in class , only those in belief state   turn out to be in OFF state. Therefore, at the end of slot , the system state vector  that correspond to class- channels is


\noindent(2) In the meanwhile, from slot  to slot , among the activated channels in class ,  amount turn out to be in OFF state. Hence by the end of slot , the vector  that correspond to class- channels is



Therefore, at the end of slot ,  .
\vspace{10pt}


\section{Proof of Lemma~\ref{lemma:steady_dist}}
\label{appen:Invar_meas}

The proof is a discrete-time version of the proof of Theorem 6.89 from \cite{Weiss_LD}. We first present a lemma which is an extension of Lemma~\ref{lemma:Kurtz_Index}.
\begin{lemma}
\label{lemma:Kurtz_Index3}
There is a neighborhood  of , with , for which if , then for any  and time , there exist positive constants  and  with,

where  and  are independent of  and .
\end{lemma}

\begin{proof}
Note that we have established, in Lemma~\ref{prof:fluid_conv}, the local convergence of the fluid approximation model  in a neighborhood . We let  and let  (recall that  is defined in proposition~\ref{prop:discrete_kurtz} with ) be such that if , then


From Proposition~\ref{prop:discrete_kurtz}, there exist positive constants  and  with,

which proves the lemma.
\end{proof}\vspace{5pt}

We let  be such that if , then  for .


We let ,  be the time slots of \emph{consecutive} hitting times into the neighborhood  from \emph{outside} of the neighborhood when the total number of users is . Similarly, we let ,  denote the time slots of \emph{exiting} the
neighborhood  from inside of the neighborhood, when the total number of users is . Hence ,  evolves as a Markov chain. In steady state,




We let  denote the random variable . For any constant , we have


Note that


Since , from Lemma~\ref{lemma:Kurtz_Index3}, there exist positive constants  and  such that for any ,


Substitute (\ref{eq:meas2}) in (\ref{eq:meas1}) we have


Therefore,  as . From (\ref{eq:Kbound}), if  is large enough, we have


Since  can be arbitrarily large, , i.e., . Since from Assumption  we know , thus from equation~(\ref{eq:frac}),

which concludes the proof.

\section{Proof of Proposition~\ref{prop:asymp}}
\label{appen:global}

For any , let  be such that for , if , then


Consider fixed , for  denote event , then


Apply Lemma~\ref{lemma:steady_dist} to (\ref{eq:invar}) we have


Since  can be arbitrary,

which proves the proposition.

\section{Proof of Lemma~\ref{lemma:EigVal}}
\label{appen:EigVal}

After some algebra, the matrix  takes the form

where matrix  is expressed as

in which only the first, last and  row have non-zero elements, and for each row, non-zero terms start at the  element.

The matrices  and  are expressed in (\ref{eq:matrix1})(\ref{eq:matrix2}).

\begin{figure*}[hb]
\hrulefill


\vspace{-0.2in}
\end{figure*}

We need the following lemma to proceed.
\begin{lemma}
\label{lemma:bound}
For any ,

\end{lemma}

\noindent \textbf{Proof:}
The proof is moved to Appendix~\ref{appen:bound}. 
\vspace{3pt}

With this lemma, we proceed to characterize the eigen values of matrix , which are given by the solution to equation , where

where the second equality is from the property of block matrices. Therefore, we have


(1) We first study the characteristic polynomial . After some algebra we have

where


The matrix  hence has eigen value  of multiplicity . Let  be any other eigen value of , we hence have , i.e., , i.e.,


We proceed to show that . We prove this by contradiction, suppose  is such that . Then taking modulus of the left hand side of equation (\ref{eq:g01}) we have

where the first equality is from triangle inequality. Applying Lemma~\ref{lemma:bound} we have,

where the first inequality is from Lemma~\ref{lemma:bound}, and the second inequality is from the fact that , and the last inequality comes from triangle Inequality. Note that inequality (\ref{eq:contra1}) contradicts (\ref{eq:g01}). Therefore each eigen values of matrix  must satisfy .
\vspace{8pt}

(2) We then study the characteristic polynomial . We derive that

where


Consider


It is clear from equation (\ref{eq:Q2zero}) that matrix  has eigen value  with multiplicity . Let  be any eigen value of , we first show the following lemma.

\begin{lemma}
\label{lemma:real_bound}
Let  be any eigen value of , then .
\end{lemma}

\begin{proof}
1) Suppose  has an eigen value of , then, from (\ref{eq:Q2zero}), . However,

leading to a contradiction. Hence  does not have  eigen value.

2) Suppose the equation  has a root  with , or , or being purely imaginary with . Hence from equation (\ref{eq:lambdah}),


Consider the  modulus of the right hand side,


The above expression contradicts the previous equation (\ref{eq:x_xi2}).
\vspace{4pt}

From 1) and 2) we conclude that  can only have solution with real part within . Therefore all eigen values of matrix  have real part within .
\end{proof}
\vspace{5pt}

We proceed to show that each eigen value  of  needs to satisfy .
\vspace{3pt}

Suppose the equation  has a root  with , then from equation (\ref{eq:lambdah}),


We let  where . From the previous lemma we know that . Some derivation shows that

where the first inequality is from the assumption that  and the fact that . Therefore


The above expression contradicts equation (\ref{eq:x_xi3}). Hence it can not be . Therefore, each eigen value  of  satisfies , which concluds the proof.

\section{Proof of Proposition~\ref{prop:discrete_kurtz}}
\label{appen:discrete_kurtz}
Consider the random variable  given , i.e.,

where  is an indicator function representing whether the belief value of the  user transits from belief value  to belief value  at the next time slot. Note that, given , the scheduling action for users in belief state  is independent of  because the scheduling decision only depends on the belief state distribution . As  increases and  stays unchanged, more users are in belief state  and the contribution of each channel to the transition of  scales down with . From the law of large numbers, if the number of users scales up while  is kept the same, we have

almost surely, where  is defined in~(\ref{eq:qz}).

\begin{lemma}
\label{lemma:onestep}
There exists a neighborhood  of  such that, for any , if , there exists a function  for which  satisfies

where  is independent of  and .
\end{lemma}

\begin{proof}
Let  be a vector with  at the  position and  in all other elements. From (\ref{eq:drift_t}),


Note that

where the second equality holds because  for all , and  for all .

Therefore


Note that once a user is activated, its belief value will only transit to  or , therefore  only for . Also note that for those channels that stay idle, there is no randomness associated with its belief transition, i.e., for them . Therefore the randomness is only associated with the channels which are activated, i.e., those with index value no smaller than . Hence, (\ref{eq:Zupdate}) becomes

where the summation  is over all the channels in belief state  that are activated, and  is the set of belief values in which channels are scheduled within the class that corresponds to belief , i.e.,


We hence have

where the last inequality holds because  as well as the union bound. Specifically, the union bound holds since


Note that, for each , the random variables  are independent. From an extension of Chebychoff's inequality (See Excercise 1.8 in \cite{Weiss_LD}) we have that, for each , there exists a positive continuous function , which does not depend on  and , with


Let  be the fraction of channels activated, under the \emph{steady state} of Optimal Relaxed Policy, in the class corresponding to belief value , i.e.,


For any , there exists a neighborhood  such that for all ,

which essentially means, under system state , the fraction of activated channels in each class will stay close to the case when system state is actually . Let , then from (\ref{eq:deviation})-(\ref{eq:deviation2}),


It is clear from the proof that  does not depend on  or . The lemma thus holds.
\end{proof}
\vspace{5pt}

\begin{lemma}
\label{lemma:onestep2}
There exists a neighborhood  of  such that, for any , if , for any , there exist positive constant  and  with

where  and  is independent of  and .
\end{lemma}

\begin{proof}
We let . From Lemma~\ref{lemma:onestep}, there exists  such that if 


We let  be such that

for all  with 
Recall that  is defined in Lemma~\ref{prof:fluid_conv}. We let  be such that, if ,  for all . We proceed to prove this statement by induction.
\vspace{3pt}

For , if , from inequality~(\ref{eq:mu_nu_bound}), there exists ,


Letting  and , the statement holds when .

Suppose the statement is true at , then there exist  and , which correspond to , for which,


Now consider the second term in (\ref{eq:t2_1}),

where the first inequality follows from triangle inequality, and the second inequality is from relationship (\ref{eq:rho_bound}).

Because  and , we have . From inequality~(\ref{eq:mu_nu_bound}), we have


Substituting (\ref{eq:t2_2}) to (\ref{eq:t2_2ineq}), we have


Hence from Equation~(\ref{eq:t2_1}) and (\ref{eq:t2_3}), there exists constants  and  that do not depend on  and  with


By induction, the lemma holds.
\end{proof}
\vspace{3pt}

Note that from union bound,


Therefore, from Lemma~\ref{lemma:onestep2}, over finite time horizon , there exist positive constants  and , which do not depend on  and , such that

which concludes the proof of Proposition~\ref{prop:discrete_kurtz}.

\vspace{5pt}


\section{Proof of Lemma~\ref{lemma:bound}}
\label{appen:bound}

\begin{proof}
From the belief value evolution (\ref{eq:evolve}) we know


Therefore


Since  for , therefore from equation (\ref{eq:lemma_diff}),

which proves the lemma.
\end{proof}

\end{document}
