\documentclass[conference]{IEEEtran}
\usepackage{graphicx}
\usepackage{amsmath}
\usepackage{amssymb}
\usepackage{fancyhdr}
\newlength{\zeroheight}
\settoheight{\zeroheight}{0}
\pagestyle{fancy}
\thispagestyle{fancy}

\begin{document}




\title{\Large \bf Synchrophasor monitoring of single line outages via area angle and susceptance}
\author{
\IEEEauthorblockN{ Atena Darvishi\hspace{2cm}Ian Dobson}
\IEEEauthorblockA{Electrical and Computer Engineering Department\\
Iowa State University,
Ames IA USA \\
darvishi@iastate.edu, dobson@iastate.edu\
P_m^{\rm R}=P_m+P_m^{\rm into}.
\label{ImR}

P_m^ {\rm Rred}&=P^{\rm R}_m- B_{{m}n}B_{nn}^{-1}P_{n},
\label{pmred}\\
B_{mm}^ {\rm Rred}&=
B_{mm}^{\rm R}-B_{mn}B_{nn}^{-1}
B_{nm}.
\label{bmmred}

P_{\rm area}=\sigma_a  P_m^ {\rm Rred}.
\label{parea}

b_{\rm area}=\sigma_a  B_{mm}^ {\rm Rred} \sigma_a ^T.
\label{bab}

\theta_{\rm area}&=\frac{\sigma_a  B_{mm}^ {\rm Rred} \theta_m}{b_{\rm area}}\notag\\
&=w  \theta_m
=w[1]  \theta_{m}[1]+w[2]  \theta_{m}[2]+...+w[k]  \theta_{m}[k]
\label{thetanew}

\theta_{\rm area}^{(i)}=\frac{\sigma_a B_{mm}^ {\rm Rred} \theta_m^{(i)}}{b_{\rm area}}=w\theta_m^{(i)}.
\label{pmuareaanglei}

\theta_{\rm area}^{[i]}=\frac{\sigma_a B_{mm}^ {{\rm Rred}(i)} \theta_m^{(i)}}{b_{\rm area}^{(i)}}=\frac{\sigma_a B_{mm}^ {{\rm Rred}(i)}\theta_m^{(i)}}{\sigma_a B_{mm}^ {{\rm Rred}(i)}\sigma_a^T }
,\label{pmuareaangleiii}

\theta_{\rm area}^{(i)}\approx \theta_{\rm area}^{[i]}.

P_{\rm area}=b_{\rm area} \theta_{\rm area}.
\label{ohm}

 P^{(i)}_{\rm area}=b^{(i)}_{\rm area} \theta^{[i]}_{\rm area},
 \label{ohmi}
 
 P^{(i)}_{\rm area}=\sigma_a (P_m+P_m^{{\rm into}(i)}- B_{mn}^{(i)}{(B^{(i)}_{nn}})^{-1}P_{n}).
 \label{pareaiex}
 
 \theta^{(i)}_{\rm area}\approx
 \theta^{[i]}_{\rm area}
 =\frac{P^{(i)}_{\rm area}}{b^{(i)}_{\rm area}}\approx 
 \frac{P_{\rm area}}{b^{(i)}_{\rm area}}
 \label{pareaiex2}
 
 \theta_{\rm area} =\frac{P_{\rm area}}{b_{\rm area}}=\frac{P_a}{b_1+b_2+b_3}
  
 \theta_{\rm area}^{(1)}
 = \theta_{\rm area}^{[1]}
 =\theta_a^{(1)}-\theta_b^{(1)}
 =\frac{P_{\rm area}^{(1)}}{b_{\rm area}^{(1)}}
 =\frac{P_a}{b_{\rm area}^{(1)}}
 =\frac{P_a}{b_2+b_3}
 
\theta_{\rm area5bus}
=0.5 \, \theta_1+0.5 \, \theta_2-0.33\,  \theta_4-0.67 \,\theta_5
\label{theta1-simpleEx}

\theta_{\rm area9bus}
=0.44\,  \theta_1+0.56\,  \theta_2-  \theta_3
\label{theta2-simpleEx}

\theta_{\rm area1} =\, & 0.79\,  \theta_1 + 0.21\,  \theta_2 - 0.42 \,\theta_3 - 0.46 \,\theta_4\notag\\&
   - 0.02 \,\theta_5 - 0.05 \,\theta_6- 0.04 \,\theta_7 - 0.01 \,\theta_8 \notag
  
\theta_{\rm area2} &=\,  0.223\,  \theta_1 + 0.006\,  \theta_2\notag\\& + 0.008 \,\theta_3 + 0.01 \,\theta_4
   + 0.02 \,\theta_5 + 0.18 \,\theta_6+ 0.59 \,\theta_7\notag\\& - 0.39 \,\theta_8 
      - 0.41 \,\theta_9 - 0.004 \,\theta_{10}- 0.03 \,\theta_{11} - 0.18 \,\theta_{12}\notag
  
In practice the measurements with very small weights could be omitted.

  \begin{figure}[h]
  \begin{center}
  \includegraphics[width=\columnwidth]{pic4Change-WeccIDGeneral}
  \caption{Area 2 of WECC system with area lines in black, north border buses in red and south border buses in blue. Layout detail is not  geographic.}
  \label{pic4Change-WeccIDGeneral}
  \end{center}
  \end{figure} 
  
 For both areas, we are interested in monitoring the north-south area stress with the area angle
when there are single non-islanding line outages, and relating changes in the area angle to the area 
susceptance and the outage severity.
We take out each line in the system in turn and calculate  the monitored area angle  and the area susceptance  in each case.


\looseness=-1
To quantify the severity of each outage, we compute the maximum power that can enter the area after the outage of each line; for more detail see
\cite{DarvishiNAPS13}. 
The real power through the area is increased by increasing the power entering at each border bus proportionally.
(Generally power enters the area at the northern border buses and leaves the area from the south border buses.)
The maximum power entering the area through the north border occurs when the first line limit inside the area is encountered.
The idea is that the more severe line outages will more strictly limit the maximum power that 
can be transferred north to south through the area. This definition of outage severity can be related to the economic effect 
of limiting the north-south transfer.

  The area angle and the area susceptance for each line outage are shown in Figure \ref{pl1Change-WeccIDGeneral} for area 1 and in Figure \ref{pl4Change-WeccIDGeneral} for area 2. 
    The similar patterns of changes in the area angles and area susceptances confirm that the inverse relationship between area angle and area susceptance usually applies. 

  
  \begin{figure}[h]
  \begin{center}
  \includegraphics[width=\columnwidth]{pl1Change-WeccIDGeneral}
  \vspace{-25pt}
   \caption{Area angle  in degrees, area susceptance  , and maximum power into the area in pu for each line outage in WECC area 1. Base case (the point at extreme right) is , pu, max power  = 46.9.
   For clarity, graph shows  multiplied by 2, and max power  multiplied by 1.5.}
  \label{pl1Change-WeccIDGeneral}
  \end{center}
\begin{center}
  \includegraphics[width=\columnwidth]{pl4Change-WeccIDGeneral}
  \vspace{-25pt}
   \caption{Area angle  in degrees, area susceptance , and max power into the area in pu for each line outage in WECC area 2. Base case (the point at extreme right) is , pu, max power  = 66.0\,pu.}  \label{pl4Change-WeccIDGeneral}
  \end{center}
  \end{figure} 
  
  \looseness =-1
 Figures~\ref{pl1Change-WeccIDGeneral} and  \ref{pl4Change-WeccIDGeneral} also show the outage severity computed as the maximum power into the area.
  Note that the line outages are sorted according to increasing  maximum power into the area (decreasing  severity). The most severe line outages are on the left hand sides of Figures~\ref{pl1Change-WeccIDGeneral} and  \ref{pl4Change-WeccIDGeneral}, and it 
  can be seen that the area angle usually increases substantially for  most of the severe line outages.  Moreover, in the middle portion of the figures with small changes in severity from the base case (the flat portion of the maximum
  power into the area), the change in area angle from the base case is usually also small.  
  This suggests, for our chosen quantification of outage severity, that large increases in area angle 
  usually indicate the severe line outages. 
  In our experience, this good  result relies on our use of  realistic line limits.
   This tracking of the severity of the outages with the area angle is imperfect, but 
    this is to be expected when trying to monitor over 700 lines in WECC area 1 and 500 lines in WECC area 2 with one scalar area angle as a single bulk area index. (Also note that we are only using a dozen or fewer synchrophasor measurements 
    to compute the area angle.)
   There are several reasons for the exceptional line outages in which the changes in area angle do not track the outage severity.
    Large generation or load inside the area can influence the maximum power entering the area under single line outage conditions,
    and the discrepancy can arise from inaccurate assessment of the outage severity with the maximum power entering the area.
    The line limits that determine the maximum power entering the area and the outage severity
   may not follow the susceptance of the lines and so the susceptance of the area and hence in these cases the area angle cannot track the outage severity. These effects are also the likely cause of the outages at the right of Figure  \ref{pl4Change-WeccIDGeneral} having a maximum power into the area larger than the base case.    

  \looseness=-1
To numerically check the assertion that   and   are close,  we compute the ratio  for each line outage. 
For WECC area 1,
  has mean  0.9999, standard deviation 0.002501, and it ranges from 0.9846 to 1.014.
For WECC area 2,
  has mean  0.9993, standard deviation 0.006082, and it ranges from 0.9236 to 1.056.


\section{Conclusion}
\label{conclusion}

 It is useful to monitor area angle by combining together synchrophasor measurements at the borders of a suitably chosen area.
The area angle and the area susceptance change when single, non-islanding line outages occur and we show that area angle and susceptance tend to change inversely
using both simple examples and two examples of areas  with hundreds of lines in a real power system.
This approximate relation between area angle and area susceptance gives intuition about how the area angle works
to detect line outages in the area.

The area angle results in a real power system also show that the amount of change in the area angle usually indicates the severity of the line outage
(the exceptions generally relate to outages of lines that are connected to generation or load inside the area).
This suggests that a threshold for changes in the area angle to distinguish severe single line outages could be set.

\newpage
\section*{Acknowledgments}
\label{ack}
We gratefully acknowledge support in part from 
DOE project ``The Future Grid to Enable Sustainable Energy Systems," an initiative of PSERC, 
and
NSF grant CPS-1135825, and the  Electric Power Research Center at Iowa State University.
We gratefully acknowledge access to the WECC data that enabled this research.
The analysis and conclusions are strictly those of the authors and not of WECC.

\newpage




  \begin{thebibliography}{9}
  
  \vspace{2pt}



 \bibitem{DobsonvoltPS12} I. Dobson,  Voltages across an area of a network, {\sl IEEE Transactions on Power Systems}, vol. 27, no. 2, May 2012, pp. 993-1002. 


 \bibitem{DobsonIREP10}  I. Dobson, New angles for monitoring areas, IREP Symposium,
Bulk Power System Dynamics and Control - VIII, Buzios,  Brazil, Aug. 2010.


 \bibitem{DobsonHICSS10} I. Dobson, M. Parashar, C. Carter, Combining phasor measurements to monitor cutset angles,  
Forty-third Hawaii International Conference on System Sciences, Kauai, Hawaii, January 2010.

\bibitem{DobsonPESGM10}  I. Dobson, M. Parashar, A cutset area concept for phasor monitoring, 
IEEE PES General Meeting, Minneapolis, MN USA, July 2010.

\bibitem{LopezPESGM12} G.J. Lopez, J.W. Gonzalez, R.A. Leon, H.M. Sanchez, I.A. Isaac, H.A. Cardona, Proposals based on cutset area and cutset angles and possibilities for PMU deployment,
IEEE PES General Meeting, San Diego CA, July 2012.


\bibitem{DarvishiNAPS13} A. Darvishi, I. Dobson, A. Oi, C. Nakazawa,
Area angles monitor area stress by responding to line outages,
 NAPS North American Power Symposium, Manhattan KS USA, September 2013.


\bibitem{SehwailNAPS12}	\looseness=-1 H. Sehwail, I. Dobson, Locating line outages in a specific area of a power system with synchrophasors, North American Power Symposium (NAPS), Urbana-Champaign IL, Sept. 2012. 

\bibitem{SehwailPS13} 	\looseness=-1 H.Sehwail, I.Dobson, Applying
synchrophasor computations to a specific area,
{\sl IEEE Trans. Power Systems}, vol\,28, no\,3, Aug 2013, pp. 3503-3504.

\bibitem{TatePS08} J.E. Tate, T.J. Overbye,
Line outage detection using phasor angle measurements,
{\sl IEEE Trans. Power Syst.}, vol~23, no~4, Nov~2008, pp. 1644-1652.


\end{thebibliography}


\end{document}
