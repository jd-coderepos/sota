\documentclass[reqno,12pt]{amsart}
\usepackage{graphics}
\usepackage{amssymb,amsmath}
\input cyracc.def
\font\tencyr=wncyr8
\def\cyr{\tencyr\cyracc}
\newsavebox{\ukrB}
\savebox{\ukrB}{{\cyr B}}
\oddsidemargin 5.75mm
\evensidemargin 5.75mm
\textheight = 45\baselineskip
\textwidth 150mm
\newtheorem{theorem}{Theorem}[section]
\newtheorem{lemma}[theorem]{Lemma}
\newtheorem{definition}[theorem]{Definition}
\newtheorem{remark}[theorem]{Remark}
\newcounter{claim}
\renewcommand{\theclaim}{\Alph{claim}}
\newenvironment{claim}{\refstepcounter{claim}\par\medskip\par\noindent{\it Claim~\theclaim.~}~\rm}{\par\smallskip\par}
\newenvironment{subproof}{\par\noindent{\sl Proof of Claim~\theclaim.~}}{\par\medskip\par}
\newenvironment{bfenumerate}{\renewcommand{\labelenumi}{{\bf\theenumi.}}\renewcommand{\labelenumii}{{\bf(\theenumii)}}\begin{enumerate}}{\end{enumerate}}
\newcounter{oq}
\newcommand{\que}{\refstepcounter{oq}\par{\bf \theoq.}~}
\newcommand{\fix}[1]{\mathit{fix}(#1)}
\newcommand{\reals}{\mathbb{R}}
\newcommand{\function}[2]{:#1 \rightarrow #2}
\newcommand{\of}[1]{\left( #1 \right)}
\newcommand{\setdef}[2]{\left\{ \hspace{0.5mm} #1 : \hspace{0.5mm} #2 \right\}}
\newcommand{\refeq}[1]{(\ref{eq:#1})}
\newcommand{\eps}{\varepsilon}

\newcommand{\EE}[1]{ {\mathbb E} \left[ #1 \right] }
\newcommand{\PP}[1]{ {\mathbb P} \left[ #1 \right] }
\newcommand{\fixx}[1]{\mathit{fix}^X(#1)}
\newcommand{\FIX}[1]{\mathit{FIX}(#1)}
\newcommand{\Fix}[1]{\mathit{Fix}(#1)}
\newcommand{\fixl}[1]{\mathit{fix}^-(#1)}
\newcommand{\calS}{{\mathcal S}}
\newcommand{\colX}{{\mathcal X}}
\newcommand{\colZ}{{\mathcal Z}}
\newcommand{\calH}{{\mathcal H}}
\newcommand{\cspace}{{\mathrm F}(X)}
\newcommand{\bspace}{{\mathrm B}(X)}
\newif\ifnotesw\noteswtrue
\newcommand{\comm}[1]{\ifnotesw {\sf #1} \fi}
\setlength{\marginparwidth}{30mm}
\setlength{\marginparpush}{-5ex}
\newcommand{\edit}[1]{\ifnotesw \marginpar [{\scriptsize\it\begin{minipage}[t]{\marginparwidth}\raggedleft#1\end{minipage}}]{\scriptsize\it\begin{minipage}[t]{\marginparwidth}\raggedright#1\end{minipage}} \fi}
\newenvironment{changed}{\edit{}}{\edit{}}
\title{Untangling planar graphs\\ from a specified vertex position
--- Hard cases}


\author{M.~Kang}
\address{Institut f\"ur Informatik, Humboldt Universit\"at zu Berlin, 
D-10099 Berlin}

\author{O.~Pikhurko}
\address{Department of Mathematical Sciences,
Carnegie Mellon University, Pittsburgh, PA 15213}
\thanks{\,Partially supported by the National Science Foundation through Grant DMS-0758057
and an Alexander von Humboldt fellowship.}

\author{A.~Ravsky}
\address{Institute for Applied Problems of Mechanics and Mathematics, 
Naukova St.\ 3{\cyr B}, Lviv 79060, Ukraine}

\author{M.~Schacht}
\address{Institut f\"ur Informatik, Humboldt Universit\"at zu Berlin, 
D-10099 Berlin}

\author{O.~Verbitsky\,}
\address{Institute for Applied Problems of Mechanics and Mathematics, 
Naukova St.\ 3{\cyr B}, Lviv 79060, Ukraine}
\thanks{\,Supported by an Alexander von Humboldt return fellowship.}


\date{22 November 2010}

\begin{document} 

\begin{abstract}
Given a planar graph ,
we consider drawings of  in the plane where edges are represented
by straight line segments (which possibly intersect).
Such a drawing is specified by an injective embedding 
of the vertex set of  into the plane.
Let  be the maximum integer   such that there exists
a crossing-free redrawing  of  which keeps  vertices
fixed, i.e., there exist  vertices  of  such that 
 for . Given a set of points , let
 denote the value of  minimized over 
locating the vertices of  on~.
The absolute minimum of  is denoted by~.

For the wheel graph , we prove that 
for every . With a somewhat worse constant factor this is as well true
for the fan graph . 
We inspect also other graphs for which it is known that .

We also show that the minimum value  of the parameter
 is always attainable by a collinear~.
\end{abstract}

\maketitle

\markleft{\sc M.~KANG, O.~PIKHURKO, A.~RAVSKY, M.~SCHACHT, AND O.~VERBITSKY}
\markright{\sc UNTANGLING PLANAR GRAPHS --- HARD CASES}

\section{Introduction}\label{s:intro}
\subsection{The problem of untangling a planar graph}
In a \emph{plane graph}, each vertex  
is a point in   and each edge  
is represented as a continuous plane curve with endpoints  and .
All such curves are supposed to be non-self-crossing and any two of them
either have no common point or share a common endvertex.
An underlying abstract graph of a plane graph is called \emph{planar}.
A planar graph can be drawn as a plane graph in many ways, and
the Wagner-F\'ary-Stein theorem (see, e.g., \cite{NRa})
states that there always exists a \emph{straight line drawing}
in which every edge is represented by a straight line segment.

Let  denote the vertex set of a planar graph . In this paper,
by a \emph{drawing} of  we mean an arbitrary injective map
. We suppose that each edge 
of  is drawn as the straight line segment with endpoints 
and .
Due to possible edge crossings and even overlaps,
 may not be a plane drawing of~. Hence it is natural to ask:
\begin{quote}
How many vertices have to be moved to obtain from \\
a plane (i.e., crossing-free) straight line drawing of~?
\end{quote}
Alternatively, we could 
allow in  curved edges without their exact specification; such a drawing 
could be always assumed to be a plane graph. Then our task would be
to \emph{straighten}  rather than eliminate edge crossings.

More formally, for a planar graph  and a drawing , let

where the maximum is taken over all plane straight line drawings  of .
Furthermore, let

In other words,  is the maximum number of vertices
which can be fixed in any drawing of  while \emph{untangling} it.

No efficient algorithm determining the parameter  is known.
Moreover, computing  is known to be NP-hard \cite{merged,Ver}.

Improving a result of Goaoc et al.~\cite{merged},
Bose et al.~\cite{Bose} showed that

for every planar graph , where here and in the rest of this paper~ 
denotes the number of vertices
in the graph under consideration. Better bounds on  are known
for cycles~\cite{PTa}, trees~\cite{merged,Bose} and, more generally, outerplanar 
graphs~\cite{merged,RVe}. In all these cases it was shown that .
For cycles Cibulka \cite{Cib} proves a better lower bound of .

Here we are interested in upper bounds on , that is, in examples of graphs
with small . Moreover, let  be an arbitrary set of  points in the 
plane and define

Note that .
This notation allows us to formalize another natural question.
Can untangling of a graph become easier if the set  of vertex positions
has some special properties (say, if it is known that  is \emph{collinear}, i.e.,
lies on a line, or is in \emph{convex position}, i.e., 
no  lies in the convex hull of )? 
This question admits several variations:
\begin{itemize}
\item
For which  can one attain equality ?
\item
Are there graphs with  small for \emph{all}~?
\item
Are there graphs such that  is for some  considerably
larger than ?
\end{itemize}

\subsection{Prior results}\label{ss:prior}
The \emph{cycle} (resp.\ \emph{path}; \emph{empty graph})
on  vertices will be denoted by  (resp.\ ; ).
Recall that the \emph{join} of vertex-disjoint graphs  and 
is the graph  consisting of the union of  and  and all
edges between  and . The graphs  (resp.\
; ) are known as \emph{wheels} (resp.\
\emph{fans}; \emph{stars}). By  we denote
the disjoint union of  copies of a graph~.

Pach and Tardos~\cite{PTa} were first who established a principal fact:
Some graphs can be drawn so that, in order to untangle them, one has
to shift almost all their vertices. In fact, 
this is already true for cycles.
More precisely,  Pach and Tardos~\cite{PTa} proved that



The best known upper bounds are of the form .
Goaoc et al.\ \cite{Goaoc}\footnote{The conference presentations \cite{Goaoc} and \cite{SWo} were subsequently combined
into the journal paper~\cite{merged}.}
showed it for certain triangulations. 
More specifically, they proved that


Shortly after \cite{Goaoc} and independently of it,
there appeared our manuscript \cite{KSchV}, which was actually 
a starting point of the current paper. For infinitely many ,
we constructed a family  of 3-connected
planar graphs on  vertices with . Though no explicit bound was
specified in \cite{KSchV}, a simple analysis of our construction
reveals that

where  denotes an arbitrary member of .
While the graphs in  are not as simple as  and the subsequent examples in the
literature, the construction of  has the advantage that this class contains graphs with
certain special properties, such as bounded vertex degrees.
By a later result of Cibulka \cite{Cib}, we have 
for every  with maximum degree and diameter bounded by a logarithmic function. 
Note in this respect that  contains graphs with bounded maximum degree that
have diameter~.

In subsequent papers \cite{SWo,Bose} examples of graphs with small 
were found in special classes of planar graphs, such
as outerplanar and even acyclic graphs.
Spillner and Wolff \cite{SWo} showed for the fan graph that

and Bose et al.\ \cite{Bose} established for the star forest with  vertices that

Finally, Cibulka \cite{Cib} proved that

for all 3-connected planar graphs.


\subsection{Our present contribution}\label{ss:contrib}
In Section \ref{s:collin} we notice that the choice of a collinear
vertex position in \refeq{Goaoc}, \refeq{fans}, and \refeq{stars}
is actually optimal for proving upper bounds on .
Specifically, we show that for any  the equality 
is attained by some collinear  (see Theorem \ref{thm:fixlfix}).

In Section \ref{s:fixx} we extend the bound
 in the strongest way with respect to specification
of vertex positions. We prove that

(see Theorem \ref{thm:FIXWnFn}).
Let us define

(while ).
With this notation, \refeq{fixx} and \refeq{fixx2} read


In Section \ref{s:hulls} we discuss an approach attempting to give
an analog of \refeq{fixx} for the aforementioned family of graphs
. A member of this family is defined as a plane graph of the
following kind. 
Let  and . Draw  triangulations, each having  vertices,
so that none of them lies inside an inner face of any other
triangulation. Connect these triangulations by some more edges making
the whole graph 3-connected.
 is the set of all 3-connected planar graphs obtainable in this way.
This set is not empty. Indeed, we can allocate the  triangulations in a
cyclic order and connect each neighboring pair by two vertex-disjoint edges
as shown in Fig.~\ref{fig:Gk4}. Note that  new edges form a cycle 
and the other  new edges participate in a cycle . 
If we remove any two vertices from the obtained graph,
each triangulation as well as the whole ``cycle'' stay connected
(since the aforementioned cycles  and  are vertex-disjoint,
at most one of them can get disconnected).

Note that, if we start with triangulations with bounded vertex degrees, 
the above construction gives us a graph with bounded maximum degree.
In this situation our argument for \refeq{fixx}
does not work. We hence undertake a different approach.

\begin{figure}
\centerline{\includegraphics{H4}}
\caption{Example of a graph in .}
\label{fig:Gk4}
\end{figure}

Given a set of colored points in the plane, we
call it \emph{clustered} if its monochromatic parts have pairwise disjoint
convex hulls. Given a set  of  points, let  denote
the maximum cardinality of a clustered subset existing in 
under any balanced coloring of  in  colors (see Definition \ref{def:CX}).
It is not hard to show (see Lemma \ref{lem:fixxC}) that

where  denotes an arbitrary graph in .
We prove that  for every , which implies that  
(Theorem~\ref{thm:FIXHn}).

Better upper bounds for  would give us better upper bounds for .
Note that  has relevance also to the star forest , namely

(see part 2 of Lemma \ref{lem:fixxC}).
Thus, if there were a set  with , the parameter 
would be far apart from .

As we do not know how close or far away the parameters  and  are
for  and , the two graph families deserve further attention.
Section \ref{s:weak} is devoted to estimation of  for  in
\emph{weakly convex position}, which means that the points in 
lie on the boundary of a convex body (including the cases that 
is in convex position and that  is a collinear set). Since
 for any  in weakly convex position, by \refeq{HnC} we
obtain  for such  (Theorem~\ref{thm:fix}). 


This result for  together with the stronger results obtained for
 and  in Section \ref{s:fixx} might suggest
that  should hold for any  whenever  is in weakly convex position.
The simplest case where we are not able to confirm this conjecture is .
By \refeq{HnC} and \refeq{kSkC} we have  for any  and ,
and bounding  from above seems harder.
Nevertheless, even here we have a rather tight bound: 
If  is in weakly convex position, then
, where 
denotes the inverse Ackermann function (Theorem~\ref{thm:stars}).


We conclude with a list of open questions in Section~\ref{s:open}. 


\section{Hardness of untangling from a collinear position}\label{s:collin}


\begin{theorem}\label{thm:fixlfix}
For every planar graph  we have  for some collinear~.
\end{theorem}
Theorem~\ref{thm:fixlfix} can
be deduced from \cite[Lemma 1]{Bose}. For the reader's convenience,
we give a self-contained proof.

\begin{proof} 
Let  denote the minimum value of  over collinear .
We have  by definition.
The theorem actually states the converse inequality .
That is, given an arbitrary drawing , we have to
show that it can be untangled while keeping at least  vertices fixed.

Choose Cartesian coordinates in the plane so that  is located between
the lines  and . Let  denote the projections 
onto the -axis and the -axis, respectively. We also suppose that the axes
are chosen so that the map  is injective. 
Let us view  as a drawing of , aligning all the vertices on the line .
By definition, there is a plane drawing  of  such that 
the set of fixed vertices 
has cardinality at least . 

Given a set  and a real , let 
denote the -neighborhood of  in the Euclidean metric. 
For each pair of disjoint edges 
in , there is an  such that .
Since  is finite, we can assume that the latter is true with the same
 for all disjoint pairs . 

We now define a drawing 
by setting

Note that  for every .
Since  is crossing-free, so is~.

Finally, define a linear transformation of the plane by 
and consider . Clearly,  is a plane drawing of  and all
vertices in  stay fixed under the transition from  to~.
\end{proof}


\section{Hardness of untangling from every vertex position}\label{s:fixx}


In Section \ref{ss:monoton} we state known results on the longest monotone subsequences
in a random permutation. These results are used in Section \ref{ss:FIX} for
proving upper bounds on  and .

\subsection{Monotone subsequences in a random permutation}\label{ss:monoton}

By a \emph{permutation} of  we will mean a sequence

where each positive integer  occurs once (that is,  determines a
one-to-one map  by ).
A subsequence , where ,
is \emph{increasing} if .
The length of a longest increasing subsequence of  will be
denoted by . 

\begin{lemma}\label{lem:random}
Let  be a uniformly random permutation of .

\noindent
\begin{bfenumerate}
\item \textbf{(Pilpel \cite{Pil})}
.

\smallskip

\item \textbf{(Frieze \cite{Fri}, Bollob\'as-Brightwell \cite{BBr})}
For any real  there is a  such that
for all  we have

\end{bfenumerate}
\end{lemma}

\noindent
Further concentration results for  are obtained in \cite{Tal,BDJ}.

Lemma \ref{lem:random} shows that 
with probability at least .
We will also need a bound for another parameter of , roughly speaking, for
the maximum total length of two non-interweaving monotone subsequences of .
Let us define this parameter more precisely.
A subsequence of a permutation  will be called \emph{monotone} if it can 
be made increasing by shifting and/or reversing (as, for example, 21543). 
This notion is rather natural if we regard  as a \emph{circular permutation}, 
i.e.,  is considered up to shifts.
Call two subsequences  and  of  
\emph{non-interweaving} if they have no common element and  has no 
subsequence  with  occurring in  
and  in . Define  to be the sum of 
the lengths of  and  maximized over non-interweaving monotone 
subsequences of .

\begin{lemma}\label{lem:random2}
Let  be a uniformly random permutation of . 
For any real  there is a  such that
for all  we have

\end{lemma}

\begin{proof}
Given a sequence  and a pair of indices ,
consider the splitting of the circular version of  into two parts
 and . Let 
and  be the reverses of  and .
Denote 

Note that  for some pair .
Since there are only polynomially many such pairs,
it suffices to show for each  that the inequality

holds with an exponentially small probability.
Denote the length of  by , so that . For each ,
note that both  and  are distributed identically to
.

Suppose first that  or  is relatively small, say, .
Then \refeq{lij} implies that 

or this estimate is true for . Provided , and hence , is large enough,
we conclude by Lemma \ref{lem:random} that \refeq{lij} happens with probability
at most .

Suppose now that  for both  and that  is large enough.
Since
,
the inequality \refeq{lij} entails that for  or  we must have

or this estimate must be true for . By Lemma \ref{lem:random}, the event \refeq{lij}
happens with probability no more than , 
where .

We see that, whatever  and  are, \refeq{ell2} holds for
any positive  and large enough~.
\end{proof}

\subsection{Graphs with small }\label{ss:FIX}

Recall that .
If  is small, this means that no special properties of the set
of vertex locations can make the untangling problem for  easy.

\begin{lemma}\label{lem:fixx}
For any 3-connected planar graph  on  vertices with maximum
vertex degree  we have

\end{lemma}

\begin{proof}
We have to prove that  for any set  of  points.
Let  and denote .
Given a point  in the plane, we define a permutation 
 describing the order in which the points in  are visible from the 
standpoint . 
If  with , we take  as the first visible point, that is,
let  be the first index in the sequence . Now, we look around starting
from the north in a clockwise direction and put  before  in 
if we see  earlier than . The ``north'' direction on the plane can be fixed 
arbitrarily. If  and  lie in the same direction from ,
we see the nearer point first, that is,  precedes  in 
whenever .

Define an equivalence relations  so that  if  and 
are obtainable from one another by a shift.
Let us show that the quotient set  is finite and estimate its cardinality.
Suppose first that not all points in  are collinear.
Let  be the set of lines passing through at least
two points in . After removal of all lines in , the plane
is split into connected components that will be called \emph{-faces}.
Any intersection point of two lines will be called an \emph{-vertex}.
The -vertices lying on a line in  split this line into \emph{-edges}.
Exactly two -edges for each line are unbounded.
It is easy to see that  whenever  and  belong to the same
-face or the same -edge. It follows that 
does not exceed the total amount of -faces, -edges, and -vertices.

Let us express this bound in terms of . If we erase
all the unbounded -edges, we obtain a crossing-free straight line drawing of
a planar graph with at most  vertices. It has less than
 edges and  faces. Restoring the unbounded
-edges, we see that the total number of -edges is less than   
and the number of -faces is less than . Therefore,

In the much simpler case of a collinear , we have .


Let  be a vertex of  with maximum vertex degree. By the Whitney theorem
on embeddability of 3-connected graphs, the neighbors of  appear around 
in the same circular order  in any plane drawing of .
Pick up a random permutation  of  and consider
a drawing  such that .
Let  be an untanglement of . Let  and denote
the set of all shifts and reverses
of the permutation  by~.

We have to estimate the number of vertices remaining fixed 
under the transition from  to , that is, the cardinality
of the set .
Let , which is the subset of 
corresponding to the fixed neighbors of . Note that
 and recall that  by our assumption.
It follows that , and we have to estimate~.

The points in  go around  in the canonical Whitney order.
This means that the indices of the corresponding vertices
form an increasing subsequence in  for some .
For each , the composition  is a random permutation of .
Recall that, irrespectively of the choice of ,
there are at most  possibilities for .
By Lemma \ref{lem:random}, every increasing subsequence of 
has length at most  with probability at least .
Thus, if  is sufficiently large, we have  for all
untanglements  of some drawing  (in fact, this is true
for almost all~). This implies the required bound .
\end{proof}

While Lemma \ref{lem:fixx} immediately gives us a bound on 
for the wheel graph, this lemma does not apply directly to the fan graph  
because it is not 3-connected and has a number of essentially different plane drawings. 
Nevertheless, all these drawings are still rather structured, which makes
analysis of the fan graph only a bit more complicated.
Indeed, denote the central vertex of  by  and let  
be the path of the other vertices.
Let  be a plane drawing of . Label each edge 
with number  and denote the circular sequence in which the labels follow each other
around  by . Split  into two pieces.
Let  be the sequence of labels starting with , ending with ,
and containing all intermediate labels if we go around  clockwise.
Let  be the counter-clockwise analog of .
Note that  and  overlap in .

\begin{lemma}\label{lem:fans}
Both  and  are monotone.
\end{lemma}

\begin{proof}
We proceed by induction on . The base case of  is obvious.
Suppose that the claim is true for all plane drawings of 
and consider an arbitrary plane drawing  of .
Let  be obtained from  by erasing 
along with the incident edges. Obviously,  is a plane drawing of .

In the drawing  of , we consider the triangle
 with vertices , , and
. Clearly, all points  for  are inside 
or all of them are outside. In both cases,  and  are neighbors in .
Therefore,  is obtainable from  by inserting  on the one
or the other side next to . It follows that  is obtained from
 either by appending  after  or by replacing  with 
(the same concerns  and ). It remains to note that
both operations preserve monotonicity.
\end{proof}

We are now prepared to obtain upper bounds on  for the wheel graph  
and the fan graph .
Note that, up to a small constant factor, these bounds
match the lower bound  (which follows, e.g., from
\cite[Theorem 4.1]{RVe}).

\begin{theorem}\label{thm:FIXWnFn}
\mbox{}

\begin{bfenumerate}
\item
.
\item
.
\end{bfenumerate}
\end{theorem}
 
\begin{proof}
The bound for  follows directly from Lemma \ref{lem:fixx} as observed before.

As for , notice that the argument of Lemma \ref{lem:fixx} becomes applicable
if, in place of the Whitney theorem, we use Lemma \ref{lem:fans}.
Let  be a random location of  on , as in the proof
of Lemma \ref{lem:fixx}. 
More precisely, let  denote the path of non-central vertices
in . We pick a random permutation  of  and set
. As established in the proof of Lemma \ref{lem:fixx},
the set  determines a set of permutations  with 
such that, from any standpoint  in the plane, the vertices 
 are visible in the circular order 
for some .

Let  be any untangling of  and 
be the associated order on the neighborhood of the central vertex .
By Lemma \ref{lem:fans},  consists of two monotone parts
 and . The set  of fixed vertices is correspondingly
split into  and .
Since  and  overlap in two elements,
 and  can have one or two common vertices. If this happens,
we remove those from .
Notice that the indices of the vertices in  and in  form
non-interweaving monotone subsequences of . 
Therefore, 
and part 2 of the theorem follows from Lemma~\ref{lem:random2}.
\end{proof}


\section{Making convex hulls disjoint}\label{s:hulls}

In Section \ref{ss:prior} we listed the few graphs for which an upper
bound  is known, namely , , , and .
By Theorem \ref{thm:FIXWnFn} in the former two cases we have a stronger
result  (note that  contains  as a subgraph).
We now consider a problem related to estimating the parameters 
and . 

\begin{definition}\label{def:CX}\rm
Let  and  be an -point set in the plane.
Given a partition , we regard  
as a coloring of  in  colors.
We will consider only \emph{balanced}  with each .
Call a set  \emph{clustered} if the monochromatic classes
 have pairwise disjoint convex hulls.
Let  denote the largest size of a clustered subset of .
Finally, define .
\end{definition}

\begin{lemma}\label{lem:fixxC}
Let  denote an arbitrary graph in .
\begin{bfenumerate}
\item
.
\item
.
\end{bfenumerate}
\end{lemma}

\begin{proof}
{\sl 1.}
Recall that  is defined as a plane graph whose vertex set
 is partitioned so that each  spans
a triangulation and these  triangulations are in the outer faces of each other.
Take  such that  and
 such that .
Consider an untanglement  of  and denote the set
of fixed vertex locations by . 
By the Whitney theorem,  is obtainable from the plane graph 
by a homeomorphism of the plane, possibly after turning some inner face of 
into the outer face.
Since  spans a triangulation in , the convex hull of 
is a triangle . Since the corresponding triangulations are pairwise
disjoint in , the triangles 's are pairwise disjoint possibly with
a single exception for some  containing all the other triangles.
Let . It follows that the convex hulls of the
's do not intersect, perhaps with an exception for a single
set . The exception may occur if  is homeomorphic to a version 
of  with different outer face.
Therefore, , where the term  corresponds to the exceptional~.

{\sl 2.}
Given an arbitrary drawing  of the star forest, 
we have to untangle it while keeping at least  vertices
fixed. Let  where each  is the vertex set
of a star component. Define a coloring  of  by .
Let  be a largest clustered subset of . 
Choose pairwise disjoint open convex sets  so that  
contains  for all . Redraw  so that, for each , 
the -th star component is contained in . It is
clear that, doing so, we can leave all non-central vertices in  fixed.
Thus, we have at least  fixed vertices.
\end{proof}

\begin{lemma}\label{lem:CX}
For any set  of  points in the plane, we have .
\end{lemma}

\begin{proof}
Let  denote the set of all balanced -colorings of , i.e.,
the set of partitions  with each .
We have .

Call a -tuple of subsets  a \emph{crossing-free
coloring} of  if the 's have pairwise disjoint convex hulls.
We do not exclude that some 's are empty and the coloring is partial, i.e.,
. Denote the set of all crossing-free colorings of 
by~.

Let . An estimate  means that

for some . Regard  and  as elements of the space
 of -colorings of , where the new color  is
assigned to the points that are uncolored in . Then \refeq{XZ} means
that the Hamming distance between  and  does not exceed .
Note that the -neighborhood of  can contain no more than 
 elements of . Therefore, an estimate 
would follow from inequality


Given a partition  of a point set ,
we call it \emph{crossing-free} if the
convex hulls of the 's are nonempty and pairwise disjoint. 
According to Sharir and Welzl \cite[Theorem 5.2]{ShW},
the overall number of crossing-free partitions of any -point set
 is at most .
In order to derive from here a bound for the number of
crossing-free \emph{colorings}, with each coloring 
we associate a partition  of the union 
so that 
is the result of removing all empty sets from the sequence .
Since  is the crossing-free partition of a subset of ,
the Sharir-Welzl bound implies that the number of all possible partitions 
obtainable in this way does not exceed .
Since  can be restored from  in
 ways, we obtain

for a constant . Thus, we would have \refeq{CB} provided

Taking logarithm of both sides, we see that the latter inequality
holds for all sufficiently large  if we set .
\end{proof}

Part 1 of Lemma \ref{lem:fixxC} and Lemma \ref{lem:CX} immediately give us
the main result of this section.

\begin{theorem}\label{thm:FIXHn}
 for an arbitrary .
\end{theorem}

\noindent
Note that the bound of Theorem \ref{thm:FIXHn} is the best upper bound on
 that we know for graphs with bounded vertex degrees.


\section{Hardness of untangling from weakly convex position}\label{s:weak}

Despite the observations made in Section \ref{s:hulls}, we do not know
whether or not  and  are close to, respectively, 
 and  for every location  of the vertex set.
We now restrict our attention to point sets  in weakly convex position,
i.e., on the boundary of a convex plane body.

We will use Davenport-Schinzel sequences defined as follows 
(see, e.g.,~\cite{ASh} for more details).
An integer sequence  is called a \emph{-Davenport-Schinzel
sequence} if the following conditions are met:
\begin{itemize}
\item
 for each ;
\item
 for each ;
\item
 contains no subsequence  of length  for any .
\end{itemize}
By a \emph{subsequence} of  we mean any sequence 
with .
The maximum length of a -Davenport-Schinzel sequence will be denoted
by . We are interested in the particular case of .

We inductively define a family of functions over positive integers:

\emph{Ackermann's function} is defined by .
This function grows faster than any primitive recursive function.
The inverse of Ackermann's function is defined by
.

Agarwal, Sharir, and Shor \cite{ASS} proved that .
Note that  grows very slowly, e.g.,  for
all  up to , which is the exponential tower of twos of height 65536. 
Thus, the bound for  is nearly linear in~.

Sometimes it will be convenient to identify a sequence  with all 
its cyclic shifts. This way , where ,
is a subsequence of . In such circumstances
we will call a sequence \emph{circular}.
Subsequences of  will be
regarded also as circular sequences. Note that the set of all circular
subsequences is the same for  and any of its shifts.
The length of  will be denoted by~.


\begin{lemma}\label{lem:circle}
Let  and  be the circular sequence consisting of  successive
blocks of the form . 
\begin{bfenumerate}
\item
Suppose that  is a subsequence of  with no
4-subsubsequence of the form , where . Then .
\item
Suppose that  is a subsequence of  with no
6-subsubsequence of the form , where . Then 
.
\end{bfenumerate}
\end{lemma}


\begin{proof}
{\sl 1.}
We proceed by double induction on  and .
The base case where  and  is arbitrary is trivial.
Let  and consider a subsequence  with no forbidden subsubsequence. 
If each of the  elements occurs in  at most once, then
 and the claimed bound is true. Otherwise, without loss of generality 
we suppose that  contains  occurrences of . 
Let  (resp.\ ) 
denote the parts of  (resp.\ ) between these  elements.
Thus, . 

Denote the number of elements with at least one occurrence in  by .
Each element  occurs in at most one of the 's because otherwise
 would contain a subsequence . It follows that .
Note that, if we append  with an element ,
it will consist of blocks . Denote the number of these blocks
by  and notice the equality .
Since  has no forbidden subsequence, we have .
If , this follows from the induction assumption because
 can be regarded a subsequence of .
If , this is also true because then .
Summarizing, we obtain .

{\sl 2.}
Let  be obtained from  by shrinking each block 
of the same elements to . Since  is a -Davenport-Schinzel 
sequence, we have . Note now that any two elements
neighboring in a shrunken block are at distance at least  in .
It easily follows that the total number of elements deleted in  is less
than~.
\end{proof}

\begin{theorem}\label{thm:fix}
Let  be an arbitrary graph in . For any  in weakly convex position we have
 
\end{theorem}

\begin{proof}
By part 1 of Lemma \ref{lem:fixxC}, it suffices to show that 
for any set  of  points on the boundary  of a convex body.
Let  be the interweaving -coloring of  where the colors appear
along  in the circular sequence  as in Lemma \ref{lem:circle}.
Suppose that  is a clustered subset of . Note that there are no
two pairs  and , ,
with intersecting segments  and . This means that
the subsequence of  induced by  does not contain any pattern
. By part 1 of Lemma \ref{lem:circle}, we have  and, hence,
 as required.
\end{proof}

\begin{remark}\rm
With a little more care, we can improve the constant factor
in Theorem \ref{thm:fix} by proving that 
for any  in weakly convex position.
\end{remark}

The rest of this section is devoted to the star forest .
This sequence of graphs is of especial interest because this is the only
example of graphs for which we know that  but are currently able
to prove neither that  nor that  for 
in weakly convex position.

The first part of the forthcoming Theorem~\ref{thm:stars} restates~\cite[Theorem 5]{Bose}
(see \refeq{stars} in Section \ref{ss:prior})
with a worse factor in front of ; we include it for an expository
purpose. Somewhat surprisingly, the proof of this part is based on  part~1 of
Lemma~\ref{lem:circle}, which we already used to prove Theorem \ref{thm:fix}.
The second part, which is of our primary interest, requires a more delicate
analysis based on part 2 of Lemma~\ref{lem:circle}.

\begin{theorem}\label{thm:stars}
Let  denote the star forest with  vertices.
For every integer  we have
\begin{bfenumerate}
\item
 for any collinear ;
\item
 for any  in weakly convex position.
\end{bfenumerate}
\end{theorem}

\begin{proof}
Denote .
Let , where each  consists of all  leaves
in the same star component and  consists of all  central vertices.

{\sl 1.}
Suppose that  consists of points  lying on a line  in this order.
Consider a drawing  such that

Let  be a crossing-free straight line redrawing of .
We have to estimate the number of fixed vertices, i.e., those vertices
participating in 
.
For this purpose we split  into four parts:  where
 (resp.\ ; ) consists of the fixed leaves adjacent to central vertices
located in  above  (resp.\ below ; on )
and  consists of the fixed central vertices.

Trivially,  and it is easy to see that . Let us estimate  and .
Label each  by the index  for which  and view 
as the sequence  defined in Lemma \ref{lem:circle}. 
Let  be the subsequence induced by the points in .
Note that  does not contain any subsequence  because
otherwise we would have an edge crossing in  (see Fig.~\ref{fig:cross}).
By part~1 of Lemma \ref{lem:circle}, we have . The same applies to .
It follows that , as claimed.



\begin{figure}
\centerline{\includegraphics{ell}}
\caption{Proof of part 1 of Theorem \protect\ref{thm:stars}: an -subsequence in .}
\label{fig:cross}
\end{figure}


{\sl 2.}
Let  be a set of  points
on the boundary  of a convex plane body . 
It is known that the boundary of a convex plane body is a rectifiable curve and,
therefore, we can speak of the length of  or its arcs.
Clearly, the convex body  plays a nominal role and can be varied once 
is fixed.
Thus, to avoid unnecessary technical complications in the forthcoming
argument, without loss of generality we can suppose that the boundary
curve  contains only a finite number of (maximal) straight line segments.
In particular, we can suppose that  contains no straight line segment at all
if  is in ``strictly'' convex position.

We will use the following terminology. A \emph{chord} is a straight line
segment whose endpoints lie on . An \emph{arrow} is a directed chord
with one endpoint called \emph{head} and the other called \emph{tail}.
Call an arrow a \emph{median} if its endpoints split  into arcs
of equal length. Fix the ``clockwise'' order of motion along  and
color each non-median arrow in one of two colors, red if the shortest way
along  from the tail to the head is clockwise and blue if it is
counter-clockwise.

Given a point  outside , we define \emph{quiver}  as follows.
For each line going through  and intersecting  in exactly two points,
 and , the  contains the arrow  directed so that the head is closer
to  than the tail. 

Given a non-median arrow , we will denote the shorter component of
 by .
Our argument will be based on the following elementary fact.


\begin{claim}\label{cl:}
Let arrows  and  be in the same quiver  and have the same color.
Suppose that  is shorter than . Then both  and 
lie in .
\end{claim}

\begin{figure}
\centerline{\includegraphics{quiver}}
\caption{Proof of Claim \protect\ref{cl:}.}
\label{fig:quiver}
\end{figure}

\begin{subproof}
Let  be the median in . Since  and  are of the same color,
the four points  are in the same component of .
The claim easily follows from the fact that the chords  and  do not cross
(see Fig.~\ref{fig:quiver}).
\end{subproof}

After these preliminaries, we begin with the proof.
Let  be a listing of points in  along . 
Fix  to be an arbitrary map satisfying \refeq{pi}.
Let  be a crossing-free redrawing of .
Look at the edges in  with one endpoint  on  
and the other endpoint elsewhere. 
Perturbing  a little at the positions not lying on  (and using the regularity
assumption made about ), we can ensure that
\begin{enumerate}
\item
any such edge intersects  
in at most two points, including 
(this is automatically true if  contains no straight line segment);
\item
if an edge intersects  in two points, 
it splits  into components having different lengths.
\end{enumerate}

Assume that  meets these conditions.
Let  be a leaf adjacent to a central
vertex . Suppose that , , and the segment
 crosses  at a point . By Condition 2,
the arrow  is not a median and hence colored in red or blue. 
We color each such  in red or blue correspondingly.

Now we split the set of fixed vertices  into five parts. 
Let  consist of the fixed central vertices,
 (resp.\ ) consist of those fixed leaves such that the edges emanating
from them are completely inside (resp.\ outside) , and
 (resp.\ ) consist of the red (resp.\ blue) fixed leaves.
By Condition 1, we have .


Trivially, . Similarly to the proof of the first part of the theorem,
notice that the subsequences of  corresponding to  and 
do not contain -subsubsequences. By part~1 of Lemma \ref{lem:circle},
we have  and . 

Finally, consider the subsequence  of  corresponding to 
and show that it does not contain any -subsubsequence.
Assume, to the contrary, that such a subsubsequence exists.
This means that  contains two interchanging subsequences
 and  whose elements belong to two different star components
of , with central vertices  and , respectively. 
Since  are red, Claim \ref{cl:} implies that, say,  and 
lie on the shorter arc of  cut off by the edge  (see Fig.~\ref{fig:ababab}).

\begin{figure}
\centerline{\includegraphics{ababab}}
\caption{Proof of part 2 of Theorem \protect\ref{thm:stars}: 
impossibility of an -subsequence in .}
\label{fig:ababab}
\end{figure}

Without loss of generality, let  be between  and  and 
be between  and . Since  and  are red and  is
crossing-free, it must be the case that  intersects 
and  intersects  (in another point). This makes a
contradiction with Claim \ref{cl:}.

Thus,  is -free and, by part~2 of Lemma \ref{lem:circle}, we have
. All the same applies to .
Summarizing, we see that , as claimed.
\end{proof}

\break

\section{Open problems}\label{s:open}
\mbox

\que
Can the parameters  and  be far apart from each other
for some planar graphs?
Say, is it possible that for infinitely many graphs we have 
with a constant ?

\que
Lemma \ref{lem:CX} states an upper bound  for any
set  of  points in the plane. A trivial lower bound is .
How to make the gap closer? By Lemma \ref{lem:fixxC}, this way we could show either
that  is close to  or that  is far from .

\que
Find upper bounds on , at least , for the cycle ,
the star forest , and the uniform binary tree.
Recall that upper bounds on  for these graphs are obtained
in \cite{PTa,Bose,Cib}, respectively 
(the uniform binary tree is just a particular instance
of the class of graphs with logarithmic vertex degrees and diameter treated in~\cite{Cib}).

\que
Let  denote the maximum of  over  in weakly convex position.
Obviously, . Note that the first inequality can be strict: 
for example,  while  for the tetrahedral graph.
Is it true that ? Currently we cannot prove this even for
graphs , cf.\ Theorem~\ref{thm:stars}.

\que
By Theorem \ref{thm:fixlfix}, for every  we have  for some collinear
. Does this equality hold for \emph{every} collinear ?
This question is related to the discussion in \cite[Section 5.1]{RVe}.


\subsection*{Acknowledgements}
We thank anonymous referees for their very careful reading of the manuscript
and suggesting several corrections and amendments.


\begin{thebibliography}{10}

\bibitem{ASh}
P.K.~Agarwal, M.~Sharir.
\newblock
Davenport-Schinzel sequences and their geometric applications.
\newblock
In: {\it Handbook of Computational Geometry}, 
J.R.~Sack and J.~Urrutia (Eds.), North-Holland, pages 1--47 (2000).

\bibitem{ASS}
P.K.~Agarwal, M.~Sharir, P.~Shor. 
\newblock
Sharp upper and lower bounds on the length of general Davenport-Schinzel 
sequences. 
\newblock
{\em Journal of Combinatorial Theory, Series A} 
52(2):228--274 (1989).

\bibitem{BDJ}
J.~Baik, P.~Deift, K.~Johansson.
\newblock
On the distribution of the length of the longest increasing subsequence 
of random permutations.
\newblock
{\em J.\ Am.\ Math.\ Soc.} 12(4):1119--1178 (1999). 

\bibitem{BBr}
B.~Bollob\'as, G.~Brightwell. 
\newblock
The height of a random partial order: Concentration of measure.
\newblock
{\em Annals of Applied Probability\/} 2:1009--1018 (1992).

\bibitem{Bose}
P.~Bose, V.~Dujmovic, F.~Hurtado, S.~Langerman, P.~Morin, D.R.~Wood.
\newblock
A polynomial bound for untangling geometric planar graphs. 
\newblock
{\em Discrete and Computational Geometry} 42(4):570--585 (2009).


\bibitem{Cib}
J.~Cibulka.
\newblock
Untangling polygons and graphs.
\newblock
{\em Discrete and Computational Geometry} 43(2):402--411 (2010).


\bibitem{Fri}
A.~Frieze.
\newblock
On the length of the longest monotone subsequence in a random permutation.
\newblock
{\em Annals of Applied Probability\/} 1(2):301--305 (1991).

\bibitem{merged}
X.~Goaoc, J.~Kratochv\'{\i}l, Y.~Okamoto, C.S.~Shin, A.~Spillner, A.~Wolff.
\newblock
Untangling a planar graph.
\newblock
{\it Discrete and Computational Geometry\/} 42(4):542--569 (2009).

\bibitem{Goaoc}
X.~Goaoc, J.~Kratochv\'{\i}l, Y.~Okamoto, C.S.~Shin, A.~Wolff.
\newblock
Moving vertices to make a drawing plane. 
\newblock
In: 
{\it Proc.\ of the 15-th International Symposium Graph Drawing.}
Lecture Notes in Computer Science, vol.\ 4875, pages 101--112. 
Springer-Verlag, 2007. 


\bibitem{KSchV}
M.~Kang, M.~Schacht, O.~Verbitsky.
\newblock
How much work does it take to straighten a plane graph out?
\newblock
E-print: {\sl http://arxiv.org/abs/0707.3373} (2007).

\bibitem{NRa}
T.~Nishizeki, Md.S.~Rahman.
\newblock
{\it Planar graph drawing.}
\newblock
World Scientific (2004).

\bibitem{PTa}
J.~Pach, G.~Tardos.
\newblock
Untangling a polygon. 
\newblock
{\it Discrete and Computational Geometry\/} 28(4):585--592 (2002).

\bibitem{Pil}
S.~Pilpel. 
\newblock
Descending subsequences of random permutations.
\newblock
{\em Journal of Combinatorial Theory, Series A} 
53(1):96--116 (1990).

\bibitem{RVe}
A.~Ravsky, O.~Verbitsky.
\newblock
On collinear sets in straight line drawings.
\newblock
E-print: {\sl http://arxiv.org/abs/0806.0253} (2008).

\bibitem{ShW}
M.~Sharir, E.~Welzl.
\newblock
On the number of crossing-free matchings, cycles, and partitions.
\newblock
{\em SIAM J.\ Comput.} 36(3):695--720 (2006).

\bibitem{SWo}
A.~Spillner, A.~Wolff.
\newblock
Untangling a planar graph.
\newblock
In: 
{\it Proc.\ of the 34-th International Conference on Current 
Trends Theory and Practice of Computer Science.} 
Lecture Notes in Computer Science, vol.\ 4910, pages 473--484. 
Springer-Verlag, 2008.


\bibitem{Tal}
M.~Talagrand.
\newblock
Concentration of measure and isoperimetric inequalities
in product spaces.
\newblock
{\em Publ.\ Math.\ Inst.\ Hautes Etud.\ Sci.} 81:73--205 (1995).

\bibitem{Ver}
O.~Verbitsky.
\newblock
On the obfuscation complexity of planar graphs.
\newblock
{\it Theoretical Computer Science\/} 396(1--3):294--300 (2008).


\end{thebibliography}

\end{document}
