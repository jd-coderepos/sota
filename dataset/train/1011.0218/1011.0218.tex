\documentclass[submission,copyright,creativecommons]{eptcs}
\providecommand{\event}{Infinity 2010} \usepackage{breakurl}             \usepackage{times}
\usepackage{psfig}
\usepackage{graphics}
\usepackage{graphicx}
\usepackage{xspace}
\usepackage{amsmath,amssymb}
\usepackage{amsfonts}
\usepackage{algorithm}
\usepackage{algorithmic}
\usepackage{array}
\usepackage{multirow}
\usepackage{gastex}

\newtheorem{proposition}{Proposition}
\newtheorem{definition}{Definition}
\newtheorem{lemma}{Lemma}
\newtheorem{theorem}{Theorem}
\newtheorem{proof}{Proof}
\newtheorem{remark}{Remark}
\numberwithin{equation}{section}

\def\tapn{A-TPN}
\def\tppn{P-TPN}
\def\ttpn{T-TPN}
\def\ttappn{\{P,T,A\}-TPN}
\def\ttappnw{\{P,T,A\}_{W}-TPN}
\def\ttappns{\{P,T,A\}_{S}-TPN}
\def\tapnw{A_{W}TPN}
\def\tppnw{P_{W}TPN}
\def\ttpnw{T_{W}TPN}
\def\tapns{A_{S}TPN}
\def\tppns{P_{S}TPN}
\def\ttpns{T_{S}TPN}
\def\tgpn{A_{WS}TPN}


\title{On interleaving in \{P,A\}-Time Petri nets with strong semantics}

\author{Hanifa Boucheneb
\institute{Laboratoire VeriForm, \'{E}cole Polytechnique de Montr\'{e}al,
P.O. Box 6079,\\ Station Centre-ville, Montr\'{e}al, Qu\'{e}bec,Canada, H3C 3A7}
\email{hanifa.boucheneb@polymtl.ca}   \and Kamel Barkaoui \institute{Laboratoire CEDRIC,
Conservatoire National des Arts et M\'etiers, 292 rue Saint Martin, Paris Cedex 03, France }
\email{kamel.barkaoui@cnam.fr} }


\def\titlerunning{On interleaving in \{P,A\}-Time Petri nets}
\def\authorrunning{H. Boucheneb \& K. Barkaoui}
\begin{document}
\maketitle


\begin{abstract}
This paper deals with the reachability analysis of \{P,A\}-Time Petri nets (\{P,A\}-TPN in short) in the context of strong semantics. It investigates the convexity of the union of state classes reached by different interleavings of the same set of transitions. In \cite{infinity08}, the authors have considered the T-TPN model and its Contracted State Class Graph (CSCG) \cite{acsd07} and shown that this union is not necessarily convex. They have however established some sufficient conditions which ensure convexity. This paper shows that for the CSCG of \{P,A\}-TPN, this union is convex and can be computed without computing intermediate state classes. These results allow to improve the forward reachability analysis by agglomerating, in the same state class, all state classes reached by different interleavings of the same set of transitions (abstraction by convex-union).
\end{abstract}


\section{introduction}
Petri nets are established as a suitable formalism for modeling
 concurrent and dynamic systems. They are used in many fields (computer science, control systems, production systems, etc.). Several extensions to time factor have
been defined to take into account different features of the system as well as its time constraints. The time constraints may be expressed in terms of stochastic delays of transitions (stochastic Petri nets), fixed values associated with places or transitions (\{P,T\}-Timed Petri nets), or intervals labeling places, transitions or arcs (\{P,T,A\}-Time Petri Nets) \cite{khansa-wodes-96,Merlin,Walter}. For \{P,T,A\}-Time Petri Nets, there are two firing semantics: Weak Time Semantics (WTS) and Strong Time Semantics (STS). For both semantics, each enabled transition has an explicit or implicit firing interval derived from time constraints associated with places, transitions or arcs of the net. A transition cannot be fired outside its firing interval, but in WTS, its firing is not forced when the upper bound of its firing interval is reached. Whereas in STS, it must be fired within its firing interval unless it is disabled. The STS is the most widely used semantics. There are also multiple-server and single-server semantics. The multiple-server semantics allows to handle, at the same time, several time intervals per place (P-TPN), per arc (A-TPN) or per transition (T-TPN) whereas it is not allowed in the single-server semantics. \par In \cite{boyer-FI-08}, the authors have compared the expressiveness of \ttappn~
models with strong (,  and weak semantics ({}, ) (see Figure \ref{fig:Exemple}). They have established that\footnote{A Petri net is bounded iff the number of tokens in each reachable marking is bounded. It is safe iff the number of tokens in each reachable marking cannot exceed one.}:\begin{itemize}
\item For the single-server semantics, bounded \ttappn~and safe \ttappn~are equally expressive w.r.t. timed-bisimilarity and then w.r.t. timed language acceptance.
\item \ttpn~and \tppn~are incomparable models.
\item \tapn~ includes all the other models.
 \item The strong semantics includes the weak one for \tppn~  and
  \tapn, but not for \ttpn.
\end{itemize}
\begin{figure}[ht!]
\centering
\includegraphics[width=0.5\textwidth]{TimePN2.eps}
\caption{Comparison of the expressiveness of \{P,T,A\}-TPNs given in \cite{boyer-FI-08}}
\label{fig:Exemple}       \end{figure}
The reachability analysis of \ttappn~is, in general, based on abstractions preserving properties of interest (markings or linear properties). In general, in the abstractions preserving linear properties, we distinguish three levels of abstraction. In the first level, states reachable by time progression may be
either represented or abstracted. In the second level, states reachable by
the same sequence of transitions independently of their firing times are
agglomerated in the same node. In the third level, the
agglomerated states are considered modulo some equivalence relation:
 the firing domain of the state class graph (SCG) \cite{BVer03}, the
bisimulation relation over the SCG of the contracted state class graph (CSCG) \cite{acsd07}, the approximations of the zone based graph (ZBG) \cite{Bou09}). An abstract state is then an equivalence class of this relation. Usually, all states within an abstract state share the same marking and the union of their time domains is convex and defined as a conjunction of atomic constraints\footnote{An atomic constraint is of the form ,  or , where ,  are real valued variables representing clocks or delays,  and  is the set of rational numbers (for economy of notation, we use operator  even if
).}. From the practical point of view, the Difference Bound Matrices (DBMs) are a useful data structure for representing and handling efficiently sets of atomic constraints \cite{Bouyer06}.
\par The classical forward reachability analysis consists of computing, on-the-fly, all abstract states that are reachable from the initial abstract state. The reachability problem is known to be decidable for bounded \ttappn~but the reachability analysis suffers from the state explosion problem. For timed models, this problem is accentuated by the fact that, in the state space abstraction, a node represents, in fact, a finite/infinite set of states (abstract state) and interleavings of concurrent transitions lead, in general, to different abstract states.
\par To attenuate the state explosion problem, the reachability analysis is usually based on an abstraction by inclusion or by convex-union. During the construction of an abstraction, each newly computed abstract state is compared with the previously computed ones. In the abstractions by inclusion, two abstract states, with the same marking, having domains such that one is included in the other are grouped into one node. In the abstractions by convex-union, two abstract states, with the same marking, having domains such that their union is convex (and then can be represented by a single DBM), are grouped into one node. Convex-union abstractions are more compact than inclusion abstractions \cite{HadjBouc-STTT-08}. However, it is known that DBMs are not closed under union and the convex-union test is a very expensive operation relatively to the test of inclusion \cite{HadjBouc-STTT-08}. The convex-union
test of  (with ) abstract states  involves computing the smallest enclosing DBM  of their union, the
difference between  and ,
and finally checking that this difference is included in .
\par Another interesting reachability analysis approach, proposed in \cite{Maler06} for a CSS-like parallel composition of timed automata, consists of computing abstract states in breadth-first manner and at each level grouping, in one abstract state, all abstract states reached by different interleavings of the same set of concurrent transitions. The authors have shown that this union is convex, and then does not need any test of convexity. To use this approach in the context of \ttappn, we need to show that the union of abstract states reached by different interleavings of the same set of transitions is convex. In \cite{infinity08}, the authors have shown that for the T-TPN model, this union is not necessarily convex in the SCG and the CSCG. This paper shows that for the \tppn, this union is not necessarily convex in the SCG but is convex in the CSCG. Finally, it shows that these results are also valid for the A-TPN model.
\par The next section is devoted to the \tppn~
model, its semantics, its SCG, its CSCG, and the proof that the union of abstract states (i.e., state classes) reached by different interleavings of the same set of transitions is not necessarily convex in the SCG but is convex in the CSCG. Moreover, this union can be computed directly without computing beforehand intermediate state classes. Section 3 extends the results shown in Section 2 to the \tapn~model. Section 4 contains concluding remarks.

\section{P-Time Petri Nets}
In this paper, for reasons of clarity, we consider safe P-Time Petri nets.

\subsection{Definition and behavior} \label{semantics}
A P-Time Petri net is a Petri net augmented with time intervals
associated with places. Formally, a P-TPN is a tuple  where: \begin{enumerate} \item  and  are nonempty and finite sets of places and transitions such that (, \item  and  map each transition to its preset and postset (,  ,
\item  is the initial marking (, \item  is the static
residence interval function ,  is the set of nonnegative rational
numbers.  specifies the lower 
and the upper  bounds of the static residence
interval in place . \end{enumerate}

\par Let  be a marking and  a transition
of . Transition  is enabled for  iff all required
tokens for firing  are present in , i.e., . The firing of  from  leads to the marking . The set of transitions enabled for  is denoted ,
i.e., . A transition   is in conflict with  in  iff . The firing of  will disable .
\par In this model, a token may die. A token of place  dies when its interval becomes empty. Dead tokens will never be used and are considered as modeling flaws that should be avoided. To detect the dead tokens, we add a special transition named  whose role is limited to die tokens.
\par The P-TPN state is defined as a triplet , where  is a
marking,  is the set of dead tokens in  and  is the residence interval function . The initial state of the P-TPN
model is  where , ,
for all . When a token is created in place , its residence interval is set to its static residence interval . The bounds of this interval decrease synchronously with time, until the token of  is consumed or dies. A transition
 can fire iff all its input tokens are available, i.e., the lower bounds of their residence intervals have reached
, but must fire, without any additional delay, if the upper
bound of, at least, one of its input tokens reaches . The firing of a transition takes no time.
\par We define the P-TPN semantics as follows: Let  and  be two states of a P-TPN,  a nonnegative real number and  a transition of the net.\\ - \ We write , also denoted , iff the state
  is reachable from  state  by a time progression of
 units, i.e., , \ , \ , and
, .
The time progression is allowed while we do not overpass residence intervals of all non dead tokens. No token may die by this time progression.\\
- \ We write  iff  state  is immediately
reachable from state  by firing transition , i.e.,
,\ \ ,
 ,\ ,
 and ,   if  and  otherwise.\\
 \ - We write  iff  state  is immediately
reachable from state  by firing transition . Transition  is immediately firable from  if there exists no transition firable from  and there is, at least, a token in  s.t. the upper bound of its interval has reached  (token to die) i.e., , ,
 , ,
 and (, ).
\par According with the above semantics, states from which transition  is firable, are timelock states\footnote{A state  is a timelock state iff no progression of time is possible and no transition is firable from .}. Therefore, transition  allows to detect timelock states and dead tokens, and also to unblock the time progression.
\par The P-TPN state space is the timed transition system , where  is the initial state of
the P-TPN and 
 is the set of reachable states of the model,  being the reflexive and transitive closure  of the relation  defined above.\\  A \emph{run} in the  P-TPN state space , starting from a state , is a maximal sequence , such that . By convention, for any state , relation  holds.
 The sequence  is called the timed trace of . The sequence  is called the untimed trace of . Runs of the P-TPN are all runs starting from the initial state . Its timed (resp. untimed) traces are timed (resp. untimed) traces of its initial state.

\subsection{The SCG and CSCG of P-TPN}
The SCG of P-TPN is defined in a similar way as the SCG of T-TPN, except that time constraints are associated with places, and tokens may die. A SCG state class is defined as a triplet  where ,   is the set of dead tokens in  and  is a conjunction of atomic constraints\footnote{An atomic constraint is of the form , where ,  are
real valued variables,  and  is the set of rational numbers (for economy of
notation, we use operator  even if ).}
characterizing the union of the residence intervals of its non dead tokens. Each place  of  has  a variable denoted  in  representing
the residence delay of its token (i.e., the waiting time before its consummation or its death).
\par From the practical point of view,  is represented by a Difference Bound Matrix (DBM). The DBM of  is a square matrix  of order , indexed by variables of  and a special variable  whose value is fixed at . Each entry  represents the atomic constraint . Hence, entries  and  represent simple atomic constraints  and , respectively. If there is no upper bound on  with ,  is set to . Entry  is set to . Though the same nonempty domain may be represented by different DBMs, they have a unique form called canonical form. The canonical form of a DBM is the representation with tightest bounds on all differences between variables, computed by propagating the effect of each entry through the DBM. It can be computed in ,  being the number of variables in the DBM, using a shortest path algorithm, like Floyd-Warshall's all-pairs shortest path algorithm \cite{Bouyer06}. Canonical forms make operations over DBMs much simpler \cite{Ben02}.
\par The initial state class is  where  is the initial marking,  and .
\par Successor state classes are computed using the following firing rule \cite{BVer03}:
Let  be a state class and  a transition of . The state
class  has a successor by  (i.e.,
 iff
  and the following formula is
 consistent\footnote{A formula  is consistent iff there is, at least, one tuple of values that satisfies, at once, all constraints of .}: 
This firing condition means that  is enabled in  and there is a state s.t. the residence delay of each input token of  is less or equal to the residence delays of all non dead tokens in .\\
If  then  is computed as follows:\begin{enumerate} \item ;
 \item ;
 \item Set  to ;\item Rename, in ,  in , for all ; \item Add constraints: ; \item Replace each variable  by  (this substitution actualizes delays (old  = new ));
\item Eliminate by substitution .
\end{enumerate}
If  is firable then its firing consumes its input tokens and creates a token in each of its output places. Step 2) means that no token may die by firing . Step 3) isolates states of  from which  is firable. Note that this firing condition implies that  and then the firing delay  of  is equal to . Step 4) renames variables associated with tokens consumed by  in . Step 5) adds constraints of the created tokens. The residence interval of a token created by  is relative to the firing date of . Step 6) updates the delays of tokens not used by . Step 7) eliminates variable .
 \medskip \medskip
\begin{figure}\medskip  \footnotesize
\begin{picture}(140,30)(5,-35)
\centering
\gasset{Nw=5.0,Nh=5.0,Nmr=4,fillgray=1}
\gasset{ExtNL=y,NLdist=1,NLangle=90} \gasset{NLangle=90}
\node(P1)(30, -5) {} \node(P2)(40, -5) {}
\gasset{NLangle=30} \node(P3)(30, -25) {}  \node(P4)(40, -25)
{}
\gasset{ExtNL=y,NLdist=1,NLangle=0}
\gasset{Nw=6,Nh=0.7,Nmr=0,fillgray=0}
\node(T1)(30,-15) {} \node(T2)(40,-15) {} \node(T3)(30,-35) {}
\node(T4)(40,-35) {} \gasset{curvedepth=0} \drawedge(P1,T1){}
\drawedge(P2,T2){} \drawedge(T1,P3){} \drawedge(T2,P4){}
\drawedge(P3,T3){} \drawedge(P4,T4){} \gasset{NLangle=180}
 \nodelabel(P3){}\nodelabel(P1) {}
\gasset{NLangle=360} \nodelabel(P2)
{}\nodelabel(P4){}
\gasset{ExtNL=n,NLdist=0,NLangle=0} \nodelabel(P1){}
\nodelabel(P2){}
\gasset{Nw=0,Nh=0,Nmr=0,fillgray=0}
\node(a)(40,-42) {} \node(b)(100,-42){ }
\gasset{Nw=5.0,Nh=5.0,Nmr=4,fillgray=1}
\gasset{ExtNL=y,NLdist=1,NLangle=90} \gasset{NLangle=90}
\node(P0)(90, -5) {} \node(P2)(120, -5) {}  \gasset{NLangle=180} \node(P1)(80, -25) {}\node(P4)(100,-25) {}  \gasset{NLangle=360}  \node(P5)(110,-25) {} \node(P3)(130, -25) {}

\gasset{ExtNL=y,NLdist=1,NLangle=0}
\gasset{Nw=6,Nh=0.7,Nmr=0,fillgray=0}
\node(T1)(90,-15) {} \node(T2)(120,-15) {}
\node(T3)(80,-35) {} \node(T5)(105,-35) {} \node(T4)(130,-35) {}
\gasset{curvedepth=0} \drawedge(P0,T1){} \drawedge(P2,T2){}
\drawedge(T1,P1){} \drawedge(T2,P3){} \drawedge(P1,T3){}
\drawedge(P3,T4){} \drawedge(T1,P4){} \drawedge(T2,P5){}
\drawedge(P4,T5){} \drawedge(P5,T5){} \nodelabel(T1) {} \nodelabel(T2)
{} \gasset{NLangle=270} \nodelabel(T3){}
\nodelabel(T5){} \nodelabel(T4){}
\gasset{ExtNL=n,NLdist=0,NLangle=0} \nodelabel(P0){}
\nodelabel(P2){}
\end{picture} \medskip \medskip
\caption{P-TPNs used to illustrate features of the
interleaving in the SCG and the CSCG}  \normalsize \vspace{-5mm}
\end{figure}
\vspace{-3mm}
\par For example, consider the P-TPN shown in Figure 2.a). From its
initial SCG state class , transition  is firable from , since   is consistent. The firing of  leads to the state class . Its formula is derived from the firing condition of  from  as follows: rename  in , add the constraint , replace  and  by  and , respectively, and finally eliminate by substitution .
\par The transition  is firable from  iff there is no possibility to reach the intervals of input places of any enabled transition without overpassing the interval of a non dead token, i.e.,     is not consistent.\\  If  is firable from  (i.e., ), its firing leads to the state class  where: , ,   is not consistent ,  is obtained from  by eliminating by substitution all variables associated with places of  (i.e., by putting  in canonical form and eliminating all variables associated with places of ).
\par Let ,  be two state classes and  a
transition. We write 
iff . The SCG of the
\emph{P-TPN} is the structure  where  is the initial state class and
 is the set of reachable state classes. \par Note that dead tokens have no effect on the future behavior. Therefore, we can abstract dead tokens when we compare state classes. Two state classes  and  are said to be equal iff they have the same set of non dead tokens (i.e., ) and the DBMs of their formulas have the same canonical form (i.e., ).
\par In the same way as for the SCG of T-TPN \cite{BVer03}, we can prove that the SCG of P-TPN is finite and  preserves linear properties.
\par According to the firing rule given above, simple atomic constraints (i.e., atomic constraints of the form  or ) are not necessary to compute the successor state classes. It follows that all classes with the same triangular atomic constraints (i.e., atomic constraints of the form ) have the same firing sequences. They can be agglomerated into one node while preserving linear
properties of the model. This kind of agglomeration has been successfully used in \cite{acsd07} for the SCG of the \ttpn.
\par Formally, we define a bisimulation relation, denoted , over the SCG of the \tppn~by: , let  and  be the DBMs in canonical form of  and , respectively,   iff  \ and \ .
\par The CSCG of the \tppn~is the quotient graph of the SCG w.r.t. . A CSCG state class is an equivalence class of . It is defined as a triplet , where  is a conjunction of triangular atomic constraints. The initial CSCG state class is  where  is the initial marking,  and .
\par The CSCG state classes are computed in the same manner as the SCG state classes, except that step 6), of the firing rule given above, is not needed because the substitution of each  by  has no effect on triangular atomic constraints (). Steps 6) and 7) are replaced by:
Put the resulting formula  in canonical form and then eliminate all constraints containing .
\subsection{Interleaving in the P-TPN state class graph}
Note that transition , used to detect timelock states and dead tokens, cannot be concurrent to any transition of . So, there is no interleaving between  and transitions of .
\par Let us first show, by means of a counterexample, that the union of the SCG state classes of a P-TPN, reached by different interleavings of the same set of transitions of , is not generally convex.
\par Consider the P-TPN shown in Figure 2.a). From its
initial SCG state class , sequences  and  lead
respectively to the SCG state classes: \\  and \\ .\\ The union of domains of  and  is obviously not convex.
\par Consider now the CSCG of the same net. From its initial CSCG state class , sequences  and  lead to the CSCG state classes:\\  and , respectively.\\ The union of domains of  and  is convex .
\par We will show, in the following, that this result is always valid for the union of all the CSCG state classes reached by different interleavings of the same set of transitions. Let us first establish the firing condition of a sequence of concurrent transitions.

\begin{proposition} \label{prop}
Let  be a CSCG state class, and  a set of transitions enabled and not in conflict in ,  the set of all interleavings of transitions of  and .
The successor of  by  is non empty (i.e., ) \footnote{ is the set of all states reachable from any state of  by a timed run supporting .} iff the following formula, denoted , is consistent:
 

\end{proposition}

\begin{proof} By assumption, all transitions of  are not in conflict (i.e., ). The firing condition of the sequence  from  adds to  the firing constraints of transitions of the sequence (for ). We add for each transition   of the sequence, a variable, denoted , representing its firing delay. The added constraints consist of five blocks. The first block fixes the firing order of transitions of . The second block means that the residence delays of tokens used by each transition  must be equal to . The third and the fourth blocks mean that the firing delay  is less or equal to the residence delays of tokens that are present (and not dead) when  is fired (i.e.,  and ). The fifth block of constraints specifies the residence delays of tokens created by  (i.e., ). Note that  denotes the residence delay of the token  created by . \normalsize
\end{proof}
\par As an example, consider the P-TPN shown in Figure 2.b) and its
initial CSCG state class . The firing condition  of the sequence  is computed as follows:\\
1) Set  to ;\\
2) Add variables  and  and the constraint ;\\
3) Add constraints specifying the firing delays of  and :  ;\\
4) Add constraints of tokens created by : ;\\
5) Add constraints specifying that the firing delay of  is less or equal to the residence delays of the tokens created by : .\\
6) Add constraints of tokens created by : \\
Then: 
 
In the same manner, we obtain the firing condition  of the sequence  from : \\


  Since  and  , it follows that:\\
  
Formula  is the firing condition of  and  from , in any order. Its domain is convex (representable by a single DBM).
The following theorem (Theorem \ref{th1}) establishes that this result is valid for any set of transitions of  not in conflict and firable from a CSCG state class. The proof of this theorem follows the same ideas as those used in the previous example to show that  can be rewritten as a conjunction of atomic constraints.

\begin{theorem} \label{th1}
Let  be a CSCG state class and  a set of transitions firable from  and not in conflict in .\\ Then  and  is a state class  where ,  and  is a conjunction of triangular atomic constraints that can be computed as follows:
\begin{itemize}
\item set  to 

\item Put  in canonical form, then eliminate variables  and variables associated with their input places. \item Rename each variable , in  .
\end{itemize}
\end{theorem}

 \begin{proof} If transitions of  are all firable from  and not in conflict then the firing of one of them cannot disable the others. So, all sequences of  are firable from . Then: .
Let us first rewrite the firing condition , given in Proposition \ref{prop}, of the sequence , so as to isolate the part that is independent from the firing order. In other words, let us show that:
     
Consider the following sub-formula, denoted , of : 
This formula implies that:
(1) \ . \\
(2) \ .\\ Then: (2') .\\
(3) \ .\\ Then: (3') .\\
Consider now the following sub-formula, denoted , of : 
From (2'), it follows that constraints (2) are redundant in the part  of  and then can be eliminated from the part  of , without altering the domain of :

Let  be the following part of : 
From (3'), it follows that constraints (3) are redundant in the part  of  and then can be added to the part  of , without altering the domain of :

Therefore,      
We have rewritten the firing condition of the sequence  so as to isolate the part   fixing the firing order from the other part, which is independent of the firing order. It follows that the firing condition of transitions of  in any order, denoted , is:    
To obtain the formula of , it suffices to put  in canonical form and then eliminates variables associated with transitions of  and their input places. \normalsize
 \end{proof}
\par Theorem \ref{th1} is also valid for unsafe P-TPNs in the context of multiple-server semantics. The proof of this claim is similar, except that markings, presets and postsets of transitions are multisets over places. In this case, a variable is associated with each token (instead of each place). Transitions can be multi-enabled. Each enabling instance of a transition is defined as a couple composed by the name of the transition and the multiset of tokens participating in its enabling. Its firing delay depends on time constraints of its tokens. A variable is associated with each enabling instance of the same transition. In the next section, we will extend the result established in Theorem \ref{th1} to the A-TPN model.

\section{A-Time Petri Nets}
The A-TPN model is the most powerful model in the class of \{P,T,A\}-TPN \cite{boyer-FI-08}. Like in P-TPN, A-TPN uses the notion of availability intervals of tokens but each token of a place  has an availability interval per output arc of , whereas, in P-TPN, each token has only one availability interval. As for P-TPN, we consider, in the following, safe A-TPN.

\par Formally, A-TPN is a tuple  where: \begin{enumerate} \item , , ,  and  are defined as for P-TPN, \item Let  be the set of input arcs of all transitions.  is the static
availability interval function.  specifies the lower 
and the upper  bounds of the static availability
interval of tokens of  for . \end{enumerate}

\par Since, in A-TPN, intervals are associated with arcs connecting places to transitions, the notion of dead tokens of the P-TPN model is replaced by dead arcs. If a place  is marked and connected to a transition , the arc  will die if the residence time of the token of  overpasses the availability interval of the arc . To detect dead arcs, we use the special transition , as for the P-TPN model.

\par Let  be the set of enabled arcs in . The A-TPN state is defined as a triplet , where  is a
marking,  is the set of dead arcs in  and  is the interval function  which associates with each enabled and non dead arc an availability interval. The initial state of the A-TPN model is  where , , for all . When a token is created in place , the availability interval of each output arc  is set to its static interval  and then decreases,  synchronously with time, until the token within  is consumed or the arc dies. A transition  can fire iff all its input arcs are not dead and have reached their availability intervals, i.e., the lower bounds of the intervals of its input arcs have reached . But, it must fire, without any additional delay, if the upper bound of, at least, one of its input arcs has reached . The firing of a transition takes no time.
\par The A-TPN state space is the timed transition system , where  is the initial state of
the A-TPN and 
 is the set of reachable states of the model,  being the reflexive and transitive closure  of the relation  defined as follows.\\
Let  be two A-TPN states, ,\\ - \ , iff , \ ,  \ and \
. The time progression is allowed while we do not overpass intervals of all non dead arcs of .\\ - \  iff  state  is immediately reachable from state  by firing transition , i.e.,
,\ \ ,
 ,\ ,
 and ,   if  and 
 otherwise. It means that all input arcs of  are enabled, not dead and  have reached their availability intervals. The firing of  consumes tokens of its input places and produces tokens in its output places (one token per output place). The consumed tokens and their output arcs are removed. The produced tokens are added to the marking. The availability intervals of their output arcs are set to their static availability intervals.\\
- \  iff  state  is immediately
reachable from state  by firing transition . Transition  is immediately firable from  if there no transition of  firable from  and there is at least an arc in  s.t. the upper bound of its interval has reached  i.e., , ,
 , ,  and (, ).
\subsection{The CSCG of the A-TPN}
The definition of the CSCG of the P-TPN is extended to the A-TPN by replacing the notion of dead tokens by dead arcs and constraints on availability of tokens by those of arcs. The CSCG state class of A-TPN is defined as a triplet  where  is a marking,  is the set of dead arcs in  and  is a conjunction of triangular atomic constraints over variables associated with non dead arcs of . Each arc  of
 has a variable, denoted  in , representing
its availability interval.
\par \indent The initial CSCG state class is:  where  is the initial marking,  and .
\par Successor state classes are computed using the following firing rule:
Let  be a state class and  a transition of . The state
class  has a successor by  (i.e.,
 iff
  and the following formula is
 consistent: 
This firing condition means that  is enabled in , its input arcs are not dead, and there is a state s.t. the input arcs  of  will reach their intervals before overpassing intervals of all non dead arcs in . \\
If  then  is  computed as follows:\begin{enumerate} \item ;
 \item 
 \item Set  to ;\item Replace variables  associated with input arcs of  by ; \item Add constraints ;
\item Put  in canonical form and then eliminate  .
\end{enumerate}
\noindent If  is firable then its firing consumes its input tokens and creates tokens in its output places (one token per output place). The consumed tokens and their output arcs are eliminated. Step 3) isolates states of  from which  is firable (i.e., states where input arcs of  reach their availability interval before overpassing the availability intervals of all non dead enabled arcs). This step implies that for all . Step 4) replaces all these equal variables by .  Steps 5) adds the time constraints of the created tokens. Step 6) puts  in canonical form before eliminating variable .
\subsection{Interleaving in the CSCG of A-TPN}
The following theorem extends, to A-TPN, the result established in Theorem \ref{th1}.


\begin{theorem} \label{th2}
Let  be a CSCG state class and  a set of transitions firable from  and not in conflict in .\\ Then  and  is a state class  where ,  and  is a conjunction of triangular atomic constraints that can be computed as follows:
\begin{itemize}
\item Set  to 

\item Put  in canonical form, then eliminate variables  and variables associated with their input places. \item Rename each variable , in  .
\end{itemize}
\end{theorem}

  \begin{proof} We first extend the firing condition of a sequence  of  given in Proposition \ref{prop} to the case of A-TPN.  is firable from  (i.e., ) iff the following formula, denoted  is consistent:
 

The firing condition of the sequence  from  adds to  for each transition   of the sequence, a variable, denoted , representing its firing delay and five blocks of constraints. The first block fixes the firing order of transitions of . The second block means that the residence delays of arcs used by each transition  must be equal to . The third and the fourth blocks mean that the firing delay  is less or equal to the residence delays of all enabled and non dead arcs present when  is fired (i.e.,  and  s.t.  and ). The fifth block of constraints specifies the residence delays of arcs enabled by  (i.e., ). The rest of the proof follows the same steps as the proof of Theorem \ref{th1}. In other words, let us show that     
Consider the following sub-formula, denoted , of : 
This formula implies that:
(1) \ . \\
(2) \ .\\ Then: (2') .\\
(3) \ .\\ Then: (3') .\\
Consider the following sub-formula, denoted , of : 
From (2'), it follows that constraints (2) are redundant in the part  of  and then can be eliminated from the part  of , without altering the domain of :

Let  be the following part of : 
From (3'), it follows that constraints (3) are redundant in the part  of  and then can be added to the part  of , without altering the domain of :

Therefore,     
The firing condition of transitions of  in any order, denoted , is obtained by eliminating the part fixing the firing order. To obtain the formula of , it suffices to put  in canonical form and then eliminate variables associated with transitions of  and their input places. \normalsize
 \end{proof}

\par The extension of this result to unsafe A-TPN is straightforward by considering multisets of tokens, multisets of enabled arcs, and associating a variable with each instance of multiple enabled arcs. Each enabled transition is defined by the name of the transition and a set of enabled arcs.

\par Using the translation into A-TPN of the P-TPN shown in Figure 2.a), we prove that the union of the SCG state classes of the A-TPN reached by different interleavings of the same set of transitions is not necessarily convex\footnote{The P-TPN is translated into A-TPN by replacing the static residence interval function  by  defined by: .}. Indeed, its initial SCG state class , sequences  and  lead
respectively to the SCG state classes:  and . The union of their domains is not convex.

\section{Conclusion}
\noindent In this paper, we have considered the \tppn~ and \tapn~
models, their SCG and CSCG. We have investigated the convexity of the union of state classes
reached by different interleavings of the same set of transitions.
We have shown that this union is not convex in the SCG but is convex in the CSCG. This result
allows to use the reachability analysis approach proposed in \cite{Maler06}, which reduces the redundancy caused by the interleaving semantics.
\par This result is however not valid for the \ttpn~\cite{infinity08}, in spite of the fact that A-TPN is the most powerful model. This could be explained by the fact that the firing interval of a transition refers to the instant when it becomes enabled in \ttpn, whereas, in \{P,A\}-TPN, it is equal to the intersection of intervals of all its input tokens/arcs. In T-TPN, the firing interval can be related to the last transition of a sequence and then dependent of the firing order. For example, consider the net shown in Figure 2.b) and suppose that intervals attached to places are moved to be attached to their output transitions. The firing of transitions  and , in any order, will enable transition . But, the firing interval of  is related to  in , whereas it is related to  in . The union of the CSCG state classes reached by  and  from the initial state class is: .  Its domain is not convex.
\par Therefore, A-TPN is more powerful than T-TPN and also more suitable for abstractions by convex-union. However, the translation of T-TPN into A-TPN is not easy and needs to add several places and transitions \cite{boyer-FI-08}, which may offset the benefits of abstractions by convex-union. The choice of the appropriate \{P,T,A\}-TPN model for a given problem should be a good compromise between the easiness of modeling the problem and the verification complexity.
\par As immediate perspective, we will use the results established here and in \cite{infinity08} to investigate the extension, to \{P,T,A\}-TPN, of the reachability approach proposed in \cite{Myers} for a variant of safe P-TPN. In this variant, there are two kinds of places (behaviour and constraint places) and each transition can have at most one behaviour place in its preset. A transition is firable, if the age of its behaviour place reaches its static residence interval. It must be fired before overpassing this interval, unless it is disabled.

\begin{thebibliography}{10}\label{bibliography}
\bibitem{Bouyer06} G. Behrmann, P. Bouyer, K. G. Larsen, and R. Pel{\'a}nek \emph{Lower and
upper bounds in zone-based abstractions of timed automata},
International Journal on Software Tools for Technology Transfer
Volume 8(3), 2006.
\bibitem{Maler06} R. Ben Salah, M. Bozga and O. Maler, \emph{On Interleaving in Timed Automata}, CONCUR'06, 465-476,
volume 4137 of LNCS, 2006.
\bibitem{Ben02}  Bengtsson, J.: Clocks, DBMs and States in Timed Systems, \emph{PhD
thesis, Dept. of Information Technology, Uppsala University},
2002.
\bibitem{BVer03} B. Berthomieu and F. Vernadat, \emph{State class constructions for branching analysis of Time Petri nets},
volume 2619 of LNCS, 2003.
\bibitem{Bou09} H. Boucheneb, G. Gardey, and O.
(H.) Roux. \emph{TCTL model checking of time Petri nets}, Journal
of Logic and Computation, 19(6):1509-1540, December 2009.
\bibitem{infinity08} H. Boucheneb and K. Barkaoui,
\emph{Covering steps graphs of time Petri nets} In Proc. of the
10th International Workshop on Verification of Infinite-State
Systems (INFINITY), 2008.
\bibitem{acsd07} H. Boucheneb and H. Rakkay, \emph{A more efficient time Petri net state space abstraction useful to model checking timed linear
properties} In journal of Fundamenta Informaticae, volume 88,
number 4, pp 469-495, 2008.
\bibitem{boyer-FI-08} M. Boyer and O. H. Roux,\emph{On the compared expressiveness of arc, place and transition time {Petri} Nets}, journal Fundamenta Informaticae, vol. 88,  3, pages 225-249, 2008.
\bibitem{khansa-wodes-96} W. Khansa,  J.-P Denat and  S. Collart-Dutilleul,
 \emph{P-Time {P}etri Nets for manufacturing systems}, International Workshop on Discrete Event Systems, WODES'96, pp 94-102, 1996.
\bibitem{HadjBouc-STTT-08} R. Hadjidj and H. Boucheneb, \emph{Improving state class constructions for {CTL*} model checking of
Time Petri Nets}, International Journal on Software Tools
Technology Transfer (STTT), volume 10, number 2, pp 167-184, 2008.
\bibitem{Merlin} P. M. Merlin, \emph{A study of the recoverability of computing systems.}, Department of Information and Computer Science, University of California, Irvine CA, 1974.
\bibitem{Myers} C. J. Myers, T. G. Rokicki, T. H.-Y. Meng, \emph{POSET timing and its application to the timed cirucits}, in IEEE Transactions on CAD, 18(6), 1999.
\bibitem{Walter} B. Walter, \emph{Timed net for modeling and analysing protocols with time},
IFIP Conference on Protocol Specification Testing and Verification, 1983.
\end{thebibliography}


\end{document}
