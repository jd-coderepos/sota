



\documentclass{mcom-l}

\usepackage{amssymb}

\usepackage{graphicx}
\usepackage{threeparttable}
\usepackage{algorithmicx,algorithm}
\usepackage{algpseudocode}







\theoremstyle{definition}
\newtheorem{sec2lemma1x}{Lemma}[section]
\newtheorem{sec2lemma2x}[sec2lemma1x]{Lemma}
\newtheorem{sec2lemma3x}[sec2lemma1x]{Lemma}
\newtheorem{sec2lemma4x}[sec2lemma1x]{Lemma}


\newtheorem{sec3_remark1}{Remark}[section]
\newtheorem{sec3_remark2}[sec3_remark1]{Remark}
\newtheorem{sec3_remark3}[sec3_remark1]{Remark}
\newtheorem{sec3_remark3x}[sec3_remark1]{Remark}
\newtheorem{sec3_remark3xx}[sec3_remark1]{Remark}
\newtheorem{sec3_remark4}[sec3_remark1]{Remark}
\newtheorem{sec3_B_remark1}[sec3_remark1]{Remark}
\newtheorem{sec3_B_remark2}[sec3_remark1]{Remark}
\newtheorem{sec3_B_remark3}[sec3_remark1]{Remark}
\newtheorem{sec3_B_remark4}[sec3_remark1]{Remark}

\newtheorem{sec3Cremark1}[sec3_remark1]{Remark}
\newtheorem{sec3Cremark2}[sec3_remark1]{Remark}
\newtheorem{sec3Cremark3}[sec3_remark1]{Remark}
\newtheorem{sec3Cremark4}[sec3_remark1]{Remark}
\newtheorem{sec3Cremark5}[sec3_remark1]{Remark}
\newtheorem{sec3thm1}{Theorem}[section]
\newtheorem{sec3thm1x}[sec3thm1]{Theorem}
\newtheorem{sec3thm1xx}[sec3thm1]{Theorem}
\newtheorem{sec3thm2}[sec3thm1]{Theorem}
\newtheorem{sec3thm2aux1}[sec3thm1]{Theorem}
\newtheorem{sec3thm2aux2}[sec3thm1]{Theorem}
\newtheorem{sec3thm2aux3}[sec3thm1]{Theorem}
\newtheorem{sec3thm2x}[sec3thm1]{Theorem}
\newtheorem{sec3thm2xx}[sec3thm1]{Theorem}

\newtheorem{sec3thm3}[sec3thm1]{Theorem}
\newtheorem{sec3thm3x}[sec3thm1]{Theorem}
\newtheorem{sec3thm4}[sec3thm1]{Theorem}
\newtheorem{sec3thm5}[sec3thm1]{Theorem}
\newtheorem{sec3CThm01}[sec3thm1]{Theorem}
\newtheorem{sec3CThm02}[sec3thm1]{Theorem}
\newtheorem{sec3CThm03}[sec3thm1]{Theorem}
\newtheorem{sec3CThm04}[sec3thm1]{Theorem}
\newtheorem{sec3CThm05}[sec3thm1]{Theorem}
\newtheorem{sec3CThm06}[sec3thm1]{Theorem}
\newtheorem{sec3CThm07}[sec3thm1]{Theorem}


\newtheorem{sec3Bcor1}{Corollary}[section]
\newtheorem{sec3Bcor2}[sec3Bcor1]{Corollary}
\newtheorem{sec3Bcor3}[sec3Bcor1]{Corollary}
\newtheorem{sec3Bcor4}[sec3Bcor1]{Corollary}
\newtheorem{sec3Bcor5}[sec3Bcor1]{Corollary}

\newtheorem{sec3Ccor1}[sec3Bcor1]{Corollary}
\newtheorem{sec3Ccor2}[sec3Bcor1]{Corollary}
\newtheorem{sec3Ccor3}[sec3Bcor1]{Corollary}
\newtheorem{sec3Ccor4}[sec3Bcor1]{Corollary}
\newtheorem{sec3Ccor5}[sec3Bcor1]{Corollary}

\newtheorem{sec3exp1}{Example}[section]
\newtheorem{sec3exp2}[sec3exp1]{Example}
\newtheorem{sec3exp3}[sec3exp1]{Example}
\newtheorem{sec3exp4}[sec3exp1]{Example}
\newtheorem{sec3exp5}[sec3exp1]{Example}
\newtheorem{sec3Bexp1}[sec3exp1]{Example}

\newtheorem{sec3lemma1}{Lemma}[section]
\newtheorem{sec3lemma2}[sec3lemma1]{Lemma}
\newtheorem{sec3lemma3}[sec3lemma1]{Lemma}
\newtheorem{sec3lemma4}[sec3lemma1]{Lemma}
\newtheorem{sec3lemma5}[sec3lemma1]{Lemma}
\newtheorem{sec3lemma6}[sec3lemma1]{Lemma}
\newtheorem{sec3lemma7}[sec3lemma1]{Lemma}
\newtheorem{sec3lemma8}[sec3lemma1]{Lemma}
\newtheorem{sec3lemma9}[sec3lemma1]{Lemma}
\newtheorem{sec3lemma10}[sec3lemma1]{Lemma}

\numberwithin{equation}{section}
\begin{document}

\title[Linear Complexity of Generalized Cyclotomic Quaternary Sequences ]{On the Linear Complexity of Generalized Cyclotomic Quaternary Sequences with Length }


\author{Minglong Qi}
\address{School of Computer Science and Technology, Wuhan University of Technology, Mafangshan West Campus, 430070 Wuhan City, China}
\curraddr{}
\email{mlqiecully@163.com}


\author{Shengwu Xiong}
\address{}
\curraddr{}
\email{}


\author{Jingling Yuan}
\address{}
\curraddr{}
\email{}
\thanks{}

\author{Wenbi Rao}
\address{}
\curraddr{}
\email{}
\thanks{}

\author{Luo Zhong}
\address{}
\curraddr{}
\email{}
\thanks{}

\subjclass[2000]{Primary 54C40, 14E20; Secondary 46E25, 20C20}



\keywords{Linear complexity, generalized cyclotomic sequences, quaternary sequences, stream cipher, generating polynomial.}

\begin{abstract}
In this paper, the linear complexity over    of   generalized cyclotomic quaternary sequences with period   is determined, where  is an odd prime such that  and . The minimal value of the linear complexity  is equal to  which is greater than  the half of the period . According to the Berlekamp-Massey algorithm, these sequences are viewed as enough good for the use in cryptography. We show also that if the character of the extension field , , is chosen so that  , , and , then the linear complexity can reach the maximal value equal to the length of the sequences.
\end{abstract}
\maketitle
\section{Introduction}
\label{sec1}
Pseudo-random sequences play important roles in fields such as communication systems, simulation, and cryptography \cite{B1,B2}. In  cryptography, sequences with good balance and high linear complexity are preferable\cite{B2,B3,B4}. According to the Berlekamp-Massey algorithm\cite{B5}, if the linear complexity is greater than the half of the period, then the sequences are viewed as enough good for cryptographic uses. Because of their  good algebraic structure\cite{B6,B8}, cyclotomic sequences of different periods and orders find many applications in cryptography and communication \cite{B4,B7,B8,B9,B10}.

A family of generalized cyclotomic quaternary sequences of length  was constructed in \cite{B11}, of which the linear complexity and the autocorrelation were studied in \cite{B10} and in \cite{B12}, respectively. In \cite{B13}, Chang et al  considered the linear complexity of quaternary cyclotomic sequences with period .  Using the technique of Fourier spectrums of  sequences, the authors of \cite{B14} determined the linear complexity of generalized cyclotomic sequences with period . The linear complexity of Whiteman\textquoteright s generalized cyclotomic binary sequences with the period  was computed out in \cite{B20}. Very recently in \cite[June 2015]{B21}, Li et al studied the linear complexity of generalized cyclotomic binary sequences with the length . Because of easy and efficient hardware implementation, researches on  linear complexity and autocorrelation of generalized cyclotomic binary and quaternary sequences are intensive \cite{B7,B8,B9,B15,B16,B17}.

 In this paper, we consider the linear complexity over  of the generalized cyclotomic quaternary sequences with period  constructed in \cite{B13}. The rest of the paper is organized as follows: in Section 2, basic definitions, notations  and related  lemmas of previous works needed to prove the main results are given. In Section 3, we give main theorems of this paper and their proofs. In Section 4, we give a brief conclusion and remarks.
 
 \section{Preliminaries}
 Let  be two distinct odd primes,  be an odd prime such that  and ,  and  be the order of  modulo . Then, by the Chinese Remainder Theorem, there is a  root of unity in the extension field . Throughout the paper, we keep the meanings of , , and , unchanged. From here and hereafter, the meanings of the following symbols are kept unchanged:  are not only two distinct odd primes, but also , which implicates that, the generalized cyclotomic quaternary sequences with the period  is of order two. Let , and  be a  root of unity in . We use , , , and , to denote a , a , a  and a second root of unity in , respectively. We define ,  where , , , , and . It is clear that, , with . In addition, let  be a common odd primitive root of  and , and  be an integer which is computed by the following congruent system:
 
 If , then by the Generalized Chinese Remainder Theorem, 
 
 
 We define the following sets:
 
 From (\ref{sec2_eq_sets_1}), the following partitions are straightforward:
 
 
 Define four parameters as follows:
 
 
 Below is the definition of the generalized cyclotomic quaternary sequences of length  constructed by Chang et al in \cite{B13},  .
   
   In (\ref{lab_seq_chang}),  and , but . Let . Then, there exists always a  mapping from  to , , such that  and .  We thus obtain a particular instance of (\ref{lab_seq_chang}):
   
    In this paper, we study the linear complexity of the sequences defined in (\ref{lab_seq_chang_ex}) over .
    
     Let  be a sequence over  with period . Then, the linear complexity of  over , denoted by , is the least integer  which satisfies next recurrent relation:
      
    for , where .
    The minimal polynomial and the generating polynomial over  
    related to  are given below respectively
    
    The following equation relates both the minimal polynomial and the generating polynomial of the sequence :
    
    From (\ref{lab_eq_GCD_MiniPoly_ex}), the linear complexity can be calculated by
    
    
    In this paper, we focus on how to compute the degree of  in (\ref{lab_eq_GCD_LC_eq_ex}) over the extension field ,  for the sequence defined in (\ref{lab_seq_chang_ex}). We will compute out the number of zeros of the generating polynomial   under the form  for , from which  the degree of  can be deduced.
    
    We write down a series of lemmas of previous works related to and necessary for proof of the main theorems.
    \begin{sec2lemma1x}\label{lab_sec2_lemma1x}Let , and  be the parameters defined in (\ref{sec2_eq_parameters}).
     \begin{enumerate} \item If , and , then , and .
     \item If , and , then , and .
     \end{enumerate}
     \end{sec2lemma1x}
     \begin{proof}
     Since  is an odd prime such that  and , and  is the order of  modulo , hence . In other words,  and . In addition,  and  are the  root of unity and the  root of unity over the extension field , respectively. From    (\ref{sec2_eq_parameters}), , and . Lemma \ref{lab_sec2_lemma1x} (1) follows from (13) in \cite{B4}. The proof of Lemma \ref{lab_sec2_lemma1x} (2) is similar to that of Lemma \ref{lab_sec2_lemma1x} (1), and mentioned at first time in \cite{B18}.
     \end{proof}
    \begin{sec2lemma2x}\cite{B10,B13}\label{lab_sec2_lemma2x}
    \begin{enumerate} \item if , then .
    \item .
    \end{enumerate}
    \end{sec2lemma2x} 
    Where . 
    \begin{sec2lemma3x}\cite{B19}\label{lab_sec2_lemma3x}
     \begin{enumerate} \item 
     if only if , and  if only if , where .
     \item
     (Quadratic Reciprocity).
     
    \end{enumerate}
    \end{sec2lemma3x}
    \begin{sec2lemma4x}\cite{B14}\label{lab_sec2_lemma4x}
      and  if only if .
    \end{sec2lemma4x}
   \section{Computing the linear complexity according to various cases}
         For easy expression of some results, we give the definition of the sets which classify the pairs of . Let  and , then
         
         As we require that , it is clear that .
         
         Let  denote the Legendre symbol of  modulo . Next, we give a series of lemmas necessary to prove the  main theorems.
            \begin{sec3lemma1}\label{lab_sec3_lemma1}
            Define a mapping , such that for . Then, the following mappings deduced from the mapping  are bijective.
            \begin{enumerate} \item
            
            \item
            
            \end{enumerate}
            \end{sec3lemma1}
            \begin{proof}
            We only prove Lemma \ref{lab_sec3_lemma1} (1) since the proof for Lemma \ref{lab_sec3_lemma1} (2) is similar. At first, we prove the range of  is such that . Let , and . Without loss of generality, suppose , and we go to prove a contradiction. Since , there is a  such that , where . There must be a pair of integers  and , with  and , such that , and . From above discussion, we have the following congruent system:
            
            From (\ref{sec3_lemma1_proof_eq1}), it follows 
            
            Remark that (\ref{sec3_lemma1_proof_eq2}) implies that , which is contradictory to the predefined condition .
            
            By the Chinese Remainder Theorem, it is obvious that the mapping  is injective. We go to prove that  is surjective as well. Recall the meaning of the symbols used in the above proof. Let , and . Hence, ,  and . Resolve the following congruent system:
            
            By the Generalized Chinese Remainder Theorem, the solution of (\ref{sec3_lemma1_proof_eq3}) is equal to
            
            It is clear that .
            \end{proof}
            
        \begin{sec3lemma2}\label{lab_sec3_lemma2}
          Let  be a mapping , such that for . Then, the following mappings related to  are bijective.
          \begin{enumerate} \item
          
          \item
          
          \end{enumerate}
          \end{sec3lemma2}
          \begin{proof}
          The proof is simple and similar to that of Lemma \ref{lab_sec3_lemma1}.
          \end{proof}
          \begin{sec3lemma3}\label{lab_sec3_lemma3}
          Let  be a mapping , such that for . Then, the following mappings related to  are bijective.
          \begin{enumerate} \item
          
          \item
          
          \end{enumerate}
          \end{sec3lemma3}
          \begin{proof}
          It is obvious.
          \end{proof}
          \begin{sec3lemma4}\label{lab_sec3_lemma4}
          
          Recall that , and .
          \end{sec3lemma4}
          \begin{proof}
          It is clear that , and . The conclusion follows from Lemma \ref{lab_sec2_lemma3x} and Lemma \ref{lab_sec3_lemma1}.
          \end{proof}
          
           We define six integer functions in order to simplify notation.
              
              Where .
              \begin{sec3lemma5}\label{lab_sec3_lemma5}
              \begin{enumerate} \item
              
              \item
              
              \end{enumerate}
              Where .
              \end{sec3lemma5}
              \begin{proof}
              We give only the proof for case (1). By Lemma \ref{lab_sec3_lemma2} and Lemma \ref{lab_sec3_lemma1}, , with , , , and .
              
              \end{proof}
            \begin{sec3lemma6}\label{lab_sec3_lemma6}
            \begin{enumerate} \item
            
            \item
            
            \end{enumerate}
            Where .
            \end{sec3lemma6}
            \begin{proof}
            Using Lemma \ref{lab_sec3_lemma3} both for  and , and the same proof method as that for Lemma \ref{lab_sec3_lemma5}, we can obtain the conclusion of the actual lemma.
            \end{proof}
            \begin{sec3lemma7}\label{lab_sec3_lemma7}
            \begin{enumerate}
            \item
            If , then
            
            \item
             If , then
             
            \end{enumerate}
            \end{sec3lemma7}
            \begin{proof}
            We only prove the expansion formula for  in case (1).  Let  where , , and . We distinguish two cases  and . It is clear that  if , and  if , according to the Legendre Symbol definition. Same situation for . We use Lemma \ref{lab_sec3_lemma1} to decompose the summation over  into the summations over .
            \begin{enumerate}
            \item .
            
            Note that  and . Using Lemma \ref{lab_sec2_lemma2x}, we eliminate the exponents  and  from  above formula:
             
            \item  . Note that in this case,   and  . The rest of the proof is same as for .
            
            \end{enumerate}
            
            \end{proof}
                 
              
              
             
            By the definition of generating polynomial in (\ref{lab_eq_GCD_GP_ex}), we can  write down the one of the sequence defined in (\ref{lab_seq_chang_ex}):
            
            
            Since  is the  root of unity in the extension field , all the zeros of the polynomial  over  are under the form , with . If the same  is also the zero of the generating polynomial in (\ref{lab_seq_chang_generating_polynomial}), it simply means that  is a common factor of the  and  . By explicitly compute out all the zeros of the generating polynomial  under the form , we can determine the degree of the corresponding minimal polynomial, from which the linear complexity of the sequence in (\ref{lab_seq_chang_ex}) is straightforward.
            
            From (\ref{lab_seq_chang_generating_polynomial}) and by  applying Lemma \ref{lab_sec2_lemma2x}, Lemma \ref{lab_sec3_lemma5}-\ref{lab_sec3_lemma6} , we can get the explicit algebraic form of :
            
            Where .
            
   \subsection{ is a quadratic residue both modulo  and modulo }  
    
     Let  and . Next lemma gives the values of  for .
     \begin{sec3lemma8}\label{lab_sec3_lemma8}
     If , then the values of  where  can be determined as follows:
     \begin{enumerate} \item .
     
     \item .
     \begin{enumerate}
     \item If  and  and , then
     
     \item If  and  and , then
     
     \item If  and  and , then
     
     \item If  and  and , then
     
     \item If  and  and , then
     
     \item If  and  and , then
     
     \item If  and  and , then
     
     \item If  and  and , then
     
     \end{enumerate}
     \item .
     \begin{enumerate}
     \item
     If  and  or  and , then
     
     \item
     If  and  or  and , then
     
     \end{enumerate}
     \item .
     \begin{enumerate}
     \item
     If  and  or  and , then
     
     \item
     If  and  or  and , then
     
     \end{enumerate}
     \end{enumerate}
     \end{sec3lemma8}
     \begin{proof}
     Using Lemma \ref{lab_sec3_lemma4} and Lemma \ref{lab_sec2_lemma2x} to eliminate the exponents ,\  , and  from (\ref{lab_seq_chang_GP_in_beta_k}), we get a new form of (\ref{lab_seq_chang_GP_in_beta_k}):
     
     Where .
     
     In (\ref{lab_seq_in_beta_k_proof}), let , , , , and . It is easy to compute  the values of , and , for all , see below:
     
     
     After having computed  and , it is straightforward to find the values of . On the other hand, computation of  is trivial. Due to limited space and similarity of proof process, we consider only the case (3) of Lemma \ref{lab_sec3_lemma8}, that is, . 
     
     By Lemma \ref{lab_sec2_lemma2x}  if , and  if .
     
     Remark that if  , then , else . Next, we compute  for  . Let ,  where . It is obvious that , , , and . By Lemma \ref{lab_sec3_lemma7}, we distinguish two cases:
     \begin{enumerate}
     \item .
     
     Note that  since . Using Lemma \ref{lab_sec2_lemma2x}, possible combination for  and  gives the following values to :
     
     \item .
      After rather similar computation, we find that the final values of  in this case are the same as that listed in (\ref{lab_sec3_L3_prf_s4x}).
     \end{enumerate}
     
     From (\ref{lab_seq_in_beta_k_proof})-(\ref{lab_sec3_L3_prf_s4x}), we obtain
     
     
     Hence, for  we get 
     
     that correspond to the case (3) in Lemma \ref{lab_sec3_lemma8}.
     \end{proof} 
     
     We below list  main theorems of this subsection:
     \begin{sec3thm1}\label{lab_MainThorem_01}
     Let ,  and . Then, the linear complexity over  of the generalized cyclotomic quaternary sequences with period  specified in (\ref{lab_seq_chang_ex}), , can be determined according to next two cases.
     \begin{enumerate}
     \item  and  and  \textbf{OR}  and  and :
     
     \item  and  and :
     
     \end{enumerate}
     \end{sec3thm1}
     \begin{proof}
       Since  and , it means that  and . Hence, by Lemma \ref{lab_sec2_lemma4x}, . For Theorem \ref{lab_MainThorem_01} (1), we only prove the part where  and  and , since proof for another part is symmetric.


       Since ,  and . Because  and , by Lemma \ref{lab_sec2_lemma1x}, .
       
       
       If , by Lemma \ref{lab_sec3_lemma8} (3),  for  or . It means that  are  common factors of both  and , where  or . The number of the common factors is equal to . On the other hand, if , by Lemma \ref{lab_sec3_lemma8} (4),  for  or . Hence, there are 's common factors between   and , that are under the form  for  or .
       
       In order to find other zeros of  with , we define eight quantities that correspond to the eight sub-cases in Lemma \ref{lab_sec3_lemma8} (2):
       
       Note that for each , from Table \ref{Lab_Table1}, there are three sub-cases in Lemma \ref{lab_sec3_lemma8} (2) where   for , that contribute in total 's zeros to  for . If we examine closely those three sub-cases, we could remark that it is required that  which is equivalent to  according to the Quadratic Reciprocity Law (See Lemma \ref{lab_sec2_lemma3x} (2)).  
       
       From Lemma \ref{lab_sec3_lemma8} (1), it is obvious that  for . By (\ref{lab_eq_GCD_LC_eq_ex}), 
       
       and for Theorem \ref{lab_MainThorem_01} (2),
       
\end{proof}
      \begin{table}[!t]
       \begin{threeparttable}[t]
       \renewcommand{\arraystretch}{1.3}
       \caption{Values of  in Lemma \ref{lab_sec3_lemma8}(2)\tnote{1}
       \label{Lab_Table1}}
       \centering
       \begin{tabular}{c|c|c|c|c|c|c|c|c|c|c|c}
       \hline
       &  &  &  &  &  &  &  &  &  &  & \\
       \hline
       0 & -1 & 0 & -1 & -2 & -4 & 2 & 0 & -2 & 0 & -2 & 0\\
       \hline
       0 & -1 & -1 & 0 & -2 &  0 & -2 & 0 & -2 & -4 & 2 & 0\\
       \hline
       -1 & 0 & 0 & -1 & -2 &  0 & -2 & 0 & 2 & 0 & -2 & -4\\
       \hline
       -1 & 0 & -1 & 0 & 2 &  0 & -2 & -4 & -2 & 0 & -2 & 0\\
       \hline
       \end{tabular}
        \begin{tablenotes}
                   \item [1]  and .
        \end{tablenotes}
       \end{threeparttable}
       \end{table}
    \begin{algorithm}
      \caption{Algorithm to compute linear complexity}\label{lab_alg01}
      \begin{algorithmic}[1]
      \Procedure{LC}{}
      \State Construct the set .
      \State Construct the set .
      \State Construct the set .
      \State Construct  by (\ref{lab_seq_chang_generating_polynomial}) .
      \State Construct the polynomial .
      \State Compute .
      \State \textbf{return} .
      \EndProcedure
      \end{algorithmic}
      \end{algorithm}
    \begin{sec3exp1}\label{lab_sec3exp1}
      We implement Algorithm \ref{lab_alg01} using the computer algebra system Maple, in particular, using the Maple built-in function \textbf{Gcdex} to compute the gcd of  and the generating polynomial  defined in (\ref{lab_seq_chang_generating_polynomial}). Given a pair of primes  and , and  such that the conditions in Theorem \ref{lab_MainThorem_01} are satisfied, let  denote the linear complexity computed by Algorithm \ref{lab_alg01} and  denote the linear complexity calculated using the formula in Theorem \ref{lab_MainThorem_01}. In Table \ref{Lab_Table2}, some numerical examples are listed. We observe that the numerical result by running the procedure in Algorithm \ref{lab_alg01} coincide the ones predicted by Theorem \ref{lab_MainThorem_01}.
      \end{sec3exp1}
      \begin{table}[!t]
         \begin{threeparttable}[t]
         \renewcommand{\arraystretch}{1.3}
         \caption{Linear Complexity Calculated By Theorem \ref{lab_MainThorem_01}}
         \label{Lab_Table2}
         \centering
         \begin{tabular}{c|c|c|c|c|c}
         \hline
         &  &  &  &  & \textbf{Condition}\\
         \hline
         41 & 79 & 5 & 4079 & 4079 & \tnote{a} \\
         \hline
         113 & 167 & 7 & 23659 & 23659 &  \\
         \hline
         89 & 263 & 11 & 29347 & 29347 &  \\
         \hline
         79 & 41 & 5 & 4079 & 4079 & \tnote{b} \\
         \hline
         167 & 113 & 7 & 23659 & 23659 &  \\
         \hline
         263 & 89 & 11 & 29347 & 29347 &  \\
         \hline
         311 & 313 & 13 & 121835 & 121835 &  \\
         \hline
         79 & 239 & 5 & 37604 & 37604 & \tnote{c} \\
         \hline
         167 & 223 & 7 & 74288 & 74288 &  \\
         \hline
         103 & 311 & 13 & 63860 & 63860 &  \\
         \hline
         \end{tabular}
         \begin{tablenotes}
         \item [a]  and  and .
         \item [b]  and  and .
         \item [c]  and  and .
         \end{tablenotes}
         \end{threeparttable}
         \end{table}
   \begin{sec3thm1x}\label{lab_MainThorem_01x}
     Let ,  and . Then, the linear complexity over  of the generalized cyclotomic quaternary sequences with period  specified in (\ref{lab_seq_chang_ex}), , can be determined according to following cases.
      \begin{enumerate} \item If  and  and  and  \textbf{OR}  and  and  and , then
      
       \item If  and  and ( or ) and  \textbf{OR}  and  and ( or ) and , then
          
       \item If  and  and ( or ) and  \textbf{OR}  and  and ( or ) and , then
            
            \end{enumerate}
     \end{sec3thm1x}
     \begin{sec3_remark2}\label{Lab_sec3_remark2_A_thm2}
     In Theorem \ref{lab_MainThorem_01x}, if  and , by Lemma \ref{lab_sec3_lemma8} (1),  and  imply that  and  are zeros of   for , respectively. For  and , again by Lemma \ref{lab_sec3_lemma8} (1),  is zero of  if only if   and,  is zero of  if only if . It is obvious that  if only if both  and  hold for the case  and , or if only if both  and  hold for the case  and 
     \end{sec3_remark2}
     \begin{sec3_remark3}\label{Lab_sec3_remark3_A_thm2}
     In Table \ref{Lab_Table3_lemma8_2case2}, les values of  for  that correspond to the eight sub-cases in Lemma \ref{lab_sec3_lemma8} (2) are listed, where  and,   and  take arbitrary values but . If ,  for the sub-cases  and  in Lemma \ref{lab_sec3_lemma8} (2); if ,  for the sub-cases  and  in Lemma \ref{lab_sec3_lemma8} (2). For above two cases  and , from Lemma \ref{lab_sec3_lemma8} (2), we can observe that the condition  is required, that is equivalent to the condition  and  or  and  by the Quadratic Reciprocity law (see Lemma \ref{lab_sec2_lemma3x} (2)). In other words, for  or , if , then  for , has  zeros. But for ,  for  has no zeros.
     \end{sec3_remark3}
      \begin{table}[!t]
      \begin{threeparttable}[t]
        \renewcommand{\arraystretch}{1.3}
        \caption{Values of  in Lemma \ref{lab_sec3_lemma8}(2)\tnote{1}
        \label{Lab_Table3_lemma8_2case2}}
        \centering
        \begin{tabular}{c|c|c|c|c|c|c|c|c|c|c|c}
        \hline
        &  &  &  &  &  &  &  &  &  &  & \\
        \hline
        0 & -1 & *\tnote{2} & * & * & * & * & 0 & * & * & * & 0\\
        \hline
        -1 & 0 & * & * & * &  0 & * & * & * & * & * & 0\\
        \hline
        \end{tabular}
         \begin{tablenotes}
              \item [1]  and .
              \item [2] * means an arbitrary values.
             
              \end{tablenotes}
        \end{threeparttable}
        \end{table}
     \begin{proof}[Proof of Theorem \ref{lab_MainThorem_01x}]
     Since  and , by Lemma \ref{lab_sec2_lemma4x},  and . If  and  or  and , then by Lemma \ref{lab_sec2_lemma1x}, . From Lemma \ref{lab_sec3_lemma8} (3),  for  has  zeros. We next prove each case in Theorem \ref{lab_MainThorem_01x}.
     \begin{enumerate}
     \item Case  and  and  and  \textbf{OR}  and  and  and . From Remark \ref{Lab_sec3_remark2_A_thm2}, neither  nor  is a zero of  for . From Remark \ref{Lab_sec3_remark3_A_thm2},  for  has  zeros. In addition that there are  zeros for , the linear complexity could be directly computed out by 
     
     \item Case  and  and ( or ) and  \textbf{OR}  and  and ( or ) and . By above discussion, these conditions imply that  has an extra zero that is either  or , so the linear complexity equals
        
     \item Case  and  and ( or ) and  \textbf{OR}  and  and ( or ) and . These conditions imply that  has two extra zeros that are   and , so the linear complexity equals
           
     \end{enumerate}
     \end{proof}
     
       \begin{sec3thm1xx}\label{lab_MainThorem_01xx}
       Let , , , and . Then, the linear complexity over  of the generalized cyclotomic quaternary sequences with period  given in (\ref{lab_seq_chang_ex}),  can be determined according to following cases.
           \begin{enumerate} \item If    and  and , then
           
            \item If   and ( or ) and , then
               
            \item If   and ( or ) and , then
                 
                 \end{enumerate}
       \end{sec3thm1xx} 
        \begin{proof}
        Since , by Remark \ref{Lab_sec3_remark3_A_thm2},  for  has no zeros. The rest of the proof is similar to that for Theorem \ref{lab_MainThorem_01x}.
        \end{proof}
 \subsection{ is a quadratic non-residue both modulo  and modulo }  
    
  \begin{sec3lemma9}\label{lab_sec3_lemma9}
  Let . Then the values of  for  can be determined as follows:
  \begin{enumerate} \item .
  
  \item . Let   and .
  \begin{enumerate}
  \item
  If  and , then
   
  \item
  If  and , then
  
  \item
  If  and , then
   
  \item
  If  and , then
    
  \end{enumerate}
  \item .
  \begin{enumerate}
  \item
  If  and  or  and , then
  
  \item
  If  and  or  and , then
  
  \end{enumerate}
  \item .
  \begin{enumerate}
  \item
  If  and  or  and , then
  
  \item
  If  and  or  and , then
  
  \end{enumerate}
  \item .
  \begin{enumerate}
  \item
  If  and  or  and , then
  
  \item
  If  and  or  and , then
  
  \end{enumerate}
  \end{enumerate}
  \end{sec3lemma9}
  \begin{proof}
  It is clear that  and . By Lemma \ref{lab_sec3_lemma1} and Lemma \ref{lab_sec3_lemma4}, . By Lemma \ref{lab_sec2_lemma2x}, it can be deduced the following equations
  
  Substitute the group of above equations into (\ref{lab_seq_chang_GP_in_beta_k}), it leads to a new form for  with :
  
  The rest of the proof is similar to that of Lemma \ref{lab_sec3_lemma8}.
  \end{proof}  
  
   \begin{sec3Bcor1}\label{lab_sec3Bcor1}
   Let , ,  and . In addition, let  and  if  and ,  and  if  and  and,  and  if  and . Then, , where  , has exactly 's zeros.
   \end{sec3Bcor1}
   \begin{proof}
   Since  and , by Lemma \ref{lab_sec2_lemma4x}, . Using Lemma \ref{lab_sec2_lemma1x}, for the case where  and  with  and , resolve next congruent system:
   
    Now consider the case where  and  with  and  and next congruent system:  
    
   And finally for the case where  and  with  and , consider the following  congruent system: 
    
    By a simple modular arithmetic computation, the congruent systems (\ref{lab_sec3Bcor1_eq1}), (\ref{lab_sec3Bcor1_eq2}) and (\ref{lab_sec3Bcor1_eq3}) give rise to the same solutions:
    
    and
    
    If  or , then by Lemma \ref{lab_sec3_lemma9} (4), , where , has 'zeros. On the other hand, if  or , then by Lemma \ref{lab_sec3_lemma9} (5), , where , has 'zeros. From above discussion, (\ref{lab_sec3Bcor1_eq4_solA}) and (\ref{lab_sec3Bcor1_eq4_solB}),  has exactly 's zeros, where .
   \end{proof}
    \begin{sec3Bcor2}\label{lab_sec3Bcor2}
    Let , ,  and . In addition, let  and  if  and ,  and  if  and  and,  and  if  and . Then,  has exactly 's zeros, where .
    \end{sec3Bcor2}
   \begin{proof}
   By Lemma \ref{lab_sec2_lemma4x}, . By Lemma \ref{lab_sec2_lemma1x}, , since . By Lemma \ref{lab_sec3_lemma9} (3),  or . In other words,   has  's zeros, where .
   \end{proof}
    \begin{sec3Bcor3}\label{lab_sec3Bcor3}
    Let , ,  and . In addition, let  and  if  and ,  and  if  and  and,  and  if  and . If , then  has at least 's zeros, where .
    \end{sec3Bcor3}
    \begin{proof}
    Remark that  and  imply  and . By Corollary \ref{lab_sec3Bcor1} and \ref{lab_sec3Bcor2}, it follows the result of Corollary \ref{lab_sec3Bcor3}.
    \end{proof}
    \begin{sec3Bcor4}\label{lab_sec3Bcor4}
      Let , ,  and . In addition, let  and  if  and ,  and  if  and  and,  and  if  and .  Then  has  's zeros, where 
      \end{sec3Bcor4}
      \begin{proof}
      By Lemma \ref{lab_sec2_lemma4x}, . By Lemma \ref{lab_sec2_lemma1x} and using the same proof method as for Corollary \ref{lab_sec3Bcor1}, it can be deduced the following equations:
       
      Substitute the above equations into Lemma \ref{lab_sec3_lemma9} (2), it leads to that in one of four sub-cases, . In other words, there are 's zeros for  where .
      \end{proof} 
       \begin{sec3thm2}\label{lab_MainThorem_02}
         Let , ,  and . In addition, let  and  if  and ,  and  if  and  and,  and  if  and . Then, the linear complexity over  of the generalized cyclotomic quaternary sequences with period  specified in (\ref{lab_seq_chang_ex}) can be determined according to following cases:
          \begin{enumerate} \item if  and  \textbf{OR}  and .
          
\item If  and  \textbf{OR}  and  .
          
\item If , then 
           
\end{enumerate}
      \end{sec3thm2}
      
      \begin{sec3_B_remark1}\label{Lab_sec3_B_remark1}
      Follow the context in Theorem \ref{lab_MainThorem_02}, and let   and  for the case where  and . It is easy to check that if  then , and if  then . By Lemma \ref{lab_sec3_lemma9} (1),  implies  that  is a zero of , and  leads to that  is a zero of  as well, where .
      \end{sec3_B_remark1}
      
      \begin{sec3_B_remark2}\label{Lab_sec3_B_remark2}
      Consider the context in Theorem \ref{lab_MainThorem_02}. Let  and  if  and , and  and  if  and . It is straightforward to verify that if  then , and if  then . Combining Remark \ref{Lab_sec3_B_remark1},  \ref{Lab_sec3_B_remark2}, and the condition   or , it prevents that  and  are zeros to , where , and the case where .
      \end{sec3_B_remark2}
      
      \begin{proof}[Proof of Theorem \ref{lab_MainThorem_02}]
      We prove each case of Theorem \ref{lab_MainThorem_02}.
      \begin{enumerate}
      \item By Remark \ref{Lab_sec3_B_remark2} and Corollary \ref{lab_sec3Bcor1},  has exactly 's zeros, where . By (\ref{lab_eq_GCD_LC_eq_ex}), 
      
      \item By Remark \ref{Lab_sec3_B_remark1}, either  or  but not both is a zero. Hence, again by Corollary \ref{lab_sec3Bcor1},  has exactly 's zeros, where . Hence,
      
      \item By Corollary \ref{lab_sec3Bcor3},  has exactly 's zeros, where . By (\ref{lab_eq_GCD_LC_eq_ex}), 
      
      \end{enumerate}
      \end{proof}
    
\begin{sec3thm2aux1}\label{lab_sec3thm2aux1}
  Let , ,  and . In addition, let  and  if  and ,  and  if  and  and,  and  if  and . Then, the linear complexity over  of the generalized cyclotomic quaternary sequences with period  specified in (\ref{lab_seq_chang_ex}) can be determined according to following cases:
  \begin{enumerate} \item if  and   \textbf{OR}  and , then
    
\item If  and   \textbf{OR}  and , then
     
\end{enumerate}
  \end{sec3thm2aux1}
  
  \begin{sec3_B_remark3}\label{Lab_sec3_B_remark3}
  In Theorem \ref{lab_sec3thm2aux1}, for the case where , if , then , and if , then . In  other words, if , then either  or  but not both is a zero of  , using Lemma \ref{lab_sec3_lemma9} (1). For the case , if , then . That is,  is a zero of . Where 
  \end{sec3_B_remark3}
  \begin{proof}[Proof of Theorem \ref{lab_sec3thm2aux1}]
  Using Remark \ref{Lab_sec3_B_remark3} and Corollary \ref{lab_sec3Bcor4}.
  \end{proof}
  
  \begin{sec3thm2aux2}\label{lab_sec3thm2aux2}
  Let , ,  and . In addition, let  and  if  and ,  and  if  and  and,  and  if  and . Then, the linear complexity over  of the generalized cyclotomic quaternary sequences with period  specified in (\ref{lab_seq_chang_ex}) can be determined according to following cases:
  \begin{enumerate} \item if  and  \textbf{OR}  and , then
      
\item If , then 
       
\end{enumerate}
    \end{sec3thm2aux2}
  \begin{proof}
   Using  Corollary \ref{lab_sec3Bcor2} and \ref{lab_sec3Bcor3}.
   \end{proof}  
     \begin{sec3thm2x}\label{lab_MainThorem_02x}
       Let , ,  and . In addition, let  if  ,   if . Then, the linear complexity over  of the generalized cyclotomic quaternary sequences with period  specified in (\ref{lab_seq_chang_ex}) can be determined according to following cases:
     \begin{enumerate} \item if  and  and  \textbf{OR }   and  and , then 
     
\item If  and ( or ) and  \textbf{OR}  and ( or ) and , then 
      
\item If  and ( or ) and  \textbf{OR}  and ( or ) and , then 
         
\end{enumerate}
       \end{sec3thm2x}
   \begin{sec3_B_remark4}\label{Lab_sec3_B_remark4}
   In Theorem \ref{lab_MainThorem_02x}, the condition  or  for the case where  and , and the one   or  for the case where  and  imply that by Lemma \ref{lab_sec3_lemma9} (1),  or  is a zero of  for , respectively. It is obvious that if , then  implies , vice versa, and  implies , vice versa.
   \end{sec3_B_remark4}
   \begin{proof}[Proof of Theorem \ref{lab_MainThorem_02x}]
     Using  Corollary \ref{lab_sec3Bcor1} and Remark \ref{Lab_sec3_B_remark4}.
   \end{proof}  
   \begin{sec3Bexp1}\label{lab_sec3Bexp1}
   We give some pairs of  which match the condition that is  and ( or ) and , and compute the linear complexity both by Algorithm \ref{lab_alg01} and the formula in Theorem \ref{lab_MainThorem_02x} (3). We found that numerical and theoretical results are identical.
   
   \end{sec3Bexp1}
    For example, for all 
    
      has same number of zeros which equals , where , but for each pair of , the linear complexity is different, that is equal to . For instance, if  and , then .
    \subsection{ is a quadratic non-residue  modulo  and a quadratic residue modulo }
    
    \begin{sec3lemma10}\label{lab_sec3_lemma10}
    Let  and . For , let  and . Then, the values of  can be determined according to various range of .
    \begin{enumerate} \item .
    
    \item .
    \begin{enumerate}
    \item
    If  and  and  or  and  and , then
     
    \item
    If  and  and  or  and  and , then
     
    \item
    If  and  and  or  and  and , then
     
    \item
    If  and  and  or  and  and , then
       
    \end{enumerate}
    \item .
    \begin{enumerate}
    \item
    If  and  and  or  and  and , then
     
    \item
    If  and  and  or  and  and , then
      
    \end{enumerate}
    \item .
    \begin{enumerate}
    \item
    If   and  and  or   and  and  or   and  and  or   and  and , then
    
    \item
    If   and  and  or  and  and  or   and  and  or  and  and , then
    
    \end{enumerate}
     \item .
       \begin{enumerate}
       \item
       If   and  and  or   and  and  or   and  and  or   and  and , then
       
       \item
       If   and  and  or   and  and  or   and  and  or   and  and , then
       
       \end{enumerate}
    \end{enumerate}
    \end{sec3lemma10}
    \begin{proof}
    Similar to the proof of Lemma \ref{lab_sec3_lemma8}.
    \end{proof} 
    \begin{sec3Ccor1}\label{lab_sec3Ccor1}
    Let , ,  and . In addition, let   if , and  if . Then,  has exactly 's zeros, where .
    \end{sec3Ccor1}
    \begin{proof}
      As for the proof of Corollary \ref{lab_sec3Bcor1}, from the conditions given in the actual Lemma, we can obtain  and  or . From Lemma \ref{lab_sec3_lemma10} (2),  for  in one of the four sub-cases, that contributes  zeros. By the same way, we get  zeros for  from Lemma \ref{lab_sec3_lemma10} (3), and  zeros for  from Lemma \ref{lab_sec3_lemma10} (5), of . Hence, the number of zeros of  where  equals 
      
    \end{proof} 
    \begin{sec3Ccor2}\label{lab_sec3Ccor2}
       Let , ,  and . In addition, let   if , and  if . Then,  has exactly 's zeros, where .
    \end{sec3Ccor2}
    \begin{proof}
    By Lemma \ref{lab_sec2_lemma4x}, . By Lemma \ref{lab_sec2_lemma1x} and referring to the proof of Corollary \ref{lab_sec3Bcor1}, we obtain  and  or . Looking at Lemma \ref{lab_sec3_lemma10} (4) with , we remark that  for  in one of the two sub-cases, that contributes  zeros. On the other hand, the condition  leads to  zeros as proved in Corollary \ref{lab_sec3Ccor1}. Hence, the number of zeros of  where  equals 
          
    \end{proof} 
    \begin{sec3Ccor3}\label{lab_sec3Ccor3}
    Let , ,  and . In addition, let   if . Then,  has exactly 's zeros, where .
    \end{sec3Ccor3}
    \begin{proof}
    Since  and , by Lemma \ref{lab_sec2_lemma4x},  and . The rest of proof is similar to that of Corollary \ref{lab_sec3Ccor2}.
    \end{proof} 
    \begin{sec3Ccor4}\label{lab_sec3Ccor4}
       Let , ,  and . In addition, let   if . Then,  has exactly 's zeros, where .
       \end{sec3Ccor4}
     \begin{proof}
     See the proof for Corollaries \ref{lab_sec3Ccor1}-\ref{lab_sec3Ccor3}.
     \end{proof} 
     \begin{sec3Ccor5}\label{lab_sec3Ccor5}
     Let  and . Then, there are no  with  such that  and .
     \end{sec3Ccor5}
     \begin{proof}
     The conditions  and  imply that . But it is impossible that  occurs. If , by the Quadratic Reciprocity Law in Lemma \ref{lab_sec2_lemma3x} (2), .With  , it leads to . Hence, , that is a contradiction.    
     \end{proof} 
     \begin{sec3CThm01}\label{lab_sec3CThm01}
    Let , ,  and . In addition, let   if , and  if .
    Then, the linear complexity over  of the generalized cyclotomic quaternary sequences with period  specified in (\ref{lab_seq_chang_ex}) can be determined according to following cases:
     \begin{enumerate} \item if  and  \textbf{OR}  and  , then
     
    
      \item If  and  \textbf{OR}  and , then
     
      \end{enumerate}
     \end{sec3CThm01}
     \begin{sec3Cremark1}\label{Lab_sec3Cremark1}
     We consider two particular conditions that are  for the case where , and  for the case where  in Theorem \ref{lab_sec3CThm01}, respectively. It is easy to check that, for the two cases if   and  respectively, then . In other words, by Lemma \ref{lab_sec3_lemma9} (1),  is a zero for  where .
     \end{sec3Cremark1}
     \begin{proof}[Proof of Theorem \ref{lab_sec3CThm01}]
     Using Corollary \ref{lab_sec3Ccor1} and (\ref{lab_eq_GCD_LC_eq_ex}) to prove Theorem \ref{lab_sec3CThm01} (1), and  Corollary \ref{lab_sec3Ccor1}, (\ref{lab_eq_GCD_LC_eq_ex}), and Remark \ref{Lab_sec3Cremark1} to prove Theorem \ref{lab_sec3CThm01} (2).
     \end{proof}
      
       \begin{sec3CThm02}\label{lab_sec3CThm02}
         Let , ,  and . In addition, let   if , and  if .
         Then, the linear complexity over  of the generalized cyclotomic quaternary sequences with period  specified in (\ref{lab_seq_chang_ex}) can be determined according to following cases: 
      \begin{enumerate} \item if  and   \textbf{OR}  and  , then 
      
      
        \item If  and   \textbf{OR}  and  , then 
    
      
    \end{enumerate}
     \end{sec3CThm02}
     \begin{sec3Cremark2}\label{Lab_sec3Cremark2}
     In Theorem \ref{lab_sec3CThm02},   implies that  is a zero of  where  for . While for ,  implicates that  is a zero, and  implies that  is a zero, of  where  .
     \end{sec3Cremark2}
     \begin{proof}[Proof of Theorem \ref{lab_sec3CThm02}]
         Using Corollary \ref{lab_sec3Ccor2} and (\ref{lab_eq_GCD_LC_eq_ex}) to prove Theorem \ref{lab_sec3CThm02} (1), and  Corollary \ref{lab_sec3Ccor2}, (\ref{lab_eq_GCD_LC_eq_ex}), and Remark \ref{Lab_sec3Cremark2} to prove Theorem \ref{lab_sec3CThm02} (2).
         \end{proof}
     
    \begin{sec3CThm03}\label{lab_sec3CThm03}
     Let , ,  and . In addition, let   if . Then, the linear complexity over  of the generalized cyclotomic quaternary sequences with period  given in (\ref{lab_seq_chang_ex}) can be determined according to following cases:
     \begin{enumerate} \item if  and  and  \textbf{OR}  and  and , then 
     
     
       \item If  and ( or ) and  \textbf{OR}  and ( or ) and , then
      
       
         \item If   and ( or ) and  \textbf{OR}  and ( or ) and , then
        
       \end{enumerate}
        \end{sec3CThm03}
    \begin{sec3Cremark3}\label{Lab_sec3Cremark3}
    In Theorem \ref{lab_sec3CThm03}, for ,  and  imply that  and  are zeros of  where , respectively. While for ,  is a zero of  if only if , and  is a zero of  if only if . It is obvious that if , then  implicate , vice versa.
    \end{sec3Cremark3}
    \begin{proof}[Proof of Theorem \ref{lab_sec3CThm03}]
            Using Corollary \ref{lab_sec3Ccor3} and (\ref{lab_eq_GCD_LC_eq_ex}) to prove Theorem \ref{lab_sec3CThm02} (1), and  Corollary \ref{lab_sec3Ccor3}, (\ref{lab_eq_GCD_LC_eq_ex}), and Remark \ref{Lab_sec3Cremark3} to prove Theorem \ref{lab_sec3CThm03} (2), and (3).
     \end{proof}
    \begin{sec3CThm04}\label{lab_sec3CThm04}
    Let , ,  and . In addition, let  if . Then, the linear complexity over  of the generalized cyclotomic quaternary sequences with period  given in (\ref{lab_seq_chang_ex}) can be determined according to following cases: 
    \begin{enumerate}
     \item if  and  and  \textbf{OR}   and  and , then 
           
    \item If  and (  or ) and  \textbf{OR}  and (  or ) and , then
             
    \item If   and ( or ) and  \textbf{OR }   and ( or ) and , then
                  
            \end{enumerate}
    \end{sec3CThm04}
    \begin{sec3Cremark4}\label{Lab_sec3Cremark4}
    In Theorem \ref{lab_sec3CThm04}, the conditions  and  signify that   and  are zeros of   for  if , respectively. If , the same conditions change into  and , respectively. It is clear that if , then  implicate , vice versa.
    \end{sec3Cremark4}
    \begin{proof}[Proof of Theorem \ref{lab_sec3CThm04}]
    Using Corollary \ref{lab_sec3Ccor4} and (\ref{lab_eq_GCD_LC_eq_ex}) to prove Theorem \ref{lab_sec3CThm04} (1), and  Corollary \ref{lab_sec3Ccor4}, (\ref{lab_eq_GCD_LC_eq_ex}), and Remark \ref{Lab_sec3Cremark4} to prove Theorem \ref{lab_sec3CThm04} (2), and (3).
    \end{proof}
    
    \begin{sec3CThm05}\label{lab_sec3CThm05}
     Let  and . Then, the linear complexity over  of the generalized cyclotomic quaternary sequences with period  specified in (\ref{lab_seq_chang_ex}) can be determined according to following cases:
     \begin{enumerate}
     \item if  and , then 
      
      \item If  or , then 
          
     \end{enumerate}
        
    \end{sec3CThm05}
   \begin{proof}
   Since  and , by Lemma \ref{lab_sec2_lemma4x}, . From Lemma \ref{lab_sec3_lemma9},  has no zeros for , except for two possible zeros which are  and , respectively. By Corollary \ref{lab_sec3Ccor5}, no  such that both  and  hold together.
   \end{proof}
 
    
 \section{Conclusion}
 In this paper, we computed out the linear complexity over   of the  generalized cyclotomic quaternary sequences of length  specified in (\ref{lab_seq_chang_ex}). The results show that the minimal value of the linear complexity  is equal to  which is greater than , the half of the period of the sequences. According to the Berlekamp-Massey algorithm \cite{B5}, these sequences are viewed as enough good for the use in cryptography. Remark that if the character of the extension field , , is chosen so that , , , , and , then by Theorem \ref{lab_sec3CThm05}, the linear complexity can reach the maximal value which is the length of the sequences, .


\bibliographystyle{amsplain}
\begin{thebibliography}{99}\bibitem{B1} M. K. Simon, J. K. Omura, R. A. Schotz,  B. K. Levitt, Spread Spectrum Communications, vol. 1, Rockville MD: Computer Science Press, 1985.
\bibitem{B2} I. B. Damgard,  On the randomness of Legendre and Jacobi sequences, Advances in Cryptograpy(CRYPTO’88) LNCS 403 (1990) 163-172, Springer-Verlag, Berlin.
\bibitem{B3} X. Yang, C. Ding,  New classes of balanced quaternary sequences and almost balanced binary sequences with optimal autocorrelation value, IEEE Trans. Inf. Theory 56 (12)  (2010) 6398-6405.

\bibitem{B4} C. Ding, Cyclic codes from cyclotomic sequences of order four, Finite Fields and Their Appl. 23 (2013) 8-24.

\bibitem{B5} J. L. Massey, Shift register synthesis and BCH decoding, IEEE Trans. Inf. Theory IT-15 (1) (1969) 122-127.

\bibitem{B6}
A. L. Whiteman,  A family of difference sets, Illinois J. Math. (1962) 107-121.

\bibitem{B7} C. Ding,  Linear comlexity of generalized cyclotomic binary sequence of order 2, Finite Fields and Their Applications 3 (1997) 159-174.

\bibitem{B8} C. Ding,  Autocorrelation values of generalized cyclotomic sequences of order two, IEEE Trans. Inf. Theory 44 (5) (1998) 1699-1702.

\bibitem{B9} C. Ding, T. Helleseth, W. Shan, On the linear complexity of Legendre sequences, IEEE Trans. Inf. Theory 44 (3) (1998) 1276-1278.

\bibitem{B10} X. Du, Z. Chen,  Linear complexity of quaternary sequences generated using generalized cyclotomic classes modulo , IEICE Trans. Fundamentals EA94-A (5) (2011) 1214-1217.
\bibitem{B11} Y. J. Kim, Y. P. Hong,  H. Y. Song, Autocorrelation of some quaternary cyclotomic sequences of length , IEICE Trans. Fundamentals E91-A (12) (2008) 3679-3684.

\bibitem{B12} P. Ke, Z. Yang,  J. Zhang, On the autocorrelation and linear complexity of some  periodic quaternary cyclotomic sequences over , IEICE Trans. Fundamentals EA94-A (11) (2011) 2472-2477.

\bibitem{B13} Z. Chang,  D. Li,  On the linear complexity of quaternary cyclotomic sequences with the period , IEICE Trans. Fundamentals EA97-A (2) (2014) 679-684.
\bibitem{B14} D. Li, Q. Wen, J. Zhang, Z. Chang, Linear complexity of generalized cyclotomic quaternary sequences with period , IEICE Trans. Fundamentals E97-A (5)  (2014)  1153-1158.

\bibitem{B15} L. Zhao, Q. Wen,  J. Zhang,   On the linear complexity of a class of quaternary sequences with low autocorrelation, IEICE Trans. Fundamentals EA96-A (5) (2013) 997-1000.

\bibitem{B16} J. Zhang, C. Zhao,  X. Ma,  On the linear complexity of generalized cyclotomic binary sequences with length , IEICE Trans. Fundamentals EA93-A (1) (2010) 302-308.

\bibitem{B17} X. Du,  Z. Chen,  Trace Representation of Binary Generalized Cyclotomic Sequences with Length , IEICE Trans. Fundamantals E94-A (2) (2011) 761-765.

\bibitem{B18} Q. Wang, D. Lin,  X. Guang, On the linear complexity of Legendre sequences over , IEICE Trans. Fundamantals E97-A (7) ( 2014) 1627-1630.

\bibitem{B19} D. M. Burton, Elementary Number Theory, fourth ed., McGraw-Hill International Editions, 1998.

\bibitem{B20}
L. Q. Hu, Q. Yue,  M. H. Wang, The linear complexity of Whiteman\textquoteright s generalized cyclotomic binary sequences with the period , IEEE Trans. Inf. Theory 58 (8) (2012) 5534-5543.

\bibitem{B21} 
D. Li, Q. Wen, J. Zhang,  L. Jiang, Linear complexity of generalized cyclotomic binary sequences with period , IEICE Trans. Fundamentals E98-A (6) (2015) 1244-1254.
\end{thebibliography}
\end{document}
