\documentclass[runningheads,a4paper]{llncs}
\usepackage{amssymb,amsmath,amscd,amstext,graphicx,texdraw,mathdots,gastex,comment}
\usepackage{color}
\setcounter{tocdepth}{3}
\newcommand{\bs}{\mathbb{S}}
\newcommand{\ulm}{\underline{m}}
\newcommand{\uln}{\underline{n}}
\newcommand{\ulk}{\underline{k}}
\newcommand{\ol}{\overline}
\newcommand{\ga}{\alpha}
\newcommand{\N}{\mathbb{N}}
\newcommand{\go}{\omega}
\newcommand{\gb}{\beta}
\newcommand{\gq}{\geqslant}
\newcommand{\sq}{\subseteq}
\newcommand{\ra}{\Rightarrow}
\newcommand{\gk}{\kappa}
\newcommand{\gs}{\sigma}
\newcommand{\gga}{\gamma}
\newcommand{\gl}{\lambda}
\newcommand{\gee}{\varepsilon}
\newcommand{\mc}{\mathcal}
\newcommand{\bx}{\mathbin{\Box}}
\newcommand{\mb}{\mathbb}
\newcommand{\tr}{\textrm}
\newcommand{\qq}{\qquad}
\newcommand{\lma}{\left(\begin{matrix}}
\newcommand{\rma}{\end{matrix}\right)}
\newcommand{\be}{\underset{\mathcal{E}}{\overset{*}{\leftrightarrow}}}
\newcommand{\beo}{\underset{\mathcal{E} _1}{\overset{*}{\leftrightarrow}}}
\newcommand{\bed}{\underset{\mathcal{E} _2}{\overset{*}{\leftrightarrow}}}
\newcommand{\bep}{\underset{\mathcal{E} '}{\overset{*}{\leftrightarrow}}}
\newcommand{\bm}{\begin{matrix}}
\newcommand{\enm}{\end{matrix}}
\newcommand{\bc}{\begin{center}}
\newcommand{\ec}{\end{center}}
\newenvironment{keywords}{
       \list{}{\advance\topsep by0.35cm\relax\small
       \leftmargin=1cm
       \labelwidth=0.35cm
       \listparindent=0.35cm
       \itemindent\listparindent
       \rightmargin\leftmargin}\item[\hskip\labelsep
                                     \bfseries Keywords:]}
     {\endlist}
\begin{document}



\title{-Colorability is Graph Automaton Recognizable}
\author{Antonios Kalampakas}
\institute{Department of Production Engineering and Management,
Laboratory of Computational Mathematics,
School of Engineering, Democritus University of Thrace,
V.Sofias 12, Prokat, Building A1, 67100Xanthi, Greece, akalampakas@gmail.com}
\maketitle
\begin{abstract}
Automata operating on general graphs have been introduced by virtue of graphoids. In this paper we construct a graph automaton that recognizes -colorable graphs.
\end{abstract}

\begin{keywords}
formal languages, automata theory, graph colorability
\end{keywords}



\section{Introduction}

Automata on general (hyper)graphs were constructed for the first time in \cite{BK3} by utilizing the algebraic properties of graphoids, i.e., magmoids  satisfying the 15 equations of graphs which are specified in \cite{BK1}. The notion of magmoids, introduced by Arnold and Dauchet in \cite{AD}, is the algebraic structure employed to generate graphs from a finite set in a role similar to that of monoids for the generation of strings. A magmoid is a doubly ranked set equipped with two operations which are associative, unitary and mutually coherent in a canonical way. In \cite{EV} Engelfriet and Vereijken proved that every graph with edges labeled  over a finite doubly ranked set  can be built from a specific finite set of elementary graphs , together with the elements of , by using the operations of graph product and graph sum. From this result it is derived that graphs can be organized into a magmoid with operations product and sum. Although, as it was observed in \cite{EV}, every hypergraph can be constructed in an infinite number of ways, this ambiguity was settled in \cite{BK1} by determining a \emph{finite} set of equations  with the property that two expressions represent the same hypergraph, if and only if, one can be transformed into the other
through them.

Commencing from this result, a \emph{graphoid}  is defined as a magmoid with a designated set of elements that satisfy the equations . Hence  can be structured into a graphoid by virtue of the set  of elementary graphs. The \emph{relational magmoid} over a set  is constructed by defining the operations of composition and sum on the set of relations from  to . This set is structured into a \emph{relational graphoid} over , by specifying a set  of relations that satisfy the equations . A relational graphoid is called abelian when a particular relation of  consists of all the transpositions in .
In \cite{BK3} graph automata, with state set , were introduced by virtue of a specific abelian relational graphoid, denoted here by , and by exploiting the fact that  is the free graphoid generated by . In the same paper it is postulated that different kinds of graphoids will produce graph automata with diverse operation and recognition capacity. In \cite{Kal2} it is shown that all abelian relational graphoids are characterized in the following way: a set  generates an abelian relational graphoid if and only if  is partitioned  into  disjoint abelian groups with operations univocally correlated with . In other words it is proved that organizing  in a relational graphoid is equivalent to partitioning the set  and structuring every class in a group.

The particular graphoid  employed in \cite{BK3} corresponds to the partitioning of  into singleton sets each one being the trivial group. Hence, due to \cite{Kal2}, it is the simplest possible abelian relational graphoid. In this paper we construct a graph automaton over  which accepts the -colorable graphs, . Hence it is manifested that graph automata are capable of recognizing important classes of graphs even when operating on the most trivial of the known graphoids. In the following section we recall basic definition for magmoids and hypergraphs. Graphoids are presented in Section \ref{S:3} and the definition of a graph automaton is given. In Section \ref{S:4} we construct the graph automaton that recognizes -colorable graphs and illustrate its operation. Conclusions and future work are discussed in the last section.


\section{Magmoids and Hypergraphs}

A doubly ranked set   is a set  together with a function  we set . In what follows we will drop the subscript   and denote a doubly ranked set simply by .
A \emph{magmoid} is a doubly ranked set 
equipped with two operations
\begin{center}
,
\quad
,
\end{center}
which are associative in the obvious way, satisfy the
distributivity law

whenever all the above operations are defined and are equipped with a
sequence of constants ,  called units, such that

hold for all  and  all .
Notice that, due to the last equation, the elements  are uniquely determined by . From now on  will be
simply denoted by . The free magmoid  generated by a doubly ranked set  is constructed in \cite{BK1}. The sets  of all relations from  to 

can be structured into a magmoid with  being the usual
relation composition and  defined as follows: for  and 

where , , , . Notice that , where  is the empty word of . The units are given by
 and .
We denote by  the magmoid  constructed in this way and call it the \emph{relational magmoid of} .


 An -(hyper)graph  with edge labels from a doubly ranked set  is a tuple   consisting of the set of nodes or vertices , the set of edges ,   the source and target functions  and  respectively,    the labeling function  such that  , for all ,  and  the sequences of begin and end nodes  and  with  and . Notice that vertices can be duplicated in the begin and end sequences of the graph and also at the sources and targets of the edges. Isomorphism between two graphs is defined in the obvious way and we shall not distinguish between two isomorphic  graphs. The set of all  -graphs over  is
denoted by   and we set . Ordinary unlabeled directed graphs are obtained as a special case of hypergraphs i.e., in the case that each hyperedge is binary (has one source and one target), every edge has the same label and the sequences  and  are the empty word.

If  is the -graph   and  is the -graph   then their \emph{product}  is the -graph that is obtained by taking the disjoint union of  and  and then identifying
the  end node of  with the  begin node of , for every ; also,  and
. The \emph{sum}  of arbitrary graphs  and  is their disjoint union with their sequences of begin nodes concatenated and similarly for their end nodes (see \cite{BK3,Kal2} for examples). For every  we denote by  the discrete graph
of rank  with nodes  and ; we write  for .  It is straightforward to verify that  with the operations  defined above is a magmoid, whose units are the graphs .

\section{Graphoids and Graph Automata}\label{S:3}


Now we present graph automata by employing the algebraic structure of graphoids as introduced in \cite{BK3}. We denote by  the discrete -graph that has a single node  and whose begin and end sequences are  ( times) and  ( times) respectively,  is the discrete -graph that has two nodes  and  and whose begin and end sequences are  and , respectively, also for every , we denote again by  the -graph having only one edge and  nodes . The edge is labeled by , and the begin (resp. end sequence) of the graph is the sequence of sources (resp. targets) of the edge, viz.  (resp.
).

\vspace{4mm}

\begin{minipage}[t]{2cm}

\begin{center}
\begin{texdraw}
\drawdim cm \linewd 0.021 \arrowheadtype t:F \arrowheadsize l:0.12
w:0.10 \move(2.4 2)\fcir f:0 r:0.03

\textref h:C v:C  \htext(2.4 0.8){} \htext(2.1
2.4){} \htext(2.1 2.1){} \htext(2.1 1.6){}
\htext(2.7 2.4){} \htext(2.7 2.1){} \htext(2.7
1.6){}


\end{texdraw}
\end{center}
\end{minipage}
\quad
\begin{minipage}[t]{2cm}
\begin{center}
\begin{texdraw}

\drawdim cm \linewd 0.021 \arrowheadtype t:F \arrowheadsize l:0.12
w:0.10 \move(4 2.2)\fcir f:0 r:0.03 \move(4 1.8)\fcir f:0 r:0.03

\textref h:C v:C  \htext(4 0.8){} \htext(3.65 2.3){}
\htext(3.65 1.8){} \htext(4.3  2.3){} \htext(4.3
1.8){}


\end{texdraw}
\end{center}
\end{minipage}
\quad
\begin{minipage}[t]{2cm}
\begin{center}
\begin{texdraw}


\drawdim cm \linewd 0.021 \arrowheadtype t:F \arrowheadsize l:0.12
w:0.10 \move(5.6 2.4)\fcir f:0 r:0.03 \move(5.6 1.6)\fcir f:0
r:0.03

\textref h:C v:C  \htext(5.6 0.8){} \htext(5.3 2.4){}
\htext(5.3 2.1){} \htext(5.3 1.6){} \htext(5.95
2.4){} \htext(5.9 2.1){} \htext(5.6 2.1){}
\htext(5.95 1.6){}


\end{texdraw}
\end{center}
\end{minipage}
\quad
\begin{minipage}[t]{3cm}
\begin{center}
\begin{texdraw}

\drawdim cm \linewd 0.021 \arrowheadtype t:F \arrowheadsize l:0.12
w:0.10      \move (7.3 1.5)\fcir f:0 r:0.03 \lvec(7.8 2)\move (7.3
1.5)\avec(7.6 1.8) \move (7.3 2.5)\fcir f:0 r:0.03 \lvec(7.8
2)\move(7.3 2.5) \avec(7.6 2.19)\move(7.8 2)

\move(7.8 1.8)\lvec(7.8 2.2)\lvec(8.2 2.2)\lvec(8.2 1.8)\lvec(7.8
1.8)\move(8.2 2)\lvec(8.7 2.5)\fcir f:0 r:0.03 \move(8.2
2)\lvec(8.7 1.5)\fcir f:0 r:0.03 \move(8.2 2)\avec(8.5
2.3)\move(8.2 2)\avec(8.5 1.7)


\textref h:C v:C  \htext(8 2){}\htext(9
2.5){}\htext(9 2){}\htext(9 1.5){}\htext(7
2.5){}\htext(7 1.5){} \htext(7 2){}\htext(8
0.8){}
\end{texdraw}
\ec

\end{minipage}

\vspace{4mm}



Engelfriet and Vereijken, in \cite{EV},  presented an algorithm that inductively constructs every graph  from the set  by using graph product and graph sum.  However, there are infinitely many ways to construct a given graph. This was overridden  by identifying a finite set  of equations  with the property that two expressions represent the same graph if and only if one can be transformed into the other through these equations \cite{BK1}. It is evident from this discussion that the equations  are satisfied in . Magmoids with such a property are called \emph{graphoids}. Formally, a graphoid  consists of a magmoid  and a set , with , ,  such that the following equations hold:\\
\begin{minipage}[b]{70pt}
  \end{minipage}  \begin{minipage}[b]{270pt}
\end{minipage}\\
\begin{minipage}[b]{170pt}
  \end{minipage}\qq\quad  \begin{minipage}[b]{140pt}
\end{minipage}\\
\begin{minipage}[b]{120pt}
  \end{minipage}\qq\quad  \begin{minipage}[b]{180pt}
\end{minipage}\\
\begin{minipage}[r]{300pt}

\end{minipage}\\
\begin{minipage}[b]{170pt}
  \end{minipage} \qq  \begin{minipage}[b]{140pt}
\end{minipage}\\
\begin{minipage}[b]{120pt}
  \end{minipage} \qq  \begin{minipage}[b]{180pt}
\end{minipage}\\
\begin{minipage}[b]{300pt}

\end{minipage}\\
\begin{minipage}[b]{100pt}
  \end{minipage} \qq\quad  \begin{minipage}[b]{200pt}
\end{minipage}\\
\begin{minipage}[b]{340pt}

\end{minipage}\\ \\
where  is defined inductively by   and represents the graph associated with the permutation
that interchanges the last  numbers with the first one \cite{BK1}. We point out that although (\ref{E:17}) is a set of equations it only has to be valid for the elements of  in order to hold for every element of a magmoid generated by  \cite{BK1}.   Thus the pair , with  is a graphoid and in fact it is the free graphoid
generated by  as it is illustrated in \cite{BK3}. Given graphoids  and , a magmoid morphism  preserving -sets, i.e.,   and , is called a morphism of graphoids.

Graphoids constructed from the magmoid of relations  over a given set  are called \emph{relational graphoids} and a relational graphoid is called abelian when  . The abelian relational graphoid  that was used for the introduction of graph automata is constructed by setting  as above and
  ,
   ,
,
 .

A \emph{nondeterministic relational graph automaton} is a structure
\bc
,
\ec
where  is the doubly ranked set of hyperedge labels,      is the finite set of states,    is a relational graphoid over ,      is the doubly ranked transition function and   are initial and final rational subsets of
. The function  is uniquely extended into a morphism of graphoids
, where  and ,
and the behavior of  is given by

where  and
. From their construction, graph automata are
finite machines due to the fact that the set of equations
- is finite.
A graph language is called recognizable whenever it is obtained as
the behavior of a graph automaton. The class of all such languages
over the doubly ranked set  is denoted by
 .


\section{A Graph Automaton Recognizing -colorable Graphs}\label{S:4}

In this section we construct a relational graph automaton over the abelian graphoid  recognizing -colorable graphs.

For , we set  with
\begin{itemize}
  \item , ,
  \item ,
  \item ,
  \item .
\end{itemize}
It is clear that the automaton  reads unlabeled -graphs with binary edges (one source and one target per edge), i.e., ordinary directed graphs.

As an example we shall illustrate the operations of  and  on the following graphs, where the label in every edge is  and thus omitted.
\bc
\begin{minipage}[b]{300pt}
\bc
\begin{texdraw}
\drawdim mm \linewd 0.21 \arrowheadtype t:H \arrowheadsize l:1.2
w:1.4

\move(42 10)\fcir f:0 r:0.4 \lvec
(32 20) \move (32 20) \avec (37 15)  \move(42 30)\fcir f:0 r:0.4 \lvec
(32 20)  \move (32 20) \avec (37 25) \move(42 30)  \avec (42 20)\move (42 30)\lvec
(42 10) \move(20 20)\fcir f:0 r:0.4  \avec (26 20)\move (20 20)\lvec
(32 20) \fcir f:0 r:0.4

\textref h:C v:C   \htext(30 5){}
\end{texdraw}\qq\qq
\begin{texdraw}
\drawdim mm \linewd 0.21 \arrowheadtype t:H \arrowheadsize l:1.2
w:1.4

\move(42 10)\fcir f:0 r:0.4 \lvec
(32 20) \move (32 20) \avec (37 15)  \move(42 30)\fcir f:0 r:0.4 \lvec
(32 20)  \move (32 20) \avec (37 25) \move(42 30)  \avec (42 20)\move (42 30)\lvec
(42 10) \move(20 20)\fcir f:0 r:0.4  \avec (26 20)\move (20 20)\lvec
(32 20) \fcir f:0 r:0.4  \move (20 20) \lvec (42 30) \move (20 20) \avec (31 25)
\move (20 20) \lvec (42 10) \move (20 20) \avec (31 15)

\textref h:C v:C   \htext(30 5){}
\end{texdraw} \qq\qq
\begin{texdraw}
\drawdim mm \linewd 0.21 \arrowheadtype t:H \arrowheadsize l:1.2
w:1.4

\move(10 10)\fcir f:0 r:0.4
\lvec (30 10) \move (10 10) \avec (20 10)  \move (10 10)
\lvec (30 20) \move (10 10)  \avec (26 18)  \move (10 10)
\lvec (30 30)  \move (10 10) \avec (26 26)

\move(10 20)\fcir f:0 r:0.4
\lvec (30 10) \move (10 20)  \avec (26 12)  \move (10 20)
\lvec (30 20) \move (10 20)   \avec (26 20) \move (10 20)
\lvec (30 30)  \move (10 20)  \avec (26 28)


\move(10 30)\fcir f:0 r:0.4
\lvec (30 10) \move (10 30)  \avec (26 14)  \move (10 30)
\lvec (30 20) \move (10 30)  \avec (26 22)  \move (10 30)
\lvec (30 30)  \move (10 30) \avec (20 30)



\move(30 10)\fcir f:0 r:0.4
\move(30 20)\fcir f:0 r:0.4
\move(30 30)\fcir f:0 r:0.4

\textref h:C v:C   \htext(20 5){}
\end{texdraw}
\ec
\end{minipage}
\ec
One of the representations of  is

where graph product and graph sum are denoted, for simplicity, by horizontal and vertical concatenation. Note that since  is the free graphoid, the operation of the automaton is the same for every representation of . The consumption of  by  gives

and an accepting sequence of transitions is

where the states are indicated in brackets. In graphical representation the states that the automaton reaches at each state are
\bc
\begin{minipage}[b]{300pt}
\bc
\begin{texdraw}
\drawdim mm \linewd 0.21 \arrowheadtype t:H \arrowheadsize l:1.2
w:1.4

\move(42 10)\fcir f:0 r:0.4 \lvec
(32 20) \move (32 20) \avec (37 15)  \move(42 30)\fcir f:0 r:0.4 \lvec
(32 20)  \move (32 20) \avec (37 25) \move(42 30)  \avec (42 20)\move (42 30)\lvec
(42 10) \move(20 20)\fcir f:0 r:0.4  \avec (26 20)\move (20 20)\lvec
(32 20) \fcir f:0 r:0.4

\textref h:C v:C   \htext(18 20){} \htext(35 20){} \htext(44 10){} \htext(44 30){}
\end{texdraw}
\ec
\end{minipage}
\ec
which is actually a proper 3-coloring of . Similarly, a representation for  is

and it is clear that there exists no successful transition of  by . A successful transition of  by  is

and the corresponding -coloring that is obtained by the state distribution of this transition is
\bc
\begin{texdraw}
\drawdim mm \linewd 0.21 \arrowheadtype t:H \arrowheadsize l:1.2
w:1.4

\move(42 10)\fcir f:0 r:0.4 \lvec
(32 20) \move (32 20) \avec (37 15)  \move(42 30)\fcir f:0 r:0.4 \lvec
(32 20)  \move (32 20) \avec (37 25) \move(42 30)  \avec (42 20)\move (42 30)\lvec
(42 10) \move(20 20)\fcir f:0 r:0.4  \avec (26 20)\move (20 20)\lvec
(32 20) \fcir f:0 r:0.4  \move (20 20) \lvec (42 30) \move (20 20) \avec (31 25)
\move (20 20) \lvec (42 10) \move (20 20) \avec (31 15)

\textref h:C v:C   \textref h:C v:C   \htext(18 20){} \htext(35 20){} \htext(44 10){} \htext(44 30){}
\end{texdraw}
\ec
Note that the elementary graph  is not necessary for the representation of  and . More generally, all planar graphs can be represented without employing . On the other hand, the non-planar graph 
is expressed as

where in the third parenthesis there are  's and  stands for the graph that is associated with the permutation

Notice that as it is shown in \cite{BK1} for every permutation we can construct, inductively by  and  a graph that represents it. The graph  is -colorable and an accepting transition of  is

Hence, if we denote by  the set of all -colorable graphs, we obtain
\begin{theorem}
For every , the graph language  is   recognizable.
\end{theorem}
\section{Conclusion and Future Work}

It is shown that -colorability is graph automaton recognizable even when the automaton operates over the most trivial abelian relational graphoid. This indicates that the graph automaton is a robust recognition mechanism and evokes numerous issues regarding the class of automaton recognizable graph languages.
\begin{itemize}
  \item The introduced graph automata are nondeterministic in the sense that if  then  is a relation in . The deterministic version of this definition is obtained by requiring that  is a function from  to , i.e.,  is an element of  which is a submagmoid of  called the magmoid of functions \cite{BK3}. It is interesting to compare the two classes and in particular to investigate the existence of a deterministic graph automaton recognizing -colorable graphs.
  \item In \cite{BK3} it is shown that the time complexity for checking if a specific graph belongs to the behavior of a graph automaton grows polynomially as a function of the number of states . It is important to calculate the complexity of the membership problem for a given graph automaton as a function of the size of the graph for both the deterministic and the nondeterministic case. Such a result would classify automaton recognizable graph languages according to their computational complexity.
  \item In \cite{Kal2} it is proved that an infinite number of non-isomorphic abelian relational graphoids exists. Two questions that naturally arise  concern  the recognition capacity of the corresponding automata as well as the existence of non-abelian or non-relational graphoids.
\end{itemize}

\begin{thebibliography}{99}
\bibitem{AD}  A. Arnold, M. Dauchet, Th\'{e}orie des magmoides. I and II, RAIRO Theoret. Inform. Appl. 12 (1978) 235-257;
13 (1979) 135-154 (in French).
\bibitem{BK1} S. Bozapalidis, A. Kalampakas,  An Axiomatization of Graphs, Acta Inform. 41 (2004) 19-61.
\bibitem{BK3} S. Bozapalidis, A. Kalampakas,  Graph Automata. Theoret. Comput. Sci. 393 (2008) 147-165.
\bibitem{Kal2} A. Kalampakas, Graph Automata: The Algebraic Properties of Abelian Relational Graphoids, LNCS 7020 (2011) 168-182.
\bibitem{EV}  J. Engelfriet,  J.J. Vereijken, Context-free graph grammars and concatenation of graphs, Acta Informatica 34 (1997) 773-803.
\end{thebibliography}




\end{document}
