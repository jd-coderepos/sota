

\documentclass[11pt,a4paper]{article}
\usepackage[hyperref]{acl2020}
\usepackage{times}
\usepackage{latexsym}
\renewcommand{\UrlFont}{\ttfamily\small}

\usepackage{microtype}

\aclfinalcopy 



\newcommand\BibTeX{B\textsc{ib}\TeX}



\usepackage{graphicx}
\usepackage{amsmath}
\usepackage{subfigure}



\title{RealFormer: Transformer Likes Residual Attention}

\author{Ruining He,  Anirudh Ravula, Bhargav Kanagal, Joshua Ainslie \\
Google Research \\
\texttt{\{ruininghe,braineater,bhargav,jainslie\}@google.com} }

  

\date{}

\begin{document}
\maketitle

\begin{abstract}
Transformer is the backbone of modern NLP models. In this paper, we propose \textbf{RealFormer}, a simple \textbf{Re}sidual \textbf{A}ttention \textbf{L}ayer Trans\textbf{former} architecture that significantly outperforms canonical Transformers on a spectrum of tasks including Masked Language Modeling, GLUE, and SQuAD. Qualitatively, RealFormer is easy to implement and requires minimal hyper-parameter tuning. It also stabilizes training and leads to models with \emph{sparser} attentions. Code will be open-sourced upon paper acceptance.
\end{abstract}


\section{Introduction}
Transformer~\citep{Vaswani-2017-attention} architectures are the backbone of numerous state-of-the-art NLP models such as BERT ~\citep{Devlin-2019-bert}, GPT~\citep{Radford-2019-gpt2}, and Meena~\citep{Adiwardana-2020-meena}, and have seen wide successes across both academia and industry. Typically, a Transformer network consists of a stack of residual layers. The original design follows a ``Post-LN'' structure which adds Layer Norm (LN) as a ``post-processing'' step for each sub-layer, as shown in Figure~\ref{fig:realformer} (a). It has been adopted by various state-of-the-art models including BERT, XLNet~\citep{Yang-2019-xlnet}, RoBERTa~\citep{Liu-2019-roberta}, ALBERT~\citep{Lan-2019-albert}, and Transformer-XL~\citep{Dai-2019-transformerxl}. Another design is to reorganize the order of sub-layers to create a ``direct'' / clean path to propagate embeddings of tokens in the input sequence through the whole network, as shown in Figure~\ref{fig:realformer} (b).\footnote{Note that a final LN module is usually added at the very top of the whole network.} This design adds LN as a ``pre-processing'' step for each sub-layer, and is often referred to as ``Pre-LN'' and used by some well-known extra large models such as GPT-2~\citep{Radford-2019-gpt2} and Megatron~\citep{Shoeybi-2019-megatron}. In some sense, Post-LN and Pre-LN are analogous to ResNet v1~\citep{He-2016-resnetv1} and ResNet v2~\citep{He-2016-resnetv2} respectively in the Computer Vision literature. Although ResNet v2 is almost always preferable to v1 for Computer Vision, it might not have been the case for Pre-LN Transformer in the NLP literature. It is likely that the particularities of self-attention modules and Transformer architectures potentially favor (at least slightly) different designs compared to traditional convolutional neural networks. 

\begin{figure*}[!h]
\centering
\includegraphics[width=\textwidth,height=\textheight,keepaspectratio]{figures/realformer.png}
\caption{Comparison of different Transformer layers: (a) The prevalent Post-LN layer used by (\emph{e.g.}) BERT; (b) Pre-LN layer used by (\emph{e.g.}) GPT-2 that creates a ``direct'' path to propagate token embeddings; (c) Our RealFormer layer that creates a ``direct'' path to propagate attention scores (by adding a simple skip edge on top of (a)).}
\label{fig:realformer}
\end{figure*}


In this paper, we propose a simple Transformer-based architecture to show that it is beneficial to create a ``direct'' path to propagate raw attention scores through the whole network. Our architecture is called \textit{\textbf{Re}sidual \textbf{A}ttention \textbf{L}ayer Trans\textbf{former}}, or \textit{\textbf{RealFormer}} in short. As shown in Figure~\ref{fig:realformer} (c), each RealFormer layer takes the raw attention scores of all attention heads from the previous layer and adds ``residual scores'' (computed the same way as attention scores in regular Transformers) on top. The sum of the two scores is then used to compute attention weights via softmax (again, as in regular Transformers).
In other words, RealFormer can be seen as adding a simple skip connection to the Post-LN Transformer. Note that it does not add any multiplication ops to the computational graph and therefore the performance is expected to be comparable.

Specifically, our main contributions include:
\begin{itemize}
\item We present RealFormer, a novel and simple model based on original Transformer (with no more than a few lines of code changes and minimal hyper-parameter tuning).
\item We demonstrate that RealFormer can be used as a drop-in replacement of Transformer in BERT, outperforming both Post-LN and Pre-LN Transformers across a wide spectrum of model sizes (from small to extra large) for Masked Language Modeling (\emph{i.e.}, pre-training).
\item We also demonstrate that RealFormer can consistently improve accuracy of downstream tasks including GLUE~\citep{Wang-2018-glue}, SQuAD v1.1~\citep{Rajpurkar-2016-squad} and SQuAD v2.0~\citep{Rajpurkar-2018-squad2}. Furthermore, it even achieves competitive downstream results when pre-trained with only half the number of epochs of the strongest baseline.
\item Finally, we demonstrate qualitatively that attention in RealFormer tends to be sparser and more correlated across layers compared to baselines, which we believe may have some regularization effects that could stabilize training and benefit fine-tuning.
\end{itemize}


\section{Related Work}
\citet{Vaswani-2017-attention} proposed Transformer initially for machine translation task in 2017 and it has profoundly changed the natural language processing field ever since. 

\citet{Radford-2018-gpt1} demonstrated that generative pre-training of a Transformer-based language model (GPT) on a diverse corpus of unlabeled text can give large gains to downstream NLP tasks that suffer from scarce labeled data. Following this thread, \citet{Devlin-2019-bert} proposed to pre-train a \emph{bidirectional} Transformer encoder (BERT) with a novel Masked Language Modeling as the main optimization objective. Since then, advances on many NLP tasks have been dominated by the self-supervised general-purpose pre-training, task-specific fine-tuning paradigm.
Following BERT, there has been a large stream of work that explores better self-supervision objectives (\emph{e.g.},~\citet{Yang-2019-xlnet,Clark-2020-electra}), larger pre-training data and better hyper-parameters (\emph{e.g.},~\citet{Liu-2019-roberta}, model parameter sharing (\emph{e.g.},~\citet{Lan-2019-albert}, multi-task pre-training (\emph{e.g.},~\citet{Sun-2019-ernie2,Raffel-2019-t5}.
These efforts typically employ a Post-LN Transformer at their core. In this paper we adopt BERT to test different Transformer architectures because it is widely used and representative of this thread of work.

Another notable thread of work focuses on improving the efficiency/scalability of Transformer. Typically, they try to reduce the \emph{quadratic} complexity of the self-attention mechanism with respect to sequence length via low-rank methods (\emph{e.g.},~\citet{Wang-2020-linformer}), fixed strided attention patterns (\emph{e.g.},~\citet{Child-2019-sparsetransformer}), learnable attention patterns (\emph{e.g.},~\citet{Kitaev-2020-reformer, Roy-2020-routingtransformer}), memory-based global \& local attention (\emph{e.g.},~\citet{Ainslie-2020-etc, Beltagy-2020-longformer, Zaheer-2020-bigbird}), and so on. 
These methods are particularly useful when dealing with long documents that go beyond the capacity of standard Transformer models. We would refer the reader to \citet{Tay-2020-survey} for a detailed survey.
RealFormer is orthogonal to these methods as it focuses on improving standard Transformer with an universal technique which can apply to these models as well. 



\section{RealFormer}
\subsection{Standard Transformer}
There is an encoder and a decoder in Transformer~\citep{Vaswani-2017-attention}. Since they work in a similar way, here we only introduce the encoder and refer the reader to the original paper for complete details.

There are two sub-layers inside each layer of a Transformer encoder. The first sub-layer contains a Multi-Head Attention module that computes output embeddings of a set of queries () by aggregating the embeddings () of a set of keys ():

where .  and  are matrices with dimension  and  is a matrix with dimension . , , and  are matrices that linearly project queries, keys, and values into the ``attention space'' of the -th head.  is a matrix that linearly transforms the concatenation of the outputs of all heads.

Attention function is typically implemented with a Scaled Dot-Product Attention module~\citep{Vaswani-2017-attention} which computes a weighted sum of the values:

where matrix  is the raw attention scores for each (query, key) pair. These scores are normalized via the Softmax function for each query and then act as weights for the corresponding vectors in .

The second sub-layer contains a fully-connected Feed-Forward Network (FFN) module with one hidden layer:

where  is an activation function usually implemented with ReLU or GeLU (\emph{e.g.},~\citet{Devlin-2019-bert}). FFN is applied to each position in the sequence separately and identically.
Finally, there are Layer Norm (LN) modules inserted into the above two sub-layers to stabilize training.

As shown in Figure~\ref{fig:realformer}, there are two canonical designs of the Transformer network which only differ in the ways they organize the modules. Post-LN is the original architecture proposed by~\citet{Vaswani-2017-attention} which normalizes the outputs at the end of each sub-layer. In contrast, Pre-LN normalizes sub-layer inputs instead and creates a direct path (without LN in the way) to propagate embeddings of the tokens in the sequence.

\subsection{Residual Attention Layer Transformer}
RealFormer closely follows the Post-LN design and simply adds a skip edge to connect Multi-Head Attention modules in adjacent layers, as shown in Figure~\ref{fig:realformer} (c). Formally, it adds , the pre-softmax attention scores from the previous layer with shape ,\footnote{Batch dimension is omitted for brevity.} as one additional input to the Multi-Head Attention module in the current layer:

where   and  is the slice of  with shape  corresponding to .
ResidualAttention adds ``residual scores'' on top of  and then computes the weighted sum as usual:

Finally, new attention scores  are passed over to the next layer.

Implementing RealFormer takes no more than adding a few lines of code to the ``backbone'' Transformer. The same idea can be straightforwardly applied even when there are more than one type of attention modules in the network. For example, there are encoder-encoder self-attention, encoder-decoder attention, and decoder-decoder self-attention modules for machine translation. In such cases, RealFormer simply creates multiple direct paths, one for each type of attention module, as long as the attention pattern / mask is the same across the layers along the path (which is almost always the case). 

As we will show in Section~\ref{sect:exp}, Post-LN Transformer tends to outperform Pre-LN Transformer for a variety of setups (given a reasonable computing budget). Therefore in this paper we choose Post-LN as the backbone for RealFormer for simplicity. Note however that it should be straightforward to switch the backbone to different Transformer variants in settings that favor different trade-offs.

\section{Experiments} \label{sect:exp}
In this section, we conduct comprehensive empirical studies on RealFormer comparing against two canonical Transformer architectures: Post-LN and Pre-LN. We evaluate the strength of RealFormer in terms of both pre-training and fine-tuning accuracy on downstream tasks with minimal (if any) hyper-parameter tuning.

\subsection{BERT}
BERT~\citep{Devlin-2019-bert} has been the standard way of transferring knowledge from large unlabeled text corpora by pre-training a bidirectional Transformer encoder. Numerous downstream NLP tasks suffering from scarcity of supervised data have benefited considerably by fine-tuning a pre-trained BERT model. This drives us to adopt BERT as the main evaluation setup for RealFormer.


\paragraph{Experiment setup.}
We follow the standard BERT pre-training setup (dataset: Wikipedia + BookCorpus, vocab: uncased 30K, max sequence length: 512\footnote{Unlike BERT which uses a reduced sequence length for the first 90\% of steps, we always use 512 for simplicity.}, dropout: 10\%, learning rate: 1e-4, learning rate schedule: warm up and then linearly decay to 0, weight decay: 0.01, optimizer: AdamW, objective: Masked Language Modeling + Next Sentence Prediction, etc.) to compare all three Transformer models: Post-LN, Pre-LN, and RealFormer. 
We experiment with Transformer architectures with a wide spectrum of sizes, from Small and Base all the way to xLarge, as detailed in Table~\ref{table:bert-arch}. For simplicity, all models are pre-trained 1M steps with a mini-batch size of 512 (except that xLarge uses 256 to avoid TPU OOM). Note that we use a larger mini-batch size than~\citet{Devlin-2019-bert}, \emph{i.e.}, doubling the amount of pre-training epochs, to show more complete behavior of different models.

\begin{table}
\centering
\begin{tabular}{lcccc}
\hline \textbf{Model} & \textbf{L} & \textbf{H} & \textbf{A} & \textbf{I} \\ \hline
BERT-Small  & 4  & 512   & 8  & 2,048 \\
BERT-Base   & 12 & 768   & 12 & 3,072 \\
BERT-Large  & 24 & 1,024 & 16 & 4,096 \\
BERT-xLarge & 36 & 1,536 & 24 & 6,144 \\
\hline
\end{tabular}
\caption{\label{table:bert-arch} Model architectures for BERT evaluation. L: \#layers, H: hidden size, A: \#heads, I: intermediate size.}
\end{table}




We use exactly the same setup for all three Transformer architectures except that for the Pre-LN Transformer we follow the initialization strategy suggested by~\citet{Radford-2019-gpt2} and~\citet{Child-2019-sparsetransformer}.\footnote{We also experimented with the initialization strategy used by BERT but with similar results.} Note that for simplicity RealFormer inherits all hyper-parameter setups from Post-LN Transformer unless otherwise specified.

All experiments are performed on 128 or 256 TPU v3 cores depending on model sizes. 


\begin{table}
\setlength{\tabcolsep}{3.5pt}
\centering
\begin{tabular}{lrrrr}
\hline \textbf{Model} & \textbf{Post-LN} & \textbf{Pre-LN} & \textbf{RealFormer} \\ \hline
BERT-Small  & 61.88\%  & 61.67\%   & \textbf{62.02\%}  \\
BERT-Base   & 70.20\%  & 69.74\%   & \textbf{70.42\%}  \\
BERT-Large  & 73.64\%  & 73.21\%   & \textbf{73.94\%}  \\
BERT-xLarge & 73.72\%  & 73.53\%   & \textbf{74.76\%}  \\
\hline
\end{tabular}
\caption{\label{table:bert-mlm} Masked Language Modeling accuracy on development set after pre-training 1M steps.\footnotemark RealFormer outperforms baselines more as model gets larger.}
\end{table}
\footnotetext{We found it to be beneficial for RealFormer to set attention scores at each layer to be the \emph{running mean} (instead of running sum) for models deeper than 24 layers and therefore used this setup for RealFormer xLarge.}


\subsubsection{Pre-training Results} \label{sec:pre-train}
To evaluate pre-trained models, we report Masked Language Modeling (MLM) accuracy\footnote{All methods achieved similar (and great) results on the Next Sentence Prediction task presumably because it is much easier.} on a randomly held-out development set. As shown in Table~\ref{table:bert-mlm}, RealFormer outperforms the two baseline Transformers considerably with the gap increasing with model size. 
Our hypothesis is that larger models are inherently harder to train (we did observe that BERT with Post-LN is unstable and sometimes even diverges for xLarge) and RealFormer can help regularize the model and stabilize training. 

We also report the pre-training curves in Figure~\ref{fig:mlm}. One interesting finding is that the Pre-LN Transformer seems to favor the combination of extra large models and a small number of steps, though it is consistently outperformed by the other two Transformers in ``regular-sized'' settings or given enough budget. 



\begin{figure*}
\subfigure[BERT-Small]{\includegraphics[width=1.61in, keepaspectratio]{figures/BERT-Small.png}}
\subfigure[BERT-Base]{\includegraphics[trim=33 0 0 0,clip,width=1.52in,keepaspectratio]{figures/BERT-Base.png}}
\subfigure[BERT-Large]{\includegraphics[trim=33 0 0 0,clip,width=1.52in,keepaspectratio]{figures/BERT-Large.png}}
\subfigure[BERT-xLarge]{\includegraphics[trim=33 0 0 0,clip,width=1.52in,keepaspectratio]{figures/BERT-xLarge.png}}
\caption{Masked Language Modeling (MLM) accuracy on development set (best viewed in color). Improvement gap of RealFormer over the best baseline tends to increase with model size. Note that these are \emph{with zero hyper-parameter tuning} for RealFormer. (As we will show in Section~\ref{sec:rq}, RealFormer can be improved considerably by simply using a larger learning rate and thereby even double the gap size over Post-LN.)}
\label{fig:mlm}
\end{figure*}





\subsubsection{Downstream Evaluation Results}
We fine-tune the above BERT-Large models with the three Transformers on a variety of downstream tasks including GLUE, SQuAD v1.1 and SQuAD v2.0 to evaluate the performance of RealFormer on both sentence-level (\emph{i.e.}, GLUE) and token-level (\emph{i.e.}, SQuAD) NLP tasks.


\begin{table}[t]
\setlength{\tabcolsep}{3.5pt}
\centering
\begin{tabular}{lccc}
\hline 
\textbf{Task}      & \textbf{Post-LN}     & \textbf{Pre-LN}       & \textbf{RealFormer}      \\ \hline
MNLI-m             & 85.96\tiny{0.11}  & 85.03\tiny{0.12}   & 86.28\tiny{0.14}  \\
MNLI-nm            & 85.98\tiny{0.14}  & 85.05\tiny{0.19}   & 86.34\tiny{0.30}  \\
QQP                & 91.29\tiny{0.10}  & 91.29\tiny{0.16}   & 91.34\tiny{0.03}  \\
QQP \small{(F1)}   & 88.34\tiny{0.15}  & 88.33\tiny{0.26}   & 88.28\tiny{0.08}  \\
QNLI               & 92.26\tiny{0.15}  & 92.35\tiny{0.26}   & 91.89\tiny{0.17}  \\
SST-2              & 92.89\tiny{0.17}  & 93.81\tiny{0.13}   & 94.04\tiny{0.24}  \\
CoLA \small{(MC)}  & 58.85\tiny{1.31}  & 58.04\tiny{1.50}   & 59.83\tiny{1.06}  \\
STS-B \small{(PC)} & 90.08\tiny{0.27}  & 90.06\tiny{0.33}   & 90.11\tiny{0.56}  \\
STS-B \small{(SC)} & 89.77\tiny{0.26}  & 89.62\tiny{0.28}   & 89.88\tiny{0.54}  \\
MRPC               & 87.50\tiny{0.67}  & 86.76\tiny{5.64}   & 87.01\tiny{0.91}  \\
MRPC \small{(F1)}  & 91.16\tiny{0.45}  & 90.69\tiny{3.16}   & 90.91\tiny{0.65}  \\
RTE                & 71.12\tiny{2.52}  & 68.59\tiny{1.52}   & 73.65\tiny{0.90}  \\ \hline
Overall            & 84.01                  & 83.47                   & \textbf{84.53}  \\
\hline
\end{tabular}
\caption{\label{table:glue} GLUE development set results of fine-tuning BERT-Large models in Table~\ref{table:bert-mlm}. Default metric: accuracy, MC: Matthews correlation, PC: Pearson correlation, SC: Spearman correlation. Overall: first average metrics within each task (if there are 1+) and then across tasks. Numbers in smaller font are standard deviations. All numbers are scaled by 100.}
\end{table}


\paragraph{GLUE.}
General Language Understanding Evaluation (GLUE) is a canonical benchmark proposed by \citet{Wang-2018-glue} for evaluating the performance of models across a diverse set of NLU tasks. Following the fine-tuning recipe in~\citet{Devlin-2019-bert}, we use a mini-batch size of 32 for all models on all GLUE tasks. For each (task, model) pair, we select number of epochs in \{2, 3, 4\} and learning rate in \{6e-6, 8e-6, 1e-5, 2e-5, 3e-5, 4e-5, 5e-5\}.\footnote{We use a slightly wider range than~\citet{Devlin-2019-bert} to better accommodate all three models.} For each setup, we run the experiment 5 times and report the best median performance and the corresponding standard deviation on the development set.



Detailed results are tabulated in Table~\ref{table:glue}. We exclude the problematic WNLI task following~\citet{Devlin-2019-bert}. For each task, we report metric(s) that are suggested by the GLUE benchmark~\citep{Wang-2018-glue}. RealFormer achieves the best overall performance and outperforms both Post-LN and Pre-LN Transformers significantly on most tasks, testifying its strength at tackling sentence-level tasks.



\paragraph{SQuAD.}
The Stanford Question Answering Dataset (SQuAD v1.1) is a reading comprehension dataset consisting of 100K crowd-sourced question-answer pairs, where the answer to each question is a segment of text from the corresponding reading passage~\citep{Rajpurkar-2016-squad}.
SQuAD v2.0, a later version, further extends with over 50K unanswerable questions written adversarially by crowdworkers to look similar to answerable ones.


\begin{table}[t]
\setlength{\tabcolsep}{0.9pt}
\centering
\begin{tabular}{lcccc}
\hline \textbf{SQuAD} &\textbf{Public}  & \textbf{Post-LN} & \textbf{Pre-LN} & \textbf{RealFormer} \\ \hline
v1.1 \small{(F1)}   & 90.9  & 91.68\tiny{0.12}  & 91.06\tiny{0.09}    & \textbf{91.93}\tiny{0.12}   \\
v1.1 \small{(EM)}   & 84.1  & 85.15\tiny{0.13}  & 83.98\tiny{0.24}    & \textbf{85.58}\tiny{0.15}   \\
v2.0 \small{(F1)}   & 81.9  & 82.51\tiny{0.12}  & 80.30\tiny{0.12}    & \textbf{82.93}\tiny{0.05}   \\
v2.0 \small{(EM)}   & 78.7  & 79.57\tiny{0.12}  & 77.35\tiny{0.16}    & \textbf{79.95}\tiny{0.08}  \\
\hline
\end{tabular}
\caption{\label{table:squad} SQuAD development set results of fine-tuning BERT-Large models in Table~\ref{table:bert-mlm}. EM: exact match. Public: Post-LN results from~\citet{Devlin-2019-bert}. Numbers in smaller font are standard deviations. All numbers are scaled by 100.}
\end{table}


We follow the fine-tuning recipe in~\citet{Devlin-2019-bert} for all three Transformer models on these two datasets without using any additional data such as TriviaQA~\citep{Joshi-2017-triviaqa}.
For both v1.1 and v2.0, we select mini-batch size in \{32, 48\}, number of epochs in \{2, 3, 4\}, and learning rate in \{2e-5, 3e-5, 4e-5, 5e-5\}. For each setup, we run the experiment 5 times and report the best median performance and the corresponding standard deviation on the development set. As we can see from Table~\ref{table:squad}, RealFormer outperforms both Post-LN and Pre-LN Transformers considerably, attesting its strength at tackling token-level tasks.


\subsubsection{Research Questions} \label{sec:rq}

\paragraph{How well does RealFormer perform with half the pre-training budget?}
Although RealFormer has outperformed both Post-LN and Pre-LN Transformers considerably when pre-training 1M steps, we are also interested in investigating its potential when the pre-training budget is more limited. For this purpose, we experiment with BERT-Large models. In particular, we take the 500K step checkpoint of the pre-trained RealFormer in Table~\ref{table:bert-mlm} and fine-tune it on GLUE and SQuAD datasets using exactly the same procedure as described above. Comparison results against the strongest baseline, Post-LN Transformer pre-trained 500K (checkpoint) and 1M steps respectively, are collected in Table~\ref{table:500k}. We can see that RealFormer with merely half the amount of pre-training epochs can beat Post-LN (1M) on GLUE with a significant margin, and almost match its performance on SQuAD.


\begin{table}
\setlength{\tabcolsep}{2.8pt}
\centering
\begin{tabular}{lccc}
\hline 
\textbf{Task}   & \multicolumn{1}{p{1.75cm}}{\centering \textbf{Post-LN} \\ \small{(500K)}}      & \multicolumn{1}{p{1.75cm}}{\centering \textbf{Post-LN} \\ \small{(1M)}}      & \multicolumn{1}{p{2cm}}{\centering \textbf{RealFormer} \\ \small{(500K)}} \\ \hline
GLUE                   & 83.84                   & 84.01                   & 84.34       \\
v1.1 \small{(F1)}      & 91.46\tiny{0.18}   & 91.68\tiny{0.12}   & 91.56\tiny{0.09}       \\
v1.1 \small{(EM)}      & 84.87\tiny{0.24}   & 85.15\tiny{0.13}   & 85.06\tiny{0.12}       \\
v2.0 \small{(F1)}      & 81.44\tiny{0.50}   & 82.51\tiny{0.12}   & 82.52\tiny{0.55}       \\
v2.0 \small{(EM)}      & 78.64\tiny{0.48}   & 79.57\tiny{0.12}   & 79.54\tiny{0.54}       \\ \hline
Overall                & 83.97                   & 84.37                   & \textbf{84.51}    \\ \hline
\end{tabular}
\caption{\label{table:500k} Downstream development set results of fine-tuning BERT-Large with Post-LN and RealFormer pre-trained with different number of steps. v*.*: SQuAD version, EM: exact match. Overall: First average across SQuAD and then GLUE. Numbers in smaller font are standard deviations. All numbers are scaled by 100.}
\end{table}



\paragraph{How well does RealFormer perform with a larger learning rate?}
As suggested by some recent works (\emph{e.g.},~\citet{Xiong-2020-preln}), Pre-LN Transformer may benefit from using larger learning rates compared to Post-LN. To this end, we follow the pre-training procedure detailed earlier and switch to a larger learning rate, 2e-4, to pre-train BERT-Large with the three Transformer models. MLM accuracy on the development set with training steps is shown in Figure~\ref{fig:lr}. We can see that 
\begin{itemize}
    \item Both Pre-LN and RealFormer can reap some benefits of using larger learning rates;
    \item RealFormer seems to benefit a bit more in this case (73.94\%  74.31\%) compared to Pre-LN (73.21\%  73.46\%). Note that Post-LN diverged with the learning rate of 2e-4.
\end{itemize}
It means that RealFormer can outperform Post-LN, the strongest baseline, actually with a prominent gap, 0.67\% (\emph{i.e.}, 74.31\% - 73.64\%) for pre-training, though with only minimal learning rate tuning.

\begin{figure}
\centering
\includegraphics[width=0.475\textwidth,keepaspectratio]{figures/learning-rate.png}
\caption{Masked Language Modeling (MLM) accuracy on development set of BERT-Large with different learning rates (best viewed in color). RealFormer seems to benefit more from using a larger learning rate compared to Pre-LN. Note that Post-LN diverged with 2e-4.}
\label{fig:lr}
\end{figure}



\paragraph{How does RealFormer qualitatively differ from the baseline Transformers?}
We conduct one empirical study to understand the differences between RealFormer and Post-/Pre-LN Transformers. 
We randomly sample 8,192 examples from the held-out development set and visualize the distribution of attention probabilities of each token (excluding padding) in these examples across all layers and all heads in the three pre-trained BERT-Base models in Table~\ref{table:bert-mlm}.
In particular, for each (token, layer, head) triplet, we compute the entropy of the attention weights (probabilities) as the ``sparsity measure'' of attentions. Intuitively, as entropy gets lower, the attention weight distribution becomes more skewed and therefore attention is sparser.

\begin{figure*}[!t]
\centering
\includegraphics[width=\textwidth,keepaspectratio]{figures/realformer-entropy.png}
\caption{Distribution of entropies of the attention probabilities of the tokens of 8,192 held-out examples using the pre-trained BERT-Base with \textbf{RealFormer} (see Section~\ref{sec:pre-train}). Attention heads in each layer are ordered by their medians of entropies for better legibility. Distributions are re color-coded based on the median of entropies: RED (median  4.5), YELLOW (1.5  median  4.5), BLUE (median  1.5). \emph{I.e.}, colder colors mean sparser attentions. There is a clear trend that higher layers tend to have \emph{sparser} attentions.}
\label{fig:realformer-entropy}
\end{figure*}

In a similar fashion to~\citet{Ramsauer-2020-hopfield}, we use violin plots to show the entropy distributions of RealFormer (see Figure~\ref{fig:realformer-entropy}). Plots for Post-LN and Pre-LN Transformers are included in Appendix (Figure~\ref{fig:post-ln-entropy} and Figure~\ref{fig:pre-ln-entropy}). Each row is a layer in BERT-Base and each column is an attention head. For better legibility, (1) for each layer, we sort the attention heads in ascending order of the median of entropies; and (2) we also color code these plots to help distinguish heads with relatively sparse attentions (BLUE: median  1.5) and relatively dense attentions (RED: median  4.5) from the rest (YELLOW: 1.5  median  4.5).
From the three figures we can see that attention tends to get sparser for later (upper) layers for all three Transformers. However, RealFormer differs from the other two in the following two ways:
\begin{itemize}
    \item RealFormer has \emph{significantly sparser} attentions for top layers (layer 9-11);
    \item RealFormer tends to have lower variance across all layers, which means that attention density is less input-dependent.
\end{itemize}
We hypothesize that the above two properties might be a sign of stableness and could benefit fine-tuning.


\begin{figure*}[!t]
\centering
\includegraphics[width=\textwidth,keepaspectratio]{figures/realformer-jsd.png}
\caption{Distribution of Jensen-Shannon Divergence (JSD) of attention probabilities in (vertically) adjacent attention heads, \emph{i.e.}, . Based on 8,192 held-out examples using the pre-trained BERT-Base with \textbf{RealFormer} (see Section~\ref{sec:pre-train}). Distributions are color-coded based on the median of JSD: RED (median  0.75), YELLOW (0.25  median  0.75), BLUE (median  0.25). \emph{I.e.}, colder color means more ``similar'' attention heads across adjacent layers.}
\label{fig:realformer-jsd}
\end{figure*}


\paragraph{How much do attention heads in layer  resemble those in layer ?}
Since RealFormer uses a residual attention scheme, it is interesting to show to what extent an attention head is ``relying on'' the corresponding head in the previous layer.
To this end, we take each of the three pre-trained BERT-Base models in Table~\ref{table:bert-mlm} and compute the Jensen-Shannon Divergence (JSD) between attention probabilities in each pair of \emph{vertically} adjacent heads, \emph{i.e.}, , for  and .
Instead of computing one scalar value for each head pair, we show the full distribution based on the tokens in 8,192 held-out examples, \emph{i.e.}, each data point is the JSD between the attention probabilities of a token at these two heads.

Figure~\ref{fig:realformer-jsd} and Figure~\ref{fig:post-ln-jsd} in Appendix demonstrate the JSD distributions for RealFormer and Post-LN Transformer\footnote{Note that JSD results from Post-LN are used only as a reference. We expect them to be ``random'' because there is no correspondence between heads in adjacent layers for Post-/Pre-LN. Proof: An equivalent Post-/Pre-LN can be constructed by permuting the order of attention heads in a layer (and the corresponding variables).} respectively via violin plots. We can see that RealFormer tends to have \emph{significantly lower} JSD values, especially for heads in middle layers. This might mean that RealFormer has some regularization advantages and provides one hypothesis for why it tends to outperform Post-LN more for larger models. Note that  can still be useful even if it has exactly the same attention probabilities with  because of the existence of the FFN sub-layer and the potential differences in value matrices (\emph{i.e.},  in Eq.~\ref{eq:attention}).


\paragraph{Can dropout outperform residual attention for regularizing large models?}
One may wonder whether increasing dropout rate can already regularize large models well so that residual attention is not necessary. To this end, we experiment with different dropout rates for pre-training BERT-Large with different Transformers (following the procedures in Section~\ref{sec:pre-train}). Results are collected in Table~\ref{table:dropout}, from which we can see that (1) RealFormer outperforms the two baselines across all dropout settings, and (2) simply increasing dropout rate can not regularize Transformer models as well as what residual attention appears to be doing.


\begin{table}
\centering
\begin{tabular}{crrrr}
\hline \textbf{Dropout} & \textbf{Post-LN} & \textbf{Pre-LN} & \textbf{RealFormer} \\ \hline
0\%   & 71.16\%  & 69.80\%  & \textbf{71.30\%}  \\
10\%  & 73.64\%  & 73.21\%   & \textbf{73.94\%}  \\
20\%  & 73.21\%  & 72.97\%   & \textbf{73.66\%}  \\
\hline
\end{tabular}
\caption{\label{table:dropout} Masked Language Modeling (MLM) accuracy on development set of BERT-Large with different dropout rates. When dropout rate is 0\%, we report the best possible number with early-stop because all models start to overfit at around 500K steps.}
\end{table}





\section{Conclusions}
We propose RealFormer, a novel and simple Transformer architecture based on the idea of residual attention.
We show that it can be used as a drop-in replacement of Transformer in BERT.
Quantitatively, it considerably outperforms two canonical Transformer architectures, Post-LN and Pre-LN, on both pre-training and downstream tasks including GLUE and SQuAD. Furthermore, on downstream tasks, RealFormer even outperforms baselines pre-trained with twice the amount of epochs. 
Qualitatively, we show that RealFormer tends to have comparatively \emph{sparser} attentions, both within heads and across heads in adjacent layers. 
Finally, we show that RealFormer can benefit significantly from hyper-parameter tuning, though the main results in this paper are not based on it.


\bibliography{acl2020}
\bibliographystyle{acl_natbib}


\appendix

\section{Appendices}
\label{sec:appendix}
\begin{figure*}[!t]
\centering
\includegraphics[width=\textwidth,keepaspectratio]{figures/post-ln-entropy.png}
\caption{Distribution of entropies of the attention probabilities of the tokens of 8,192 held-out examples using the pre-trained BERT-Base with \textbf{Post-LN Transformer} (see Section~\ref{sec:pre-train}). Attention heads in each layer are ordered by their medians of entropies for better legibility. Distributions are color-coded based on the median of entropies: RED (median  4.5), YELLOW (1.5  median  4.5), BLUE (median  1.5). \emph{I.e.}, colder colors mean sparser attentions. Note that here top layers (layer 9-11) tend to have larger entropies compared to RealFormer, which means that attentions are relatively \emph{denser}.}
\label{fig:post-ln-entropy}
\end{figure*}

\begin{figure*}[!t]
\centering
\includegraphics[width=\textwidth,keepaspectratio]{figures/pre-ln-entropy.png}
\caption{Distribution of entropies of the attention probabilities of the tokens of 8,192 held-out examples using the pre-trained BERT-Base with \textbf{Pre-LN Transformer} (see Section~\ref{sec:pre-train}). Attention heads in each layer are ordered by their medians of entropies for better legibility. Distributions are color-coded based on the median of entropies: RED (median  4.5), YELLOW (1.5  median  4.5), BLUE (median  1.5). \emph{I.e.}, colder colors mean sparser attentions. Note that here top layers (layer 9-11) tend to have larger entropies compared to RealFormer, which means that attentions are relatively \emph{denser}.}
\label{fig:pre-ln-entropy}
\end{figure*}

\begin{figure*}[!t]
\centering
\includegraphics[width=\textwidth,keepaspectratio]{figures/post-ln-jsd.png}
\caption{Distribution of Jensen-Shannon Divergence (JSD) of attention probabilities in (vertically) adjacent attention heads, \emph{i.e.}, . Based on 8,192 held-out examples using the pre-trained BERT-Base with \textbf{Post-LN Transformer} (see Section~\ref{sec:pre-train}). Distributions are color-coded based on the median of JSD: RED (median  0.75), YELLOW (0.25  median  0.75), BLUE (median  0.25). \emph{I.e.}, colder color means more ``similar'' attention heads across adjacent layers.}
\label{fig:post-ln-jsd}
\end{figure*}

\end{document}
