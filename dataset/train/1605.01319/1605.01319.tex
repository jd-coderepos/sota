\documentclass[
final
]{dmtcs-episciences}




\usepackage{subfigure}
\usepackage{mathrsfs}







\usepackage{amsmath,amssymb,amsfonts,euscript}
\usepackage{algorithmic, algorithm}
\usepackage{graphicx}
\usepackage[ansinew]{inputenc}
\usepackage{graphicx}
\usepackage{color}
\usepackage{mathrsfs}






\newtheorem{prethm}{{\bf Theorem}}
\renewcommand{\theprethm}{{\arabic{prethm}}}
\newenvironment{thm}{\begin{prethm}{\hspace{-0.5
				em}{\bf}}}{\end{prethm}}


\newtheorem{prepro}{{\bf Theorem}}
\renewcommand{\theprepro}{{\Alph{prepro}}}
\newenvironment{pro}{\begin{prepro}{\hspace{-0.5
				em}{\bf}}}{\end{prepro}}

\newtheorem{preprop}{{\bf Proposition}}
\renewcommand{\thepreprop}{{\arabic{preprop}}}
\newenvironment{prop}{\begin{preprop}{\hspace{-0.5
				em}{\bf}}}{\end{preprop}}

\newtheorem{precor}{{\bf Corollary}}
\renewcommand{\theprecor}{{\arabic{precor}}}
\newenvironment{cor}{\begin{precor}{\hspace{-0.5
				em}{\bf}}}{\end{precor}}


\newtheorem{preconj}{{\bf Conjecture}}
\renewcommand{\thepreconj}{{\arabic{preconj}}}
\newenvironment{conj}{\begin{preconj}{\hspace{-0.5
				em}{\bf}}}{\end{preconj}}


\newtheorem{predefi}{{\bf Definition}}
\renewcommand{\thepredefi}{{\arabic{predefi}}}
\newenvironment{defi}{\begin{predefi}{\hspace{-0.5
				em}{\bf}}}{\end{predefi}}





\newtheorem{preremark}{{\bf Remark}}
\renewcommand{\thepreremark}{{\arabic{preremark}}}
\newenvironment{remark}{\begin{preremark}\rm{\hspace{-0.5
				em}{\bf}}}{\end{preremark}}



\newtheorem{preexample}{{\bf Fact}}
\renewcommand{\thepreexample}{{\arabic{preexample}}}
\newenvironment{example}{\begin{preexample}\rm{\hspace{-0.5
				em}{\bf}}}{\end{preexample}}

\newtheorem{prelem}{{\bf Lemma}}
\renewcommand{\theprelem}{{\arabic{prelem}}}
\newenvironment{lem}{\begin{prelem}{\hspace{-0.5
				em}{\bf}}}{\end{prelem}}

\newtheorem{prelam}{{\bf Lemma}}
\renewcommand{\theprelam}{{\Alph{prelam}}}
\newenvironment{lam}{\begin{prelam}{\hspace{-0.5
				em}{\bf}}}{\end{prelam}}


\newtheorem{preprob}{{\bf Problem}}
\renewcommand{\thepreprob}{{\arabic{preprob}}}
\newenvironment{prob}{\begin{preprob}{\hspace{-0.5
				em}{\bf.}}}{\end{preprob}}


\newtheorem{preproof}{{\bf Proof}}
\renewcommand{\thepreproof}{}
\newenvironment{proof}[1]{\begin{preproof}{\rm
			#1}\hfill{}}{\end{preproof}}
\newenvironment{prooff}[1]{\begin{preproof}{\rm
			#1}}{\end{preproof}}

\newtheorem{preali}{{\bf Proof of Theorem 1.}}
\renewcommand{\thepreali}{}
\newenvironment{ali}[1]{\begin{preali}{\rm
			#1}\hfill{}}{\end{preali}}
\newenvironment{alif}[1]{\begin{preali}{\rm
			#1}}{\end{preali}}

\newtheorem{prealii}{{\bf Proof of Theorem 2.}}
\renewcommand{\theprealii}{}
\newenvironment{alii}[1]{\begin{prealii}{\rm
			#1}\hfill{}}{\end{prealii}}
\newenvironment{aliif}[1]{\begin{prealii}{\rm
			#1}}{\end{prealii}}


\newtheorem{prealiii}{{\bf Proof of Theorem 3.}}
\renewcommand{\theprealiii}{}
\newenvironment{aliii}[1]{\begin{prealiii}{\rm
			#1}\hfill{}}{\end{prealiii}}
\newenvironment{aliiif}[1]{\begin{prealiii}{\rm
			#1}}{\end{prealiii}}


\newtheorem{prealiiii}{{\bf Proof of Theorem 5.}}
\renewcommand{\theprealiiii}{}
\newenvironment{aliiii}[1]{\begin{prealiiii}{\rm
			#1}\hfill{}}{\end{prealiiii}}
\newenvironment{aliiiif}[1]{\begin{prealiiii}{\rm
			#1}}{\end{prealiiii}}


\newtheorem{prealiiiii}{{\bf Proof of Theorem 6.}}
\renewcommand{\theprealiiiii}{}
\newenvironment{aliiiii}[1]{\begin{prealiiiii}{\rm
			#1}\hfill{}}{\end{prealiiiii}}
\newenvironment{aliiiiif}[1]{\begin{prealiiiii}{\rm
			#1}}{\end{prealiiiii}}






\author{Arash Ahadi\affiliationmark{1}
	\and Ali Dehghan \affiliationmark{2}
	\thanks{E-mail Addresses: 
		   arashahadi@mehr.sharif.edu (Arash Ahadi)  (Ali Dehghan).}}
\title[-SAT problem and its applications in dominating set problems]{-SAT problem and its applications in dominating set problems}


\affiliation{
Department of
	Mathematical Sciences, Sharif University of Technology, Tehran,
	Iran\\
	Systems and Computer Engineering Department, Carleton University, Ottawa,   Canada
}
\keywords{-SAT; Computational complexity; Independent dominating set;  Perfect codes; Regular graphs; Incidence coloring.}



\received{2016-5-5}

\revised{2019-3-7}

\accepted{2019-8-5}
\begin{document}
\publicationdetails{21}{2019}{4}{9}{1464}
\maketitle
\begin{abstract}{
The satisfiability problem is known to be -complete in general and for many restricted
cases. One way to restrict instances of -SAT is to limit the number of times a variable can be occurred. It was shown that for an instance of 4-SAT with the property that every variable appears in exactly 4 clauses (2 times negated and 2 times not negated), determining whether there is an assignment for variables  such that every clause contains exactly two true variables and two false variables is -complete. In this work, we show that deciding the satisfiability of 3-SAT with the property that every variable appears in exactly four clauses (two times negated and two times not negated), and each clause contains at least two distinct variables is -complete. We call this problem -SAT. For an  -regular graph  with ,  it was asked in [Discrete Appl. Math., 160(15):2142--2146, 2012] to determine whether for a given independent set   there is an independent dominating set  that dominates  such that ?
As an application of -SAT problem  we show that  for every , this problem is  -complete.
Among other results, we study the relationship between 1-perfect codes and the incidence coloring of graphs and as  another application of our complexity results,
we  prove
that for a given cubic graph  deciding whether  is 4-incidence colorable  is -complete.
}\end{abstract}



\section{Introduction}

The satisfiability problem is known to be -complete in general and for many restricted
cases, for example see \cite{MR3544062,  MR3386014, MR3864719, MR2184613, MR3810276, MR2500722}. Finding the strongest possible restrictions under which the satisfiability problem remains -complete
is important since this can make it easier to prove the -completeness
of new problems by allowing easier reductions.
An instance of -SAT is a set of clauses that are disjunctions of exactly  literals. The problem is to determine
whether there is an assignment of truth values to the variables such that all the clauses are satisfied.
One way to restrict instances of -SAT is to limit the number of times a variable can be occurred.
Consider a 4-SAT formula with the property that each clause contains four variables and each variable appears four times in the formula, twice negated and twice not negated,
determining whether there is a truth assignment for the formula such that in each clause there are exactly two true literals is -complete. In this work, we show that a similar version of this problem is -complete \cite{Puzzle}.


\subsection{-SAT problem}

For a given formula   a truth assignment is a mapping which
assigns to each variable one of the two values  or .
A truth assignment satisfies a clause  if  contains at
least one literal whose value is .
A truth assignment satisfies a CNF formula (a
Boolean formula in Conjunctive Normal Form) if it satisfies
each of its clauses. Given a CNF formula , the satisfiability
problem asks to determine if there is a truth assignment
satisfying .
We show that deciding the satisfiability of 3-SAT with the property that every variable appears in exactly four clauses (two times negated and two times not negated), is -complete. We call this problem -SAT.
\\ \\
{\em -SAT problem.}\\
\textsc{Instance}: A 3-SAT formula  such that  every variable appears in exactly four clauses (two times negated and two times not negated),  also each clause contains at least two distinct variables. \\
\textsc{Question}: Is there a truth assignment for the variables  of formula  such that each clause in  has at least one  true literal?
\\ \\
Note that if we consider  3-SAT problem  with the property that every variable appears in two clauses (one time positive and one time negative), then the problem is always  satisfiable.
Also, Tovey in \cite{tovey1984simplified} showed that
instances of 3-SAT in which every variable occurs three times are always satisfiable (this is an immediate corollary
of Hall's Theorem).
Also, the following similar restriction is mentioned in the book Computational Complexity by Papadimitriou (page
183): ``Allowing clauses of size two and three with each variable appearing three times and each
literal at most two is  -complete (but if all clauses have size three it is in )." Motivated by above results, in this work  we study the computational complexity of -SAT problem.

\subsection{Independent dominating sets}

Suppose that  is a  graph and let .
We say  is a dominating set for , if
for every vertex , we have  or there is a vertex  such that . 
For a given graph  and independent set , finding an independent dominating set  that dominates  has a lot of applications in the concept of dynamic coloring of graphs, see for example \cite{MR2935408, dehhh, MR3679602}. 
Motivated by those applications, 
for an -regular graph ,  it was asked in \cite{dehhh} to determine whether for a given independent set , there is an independent dominating set  that dominates  such that ?
As an application of -SAT problem we show that  for every , this problem is  -complete.




\begin{thm}\label{thm1}\\
	(i) -SAT problem is -complete.\\
	(ii) Let  be a fixed integer. Given , where  is an -regular graph and  is  a  maximal independent set of , it  is -complete to determine whether
	there is an independent dominating set  that dominates  such that .
\end{thm}


The vertex set of every graph without isolated vertices can be
partitioned into two dominating sets \cite{ID3}.
For any  , Heggernes and Telle   showed that it is -complete to determine whether a graph can
be partitioned into  independent dominating sets \cite{HT}.
It was shown that it is -hard to determine the chromatic index of a given -regular graph for any  \cite{MR689264}. Heggernes and Telle  reduced this problem to their problem. For a given -regular graph  they construct a graph  such that the chromatic index of  is  if and only if the vertices of   can be partitioned into  independent dominating sets.
It was shown in Appendix that the graph  can be partitioned into  independent dominating sets. Determining the computational complexity of deciding whether the vertices of a given connected cubic graph  can be partitioned into a number of independent dominating sets is unsolved and has a lot of applications in proving the  -hardness results for other problems.
Here, we focus on this problem and present an application.

We show that deciding whether the vertices of a given  graph  can be partitioned into a number of  independent dominating sets is  -complete, even for restricted class of graphs.

\begin{thm}\label{thm4}
	\\
	 For a given connected graph  with at most two numbers in its degree set, determining  whether the vertices of  can be partitioned into a number of independent dominating sets is  -complete.\\
	 Determine  whether the vertices of a given  3-regular graph can be partitioned into a number of independent dominating sets is  -complete.
\end{thm}


\subsection{Incidence coloring}

There is a close relationship between 1-perfect codes and the incidence coloring of graphs. We will use this relationship and prove a new complexity result for  the incidence coloring of  cubic graphs.
An {\it incidence} of a graph  is a pair  with , , such that  and  are
incident. Two distinct incidences  and  are adjacent if one of the following holds:\\
 , or\\
  and  are adjacent and .\\
An {\it incidence coloring} of a graph
 is a mapping from the set of incidences to a color set such that adjacent incidences of
 are assigned distinct colors. The {\it incidence chromatic number} is the minimum number of
colors needed and denoted by .

The concept of incidence coloring was first introduced by Brualdi and   Massey  in 1993 \cite{Kh2}. They said that
determining the incidence chromatic number of a given  cubic graph is an interesting question. After that the incidence coloring of cubic graphs were investigated by several authors \cite{Kh5, MR3636883, Kh4, Kh6}.
In 2005 Maydanskiy proved that the Incidence Coloring Conjecture\footnote{The incidence coloring conjecture  states that any graph can be incidence-colored with
	 colors, where  is the maximum degree of the graph.} holds for any graph with
 \cite{Kh4}. Therefore, for  a given cubic graph , .
For a graph  with , if the degree of any vertex of  is 1 or 3, then the graph  is called a semi-cubic
graph. In 2008, it was shown that it is -complete to determine if a semi-cubic graph is -incidence colorable \cite{Kh5}. 
Furthermore, recently Janczewski {\it et al.} proved that the incidence 4-coloring problem for semi-cubic bipartite graphs is -complete \cite{janczewski2017incidence}. 
Here, by using the relationship between 
incidence coloring of graphs and independent dominating sets,
we improve the previous complexity results  and show the following theorem.


\begin{thm}\label{TF}
	For a given 3-regular graph  deciding whether  is 4-incidence colorable  is -complete.
\end{thm}

\section{Notation}

We follow \cite{MR1567289, MR1367739} for terminology and
notation are not defined here, and we denote  by .
We denote the vertex set and the edge set of
 by  and , respectively. The maximum degree
and minimum degree of  are denoted by  and .
Also, for every  and , ,  and   denote the degree of ,  the neighbor set of 
and the set of vertices of  which has a neighbor in , respectively.
We say that a set of vertices are {\it independent} if there is no edge
between these vertices.
The {\it independence number}, , of a graph  is the size of a largest independent set of
.
A {\it clique} in a  graph   is a subset of its vertices such that every two vertices in the subset are connected by an edge.
A {\it dominating set} of a graph  is a subset  of  such that every vertex
not in  is joined to at least one vertex  of .
For , a {\it proper edge -coloring} of  is a function , such that if  share a common endpoint,
then  and  are different.
The smallest integer  such that
 has a proper edge -coloring is called the {\it chromatic index} of  and denoted by . By Vizing's theorem   the chromatic index of a graph  is equal to either  or  \cite{MR0180505}.


\section{Proofs}

Here, we show that   -SAT problem is -complete. Next, by using  that complexity result we prove that if 
 is a fixed integer, then for a given , where  is an -regular graph and  is  a  maximal independent set of , it  is -complete to determine whether
there is an independent dominating set  that dominates  such that .

\begin{ali}{
		It was shown that -SAT is -complete \cite{Puzzle}. 
\\ \\
{\em -SAT.}\\
\textsc{Instance}: A 4-SAT formula  such that  each variable appears four
times in the formula, twice negated and twice not negated.\\
\textsc{Question}: Is there a truth assignment for the variables  of formula   such that in each clause there are exactly two true
literals?
\\ \\		
We prove the two parts of the theorem together.
		Assume that  is a fixed integer. Let  be a given formula in -SAT problem. Assume that  has  the set of variables  and the set of clauses .
		We transform the formula  into a
		formula   such that in 
		each variable appears four
		times in the formula, twice negated and twice not negated. Also,
		 has a  satisfying assignment  if and only if  has a  satisfying assignment such that in each clause there are exactly two true literals. Next, we transform the formula  into  an  -regular  graph  with a maximal independent set 
		such that the graph  has an independent dominating set  for  if and only if the formula  has a  satisfying assignment.
		Our proof consists of five steps.\\
		{{\bf Step 1.}}\\
		For every clause  in , consider the ten clauses  in  (for each variable ,  means one of  or ) such that the number of negative literals in each of the clauses in  is not exactly 2.
		In other words, for instance  and  are in , but  is not in .
		
		Since in  each variable appears four
		times, twice negated and twice not negated. In  every literal appears 20 times (each variable appears 40 times). Also,  has a  satisfying assignment such that in every clause there are exactly two true literals if and only if  has a satisfying assignment (there is at least one true literal in each clause).\\
		{{\bf Step 2.}}\\
		For every clause  in , consider two new variables ,  and put the following four clauses in :
		
		\begin{center}
			, .
		\end{center}
		
		It is easy to see that the formula  has a  satisfying assignment if and only if the formula  has a  satisfying assignment.\\
		In the formula  some of the variables appear 40
		times. Call them old  variables. For each old variable , consider the new variables  and for every , , put the following clause in :
		
		\begin{center}
			   if   is odd   otherwise. 
		\end{center}
		
		Without loss of generality suppose that the old variable  appears negated in  and appears not negated in . For each ,  replace  in  (respect.  in ) with  ( respect. ). Call the resulting formula . In  each variable appears four
		times in the formula, twice negated and twice not negated. It is easy to see that  has a  satisfying assignment if and only if  has a  satisfying assignment.\\
		{{\bf Step 3.}}\\
		Let  be a finite family of (not necessarily distinct) subsets
		of a finite set . A system of distinct representatives (SDR) for the family  is a
		set  of distinct elements of  such that  for all . Hall's Theorem says that  has a system of distinct representatives if
		and only if  for all subsets  of  \cite{MR1367739}.
		
		Let  be a given formula with the set of variables  and the set of clauses .
		Let  and for every clause , .
		In  each variable appears four
		times in the formula, twice negated and twice not negated.
		Consider any union of  of the sets . Since each  contains at least 2 distinct elements and no literal is contained in more than 2 sets, the union contains at least  distinct
		elements. Therefore, by   Hall's Theorem, there exists a system of distinct representatives
		of . For each clause  denote its representative literal by . Note that there is a polynomial-time algorithm which finds an SDR, when ever it exists.\\
		{{\bf Step 4.}}\\
		For every variable , put a copy of the complete bipartite graph  with the vertex set , where  and .
		Also, for every clause  put the vertex .
		Join the vertex  to one of the vertices  or . Also, join the vertex  to one of the vertices  or  and join the vertex  to one of the vertices  or , such that in the resulting graph .
		
		Next, for every clause  join the vertex  to the vertices .
		Call the resulting graph . In  the degree of each clause vertex  is  and .\\
		{{\bf Step 5.}}\\
		Let  be a complete graph with the vertices . Let  and . Consider two copies of . For every vertex  with , put a copy of  and join the vertex  to the vertex  of the first copy of the graph , and join the vertex   to the vertex  of the second copy of the graph .
		Call the resulting -regular graph .
		
		In the following we introduce the members of maximal independent set .
		
		{{\bf  Members of .}}\\
		{{\bf Step 1.}}\\
		For every subgraph    of  put the set of vertices  in .\\
		{{\bf Step 2.}}\\
		For every subgraph   of  put the vertices  and  in . ( was introduced in Step 5, in the construction of ).
		
		Let  be an independent dominating set for  and
		suppose that  is an arbitrary clause. Without loss of generality suppose that . By the structure of  at least one of the vertices of the set  is in . On the other hand, for every variable , since we put a copy of complete bipartite graph  with the vertex set , where  and  in .
		Therefore, if  contains a vertex from  , then it does not have any vertex from  and vice versa.
		First, suppose that  has an independent dominating set  for .
		Let  be a function such that  if and only if at least one of the vertices  is in .
		
		
		It is easy to see that
		 is a   satisfying assignment for . Next, let   be
		a   satisfying assignment for . For every , put   in  if and only if . It is easy to extend this set into an independent dominating set for . This completes the proof.
}\end{ali}


Here, we prove that deciding whether the vertices of a given  graph  can be partitioned into a number of  independent dominating sets is  -complete, even for connected graphs with at most two numbers in their degree set and  3-regular graphs.



\begin{alii}{
		 It was shown that  3-colorability of planar 4-regular
		graphs  is NP-complete \cite{MR573644}. For a given 4-regular
		graph  with  vertices we construct a  graph  with degree set 
		such that  the vertices of  can be partitioned into a number of  independent dominating sets is  if and only if  is 3-colorable. Define:
		\\ \\
		\begin{tabular}{r  l}
			  & 
			\\
			 & \\
			&\\
			&
		\end{tabular}
		\\ \\
		First, suppose that  is 3-colorable and let  be a proper vertex coloring.
		Consider the following partition for the vertices of :
		
		,
		
		
		,
		\\ \\
		where .
		It is easy to see that theses sets are disjoint independent dominating sets for  and a partition for the vertices of .
		
		
		Now, assume that  is not 3-colorable.
		Let  and .
		To the contrary suppose that  is a partition of the vertices of  and each  is an independent dominating set for .
		Consider the following partition for the vertices of :
		
		
		.
		
		.
		\\ \\
		By the structure of  for every independent dominating set  and  we have .
		Therefore, for each , , so .
		Therefore .
		On the other hand, since  is not 3-colorable, therefore for every independent dominating set , we have . Consequently . But this is a contradiction. This completes the proof.
		\\
		\\
		  Given a graph ,
		a subset  of its vertex set is a 1-perfect code if
		 is an independent set and every vertex not in  is at distance one from exactly one vertex of .
		In other words:
		
		\begin{center}
			 is 1-perfect code 
		\end{center}
		
		It was shown that for a given 3-regular graph  determining whether the vertices of  can be partitioned into l-perfect codes is -complete \cite{Kh}.
		Note that every l-perfect code in a 3-regular graph on  vertices has size .
		So the vertices of a given 3-regular graph  can be partitioned into l-perfect codes if and only if the vertices of  can be assigned 4 different colors in such a way that closed neighborhood of each vertex is
		assigned all 4 colors, i.e.,   (The square of a graph , denoted by
		, is the graph obtained from  by adding a new edge joining each pair of vertices at
		distance 2). Therefore, from \cite{Kh} we have the following corollary:
		
		\begin{cor}
			For a given 3-regular graph  determining whether   is -complete.
		\end{cor}
		
		
		Let  be a 3-regular graph. Let . Then the vertices of  can be partitioned into l-perfect codes  if and only if the vertices of  can be partitioned into independent dominating sets. This completes the proof.
}\end{alii}

Next, we show that for a given 3-regular graph  deciding whether  is 4-incidence colorable  is -complete.

\begin{aliii}{
		An {\it strong vertex coloring} of graph  is a proper vertex coloring of  such that for any ,  and  are
		assigned distinct colors. If  is an strong vertex coloring of  and , then  is called {\it -strong-vertex colorable} and  is a {\it -strong-vertex coloring} of , where  is a color set.
		
		It was shown that for a given graph  whose vertices have degree equal to  or 1 is -incidence
		colorable if and only if  is -strong-vertex colorable \cite{Kh5}. Since for a given 3-regular graph  determining whether   is -complete (for more details, see Part  in the proof of Theorem \ref{thm4}), thus for a given 3-regular graph  deciding whether  is 4-incidence colorable is -complete.
}\end{aliii}



\section{Conclusion and Future Works}


\subsection{-SAT problem}

In this work, we proved that  deciding the satisfiability of 3-SAT with the property that every variable appears in exactly 4 clauses (2 time negated and 2 times not negated) and each clause contains at least two distinct variables is -complete. We called this problem -SAT.
Note that if we consider  3-SAT problem  with the property that every variable appears in 2 clauses (1 time positive and 1 time negative), then the problem is always  satisfiable.
Also, Tovey in  \cite{tovey1984simplified} showed that
instances of 3-SAT in which every variable occurs three times are always satisfiable.
It is interesting to determine the complexity of  -SAT problem when each clause has exactly three distinct variables. 
\\ \\
\textbullet    Determine the computational complexity of -SAT problem when each clause has exactly three distinct variables. 

\subsection{Independent dominating sets}
In this work, as an application of -SAT problem, we proved that for each , the following problem is -complete: "Given , where  is an -regular graph and  is  a  maximal independent set of , determine whether
there is an independent dominating set  for  such that ".
Regarding this result, solving the following question can be interesting.
\\ \\
\textbullet    Determine the computational complexity of deciding whether a given  regular graph has two disjoint independent dominating sets.
\\ \\
We proved that determine  whether the vertices of a given  3-regular graph can be partitioned into a number of independent dominating sets is  -complete. However, one further step does not seem trivial.
\\ \\
\textbullet  Determine the computational complexity of deciding whether  the vertices of a given  connected regular graph can be partitioned into a number of independent dominating sets.
\\ \\
In \cite{dehhh}, it was  proved that
if  is a non-empty graph, and  is an independent set of , then there exists  such that,  is an independent dominating set for  and
.
\\ \\
\textbullet    Determine  nontrivial upper bounds for the minimum cardinality of  and also  among all  two  independent dominating sets  and  of a graph  for some important family of graphs such as regular graphs.

\subsection{Incidence coloring}


We showed that for a given 3-regular graph  deciding whether  is 4-incidence colorable  is -complete. The complexity of that problem for the family of planar 3-regular graphs can be interesting.


\section{Appendix}
Here, we show that the vertices of  can be partitioned into  independent dominating sets.
First, we introduce the construction of . For a given -regular graph , define:


Now, consider the following useful lemma which will be used in our proof.
\\ \\
{\bf Lemma 1}. Let  be a -regular graph. There is function  such that:\\
1. For every edge  in , .\\
2. For each vertex  in , for every two edges  and  incident with , .
\\ \\
{\bf Proof of Lemma 1}. Consider the bipartite graph  ( is obtained from  by replacing each edge with a path with
exactly one inner vertex).   Since for every bipartite  graph ,  (see for example \cite{MR1367739}). Therefore . Consequently, there is function . 
\\

Partitioning the vertices of a graph into  independent dominating sets is equivalent to a -labeling of the vertices such that each vertex has no neighbors labeled the same as itself and at least one neighbor labeled with each of the other  labels.\\
Let  be a graph which is constructed from  and  be a function  such that for every edge  in ,  and for each vertex  in , for every two edges  and  incident with , .
Consider the following  labeling for the vertices of  :

, .
\\
For every  define  such that

.

It is easy to see that  is a -labeling of the vertices such that each vertex has no neighbors labeled the same as itself and at least one neighbor labeled with each of the other  labels. So, the vertices of  can be partitioned  into  independent dominating sets. 

\bibliographystyle{plain}
\bibliography{Dynamic}


\end{document}