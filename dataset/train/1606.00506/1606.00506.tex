\begin{figure}[t]
	
	\textbf{Syntax}
	
	
	\textbf{Dynamics}
	\begin{mathpar}
		\inference[\rtit{mVer}]{}{\vV \traSS{\act} \vV}\and
		\inference[\rtit{mAct}]{}{\prf{\act}{\mV}  \traSS{\act} \mV} \and
		\inference[\rtit{mNAct}]{\actt \neq \act}{\prf{\notAct{\act}}{\mV}  \traSS{\actt} \mV}
		\\
		\inference[\rtit{mSelL}]{\mV \traSS{\act} \mV'}{\esel{\mV}{\mVV}  \traSS{\act} \mV'} \and
		\inference[\rtit{mSelR}]{\mVV \traSS{\act} \mVV'}{\esel{\mV}{\mVV}  \traSS{\act} \mVV'} 
		\and
		\inference[\rtit{mConj}]{\mV \traSS{\act} \mV' & \mVV \traSS{\act} \mVV'}{\mtimes{\mV}{\mVV}  \traSS{\act} \mtimes{\mV'}{\mVV'}} \end{mathpar}
	
	\textbf{Instrumentation}
	\begin{mathpar}
		\inference[\rtit{iMon}]{p \traSS{\act} p' & \mV \traSS{\act} \mV'}{\sys{\mV}{p} \traSS{\act} \sys{\mV'}{p'}} 
		\and
		\inference[\rtit{iTer}]{p \traSS{\act} p' & \mV \traSSN{\act}  }{\sys{\mV}{p} \traSS{\act} \sys{\stp}{p'}}
		\and
		\inference[\rtit{iAsy}]{p \traSS{\acttau} p'}{\sys{\mV}{p} \traSS{\acttau} \sys{\mV}{p'}}
\end{mathpar}
	
	\caption{Monitors and Instrumentation}
	\label{fig:monit-instr}
	\end{figure}

Figure~\ref{fig:monit-instr} describes  the monitoring framework used to analyse servers purporting to adhere to some advertised contract.
It defines the syntax of these monitors, which follow the general structure used in earlier works \cite{FrancalanzaAI:2015:uHML,BLC:2011:RVL:2000799.2000800:RV-LTL} whereby monitors may reach any one of the three verdicts \Verd, namely acceptance, rejection, or the inconclusive verdict. In addition to the basic prefixing patterns used in  \cite{FrancalanzaAI:2015:uHML,Fra16}, we here also use action complementation, , to denote any action \emph{apart from} .  As in \cite{FrancalanzaAI:2015:uHML,Fra16}, a monitor is allowed to branch, \esel{\mV}{\mVV}, depending on the actions observed at runtime.  We also find it convenient to express a merge monitor operator that facilitates the composition of monitor specifications, \mtimes{\mV}{\mVV}.  

The semantics of a monitor is given in terms of the LTS defined by the rules in Figure~\ref{fig:monit-instr}. This is best viewed as the evolution of a monitor in response to a (finite) execution trace , consisting of a sequence of actions .  Verdicts are irrevocable when reached, and do not change upon viewing further actions in the trace (rule \rtit{mVer}). Prefixing releases the guarded monitor when the expected pattern is encountered (rules \rtit{mAct} and \rtit{mNAct}). The rules  \rtit{mSelL} and \rtit{mSelR} describe left and right monitor branching as expected, whereas rule \rtit{mConj} describes the synchronous evolution of merged monitors.

A \emph{monitored server contract} consists of a server  that is instrumented with a monitor \mV, denoted as \sys{\mV}{p}.  The behaviour of monitored contracts is defined as an LTS through the rules stated in Figure~\ref{fig:monit-instr}, and relies on the \resp LTSs of the monitor and the server.  Rule \rtit{iMon} states that if a server can transition with action  and the  monitor can follow this by transitioning with the same action, then in an instrumented server  they transition in lockstep. However, if the monitor cannot follow such a transition the instrumentation forces it to terminate with an inconclusive verdict, \stp, while the process is allowed to proceed unaffected; see rule \rtit{iTer}.  Finally, rule \rtit{iAsy} allows a contract to evolve independently from the monitor when performing silent  moves (which are unobservable to the monitor).  We refer to a sequence of transitions from a monitored contract as a \emph{monitored computation} and use the standard notation  that abstracts over -moves in trace . 

A few comments are in order.  First, we highlight the fact that in the operational semantics for monitored systems of \figref{fig:monit-instr},  the monitor does \emph{not} have access to the internal state of the server generating the trace, and its observations are limited to the execution that the server chooses to exhibit at runtime.  This is meant to model the RV scenarios mentioned in \secref{sec:introduction}, where the source of the executing  system cannot be analysed: from the point of view of the runtime monitoring and verification, the server description is merely used to generate traces.  Second, we note that, in a monitored server setup, any visible behaviour is instigated by the server, relegating the instrumented monitor to a \emph{passive} role that merely follows the server actions. Stated otherwise, the server  \emph{drives} the behaviour in a monitored system  and dictates  the execution path that the monitor can analyse at runtime.

In what follows, we explain how monitors work through a series of examples. The exposition focuses on monitors that produce rejection verdicts, but the discussion can be extended to acceptance verdicts in a straightforward manner.


\begin{example} \label{ex:monit-contract}
   The monitor  checks for violations from  contracts that are expected to adhere to (\ie be supercontracts of) the contract  . In fact, the monitor reaches a rejection verdict whenever  a contract either emits an action that is not  at runtime, \prf{\notAct{\coact{a}}}{\no}, or else  emits an action that is not  following action , \prf{\coact{a}}{\prf{\notAct{b}}{\no}}.  Consider the server ; when instrumented with our monitor we can observe the following monitored computation whereby the monitor reaches a rejection verdict, \no.
   
   By contrast, when the server   is instrumented with the monitor, no rejection verdict is reached; in particular, the final transition below is derived using rule \rtit{iTer} because .
   
We emphasise the fact that monitor termination through rule \rtit{iTer} is crucial to avoid unwanted detections.  Consider a variant of the earlier  monitor,  , which now reports violations whenever it observes the trace consisting of the action  followed by the action . When composed with the system  we observe the following monitored computation.

   At transition (\ref{eq:iter}), the server can perform an action, , that the monitor is not able to follow (\ie it is \emph{not} specified how the monitor should behave at that point should it observe action ).  Accordingly, the semantics instructs the monitor to terminate (prematurely) with an inconclusive verdict.   There are two instrumentation alternatives that could have been adopted, both of which are arguably wrong from a monitoring perspective.  The first option would have been to prohibit the server from exhibiting action , which goes against the tenet that the monitor should adopt a passive role and not interfere with the execution of the program it monitors.  The second option is arguably even worse: we could have let the server transition and left the monitor in its present state, \ie    
   , 
   but then this would have led to an \emph{unspecified/erroneous} detection at the next transition 
   .
    \exqed
\end{example}

\begin{example} \label{ex:monit-contract2}
  The server  is \emph{not} a supercontract of  according to \defref{def:server-pre}.  Crucially, however, in an RV setting, monitor detection depends on the runtime behaviour exhibited by the server.  This contrasts with other forms of verification which may be allowed to explore \emph{all} the execution paths of a server under scrutiny.\footnote{In the general case, a pre-deployment verification technique may also analyse \emph{infinite} paths.}
  
  In the first monitored computation above, the server exhibits the behaviour described by the trace , which prohibits the monitor from detecting any violations. However, the same server exhibits a different trace  in the second monitored computation which permits monitor detection.  The rejection verdict is in fact reached after the first visible transition on action , and then preserved throughout the remainder of the computation. \exqed



\end{example}


\begin{example} \label{ex:monit-contract-plus}
  We can monitor for violations of the contract  by composing two submonitors that monitor for the constituents.  Specifically, since the monitor  checks for violations of contract \prf{c}{\nil} and, the minimally extended monitor  checks for violations of \prf{\coact{a}}{\prf{b}{\nil}} as discussed in \exref{ex:monit-contract}, we can construct the composite monitor  to monitor for violations of .  
  
  When the composite monitor is instrumented with the contract it is expected to monitor for, we note that it does not reach a rejection along \emph{every} (parallel) submonitor.  
  
  By contrast, the violating contract above generates a rejection along every submonitor. \exqed 
\end{example}

 \exref{ex:monit-contract-plus} clearly suggests a definition of monitor rejection.

\begin{definition}[Rejection] \label{def:rej} A monitor  is in a rejection state, denoted as \rej{\mV}, whenever it is of the form .  We overload this predicate to denote a server  being rejected by a monitor , defined formally as  
	
\end{definition}


\begin{example} \label{ex:monit-rej}  The monitor   rejects server ,  as well as server ,  because both may exhibit an execution trace that leads the monitor to a rejection state.  By contrast,   does \emph{not} reject server . Recalling monitor  from \exref{ex:monit-contract-plus}, we can also state that it rejects server , \rej{\prf{b}{\nil},\mV}. \exqed
\end{example}







