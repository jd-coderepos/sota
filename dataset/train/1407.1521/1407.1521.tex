





We now consider a very simple type of protocols that we call \emph{fire-and-forward} protocols.
For convenience, in this model we allow nodes to receive and transmit messages at the same step. 
(Later we will explain how this condition can be removed.) In 
a fire-and-forward protocol, at any time , any node  can either be idle or
make one of two types of transmissions:
\begin{description}
\item{\emph{Fire}:}  can transmit its own rumor, or
	\item{\emph{Forward}:}  can transmit the rumor received in step , if any.
\end{description} 
In Section~\ref{sec: nlogn randomized} we show that there exists a randomized
fire-and-forward protocol that accomplishes information gathering in time .
This raises the question whether this running time can be achieved by a
deterministic fire-and-forward protocol. 
(As before, in the deterministic case we assume that the nodes are labelled .)
There is a trivial deterministic fire-and-forward protocol with running time : release
all rumors one at a time, spaced at intervals of length . In this section we show that
this can be improved to  and that this bound is optimal.

When a node does not forward the rumor received in the previous step, we say that
this rumor is \emph{dropped}. Note that we allow a node to drop a received rumor even
if it does not fire. We will extend the definition of collision to include the
situation when a node attempts to fire right after receiving a rumor (in which case
nothing will be transmitted).

Of course, fire-and-forward protocols use only bounded messages. The key property of
fire-and-forward protocols is that any rumor, once fired,
moves up the tree one hop per step, unless either it collides, or is dropped,
or it reaches the root. 

If rumors fired from two nodes collide at all, they will collide at
their lowest common ancestor. This happens only when the
difference in times between these two firings is equal to the difference
of their depths in the tree. More precisely, let  be the tree on input,
denote by  the depth of a node  in , and suppose that some node
 fires its rumor at time . Then this rumor will reach the root if
no other node  fires at time .

The fire-and-forward protocol we develop in this section is \emph{oblivious}, in the sense that
(i) the decision whether to fire or not depends only on the label of the node and the
current time, and (ii) if a node received a rumor in the previous step and it
is supposed to fire at the current step, then no message is transmitted. (This can be
thought of as extending the definition of collisions.)
In fact, it is not hard
to see that any protocol can be turned into an oblivious one without affecting its
asymptotic running time. The idea is that leaves of the tree receive no information at all
during the computation. For any fire-and-forward protocol  that runs in time , and for
any tree , imagine that we
run this protocol on the tree  obtained by adding a leaf to any
node  and giving it the label of . Label the original nodes with the remaining labels.
This at most
doubles the number of nodes, so  will complete in time  on .
(We tacitly assume here that  for any constant , which is true for the
time bounds we consider.)
In the execution of  on  the leaves receive no information and all rumors
from the leaves will reach the root. This implies that
if we apply  on  and ignore all information received during the computation,
the rumors will also reach the root. In other words, after this modification, we obtain
an oblivious protocol  with running time .



\subsection{An  Upper Bound}

We now present our -time fire-and-forward protocol. As explained earlier, this protocol 
should specify a set of firing times for each label, so that for any mapping , 
that maps each label to the depth of the node with this label,
each node will have at least one firing time for which there will not be a collision
along the path to the root. We want each of these firing times to be at most .
To this end, we will partition all labels into batches, each of size 
roughly , and show that for any batch we can define such collision-avoiding
firing times from an interval of length .
Since we have about  batches, this will give us running time .

Our construction of firing times is based on a concept of dispersers, defined below,
which are reminiscent of various \emph{rulers} studied in number theory,
including Sidon sequences~\cite{wikipedia_sidon},
Golomb rulers~\cite{wikipedia_golomb}, or sparse rulers~\cite{wikipedia_sparse}. 
The particular construction we give in the paper is, in a sense, a
multiple set extension of a Sidon-set construction 
by Erd{\"o}s and Tur{\'a}n~\cite{Erdos_Turan_41}.

We now give the details. For  and , let
. Let also  be a positive integer.
A set family  is called an \emph{-disperser} if 
for each function  and
each  we have .
The intuition is that  represents the set of firing times of node  and
  represents 's depth in the tree.
Then the disperser condition says that some firing in  will not collide
with firings of other nodes.




\begin{lemma}\label{lem: disperser}
There exists an -disperser with  and .
\end{lemma}

\begin{proof}
Let  be the smallest prime such that . 
For each  and  define

We claim that for any  and any  the equation 
 has at most two solutions .
For the proof, fix  and one solution . Suppose that
 is a different solution. Thus we have
. After substituting and 
rearranging, this can be written as

The expression on the left-hand side is strictly between  and , so
both sides must be equal . This implies that

From equation~(\ref{eqn: n^1.5 first eq}), the assumption that
 implies that  and .
We can then divide the two equations, getting

With addition and multiplication modulo ,  is a field. Therefore 
for any  and , and any , equations (\ref{eqn: n^1.5 first eq}) and (\ref{eqn: n^1.5 third eq})
uniquely determine  and , completing the proof of the claim.

\smallskip

Now, let  and . 
By Bertrand's postulate we have , which implies
that  and .
For each , define . 
It is sufficient to show that the sets  satisfy the
condition of the -disperser. 

The definition of the sets  implies that  for each .
Fix some some  and  from the definition of dispersers. It remains to
verify that .
For  and , we say that  \emph{kills} 
if .
Our earlier claim implies that any  kills at most two values in .
Thus all indices  kill at most  integers in
, which implies that there is some  that is not killed by 
any . For this , we will have
,
completing the proof that  is indeed an -disperser.
\end{proof}


We now describe our algorithm.



\begin{myalgorithm}{}
Let  be the -disperser from Lemma~\ref{lem: disperser}. 
We partition all labels (and thus also the corresponding nodes) arbitrarily
into batches , for , with each batch  having 
nodes (except the last batch, that could be smaller).
Order the nodes in each batch arbitrarily, for example according to increasing labels.

The algorithm has  phases. Each phase  consists of  steps in the time interval
. In phase , the algorithm transmits rumors from batch ,
by having the -th node in  fire at each time , for .
Note that in the last  steps of each phase none of the nodes fires.
\end{myalgorithm}



\emparagraph{Analysis.}
We now show that Algorithm~ correctly performs gathering in any -node
tree in time .  Since , we have
. Also, , so the total run time of the protocol is .

It remains to show that during each phase  each
node in  will have at least one firing that will send its rumor to the root  without collisions.
Fix some tree  and let  be the depth of the
th node in batch .
For any batch  and any , if  is the
th node in  then  will fire at times , for .
From the definition of dispersers, there is
 such that  for each .
This means that the firing of  at time  will not collide with any
firing of other nodes in batch . Since the batches are separated by empty intervals
of length , this firing will not collide with any firing in other batches.
So 's rumor will reach . 

Summarizing, we obtain our main result of this section.



\begin{theorem}\label{thm: n^1.5 fire-and-forward upper bound}
There is a fire-and-forward protocol for information gathering in trees
with running time .
\end{theorem}

It remains to explain how we can extend this result to the model where
nodes are not allowed to receive and transmit at the same time. We only
give a sketch of the argument. The idea is that if some node  transmits a rumor
 and receives a rumor , and if  fired at time , then 
 fired at time . We can then extend the
definition of collisions to include this situation. Incorporating this into
the construction from Lemma~\ref{lem: disperser}, any index  may now
kill more than two 's, but not more than four. So taking ,
we still will always have an  that is not killed by any .
The details will be provided in the full paper.



\subsection{An  Lower Bound}

In this section we show that any fire-and-forward protocol needs time  to
deliver all rumors to the root for an arbitrary tree. Fix some fire-and-forward protocol
. Without loss of generality, as explained earlier in this section, we can
assume that  is oblivious, namely that the set  of firing times of each
node  is uniquely determined by its label. 

Let  be the running time of .
We first give the proof under the assumption that the total number of firings
of , among all nodes in , 
is at most . Later we will show how to extend our
argument to protocols with an arbitrary number of firings.

We will show that if  then
 will fail even on a ``caterpillar'' tree,
consisting of a path  of length  with  leaves attached to the
nodes of this path. (For convenience we use  nodes
instead of , but this does not affect the asymptotic lower bound.)
This path  is fixed, and the label assignment to these nodes is not important
for the proof, but, for the ease of reference,
we will label them , in order in which they appear on the path,
with node labeled  being the root. 
The leaves have labels from the set . 
To simplify the argument we will identify
the labels with nodes, and we will refer to the node with label  simply
as ``node ''. (See Figure~\ref{fig: caterpillar}.)
The objective is to show that there is a way to attach
the nodes from  to  to make  fail, which means that
there is at least one node  whose all firings will collide with
firings from other nodes.

\begin{figure}[ht]
\begin{center}
\includegraphics[width=4.5in]{caterpillar.pdf}
\caption{A caterpillar graph from the proof, for . 
The nodes on the path  are represented by 
rectangles, and the leaves are represented by circles. In this example,
the root is  and .
}
\label{fig: caterpillar}
\end{center}
\end{figure}


Without loss of generality, assume that  is a multiple of .
We let , and we divide the time range  into 
bins of size , where the th bin is the interval , for .
If a node  fires at time , we say that a node  
\emph{covers} this firing if  has a firing at time  such that  and  are
in the same bin.
For a set  of firings of , denote by 
 the set of nodes that cover the firings in .

\paragraph{Claim~A:} For each node  there is a set of firings 
such that .

\smallskip

The proof of Claim~A is by contradiction. Suppose that there exists a node 
with the property that for each  we have .
Then Hall's theorem would imply that there is a perfect matching between the firing times 
in  that cover these firings.
Let the firing times of   be ,
and for each , let  be the node matched to  in this matching.
By the definitions of bins and covering, each 
fires at some time , where .  We can then construct
a caterpillar tree by
attaching  to node  and attaching each  to node  on . 
In this caterpillar tree, the firing of  at each time  will collide with
the firing of  at time . So the rumor from  will not
reach the root, contradicting the correctness of . 
This completes the proof of Claim~A.

\smallskip

For each , we now fix a set  from Claim~A.
Let  be the set of ordered pairs  of different nodes 
for which
there is a bin which contains a firing from  and a firing from ,
in this order in time.

We will bound  in two different ways. On the one hand, Claim~A implies that 
each  appears as the first element in fewer than  pairs in . 
Adding up over all , and using the assumption that the total number of firings
is at most , we have .

On the other hand, we can also establish a lower bound on , as follows.
Choose a specific representative firing  from each . 
For each bin , let  be the number of representatives in the th bin.
Any two representatives in bin  contribute one pair to . 
So , for .
Since , if we let all  take real values then
the value of  will be minimized when all  are equal,
which implies that , for some .
Thus we get that .

Combining the bounds from the last two paragraphs, we obtain that
, which implies that .
This completes the proof of the lower bound, with the assumption that the total
number of firings does not exceed .

\smallskip

We now consider the general case, without any assumption on the total number
of firings. 
Suppose that  is some protocol with running time  .
Using a probabilistic argument and a reduction to the above special case,
we show that then we can construct a 
caterpillar tree and a node  for which  will fail.

Choose some  such that  and
. (For example, we can take .)
Let  be a random set of nodes, where each node 
is included in  independently with probability .  
Let also  be the set of times at which only nodes from  fire in . 
We claim that with probability , the following two 
properties will hold:
\begin{description}
\item{(i)} , and
\item{(ii)} The total number of firings in  is at most .
\end{description}
To justify this claim, note the expected cardinality of  is ,
so that probability that (i) is not true is
, because  and the expectation of 
 is .
To justify (ii), consider some time  where  has  firings.
Then the probability that  is  , so
the contribution of  to the expected number of firings in  is .
Therefore the probability that (ii) is violated is .

We can thus conclude that for some choice of  both properties (i) and (ii)
hold. Let  be this choice, , and let  be the corresponding
set of times .

We convert  into a protocol  for labels in , as follows.
(Note that  may not be of the form , but that does not affect the validity of our argument.)
For each time , if  then the firings in  are the same as in ;
if  then no node in  fires.
By our earlier argument, there must be a caterpillar tree  with nodes from  and a node 
for which  fails. 

Let  be a modified tree obtained from  by adding all nodes that are not in 
as children of the parent of .
Consider now a firing of  in  at time .
If , then this firing  collides with another firing from . 
If , then, by the definition of , there is a node  outside of 
that fires at the same time (otherwise  would be included in ) and is a sibling
of  in . So again, this firing from  will collide with the firing of .
We can thus conclude that the rumor from  will not reach the root, completing the proof of
the lower bound.

We thus obtain our lower bound.



\begin{theorem} \label{thm: fireandforward lower bound}
If  is a deterministic fire-and-forward protocol for information gathering
in trees, then the running time of  is .
\end{theorem}
