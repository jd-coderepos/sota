\documentclass[letterpaper, 11pt]{article}
\usepackage[top=1in, bottom=1in,margin=1in]{geometry}

\usepackage{amsmath,amsthm,amsfonts, amssymb}
\theoremstyle{plain}
\newtheorem{thm}{Theorem}
\newtheorem{lem}{Lemma}
\newtheorem{prop}{Proposition}
\newtheorem*{cor}{Corollary}
\theoremstyle{definition}
\newtheorem{defn}{Definition}
\newtheorem{conj}{Conjecture}
\newtheorem{exmp}{Example}
\newtheorem{remark}{Remark}

\usepackage{longtable}

\usepackage{graphicx}
\graphicspath{{./fig/}}

\usepackage[labelformat=simple]{subcaption}
\renewcommand\thesubfigure{(\alph{subfigure})}

\usepackage{authblk}

\usepackage[toc,page]{appendix}

\usepackage{hyperref}
\usepackage[usenames, dvipsnames]{color}



\title{Learning the LMP-Load Coupling From Data:\\ A Support Vector Machine Based Approach}
\author[1]{Xinbo Geng\thanks{gengxbtamu@tamu.edu}}
\author[1]{Le Xie\thanks{le.xie@tamu.edu}}
\affil[1]{Department of Electrical and Computer Engineering, Texas A\&M University}

\begin{document}
\maketitle


\begin{abstract}
This paper investigates the fundamental coupling between loads and locational marginal prices (LMPs) in security-constrained economic dispatch (SCED). 
Theoretical analysis based on multi-parametric programming theory points out the unique one-to-one mapping between load and LMP vectors. 
Such one-to-one mapping is depicted by the concept of system pattern region (SPR) and identifying SPRs is the key to understanding the LMP-load coupling.
Built upon the characteristics of SPRs, the SPR identification problem is modeled as a classification problem from a market participant's viewpoint, and a Support Vector Machine based data-driven approach is proposed. 
It is shown that even without the knowledge of system topology and parameters, the SPRs can be estimated by learning from historical load and price data.
Visualization and illustration of the proposed data-driven approach are performed on a 3-bus system as well as the IEEE 118-bus system. 
\end{abstract}


\section{Introduction} \label{sec:introduction}
A fundamental issue with electricity market operation is to understand the impact of operating conditions (e.g. load levels at each bus) on the locational marginal prices (LMPs).  
This paper examines this key issue of the relationship between nodal load levels and LMPs. 
This issue is further compounded by the increasing levels of demand response and variable resources in the grid.

In the power systems literature, reference \cite{Conejo2005} is among the pioneering works that uses perturbation techniques to compute the sensitivities of the dual variables in SCED (e.g. LMPs) with respect to parameters (e.g. the nodal load levels).
This sensitivity calculation method is widely used in subsequent researches. However, this approach is valid only for small changes and the marginal generator stays the same. Reference \cite{Li2007} observed the ``step changes'' of LMPs with respect to increasing system load level and discovered that new binding constraints (transmission or generation) are the reason of the ``step changes''.
This is followed by further analysis on identifying the critical load levels (CLLs) that trigger such step changes of LMPs \cite{Li2009}, \cite{Bo2009}, \cite{Bo2011}. 
This line of work assumes that the system load change is distributed to each bus proportional to the base case load, which, in many instances, do not necessarily represent the real-world situations. 
Reference \cite{Zhou2011} analyzed this problem using quadratic-linear programming (QLP) and the concepts of \emph{system patterns} and \emph{system pattern regions (SPRs)} were first introduced. The SPRs depict the relationship between loads and LMPs in the whole load space, which is not confined in a small neighborhood of an operating point or constrained by a specific load distribution pattern. 
This paper is inspired by \cite{Zhou2011} but focuses on the case of piecewise linear generation costs, instead of the quadratic cost case in \cite{Zhou2011}.
The reason that we study the piecewise linear cost case is that piecewise linear cost curves are often quite representative of the market practice in the real world. In addition, some new theoretical results based on piecewise linear cost curves are derived, and are generalizable towards quadratic cost cases.









Characterizing the SPRs would provide important insights to both system operators and market participants. Reference \cite{Ji} advances the theory of SPR from system operator's perspective where the knowledge of system topology and parameters is available. For market participants, such knowledge is not necessarily available. Our previous work \cite{Geng2015} examines the issue from market participant's viewpoint and applies the geometric features of SPRs to identify them. 

This paper significantly advances our previous work by (1) completing the theoretical characterization of SPRs as a function of nodal load levels; (2) proposing a computational algorithm to identify SPRs using historical data; (3) introducing the posterior probabilities of SPRs with the presence of uncertain system parameters such as transmission limits; and (4) extending the algorithm to consider practical factors such as partial load information and loss component of LMPs.
 


The rest of the paper is organized as follows. 
Section \ref{sec:theoretical_analysis} provides the analysis of LMP-load coupling in SCED problem from the viewpoint of MLP theory, with an illustrative example. 
Section \ref{sec:sprs_with_varying_parameters} illustrates the changes of SPRs given changes of system parameters such as transmission limits. 
Based on the theoretical analysis, a data-driven algorithm for market participants to identify SPRs is described in Section \ref{sec:extended_data_driven_approach}. 
Section \ref{sec:case_study} illustrates the performance of the algorithm on the IEEE 118-bus system.
Section explores the impact of nodal load information, and Section \ref{sec:discussions} provides critical assessment
of the proposed method. Concluding remarks and future works
are presented in Section \ref{sec:conclusion}.











\section{Theoretical Analysis} \label{sec:theoretical_analysis}
\subsection{Notations} \label{sub:notations}
The notations of this paper are summarized below: mathematical symbols in hollowed-out shapes (e.g. ) represent spaces and symbols in Calligra font (e.g. ) stand for sets. The superscript ``'' indicates the variable is optimal, ``'' denotes estimated values (e.g. ). Variables with ``'' are expectations or average values (e.g. ). ``'' denotes the transpose of a vector or matrix (e.g. ). The subscript ``'' represents the th element of the vector (e.g. ), and the superscript ``'' represents the th element in a set (e.g. ).
The vector of  ones, matrix of  zeros and the  identity matrix are denoted by  and  and  respectively.



\subsection{Security Constrained Economic Dispatch} \label{par:reivew_sced}
In real-time energy market operations, the LMPs are the results from the security-constrained economic dispatch (SCED), which is formulated as follows: 

& \underset{P_G^{(k)}}{\min} && \sum_{i=1}^{n_b} c_i(P_{G_i}^{(k)}) & \label{eqn:static_sced_obj} \\
& \text{s.t.} && \sum_{i = 1}^{n_b}{P_{G_i}^{(k)}} = \sum_{j=1}^{n_b}{P_{D_j}^{(k)}} & & :\lambda_1 \label{eqn:static_sced_balance} \\
& && -F^+ \le H(P_G^{(k)}-P_D^{(k)}) \le F^+ & & :\mu^+, \mu^- \label{eqn:static_sced_transmission}\\
& && P_G^- \le P_G^{(k)} \le P_G^+ & & : \eta^+, \eta^- \label{eqn:static_sced_generation}

where  is the generation vector at time , and  is the load vector at time . We assume there are both generation and load at each bus. Let  denote the number of buses and  denote the number of transmission lines, then . 
 is the shift factor matrix.

This formulation considers each snapshot independently, therefore it is called \emph{static SCED} in this paper. For simplicity, we write  and  as  and  when discussing the static SCED.




The objective of SCED is to minimize the total generation cost and satisfy the transmission and generation capacity constraints while keeping the real-time balance between supply and demand. The generation cost function  of generator  is increasing and convex, and it is usually regarded as a  quadratic function or approximated by a piecewise linear function. 
To better reflect the current practice in electricity markets, this paper studies the SCED problem with piecewise linear generator bidding functions. And for the consideration of simplicity, the simplest form, i.e.  is being considered in this paper. 

A fundamental concept in electricity markets is the \emph{Locational Marginal Price. The LMP  at bus  is defined as the change of total system cost if the demand at node  is increased by 1 unit \cite{Kirschen2005}.} According to \cite{Wu1996}, the LMP vector  can be calculated by the following equation:




For better understanding, we start with the simplest case of
static SCED. More elaborated SCED formulations are in Section
\ref{sub:ramp_constraints}. Since the line losses are not explicitly modeled the
SCED formulation, the LMPs in this paper do not contain the
loss components. Further discussions on the loss component
are in Section \ref{sub:lmps_with_loss_components}.


\subsection{SCED Analysis via MLP} \label{sub:SCED_analysis_via_MLP}
In real-world market operations, the parameters associated with the SCED above are typically time-varying. Therefore, it is essential to 
understand the effects of parameters on the optimality of the problem. Multi-parametric Programming (MP) problem aims at exploring the characteristics of an optimization problem which depending on a \emph{vector of parameters} \cite{Borrelli2003}. Multi-parametric Linear Programming (MLP) theory, which is the foundation of this paper, pays special attention to Linear Programming (LP) problems.


In this paper, we would like to understand the impact of parameters (i.e., load levels, line capacities, etc) on the outcome of SCED (namely, the prices). We pose the problem in view of MLP, and analyze the theoretical properties.


In the reality, the LMP vector depends upon a number of factors, including: (1) the loads in the system; (2) line flow limits; (3) ramp constraints; (4) generation offer prices; (5) topology of the system; (6) unit commitment results. We first focus on the relationship between loads and LMPs assuming the other five factors remain unchanged; then Section \ref{sec:sprs_with_varying_parameters} takes the line flow limits and ramp constraints into account; the influence of generation offer prices is explored in Section \ref{sub:on_generation_offer_prices}. Future work will investigate the impacts of unit commitment results and the system topology changes on the prices.


Consider the static SCED in the standard MLP form\footnote{In other references (e.g. \cite{Adler1992a, Gal1972}), the primal form of the MLP problem is different. For the consideration of convenience of analyzing SCED problem, we follow the formulations in \cite{Borrelli2003}. Those two forms are interchangeable.}:

  \text{Primal:} & \min\{c^\intercal P_G: AP_G + s = b + WP_D, s\ge 0\} \\
  \label{eqn:SCED_MLP_Dual}
  \text{Dual:} &\max\{-(b+WP_D)^\intercal y: A^\intercal y=-c, y\ge 0\} 

where:

The load vector  is the vector of parameters , and the load space  is the parameter space . Since not every  in the load space leads to a feasible SCED problem,   denotes the set of all feasible vectors of loads. \cite{Gal1972} shows that  is a convex polyhedron in .

\begin{defn}[Optimal Partition/System Pattern]
\label{defn:optimal_partition}
For a load vector , we could find a finite optimal solution  and . Let  denote the index set of constraints where  for Eqn. (\ref{eqn:SCED_MLP_Primal}). The \emph{optimal partition}  of the set  is defined as follows:

  \mathcal{B}(P_D) &:= \{i: s_i^* = 0 \text{ for } P_D \in \mathcal{D}  \} \\
  \mathcal{N}(P_D) &:= \{j: s_j^* > 0 \text{ for } P_D \in \mathcal{D} \} 

Or in the dual form:

  \mathcal{B}(P_D) &:= \{i: y_i^* > 0 \text{ for } P_D \in \mathcal{D}  \} \\
  \mathcal{N}(P_D) &:= \{j: y_j^* = 0 \text{ for } P_D \in \mathcal{D} \} 

Obviously,  and . The optimal partition  divides the index set into two parts: binding constraints  and non-binding constraints .
In SCED, the \emph{optimal partition} represents the status of the system (e.g. congested lines, marginal generators), and is called \emph{system pattern}.
\end{defn}

\begin{defn}[Critical Region/System Pattern Region]
The concept \emph{critical region} refers to the set of vectors of parameters which lead to the same optimal partition (system pattern) :

For the consideration of consistency, the \emph{critical region} is called \emph{system pattern region} (SPR) in this paper.
\end{defn}
According to the definitions, each SPR is one-to-one mapped to a system pattern, the SPRs are therefore disjoint and the union of all the SPRs is the feasible set of vectors of loads: .
All the SPRs together represent a specific partition of the load space. The features of SPRs, which directly inherit from critical regions in MLP theory, are summarized as follows:
\begin{thm}
\label{thm:convexSPR}
The load space could be decomposed into many SPRs. Each SPR is a convex polytope. The relative interiors of SPRs are disjoint convex sets and each corresponds to a unique system pattern \cite{Zhou2011}. 
There exists a separating hyperplane between any two SPRs \cite{Geng2015}.
\end{thm}


\begin{lem}[Complementary Slackness]
\label{lem:complementary_slackness}
According to complementary slackness:

A_{\mathcal{B}} P_G^* &= (b+WP_D)_{\mathcal{B}} \label{eqn:B_primal}\\
A_{\mathcal{N}}P_G^* &< (b+WP_D)_{\mathcal{N}} \label{eqn:N_primal}\\
A_{\mathcal{B}}^\intercal y_{\mathcal{B}} &= -c, y_{\mathcal{B}} > 0 \label{eqn:determine_y}\\
y_{\mathcal{N}} &= 0  

where the  is the sub-matrix or the sub-vector whose row indices are in set , same meaning applies for .
\end{lem}

\begin{remark}
The supply-demand balance equality constraint in is rewritten as two inequalities in Eqn. (\ref{eqn:SCED_MLP_Primal}). These two inequalities will always be binding and appear in the binding constraint set  at the same time. One of them is redundant and therefore eliminated from the set . \emph{In the remaining part of the paper, set  denotes the set after elimination. }
\end{remark}
\begin{remark}
If the problem is not degenerate, the cardinality of binding constraint set  is equal to the number of decision variables (i.e. number of generators ) \footnote{This is consistent with the statement that the number of marginal generators equals to the number of congested lines plus one.}.
The matrix  is invertible and  is uniquely determined by .
\end{remark}

\begin{remark}
SCED problems with different generation costs will have different SPRs.
For a system pattern , its SPR would remain the same as
long as the generation cost vector  satisfies Eqn. (\ref{eqn:determine_y}).
\end{remark}

\begin{lem}
Within each SPR, the vector of LMPs is unique \cite{Ji}\cite{Geng2015}.
\end{lem}
The proof of this lemma follows Eqn. (\ref{eqn:determine_y}) (dual form of system pattern definition).
Since the system pattern  is unique within an SPR , therefore the solution  is unique for any 
And the vector of LMPs can be calculated using Eqn. (\ref{eqn:LMP}). This lemma also illustrates that the LMP vectors are discrete by nature in the case of linear costs.


\begin{thm}
\label{thm:diff_LMPs}
If the SCED problem is not degenerate, then different SPRs have different LMP vectors.
\end{thm}
The proof of Theorem \ref{thm:diff_LMPs} turns out to be non-trivial, and is described as follows. 
If two SPRs have the same LMP: , their energy components are the same because of the entry-wise equality, then Eqn. (\ref{eqn:LMP}) suggests that the congestion components should also be the same: . Given the fact that the null space of  is always non-empty\footnote{. The equality holds if and only if the topology of the system is a tree, where .}, a critical question arises: ``is it possible that  belongs to the null space of ?'' Or equivalently, ``is it possible that different congestion patterns have the same LMP vector?''
We show that the answer is ``no''. A complete proof of the theorem is provided in Appendix \ref{sec:proof_diff_LMP_diff_SPR}.

\subsection{An Illustrative Example} \label{sub:an_illustrative_example}
The 3-bus system in Fig. \ref{fig:3Bus2GeneSystem} serves as an illustrative example in this paper. It was first analyzed using the Multi-Parametric Toolbox 3.0 (MPT 3.0) \cite{Herceg2013}, results are shown in Fig. \ref{fig:SPR_3bus_mpt}. A Monte-Carlo simulation is conducted, with load vectors colored according to their LMPs. The theoretical results are verified by the Monte-Carlo simulation results.
Notice that  and  could be negative. This is for the consideration of renewable resources in the system, which are typically considered as negative loads. 

\begin{figure}[htbp]
  \centering
  \includegraphics[width=0.6\linewidth]{3BusSystem2GenRamp.png}
  \caption{3-bus System}
  \label{fig:3Bus2GeneSystem}
\end{figure}  


\begin{figure}[htbp]
  \centering
  \begin{subfigure}[t]{0.49\linewidth}
  \centering
  \includegraphics[width=\linewidth]{3Bus2GenSystemline606080.png}
  \caption{Theoretical Results Using MPT 3.0}
  \label{fig:SPR_3bus_mpt}
  \end{subfigure}
  \begin{subfigure}[t]{0.49\linewidth}
  \centering
  \includegraphics[width=\linewidth]{3bus2genNO.png} 
  \caption{Monte-Carlo Simulation}
  \label{fig:SPR_3busNO_mc}
  \end{subfigure}
  \caption{SPRs of the 3-bus System (Static SCED)}
\end{figure}  



\section{SPRs with Varying Parameters} \label{sec:sprs_with_varying_parameters}

Section \ref{sub:SCED_analysis_via_MLP} shows construction properties of the load space with fixed parameters of the system (e.g. transmission constraints). However, these parameters might be time-varying due to reasons like dynamic ratings or active ramping constraints. This subsection reveals more features of SPRs with respect to varying factors in the system.


\begin{lem}[Analytical Form of SPRs]
Let  represent the sub-vector , where  is the sub-matrix of the identity matrix  whose row indices are in set .
Then the analytical form of the SPRs could be solved from Eqn. (\ref{eqn:B_primal}) and Eqn. (\ref{eqn:N_primal}) as follows:

\end{lem}
We can calculate the analytical expressions of the SPRs using Eqn. (\ref{eqn:SPR_analytical_form}). An illustrative example with complete details is provided in Appendix \ref{sec:analytical_form_of_the_sprs_of_the_3_bus_system_in_fig_}.

\begin{remark}
Eqn. (\ref{eqn:SPR_analytical_form}) could be written as:

This indicates the shape of the SPR  only depends on two factors: (1) the corresponding system pattern ; (2) matrices  and , namely the shift factor matrix  according to Eqn. (\ref{eqn:details_of_the_A_W_b}).
Small changes of vector  only parallel-shift the SPRs' boundaries.
\end{remark}


\subsection{Dynamic Line Rating} \label{sub:dynamic_line_rating}
\emph{Dynamic line rating} (DLR), contrary to the \emph{static line rating} (SLR),  refers to the technology that optimizes the transmission capacity based on the real-time conditions such as ambient temperature and wind speed \cite{douglass1996real}.
It is considered to be more adaptive in maximizing the line potential while keeping the secure grid operation.

From dispatch point of view, DLR can be represented by the changes of transmission limits  in Eqn. (\ref{eqn:static_sced_transmission}). It changes the vector  in Eqn. (\ref{eqn:details_of_the_A_W_b}) and thus translate the boundaries of SPRs.

The 3-bus system in Fig. \ref{fig:3Bus2GeneSystem} with different transmission limits is analyzed via MPT 3.0. Compared with the standard transmission limits , when we increase the limits by  (Fig. \ref{fig:line_11}), SPR \#3 expands but SPR \#1, \#2 and \#4 shrink; when we decrease the limits by  (Fig. \ref{fig:line_09}), SPR \#3 shrinks but SPR \#1, \#2 and \#4 expand.
This verifies the claim that dynamic line ratings only shift the boundaries without altering the shapes of SPRs. 
The implication of having DLR is that SPRs in Fig. \ref{fig:spr_3bus_dlr_mc} are overlapping instead of completely separable in Fig. \ref{fig:SPR_3busNO_mc}. Details of the Monte-Carlo simulation are provided in Section \ref{ssub:3_bus_system_dlr}.

\begin{figure}[htbp]
  \centering
\begin{subfigure}[t]{0.49\linewidth}
  \centering
  \includegraphics[width=\linewidth]{3Bus2GenSystemline666688.png} 
  \caption{Line Limits: (66, 66, 88)}
  \label{fig:line_11}
  \end{subfigure}
  \begin{subfigure}[t]{0.49\linewidth}
  \centering
  \includegraphics[width=\linewidth]{3Bus2GenSystemline545472.png} 
  \caption{Line Limits: (54, 54, 72)}
  \label{fig:line_09}
  \end{subfigure}  
\caption{SPRs of the 3-bus System (Static SCED with DLRs)}
  \label{fig:SPR_3bus_dlr}
\end{figure}

\begin{figure}[htbp]
  \centering
  \includegraphics[width=0.6\linewidth]{3bus2genDLR}
  \caption{Monte-Carlo Simulation (Static SCED with DLRs)}
  \label{fig:spr_3bus_dlr_mc}
\end{figure}






\subsection{Ramping Constraints} \label{sub:ramping_constraints}
The analysis of SPRs can also be generalized to the dispatch models that include inter-temporal constraints such as ramping:

In Eqn.(\ref{eqn:ramp_const}),  and  represent the ramp up and down constraints of generators.


Adding ramp constraints to the static SCED problem is equivalent with replacing the generation capacity constraints Eqn. (\ref{eqn:static_sced_generation}) with:

When the ramp capacity is not binding, i.e.  and , the SCED problem is the same as the case where no ramp constraints are considered. The SPRs would be exactly the same as in Fig. \ref{fig:SPR_3bus_mpt} and \ref{fig:SPR_3busNO_mc}.
However, active ramp constraints change the actual generation constraints, and therefore change the parameter  in Eqn.(\ref{eqn:details_of_the_A_W_b}). 
This leads to parallel shift of the boundaries of SPRs. The impacts of ramping constraints on SPRs is similar with the case of dynamic line ratings.


The 3-bus system, again, is analyzed via both MPT 3.0 and Monte-Carlo simulation. 
Fig. \ref{fig:ramp_3030} and \ref{fig:ramp_100100} demonstrate the cases where ramp constraints are active. SPRs look similar with parallel changes on the boundaries. 
When analyzing the load and LMP data, we will again see the overlapping SPRs (Fig. \ref{fig:spr_ramp_mc}). 


\begin{figure}[htbp]
  \centering
  \begin{subfigure}[t]{0.49\linewidth}
  \centering
  \includegraphics[width=\linewidth]{3Bus2GenSystem_ramp_3030.png}
  \caption{Previous Generation: (30; 30)}
  \label{fig:ramp_3030}
  \end{subfigure}
  \begin{subfigure}[t]{0.49\linewidth}
  \centering
  \includegraphics[width=\linewidth]{3Bus2GenSystem_ramp_100100.png}
  \caption{Previous Generation: (100; 100)}
 \label{fig:ramp_100100}
  \end{subfigure}  
  \caption{SPRs of the 3-bus System (SCED with Ramp constraints)}
  \label{fig:spr_ramp_mpt}
\end{figure}

\begin{figure}[htbp]
  \centering
  \includegraphics[width=0.6\linewidth]{3bus2GenRamp.png}
  \caption{Monte-Carlo Simulation (SCED with Ramp constraints)}
  \label{fig:spr_ramp_mc}
\end{figure}








\section{A Data-driven Approach to Identifying SPRs} \label{sec:extended_data_driven_approach}
The SPRs depict the fundamental coupling between loads and LMP vectors. Massive historical data could help market participants estimate SPRs, understand the load-LMP coupling and then forecast LMPs. This section proposes a data-driven method to identify SPRs, which is a significant improvement of the basic method in \cite{Geng2015} by considering varying system parameters and the probabilistic nature of system parameters.

\subsection{The SPR Identification Problem} \label{sub:data_driven_approach_revisited}
\subsubsection{SPR Identification as a Classification Problem} \label{ssub:model_as_a_classification_problem}
A \emph{classifier} is an algorithm to give a \emph{label}  to each \emph{feature} vector . 
The feature vectors sharing the same labels belong to the same \emph{class}. The objective of the classification problem is to find the best classifier which could classify each feature vector accurately. For the parametric classifiers, there is always a training set, i.e. a group of feature vectors whose labels are known. There are two steps in a classification problem: training and classifying. \emph{Training} usually means solving an optimization problem over the training set to find the best parameters of the classifier. And \emph{classifying} is to classify a new feature vector with the trained classifier.

According to Section \ref{sub:SCED_analysis_via_MLP},  the load vectors in an SPR share many common features (e.g. vectors of LMPs). Theorem \ref{thm:diff_LMPs} proved that the LMP vectors are distinct for different SPRs. Therefore, one SPR can be regarded as a \emph{class} and the LMP vector is the \emph{label} of each class. Theorem \ref{thm:convexSPR} proves the existence of the separating hyperplanes. Since each separating hyperplane labels two SPRs with different sides, it turned out that the separating hyperplanes are classifiers and the key of identifying SPRs is to find optimal hyperplanes, which is exactly the objective of Support Vector Machine (SVM).




\subsubsection{SPR Identification with SVM} \label{ssub:spr_identification_with_svm}
Suppose there is a set of labeled load vectors for training and those load vectors belong to only \emph{two} distinct SPRs (labels ).
Then the SPR identification problem with a \emph{binary} SVM classifier (separable case) is stated below:

  \min_{w,b} & \qquad \frac{1}{2} w^\intercal  w   \label{eqn:separable_SVM_obj} \\ 
  \text{s.t} & \qquad y^{(i)}(w^\intercal  P_D^{(i)}-b) \ge 1, y^{(i)} \in \{-1,1\} \label{eqn:separable_SVM_cons}	

The word ``binary'' here specifies only two classes (i.e. SPRs) are being considered. Eqn. (\ref{eqn:separable_SVM_cons}) is feasible only when the two SPRs are not overlapping and there exists at least one hyperplane thoroughly separating them (separable case). For any load vector  in the load space, 
 represents the separating hyperplane where  is the norm vector to the hyperplane. Two lines satisfying  separate all the training data and formulate an area with no points inside. This empty area is called \emph{margin}. The width of the margin is , which is the distance between those two lines. The optimal solution refers to the separating hyperplane which maximizes the width of the margin , therefore the objective of the binary SVM classifier is to minimize the norm of vector .
\begin{figure}[htbp]
  \centering
  \includegraphics[width=0.5\linewidth]{SVM_SPR}
  \caption{SPR Identification Problem with SVM (Separable Case)}
  \label{fig:SVM_SPR}
\end{figure}






Due to the existence of multiple SPRs, multi-class classifiers are needed. Since Theorem \ref{thm:convexSPR} guarantees the existence of separating hyperplanes between every pair of SPRs, the ``one-vs-one'' multi-class SVM classifier is incorporated in the data-driven approach to identifying SPRs. Detailed procedures are summarized in Section \ref{sub:extended_data_driven_approach}.






\subsection{A Data-driven Approach} \label{sub:extended_data_driven_approach}
\subsubsection{SPR Identification with Varying System Parameters} \label{ssub:spr_identification_with_varying_system_parameters}
When the system parameters are varying (e.g. dynamic line ratings), two SPRs may overlap with each other. The SPR identification problem is no longer a separable case as in Section \ref{ssub:spr_identification_with_svm}. The SVM classifier needs to incorporate \emph{soft margins} to allow some tolerance of classification error. The slack variable  is added to Eqn. (\ref{eqn:separable_SVM_obj}) and penalties of violation  are added in the objective function. Large  indicates low extent of tolerance.

  \min_{w,b,s} & \qquad { \frac{1}{2} w^\intercal  w + C\sum_i s^{(i)} } \label{eqn:non_sep_SVM_obj} \\ 
  \text{s.t} & \qquad y^{(i)}(w^\intercal  P_D^{(i)} -b) \ge 1-s^{(i)} \label{eqn:non_sep_SVM_cons} \\
 			 & \qquad s^{(i)} \ge 0, y^{(i)} \in \{-1,1\} \nonumber	

\begin{figure}[htbp]
  \centering
  \includegraphics[width=0.5\linewidth]{SVM_SPR_non}
  \caption{SPR Identification Problem with SVM (Non-Separable Case)}
  \label{fig:SVM_SPR_non}
\end{figure}





\subsubsection{Fitting Posterior Probabilities} \label{ssub:fitting_posterior_probabilities}

The posterior probability is the probability that the hypothesis is true given relevant data or observations.
In the classification problem, the posterior probability can be stated as: .

Estimating the posterior probability is very helpful in practical problems \cite{Platt1999}. 
When identifying SPRs, knowing the posterior probability  is not only about knowing the classification result  ( belongs to SPR\#i), but also understanding the confidence or possible risk. The market participants could accordingly adjust their bidding strategy and reduce possible loss.

Although the posterior probabilities are desired, the standard SVM algorithm provides an uncalibrated value which is not a probability as output \cite{Platt1999}. Modifications are needed to calculate the \emph{binary} posterior probabilities . 
Common practice is to add a link function to the binary SVM classifier and train the data to fit the link function. Some typical link functions include sigmoid functions \cite{Platt1999} and Gaussian approximations \cite{hastie1998classification}. In this paper, the sigmoid link function is selected due to its general better performance than other choices \cite{Platt1999}. 

In general, there are more than two SPRs. What we really want to know is the \emph{multi-class} posterior probabilities . For short, we will use  to represent multi-class posterior probabilities. \cite{hastie1998classification} proposed a well-accepted algorithm to calculate multi-class posterior probabilities from pairwise binary posterior probabilities. This algorithm is incorporated in our approach and briefly summarized in Appendix \ref{sec:hastie_tibshirani_s_algorithm}.


\subsubsection{The Data-driven Approach} \label{ssub:extended_data_driven_approach}
There are three steps in the proposed data-driven approach (Fig. \ref{fig:flowchart}):
\begin{figure}[htbp]
  \centering
  \includegraphics[width=\linewidth]{flowchart}
  \caption{The Data-driven Approach}
  \label{fig:flowchart}
\end{figure}

\paragraph{Training} \label{par:step_1}
Suppose there are  different SPRs in the training data set. Each time two SPRs are selected, trained and we get a \emph{binary} SVM classifier. This pairwise training procedure is repeated  times and we collect  binary classifiers, namely the  separating hyperplanes between any two out of  SPRs.




\paragraph{Classifying/Predicting} \label{par:step_3}
Given load forecast , we could use the ``\emph{max-vote-wins}'' algorithm to get the classification results:
each binary classifier provides a classification result (vote) for the load forecast , the SPR which collects the most votes will be the final classification result. The load forecast  is therefore pinpointed to an SPR.
The LMP forecast: 
 where  is the index of the SPR winning the most votes. This step is independent of the data post-processing procedure.
\paragraph{Data Post-processing} \label{par:step_2}
Calculate posterior probabilities  for  by applying Platt's algorithm and then Hastie and Tibshirani's algorithm\footnote{Details of these two algorithms are summarized in Appendix \ref{sec:platt_s_algorithm_} and \ref{sec:hastie_tibshirani_s_algorithm}.}.
It is worth noting that the proposed approach is generalizable to many other scenarios with overlapping SPRs in the data, possible extensions are discussed in Section \ref{sub:on_posterior_probabilities}.









\section{Case Study} \label{sec:case_study}
In this section, we illustrate the proposed data-driven approaches on two systems.
\subsection{Performance Metrics} \label{sub:performance_measurements}
We first introduce the performance metrics.
\subsubsection{5-fold Cross Validation} \label{ssub:cross_validation}
To evaluate the performance of the model to an independent data set and avoid overfitting, the \emph{-fold cross validation} technique is being used.
In -fold cross-validation, the overall data set is randomly and evenly partitioned into  subsets. Every time a subset is chosen as validation data set, and the remaining  subsets are used for training. This cross-validation process is repeated  times ( folds), and each subset serves as the validation data set once.
The 5-fold cross validation is being used in this paper.



\subsubsection{Classification Accuracy} \label{ssub:classification_accuracy_eval}
\emph{Classification accuracy} is the most common criteria to evaluate the performance of classifiers. The classification accuracy  is the ratio of the correctly classified points in the validation data set.
When incorporating -fold cross validation, the classification accuracy of each fold () is calculated first, then the overall performance of the method is evaluated by the average classification accuracy 
.


\subsubsection{LMP Forecast Accuracy} \label{ssub:lmp_forecast_accuracy}
The proposed approach forecasts the LMP at every bus. The performance of LMP forecast at bus  is evaluated by the nodal LMP forecast accuracy , which is the average forecast accuracy of all the validation data points ()

The overall LMP forecast accuracy  evaluates the performance of LMP forecast for the whole system. It is the average of all the nodal LMP forecast accuracy  ():







\subsection{Static SCED with Static Line Ratings} \label{sub:case_studies_of_static_sced_with_static_line_ratings}
This section explores the simplest case: static SCED with SLRs. Since \cite{Geng2015} discusses the 3-bus system as well as the IEEE 24-bus system, we only examine the data-driven approach on an 118-bus system. The same dataset generated in this section is used in Section \ref{sub:how_the_nodal_load_levels_help_us_} as well.

\paragraph{System Configuration} \label{par:system_configuration}
Most of the system settings follow the \emph{IEEE 118-bus, 54-unit, 24-hour system} in \cite{Technology} but with the following changes: (1) the lower bounds of generations are set to zero, but the upper bounds of generators remain the same as in \cite{Technology}; (2) generation costs are linear. Details of the parameters are summarized in \cite{Geng2015c}.



\paragraph{Load} \label{par:Load}
\cite{Technology} also provides an hourly system load profile and a bus load distribution profile. With linear interpolation, the hourly system load profile is modified to be 5-min based. To account for the variability of loads, we assume the load at each bus follows normal distribution . The expectation  of each nodal load is calculated from the system load profile and bus load distribution profile, the standard deviation  is set to be  of the expectation. 1440 (5 days, 5-min based) load vectors are generated, then Matpower \cite{zimmerman2011matpower} solves these 1440 SCED problems and records 1440 LMP vectors. These 1440 load vectors and LMP vectors are the training and validation data.


\paragraph{Simulation Results} \label{par:simulation_results}
Results are summarized in Table \ref{tab:classification_accuracy_118bus}.
The classification accuracy is around 67\% but the LMP forecast is satisfying. When the classification result of a load vector is correct, the LMP forecast is correct for every bus, i.e. . 
It is worth noting that even if the classification fails, the overall LMP forecast still has accuracy about 90\%. This is because the classification errors happen between one SPR and it neighbors. LMPs of adjacent SPRs are similar due to the fact that only one active constraint is different (Lemma \ref{lem:adjacent_SPRs}). Therefore, the LMP forecast result is much more accurate than classification.

\begin{table}[htbp]
  \caption{Results of the 118-bus System (Static SCED with SLR)}
  \label{tab:classification_accuracy_118bus}
  \centering

  \begin{tabular}{l|cc}
  \hline

  \hline
  \textbf{Fold} & \textbf{Classification} & \textbf{LMP Forecast} \\
  \hline
    1 & 64.24\% & 96.82\% \\
    2 & 67.36\% & 96.71\% \\
    3 & 64.93\% & 96.95\% \\
    4 & 71.18\% & 97.34\% \\
    5 & 65.63\% & 96.84\% \\
  \hline
  \textbf{avg} & 66.67\% & 96.93\% \\
  \hline

  \hline
  \end{tabular}
\end{table}



\subsection{Static SCED with Dynamic Line Ratings} \label{sub:case_study_dlr}
\subsubsection{3-bus System} \label{ssub:3_bus_system_dlr}
We start with an illustrative 3-bus system example. This succinct example provides key insights and visualization of the proposed method.

\paragraph{Data} \label{par:3bus_data}
The parameters of the 3-bus system are presented in Fig. \ref{fig:3Bus2GeneSystem}.
The data set is generated using Matpower with the following assumptions: 
(1) the load vector is evenly distributed in the load space;
and (2) the transmission limits  is time-varying: for simplicity, we utilize the following model to calculate the real-time transmission limits :

 is the ``standard'' transmission limits and . It is the same as the case of static line ratings.  
 represents the major factor (e.g. ambient temperature or wind speed) that impacts the transmission capacities. All the data generated is visualized in Fig. \ref{fig:spr_3bus_dlr_mc}.








\paragraph{Simulation Results} \label{par:classification}
Table \ref{tab:classification_accuracy_3bus} summarizes the classification and LMP forecast accuracies. The accuracies are around 95\% because of the overlapping SPRs.


\begin{table}[htbp]
  \caption{Results of 3-bus System (5-fold Validation)}
  \label{tab:classification_accuracy_3bus}
  \centering

  \begin{tabular}{l|cccc}
  \hline

  \hline
  \textbf{Fold} & \textbf{Classification} & \textbf{LMP Forecast}  \\
  \hline
  1 & 93.967\% & 96.218\% \\
  2 & 93.236\% & 96.054\% \\
  3 & 94.150\% & 95.767\%\\
  4 & 95.612\% & 96.700\%\\
  5 & 94.150\% & 96.405\%\\
  \hline
  \textbf{avg} & 94.23\% & 96.23\%\\
  \hline

  \hline
  \end{tabular}
\end{table}



\paragraph{Posterior Probabilities} \label{par:posterior_probability_estimation}
The posterior probabilities are visualized. 
The posterior probabilities of an SPR compose a surface (Fig. \ref{fig:3bus_dlr_post_lmp_205030} and Fig. \ref{fig:3bus_dlr_post_lmp_205030}). When putting all the 5 surfaces of 5 SPRs together (shown in Fig. \ref{fig:3bus_dlr_post_all}), the five surfaces intersect with each other and formulate some ``mountains'' and ``valleys''. The ``mountains'' correspond to the inner parts of SPRs, where the overlapping of SPRs is almost impossible to happen. And the  ``valleys'' always locate at the boundaries among SPRs.

\begin{figure}[htbp]
  \centering
  \begin{subfigure}[t]{0.49\linewidth}
  \centering
  \includegraphics[width=\linewidth]{3bus_dlr_post_lmp_505050.png}
  \caption{SPR\#3: LMP = (50,50,50)}
  \label{fig:3bus_dlr_post_lmp_505050}
  \end{subfigure}
  \begin{subfigure}[t]{0.49\linewidth}
  \centering
  \includegraphics[width=\linewidth]{3bus_dlr_post_lmp_205030.png}
  \caption{SPR\#4: LMP = (20,50,35)}
  \label{fig:3bus_dlr_post_lmp_205030}
  \end{subfigure} 
  \caption{Posterior Probabilities of Two SPRs}  
  \label{fig:visualize_posteriors} 
\end{figure}


\begin{figure}[htbp]
  \centering
  \includegraphics[width=0.8\linewidth]{3bus_dlr_post_all} 
  \caption{Posterior Probability Surfaces}
  \label{fig:3bus_dlr_post_all} 
\end{figure}








\subsubsection{118-bus System} \label{ssub:118_bus_system_dlr}
A more comprehensive case study is conducted on the 118-bus system to evaluate the performance and computational burden of the data-driven approach on a complex system with realistic settings.


\paragraph{System Configuration} \label{par:system_configuration}
The only difference from the system configuration in Section \ref{sub:case_studies_of_static_sced_with_static_line_ratings} is about transmission limits. To consider DLR, we use the same model as Eqn. (\ref{eqn:model_dlr}).  is the same as the transmission limits in \cite{Technology} and .











\paragraph{Performance} \label{par:classification_accuracy}
The algorithm is implemented using the Statistics and Machine Learning Toolbox of Matlab.
Table \ref{tab:avg_cal_time_per_step} summarizes the computation time of each step in the data-driven approach on a PC with Intel i7-2600 8-core CPU@3.40GHz and 16GB RAM memory. There are 181 SPRs found in 1152 points for training,  SVM classifiers are trained in 58.72 seconds. On average, one SVM classifier is trained within  seconds. This is because most of the SPRs are completely separable, these cases will be solved in an extremely short time. Those adjacent SPRs are overlapping and are the major source of the computational burden.

\begin{table}[htbp]
  \caption{Average Computation Time (in seconds)}
  \label{tab:avg_cal_time_per_step}
  \centering

  \begin{tabular}{l|cccc}
  \hline

  \hline
  \textbf{Steps} & \textbf{Computation Time (s)}\\
  \hline
  \textbf{(a) training}  & 58.73 \\
  \textbf{(b) predicting} (288 points)  & 26.8504  \\
  \textbf{(c) data post-processing} & 701.22 \\
  \hline

  \hline
  \end{tabular}
\end{table}

\begin{table}[htbp]
  \caption{Results of the 118-bus System (Dynamic Line Rating)}
  \label{tab:classification_accuracy_118bus}
  \centering

  \begin{tabular}{l|cc}
  \hline

  \hline
  \textbf{Fold} & \textbf{Classification} & \textbf{LMP Forecast} \\
  \hline
    1 & 61.11\% & 95.11\% \\
    2 & 59.38\% & 94.53\% \\
    3 & 60.76\% & 95.24\% \\
    4 & 51.39\% & 93.34\% \\
    5 & 55.90\% & 94.22\% \\
  \hline
  \textbf{avg} & 57.71\% & 94.49\% \\
  \hline

  \hline
  \end{tabular}
\end{table}














\subsection{Case Studies with Ramp Constraints} \label{sub:ramp_constraints}
\paragraph{Settings} \label{par:settings}
The parameters of the 118-bus system are the same as in Section \ref{sub:case_studies_of_static_sced_with_static_line_ratings}. And the ramp capacities of generators follows the simplified assumption below: each generator could ramp up (down) to its generation limits in 15 min. For example, a generator with MW and MW, its ramp capacity is: MW/min. This setting is called  in Table \ref{tab:118_bus_diff_ramp}.
Due to the temporal coupling of SCED with ramp constraints, a daily load curve is necessary.
The settings of loads are the same as in Section \ref{sub:case_studies_of_static_sced_with_static_line_ratings}. 1440 SCED problems are solved consecutively with Matpower, and 1440 load vectors and LMP vectors are recorded.
\paragraph{Simulation Results} \label{par:simulation_results}
The classification and LMP forecast accuracies are summarized in Table. \ref{tab:118_bus_diff_ramp}.
With the ramp settings above, the classification and LMP forecast are satisfying. 
However, different ramp settings would change the results dramatically. As shown in Table. \ref{tab:118_bus_diff_ramp}: when generators ramp up/down 2 times faster (), the ramp constraints would rarely be active, then it is the same as static SCED; when generators ramp up/down 2 times slower (), the actual generation upper/lower bounds are determined by previous dispatch results and ramp constraints. Generation limits become time-varying and the SPRs are overlapping. This explains the unsatisfying results when the system is lack of ramp capacities. Furthermore, varying SPRs could also explain the price spikes during ramping up hours in the morning and ramping down hours in the early evening.

\begin{table}[htbp]
  \caption{Results on 118-bus System with Different Ramp Settings}
  \label{tab:118_bus_diff_ramp}
  \centering

  \begin{tabular}{l|ccc}
  \hline

  \hline
   & 0.5 & 1 & 2 \\
  \hline
  \textbf{LMP Forecast} & 44.57\% & 85.10\% & 96.33\% \\
  \hline

  \hline
  \end{tabular}
\end{table}


\section{The Impact of Nodal Load Information} \label{sec:impact_of_nodal_load_information}
We would like to point out that one possible contribution of this paper is to consider the LMP changes due to nodal
load variations. This section dedicates to a detailed discussion about the impact of nodal load information on the understanding of LMP changes. We first demonstrate the benefits of having nodal load information in Section \ref{sub:how_the_nodal_load_levels_help_us_}; then Section \ref{sub:partial_load_information} illustrates the effects of incomplete load information and the attempts to solve the issue.

To concentrate on the effects that incomplete load information brings, we make the following assumptions: (1) transmission limits are constant, no dynamic line ratings are being considered; (2) ramp constraints are not taken into account.


\subsection{On Nodal Load Levels} \label{sub:how_the_nodal_load_levels_help_us_}
Previous literature such as \cite{Li2007} studied the impact of system load levels on the LMPs. An important concept ``critical load level'' is defined as the system load level where the step changes of LMPs happen. Many LMP forecast methods were proposed based on identifying CLLs. But the definition of CLL assumes that the nodal load levels of all the buses change \emph{proportionally}. This assumption constrains the load vectors in the load space to be on a straight line, and the CLLs are indeed the intersection points of the straight line with the boundaries of SPRs.


We would like to point out that one possible contribution of this paper is to consider the LMP changes due to \emph{nodal-level} load variations.
Contrary to CLL-based methods, which solve a one-dimensional problem, the proposed SVM-based method could explore all the dimensions of the load space and is indeed a generalization of the CLL-based method.

Consider the SPR identification problem with only one feature vector: the total demand of the system. 
Fig. \ref{fig:cll_ID} illustrates the problem formulation. Since only the total demand  is available, the load vectors in the original SPRs are projected to the axis of total demand. Because this is a one-dimension problem, the SVM classifier degenerates to the case that there is only one decision variable , the direction of the separating hyperplane  is represented by the positivity of . The objective becomes finding the optimal value  which has the least overlapping points of different LMPs.

  \min_{b,s} & \qquad { \sum_i s^{(i)} } \label{eqn:SVM2CLL_obj} \\ 
  \text{s.t} & \qquad y^{(i)}( P_D^{(i)} -b) \ge 1-s^{(i)} \label{eqn:SVM2CLL_cons} \\
  & \qquad  s^{(i)} \ge 0, y^{(i)} \in \{-1,1\} \nonumber 


\begin{figure}[htbp]
  \centering
  \includegraphics[width=0.5\linewidth]{SVM_SPR_1D}
  \caption{Identifying Critical Load Levels}
  \label{fig:cll_ID}
\end{figure}

We compare this CLL-based method and SVM-based method on the 3-bus system and 118-bus system. Results are demonstrated in Table. \ref{tab:LMP_forecast_lse_3bus_alllmp}, \ref{tab:cll_svm_118bus} and Fig. \ref{fig:svm_cll_nodal_compare}. The performance of both methods are close for the nodal LMP forecast of many buses, 
but the CLL-based method failed to provide correct forecast of some specific buses (e.g. bus 23 in Fig \ref{fig:svm_cll_nodal_compare}), while the SVM-based method provides much better results. The SVM-based method is also better on forecasting high prices.

\begin{table}[htbp]
  \caption{Comparison of CLL and SVM (118-bus system)}
  \label{tab:cll_svm_118bus}
  \centering

  \begin{tabular}{l|cc}
  \hline

  \hline
  \textbf{LMP Forecast} & CLL & SVM  \\ 
  \hline
  \textbf{Overall} & 94.82\% & 95.95\%\\
  \textbf{Price  \P_{D_1}P_{D_2}+P_{D_3}P_{D_2}P_{D_2}P_{D_3}iiiP_{D_i}\sum_{i=1}^3 P_{D_i}P_{D_2}P_{D_3}2^{n_g-1}\times C_{n_g+n_l}^{n_g-1}\sim 10^6\pi = (\mathcal{B}, \mathcal{N})\mathcal{S}_{\pi}cc_3 < 2c_2 - c_1 = 80c = (20,50,65)c = (20,50,100)P_D^{(k)}kt_kP_D^{(k)}N_{+}N_{-}\hat{p_i} = \mathbb{P}(y=i | P_D \text{ and } i \in \{1,2,\cdots,n\} )\hat{p_i}\hat{\mu_{ij}}r_{ij}\hat{\mu_{ij}}r_{ij}r_{ij}\hat{p_i}\hat{\mu_{ij}} = \hat{p_i}/(\hat{p_i}+\hat{p_j} ) i=1,2,\cdots,n,1,2,\cdots\hat{p_i} \leftarrow \hat{p_i}/\sum_{j=1}^n {\hat{p_j}}\hat{\mu_{ij}} = \hat{p_i}/(\hat{p_i}+\hat{p_j} ) \mathbf{\hat{p}}/\sum{\hat{p_i}}\mathbf{\hat{p}}\mathbf{\hat{p}}L: \mathbb{R}^{n_b}\times \mathbb{R} \times \mathbb{R}^{n_l}\times \mathbb{R}^{n_l}\times \mathbb{R}^{n_g}\times \mathbb{R}^{n_g} \rightarrow \mathbb{R}\mu_+, \mu_-, \eta_+, \eta_- \ge 0\mu = \mu_+ - \mu_-\eta = \eta_+ - \eta_-\lambdaf: \mathbb{R}^{n}\rightarrow \mathbb{R}mf_k = c_k^\intercal xk = 1,2,\cdots, mD = \text{dom} fD_k = \text{dom } f_kD = \cup_k D_kDD_ii\ne jD_i \cap D_j \ne \emptyset\text{relint } D_i \cap \text{relint } D_j = \emptysetff_if_ji\ne j)\nabla^\intercal f_i = \nabla^\intercal f_jf_if_j\forall x_i \in \text{relint } D_i, \forall x_j \in \text{relint } D_jf(x_j) = f(x_i) + \nabla^\intercal f(x_i)\cdot (x_j - x_i)\forall x_i \in \text{relint } D_i, \forall x_j \in \text{relint } D_jx_ix_jD_kD_iD_ji\ne j \ne k\nabla^\intercal f_k = \nabla^\intercal f_i = \nabla^\intercal f_jf(x_k) = f(x_i) + \nabla^\intercal f(x_i)\cdot (x_k - x_i)D_iD_j\text{dim}(D_i) = \text{dim}(D_j) = d \ge 2D_iD_jD_i \cap D_j \ne \emptyset\text{dim}(D_i \cap D_j) = d-1S_{\pi}\mathbb{D}f^*(P_D) = c^\intercal  P_G^*(P_D)\mathcal{D}P_G^*P_DS_iS_j\pi_i = (\mathcal{B}_i, \mathcal{N}_i ) \pi_j = (\mathcal{B}_j, \mathcal{N}_j )S_iS_j\mathcal{B}_i\mathcal{B}_jS_iS_j\mathcal{B}_i\mathcal{B}_jk\ge 2S_i \cap S_jk\text{dim}  S_i \cap S_j = \text{dim}  S_i - k < \text{dim}  S_i - 1\text{dim}  S_i \cap S_j = \text{dim}  S_i -1f^*\mathcal{D}f^*f^*i,j\lambda^{(i)} = \lambda^{(j)}\lambda^{(i)} = \lambda^{(j)}\lambda_1^{(i)} = \lambda_1^{(j)}H^\intercal \mu = \lambda - \lambda_1 \mathbf{1}\eta^{(i)} = \eta^{(j)} i,jikk\ne jik\lambda^{(i)} = \lambda^{(k)}\eta^{(i)} = \eta^{(k)}ik\eta = \eta^{(i)} = \eta^{(k)}ij1ij2jiij\mathcal{C}A_{\mathcal{B}}A_{\mathcal{B}}ik[\mathbf{1}_{n_b}^\intercal; G]ikE = [\mathbf{1}_{n_b}^\intercal; G]H_{\mathcal{C}}ikh_11Hh_22Hh_1, h_2 \in \mathbb{R}^{1\times n_b} n_bA_{\mathcal{B}_i}^{-1}A_{\mathcal{B}_k}^{-1}\beta_3^{(k)}, \beta_3^{(i)} \in \mathbb{R}^{n_b \times 1}A_{\mathcal{B}_k} A_{\mathcal{B}_k}^{-1} = \mathbf{I}y_{\mathcal{B}_i}\lambda^{(i)} = \lambda^{(k)}H^\intercal (\mu^{(i)} - \mu^{(k)}) = \mathbf{0}(A_{\mathcal{B}_k})^{-1}A_{\mathcal{B}_k} = \mathbf{I}\mu_1 > 0E B_1^{(k)} = \mathbf{I}(B_1^{(k)})^\intercal = (E E^\intercal)^{-1} EEH\mathcal{B}\mathcal{N}\{1,2,\cdots,14\}1212\begin{bmatrix}
    1\\
     2\\
    13\\
    14
\end{bmatrix}\begin{bmatrix}  
    0.8944  &  0.4472\\
    0.4472  &  0.8944\\
   -0.7071 &   0.7071\\
    0.7071  & -0.7071\\
    0.7071 &   0.7071\\
   -0.7071  & -0.7071 
  \end{bmatrix}\begin{bmatrix}
    P_{D_2} \\ P_{D_3}
  \end{bmatrix} \le \begin{bmatrix}
       80.4984\\
   80.4984\\
  169.7056\\
  169.7056\\
   70.7107\\
         0
  \end{bmatrix}\begin{bmatrix}
    20\\
    20\\
    20
  \end{bmatrix}\begin{bmatrix}
  1\\
  2\\
  4\\
  14
\end{bmatrix}\begin{bmatrix}  
         0   & 1.0000\\
         0  & -1.0000\\
    0.4472  &  0.8944\\
   -0.4472  & -0.8944
  \end{bmatrix}\begin{bmatrix}
    P_{D_2} \\ P_{D_3}
  \end{bmatrix} \le \begin{bmatrix}
  140.0000\\
  -80.0000\\
  147.5805\\
  -80.4984
  \end{bmatrix}\begin{bmatrix}
    20\\
    50\\
    80
  \end{bmatrix}\begin{bmatrix}
  1\\
  2\\
  9\\
  14
\end{bmatrix}\begin{bmatrix}  
         0  & -1.0000\\
         0  &  1.0000\\
    0.7071  &  0.7071\\
   -0.7071  & -0.7071
  \end{bmatrix}\begin{bmatrix}
    P_{D_2} \\ P_{D_3}
  \end{bmatrix} \le \begin{bmatrix}
  -20.0000\\
   80.0000\\
  176.7767\\
  -70.7107
  \end{bmatrix}\begin{bmatrix}
    50\\
    50\\
    50
  \end{bmatrix}\begin{bmatrix}
  1\\
  2\\
  3\\
  14
\end{bmatrix}\begin{bmatrix}  
    1.0000 &        0\\
    0.7071 & -0.7071\\
   -0.7071 &   0.7071\\
   -1.0000  &       0
  \end{bmatrix}\begin{bmatrix}
    P_{D_2} \\ P_{D_3}
  \end{bmatrix} \le \begin{bmatrix}
 -100.0000\\
 -169.7056\\
  205.0610\\
  120.0000
  \end{bmatrix}\begin{bmatrix}
    20\\
    50\\
    35
  \end{bmatrix}\begin{bmatrix}
  1\\
  2\\
  8\\
  14
\end{bmatrix}\begin{bmatrix}  
         0  &  1.0000\\
    0.7071 &  -0.7071\\
         0 &  -1.0000\\
   -0.7071  &  0.7071
  \end{bmatrix}\begin{bmatrix}
    P_{D_2} \\ P_{D_3}
  \end{bmatrix} \le \begin{bmatrix}
 -100.0000\\
  275.7716\\
  120.0000\\
 -169.7056
  \end{bmatrix}\begin{bmatrix}
    20\\
    50\\
    -10
  \end{bmatrix}\begin{bmatrix}
  1\\
  2\\
  5\\
  13
\end{bmatrix}\begin{bmatrix}  
    1.0000   &      0\\
    0.7071  & -0.7071\\
   -0.7071  &  0.7071\\
   -1.0000  &       0
  \end{bmatrix}\begin{bmatrix}
    P_{D_2} \\ P_{D_3}
  \end{bmatrix} \le \begin{bmatrix}
 -100.0000\\
 -169.7056\\
  205.0610\\
  120.0000
  \end{bmatrix}\begin{bmatrix}
    20\\
    -60\\
    100
  \end{bmatrix}\begin{bmatrix}
  1\\
  2\\
  4\\
  5
\end{bmatrix}\begin{bmatrix}  
     0   &-1\\
     1  &  0\\
     0  &   1\\
    -1  &   0
  \end{bmatrix}\begin{bmatrix}
    P_{D_2} \\ P_{D_3}
  \end{bmatrix} \le \begin{bmatrix}
 -140.0000\\
   50.0000\\
  190.0000\\
  100.0000
  \end{bmatrix}\begin{bmatrix}
    20\\
    50\\
    100
  \end{bmatrix}\begin{bmatrix}
  1\\
  2\\
  4\\
  10
\end{bmatrix}\begin{bmatrix}  
   -1.0000  &       0\\
   -0.4472  & -0.8944\\
    1.0000 &        0\\
    0.4472 &   0.8944
  \end{bmatrix}\begin{bmatrix}
    P_{D_2} \\ P_{D_3}
  \end{bmatrix} \le \begin{bmatrix}
  -50.0000\\
 -147.5805\\
  170.0000\\
  192.3018
  \end{bmatrix}\begin{bmatrix}
    20\\
    60\\
    100
  \end{bmatrix}\begin{bmatrix}
  1\\
  2\\
  9\\
  10
\end{bmatrix}\begin{bmatrix}  
    1.0000 &        0\\
   -1.0000 &        0\\
   -0.7071 &  -0.7071\\
    0.7071 &   0.7071
  \end{bmatrix}\begin{bmatrix}
    P_{D_2} \\ P_{D_3}
  \end{bmatrix} \le \begin{bmatrix}
  230.0000\\
 -170.0000\\
 -176.7767\\
  212.1320
  \end{bmatrix}\begin{bmatrix}
    100\\
    100\\
    100
  \end{bmatrix}\begin{bmatrix}
  1\\
  2\\
  3\\
  10
\end{bmatrix}\begin{bmatrix}  
    1.0000  &       0\\
   -0.8944  & -0.4472\\
   -1.0000  &       0\\
    0.8944  &  0.4472
  \end{bmatrix}\begin{bmatrix}
    P_{D_2} \\ P_{D_3}
  \end{bmatrix} \le \begin{bmatrix}
  290.0000\\
 -214.6625\\
 -230.0000\\
  237.0232
  \end{bmatrix}\begin{bmatrix}
    20\\
    180\\
    100
  \end{bmatrix}$ \\
  & & & \\
  \hline

  \hline
\end{longtable}




\section{System Pattern Regions in 3D Space} \label{sec:system_pattern_regions_in_3d_space}
For better illustration, we only visualized the 2-dimension SPRs in previous sections. But the SPRs are usually polyhedrons in high-dimension space. The visualization of the 3-dimension SPRs could help readers get more intuition on the high-dimension SPR/polyhedron.
One load is added to the 3-bus system in Fig. \ref{fig:3Bus2GeneSystem}, and the new 3-bus 2-generator 3-load system is shown in Fig. \ref{fig:3Bus2Gen3Load}. Since there are 3 loads in the system, the SPRs locate in the 3D space. Similar with the 2D case, the SPRs are polyhedrons and there exists a separating hyperplane between any two SPRs.

\begin{figure}[htbp]
  \centering
  \includegraphics[width=0.6\linewidth]{3Bus2Gen3Load.png}
  \caption{3-bus 2-generator 3-load System}
  \label{fig:3Bus2Gen3Load}
\end{figure}

\begin{figure}[htbp]
  \centering
  \includegraphics[width=\linewidth]{3busSystem-3D.png}
  \caption{3D System Pattern Regions}
\end{figure}

\begin{figure}[htbp]
  \centering
  \includegraphics[width=0.45\linewidth]{3busSystem_set_1_2.png}
  \includegraphics[width=0.45\linewidth]{3busSystem_set_1_3.png} \\
  \includegraphics[width=0.45\linewidth]{3busSystem_set_1_4.png} 
  \includegraphics[width=0.45\linewidth]{3busSystem_set_2_3.png} \\
  \includegraphics[width=0.45\linewidth]{3busSystem_set_2_4.png}
  \includegraphics[width=0.45\linewidth]{3busSystem_set_3_4.png}
  \caption{Pairwise Visualization of System Pattern Regions}
\end{figure}


\newpage
\bibliographystyle{IEEEtran}
\bibliography{MyCollection}





























\end{document}
