 
\documentclass[12pt]{article}
\usepackage{fullpage,url}
  \usepackage{amssymb}
	\usepackage{graphicx}
	\newtheorem{prop}{Proposition}
\newtheorem{property}{Property}
\newtheorem{theorem}{Theorem}
\newtheorem{fact}{Fact}
\newtheorem{proof}{Proof}
	\def\dist{\mathrm{dist}}
\def\diag{\mathrm{diag}}
 
\def\arccosh{\mathrm{arccosh}}
\def\sinh{\mathrm{sinh}}
\def\cosh{\mathrm{cosh}}
\def\st{:}
\def\ball{\mathrm{ball}}
\def\calD{\mathcal{D}}
\def\calP{\mathcal{P}}
\def\calB{\mathcal{B}}
\def\calH{\mathcal{H}}
\def\calU{\mathcal{U}}
\def\calK{\mathcal{K}}
\def\calL{\mathcal{L}}
\def\dequal{{\dot=}}

\def\dzb{\mathrm{d}\bar{z}}


\def\Bi{\mathrm{Bi}}
\def\vor{\mathrm{Vor}}
\def\bbX{\mathbb{X}}
\def\bbH{\mathbb{H}}
\def\bbR{\mathbb{R}}
\def\bbB{\mathbb{B}}
\def\bbS{\mathbb{S}}
\def\bbL{\mathbb{L}}
\def\bbC{\mathbb{C}}
\def\bbU{\mathbb{U}}

\def\SO{\mathrm{SO}}
\def\DT{\mathrm{DT}}

\def\PSU{\mathrm{PSU}}
\def\innerL#1#2{{\langle #1,#2\rangle}_L}
\def\inner#1#2{{\langle #1,#2\rangle}}
\def\Inner#1#2{{\left\langle #1,#2\right\rangle}}

\def\InnerC#1#2{{\left\langle #1,#2\right\rangle}_\mathrm{C}}
\def\innerC#1#2{{\langle #1,#2\rangle}_\mathrm{C}}


\def\InnerE#1#2{{\left\langle #1,#2\right\rangle}_\mathrm{E}}
\def\innerE#1#2{{\langle #1,#2\rangle}_\mathrm{E}}
\def\innerL#1#2{{\langle #1,#2\rangle}_\mathrm{L}}

\def\ceil#1{{\lceil {#1}\rceil}}
\def\floor#1{{\lfloor {#1}\rfloor}}

\def\ballE{\mathrm{Ball}_E}
\def\Vor{\mathrm{Vor}}

\def\ds{\mathrm{d}s}
\def\dx{\mathrm{d}x}
\def\dy{\mathrm{d}y}
\def\dt{\mathrm{d}t}
\def\PSL{\mathrm{PSL}}
\def\PGL{\mathrm{PGL}}

\def\Pow{\mathrm{Pow}}
\def\pow{\mathrm{\Pi}}
\def\path#1{}
\def\calS{\mathcal{S}}

\def\Isom{\mathrm{Isom}}
 
\def\bbB{\mathbb{B}}
\def\myvec#1#2{ \left[\begin{array}{c}#1\cr #2\end{array}\right]}
 \def\calF{\mathcal{F}}
 
\begin{document}

\title{Further results on the hyperbolic Voronoi diagrams}

\date{April 2014} 

\author{Frank Nielsen\thanks{Ecole Polytechnique, France 
Sony Computer Science Laboratories, Japan.
Email: {\tt Frank.Nielsen@acm.org}}
\and
Richard Nock\thanks{NICTA, Australia.
UAG CEREGMIA, France.
Email: {\tt rnock@martinique.univ-ag.fr}}
}

\maketitle


\begin{abstract}
In Euclidean geometry, it is well-known that the -order Voronoi diagram in  can be computed from the vertical projection of the -level of an arrangement of hyperplanes tangent to a convex potential function in : the paraboloid.
Similarly, we report  for the Klein ball model of hyperbolic geometry such a {\em concave} potential function: the northern hemisphere.
Furthermore, we also show how to build the hyperbolic -order diagrams as  equivalent clipped power diagrams in .
We investigate the hyperbolic Voronoi diagram in the hyperboloid model and show how it reduces to a Klein-type model using central projections.
\end{abstract}

\noindent {\bf Keywords}:\\
Voronoi diagram; hyperbolic geometry; clipping.
 


 



\section{Introduction}
Hyperbolic geometry is a consistent geometry where the Euclidean Playfair's parallel postulate is discarded and replaced by the existence of many lines  not intersecting another given line   and passing through a given point  (the 's are said {\em ultra-parallel}\footnote{{\em Parallel} lines intersect at infinity in hyperbolic geometry.} to ).
Hyperbolic geometry can be studied using various models~\cite{VHVD-2014}: Poincar\'e disk or upper plane conformal models, Klein  non-conformal model disk model, hyperboloid conformal model, etc.
From the viewpoint of computational geometry, we prefer to use Klein model where lines/bisectors are Euclidean straight~\cite{HVDeasy-2010} and then convert the output to the desired model for visualization or navigation purposes~\cite{VHVD-2014}. 
We report further novel results for constructing hyperbolic Voronoi diagrams (HVDs) in Klein model~\cite{HVDeasy-2010} and present yet another approach to get Klein-type affine bisectors/diagrams from the hyperboloid\footnote{\underline{Hyperbol}ic geometry stems from the \underline{hyperbol}oid model.} model.

\section{HVDs from lower envelopes}


The {\em Voronoi diagram} of  a set  of  points  in  w.r.t.  can be computed equivalently as the {\em minimization diagram} of  functions by observing that
 where , .
Thus the {\em combinatorial structures} are congruent: .
Furthermore, this minimization diagram amounts to compute the {\em lower envelope} of  graph functions in :
 .

Let  denotes the Euclidean inner product.
In the Klein model~\cite{HVDeasy-2010}, the distance between two points  and  in the open unit ball domain   is  where
 for  is a monotonically increasing function.
Since the Voronoi diagram does not change by composing the distance with a monotonous function, we consider the equivalent Klein distance .
To each point  corresponds a function .
Since the denominator  is common to all functions, the minimization diagram is equivalent to the minimization diagram
of .
The graph  are {\em hyperplanes} in  defined on , and the lower envelope can thus be computed from the intersection of  halfspaces , yielding the Voronoi unbounded polytope in .

\begin{theorem}
The HVD of  points can be computed in the Klein model as the intersection of  half-spaces in  and by projecting vertically ( ) the polytope on , and clipping it with the unit ball domain: .
\end{theorem}




\section{Lifting sites to a potential function}


In Euclidean (and more generally Bregman geometry), the Voronoi polytope is built by lifting points to tangent hyperplanes to a {\em potential function}  at site locations. This is the paraboloid lifting transformation:   ( for a convex Bregman generator ).

\begin{theorem}
In the Klein ball model, the {\em potential function} for lifting generators to hyperplanes is the {\em concave} function 
 restricted to .
\end{theorem}

Proof:
Let us identify the  hyperplane equation  with
the hyperplane tangent at  to a potential function : 
.
We have  and the remaining term (independent of ) is .
The anti-derivative of  is , and the constant  solves to zero.
This is the equation  of the northern hemisphere for .
Observe that the hyperplanes tend to become vertical as we near the boundary domain , and are vertical at the boundary.
 
\section{-order hyperbolic Voronoi diagrams}

Since the Klein bisector is affine, the -order HVD is affine. We present two construction methods.


\subsection{-HVDs from levels of an  arrangement of hyperplanes}
This is a straightforward generalization of the Euclidean procedure using the  potential function.
The -order HVD is a {\em cell complex} that can be built by projecting to  all the -dimensional cells at -level of the arrangement of the site hyperplanes  of   and clipping the structure to .
Figure~\ref{fig:example} displays some -order diagrams and illustrates some degenerate cases.

\def\ttt{0.33\columnwidth}
\begin{figure}\centering
\begin{tabular}{cc}
\includegraphics[bb=0 0 512 512,width=\ttt]{Figures/Klein-1-Voronoi-border_1.png} &
\includegraphics[bb=0 0 512 512,width=\ttt]{Figures/Klein-2-Voronoi-border_1.png} \\
(a) & (b) \\
\includegraphics[bb=0 0 1024 1024,width=\ttt]{Figures/Vor-Klein-9.png} &
\includegraphics[bb=0 0 1024 1024,width=\ttt]{Figures/Deg2-Vor-Klein-8.png} \\
(c) & (d)
\end{tabular}
\caption{HVD for  (a) and  (b).
HVD with all unbounded cells (c), and pencil of parallel bisectors intersecting at  (d).
}\label{fig:example}\end{figure}


\subsection{-HVDs from power diagrams}

Consider all subsets of size ,  with .
Those {\em subset generators} partition the space into {\em non-empty -order Voronoi cells}: 


Observe that  iff .
In Klein model with , we define the function , and
.
By identifying those hyperplane equations with the generic power diagram hyperplane  for a ball centered at  and radius  ( may be imaginary when ), we transform each -subset  in Klein model into a weighted point (or ball) :
 and .
This method is only practical if when we consider all subsets  that yields non-empty cells, otherwise we have  too many balls to be tractable!



\section{HVDs from the hyperboloid model}
Consider the symmetric bilinear form  in Minkowski space : . The hyperboloid model is defined on the upper sheet domain  (interpreted as a sphere  of imaginary radius ).
For , we denote  its point obtained by vertically rising  on : , called Weierstrass coordinates. 
The hyperbolic distance is expressed by  and is equivalent to .
For two points  and  on , the bisector equation is .
The bisector is an hyperbola of equation .
This hyperbola bisector is contained in a hyperplane  of  passing through the origin :
.
The Klein disk model is obtained from  by a central projection  from the origin to the hyperplane : . The disk center touches the apex of .
Let .
Multiplying  by , we have the bisector written as
, an affine bisector in .

Now consider  the {\em generic} central projection of  from  to the hyperplane  so that .
We have
.
Choosing  and  yields the same construction procedure but the clipping of the equivalent power diagram~\cite{HVDeasy-2010} need to be done on a disk of size  since
, .

Note that  clipping may destroy bounded cells of the affine diagram as illustrated in Figure~\ref{fig:clipping}.
Thus a remaining open question is to report an optimal output-sensitive construction of the -order HVDs.

A video illustrating the hyperbolic Voronoi diagrams using the five common models of hyperbolic geometry is available online~\cite{HVDvideo}. 
 
\begin{figure}\centering
\begin{tabular}{ccc}
\includegraphics[width=0.3\textwidth]{Figures/exHVD-P.pdf} &
\includegraphics[width=0.3\textwidth]{Figures/exHVD-K.pdf} &
\includegraphics[width=0.3\textwidth]{Figures/exHVD-PDunboundedcells.pdf} 
\\
(a) & (b) & (c)
\end{tabular}
\caption{The hyperbolic Voronoi diagram in conformal Poincar\'e disk (a) is obtained by a radial scaling transformation of the HVD in non-conformal Klein disk (b) that is itself built as an equivalently clipped power diagram (c). Observed that some bounded cells of the power diagram are cut by the boundary cutting circle.
}\label{fig:clipping}\end{figure}
 
 
\nocite{*}

 
\bibliographystyle{plain}
 

 
\begin{thebibliography}{1}

\bibitem{HVDeasy-2010}
Frank Nielsen and Richard Nock.
\newblock Hyperbolic {V}oronoi diagrams made easy.
\newblock In B.~O.~Apduhan et~al., editor, {\em International Conference on
  Computational Science and Its Applications}, pages 74--80. IEEE, 2010.

\bibitem{nielsen2012hyperbolic}
Frank Nielsen and Richard Nock.
\newblock The hyperbolic {V}oronoi diagram in arbitrary dimension.
\newblock {\em arXiv preprint arXiv:1210.8234}, 2012.

\bibitem{VHVD-2014}
Frank Nielsen and Richard Nock.
\newblock Visualizing hyperbolic {V}oronoi diagrams.
\newblock In {\em Symposium on Computational Geometry}. ACM, 2014.
\newblock \url{http://www.youtube.com/watch?v=i9IUzNxeH4o}.

\bibitem{HVDvideo}
Frank Nielsen and Richard Nock.
\newblock Visualizing hyperbolic {V}oronoi diagrams.
\newblock In {\em Proceedings of the Thirtieth Annual Symposium on
  Computational Geometry}, SOCG'14, pages 90:90--90:91, New York, NY, USA,
  2014. ACM.

\end{thebibliography}




\end{document}
