


\documentclass[copyright]{eptcs}
\usepackage{amssymb}
\providecommand{\event}{DCFS 2010}

\def\newrmtheorem#1{\@ifnextchar[{\@ormthm{#1}}{\@nrmthm{#1}}}

\newtheorem{theorem}{\bf Theorem}[section]
\newtheorem{lemma}[theorem]{\bf Lemma}        \newtheorem{sublemma}[theorem]{\bf Sublemma}	\newtheorem{proposition}[theorem]{\bf Proposition}  \newtheorem{corollary}[theorem]{\bf Corollary}
\newtheorem{definition}[theorem]{\bf Definition}
\newtheorem{clim}[theorem]{\bf Claim}
\newtheorem{conjecture}[theorem]{\bf Conjecture}
\newtheorem{example}[theorem]{\bf Example}


\newcommand{\cA}{\mathcal{A}}
\newcommand{\cB}{\mathcal{B}}
\newcommand{\cC}{\mathcal{C}}
\newcommand{\cD}{\mathcal{D}}
\newcommand{\zo}{\{0,1\}}
\newcommand{\zos}{\{0,1\}^*}
\newcommand{\lex}{<_{\mathrm{lex}}}
\newcommand{\mod}{\rm mod}
\newcommand{\ord}{{o}}
\newcommand{\pre}{\mathrm{pre}}


\def\Bbox{
{\unskip\nobreak\hfil\penalty50
\hskip1em\hbox{}\nobreak\hfil{\lower .5pt \hbox{}}
\parfillskip=0pt \finalhyphendemerits=0 \par}
}

\def\eop{
\ifmmode {\hbox{\Bbox}} \else \Bbox \fi
}


\title{Representing Small Ordinals by Finite Automata}



\def\titlerunning{Small Ordinals and Finite Automata}
\def\authorrunning{Z. \'Esik}


\author{Z. \'Esik\thanks{Supported in part by grant no. K 75249 of the National Foundation of
Hungary for Scientific Research (OTKA) and the T\'AMOP 4.2.1/B program of the Hungarian National Development Agency.}
\institute{
Dept. of Computer Science\\
University of Szeged\\
Hungary}
\email{ze@inf.u-szeged.hu}
}

\begin{document}

\maketitle


\begin{abstract}
It is known that an ordinal is the order type of the lexicographic
ordering of a regular language if and only if it is less than
. We design a polynomial time algorithm that 
constructs, for each  well-ordered regular language 
with respect to the lexicographic ordering, given by a deterministic 
finite automaton, the Cantor Normal Form of its order type. It follows that 
there is a polynomial time algorithm to decide whether two  
deterministic finite automata accepting well-ordered regular 
languages accept isomorphic languages.    
We also give estimates on the size of the smallest automaton
representing an ordinal less than , together with
an algorithm that translates each such ordinal to an automaton.
\end{abstract} 



\section{Introduction}

One of the basic decision problems in the theory of automata and languages
is the equivalence or equality problem that asks if two specifications
define equal languages. In this paper we study the related ``isomorphism problem''
of deciding whether the lexicographic orderings of the languages 
defined by two specifications are isomorphic, i.e., whether 
the two languages determine ``isomorphic dictionaries''.  


The study of lexicographic orderings of regular languages, or equivalently, 
lexicographic orderings of the leaves of regular trees goes back
to \cite{Courcelle}. 
Thomas \cite{Thomas} has shown without giving any 
complexity bounds that it decidable whether the lexicographic orderings of two regular languages 
(given by finite automata or regular expressions) are isomorphic.
In contrast, the results in \cite{BEregwords} imply that there is an exponential 
algorithm to decide whether the lexicographic orderings 
of two regular languages, given by deterministic finite automata (DFA) 
are isomorphic. In contrast, no such algorithm exists
for context-free languages, cf. \cite{EsCF}.
In this paper, one of our aims is to show that there is a polynomial time 
algorithm to decide for DFA accepting lexicographically well-ordered 
languages, whether they accept isomorphic languages with respect
the lexicographic order.


The ordinals that arise as order types of lexicographic well-orderings 
of regular languages are exactly the ordinals less than ,
cf. \cite{BEord,Heilbrunner}. 
The Cantor Normal Form (CNF) \cite{Rosenstein} of any such nonzero ordinal 
takes the form

where  and  are integers such that ,
, , and .
We provide an algorithm that, given an ``ordinal automaton'' 
representing a well-ordering, computes its CNF. 

We also give estimates on the size of the smallest ordinal automaton
representing an ordinal less than , together with
an algorithm that translates such an ordinal to an automaton.

In the main part of the paper we will restrict ourselves to DFA over
the binary alphabet  accepting a complete prefix language 
(complete prefix code). However, this restriction  
is only a technical convenience and is not essential for the results.


 


\section{Lexicographic orderings}


Suppose that  is an alphabet linearly ordered by the relation .
We define the lexicographic ordering  of the set  
by  iff  is either a proper prefix of  or  and  
are of the form ,  with  in .
When , we obtain a (strict) linear ordering 
, called the \emph{lexicographic ordering of }.
 It is known that if  has two or more letters,
then every countable linear ordering is isomorphic to the
linear ordering  of some language 
, see e.g. \cite{BEord}. 
Moreover, we may restrict ourselves to prefix 
languages,
for if  where the 
alphabet is ordered as indicated, then   is isomorphic to ,
where  is a new letter which is  lexicographically less than any other letter.
Further, we may restrict ourselves to the binary alphabet, 
since each ordered alphabet of  letters can be encoded by words 
over  of  length  in an order preserving 
manner. 
Actually, it suffices to consider 
\emph{complete} prefix languages  having the 
property that for any ,  is in the set  
of all prefixes of words in  iff . 

Suppose that  is a complete prefix language.
We define the complete binary tree  to be the tree whose 
vertices are the words in , 
such that each vertex  is either a leaf or has 
two successors, the words  and . When  is the 
empty language,  is the empty tree. Note that  
is an ordered tree, since the successors  
of a non-leaf vertex  are ordered by . 
The linear ordering  is just the ordering 
of the leaves of . Note that each infinite branch
of  determines an -word over . 
Below we will make use of the following 
simple fact, see also \cite{Carayoletal}. 

\begin{lemma}
\label{lem-intro}
Suppose that  and consider the tree .
Then  is a well-ordering iff the -word 
determined by each infinite branch of  contains a 
finite number of occurrences of .
\end{lemma}


Call a linear ordering regular if it is isomorphic to the 
lexicographic ordering of a regular (complete prefix) language over some 
ordered alphabet, or equivalently, over the alphabet .
A \emph{regular well-ordering} is a regular linear ordering that is 
a well-ordering. 

Regarding linear orderings and ordinals, we will use standard 
terminology. Below we review some simple facts for 
linear orderings and ordinal arithmetic
(restricted to ordinals less than ). For all 
unexplained notions we refer to \cite{Rosenstein}.

Suppose that  and  are disjoint 
(strict) linear orderings.  
Then the \emph{ordered sum}  is the linear ordering 
, where the restriction of  to  is the relation 
and similarly for , and where  holds for all  and .
It is known that if  and  are well-ordered of order type
 and , respectively, where  and  are ordinals, 
then  is well-ordered of order type . 
In addition to sum, we will make use of the product operation. 
Given  and  as above, let us define the following 
linear order  of the set : For all , 
 iff , or  and . 
When  are well-ordered of order type , respectively, then 
 is also well-ordered of order type . 

As mentioned in the Introduction, it is known that a well-ordering 
is regular iff its order type is less than the ordinal .
The Cantor Normal Form (CNF) \cite{Rosenstein} 
of each nonzero ordinal less than this bound is of the form

where  and  and  are integers with  
,  for all . The 
exponent  is called the \emph{degree}.  

In order to compute the CNF of the 
sum of two nonzero ordinals less than , it is helpful to know 
that  whenever . 
Thus, when 

then the CNF of  can be computed as follows. 
First, suppose that  are all greater than 
and . If , then 
is

If , then  is

Finally, suppose that . In that case  is 

In order to compute the product , it suffices to know 
that product distributes over sum on the left, and if  is 
the ordinal given above, then . 

 










\section{Ordinal automata}

We will be considering DFA ,
where  is the finite set of states,  is the
input alphabet,  is a partial function
, the transition function,
 is the initial state, and  
is the set of final states. As usual, we extend  
to a partial function  
and write  for , for  and
.


The \emph{language  accepted by the DFA} 
 is the set . 
As usual, we call an automaton  \emph{trim}
if each state  is both accessible and
co-accessible, i.e., when there exist words  with  and . It is well-known that 
if  is nonempty, then  is equivalent to a 
trim automaton that can be easily constructed from  
by removing all states that are not accessible or co-accessible.
\emph{To avoid trivial situations, we will only consider 
automata that accept a nonempty language, so that we 
may restrict ourselves to trim automata}.

A trim automaton 
accepts a prefix language iff neither  nor  is defined 
when .  Moreover, assuming that 
this holds,  accepts a complete prefix language iff
for every , both  
and  are defined. We will call such trim 
automata \emph{complete prefix automata} (CPA). It is clear that 
for each trim automaton  accepting a prefix language 
one can construct a CPA  
with   such that  is isomorphic to 
. To this end, for each state  we 
form the unique sequence of states  
such that for each ,  or , 
moreover, exactly one of  and  is defined, and finally
either  (in which case neither  nor  is defined), 
or both  and  are defined. If , then we remove 
the transitions used to form this sequence and declare  to be a final state. 
If , then we replace the transition originating in 
by the two transitions , .
Finally, we remove states that are not accessible or co-accessible. 
 

Suppose that   is a DFA. By the \emph{size} of  
we will mean the number of states in . 
The strongly connected components of  
are defined as usual. We say that a strongly connected component  is \emph{trivial} 
if  consists of a single state  and .
Otherwise  is called \emph{nontrivial}. 
We impose  the usual  
partial order on strongly connected components by defining  iff 
there exist some  and  with . The \emph{height}
of a nontrivial strongly connected component  is the length  of the 
longest sequence  of nontrivial strongly connected 
components such that  and . 
From Lemma~\ref{lem-intro} we immediately have:

\begin{proposition}
A CPA  accepts a well-ordered 
language iff for each nontrivial strongly connected component
 and  it holds that  (and
of course ).
\end{proposition}

We conclude that there is a simple algorithm to decide 
whether a CPA  accepts a well-ordered language which 
runs in polynomial time in the size of the automaton,
see also \cite{BEscattered,BloomZhang}.   
It is trivial to extend this result to automata over 
larger alphabets.


\begin{definition}
\label{def-ordinal aut}
An \emph{ordinal automaton} (OA) is a CPA 
 
such that whenever  belongs to a nontrivial
strongly connected component ,  does not
belong to .
\end{definition}

By the previous proposition, a CPA  is an OA
iff it accepts a well-ordered (complete prefix) 
language. For an OA , we call the order type of  
the \emph{ordinal represented by }, denoted . 

\begin{lemma}
\label{lem-omega n}
For each , there is an OA  of size  representing .
\end{lemma} 

{\sl Proof.}\  Let  have states  with transitions 
 and  for all .
The initial state is  and the only final state is . \eop 

\begin{example}
Consider the ordinal . 
An ordinal automaton representing  has  states, ,
where  is the initial state and  is the only final state. 
The transitions are defined by , , 
, and ,  for 
. 
\end{example} 

We end this section with a construction converting a nonzero ordinal 
to an OA. First, for each , we construct a CPA  
having a single final state which accepts a language of  words.
 The CPA  has a single state which is both 
initial and final, and no  transitions.  
If  is even, say , consider   and add a new initial state  
together with  transitions  to the old initial state . 
The only final state is the final state of . 
If  for some , then consider  with initial state  
and final state . We add two new states  and  and new transitions 
, , . 

Now let the CNF of  be . When  and , then 
we may take the OA  of Lemma~\ref{lem-omega n},
we have that . 
So suppose that  or . For each , consider the 
automaton  constructed above
with initial state  and final state , say. We may assume 
that the state sets of the  are pairwise disjoint. 
Then we form the ``ordered sum'' of the ,  by adding 
 new states , transitions ,
,  and .  
Finally, take the automaton  
of Lemma~\ref{lem-omega n}, and identify its state  with  for all 
. 
The resulting OA has  states and represents ,
where 
 and  
, 

for all .

\section{From ordinal automata to CNF}

For this section, fix an OA 
. 
For each , let us denote by  the automaton 
, where ,
 is the restriction of  to ,
and .


The following lemma is clear.

\begin{lemma}
For each state ,  is also 
an ordinal automaton.
\end{lemma}


For each , we let  denote the order type of  
. By the above lemma,  is a (nonzero) 
ordinal for each .



\begin{lemma}
\label{lem-mon}
For all  and , 
 Thus, if 
and  belong to the same strongly 
connected component, then . 
\end{lemma} 

{\sl Proof.}\  The function ,  defines an order embedding 
of the linear ordering  into . \eop 


\begin{proposition}
\label{prop-strongly connected}
If  is a \emph{nontrivial} strongly connected component,
then there is an integer  such that 
for all  it holds that .
Moreover, for each  the degree of  is at most , 
and there is some state  such that the degree of 
 is . 
\end{proposition}

{\sl Proof.}\  
By Definition~\ref{def-ordinal aut}, we can arrange the states in  
in a sequence  such that  
for all . We also know that  for all . Thus,
 
Since , this is possible only if  
for some . It follows now by Lemma~\ref{lem-mon} that 
 for all . Using the formula (\ref{eq-sc}), 
it follows that the degree of each  is at most . 
Moreover, there is at least one  such that the degree of 
 is exactly , since otherwise the degree
of  would be less than . \eop 

When  is a strongly connected component, trivial or not, 
we let  denote the ordinal  for . 

Suppose that the strongly connected component containing  is trivial. 
\emph{Below we will say that a word  {\bf leads} from  to a 
strongly connected component } if  
but  does not belong to any nontrivial strongly connected 
component whenever  is a proper prefix of .
When  is a trivial strongly connected component consisting
of a single final state , then we also say that  leads from
 to the final state . The following fact is clear.


\begin{proposition}
\label{prop-sum}
Suppose that the strongly connected component of the state  is trivial. 
Then let  denote in lexicographic order 
all the words leading from  to a nontrivial strongly 
connected component, or to a final state.\footnote{The number of such 
words is clearly finite.} Then 

Thus, the degree of  is the maximum degree of the ordinals 
, .
\end{proposition}

We now prove a stronger version of Proposition~\ref{prop-strongly connected}. 

\begin{proposition}
\label{prop-strongly connected 2}
If  is a nontrivial strongly connected component of  height , 
then .
\end{proposition}


{\sl Proof.}\  
Suppose that  is a nontrivial strongly connected 
component of height . Clearly, . We argue by induction on  
to prove that . This is clear when 
, since by Proposition~\ref{prop-strongly connected}, 
 for some . Suppose now
that . Then let  be a nontrivial strongly 
connected component of height  accessible from a state 
of  by some word. Then there exists a state  
with  such that  is accessible from 
by some word. Since , the degree of 
 is at least . By (\ref{eq-sc}),  
. 




Next we show that for any nontrivial strongly connected component  of height ,
. This is clear when . Supposing , 
by Propositions~\ref{prop-sum} and the 
induction hypothesis we know that the degree
of  is at most  for each . Thus, by 
Proposition~\ref{prop-strongly connected}, . 
\eop 


\begin{corollary}
\label{cor-omega n}
If the degree of  is , then  has at least  states. 
\end{corollary}

{\sl Proof.}\  
This is clear when . Suppose now that . 
Since the degree of  is ,  has at least 
one nontrivial strongly connected component of height ,
and thus at least one nontrivial strongly connected component of height  for 
every .  Together with a final state, this gives at least  states. \eop 












As a corollary of the above facts, there is an algorithm 
that computes the CNF of the ordinal  represented
by the ordinal automaton . First, using some  standard polynomial time
algorithm,  we determine the set   
of all nontrivial strongly connected components together with all 
trivial strongly connected components consisting of 
a single final state. We also determine  for each nontrivial 
strongly  connected component 
by computing the height  of . We set 
for all strongly connected components  consisting
of a single final state.  If the initial state belongs to
some , then . 
Otherwise let  denote the maximum of the heights of the 
nontrivial strongly connected components, and let  if there 
is no nontrivial strongly connected component. Let  denote the set 
of all nontrivial strongly connected components in 
of maximum height . 
Using the algorithms specified in the Appendix as subroutines
with suitable parameters,
we determine for each  the number  of all words  
leading from the initial state  to , together with the 
lexicographically greatest such word . Then we define  as the lexicographically 
greatest word among the  and .
By Proposition~\ref{prop-strongly connected 2}
and Proposition~\ref{prop-sum},  for some 
unknown ordinal  of degree . 

In the next step, we consider the set  of all 
strongly connected components  in  of height , 
and for each , we compute the number  of all those words 
leading from  to  that are \emph{lexicographically greater
than} , together with the lexicographically greatest such word
, if any. Then , 
where  is the sum of the integers , ,
 and  is some unknown ordinal of degree . 
We also determine the lexicographically greatest 
word in the set consisting of  and all words ,  
such that , and we denote this word by . 

Repeating the procedure, before the last step we know that 

where  is an unknown finite ordinal. Moreover, we have computed a word 
. In the last step, we consider the set  of those connected
components in  that consist of a single final state. We determine
for each  the number of all words leading from  
to  that are lexicographically greater than . Then 
.

We conclude that .
To get the CNF of , we remove all summands  with . 

The length of each word  determined in the above algorithm is 
bounded by the size of  and can be determined in 
polynomial time. Similarly, the length
of the binary representation of each  is at most the size of ,
and each  can be computed in polynomial time in the size of .
Thus, the overall algorithm runs in polynomial time.  
We have proved:

\begin{theorem}
There is a polynomial algorithm that, given an ordinal automaton
, computes the CNF of the ordinal  represented by .
\end{theorem}


\begin{corollary}
There is a polynomial time algorithm to decide for 
ordinal automata  and  whether 
, i.e., whether 
and  are isomorphic.
\end{corollary}

{\sl Proof.}\  We compute in polynomial time the CNFs of  and 
and check whether they are identical. \eop 




\section{Minimal ordinal automata}


For a nonzero ordinal , let  
denote the minimum number  such that  for 
some -state OA . In this section we reduce the determination
of the function  to another problem on automata
and give some estimations on  in terms of the 
CNF of . 



\begin{definition}
\label{def-f}
Let  be positive integers. 
Then we let 
 
denote the minimal number of states of a CPA
 having no  nontrivial strongly connected component 
with the following properties:
\begin{enumerate}
\item , where the states  are pairwise different.
\item For each  with , the language  has exactly  words: 

\end{enumerate}
\end{definition} 
Note that . 
Also, when ,  is the minimum number of states of a CPA 
accepting a language of  words. In particular, . 


\begin{lemma}
\label{lem-minoa}
Suppose that  is a nonzero ordinal 
with CNF  . 
Then there is a OA of size  
representing .   
\end{lemma}

{\sl Proof.}\  We take a CPA  having no nontrivial strongly connected 
component as in Definition~\ref{def-f}, having  states, 
 and the automaton  constructed 
in Lemma~\ref{lem-omega n}. Then we identify  with  for all 
.
\eop


 




\begin{theorem}
Suppose that the Cantor normal form of a nonzero ordinal  is 
. 
Then  where .
\end{theorem} 



{\sl Proof.}\  
We have already shown that . Thus, it 
remains to prove that .


Suppose that  is an OA with  having a 
least number of states among all such automata.
By Corollary~\ref{cor-omega n},  must have at least  states. 
Thus, when  and , , 
since . So from now on we assume that  or . 


By the proof of Corollary~\ref{cor-omega n},  has at least one nontrivial 
strongly connected component  of height  for each  with , 
and of course  at least one final state. 
It is not possible that a nontrivial strongly connected 
component  of height , say,  contains two or more states, since otherwise 
we could select a state  of  such that at least 
one strongly connected component  of height  is accessible 
from  by some word  (i.e., ), and redirect any transition going to  to 
the selected  state . After that, 
we could remove all states in , the resulting ordinal 
automaton would still represent the same ordinal, by 
Proposition~\ref{prop-sum} and Proposition~\ref{prop-strongly connected 2}. 
Similarly, for each , there must be a single nontrivial
strongly connected component of height . Indeed, if  and  
were different nontrivial strongly connected components of the 
same height , then we could remove  and redirect every transition 
originally going to some state in  to a state in ; 
the resulting smaller OA would represent the 
same ordinal. Clearly,   has a single final state. 
Also, if a state  forms a nontrivial strongly connected component 
of height , and  is either the state that forms the single 
nontrivial strongly connected component of height  if  
or  is the single final state if , and if  is not ,
then we can redirect this transition from  under  to . The OA obtained after removing 
those states that possibly become inaccessible from the initial state 
 still represents . 

In conclusion, we have that  contains a subautomaton consisting 
of states  such that  is the final state
and for each ,  forms a nontrivial strongly connected component of height .
Moreover,  and  for all . 
Let . 
None of the states in  is contained in any nontrivial strongly
connected component, and each state is accessible from 
by some word. Moreover, from each state  there is at least 
one word leading to some connected component , 
trivial or not. We claim that if  or  for some  
and , then there exists some  
with , i.e.,  appears in the CNF of .
Indeed, if , say,  but  is not in the set ,
then we can remove state  and redirect all transitions going to 
 to , the resulting smaller OA still represents ,
a contradiction.  




Since  or ,  the initial state  is not in 
(since otherwise  would be a power of ). 
Let us order the set  of all words leading from  to 
a strongly connected component ,  
lexicographically. We know that 
for each , . 
Then, by Proposition~\ref{prop-sum}, in order to have , 
for each  with  there must be exactly  
words  with  and such that 
there is no lexicographically greater word  with  
. This means that 
by removing all states in  and all transitions 
originating in the states belonging , the resulting 
automaton has at least  states, and thus 
 has at least  states. \eop 


\begin{corollary}
For each , there is up to isomorphism a unique OA with  states representing , 
the automaton  constructed in Lemma~\ref{lem-omega n}. 
\end{corollary}



\subsection{The function }
\label{sec-comb}

In this section, we give some estimations on the function  introduced 
above. 

\begin{proposition}
For all positive integers , 

\end{proposition}

{\sl Proof.}\  This is clear when , so assume that . 
To prove the upper bound, for each  consider a CPA  of size  without nontrivial strongly connected components 
and  having a single final state 
 which accepts a language 
of  words. Without loss of generality, we may assume that the 
sets  are pairwise disjoint. 
Let  be the ordered sum of the  constructed as above. 
Then for each , there are exactly  words taking the initial state  
to , and whenever  and  with , 
it holds that . Since  has 
states, we conclude that . 

To prove the lower bound, consider the automaton  obtained from 
by collapsing the final states  into a single final state. 
Then  accepts a language of  words
and has  states. Thus, 
. \eop 
 

In the rest of this section, we consider the case when .
In this case,  is a function on the positive integers. It is
not difficult to see that for each , 
 is the length of the shortest \emph{addition chain} \cite{Guy} 
representing , 
i.e.,  is the least integer  for which there there exist 
different integers  such that for each 
 there exist  with .
Addition chains have a vast literature
\cite{Sloane} . It is not difficult to show that
 is at most the sum of  and the number  of occurrences 
of the digit  in the binary representation of . If  is a power 
of , then . 
In the first paper \cite{Schonhage} published in the journal TCS, it was shown that  
is at least , where  is defined as above.  
By \cite{Downeyetal}, it is an NP-complete problem to decide for 
integers  whether  holds.  















\section{Conclusion and open problems}


We have shown that there is a polynomial time algorithm
to decide if two ordinal automata represent the same ordinal.
Since it is decidable in polynomial time whether the 
lexicographic ordering of the language accepted by a 
DFA is well-ordered, and since every DFA accepting a 
well-ordered regular language can be transformed in
polynomial time to an ordinal automaton, the restriction 
to ordinal automata was inessential.

A linear ordering is called \emph{scattered} if it does not have
a subordering isomorphic to the dense ordering of the rationals.
By Hausdorff's theorem \cite{Rosenstein}, every linear ordering is a dense 
sum of scattered linear orderings. Call a language 
scattered if its lexicographic ordering has this
property.

Hausdorff classified countable linear orderings according 
to their rank. It follows from results proved in
\cite{Heilbrunner} that the rank of the 
lexicographic ordering of a scattered regular language is always
finite. It is known (cf. \cite{BEscattered}) 
that a CPA  accepts a scattered regular language
iff for each nontrivial strongly connected component  
and , either  or .
It would be interesting to know whether there is 
a polynomial time algorithm to decide whether two DFA 
accepting scattered languages accept isomorphic languages.


\paragraph*{Acknowledgement}
The author would like to thank all three referees for suggesting improvements
and Szabolcs Iv\'an for the references on addition chains. 

\begin{thebibliography}{nn}




\bibitem{BEscattered}
S. L. Bloom and  Z. \'Esik,
Deciding whether the frontier of a regular tree is scattered,
\emph{Fund. Inform.}, 55(2003), 1--21. 

\bibitem{BEregwords}
S. L. Bloom and  Z. \'Esik, 
The equational theory of regular words.  
{\em Inform. and Comput.},  197(2005), 55--89.


\bibitem{BEord}
S. L. Bloom and Z. \'Esik, 
 Regular and algebraic words and ordinals.  
{\em Algebra and coalgebra in computer science},
LNCS 4624, Springer, Berlin, 2007,  1--15. 

\bibitem{BloomZhang}
S. L. Bloom and  Y. D.  Zhang,
A note on ordinal DFAs, arXiv:1005.2329v1, May 2010. 

\bibitem{Carayoletal}
L. Braud and A. Carayol,
Linear orders in the pushdown hierarcy, ICALP 2010, to appear. 

\bibitem{Courcelle}
B. Courcelle, Frontiers of infinite trees.  
{\em RAIRO Inform. Th\'eor.},  12(1978), 319--337.

\bibitem{Downeyetal}
P. Downey, B. Leony and R. Sethi, \emph{Computing sequences
with addition chains}, SIAM J. Comput. 3(1981),  121--125.

\bibitem{EsCF}
Z. \'Esik, An undecidable property of context-free languages,
arXiv:1004.1736v1, 2010.

\bibitem{Guy}
R. K. Guy, \emph{Unsolved Problems in Number Theory}, Springer, 2004. 

\bibitem{Heilbrunner}
S. Heilbrunner,  An algorithm for the solution of fixed-point equations for infinite words.  
{\em RAIRO Inform. Th\'eor.},  14(1980), 131--141.

\bibitem{Rosenstein}
J. G. Rosenstein, {\em Linear Orderings}. Pure and Applied Mathematics, 98. Academic Press, 
1982.

\bibitem{Schonhage} 
A. Sch\"onhage,  A lower bound on the length of addition chains,
\emph{Theoret. Comput. Sci.}, 1(1975), 1--12. 

\bibitem{Sloane}
N. J. A. Sloane, Length of shortest addition chain for .
\emph{ The On-Line Encyclopedia of Integer Sequences},
http://www.research.att.com/~njas/sequences/A003313.


\bibitem{Thomas}
W. Thomas, On frontiers of regular trees.  
{\em RAIRO Inform. Th\'eor. Appl.},  20(1986), 371--381.


\end{thebibliography}



\section*{Appendix}




Suppose that  with 
is a DFA having no nontrivial strongly connected component over 
the binary alphabet  such that all final states 
are sinks, i.e., whenever  is a final state, neither  nor  
is defined.


{\bf Algotithm 1} {\em Input}: A word  (of length less than ) such that 
neither  nor  is defined. 

{\em Output}: The number of different words  accepted by
 such that .

{\em Method}: 
Let  and  denote the  transition
matrices of  with respect to the letters  and ,
respectively. 
Let , where each  is either  or .
For each  with  and ,
consider the sum

where matrix sum and product are computed in the semiring of natural numbers. 
Since each word of length  or longer induces the 
empty partial function on the set of states, is clear that for each , 
 is the number of words accepted by  with 
initial state  and final state  of the form 
. 

Let  denote the -dimensional row vector whose first entry is 
and whose other entries are , and let  denote the -dimensional 
- column vector whose th component is  iff ,
for .  
Then,  is the number of all words 
accepted by  lexicographically greater than .
By the above consideration, and since each number 
occurring in the computation is at most  that 
can be represented by  bits, this number can be computed
in polynomial time in the number  of states.



{\bf Algorithm 2}
{\em Input}:  A state .

{\em Output}: The lexicographically greatest word  with .

{\em Method}: First, in polynomial time, compute the  
binary reachability matrix  such that  iff there is a word
 with . Then form a finite sequence of states 
  together with letters 
such that , and if  then 
 and  if ; and  
and  otherwise. 
The length  of this sequence is at most  
and  is the lexicographically greatest word  
with .



\end{document}
