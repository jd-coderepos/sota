






\documentclass[10pt]{article}
\usepackage{mathrsfs}
\usepackage{amsfonts,enumerate}
\usepackage{amsmath,amssymb,amsthm,graphicx,verbatim,psfrag,url,epstopdf}
\setlength{\textheight}{8.75in}
\setlength{\textwidth}{7.000in}
\setlength{\topmargin}{-0.200in}
\setlength{\oddsidemargin}{-0.375in}
\setlength{\evensidemargin}{-0.375in}
\setlength{\columnsep}{2pc}
\setlength{\headheight}{0.200in}
\flushbottom





\def\S{\mathcal {S}}
\def\M{\mathcal {M}}
\def\N{\mathcal {N}}
\def\I{\mathcal {I}}
\def\A{\mathcal {A}}
\def\B{\mathcal {B}}
\def\H{\mathcal {H}}
\def\X{\mathcal {X}}
\def\G{ G(p)}
\def\P{\mathscr{P}}
\def\C{\mathscr{C}}
\def\K{\mathscr{K}}
\def \p {\mathbf{p}}
\newcommand{\rmnum}[1]{\romannumeral #1}
\newcommand{\Rmnum}[1]{\expandafter\@slowromancap\romannumeral #1@}



\newtheorem{thm}{Theorem}
\newtheorem{clm}{Claim}
\newtheorem{lem}{Lemma}
\newtheorem{dfn}{Definition}
\newtheorem{obs}{Proposition}
\newtheorem{cor}[thm]{Corollary}
\newtheorem{conjecture}{Conjecture}



\begin{document}
\title{Maxwell-independence: a new rank estimate for the -dimensional generic rigidity matroid}
\author{Jialong Cheng\footnotemark[1]{  }\footnotemark[2]{}, Meera Sitharam\footnotemark[1]{}}
\date{}

\footnotetext[1]{University of Florida, Gainesville, FL 32611-6120, US, research supported in part by NSF grant DMSO714912 and
CCF 0610136, and a gift from SolidWorks Corporation
}

\footnotetext[2]{corresponding author: jicheng@cise.ufl.edu}
\maketitle

\begin{abstract}
The problem of combinatorially determining the rank of the 3-dimensional
bar-joint {\em rigidity matroid} of a graph is an important open problem in
combinatorial rigidity theory. Maxwell's condition states that the edges of a graph  are {\em independent} in its -dimensional generic rigidity matroid only if  the number of edges   , and  this holds for every induced subgraph with at least  vertices. We call such graphs {\em Maxwell-independent} in  dimensions.\footnote[2]{
{\bf Note:} Maxwell-independent graphs are called ``-sparse''  and ``-sparse'' in the literature (see \cite{JacksonBound2011}, \cite{LeeStreinu2008}). But we note that dense and sparse graphs have a variety of different meanings in graph theory. Our terminology is motivated by Maxwell's observation in 1864 that every graph  that is rigid in  dimensions must contain a Maxwell-independent subgraph that has at least  edges \cite{maxwell:equilibrium:1864}.}
Laman's theorem shows that the converse holds for  and thus every maximal Maxwell-independent set of  has size equal to the  rank of the -dimensional generic rigidity matroid. While this is
false for , we show that every maximal, Maxwell-independent set of a graph  has size at least the rank of the -dimensional generic rigidity matroid of . This answers a question posed by Tib\'or Jord\'an at the 2008 rigidity workshop at BIRS \cite{bib:birs}.

Along the way, we construct subgraphs (1) that yield alternative formulae for a rank upper bound for Maxwell-independent graphs and (2) that contain a maximal (true) independent set. We extend this bound to special classes of non-Maxwell-independent graphs. One further consequence is a simpler proof of
correctness for existing algorithms that give rank bounds.
\end{abstract}


\begin{comment}
\begin{frontmatter}
\title{Maxwell-independence: a new rank estimate for the -dimensional generic rigidity matroid}

\author[au1,au2]{Jialong Cheng\corref{au3}}
\author[au1,au2]{Meerea Sitharam}

\address[au1]{Department of CISE,
University of Florida, Gainesville, FL 32611-6120, US}

\cortext[au3]{corresponding author: jicheng@cise.ufl.edu}

\fntext[au2]{Supported in part by NSF grant DMSO714912 and
CCF 0610136, and a gift from SolidWorks Corporation}




\begin{abstract}
The problem of combinatorially determining the rank of the 3-dimensional
bar-joint {\em rigidity matroid} of a graph is an important open problem in
combinatorial rigidity theory. Maxwell's condition states that the edges of a graph  are {\em independent} in its -dimensional generic rigidity matroid only if  the number of edges   , and  this holds for every induced subgraph with at least  vertices. We call such graphs {\em Maxwell-independent} in  dimensions.\footnote[2]{
{\bf Note:} Maxwell-independent graphs are called ``-sparse''  and ``-sparse'' in the literature (see \cite{JacksonBound2011}, \cite{LeeStreinu2008}). But we note that dense and sparse graphs have a variety of different meanings in graph theory. Our terminology is motivated by Maxwell's observation in 1864 that every graph  that is rigid in  dimensions must contain a Maxwell-independent subgraph that has at least  edges \cite{maxwell:equilibrium:1864}.}
Laman's theorem shows that the converse holds for  and thus every maximal Maxwell-independent set of  has size equal to the  rank of the -dimensional generic rigidity matroid. While this is
false for , we show that every maximal, Maxwell-independent set of a graph  has size at least the rank of the -dimensional generic rigidity matroid of . This answers a question posed by Tib\'or Jord\'an at the 2008 rigidity workshop at BIRS \cite{bib:birs}.

Along the way, we construct subgraphs (1) that yield alternative formulae for a rank upper bound for Maxwell-independent graphs and (2) that contain a maximal (true) independent set. We extend this bound to special classes of non-Maxwell-independent graphs. One further consequence is a simpler proof of
correctness for existing algorithms that give rank bounds.
\end{abstract}
\begin{keyword}
Maxwell's condition \sep Maxwell-independent \sep independent \sep rank \sep  rigidity matroid \sep cover \sep inclusion-exclusion count \sep upper bound
\end{keyword}

\end{frontmatter}
\end{comment}




\section{Introduction}\label{sec:intro}
It is a long open problem to combinatorially characterize the -dimensional bar-joint rigidity of graphs.
The problem is at the intersection of combinatorics and
algebraic geometry, and  crops up in practical algorithmic applications ranging from
mechanical computer aided design to
molecular modeling.

\medskip\noindent
The problem is equivalent to combinatorially determining
the generic rank of the -dimensional bar-joint rigidity matrix of a graph.
The {\em -dimensional bar-joint rigidity matrix} of a graph  , denoted ,
is a matrix of indeterminates. Let  represent the coordinate position  of
the {\em joint} corresponding to a
vertex .
The matrix  has one row for each edge  and  columns for
each vertex .
The row corresponding to 
represents the {\em bar}
connecting  to  and has  non-zero indeterminate entries
 (resp. ), in the  columns corresponding to 
(resp. ) and zero in the other entries.

\medskip\noindent
A subset of edges , or a subgraph , of a graph  is said to be {\em independent} (we drop ``bar-joint'' from now on) in -dimensions, when the set of rows of  corresponding to  is {\em generically independent}, or independent for a generic instantiation of the indeterminate entries.
This yields the -dimensional {\em generic rigidity matroid} associated with a graph . The graph is {\em rigid} if the number of generically independent rows or
the rank of   is maximal, i.e., , where
 is the number of rotational and translational degrees-of-freedom of a rigid body in  \cite{graver:servatius:rigidityBook:1993}.

\medskip\noindent
Clearly, the number of edges of  is a trivial upper bound on the generic rank of
, or alternatively the {\it rank of the -dimensional rigidity matroid of }, which we denote by rank. Thus, a graph is independent in  dimensions only if   does not exceed ; and  this holds for every induced subgraph with at least  vertices. This is called {\em Maxwell's condition in  dimensions} \cite{maxwell:equilibrium:1864}, and we call such graphs (or their edge sets)  {\em Maxwell-independent} in  dimensions. 

\medskip\noindent
In other words, Maxwell's condition states that for any subset of edges of
, independence implies Maxwell-independence. For , the famous Laman's theorem states that the converse is also true. I.e., Maxwell-independence implies independence.
Thus (1) the rank of the -dimensional generic rigidity matroid of a graph  is exactly the size of any maximal, Maxwell-independent set (here, by {\em maximal} we mean that no edge can be added without violating Maxwell-independence) and (2) all maximal, Maxwell-independent sets of  must have the same number of edges.

\medskip\noindent
For , however, different maximal, Maxwell-independent sets may
have different sizes, see Figure \ref{fig:banana}.
I.e, for , the collection of Maxwell-independent sets does
not yield a matroid.
Clearly, any maximal independent subgraph of 
is itself Maxwell-independent, so the rank of the generic rigidity matroid of a
graph is at most the
size of {\em some} maximal Maxwell-independent set and this
generalizes to any dimension.
But this only yields the trivial upper bound, i.e., number of edges, for Maxwell-independent graphs.
For other special classes of graphs such as graphs of
bounded degree, graphs that satisfy certain covering conditions
etc., alternative combinatorial formulae are known \cite{JacksonJordanrank:2006, JacksonJordansparse:2005}, that give better bound than the number of edges in some cases.

\begin{center}
\begin{figure}[!h]
\begin{center}
\scalebox{0.3}[0.3]{\includegraphics{banana-eps-converted-to.pdf} \includegraphics{banana_s1-eps-converted-to.pdf} \includegraphics{banana_s2-eps-converted-to.pdf}}
\end{center}
\caption{The graph on the left is called a {\em double-banana} and consists of two 's intersecting on an edge. The graphs on the middle and the right are two maximal Maxwell-independent sets of different sizes for the graph on the left (the middle is of size  and the right is of size )
.}\label{fig:banana}
\end{figure}
\end{center}
\medskip
\noindent
This leads to the following natural question concerning the rank of the -dimensional generic rigidity matroid. The question was posed by Tib\'or Jord\'an during the 2008 BIRS rigidity workshop \cite{bib:birs}.

\medskip
{\em Question ():}
Does {\it every} maximal, Maxwell-independent subgraph (subsets of edges) of a graph   have size at least the rank of the -dimensional generic rigidity matroid of ?

Note that the answer to Question () would be obvious if every maximal Maxwell-independent set of a given graph  contains a maximal independent set of . However, this is not the case. See Figure \ref{fig:bananaBar}.

\begin{center}
\begin{figure}[!h]
\begin{center}
\scalebox{0.3}[0.3]{\includegraphics{banana_bar-eps-converted-to.pdf}\includegraphics{banana_s1-eps-converted-to.pdf} }
\end{center}
\caption{On the left is a double-banana-bar, which consists of a double-banana and a bar connecting two vertices from each banana. Notice that this double-banana-bar is rigid, thus every maximal independent set in it has  edges. On the right we have a maximal Maxwell-independent set of the double-banana-bar, which has  edges. The figure on the right is dependent, so every maximal independent set of it has size less than . So the right figure cannot contain a maximal independent set of size . }\label{fig:bananaBar}
\end{figure}
\end{center}

\medskip
\noindent
Our main result (Theorem \ref{thm:main}) in Section \ref{sec:main} gives an affirmative answer to Question () for . Bill Jackson \cite{JacksonBound2011} has extended this result up to . His proof is by contradiction and is hence nonconstructive. Our proof is constructive: for Maxwell-independent graphs, we give combinatorial formulae based on inclusion-exclusion (IE) counts upper bounding the rank; and we construct subgraphs ({\em independence assignments}) whose sizes meet this bound, and moreover contain a maximal true independent set (Theorems \ref{thm:weakrankIE} and \ref{thm:propermaximal}); this construction is of algorithmic interest. The construction leads to alternative upper bounds on rank related to Dress' formula (\cite{bib:Dress}, Section \ref{sec:knownbounds}) for certain classes of non-Maxwell-independent graphs that admit certain types of covers in Section \ref{sec:nonMaxwell} (Theorems \ref{thm:complete2thin} and \ref{thm:properComplete2thin}). However, algorithms for computing these covers are beyond the scope of this paper.
\begin{comment}
it uses Lemma \ref{lem:comb}(\ref{thm:9tree}) which elucidates the structure of a special type of cover that can always be found for Maxwell-independent but
non-Maxwell-rigid graphs. This structure is of independent interest and is exploited (by Proposition \ref{obs:rankIE}) employing a commonly used rank inclusion-exclusion (IE) count. Our best bound (in Theorem \ref{thm:weakrankIE}) is employed in answering both Questions 1 (Theorem \ref{thm:main}) and 2 (Proposition \ref{obs:rankIE} and Theorem \ref{thm:propermaximal}). 
\end{comment}



Several algorithms exist for combinatorially recognizing certain types of dependences for   (\cite{bib:survey, andrewThesis,sitharam:zhou:tractableADG:2004}
). The simplest of these algorithms is a
minor modification (\cite{andrewThesis}) of Jacobs and Hendrickson's (\cite{Jacobs97analgorithm}) pebble game for , and finds a maximal Maxwell-independent set (it may be neither the minimum sized one nor the 
maximum sized one). The techniques developed in this paper simplify the proofs of correctness for these algorithms.

\medskip

\medskip\noindent
In Section \ref{betterbound}, we also relate our bounds to existing bounds and conjectures.
In the concluding Section \ref{conclusion}, we pose open problems.






\section{Main Result and Proof}\label{sec:main}
In this section, we state and give the proof of the following main theorem. Note that Sections \ref{sec:main} and \ref{betterbound} deal exclusively with  and we use rank to denote the rank of the -dimensional generic rigidity matroid of graph .


\begin{thm}\label{thm:main}
Let  be a maximal Maxwell-independent subgraph of a graph  and  be a maximal independent set of the -dimensional generic
rigidity matroid of . Then , where  denotes the edge set of .
\end{thm}

\medskip
\noindent The proof requires a few definitions.


\begin{dfn}\label{dfn:Maxwell}
The {\em Maxwell count} for a graph  in  dimensions is .  is said to be \emph{Maxwell-rigid} in  dimensions, if there exists a Maxwell-independent subset  such that
the Maxwell count of  is at most . As exceptions, -cliques  () are considered to be Maxwell-independent and Maxwell-rigid.

A subgraph  induced by  is
said to be a {\em component} of , if it is Maxwell-rigid. In addition,  is called a {\em vertex-maximal component} of , if it is Maxwell-rigid and there is no proper superset of  that also induces a Maxwell-rigid subgraph of . A component with  vertices consists of a single edge of the graph, and we call it an {\em edge component}, or {\em trivial} component. Other components are called {\em non-trivial} components.
\end{dfn}





\medskip\noindent
The following concepts of covers and inclusion-exclusion formulae on covers from  \cite{crapo:structuralRigidity:1979, sitharam:zhou:tractableADG:2004, andrewThesis, bib:survey, lovasz:yemini, JacksonJordansparse:2005, JacksonJordanrank:2006} are important for the proof of Theorem \ref{thm:main}.


A \emph{cover} of a graph  is a collection 
of pairwise incomparable induced subgraphs  of , each with at least two vertices, such that , where  is the edge set of subgraph .  denotes the vertex set of . Let  denote the graph ,  and  denote the graph , . 


Given a graph  with a cover    , we use  to denote the set of all pairs of vertices  such that  for some . Denote by  the number of elements in  that contain both  and . Then we can define two different inclusion-exclusion formulae on covers as follows, where the first is used in the proof of Theorem \ref{thm:main} and the second is used later in the paper:



\begin{dfn}\label{dfn:IE}
Given a graph , let ,  be a cover of  where  are edge components and  are subgraphs with at least  vertices.
The \emph{rank inclusion-exclusion (IE) count} of cover  is defined as the following:


The \emph{full rank inclusion-exclusion (IE) count} of cover  in is defined as

\end{dfn}

\begin{comment}
The {\em Dress rank inclusion-exclusion (IE) count} of cover  is defined as the following:

The \emph{strong rank inclusion-exclusion (IE) count} of cover  is defined as the following:

\end{comment}

\medskip
\noindent
The relationships between the two types of IE count defined in Definition \ref{dfn:IE} will be discussed in Section \ref{sec:knownbounds}. The proof of Theorem \ref{thm:main} only uses IE.

Now we are ready to prove Theorem \ref{thm:main}.
\begin{proof}(of Theorem \ref{thm:main})
First, notice that if  itself is independent, we are done.
Similarly, if  is Maxwell-rigid, then we have  rank = , hence we are done. 


Let  with  rank be a maximal independent set of . Without loss of generality, let . Let . Thus   . Here  means the linear span of those rows of the rigidity matrix  corresponding to .



Consider a cover   of  by the {\em complete} collection of vertex-maximal components, where  are edge components and      are non-trivial components. Next we show that for each edge  in , there exists at least one non-trivial component  such that  and .


Since  ,    is not an edge component of . Hence if  and  lie inside any component of , the component must be non-trivial. If no component  contains both  and , then in fact no vertex-maximal component of  contains both  and , since  is the complete collection of vertex-maximal components of . Next we will show that  is Maxwell-independent.

Suppose not. We know there is a violation to Maxwell's condition in  and this must be caused by the addition of , since  is Maxwell-independent. To violate Maxwell's condition, both endpoints of  must lie inside a same non-trivial Maxwell-rigid subgraph of , and every non-trivial Maxwell-rigid subgraph of  lies inside a non-trivial vertex-maximal component of . This contradicts the fact that no vertex-maximal component of  contains both  and . Hence  is Maxwell-independent, contradicting the maximality of . So for each edge  in , there exists at least one non-trivial component  such that  and . 


Denote by  the set of edges of  both of whose endpoints are in . Hence 





Take  and  as defined earlier in the section. We get


Since each  is Maxwell-rigid, adding any  into  causes the number of edges in  to exceed  and in turn indicates the existence of a true dependence. However,    , since   . It follows that  was already dependent even before  was added. I.e., to obtain an independent set in , at least  edges must be removed from . So we have



Plugging \eqref{eqn:MI} into \eqref{eqn:Md}, we have 
 

From Proposition \ref{obs:vmmr}(\ref{obs:vmmr1}) below, we know that the cover  is 2-thin. Then we can apply Theorem \ref{thm:weakrankIE} below and obtain the following:
 
Then, using \eqref{eqn:sum} and \eqref{eqn:IIMDecomp}, we obtain that 
 which proves Theorem \ref{thm:main}. 
\end{proof}

\medskip\noindent
Note that the proof of Theorem \ref{thm:main} uses a cover by the complete collection of vertex-maximal components. This not only implies 2-thinness of the cover, but also strong 2-thinness. However, 2-thinness (Proposition \ref{obs:vmmr}(\ref{obs:vmmr1})) is sufficient for proving Theorem \ref{thm:main}. Strong 2-thinness is used in Section \ref{betterbound}.




In the remainder of this section, we state and prove Proposition \ref{obs:vmmr} and Theorem \ref{thm:weakrankIE} and the required lemmas. The following concept, as defined in \cite{JacksonJordanrank:2006}, is needed to state Proposition \ref{obs:vmmr}.

\begin{dfn}\label{dfn:2thin}
Let  be a cover . We say  is {\em -thin} if  for all .  We say a -thin cover  is {\em strong 2-thin} if for all , whenever , then  and  in fact share an edge.
\end{dfn}


Next, we prove a lemma illustrating an elementary, but useful property of the union of two Maxwell-rigid graphs.
\begin{lem}\label{lem:join}
\begin{enumerate}[(a)]
\item \label{lem:join1} Given Maxwell-rigid graphs  and , if    consists of two vertices  and  and     , then    is also Maxwell-rigid. 
\item \label{lem:join2} Given Maxwell-independent graph  and two Maxwell-rigid subgraph  and  of , if   , then    is also Maxwell-rigid.
\end{enumerate}

\end{lem}
\begin{proof}
\begin{enumerate}[(a)]
\item Let  be a Maxwell-independent subgraph of  with    edges and  be a Maxwell-independent subgraph of  with    edges. We show next that    is Maxwell-independent.

 Suppose    is Maxwell-dependent. Then there exists     such that  has Maxwell count less than . Since both  and  are Maxwell-independent, it is clear that  and . Let  such that  and . Then  and  both have Maxwell count at least 6. To make their union have Maxwell count less than 6,  and  must share at least two vertices. Since  consists of two vertices  and , we know    consists of at most two vertices  and .

Since , it can be seen that in order to make  of Maxwell count less than , at least one of  and  will have Maxwell count less than , which together with the fact that  and  violates Maxwell-independence of  or . Hence    is Maxwell-independent. Notice that    has enough edges to be Maxwell-rigid and thus    is also Maxwell-rigid.

\item Since  is Maxwell-independent, we know both   and  are Maxwell-independent. Then can we calculate the Maxwell count of  as follows. We know (1)  and  each have Maxwell count  and (2)    has Maxwell count at least  since    is Maxwell-independent and has at least  vertices. Thus the Maxwell count of    is at most . Together with the fact that    is Maxwell-independent, we know    is Maxwell-rigid.
\end{enumerate}
\end{proof}


This following proposition gives a useful property of a cover of a Maxwell-independent subgraph by vertex-maximal components.
\begin{obs}\label{obs:vmmr}
Let  be a Maxwell-independent graph. Let   be a cover of  by vertex-maximal components, where  are edge components and  are non-trivial components. Then
\begin{enumerate}[(a)]
\item\label{obs:vmmr1}  is a 2-thin cover of .
\item\label{obs:vmmr2}  is strong 2-thin.

\begin{comment}
\item\label{obs:vmmr3} Suppose  is the {\em complete} collection of vertex-maximal components of . Let  be a maximal independent set of  and extend  to a maximal independent set  of . Let . Then for each edge  in , there exists at least one non-trivial component  such that  and .
Denote by  the set of edges of  both of whose endpoints are in .
Then we have 
\end{comment}
\end{enumerate}
\end{obs}
\begin{proof}
\begin{enumerate}[(a)]
\item Edge components do not affect the cover being 2-thin or not. 
Let  and  be two non-trivial vertex-maximal components in . Suppose  and  share at least  vertices. Then from Lemma \ref{lem:join}(\ref{lem:join2}), we know    is Maxwell-rigid, violating the fact that  and  are vertex-maximal components.


\item  Again, edge components do not affect the cover being strong 2-thin or not. Let  and  be two non-trival vertex-maximal components in . If  and  share two vertices but do not share an edge, then from Lemma \ref{lem:join}(\ref{lem:join1}),    is Maxwell-rigid, which violates the vertex-maximal property of  and .





\end{enumerate}
\end{proof}




\medskip\noindent
Next we prove a lemma about the structure of a 2-thin cover of a Maxwell-independent graph.
We first need the following definition of {\em 2-thin component graph}.


\begin{dfn}\label{dfn:componentGraph}
Given graph , let     be a 2-thin cover of  by components of . The {\em 2-thin component graph}  of  ({\em component graph} for short) is defined as follows.     , where  consists of {\em component nodes} , one for each component  in ; and  consists of {\em edge nodes} , one for each edge  shared by at least two components in . The edges in  are of the form , where   ,   , and  is a shared edge of .
\end{dfn}
\noindent
Figure \ref{compG} shows how to obtain a 2-thin component graph from a graph and a cover by its vertex-maximal components.

\medskip
\noindent
Note that components sharing only vertices are non-adjacent in the component graph. Edge components have degree zero and become disconnected nodes in the component graph. See Figure \ref{compG}.

\begin{center}
\begin{figure}[!h]
\begin{center}
\includegraphics[width=.5\textwidth]{compGraph.png}
\end{center}
\caption{The figure on the left represents the vertex-maximal components of a graph. On the right side is its 2-thin component graph, where circles represent component nodes and squares represent edge nodes. Note that the 2-thin component graph may not be connected.}\label{compG}
\end{figure}
\end{center}

\begin{comment}
\begin{center}
\begin{figure}[h]
\begin{center}
\psfrag{Con}{Two components share an edge}\psfrag{Dis}{Two components do not share an edge}
\scalebox{0.5}[0.5]{\includegraphics{compGraph}}
\end{center}
\caption{The figure on the left represents the vertex-maximal components of a graph. On the right side is its 2-thin component graph, where circles represent component nodes and squares represent edge nodes. Note that the 2-thin component graph may not be connected.}\label{compG}
\end{figure}
\end{center}
\end{comment}

\noindent
Lemma \ref{lem:comb}(\ref{thm:9tree}) below states an important property of component graphs of Maxwell-independent graphs. Specifically,  these 2-thin component graphs generalize the concept of partial -trees (also called {\em tree-width  graphs}) and Henneberg constructions \cite{graver:servatius:rigidityBook:1993}, which we define below.

\medskip
\noindent
\begin{dfn}\label{dfn:comppartialmTree}
Let  be a positive integer. Then a 2-thin component graph is called a {\em generalized partial -tree} if it can be reduced to an empty graph by a sequence of the 
following two operations: (i) removal of a component node of degree at most  and 
(ii) removal of an edge node of degree one. 
\end{dfn}



Now we are ready to state the lemma. \begin{lem}\label{lem:comb} If  is a Maxwell-independent graph and  is a 2-thin cover of  by components of , then
\begin{enumerate}[(a)]
\item \label{thm:degree}
the component nodes of any subgraph of the 2-thin component graph  have average degree strictly less than .
 \item \label{thm:9tree}
any subgraph of the 2-thin component graph  of  is a generalized partial -tree.
 \end{enumerate}
\end{lem}
\begin{proof}
\begin{enumerate}[(a)]
\item\label{lem:comb1} First we remove all edge components of  and show the remainder of the component graph has average degree .

Let  be any subgraph of the 2-thin component graph . Let  denote 's corresponding subgraph in .
Let  be  restricted to . Let  and  be the shared vertex and shared edge sets of component  of , i.e.,  and  are shared by other components  of . Let  and  be the entire sets of such shared vertices and shared edges in . Let  and  denote the number of components  of  that share  and  respectively. Since the Maxwell count of each  is  (they are all non-trivial), the Maxwell count of  can be calculated as follows:


Suppose the Maxwell count of  is . We have



Consider any shared vertex  in . Denote by  the set of indices of components containing . In this proof, since the context is clear, we refer to ,  as a component containing . The collection of all  components of  meeting at  forms a subgraph . Since  is Maxwell-independent,  is also Maxwell-independent. Let  be the number of shared edges incident at  in component  and  be the number of shared edges that are incident at . Then the Maxwell count of  can be computed as follows:
\begin{itemize}
\item there are  components, which contributes ;
\item  is shared by  components, and the contribution is ;
\item each shared edge in a component  contributes  to the Maxwell count, and altogether the shared edges contribute 
\item for each shared edge , vertex  contributes 
\item for the set of shared vertices that are not part of any shared edge in , their contribution is  for a non-negative number ;
\end{itemize}
Thus the Maxwell count of  is:

Since  is Maxwell-independent, we know:

Since  , we know

Summing over all shared vertices in , we have:

Since ,  and , we know

Plugging into \eqref{eqn:MC}, we have:

Since , we have:

We now observe that the component nodes in  must have average degree strictly less than . Otherwise, , leading to a contradiction that

This proves (\ref{lem:comb1}).

\item This follows immediately from (\ref{lem:comb1}).

\end{enumerate}


\end{proof}

Next we establish a condition on the cover of a Maxwell-independent graph such that the IE count in Definition \ref{dfn:IE} gives an upper bound on rank. This condition is called an {\em independence assignment}.

\begin{dfn}\label{dfn:assign}
Given a graph  and a cover  of , we say  has an {\em independence assignment} ; , if there is an independent set  of  and maximal independent set
 of each of the 's, such that  {\em restricted to} , (denoted ), is contained in  and for any , 
is missing from at most one of the 's whose corresponding  contains . When  is clear, we also say there is an independence assignment for . \end{dfn}


\medskip
\noindent
The next lemma shows the existence of an independence assignment for Maxwell-independent graphs.

\begin{lem}\label{lem:assign}
If  is Maxwell-independent and  is a 2-thin cover of  by components of , then  has an independence assignment.
\end{lem}
\begin{proof} (of Lemma \ref{lem:assign}).
In fact, we can construct an independence assignment if the 2-thin component graph of  is a generalized partial -tree.
From Lemma \ref{lem:comb}(\ref{thm:9tree}), we know that any subgraph of the 2-thin component graph  of  is a generalized partial
-tree, which is automatically a generalized partial -tree. Let   be the component nodes of  listed in reverse order from the removal order in Definition \ref{dfn:comppartialmTree}. We use induction to prove that there is always an independence assignment for .

If  has only one component, it is clear that we can find an independence assignment.

Suppose there is an independence assignment ;   for a subgraph  of  containing the component nodes . After adding  to form , we need to find , which is a maximal independent set of , and  for  such that ;   is an independence assignment.

First we take  for  and let  be the set of edges of  that are shared by other components. Since ,  is independent for , because for , a minimum-size graph that is not independent will have at least  edges. Thus we can extend  to a maximal independent set  of . Now let    , then (1)  spans all edges in , and (2) every edge  in  that is shared by at least two components in   is missing in at most  of the 's sharing , since (a) ;   is an independence assignment for  and (b)  contains all shared edges of . If  is already independent, we have our independence assignment. Otherwise we can remove a minimum number of edges from  until it is independent.

\end{proof}

The following theorem gives an alternative combinatorial upper bound on rank of rigidity matroid of Maxwell-independent graphs. This also completes the proof for Theorem \ref{thm:main}.

\begin{thm}\label{thm:weakrankIE}
Let  be a Maxwell-independent graph and ,  be a 2-thin cover of  by components of . Then .


\end{thm}
\begin{proof}

When  is a 2-thin cover, we can apply Lemma \ref{lem:assign} and obtain that  has an independence assignment.

First we remove all edge components of  to obtain a new graph . Now the existence of an independence assignment directly implies that      .


Next we consider the edge components . If we add the contributions of all of them to both sides of the inequality, the left hand side becomes , and the right hand side becomes  , which is at least the rank of , since    .

\end{proof}









\section{Alternative Upper Bounds Using IE Counts}\label{betterbound}
\subsection{Relation to Known Bounds and Conjectures Using IE Counts}\label{sec:knownbounds}
\noindent Decomposition of graphs into covers is a natural way of approaching a combinatorial characterization of -dimensional rigidity. So far, the
inclusion-exclusion(IE) count method for covers has been used by many in the
literature (see \cite{crapo:structuralRigidity:1979, sitharam:zhou:tractableADG:2004, andrewThesis, bib:survey, lovasz:yemini, JacksonJordansparse:2005, JacksonJordanrank:2006}). The most explored decompositions are the 2-thin covers.

\medskip\noindent
We defined two types of rank IE counts in Definition \ref{dfn:IE}, with IE being used in the proof of Theorem \ref{thm:main}. Our Theorem \ref{thm:propermaximal} below in Section \ref{sec:proper}, will show that for a {\em specific, not necessarily independent} cover, a slightly different inclusion-exclusion count is equal to IE count, which in turn gives a rank upper bound for Maxwell-independent graphs. 


\medskip\noindent
Besides the IE count, other IE counts have also been explored in the aforementioned literature.
In 1983, Dress et al \cite{bib:Dress,bib:Tay84} conjectured that the minimum of the IE count taken over all 2-thin covers is an upper bound on the rank of the 3-dimensional generic rigidity matroid. However,
this conjecture was disproved for general graphs by Jackson and Jord\'an in
\cite{Jackson03thedress}. 

\medskip\noindent
Although Dress' conjecture is false, the IE count can be an upper bound of the rank if the cover is special: it is shown in \cite{JacksonJordanrank:2006} that the minimum of the IE count taken over all {\em independent} 2-thin covers is an upper bound on the rank. Here, an {\em independent 2-thin cover}  is one for which the edge set given by the pairs in the shared part  is independent. It is also shown that to achieve the upper bound, the covers need not be independent, but can be obtained as iterated, or recursive version of independent covers. 

\medskip\noindent
We have no examples where our bound in Theorem \ref{thm:propermaximal} is better than the above mentioned bound from \cite{JacksonJordanrank:2006}, which was conjectured to be tight when restricted to non-rigid graphs and covers of size at least . Hence any such examples would be counterexamples to their conjecture.
However, our formula provides an alternative way of computing a rank upper bound using not necessarily independent covers.

\medskip\noindent
In Section \ref{sec:nonMaxwell}, we use the same IE count
over another special cover, which is a {\em specific non-iterated, non-independent} cover, to obtain rank bounds on Maxwell-dependent graphs. 
Again, we have no examples where our
bound is better than the above mentioned bound in \cite{JacksonJordanrank:2006}, which was conjectured to be tight. Hence any such examples would be counterexamples to their
conjecture. Our bound gives an alternative method using a specific,
non-iterated, not necessarily independent cover by (proper) vertex-maximal components. However, the catch is that these covers may not exist for general graphs.




\subsection{Alternative Upper Bounds for Maxwell-Independent Graphs}\label{sec:proper}
In this section, we give alternative combinatorial bounds on the rank of the generic rigidity matroid of Maxwell-independent graphs in  dimensions.


Notice that if  is a Maxwell-independent graph with a cover     by vertex-maximal components, then    and thus        IE  rank.


However, when a graph  is Maxwell-rigid, there is a single vertex-maximal component namely  itself, so the above bound is uninteresting. In this case, we use the cover of  by ``proper'' vertex-maximal components:

\begin{dfn}\label{dfn:properMax}
Given graph , an induced subgraph is \emph{proper vertex-maximal, Maxwell-rigid} if it is Maxwell-rigid and the only graph that properly contains this
subgraph and is Maxwell-rigid is  itself.
\end{dfn}

\noindent Since the collection of proper vertex-maximal components may not be a 2-thin cover even for Maxwell-independent graphs, Theorem \ref{thm:weakrankIE} does not directly apply. The following theorem deals with cases that are relatively minor variations of Theorem \ref{thm:weakrankIE}.


\begin{thm}\label{thm:propermaximal}
Let  be a Maxwell-independent graph and     be a cover of  by proper vertex-maximal components. Then we have:
\begin{enumerate}
\item If  is strong -thin, then       IE  rank.
\item If  is -thin but not strong -thin,  consists entirely of two non-trivial components  and  in  s.t.      and hence       rank.
\item Otherwise, there exist two non-trivial components  and  in , s.t.      and hence       rank.
\end{enumerate}

\end{thm}
\begin{proof}
\begin{enumerate}
\item When  is strong -thin, we know    and thus       IE. Then it follows from Theorem \ref{thm:weakrankIE} that IE  rank.

\item When  is -thin but not strong -thin, we know there exist two proper vertex-maximal components  and , s.t.    has two vertices but no edge. From Lemma \ref{lem:join}(\ref{lem:join1}), we know    is Maxwell-rigid. Since  and  are both proper vertex-maximal, we know     . Since  is Maxwell-independent, we know     . Since the cover is -thin, no other non-trivial vertex-maximal component can exist. Hence  and  are the only two non-trivial components in  and it follows that rank    and hence        rank.




\item When  is not -thin, i.e., there exist  and  such that their intersection has at least  vertices. From Lemma \ref{lem:join}(\ref{lem:join2}), we know the union of  and  is also Maxwell-rigid. Since  and  are both proper vertex-maximal, we know     . Since  is Maxwell-independent, we know     .

It remains to show that       rank. To show this, we can start from a maximal independent set  of   , and expand it to maximal independent sets  of  and  of . It is clear that  spans the graph   , and hence         rank.




\end{enumerate}


\end{proof}

\subsection{Removing the Maxwell-Independence Condition}\label{sec:nonMaxwell}
We now give rank bounds for Maxwell-dependent graphs using the IE count. We start with the following simple but useful property of edge-sharing, Maxwell-rigid subgraphs.

\begin{lem}\label{obs:maximal}
Given graph , let  and  be two subgraphs of  s.t.    consists of two vertices ,  and an edge .
\begin{enumerate}[(a)]
\item\label{obs:maximal1} If  is a vertex-maximal component of  and there is a  Maxwell-independent subgraph  of  s.t.    and , then every maximal Maxwell-independent subgraph of  contains .
\item\label{obs:maximal2} If  is a proper vertex-maximal component of  and there is a  Maxwell-independent subgraph  of  s.t.    and , then one of following holds:
(1)     , or
(2) every maximal Maxwell-independent subgraph of  contains .
\end{enumerate}
\end{lem}
\begin{proof}
\begin{enumerate}[(a)]
\item Suppose there is a Maxwell-independent subgraph  of 
such that    is Maxwell-dependent. Then there must be a subgraph  of  such that  has Maxwell count . Then it follows from Lemma \ref{lem:join}(\ref{lem:join1}) that   is also Maxwell-rigid, a contradiction to the vertex-maximality of .
\item Statement follows from (\ref{obs:maximal1}) and the proper vertex-maximality of . 
\end{enumerate}
\end{proof}

Next we give two similar theorems with similar proofs. The first theorem, Theorem \ref{thm:complete2thin}, gives a rank bound for graphs for which the complete collection of vertex-maximal components forms a -thin cover. The second, Theorem \ref{thm:properComplete2thin}, concerns proper vertex-maximal components.

\begin{thm}\label{thm:complete2thin}
For a graph , if the complete collection    ,  of vertex-maximal components forms a 2-thin cover, then IE is an upper bound on , i.e.,  \end{thm}

\begin{proof}
We first consider the case where there are no edge components.

First, we show that the cover  is strong 2-thin. Suppose not, then there exists  such that . Suppose further that  and  both contain  and . From Lemma \ref{lem:join}(\ref{lem:join1}), we know  is Maxwell-rigid, contradicting the fact that  and  are vertex-maximal, Maxwell-rigid.  Hence the cover  is strong 2-thin and IE can be rewritten as .

We need the following claim (which is also used for proving Theorem \ref{thm:properComplete2thin}).

\medskip
\begin{clm}\label{clm:2thin}
For a graph   , if the complete collection   , , ,  of (proper) vertex-maximal components forms a strong 2-thin cover, then there is a maximal Maxwell-independent subgraph  of  s.t. IE   and hence IE  .
\end{clm}

\begin{proof}
We show the claim for the case where  consists of vertex-maximal components. However, along the way, we point out the slight differences for the case where  consists of proper vertex-maximal components, making the claim applicable also to Theorem \ref{thm:properComplete2thin}.

We first construct a subgraph   with  equal to IE as follows. For , denote by  a {\em maximum} sized Maxwell-independent subgraph of . Then from Lemma \ref{obs:maximal}(\ref{obs:maximal1}), we know that for any edge , there is at most one , such that  is Maxwell-dependent. ({\em Note:} from Lemma \ref{obs:maximal}(\ref{obs:maximal2}), even if  is a cover by complete collection of proper vertex-maximal components, when there are no two components  and  s.t.     , it still holds that for any edge , there is at most one , such that  is Maxwell-dependent.)



\medskip
\noindent Thus, edges of component  can be divided into four parts:
\begin{itemize}
\item : the set of edges  in  that are present in each  for which  contains ;
\item : the set of edges  in  for which there is exactly one  where  , i.e.,  is Maxwell-dependent;
\item : the set of edges  in , and present in all other 's, where  contains .
\item : .
\end{itemize}

\begin{comment}
\begin{center}
\begin{figure}[h]
\begin{center}
\psfrag{Ci}{\huge }\psfrag{P1}{\huge }\psfrag{P2}{\huge
}\psfrag{P3}{\huge }\psfrag{P4}{\huge }
\includegraphics[width=.5\textwidth]{four_part}
\end{center}
\caption{Illustration of 4 parts of edges in  in the proof of Theorem \ref{thm:complete2thin}: the solid lines inside the circle
represent the edges that are in , and the dashed lines inside the circle
represent edges that are not in  but in . The solid lines outside the circle are present in all (other) 's where  contains . The dashed lines outside the circle are edges that are absent in exactly one   .}
\label{four_part}
\end{figure}
\end{center}
\end{comment}


Let . Now we construct  as follows. First, let . Then we construct the edge set  by removing all edges in  and  from . Thus     , where  denotes  restricted to .

Now note that     , which is exactly IE, since  is strong -thin.
In the following we show that this number is at least rank by showing that  is a maximal Maxwell-independent subgraph of  and using Theorem \ref{thm:main}.
\begin{enumerate}[(I)]

\item  is Maxwell-independent. Suppose not, then we can find a minimal subgraph  that is Maxwell-dependent. Since  is picked in such a way that every  is Maxwell-independent, we know  cannot be inside any . Because  is minimal, we know there exists  that (1) contains all vertices of  and (2) is Maxwell-independent with Maxwell count .
Then  is a component that is not contained in any , since  is not inside any , and removing an edge from  does not make it inside any  either. That is a contradiction to the fact that  is the {\em complete} collection of vertex-maximal components of . ({\em Note:} this contradiction would hold even if  is a cover by complete collection of proper vertex-maximal components.)

\item  is a maximal Maxwell-independent subgraph of . In order to show this, we first notice that for every , every maximal Maxwell-independent subgraph  of  contains , which follows from the statements that (1) there exists a  s.t  is Maxwell-dependent and (2) Lemma \ref{obs:maximal}(\ref{obs:maximal1}). ({\em Note:} from Lemma \ref{obs:maximal}(\ref{obs:maximal2}), even if  is a cover by complete collection of proper vertex-maximal components, when there are no two components  and  s.t.     , it still holds that for every , every maximal Maxwell-independent subgraph of  contains .)





Suppose there is an edge  such that  is  Maxwell-independent. Then  (which denotes  restricted to ) is also Maxwell-independent. Since  for some , we know  or . In fact every edge  for some  is also in  for some , without loss of generality, we choose a component  such that  or . 
Notice that there is an extension of  into a maximal Maxwell-independent subgraph  of , which must contain all edges in  as shown in the previous paragraph, i.e.,  contains   . Since     , we know  has size larger than , which is a contradiction to the fact that  is a maximum sized Maxwell-independent subgraph of . Hence  is maximal Maxwell-independent.

\end{enumerate}

\medskip\noindent
Thus we know  is a maximal Maxwell-independent set of . From Theorem \ref{thm:main}, we know . As noticed before, the IE count of the cover  is equal to , hence we have
   . 

\end{proof}

Returning to the proof of Theorem \ref{thm:complete2thin}, we first notice that Claim \ref{clm:2thin} completes the proof, when there are no edge components in the cover .

With edge components in the cover, notice that each edge component contributes  to the left hand side but contributes at most  to the right hand side. Thus the inequality still holds.

\end{proof}


The next theorem extends the bound in Theorem \ref{thm:complete2thin} to covers by proper vertex-maximal components.

\begin{thm}\label{thm:properComplete2thin}
For a graph , if the complete collection   , ,  of proper vertex-maximal components forms a 2-thin cover, then the IE count of the cover  is an upper bound on , i.e., 
\end{thm}

\begin{proof}
When  is not Maxwell-rigid, the proof is the same as in Theorem \ref{thm:complete2thin}.

When  is Maxwell-rigid, we first show the theorem for the case where there are no edge components. There are two further cases:
\begin{description}

\item[Case 1.] There exist two components  and  s.t.     . In this case, all other non-trivial components in the cover can only be  or . For every edge  in those components, we know (1) if  , then  contributes to  to both the left hand side and right hand side of the inequality; and (2) if  ,  then  contributes to  to the left hand side, and  or  to the right hand side of the inequality. 

Thus if we can show that IE count on  is an upper bound on the rank of , then the theorem holds. Note that IE count on  is equal to , and from the axiom  of abstract rigidity matroid (see \cite{graver:servatius:rigidityBook:1993}), we know  is not rigid and thus rank is at most . Hence IE count on  is an upper bound on the rank of .


\item[Case 2.] For any two components  and , we have     . In this case, we know the cover is strong -thin, since otherwise, there exist two components  and  whose intersection is a pair of vertices without an edge. From Lemma \ref{lem:join}(\ref{lem:join1}), we know  is Maxwell-rigid. Since both  and  are proper vertex-maximal components, we know     , a contradiction.

Next, we apply Claim \ref{clm:2thin} of Theorem \ref{thm:complete2thin} to complete the proof of Theorem \ref{thm:properComplete2thin} where there are no edge components.

\begin{comment}
Next, we can 

follow the proof of Theorem \ref{thm:complete2thin}. First we can obtain that for every edge , there is at most one  where  contains  s.t.  is Maxwell-dependent. Otherwise, if there is an edge  s.t. (1) there are two components  and  containing  and (2)  and  are both Maxwell-dependent, then from Lemma \ref{obs:maximal}(\ref{obs:maximal2}), we can see that   , a contradiction.

Then we can divide edges of  into , , , and . Following the proof of Theorem \ref{thm:complete2thin}, we can construct a subgraph  with   IE. Using proper maximality and completeness of the cover, we can show that  is Maxwell-independent.

Moreover, we can obtain that for every , every maximal Maxwell-independent subgraph  of  contains . 
Otherwise, we can consider the component  whose  does not  and from Lemma \ref{obs:maximal}(\ref{obs:maximal2}), we know   , a contradiction.

The rest of the proof is the same as the proof of Theorem \ref{thm:complete2thin}.



\end{comment}
\end{description}


Now we can consider the case with edge components in the cover and notice that each edge component contributes  to the left hand side but contributes at most  to the right hand side. Thus the inequality still holds.
\end{proof}



\noindent
{\bf Remark:}
(I) In fact, in Theorems \ref{thm:complete2thin} and \ref{thm:properComplete2thin}, when  is not Maxwell-rigid or  has at least  non-trivial components
in the strong 2-thin cover , it turns out that we do not {\em need}
Theorem \ref{thm:main} to show that the IE count
of the cover  is an upper bound on . This is because we can show that  constructed in Theorem
\ref{thm:complete2thin} is in fact a {\em maximum-size} Maxwell-independent subgraph
of . Otherwise we can find a maximal Maxwell-independent subgraph
 such that . Then there must be some  such
that . We know  is Maxwell-independent in
every Maxwell-independent set of  and since  is
Maxwell-independent, hence  is also
Maxwell-independent with size greater than , which is
. That is a contradiction to the fact that  is a maximum sized
Maxwell-independent subgraph of . (II) We can use the maximum sized Maxwell-independent subgraph  constructed in Theorems \ref{thm:complete2thin} and \ref{thm:properComplete2thin} to test Maxwell-rigidity.




\section{Open Problems}
\label{conclusion}

\subsection{Extending Rank bound to Higher Dimensions}\label{sec:higher}

The definition of maximal Maxwell-independent set extends to all dimensions, leading to the following conjecture.
\begin{conjecture}\label{conj:kdim}
For any dimension , the size of any maximal Maxwell-independent set gives an upper bound on the rank of the generic rigidity matroid of a graph .
\end{conjecture}


\medskip\noindent
Moreover, the definition of 2-thin component graphs can also be extended to  dimensions.

\begin{dfn}\label{dfn:dDimcomponentGraph}
Given , let    be a -thin cover of , i.e.,    for all . The {\em -thin component graph}  of  contains a {\em component node} for each subgraph induced by  in  and whenever  and  share
a complete graph  in , their corresponding component nodes in  are connected via an {\em edge
node}. The {\em degree} of a component node is defined to be the number of its adjacent edge nodes.
\end{dfn}

\medskip\noindent
To show Conjecture \ref{conj:kdim}, Proposition \ref{obs:vmmr} will have to be shown for -thin covers and it is sufficient to show that the  -thin component graphs of Maxwell-independent
sets are generalized partial -trees. However, we conjecture one possible generalization of the
strongest bound that we are able to show in the proof of Lemma \ref{lem:comb}(\ref{thm:degree}).

\begin{conjecture}\label{conj:kdimTree}
For a Maxwell-independent graph with a -thin cover  in  dimensions
the average degree of the component nodes of any subgraph of the -thin component graph is strictly
smaller than .
\end{conjecture}

For  this bound says that for Maxwell-independent sets, the average degree of the component nodes in the component graph is at most . For , however, we do not know of an example where all nodes have degree . In fact, we do not even know of an example with average degree . We state this as a conjecture for generalized body-hinge frameworks.
 
 \begin{conjecture}\label{conj:bodyhinge}
 In a -dimensional independent generalized body-hinge framework (where several bodies can meet at a hinge and several hinges can share a vertex), the average number of hinges per body is less than .
 \end{conjecture}
 
 
\medskip\noindent
 Lemma \ref{lem:comb}(\ref{thm:degree}) shows that there is no subgraph of the 2-thin component graph where each
 component node has at least  shared edges.
 A natural question is whether the counts for the so-called ``identified'' body-hinge
 frameworks can be used \cite{tay:rigidity1984, WhWh87,KatohTanigawa2009, Tanigawa:2012}, treating the component nodes as bodies
 and the shared edges as hinges. However, while identified body-hinge
 frameworks account for {\em several} component nodes sharing an edge (as we have here), generalized body-hinge structures may additionally have shared edges that have common vertices, hence the
 generic, identified body-hinge counts may not apply.


\subsection{Stronger Versions of Independence}\label{sec:stronger}
Even for Maxwell-independent graphs, the rank bounds of our Theorem \ref{thm:main}
can be arbitrarily bad. Even a simple
example of 2 bananas without the hinge edge has a single maximal
Maxwell-independent set of size 18 (which is the bound given by all of our
theorems), but its rank is only 17. Another example is the so-called ``-banana'': it is formed by joining  's on an edge and then removing that shared edge. In the -banana, the whole graph is Maxwell-independent, so itself is the unique maximal Maxwell-independent set. This maximal Maxwell-independent set exceeds the rank of the -dimensional generic rigidity matroid of -banana by .

\medskip\noindent Theorem \ref{thm:propermaximal} give alternative upper bounds for Maxwell-independent graphs. (In fact, Theorem \ref{thm:propermaximal} leads to a recursive method of obtaining a rank bound by recursively decomposing the graph into proper vertex-maximal
components. As one consequence, it gives an alternative, much simpler proof of
correctness for an existing algorithm called the Frontier Vertex algorithm (first version) that is based on
this decomposition idea as well as other ideas in this
chapter such as the component graph \cite{bib:survey}.)

\medskip\noindent
A natural open problem is to improve the bound in Theorem \ref{thm:main} directly by considering other
notions of independence that are stronger than Maxwell-independence. (Algorithms in \cite{sitharam:zhou:tractableADG:2004, bib:survey}
suggest and use stronger notions than Maxwell-independence,
but the algorithms usually
use some version of an inclusion-exclusion formula. They do not
provide explicit maximal sets of edges satisfying the stronger notions of
Maxwell-independence.
Neither do they prove that all such sets provide good bounds.)


\subsection{Bounds for Maxwell-Dependent Graphs Using 2-Thin Covers}\label{sec:Maxwelldependent}

\medskip\noindent While Theorem \ref{thm:propermaximal} gives a strong rank bound for Maxwell-independent graphs, Theorem \ref{thm:complete2thin} and Theorem \ref{thm:properComplete2thin} give much weaker bounds for Maxwell-dependent graphs because a collection of (proper) vertex-maximal, Maxwell-rigid subgraphs may be far from being a 2-thin cover. For example, in Figure \ref{fig:n2thin} we have  's and the neighboring 's share an edge with each other. There are two vertex-maximal, Maxwell-rigid subgraphs, each of which consists of  's with a shared edge.

\begin{center}
\begin{figure}[!h]
\begin{center}
\scalebox{0.8}[0.8]{\includegraphics{n2thin-eps-converted-to.pdf}}
\end{center}
\caption{A cover of vertex-maximal components that is not 2-thin. The circles are 's and the two larger ellipses are vertex-maximal, Maxwell-rigid subgraphs that form the cover.}
\label{fig:n2thin}
\end{figure}
\end{center}

\medskip\noindent
While many other 2-thin covers exist, the completeness as well as (proper) vertex-maximality
are important ingredients in the proofs of these theorems.
One possibility is to use 2-thin covers that are a subcollection of
(proper) vertex-maximal, Maxwell-rigid subgraphs. Another is to use collections of not necessarily vertex-maximal, but Maxwell-rigid subgraphs in which no proper subcollection of  or more subgraphs has a Maxwell-rigid union.


\medskip\noindent
Another notion that can be used involves the following definition of {\em strong Maxwell-rigidity}:
\begin{dfn}
A graph  is \emph{strong Maxwell-rigid} if for all maximal Maxwell-independent edge sets , we have .
\end{dfn}

It is tempting to use the approach in Theorem \ref{thm:complete2thin} to show that the IE count for a cover by vertex-maximal, strong Maxwell-rigid subgraphs is a new upper bound on the rank. We conjecture the -thinness of the cover, which is a crucial property explored in proving Theorem \ref{thm:complete2thin}.
\begin{conjecture}\label{conj:strong2co}
Any cover of a graph by a collection of vertex-maximal, strong Maxwell-rigid subgraphs is a 2-thin cover.
\end{conjecture}



\medskip\noindent
However, the idea in the proof of Theorem \ref{thm:complete2thin} will not work because the set , constructed in the proof of Theorem \ref{thm:complete2thin} that is of size equal to the IE count, can now be of smaller size than {\em any} maximal Maxwell-independent set of  as in the example of Figure \ref{smr_counter}.

\medskip\noindent
{\bf Example(Figure \ref{smr_counter}):} there are
five rings of 's, where each ring consists of  's. In the graph,
every  is a vertex-maximal strong Maxwell-rigid subgraph, and the
IE
count for the cover  is . Here the  is the number of 's and  is the total number of shared edges. But if we take 
edges in every  except  such that the missing edges are not shared, then we obtain a set  that is Maxwell-dependent. From  we drop one edge  of  and add one missing edge  to the  that shares  with . Then we get a set  that is a {\it minimum-size} maximal Maxwell-independent set of . The size of  is , where  is the number of edges in each ring, not counting the edges in  that are unshared in that ring. 

\medskip\noindent 
Hence in the Figure \ref{smr_counter} example, the IE count is less than the size of any maximal Maxwell-independent set, so the latter cannot be used as a bridging inequality
as in Theorem \ref{thm:complete2thin}.
However, the IE count does seem to give a direct upper bound on the rank (it is equal to the rank) hence a different proof idea might yield the required bound on rank.
\begin{center}
\begin{figure}[!h]
\begin{center}
\includegraphics[width=.5\textwidth]{smr_counter.png}
\end{center}
\caption[A counterexample to show that IE count of cover  by vertex-maximal, {\em strong} Maxwell-rigid subgraphs turns out to be smaller than the size of any maximal Maxwell-independent set.]{A counterexample to show that IE count of cover  by vertex-maximal, {\em strong} Maxwell-rigid subgraphs turns out to be smaller than the size of any maximal Maxwell-independent set.
Start with a , denoted . Each of  pairs of edges of  is extended into a ring of  's, where each ring is formed by closing a chain of 's where the neighboring 's share an edge (bold) with each other. In each
of the  rings, every  shares an edge with each of its two neighboring 's and these two edges are non-adjacent. Note that in the figure, only one of the five rings is shown.}
\label{smr_counter}
\end{figure}
\end{center}

\subsection{Algorithms for Various Maximal Maxwell-Independent Sets}
So far the emphasis has been to find good upper bounds on rank and Theorem \ref{thm:main} shows that the {\it minimum-size} maximal Maxwell-independent set of a graph  is at least . A natural open problem is to give an algorithm that constructs a minimum-size,
maximal Maxwell-independent set of an arbitrary graph.

\medskip\noindent
Note that Maxwell-rigidity requires the {\it maximum} Maxwell-independent set to be of size . Although the maximum Maxwell-independent set is trivially as big as the rank (and is not directly relevant to finding good bounds on rank), covers by Maxwell-rigid components have played a role in some of the theorems above (Theorems \ref{thm:propermaximal}, \ref{thm:complete2thin}, \ref{thm:properComplete2thin}) that give useful bounds on rank. Recall that Hendrickson \cite{Hendrickson92conditionsfor} gives an algorithm to test -dimensional Maxwell-rigidity by finding a maximal Maxwell-independent set that is automatically maximum for . While an extension of Hendrickson \cite{Hendrickson92conditionsfor} to  dimensions given in \cite{andrewThesis} finds {\em some} maximal Maxwell-independent set, it is not guaranteed to be maximum (or minimum). Thus another question of interest is whether maximum Maxwell-independent sets can be characterized in some natural way.



\subsection*{Acknowledgement}
We thank Bill Jackson and an anonymous reviewer for a careful reading and many constructive suggestions to improve the presentation.



\bibliographystyle{plain}


\bibliography{biblio}
\end{document}
