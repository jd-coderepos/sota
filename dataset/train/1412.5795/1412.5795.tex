\documentclass[11pt]{llncs}


\usepackage{latexsym}
\usepackage[pdf]{pstricks}
\usepackage{pstricks-add}
\usepackage{auto-pst-pdf}
\usepackage{pst-node}
\usepackage{pst-tree}
\usepackage{amssymb}
\usepackage{amsmath,bm}
\usepackage{stmaryrd}
\usepackage{xspace}
\usepackage{fullpage}

\newcommand{\e}[1]{{\bf #1}}
\newcommand{\logic}[1]{\textsc{#1}\xspace}
\newcommand{\FOL}{\logic{Fo}}
\newcommand{\domain}{\Delta^{\inte}}
\newcommand{\Var}[1]{\textit{Var}(#1)}
\newcommand{\g}[1]{\logic{#1}}
\newcommand{\inte}{\f{I}}
\newcommand{\tup}[1]{(#1)}
\newcommand{\card}[1]{\#(#1)}
\newcommand{\set}[1]{\{#1\}}
\newcommand{\f}[1]{\mathcal{#1}}
\newcommand{\dllite}{\g{DL-Lite}\xspace}
\newcommand{\LogSpace}{\textsc{LogSpace}\xspace}
\newcommand{\PTime}{\textsc{PTime}\xspace}
\newcommand{\NP}{\textsc{NP}\xspace}
\newcommand{\ex}{\hspace*{\fill}}
\newcommand{\dlliterc}{\g{DL-Lite}_{R,\sqcap}\xspace}
\renewcommand{\vec}[1]{\bar{#1}} \pagestyle{empty}

\title{The Expressive Power of }
\author{
Camilo Thorne
}
\institute{
\begin{tabular}{c}
IBM CAS Trento - Trento RISE\\
Piazza Manci 17\\
38123, Povo di Trento (Italy)\\
{\tt c.thorne.email@trentorise.eu}
\end{tabular}
}


\begin{document}


\maketitle



\begin{abstract}
Description logics are knowledge representation formalisms
that provide the formal underpinning of the semantic 
web and in particular of the OWL web ontology language.
In this paper we investigate the
expressive power of , and some of
its computational properties. We rely on simulations
to characterize the absolute expressive power of 
as a {\em concept language}, and to show that disjunction is not expressible.
We also show that no simulation-based closure property exists
for  assertions. Finally, we show
that query answering of unions of conjunctive queries is
\NP-complete.
\end{abstract}

\section{Introduction}\label{intro}


Description logics (DLs) are knowledge representation formalisms
that provide the formal underpinning of the semantic 
web and in particular of the OWL web ontology language\footnote{http://www.w3.org/TR/owl-features/}.
In this paper we are interested in investigating the
expressive power of the DL known as \g{DL-Lite}
\cite{Calvanese2006B}. The \g{DL-Lite} family of logics, of which
 makes part,
has been proposed by Calvanese et al. as a foundation of
ontology-based data access systems. They are intended
\cite{Calvanese2005A,Calvanese2007D}
as the least expressive DLs capable of capturing the main
features of conceptual modelling languages such as UML\footnote{http://www.omg.org/uml/}.
By the expressive power of a DL we understand
\textit{(i)} the
computational complexity of its reasoning problems
and \textit{(ii)} its model-theoretic properties.
As most DLs,  
is contained in , the 2-variable
fragment of \FOL and is therefore decidable
\cite{Borgida1996,Hudstadt2004,DLHandbook}. However,
its expressive power is still not known completely.

DLs model
domains in terms
of concepts (representing classes of objects), 
and binary relations known as roles (representing relations and
attributes of objects) \cite{DLHandbook}, all of which
are structured
into hierarchies by concept and role inclusion assertions.
Extensional information (the data), by contrast,
is conveyed by membership assertions. 
This information can be accessed by posing suitable \FOL
formulas, viz., unions of conjunctive queries.
This crucial reasoning problem is known as the knowledge
base query answering problem.

The main contributions of this paper consist, on the one hand,
in determining the (so-called) combined complexity of 's
query answering problem and, on the other hand,
to define what
we call  simulations.
This relation stems from the notion of bisimulations
(see e.g. \cite{vanBenthem2006}) for modal logics, known
to hold for the DL  \cite{DLHandbook}, that
has been proposed \cite{deRijke1997} as a means
of characterizing the (absolute)
expressivity of arbitrary DLs as concept languages.

The structure of this paper is as follows.
Section~\ref{two} recalls \textit{(i)} 's syntax
and semantics and \textit{(ii)} those of unions of conjunctive queries.
In section~\ref{two} we characterize the 
combined complexity of answering unions of conjunctive queries
over  knowledge bases.
In section~\ref{three} we introduce the notion
of  simulations
and show that a \FOL formula is equivalent to a 
concept when and only when it is closed under  simulations.
In section~\ref{four} we show that no such closure property
exists for assertions.
Finally, in section~\ref{five} we sum up our conclusions.

\section{Preliminaries}\label{one}


The syntax of
 is defined by the grammar:
\begin{itemize}
\item ,
\item  (left concepts),
\item  (right concepts),
\end{itemize}
where  stands for an atomic concept symbol (a unary predicate),
 for an atomic role symbol (a binary predicate) and  for its inverse.

Concepts combine into {\em concept inclusion assertions} of the form
, where  is a left concept,  is a right concept
and  is the {\em subsumption} relation. Roles into
{\em role inclusion assertions} of the form .
A {\em teminology}  (TBox) is a set of such assertions.
A {\em membership assertion} is an assertion of the form 
 or , where  are object (or individual) constants.
We denote  any set of membership assertions (ABox).
The integer
 denotes the number of (distinct)
tuples occuring among the atoms in .
The integer  the number of axioms in the terminology.
A {\em knowledge base} is a pair .

Let  denote a countable infinite set of constants.
The semantics of 
is based on \FOL\xspace {\em interpretations} , where
 is a non-empty {\em domain}.
Interpretations map each constant  to itself,
each atomic concept  to 
and each atomic role  to 
such that the following conditions hold:
\begin{itemize}
\item ,
\item , 
\item ,
\item ,
\item , and
\item .
\end{itemize}

We say that  {\em models} an assertion  (resp. ),
and write  (resp. ),
whenever  (resp. )
and a TBox
, and write , whenever it is a model of all of its
assertions. We say that it {\em models} a membership assertion
 (resp. ), and write  (resp. ), 
whenever 
(resp. ) 
and an ABox , and write ,
when it {\em models} all of its membership assertions.
Finally, we say that it is a {\em model} of a KB ,
and write ,
if it is a model of both  and .

The semantics \FOL formulas
is defined, we recall, in the usual terms
of satisfaction w.r.t. interpretations .
Let  be a \FOL formula and let
 denote the set of its variables.
An {\em assignment} 
for  relative to  is a
function ,
that can be recursively extended in the standard
way to complex formulas (see, e.g., \cite{CoriLascar}).
It is said to {\em satisfy} an atom  w.r.t. 
iff . This definition
is recursively extended
to complex formulas \cite{CoriLascar}. If  satisfies  w.r.t. ,
we write .
An interpretation  is said to be a {\em model} of
, written , 
if there exists an assignment  s.t. .

A {\em union of conjunctive queries} (UCQ)
of arity 
is a (positive existential) \FOL formula of the form
 
where 
 is a sequence of 
{\em distinguished variables} and
the s, for , are conjunctions of atoms.
A UCQ is said to be {\em boolean} if 
is an empty sequence. The integer 
denotes the number of symbols of .


Let  be a KB and  a UCQ of arity . 
KB  is said to {\em entail}
, written 
,
iff for all interpretations , 
implies that .
The {\em certain answers} of a UCQ  over KB 
 are defined as the set
, where
 denotes the instantiation of  in 
by a sequence of constants . The associated decision
problem is known as the KB {\em query answering} problem (\g{QA}) 
and is
defined as follows:
\begin{itemize}
\item given , a UCQ  of arity  and a
KB ,
\item does ?
\end{itemize}

When  and  are fixed we speak about the 
{\em data complexity} of \g{QA}, when only  about its {\em KB complexity}, 
when  and  are fixed about its {\em query complexity}
and finally, when none is fixed, about its {\em combined complexity}.
It is known \cite{Calvanese2007C} that  is
in \LogSpace in data complexity, \PTime-complete in KB complexity and
\NP-complete in query complexity, but its combined complexity remains unknown.

\section{Combined Complexity of \g{QA}}\label{two}


A {\em perfect reformulation} is an algorithm
that takes as input a DL TBox  and a UCQ 
and rewrites  w.r.t.  into a UCQ
 s.t., for every DL ABox
 and every  it holds that:
 iff ,
where  denotes the interpretation built out of 
(i.e.,  seen as a \FOL interpretation).

\begin{proposition}
{\bf (Calvanese et al. 2006)}
A perfect reformulation exists for .
\end{proposition}

\begin{theorem}
QA for  is \NP-complete in combined complexity.
\end{theorem}

\proof 
(\e{Membership}) Let  be a
KB and let  be the grounding of a UCQ .
First, consider:

We know that
 can be "compiled" into  by a perfect reformulation, yielding a
UCQ
. Guess, therefore, a disjunct
, for some .
This can be done in time constant in  and .
Clearly,  iff
, for
some assignment . 
Guess now an assignment . 
This can be done in time constant in, ultimately, .
Finally, check in
time polynomial on  and  whether
.

(\e{Hardness}) By reduction from the graph homomorphism problem, where, given
two graphs  and 
we ask whether there exists an homomorphism  from  to . 
A graph homomorphism, we recall, is a function  s.t.
for all , .
This problem is known to the \NP-complete.
We will consider  KBs with empty TBoxes.
Polynomially encode  and  as follows:
\begin{itemize}
\item for each , add the fact 
to the ABox ,
\item for each , add the ground atom 
to the boolean UCQ , which is the conjunction of such atoms.
\end{itemize}

We now claim that
there exists an homomorphism  from graph  to graph 
iff .

Since there is a perfect reformulation for , then
 iff .
Now, clearly, .
Thus, the interpretation function  can be seen as an homomorphism mapping
 to . Finally, given that  encodes , the claim
follows. \qed

\section{ Simulations}\label{three}


Given two interpretations  and , a 
{\em  left}  or {\em right simulation}  
is a relation  s.t., for every , every
\footnote{Observe that the clause for  follows implicitly from the first two.}:
\begin{itemize}
\item if  and , then .
\item if  and forall  there is some  
s.t. , then there exists an  s.t.
.
\item if  and , then .
\item if  and forall  there exists no  s.t. 
, then there is no  s.t.
.
\item if  and forall  there exists an  s.t. 
, then there is an  s.t.
 and .
\end{itemize}

A {\em  simulation}  is either a left, a
right or a combination of both simulations (i.e., their union). 
If a  simulation  exists
among two interpretations  and  we say that
they are {\em DL-similar} and write .

We say that a \FOL formula  is {\em closed under 
simulations} iff for every two interpretations  and ,
if  and , then .

We say that a \FOL formula  {\em entails} 
a  concept ,
written ,
iff for all ,  implies
that , and conversely, that
 {\em entails} , written , whenever, for all
,  implies .
If both entailments hold, we say that they are {\em equivalent}.

\begin{lemma}
If A \FOL formula  is closed
under \g{DL-Lite} simulations, then it is 
equivalent to a \g{DL-Lite}
right hand or left hand side concept.
\end{lemma}

\proof Let  be a \g{FOL} formula closed under
 simulations. Let  denote the set of consequences in 
 of 
a \FOL formula , i.e., . 
By compactness for DLs \cite{DLHandbook} the set of concepts
 has a model iff every finite  
has a model, whence
the concept  should have a model too.
We claim that  is equivalent to . Clearly,
. We claim now that

Assume that , 
for an arbitrary intrepretation .
Then, there exists a
 s.t. .
Put now . Then, for every . 
Hence for every 
there exists an interpretation  
s.t.  and .
The idea now is to build an interpretation  from the
s:
\begin{itemize}
\item ,
\item  extends each , for .
\end{itemize}
Define now a \g{DL-Lite} simulation  by putting:
\begin{center}
\begin{tabular}{c}
 iff for every concept ,  implies .
\end{tabular}
\end{center}
We now claim that  is a  simulation
between  and  and a fortiori that
.
We prove this by induction on :
\begin{itemize}
\item Basis:
\begin{itemize}
\item The property trivially holds for basic concepts. 
\item . Let ,
. By definition of , , that is,
.
\item . Let 
and  such that there is some 
such that . Now, , so
 and hence there is some 
such that .
\item . This is proven by combining the two
previous cases.
\end{itemize}
\item Inductive step:
\begin{itemize}
\item . Let  s.t.
exists  and
. ,
therefore,  by definition
and so there is an  such that 
and . Suppose that
. By induction hypothesis, .
Thus, by definition of ,
.
\item  (trivial).
\end{itemize}
\end{itemize}

Therefore,  
and since by assumption  is closed under
 simulations, .
This means that claim (\ref{eq:e}) holds. \qed


\begin{lemma}
If a \FOL formula  is equivalent to a \g{DL-Lite}
right hand or left hand side concept, then it is closed
under  simulations.
\end{lemma}

\proof Let  be s.t. . Let  be an interpretation
DL-similar to . Let
 and assume that
. 
We prove now, by induction on , that :
\begin{itemize}
\item Basis:
\begin{itemize}
\item . Let .
Then, , whence (by definition)
.
\item  (analogous argument).
\item . Let
. Then there exists
 s.t. , whence, by definition of
\g{DL-Lite} simulations , 
there is an  s.t. , 
that is, s.t. .
\item  (analogous argument).
\end{itemize}
\item Inductive step:
\begin{itemize}
\item . 
Suppose that . Therefore there
is some  s.t.  and .
By induction hypothesis this implies that , whence
 as well.
\item . By induction hypothesis
the property holds for  and . Now:

\end{itemize}
Therefore, since  is equivalent to ,
, as desired. \qed
\end{itemize}

\begin{theorem}
A \FOL formula  is equivalent to a 
right hand or left hand side concept iff it is closed
under \g{DL-Lite} simulations.
\end{theorem}

\begin{example}
The \FOL formula  is not equivalent to any
 concept, because
it is not closed under  simulations.

\begin{center}
\begin{pspicture}(10,4.5)
\psline{->}(2.8,2.5)(0.5,3.3)
\psline{->}(2.8,2.5)(0.5,1.7)
\psellipse(0.5,2.5)(0.4,1.3)
\put(0,4.1){}
\put(9.5,4.1){}
\put(5,2.5){}
\put(1.3,2.5){}
\put(7.8,2.5){}
\put(0.8,3.6){}
\cput[doubleline=false](3,2.5){}
\put(2.8,1.8){}
\put(0.3,3.35){}
\put(0.3,1.5){}
\put(6.7,2.6){}
\psline[linearc=.25]{<-}(6.8,2.5)(8,3)(9.5,2.5)
\put(9.5,2.6){}
\psframe(0,1)(3.5,4)
\psframe(6.5,1)(10,4)
\psline[linestyle=dashed,linearc=.5](0.8,3.5)(5,3.5)(6.7,2.5)
\psline[linestyle=dashed,linearc=.5](3.4,2.5)(5,1.5)(9.5,2.5)
\end{pspicture}
\end{center}
As the reader can see,  is a  simulation
there \textit{(i)} , \textit{(ii)} 
and \textit{(iii)} .
Now, clearly,
, but
, since
. \ex
\end{example}

\section{Some Negative Results}\label{four}


\begin{proposition}
Disjunction is not expressible in .
\end{proposition}

\proof  is contained in \g{HORN} 
(the set of \FOL horn clauses)\cite{Calvanese2007C,Calvanese2007E},
which cannot express disjunctions of the form
. 
Otherwise, let  
and 
be two Herbrand models of . Clearly,  and 
are minimal (w.r.t. set inclusion) models of  s.t. .
But this is impossible, since \g{HORN}
verifies the least (w.r.t. set inclusion) 
Herbrand model property \cite{CoriLascar}. \qed

\begin{theorem}
There is no relation  over interpretations 
such that,
for every \FOL sentence ,  is equivalent to a
 assertion iff it is closed under 
the relation .
\end{theorem}

\proof Recall that a \FOL sentence is a 
\FOL formula with no free variables.
Suppose the contrary and consider the sentence
. Let  and  be two 
structures s.t.  and suppose that
. Then, obviously,  too.
But then:
\begin{center}
\begin{tabular}{l}
 implies , and\\
 implies .
\end{tabular}
\end{center}
That is,   is closed under  and is a fortiori
equivalent to some
 assertion.
But this is impossible, because disjunction is not expressible
in . \qed

\section{Conclusions}\label{five}


In this paper we have shown four things:
\textit{(i)} Answering UCQs over  KBs is \NP-complete in
combined complexity.
\textit{(ii)} A simulation relation among interpretations, viz., a  simulation, 
can be used to characterize the expressive power of  as a
concept language.
\textit{(iii)} \FOL formulas that are closed under  simulations
are equivalent to a (left or right)  concept.
\textit{(iv)} This closure property holds only w.r.t. concepts, but not w.r.t.
assertions. 
Simulations, in particular, can be generalized, with minor adjustments, to
the whole \g{DL-Lite} family of DLs, although, since all of them are in
\g{HORN}, no such closure property exists for their assertions.

\bibliographystyle{plain}
\bibliography{/home/camilo/Documents/bibs/my_bib}


\end{document}
