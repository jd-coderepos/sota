\documentclass[11pt]{article}
\pdfoutput=1
\usepackage{verbatim,url,enumerate,color,paralist}
\usepackage{amsmath,amsfonts}
\usepackage{epsfig,amssymb,amstext,xspace,theorem}
\usepackage{algorithm}
\usepackage{algorithmicx,algpseudocode}
\usepackage{pifont}

\usepackage{caption}
\usepackage{slashbox}
\usepackage{graphicx}
\usepackage{tikz}
\newcommand*\circled[1]{\tikz[baseline=(char.base)]{
            \node[shape=circle,draw,inner sep=2pt] (char) {#1};}}

\newcommand{\qed}{\hspace*{\fill}}

\newcommand{\bs}{\backslash}
\newcommand{\IGNORE}[1]{}

\newtheorem{theorem}{Theorem}[section]
\newtheorem{lemma}[theorem]{Lemma}
\newtheorem{corollary}[theorem]{Corollary}
\newtheorem{definition}[theorem]{Definition}
\newtheorem{observation}[theorem]{Observation}
\newtheorem{proposition}[theorem]{Proposition}
\newtheorem{claim}[theorem]{Claim}
\newtheorem{fact}[theorem]{Fact}
\newtheorem{remark}[theorem]{Remark}



\newenvironment{proof}[1][Proof. ]{\noindent {\bf #1 }}{\qed}
\begin{document}

\title {On the Metric - Path Traveling Salesman Problem}
\author{
Zhihan Gao\thanks{
        (z9gao@uwaterloo.ca)
	Dept.\ of Comb.\ \& Opt.,
        University of Waterloo, Waterloo, Ontario N2L3G1, Canada.
	}
}


\date{}
\maketitle



\begin{abstract}
We study the metric - path Traveling Salesman Problem (TSP).
[An, Kleinberg, and Shmoys, STOC 2012] improved on
the long standing -approximation factor and
presented an algorithm that achieves an approximation factor of
.
Later [Seb\H{o}, IPCO 2013] further improved the approximation factor
to .
We present a simple, self-contained analysis that
unifies both results;
our main contribution is a \emph{unified correction vector}.
Additionally, we compare two different linear programming (LP) relaxations of
the - path TSP, namely,
the path version of the Held-Karp LP relaxation for TSP
and a weaker LP relaxation, and we show that both LPs have
the same (fractional) optimal value. Also, we show that the minimum cost of integral solutions of the
two LPs are within a factor of  of each other.
Furthermore, we prove that a half-integral solution of the stronger LP-relaxation of
cost  can be rounded to an integral solution of cost at most .
Finally, we give an instance that presents obstructions to
two natural methods that aim for an approximation factor of .
\end{abstract}




\section{Introduction}

The metric Traveling Salesman Problem (TSP) is a celebrated
problem in Combinatorial Optimization, see \cite[Chapter 58]{Sch03}, \cite{BB08}. One important variant of TSP
is the (metric) - path TSP. Let  be a complete graph  with
nonnegative metric edge costs , i.e.,  satisfies the triangle inequality.
Given two fixed vertices  in , the
\emph{- path TSP} is to find a minimum-cost Hamiltonian path
from  to  in .

Hoogeveen \cite{hoogeveen91} gave an - path TSP variant of
Christofides' approximation algorithm for the TSP \cite{christofides76}, and obtained an approximation factor
of . There was no improvement in this approximation factor
for over two decades until An, Kleinberg, and Shmoys \cite{AKS12}
improved the approximation factor to .
One of the key new contributions of \cite{AKS12} is to
design and analyse a randomized version of Christofides' algorithm.
The analysis introduced the notion of a correction vector for
the - path TSP.
Most recently, Seb\H{o} \cite{sebo13} further improved the
analysis and obtained a better approximation factor of .
\cite{sebo13} introduced a correction vector different from that
of \cite{AKS12}, and this is one reason why the analysis in \cite{sebo13}
gives a better approximation factor. Informally speaking,
a better correction vector provides a better approximation factor.
In this paper, we give a unified presentation of the results from both
\cite{AKS12} and \cite{sebo13} by introducing a new correction vector
that we call the \emph{unified correction vector}.
Our correction vector is simple and it leads to short derivations of
the approximation factors of both \cite{AKS12} and \cite{sebo13}. The difference between our correction vector and
the previous ones is that it assigns the value one to the minimum-cost
edge in each so-called -narrow cut, whereas the correction vectors used in \cite{AKS12} and \cite{sebo13} are fractional
on each -narrow cut. We mention that Vygen's \cite{vygen13} comprehensive recent survey discusses
the common points of the analysis of \cite{AKS12} and \cite{sebo13},
and the survey sketches short proofs of both approximation factors;
however, \cite{vygen13} uses the same correction vectors as \cite{AKS12} and \cite{sebo13}.



An et al. \cite{AKS12} and Seb\H{o} \cite{sebo13} use two
different LP relaxations of the - path TSP in their algorithms.
\cite{AKS12} uses the path version of the Held-Karp LP relaxation for TSP,
whereas \cite{sebo13} uses a weaker LP relaxation. This motivates a comparison of these two LP relaxations. We mention that Seb\H{o} proves an approximation factor of  for a more
general problem, namely, the \emph{connected T-join problem}, and the LP in
his paper is a relaxation of this problem.
We show that both LPs for the - path TSP have the same (fractional) optimal value.
Also, we show that the minimum cost of integral solutions of the
two LPs are within a factor of  of each other; moreover,
we present an example to show that the factor of  is tight.
We prove this result by showing that
a half-integral solution of the stronger LP-relaxation of cost 
can be rounded to an integral solution of cost at most .




For the - path TSP, it is known that the integrality ratio
of the path version of the Held-Karp LP relaxation has a lower bound
of . All of the algorithms mentioned above are LP-based.
This leads to the best known upper bound  on the integrality
ratio of the LP relaxation. A natural open question is to close this
gap by designing an LP-based -approximation algorithm
for the - path TSP. Given a connected graph  with unit
edge costs and two fixed vertices  and , the \emph{-
path graph-TSP} is to find a minimum-cost Hamiltonian path from 
to  in the metric completion of . For this critical special
case of the - path TSP, the integrality ratio of the corresponding
LP relaxation has been resolved already. The first
-approximation algorithm was given by Seb\H{o} and Vygen
\cite{SV12} using ear decompositions.  Gao \cite{Gao13} designed
another, conceptually simpler, LP-based
-approximation algorithm.
The analysis of the -approximation factor of \cite{Gao13}
uses the graphic property only for one point: to guarantee that the
cost of a special spanning tree constructed in the algorithm is at
most the optimum of the LP relaxation. A natural question is whether we can
extend this graphic LP-based approximation algorithm and analysis
to the general metric case. Unfortunately, we present an instance
that shows that that is not possible.  Moreover, our instance also illustrates that probabilistic methods
are relevant for the analysis of improved LP-based approximation
algorithms.  This instance may shed some
light on how to design a better approximation algorithm for the
- path TSP.

The paper is organized as follows. Section~\ref{sec:pre} has some
notation and basic results. Section~\ref{sec:ucv} presents our
unified correction vector. Section~\ref{sec:LPs} shows the relationship
of two different LP relaxations of the - path TSP.
Section~\ref{sec:BI} discusses an instance that points to some
of the obstructions for obtaining better approximation factors.



\section{Preliminaries}\label{sec:pre}

Let  be a complete graph.
Let  be two fixed vertices in .
We call a nonempty, proper subset of vertices  a \emph{cut};
thus, .
In particular, if ,
then we call  an \emph{- cut}.
For , let  denote the set of edges
that have one end in , thus,
.
If , then we use  instead of .
Let  denote the set of edges induced by , thus,
.
For any two sets  and ,
we use  to denote . For a vector , we define 
for any subset  of . When there is no risk of confusion, we will use the same notation  for a subgraph  and its edge set .

For any probabilistic event ,
we use  to denote the probability of occurrence of .
For a random variable ,
the expectation of  is denoted by .



\subsection{Linear programs}

The path version of the Held-Karp relaxation for
the - path TSP is defined as follows:



The spanning tree polytope is shown as follows:



\begin{lemma}
Every solution  of (L.P.1) lies in the spanning tree polytope (L.P.2).
\end{lemma}
\begin{proof}
By the degree constraint for each vertex in (L.P.1), we have . Now consider the second set of constraint in (L.P.2). If  is even, by the degree and cut constraints in (L.P.1), . Otherwise, . Similarly, . This completes the proof.
\end{proof}



\subsection{-joins}

Let  be a nonempty subset of  with  even.
For , if the set of odd degree vertices of
the graph  is , then we call  a \emph{-join}.
For any , if  is odd (even),
then we call  a \emph{-odd cut} (\emph{-even cut}).
The following LP formulates the problem of
finding a -join of minimum cost:
\medskip

T
\begin{lemma}\cite{EJ01}\label{lem:Tjoin}
The optimal value of (L.P.3) is the same as the minimum cost of a -join.
\end{lemma}

Let  be a spanning tree. \emph{The set of wrong degree vertices} of  is defined as .
\begin{lemma}\cite{AKS12}\label{lem:simpleParity}
Let  be the set of wrong degree vertices of a spanning tree . Let  be an - cut. If  is -odd, then  is even.
\end{lemma}

The proof can be also found in \cite[Lemma 2.1]{CFG12}. But for the sake of completeness, we present a proof here.

\vspace{3mm}

\begin{proof}
Since , we have  has the same parity as . Without loss of the generality, we assume . By the definition of , we know that  is the set of vertices  in  such that  is odd. If  is odd, then . In this case, since  is -odd,  is odd.  Hence, we have an even number of vertices  in  such that  is odd, which implies that  is even. Otherwise,  is even. Then, . This implies that  is even. Similarly,  is even.
\end{proof}


\subsection{Polyhedra and convex decomposition}

Let

Let  be a feasible solution of . For a constraint  in , we say  is \emph{tight} at this constraint if . Let  be two distinct feasible solutions of . If there exists a  and 
such that , we say
 is \emph{in some convex decomposition} of  in .

From the geometry of polyhedra, we have the following characterization of the convex
decompositions.

\begin{lemma}\label{lem:decomposition}
The solution  is in some convex decomposition of  in  if and only if  is tight at the constraints of  where  is tight.
\end{lemma}

\subsection{Christofides' algorithm for - path TSP}
Hoogeveen \cite{hoogeveen91} gave a variant of Christofides' algorithm to achieve the first approximation
factor of  for the - path TSP.\\

\noindent \textbf{Christofides' algorithm for - path TSP}\\
\noindent Compute a minimum-cost spanning tree . Let  be the set of wrong degree vertices of .
Find a minimum-cost -join . Then, the union  of  and  (that keeps the duplicated edges) forms a
connected graph that has even degree at all nodes except  and . One can then take the Eulerian traversal that starts
at  and ends at , and shortcut it, to obtain
an - path visiting all vertices of no greater cost.
\begin{theorem}\cite{hoogeveen91}
Christofides' algorithm for - path TSP achieves an approximation factor of .
\end{theorem}

For the sake of completeness, we present a nice proof from Seb\H{o} and Vygen \cite{SV12}.

\vspace{3mm}

\begin{proof}
Let  be an optimal solution of - path TSP. Let  be as in the algorithm. Let  be the - path in 
 and  be the -join in . Since  is a spanning tree,
 we know . So, we only need to prove .
 This follows from the fact that  can be partitioned into three -joins:
 one is , one is ,
 and one is the union of  and . One can check that each of these edge sets is a -join
 by using the fact that  is the set of wrong degree vertices of .
  Then, . This completes the proof.
\end{proof}



\section{Unified correction vector}\label{sec:ucv}

An et al. \cite{AKS12} designed a randomized Christofides' algorithm for the - path~TSP, and
they proved an approximation factor of
 by analysing this algorithm.
Their algorithm and their analysis were based on the LP relaxation (L.P.1).
Seb\H{o} \cite{sebo13} presented a new analysis of this randomized algorithm
and improved the approximation factor to .
The algorithm and analysis of \cite{sebo13} were based on
a different LP relaxation, see (L.P.4) in Section \ref{sec:LPs}.
In Section \ref{sec:LPs}, we prove that
(L.P.1) and (L.P.4) have the same optimal value.
This result together with a few more observations implies that
(L.P.4) can be replaced by (L.P.1) in
the algorithm and analysis of \cite{sebo13}
to achieve the same approximation factor of .
In this section,
we prove the approximation factor of \cite{AKS12}; also,
we prove the -approximation factor of \cite{sebo13}
based on (L.P.1) rather than (L.P.4).





\bigskip

\noindent \textbf{Randomized Christofides' algorithm:} \\
Solve the LP relaxation (L.P.1) to get an optimal solution . Since  is in the spanning tree polytope,
there exists a convex decomposition of spanning trees
 such that
 where ,  and 
is the edge incidence vector of .
Such a decomposition can be found in polynomial time, see
Theorem 51.5 of \cite{Sch03}. We sample a spanning tree  from these spanning trees according to
the probability defined by the coefficient  of
each spanning tree in the convex combination.
Let  denote the set of the wrong degree vertices of .
Then, as in the Christofides' algorithm,
a minimum-cost -join  is added to fix
the wrong degree vertices of .

The expected cost of the random solution of the algorithm
is the sum of the expected cost of , which is the cost of ,
and the expected cost of the -join .
Any feasible solution of the -join polyhedron provides
a cost upper bound for the -join .
An et al. \cite{AKS12} introduced correction vectors to construct a special type of fractional -join. A \emph{correction vector} for a -narrow cut  is an edge vector  that satisfies , where the definition of -narrow cut will be given next. The correction vectors were further analyzed in \cite{sebo13}
to obtain a better approximation factor. In this section, we present a unified correction vector
to derive the results of both \cite{AKS12} and \cite{sebo13}.

The following key definition is introduced in \cite{AKS12}.
Let . If an - cut  satisfies
, we call it a \emph{-narrow cut}.
Let  be the set of all -narrow cuts
that contain . It turns out that -narrow cuts have
a nice structural property.

\begin{lemma}\cite{AKS12} \label{lem:nestedCuts}
Let ,  be two distinct cuts in .
Then either  or .
\end{lemma}

For the sake of completeness, we present a proof.
\vspace{3mm}

\begin{proof}
Suppose that the statement is false.
Then both  and  are nonempty.
Note that both  and  are
-even. Hence,

by the constraints in (L.P.1).
However, .
This is a contradiction.
\end{proof}

Thus, we can use  to denote all of
the -narrow cuts containing  such that
.
Note that .
Define  for 
where  and .
Each  is nonempty and .
We call  \emph{the partition derived by
the -narrow cuts }.

Let  denote the edge incidence vector of the edge~set of .
For any , we let  be
an edge in  of minimum cost.
Let  denote the edge incidence vector of ,
i.e., , and  if .
Our \emph{unified correction vector} is defined as  for each ,
i.e., the unified correction vector simply assigns
the value one to the minimum-cost edge in each -narrow cut. In contrast, the correction vectors used
in \cite{AKS12} and \cite{sebo13} are fractional but sum up to at least one for each -narrow cut.


Let  and  be real parameters between  and ,
whose specific values are given later.
Recall that  is the random spanning tree in the
randomized Christofides' algorithm. Our fractional feasible -join solution with unified correction
vectors, called \emph{unified fractional -join}, is as follows:


\medskip

\noindent \textbf{Unified fractional -join: }	

where  satisfy the following condition:


Let us derive the settings of  and  in (\ref{settings}).
The purpose of the unified fractional -join  is to provide an upper bound on the cost of the minimum-cost -join  in the
randomized Christofides' algorithm. By Lemma \ref{lem:Tjoin}, it suffices to make  feasible
for the -join polyhedron (L.P.3). This requires special settings of  and .

Consider the cut constraints in (L.P.3). Let  be a -odd cut. First we need to make sure that for any
, the coefficient  is nonnegative. Since  for any , it
suffices to set , i.e., .


Suppose that  is an - cut.
Note that  is -odd.
Hence, by Lemma \ref{lem:simpleParity},  is even.
If  is not a -narrow cut, then
. By the assumption that , we have  in this case.
If  is a cut in , then .

Now the only remaining case is that  is -even. Then  by (L.P.1).
Since  is a spanning tree, we have .
This implies . Hence, in this case,
it suffices to set .

Hence, we have the following result by the analysis above.

\begin{lemma}\label{lem:feasible}
The unified fractional -join  is
a feasible solution of the -join polyhedron (L.P.3).
\end{lemma}

Lemma \ref{lem:feasible} shows that the expected cost of the minimum-cost -join  computed by
the randomized Christofides' algorithm is
at most the expected cost of the unified fractional -join. Hence, the expected cost of the solution of the randomized
Christofides' algorithm is upper bounded by the optimal value of (L.P.1) plus the expected cost of the
unified fractional -join.  In Section \ref{sec:AKS'results} and Section \ref{sec:sebo's results}, we will present two different analyses of the
expected cost of the unified fractional -join to derive two different approximation factors from \cite{AKS12} and \cite{sebo13} for the randomized
Christofides' algorithm.


\begin{remark}
From the analysis above,  the cost analysis of the
unified fractional -join is critical for proving
an approximation factor for the randomized Christofides' algorithm.
If we can get a better upper bound on the cost of the unified fractional -join,
then the approximation factor can be further improved.
\end{remark}

The following lemma is used in the analysis of the expected cost of the unified fractional -join in
Section \ref{sec:AKS'results} and Section \ref{sec:sebo's results}.

\begin{lemma}\cite{AKS12}\cite{sebo13}\label{lem:probBound}
Let  be the random spanning tree and
 be the set of wrong degree vertices of  in the
randomized Christofides' algorithm.
Let , i.e.,  is a -narrow cut.
Then
\begin{itemize}
\item[(i)]
, and
\item[(ii)]
.
\end{itemize}
\end{lemma}

\vspace{2mm}

For the sake of completeness, we present a proof.
\vspace{3mm}

\begin{proof}
Since  is a spanning tree,  always holds.
So . Then

Note that  follows from the fact that  since  is a random tree in the convex decomposition
of spanning trees for  where the coefficients of the spanning trees define the probability distribution. Thus, we have
.
This proves the first inequality.

Now consider the second inequality.
By Lemma \ref{lem:simpleParity},  is even if  is -odd.
This means .
\end{proof}



\subsection{AKS' -approximation via
unified correction vector} \label{sec:AKS'results}

First, we present two lemmas needed for the cost analysis of the randomized Christofides' algorithm.
\begin{lemma}\label{base_inj}
Let  be a spanning tree with  vertices. Let  be a family of subsets
of the vertex set of  such that
 and . There exists a bijection from  to  such that
each cut  is mapped to an edge of  in .
\end{lemma}

\begin{proof}Without loss of generality, we can assume that the vertex set of  is 
and  for . We prove the result by induction on .
The statement is clearly true for .
Suppose .
Consider the vertex .

We first pick the edge  of  incident with 
in the unique path of  between  and . We map  to this edge . Let  be the graph obtained from  by
contracting  and  into a single vertex . Note that  is a connected graph with  edges. This implies
that  is a spanning tree with  vertices  where  for  and .
Note that  is a subset of  for . Hence, we can define
the rest of the bijection by applying the induction hypothesis
to the spanning tree  on these  vertices.











\end{proof}


\begin{lemma}\label{lem:ineq1}

\end{lemma}

\begin{proof}
Let  be a minimum-cost spanning tree on .
Consider the partition  derived by .
We contract every  into a single vertex.
Then the resulting graph obtained from  is connected.
Let  be a spanning tree of the contracted graph.
Applying Lemma \ref{base_inj} to ,
we construct an injective mapping  from 
to the edge set of  such that 
for each .
Note that .
Then 
since  is in the spanning tree polytope. The first inequality follows from the fact that
 is the minimum-cost edge in .
\end{proof}

\begin{theorem}\cite{AKS12}\label{thm:AKS}
The randomized Christofides' algorithm achieves an
approximation factor of .
\end{theorem}
\begin{proof}
Since  is a random spanning tree based on the convex decomposition of spanning trees for , we have . Hence, the expected cost of the solution of the randomized
Christofides' algorithm is upper bounded by the optimal value of (L.P.1) plus the expected cost of the
minimum-cost -join . By Lemma \ref{lem:Tjoin} and Lemma \ref{lem:feasible}, the expected cost of  is
at most the expected cost of the unified fractional -join.

The last equality follows from the fact that  by (\ref{settings}). The value of  that maximizes
the expression is . Hence, the upper bound on the expected cost of the unified fractional
-join is at most . Substitute ,  from (\ref{settings}) into the upper bound.
Minimizing with respect to  gives  with optimal settings:  ,
, .
Therefore, the optimal value of (L.P.1) plus this upper bound  on the expected cost of the unified fractional -join leads to
the approximation factor of
 that was first proved in \cite{AKS12}.
\end{proof}


In \cite{AKS12}, the correction vector is constructed by using flow
computations to map the optimal LP solution  to the -narrow cuts. In contrast, our unified correction vector simply assigns the
value one to the minimum-cost edge in each -narrow cut. We avoid the flow computation argument of \cite{AKS12} by using Lemma \ref{base_inj}.



\subsection{Seb\H{o}'s -approximation via
unified correction vector}\label{sec:sebo's results}

Let  be the - path in .
Seb\H{o} \cite{sebo13} points out the crucial fact that
 is a -join for the set of wrong degree vertices  of . Recall that  is the
minimum-cost -join in the randomized Christofides' algorithm. This implies that .
Note that .

It turns out that  also serves as
an upper bound in another cost inequality similar to (\ref{ineq1});
see the following lemma.

\begin{lemma}\label{lem:ineq2}

\end{lemma}
\begin{proof}
Let ; thus,  is a -narrow cut.
If ,
then let  denote the unique edge in .
Recall that a -narrow cut is an - cut, and therefore
 must be in  since  is the - path in .
Moreover, observe that  is one of the two connected components of
.
Hence, for distinct 
such that  and ,
the edges  and  must be distinct
(otherwise,  and  would have
the same connected components, contradicting the fact that 
are distinct sets containing ).
Then

By Lemma  \ref{lem:probBound},

\end{proof}

\begin{theorem}\cite{sebo13}
The randomized Christofides' algorithm achieves an
approximation factor of .
\end{theorem}
\begin{proof}
By an argument similar to the one in the proof of Theorem \ref{thm:AKS}, we are only concerned with the expected cost of the unified fractional -join, which bounds the expected cost of the minimum-cost -join  in the randomized Christofides' algorithm.

The last equality follows from the fact that  by (\ref{settings}). The value of  that maximizes the expression is . Hence, the upper bound on the expected cost of the unified fractional
-join is at most . Substitute ,  from (\ref{settings}) into (\ref{inq:rawuppbound}). Then the coefficients of the terms in (\ref{inq:rawuppbound}) only depend on . Denote the coefficient of the last term in (\ref{inq:rawuppbound}) by  where . Then the bound can be written as .
Note that . Assume . So 
and . Since , we have

 maximizes the expression when . So . Minimizing
the upper bound in (\ref{inq:maxmin}) with respect to  gives  with optimal settings:
 ; moreover, .
Therefore, the optimal value of (L.P.1) plus this upper bound  leads to
the approximation factor of
 that was first proved in \cite{sebo13}.
\end{proof}




\section{Linear programming relaxations of the - path TSP}\label{sec:LPs}

In this section, we investigate the relationship between two different LP
relaxations of the - path TSP. Let  be a connected
graph with nonnegative edge costs , and
let  and  be two fixed vertices.
For a partition  of
the vertex set ,
let  denote .
Let  be the metric completion of  with metric costs .
As mentioned in Section \ref{sec:ucv}, (L.P.1) is a linear
programming relaxation of the - path TSP on .  Let 
be the graph obtained from  by doubling every edge of . The
- path TSP on  is equivalent to
the problem of finding a minimum-cost trail in  from  to 
visiting every vertex at least once (multiple visits are allowed for the vertices but not the edges).
Thus, the problem is to find a minimum-cost connected
spanning subgraph of  with  as the odd-degree vertex set.
Hence, the following (L.P.4) is another LP relaxation of
the - path TSP.

\medskip

Note that (L.P.4) is defined on the original graph  but (L.P.1)
is defined on the metric completion  of .


In this section, we show that both LPs, (L.P.1) and (L.P.4), have
the same (fractional) optimal value, see Corollary~\ref{cor:equiv}.
But these two LPs can differ with respect to integral solutions.
Observe that the integral solutions of (L.P.1) are
exactly the - Hamiltonian paths of ;
this follows because an integral solution induces a graph
that is connected, has degree one at , and has
degree two at all other vertices.
The integral solutions of (L.P.4) need not correspond to
the - Eulerian paths of ; see the example shown
in Figure~\ref{tightExample}.

Let  denote the optimal value  of (L.P.), for .
Let  denote the minimum cost of
an integral solution that satisfies all constraints of (L.P.),
for .
We call  the \emph{optimal integral value} of (L.P.).
The following table summarizes the relationship between the two LPs;
the new results of this section appear in the last two columns.


\begin{center}
\setlength{\tabcolsep}{1pt}
\scalebox{0.75}{
\begin{tabular}{ | c | c | c | c | c |}
\hline
   LPs &  Graph & Costs & Optimum & Optimal Integral Value \\
\hline
  (L.P.1) & \small {: metric completion of  } & \small {: metric extension of  }&  &  \\ \hline
  (L.P.4) &  &  &   & \small {} \\
  \hline
\end{tabular}
}
\end{center}


To obtain these results, we need an edge-splitting lemma.
Let  be a multigraph, i.e., two adjacent vertices in  may be
connected by one or more edges.
Let . The \emph{splitting operation} on
 at the vertex  is defined as follows:

\begin{itemize}
\item Remove  and then add  if .
\end{itemize}

If , then we remove the loop formed by adding ; note that
this removal of the loop has no effect on the edge-connectivity of the graph. We use the following
result to prove Lemma \ref{lem:17trans}; see \cite[Theorem ]{Frank92}.

\begin{lemma}\cite{Lov74}\cite[Ex. 6.51]{Lov79}\label{lem:splitting}
Let  be a multigraph with even degree at each vertex.
Let  and let .
Let  be a positive integer. If


then the edges incident with  can be partitioned into
 disjoint edge pairs , 
such that the multigraph obtained by applying
the splitting operation to any one of these edge pairs (at the vertex )
still satisfies (\ref{cond}).
\end{lemma}

\begin{lemma}\label{lem:17trans}
Let  be a rational solution of (L.P.4) of cost .
Then there exists a solution  of (L.P.1) with cost at most .
Moreover, if  is an integral solution, then  is half-integral.
\end{lemma}
\begin{proof}
The first part of this statement follows from the parsimonious property shown in \cite{BT97}. However, to show the second part of the statement, we present a proof for the first part as well.

Define an edge vector  on  as follows:

Since  is the metric completion of , we know .
Then we construct  from  as follows:



By the constraints of (L.P.4) and
the fact that ,
we have  for each cut .
Let  be a positive integer such that  is integral.
Consider the multigraph  with 
number of edges between  and .
Then .

By using Lemma \ref{lem:splitting}, we apply splitting operations at every
vertex until the degree of every vertex is exactly . We claim
that this procedure can be applied such that the number of edges
between  and  is . To see this, consider a splitting
operation at  or , say ; note that splitting at other vertices
does not decrease the number of edges between  and . There are at
least  feasible splitting pairs available at  (since otherwise there is no
need to do a splitting operation at , i.e., ). This implies
that we can always choose a splitting pair such that at least 
edges between  and  are preserved.

Let  be the edge vector associated with the resulting graph after
splitting, i.e.,  equals the number of edges between
 and  in the resulting graph. Furthermore, let .
Then  for each cut ,  for
each vertex , and . Consider two different
vertices . We know  and
. This implies . In
particular, . Construct  from  as follows:



By the properties obtained for , we have  is a feasible
solution of (L.P.1). Note that the splitting operations never
increase the total cost since the edge costs are metric on .
Therefore, the cost of  is at most . In particular,
if  is integral, we can set  in the procedure.
In this case,  is half-integral.
\end{proof}

Conversely, any feasible solution of (L.P.1) can be transformed to
a feasible solution of :
the idea is to replace each edge  in 
by a shortest - path in .
Note that every solution of (L.P.1) is a feasible solution of
the spanning tree polytope. Hence,
it can be seen that the transformed solution is feasible for (L.P.4),
and, in particular, it satisfies the partition constraints in (L.P.4).
Hence,
 

By Lemma \ref{lem:17trans}, we have the following result.

\begin{corollary}\label{cor:equiv}
.
\end{corollary}

However, (L.P.1) and (L.P.4) may differ
in terms of the integral optimal value.
Consider the graph with unit edge costs in Figure~\ref{tightExample};
this is meant to be the original graph  in the instance
of the - path TSP.

\begin{figure}[h]
\begin{center}
\includegraphics[scale=0.5]{2paths.pdf}\\
  \caption{Tight Example}
  \label{tightExample}
\end{center}
\end{figure}

Note that (L.P.4) is defined on the original graph but (L.P.1) is
defined on the metric completion. Let  be the length of the
middle path in Figure \ref{tightExample}. It is not hard to see
that  but  when  is sufficiently large. (For (L.P.4), consider the integral solution with value  for
every edge of the original graph.) In this case,
.
Interestingly,  can be proved to be an upper bound for this
ratio. This example shows that the upper bound of  is tight. To prove this upper bound, we present an algorithm to round a half-integral solution of (L.P.1) to an integral one by increasing the cost by a factor of at most .

Apply the randomized Christofides' algorithm to a half-integral
solution  of (L.P.1). Let  be the random spanning tree obtained
from . Let  be a minimum-cost -join for the set of wrong degree
vertices  of .

\begin{lemma}\label{lem:fea-half}
 for any  -odd cut .
\end{lemma}
\begin{proof}
For any vertex ,  is integral by
the constraints of (L.P.1).
Since  is half-integral,  implies that  is integral.
Suppose  for some -odd cut . Then we have
. By the constraints of (L.P.1),  must be an
- cut. Note that  and  since  is a random spanning tree. This implies  always holds. However, since  is an - cut and
also a -odd cut, we have  is even by Lemma \ref{lem:simpleParity}. This is a contradiction.
\end{proof}

\begin{theorem}\label{halfAlgorithm}
If the input is a half-integral solution  of (L.P.1), then the
randomized Christofides' algorithm outputs a Hamiltonian -
path with cost at most .
\end{theorem}
\begin{proof}
By Lemma \ref{lem:fea-half},  is a feasible solution
of the -join polyhedron (L.P.3).
This means .
Therefore .
\end{proof}

\medskip

Now we are ready to prove the ratio for the optimal integral values
of the two LPs.

\begin{theorem}\label{equivLP41}
.
Moreover, the bounds are tight.
\end{theorem}
\begin{proof}
The lower bound is due to (\ref{LP4lesLP1}). Now consider the upper
bound.  Let  be an optimal integral solution of (L.P.4). By
Lemma \ref{lem:17trans}, there exists a half-integral solution 
of (L.P.1) such that . By Theorem \ref{halfAlgorithm},
we can get an - Hamiltonian path with cost at most .
This means .

The tight example for the upper bound is shown in Figure
\ref{tightExample}. For the tightness of the lower bound, consider
the graph  consisting of one path connecting  and  where
every edge has unit cost.
\end{proof}




\section{Counterexample to two approaches}\label{sec:BI}
For - path TSP, the main question is whether there exists a -approximation algorithm. When addressing this problem, two natural questions arise:

\begin{itemize}
\item \cite{Gao13} presented a simple -approximation algorithm for the - path TSP in the graphic case. Does it extend to give the same approximation factor for the general metric case?
\item Does every spanning tree in a given convex decomposition of an optimal solution  of (L.P.1) achieve a -approximation factor by adding a minimum-cost -join to fix the wrong degree vertices ?
\end{itemize}

The first question concerns the extension of the algorithm for the graphic case. The second question focuses on the role of randomness and probabilistic methods in the analysis of the recent LP-based approximation algorithms. We answer these questions negatively by providing a counterexample.
In the following, we make the questions more precise and then show how our counterexample serves as a negative answer.


Let  be the metric completion of some connected graph  with unit edge costs .
The - path TSP defined on  is called - path graph-TSP. In this important special case,
the gap between the upper bound and lower bound of the LP integrality ratio has been closed. The first -approximation algorithm
for the - path graph-TSP was given by \cite{SV12} using sophisticated techniques. \cite{Gao13} presented another -approximation
algorithm which was conceptually simpler than that in \cite{SV12}.


Let  be an optimal solution of the (L.P.4) defined on . Note that  for  in this case.
Let  be an - cut. If , we call it a \emph{narrow cut}, which is exactly a -narrow
cut as defined in Section \ref{sec:ucv}. Note that the narrow cuts containing  still have the nice structural
property of Lemma \ref{lem:nestedCuts} even when  is an optimal solution of (L.P.4). We recall some notation
from Section \ref{sec:ucv}. The cuts  are all the narrow cuts containing  such that
. Define 
for  where  and . Note that each  is nonempty and .
It is shown in \cite{Gao13} that  restricted on each  is connected and also there exists at least one edge between each two
consecutive  and  in .

We sketch the -approximation algorithm in \cite{Gao13}. The algorithm constructs a minimal spanning tree on each  and then connects them together by a unit cost edge between each two consecutive  and . This results in a spanning tree on , which is called a \emph{good spanning tree}. Then a minimum-cost -join  is added to correct the wrong degree vertices of the good spanning tree. Since every edge in  has unit cost, the good spanning tree has minimum cost, which is at most . Furthermore, it is shown in \cite{Gao13} that the minimum-cost -join  has cost at most . This gives a -approximation factor in total.

The only part in the analysis using the graphic property is
that the good spanning tree has cost at most . A natural extension of the definition of a good spanning tree would be as follows:
\begin{itemize}
\item In the general metric case, a good spanning tree is constructed by connecting the minimum-cost spanning tree in each  with a minimum-cost edge from  to .
\end{itemize}
If the cost of this ``extended" good spanning tree is bounded above by  in the general metric case,
then it gives us a -approximation factor for - path TSP.
Unfortunately, this is not true. To show this, we present our counterexample, a complete graph 
with metric edge costs  and vertex set  where . The metric edge costs  are given by the metric completion of the costs indicated in Figure \ref{optSoln} below. Note that for every edge  in Figure \ref{optSoln},  is exactly the edge cost value shown in that figure.

Figure \ref{optSoln} shows the support graph of a feasible solution  of (L.P.4),
where the first number on each edge denotes the  value and the second number denotes the cost of the edge.

\begin{figure}[h]
\begin{center}
\includegraphics[scale=0.5]{8nodes.pdf}\\
  \caption{Support graph of  with edge  values and edge costs}
  \label{optSoln}
\end{center}
\end{figure}

\begin{lemma}\label{lem:feaExtreme}
 is an optimal solution for (L.P.4) with respect to .
Furthermore,  is an extreme point of the polyhedron of (L.P.4) on .
\end{lemma}
\begin{proof}
To show the optimality of  for (L.P.4), it is sufficient to prove that  is an optimal
solution of (L.P.1) by Corollary \ref{cor:equiv}. We use complementary slackness conditions to prove the optimality of  for (L.P.1).
Let  be the set of all - cuts and  be the set of all -even cuts. Let .



The following dual solution  witnesses the optimality of  to (L.P.1) by the complementary slackness conditions:
\begin{itemize}
\item , , and  for any other edge 
\item  and  for any other 
\item 
\end{itemize}
Hence  is also an optimal solution of (L.P.4).

Denote the polyhedron of (L.P.4) on  by . We now show that  is an extreme point of . Otherwise,
there exists  and  such that  for some
.

Clearly, for any edge  not in the support graph of , we have  by Lemma \ref{lem:decomposition}.
We also
apply Lemma \ref{lem:decomposition} to  for each vertex , and the cuts .
Then,  for  and  for other vertices, and  for .
Hence,  for each . Let . By the -values on the edges in  and the values  for , we have .
Now consider  and . Then

Hence, . By checking each edge, . This is a contradiction. Therefore,  is an extreme point of . Note that
the analysis above also shows that  is an extreme point of the polytope of (L.P.1) on .

\end{proof}


\begin{figure}[h]
\begin{center}
\includegraphics[scale=0.5]{cross.pdf}\\
  \caption{Cost of the good spanning tree}
  \label{COGST}
\end{center}
\end{figure}

The cost of the corresponding good spanning tree is  and is shown in Figure \ref{COGST}.
The number on the edge between  and  in Figure \ref{COGST} is the edge cost.
The numbers below the dashed narrow cuts are the minimum costs of the edges crossing the
 narrow cuts to connect two consecutive parts. By Lemma \ref{lem:feaExtreme}, we know the optimal value of (L.P.4) is .
 So, we can see that the cost of the good spanning tree
 is strictly larger than the optimal value of (L.P.4). This refutes the statement that the cost of the ``extended"
good spanning tree can be upper bounded by .

Interestingly, this instance also illustrates that probabilistic methods are
important for the analyses of improved LP-based approximation
algorithms such as the  ``randomized Christofides' algorithm" or its
deterministic version the ``best-of-many Christofides' algorithm"
(see \cite{AKS12}). The randomized Christofides' algorithm obtains a better approximation factor
by sampling a spanning tree  from the convex decomposition of .  However, is it true that for an arbitrary spanning tree in the support of a given convex decomposition,
the cost of the spanning tree plus a minimum-cost  -join is at most  ?
In the rest of this section, via the instance , we show this statement is false in general.

We recall the optimal solution  of (L.P.1) on  with metric costs . We know that  is in the spanning
tree polytope (L.P.2). The tight constraints of  for the inequality constraints of (L.P.2) are illustrated as dashed circles in the Figure \ref{TreeJB} except
the tight constraints for , , .
\begin{figure}[h]
\begin{center}
\includegraphics[scale=0.5]{tight.pdf}\\
  \caption{Tree }
  \label{TreeJB}
\end{center}
\end{figure}

By Lemma \ref{lem:decomposition}, the tree  with the dark edges in the graph of Figure \ref{TreeJB} is in some convex decomposition of  in (L.P.2),
i.e.,  is a spanning tree in the support of some convex decomposition of . Let  be the set of wrong degree vertices of , i.e., .  is a minimum-cost -join with cost . Hence, the total cost
of the disjoint union of  and  is , which is larger than  times the optimal value  of (L.P.1). This shows the importance of the probabilistic techniques in the analysis of
the ``randomized Christofides' algorithm" or its
deterministic version the ``best-of-many Christofides' algorithm". Note that the minimum-cost -join  to
fix the wrong degree vertices of  is also larger than half of the optimal value  of (L.P.1).

\medskip
\noindent
{\bf Acknowledgements}.
The author is grateful to Joseph Cheriyan for indispensable help, and to Zachary Friggstad for stimulating discussions.  Many thanks to Nishad Kothari, Andr\'{e} Linhares, Abbas Mehrabian, Andr\'{a}s Seb\H{o}, Chaitanya Swamy and Jens Vygen for their useful comments. The author also would like to thank the anonymous reviewers for their valuable advice for improving the presentation.



\bibliographystyle{alpha}
\bibliography{notes}


\end{document}
