\documentclass {journal}

\setlength {\parindent} {0 pt}
\setlength {\parskip} {1.5 ex plus 0.5 ex minus 0.2 ex}

\usepackage {a4wide}
\usepackage {amssymb}
\usepackage [usenames] {color}
\usepackage{graphicx}
\usepackage {enumerate}
\usepackage {citesort}


\usepackage [usenames] {color}
\usepackage [left,pagewise] {lineno}

\usepackage{fancyhdr}

\definecolor {infocolor} {rgb} {0.6,0.6,0.6}
\renewcommand\linenumberfont{\normalfont\normalsize\textcolor{infocolor}}

\definecolor {sepia} {rgb} {0.75,0.30,0.15}

\everymath{\color{sepia}}

\usepackage {amsmath}
\usepackage {xspace}

\newtheorem {theorem} {Theorem}
\newtheorem {problem} {Problem}
\newtheorem {lemma} {Lemma}
\newtheorem {observation} {Observation}
\newtheorem {corollary} {Corollary}
\newtheorem {claim} {Claim}
\newtheorem {definition} {Definition}
\newenvironment {proof}{\textbf {Proof:}}{\hfill \ensuremath {\boxtimes}}



\newcommand{\maxno}[1]{\ensuremath{T_{#1}(n)}}
\newcommand{\minno}[1]{\ensuremath{t_{#1}(n)}}
\newcommand{\avgno}[1]{\ensuremath{\mu_{#1}(n)}}
\newcommand{\maxnok}{\maxno{k}}
\newcommand{\minnok}{\minno{k}}
\newcommand{\avgnok}{\avgno{k}}

\newcommand{\Pts}{\cal P}

\newcommand {\hodt}{higher order Delaunay triangulation}
\newcommand {\hodts}{higher order Delaunay triangulations}
\newcommand {\Hodts}{Higher order Delaunay triangulations}
\newcommand {\fodt}{first order Delaunay triangulation}
\newcommand {\fodts}{first order Delaunay triangulations}
\newcommand {\Fodts}{First order Delaunay triangulations}
\newcommand {\fod}{first order Delaunay}
\newcommand {\ordero}{\mbox{order-}}
\newcommand {\orderk}{\mbox{order-}}

\newcommand{\bE}{{\mathbb E}}
\newcommand{\bP}{{\mathbb P}}
\newcommand{\bD}{{\mathbb D}}

\def\ex{{\rm\bf E}}
\def\pr{{\rm\bf Pr}}

\newcommand{\marrow}{\marginpar[\hfill]{}}
\newcommand{\niceremark}[3]{\textcolor{red}{\textsc{#1 #2:} \marrow\textsf{#3}}}
\newcommand{\maria}[2][says]{\niceremark{Maria}{#1}{#2}}
\newcommand{\dieter}[2][says]{\niceremark{Dieter}{#1}{#2}}
\newcommand{\rodrigo}[2][says]{\niceremark{Rodrigo}{#1}{#2}}

\newcommand{\futureWork}[2][says]{\niceremark{Rodrigo/Future work}{#1}{#2}}





\graphicspath{{./figures/}}

\pagestyle{fancy}
\fancyhead[LE,RO]{\small \thepage}
\fancyhead[LO,RE]{}
\fancyfoot[C]{}


\begin{document}

\title{On the Number of Higher Order Delaunay Triangulations}


\author{Dieter Mitsche\thanks{Centre de Recerca Matem\`{a}tica, Universitat Aut\`{o}noma de Barcelona,
  \tt{dmitsche@crm.cat}.}
\and
Maria Saumell\thanks{Dept. Matem\`{a}tica Aplicada II, Universitat Polit\`{e}cnica de Catalunya,
   \tt{maria.saumell@upc.edu}.}
\and
Rodrigo I. Silveira\thanks{Dept. Matem\`{a}tica Aplicada II, Universitat Polit\`{e}cnica de Catalunya,
   \tt{rodrigo.silveira@upc.edu}.}
}



\maketitle

\begin{abstract}
\Hodts\ are a generalization of the Delaunay triangulation which
provides a class of well-shaped triangulations, over which extra
criteria can be optimized. A triangulation is order- Delaunay
if the circumcircle of each triangle of the triangulation contains
at most  points. In this paper we study lower and upper bounds
on the number of \hodts,  as well as their expected number for
randomly distributed points. We show that arbitrarily large point
sets can have a single \hodt, even for large orders, whereas for
\fodts, the maximum number is . Next we show that
uniformly distributed points have an expected number of at least
  \fodts, where  is an analytically
defined constant (), and for , the
expected number of order- Delaunay triangulations (which are
not order- for any ) is at least ,
where  can be calculated numerically.
\end{abstract}

 \section{Introduction}

A triangulation is a decomposition into triangles. In this paper
we are interested in triangulations of point sets in the Euclidean
plane, where the input is a set of points in the plane, denoted
, and a triangulation is defined as a subdivision of the
convex hull of  into triangles whose vertices are the points
in .

It is a well-known fact that  points in the plane can have many different triangulations.
For most application domains, the choice of the triangulation is important, because different triangulations can have different effects.
For example, two important fields in which triangulations are frequently used are finite element methods and terrain modeling.
In the first case, triangulations are used to subdivide a complex domain by creating a mesh of simple elements (triangles), over which a system of differential equations can be solved more easily.
In the second case, the points in  represent points sampled from a terrain (thus each point has also an elevation), and the triangulation provides a bivariate interpolating surface, providing an elevation model of the terrain.
In both cases, the shapes of the triangles can have serious consequences on the result.
For mesh generation for finite element methods, the aspect ratio of the triangles is particularly important, since elements of large aspect ratio can lead to poorly-conditioned systems.
Similarly, long and skinny triangles are generally not appropriate for surface interpolation because they can lead to interpolation from points that are too far apart.

In most applications, the need for \emph{well-shaped}
triangulations is usually addressed by using the \emph{Delaunay}
triangulation. The Delaunay triangulation of a point set  is
defined as a triangulation where the vertices are the points in
 and the circumcircle of each triangle (that is, the circle
defined by the three vertices of each triangle) contains no other
point from . The Delaunay triangulation has many known
properties that make it the most widely-used triangulation. In
particular, there are several efficient and relatively simple
algorithms to compute it, and its triangles are considered
\emph{well-shaped}. This is because it maximizes the minimum angle
among all triangle angles, which implies that its angles are---in
a sense---\emph{as large as possible}. Moreover, when the points
are in general position (that is, when no four points are
cocircular and no three points are collinear) it is uniquely
defined. However, this last property can become an important
limitation if the Delaunay triangulation is suboptimal with
respect to other criteria, independent of the shape of its
triangles, as it is often the case in applications.

To overcome this limitation, Gudmundsson et al. proposed
\emph{\hodts}~\cite{ghk-hodt-02}. They are a natural
generalization of the Delaunay triangulation that provides
well-shaped triangles, but at the same time,  flexibility to
optimize some extra criterion. They are defined by allowing up to
 points inside the circumcircles of the triangles (see Figure~\ref{fig:FigOrders}). For ,
each point set in general position has only one higher order
Delaunay triangulation, equal to the Delaunay triangulation. As
the parameter  is increased, more points inside the
circumcircles imply a reduction of the shape quality of the
triangles, but also an increase in the number of triangulations
that are considered. This last aspect makes the optimization of
extra criteria possible, thus providing triangulations that are a
compromise between well-shaped triangles and optimality with
respect to other criteria.



   \begin{figure}[tb]
\centering
\includegraphics{FigOrders}
\caption{Left: A Delaunay triangulation (). Center: an
order- triangulation (with useful-1, non-Delaunay, edges in
gray). Right: an order- triangulation, with order-1 triangles
in light gray and order-2 triangles in medium gray.}
\label{fig:FigOrders}
\end{figure}


Therefore the importance of \hodts\ lies in multi-criteria
triangulations. Their major contribution is providing a way to
optimize over a---hopefully large---class of well-shaped
triangulations.

A particularly important subclass of \hodts\ are the
\emph{\fodts}, that is, when . It has been observed that
already for , a point set with  points can have an
exponential number of different triangulations~\cite{s-opt-09}.
This, together with the fact that for  the shape of the
triangles is as close as possible to the shape of the Delaunay
triangles (while allowing more than one triangulation to choose
from), make first order Delaunay triangulations especially
interesting. In fact, \fodts\ have been shown to have a special
structure that facilitates the optimization of many
criteria~\cite{ghk-hodt-02}. For example, it has been shown that
many criteria related to measures of single triangles, as well as
some other relevant parameters like the number of local minima,
can be optimized in  time for . In a recent
paper~\cite{kls-ofodt-09}, Van Kreveld et al. studied several
types of more complex optimization problems, constrained to
\fodts. They showed that many other criteria can be also optimized
efficiently for , making \fodts\ even more appealing for
practical use.

For larger values of , fewer results are known.
The special structure of \fodts\ is not present anymore, which complicates exact optimization algorithms.
Several heuristics and experimental results have been presented for optimization problems related to terrain modeling,
showing that very small values of  () are enough to achieve important improvements for several terrain criteria~\cite{bg-drthodt-08,bg-sfthodt-08,kkl-grtho-07}.

However, despite the importance given to finding algorithms to
optimize over \hodts, it has never been studied before how many
\hodts\ there can be in the first place. In other words, it is not
known what the minimum and maximum number of different
triangulations are, as functions of  and , not even for the
simpler (but---in practice---most important) case of \fodts.

The problem of determining bounds on the number of \hodts\ is of
both theoretical and practical interest.

From a theoretical point of view, determining how many triangulations a point set has is one of the most intriguing problems in combinatorial geometry, and has received a lot of attention (e.g. \cite{ahhhk-npg-06,sw-rtpps-06,ss-bubnt-03}).
\Hodts\ are a natural and simple generalization of the Delaunay triangulation, hence the impact of such generalization on the number of triangulations is worth studying.

From a more practical point of view, knowing the number of
triangulations for a given  gives an idea of how large the
solution space is when optimizing over this class of well-shaped
triangulations. Ideally, one expects to have many different
triangulations to choose from, in order to find one that is
\emph{good} with respect to other criteria.

Up to now, only trivial bounds were known: every point set has at
least one order- Delaunay triangulation, for any  (equal to
the Delaunay triangulation), and there are point sets of size 
that have  triangulations, already for . In
this paper we present the first non-trivial bounds on the number
of \hodts. Given the practical motivation mentioned above, we are
mostly interested in results that have practical implications for
the use of \hodts. Thus low values of  are our main concern.
Our ultimate goal---achieved partially in this paper---is to
determine to what extent the class of \hodts\ (for small values of
, which has the best triangle-shape properties), also provides
a \emph{large} number of triangulations to choose from.


\paragraph{Results} We study lower and upper bounds on the number of \hodts, as well
as the expected number of order- Delaunay triangulations for
uniformly distributed points. Let  denote the maximum
number of order- Delaunay triangulations that a set with 
points can have, and let  denote the minimum number of
order- Delaunay triangulations that a set with  points can
have. First we show that the lower bound  is
tight. In other words, there are arbitrarily large point sets that
have a single \hodt, even for large values of . Next we show
that, for \fodts, . Since these extreme cases
do not describe an average situation when \hodts\ are used, we
then study the number of \hodts\ for a uniformly distributed point
set. Let  denote the number of order- (and not order-
for any ) Delaunay triangulations of a uniformly
distributed point set of size . We show that , where  is an analytically defined
constant (). We also prove that, for
constant values of  ,
where  can be calculated numerically (asymptotics are with
respect to ). The result has interesting practical
consequences, since it implies that it is reasonable to expect an
exponential number of \hodts\ for any .


\paragraph{Related work}
As mentioned earlier, there is no previous work on counting \hodts.
A related concept, the \emph{higher order Delaunay graph}, has been studied by Abellanas et al.~\cite{apgh-sgtph-08}.
The \emph{order- Delaunay graph} of a set of points  is a graph with vertex set  and an edge between two points  when there is a circle through  and  containing at most  other points from .
Abellanas et al. presented upper and lower bounds on the number of edges of this graph. However, since a triangulation that is a subset of the order- Delaunay graph does not need to be an order- Delaunay triangulation, it is difficult to derive good bounds for \hodts\ based on them.

There is an ample body of literature on the more general problem of counting \emph{all} triangulations.
Lower and upper bounds on the number of triangulations that  points can have have been improved many times over the years, with the current best ones establishing that there are point sets that have ~\cite{ahhhk-npg-06} triangulations, whereas no point set can have more than ~\cite{sw-rtpps-06}.



In relation to our expected case analysis of the number of \hodts, it is worth mentioning that many properties of the Delaunay triangulation---and related proximity graphs---of random points have been studied.
The expected behavior of properties of the Delaunay triangulation that have been considered include the average and maximum edge length~\cite{m-hpppp-70,bey-eedt-91a}, the minimum and maximum angles~\cite{bey-eedt-91a}, and its expected weight~\cite{cl-aldt-84}.
Expected properties of other proximity graphs, such as the Gabriel graph and some relatives, are investigated in~\cite{d-essgc-88,c-psepg-92,ms-pggrg-80}.


\paragraph{Outline} This paper is structured as follows.
The next section presents some previous results related to \hodts,
needed for the following sections. In Section~\ref{sec:bounds} we
give lower and upper bounds for the number of \hodts.
Section~\ref{sec:ExpNumTriang} deals with the expected number of
\hodts. Finally, some concluding remarks are made in
Section~\ref{sec:discussion}.

\section{\Hodts}
\label{sec:hodts} We begin by introducing \hodts\ more formally,
and presenting a few properties that will be used throughout the
paper. From now on, we assume that point sets are in general
position.



\begin{definition}\label{def:Order}
(from \cite{ghk-hodt-02}) A triangle  in a point set  is \emph{order- Delaunay} if its circumcircle  contains at most  points of . A triangulation of  is \emph{order- Delaunay} if every triangle of the triangulation is order-.
\end{definition}


Note that if a triangle or triangulation is order-, it is also order- for any . A simple corollary of this is that, for any point set and  any , the Delaunay triangulation is an order- Delaunay triangulation.

\begin{definition}
(from \cite{ghk-hodt-02}) An edge  is an order-
Delaunay edge if there exists a circle through  and  that
has at most  points of  inside. An edge 
is a \emph{useful} order- Delaunay edge (or simply
\emph{useful- edge}) if there is an \orderk\ Delaunay
triangulation that contains .
\end{definition}

The useful order of an edge can be checked using the following lemma, illustrated in Figure~\ref{fig:pre-usefulOrder}.


\begin{figure}[tb]
\centering
\includegraphics{pre-usefulOrder}
\caption{The useful order of edge  is determined by
the lowest order of triangles  and . In the example, the (lowest) useful order of
 is .} \label{fig:pre-usefulOrder}
\end{figure}

\begin{lemma}
  \label {lem:useful_edge_test}
      (from \cite{ghk-hodt-02}) Let  be an order- Delaunay edge, let  be the point to the left\footnote{
      We sometimes treat edges as directed, to be able to refer to the right or left side of the edge.
      The \emph{left} side of  denotes the halfplane defined by the line
      supporting , such that
       a polygonal line defined by ,  and a point interior to that halfplane, makes a counterclockwise turn.
       In the \emph{right} side, the turn is clockwise.}
       of , such that the circle  contains no points to the left of .
       Let  be defined similarly but to the right of . Edge  is
       useful- if and only if  and  are order- Delaunay triangles.
\end{lemma}

The concept of a \emph{fixed edge} is important in order to study
the structure of \hodts.

\begin{definition}
Let  be a point set and  its Delaunay triangulation. An
edge of  is \emph{-fixed} if it is present in every
order- Delaunay triangulation of .
\end{definition}

Some simple observations derived from this are that the convex hull edges are always -fixed, for any , and that all the Delaunay edges are -fixed.


\Fodts\ have a special structure.
  If we take all edges that are 1-fixed, then the resulting subdivision has only triangles and
  convex quadrilaterals (and an unbounded face). In the convex quadrilaterals,
  both diagonals are possible to obtain a \fodt\ (see Figure~\ref{fig:FigOrders}, center).
  We say that both diagonals are \emph{flippable}, and similarly we call the quadrilateral \emph{flippable}.
More formally, based on results in~\cite{ghk-hodt-02}, we can make the following observation.

\begin{observation}
\label{obs:Order1Edge}
Let  be a useful order-1 Delaunay edge in an order-1 Delaunay triangulation, such that  is not a Delaunay edge.
Then flipping  results in a Delaunay edge. Moreover, the four edges (different from ) that bound the two triangles adjacent to  are 1-fixed edges.
\end{observation}

An implication of this special structure is that instead of
counting triangulations, we can count flippable quadrilaterals or,
equivalently, useful-1 edges that are not Delaunay.

\begin{corollary}
\label{cor:flippable}
Let  be a point set. If  has  flippable quadrilaterals, then  has exactly  order-1 Delaunay triangulations.
\end{corollary}

For , the structure is not so simple anymore and it seems
difficult to provide an exact expression for the number of
order- Delaunay triangulations in terms of the number of
useful- edges. However, we can derive a lower bound by combining a number of known results, as follows.
First we need some extra
definitions and previous results:

\begin{definition}
(from~\cite{ghk-hodt-02}) The \emph{hull} of an order- Delaunay
edge  () is the closure of the union of
all Delaunay triangles whose interior intersects 
\end{definition}

\begin{lemma} \label{lem:sizehull}
(from~\cite{ghk-hodt-02}) The hull of an order- Delaunay edge
() is a simple polygon consisting of at most 
vertices.
\end{lemma}

\begin{lemma} \label{lem:trianghull}
(from~\cite{ghk-hodt-02}) Let  be a useful- edge
(and not useful- for any ), with  There exists
an order- (and not order- for any ) Delaunay
triangulation of the hull of  that contains

\end{lemma}

\begin{lemma} \label{lem:UsefkIntersOrder0}
Let  be an order- edge. The number of useful-
edges () that intersect  is at most

\end{lemma}

\begin{proof}
It follows directly from the proof of Lemma~8
in~\cite{ghk-hodt-02}.
\end{proof}


We have now the necessary tools to prove the following lower bound on the number of order- triangulations, expressed as a function of the number of useful- edges.

\begin{lemma}\label{lem:k_flippable}
Let  be a point set and let  for  be the number
of useful- edges (which are not useful- for any ) of
 Then  has at least  order- (and
not order- for any ) Delaunay triangulations, where

\end{lemma}

\begin{proof}
Let  denote the set of useful- edges (which are not
useful- for any ) of  and let  denote the
cardinal number of  We select a subset  of the edges
of  in the following way: We pick an edge  of  we
remove all the edges in  whose hull intersects the hull of
 in at least one Delaunay triangle, and we repeat until 
does not contain any edge.

Let  be an edge in  and  be an edge in  whose
hull intersects the hull of  in at least one Delaunay triangle
 Then  intersects at least one edge of  By
Lemma~\ref{lem:sizehull}, all Delaunay triangles included in the
hull of  contain at most  edges. By
Lemma~\ref{lem:UsefkIntersOrder0}, each of them intersects at most
 useful- edges. Hence if  is selected, at most
 edges in  are removed. Therefore 
contains at least  edges.

Each non-empty subset of  gives rise to a different
order- (and not order- for any ) Delaunay
triangulation proceeding as follows: If an edge  is in the
subset, we triangulate the hull of  as in
Lemma~\ref{lem:trianghull}, that is, we use an order- (and not
order- for any ) Delaunay triangulation containing 
If an edge  is not in the subset , we triangulate the hull of
 using the Delaunay triangles crossed by 
Finally, we complete the
triangulation by adding Delaunay triangles in the regions that
have not been triangulated (that is, computing a constrained Delaunay
triangulation). This construction is consistent because the hulls
of the edges in  can only intersect in points and boundary
edges, and because the boundary edges of the hulls belong to the
Delaunay triangulation.
\end{proof}




\section{Lower and upper bounds}
\label{sec:bounds}

In this section we derive upper and lower bounds on the number of \fodts.
As mentioned in the introduction, due to the practical motivation of this work, we are mostly interested in lower bounds.
However, for completeness and because the theoretical question is also interesting, this section also includes a result on upper bounds.

The main question that we address in this section is: what is the minimum number of \hodts\ that  points can have?
Are there arbitrarily large point sets that have only  \hodts?
To our surprise, the answer to the second question is affirmative.

The lemma below presents a construction that has only one \hodt, regardless of the value of , for any .
Note that this implies that for any value of  of practical interest, there are point sets that have no other order- Delaunay triangulation than the Delaunay triangulation.

\begin{figure}[tb]
\centering
\includegraphics{ExampleOnly1Triangulation}
\caption{Construction of a point set (left) whose only order- Delaunay triangulation is the Delaunay triangulation (right).}
\label{fig:ExampleOnly1Triangulation}
\end{figure}

\begin{lemma}
\label{lem:Only1}
Given any  and any  such that , there are point sets with  points in general position that have only one order- Delaunay triangulation.
\end{lemma}

\begin{proof}
We give a construction with  points that can be shown to have only one order- Delaunay triangulation, for any value of  . The construction is illustrated in Figure~\ref{fig:ExampleOnly1Triangulation}.

To simplify the explanation, in the following we assume that  is a multiple of 3.
Since any order- Delaunay triangulation is also order- for all , for the proof it is enough to use .


We start with a triangle . Then we add
three groups of points, which we will denote with letters ,
, and . The points in the first group are
denoted , where . These points are initially
placed on a circle  that goes through , as shown in the
figure; they are sorted from top to bottom. The second group
comprises  points , placed very close to each
other on a circle  through , as shown in the figure. In
addition, we must also make sure that the points 
are close enough to  in order to be contained inside
. Finally, the points in the third group, , are placed very close to each other on a third
circle , which goes through . The important properties
of these circles are:
(i)  contains  and all the points ,
(ii)  and  contain all the points of the type ,
and
(iii)  contains  and all points of type .

Clearly, the point set as constructed is degenerate, but this can be easily solved by applying a slight perturbation to each point, without affecting the properties just mentioned.
Moreover, the perturbation can be made such that the Delaunay triangulation of the point set looks like the one in the right of Figure~\ref{fig:ExampleOnly1Triangulation}.

We now argue that all the edges in the Delaunay triangulation are
-fixed, by considering the different types of edges that,
potentially, could cross a Delaunay edge to make it non-fixed.
Suppose an edge of the shape  is not
-fixed. Then there must be some triangulation in which the edge
is crossed by some other useful order- edge. Such edge can be
of three types: (i) it connects two points , , (ii) it
connects two points ,  (or ), or (iii) it connects
two points ,  (or ). An edge of the type
 that crosses  must be an
edge of the shape  or force a such an edge
to appear in the triangulation. However, the circumcircle of the
triangle defined by any three consecutive points , ,
 contains at least  points because it is a slightly
perturbed version of . Thus no such edge can be part of an
order- triangulation. A similar situation occurs with any edge
of the shape , since it forces a triangle of
the form  (or ).
Finally, edges of type  force a triangle of
the form  (or ), whose
circumcircle includes at least as many points as contained in
, hence cannot be part of an order-
triangulation either. Therefore all the edges of the shape
 are -fixed. Similar arguments can be used
to show that the edges in the other groups are also -fixed,
hence no other order- triangulation can exist.
\end{proof}


Having determined that some point sets can have as little as one \fodt, it is reasonable to ask what is the \emph{maximum} number of \fodts\ that a point set can have.
The following lemma gives a precise---and tight---bound on the maximum number of \fodts.

\begin{lemma} \label{lem:maxfodt}
Every point set  with  points in general position has at most
 \fodts, and this bound is tight.
\end{lemma}
\begin{proof}
To see that no point set can have more than  flippable
quadrilaterals, observe that the subdivision of the convex hull of
 induced by the fixed edges is a plane graph. It follows
from Euler's equation that any triangulation has at most 
triangles. Since each quadrilateral is formed by two triangles,
there can be at most  quadrilaterals.

Now we give a construction with  (for  any multiple of 4)
points that has  flippable quadrilaterals, thus a total of
 \fodts. The construction is illustrated in
Figure~\ref{fig:maxNumberOrder1bis}, and consists of a series of
points placed on the vertices of concentric squares with the same
orientation. Clearly, the edges in
Figure~\ref{fig:maxNumberOrder1bis} are Delaunay edges and the
four vertices of each quadrilateral are cocircular. If we apply a
small perturbation to the point set so that it reaches a general
position, one of the diagonals of each quadrilateral becomes a
Delaunay edge, while the other one becomes a useful order-1
(non-Delaunay) edge. Therefore, all quadrilaterals are flippable.
\end{proof}

\begin{figure}[tb]
\centering
\includegraphics{maxNumberOrder1bis}
\caption{Construction achieving the maximum number of \fodts. Left
and right: point set and flippable quadrilaterals, for points not
in general position.} \label{fig:maxNumberOrder1bis}
\end{figure}



\section{Expected number of triangulations}
\label{sec:ExpNumTriang} Let  be a set of  points
uniformly distributed in the unit square. In this section we give
lower bounds on the expected number of \hodts\ of 

Note  that the events that four points in 
are cocircular and that three points in  form a right angle
happen with probability zero, and hence we can safely ignore these cases. Throughout this section we will use the notation  if .

We start with \fodts. We aim to compute the probability that two
randomly chosen points  in  form a useful-1,
non-Delaunay, edge. Assume that the edge is directed
. Let  be the point to the left of
, such that the circle  contains no
points to the left of , and let  be the
point to the right of , such that the circle
 contains no points to the right of
. Let  be the event defined as
follows: the edge  is useful-1 (but not Delaunay),
 is to the right of  and the circle
 contains no points of  to the right of
,  is to the left of 
and the circle  contains no points of  to the left
of . It is well-known that 
belongs to the Delaunay triangulation of  if and only if
. Thus the event  can be
decomposed into the disjoint union

where  denotes the event  with the
additional conditions that    denotes the event  with the
 conditions that   and  denotes the event  with
the  conditions that   (in all cases we must have ).  Consequently,


\begin{lemma}\label{lem_e1_e2}
 and , where  and .
\end{lemma}
\begin{proof}
Let us first compute 

Let  (respectively, )  be the interior of the set
consisting of all points in  that are to
the left (resp. right) of  Let 
(respectively, ) denote the interior of the set containing
all points in  (resp. ) that are to the right
(resp. left) of  and do not lie in 
(resp. ) (see Figure~\ref{fig:Events}, left).
 Since  is the point such that the
circle  contains no points to the left of
 the region  is empty of points in
 In order for the edge  to be useful-1, the region  also has to be empty of points in 
Analogously, under the hypothesis of , the regions
 and  are empty of points in  It is not
difficult to see that the reverse implications also hold.
Therefore, the event  is equivalent to the event
that    and  do not contain any point in


\begin{figure}[tb]
\centering \scalebox{0.82}{\begin{picture}(0,0)\includegraphics{regions_6.pstex}\end{picture}\setlength{\unitlength}{4144sp}\begingroup\makeatletter\ifx\SetFigFont\undefined \gdef\SetFigFont#1#2#3#4#5{\reset@font\fontsize{#1}{#2pt}\fontfamily{#3}\fontseries{#4}\fontshape{#5}\selectfont}\fi\endgroup \begin{picture}(5436,1415)(36,-3086)
\put(631,-3031){\makebox(0,0)[b]{\smash{{\SetFigFont{10}{12.0}{\familydefault}{\mddefault}{\updefault}{\color[rgb]{0,0,0}}}}}}
\put(1171,-1861){\makebox(0,0)[b]{\smash{{\SetFigFont{10}{12.0}{\familydefault}{\mddefault}{\updefault}{\color[rgb]{0,0,0}}}}}}
\put(316,-2266){\makebox(0,0)[b]{\smash{{\SetFigFont{10}{12.0}{\familydefault}{\mddefault}{\updefault}{\color[rgb]{0,0,0}}}}}}
\put(766,-2356){\makebox(0,0)[b]{\smash{{\SetFigFont{10}{12.0}{\familydefault}{\mddefault}{\updefault}{\color[rgb]{0,0,0}}}}}}
\put(991,-2536){\makebox(0,0)[b]{\smash{{\SetFigFont{10}{12.0}{\familydefault}{\mddefault}{\updefault}{\color[rgb]{0,0,0}}}}}}
\put(1396,-2356){\makebox(0,0)[b]{\smash{{\SetFigFont{10}{12.0}{\familydefault}{\mddefault}{\updefault}{\color[rgb]{0,0,0}}}}}}
\put(721,-1951){\makebox(0,0)[b]{\smash{{\SetFigFont{10}{12.0}{\familydefault}{\mddefault}{\updefault}{\color[rgb]{0,0,0}}}}}}
\put(1126,-2896){\makebox(0,0)[b]{\smash{{\SetFigFont{10}{12.0}{\familydefault}{\mddefault}{\updefault}{\color[rgb]{0,0,0}}}}}}
\put(4771,-3031){\makebox(0,0)[b]{\smash{{\SetFigFont{10}{12.0}{\familydefault}{\mddefault}{\updefault}{\color[rgb]{0,0,0}}}}}}
\put(5311,-1861){\makebox(0,0)[b]{\smash{{\SetFigFont{10}{12.0}{\familydefault}{\mddefault}{\updefault}{\color[rgb]{0,0,0}}}}}}
\put(4726,-2086){\makebox(0,0)[b]{\smash{{\SetFigFont{10}{12.0}{\familydefault}{\mddefault}{\updefault}{\color[rgb]{0,0,0}}}}}}
\put(4771,-2356){\makebox(0,0)[b]{\smash{{\SetFigFont{10}{12.0}{\familydefault}{\mddefault}{\updefault}{\color[rgb]{0,0,0}}}}}}
\put(4343,-2221){\makebox(0,0)[b]{\smash{{\SetFigFont{10}{12.0}{\familydefault}{\mddefault}{\updefault}{\color[rgb]{0,0,0}}}}}}
\put(5208,-2356){\makebox(0,0)[b]{\smash{{\SetFigFont{10}{12.0}{\familydefault}{\mddefault}{\updefault}{\color[rgb]{0,0,0}}}}}}
\put(5086,-2671){\makebox(0,0)[b]{\smash{{\SetFigFont{10}{12.0}{\familydefault}{\mddefault}{\updefault}{\color[rgb]{0,0,0}}}}}}
\put(5401,-2626){\makebox(0,0)[b]{\smash{{\SetFigFont{10}{12.0}{\familydefault}{\mddefault}{\updefault}{\color[rgb]{0,0,0}}}}}}
\put(2431,-2986){\makebox(0,0)[b]{\smash{{\SetFigFont{10}{12.0}{\familydefault}{\mddefault}{\updefault}{\color[rgb]{0,0,0}}}}}}
\put(2971,-1816){\makebox(0,0)[b]{\smash{{\SetFigFont{10}{12.0}{\familydefault}{\mddefault}{\updefault}{\color[rgb]{0,0,0}}}}}}
\put(2521,-1906){\makebox(0,0)[b]{\smash{{\SetFigFont{10}{12.0}{\familydefault}{\mddefault}{\updefault}{\color[rgb]{0,0,0}}}}}}
\put(3039,-2221){\makebox(0,0)[b]{\smash{{\SetFigFont{10}{12.0}{\familydefault}{\mddefault}{\updefault}{\color[rgb]{0,0,0}}}}}}
\put(2566,-2243){\makebox(0,0)[b]{\smash{{\SetFigFont{10}{12.0}{\familydefault}{\mddefault}{\updefault}{\color[rgb]{0,0,0}}}}}}
\put(2544,-2536){\makebox(0,0)[b]{\smash{{\SetFigFont{10}{12.0}{\familydefault}{\mddefault}{\updefault}{\color[rgb]{0,0,0}}}}}}
\end{picture} }
\caption{Left: the event  and the regions ,
,  and . Middle: in the event , the
region  is a circular sector minus a triangle. Right: the
event  and the regions , ,  and
.} \label{fig:Events}
\end{figure}

Now let us denote by  the radius of the circle  and
by  the length of the edge  (see
Figure~\ref{fig:Events}, center). A straightforward calculation
leads to the following expression for the area of :


In order to compute , we will be interested in
having certain areas being empty of  points, which happens with
probability  (for  the area in
question). Since the contribution of areas  of size 
is  for some  (which is far less
than the asymptotic value of the integrals, as we shall see
below), for any constant  we can safely assume in the
integrals below the asymptotic equivalence
,
without affecting the first order terms of the asymptotic behavior
of the integral~\footnote{In fact, this formula arises in a
homogeneous Poisson point process of intensity  in the unit
square, and it is not surprising that both distributions give the
same asymptotic results (see the ideas of depoissonization given
in~\cite{Penrose03}).}.


Observe that  may take values from  to  and
that the probability density of the event 
is . Notice also that
 and that the event of having a radius 
has probability density

since it corresponds to the negative derivative  of the function


Denoting by  the radius of the circle , we obtain analogous expressions for .

Now we have all the necessary ingredients to develop an expression
for 


Since classical methods for asymptotic integration fail for the
integral given by~\eqref{eq:case1_formula} (the derivative of the
exponent is infinity at the point where the exponent maximizes),
we apply the following change of variables:  ,
, . The
integral~\eqref{eq:case1_formula} then becomes (replacing the
integration limit  by , which can be done since
the dominant contribution comes from small values of )



Given that it does not seem possible to evaluate this integral
symbolically, we resort to applying numerical methods. For reasons
of numerical stability (especially in the case of
 below) we apply another change of variables
, :



Solving this integral numerically (using Mathematica), we obtain
that , where .

Let us now consider  Let , ,  and
 be defined as in the event  (see
Figure~\ref{fig:Events}, right). By the same arguments, the event
 is equivalent to the event that the regions 
  and  are empty of points in  Analogous
observations as in the previous case yield



As before, we apply the substitution ,
,  to the
integral~\eqref{eq:case2_formula} and obtain



For reasons of numerical stability, we again apply the change of
variables  and
 and obtain


Solving this integral numerically (using Mathematica), we obtain
that  , where .
\end{proof}

Denote by  the random variable counting the number of
useful-1 (and not Delaunay) edges. We have the following
corollary:
\begin{corollary}\label{cor:useful_1}
 where .
\end{corollary}
\begin{proof}
Since  and  are symmetric, we
obviously have that 
Hence,  Since for a
fixed edge  there are  ways to
choose the points  and  to the left and to the right of
, and these events are all disjoint, the edge
 is useful-1 (and not Delaunay) with probability
. Hence,  where .
\end{proof}

Recall that  denotes the number of order- (and not
order- for any ) Delaunay triangulations of a uniformly
distributed point set. We can now state the following theorem:

\begin{theorem}\label{thm:triangulations1}
Given  points distributed uniformly at random in the unit
square, , where 
\end{theorem}
\begin{proof}
By Corollary~\ref{cor:flippable}, . Now, by
Jensen's inequality, , and the
result follows by Corollary~\ref{cor:useful_1}.
\end{proof}

Combining the ideas for the case  with the result from
Lemma~\ref{lem:k_flippable}, we obtain the following
generalization for constant values of :

\begin{theorem}
Given  points distributed uniformly at random in the unit
square, for any constant value of   where  is a constant that can be calculated
numerically.
\end{theorem}
\begin{proof}
Denote by  the number of useful- edges (which are not
useful- for any ) for any constant . We want to know the value of .

In order for an edge  to be useful- (and not
useful- for ), using the notation of
Figure~\ref{fig:Events}, first observe that the regions  and
 have to be empty of points. Moreover, either the region  has to contain exactly  points ( is
excluded), whereas the region  can contain any
number of points  ( is excluded), or vice
versa. For any constant , the probability of having exactly 
points in an area  of size  (as before, for constant 
only such areas count for the asymptotic behaviour of the
integrals) is . Thus,
defining the events   and
 analogously as in Section~\ref{sec:ExpNumTriang},



Now, since

and

after applying the substitutions , , , in these new factors  disappears and the
integral again yields . The same argument also
holds for  and by the same reasoning as in
the case of useful-1 edges, we obtain that the expected number of
useful- edges (that are not useful- for any ) is  for any constant . We point out that using our formula the
constant  can be calculated numerically.

By Lemma~\ref{lem:k_flippable},  where
 Therefore  and as before, by Jensen's inequality,

\end{proof}


\section{Discussion and further work}
\label{sec:discussion} We have given the first non-trivial bounds
on the number of \hodts. We showed that there are sets of 
points that have only one \hodt\ for values of , and that no point set can have
more than  \fodts. Moreover, we showed that for any
constant value of  (in particular, already for ) the
expected number of order- triangulations of  points
distributed uniformly at random is exponential. This supports the
use of \hodts\ for small values of , which had already been
shown to be useful in several applications related to terrain
modeling~\cite{kkl-grtho-07}. From a more theoretical perspective,
it would be interesting to obtain tighter bounds on the expected number of
order- Delaunay triangulations for uniformly distributed points.
\paragraph{Acknowledgments} We are grateful to Nick Wormald for his help on dealing with the integrals of Section~\ref{sec:ExpNumTriang}. We would also like to thank Christiane Schmidt and Kevin Verbeek for helpful discussions.
M. S. was partially supported by projects MTM2009-07242 and Gen. Cat. DGR
2009SGR1040.
R.I.S. was supported by the Netherlands Organisation for
  Scientific Research (NWO).

\bibliographystyle{abbrv}

\begin{thebibliography}{99}

\bibitem{apgh-sgtph-08}
M.~Abellanas, P.~Bose, J.~Garc\'{\i}a, F.~Hurtado, C.~M.
Nicol\'as, and
  P.~Ramos.
\newblock On structural and graph theoretic properties of higher order
  {D}elaunay graphs.
\newblock {\em Internat. J. Comput. Geom. Appl.}, 19(6):595--615, 2009.

\bibitem{ahhhk-npg-06}
O.~Aichholzer, T.~Hackl, C.~Huemer, F.~Hurtado, H.~Krasser, and
B.~Vogtenhuber.
\newblock On the number of plane graphs.
\newblock In {\em 17th Annual ACM-SIAM Symposium on Discrete Algorithms}, pages
  504--513. ACM, 2006.

\bibitem{bey-eedt-91a}
M.~Bern, D.~Eppstein, and F.~Yao.
\newblock The expected extremes in a {Delaunay} triangulation.
\newblock {\em Internat. J. Comput. Geom. Appl.}, 1:79--91, 1991.

\bibitem{bg-drthodt-08}
A.~Biniaz and G.~Dastghaibyfard.
\newblock Drainage reality in terrains with higher-order {D}elaunay
  triangulations.
\newblock In P.~van Oosterom, S.~Zlatanova, F.~Penninga, and E.~M. Fendel,
  editors, {\em Advances in 3D Geoinformation Systems}, pages 199--211.
  Springer Berlin Heidelberg, 2008.

\bibitem{bg-sfthodt-08}
A.~Biniaz and G.~Dastghaibyfard.
\newblock Slope fidelity in terrains with higher-order {D}elaunay
  triangulations.
\newblock In {\em 16th International Conference in Central Europe on Computer
  Graphics, Visualization and Computer Vision}, pages 17--23, 2008.

\bibitem{cl-aldt-84}
R.~C. Chang and R.~C.~T. Lee.
\newblock On the average length of {Delaunay} triangulations.
\newblock {\em BIT}, 24:269--273, 1984.

\bibitem{c-psepg-92}
R.~J. Cimikowski.
\newblock Properties of some {Euclidian} proximity graphs.
\newblock {\em Pattern Recogn. Lett.}, 13(6):417--423, 1992.

\bibitem{kkl-grtho-07}
T.~de~Kok, M.~van Kreveld, and M.~L{\"o}ffler.
\newblock Generating realistic terrains with higher order {D}elaunay
  triangulations.
\newblock {\em Comput. Geom.}, 36:52--65, 2007.

\bibitem{d-essgc-88}
L.~Devroye.
\newblock On the expected size of some graphs in computational geometry.
\newblock {\em Comput. Math. Appl.}, 15:53--64, 1988.

\bibitem{ghk-hodt-02}
J.~Gudmundsson, M.~Hammar, and M.~van Kreveld.
\newblock Higher order {D}elaunay triangulations.
\newblock {\em Comput. Geom.}, 23:85--98, 2002.

\bibitem{ms-pggrg-80}
D.~W. Matula and R.~R. Sokal.
\newblock Properties of {Gabriel} graphs relevant to geographic variation
  research and clustering of points in the plane.
\newblock {\em Geogr. Anal.}, 12(3):205--222, 1980.

\bibitem{m-hpppp-70}
R.~E. Miles.
\newblock On the homogenous planar {Poisson} point-process.
\newblock {\em Math. Biosci.}, 6:85--127, 1970.

\bibitem{Penrose03}
M.~Penrose.
\newblock {\em Random Geometric Graphs}.
\newblock Oxford University Press, 2003.

\bibitem{ss-bubnt-03}
F.~Santos and R.~Seidel.
\newblock A better upper bound on the number of triangulations of a planar
  point set.
\newblock {\em J. Combin. Theory Ser. A}, 102(1):186--193, 2003.

\bibitem{sw-rtpps-06}
M.~Sharir and E.~Welzl.
\newblock Random triangulations of planar point sets.
\newblock In {\em 22nd Annual Symposium on Computational Geometry}, pages
  273--281. ACM, 2006.

\bibitem{s-opt-09}
R.~I. Silveira.
\newblock {\em Optimization of polyhedral terrains}.
\newblock PhD thesis, Utrecht University, 2009.

\bibitem{kls-ofodt-09}
M.~van Kreveld, M.~L{\"o}ffler, and R.~I. Silveira.
\newblock Optimization for first order {D}elaunay triangulations.
\newblock {\em Comput. Geom.}, 43(4):377--394, 2010.

\end{thebibliography}

\end{document}
