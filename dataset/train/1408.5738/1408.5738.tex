
\documentclass[a4paper, 9pt, twocolumn]{IEEEtran}      




\usepackage{amsmath,amssymb,enumerate,soul}
\usepackage{epstopdf}
\usepackage{times} \usepackage[dvips]{graphicx}
\usepackage{psfrag}
\usepackage{multirow} \usepackage{cite} \usepackage{float}
\usepackage[x11names]{xcolor}
\usepackage{colortbl}

\let\proof\relax \let\endproof\relax
\usepackage{amsthm}
\theoremstyle{plain}
\newtheorem{ass}{Assumption}
\newtheorem{thm}{Theorem}
\newtheorem{props}{Proposition}
\newtheorem{cor}{Corollary}
\newtheorem{rmrk}{Remark}

\theoremstyle{definition}
\newtheorem{exmple}{Example}

\newtheoremstyle{cited}{3pt}{3pt}{\itshape}{}{\bfseries}{.}{.5em}{\thmname{#1} \thmnumber{#2} \normalfont(\thmnote{\normalfont#3})}\theoremstyle{cited}
\newtheorem{lemma}{Lemma}


\newcommand{\arctanh}{\operatorname{arctanh}}
\newcommand{\dom}{\ensuremath{\text{dom}\,}}
\newcommand{\avg}{\ensuremath{\text{avg}\,}}
\newcommand{\diag}{\ensuremath{\text{diag}\,}}
\newcommand{\R}[2]{\ensuremath{\mathbb{R}^{#1}_{#2}}}
\newcommand{\Zp}{\ensuremath{\mathbb{Z}_{\geq 0}}}
\newcommand{\K}{\ensuremath{\mathcal{K}}}
\newcommand{\Kinf}{\ensuremath{\mathcal{K}_{\infty}}}
\newcommand{\KL}{\ensuremath{\mathcal{KL}}}
\newcommand{\squeezeup}{\vspace{-3mm}}
\graphicspath{{Figures/}}

\title{\LARGE \bf
Stabilization of nonlinear systems using event-triggered output feedback controllers
}

\author{Mahmoud Abdelrahim, Romain Postoyan, Jamal Daafouz and Dragan Ne\v{s}i\'c\thanks{M. Abdelrahim, R. Postoyan and J. Daafouz are with the Universit\'{e} de Lorraine, CRAN, UMR 7039 and the CNRS, CRAN, UMR 7039, France {\tt\small \{othmanab1,romain.postoyan,jamal.daafouz\}@} {\tt\small univ-lorraine.fr}. Their work is partially supported by the ANR under the grant COMPACS (ANR-13-BS03-0004-02).}
\thanks{D. Ne\v{s}i\'c is with the Department of Electrical and Electronic Engineering, the University of Melbourne, Parkville, VIC 3010, Australia {\tt\small dnesic@unimelb.edu.au}. His work is supported by the Australian Research Council under the Discovery Projects and Future Fellowship schemes.}
}

\begin{document}

\maketitle

\begin{abstract}
The objective is to design output feedback event-triggered controllers to stabilize a class of nonlinear systems. One of the main difficulties of the problem is to ensure the existence of a minimum amount of time between two consecutive transmissions, which is essential in practice. We solve this issue by combining techniques from event-triggered and time-triggered control. The idea is to turn on the event-triggering mechanism only after a fixed amount of time has elapsed since the last transmission. This time is computed based on results on the stabilization of time-driven sampled-data systems. The overall strategy ensures an asymptotic stability property for the closed-loop system. The results are proved to be applicable to linear time-invariant (LTI) systems as a particular case.
\end{abstract}

\section{Introduction}
Networked control systems (NCS) are systems in which the communication between the plant and the controller occurs through a shared digital channel. Since the network has a limited bandwidth and is typically used by other tasks, it is essential to develop communication-aware control strategies. Event-triggered control is a relevant paradigm in this context as it adapts transmissions to the current state of the plant, see \textit{e.g.} \cite{Arzen1999simple, Astrom2002comparison, Tabuada2007event, Romain2011unifying, Heemels2012Introduction} and the references therein. In that way, transmissions only occur when it is needed according to the control objectives.

A fundamental issue in the implementation of event-triggered controllers is to ensure the existence of a minimum amount of time between two consecutive transmissions to respect hardware limitations. This task becomes particularly challenging when we have to design the controller using only an output of the system and not the full state vector (see \cite{Donkers2012output}), in particular when we aim to guarantee asymptotic stability properties. To the best of our knowledge, this problem has been first addressed in \cite{Kofman2006level} and then in \cite{Donkers2012output,Lehmann2011event, Peng2013output,Tallapragada2012event-CDC,Forni2014event,Zhang2013event,Meng2014event} for LTI systems and in \cite{Yu2012event} for nonlinear systems.

In this paper, we design output feedback event-triggered controllers for nonlinear systems which guarantee a (global) asymptotic stability property and the existence of a uniform strictly positive lower bound on the inter-transmission times. The proposed strategy combines the event-triggering condition of \cite{Tabuada2007event} adapted to output measurements and the results on time-driven sampled-data systems in \cite{Nesic2009explicit}. Indeed, the event-triggering condition is only (continuously) evaluated after  units of times have elapsed since the last transmission, where  corresponds to the \textit{maximum allowable sampling period} (MASP) given by \cite{Nesic2009explicit}. This two-step procedure is justified by the fact that the adaption of the event-triggering condition of \cite{Tabuada2007event} to output feedback on its own can lead to Zeno phenomenon (see \cite{Donkers2012output}). Although the rationale is intuitive, the analysis is not trivial as we show in the paper. This type of triggering rule has been used in \cite{Abdelrahim2013Event} to stabilize nonlinear singularly perturbed systems under a different set of assumptions. Similar approaches have been followed in \cite{Forni2014event}, \cite{Mazo2011decentralized}, \cite{Wang2012asynchronous} to enforce a lower bound on the inter-transmission times in different contexts, mainly for linear systems. Note that the idea of enforcing a given time between two jumps is linked to time regularization techniques, see \cite{Johansson1999simulation}.

Our results rely on similar assumptions as in \cite{Nesic2009explicit} which allow us to derive both local and global results. These conditions are shown to be always verified by LTI systems that are stabilizable and detectable, in which case these are reformulated as a linear matrix inequality (LMI). Contrary to \cite{Forni2014event}, the approach is applicable to nonlinear systems and the output feedback law is not necessarily based on an observer. Compared to \cite{Yu2012event}, we rely on a different set of assumptions and we conclude a different stability property. In addition, we show that our results are applicable to any LTI systems that are stabilizable and detectable, which is a priori not the case of \cite{Yu2012event}. Furthermore, we apply our results to Lorenz model, which is nonlinear and which does not satisfy the conditions of \cite{Yu2012event}. Unlike \cite{Tallapragada2012event-CDC}, where LTI systems have been studied, we do not necessarily consider observer-based output feedbacks and the triggering condition does not a priori rely on estimates of the unmeasured states. The latter has the advantage to lighten the implementation since the triggering mechanism only needs to have access to the output of the plant, and not the controller variable. Finally, in the particular case of LTI systems, we conclude a global asymptotic stability property as opposed to ultimate boundedness in \cite{Donkers2012output}. It has to be noted that the event-triggering mechanism that we propose is different from the periodic event-triggered control (PETC) paradigm, see \textit{e.g.} \cite{Heemels2013PETC}, \cite{Romain2013periodic}, where the triggering condition is verified only at some periodic sampling instants. In our case, the triggering mechanism is \textit{continuously} evaluated, once  units of time have elapsed since the last transmission. The first results of this work have been presented in \cite{Abdelrahim2014stabilization}. In comparison to our previous work: we provide all the proofs of the results; we show how the proposed technique can be fruitfully employed in the context of state feedback control as a special case, to directly tune the lower bound on the inter-transmission times; we apply the results on a different physical nonlinear example to better motivate our results and we compare our proposed triggering mechanism with the existing results on a linear output feedback example.



\section{Preliminaries} \label{sec: preliminaries}
Let ,  and . A continuous function  is of class  if it is zero at zero, strictly increasing, and it is of class  if in addition  as . A continuous function  is of class  if for each ,  is of class , and, for each ,  is decreasing to zero. We denote the minimum and maximum eigenvalues of the symmetric matrix  as  and , respectively. We write  to denote the transpose of . We use  to denote the identity matrix of dimension . We write  to represent the vector  for  and . For a vector , we denote by  its Euclidean norm and for a matrix , we denote by . We will consider locally Lipschitz Lyapunov functions (that are not necessarily differentiable everywhere), therefore we will use the generalized directional derivative of Clarke which is defined as follows. For a locally Lipschitz function  and a vector , . For a continuously differentiable function ,  reduces to the standard directional derivative , where  is the (classical) gradient. We will invoke the following result, see Lemma II.1 in \cite{Liberzon2012Lyapunov}.

\begin{lemma}[Lemma II.1 \cite{Liberzon2012Lyapunov}] \label{lma: clarke}
Consider two functions  and  that have well-defined Clarke derivatives for all  and . Introduce three sets , , . Then, for any , the function  satisfies  for all ,  for all  and  for all . \hfill 
\end{lemma}

In this paper, we consider hybrid systems of the following form using the formalism of \cite{Teel}

where  is the state,  is the flow map,  is the flow set,  is the jump map and  is the jump set. The vector fields  and  are assumed to be continuous and the sets  and  are closed. The solutions to system (\ref{eq: Prelim-hybrid-model}) are defined on so-called hybrid time domains. A set  is called a \textit{compact hybrid time domain} if  for some finite sequence of times  and it is a \textit{hybrid time domain} if for all  is a compact hybrid time domain. A function  is a hybrid arc if  is a hybrid time domain and if for each  is locally absolutely continuous on . A hybrid arc  is a solution to system (\ref{eq: Prelim-hybrid-model}) if: (i) ; (ii) for any ,  and  for almost all ; (iii) for every  such that ,  and . A solution  to system (\ref{eq: Prelim-hybrid-model}) is \textit{maximal} if it cannot be extended, \textit{complete} if its domain, , is unbounded, and it is \textit{Zeno} if it is complete and .


\section{Problem statement} \label{sec: problem-statement}
Consider the nonlinear plant model

where  is the plant state,  is the control input,  is the measured output of the plant and we focus on general dynamic controllers of the form

where  is the controller state. We emphasize that the -system is not necessarily an observer. Moreover, (\ref{eq: xdot_c-u}) captures static feedbacks as a particular case by setting . We follow an emulation approach in this paper. Hence, we assume that the controller (\ref{eq: xdot_c-u}) renders the origin of system (\ref{eq: xdot_p-y}) globally asymptotically stable in the absence of network. Afterwards, we take into account the communication constraints and we synthesize the triggering condition. In particular, we consider the scenario where controller (\ref{eq: xdot_c-u}) communicates with the plant via a digital channel. Hence, the plant output and the control input are sent only at transmission instants . We are interested in an event-triggered implementation in the sense that the sequence of transmission instants is determined by a criterion based on the output measurement, see Figure \ref{fig:output-controller}.
\begin{figure}[h!]
\centering \scriptsize
\psfrag{Plant}[][][1]{Plant}
\psfrag{ETC}[][][1]{Event-triggering}
\psfrag{mechanism}[][][1]{mechanism}
\psfrag{Controller}[][][1]{Controller}
\psfrag{yt}[][][1]{}
\psfrag{yk}[][][1]{}
\psfrag{ut}[][][1]{}
\psfrag{uk}[][][1]{}
\includegraphics[width=7cm,height=3cm]{output-controller.eps}
\caption{Event-triggered control schematic \cite{Donkers2012output}}\label{fig:output-controller}
\end{figure}
At each transmission instant, the plant output is sent to the controller which computes a new control input that is instantaneously transmitted to the plant. We assume that this process is performed in a synchronous manner and we ignore the computation times and the possible transmission delays. In that way, we obtain

\noindent where  and  respectively denote the last transmitted values of the plant output and the control input. We assume that zero-order-hold devices are used to generate the sampled values  and , which leads to   and . We introduce the network-induced error , where  and  which are reset to  at each transmission instant.

We model the event-triggered control system using the hybrid formalism of \cite{Teel} as in \cite{Donkers2012output}, \cite{Forni2014event}, \cite{Romain2011unifying}, for which a jump corresponds to a transmission. In that way, the system is modeled as
15pt]
  \left(\begin{array}{c} x^{+}\\ e^{+}\\ \tau^{+}\end{array} \right) &=& \left(\begin{array}{c} x\\ 0\\ 0\end{array} \right) &\hspace{15pt} (x,e,\tau) \in D,
\end{array}
 \label{Vx-bounds-out}
  \underline{\alpha}(|x|) \leq V(x) \leq \overline{\alpha}(|x|),
 \label{V-dot-out}
  \langle\nabla V(x), f(x,e)\rangle \leq -\alpha(|x|) - H^{2}(x) - \delta(y) + \gamma^{2} W^{2}(e)
\label{W-dot-out}
  \langle\nabla W(e), g(x,e)\rangle \leq LW(e) + H(x).
 \label{eq:tig-condn}
  \gamma^{2} W^{2}(e) \leq \delta(y) \text{ or } \tau \in [0,T],
\label{flow-jump-sets-out}
\arraycolsep=1pt\def\arraystretch{1.3}
\begin{array}{lcll}
C &=& \Big\{(x,e,\tau): &\gamma^{2} W^{2}(e) \leq \delta(y) \text{ or } \tau \in [0,T]\Big\} \6pt]
& & & \Big(\gamma^{2} W^{2}(e) \geq \delta(y) \text{ and } \tau = T\Big) \Big\}.
\end{array}
 \label{T-phi}
\mathcal{T}(\gamma, L) := \left\{
    \begin{array}{ll}
    \frac{1}{Lr}\arctan(r)& \hspace{10pt} \gamma > L \\
    \frac{1}{L}& \hspace{10pt} \gamma = L \\
    \frac{1}{Lr}\arctanh(r)& \hspace{10pt} \gamma < L
    \end{array}
    \right.
 \label{eq-thm-clock-out}
|\phi_{x}(t,j)| \leq \beta(|(\phi_{x}(0,0), \phi_{e}(0,0))|, t + j) \hspace{10pt} \forall(t,j) \in \dom \phi,
\label{plant-linear-ex-out}
\dot{x}_{p} = A_{p}x_{p} + B_{p}u, \hspace{30pt} y = C_{p}x_{p},
\label{controller-linear-ex-out}
\dot{x}_{c} = A_{c}x_{c} + B_{c}y, \hspace{30pt} u = C_{c}x_{c} + D_{c}y,
 \label{eq: hybrid-model-linear}
\begin{array}{rllllll}
\left(\begin{array}{c} \dot{x}\\ \dot{e}\\ \dot{\tau}\end{array} \right) &=& \left(\begin{array}{c} \mathcal{A}_{1}x + \mathcal{B}_{1}e\\ \mathcal{A}_{2}x +  \mathcal{B}_{2}e\\ 1\end{array} \right) &\hspace{10pt} (x,e,\tau)\in C \
\begin{flushleft}
\linespread{2.5}\selectfont
where , ,  and .
\end{flushleft}
\vspace{6pt}
We obtain the following result.
\begin{props}\label{prop: linear-systems}
Consider system (\ref{eq: hybrid-model-linear}). Suppose that there exist  and a positive definite symmetric real matrix  such that

where . Then Assumption \ref{ass-clock-out} holds with

\hspace*{\fill} 
\end{props}
\noindent Proposition \ref{prop: linear-systems} provides a sufficient condition, namely (\ref{eq: prop-LMI}), for the verification of Assumption \ref{ass-clock-out}, which thus allows us to apply the results of Section \ref{sec: main-results}. It has to be noted that the LMI (\ref{eq: prop-LMI}) can always be satisfied when system (\ref{plant-linear-ex-out}) is stabilizable and detectable. Indeed, in this case, we can select the controller (\ref{controller-linear-ex-out}) such that  is Hurwitz.  Noting that (\ref{eq: prop-LMI}) is equivalent to the following inequalities, by using the Schur complement of (\ref{eq: prop-LMI}) (see Section A.5.5 in \cite{Boyd}), . We see that we can select the matrix  such that  is negative definite. It then suffices to choose  sufficiently large to ensure the last inequality.


\begin{exmple}\label{sec: illutrative-example}
We apply the result in this section to Example 2 in \cite{Donkers2012output}. We obtain  and we obtain the values , ,  by solving the LMI (\ref{eq: prop-LMI}) using the SEDUMI solver with the YALMIP interface. The guaranteed minimum inter-transmission time is , by using (\ref{T-phi}). Table \ref{tbl:linear-out-emulation} provides the minimum and the average inter-transmission times, respectively denoted as  and , for 100 randomly distributed initial conditions such that  and .  The provided values of  in Table \ref{tbl:linear-out-emulation} indicates that the generated amount of transmissions by our proposed triggering mechanism is approximately 100 times less than the amount given by \cite{Donkers2012output}. Moreover, the stability property achieved in \cite{Donkers2012output} is a practical stability property, while we ensure a global asymptotic stability property. We note that the results in \cite{Yu2012event} are not applicable to this system because condition (3) of Proposition 1 in \cite{Yu2012event} is not satisfied. We believe that the comparison with \cite{Tallapragada2012event-CDC} is not relevant since the triggering mechanism is different and the dynamic controller in \cite{Tallapragada2012event-CDC} is based on an observer.
\hfill 



\begin{table}[H]
\begin{center}
\begin{tabular}{ c | c | c }
            & Donkers \& Heemels \cite{Donkers2012output}  & Our proposed mechanism \\ \hline
  Guaranteed & \multirow{2}{*}{6.5}  &  \multirow{2}{*}{0.017}      \\
  lower bound & & \\ \hline
    &   &  0.017 \\ \hline
    &   & 0.0202
  \end{tabular}
  \caption{Simulation results for 100 randomly distributed initial conditions such that  and  for a simulation time of 20 seconds.}
  \label{tbl:linear-out-emulation}
\end{center}
\end{table}
\end{exmple}

\section{State feedback controllers}\label{sec: linear-state-feedback}
The technique proposed in Section \ref{sec: main-results} is also relevant in the context of state feedback control, \textit{i.e.} when , as the constant  in (\ref{flow-jump-sets-out}) can be used to directly tune the minimum inter-transmission time (up to  in (\ref{T-phi})). It has to be noted that in this case, we can replace  in (\ref{eq:tig-condn}) by  when Assumption \ref{ass-clock-out} holds. The following result is a direct consequence of Theorem \ref{thm: clock-output}.
\begin{cor}
Suppose that Assumption \ref{ass-clock-out} holds and consider system (\ref{eq: hybrid-model}) with  and the flow and jump sets defined as
6pt]
D &=& \Big\{q: &\Big(\gamma^{2} W^{2}(e) = \sigma(\alpha(|x|) + H^{2}(x) + \delta(x)) \text{ and } \tau \geq T\Big) \text{ or } \
where ,  and  is such that . Then, the conclusions of Theorem \ref{thm: clock-output} hold.
\hfill 
\end{cor}

\begin{exmple}
We illustrate the interest of our proposed triggering condition. Consider the LTI system, as in \cite{Tabuada2007event}, , where ,  and . Since the pair () is stabilizable, we take the control input  with  as in \cite{Tabuada2007event}. By following similar lines as in Section \ref{sec: application-linear}, we derive the LMI (\ref{eq: prop-LMI}) with ,  and . Hence, and by solving the resulted LMI, we obtain the numerical values  which lead to . For comparison, we set  and we ran simulations for 200 randomly distributed initial conditions such that  and . Table \ref{tbl:tau-min-avg} provides the generated minimum and average inter-transmission times by both the proposed triggering strategy and the triggering condition in \cite{Tabuada2007event}, \textit{i.e.} with . We note that our proposed mechanism produces larger values of , . To spotlight the effect of the time-triggered part in our proposed triggering mechanism, the enforced lower bound  is plotted in Figures 2, 3 versus the generated inter-transmission times by both our proposed triggering mechanism and the triggering condition in \cite{Tabuada2007event} respectively, for one initial condition. \hfill 

\begin{table}[H]
\begin{center}
  \begin{tabular}{ l | c | c }
                & Our proposed triggering mechanism & Tabuada \cite{Tabuada2007event}  \\ \hline
   & 0.075                                   & 0.0543 \\ \hline
   & 0.0772                                  & 0.0659
  \end{tabular}
  \caption{Minimum and average inter-execution times for 200 initial conditions such that  and  for a simulation time of 10 s.}
  \label{tbl:tau-min-avg}
\end{center}
\end{table}

\begin{figure}[H]
\centering
\psfrag{T}[][][1]{\scriptsize }
\includegraphics[width=\linewidth,height=4cm]{tau-state-T-eq-0075.eps}\caption{Inter-transmission times with .}
\label{fig:tau-state-clock}
\end{figure}

\begin{figure}[H]
\centering
\psfrag{T}[][][1]{\scriptsize }
\includegraphics[width=\linewidth,height=4cm]{tau-state-T-tab.eps} \caption{Inter-transmission times with \cite{Tabuada2007event}.}
\label{fig:tau-state-Tabuada}
\end{figure}
\end{exmple}


\section{Conclusion} \label{sec: conclusions}
We have developed output-based event-triggered controllers for the stabilization of nonlinear systems. The proposed technique ensures an asymptotic stability property and enforces a minimum amount of time between two consecutive transmission instants. The required conditions are shown to be satisfied by any stabilizable and detectable LTI systems. We show in \cite{Abdelrahim2014codesign} that these results can be used as a starting point to address the challenging co-design problem in which the output feedback law is not obtained by emulation but is jointly synthesized with the triggering condition.

\section*{Appendix}
\noindent\textbf{Proof of Theorem \ref{thm: clock-output}.}
First, we prove the result when Assumption \ref{ass-clock-out} holds globally. Let  be the solution to

where ,  for some  and  come from Assumption \ref{ass-clock-out}. We denote  the time it takes for  to decrease from  to . This time  is a continuous function of  which is decreasing in  and  (by invoking the comparison principle). In addition, it holds that  as  tends to  (where  is defined in Section \ref{sec: main-results}), like in \cite{Nesic2009explicit}. As a consequence, since , there exists  such that . We fix the couple . Let . We define for all , . Let , we obtain, in view of (\ref{eq: hybrid-model}) and the fact that  is positive definite,

where . Let  and suppose that . As a consequence it holds that . Indeed,  is strictly decreasing in , in view of (\ref{zeta-dot}), and  as . As a consequence  implies that . Hence,  in view of (\ref{flow-jump-sets-out}) since . Consequently, in view of page 100 in \cite{Teel-Praly-mcss-00}, Lemma \ref{lma: clarke}, Assumption \ref{ass-clock-out} and the definition of the function , , where . Hence, by following similar arguments as in the proof of Theorem 1 in \cite{Nesic2009explicit} since  is continuous and positive definite and  is positive definite and radially unbounded, there exists a continuous positive definite function  such that

When  and , we have . As above, in view of Lemma \ref{lma: clarke}, Assumption \ref{ass-clock-out} and (\ref{zeta-dot}) and by following the same lines as in the proof of Theorem 1 in \cite{Nesic2009explicit}, we obtain . Using the fact that , . Recall that , it holds that . By using the same argument as in (\ref{Rdot1}), we derive that , where . Since  for all  in view of (\ref{zeta-dot}), it holds that . We deduce that there exists a continuous positive definite function  such that . In view of the last inequality, (\ref{Rdot1}) and Lemma \ref{lma: clarke}, when . Consequently, it holds that, for all 

where  is continuous and positive definite. Let  be a solution to (\ref{eq: hybrid-model}), (\ref{flow-jump-sets-out}). In view of (\ref{R-all-clark}) and by definition of the Clarke's derivative (see for instance page 99 in \cite{Teel-Praly-mcss-00}), it holds that, for all  and for almost all  (where )

Thus, in view of (\ref{R-all-D}), (\ref{R-all-C}) and since inter-jump times are lower bounded by  in view of (\ref{flow-jump-sets-out}), we conclude that, by following the same lines as in the end of the proof of Theorem 1 in \cite{Nesic2009explicit}, there exists  such that for any solution  to (\ref{eq: hybrid-model}), (\ref{flow-jump-sets-out}) and any , . In view of Assumption \ref{ass-clock-out} and since  is continuous (since it is locally Lipschitz) and positive definite, there exists  such that  for all  according to Lemma 4.3 in \cite{Khalil}. As a result, in view of Assumption \ref{ass-clock-out}, (\ref{zeta-dot}) and the definition of the function , it holds that, for all , , where . Hence, we deduce that (\ref{eq-thm-clock-out}) holds for any solution  to (\ref{eq: hybrid-model}), (\ref{flow-jump-sets-out}) ad for all , where .

We now investigate the completeness of the maximal solutions to system (\ref{eq: hybrid-model}), (\ref{flow-jump-sets-out}). Let  be a maximal solution to (\ref{eq: hybrid-model}), (\ref{flow-jump-sets-out}). We first show that  is nontrivial, \textit{i.e.} its domain contains at least two points (see Definition 2.5 in \cite{Teel}). According to Proposition 6.10 in \cite{Teel}, it suffices for that purpose to prove that  for any , where  and  is the tangent cone\footnote{The tangent cone to a set  at a point , denoted , is the set of all vectors  for which there exist  with  as  such that  (see Definition 5.12 in \cite{Teel}).} to  at . Let . If  is in the interior of ,  and the required condition holds. If  is not in the interior of , necessarily  as , in this case  and we see that , in view of (\ref{eq: hybrid-model}). Hence,  is nontrivial according to Proposition 6.10 in \cite{Teel}. In view of (\ref{eq: hybrid-model}), (\ref{flow-jump-sets-out}) and (\ref{eq-thm-clock-out}),  and  cannot explode in finite time. Recall that the network-induced error is   with ,  for  and  where we write  with . Hence, in view of (\ref{eq: xdot_p-y}), (\ref{eq: xdot_c-u}), (\ref{eq-thm-clock-out}) and since ,  are continuous, it holds that, for all  and ,
6pt]
                      &\leq& |g_{p}(\phi_{x_{p}}(t_{j},j))| + |g_{p}(\phi_{x_{p}}(t,j))| \
Similarly, we obtain, for all  and 
6pt]
                      &\quad& + |g_{c}(\phi_{x_{c}}(t,j), g_{p}(\phi_{x_{p}}(t_{j},j))| \3pt]
                                                         |z_{2}|\,\leq \,\max|g_{p}(z_{1})|\end{subarray}}{2\hspace{12pt}\max|g_{c}(z_{1},z_{2})|.}
\end{array}

  \langle\nabla W(e), \mathcal{A}_{2}x + \mathcal{B}_{2}e\rangle \leq |\mathcal{A}_{2}x| + |\mathcal{B}_{2}||e|.
 \label{vx-dot}
\begin{array}{lcl}
  \langle\nabla V(x), \mathcal{A}_{1}x + \mathcal{B}_{1}e\rangle &=& x^{T}(\mathcal{A}_{1}^{T}P + P\mathcal{A}_{1})x + x^{T}P\mathcal{B}_{1}e \
By post- and pre-multiplying LMI (\ref{eq: prop-LMI}) respectively by the state vector  and its transpose, we obtain

Expanding the last inequality yields
2pt]
  - x^{T}\mathcal{A}_{2}^{T}\mathcal{A}_{2}x - \varepsilon_{1}x^{T}\overline{C}_{p}^{T}\overline{C}_{p}x + \mu e^{T}e
\end{array}
\label{eq: proof-prop-lmi}
\begin{array}{rrrrr}
  x^{T}(\mathcal{A}_{1}^{T}P + P\mathcal{A}_{1})x + x^{T}P\mathcal{B}_{1}e + e^{T}\mathcal{B}_{1}^{T}Px \leq -\varepsilon_{2}|x|^{2} \
As a result, in view of (\ref{vx-dot}), (\ref{eq: proof-prop-lmi}), condition (\ref{V-dot-out}) is verified with ,  and . Thus, Assumption \ref{ass-clock-out} holds. \hfill 




\bibliographystyle{IEEEtran}
\bibliography{D:/Mahmoud/References}
\end{document}
