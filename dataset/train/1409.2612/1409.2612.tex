\documentclass{article}[12pt]
\usepackage{a4wide}
\usepackage{amsfonts,amsmath,latexsym,times}
\newtheorem{lemma}{Lemma}
\newtheorem{theorem}{Theorem}
\newtheorem{proposition}{Proposition}
\newtheorem{corollary}{Corollary}
\newtheorem{claim}{Claim}
\newenvironment{proof}{\noindent {\bf Proof \ }}{\hfill ~}
\def\N{\mathbb{N}}

\usepackage{color}
\newcommand{\weg}[1]{}
\newtheorem{definition}{Definition}
\newcommand{\hans}[1]{}
\renewcommand{\phi}{\varphi}
\newcommand{\m}{{\mathcal M}}
\newcommand{\imp}{\rightarrow}
\newcommand{\powerset}{\mathcal P}


\begin{document}
\title{A simple proof of the completeness of }
\author{Philippe Balbiani\thanks{Institut de recherche en informatique de Toulouse} \ and Hans van Ditmarsch\thanks{Laboratoire lorrain de recherche en informatique et ses applications}}
\date{\today}
\maketitle
\begin{abstract}
We provide a simple proof of the completeness of arbitrary public announcement logic . The proof is an improvement over the proof found in \cite{balbianietal:2008}.
\end{abstract}
\section{Introduction}
In \cite{balbianietal:2008} Arbitrary Public Announcement Logic () is presented. This is an extension of the well-known public announcement logic \cite{plaza:1989} with quantification over announcements. The logic is axiomatized, but the completeness proof may be considered rather complex. The completeness is shown by employing an infinitary axiomatization, that is then shown to be equivalent (it produces the same set of theorems) to a finitary axiomatization. The completeness proof in \cite{balbianietal:2008} contained an error in the Truth Lemma. The lemma is as follows: \weg{\begin{quote}} {\em Let  be a formula in . Then for all maximal consistent theories
 and for all finite sequences 
of formulas in  such that , , :   iff .} \weg{\end{quote}} The proof is by induction on . The problem is that in expression , the restriction 
of the canonical model  cannot be assumed to exist: although we have assumed that
, , and that , we did not assume that , \dots, and that . The latter would be needed to guarantee that existence. But the induction was only on  and not on , \dots, and  as well. This error has been corrected in \cite{philippe.corrected:2014}, by an expanding the complexity measure used in the Truth Lemma to include the formulas in the sequence , ,  as well. 

Another source of confusion in \cite{balbianietal:2008}, although there was no error involved, concerned the employment of maximal consistent theories (instead of maximal consistent sets, a more common term in modal logic), and a number of properties shown for maximal consistent theories. While repairing the completeness proof, and while also considering additional properties of the canonical model, we found another completeness proof, that the reader may consider more direct and more elegant than the one in \cite{balbianietal:2008,philippe.corrected:2014}. This is presented in this work, including some further results for the canonical model.

\section{Syntax}
Let  be a countable set of atoms (with typical members denoted , , etc) and  be a countable set of agents (with typical members denoted , , etc).

\begin{definition}[Language of APAL]
The set  of all formulas (with typical members denoted , , etc) is inductively defined as follows: 
\begin{itemize}
\item .
\end{itemize}
\end{definition}
We define the other Boolean constructs as usual.
The formulas ,  and  are obtained as abbreviations:  for ,  for  and  for .
We adopt the standard rules for omission of the parentheses.
Given a formula , the set of all subformulas of  is denoted by  (an elementary inductive definition is omitted).
We will say that a formula  is {\em -free} iff  contains no formula of the form .
A formula  is said to be -free iff  contains no formula of the form .
We will say that a formula  is {\em epistemic} iff  is both -free and -free.
The set  is the set of all -free formulas.
The set  is the set of all epistemic formulas.

Of crucial importance in the completeness proof is a proper complexity measure on formulas. The one we need is based on a partial order  providing a weighted count of the number of symbols, and on a partial order  counting the number of stacked  operators in a formula.

\begin{definition}[Size]
The size of a formula , in symbols , is the non-negative integer inductively defined as follows:
\begin{itemize}
\item ,
\item ,
\item ,
\item ,
\item ,
\item ,
\item .
\end{itemize}
The -depth of a formula , in symbols , is the non-negative integer inductively defined as follows:
\begin{itemize}
\item ,
\item ,
\item ,
\item ,
\item ,
\item ,
\item .
\end{itemize}
We define the binary relations ,  , and  between formulas in the following way:
\begin{itemize}
\item  iff .
\item  iff .
\item  iff either , or  and .
\end{itemize}

\end{definition}
The next two lemmas combine a number of results on these binary relations.
Their proofs are obvious and have been omitted.
\begin{lemma} \label{gather.lemma} Let  be formulas.
\begin{itemize}
\item \label{lem_1b}
 is a well-founded strict partial order between formulas.
\item \label{lem_1c}
 is a well-founded strict partial order between formulas.
\item \label{lem_1}
 is a well-founded strict partial order between formulas.
\item \label{lem_1bh}
If  then .
\item \label{lem_hans} If  then .
\item \label{lem_1a}
If  is epistemic, then .
\item If  is epistemic, then .
\end{itemize}
\end{lemma}





\begin{lemma}\label{lem_2}
Let  be formulas and .
\begin{enumerate}
\item ,
\item ,
\item .
\end{enumerate}
\end{lemma}

The relation  has been tailored in order to ensure exactly the properties of Lemma~\ref{lem_2}.
Without the curious factor  in  these properties would not hold.
Given the previous lemmas, we can now list all the cases later used in the Truth Lemma.
\begin{corollary} In cases  and ,  is epistemic.

\end{corollary}
\begin{definition}[Necessity form]
Now, let us consider a new atom denoted .
The set  of {\em necessity forms} (with typical members denoted , , etc) is inductively defined as follows---where  is a formula.
\begin{itemize}
\item .
\end{itemize}
\end{definition}
\section{Semantics}


We introduce the structures and give a semantics for the logical language on these structures. The material in this section (as also the logical language in the previous section, and the axiomatization in the next section) is as in \cite{balbianietal:2008}.

\begin{definition}[Model] \label{def.model}
A model  consists of a nonempty {\em domain} , an {\em accessibility function}  associating to each  an equivalence relation  on , and a {\em valuation function}  --- where  denotes the valuation of atom .
For , we write .
\end{definition}

\begin{definition}[Semantics]
Assume a model . We inductively define the truth set .
 
where model  is such that 
\end{definition} 
\section{Axiomatization}
An axiomatic system consists of a collection of axioms and a collection of inference rules.
Let us consider the following axiomatic system:
\begin{definition}[Axiomatization ]
\begin{description}
\item[] all instantiations of propositional tautologies,
\item[] ,
\item[] ,
\item[] ,
\item[] ,
\item[] ,
\item[] ,
\item[] ,
\item[] ,
\item[] ,
\item[] ,
\item[] ,
\item[] ,
\item[] if  is epistemic, then ,
\item[] ,
\item[] ,
\item[] ,
\item[] , \item[] :  is epistemic.
\end{description}


\bigskip

\noindent
Let  be the least subset of  containing -- and closed under --. An element of  is called a {\em theorem}.
\end{definition}
In \cite{balbianietal:2008} other (finitary) axiomatizations are also given, that are then shown to be equivalent to  (they define the same set of theorems as ). For the completeness proof, we have chosen the most convenient form, with the infinitary rule .  Some of the axioms and rules in the axiomatization  are derivable from the other axioms and rules, again, see \cite{balbianietal:2008} for details. It concerns the following rules and axioms (where  should be seen as the abbreviation of ):
\begin{description}
\item[] ;
\item[] ; 
\item[] . \end{description}
\section{Canonical model}
\begin{definition}[Theory]
A set  of formulas is called a theory iff it satisfies the following conditions:
\begin{itemize}
\item  contains ,
\item  is closed under  and .
\end{itemize}
A theory  is said to be {\em consistent} iff .
A set  of formulas is {\em maximal} iff for all formulas ,  or . 
\end{definition}
Obviously, the smallest theory is  whereas the largest theory is .
The only inconsistent theory is .
The reader may easily verify that a theory  is consistent iff for all formulas ,  or .
Moreover, for all maximal consistent theories ,
\begin{itemize}
\item ,
\item  iff ,
\item  iff  or .
\end{itemize}
Theories are closed under  and  but not under the derivation rules , , and  for a specific reason. Obviously, by definition, all derivation rules preserve theorems. Semantically, we could say that they all preserve validities. Now, unlike , , and , the derivation rules  and  also preserve truths. That is the reason!
In the setting of our axiomatization based on the infinitary rule (R4), we will say that a set  of formulas is consistent iff there exists a consistent theory  such that .
Obviously, maximal consistent theories are maximal consistent sets of formulas. Under the given definition of consistency for sets of formulas, maximal consistent sets of formulas are also maximal consistent theories.
\begin{definition}
For all formulas  and for all , let 
\end{definition}
The proofs of the following lemmas can be found in~\cite{balbianietal:2008} (Lemmas~ and~).
\begin{lemma}\label{lem_7}
Let  be a formula and .
For all theories ,
\begin{itemize}
\item  is a theory containing  and ,
\item  is a theory,
\item  is a theory.
\end{itemize}
\end{lemma}
\begin{lemma}\label{lem_7bis}
Let  be a formula.
For all theories ,  is consistent iff .
\end{lemma}
\begin{lemma}\label{lem_9}
Each consistent theory can be extended to a maximal consistent theory.
\end{lemma}
The proof of the next lemma uses axioms --.
\begin{lemma}\label{equivalence}
Let .
For all maximal consistent theories ,
\begin{itemize}
\item ,
\item if  and , then ,
\item if , then .
\end{itemize}
\end{lemma}
Next lemma is usually called ``Diamond Lemma''.
Its proof is very classical and uses Lemmas~\ref{lem_7}, \ref{lem_7bis} and~\ref{lem_9}.
\begin{lemma}\label{diamond_lemma}
Let  be a formula and .
For all theories , if , then there exists a maximal consistent theory  such that  and .
\end{lemma}
The next three lemmas were not found in \cite{balbianietal:2008}.
\begin{lemma}\label{lem_mcs}
Let  be a formula.
For all maximal consistent theories , if , then  is a maximal consistent theory.
\end{lemma}
\begin{proof}
Suppose .
If  is not consistent, then .
Hence, .
Thus, .
Since  is consistent, : a contradiction.
If  is not maximal, then there exists a formula  such that  and .
Therefore,  and .
Since  is maximal,  and .
Consequently, .
Hence, using , .
Since  is consistent, .
Since , .
Thus, : a contradiction.
\end{proof}
\begin{lemma}\label{lem_in_diamond}
Let  be formulas.
For all maximal consistent theories ,  iff  and .
\end{lemma}
\begin{proof}
 Suppose .
Hence, .
Thus, using , .
By Lemma~\ref{lem_mcs},  is a maximal consistent theory.
Suppose .
Since  is maximal, .
Therefore, .
Consequently, .
Since  is consistent, : a contradiction.
\\
 Suppose  and .
By Lemma~\ref{lem_mcs},  is a maximal consistent theory.
Suppose .
Since  is maximal, .
Hence, .
Thus, .
Since  is consistent, : a contradiction.
\end{proof}
\begin{lemma}\label{commutatif}
Let  be a formula and .
For all theories , if , then .
\end{lemma}
\begin{proof}
Suppose .
For all formulas , the reader may easily verify that the following conditions are equivalent:
\begin{enumerate}
\item ,
\item ,
\item ,
\item ,
\item ,
\item ,
\item .
\end{enumerate}
\end{proof}
\begin{definition}[Canonical model]
The {\em canonical model}  is defined as follows:
\begin{itemize}
\item  is the set of all maximal consistent theories;
\item  is the function assigning to each agent  the binary relation  on  defined as 
\item  is the function assigning to each atom  the subset  of  defined as 
\end{itemize}
\end{definition}
It will be clear that the canonical model is a model according to Definition~\ref{def.model}.  By Lemma~\ref{lem_9},  is a non-empty set, and by Lemma~\ref{equivalence} the binary relation  is an equivalence relation on  for each agent .
\section{Completeness}
The main result of this Section is the proof of 's Truth Lemma (Lemma~\ref{equ_sem}).
This proof is different from and simpler than the proof presented in \cite{balbianietal:2008}.

\begin{definition}
Let  be a formula. Condition  is defined as follows.  \begin{quote} For all maximal consistent theories ,   iff . \end{quote}
Condition  is defined as follows. \begin{quote} For all formulas , if , then . \end{quote}
\end{definition}

Our new proof of 's Truth Lemma is done by using an -induction on formulas.
More precisely, we will demonstrate that

\begin{lemma}\label{induction_bb}
For all formulas , if , then .
\end{lemma}
\begin{proof}
Suppose .
Let  be a maximal consistent theory.
We consider the following  cases.


\medskip \noindent 
{\bf Case .}
\hans{Chang\'e}  holds, as  iff , by the definition of the canonical model and the semantics of propositional atoms. 

\medskip  \noindent 
{\bf Case .}
 holds, as  and , by the definition of the canonical model and the semantics of . 

\medskip  \noindent 
{\bf Case .}
The reader may easily verify that the following conditions are equivalent.
The induction using  is used between step 2.\ and step 3. A similar inductive argument is also used in all following cases. 
\begin{enumerate}
\item ,
\item ,
\item ,
\item .
\end{enumerate}
Hence,  iff .


\medskip \noindent 
{\bf Case .}
The reader may easily verify that the following conditions are equivalent:
\begin{enumerate}
\item ,
\item , or ,
\item , or ,
\item .
\end{enumerate}
Hence,  iff .


\medskip \noindent 
{\bf Case .} The reader may easily verify that the following conditions are equivalent.
The implication from step 2.\ to step 1.\ is by Lemma \ref{diamond_lemma}.
\begin{enumerate}
\item ,
\item for all maximal consistent theories , if , then ,
\item for all maximal consistent theories , if , then ,
\item .
\end{enumerate}
Hence,  iff .


\medskip



\medskip \noindent 
{\bf Case .}
The reader may easily verify that the following conditions are equivalent. Between step 1.\ and step 2., use axiom  , so that   iff   (similar justifications apply in the other cases of form ).
\begin{enumerate}
\item ,
\item , or ,
\item , or ,
\item .
\end{enumerate}
Hence,  iff .


\medskip \noindent 
{\bf Case .}
The reader may easily verify that the following conditions are equivalent:
\begin{enumerate}
\item ,
\item ,
\item ,
\item .
\end{enumerate}
Hence,  iff .


\medskip \noindent 
{\bf Case .}
The reader may easily verify that the following conditions are equivalent. In the crucial equivalence between step 2.\ and 3.\ we use that , a consequence of Lemma \ref{lem_2} (the  depth is the same for both formulas). 
\begin{enumerate}
\item ,
\item , or ,
\item , or ,
\item .
\end{enumerate}
Hence,  iff .


\medskip \noindent 
{\bf Case .}
The reader may easily verify that the following conditions are equivalent:
\begin{enumerate}
\item ,
\item , or ,
\item , or 
\item .
\end{enumerate}
Hence,  iff .


\medskip \noindent 
{\bf Case .}
The reader may easily verify that the following conditions are equivalent (again, a crucial step is between 2.\ and 3.\, where we can use induction on  because of Lemma \ref{lem_2}): 
\begin{enumerate}
\item ,
\item , or ,
\item , or ,
\item .
\end{enumerate}
Hence,  iff .


\medskip \noindent 
{\bf Case .}
The reader may easily verify that the following conditions are equivalent (and once more, a crucial step is between 2.\ and 3.\, where we use Lemma \ref{lem_2}): 
\begin{enumerate}
\item ,
\item ,
\item ,
\item .
\end{enumerate}
Hence,  iff .


\medskip \noindent 
{\bf Case .}
The reader may easily verify that the following conditions are equivalent. Between 1.\ and 2., we use derivation rule  on the necessity form  and closure of maximal consistent sets under . Between step 2.\ and step 3.\, we use the complexity measure , where we now simply observe that  contains one  less than . Between step 3.\ and step 4., we use the semantics of arbitrary announcements  and of announcements : we note that   is by the semantics equivalent to:  implies .
\begin{enumerate}
\item ,
\item for all epistemic formulas , ,
\item for all epistemic formulas , ,
\item .
\end{enumerate}
Hence,  iff .


\medskip \noindent 
{\bf Case .}
The reader may easily verify that the following conditions are equivalent.
The equivalence between step 2.\ and step 3.\ follows from the fact that for all epistemic formulas , .
\begin{enumerate}
\item ,
\item for all epistemic formulas , ,
\item for all epistemic formulas , ,
\item .
\end{enumerate}
Hence,  iff .
\end{proof}
\begin{lemma}[Truth Lemma] \label{equ_sem}
Let  be a formula.
For all maximal consistent theories ,
\begin{itemize}
\item  iff .
\end{itemize}
\end{lemma}
\begin{proof}
By Lemma~\ref{induction_bb}, using the well-foundedness of the strict partial order  between formulas.
\end{proof}


\medskip

\noindent Now, we are ready to prove the completeness of .
\begin{proposition}\label{pro_complete}
For all formulas , if  is valid, then .
\end{proposition}
\begin{proof}
Suppose  is valid and .
By Lemmas~\ref{lem_7}, \ref{lem_7bis} and~\ref{lem_9}, there exists a maximal consistent theory  containing .
By Lemma~\ref{equ_sem}, .
Thus, .
Therefore, .
Consequently,  is not valid: a contradiction.
\end{proof}
\section{Conclusion}
We have provided an alternative, simpler, completeness proof for the logic . The proof is considered simpler, because in the crucial Truth Lemma we do not need to take finite sequences of announcements along. Instead, it can proceed by -induction on formulas.
We consider this result useful, as the completeness proofs of various other logics employing arbitrary announcements or other forms of quantifiying over announcements may thus also be simplified, and as it may encourage the developments of novel logics with quantification over announcements. We acknowledge useful discussions on the completeness of  with Jie Fan, Wiebe van der Hoek, and Barteld Kooi.

\bibliographystyle{plain}
\bibliography{biblio2014}
\end{document}
