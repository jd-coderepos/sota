\documentclass[10pt, conference]{ieeeconf}
\IEEEoverridecommandlockouts
\overrideIEEEmargins
\usepackage{enumerate}
\usepackage{array}
\usepackage{amsfonts,amssymb,amsxtra,balance}
\usepackage{mathrsfs}
\usepackage{calc,ifthen}
\usepackage{mathrsfs}
\usepackage{graphicx}
\usepackage{amssymb}


\newcommand{\mr}{\mathrm}

\newtheorem{condition}{Condition}
\newtheorem{assumption}{Assumption}
\newtheorem{Theorem}{Theorem}
\newtheorem{proposition}{Proposition}
\newtheorem{corollary}{Corollary}
\newtheorem{definition}{Definition}
\newtheorem{lemma}{Lemma}
\newtheorem{remark}{Remark}
\newtheorem{fact}{Fact}



\newcommand{\Gstas}{G_2}


\newcommand{\sym}{ \mbox{sym} }
\newcommand{\col}{ \mbox{col} }
\newcommand{\bfidot}[1]{\!\stackrel{\dot {\frown}} {\bf i}_{\!#1}}
\newcommand{\idot}[1]{\stackrel{\dot {\frown}} {i}_{\!#1}}
\def\cala{{\bf Q}}
\def\calA{{\cal A}}
\def\calJ{{\cal J}}
\def\cali{{\cal I}}
\def\cald{{\cal D}}
\def\calM{{\cal M}}
\def\hal{{1 \over 2}}
\def\bfq{{\bf q}_C}
\def\bfm{{\bf M}}
\def\bfv{{\bf v}_}
\def\bfi{{\bf i}_}
\def\bfc{{\bf C}}
\def\bfl{{\bf L}}
\def\Linf{{\cal L}_\infty}
\def\Lt{{\cal L}_2}
\def\L2e{{\cal L}_{2e}}
\def\bul{\noindent }
\def\rea{\mathbb{R}}
\def\sign{\mbox{sign}}
\def\htheta{\hat{\theta}}
\def\diag{\mbox{diag}}
\def\blodia{\mbox{block\;diag}}
\def\rank{\mbox{rank}}

\def\calc{{\mathcal C}}
\def\calp{{\mathcal P}}
\def\cala{{\mathcal A}}
\def\calb{{\mathcal B}}
\def\cale{{\mathcal E}}
\def\call{{\mathcal Q}}
\def\calr{{\mathcal R}}
\def\calj{{\mathcal J}}
\def\calm{{\mathcal Q}_x}
\def\calv{{\mathcal Q}_{\omega_m}}
\def\caln{{\mathcal N}}
\def\calt{{\mathcal T}}
\def\calq{{\mathcal Q}}
\def\hal{{1 \over 2}}
\def\C{\mathrm{C}}
\def\I{\mathrm{I}}

\def\cas{{\cal C}}
\def\pit{\pi_\theta}
\def\begequarrs{}
\def\begequarr{}
\def\begarr{\begin{array}}
	\def\endarr{\end{array}}
\def\begequ{}
\def\lab{\label}
\def\begdes{\begin{description}}
	\def\enddes{\end{description}}
\def\begenu{\begin{enumerate}}
	\def\begite{\begin{itemize}}
		\def\endite{\end{itemize}}
	\def\endenu{\end{enumerate}}
\def\rem{\vspace{0.2cm} \noindent {\bf Remark }}
\def\lef[{\left[\begin{array}}
	\def\rig]{\end{array}\right]}
\def\qed{\hfill}
\def\begcen{\begin{center}}
	\def\endcen{\end{center}}
\def\begrem{\begin{remark}\rm}
	\def\endrem{\end{remark}}
\def\begcas{\begin{cases}}
	\def\endcas{\end{cases}}

\title{\LARGE \bf {Robust PI Passivity--based Control of Nonlinear Systems: Application to Port--Hamiltonian Systems and Temperature Regulation}}\date{}
\author{S. Aranovskiy R. Ortega and R. Cisneros\thanks{S. Aranovskiy is with the Department of Control Systems And Informatics, Saint-Petersburg University ITMO :  {\tt\small s.aranovskiy@gmail.com}}\thanks{R. Ortega and R. Cisneros are with the LSS-Supelec, 3, Rue Joliot-Curie, 91192 Gif--sur--Yvette, France : {\tt\small ortega@lss.supelec.fr} and {\tt\small rafael.cisneros@lss.supelec.fr} }}
\begin{document}
\maketitle
\thispagestyle{empty}
\pagestyle{empty}
	
\begin{abstract}
This paper deals with the problem of control of partially known nonlinear systems, which have an open--loop stable equilibrium, but we would like to add a {\em PI controller} to regulate its behavior around another operating point. Our main contribution is the identification of a class of systems for which  a globally stable PI can be designed knowing  {\em only} the systems input matrix and measuring {\em only} the actuated coordinates. The construction of the PI is done invoking passivity theory. The difficulties encountered in the design of {\em adaptive} PI controllers with the existing theoretical tools are also discussed. As an illustration of the theory, we consider port--Hamiltonian systems and a class of thermal processes.
\end{abstract}




\section{INTRODUCTION}
\lab{sec1}	
In many practical applications the plant to be controlled has an open--loop stable equilibrium, {\em e.g.}, at the origin, and we would like to add a controller to regulate its behavior around another operating point. In the case of linear systems the dynamics remains invariant under coordinate shifts, therefore this task can be easily accomplished using the incremental model of the plant. Unfortunately, this is not the case for nonlinear systems, for which there is no obvious advantage of working with the incremental model.

Another common requirement in applications is the use of standard proportional-integral (PI) controllers, which overwhelmingly dominate the industrial market. Although commissioning a PI to operate around a single operating point is relatively easy, the performance will be below par in wide operating regimes, which is the scenario in modern high--performance applications. To overcome this drawback the current practice is to re--tune the gains of the PI controllers based on a linear model of the plant evaluated at various operating points, a procedure known as gain--scheduling. There are several  disadvantages of gain--scheduling including the need to switch (or interpolate) the controller gains  and the non--trivial definition of the regions in the plants state space where the switching takes place---both problems are exacerbated if the dynamics of the plant is highly nonlinear. Another common commissioning procedure is to use auto--tuners,  that heavily rely on the availability of a ``good" linear approximation of the plant dynamics. To avoid the need to rely on linearization it is necessary to develop a procedure to design robust PI controllers for nonlinear systems with uncertain parameters.

Motivated by the discussion above in this paper we identify a class of (input affine) nonlinear systems for which it is possible to design a PI controller with the following features.
\begite
\item[F1] Regulation of the closed--loop system around the desired (non--zero) operating point should be guaranteed.
\item[F2] The PI controller should be {\em robust}, in the sense that reduced knowledge of the system parameters is required.
\item[F3] To simplify the controllers commissioning, a well defined admissible range of variation for the PI proportional and integral gains, preserving closed--loop stability, should be provided.
\endite
We propose the construction of a PI controller with the features F1--F3 for plants with unknown dynamics verifying the following assumptions.
\begite
\item[A1] The open--loop system is {\em unknown} but has a stable equilibrium at the origin.
\item[A2] The desired equilibrium belongs to the assignable set and admits a {\em convex} Lyapunov function.
\item[A3] The Lyapunov function is the sum of two functions, depending on the un--actuated and actuated coordinates, respectively. The first function is {\em unknown} while the second one is separable and linearly parameterized in terms of some {\em unknown parameters}.
\item[A4] The input matrix is constant, known and has  zero rows, where  and  are the dimensions of the state  and  input vectors, respectively.
\endite
As indicated in the article's title we exploit the fundamental property of {\em passivity} to design the proportional-integral passivity based controller, which will be referred in the sequel as PI--PBC. The first step in the design is to, relying on A1 above, invoke the celebrated theorem of Hill and Moylan \cite{VAN} to identify a suitable passive output for the system,  with storage function the Lyapunov function of the open--loop system. Since our interest is the regulation of non--zero equilibria, we then use the results of \cite{JAYetal} to create a new passive output for the incremental model with a storage function that has a minimum at the desired equilibrium. As shown in \cite{JAYetal}, feeding back the passive output through a PI controller ensures stability of the desired equilibrium {\em for all} positive definite PI gains. It is important to underscore that, since the passivity property has been established for the incremental model, the equilibrium can also be stabilized setting the control input equal to the (constant) value that assigns the equilibrium, say , which is univocally defined. However, this open--loop control will, obviously, be non--robust. In the robustness context of the present paper neither the plant dynamics nor the Lyapunov function are known and, consequently, we cannot compute neither the passive output nor . It is at this point that we invoke A3 and A4 above to prove that, under these assumptions, it is possible to define ranges for the proportional and integral gains that make the PI--PBC implementable and, consequently, guarantee stability of the equilibrium. Another important feature of the proposed PI--PBC is that it requires only partial measurement of the state, namely, only the  state variables associated to the non--zero rows of the input matrix, referred in the sequel as {\em actuated} coordinates.

Several practical applications of PI--PBC have been reported in the literature. This include, RLC circuits \cite{CASetal}, power converters \cite{HERetal}, fuel cells \cite{TALetal}, electric drives \cite{MARALE} and mechanical systems \cite{SANetal}. In \cite{DONJUN} a procedure to add an integral action to a non--passive output for a class of port--Hamiltonian systems was first proposed, and later extended in \cite{ORTROM}, \cite{ROMDONORT}. To the best of our knowledge, the present paper is the first attempt to design PI--PBCs with guaranteed stability properties for systems with partially known dynamics.

A natural question that arises at this point is the incorporation of adaptation in the design of the PI (or PID). In the  power converter application of \cite{HERetal} a parameter that enters in the definition of the passive output, {\em i.e.}, the load resistance, is adaptively identified---however, all other parameters are assumed to be known. In the interesting paper \cite{ANTAST} it is shown that it is possible to adaptively estimate  for a general nonlinear system with scalar input, keeping the estimate in a known interval, provided the passive output is known. In spite of a large number of publications the problem of designing a {\em provably stable} adaptive PID for systems with unknown parameters remains, as far as we know, open. The difficulty of this task was identified already in 1984 in \cite{ORTKEL}. As is well--known  \cite{SASBOD}, the stability of indirect adaptive methods relies on parameter convergence that, in its turn, requires persistency of excitation---a property that is not satisfied in the regulation tasks where PI control is effective. On the other hand, the application of direct methods is stymied by the absence of a suitable parameterization of this structure--constrained controller. For the PI--PBC studied in this paper the main difficulty is the need to estimate two objects, that appear multiplicatively in the Lyapunov analysis: the passive output and the ideal control signal  . This point is further elaborated in Subsection \ref{subsec52}.

The remaining of the paper is organized as follows. Section \ref{sec2} presents the problem formulation. To streamline the presentation of the main result, which is given in  Section \ref{sec4}, some preliminary lemmata are given in Section \ref{sec3}. In Section \ref{sec5} we discuss the reasons that stymie the use of adaptation and the inability to state a robustness result based on continuity and approximate prior knowledge of the plant. Section \ref{sec6} is devoted to application of the proposed PI--PBC for port--Hamiltonian (pH) systems \cite{VAN}  and a temperature regulation problem. The paper is wrapped--up with concluding remarks in  Section \ref{sec7}.
\vspace{2mm}

\noindent {\bf Notations}  is the  identity matrix and  is an
 matrix of zeros,  is an --dimensional column vector of zeros. Given , we denote with  the --dimensional column vector and  the diagonal  matrix with
elements . For ,   is the Euclidean norm. For mappings of scalar argument ,  and  denote first and second order differentiation, respectively. For mappings ,  and . For the distinguished element  and any mapping  we denote  and the error signal  . 
\section{PROBLEM FORMULATION}
\lab{sec2}	
In this section we formulate the control problem addressed in the paper, enunciate the assumptions made on the plant to solve it and make some remarks on these assumptions.
\subsection{Robust PI control problem}
\lab{subsec21}	
Consider the nonlinear, input affine, system

where , , ,  is an {\em unknown} smooth mapping,  is {\em constant} verifying .

The following is a key assumption made throughout the paper.

\begin{assumption} 
\label{as:g}
The matrix  has  zero rows. Without lost of generality\footnote{See R6 in the next subsection and Subsection \ref{subsec51} for more general forms of .} it is assumed of the form
\begequ
\lab{g}
	G=\begin{bmatrix} \mathbf{0}_{(n-m)\times m} \\ G_2\end{bmatrix},
\endequ
where  is {\em known}.
\end{assumption}

This assumption can be easily {\em obviated} introducing state and input changes of coordinates. Indeed, it is well--known---see, {\em e.g.}, Theorem 2 of Section 2.7 of \cite{LANTIS}---that for any full rank, matrix  there exists (elementary) full rank matrices  and   such that

Consequently, introducing  and  the system \eqref{systform} takes the desired form

where . We should note, however, that a change of state representation destroys---in general---the {\em original structure} of the system, whose knowledge may be critical for the verification of the second assumption below. This fact is clearly illustrated in the physical examples considered in Section \ref{sec6}. For this reason, we prefer to leave it as an standing assumption. 

Motivated by Assumption \ref{as:g} we find convenient to define a partition of the state vector into its un--actuated and actuated components as

It is assumed that {\em only  is available for measurement.}

The unforced system, that is, ,
has a stable equilibrium at the origin with a {\em partially known} Lyapunov function.  We are interested in controlling the system with a PI at a non--zero equilibrium---a situation that arises in most practical applications.
Thus, we are given a desired equilibrium point, , and the control goal is to ensure {\em stability} of this equilibrium using a PI control law of the form
\begequarrs
\dot{z} &= - K_I \psi(x_a,x^*)\\
u &= - K_P  \psi(x_a,x^*) + z,		
\endequarrs
where  is the controller state,  and  are tuning gains and  is a mapping designed with the partial knowledge of the aforementioned Lyapunov function.

The following, practically reasonable, assumption is made throughout the paper.
\begin{assumption} 
\label{as:xstar}
The desired equilibrium point  belongs to the assignable equilibrium set, that is,

\end{assumption}
\subsection{Assumptions on the open--loop plant}
\lab{subsec22}
The following assumption identifies the class of vector fields  for which we provide an answer to the problem.\\

\begin{assumption} 
\label{as:H}
For the system \eqref{systform} there exists a twice--differentiable, positive definite function , verifying the following.\\
\begin{enumerate}[(i)]
	\item .\\
	\item .\\
	\item The function  is of the form

with
\begequ
\lab{h}
H_a(x_a) = \sum_{i=n-m+1}^n{d_i \phi_i(x_i)},
\endequ
where the function  and the constants  are {\em unknown} but the  the functions   are {\em known}.\\
\item The functions  and   are {\em convex}.
\end{enumerate}
\end{assumption}
\subsection{Discussion}
\lab{subsec23}
The following remarks regarding the assumptions are in order.

\begite
\item[R1] Although the set  is not known, it is reasonable to assume that we have enough prior knowledge about the plant to select the desired operating point as a feasible equilibrium. Hence, Assumption \ref{as:xstar} is reasonable.

\item[R2] A corollary of Assumption \ref{as:xstar}  is that  the constant input , that assigns the equilibrium, is uniquely defined as

Notice that, without knowledge of , this constant cannot be computed.

\item[R3] Since the open--loop system \eqref{systform} has a stable equilibrium at the origin Assumption \ref{as:H} (i) follows as a corollary of Lyapunov's converse theorems \cite{KHA}. As will become clear below Assumption  \ref{as:H} (ii) and (iv) are required to prove passivity of the incremental model as done in \cite{JAYetal}. 

\item[R4] We underscore that no assumption, beyond twice differentiability and convexity, is imposed on the unknown component  of the  Lyapunov function of the open--loop system . On the other hand, stricter conditions are imposed on the second component , with uncertainty captured by the unknown constants .

\item[R5] Assumptions  \ref{as:H} (iii) and Assumption  \ref{as:g} are the key requirements imposed on the plant to design the robust PI--PBC. This assumption is satisfied by a large class of physical systems, including a class of port--Hamiltonian \cite{VAN}  and thermal systems studied in Section \ref{sec6}.

\item[R6] Regarding Assumptions  \ref{as:g}, in the more general case when  is not of the form \eqref{g} an additional shuffling of the rows of  is needed in the design. This procedure is explained in  Subsection \ref{subsec51}.

\item[R7] For quadratic Lyapunov functions of the form , with , Assumption  \ref{as:H} (ii) is satisfied if the open--loop system is {\em convergent} in the sense of Demidovich \cite{PAVetal}. That is, if it satisfies
  
\endite
\section{PRELIMINARY LEMMATA}
\lab{sec3}	
Unless otherwise indicated, throughout the rest of the paper Assumption \ref{as:g} holds. Define for the system \eqref{systform} the output

where 
\begequarrs
D & := & \lef[{cccc} d_{n-m+1} & 0 & \dots & 0 \\ 0 & d_{n-m+2} & \dots & 0 \\ \vdots & \vdots & \vdots & \vdots \\ 0 & 0 & \dots & d_n \rig]>0 \\
\Phi(x_a) & :=  & \lef[{c} \phi'_{n-m+1}(x_{n-m+1}) \\ \vdots \\ \phi'(x_n) \rig].
\endequarrs
A corollary of the theorem of Hill and Moylan \cite{VAN} is that, if Assumption \ref{as:H} (i) holds, the system \eqref{systform}, \eqref{eq:youtput} defines a passive mapping  with storage function .

To operate the system at a non--zero equilibrium it is necessary to shift the minimum of the storage function and define the passivity property between the incremental input and the output error. Towards this end, we
recall Proposition 1 of \cite{JAYetal} and state it as a lemma below.
\begin{lemma} 
\lab{lem1}
Consider the incremental model of the system \eqref{systform}, \eqref{eq:youtput}

where  is the incremental input. Under Assumptions \ref{as:g}--\ref{as:H} the mapping   is passive with storage function  given by

where  is a constant that ensures . More precisely,
\begequ
\lab{doth0}
\dot{U} = -Q(x) + e^\top \tilde{u},
\endequ
where  is defined in Assumption \ref{as:H} (ii).
\end{lemma}

One of the main interests of passive systems is that they can be globally stabilized with PI controls (with arbitrary positive definite gains). This well--known fact is stated in the lemma below, whose proof is given to streamline the presentation of our main result.

\begin{lemma} 
\lab{lem2}
Consider the system \eqref{systform} verifying Assumptions \ref{as:g}--\ref{as:H} in closed--loop with the PI--PBC
\begequ
\lab{u}
\begin{aligned}
                 e&=G_2^\top D \widetilde {\Phi}(x_a)\\
                 \dot{z} &= - K_I e\\
		u &= - K_P e + z.
	\end{aligned}

Consider the Lyapunov function candidate
\begequ
\lab{w}
	W(\tilde{z},x) = U(x) + \frac{1}{2}\tilde{z}^\top \Lambda_I\tilde{z},
\endequ
where . The time derivative of the Lyapunov function along the trajectories of the closed--loop system is

Setting  yields

The proof is complete invoking standard Lyapunov arguments \cite{KHA}.
\end{proof}
\section{THE ROBUST PI--PBC}
\lab{sec4}	
As indicated in R4 of Subsection \ref{subsec23} the matrix  is unknown. Hence, the error signal  cannot be constructed and the PI--PBC  \eqref{u} is not implementable.	
This motivates our main result given below.
\begin{proposition} \lab{pro1}
Consider system \eqref{systform} verifying Assumptions 1--3 in closed--loop with the robust PI--PBC

with the controller gains
\begequarr
\nonumber
K_P & = & G_2^{-1} \Gamma_P \\
K_I & = & G_2^{-1} \Gamma_I.
\lab{gaipi}
\endequarr
{\em For all} diagonal, positive definite matrices   and   we have the following.
\begite
\item[(i)] All trajectories are bounded and the equilibrium point  is {\em globally stable} (in the sense of Lyapunov).
\item[(ii)] The augmented error signal  defined in \eqref{ea} verifies \eqref{limea}.  
\item[(iii)] If  is a detectable output for the closed--loop system then the equilibrium point is globally {\em asymptotically} stable.
\endite
\end{proposition}
\begin{proof}
Some simple manipulations prove that

where we defined the matrix
\begequ
\lab{lamp1}
\Lambda_P := G_2^{-1} \Gamma_P D^{-1} G_2^{-\top},
\endequ
and used the definition of  in \eqref{u}. Invoking Sylvester's Law of Inertia \cite{LANTIS}, and the fact that  and  are diagonal and positive definite, we have that . 

Next choose
\begequ
\lab{lami1}
\Lambda_I := G_2^\top D \Gamma_I^{-1} G_2,
\endequ
that is, also, positive definite for all diagonal, positive definite . Then

Replacing \eqref{lamp} and \eqref{lami} in the controller equations yields

Consequently, the time derivative of the Lyapunov function \eqref{dotw} becomes now
\begequ
\lab{dotw1}
	\dot{W} = -Q(x) - e^\top \Lambda_P e,
\endequ
completing the proof.
\end{proof}

To obtain an implementable version of the robust PI--PBC it was necessary to carry--out two tasks. First, to make the damping injection, introduced by the proportional term, function of the unknown matrix . Indeed, replacing \eqref{lamp1} in \eqref{lamp} we get

Second, make the gain  of the Lyapunov function \eqref{w} also a function of ---see \eqref{lami1}. 

An important observation is that, even though the controller only requires measurement of the actuated terms of the state ,  it achieves regulation of the full state vector.
\section{ADDITIONAL REMARKS ON THE PI--PBC}
\lab{sec5}	
In this section we explain how to proceed when  is not of the form \eqref{g}, discuss the reasons that stymie the use of adaptation and the inability to state a robustness result based on continuity and approximate prior knowledge of the matrix .
\subsection{General  (with  zero rows)}
\lab{subsec51}	


Instrumental to design the robust PI--PBC was the particular form of  defined in Assumption \ref{as:H} (iii). In view of the construction of the robust PI--PBC, it is clear that if  is not of the form  \eqref{g} the assumption must be modified redefining the actuated and un--actuated coordinates. 

To avoid cluttering the notation we will explain the procedure only for the case when  and ---the general case follows {\em verbatim}. Assume, furthermore, that  is of the form

The form of  given in Assumption \ref{as:H} (iii) must be, accordingly, modified to

In this case the passive output  for the incremental model becomes

where

The robust PI--PBC is given by

where  and  are arbitrary, diagonal, positive definite matrices.


Before closing this subsection we remark that our construction {\em critically} relies on the assumption of existence of  zero rows in . Indeed, it is possible to show that if this is not the case, even assuming  of the form

and defining , it is not possible to find an  positive definite matrix , which will depend on , such that the matrix  is {\em independent} of . The fact that this is {\em not possible} for all matrices  is obvious considering the counterexample . Hence, the assumption of existence of  zero rows in  is {\em necessary} to solve the problem. 
\subsection{Difficulties for adaptation}
\lab{subsec52}	
A natural alternative to the robust PI--PBC presented above is to assume a parametrisation of  and try to estimate this parameters or, in a direct approach, estimate the matrix  that defines the passive output. The indirect approach, as is well--known,  relies on parameter convergence that requires persistency of excitation---a property that is not satisfied in the regulation tasks where PI control is effective.

Let us see what are the difficulties for the application of a direct adaptation approach. Towards this end, we propose the adaptive PI--PBC

where the parameter adaptation law  is to be defined.\footnote{Notice that, in contrast to the robust PI--PBC, we have assumed that the full state is measurable.} Defining  the last two equations of the controller may be written in the form

The time derivative of the Lyapunov function \eqref{w} with  is now

where we underscore the presence of the last right hand term. If  were known the standard procedure of augmenting the Lyapunov function with a term  and cancelling the sign--indefinite term with a suitable choice of  would do the job. Alas,  is not known, hence this approach is not feasible. 

Adding an adaptation for the constant  is also not a trivial task, because of the bilinear nature of the joint estimation problem.
\subsection{Comments on robustness based on continuity}
\lab{subsec53}	
The availability of a {\em bona fide} Lyapunov function for the known parameters PI--PBC, {\em i.e.}, , suggests that stability will be preserved if the matrix  is replaced by a ``good", {\em constant} estimate of it, say . More precisely, it is expected that replacing the controller \eqref{u} by

where

would ensure stability if  is sufficiently small. Unfortunately, since the Lyapunov function is {\em not strict}, this conjecture cannot be analytically validated. Indeed, in this case the time derivative of the Lyapunov function \eqref{w} with  is now

While the term  can be dominated for ``small" , there is no way we can dominate the remaining term  and the Lyapunov analysis cannot be completed with standard techniques.

This unfortunate situation does not mean, of course, that a continuity result of this type cannot be established. It simply reveals our inability to do it with the tools used to analyze the ideal case.
\section{EXAMPLES}
\lab{sec6}	
Two examples of physical systems  which are amenable for robust PI--PBC are given below. Attention is concentrated on the verification of Assumption \ref{as:H}. Hence, unless otherwise indicated, Assumption \ref{as:g} is not imposed.
\subsection{A class of port--Hamiltonian system}
\lab{subsec61}	
\begin{proposition}
The pH system

with {\em constant} interconnection   and damping  matrices satisfies Assumption \ref{as:H} (i) and (ii).
\end{proposition}
\begin{proof}
Assumption  \ref{as:H}  (i) is obviously satisfied because

Similarly, Assumption  \ref{as:H}  (ii) holds since
\begequarrs
&&[ \nabla H(x)- \nabla H(x^*)]^\top  ({\cal J} -{\cal R} )[\nabla H(x)- \nabla H(x^*)] =\\
&& -[ \nabla H(x)- \nabla H(x^*)]^\top {\cal R} [\nabla H(x)- \nabla H(x^*)]\leq 0.
\endequarrs
\end{proof}
\subsection{Application to temperature regulation}
\lab{subsec62}	
In this subsection we design a robust PI--PBC for the temperature regulation of a class of thermal systems---the so--called, rapid thermal processes.
\subsubsection{System Description}
Similarly to \cite{ebert2004model, schaper1992modeling} we consider the following model of rapid thermal processes
 
where   represents the vector of temperatures,  and    are, respectively, the vectors of temperatures related to the radiation heat emission from environment and the convection air flows. 
The constant matrices   are {\em Hurwitz}  and contain the radiation and the convection heat transfer coefficients. Also, the entries of   correspond to the heat transfer coefficients of the heating elements. Finally,  is the controlled power applied to the heating elements. 

In the model above, as in \cite{schaper1992modeling}, it is considered that  \eqref{tsys}  is heated almost uniformly so that the contribution of energy from conduction is too small with respect to the radiation transfer. Hence, the conduction heat transfer is neglected. 

To simplify the notation we re--write the system \eqref{tsys} in the form 

where

Unlike   and  that are highly uncertain, the input matrix ---that is defined by the induced heat flow---can be precisely identified. The {\em control objective} is then to design a robust PI, {\em i.e.}, that does not require the knowledge of  the uncertain parameters, to regulate the process around some desired temperature, which is {\em not equal} to the open--loop equilibrium, but belongs to the assignable equilibrium set, that is,
\begequ
\lab{tsta}
T^\star \in \left\{T \in \rea^n_{\geq 0}\ |\ G^\perp [A_1 \Psi(T)+A_2 T + E] = 0 \right\},
\endequ
where  is a full-rank left-annihilator of .

To place the problem in the context of Proposition 1 we first shift the equilibrium of the open--loop system to the origin and then proceed to verify Assumption \ref{as:H}. For, we  introduce the standard change of coordinates

where  is the open--loop equilibrium that satisfies
\begequ
\lab{opelooequ}
A_1 \Psi(\bar T)+A_2 \bar T+E=0.
\endequ
Thus, the system \eqref{tsys} in the new coordinates takes the form \eqref{systform} with

Associated to the desired temperature  we define the equilibrium to be stabilised
\begequ
\lab{xsta}
x^\star := T^\star - \bar T.
\endequ
\subsubsection{Passivity of the thermal system}


The lemma below identifies conditions under which the system  \eqref{tsys} satisfies Assumption \ref{as:H} {\em without} imposing Assumption \ref{as:g}, that is, avoiding the partition of the coordinates into actuated and un--actuated. Towards this end, the following assumption is needed.

\begin{assumption}
\lab{ass4}
The matrix  is {\em diagonally stable} \cite{KASBHA}. That is, there exists ,  such that

Moreover, the matrix  verifies
\begequ
\lab{moncon}
A_2^\top P \diag\{T_i^3\}+  \diag\{T_i^3\} P A_2  \leq 0.
\endequ
\end{assumption}
\vspace{0.2cm}

Conditions for diagonal stability of a matrix have been studied intensively, see \cite{KASBHA} for a survey. Necessary and sufficient conditions were first reported in \cite{BARetal}---see also \cite{SHOetal} for a simple proof. A sufficient condition, given  in \cite{Farina},  is that it is a Metzler matrix (namely, its non diagonal elements are nonnegative).  

Since  is Hurwitz, condition \eqref{moncon} is trivially satisfied if  is {\em diagonal}, which is the case in some physical examples.

\begin{lemma} \label{lem3}
If Assumption \ref{ass4} holds  the vector field \eqref{vecfie} satisfies Assumption \ref{as:H} with
\begequ
\lab{hh}
H(x)=\sum_{i=1}^n p_i \phi_i(x_i)+k,
\endequ
where
\begequ
\lab{phii}
\phi_i(x_i)=\frac{1}{5}(x_i+\bar T_i)^5  - \Psi_i(\bar T_i) x_i
\endequ
and

\end{lemma}
\begin{proof}
Point (iii) of  Assumption \ref{as:H} is trivially satisfied by \eqref{hh}.

We proceed now to prove point (i). Replacing \eqref{phii} in \eqref{hh} and grouping terms yields

Now, notice that

where
\begequ
\lab{phi}
\Phi(x):=\Psi(x+\bar T)-\Psi(\bar T).
\endequ
On the other hand, from \eqref{opelooequ} it follows that the systems vector field may be written as

Consequently,
\begequarrs
[\nabla H(x)]^\top f(x,\theta) & = & \Phi^\top(x)P[ A_1 \Phi(x)+A_2 x ]\\
& = &  -\Phi^\top(x) S  \Phi(x)+ \Phi^\top(x) P A_2 x,
\endequarrs
where we have used \eqref{eq:lmi} to obtain the second identity. Now, condition \eqref{moncon} ensures that the function 

is monotonically decreasing \cite{PAVetal}. That is, for all ,

Consequently,

completing the proof of point (i).

To prove point (ii) we notice that

while

Hence, the claim is established invoking the same arguments used above and defining


Finally, the second derivative of the functions  yields

which is non--negative because . Hence, the functions  are convex as requested by condition (iv) of Assumption \ref{as:H}. This completes the proof.\\
\end{proof}



Direct application of Lemma 1 leads to the following. 
\begin{corollary}
If Assumption \ref{ass4} holds, the thermal system \eqref{tsys} defines a passive map 
with storage function , where 
\begequarrs
e & = & G^\top P \tilde \Phi(x) \\
U(x) & = & H(x) - x^\top P \Phi(x^*) - H(x^*) + (x^*)^\top P \Phi(x^*).
\endequarrs
\end{corollary}
\subsubsection{Robust PI--PBC of the thermal system}
To present the robust PI--PBC for systems verifying Assumption \ref{as:g} we partition the vector of temperatures into its un--actuated and actuated components

partition  as

and do the same with the vector function . 

The following proposition is a consequence of Lemma \ref{lem3} and Proposition \ref{pro1}.
\begin{proposition}
\lab{pro2}
Consider the system \eqref{tsys} verifying Assumptions \ref{as:g} and \ref{ass4}. Fix any desired temperature  verifying \eqref{tsta} and define the PI--PBC 

and the controller gains  and   are given by \eqref{gaipi}.  {\em For all} diagonal, positive definite matrices   and   all trajectories are bounded and the equilibrium point  is {\em globally asymptotically stable}.
\end{proposition}
\begin{proof}
The proof of stability is established invoking item (i) of Proposition 1 and identifying

To prove asymptotic stability we invoke item (ii) and observe that the augmented error signal \eqref{ea} is given in this case by

Since  verifies \eqref{limea} and  is positive definite we conclude that  and consequently .  
\end{proof}



Physically,  considering  matrix  as \eqref{g}  means that for  heating elements there are  measured points that are not directly heated by these elements.







\section{Concluding Remarks}
\lab{sec7}	
In this work we identify a class of nonlinear systems for which it is possible to design robust PI controllers with guaranteed stability properties. The class consists of input affine systems with known, constant input matrix  and  zero rows. We assume that only the states associated to the non--zero rows of   are measurable.The systems  have an open--loop stable equilibrium, but is different from the desired operating point. To handle this situation, we follow \cite{JAYetal} and generate new passive outputs for the incremental model, hence the name PI--PBC. Associated to the open--loop stable equilibrium a  Lyapunov function of the form \eqref{h} is assumed to exist. We underscore that, besides convexity, there is no assumption on the function , which is unknown. Moreover,  the controller does not require the measurement of . The functions  are assumed convex and known, but the coefficient  are unknown. Under these conditions, we show that, for a well identified class of PI tuning gains, see \eqref{gaipi}, global stability of the proposed PI--PBC is guaranteed. Conditions that ensure global asymptotic stability, are also derived.  

\section*{Acknowledgments}


This work was supported by the Ministry of Education and Science of Russian Federation (Project 14.Z50.31.0031).


\bibliographystyle{IEEEtran}
\bibliography{IEEEabrv,bibliography}

\end{document}
