\documentclass{article}

\usepackage[utf8]{inputenc}
\usepackage{amsmath}
\usepackage{hyperref}

\title{ 
  Polcovar:  
  Software for Computing the Mean and Variance of Subgraph Counts
  in Random Graphs\footnote{The software is available at \url{https://github.com/kunegis/polcovar/}}
}  

\begin{document}

\author{ Jérôme Kunegis }
\maketitle

\begin{abstract}
The mean and variance of the number of appearances of a given subgraph 
in an Erdős--Rényi random graph over  nodes are rational polynomials
in  \cite{gp6}.   
We present a piece of software named Polcovar (from \emph{polynomial}
and \emph{covariance}) that computes the exact rational coefficients of
these polynomials in function of the graph .  
\end{abstract}

\section{Introduction}
Large networks can be characterised by the number of times specific
subgraphs appear in them.  For instance, the number of triangles
measures the clustering in a network and the number of wedges
(i.e.\ 2-stars / 2-paths) characterises the inequality of the degree
distribution.  The number of vertices and edges in a graph, too, can be
characterised in this way as the number of times the complete graphs on
one and two nodes appear as subgraphs.  To assess whether a graph
contains many or few subgraphs for its size, the subgraph count must be
compared to the expected subgraph count in random graphs.  To do this,
we must know the distribution of subgraph counts.  Under mild conditions
which are valid for all examples given below, the subgraph count becomes
normal in the large graph limit.  Consequently, the distribution of
subgraph counts is characterised by its mean and average.  Expressions
for the mean and variance of subgraph counts are given in \cite{gp6},
and are always rational polynomials in the number of nodes  of the network.
In this paper, we present Matlab software for computing the exact
coefficients rational
of these polynomials, as their expressions are usually too unwieldy to
be computed by hand. 

\section{Subgraph Counts}
We consider a random graph  on  nodes distributed according to the
Erdős--Rényi model with parameter , i.e., each edge in  exists with
probability  \cite{b569}. Let  be a pattern, i.e. a small graph with
 vertices and  edges, and  the number of times this pattern
appears as a subgraph of .  
The mean and variance of  are then given by the following
expressions~\cite{gp6}:

where  is the falling factorial\footnote{defined
  as }, 
the sum is over all 
graphs  containing two differently colored copies of  (which might
overlap), 
 is the automorphism group of the graph , 
and  is the group of automorphisms of  that preserve
the edges of the underlying distinct copies of . 

Although the normal limit for  is only true when the
graph  is \emph{strictly balanced}, the expressions for the mean and
variance are always correct.  Note also that they are true exactly for
any , not just in the large  limit.  

Alternatively, the following expression can be used, which gives the
same result.  It is this expression that we implement in our code.  

where the inner sum is over all pairs of -permutations, and 
denotes the number of edges in the overlay of H permuted by  and 
permuted by  which share  nodes.

\section{Special Cases}
For specific small graphs , we get the following exact results. 

\paragraph{Node Count}
Taking  as the graph with one node gives the number of
nodes. Plugging this graph into the general form expression gives
 and .  In other words, the
number of nodes is always exactly , as expected. 

\paragraph{Edge Count}
Edges are always independent of each other and therefore the binomial
approximation for the number of edges  is exact.

These expressions can be derived both by the general form we gave above,
and by the fact that the number of edges is a binomial distribution. 

\paragraph{Triangle Count}
In a random -graph with parameter , the number of triangles 
has mean and variance given by


The expressions follow from the general form given below. 

\paragraph{Wedge Count}
The number  of wegdes (i.e., pairs of edges sharing one endpoint,
also known as 2-stars or 2-paths) has
the following distributions when :



The expressions follow from the general form given below. 

\paragraph{Other Patterns}
For the number  of squares we get:

For the number  of 4-cliques we get: 


\section{Proof Outline}
A complete proof can be found in \cite{gp6}.  We here outline the proof
as a starting point. The total number of possible subgraphs  in a graph
with  vertices is 

Define the random variables  to denote the
presence or absence of each possible pattern .  
Then, 

The expected value of each  is given by

Thus, the expected value of  can be expressed as

To compute the variance we exploit the fact that the variance equals the
expected value of the square minus the square of the expected value: 

Then, each possible pair corresponds to one possible pattern graph ,
of which the possible number is
, and each exists
with independently with probability .  From this follows
the given expression. 

\section{Extension to Covariances}
The method can be extended to covariances between the count statistics
of different patterns.  As an example:


\section{The Software}
Our code is written in the programming language Matlab, and contains two
entry points, the function \texttt{polcovar\_mu()} that computes the
mean and the function \texttt{polcovar\_sigma()} that computes the
variance or covariance. 

\begin{verbatim}
r = polcovar_mu(H);
r = polcovar_sigma(H1, H2);
\end{verbatim}

The input graphs \texttt{H} must be given as  adjacency matrices.
The function \texttt{polcovar\_sigma()} expects two graphs \texttt{H1} and
\texttt{H2} and returns the covariance of their subgraph counts.  To compute
the variance, pass the same adjacency matrix as both arguments.  All
input matrices must be symmetric 0/1 matrices with zero diagonals. 
All computations are valid for Erdős--Rényi graphs with . 

The return values are rational polynomials in form of 
matrices, where  is the degree, coded in the following way:

representing the following rational polynomial in :

All fractions  are returned in simplified form. 

\subsection{Example}
The following example uses Polcovar to compute the mean and standard
deviation of the number of triangles in a random graph with 1,000,000
nodes.

\begin{verbatim}
H = [ 0 1 1; 1 0 1; 1 1 0]

r_mu = polcovar_mu(H)
r_sigma = polcovar_sigma(H, H)

n = 1000000
mu = polyval(r_mu(1,:) ./ r_mu(2,:), n)
sigma = polyval(r_sigma(1,:) ./ r_sigma(2,:), n)
sigma_stddev = sqrt(sigma)
\end{verbatim}
This will compute that a random graph with 1,000,000 nodes can be
expected to contain 
triangles.   

\section*{Acknowledgements}
We thank Thomas Sauerwald from the University of Cambridge. 

\bibliographystyle{plain}
\bibliography{polcovar}

\end{document}
