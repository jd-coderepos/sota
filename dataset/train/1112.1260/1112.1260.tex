\documentclass{comjnl}

\usepackage{amsmath,amssymb}
\usepackage{graphicx} 
\usepackage{color}
\usepackage{dsfont}
\usepackage{graphicx}
\usepackage{subfigure}
\usepackage{stmaryrd}
\usepackage{color}
\usepackage{url}


\begin{document}

\title[Steganography: a class of secure and robust algorithms]{Steganography: a class of secure and robust algorithms}

\author{Jacques M. Bahi, Jean-Fran\c{c}ois Couchot, and Christophe Guyeux\thanks{Authors in alphabetic order}}
\affiliation{University of Franche-Comt\'{e}, Computer Science Laboratory, Belfort, France} \email{\{jacques.bahi, jean-francois.couchot, christophe.guyeux\}@univ-fcomte.fr}
\shortauthors{J. M. Bahi, J.-F. Couchot, and C. Guyeux}




\received{30 June 2011}
\revised{14 October 2011}



\newcommand{\Nats}[0]{\ensuremath{\mathbb{N}}}
\newcommand{\Z}[0]{\ensuremath{\mathbb{Z}}}
\newcommand{\R}[0]{\ensuremath{\mathbb{R}}}
\newcommand{\Bool}[0]{\ensuremath{\mathds{B}}}
\newcommand{\StratSet}[0]{\ensuremath{\mathbb{S}}}



\begin{abstract}
This research work presents a new class of non-blind information hiding
algorithms that are stego-secure and robust.
They are based on some finite domains iterations having the Devaney's 
topological chaos property.
Thanks to a complete formalization of the approach we prove   
security against watermark-only attacks
of a large class of steganographic algorithms.
Finally a complete study of robustness is given in frequency DWT and DCT domains.
\end{abstract}

\maketitle 



\maketitle




\section{Introduction}\label{sec:intro}



This work focus on non-blind  binary information hiding chaotic schemes: 
the  original host  is  required  to  extract the  binary hidden
information. This  context is indeed not
as restrictive as it could primarily appear.
Firstly, it allows to prove 
the authenticity of a document sent  through the  Internet 
(the original document is stored whereas the stego content is sent). 
Secondly, Alice and Bob can establish an hidden channel into a
streaming video 
(Alice and Bob both have the same movie, and  Alice hide information into 
the frame number   iff the binary digit
number  of its hidden message is  1).
Thirdly, based on a similar idea, a same
given image can be marked  several times by using various secret parameters
owned both by Alice and Bob. Thus more than one bit can be embedded into a given
image  by using  this work. Lastly,  non-blind watermarking  is
useful in  network's anonymity and intrusion  detection \cite{Houmansadr09}, and
to protect digital data sending through the Internet \cite{P1150442004}.
Furthermore,  enlarging the given payload  of a data  hiding scheme leads
clearly to  a degradation of its  security: the smallest the  number of embedded
bits is, the better the security is.


Chaos-based approaches are frequently proposed to improve
the quality of schemes in information 
hiding~\cite{Wu2007,Liu07,CongJQZ06,Zhu06}.
In these works, the understanding of chaotic systems
is almost intuitive: a kind of noise-like spread system
with sensitive dependence on initial condition.
Practically, some well-known chaotic maps are used
either in the data encryption stage~\cite{Liu07,CongJQZ06}, 
in the embedding into the carrier medium,
or in both~\cite{Wu2007,Wu2007bis}.
Methods referenced above are almost based on two
fundamental chaotic maps, namely the Chebychev and logistic maps, which range in . 
To avoid justifying that functions which are chaotic in 
still remain chaotic in the computing representation (\textit{i.e.},
floating numbers) we argue that functions should be iterated on finite domains.
Boolean discrete-time dynamical systems (BS) are thus iterated.

Furthermore, previously referenced works often focus on  
discretion and/or robustness properties, but they do not consider security.  
As far as we know, stego-security~\cite{Cayre2008} and chaos-security 
have only been proven 
on the spread spectrum watermarking~\cite{Cox97securespread},
and on
the dhCI algorithm~\cite{gfb10:ip}, which is notably based on iterating  
the negation function.
We argue that other functions can provide algorithms as secure as the dhCI one.
This work generalizes thus this latter algorithm and formalizes all
its stages. Due to this formalization, we address the proofs of the two
security properties for a large class of steganography approaches.  




This research work is organized as follows.
It firstly introduces the new class of algorithms (Sec.~\ref{sec:formalization}), which is the first contribution.
Next, the Section~\ref{sec:security} presents a 
State-of-the-art  in information hiding security and shows how secure  
is our approach. The proof is the second contribution.
The chaos-security property is studied in Sec.~\ref{sec:chaossecurity}
and instances of algorithms guaranteeing that desired property are presented.
This is the fourth contribution. 
Applications in frequency domains (namely DWT and DCT embedding) 
are formalized and corresponding experiments are 
given in Sec.~\ref{sec:applications}. 
This shows the applicability of the whole approach.
Finally, conclusive remarks and perspectives are given in Sec.~\ref{sec:concl}.


\section{Information hiding algorithm: formalization}
\label{sec:formalization}

As far as we know, no result rules that the chaotic behavior of a function 
that has been established on  remains on the floating numbers.
As stated before, this work presents the alternative to iterate a Boolean map:
results that are theoretically obtained in that domain are preserved  
during implementations.
In this section, we first give some recalls on Boolean discrete
dynamical Systems (BS). With this material, next sections formalize the information hiding algorithms based on 
these Boolean iterations.




\subsection{Boolean discrete dynamical systems}\label{sub:bdds}



Let us denote by  the interval of integers:
, where . 


Let  be a positive integer. A Boolean discrete-time 
network is a discrete dynamical
system defined from a {\emph{Boolean map}}
 s.t. 

{\emph{and an iteration scheme}}: parallel, serial,
asynchronous\ldots 
With the parallel iteration scheme, 
the dynamics of the system are described by 
where .
Let thus  to  
be defined by

with the \emph{asynchronous} scheme,
the dynamics of the system are described by 
where  and  is a {\emph{strategy}}, \textit{i.e.}, a sequence 
in .
Notice that this scheme only modifies one element at each iteration.

Let  be the map from  to 
itself s.t.
 
where  for all  in . 
Notice that parallel iteration of  from an initial point
 describes the ``same dynamics'' as the asynchronous
iteration of  induced by the initial point  and the strategy
.


Finally, let  be a map from  to itself. The
{\emph{asynchronous iteration graph}} associated with  is the
directed graph  defined by: the set of vertices is
; for all  and ,
 contains an arc from  to . 























We have already established~\cite{GuyeuxThese10} that we can define a
distance  on  such that 
 is a continuous and chaotic function according to 
Devaney~\cite{Devaney}.
The next section focus on  the coding step of  the steganographic algorithm
based on  iterations. 

\subsection{Coding}\label{sub:wmcoding}
In what follows,  always stands  
for a digital content we wish to hide into a digital host .


The data hiding scheme presented here does not constrain media to have 
a constant size. It is indeed sufficient to provide a function and a strategy 
that may be  parametrized with the size of the elements to modify. 
The \emph{mode} and the \emph{strategy-adapter} defined below achieve 
this goal.  

\begin{definition}[Mode]
\label{def:mode}
A map , which associates to any  an application 
, is called a \emph{mode}.
\end{definition}






For instance, the \emph{negation mode} is defined by the map that
assigns to every integer  the function 




\begin{definition}[Strategy-Adapter]
  \label{def:strategy-adapter}
  A \emph{strategy-adapter}\index{configuration} is a function  
  from  to the set of integer sequences, 
  which associates to  a sequence 
  .
\end{definition}

Intuitively, a strategy-adapter aims at generating a strategy 
 where each term  belongs to 
. Moreover it may be parametrized
in order to depend on digital media to embed. 

For instance, let us define the  \emph{Chaotic Iterations with Independent Strategy}
(\emph{CIIS}) strategy-adapter.
The CIIS strategy-adapter with parameters 

is the function that associates to any   the sequence
 defined by: 



 \begin{itemize} 
 \item :  is the real number whose binary decomposition is equal to the bitwise exclusive or (xor)
   between the binary decompositions of  and of  ;
 \item ;
 \item ;
 \item .
 \end{itemize}
where  is the
piecewise linear chaotic map~\cite{Shujun1}, 
recalled in what follows:

\begin{definition}[Piecewise linear chaotic map]
 \label{def:fonction chaotique linéaire par morceaux}
Let  be a control parameter. 
The \emph{piecewise linear chaotic map} is the map 
defined by: 
 
 \end{definition}

Contrary to the logistic map, the use of this piecewise linear chaotic map 
is relevant in cryptographic usages \cite{Arroyo08}.

Parameters of CIIS strategy-adapter will be instantiate as follows: 
 is the secret embedding key,  is the secret message, 
 is the threshold of the piecewise linear chaotic map,
which can be set as  or can act as a second secret key.
Lastly,  is for the iteration number bound:
enlarging its value improve the chaotic behavior of the scheme,
but the time required to achieve the embedding grows too.

Another strategy-adapter is the 
\emph{Chaotic Iterations with Dependent Strategy} (CIDS) 
with parameters , 
which is the function that maps any  to
the sequence  defined by:
\begin{itemize}
\item , if  and , 
  then , else ;
\item .
\end{itemize}




Let us notice that the terms of  that may be replaced by terms taken
from  are less important than other: they could be changed 
without be perceived as such. More generally, a 
\emph{signification function} 
attaches a weight to each term defining a digital media,
w.r.t. its position :

\begin{definition}[Signification function]
A \emph{signification function} is a real sequence 
. \end{definition}













For instance, let us consider a set of    
grayscale images stored into portable graymap format (P3-PGM):
each pixel ranges between 256 gray levels, \textit{i.e.},
is memorized with eight bits.
In that context, we consider 
  to be the -th term of a signification function 
. 
Intuitively, in each group of eight bits (\textit{i.e.}, for each pixel) 
the first bit has an importance equal to 8, whereas the last bit has an
importance equal to 1. This is compliant with the idea that
changing the first bit affects more the image than changing the last one.





\begin{definition}[Significance of coefficients]\label{def:msc,lsc}
Let  be a signification function, 
 and  be two reals s.t. . Then 
the \emph{most significant coefficients (MSCs)} of  is the finite 
  vector , 
the \emph{least significant coefficients (LSCs)} of  is the 
finite vector , and 
the \emph{passive coefficients} of  is the finite vector  such that:

 \end{definition}

For a given host content ,
MSCs are then ranks of   that describe the relevant part
of the image, whereas LSCs translate its less significant parts.
We are then ready to decompose an host  into its coefficients and 
then to recompose it. Next definitions formalize these two steps. 

\begin{definition}[Decomposition function]
Let  be a signification function, 
 the set of finite binary sequences,
 the set of finite integer sequences, 
 and  be two reals s.t. .  
Any host  can be decomposed into 

where
\begin{itemize}
\item , , and  are coefficients defined in Definition  
\ref{def:msc,lsc};
\item ;
 \item ;
 \item .
 \end{itemize}
The function that associates the decomposed host to any digital host is 
the \emph{decomposition function}. It is 
further referred as  since it is parametrized by 
, , and . Notice that  is a shortcut for .
\end{definition} 


\begin{definition}[Recomposition]
Let 
 s.t.
\begin{itemize}
\item the sets of elements in , elements in , and 
elements in  are a partition of ;
\item , , and .  
\end{itemize}
One can associate the vector 

\noindent where 
 is the usual basis of the vectorial space  (that is to say, , where  is the Kronecker symbol).
The function that associates  to any 
 following the above constraints 
is called the \emph{recomposition function}.
\end{definition}

The embedding consists in the replacement of the values of 
 of 's LSCs  by . 
It then composes the two decomposition and
recomposition functions seen previously. More formally:


\begin{definition}[Embedding media]
Let  be a decomposition function,
 be a host content,
 be its image by , 
and  be a digital media of size .
The digital media  resulting on the embedding of  into  is 
the image of 
by  the recomposition function .
\end{definition}

Let us then define the dhCI information hiding scheme
presented in~\cite{gfb10:ip}:

\begin{definition}[Data hiding dhCI]
 \label{def:dhCI}
Let  be a decomposition function,
 be a mode, 
 be a strategy adapter,
 be an host content,\linebreak
 
be its image by ,
 be a positive natural number,  
and  be a digital media of size .


The dhCI dissimulation  maps any
  to the digital media  resulting on the embedding of
 into , s.t.

\begin{itemize}
\item we instantiate the mode  with parameter , leading to 
  the function ;
\item we instantiate the strategy adapter  
with parameter  (and possibly some other ones);
this instantiation leads to the strategy .

\item we iterate  with initial configuration ;
\item  is finally the -th term of these iterations.
\end{itemize}
\end{definition}


To summarize, iterations are realized on the LSCs of the
host content
(the mode gives the iterate function,  
the strategy-adapter gives its strategy), 
and the last computed configuration is re-injected into the host content, 
in place of the former LSCs.
























Notice that in order to preserve the unpredictable behavior of the system, 
the size of the digital medias is not fixed.
This approach is thus self adapted to any media, and more particularly to
any size of LSCs. 
However this flexibility enlarges the complexity of the presentation: 
we had to give Definitions~\ref{def:mode} and~\ref{def:strategy-adapter} 
respectively of mode and strategy adapter.

\begin{figure}[ht]
\centering
\includegraphics[width=8.5cm]{organigramme2.eps}
\caption{The dhCI dissimulation scheme}
\label{fig:organigramme}
\end{figure}


Next section shows how to check whether a media contains a watermark.


\subsection{Decoding}\label{sub:wmdecoding}

Let us firstly show how to formally check whether a given digital media  
results from the dissimulation of  into the digital media . 



\begin{definition}[Watermarked content]
Let  be a decomposition function,
 be a mode, 
 be a strategy adapter, 
 be a positive natural number,  
 be a digital media, and 
 be the 
image by   of  a digital media . 
Then  is \emph{watermarked} with  if
the image by  of  is 
, where 
 is the right member of .
\end{definition}


Various decision strategies are obviously  possible to determine whether a given
image  is watermarked or not, depending  on the eventuality
that the considered image may have  been attacked.
For example, a  similarity percentage between 
and   can  be  computed and compared  to a  given
threshold. Other  possibilities are the use of  ROC curves or
the definition of a null hypothesis problem.











The next section recalls some security properties and shows how the 
\emph{dhCI dissimulation} algorithm verifies them.

\section{Security analysis}\label{sec:security}
\subsection{State-of-the-art in information hiding security}\label{sub:art}



As far as we know, Cachin~\cite{Cachin2004}
produces the first fundamental work in information hiding security:
in the context of steganography, the attempt of an attacker to distinguish 
between an innocent image and a stego-content is viewed as an hypothesis 
testing problem.
Mittelholzer~\cite{Mittelholzer99} next proposed the first theoretical 
framework for analyzing the security of a watermarking scheme.
Clarification between  robustness and security 
and classifications of watermarking attacks
have been firstly presented by Kalker~\cite{Kalker2001}.
This work has been deepened by Furon \emph{et al.}~\cite{Furon2002}, who have translated Kerckhoffs' principle (Alice and Bob shall only rely on some previously shared secret for privacy), from cryptography to data hiding. 

More recently~\cite{Cayre2005,Perez06} classified the information hiding  
attacks into categories, according to the type of information the attacker (Eve)
has access to:
\begin{itemize}
\item in Watermarked Only Attack (WOA) she only knows embedded contents ;
\item in Known Message Attack (KMA) she knows pairs  of embedded
  contents and corresponding messages;
\item in Known Original Attack (KOA) she knows several pairs  
  of embedded contents and their corresponding original versions;
\item in Constant-Message Attack (CMA) she observes several embedded
  contents ,\ldots, and only knows that the unknown 
  hidden message  is the same in all contents.
\end{itemize}

To the best of our knowledge, 
KMA, KOA, and CMA have not already been studied
due to the lack of theoretical framework.
In the opposite, security of data hiding against WOA can be evaluated,
by using a probabilistic approach recalled below.




\subsection{Stego-security}\label{sub:stegosecurity}


In the Simmons' prisoner problem~\cite{Simmons83}, Alice and Bob are in jail and
they want to,  possibly, devise an escape plan by  exchanging hidden messages in
innocent-looking  cover contents.  These  messages  are to  be  conveyed to  one
another by a common warden named Eve, who eavesdrops all contents and can choose
to interrupt the communication if they appear to be stego-contents.

Stego-security,  defined in  this well-known  context, is  the  highest security
class in Watermark-Only  Attack setup, which occurs when Eve  has only access to
several marked contents~\cite{Cayre2008}.


Let  be the set of embedding keys,  the probabilistic model of
 initial  host contents,  and  the  probabilistic model  of 
marked contents s.t. each host  content has  been marked
with the same key  and the same embedding function.

\begin{definition}[Stego-Security~\cite{Cayre2008}]
\label{Def:Stego-security}  The embedding  function  is \emph{stego-secure}
if   is established.
\end{definition}







 Stego-security  states that  the knowledge  of   does  not help  to make  the
 difference  between  and  .  This  definition implies  the following
 property:
  
 This property is equivalent to  a zero Kullback-Leibler divergence, which is the
 accepted definition of the "perfect secrecy" in steganography~\cite{Cachin2004}.


\subsection{The negation mode is stego-secure}
To make this article self-contained, this section recalls theorems and proofs of stego-security for negation mode published in~\cite{gfb10:ip}.

\begin{proposition} \emph{dhCI dissimulation}  of Definition \ref{def:dhCI} with
negation mode and  CIIS strategy-adapter is stego-secure, whereas  it is not the
case when using CIDS strategy-adapter.
\end{proposition}


\begin{proof}   On   the    one   hand,   let   us   suppose    that     when  using  \linebreak CIIS.
We  prove  by  a
mathematical   induction   that   .

The     base     case     is     immediate,     as     . Let us now suppose that the statement   holds  until for  some . 
Let     and   \linebreak    (the digit  is in position ).

So    
 where  
 is again the bitwise exclusive or. 
These  two events are  independent when
using CIIS strategy-adapter 
(contrary to CIDS, CIIS is not built by using ),
 thus:
 

According to the
inductive    hypothesis:   .  The set  of events  for   is  a partition  of  the universe  of possible,  so
.                  Finally,
,   which    leads   to   .   This  result  is  true  for all   and then for .

Since  is  that is proven to be equal to ,
we thus  have established that, 
 
So   dhCI   dissimulation   with   CIIS
strategy-adapter is stego-secure.

On  the  other  hand,  due  to  the  definition  of  CIDS,  we  have  \linebreak
. 
So   there  is   no  uniform  repartition   for  the stego-contents .
\end{proof}



To sum up, Alice  and Bob can counteract Eve's attacks in  WOA setup, when using
dhCI dissimulation with  CIIS strategy-adapter.  To our best  knowledge, this is
the second time an information hiding scheme has been proven to be stego-secure:
the   former  was   the  spread-spectrum   technique  in   natural  marking
configuration with  parameter equal to 1 \cite{Cayre2008}.





\subsection{A new class of -stego-secure schemes}

Let us prove that,
\begin{theorem}\label{th:stego}
Let  be positive,
 be any size of LSCs, 
,
 be an image mode s.t. 
 is strongly connected and 
the Markov matrix associated to  
is doubly stochastic. 
In the instantiated \emph{dhCI dissimulation} algorithm 
with any uniformly distributed (u.d.) strategy-adapter 
that is independent from ,  
there exists some positive natural number  s.t.
. 
\end{theorem}


\begin{proof}   
Let  be the bijection between  and 
 that associates the decimal value
of any  binary number in .
The probability  for  is thus equal to 
 further denoted by .
Let , 
the probability   is 

\noindent 
where  is true iff the binary representations of 
 and  may only differ for the  -th element,
and where 
 abusively denotes, in this proof, the -th element of the binary representation of 
.

Next, due to the proposition's hypotheses on the strategy,
 is equal to  
.
Finally, since  and  are constant during the 
iterative process  and thus does not depend on , we have 


Since 
 is equal to  where   is the Markov matrix associated to
  we thus have



First of all, 
since the graph  is strongly connected,
then for all vertices  and , a path can
be  found to  reach   from   in at  most   steps.  
There  exists thus  s.t.
.  
As all the multiples  of  are such that 
, 
we can  conclude that, if
 is the least common multiple of  thus 
 and thus 
 is a regular stochastic matrix.


Let us now recall the following stochastic matrix theorem:
\begin{theorem}[Stochastic Matrix]
  If  is a regular stochastic matrix, then  
  has an unique stationary  probability vector . Moreover, 
  if  is any initial probability vector and 
   for  then the Markov chain 
  converges to  as  tends to infinity.
\end{theorem}

Thanks to this theorem,  
has an unique stationary  probability vector . 
By hypothesis, since  is doubly stochastic we have 

and thus .
Due to the matrix theorem, there exists some 
 s.t. 

and the proof is established.
Since  is  the method is then -stego-secure
provided the strategy-adapter is uniformly distributed.
 \end{proof}

This section has focused on security with regards to probabilistic behaviors. 
Next section studies it in the perspective of topological ones.




\section{Chaos-security}\label{sec:chaossecurity}


 To check whether an existing data hiding scheme is chaotic or not, we propose firstly to write it as an iterate process . It is possible to prove that this formulation can always be done, as follows. Let us consider a given data hiding algorithm. Because it must be computed one day, it is always possible to translate it as a Turing machine, and this last machine can be written as  in the following way. Let  be the current configuration of the Turing machine (Fig.~\ref{Turing}), where  is the paper tape,  is the position of the tape head,  is used for the state of the machine, and  is its transition function (the notations used here are well-known and widely used). We define  by:
 \begin{itemize}
 \item , if  ;
 \item ,  if .
 \end{itemize}
 Thus the Turing machine can be written as an iterate function  on a well-defined set , with  as the initial configuration of the machine. We denote by  the iterative process of a data hiding scheme .

 \begin{figure}[h!]
   \centering
\includegraphics[width=8.5cm]{Turing.eps}
 \caption{Turing Machine}
 \label{Turing}
 \end{figure}


Let us now define the notion of chaos-security.
Let  be a topology on . So the behavior of this dynamical system can be studied to know whether or not the data hiding scheme is unpredictable. This leads to the following definition.

\begin{definition}
\label{DefinitionChaosSecure}
An information hiding scheme  is said to be chaos-secure on  if its iterative process  has a chaotic behavior, as defined by Devaney, on this topological space.
\end{definition}



Theoretically speaking, chaos-security can always be studied, as it only requires that the two following points are satisfied.
\begin{itemize}
\item Firstly, the data hiding scheme must be written as an iterate function on
  a set ;
As illustrated by the use of the Turing machine, it is always possible to satisfy this requirement; It is established here since we iterate  as defined in
Sect.~(\ref{sub:bdds});

\item Secondly, a metric or a topology must be defined on ; This is always possible, for example, by taking for instance the most relevant one, that is the order topology.
\end{itemize}


Guyeux has recently shown in~\cite{GuyeuxThese10}  that chaotic
iterations of  with  the
vectorial negation  as  iterate  function 
have   a  chaotic  behavior.
As a corollary, we deduce that the dhCI  dissimulation algorithm 
with negation mode and  CIIS strategy-adapter is chaos-secure.

However, all these results suffer from only relying on the vectorial
negation function. This problem has been theoretically tackled 
in~\cite{GuyeuxThese10} which provides the  following theorem.

\begin{theorem}
\label{Th:Caracterisation des  IC chaotiques} Functions  such that   is chaotic according to  Devaney, are functions
such that the graph  is strongly connected.
\end{theorem}
\noindent We deduce from this theorem that functions whose  
graph is strongly connected are sufficient to provide new instances of 
dhCI dissimulation that are chaos-secure.


Computing a mode  such that the image of  (\textit{i.e.}, ) 
is a function with a strongly connected graph of iterations 
has been previously studied (see~\cite{bcgr11:ip} for instance). 
The next section presents a use of them in our steganography context. 

















\section{Applications to frequential domains}
\label{sec:applications}
We are then left to provide an u.d. strategy-adapter that is independent
from the cover, an image mode  whose iteration
graph  is strongly connected and whose Markov
matrix is doubly stochastic.

First, the  strategy adapter (see Section~\ref{sub:wmcoding})
has the required properties:
it does not depend on the cover and the proof that its outputs
are u.d. on  
is left as an exercise for the reader.
In all the experiments parameters  and  are randomly 
chosen in  and 
respectively.
The number of iteration is set to , where  is the number of LSCs 
that depends on the domain.  
 
Next,~\cite{bcgr11:ip} has presented an iterative approach to generate image
modes  such that  is strongly connected. Among these
maps, it is obvious to check which verifies or not the doubly
stochastic constrain.
For instance, in what follows we consider the mode
 s.t. its th component is
defined by


Thanks to~\cite[Theorem 2]{bcgr11:ip} we deduce that its iteration graph 
 is strongly connected. 
Next, the Markov chain is stochastic by construction. 

Let us prove that its Markov chain is doubly stochastic by induction on the 
length .
For  and  the proof is obvious. Let us consider that the 
result is established until  for some .

Let us then firstly prove the doubly stochasticity for .
Following notations introduced in~\cite{bcgr11:ip}, 
let   and  denote
the subgraphs of  induced by the subset 
and  of  respectively.
 and    are isomorphic to .
Furthermore, these two graphs are linked together only with arcs of the form
 and 
.
In  the number of arcs whose extremity is 
is  the same than the number of arcs whose extremity is  
augmented with 1, and similarly for .
By induction hypothesis, the Markov chain associated to  is doubly stochastic. All the vertices  have thus the same number of 
ingoing arcs and the proof is established for  is .

Let us then  prove the doubly stochasticity for .
The map  is defined by 
.
With previously defined  notations, let us focus on 
 and    which are isomorphic to . 
Among configurations of , only four suffixes of length 2 can be
obviously observed, namely, , ,  and .
Since 
, , 
, and , the number of 
arcs whose extremity is 
\begin{itemize}
\item 
 is the same than the one whose extremity is  in  augmented with 1 (loop over configurations );
\item 
 is the same than the one whose extremity is  in  augmented with 1 (arc from configurations 
 to configurations 
);
\item 
 is the same than the one whose extremity is  in  augmented with 1 (loop over configurations );
\item 
 is the same than the one whose extremity is  in  augmented with 1 (arc from configurations 
 to configurations 
).
\end{itemize}
Thus all the vertices  have  the same number of 
ingoing arcs and the proof is established for .




\subsection{DWT embedding}

Let us now explain how the dhCI dissimulation can be applied in
the discrete wavelets domain (DWT).
In this paper, the Daubechies family of wavelets is chosen: 
each DWT decomposition depends on a decomposition level and a coefficient
matrix (Figure~\ref{fig:DWTs}):  means approximation coefficient,
when  denote respectively diagonal, 
vertical, and horizontal detail coefficients. 
For example, the DWT coefficient \textit{HH}2 is the matrix equal to the 
diagonal detail coefficient of the second level of decomposition of the image.

\begin{figure}[htb]
\centerline{
\includegraphics[width=3.7cm]{DWTs.eps}
}
\caption{Wavelets coefficients.}
\label{fig:DWTs}
\end{figure}






The choice of the detail level is motivated by finding
a good compromise between robustness and invisibility.
Choosing low or high frequencies in DWT domain leads either to a very
fragile watermarking without robustness (especially when facing a
JPEG2000 compression attack) or to a large degradation of the host
content. 
In order to have a robust but discrete DWT embedding, 
the second detail level 
(\textit{i.e.}, ) 
that corresponds to the middle frequencies,
has been retained.






Let us consider the Daubechies wavelet coefficients of a third
level decomposition as represented in Figure~\ref{fig:DWTs}. 
We then translate these float coefficients into their 32-bits values.
Let us define the  significance function  that associates to any index  in this sequence of bits the following numbers:
\begin{itemize}
\item  if  is one of the three last bits of any index of
  coefficients in  , , or in ;
\item  if  is an index of a coefficient in  
  , , or in ;
\item  otherwise.
\end{itemize}

According to the definition of significance of coefficients 
(Def.~\ref{def:msc,lsc}), if  is ,  LSCs are the
last three bits of coefficients in 
,, and .
Thus, decomposition and recomposition functions are fully defined
and dhCI dissimulation scheme can now be applied.

Figure \ref{fig:DWT} shows the result of a
dhCI dissimulation embedding into DWT domain. 
The original is the image 5007 of the BOSS contest~\cite{Boss10}.
Watermark  is given in Fig.~\ref{(b) Watermark}.


From a random selection of 50 images into the database from the BOSS 
contest~\cite{Boss10}, we have applied the previous algorithm with mode  
defined in Equation~(\ref{eq:fqq}) and with the negation mode. 

\begin{figure}[ht]
  \centering
\subfigure[Original Image.]{\includegraphics[width=0.24\textwidth]
    {5007.eps}\label{(a) Original 5007}}\hspace{1cm}

\subfigure[Watermark .]{\includegraphics[width=0.08\textwidth]{invader.eps}\label{(b) Watermark}}\hspace{1cm}

\subfigure[Watermarked Image.]{\includegraphics[width=0.24\textwidth]{5007_bis.eps}\label{(c) Watermarked 5007}}

\caption{Data hiding in DWT domain}
\label{fig:DWT}
\end{figure}





\subsection{DCT embedding}
Let us denote by  the original image of size , and by 
the hidden message, supposed here to be a binary image of size . The image  is transformed from the spatial
domain to DCT domain frequency bands,
in order to embed  inside it.  
To do so, the host image is firstly divided into 
image blocks as given below:

Thus, for each image block,
a DCT is performed and the coefficients in the frequency bands 
are obtained as follows:
.

To define a discrete but robust scheme, only the four following coefficients of each  block in position  will be possibly modified:   or . 
This choice can be reformulated as follows. 
Coefficients of each DCT matrix are re-indexed by using a southwest/northeast diagonal, such that ,\linebreak , , , ..., and   .
So the signification function can be defined in this context by:
\begin{itemize}
\item if  mod  and , then ;
\item else if  mod  and , then ;
\item else .
\end{itemize}
The significance of coefficients are obtained for instance with 
 leading to the definitions of MSCs, LSCs, and passive coefficients.
Thus, decomposition and recomposition functions are fully defined and dhCI dissimulation scheme can now be applied.




\subsection{Image quality}
This section focuses on measuring visual quality of our steganographic method.
Traditionally, this is achieved by quantifying the similarity 
between the modified image and its reference image.
The Mean Squared Error (MSE) and the Peak Signal to Noise
Ratio (PSNR) are the most widely known tools that provide such a metric.
However, both of them do not take into account Human Visual System (HVS)
properties. 
Recent works~\cite{EAPLBC06,SheikhB06,PSECAL07,MB10} have tackled this problem 
by creating new metrics. Among them, what follows focuses on PSNR-HVS-M~\cite{PSECAL07} and BIQI~\cite{MB10}, considered as advanced visual quality metrics.  
The former efficiently combines PSNR and  visual between-coefficient contrast masking of DCT basis functions based on HVS. This metric has 
been computed here by using the implementation given at~\cite{psnrhvsm11}.
The latter allows to get a blind image quality assessment measure, 
\textit{i.e.}, without any knowledge of the source distortion.
Its implementation is available at~\cite{biqi11}.


\begin{table}
\begin{center}
\begin{tabular}{|c|c|c|c|c|}
\hline
Embedding & \multicolumn{2}{|c|}{DWT} 
 & \multicolumn{2}{|c|}{DCT} \\
\hline
Mode &  & neg. &  & neg. \\
\hline
PSNR & 42.74     & 42.76     &  52.68      &  52.41   \\
\hline
PSNR-HVS-M & 44.28  & 43.97 & 45.30 & 44.93 \\
\hline
BIQI & 35.35 & 32.78 & 41.59 & 47.47 \\
\hline
\end{tabular}
\end{center}
\caption{Quality measeures of our steganography approach\label{table:quality}} 
\end{table}



Results of the image quality metrics 
are summarized into the Table~\ref{table:quality}.
In wavelet domain, the PSNR values obtained here are comparable to other approaches
(for instance, PSNR are 44.2 in~\cite{TCL05} and 46.5 in~\cite{DA10}), 
but  a real improvement for the discrete cosine embeddings is obtained 
(PSNR is 45.17 for~\cite{CFS08}, it is always lower than 48 for~\cite{Mohanty:2008:IWB:1413862.1413865}, and always lower than 39 for~\cite{MK08}).
Among steganography approaches that evaluate PSNR-HVS-M, results of our approach 
are convincing. Firstly, optimized method developed along~\cite{Randall11} has a PSNR-HVS-M equal to 44.5 whereas our approach, with a similar PSNR-HVS-M, should be easily improved by considering optimized mode. Next, 
another approach~\cite{Muzzarelli:2010} have higher PSNR-HVS-M, certainly, but
this work does not address robustness evaluation whereas our approach is complete.
Finally, as far as we know, this work is the first one that evaluates the BIQI metric in the steganography context. 

 

With all this material, we are then left to evaluate the robustness of this 
approach. 







\subsection{Robustness}
Previous sections have formalized frequential domains embeddings and
has focused on the negation mode and  defined in Equ.~(\ref{eq:fqq}).
In the robustness given in this continuation, {dwt}(neg), 
{dwt}(fl), {dct}(neg), {dct}(fl) 
respectively stand for the DWT and DCT embedding 
with the negation mode and with this instantiated mode.
 
For each experiment, a set of 50 images is randomly extracted 
from the database taken from the BOSS contest~\cite{Boss10}. 
Each cover is a  grayscale digital image and the watermark  
is given in Fig~\ref{(b) Watermark}. 
Testing the robustness of the approach is achieved by successively applying
on watermarked images attacks like cropping, compression, and geometric 
transformations.
Differences between 
 and  are 
computed. Behind a given threshold rate, the image is said to be watermarked.  
Finally, discussion on metric quality of the approach is given in 
Sect.~\ref{sub:roc}.

 


Robustness of the approach is  evaluated by
applying different percentage of cropping: from 1\% to 81\%.
Results are presented in Fig.~\ref{Fig:atck:dec}. Fig.~\ref{Fig:atq:dec:img}
gives the cropped image 
where 36\% of the image is removed.
Fig.~\ref{Fig:atq:dec:curves} presents effects of such an attack.
From this experiment, one can conclude that all embeddings have similar 
behaviors.
All the percentage differences are so far less than 50\% 
(which is the mean random error) and thus robustness is established.



\begin{figure}[ht]
  \centering
\subfigure[Cropped Image.]{\includegraphics[width=0.24\textwidth]
    {5007_dec_307.eps}\label{Fig:atq:dec:img}}\hspace{2cm}
  \subfigure[Cropping Effect]{
\includegraphics[width=0.5\textwidth]{atq-dec.eps}\label{Fig:atq:dec:curves}}
\caption{Cropping Results}
\label{Fig:atck:dec}
\end{figure}


\subsubsection{Robustness against compression}

Robustness against compression is addressed
by studying both JPEG  and JPEG 2000 image compressions.
Results are respectively presented in Fig.~\ref{Fig:atq:jpg:curves}
and Fig.~\ref{Fig:atq:jp2:curves}.
Without surprise, DCT embedding which is based on DCT 
(as JPEG compression algorithm is) is  more 
adapted to JPEG compression than DWT embedding.
Furthermore, we have a similar behavior for the JPEG 2000 compression algorithm, which is based on wavelet encoding: DWT embedding naturally outperforms
DCT one in that case.  


\begin{figure}[ht]
  \centering
  \subfigure[JPEG Effect]{
\includegraphics[width=0.45\textwidth]{atq-jpg.eps}\label{Fig:atq:jpg:curves}}
  \subfigure[JPEG 2000 Effect]{
\includegraphics[width=0.45\textwidth]{atq-jp2.eps}\label{Fig:atq:jp2:curves}}
\caption{Compression Results}
\label{Fig:atck:comp}
\end{figure}



\subsubsection{Robustness against Contrast and Sharpness Attack}
Contrast and Sharpness adjustment belong to the the classical set of 
filtering image attacks.
Results of such attacks are presented in 
Fig.~\ref{Fig:atq:fil} where 
Fig.~\ref{Fig:atq:cont:curve} and Fig.~\ref{Fig:atq:sh:curve} summarize 
effects of contrast and sharpness adjustment respectively. 


\begin{figure}[ht]
  \centering
  \subfigure[Contrast Effect]{
\includegraphics[width=0.45\textwidth]{atq-contrast.eps}\label{Fig:atq:cont:curve}}
  \subfigure[Sharpness Effect]{
\includegraphics[width=0.45\textwidth]{atq-flou.eps}\label{Fig:atq:sh:curve}}
\caption{Filtering Results}
\label{Fig:atq:fil}
\end{figure}

\subsubsection{Robustness against Geometric Transformation}
Among geometric transformations, we focus on  
rotations, \textit{i.e.}, when two opposite rotations 
of angle  are successively applied around the center of the image.
In these geometric transformations,  angles range from 2 to 20 
degrees.  
Results are presented in Fig.~\ref{Fig:atq:rot}: Fig.~\ref{Fig:atq:rot:img}
gives the image of a rotation of 20 degrees whereas
Fig.~\ref{Fig:atq:rot:curve} presents effects of such an attack.
It is not a surprise that results are better for DCT embeddings: this approach
is based on cosine as rotation is. 



\begin{figure}[ht]
  \centering
  \subfigure[20 degrees Rotation Image]{
\includegraphics[width=0.25\textwidth]{5007_rot_10.eps}\label{Fig:atq:rot:img}}

  \subfigure[Rotation Effect]{
\includegraphics[width=0.45\textwidth]{atq-rot.eps}\label{Fig:atq:rot:curve}}

\caption{Rotation Attack Results}
\label{Fig:atq:rot}
\end{figure}

\subsection{Evaluation of the  Embeddings}\label{sub:roc}
We are then left to set a convenient threshold that is accurate to 
determine whether an image is watermarked or not.
Starting from a set of 100 images selected among the Boss image Panel,
we compute the following three sets: 
the one with all the watermarked images ,
the one with all successively watermarked and attacked images ,
and the one with only the attacked images .
Notice that the 100 attacks for each images 
are selected among these detailed previously.


For each threshold  and a given image 
, 
differences on DCT are computed. The image is said to be watermarked
if these differences are less than the threshold. 
In the positive case and if  really belongs to 
 it is a True Positive (TP) case.  
In the negative case but if  belongs to 
 it is a False Negative (FN) case.  
In the positive case but if   belongs to 
, it is a False Positive (FP) case.  
Finally, in the negative case and if  belongs to
, it is a True Negative (TN).  
The True (resp. False) Positive Rate  (TPR) (resp. FPR) is thus computed 
by dividing the number of TP (resp. FP) by 100.

\begin{figure}[ht]
\begin{center}
\includegraphics[width=7cm]{ROC.eps}
\end{center}
\caption{ROC Curves for DWT or DCT Embeddings}\label{fig:roc:dwt}
\end{figure}

The Figure~\ref{fig:roc:dwt} is the Receiver Operating Characteristic (ROC) 
curve. 
For the DWT, it shows that best results are obtained when the threshold 
is 45\% for the dedicated function (corresponding to the point (0.01, 0.88))
and  46\% for the negation function (corresponding to the point (0.04, 0.85)).
It allows to conclude that each time LSCs differences between
a watermarked image and another given image   are less than 45\%, we can claim that 
 is an attacked version of the original watermarked content.
For the two DCT embeddings, best results are obtained when the threshold 
is 44\% (corresponding to the points (0.05, 0.18) and (0.05, 0.28)).

Let us then give some confidence intervals for all the evaluated attacks. The
approach is resistant to:
\begin{itemize}
\item all the croppings where percentage is less than 85;
\item compressions where quality ratio is greater 
  than 82 with DWT embedding and 
  where quality ratio is greater than 67 with DCT one;
\item contrast when strengthening belongs to  
(resp. )  in DWT (resp. in DCT) embedding;
\item all the rotation attacks with DCT embedding and a rotation where
angle is less than 13 degrees with DWT one.
\end{itemize}


\section{Conclusion}\label{sec:concl}
This paper has proposed a new class of secure and robust information hiding 
algorithms.
It has been entirely formalized, thus allowing both its theoretical security 
analysis, and the computation of numerous variants encompassing spatial and 
frequency domain embedding.
After having presented the general algorithm with detail, we have given
conditions for choosing mode and strategy-adapter making the whole
class  stego-secure or -stego-secure.
To our knowledge, this is the first time such a result has been established.

Applications in frequency domains (namely DWT and DCT domains) have finally be
formalized.
Complete experiments have allowed us 
first to evaluate how invisible is the steganographic method (thanks to the PSNR computation) and next to verify the robustness property against attacks.
Furthermore, the use of ROC curves for DWT embedding have revealed very high rates
between True positive and False positive results.

In future work, our intention is to find the best image mode with respect to  
the combination between  DCT and DWT based steganography
algorithm. Such a combination topic has already been addressed
(\textit{e.g.}, in~\cite{al2007combined}), but never with objectives
we have set.


Additionally, we will try to discover new topological properties for the dhCI
dissimulation schemes.
Consequences of these chaos properties will be drawn in the context of 
information hiding security.
We will especially focus on the links between topological properties and classes
of attacks, such as KOA, KMA, EOA, or CMA.

Moreover, these algorithms will be compared to other existing ones, among other
things by testing whether these algorithms are chaotic or not.
Finally we plan to verify the robustness of our approach 
against statistical steganalysis methods~\cite{GFH06,ChenS08,DongT08,FridrichKHG11a}.




\bibliographystyle{compj}

\bibliography{abbrev2,mabase,biblioand2}

\end{document}
