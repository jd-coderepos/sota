\documentclass[amsthm]{elsart}

\usepackage{graphicx}
\usepackage{indentfirst}
\usepackage{url}
\usepackage{color}
\usepackage{yjsco}
\usepackage{natbib}
\usepackage{amsmath}
\usepackage{amsfonts}
\usepackage{amsthm}
\usepackage{amssymb}
\usepackage{bm}
\usepackage{enumitem}

\usepackage{comment}
\usepackage{algpseudocode}

\graphicspath{{img/}}
\def \coeffs{{\rm coeffs}}
\def \codim {{\rm codim}}
\def \OWD  {{\rm OWD}}
\def \Res  {{\rm Res}}
\def \MCproj  {{\rm MCproj}}
\def \cs  {{\rm cs}}
\def \cso  {{\rm cso}}
\def \ms  {{\rm ms}}
\def \mso  {{\rm mso}}
\def \discrim  {{\rm discrim}}
\def \sqrfree  {{\rm sqrfree}}
\def \sqk  {{\rm sqk}}
\def \lc  {{\rm lc}}
\def \Bproj  {{\tt Bp}}
\def  \zero {{\rm Zero}}
\def  \Nproj {{\tt Np}}
\def  \NTproj {{\tt NTp}}
 \def  \AlgNproj {{\tt Nproj}}
\def  \Hproj {{\tt Hp}}
\def  \Npl {{\tt NpL}}
\def  \Nfproj {{\tt Nfp}}
\def \Proineq {{\tt DPS}}
\def \ProCop {{\tt ProCop}}
\def \Bprojection {{\tt Bproj}}
\def \HNprojection {{\tt HNproj}}
\def  \HNproj {{\tt HNp}}
\def \Nprojection {{\tt Nproj}}
\def \Hprojection {{\tt Hproj}}
\def \Findk {{\tt Findk}}
\def \SRes {{\tt SRes}}
\def \Findinf {{\tt Findinf}}
\def \m  {{\rm m}}
\def \RR {{\mathbb R}}
\def \ZZ {{\mathbb Z}}
\def \AA {{\mathbb A}}
\newcommand{\va}{\bm{\alpha}}
\newcommand{\vb}{\bm{\beta}}
\newcommand{\xx}{\bm{x}}
\newcommand{\yy}{\bm{y}}
\newcommand{\zz}{\bm{z}}
\newcommand{\XX}{\bm{X}}
\newcommand{\YY}{\bm{Y}}
\def \rr {{\mathcal R}}
\def \RAGlib {{\tt RAGlib}}
\def \FI {{\tt FI}}
\def \QEPCAD {{\tt QEPCAD}}
\def \PCAD {{\tt PCAD}}
\def \TwoPro {{\tt PSD-HpTwo}}
\def \TwoHp {{\tt HpTwo}}
\def \RAGMaple {{\tt RAGMaple}}
\def \Raglib {{\tt Raglib}}
\def \FI {{\tt FI}}
\def \QEPCAD {{\tt QEPCAD}}
\def \PCAD {{\tt PCAD}}
\def \TCPT {{\tt CMT}}
\def \TCPTC {{\tt TCPTC}}
\def \OO {{\mathcal{O}}}

\newcommand\numberthis{\addtocounter{equation}{1}\tag{\theequation}}
\newtheorem{example}{\qquad Example}[section]
\newtheorem{ex}{Example}   \renewcommand{\algorithmicrequire}{\textsf{Input:}}
\renewcommand{\algorithmicensure}{\textsf{Output:}}


\begin{document}


\begin{frontmatter}

\title{\textbf{Open Weak CAD and Its Applications}}

\thanks{This research was partly supported by US National Science Foundation Grant 1319632, National Science Foundation of China Grants 11290141, 11271034 and 61532019, and China Scholarship Council.}

\author{Jingjun Han, Liyun Dai}
\ead{hanjingjunfdfz@gmail.com, dailiyun@pku.edu.cn}
\address{LMAM  School of Mathematical Sciences\\  Beijing International Center for Mathematical Research\\Peking University, Beijing 100871, China}

\author{Hoon Hong}
\address{Department of Mathematics, North Carolina State University, Raleigh NC 27695, USA}
\ead{hong@ncsu.edu}
\ead[url]{www.math.ncsu.edu/\~{}hong}

\author{Bican Xia}
\address{LMAM  School of Mathematical Sciences, Peking University, Beijing 100871, China}
\ead{xbc@math.pku.edu.cn}
\ead[url]{www.math.pku.edu.cn/is/\~{}xbc/}


\begin{abstract}
The concept of open weak CAD is introduced. Every open CAD is an open weak CAD. On the contrary, an open weak CAD is not necessarily an open CAD. An algorithm for computing projection polynomials of open weak CADs is proposed. The key idea is to compute the intersection of projection factor sets produced by different projection orders. The resulting open weak CAD often has smaller number of sample points than open CADs.

The algorithm can be used for computing sample points for all open connected components of  for a given polynomial . It can also be used for many other applications, such as testing semi-definiteness of polynomials and copositive problems. In fact, we solved several difficult semi-definiteness problems   efficiently by using the algorithm. Furthermore, applying the algorithm to copositive problems, we find an explicit expression of the polynomials producing open weak CADs under some conditions, which significantly improves the efficiency of solving copositive problems.




\end{abstract}
\begin{keyword}
Open weak CAD, open weak delineable, CAD projection, semi-definiteness, copositivity.
\end{keyword}

\end{frontmatter}




\section{Introduction}
\label{secc:intro}
In this paper, we introduce the concept of {\em\ open weak CAD}, provide an algorithm for computing corresponding projection polynomials, and illustrate its usefulness through various applications. In the following, we elaborate on the above sentence.

Cylindrical Algebraic Decomposition (CAD)\ is a fundamental concept and tool for computational real algebraic geometry with numerous applications. It was introduced by
\citet{collins1,Caviness1998} and have been improved
by  many \citep{McCallum, Hong:90a, Collins_Hong:91, Hong:92a, LS93, Ren, BPR96,  McCallumeq, AnaiW01, Brown01a, Strzebonski06, Hong_Safey2012,Chen_Maza2014}.


For polynomials in  variables, a CAD is a finite collection of sign-invariant cells satisfying the following two requirements : (a) the cells constitute a  decomposition  of the whole -dimensional space.  (b) the cells  are cylindrically arranged.  Such a decomposition is typically constructed in two stages: (1)\  compute   a kind of triangular set of  polynomials through repeated projection, resulting  in so-called projection polynomials (2)\ carrying back substitutions by repeatedly solving the projection polynomials.

It  was immediately observed that the computation   of CAD can be     time-consuming. Hence there have been intensive and diverse effort to improve its computational efficiency.  For instance, it was observed that the algorithms spend  huge amount of time on  constructing low dimensional cells and that those cells are often not needed in various applications. Hence a relaxed notion called {\em\ open CAD} or {\em\ generic CAD} was introduced, where  low dimensional cells are ignored. Consequently, it relaxes the requirement (a) that the cells constitute a decomposition of the {\em whole\/} space.

In this paper, we introduce a further relaxed notion called {\em open weak CAD}, where we  also relax  the requirement (b)  that the cells  are cylindrically arranged.   Technically, the cylindricity is intimately related to  delineability.   We replace the original delineability requirement with a weaker version in such a way that it still captures sufficient amount of geometric information needed in various applications.

Furthermore, we  provide an algorithm for computing projection polynomials of open weak CADs. The key idea is to compute the intersection of projection factor sets produced by different projection orders. The resulting open weak CAD often has smaller number of cells than open CADs.



We illustrate the usefulness of open weak CAD theory and algorithm by tackling several application problems. First we show how to compute sample points for all open connected components of  for a given polynomial . Next we show how to test polynomial inequalities. Finally we  show how to tackle  co-positiveness problems; we find an explicit expression of the polynomials producing open weak CADs under some conditions, which significantly improves the efficiency of solving copositive problems.








The structure of this paper is as follows. In Section \ref{sec:problem}, we introduce a notion of  open weak delineability  and open weak CAD and then state the problem of finding projection polynomials. In Section \ref{sec:pre}, we review basic definitions, lemmas and concepts of CAD. In Section \ref{sec:refined}, we introduce several properties of open weak delineable, provide an algorithm for computing projection polynomials of open weak CADs  and prove its correctness. In Section \ref{sec:reduced_open_cad}, we apply open weak CAD theory and algorithm to  compute open sample. In Section \ref{sec:improved}, we apply open weak CAD theory and algorithm and some previous work \citep{han2016proving} to prove  polynomial inequalities.    In Section \ref{sec:copositive}, we again  apply open weak CAD theory and algorithm to  test co-positiveness. Section \ref{sec:applicat}, we provide  several examples which demonstrate the effectiveness of the algorithms. 



 \section{Problem: Open weak CAD}\label{sec:problem}



 In this section, we give a precise statement of the problem.  We begin by  introducing several notions.

 \begin{defn}[Open weak delineable] \label{def:weakopendeli}
Let   and let  be an open set of . We say that  is  {\em open weak delineable} (OWD) on  if, for any open connected component  defined by , we have either

where .
Let  for  We say
that  is {\em open weak delineable\/} over~ in , if  is open weak delineable on any open connected component of  in .
\end{defn}

\begin{ex}\label{ex:wod}
Let 
The plot of  is given by
\begin{figure}[ht]
\begin{centering}
        \includegraphics[width=2.0in]{example1}
\caption{Example \ref{ex:wod}\label{fig:ex21} }
\end{centering}
\end{figure}

\noindent
Note that  is analytically delineable (\citet{collins1}, \citet{McCallum2}) and also open weak delineable on the set .
Note that  is {\em not\/} analytically delineable  but {\em is\/} open weak delineable on the set . Note also that  is open weak delineable over~ in 
\end{ex}





\begin{defn}[Open weak CAD]
Let . A decomposition of , , is called an {\em open weak CAD\/} of  in  if and only if  is open weak delineable on every -dimensional open set in the decomposition. \end{defn}

\begin{ex}
Let  be the  polynomial from Example~\ref{ex:wod}, 
is an open weak CAD of  in .
\end{ex}

Finally, we are ready to state the problem precisely.\medskip

\noindent \textbf{Problem}. (Projection polynomials of open weak CAD) Devise an algorithm with the following specification.
\medskip
\begin{description}[leftmargin=3em,style=nextline,itemsep=0.5em]
\item[\sf In:]   
\item[\sf Out:]   where 
such that  is open weak delineable over  in .
\end{description}

\begin{ex}\label{ex:1}
Consider the following polynomial. \medskip
\begin{description}[leftmargin=3em,style=nextline,itemsep=0.5em]
\item[\sf In:]   
\item[\sf Out:]  
\item[]             
\end{description} \medskip
The left plot in Figure~\ref{fig:ex1ab} below shows the open weak CAD of   produced by  and~.
The factor  in  does not have real root and thus it does not contribute to the open weak CAD.
\end{ex}
\begin{figure}[h]
\begin{center}
\includegraphics[width=2.5in]{ex1a.jpg}
\includegraphics[width=2.5in]{ex1b.jpg}
\end{center}
\caption{\label{fig:ex1ab}Example \ref{ex:1} }
\end{figure}


\begin{rem} For comparison, if we apply an open CAD  algorithm on the above , one would obtain the following
output\medskip
\begin{description}[leftmargin=3em,style=nextline,itemsep=0.5em]
\item[\sf Out:]  
\item[]          
\end{description}  \medskip
The right plot in Figure~\ref{fig:ex1ab} shows the open  CAD of   produced by  and~.
Note that it has more cells than the open weak CAD (on the left).
\end{rem}


\begin{rem}
It is natural to wonder whether the multivariate discriminants of  always produce open weak CADs. Unfortunately, this is not true since the discriminant  (the multivariate discriminant of  with respect to ) may vanish identically and thus does not always produces an open weak CAD of .
One may also wonder whether if the multivariate discriminants of  do produce open weak CADs, then they would be the smallest open weak CADs. Unfortunately this is not true  either. In Example \ref{ex:wod}, it has been shown that  produces an open weak CAD of  with  open intervals. But the discriminant  produces an open weak CAD of  with  open intervals.
\end{rem}

\begin{rem}
The output of the above problem is a list of ``projection'' polynomials, not an open weak CAD. As usual, we can compute sample points in an open weak CAD of  from the projection polynomial  by standard lifting technique. We say that  produce an open weak CAD. Thus, in this paper, we will focus ourselves on the problem of finding projection polynomial of Open Weak CAD.
\end{rem}














\section{Preliminaries}\label{sec:pre}
If not specified, for a positive integer , let  be the list of variable  and  and  denote the point  and , respectively.





\begin{defn}
Let , denote by  and  the {\em leading coefficient} and the {\em discriminant} of  with respect to (w.r.t.) , respectively. The set of real zeros of  is denoted by . Denote by  or  the common real zeros of 
The {\em level} for  is the biggest  such that  where  is the degree of  w.r.t. .
        For polynomial set ,  is the set of polynomials in  with level .
\end{defn} \begin{comment}
\begin{defn}
        Let  be the symmetric permutation group of . Define  to be the subgroup of , where any element  of  fixes , {\it i.e.},  for .
\end{defn}
\end{comment}

\begin{defn}\label{de:sqrfree}
If  can be factorized in  as:
        
        where , ,  and  are
pairwise different irreducible primitive polynomials with positive leading coefficients under a suitable ordering and positive degrees in ,
then define
        
        If  is a constant, let  
\end{defn}


In the following, we introduce some basic known concepts and results of CAD. The reader is referred to \cite{collins1,Hong:90a,Collins_Hong:91,McCallum1,McCallum2,Brown01a} for a detailed discussion on the properties of CAD.

\begin{defn}\citep{collins1,McCallum1}\label{def:delineable}
        An -variate polynomial  over the reals is said to be {\em delineable} on a subset  (usually connected) of  if
        (1) the portion of the real variety of  that lies in the cylinder  over  consists of the union of the graphs of some  continuous functions  from  to ; and
        (2) there exist integers  s.t. for every , the multiplicity of the root  of  (considered as a polynomial in  alone) is .
\end{defn}


\begin{defn}\citep{collins1,McCallum1}
        In the above definition, the  are called the real root functions of  on , the graphs of the  are called the -{\em sections} over , and the regions between successive -sections are called -{\em sectors}.

\end{defn}

\begin{thm}\label{thm:McCallum}\citep{McCallum1,McCallum2}
        Let  be a polynomial in  of positive degree and  is a nonzero polynomial. Let  be a connected submanifold of  on which  is degree-invariant and does not vanish identically, and in which  is order-invariant. Then  is analytic delineable on  and is order-invariant in each -section over .
\end{thm}
Based on this theorem, McCallum proposed the projection operator \MCproj, which consists of the discriminant of  and all coefficients of .

\begin{thm}\label{thm:Brown}\citep{Brown01a}
        Let  be a -variate polynomial of positive degree in  such that  . Let  be a connected submanifold of  in which  is order-invariant, the leading coefficient of  is sign-invariant, and such that  vanishes identically at no point in .  is degree-invariant on .
\end{thm}
Based on this theorem, Brown obtained a reduced McCallum projection in which only leading coefficients, discriminants and resultants appear. The Brown projection operator is defined as follows.
\begin{defn} \label{def:brown projection}\citep{Brown01a}
        Given a polynomial , if  is with level ,
        the Brown projection operator for  is
        
        Otherwise, .
        If  is a polynomial set with level , then
        
        Define
        
\end{defn}

The following definition of {\em open CAD\/}  is essentially the GCAD introduced in \cite{Strzebonski}. For convenience, we use the terminology of open CAD in this paper.

\begin{defn} Open CAD \label{def:opencad}
        For a polynomial , an open CAD defined by  under the order  is a set of sample points in  obtained through the following three phases:
\begin{enumerate}
       \item[(1)] [Projection] Use the Brown projection operator on ,
                  let  \item[(2)] [Base] Choose one rational point in each of the open intervals defined by the real roots of ; \item[(3)] [Lifting] Substitute each sample point of  for  in  to get a univariate polynomial  and then, by the same method as Base phase, choose sample points for . Repeat the process for  from  to .
\end{enumerate}
\medskip
Sometimes, we say that the polynomial set  produces the open CAD or simply,  is an open CAD.
\end{defn}

\section{Open Weak CAD: Properties and Algorithm}\label{sec:refined}
In this section, we derive some basic properties of Open weak CAD, describe an algorithm (Algorithm \ref{alg:openweakcad}) for computing open weak CAD
and prove its correctness (Theorem~\ref{thm:openweakcad}). 

We first prove two simple but useful properties of open weak delineable. The first one is a transitive property.\begin{prop}  \label{prop:weaktransitive}
 Let , and  be an open set of . Suppose that there exists  and a polynomial  such that  is open weak delineable on , and  is open weak delineable on . Then  is open weak delineable on .
\end{prop}
\begin{proof}
Let  be any open connected component defined by  such that . We have  since . Let  be any open connected component defined by  such that . Now,  and  since  is open weak delineable on , and~ is open weak delineable on . Hence, .
\end{proof}
Before stating the next property, let us take an example to illustrate our motivation. Let . In this case  has only one open component . Let , , , . It is clear that  , and  is OWD over  () in , respectively. Since  and ,
  is OWD over . We note that  is not OWD over , while  and  only differ at a closed set of codimension . In general, we have the following Proposition.
\begin{prop} \label{prop:weakgcd}
 Let , suppose  is OWD over  in ,  is OWD over  in .\end{prop}

\begin{proof}
Let  be any open connected component defined by ,  be any open connected set defined by . Let , . It is clear that , . For any , we assume that . There exists an open set  containing , such that . Since  is OWD over , either  or . Hence, either  or  since  is a connected set, and it can not be partitioned into two nonempty subsets which are open. Therefore,  is OWD on , and  is OWD over~.
\end{proof}

Now the following two theorems follow immediately from the above Propositions. The first one states that the set 
 is nonempty, so the problem proposed in Section 2 makes sense. \begin{thm}\label{thm:owdnepty}
Let ,  is OWD over  in . As a result, the set  is nonempty.
\end{thm}
\begin{proof}
By Theorem \ref{thm:McCallum} and Theorem \ref{thm:Brown},  is OWD over , and  is OWD over  (). By Proposition \ref{prop:weaktransitive},  is OWD over  in .
\end{proof}
The next theorem says that there is a minimal element in  in some sense.
 In the following, we call the number of the open components in  defined by 
 the {\em scale} of the open weak CAD of  defined by  in .
\begin{thm}
Let , there exists , such that any , . In particular, the scale of the open weak CAD of  defined by  in  is minimal.
\end{thm}
\begin{proof}
  Otherwise, for any , there exists  such that . By Proposition \ref{prop:weakgcd}, . Thus, we can find a sequence of polynomials  (), such that the descending chain of closed sets  is not stationary, which contradicts with the well-known fact that  is noetherian under the Zariski Topology.
\end{proof}

We want to obtain an element in  as small as possible. A natural way is to apply Theorem \ref{thm:owdnepty} and Proposition \ref{prop:weakgcd}. Let us take  as an example. According to Theorem \ref{thm:owdnepty},  is OWD over  and  in , respectively. According to Proposition \ref{prop:weakgcd},  is OWD over  in .
If we want to obtain an element in  from , the simplest way is to apply Brown's projection operator directly, . But it is quite possible that  is more complicated than , since the degree of  is twice as much as that of .
For a polynomial , whether  or not is only dependent on the real zeros of . It indicates us that, instead of computing  directly, we may find a polynomial , such that  and  are almost the same, and . Roughly speaking, let , , . For simplicity, we suppose that  is an irreducible polynomial. It is clear that . Since , intuitively,  is a closed set of codimension . If  is not semi-definite, one can show that  is a closed set of codimension . Thus, the two sets  and  are almost the same. We will show that  is ``almost'' OWD over  in .

In order to state our results precisely, we introduce the following definitions.


\begin{defn}
Let  be a polynomial set, where . We say that  is a polynomial set of level  if . Define

Let , define

It is clear that  if and only if .

Let , where . Define
 to be the set of all the coefficients of all the polynomials  in  with respect to the indeterminates .

Let  define

We say that a polynomial  is a common factor of , if  is a factor of  for every .

We say that a polynomial set  of level  has codimension at least two and denote it by , if for any open connected set ,  is still open connected.
\end{defn}
Lemma \ref{lem:codim2} below gives a description of a polynomial set of codimension at least two. Before proving the lemma, we introduce the following result.

\begin{lem}\label{thm:1} \citep{han2016proving}
We have
\begin{enumerate}
\item Let  and  be coprime in . For any connected open set  of , the open set  is also connected.

\item Suppose  is a non-zero squarefree polynomial and  is a connected open set of . If  is semi-definite on , then  is also a connected open set.
\end{enumerate}
\end{lem}

\begin{lem}\label{lem:connected} Let  where ,  Suppose  has no real zeros in a connected open set , then the open set  is also connected.
\end{lem}
\begin{proof}
Without loss of generality, we may assume that . If  the result is obvious. The result of case  is just the claim of Lemma \ref{thm:1}. For , let  and  , then  and . Let , .
Since , we have . Notice that the closure of  equals the closure of , it suffices to prove that  is connected, which follows directly from Lemma \ref{thm:1} and the induction.
\end{proof}

\begin{lem}\label{lem:codim2}
Let  be a polynomial set of level .  if and only if for any common factor  of ,  is semi-definite on .
\end{lem}
\begin{proof}
  If ,  is connected for any open connected set . It is obvious that , and
  
  Notice that the closure of  equals the closure of ,  is open connected. In particular,  is open connected, and  is semi-definite on  since  is sign invariant on .

If for any common factor  of ,  is semi-definite on . Let , , , . By assumption,  is semi-definite on , and  is open connected by Lemma \ref{thm:1}. According to Lemma \ref{lem:connected},

is open connected since .
\end{proof}

By Lemma \ref{lem:codim2}, any common factor  of a polynomial set  of codimension at least 2 is semi-definite, by Theorem 4.5.1 in \citep{bochnak2013real}, . In fact, we can show that . That's why we call  has codimension at least 2.

\begin{defn}\label{def:generalweakopendeli}
Let ,  for  and  is a polynomial set of level , . We say that  is \emph{OWD over}  \emph{w.r.t.}  (in ), if for any open connected component  of  in ,  is OWD on . We also say that  is \emph{OWD} over  \emph{in general}.
\end{defn}\begin{rem}
If  is OWD over , it is clear that  is OWD over  w.r.t. . If  is OWD over  w.r.t. , and , by definition,  is OWD over  w.r.t.  since  is OWD on  for any open connected component  of  in . In particular, if  is OWD over  (w.r.t. ),  is OWD over  w.r.t.  for any polynomial set  of level .
\end{rem}

The following lemma shows that the above definition is just a variant of OWD when . In the rest of this paper, we will switch the two notations freely. \begin{lem}\label{lem:owdeq}
  Let , ,  is a polynomial set of level . If  is OWD over  w.r.t. ,  is OWD over . If ,  is OWD over  if and only if  is OWD over  w.r.t. . \end{lem}
\begin{proof}
If  is OWD over  w.r.t. . Let  be any open connected component of , there there exists a unique open connected component  of , such that . Since , .   is OWD on  since  is OWD on .

If , and  is OWD over . Let  be any open connected component of , and  be any open connected component of , such that . We have . Now , and  is open connected since . Thus, . Hence, , and  is OWD on .
\end{proof}
\begin{comment}
      and 

   We only need to prove that there is a one to one correspondence between open connected component  of  and open connected component  of , and .

  Let  be any open connected component of , there exists a unique open connected component  of , such that . On one side, since , . On the other side, , and  since  is open connected by the definition of . Thus, .

  Let  be any open connected component of , there exists an open connected component  of , such that . Repeat the above argument again, we see that .
\end{comment}

The following two theorems are analogous to Proposition \ref{prop:weaktransitive} and Proposition \ref{prop:weakgcd}. \begin{thm}\label{thm:weaktran}
  Let , , , ,  is a polynomial set of level . Suppose  is OWD over  w.r.t. , and  is OWD over .  is OWD over  w.r.t. . Furthermore, if ,   \end{thm}
\begin{proof}
  Let  be any open component of , . We prove that  is OWD on . Let  be any open component of , suppose . It is clear that . Let  be any open component of , such that .  is nonempty since  is a nonempty open set. Thus,  since  is OWD over  w.r.t. . We only need to show that . Since  and  is OWD over , . For any , let  such that .  implies that . Thus, there exists a polynomial , such that . Let  be a neighborhood of , such that , , and
   . This indicates that 

 If , let  be any common factor of . It is clear that  is a common factor of . Since ,  is semi-definite on  by Lemma \ref{lem:codim2}. By Lemma \ref{lem:codim2} again, .
The theorem is proved.
\end{proof}
As a special case of Theorem \ref{thm:weaktran}, when , and the set  has only one polynomial.  is OWD over  w.r.t. . In particular,  is OWD over  w.r.t. .

One benefit of Theorem \ref{thm:weaktran} is that we can reduce the computational complexity when we apply Brown's projection operator. Namely, suppose ,  is OWD over , where  is an irreducible polynomial and is semi-definite on . If we apply Brown's projection operator directly,  is OWD over . Now we use Theorem \ref{thm:weaktran} to get a simpler but stronger result. By Lemma \ref{lem:owdeq},  is OWD over  w.r.t. . According to Theorem \ref{thm:weaktran},  is OWD over  w.r.t. . By Lemma \ref{lem:owdeq} again,  is OWD over  which is a factor of .

\begin{thm}\label{thm:weakgcd}
   Let , ,  is a polynomial set of level  (), , . Suppose  is OWD over  w.r.t. .  is OWD over  w.r.t. , and  for any . Furthermore, if , .
\end{thm}\begin{proof}
Let  be any open connected component defined by ,  be the set of open components of  in . By definition, , and

Let  be any open connected set defined by , 

Since  is OWD over  w.r.t. , .
For any , and ,  since 
Let 

We have , and

By Lemma ,  is open connected, and can not be
partitioned into two nonempty subsets which are open. Hence, either  or .
Since 
and , , either  or . Thus, either  or . Therefore,  is OWD over  w.r.t. .

We have

for any .

Since , any common factor  of  must be a common factor of  for some . If , by Lemma \ref{lem:codim2},  is semi-definite on . By Lemma \ref{lem:codim2} again, .
\end{proof}
\begin{comment}

 and  is OWD over  over
Let \zero(pQ^2)=\zero(p)\cup\zero(Q^2)\subseteq\zero(p)\cup\zero(p_i'Q_i^2)=\zero(p_iQ_i^2),\Hproj(f,[w,z])=\gcd(\Hproj(f,[w,z],z),\Hproj(f,[w,z],w)),{\Hproj^{\vartriangle}(f,[w,z],w)}=\frac{\Hproj(f,[w,z],w)}{\Hproj(f,[w,z])},{\Hproj^{\vartriangle}(f,[w,z],z)}=\frac{\Hproj(f,[w,z],z)}{\Hproj(f,[w,z])}.{\Hproj^{\ast}(f,[w,z])}=\{{\Hproj^{\vartriangle}(f,[w,z],w)},{\Hproj^{\vartriangle}(f,[w,z],z)}\}\Hproj(f,[w,z,y],y)=\Bproj(\Hproj(f,[w,z]),y){\Hproj^{\ast}(f,[w,z,y],y)}=\coeffs({\Hproj^{\ast}(f,[w,z])},[y])\Hproj(f,[w,z,y],z),\, \Hproj(f,[w,z,y],w),\, \Hproj(f,[w,z,y]),{\Hproj^{\ast}(f,[w,z,y],z)},\,\Hproj^{\ast}(f,[w,z,y],w),{\Hproj^{\vartriangle}(f,[w,z,y],y)},{\Hproj^{\vartriangle}(f,[w,z,y],z)},\,\Hproj^{\vartriangle}(f,[w,z,y],w).
\Hproj^{\ast}(f,[w,z,y])=&\Hproj^{\vartriangle}(f,[w,z,y],y)\Hproj^{\ast}(f,[w,z,y],y)\bigcup\\
& \Hproj^{\vartriangle}(f,[w,z,y],z)\Hproj^{\ast}(f,[w,z,y],z)\bigcup\\
&\Hproj^{\vartriangle}(f,[w,z,y],w)\Hproj^{\ast}(f,[w,z,y],w)
h_1(f)=\Hproj(f,[w,z,y])\Hproj^{\ast}(f,[w,z,y])^2.
                 \Bproj(f,[y_i])    &=\Res(\sqrfree(f),\frac{\partial \sqrfree(f)}{\partial y_{i}}, y_{i}),\\
                \Hproj(f,[\yy],y_i) &=\Bproj(\Hproj(f,[\hat{\yy}_i],[y_i]),\\
                \Hproj(f,[\yy])     &=\gcd(\Hproj(f,[\yy],y_1),\ldots,\Hproj(f,[\yy],y_m)),\\
                \Hproj(f,[~])       &=f,
    
\Hproj^{\ast}(f,[~])&=\{1\},\\
{\Hproj^{\ast}(f,[\yy],y_i)}&=\coeffs({\Hproj^{\ast}(f,[\hat{\yy}_i])},y_i),\\
{\Hproj^{\vartriangle}(f,[\yy],y_i)}&=\frac{\Hproj(f,[\yy],y_i)}{\Hproj(f,[\yy])},\\
{\Hproj^{\ast}(f,[\yy])}&=\bigcup_{i=1}^m {\Hproj^{\vartriangle}(f,[\yy],y_i)}{\Hproj^{\ast}(f,[\yy],y_i)}.

\Hproj(f,\left[  x_{1},x_{2}\right]  )  & =\gcd\left(  \Hproj\left(  f,\left[
x_{1},x_{2}\right]  ,x_{1}\right)  ,\Hproj\left(  f,\left[  x_{1},x_{2}\right]  ,x_{2}\right)  \right),  \\
\Hproj\left(  f,\left[  x_{1},x_{2}\right]  ,x_{1}\right)    &
=\Bproj(\Hproj(f,\left[  x_{2}\right]  ),[x_{1}]),\\
\Hproj\left(  f,\left[  x_{1},x_{2}\right]  ,x_{2}\right)    &
=\Bproj(\Hproj(f,\left[  x_{1}\right]  ),[x_{2}]),\\
\Hproj(f,\left[  x_{2}\right]  )  & =\Hproj(f,[x_{2}],x_{2}),\\
\Hproj(f,\left[  x_{1}\right]  )  & =\Hproj(f,[x_{1}],x_{1}),\\
\Hproj(f,[x_{2}],x_{2})  & =\gcd(\Bproj(\Hproj(f,[~]),[x_{2}])),\\
\Hproj(f,[x_{1}],x_{1})  & =\gcd(\Bproj(\Hproj(f,[~]),[x_{1}])),\\
\Hproj(f,[~])  & =f.

\Hproj(f,\left[  x_{1},x_{2}\right]  )=\gcd\left(  \Bproj(\Bproj(f,[x_{2}]),[x_{1}]),\Bproj(\Bproj(f,[x_{1}]),[x_{2}])\right).

{\Hproj^{\ast}(f,[~])}&=\{1\},\\
{\Hproj^{\ast}(f,[x_1],x_1)}&=\coeffs({\Hproj^{\ast}(f,[~])},x_1)=\{1\},\\
{\Hproj^{\vartriangle}(f,[x_1],x_1)}&=\frac{\Hproj(f,[x_1],x_1)}{\Hproj(f,[x_1])}=1,\\
{\Hproj^{\ast}(f,[x_1])}&={\Hproj^{\vartriangle}(f,[x_1],x_1)}{\Hproj^{\ast}(f,[x_1],x_1)}=\{1\}.
{\Hproj^{\ast}(f,[x_2])}={\Hproj^{\vartriangle}(f,[x_2],x_2)}{\Hproj^{\ast}(f,[x_2],x_2)}=\{1\}.
{\Hproj^{\ast}(f,[x_2,x_1],x_2)}&=\coeffs({\Hproj^{\ast}(f,[x_1])},x_2)=\{1\},\\
{\Hproj^{\ast}(f,[x_2,x_1],x_1)}&=\coeffs({\Hproj^{\ast}(f,[x_2])},x_1)=\{1\},\\
{\Hproj^{\vartriangle}(f,[x_2,x_1],x_1)}&=\frac{\Hproj(f,[x_2,x_1],x_1)}{\Hproj(f,[x_2,x_1])},\\
{\Hproj^{\vartriangle}(f,[x_2,x_1],x_2)}&=\frac{\Hproj(f,[x_2,x_1],x_2)}{\Hproj(f,[x_2,x_1])},\\
{\Hproj^{\ast}(f,[x_2,x_1])}&=\bigcup_{i\in\{1,2\}}{\Hproj^{\vartriangle}(f,[x_2,x_1],x_i)}{\Hproj^{\ast}(f,[x_2,x_1],x_i)}\\
&=\{{\Hproj^{\vartriangle}(f,[x_2,x_1],x_1)},{\Hproj^{\vartriangle}(f,[x_2,x_1],x_2)}\}.

\Hproj(f,[x_n,\ldots,x_{j+1}]) \;\;\;\;\;\; \text{and}\;\;\;\;\;\;
{\Hproj^{\ast}(f,[x_n,\ldots,x_{j+1}])}
h_j(f)=\Hproj(f,[x_n,\ldots,x_{j+1}])\cdot{\Hproj^{\ast}(f,[x_n,\ldots,x_{j+1}])}^2.\begin{array}{ll}
\Hproj(f,[x_3])&=\Hproj(f,[x_3],x_3)=\Bproj(\Hproj(f,[~]),[x_3])=\Bproj(f,[x_3])\\
& =(x_2^2+x_1^2-1)(25x_2^2+12x_2x_1+20x_1^2-6x_2-4x_1-15),\\
\Hproj(f,[x_3, x_2],x_2) & =\Bproj(\Hproj(f,[x_3]),[x_2])=\Bproj(\Hproj(f,[x_3],x_3), [x_2])\\
  & = (x_1-1)(x_1+1)(29x_1^2-4x_1-24)(13x_1^2-4x_1-8), \\
\Hproj(f,[x_3, x_2],x_3) & = \Bproj(\Hproj(f,[x_2]),[x_3])=\Bproj(\Hproj(f,[x_2],x_2), [x_3])\\
  & = (x_1-1)(x_1+1)(29x_1^2-4x_1-24)(20x_1^2-4x_1-15),\\
\Hproj(f,[x_3,x_2])&=\gcd(\Hproj(f,[x_3, x_2],x_2),\Hproj(f,[x_3, x_2],x_3))\\
&=(x_1-1)(x_1+1)(29x_1^2-4x_1-24). \\
\\
{\Hproj^{\ast}(f,[x_3])}&={\Hproj^{\vartriangle}(f,[x_3],x_3)}{\Hproj^{\ast}(f,[x_3],x_3)}=\{1\},\\
{\Hproj^{\vartriangle}(f,[x_3,x_2],x_2)}&=\frac{\Hproj(f,[x_3,x_2],x_2)}{\Hproj(f,[x_3,x_2])}=13x_1^2-4x_1-8,\\
{\Hproj^{\vartriangle}(f,[x_3,x_2],x_3)}&=\frac{\Hproj(f,[x_3,x_2],x_3)}{\Hproj(f,[x_3,x_2])}=20x_1^2-4x_1-15,\\
{\Hproj^{\ast}(f,[x_3,x_2])}&=\{{\Hproj^{\vartriangle}(f,[x_3,x_2],x_2)},{\Hproj^{\vartriangle}(f,[x_3,x_2],x_3)}\}\\
&=\{13x_1^2-4x_1-8,20x_1^2-4x_1-15\}.
\end{array}\begin{array}{ll}
h_1(f) &=\Hproj(f,[x_3,x_2]){\Hproj^{\ast}(f,[x_3,x_2])}^2\\
    & =(x_1-1)(x_1+1)(29x_1^2-4x_1-24)((20x_1^2-4x_1-15)^2+(13x_1^2-4x_1-8)^2),\\
h_2(f) &=\Hproj(f,[x_3]){\Hproj^{\ast}(f,[x_3])}^2 \\
    &=(x_2^2+x_1^2-1)(25x_2^2+12x_2x_1+20x_1^2-6x_2-4x_1-15).
\end{array}
h_j(f)=\Hproj(f,[x_n,\ldots,x_{j+1}])\cdot{\Hproj^{\ast}(f,[x_n,\ldots,x_{j+1}])}^2,h_1(f) =(x_1-1)(x_1+1)(29x_1^2-4x_1-24).\Hproj(f,[x_n,\ldots,x_{k+1}])|\Hproj(f,[x_n,\ldots,x_{k+1}],x_{k+1})|\Bproj(f,[x_n,\ldots,x_{k+1}]),q_k|\Bproj(f,[x_n,\ldots,x_{k+1}]);
\zero(h_k)=&\zero(\Hproj(f,[x_n,\ldots,x_{k+1}]))\cup\zero({\Hproj^{\ast}(f,[x_n,\ldots,x_{k+1}])})\\
\subseteq&\zero(\Bproj(f,[x_n,\ldots,x_{k+1}])).
\Hproj(f,[x_n])=\Hproj(f,[x_n],x_n)=\Bproj(f,[x_n]),\,{\Hproj^{\ast}(f,[x_n])}=\{1\}.h_{n-1}=\Hproj(f,[x_n]){\Hproj^{\ast}(f,[x_n])}^2=\Hproj(f,[x_n]),\, \zero(h_{n-1})=\zero(\Bproj(f,[x_n])).\Hproj(f,[x_n,\ldots,x_{j+1}],x_{j+1})=\Bproj(\Hproj(f,[x_n,\ldots,x_{j+2}]),[x_{j+1}]),\Hproj(f,[x_n,\ldots,x_{j+2}])|\Bproj(f,[x_n,\ldots,x_{j+2}]).\Bproj(\Bproj(f,[x_n,\ldots,x_{j+2}]),[x_{j+1}])=\Bproj(f,[x_n,\ldots,x_{j+1}]),\Hproj(f,[x_n,\ldots,x_{j+1}])|\Bproj(f,[x_n,\ldots,x_{j+1}]).q_{j+1}|\Bproj(f,[x_n,\ldots,x_{j+2}]).\lc(q_{j+1},x_{j+1})|\lc(\Bproj(f,[x_n,\ldots,x_{j+2}]),x_{j+1})|\Bproj(f,[x_n,\ldots,x_{j+1}]).{\Hproj^{\vartriangle}(f,[x_n,\ldots,x_{j+1}],x_{j+1})}|\Hproj(f,[x_n,\ldots,x_{j+1}],x_{j+1})|\Bproj(f,[x_n,\ldots,x_{j+1}]).
&\zero({\Hproj^{\ast}(f,[x_n,\ldots,x_{j+1}])})\\
\subseteq &\zero(q_j)=\zero(\lc(q_{j+1},x_{j+1}))\cup\zero({\Hproj^{\vartriangle}(f,[x_n,\ldots,x_{j+1}],x_{j+1})})\\
\subseteq&\zero(\Bproj(f,[x_n,\ldots,x_{j+1}])).
\va_j\boxplus S=\{(\alpha_1,\ldots,\alpha_j,\beta)\mid \beta\in S\}.
B_1 & =[f, \Bproj(f,[x_n]),\dots, \Bproj(f,[x_n,\dots,x_2])],\\
B_2 &=[1,\ldots,1],\\
S   & ={\tt SPOne}(\Bproj(f,[x_n,\dots,x_2]),1),
(2\le i\le n)f\in\ZZ[\xx_n]f\Hproj(f,[x_n,\ldots,x_i])\Hproj(f,i)\Hproj(f,i)\in{\Hproj^{\ast}(f,[x_n,\ldots,x_i])}f\Hproj(f,[x_n,\ldots,x_i]){\Hproj^{\ast}(f,[x_n,\ldots,x_{i}])}f\Hproj(f,[x_n,\ldots,x_i])\Hproj(f,i)f(\xx_n)[x_n,\ldots,x_{j+1}]\RR^nOSOS=S_{\Hproj(f,[x_n,\ldots,x_{j+1}]),\Hproj(f,j+1)}\RR^{j}f(\xx_n)\in\ZZ[\xx_n]S_{\Hproj(f,[x_n,\ldots,x_{j+1}]),\Hproj(f,j+1)}\RR^{j}\RR^{n}P_j:=S_{\Hproj(f,[x_n,\ldots,x_{j+1}]),\Hproj(f,j+1)}ij+2n+1P_{i-1}:=\emptyset\vaP_{i-2}i\le nP_{i-1}:=P_{i-1}\bigcup (\va\boxplus {\tt SPOne}(\Hproj(f,[x_n,\ldots,x_{i}])(\va,x_{i-1}), \Hproj(f,i)(\va,x_{i-1})))P_{i-1}:=P_{i-1}\bigcup (\va\boxplus {\tt SPOne}(f(\va,x_{n}),\Hproj(f,n+1)(\va,x_n)))P_nf(\xx_n)[x_n,\ldots,x_{j+1}]f(\xx_n)P_jU\subseteq\RR^nf\neq0k\pi_{k}^n(U)\cap P_k\neq\emptysetk=jS\subseteq\RR^j\Hproj(f,[x_n,\ldots,x_{j+1}])\neq0S\cap \pi_{j}^n(U)\neq\emptysetf\Hproj(f,[x_n,\ldots,x_{j+1}])\Hproj(f,j+1)S\backslash\zero(\Hproj(f,j+1))\subseteq \pi_{j}^n(U)\va \in P_j\va\in S\backslash\zero(\Hproj(f,j+1))\va\in \pi_{j}^n(U)\cap P_jk=j,j+1,\ldots,ik=i+1\va\in P_i\va\in \pi_{i}^n(U)\Hproj(f,i+1)(\va)\neq0S\subseteq\RR^i\Hproj(f,[x_n,\ldots,x_{i+1}],x_{i+1})\neq0\va\va\in S\cap\pi_{i+1}^n(U)S\cap\pi_{i}^n(U)\neq \emptyset{\pi_{i}^{i+1}}^{-1}(S)\cap\pi_{i+1}^n(U)S'\Hproj(f,[x_n,\ldots,x_{i+2}])\neq0\Hproj(f,[x_n,\ldots,x_{i+2}])=fi=n-1S'\cap {\pi_{i}^{i+1}}^{-1}(S)\cap\pi_{i+1}^n(U)\neq\emptysetf\Hproj(f,[x_n,\ldots,x_{i+2}])\Hproj(f,i+2)\Hproj(f,[x_n,\ldots,x_{i+2}])\Hproj(f,[x_n,\ldots,x_{i+1}],x_{i+1})S\subseteq\pi_{i}^{i+1}(S')\va\in\pi_{i}^{i+1}(S')U_{\va}\subseteq\RR(\va,U_{\va})\in S'\Hproj(f,i+1)(\va)\neq0\lc(\Hproj(f,i+2),x_{i+1})(\va)\neq0\Hproj(f,i+2)(\va,x_{i+1}){\tt OpenSP}(\overline{\Hproj}(f,i),{\Hproj^{\vartriangle}}(f,i),OS)ff=(x_3^2+x_2^2+x_1^2-1)(4x_3+3x_2+2x_1-1)\in \ZZ[x_1, x_2,x_3]S_{\Hproj(f,[x_3,x_2]),\Hproj(f,2)}=\{-2,-\frac{27}{32},0,\frac{63}{64},2\}\RRP_1 :=\{-2,-\frac{27}{32},0,\frac{63}{64},2\}P_1\va_1,\ldots, \va_{5}P_2 := \emptysetP_2:=P_2\bigcup (\va_1\boxplus {\tt SPOne}(\Hproj(f,[x_3])(\va_1,x_2),\Hproj(f,3)(\va_1,x_2)))P_2:=P_2\bigcup (\va_2\boxplus {\tt SPOne}(\Hproj(f,[x_3])(\va_2,x_2),\Hproj(f,3)(\va_2,x_2)))\vdotsP_2:=P_2\bigcup (\va_5\boxplus {\tt SPOne}(\Hproj(f,[x_3])(\va_5,x_2),\Hproj(f,3)(\va_5,x_2)))P_2\va_1,\ldots, \va_{13}P_3 := \emptysetP_3:=P_3\bigcup (\va_1\boxplus {\tt SPOne}(f(\va_1,x_3),\Hproj(f,4)(\va_1,x_3)))P_3:=P_3\bigcup (\va_2\boxplus {\tt SPOne}(f(\va_2,x_3),\Hproj(f,4)(\va_2,x_3)))\vdotsP_3:=P_3\bigcup (\va_{13}\boxplus {\tt SPOne}(f(\va_{13},x_3),\Hproj(f,4)(\va_{13},x_3)))P_3\va_1,\ldots, \va_{36}P_3f(\xx_n)f(\xx_n)\Hproj(f,[x_n,\ldots,x_j])\RR^{j-1}f\RR^n\Hprojm\{x_1,\ldots,x_n\}mm=2[x_n,x_{n-1}][x_{n-2},x_{n-3}]f \in \ZZ[\xx_n]nff\neq0\RR^{n}g:=fL_1:=\{f\}L_2:=\{1\}h:=1i\ge 3L_1:=L_1\bigcup \{\Hproj(g,[x_i]),\Hproj(g,[x_i,x_{i-1}])\}h:=\lc(h,[x_i])L_2:=L_2\bigcup \{h\}h:=\lc(h,[x_{i-1}]){\Hproj^{\vartriangle}(g,[x_i,x_{i-1}],x_{i-1})}g:=\Hproj(g,[x_{i},x_{i-1}])i:=i-2i=2L_1:=L_1\bigcup \{\Hproj(g,[x_i])\}h:=\lc(h,[x_i])L_2:=L_2\bigcup \{h\}g:=\Hproj(g,[x_{i}])S{\tt SPOne}(L_1^{[1]},L_2^{[1]})C{\tt OpenSP}(L_1,L_2,S)C\Hproj(f,[x_{n},x_{n-1})]\neq \Bproj(f,[x_{n},x_{n-1}])n>3\Hprojm\Hproj\Nprojf\in \ZZ[\xx_{n}]n\NprojLn\Nprojf(\xx_n) \in \ZZ[\xx_n]f(\xx_n) \ge0\RR^n\Nprojf\in \ZZ[x_1,\dots,x_n]n[\yy]=[y_1,\dots,y_{m}]1\le m\le n y_i \in \{x_1,\dots,x_n\}1\le i\le my_i\neq y_ji\neq jm (m\ge2)i (1\le i \le m)\Nproj(f,[\yy],y_i)\Nproj(f,[\yy])\Nproj(f,i)\hat{[\yy]}_i=[y_1,\ldots,y_{i-1},y_{i+1},\ldots,y_m]n\ge2f\in \ZZ[\xx_n]U\Nproj(f,[x_{n}])\neq0\RR^{n-1}\Nproj_{1}(f,[x_n])UfV=U\backslash \bigcup_{h\in \Nproj_{1}(f,[x_n])}\zero(h)n\ge2f\in\ZZ[\xx_{n}]nU\Nproj(f,[x_{n}])\neq0\RR^{n-1}f(\xx_n)U\times \RR\Nproj_{1}(f,[x_n])Uf\in\ZZ[\xx_{n}]nU\Nproj(f,[x_{n}])\neq0\RR^{n-1}\Nproj_{1}(f,[x_n])UfU\backslash\zero({\Nproj}(f,n))f\Bproj(f,[x_n])\{1\}j2\le j\le nf(\xx_n)\in \ZZ[\bm{x}_n]U\Nproj(f,[x_n,\ldots,x_j])\neq0\RR^{j-1}\bigcup_{i=0}^{n-j} \Nproj_{1}(f,[x_{n-i}])U\times \RR^{n-j}f(\xx_n)S=U\backslash\zero(\Nproj(f,j))j2\le j\le nf(\xx_n)\in \ZZ[\bm{x}_n]U\Nproj(f,[x_n,\ldots,x_j])\neq0\RR^{j-1}\bigcup_{i=0}^{n-j} \Nproj_{1}(f,[x_{n-i}])U\times \RR^{n-j}f(\xx_n)S\overline{\Nproj}(f,j)\widetilde{\Nproj}(f,j)f\Bproj(f,[x_{n}])\neq0\RR^{n-1}\overline{\Hproj}(f,n){\Hproj^{\vartriangle}}(f,n)f\Nproj(f,[x_{n}])\neq0\RR^{n-1}\overline{\Nproj}(f,n)\widetilde{\Nproj}(f,n)f\in\ZZ[\xx_{n}]nU\Nproj(f,[x_{n},\ldots,x_j])\neq0\RR^{j-1}S=U\backslash\zero(\{\Nproj(f,[x_n,\dots,x_j],x_t)\mid t=j,\ldots,n\})f(\xx_n)U\times \RR^{n-j+1}(1)\bigcup_{i=0}^{n-j} \Nproj_{1}(f,[x_{n-i}])U\times \RR^{n-j}(2)\va \in Sf(\va,x_j,\ldots,x_n)\RR^{n-j+1}f\in\ZZ[\xx_{n}]nU\Nproj(f,[x_{n},\ldots,x_j])\neq0\RR^{j-1}S=U\backslash\zero(\Nproj(f,j))f(\xx_n)U\times \RR^{n-j+1}(1)\bigcup_{i=0}^{n-j} \Nproj_{1}(f,[x_{n-i}])U\times \RR^{n-j}(2)\va \in Sf(\va,x_j,\ldots,x_n)\RR^{n-j+1}jj=n-1n\ge3f\in\ZZ[\xx_{n}]nU\Nproj(f,[x_{n},x_{n-1}])\neq0\RR^{n-2}S=U\backslash \zero(\Nproj(f,n-1))f(\xx_n)U\times \RR^2(1)\Nproj_{1}(f,[x_n])\Nproj_{1}(f,[x_{n-1}])U\times \RR(2)\va\in Sf(\va,x_{n-1},x_n)\RR^2f \in \ZZ[\xx_n]\forall \va_n\in \RR^nf(\va_n)\ge0.n\le2\Proineq(f(x_n))L_1:=\Nproj_{1}(f,[x_n])\bigcup \Nproj_{1}(f,[x_{n-1}])L_2:=\Nproj(f,[x_{n},x_{n-1}])gL_1(g)=C_{n-2}:=L_2[x_{n-2},\ldots,x_2]\zero(\Nproj(f,n-1))\cap C_{n-2}=\emptyset\exists \va_{n-2}\in C_{n-2}\Proineq(f(\va_{n-2},x_{n-1},x_n))n\times nA_n\xx_nA_n\xx_n^T\ge0\xx_n\xx_nA_n\xx_n^TA_nxyn\le6n\xx_nA_n\xx_n^T(x_1^2,\ldots,x_n^2)A_n(x_1^2,\ldots,x_n^2)^T\OO(n^2 4^n)nnf\in\ZZ[\xx_{n}]nU\Nproj(f,[x_{n},\ldots,x_j])\neq0\RR^{j-1}S=U\backslash\zero(\{\Nproj(f,[x_n,\dots,x_j],x_t)\mid t=j,\ldots,n\})f(\xx_n)U\times \RR^{n-j+1}(1)\bigcup_{i=0}^{n-j} \Nproj_{1}(f,[x_{n-i}])U\times \RR^{n-j}(2)\va \in Sf(\va,x_j,\ldots,x_n)\RR^{n-j+1}a,b,c\in \ZZ,d,e,f\in\ZZ[\zz_{n}]a\neq0c\neq 0Fyd,e2f4\Bproj(F,[x,y])F_1=a(4acy^4+4aey^2+4af-b^2y^4-2by^2d-d^2)cy^4+ey^2+fF_1\Nproj(F,[x])=F_1Fcy^4+ey^2+f\Nproj(F,[x])\neq0cy^4+ey^2+fFF_2=c(4cx^4a+4cdx^2+4fc-b^2x^4-2bx^2e-e^2)ax^4+dx^2+fF_2\Nproj(F,[y])=F_2Fax^4+dx^2+f\Nproj(F,[y])\neq0ax^4+dx^2+fFax^4+dx^2+fcy^4+ey^2+fT_2\Nproj(F,[x,y])=\gcd(\Bproj(\Nproj(F,[x]),y)\Bproj(\Nproj(F,[y]),x))=(4ac-b^2)(4afc-ae^2-d^2c+edb-fb^2)\neq0\Res(\Nproj(F,[x]),y)T_2T_1\Nproj(F,[x])\neq0T_2cy^4+ey^2+fT_1FT_1d,e2f4\Nproj(F,[x])\Nproj(F,[x,y])4I\{1,\dots,n\}iII[i]\in \{1,\dots, n\} I[i]< I[i+1]i=1,\dots,|I|-1I\{1,\dots, n\}m=|I|\xx_I=[x_{I[1]},\ldots,x_{I[m]}]\overline{\xx_I}=\{x_i\mid i\not\in I\}A_I=(a_{{I[i],I[j]}})_{1\le i,j\le m}A_{n}\xx_nA_{n+1}=(a_{i,j})_{(n+1)\times(n+1)}a_{i,j}=a_{j,i}\in \ZZ[\zz_s]1\le i,j\le n+1F(\xx_n)=f(x_1^2,\ldots,x_n^2)I\{1,\dots, n\}mp_1,\ldots,p_{m+1}\in \ZZ[\zz_s,\overline{\xx_I}]F(\xx_n)=(\xx_I^2,1)P_I(\xx_I^2,1)^TP_{[1,\dots,m]}P_{m+1}(x_{m+1},\ldots,x_n)P_{m+1}P_{n+1}=A_{n+1}F(\xx_n)=(x_1^2,\ldots, x_i^2,1)P_{i+1}(x_1^2,\ldots,x_i^2,1)^TF(x_1,x_2,x_3)=x_1^4+2x_2^4+4x_1^2x_2^2-2x_1^2x_3^2+4x_2^2x_3^2+8x_1^2z^2+5x_3^4+z^4I=[1,2]\xx_I=[x_1,x_2],\overline{\xx_I}=\{x_3\}I=[1]\xx_I=[x_1],\overline{\xx_I}=\{x_2,x_3\}RnPk<nQ\in \RR^{k\times (n-k)}M\in \RR^{(n-k)\times k }N\in \RR^{(n-k)\times (n-k) }RMM^{(i,j)}ijMf\in\ZZ[\zz_s][\xx_n]F(\xx_n)=f(x_1^2,\ldots,x_n^2)F(\xx_n)|_{x_i=0} i\in \{1,\dots,n\}\det(A_I)=P_{I}^{(|I|+1,|I|+1)}I\{1,\ldots,n \}\gcd(A_{I}^{(1,1)},\dots,A_{I}^{(|I|,|I|)})=1|I|\ge2\{1,\ldots,n \}\gcd(P_{I}^{(1,1)},\dots,P_{I}^{(|I|,|I|)})=1|I|\ge2\{1,\ldots,n \} \det(P_{I})I\{1,\ldots,n \}\gcd(\det(P_{I}),\det(A_{I}))=1I\{1,\ldots,n\}\Nproj(F,[\xx_n])=\det(A_{n})\det(A_{n+1})nn=1F(\xx_1)=a_{1,1}x_1^4+2a_{1,2}x_1^2+a_{2,2}a_{1,1}\neq0
, a_{2,2}\neq0, a_{1,1}a_{2,2}-a_{1,2}^2\neq0a_{2,2}a_{1,1}a_{2,2}-a_{1,2}^2\Nproj(F,[\xx_1])=a_{1,1}(a_{1,1}a_{2,2}-a_{1,2}^2)=\det(A_1)\det(A_2)k1\le k < nn=kI\{1,\dots,n\}|I|=n-1I=[1,\dots,n-1]A_{I}=(a_{i,j})_{1\le i,j< n}B=(a_{1,n},\ldots,a_{n-1,n})C=(a_{1,n+1},\ldots,a_{n-1,n+1})D=a_{n,n}x_n^4+2a_{n,n+1}x_n^2+a_{n+1,n+1}F(\xx_n)F|_{x_i=0}i\in I\det(A_I')\gcd(A_{I'}^{(1,1)},\dots,A_{I'}^{(|I'|,|I'|)})=1\gcd(P_{I'}^{(1,1)},\dots,P_{I'}^{(|I'|,|I'|)})=1\gcd(\det(P_{I'}),\det(A_{I'}))=1I'I|I'|\le n-1 \Nproj(F(\xx_n),[\xx_I])=\det(A_I)\det(P_{I})\det(P_{I})\det(A_{I})=P_{I}^{(n,n)}\lambda=(a_{n,n}-BA_I^{-1}B^T)\mu=a_{n,n+1}-BA_I^{-1}C^T\nu=a_{n+1,n+1}-CA_I^{-1}C^T\det(A_I)\lambda\det(A_I)\nuA_{n+1}nH=\det(P_{I})\det(A_I)A_{n+1}^{(n+1,n+1)}1\le i\le n,\gcd(A_{n+1}^{(1,1)},\cdots, A_{n+1}^{(n,n)})=1\gcd(A_{n}^{(1,1)},\cdots, A_{n}^{(n,n)})=1\det(A_{n})\det(A_{n+1})\gcd(\det(A_{n}),\det(A_{n+1}))=1f(\xx_n)=(\xx_n,1)A_{n+1}(\xx_n,1)^TF(\xx_n)=f(x_1^2,\ldots,x_n^2)a_{i,j}f(\xx_{n})a_{i,j}=a_{j,i}a_{i,j}f(\xx_{n})a_{i,j}=a_{j,i}F(\xx_n)=f(x_1^2,\ldots,x_n^2)\Nproj(F,[\xx_n])=\det(A_{n})\det(A_{n+1})m|I|=mP_{I}^{(i,i)}\det(P_{I})\ZZ[\bm{a}_{i,j}][\xx_n]n\ge m\bm{a}_{i,j}=(a_{1,1},\ldots,a_{n+1,n+1})\det(P_{I})nn=mm\le l\le n-1l=nI=[1,2,\ldots,m]F(\xx_n)p_{i,n}=\sum_{j=m+1}^na_{i,j}x_j^2+a_{i,n+1}1\le i\le mp_{m+1,n}=\sum_{j=m+1}^n a_{j,j}x_j^4+\sum_{j=m+1}^n 2a_{n+1,j}x_j^2+\sum_{m+1\le i<j\le n} 2a_{i,j}x_i^2x_j^2+a_{n+1,n+1}B=(a_{1,n},\ldots,a_{m,n})C=(p_{1,n-1},\ldots,p_{m,n-1})D=p_{m+1,n}=a_{n,n}x_n^4+2a_{n,n+1}x_n^2+2\sum_{m+1\le i< n} a_{i,n}x_i^2x_n^2+a'_{n+1,n+1}a'_{n+1,n+1}\deg(a'_{n+1,n+1},x_n)=0P_{I,F}\det(P_{I,F})\lambda=(a_{n,n}-BA_I^{-1}B^T)\mu=a_{n,n+1}+\sum_{m+1\le i< n} a_{i,n}x_i^2-BA_I^{-1}C^T\nu=a'_{n+1,n+1}-CA_I^{-1}C^T \det(A_I)\lambda=\det(P_{I,G})\det(A_I)\nu=\det(P_{I,H})\det(A_I)\lambda\det(A_I)\nu\deg(\det(A_I)\mu,a_{n,n+1})>0\deg(\det(A_I)\lambda,a_{n,n+1})=\deg(\det(A_I)\nu,a_{n,n+1})=0\det(A_I)\mu\neq \pm (\det(A_I)\lambda\cdot \det(A_I)\nu+1)\det(A_I)\mu\neq \pm (\det(A_I)\lambda+\det(A_I)\nu)\mathcal{R}\pm1a,b,c\in \mathcal{R}b\neq \pm (ac+1)b\neq \pm (a+c)a,c\mathcal{R}T(x)=ax^4+bx^2+c\mathcal{R}[x]T(x)=g(x)h(x)g,h\mathcal{R}[x]\alpha\in \mathcal{R}T(x)-\alphaT(x)(x^2-\alpha^2)T\deg(g)=\deg(h)=2g=g_0+g_1x+g_2x^2h=h_0+h_1x+h_2x^2g_i,h_i\in\mathcal{R}Tghc=g_0h_00=g_0h_1+g_1h_0h_1g_2+h_2g_1=0c|g_0c\nmid h_0h_1\neq0lc^l|h_1l+1c^{l+1}|g_1h_1g_2+h_2g_1=0c|g_2c|gh(a,c)=1h_1=0g_1=0g_0=c,h_0=1(g_2=\pm a,h_2=\mp1)(g_2=\pm1,h_2=\mp a)b\neq \pm (ac+1)b\neq \pm (a+c)g(\xx_{n})=\sum_{1\le i,j\le n} a_{i,j}x_ix_j=\xx_{n}A_{n}\xx_{n}^Ta_{i,j}a_{i,j}=a_{j,i} (1\le i,j\le n)G(\xx_n)=g(x_1^2,\ldots,x_n^2)\Nproj(G,[\xx_n])=\det(A_{n})\Nproj(G,[\xx_n])=\gcd(\Nproj(G,[\xx_n],1 ),\ldots,\Nproj(G,[\xx_n],n))=\det(A_{n}).g(\xx_{n})=\sum_{1\le i,j\le n} a_{i,j}x_ix_j=\xx_{n}A_{n}\xx_{n}^Ta_{i,j}a_{i,j}=a_{j,i} (1\le i,j\le n)A_{n}=(a_{i,j})_{i,j=1}^{n}\Hproj(g,[\xx_n])=\det(A_{n})=\discrim(g,[\xx_n])\discrim(g,[\xx_n])g\ZZ[\bm{a}_{i,j}]g\Hproj(g,[\xx_n])gnd\bm{\alpha}=(\alpha_1,\ldots,\alpha_n)|\bm{\alpha}|=\sum_{i=1}^n \alpha_i\bm {x}^{\bm{\alpha}}=\prod_{i=1}^n x_i^{\alpha_i}\{\bm{C}_{\bm{\alpha}}\}=\{C_{\bm{\alpha}}||\bm{\alpha}|=d\}N=(\begin{subarray}{c}n+d-1\\n-1 \end{subarray})\discrim(g,[x_n,\ldots,x_1])g(\xx_n)d\Hproj(g,[\xx_n])g(x,y,z)dg(x_1,\ldots,x_n)a_{i,j}f(\xx_{n})a_{i,j}=a_{j,i}f(\xx_n)a_{i,j}a_{i,j}=a_{j,i} (1\le i,j\le n+1)A_{n}=(a_{i,j})_{i,j=1}^{n}A_{n+1}=(a_{i,j})_{i,j=1}^{n+1}F(\xx_n)=f(x_1^2,\ldots,x_n^2)\Nproj(F,[\xx_n])=\det(A_{n})\det(A_{n+1})g(\xx_{n})=\sum_{1\le i,j\le n} a_{i,j}x_ix_j=\xx_{n}A_{n}\xx_{n}^Ta_{i,j}a_{i,j}=a_{j,i} (1\le i,j\le n)A_{n}=(a_{i,j})_{i,j=1}^{n}G(\xx_n)=g(x_1^2,\ldots,x_n^2)\Nproj(G,[\xx_n])=\det(A_{n})\overline{\Nproj}(F,{n-1})=\{f,\Nproj(f,[x_{n}]), \ldots, \Nproj(f,[x_{n},\ldots,x_2)]\}f(\xx_n)f(\xx_n)F(\xx_n) \in \ZZ[\xx_n]n\ge1x_n\prec x_{n-1}\cdots \prec x_1QF(\xx_n) \ge0\RR^nF\in Qi1n\TCPT(F(x_1,x_2,\ldots,x_n)|_{x_i=0},Q)=Q:=Q\cup (F(x_1,x_2,\ldots,x_n)|_{x_i=0})g_n(x_n):=\det(P_{n})F(\xx_n)=(x_1^2,\ldots, x_{n-1}^2,1)P_{n}(x_1^2,\ldots,x_{n-1}^2,1)^TO:={\tt SPOne}(g_n,1)i2nS:=\emptysetg_{n-i+1}(x_{n-i+1},\ldots,x_n):=\det(P_{n-i+1})F(\xx_n)=(x_1^2,\ldots,x_{n-i}^2,1)P_{n-i+1}(x_1^2,\ldots,x_{n-i}^2,1)^T\vaOS:=S\bigcup (\va\boxplus {\tt SPOne}(g_{n-i+1}(x_{n-i+1},\va),1))O:=S\exists \va_{n}\in OF(\va_{n})<0f(\xx_n)f(\xx_n)f(\xx_n)f(\xx_n)\det(P_k)1\le k\le nP_kk^2\det(P_k)\det(P_k)g_k\OO(k^2(k-2)^3+k^2(n-k)^2)g_k(x_k,\va)g_k(x_k,\va)\OO(1)(g_k(x_k,\va),1)O3^{i-1}g_k(x_k,\va)\OO(k^2)\va18F(\va_n)3^n18\OO(n^2)Fn^218\OO(n^23^n)8-19\OO(n^23^n)Q\sum_{k=0}^{n} {n \choose k}=2^n.\OO(n^22^{2n}).(g_k(x_k,\va),1)\OO(3^n n^2)a\prec b\prec c\prec xf13215z\succ y \succ xf11387100f(x,y,z)8\Bproj(f,[z,y])\Bproj(f,[y,z])\Hproj(f,[y,z])\Hproj(f,[y,z])\Bproj(f,[z,y])\Bproj(f,[y,z])3044\TwoHp0.13 0.29262{\tt open\ CAD}0.19 3.11 48653{\tt randpoly([seq(x[i],}{\tt i=1..5)], degree=3)+add(x[i]^2,i=1..5)-1}105310\TwoHp2.87 3.512894{\tt open\ CAD}0.76 12.0178024453x_{n+1}=x_1\inftyn58\infty\infty\infty\infty11\infty\infty\infty\infty17\infty\infty\infty\infty\infty23\infty\infty\infty\infty\infty30\infty\infty\infty\infty\inftyn\TCPT\TwoPro\RAGlib1.343.958.25\Proineq\infty\infty\infty\infty\FI\infty\infty\infty\infty\PCAD\infty\infty\infty\infty\QEPCAD\infty\infty\infty\infty\inftyG(\bm{x}_{n})n20\infty\infty\infty\infty30\infty\infty\infty\inftyx_{3m+2+r}=x_rm=1n=5m12\infty\infty\infty3\infty\infty\infty\infty4\infty\infty\infty\infty\infty400034f(\xx_n)$ is symmetric, the new projection operator \Hproj\ cannot reduce the projection scale and the number of sample points.
Thus, it is reasonable to conclude that the complexity of \TwoPro\ is still doubly exponential.
\end{rem}

\section*{Acknowledgements}
Part of this paper is written while the first author visited Princeton University and North Carolina State University, and he would like to thank the institutes for their hospitality. He would like to thank ``Training, Research and Motion" (TRAM) network for supporting him visiting Princeton University. The first author would also like to thank his advisor Gang Tian for his constant support and encouragement.

The authors  thank M. Safey El Din who provided us several examples and communicated with us on the usage of \RAGlib.
The authors would like to convey their gratitude to all the four referees of our ISSAC'2014 paper and two referees of this paper, who provided their valuable comments, advice and suggestion, which helped improve this paper greatly on not only the presentation but also the technical details.

\bibliographystyle{elsart-harv}
\begin{thebibliography}{32}
\expandafter\ifx\csname natexlab\endcsname\relax\def\natexlab#1{#1}\fi
\expandafter\ifx\csname url\endcsname\relax
  \def\url#1{\texttt{#1}}\fi
\expandafter\ifx\csname urlprefix\endcsname\relax\def\urlprefix{URL }\fi

\bibitem[{Anai and Weispfenning(2001)}]{AnaiW01}
Anai, H., Weispfenning, V., 2001. Reach set computations using real quantifier
  elimination. In: HSCC. pp. 63--76.

\bibitem[{Andersson et~al.(1995)Andersson, Chang, and
  Elfving}]{andersson1995criteria}
Andersson, L.-E., Chang, G., Elfving, T., 1995. Criteria for copositive
  matrices using simplices and barycentric coordinates. Linear Algebra and Its
  Applications 220, 9--30.

\bibitem[{Basu et~al.(1996)Basu, Pollack, and Roy}]{BPR96}
Basu, S., Pollack, R., Roy, M.-F., 1996. On the combinatorial and algebraic
  complexity of quantifier elimination. Journal of ACM 43~(6), 1002--1045.

\bibitem[{Basu et~al.(1998)Basu, Pollack, and Roy}]{basu1998new}
Basu, S., Pollack, R., Roy, M.-F., 1998. A new algorithm to find a point in
  every cell defined by a family of polynomials. In: Quantifier elimination and
  cylindrical algebraic decomposition. Springer, pp. 341--350.

\bibitem[{Bochnak et~al.(2013)Bochnak, Coste, and Roy}]{bochnak2013real}
Bochnak, J., Coste, M., Roy, M.-F., 2013. Real algebraic geometry. Vol.~36.
  Springer Science \& Business Media.

\bibitem[{Brown(2001)}]{Brown01a}
Brown, C., 2001. Improved projection for cylindrical algebraic decomposition.
  J. Symb. Comput. 32~(5), 447--465.

\bibitem[{Chen and Maza(2014)}]{Chen_Maza2014}
Chen, C., Maza, M.~M., 2014. Cylindrical algebraic decomposition in the
  regularchains library. In: Mathematical Software ?ICMS 2014. Lecture Notes in
  Computer Science. Vol. 8592. pp. pp 425--433.

\bibitem[{Collins(1975)}]{collins1}
Collins, G.~E., 1975. Quantifier elimination for real closed fields by
  cylindrical algebraic decompostion. In: Automata Theory and Formal Languages
  2nd GI Conference Kaiserslautern, May 20--23, 1975. Springer, pp. 134--183.

\bibitem[{Collins(1998)}]{Caviness1998}
Collins, G.~E., 1998. Quantifier elimination by cylindrical algebraic
  decomposition--twenty years of progress. In: Quantifier elimination and
  cylindrical algebraic decomposition. Springer, pp. 8--23.

\bibitem[{Collins and Hong(1991)}]{Collins_Hong:91}
Collins, G.~E., Hong, H., 1991. Partial cylindrical algebraic decomposition for
  quantifier elimination. Journal of Symbolic Computation 12~(3), 299--328.

\bibitem[{Faug{\`e}re et~al.(2008)Faug{\`e}re, Moroz, Rouillier, and
  El~Din}]{faugere2008classification}
Faug{\`e}re, J.-C., Moroz, G., Rouillier, F., El~Din, M.~S., 2008.
  Classification of the perspective-three-point problem, discriminant variety
  and real solving polynomial systems of inequalities. In: Proceedings of the
  twenty-first international symposium on Symbolic and algebraic computation.
  ACM, pp. 79--86.

\bibitem[{Hadeler(1983)}]{hadeler1983copositive}
Hadeler, K., 1983. On copositive matrices. Linear Algebra and its Applications
  49, 79--89.

\bibitem[{Han(2011)}]{han2011}
Han, J., 2011. An Introduction to the Proving of Elementary Inequalities. Vol.
  123. Harbin Institute of Technology Press.

\bibitem[{Han(2016)}]{han2016multivariate}
Han, J., 2016. Multivariate discriminant and iterated resultant. Acta
  Mathematica Sinica, English Series 32~(6), 659--667.

\bibitem[{Han et~al.(2014)Han, Dai, and Xia}]{han2014constructing}
Han, J., Dai, L., Xia, B., 2014. Constructing fewer open cells by gcd
  computation in cad projection. In: Proc. ISSAC'2014. ACM, pp. 240--247.

\bibitem[{Han et~al.(2016)Han, Jin, and Xia}]{han2016proving}
Han, J., Jin, Z., Xia, B., 2016. Proving inequalities and solving global
  optimization problems via simplified cad projection. Journal of Symbolic
  Computation 72, 206--230.

\bibitem[{Hiriart-Urruty and Seeger(2010)}]{hiriart2010variational}
Hiriart-Urruty, J.-B., Seeger, A., 2010. A variational approach to copositive
  matrices. SIAM review 52~(4), 593--629.

\bibitem[{Hong(1990)}]{Hong:90a}
Hong, H., 1990. An improvement of the projection operator in cylindrical
  algebraic decomposition. In: International Symposium of Symbolic and
  Algebraic Computation (ISSAC-90). ACM, pp. 261--264.

\bibitem[{Hong(1992)}]{Hong:92a}
Hong, H., 1992. Simple solution formula construction in cylindrical algebraic
  decomposition based quantifier elimination. In: International Conference on
  Symbolic and Algebraic Computation ISSAC-92. pp. 177--188.

\bibitem[{Hong and Safey El~Din(2012)}]{Hong_Safey2012}
Hong, H., Safey El~Din, M., Jul. 2012. Variant quantifier elimination. J. Symb.
  Comput. 47~(7), 883--901.

\bibitem[{Liska and Steinberg(1993)}]{LS93}
Liska, R., Steinberg, S., 1993. Applying quantifier elimination to stability
  analysis of difference schemes. Comput. J. 36~(5), 497--503.

\bibitem[{{McCallum}(1984)}]{McCallum}
{McCallum}, S., 1984. An improved projection operator for {C}ylindrical
  {A}lgebraic {D}ecomposition. Ph.D. thesis, University of Wisconsin-Madison.

\bibitem[{McCallum(1988)}]{McCallum1}
McCallum, S., 1988. An improved projection operation for cylindrical algebraic
  decomposition of three-dimensional space. J. Symb. Comput. 5~(1), 141--161.

\bibitem[{McCallum(1998)}]{McCallum2}
McCallum, S., 1998. An improved projection operation for cylindrical algebraic
  decomposition. In: Quantifier Elimination and Cylindrical Algebraic
  Decomposition. Springer, pp. 242--268.

\bibitem[{McCallum(1999)}]{McCallumeq}
McCallum, S., 1999. On projection in cad-based quantifier elimination with
  equational constraint. In: Proc. ISSAC. pp. 145--149.

\bibitem[{Murty and Kabadi(1987)}]{murty1987some}
Murty, K.~G., Kabadi, S.~N., 1987. Some np-complete problems in quadratic and
  nonlinear programming. Mathematical programming 39~(2), 117--129.

\bibitem[{Parrilo(2000)}]{parrilo2000structured}
Parrilo, P.~A., 2000. Structured semidefinite programs and semialgebraic
  geometry methods in robustness and optimization. Ph.D. thesis, CIT.

\bibitem[{Renegar(1992)}]{Ren}
Renegar, J., 1992. On the computational complexity and geometry of the first
  order theory of the reals. Journal of {S}ymbolic {C}omputation 13~(3),
  255--352.

\bibitem[{Safey El~Din(2007)}]{el2007testing}
Safey El~Din, M., 2007. Testing sign conditions on a multivariate polynomial
  and applications. Mathematics in Computer Science 1~(1), 177--207.

\bibitem[{Safey El~Din and Schost(2003)}]{safey2003polar}
Safey El~Din, M., Schost, {\'E}., 2003. Polar varieties and computation of one
  point in each connected component of a smooth real algebraic set. In: Proc.
  ISSAC'2003. ACM, pp. 224--231.

\bibitem[{Strzebo{\'n}ski(2000)}]{Strzebonski}
Strzebo{\'n}ski, A., 2000. Solving systems of strict polynomial inequalities.
  J. Symb. Comput. 29~(3), 471--480.

\bibitem[{Strzebonski(2006)}]{Strzebonski06}
Strzebonski, A., 2006. Cylindrical algebraic decomposition using validated
  numerics. J. Symb. Comput. 41~(9), 1021--1038.

\end{thebibliography}

\end{document}
