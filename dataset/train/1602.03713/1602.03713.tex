\documentclass[11pt]{article}

\usepackage[utf8]{inputenc}
\usepackage[T1]{fontenc}
\usepackage{lmodern}

\usepackage{makeidx}
\usepackage{graphicx,comment}
\usepackage[linesnumbered,vlined]{algorithm2e}
\newcommand{\todo}[1]{\typeout{TODO: #1}\textbf{[[[ #1 ]]]}}

\usepackage{amsmath}
\usepackage{amsthm}
\usepackage{amsfonts, amssymb}
\usepackage{cite}
\usepackage{fullpage}

\newtheorem{theorem}{Theorem}[section]
\newtheorem{lemma}[theorem]{Lemma}
\newtheorem{corollary}[theorem]{Corollary}
\newtheorem{claim}[theorem]{Claim}
\newtheorem{definition}[theorem]{Definition}
\newtheorem{axiom}[theorem]{Axiom}

\newcommand{\concept}[1]{\textbf{#1}}
\newcommand{\etal}{\textit{et al.}\xspace}
\newenvironment{lemma-repeat}[1]{\begin{trivlist}
\item[\hspace{\labelsep}{\bf\noindent Lemma \ref{#1} }]\em }{\end{trivlist}}
\newenvironment{theorem-repeat}[1]{\begin{trivlist}
\item[\hspace{\labelsep}{\bf\noindent Theorem \ref{#1} }]\em }{\end{trivlist}}

\renewcommand\topfraction{0.85}
\renewcommand\bottomfraction{0.85}
\renewcommand\textfraction{0.1}
\renewcommand\floatpagefraction{0.85}
\newcommand*\samethanks[1][\value{footnote}]{\footnotemark[#1]}

\newcommand{\Amount}{t}
\newcommand{\Deal}{request}
\newcommand{\Vault}{vault}
\newcommand{\Bank}{bank}
\newcommand{\Budget}{budget}
\newcommand{\InCover}{\texttt{InCover}}
\newcommand{\NotInCover}{\texttt{NotInCover}}
\newcommand{\ceil}[1]{\lceil #1 \rceil}

\begin{document}
\begin{titlepage}
\title{A Distributed -Approximation for\\ Vertex Cover in  Rounds}
\author{Reuven Bar-Yehuda\thanks{Technion, Department of Computer Science, \texttt{reuven@cs.technion.ac.il}, \texttt{ckeren@cs.technion.ac.il}, \texttt{gregorys@cs.technion.ac.il}. Supported in part by the Israel Science Foundation (grant 1696/14).}
\and Keren Censor-Hillel\samethanks \and Gregory Schwartzman\samethanks}
\maketitle

\begin{abstract}
We present a simple deterministic distributed -approximation algorithm for minimum weight vertex cover, which completes in  rounds, where  is the maximum degree in the graph, for any  which is at most . For a constant , this implies a constant approximation in  rounds, which contradicts the lower bound of [KMW10].
\end{abstract}

\thispagestyle{empty}
\end{titlepage}

\section{Introduction}
We present a simple deterministic distributed -approximation algorithm for minimum weight vertex cover (MWVC), which completes in  rounds, where  is the maximum degree in the graph, for any  which is at most , and in particular . If  then our algorithm simply requires  rounds.
Our algorithm adapts the \emph{local ratio} technique~\cite{BarYehudaE1985} to the distributed setting in a novel simple manner. Roughly speaking, in the simplest form of this technique, one repeatedly reduces the same amount of weight from both endpoints of an arbitrary edge, while not going below zero for any vertex. Terminating this process at the time in which for every edge there is at least one endpoint with no remaining weight, gives that the set of vertices with no remaining weight is a -approximation for MWVC. This can be extended to produce a -approximation if instead the process terminates at the time in which for every edge there is at least one endpoint with a remaining weight of at most an  fraction of its initial weight, where .

The challenge in translating this framework to the distributed setting is that the weights we can reduce from endpoints of neighboring edges must depend on each other. This is because we need to make sure that no weight goes below zero. However, as common to computing in this setting, we cannot afford long chains of dependencies, as these directly translate to a large number of communication rounds. Our key method is to divide the weight of a vertex into two parts, a  from which it initiates requests for weight reductions with its neighbors, and a  from which it reduces weight in response to requests from its neighbors. Carefully balancing these two reciprocal weight reductions at each vertex gives the claimed  approximation factor and  time complexity.

In fact, in our distributed algorithm, each vertex  with degree  completes in  rounds if , and in  rounds otherwise (and requires no knowledge of  or ). The algorithm also works in anonymous networks, i.e., no IDs are required. Moreover, the vertices are not required to start at the same round: as long as each vertex starts no later than after the first message has been sent to it, then each vertex completes within  rounds after it starts (or in  rounds if ).
Finally, provided that the weights of all vertices as well as the ratio between the maximal and minimal weights fit in  bits, our algorithm can be modified to work in the CONGEST model.

For any constant , our algorithm provides a constant approximation in  rounds. Apart from improving upon the previous best known complexity for distributed -approximation algorithm for minimum weight vertex cover and providing a new way of adapting the sequential local ratio technique to the distributed setting, our algorithm has the consequence of contradicting the lower bound of~\cite{KMW10}. The latter states that a constant approximation algorithm requires  rounds. Its refutation implies that the current lower bound is  from~\cite{KMW04}, which means that our algorithm is tight.

In Section~\ref{sec:refutation} we pinpoint the flaw in the lower bound of~\cite{KMW10}. This also includes refuting the second result of~\cite{KMW10}, which is a lower bound in terms of , of  rounds for a constant approximation algorithm. Roughly speaking, we claim that the statement of the main theorem is only correct for some smaller range of parameters than claimed, and hence, in particular, one cannot apply it for a number of rounds that is  or .
We emphasize that, as far as we are aware, this bug \emph{does not} occur in the previous version of the lower bound~\cite{KMW04}, implying that the current lower bounds are  in terms of , and  in terms of .

\paragraph{Related Work:}
Minimum vertex cover is known to be one of Karp's 21 NP-hard problems~\cite{Karp72}. For the unweighted case, a simple polynomial-time -approximation algorithm is obtained by taking the endpoints of a greedy maximal matching (see, e.g.,~\cite{Cormen2009,GareyJ79}). For the weighted case, the first polynomial-time -approximation algorithm was given in~\cite{NemhauserT75} and observed by~\cite{Hochbaum82}. The first linear-time -approximation algorithm is due to~\cite{BarYehudaE81} using the primal-dual framework, and~\cite{BarYehudaE1985} gives a linear-time -approximation local-ratio algorithm. Conditioned on the unique games conjecture, minimum vertex cover does not have a  polynomial-time approximation algorithm~\cite{KhotR08}.

In the distributed setting, an excellent summary of approximation algorithms is given in~\cite{AstrandS10}, which we overview in what follows. For the unweighted case, it is known how to find a -approximation in  rounds~\cite{HanckowiakKP01} and in  rounds~\cite{PanconesiR01}. With no dependence on ,~\cite{AstrandFPRSU09} give a -round -approximation algorithm, and~\cite{PolishchukS09} give an -round -approximation algorithm. The maximal matching algorithm of~\cite{BarenboimEPS12} gives a -approximation for vertex cover in  rounds. This can be made into a -approximation within  rounds~\cite{PettiePersonal}.

\sloppy{
For the weighted case,~\cite{GrandoniKP08,KoufogiannakisY09} give randomized -approximation algorithms in  rounds. In~\cite{PanconesiR01}, a -approximation algorithm which requires  rounds, and in~\cite{KhullerVY94}, a -approximation algorithm is given, requiring  rounds. With no dependence on ,~\cite{KuhnMW06} give a -approximation algorithm in ,~\cite{AstrandFPRSU09} give a -approximation algorithm in  rounds for , and~\cite{AstrandS10} give a -approximation algorithm in  rounds, where  is the maximal weight.
}
\section{A local ratio template for approximating MWVC}
\label{sec:LR}
In this section we provide the template for using the local-ratio technique for obtaining a -approximation for MWVC. This template does not assume any specific computation model and only describes the paradigm and correctness. It can be proven either using the primal-dual framework~\cite{BarYehudaE81}, or the local-ratio framework~\cite{BarYehuda00}, which are known to be equivalent~\cite{BarYehudaR05}. A similar idea, though in the primal-dual framework, was given in~\cite{KhullerVY94} which obtained a -approxiamtion as well, but with a larger number of rounds. Our distributed implementation is more efficient and allows us to obtain a faster algorithm.
Here we provide the template and proof for completeness. In Section~\ref{sec:dist} we provide a distributed implementation of the template and analyze its running time.

We assume a given weighted graph , where  is an assignment of weights for the vertices.
Let  be a function that assigns weights to edges. We say that  is \emph{-valid} if for every , , i.e., the sum of weights of edges that touch a vertex is at most the weight of that vertex in .

Fix any -valid function . Define  by , and let  be such that . Since  is -valid, it holds that  for every .

Let , where . The following theorem states that if  is a vertex cover, then it is a -approximation for MWVC.
\begin{theorem}
\label{thm:lr}
Fix  and let  be a -valid function. Let  be the sum of weights of vertices in a minimum weight vertex cover  of . Then . In particular, if  is a vertex cover then it is a -approximation for MWVC for .
\end{theorem}

\begin{proof}
For every  we have that , which implies that . For every  it holds that , and therefore . Put otherwise, for every  we have . This gives:

The above is at most  because . To see why , associate each edge  with its endpoint  in  (choose an arbitrary endpoint if both are in ). The weight  of each  is at least , because it is at least . Hence, .
Hence the sum of weights in  is at most a factor  larger than . Since , we have that , which completes the proof.
\end{proof}

In the next section, we show how to implement efficiently in a distributed setting an algorithm that finds a function  that is -valid, for which the set  is a vertex cover. This immediately gives a distributed -approximation for MWVC.


\section{A fast distributed implementation}
\label{sec:dist}
Our goal in this section is to find a -valid function  such that  is a vertex cover. Since every vertex knows whether it is in , this immediately gives a distributed -approximation algorithm for MWVC. Our algorithm is deterministic and requires for every vertex  only  rounds, where  is the degree of  in , or  if . Here  where , which means that .
For clarity of presentation, in this section we describe an implementation for the LOCAL model. In Section~\ref{sec:discussion} we show how this can be easily be adapted to the CONGEST model in which the message size is limited to  bits, provided that the initial weights of the vertices and the ratio between the maximal and minimal weights can be expressed by  bits.

In our algorithm, each vertex converges to agreeing with each of its neighbors on a function  that is -valid, by iterating the process of decreasing the weight of neighbors by the same amount, until either its weight is below a small fraction of its original weight or it has no more neighbors in the graph induced by the vertices that remain so far. This would imply that the set of vertices whose weight decreased below the above threshold is a vertex cover, and by Theorem~\ref{thm:lr} its weight is a -approximation to the weight of a minimal vertex cover.

\paragraph{Overview of Algorithm~\ref{alg}:} The algorithm consists of iterations, each of which has a constant number of communication rounds. Each vertex  splits its current weight  into two amounts. The first amount is , which is equal to its threshold , and the second amount is , which contains the rest of the weight . Notice that  because  and therefore these amounts are well-defined.

In each iteration, vertex  sends a  request to its neighbor , which is the amount in its  divided by the current number of neighbors of . This guarantees that any weight decrease that results for  from this part does not exceed its total remaining weight. The second amount is used to respond to  requests from its neighbors. The vertex  processes these requests one by one in any arbitrary order, and responds with the amount  which is the largest amount by which  can currently decrease its weight, and no more than the request . This amount is decreased from , and hence it is also guaranteed that decreasing this amount does not exceed the total remaining weight.

Once the weight of  reaches its threshold,  completes its algorithm, returning  after notifying its neighbors of this fact. If the weight is still above the threshold then  only removes its edges to neighbors that notified they are returning . This gives that each edge has at least one endpoint returning , and hence the set of all such vertices is a vertex cover, and by Theorem~\ref{thm:lr} its weight is a -approximation to the weight of a minimal vertex cover. The analysis of the number of rounds is based on the observation that for each of the neighbors  of , either it decreases its weight by responding to  with the entire amount and thereby contributes to decreasing the weight of  by this amount, or it does not have the required budget in its , in which case it contributes to decreasing the number of neighbors of  by .

We proceed by the full pseudocode, followed by an explicit analysis.


\RestyleAlgo{boxruled}
\LinesNumbered
\begin{algorithm}[htbp]
	\label{alg}
	\caption{A distributed -approximation algorithm for MWVC, code for vertex .}
	\\
	\\
	\\
	\\
	\\
	\While{}
	{
		\\
		\\
		\\
		\ForEach{}
		{		
			\\
			Send  to \\
			Let  be the response from \\
			\\
			\If{}
			{
				
			}
		}
		Let  be an order of \\
		\ForEach{}
		{
			Let  be received from \\
			\\
			Send  to 
		}
		\\
		\\
		\\
		\\
		\If{}
		{
            		Send  to all neighbors\\
            		Return \\
        		}
        		\ForEach{ received from }
        		{
            		\\
            		
        		}
        		\If{}
        		{
            		Return 
        		}
    	}	
\end{algorithm}

\begin{theorem}
\label{thm:alg}
For every , Algorithm~\ref{alg} is a deterministic distributed -approximation algorithm for MWVC in which each vertex  with degree  completes in  rounds if , and in  rounds otherwise.
\end{theorem}
We split the proof into two parts, showing correctness and complexity separately. We begin by showing correctness in the following lemma. Essentially, we show that the algorithm finds a function  that is -valid and for which  is a vertex cover.

\begin{lemma}
\label{lemma:approx}
Algorithm~\ref{alg} is a deterministic distributed -approximation algorithm for MWVC.
\end{lemma}
\begin{proof}
We first show that Algorithm~\ref{alg} is a -approximation algorithm for MWVC. That is, we claim that the set  is a vertex cover, and that , where  for some optimal vertex cover . For this, we show that the sum of amounts deducted by neighbors can be used to define a -valid function over the edges. This will be exactly the function according to which the vertices decide whether to output  or .

For every  and every , let . Let . We claim that  is -valid, i.e., for every vertex  it holds that . Let  be the value of  when  returns, that is,  participates in iterations . For each iteration  it holds that

where  is the set of neighbors of  at the beginning of iteration .
Further, since for , , we have that


Since  it holds that . Since , we have that .
This gives that , and hence .

This proves that  is -valid, which gives that for   it holds that , where  for some optimal vertex cover , by Theorem~\ref{thm:lr}. This is because a vertex  outputs  at the end of iteration  if and only if . It remains to show that  is a vertex cover. To see why, consider an edge . We claim that if  have both returned by the end of iteration , then at least one of them is in . This is because otherwise , which implies that both have not returned yet. This completes the proof that  is indeed a -approximation for MWVC.
\end{proof}

It remains to bound the number of rounds. We do so in the following lemma, in which we show that in each iteration either enough weight is reduced or enough neighbors enter the vertex cover.
\begin{lemma}
\label{lemma:complexity}
In Algorithm~\ref{alg}, each vertex  with degree  completes in  rounds if , and in  rounds otherwise.
\end{lemma}
\begin{proof}
Let  be a parameter to be chosen later. Let  be an iteration at the beginning of which a vertex  has not yet returned. We claim that either
 or . To see why, suppose . This means that for at least  vertices , it holds that , and hence


Next, we claim that  returns after at most  iterations of the algorithm. This is because at most  of the iterations  can be such that  (since  returns when ), and at most  iterations  can be such that  (since  returns when ).

Finally, we set  as follows. If  we set . This guarantees  (an isolated vertex simply outputs \NotInCover) and gives  rounds for  to complete.

Otherwise, we set . Since , it holds that  is well defined (as ) and that . It also holds that  which is used in what follows.
This gives that vertex  returns after at most  iterations, where

where the last inequality follows because  (and since  is at most  and so ) and  dominates , completing the proof.
\end{proof}

Theorem~\ref{thm:alg} follows directly from Lemmas~\ref{lemma:approx} and ~\ref{lemma:complexity}.

\section{Adaptation to the CONGEST model}
\label{sec:discussion}
Our algorithm is described for the LOCAL model, but can be easily adapted to the CONGEST model in which the message size is limited to  bits, provided that the initial weights of the vertices and the ratio between the maximal and minimal weights can be expressed by  bits. In order to accommodate -bit messages, we slightly modify the messages that are sent as follows. First, in an initial round, each vertex  sends  to all of its neighbors. Then, instead of sending  to neighbor  in some iteration , vertex  only needs to send  to its neighbor  and  can locally compute  since all vertices know the value of  as part of their algorithm.

Second, we need to handle the messages of type . In general, this amount can be an arbitrary fraction which might not fit in  bits. However, we notice that we can avoid sending this explicit amount. To do this, we slightly modify  to be . Then, upon receiving a  message, if  then vertex  replies with a predefined message \emph{accept}, and otherwise,  responds with the maximal integer  such that . The amount  can be locally computed by , and  can infer that  returns . This is because the remainder of weight in vertex  will be another value of at most  on top of the at most  value which might remain in , summing to no more than , as needed.

\section{Discussion of~\cite{KMW10}}
\label{sec:refutation}
The main result of~\cite{KMW10} is Theorem , which states the following:

\paragraph{Theorem 9 from~\cite{KMW10}.}
\textit{
For every constant , there are graphs , such that in  communication rounds, every distributed algorithm for the minimum vertex cover problem on  has approximation ratios at least 
where  and  denote the number of nodes and the highest degree in , respectively.
}

~\\

The argument in~\cite{KMW10} is that in order for the above approximation factors to be constant, the number of rounds, , has to be  and , respectively.

However, we argue that the above lower bounds only hold under the conditions that  and , respectively. This means that they cannot be applied to  or , and therefore do not imply the claimed bounds for constant approximation factors.

To justify our claim, we elaborate upon the proof of the theorem.
Previous lemmas in the paper\footnote{We refer the reader to~\cite{KMW10} for exact details.} show that the approximation factor of any -round algorithm is , where  satisfies the following two constraints\footnote{We use the notation  as this is the notation in~\cite{KMW10}. Notice that it is unrelated to the function  that we use in our framework in previous sections of this paper.}. First, it holds that  and second, it holds that .

The first constraint implies that

Hence, in order to deduce that , it needs to hold that . However, for this to happen, it must be that , and in particular  has to be within .

The second constraint implies that

Hence, in order to deduce that , it needs to hold that . However, for this to happen, it must be that , and in particular  has to be within .

We emphasize again that this last step in the proof of the lower bound is different in the previous version~\cite{KMW04}, and hence we do not suggest that there is a flaw in~\cite{KMW04}.

\paragraph{Acknowledgements:} We thank Seri Khoury and Dror Rawitz for many discussions and helpful suggestions.

\bibliographystyle{alpha}
\bibliography{paper}

\end{document}
