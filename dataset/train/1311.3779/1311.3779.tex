\documentclass[conference]{IEEEtran}







\usepackage{amssymb}  \usepackage{amsmath,enumerate} 

\usepackage{graphicx} 






\newcommand{\epss}{\varepsilon}

\newcommand{\msu}{\mathcal{U}}
\newcommand{\msv}{\mathcal{V}}
\newcommand{\msm}{\mathcal{M}}
\newcommand{\msn}{\mathcal{N}}
\newcommand{\msw}{\mathcal{W}}




\newcommand{\eg}{\emph{e.g.}~}
\newcommand{\ie}{\emph{i.e.}~}
\newcommand{\wrt}{\emph{w.r.t.}~}
\newcommand{\etc}{\emph{etc}~}
\newcommand{\ifft}{\emph{iff}~}
\newcommand{\hsp}{\hspace{2pt}}
\newcommand{\hspm}{\hspace{-2pt}}
\newcommand{\hspu}{\hspace{-1pt}}
\newcommand{\hint}{\emph{hint:}~}
\newcommand{\nato}{{\mathbb{n}_0}}
\newcommand{\reell}{{\mathbb{R}}}
\newcommand{\tab}{\hspace*{1em}}
\newcommand{\ball}{\mathcal{B}}
\newcommand{\powerset}{\mathcal{P}}
\newcommand{\co}{\overline{\text{co}}}
\newcommand{\acl}{\bar{A}}
\newcommand{\numn}{\boldsymbol{n}}
\newcommand{\numm}{\boldsymbol{m}}
\newcommand{\nump}{\boldsymbol{p}}
\newcommand{\numl}{\boldsymbol{l}}
\newcommand{\numnone}{\boldsymbol{n-1}}
\newcommand{\numr}{\boldsymbol{r}}
\newcommand{\om}{\omega}
\newcommand{\kont}{\mathit{Con}}
\newcommand{\nr}{n-r}
\newcommand{\numro}{\boldsymbol{r_0}}
\newcommand{\numno}{\boldsymbol{n_0}}



\newtheorem{definition}{definition}[section]
\newtheorem{theorem}{theorem}[section]
\newtheorem{lemma}{lemma}[section]
\newtheorem{corollary}{corollary}[section]
\newtheorem{proposition}{proposition}[section]
\newtheorem{remark}{remark}[section]
\newtheorem{reminder}{reminder}[section]
\newtheorem{example}{example}[section]


\newcommand{\tred}[1]{\textcolor{red}{#1}}








\hyphenation{op-tical net-works semi-conduc-tor}


\begin{document}
\title{On Pole Placement and Invariant Subspaces}




\author{\IEEEauthorblockN{Naim Bajcinca}
\IEEEauthorblockA{Max Planck Institute for \\  Dynamics of Complex Technical Systems\\
Sandtorstr. 1, 39106, Magdeburg, Germany}
\tt\small Email:  bajcinca@mpi-magdeburg.mpg.de\vspace{-12pt}
}




\maketitle



\begin{abstract}
The classical eigenvalue assignment problem is revisited in this
note. We derive an analytic expression for pole placement
which represents a slight generalization of the celebrated Bass-Gura and Ackermann formulae, and also is closely related to the modal procedure of Simon and Mitter.
\end{abstract}
\IEEEpeerreviewmaketitle

\vspace{-5pt}
\section{Introduction}
\label{Introduction}
\vspace{-2pt}


For a single-input linear time-invariant system
 with ,
,  , the
solution to the pole placement problem provides the feedback gain  in ,
such that the open-loop eigenvalues  are shifted to
some prespecified values  where ,
\cite{Kailat80, Ackermann:1972vl, SimMit68}, etc. In this note, we
utilize a left eigenvector assignment procedure for a
controllable pair  to derive the following pole placement analytic expression:

where ,  and
 represent the design
parameters that independently assign  and  eigenvalues,
respectively, and are defined as follows: Let
 be the specified closed-loop
characteristic polynomial, where  and
 are real polynomials in  with leading
coefficients equal to one, and let them host the desired  and 
eigenvalues, respectively. Then, ,
where ,
,
and  represents the controller canonical state-space
transformation matrix \cite{Kailat80}, while  is the matrix polynomial corresponding to .
To the best author's knowledge, \eqref{polplac} has
not appeared in that form previously in the control literature and
could be of interest in the sense that it includes both the Ackermann
and Bass-Gura formulae as special cases. Indeed, it will be shown
later in the paper that for  we obtain
the Ackermann formula,
and for  we can link \eqref{polplac} to the Bass-Gura formula. We also
stress its close relationship to the procedure of Simon and
Mitter.

\emph{Preliminaries \& Notation:~}  stands for the open
left-hand complex half-plane. By  we denote the
multiset of the eigenvalues of the matrix .  is an eigenpair of 
(i.e. ) if and only if  is an
eigenpair of the similar matrix of . A real matrix  can be
factorized into a product , where  is an orthogonal matrix
and  is lower quasitriangular (i.e., block lower triangular with
 or/and  blocks along the diagonal), representing the real
Schur decomposition \cite{qref:matrix-analysis-HJ-vol1}.
 is said to be invariant if
there exists a matrix  such that , where
.  The controllability
matrix  of the pair  is denoted by .
Finally, we use the shorthand:
 for  in  to indicate
;  allows  to take also
the value .
















\vspace{-2pt}
\section{Spectrum assignment}
\label{sec:Eigenvalue assignment}


Consider the state space representation of a finite-dimensional controllable  single-input linear time invariant  system:
. It is well-known that for any arbitrary multiset  of
{self-conjugate} eigenvalues   in
, there exists always a unique state feedback gain  which solves the pole assignment problem \cite{Kailat80}. In the sequel, we provide an original method for computation of .


Let  be a left  eigenvector of the closed-loop system matrix  corresponding to an arbitrary eigenvalue . Then, with , we claim:

whereby in light of implementation, care has to be taken in selecting a pair  and  that guarantee a real outcome .
Observe, that  the right-hand side statement in
\eqref{eq2:statefeedback} results from the fact that  is a left eigenvector of , as well, and the
condition , which is guaranteed by the
controllability of the pair . Indeed, if the opposite would
hold true, i.e. if , we would have:
 for all
, indicating that  is an eigenvalue of  and 
simultaneously, i.e. it cannot be shifted by any , which
contradicts the controllability of .


Furthermore, equation \eqref{eq2:statefeedback} reveals that the
remainder eigenvalues in the multiset  are
uniquely specified by the left eigenvector . Hence, it
is natural to pose the spectrum assignment in terms of computing
the eigenvector  such that a prespecified multiset of
self-conjugate (not necessarily distinct) eigenvalues
 are assigned to



To this end, we start with the characteristic polynomial of the {closed loop} matrix , which (with a little of technical effort) is shown to be given by:

Next, consider the controller canonical form , with ,  , and 
Here,  \cite{Kailat80} indicates the transformation , where, for convenience, we denote by
 and  the
open-loop and closed-loop controllability matrix \cite{Kailat80}, respectively.
The characteristic polynomial of  then reads:

Following the discussion related to equation \eqref{eq2:statefeedback}, if we let

represent the desired left eigenvector, and  the corresponding eigenvalue of the closed loop
 in the -coordinates, then from \eqref{eq:chpeigs} we get

where we introduce:
.
From \eqref{canocical char pol} it is obvious that the eigenvalues  of the closed-loop matrix  (that is, of  , as well) are independent of the parameters , and they are entirely determined by the left eigenvector . On the other hand, let \eqref{canocical char pol}  be specified by a desired closed-loop characteristic polynomial of the form:

Equation \eqref{canocical char pol} says that  hosts the parameters of the polynomial , where
. Explicitly, it
can be checked that , as defined in \eqref{eqgama10}, is given by the recursive algorithm:
 for ,
where, in accordance with our adoption in \eqref{eqgama10}: . Moreover, with  and  being similar, we have



This represents our initial pole assignment formula. Next, we
generalize it and demonstrate its relationship to the Bass-Gura and
Ackermann formulae. First, it is readily verified that
\vspace{-10pt}

indicating that all the closed-loop eigenvalues in  are ``encoded'' in the (real) vector , whereas  carries the information about . Then, the Bass-Gura formula:

results immediately, if we rewrite \eqref{eq:poleplacement} as:
,
with the term  shifted right most. 

Equation \eqref{eq:bg1} can be interpreted as ``pulling out'' or ``carrying over'' the eigenvalue  from  via the factor , this necessarily introducing . By proceeding in the same way, one can pullout the eigenvalue  from  by means of ,  from  via , and so on. Hence, we can introduce

using: 
where the  zeros (for ) result due to the ``absence''
of the eigenvalues  in ,
while the  non-zero terms carry the information about .
In this sense, by substituting \eqref{eq:poleplacement_4_2} into  \eqref{eq:bassguraformula}, our spectrum assignment formula \eqref{eq:poleplacement} can be set in the general form:

which can be slightly generalized to

with  and otherwise:

Clearly, equation \eqref{eq:poleplacement_gen_34} represents the generalized form of our initial expression in \eqref{eq:poleplacement}. For  the vector  is simply \emph{defined by the coefficients of the polynomial }, where

The definition of  (i.e. reflecting the Bass-Gura formula with , c.f. \eqref{eq:bg1}) \emph{represents an exception} to this rule.

Now, consider  the special case with  and let  denote the real matrix polynomial corresponding to the desired characteristic polynomial  from \eqref{eq:descp}.
Then, using  from \eqref{eq:gammar}, and:
,
we obtain  the Ackermann formula directly from \eqref{eq:poleplacement_gen}:


\subsection{Comments}



\emph{(i)}~Expressions \eqref{eq:poleplacement_gen}
and \eqref{eq:poleplacement_gen_34} provide a direct link of the
Bass-Gura and Ackermann formulae. Moreover, it represents a
generalization thereof: the former one results with  (leading to the definition \eqref{eq:bg1} for ), while the latter one   for  in  \eqref{eq:poleplacement_gen}. Notice that from \eqref{eq:ack} we immediately obtain





\emph{(ii)}~The desired conjugate eigenpairs should be ``encoded'' jointly in \eqref{eq:poleplacement_gen}, either in the real vector  or in the real matrix polynomial  to benefit from the numerical computation with real numbers. Therefore, without loss of generality we may consider

as the general form of our spectrum assignment formula. In this sense, it is also convenient to use a real  in
\eqref{eq2:statefeedback}.

\emph{(iii)}~If  in \eqref{eq2:statefeedback} is selected to be the left eigenvector of the open-loop matrix  corresponding to a real eigenvalue, say , then from \eqref{eq:Acl} we have , with  referring to a real shift. The remainder open-loop eigenvalues  are thereby unaltered, as for any right eigenvector  of  corresponding to the eigenvalue , we have ,  (as a consequence of ). In this case we retain:

which represents the well-known result of Simon and Mitter
\cite{SimMit68} (cf.~{pp.\,338}). It is important to observe in this
case the geometric interpretation of the vector term  in
\eqref{eq2:statefeedback}: it is {orthogonal to the invariant
  subspace} corresponding to the {eigenvalues that remain
  unchanged}. We discuss this more generally in the next section.




\emph{(iv)}~Finally, due to the presence of the factor
, which for large  is typically ill-conditioned,
related well-known numerical robustness problems are inherent in the
expression \eqref{eq:poleplacement_gen}. In the sequel, we discuss the
avoidance of such difficulties. 











\subsection{Partial spectrum assignment}
\label{Partial pole placement}
Next, we consider the usability of the vector  in the context of the partial spectrum assignment and a sequential spectrum assignment based thereon, which consists in shifting a submultiset of open-loop self-conjugate eigenpairs, say , to some prescribed self-conjugate , while keeping the remainder -ones of  unaltered (). 

To this end, consider the operator description of :
\vspace{-5pt}

\vspace{-5pt}
 corresponding to the real Schur decomposition:

where  (i.e.  and  are complementary subspaces), ,  is the -invariant subspace (i.e. ) corresponding to the eigenvalues in , and  is orthogonal (i.e.,  and  are mutually orthogonal subspaces).
Next, introducing 
in terms of  in \eqref{eq:poleplacement_gen_rer}, it can be readily checked that the block-triangular form is preserved under feedback \cite{Saad86}:

Note that due to the re-appearance of  in the diagonal, the eigenvalues in   remain unaltered in , while those from   change subject to the parameter  in the term .  The latter expression suggests using the Ackermann formula for computation of  in shifting the eigenvalues  of  to :

In words, {if  is fixed perpendicularly to the invariant subspace , then the corresponding open-loop eigenvalues remain unchanged} if we apply the feedback of the form \eqref{eq:poleplacement_gen_rer} with \eqref{eq:kappa} and \eqref{eq:poleplacement_gen_gen}. This fact provides a geometric interpretation for the term  in the expression \eqref{eq:poleplacement_gen_rer}.

With reference to \eqref{eq:geom1},  it is easily seen that the invertibiliy of the controllability matrix  in \eqref{eq:poleplacement_gen_gen} requires

which refers to the projected subsystem  onto the subspace  \cite{Saad86}.

\subsection{Sequential spectrum assignment}
\label{Sequential spectrum assignment}
Comment \emph{(iv)} indicates the difficulties with the invertibility
of the underlying controllability matrix, while in the previous
section we saw that the latter is reduced due to the projection of the
system matrix onto a subspace of a lower dimension. This idea can now be utilized sequentially as suggested by the
following algorithm. Let

where  includes a submultiset of self-conjugate open-loop
eigenvalues, and  the corresponding desired self-conjugate
closed-loop eigenvalues. In other words, the eigenvalues in 
are to be shifted to  for all . Then, introduce:

with , , , ,

where  represents the -invariant subspace corresponding to the eigenvalues ,  is orthogonal to  in ,

and  is the characteristic polynomial corresponding to the desired eigenvalues in .
Effectively, we obtain: 
In words, the vector  is set perpendiculary to the invariant
subspaces  corresponding to the unaltered eigenvalues at
the  iteration, while the Ackermann formula is used to
design the feedback gain  for the assignment of the eigenvalues
 in the projected subspace. This procedure is repeated
sequentially. Thereby,  represents the closed-loop system
matrix up to the  iteration. Finally, if
 includes a pair of conjugated poles only, then this algorithm
reduces to the Ackermann's method of invariant planes \cite{ackermann93}.

\vspace{-5pt}
\section{Conclusion}
This short note introduces a slightly
generalized version of pole placement formulae and discusses its
relationships to Ackermann, Bass-Gura and Simon\,\&\,Mitter
algorithms. It extends and completes initial ideas of \cite{BajAJC}. The author thanks Dietrich Flockerzi for  useful discussions.

\vspace{-5pt}

\bibliographystyle{plain}
\bibliography{bib/aigaion_fgrs_export_2012_05_30}

\end{document} 