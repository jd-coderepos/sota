\documentclass{llncs}
\usepackage{amsmath,amssymb,latexsym,epsfig,llncsdoc,graphicx,wrapfig,subfigure,ifthen}
\usepackage{amsmath}
\usepackage{amsfonts}
\usepackage{amssymb}
\usepackage{amssymb}
\usepackage{setspace,cite,multirow}


\newtheorem{obs}{Observation}
\newtheorem{lem}{Lemma}
\newtheorem{cor}{Corollary}
\newtheorem{thm}{Theorem}
\newtheorem{cla}{Claim}

\date{}

\title{Cutting a Convex Polyhedron Out of a Sphere \thanks{An earlier version appeared in Proc. WALCOM 2010, LNCS, Springer, 2010.}}
\author{Syed Ishtiaque Ahmed, Masud Hasan, and Md. Ariful Islam}

\institute{
Department of Computer Science and Engineering\\
Bangladesh University of Engineering and Technology\\
Dhaka-1000, Bangladesh\\
\texttt{http://www.buet.ac.bd/cse}\\ 
\email{\texttt{ishtiaque@csebuet.org, masudhasan@cse.buet.ac.bd, arifulislam@csebuet.org}}
}


\pagestyle{plain}


\begin{document}

\maketitle{}


\begin{abstract}
Given a convex polyhedron  of  vertices inside a sphere , we give an -time algorithm that cuts  out of 
by using guillotine cuts and has cutting cost  times the optimal.

\smallskip

\noindent{\bf Keywords:} Approximation algorithm, guillotine cut, polyhedra cutting
\end{abstract}

\section{Introduction} 

The problem of cutting a convex polygon  out of a piece of planar material  
( is already drawn on )
with minimum total cutting length is a well studied problem in computational geometry.
The problem was first introduced by Overmars and Welzl in 1985~\cite{aa}
but has been extensively studied in the last decades~\cite{AIH,aa,ab,ac,ae,ag,ah,ai,am,JK03}
with several variations, such as  and  are
convex or non-convex polygons,  is a circle, and the cuts are line cuts or ray cuts.
This type of cutting problems have many industrial applications such as in metal sheet cutting, paper cutting,
furniture manufacturing, ceramic industries, fabrication, ornaments, and leather industries.
Some of their variations also fall under \emph{stock cutting problems}~\cite{ab}.

If  is another convex polygon with  edges, this problem with line cuts 
has been approached in various ways~\cite{aa,ab,ac,ad,ae,af,ah,ai}.
If the cuts are allowed only along the edges of ,
Overmars and Welzl~\cite{aa} proposed an -time algorithm for this problem with optimal cutting length,
where  is the number of edges of .
The problem is more difficult if the cuts are more general, i.e.,
they are not restricted to touch only the edges of .
In that case, Bhadury and Chandrasekaran showed that the problem has optimal solutions
that lie in the algebraic extension of the input data field~\cite{ab},
and due to this algebraic nature of this problem,
an approximation scheme
is the best that one can achieve~\cite{ab}.
They also gave an approximation scheme
with pseudo-polynomial running time~\cite{ab}.

After the indication of Bhadury and Chandrasekaran~\cite{ab} to
the hardness of the problem, many people have given polynomial time approximation algorithms.
Dumitrescu proposed an -approximation algorithm with
 running time~\cite{ae,ad}. Then, Daescu and
Luo~\cite{af} gave the first constant factor approximation
algorithm with ratio ,
where  and  are the perimeters of  and the minimum area bounding rectangle of  respectively.
Their algorithm has a running time of
. The best known constant factor
approximation algorithm is due to Tan~\cite{ac} with an approximation ratio of
 and running time of . In the same
paper~\cite{ac}, the author also proposed an -approximation
algorithm with improved running time of .
As the best known result so far, very recently, Bereg, Daescu and
Jiang~\cite{ah} gave a polynomial time approximation scheme (PTAS)
for this problem with running time .
Recently, Ahmed et.al.~\cite{AIH} have given similar constant factor and -factor
approximation algorithms where  is a circle. 
As observed in~\cite{AIH}, algorithms for  being a convex polygon are not
easily transferred for  being a circle, as the running time of the
formers depend upon the number of edges of .


For ray cuts, Demaine, Demaine and Kaplan~\cite{ag} gave a linear time algorithm to decide
whether a given polygon  is \emph{ray-cuttable} or not.
For optimally cutting  out of  by ray cuts, 
if  is convex and  is non-convex but ray-cuttable, 
then Daescu and Luo~\cite{af} gave an almost linear time
-approximation algorithm.
If  is convex, then they gave a linear time -approximation algorithm.
Tan~\cite{ac} improved the approximation ratio for both cases
as  and , respectively, but with much higher running time of .
See Table~\ref{fi:comparison} for a summary of these results.



\begin{table*}[htbp]
\begin{center}
{\scriptsize
\begin{tabular}{|c|c|c|c|c|c|c|}
\hline
Dim. & Cut Type &  &  & Approx. Ratio & Running Time & Reference \\
\hline
\multirow{11}{*}{2D} 
& \multirow{5}{*}{Line} & Convex & Convex &  &   & \cite{ae,ad} \\
\cline{3-7}
& & Convex & Convex &  &  & \cite{af} \\
\cline{3-7}
& & Convex & Convex & 7.9 &  & \cite{ac} \\
\cline{3-7}
& & Convex & Convex &  &  & \cite{ah} \\
\cline{3-7}
& & Circle & Convex &  &  & \cite{AIH} \\
\cline{3-7}
& & Circle & Convex &  &  & \cite{AIH} \\
\cline{2-7}
& \multirow{5}{*}{Ray} & - & Non-convex & Ray-cuttable? &  & \cite{ag} \\
\cline{3-7}
& & Convex & Convex &  &  & \cite{af} \\
\cline{3-7}
& & Convex & Non-convex &  &  & \cite{af} \\
\cline{3-7}
& & Convex & Convex &  &  & \cite{ac} \\
\cline{3-7}
& & Convex & Non-convex &  &  & \cite{ac} \\
\hline
\multirow{2}{*}{{\bf 3D}} & Hot-wire & - & Non-convex & Cuttable? &  & \cite{JK03}\\
\cline{2-7}
& {\bf Guillotine} & {\bf Sphere} & {\bf Convex} & {\boldmath} & {\boldmath} & {\bf This paper} \\
\hline
\end{tabular}
}\end{center}
\caption{Comparison of the results.}
\label{fi:comparison}
\end{table*}


\paragraph{Our results}
The generalization of this problem in 3D is very little known.
To the best of our knowledge, the only result is to decide whether a 
polyhedral object can be cut out form a larger block using continuous hot wire cuts~\cite{JK03}.
In this paper we attempt to generalize the problem in 3D.
We consider the problem of cutting a convex polyhedron  which is 
fixed inside a sphere  by using only guillotine cuts with minimum total cutting cost.
A \emph{guillotine cut}, or simply a \emph{cut}, is a plane that does not pass through  
and partitions  into two smaller convex pieces.
After a cut is applied,  is updated to the piece that contains .
The \emph{cutting cost} of a guillotine cut is the area of the newly created face of .
We give an -time algorithm that cuts  out of  by using only guillotine cuts and
has cutting cost no more than  times the optimal cutting cost.
Also see Table~\ref{fi:comparison}.


The rest of the paper is organized as follows.
We give some preliminaries in Section~\ref{se:pre}, 
then Section~\ref{se:algo} gives the algorithms, and finally Section~\ref{se:con} concludes the paper 
with some future work.


\section{Preliminaries}
\label{se:pre}
A cut is a \emph{vertex/edge/face cut} if it is tangent to  at a single vertex/a single edge/a face respectively.
We call  to be \emph{cornered} (within ) if it does not contain the center  of , otherwise it is called \emph{centered}.
For cornered , the \emph{D-separation} of  is the minimum-cost (single) cut that separates  from .



We represent an \emph{orthogonal projection} of a convex polyhedron  by the corresponding 
\emph{projection direction} coming towards the origin from a \emph{view point} at infinity. 
A face  of  is visible in an orthogonal projection if the view point
lie in the half space that is defined by the supporting plane of  and does not contain .
An orthogonal projection of  is called  \emph{non-degenerate}
if the projection direction is not parallel to any face of .
An orthogonal projection of  is a convex polygon.
If the projection is non-degenerate, then each edge of the projected convex polygon corresponds to an edge of . 



\section{The algorithm} 
\label{se:algo}
Let  be the optimal cutting cost.
We shall have two phases in our algorithm: \emph{box cutting phase} and \emph{carving phase}.
In the box cutting phase, we shall cut a minimum volume rectangular box  containing  out of 
with cutting cost no more than a constant factor of .
Then in the carving phase we shall cut  out of  with cutting cost bounded by  times of .


\subsection{Box cutting phase} 
We first deal with cornered .
If  is cornered, we shall first apply the D-separation to .
The following lemma gives a characterization of the D-separation, which will help finding it quickly.

\begin{lemma}
\label{le:closest_x}
For cornered , let  be the closest point of  from .
Then the D-separation of  is the plane that is perpendicular to the line segment  at .
\end{lemma}

\begin{proof}
A D-separation is a tangent to  that separates  from  and is farthest form .
Let  be the plane that is perpendicular to the line segment  at .
To prove that  is the D-separation, we first prove that  is tangent to .
Suppose not. 
Then there exists some portion of  in the neighborhood of  that lies in the half space of  containing .
Then there must be a point  in that portion that is closer to  than ,
which is a contradiction that  is closest to .

We next prove that  is the farthest tangent of  from  that separates  from .
To separate  from ,  must intersect .
Now, any other plane that intersects  and is not perpendicular to  at 
is closer to  than .
Therefore,  is the farthest.
\qed
\end{proof}

Observe that since  is convex, the  closest point  of  from  is unique,
and therefore, the D-separation of  is also unique.
However,  can be a vertex, or a point of an edge or a face.

\begin{lem}
The D-separation can be found in  time.
\end{lem}

\begin{proof}
By Lemma~\ref{le:closest_x}, we need to find the closest point  of  from .
We first find the closest vertex  from  in  time.
Then for each edge , we find the closest point  of  from  as follows:
Let  be the line passing through .
Draw a line segment  perpendicular to .
If  is a point of , then  is ,
otherwise  is the end point of  that is closer to .
Finding  can be done in constant time. 
For all edges of , it takes  time.
Similarly, for each face , we find the closest point  of  from  as follows:
Let  be the supporting plane of .
Draw a line segment  perpendicular to .
If  is a point of , then  is ,
otherwise  is among the closest point of the edges of  or among the vertices of .
Finding  can be found in  time, where  is the number of edges of .
For all faces of , it takes  time.
Finally,  is the closest among , all 's and 's.
\qed
\end{proof}

\iffalse*****************************************

By Lemma~\ref{le:closest_x} we need to find the plane that is perpendicular to the line segment 
at , where  is the closest point of  from .
To check whether  is a vertex of , 
among all the vertices of  let  be closest to .
Let  be the plane 
perpendicular to  at . If  is tangent to , 
then  is the D-separation.
Checking  to be a tangent of  can be easily done in  time, 
where  is the degree of .

To check whether  is a point of an edge  (similarly, of a face ) of ,
for each edge  (each face ) we draw the line segment 
perpendicular to the line  passing through  (perpendicular to the supporting plane  of ).
If  is a point of  (), then  is 
and the D-separation is the plane perpendicular to  at .
Checking whether  to be a point of  can be easily done in  time, 
where  is the number of edges of .
Over all faces of , this checking requires a total time of .
\qed
\end{proof}

*********************************************************\fi


For cornered , after the D-separation is applied,  is a spherical segment
and let  be the radius of the base circle of that segment.
The following lemma relates for cornered  the cost of D-separation and .

\begin{lem}
\label{le:cornered_lb}
For cornered , cost of the D-separation, which is , is at most .
\end{lem}

\begin{proof}
Consider an optimal cutting sequence  with cutting cost . 
 must separate  from . 
However, it may do that by using a single cut  or by using more than one cut. 
If it uses a single cut, then it is in fact doing the the D-separation, 
since the D-separation is the minimum cost single cut that can separate  from .
Therefore, .

If  uses more than one cut, then let  
with  being the first cut that separates  from . 
Observe that  can not be the very first cut of , 
since otherwise it is doing a D-separation and we are in the previous case.
Replace  by a single cut  whose plane is the same as that of .
We will show that cost of  is smaller than the total cost of .

Consider the first two cuts  and .
Replace  and  by a single cut  whose plane is the same as that of .
Since  does not separate  from , the portion of  that is created due to 
and that does not contain  is no larger than a half sphere of .
It implies that the portion of  that is not present in  is smaller than 
(also see Fig.~\ref{fi:replace}).
That means the cost of  is smaller than the total cost of  and .
Similarly, we can show that replacing  and  with a 
single cut  in the plane of  has smaller cost than the total cost of  and .
Repeating this for  times would show that  has smaller cost than the total cost of .
Therefore, using more than one cut to separate  from  is even worse than using a single cut,
and we already proved that an optimal way to use a single cut is to use the D-separation.
Thus the lemma holds.
\qed
\end{proof}

\iffalse


The above fact also implies that,
to separate  from  if a single cut is used that is not a D-separation,
then it must have cost more than the D-separation, since D-separation is the minimum such cut.
If more than one cut are used, then their total cost would be even higher.
\qed
\end{proof}

\fi

\begin{figure}
\begin{center}
\begin{picture}(0,0)\includegraphics{replace.pstex}\end{picture}\setlength{\unitlength}{2960sp}\begingroup\makeatletter\ifx\SetFigFont\undefined \gdef\SetFigFont#1#2#3#4#5{\reset@font\fontsize{#1}{#2pt}\fontfamily{#3}\fontseries{#4}\fontshape{#5}\selectfont}\fi\endgroup \begin{picture}(2430,2431)(3579,-2783)
\put(4426,-1111){\makebox(0,0)[lb]{\smash{{\SetFigFont{9}{10.8}{\rmdefault}{\mddefault}{\updefault}}}}}
\put(4876,-2311){\makebox(0,0)[lb]{\smash{{\SetFigFont{9}{10.8}{\rmdefault}{\mddefault}{\updefault}}}}}
\put(4126,-1786){\makebox(0,0)[lb]{\smash{{\SetFigFont{9}{10.8}{\rmdefault}{\mddefault}{\updefault}}}}}
\put(4876,-1561){\makebox(0,0)[lb]{\smash{{\SetFigFont{9}{10.8}{\rmdefault}{\mddefault}{\updefault}}}}}
\put(5476,-511){\makebox(0,0)[lb]{\smash{{\SetFigFont{9}{10.8}{\rmdefault}{\mddefault}{\updefault}}}}}
\end{picture} \caption{2D view of  and . 
Broken line represents the part of  that is not in ;
This portion is smaller than .}
\label{fi:replace}
\end{center}
\end{figure}



We now deal with centered . A lemma that is similar to the previous one and gives lower bound
for centered  is the following.

\begin{lem}
\label{le:centered_lb}
For centered , it holds that , where  is the radius of .
\end{lem}

\begin{proof}
For centered ,  remains a sphere.
Since  contains the center  of , any cutting sequence, starting from the boundary of ,
must ``wrap''  and finally get out of  by a plane different from the starting plane.
That means the wrapping must enclose the center .
In the best case when  is simply a point that lies in the center  of , 
the cutting sequence must traverse at least  area to reach  
and then to traverse another  area to finish the cutting. 
In the worst case, when  is almost the sphere , the sequence must traverse the whole area of ,
which is .
\qed
\end{proof}


We next find a minimum volume rectangular bounding box  of  in  time 
by the algorithm of O'Rourke~\cite{O85}.
Then we cut out this box from  by applying six cuts along the six faces of .

\begin{lem}
\label{le:garbage1}
Cost of cutting  out of  is at most  for cornered  and at most  for centered .
\end{lem}

\begin{proof}
Consider  before  was cut out of it.
We denote the area of  by \footnote{In the subsequent text, we use  to denote the area of a 3D object.}.
For cornered , since  is no bigger than a half sphere, it holds that ,
which by Lemma~\ref{le:cornered_lb} becomes . 
For centered , since we do not apply D-separation, we have ,
which by Lemma~\ref{le:centered_lb} becomes .
While cutting along the faces of , for each cut  let  be the portion of 
that does not contain . 
Let  be the portion of the surface of  that is ``inherited'' from , 
i.e., that was a part of the surface of .
One important observation is that the cost of  is no more than the area of .
Moreover, over all six cuts, sum of these inherited surface area is .
Therefore, the lemma holds.
\qed
\end{proof}

Once the minimum area bounding box  has been cut, a lower bound on  can be 
given in terms of the area of .


\begin{lem}
\label{le:lb}
, where  is the area of .
\end{lem}

\begin{proof}
Let  be a maximum area face of .
Project  orthogonally from the direction perpendicular to .
 projects to a convex polygon .
In this projection,  is the minimum area bounding rectangle of ,
since otherwise we could rotate the four faces of  that are not perpendicular to 
and would get a bounding rectangle smaller than , which in turn would give a bounding box smaller than ,
but that would be a contradiction that  is the smallest bounding box.
It implies that the area of  is at least .
Now,  is at least twice the area of , and . 
Therefore, .
\qed
\end{proof}

\subsection{Carving phase} 

Let  be the portion of  that we would achieve if  were removed from .
Then,  is a polyhedral object.  may be convex or non-convex and possibly disconnected.
The \emph{outer} surface of  is the surface of  that existed in  when  was not removed from .
Our idea is to apply an edge cut through each edge of , 
and we shall do that in two types of rounds: \emph{face rounds} and \emph{edge rounds}.
Face rounds will find polygonal chains that will partition the faces of   into smaller 
sets and edge rounds will apply edge cuts through the edges of those polygonal chains.
There will be  face rounds,
and within each face round there will be a number of edge rounds but their total cost will be . 
Once we have applied edge cuts through all the edges of , each face  of  will have a small ``cap''-like
portion of  over it, which we shall cut at a cost of the area of  to get , 
giving a cost of  for all faces.

\subsubsection{Face rounds}
Let  be a set of faces of .
From now on, we use the term \emph{face set} to represent a set of faces of .
At the very first face round ,  consists of all the faces of .
We find a chain of edges  that will partition  into two smaller
face sets  and  by the following lemma.


\begin{lem}
\label{le:face_equal_partition}
Let  be the number of faces in a face set .
It is always possible to find in  time a non-degenerate orthogonal projection of  
such that the two sets of visible and invisible faces of  
contain at least  faces each.
\end{lem}

\begin{proof}
For this proof we shall move on to the surface of an origin-centered sphere .
For each face , its  \emph{normal point} is the intersection point of  
and the outward normal vector of  when the vector is translated to the origin.
Each point of  also represents an orthogonal projection direction of .
So, a non-degenerate orthogonal projection of  can be represented 
by a great circle of  that does not pass through the normal points of the faces of .
We need one such great circle satisfying an additional criterion that its two hemispheres contain 
at least  normal points each.
There exists infinitely many such great circles and one of them can be found in  time
as follows. 
Take as \emph{poles} any two antipodal points that are not normal points of the faces of .
Take a great circle  through these two poles and rotate it around these poles until 
the number of normal points in its two hemisphere differ by at most one.
If it happens that some normal points fall on  when we stop,
then slightly change the poles as well as  so that the normal points on  are distributed into two hemispheres as necessary.
For running time, all we need to do is to sort the normal points according to their angular distance 
with the plane of initial position of  at the origin.
The resulting projection is the one from the perpendicular direction of the plane of final position of .
\qed
\end{proof}

The projection direction achieved by the above lemma is called the \emph{zone direction} of .
 is the chain of edges in the boundary of the above projection
whose corresponding edges in  have both adjacent faces (one is visible and another is invisible) in .
We call  a \emph{separating chain} of .
 and  are the two sets of faces separated by .
In the next face round , we shall apply Lemma~\ref{le:face_equal_partition} for each of  and 
recursively and thus get two separating chains and four  face sets.
We shall repeat the same procedure for each of these four face sets.
We shall continue like this until each face set has only one face.
Clearly, we need  face rounds.

\subsubsection{Edge rounds}
Let  be the separating chain of a particular face round
with its two ends, which are two vertices of  and , touching the outer surface of .
Observe that for the very first face round ,  is a cycle and the two ends are the same.
We shall apply edge cuts through the edges of  such that all of them are parallel to a particular direction.
Such a direction can be the corresponding zone direction.
We shall call this set of  edge cuts a \emph{zone} of cuts and their direction of cuts the \emph{zone cut direction}.
We shall apply these cuts in  edge rounds.
At the very first edge round , we apply an edge cut through  in the zone cut direction.
This cut will partition the edges of  into two sub chains of size at most . 
In the next edge round , we apply two edge cuts through the two middle edges of these two sub chains,
which will result into four sub chains.
Then in the next round we apply four similar cuts to the four sub chains.
We continue like this until each sub chain has only one edge.
Clearly, we need  edge rounds for . 

\begin{lem}
After all the face rounds and their corresponding edge rounds are completed, all edges of  get an edge cut.
\end{lem}

\begin{proof}
Let  be an edge that does not get an edge cut through it. 
Then the two adjacent faces of  are in the same face set.
But that is a contradiction that each face set has only one face.
\qed
\end{proof}


\subsubsection{Analysis}
We are now ready to find the approximation ratio and the running time of our algorithm.

We define the \emph{box area} of a face set  as follows.
When  contains all faces of , its box area is ---the whole surface area of .
Zone of cuts through the separating chain of  partitions  into  and 
and  into two components, say  and , respectively.
Then the \emph{box area} of  () is the outer surface area of  (),
which we denote by by  ().
Observe that .
Box area of any subsequent face set is similarly defined.
Moreover, two face sets from the same face round have their box areas disjoint
and in any face round sum of all box area is at most .


The following lemma bounds the cutting cost of an edge rounds in a particular face round.

\begin{lem}
Let  be the separating chain with  edges of an arbitrary face set  
to which we apply  edge rounds.
Let  be the box area of .
At each edge round , total cost of  cuts is .
Over all  edge rounds, total cost is .
\end{lem}


\begin{proof}
This proof is similar to that of Lemma~\ref{le:garbage1}.
Consider a particular edge round .
For each cut  the cost of  is no more than the portion of  that is thrown away by .
Moreover, these cuts are pairwise disjoint, 
since they can at best intersect the cut which is in between them and was applied in -th round.
It implies that the total cost of  cuts is at most .
Since , the second part of the lemma follows.
\qed
\end{proof}

The next lemma bounds the total cutting cost over all face rounds.

\begin{lem}
At each face round , total cost of  zones of cuts is .
Over all  face rounds, the total cost is .
\end{lem}

\begin{proof}
At each face round , we apply  zones of cuts to  face sets.
By the previous lemma, for a particular face set , ,
cost of the zone of cuts applied to it is at most .
Since , cost of all zone cuts is 
.
Over all  face rounds, the total cost is ,
which  by Lemma~\ref{le:lb} is . 
\qed
\end{proof}


We now see the running time of our algorithm.
Running time in face round  involves finding  separating chains,
each of size  , 
plus applying a zone of cuts to each of them. 
Each separating chain 
can be found in  time by Lemma~\ref{le:face_equal_partition}.
Each cut needs to update , which can be done in  time assuming that  is 
represented by suitable data structures~\cite{berg}.
It gives that a zone of cuts needs  time.
So, in round  total time is .
Over all  rounds, it becomes .

We summarize the result in the following theorem.

\begin{thm}
Given a convex polyhedron  fixed inside a sphere ,  can be cut out of  by using only
guillotine cuts in  time with cutting cost  times the optimal,
where  is the number of vertices of .
\end{thm}


\section{Conclusion}
\label{se:con}
In this paper, we have given an -time algorithm that cuts a convex polyhedron 
with  vertices from a sphere , where  is fixed inside , by using guillotine cuts
with cutting cost  times the optimal.


This problem is well studied in 2D, where the series of results include several  
and constant factor approximation algorithms and a PTAS.
The key ingredients of the 2D algorithms involve three major steps: 
(1) take some approximate vertex cuts through the vertices of , 
(2) use dynamic programming to find an optimal cutting sequence among the edge cuts,
and the vertex cuts taken in step (1), and 
(3) show that the cutting cost of the sequence obtained in step (2)
is within the desired factor of the optimal.
Using the idea of 2D algorithms may be a way to improve the approximation ratio of our algorithm.
Among the above three steps, it may not be difficult to generalize steps (1) and (3) for 3D,
but the most difficult part we find is the applying a dynamic programming.

An immediate future work would be to find approximation algorithms when  is another convex polyhedron.
Recently, Ahmed et.al.~\cite{ABHK10} have studied a more generalized version of the problem in 2D
where the polygon  is \emph{not fixed} inside a circle .
For that problem they have given several constant factor approximation algorithms. 
It would be interesting to study that version of the problem in 3D.




\bibliographystyle{abbrv}

\bibliography{bib2_short}


\end{document}