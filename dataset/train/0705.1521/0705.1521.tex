\documentclass[11pt]{article}
\usepackage{amsfonts,epsfig,amssymb,amsmath,a4wide}

   
\newcommand{\MSOA}{\mbox{MSO}_1}



\newtheorem{theorem}{Theorem}[section]
\newtheorem{example}[theorem]{Example}
\newtheorem{proposition}[theorem]{Proposition}
\newtheorem{corollary}[theorem]{Corollary}
\newtheorem{lemma}[theorem]{Lemma}
\newtheorem{fact}[theorem]{Fact}
\newtheorem{definition}[theorem]{Definition}
\newtheorem{remark}[theorem]{Remark}
\newtheorem{exercise}[theorem]{Exercise}
\newtheorem{problem}[theorem]{Problem}
\newtheorem{solution}[theorem]{Solution}
\newtheorem{note}[theorem]{Note}
\newtheorem{question}[theorem]{Question}

\newtheorem{alg}[theorem]{Algorithm}
\newenvironment{proof}{\noindent{\bf Proof~}}{\null\hfill \par\medskip}
\newenvironment{proofsk}{\noindent{\bf Proof Sketch~}}{\null\hfill 
\par\medskip}









\begin{document}


\title{A note on module-composed graphs}

\author{Frank Gurski\thanks{Heinrich-Heine Universit\"at D\"usseldorf,
Department of Computer Science, D-40225 D\"usseldorf, Germany,
E-Mail: gurski-corr@acs.uni-duesseldorf.de, }}



\bigskip

\maketitle

\begin{abstract}
In this paper we consider module-composed graphs, i.e. graphs which can be defined by a sequence of one-vertex insertions , such
that the neighbourhood of vertex , , forms a module (a homogeneous set) of the graph defined by vertices .

We show that
module-composed graphs are HHDS-free and thus homogeneously orderable, weakly chordal, and perfect. Every bipartite distance hereditary graph, every (co-)-free graph and thus every trivially perfect graph is module-composed. We give an  time algorithm to decide whether a given graph  is module-composed and construct a corresponding 
module-sequence.

For the case of bipartite graphs, module-composed graphs
are exactly distance hereditary graphs, which implies
simple linear time algorithms for their recognition and construction of a corresponding module-sequence.

\bigskip
\noindent
{\bf Keywords:} graph algorithms, homogeneous sets, HHD-free graphs, distance hereditary graphs, bipartite graphs
\end{abstract}








\section{Preliminaries}




Let  be a graph.
For some vertex  we denote the {\em neighbourhood} of  by
.  is called a {\em module (homogeneous set)} of , if
and only if for all : , i.e.  and  have
identical neighbourhoods outside .  is called a {\em trivial module},
if ,  , or , see \cite{CH94}. A graph  is called {\em prime}
if every module of  is trivial.
A module  is {\em maximal} if there is no non-trivial module  such that .
A module is called {\em strong} if it does not overlap
with any other module. 

While the set of modules of a graph  can be exponentially large, the set of strong modules is linear
in the number of vertices. The inclusion order of the set of all strong modules
defines a tree-structure which  is denoted as {\em modular decomposition} , see \cite{MR84}.
The root of  represents the graph  and the leaves of  correspond to
the vertices of . Every inner node, i.e. non-leaf node,  of  corresponds to an induced subgraph
of  consisting of the leaves of  in subtree with root , which is called the {\em representative graph} of  
and is denoted by . Vertex set
 is a strong module of . For some inner node  of , the {\em quotient graph}   is
obtained by substituting in  every strong module, represented by some child of  in , by a
single vertex. For some inner node  of  , quotient graph  is either an independent set
( is denoted as {\em co-join node}), a clique ( is denoted as {\em join node}), or a prime graph ( is denoted as
{\em prime node}).

For , we define by
 the subgraph of  induced by the vertices of .
For some graph , we denote its edge complement by co-. For a set of graphs , we denote by {\em -free graphs} the set of all graphs that do not contain a graph of 
as an induced subgraph.

In Table \ref{gr} we show some special graphs to which we refer during the paper.
A {\em hole} is a chordless cycle with at least five vertices. 
A {\em -sun} is a chordal graph  on  vertices for some  whose vertex
set can be partitioned into   such that  and  
is an independent set. Additionally vertex 
is adjacent to vertex  if and only if  or .
 is called a  if it is a -sun for some .
If graph   is a clique, then  is called a {\em complete -sun}.



\begin{table}[ht]
\begin{center}
\begin{tabular}{ccccccc}  
\\
\epsfig{figure=c5.eps,width=1.5cm}  && \epsfig{figure=hole.eps,width=1.5cm} && \epsfig{figure=house.eps,width=1.5cm} &&   \epsfig{figure=gem2.eps,width=1.5cm}        \\
 &&  hole && house && gem  \\ \\
\epsfig{figure=do.eps,width=1.8cm} && \epsfig{figure=cc3.eps,width=1.8cm}  && \epsfig{figure=3-sun.eps,width=1.6cm} && \epsfig{figure=kk22.eps,width=2.8cm}\\
domino &&  co-()       &&  3-sun  && co- 
\end{tabular}
\end{center}
\caption
{Special graphs}
\label{gr}
\end{table}






\section{Module-composed graphs}



There are several graph classes which are defined by a sequence of one-vertex
extentions of restricted form. Some well known examples are trees, co-graphs,
and distance hereditary graphs, see \cite{Rao07} for a survey. We next analyze
a closely related but new concept.

Graph  is {\em module-composed}, if and only if there exists a linear 
ordering , such that for every  the 
neighbourhood of vertex  in graph 
 forms a module. 
For some module-composed graph ,  is called a {\em module-sequence} for .

The definition of module-composed graphs
was introduced \cite{AGKKW06} for computing
connectivity ratings for vertices in special graph classes, 
see also \cite{AKKW06}.
We first recall the following easy but important lemma from \cite{AGKKW06}.

\begin{lemma}[Induced subgraph]\label{Li}
If a graph  is module-composed, then every induced subgraph of  is also 
module-composed.
\end{lemma}

Given two module-sequences  for two graphs  and , 
sequence  and  is a possible module-sequence for the disjoint union of these two graphs.

\begin{lemma}[Disjoint union]\label{Ld}
For two module-composed graphs , the disjoint union   is also 
module-composed.
\end{lemma}


The following observation follows from Lemma \ref{Li} and the definition of 
module-composed graphs.

\begin{lemma} \label{Lx}
A graph  is  module-composed, if and only if there exists a
vertex  such that  is a module in graph  and graph  
is module-composed.
\end{lemma}


By Lemma \ref{Lx} the following graphs (see Table \ref{gr}) are  not module-compo\-sed,
since none of them contains a vertex  such that  is a module in graph :

,  (i.e. holes), co-,  (i.e. anti-holes), house,  domino,  
co-(), 3-sun, co-.




The example of graph co- shows that not every co-graph\footnote{A {\em co-graph} is either a  single vertex ,
the disjoint union  of two co-graphs , or the join   of two co-graphs , which connects every vertex of  with every vertex of . } is module-compo\-sed. Graph co- can even be used to characterize those co-graphs which are module-composed.


\begin{lemma}\label{Lco}
Let  be a co-graph. The following conditions are equivalent.
\begin{enumerate}
\item  is module-composed.

\item  is (co-)-free.
\end{enumerate}
\end{lemma}


\begin{proof}
If  is module-composed, then by Lemma \ref{Li} it obviously contains no co- as induced subgraph.

Let  be (co-)-free co-graph. Then there exists a co-graph expression  defined by the three co-graph operations (single vertex ,
disjoint union  of two co-graphs , join   of two co-graphs ) for .  Any subexpression  and  
are also feasible for a module-sequence. 

Let 
be a subexpression of . Since the graph defined by  contains no co- as an induced subgraph either graph defined by  or that by  defines a subgraph of , i.e. the disjoint union of a clique 
on two vertices and a clique on one vertex. Let us assume that   does so.  This allows us to define
a module decomposition for  as follows. We start with
a module-sequence for , which exists by induction, proceed with
the vertices of  and finish with vertex of graph ,
which leads a module-sequence for graph defined by .
\end{proof}

Co-graphs are exactly -free graphs which implies our next corollary.

\begin{corollary}
(co-)-free graphs are module-composed.
\end{corollary}


Further it is  known that trivially perfect\footnote{
A graph is {\em trivially perfect} if for every induced subgraph  of ,
the size of the largest independent set in  equals the number of  all maximal cliques in .} graphs are exactly  -free graphs
\cite{Gol78}, which obviously form a subclass of co--free graphs.



\begin{corollary}
Trivially perfect graphs are module-composed.
\end{corollary}






Next we conclude results on super classes
of module-composed graphs.

It is easy to see that the house, every hole and the domino are not module-composed.
By a result shown in \cite{Far83} each sun contains a complete sun as induced subgraph, which
is obviously not module-composed. By  Lemma \ref{Li} the next result follows.  


\begin{lemma}
Module-composed graphs are HHDS-free\footnote{(house,hole,domino,sun)-free}.
\end{lemma}


Since HHDS-free graphs are perfect\footnote{A graph  is {\em perfect} if, for every induced subgraph  of , the chromatic number of  is equal to the size of a maximum clique of .}, the same holds true for module-composed graphs.

\begin{corollary}
Module-composed graphs are perfect.
\end{corollary}

Further, HHDS-free graphs are homogeneously orderable  by the results shown in \cite{BDN97}, which implies the same for module-composed graphs.

\begin{corollary}
Module-composed graphs are homogeneously orderable.
\end{corollary}


Since the graph  is module-composed but not chordal, we conclude that 
module-composed graphs are not chordal, but they are weakly chordal\footnote{A graph is {\em weakly chordal} if it does not 
contain any induced cycles of length greater than four or their complements.}, since they
are HHD-free\footnote{(house,hole,domino)-free} and HHD-free graphs are weakly chordal.




\begin{corollary}
Module-composed graphs are weakly chordal.
\end{corollary}




\section{Algorithms for module-composed graphs}




Next we give a polynomial time algorithm to recognize module-composed graphs.
Our algorithm is based on Lemma \ref{Lx}. In order to find some vertex  that satisfies the 
conditions of Lemma \ref{Lx}, we use a modular decomposition \cite{CH94} 
in our following Algorithm \ref{a2}. A basic observation is that for every connected module-composed
graph  vertex  is either a child or a grandchild of the root of .






\begin{alg} \label{a2} ~
\hrule

\medskip
\noindent
Input: Graph 

\noindent
Output: Module-sequence  or the answer NO

\begin{tabbing}
(11) \= d \= c \=  d \=  d \=  d \=  d\kill
(1)  \>  mod-com()   \\
(2)  \> \> if ( disconnected)\\
(3)  \> \> \> for every connected component  of : mod-com(); \\
(4)  \> \> else \{ \\
(5)  \> \> \> construct  with root ; \\
(6)  \> \> \> if ( is join node) \{ \\
(7)  \> \> \> \> if ( child  of  which is a leaf in ) \{  \\
(8)  \> \> \> \>    \>for every such child  of   \{; ; \}\\
(9)  \> \> \> \> \>mod-com();\}\\
(10)  \> \> \> \> else if ( child  of  labeled by co-join and a child  of  which \\
(11)  \> \> \> \> is a leaf in ) \{  \\
(12)  \> \> \> \>    \>for every such vertex   \{; ; \}\\
(13) \> \> \> \> \>mod-com(); \} \\ 
(14)  \> \> \> \} \\
(15)  \> \> \> else if ( is prime node) \{ \\
(16)  \> \> \> \> if ( child  of  which is a leaf in  and corresponds
to a vertex \\
(17)  \> \> \> \> of degree 1 in quotient graph ) \{  \\
(18)  \> \> \> \>    \>for every such child  of   \{; ; \}\\
(19)  \> \> \> \> \>mod-com();\}\\
(20)  \> \> \> \> else if ( child  of  labeled by co-join and corresponds
to a vertex  \\
(21)  \> \> \> \> of degree 1 in quotient graph  and a child  of  which is a \\
(22)  \> \> \> \> leaf in ) \{  \\
(23)  \> \> \> \> \>for every such vertex   \{; ; \}\\
(24)  \> \> \> \> \>mod-com(); \} \\ 
(25)  \> \> \> \} \\
(26)  \> \> \> else\\
(27)  \> \> \> \> return NO;\\
(28)  \> \> \} \\
\end{tabbing}

\vspace{-0.5cm}
\hrule
\end{alg}





The construction of the modular decomposition  in Line (5) of Algorithm \ref{a2} 
can be realized in time  by \cite{CH94,MS99}. 


\begin{theorem}
Given a graph , one can decide in time  whether
 is module-composed, and in the case of a positive answer, constructs  a module-sequence.
\end{theorem}






Since module-composed graphs are HHD-free, we conclude by the results shown in
\cite{JO88} the following theorem.

\begin{theorem}
For every module-composed graph which is given together with a module-sequence the size of a largest independent set, the size of a largest clique, 
the chromatic number and the minimum number of cliques covering the graph
can be computed in linear time.
\end{theorem}






\section{Independent module-composed graphs}



Next we want to characterize module-composed graphs for a restricted case.

A graph  is {\em independent module-composed}, if and only if there exists a linear 
ordering , such that for every  
the neighbourhood of vertex  in graph  
forms a module which is an independent set.



It is easy to see that independent module-composed graphs do not contain
any of the graphs of Table \ref{gr} as induced subgraph.


\begin{lemma}
Independent module-composed graphs are HHDG-free\footnote{(house,hole,domino,gem)-free}.
\end{lemma}

HHDG-free are also known as distance hereditary graphs \cite{HM90,BM86}.
Examples for distance hereditary graphs are co-graphs and trees.
For the case of bipartite graphs\footnote{A graph is bipartite if it is -free, for .}, the notion module-composed 
even is equivalent to the notion of distance hereditary. 

\begin{theorem}[\cite{AGKKW06}]\label{bipm}  Let  a bipartite graph. The following conditions are equivalent.
\begin{enumerate}
\item  is module-composed.

\item  is domino and hole free.

\item  is distance hereditary.

\item  is   -chordal\footnote{ A graph is -chordal if each cycle
of length at least  has at least  chords.}.
\end{enumerate}
\end{theorem}

For general graphs Theorem \ref{bipm} does not hold true, since 
there are module-composed graphs which are not distance hereditary, e.g. the gem and there
are  distance hereditary graph which are not module-composed, e.g. the
co-().










\bigskip
The problem to decide whether a given graph is bipartite distance hereditary
and to construct a corresponding pruning sequence can be done in linear time
by the well known characterization for bipartite graphs as 2-colorable graphs and
the linear time recognition algorithms for distance hereditary graphs shown in \cite{HM90,BM86}.
By Theorem \ref{bipm}, this immediately implies a linear time algorithms for recognizing
independent module-composed graphs. A corresponding module-sequence can be constructed 
in linear time from a pruning sequence as shown in \cite{AGKKW06}.
Since both known linear time recognition algorithms for distance hereditary graphs shown 
in \cite{HM90,BM86} are based on the fact that the neighbourhood of every vertex in 
a  distance hereditary graph is a co-graph and additional conditions, both algorithms 
are not simple.


In \cite{JO88} it is shown that for HHD-free graphs every Lex-BFS (Lexicographic Breadth First Search) 
ordering is a semi perfect elimination ordering, i.e. every vertex  is no midpoint  of an
induced  in graph 
. In the case of bipartite graphs 
this ordering obviously is even an independent module-sequence.

\begin{theorem}
Given an independent module-composed graph , every  Lex-BFS ordering  
constructs in  time  an independent module-sequence for .
\end{theorem}



To decide whether a given graph is bipartite distance hereditary can be
done by Corollary 5 shown in \cite{BM86} using the fundamental search strategy 
of BFS (Breadth First Search) which produces a classification of the vertices into levels, 
with respect to a start vertex . Level  is the set of vertices
with distance  to vertex  and is denoted by .



\begin{theorem}[Corollary 5 of \cite{BM86}]
Let  be a connected graph and let  be a vertex of . Then 
is bipartite distance hereditary if and only if all levels 
are edgeless, and for every vertices , in  and neighbours
 and  of  in , we have ,
and further  and  are either disjoint
or one is contained in the other.
\end{theorem}



A BFS starting at a vertex  can compute the level sets  
in time  and using these levels,
the conditions of Corollary 5 of \cite{BM86} can be verified in the same time.

A BFS numbering  of the vertices with respect to some
vertex  can be used to obtain a
module-sequence  as follows. We start with , .
For the first  vertices we obviously 
can choose . For the vertices of , ,
we know that their neighbours in set  are modules which can be ordered
by a series of inclusions . We rearrange the order of 
the vertices in  with respect to  such that for every such series of inclusions 
 if and only if .
This obviously leads a module-sequence for graph  if  is bipartite distance hereditary.




\begin{theorem}
Given a graph , one can decide using  BFS in time  whether
 is independent module-composed, and in the case of a positive answer, 
construct a module-sequence.
\end{theorem}


On bipartite distance hereditary graphs, and so on independent module-composed graphs, the 
path-partition problem \cite{YC98},
hamiltonian circuit and path problem \cite{MN93},
and the computation of shapley value ratings  \cite{AGKKW06}
can be solved in polynomial time.

It is well known that distance hereditary graphs and
thus independent module-composed graphs have clique-width at most 3 \cite{GR00}. This implies that
all graph properties which are expressible in monadic second order logic with
quantifications over vertices and vertex sets (-logic) are decidable in
linear time on independent module-composed graphs \cite{CMR00}. Some of these problems are
partition into  independent sets or cliques, -dominating set, -achromatic
number, for every fixed integer .

Furthermore, there are a lot of NP-complete graph problems which are not expressible in
-logic like chromatic number, partition problems,  vertex disjoint paths, and
bounded degree subgraph problems but which can also be solved in polynomial time on clique-width
bounded graphs and thus on bipartite distance hereditary graphs \cite{EGW01a,GW06}. 

Note that general module-composed graphs are of unbounded clique-width. 
For example every graph which can be constructed from a single
vertex by a sequence of one vertex extentions by a domination vertex\footnote{A vertex  is a {\em dominating vertex} of , if it is adjacent to all other vertices in .} or a
pendant vertex\footnote{A vertex  of degree one is called a {\em pendant vertex} of .} is 
obviously module-composed. But the set of all such defined graphs
have unbounded clique-width  \cite{Rao07}.


\section{Graph class inclusions}




In Table \ref{grcl} we summarize the relation of module-composed graphs
and related graph classes. For the definition and relations of special graph classes
we refer to the survey of Brandst\"adt  et al. \cite{BLS99}.

\begin{table}
\begin{center}
\epsfig{figure=classes.eps,width=11.0cm}
\caption{Inclusion of special graph classes}\label{grcl}
\end{center}
\end{table}








\bibliographystyle{alpha}
\bibliography{/home/gurski/bib.bib}




\end{document}
