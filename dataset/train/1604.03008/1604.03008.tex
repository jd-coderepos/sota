

This section provides a proof of Theorem~\ref{thm:minor}. We describe the algorithm in Subsection~\ref{sec:description}, we prove its correctness in Subsection~\ref{sec:correctness}, and we analyze its running time in Subsection~\ref{sec:runtime}.


\subsection{Description of the algorithm} 
\label{sec:description}

Let    be the display graph of a collection  of unrooted phylogenetic trees, and let . By Theorem~\ref{th:5-approx} and Theorem~\ref{Theo1ext}, we may assume that we are given a
nice tree-decomposition  of  of width at most , as otherwise we can safely conclude that  is not compatible. Let   be the root of , and recall that .

Our objective is to build a compatible supertree  of , if such exists. (We would like to note that there could exist an exponential number of compatible supertrees; we are just interested in constructing {\sl one} of them.) As it is usually the case of dynamic programming algorithms on tree-decompositions, for building  we process  in a bottom-up way from the leaves to the root, where we will eventually decide whether a solution exists or not. We first describe the data structure used by the algorithm along with a succinct intuition behind the defined objects, and then we proceed to the description of the dynamic programming algorithm itself.






\paragraph{\textbf{\emph{Description of the data structure}}.} Before defining the dynamic-programming table associated with every node  of , we need a few more definitions.

\begin{definition}\label{def:supertree}
Given a node  of , its graph , and  a subset , a  \emph{-supertree} is a tuple  such that
\begin{itemize}
\item[]  is a tree containing at most  vertices,
\item[] , called the \emph{vertex-model function}, associates every  with a subset  such that
\begin{itemize}
\item[]  is connected and if  is a labeled vertex, then , and
\item[] if  and  are two vertices of  such that , then ,
\end{itemize}

\item[] , called \emph{the edge-model function}, associates a subset of colors with every edge of , and

\item[] , called the \emph{vertex-representative function}, selects, for each vertex , a \emph{representative}  in the vertex-model .
\end{itemize}

\noindent Moreover, we say that a -supertree  is \emph{valid} if
\begin{itemize}
\item[] for every  such that , then the unique edge  between  and  exists in  and satisfies .
\end{itemize}

\noindent For a node  of , we define a \emph{-supertree} as a -supertree and a \emph{-supertree} as a -supertree.
\end{definition}




To give some intuition on why -supertrees capture partial solutions of our problem, let us assume that  is a compatible supertree of  and consider a node  of . Then
we can define a -supertree  as follows:



\begin{tight_enumerate}
\item [] For every vertex ,  can be chosen as any element in the set ,
\item [] , where ,
\item [] for every vertex , ,  where  is the vertex-model of  in , and
\item[]  for every edge ,
 if
there exist an edge , with  , and
an edge 
such that  is incident to a vertex of  and to a vertex of , and
 is on the unique path  in  between the vertices incident to .
\end{tight_enumerate}

\vspace{-.15cm}
The edge-model function  introduced in Definition~\ref{def:supertree} allows to keep track, for every edge , of the set of trees in  containing an edge having  as an edge-model.
Observe that the size of a vertex-model  in  of some vertex  may depend on  (so, a priori,  we may need to consider a number of vertex-models of size exponential in ). We overcome this problem via the vertex-representative function , which allows us to store a tree  of size at most . This tree  captures how the vertex-models in  ``project'' to the current bag, namely , of the tree-decomposition of the display graph.








Before we describe the information stored at each node of the tree-decomposition, we need three more definitions.

\begin{definition}\label{def:shadow}
A tuple  is called a \emph{shadow} -supertree
if there exists a -supertree  such that
\begin{itemize}
\item[]  is a tree obtained from  by subdividing every edge once, called \emph{shadow tree}. The new vertices are called \emph{shadow vertices} and denoted by , while the original ones, that is, , are denoted by ,
\item[] for every ,  is a subset of  such that  is connected and such that , where we licitly consider the vertices in   as a subset of . Furthermore, if  with , then ,
\item[]  such that for every , if  and  are the neighbors of  in , then , and
\item[]  such that for every , .
\end{itemize}
\end{definition}

We say that  is a \emph{shadow} of . Note that  may have more than one shadow satisfying Definition~\ref{def:shadow}.

\begin{definition}\label{def:restriction}
Let  be a -supertree. The \emph{restriction} of  to a subset of vertices   is defined as the -supertree , where
\begin{itemize}
\item[]  , where ,
\item[] for every , ,
\item[]  for every , , where  is the unique path in  between the vertices incident to , and
\item[]  for every , .
\end{itemize}
If  is a -supertree and , we define a \emph{shadow restriction of  to } as a shadow of , and we denote it by
 .
 \end{definition}

\begin{definition}\label{def:equivalent} Two -supertrees  and  are \emph{equivalent}, and we denote it by  , if  there exists an isomorphism  from  to  such that
\begin{itemize}
\item[]  , , ,
\item[] , , and \item[] , .
\end{itemize}
\end{definition}

Every  node  of  is associated with a set  of pairs  , called \emph{colored shadow -supertrees}, where  is a shadow -supertree and  is the so-called \emph{coloring function}.
The dynamic programming algorithm will maintain the following invariant:

\begin{invariant} \label{inv:dp-minor}
{A colored shadow -supertree  belongs to  if and only if there exists a valid -supertree  such that}

\begin{itemize}
\item[\emph{\textbf{(1)}}] ,
\item[\emph{\textbf{(2)}}] for every , a color  if and only if there exists  with   such that , and
\item[\emph{\textbf{(3)}}] for every  with neighbors  and  in , a color  if there exists  with  and  such that
  the unique path between  and  in  uses at least one vertex of .
\end{itemize}
\end{invariant}

Intuitively, condition \textbf{(2)} of Invariant~\ref{inv:dp-minor} guarantees that for every vertex , we can recover the set of trees for which  has already appeared in a vertex-model of a vertex of .  On the other hand, condition \textbf{(3)} of Invariant~\ref{inv:dp-minor} is useful for the following reason. When a vertex is forgotten in the tree-decomposition, we need to keep track of its ``trace'', in the sense that the colors given to the corresponding shadow vertex guarantee that the algorithm will  construct vertex-models appropriately.
If  is a coloring function satisfying conditions \textbf{(2)} and \textbf{(3)}, we say that  is \emph{consistent} with .

\smallskip

 For , we denote by  the unique colored shadow -supertree. From the above description, it follows that the collection  is compatible  if and only if . Indeed, for  we have that  and . In that case, the only condition imposed by Invariant~\ref{inv:dp-minor} is the existence of a valid -supertree. Then, by Definition~\ref{def:supertree}, the existence of such a supertree is equivalent to the existence of a compatible supertree  of  in which the vertex-models and edge-models are given by the functions  and , respectively. Finally, note that the first condition of Definition~\ref{def:supertree}, namely that , is not a restriction on the set of solutions, as we may clearly assume that the size of a compatible supertree is always at most the size of the display graph.






\paragraph{\textbf{\emph{Description of the dynamic programming algorithm}}.} Let  be a nice tree-decomposition of the display graph  of . We proceed to describe how to compute the set   for every node . For that, we will assume inductively that, for every descendant  of , we have at hand the set  that has been correctly built.  We distinguish several cases depending on the type of node :


\begin{figure}[t]
\vspace{-.1cm}
\begin{center}
\vspace{.25cm}
 \scalebox{.85}{\begin{tikzpicture}

\def\x{7}
\def\y{0}

\capt{-1+\x,-4+\y}{(iii)};
\node[vertex,fill=black,label={0:}, inner sep=0pt,minimum size=6pt] (x) at (\x,\y) {};
\outb{x}

\draw (0+\x,-0.5+\y) -- +(0,-0.5);
\draw (0+\x,-1+\y) -- +(0.1,0.15);
\draw (0+\x,-1+\y) -- +(-0.1,0.15);

\def\x{7}
\def\y{-2}
\node[vertex,fill=black,label={0:}, inner sep=0pt,minimum size=6pt] (x) at (\x,\y) {};

\node[vertex,fill=black,label={0:}, inner sep=0pt,minimum size=6pt] (a) at (\x,\y-1) {};
\draw (x) --(a);
\node[circle,draw=black,fill=white, inner sep=0pt,minimum size=6pt,label={0:}] at (\x,\y-1/2) {};
\outb{x}



\def\x{9.5}
\def\y{0}
\capt{0+\x,-4+\y}{(iv)};
\node[vertex,fill=black,label={180:}, inner sep=0pt,minimum size=6pt] (x) at (\x,\y) {};
\node[circle,draw=black,fill=white, inner sep=0pt,minimum size=6pt, label={0:}] (y) at (\x+2,\y) {};
\draw (x) -- (y);


\outb{x}
\outb{y}

\draw (1+\x,-0.5+\y) -- +(0,-0.5);
\draw (1+\x,-1+\y) -- +(0.1,0.15);
\draw (1+\x,-1+\y) -- +(-0.1,0.15);

\def\x{9.5}
\def\y{-2}
\node[vertex,fill=black,label={180:}, inner sep=0pt,minimum size=6pt] (x) at (\x,\y) {};
\node[circle,draw=black,fill=white, inner sep=0pt,minimum size=6pt, label={0:}] (y) at (\x+2,\y) {};
\node[vertex,fill=black,label={90:}, inner sep=0pt,minimum size=6pt] (b) at (\x+1,\y) {};
\node[vertex,fill=black,label={0:}, inner sep=0pt,minimum size=6pt] (a) at (\x+1,\y-1) {};
\draw (x) -- (y);
\draw (a) -- (b);
\outb{x}
\outb{y}

\node[circle,draw=black,fill=white, inner sep=0pt,minimum size=6pt,label={-90:}] at (\x+0.5,\y) {};
\node[circle,draw=black,fill=white, inner sep=0pt,minimum size=6pt,label={0:}] at (\x+1,\y-0.5) {};

\def\x{2.5}
\def\y{0}
\capt{0+\x,-4+\y}{(ii)};
\node[vertex,fill=black,label={180:}, inner sep=0pt,minimum size=6pt] (x) at (\x,\y) {};
\node[circle,draw=black,fill=white, inner sep=0pt,minimum size=6pt,label={0:}] (y) at (\x+2,\y) {};
\draw (x) -- (y);
\outb{x}
\outb{y}



\draw (1+\x,-0.5+\y) -- +(0,-0.5);
\draw (1+\x,-1+\y) -- +(0.1,0.15);
\draw (1+\x,-1+\y) -- +(-0.1,0.15);




\def\x{2.5}
\def\y{-2}
\node[vertex,fill=black,label={180:}, inner sep=0pt,minimum size=6pt] (x) at (\x,\y) {};
\node[circle,draw=black,fill=white, inner sep=0pt,minimum size=6pt,label={0:}] (y) at (\x+2,\y) {};
\node[vertex,fill=black,label={-90:}, inner sep=0pt,minimum size=6pt] (a) at (\x+1.34,\y) {};
\draw (x) -- (y);

\outb{x}
\outb{y}

\node[circle,draw=black,fill=white, inner sep=0pt,minimum size=6pt,label={-90:}] at (\x+0.66,\y) {};


\def\x{0}
\def\y{0}
\capt{-1+\x,-4+\y}{(i)};
\node[vertex,fill=black,label={0:}, inner sep=0pt,minimum size=6pt] (x) at (\x,\y) {};
\outb{x}

\draw (0+\x,-0.5+\y) -- +(0,-0.5);
\draw (0+\x,-1+\y) -- +(0.1,0.15);
\draw (0+\x,-1+\y) -- +(-0.1,0.15);


\def\x{0}


\def\y{-2}
\node[vertex,fill=black,label={0:}, inner sep=0pt,minimum size=6pt] (x) at (\x,\y) {};
\outb{x}


\end{tikzpicture}
 }
\end{center}
\vspace{-.3cm}
\caption{The four possible cases (i-iv) in the dynamic programming algorithm. The configurations above correspond to , while the ones below correspond to . Full dots correspond to vertices in , the other ones being in .\label{fig:minor}}
\end{figure}


\begin{enumerate}
\item {\bf  is a leaf with :} , where  is a tree with only one vertex , , , , and .

\medskip
\item {\bf  is an introduce-vertex node such that the introduced vertex   is unlabeled:}
For every element  of ,
we add  to  the elements  of the form  that can be built according to one of the following four cases. For all of them, we define the vertex-representative function such that   for some vertex , and  for every , . The different cases depend on this vertex .


\medskip
\begin{itemize}
\item[(i)] {\bf  such that   and .} See Figure~\ref{fig:minor}(i) for an example.  We define .  Let us define  , , and .
\begin{itemize}
\item \emph{Definition of the vertex-model function}:  is connected, contains ,
  and for every , . For every , .
\item \emph{Definition of the edge-model function}: For every , .
\item\emph{ Definition of the coloring function}: For every , .
\end{itemize}

\medskip
\item[(ii)] \textbf{  and  subdivides an edge  of  with }. See Figure~\ref{fig:minor}(ii) for an example.
Since  is a shadow tree, assume w.l.o.g. that  and .
Then  is obtained from  by removing the edge , adding two vertices  and  and three edges , , and . Let us define  , , and .
\begin{itemize}
\item \emph{Definition of the vertex-model function}:  is connected, contains ,
  and for each , .
  For each ,  is connected,  , and
 if  is unlabeled, then .
For each  with , .

\item \emph{Definition of the edge-model function}: For each , . Also,
  .

\item \emph{Definition of the coloring function}: For each , .   and .
For each ,    and . Finally,  and .

\end{itemize}

\medskip
\item[(iii)] \textbf{  with  and  is connected to a vertex }. See Figure~\ref{fig:minor}(iii) for an example.
  is obtained from  by adding two vertices  and  and two edges  and . Let us define , , and .
\begin{itemize}
\item \emph{Definition of the vertex-model function}:
 is connected, contains , and for each , .
For each ,  is connected, , and
 if  is unlabeled, then .
For each  with , .

\item \emph{Definition of the edge-model function}:
For each , , and
  .

\item \emph{Definition of the coloring function}:
  For each , .
  For each ,  and .
\end{itemize}

\medskip
\item[(iv)] \textbf{ with  and  subdivides an edge  of }. See Figure~\ref{fig:minor}(iv) for an example. Again, we may assume that  and .
Then  is obtained from  by removing the edge ,
adding four vertices  and , and five edges , , , , and . Let us define , , and .
\begin{itemize}
\item \emph{Definition of the vertex-model function}:
 is connected, contains  and, for every , .
For each ,  is connected,  , and
 if  is unlabeled, then .
For each  with , .

\item \emph{Definition of the edge-model function}:
For each edge , .
, and
.

\item \emph{Definition of the coloring function}:
For every , .
For every ,  and .
For every ,  and .
For every ,    and .
\end{itemize}
\end{itemize}

\medskip
\item {\bf  is an introduce-vertex node such that the introduced vertex   is labeled:} This case is very similar to Case~2 but, as vertex  is a leaf, only Case~2(iii) and Case~2(iv) can be applied. In both cases, we further impose that   and .

\medskip
\item {\bf  in an introduce-edge node for an edge  with :}
Let  be an element of  such that there exist  and  such that  and .
We construct  as an element of  as follows:
 .
For every , .
For every , .
.
For every , .

\medskip
\item
{\bf  is a forget-vertex node for a vertex :} Let  be an element of .
We construct  as an element of  as follows: .
For every , .
For every , if  and  are the neighbors of  in , then  on the path between  and  in  and .

\medskip
\item
{\bf  is a join node:}
Let  be an element of  and
let  be an element of  such that
for every ,  and
for every , . We construct  as an element of  as follows:
For every , , and
 for every , .

\end{enumerate}





