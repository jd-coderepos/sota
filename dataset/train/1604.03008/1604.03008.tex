

This section provides a proof of Theorem~\ref{thm:minor}. We describe the algorithm in Subsection~\ref{sec:description}, we prove its correctness in Subsection~\ref{sec:correctness}, and we analyze its running time in Subsection~\ref{sec:runtime}.


\subsection{Description of the algorithm} 
\label{sec:description}

Let   $D=(V_D,E_D)$ be the display graph of a collection $\mathcal{T}=\{(T_1, \phi_1), (T_2, \phi_2), \ldots, (T_k, \phi_k)\}$ of unrooted phylogenetic trees, and let $n = |V(D)|$. By Theorem~\ref{th:5-approx} and Theorem~\ref{Theo1ext}, we may assume that we are given a
nice tree-decomposition $(\mathsf{T},\mathcal{B})$ of $D$ of width at most $5k+4$, as otherwise we can safely conclude that $\mathcal{T}$ is not compatible. Let  $t_r$ be the root of ${\sf T}$, and recall that $B_{t_r} = \es$.

Our objective is to build a compatible supertree $\widehat{T}$ of $\mathcal{T}$, if such exists. (We would like to note that there could exist an exponential number of compatible supertrees; we are just interested in constructing {\sl one} of them.) As it is usually the case of dynamic programming algorithms on tree-decompositions, for building $\widehat{T}$ we process $({\sf T},\mathcal{B})$ in a bottom-up way from the leaves to the root, where we will eventually decide whether a solution exists or not. We first describe the data structure used by the algorithm along with a succinct intuition behind the defined objects, and then we proceed to the description of the dynamic programming algorithm itself.






\paragraph{\textbf{\emph{Description of the data structure}}.} Before defining the dynamic-programming table associated with every node $t$ of $(\mathsf{T},\mathcal{B})$, we need a few more definitions.

\begin{definition}\label{def:supertree}
Given a node $t$ of $(\mathsf{T},\mathcal{B})$, its graph $G_t = (V_t,E_t)$, and  a subset $Z\subseteq V_t$, a  \emph{$(Z,t)$-supertree} is a tuple $\mathfrak{T}=(T,\varphi,\psi,\rho)$ such that
\begin{itemize}
\item[$\bullet$] $T$ is a tree containing at most $|B_t|+|Z|$ vertices,
\item[$\bullet$] $\varphi : Z \rightarrow 2^{V(T)}$, called the \emph{vertex-model function}, associates every $v\in Z$ with a subset $\varphi(v)$ such that
\begin{itemize}
\item[$\circ$] $T[\varphi(v)]$ is connected and if $v$ is a labeled vertex, then $|\varphi(v)|=1$, and
\item[$\circ$] if $u$ and $v$ are two vertices of $Z$ such that $c(u)=c(v)$, then $\varphi(u)\cap\varphi(v)=\emptyset$,
\end{itemize}

\item[$\bullet$] $\psi : E(T) \rightarrow 2^{[k]}$, called \emph{the edge-model function}, associates a subset of colors with every edge of $T$, and

\item[$\bullet$] $\rho: Z \to V(T)$, called the \emph{vertex-representative function}, selects, for each vertex $v \in Z$, a \emph{representative} $\rho(v)$ in the vertex-model $\varphi(v) \subseteq V(T)$.
\end{itemize}

\noindent Moreover, we say that a $(Z,t)$-supertree $(T,\varphi,\psi,\rho)$ is \emph{valid} if
\begin{itemize}
\item[$\bullet$] for every $\{u,v\} \in E_t$ such that $u, v \in Z$, then the unique edge $e$ between $\varphi(u)$ and $\varphi(v)$ exists in $T$ and satisfies $c(\{u,v\}) \in \psi(e)$.
\end{itemize}

\noindent For a node $t$ of $(\mathsf{T},\mathcal{B})$, we define a \emph{$B_t$-supertree} as a $(B_t,t)$-supertree and a \emph{$V_t$-supertree} as a $(V_t,t)$-supertree.
\end{definition}




To give some intuition on why $(Z,t)$-supertrees capture partial solutions of our problem, let us assume that $\widehat{T}$ is a compatible supertree of $\mathcal{T}$ and consider a node $t$ of $(\mathsf{T},\mathcal{B})$. Then
we can define a $B_t$-supertree $\mathfrak{T}=(T,\varphi,\psi,\rho)$ as follows:



\begin{tight_enumerate}
\item [$\bullet$] For every vertex $v \in B_t$, $\rho(v)$ can be chosen as any element in the set $\widehat{\varphi}(v)$,
\item [$\bullet$] $T=\widehat{T}|_Y$, where $Y = \underset{v \in B_t}{\bigcup} \rho(v)$,
\item [$\bullet$] for every vertex $v \in B_t$, $\varphi(v) = V(T) \cap \widehat{\varphi}(v)$,  where $\widehat{\varphi}(v)$ is the vertex-model of $v$ in $\widehat{T}$, and
\item[$\bullet$]  for every edge $e\in E(T)$,
$i\in \psi(e)$ if
there exist an edge $\{u,v\}\in E_t$, with  $c(\{u,v\})=i$, and
an edge $f\in E(\widehat{T})$
such that $f$ is incident to a vertex of $\widehat{\varphi}(u)$ and to a vertex of $\widehat{\varphi}(v)$, and
$f$ is on the unique path  in $\widehat{T}$ between the vertices incident to $e$.
\end{tight_enumerate}

\vspace{-.15cm}
The edge-model function $\psi$ introduced in Definition~\ref{def:supertree} allows to keep track, for every edge $e\in E(T)$, of the set of trees in $\mathcal{T}$ containing an edge having $e$ as an edge-model.
Observe that the size of a vertex-model $\widehat{\varphi}(v)$ in $\widehat{T}$ of some vertex $v\in V_D$ may depend on $n$ (so, a priori,  we may need to consider a number of vertex-models of size exponential in $n$). We overcome this problem via the vertex-representative function $\rho$, which allows us to store a tree $T$ of size at most $2k$. This tree $T$ captures how the vertex-models in $\widehat{T}$ ``project'' to the current bag, namely $B_t$, of the tree-decomposition of the display graph.








Before we describe the information stored at each node of the tree-decomposition, we need three more definitions.

\begin{definition}\label{def:shadow}
A tuple $\mathfrak{T}_s=(T_s,\varphi_s,\psi_s,\rho_s)$ is called a \emph{shadow} $B_t$-supertree
if there exists a $B_t$-supertree $\mathfrak{T}=(T,\varphi,\psi,\rho)$ such that
\begin{itemize}
\item[$\bullet$] $T_s$ is a tree obtained from $T$ by subdividing every edge once, called \emph{shadow tree}. The new vertices are called \emph{shadow vertices} and denoted by $S(T_s)$, while the original ones, that is, $V(T_s) \setminus S(T_s)$, are denoted by $O(T_s)$,
\item[$\bullet$] for every $v \in B_t$, $\varphi_s(v)$ is a subset of $V(T_s)$ such that $T_s[\varphi_s(v)]$ is connected and such that $\varphi(v) = \varphi_s(v) \cap O(T_s)$, where we licitly consider the vertices in $\varphi(v)$  as a subset of $O(T_s)$. Furthermore, if $u,v \in B_t$ with $c(u) = c(v)$, then $\varphi_s(u) \cap \varphi_s(v) = \emptyset$,
\item[$\bullet$] $\psi_s: E(T_s) \to 2^{\dotsss{k}}$ such that for every $s \in S(T_s)$, if $x$ and $y$ are the neighbors of $s$ in $T_s$, then $\psi_s(\{x,s\}) = \psi_s(\{s,y\}) = \psi(\{x,y\})$, and
\item[$\bullet$] $\rho_s:B_t \to V(T_s)$ such that for every $v \in B_t$, $\rho_s(v) = \rho(v)$.
\end{itemize}
\end{definition}

We say that $\TT_s$ is a \emph{shadow} of $\TT$. Note that $\TT$ may have more than one shadow satisfying Definition~\ref{def:shadow}.

\begin{definition}\label{def:restriction}
Let $\mathfrak{T}=(T,\varphi,\psi,\rho)$ be a $(Z,t)$-supertree. The \emph{restriction} of $\mathfrak{T}$ to a subset of vertices $Y\subseteq V_t$  is defined as the $(Y,t)$-supertree $\mathfrak{T}|_Y=(\tilde{T},\tilde{\varphi},\tilde{\psi},\tilde{\rho})$, where
\begin{itemize}
\item[$\bullet$]  $\tilde{T}=T|_Z$, where $Z=\{\rho(v)\mid v\in Y\}$,
\item[$\bullet$] for every $v \in Y$, $\tilde{\varphi}(v) = \varphi(v) \cap V(T|_{Y})$,
\item[$\bullet$]  for every $ e\in E(\tilde{T})$, $\tilde{\psi}(e) = \bigcup_{f \in E(P_e)} \psi(f)$, where $P_e$ is the unique path in $T$ between the vertices incident to $e$, and
\item[$\bullet$]  for every $ v \in Y$, $\tilde{\rho}(v) = \rho(v)$.
\end{itemize}
If $\TT$ is a $(Z,t)$-supertree and $B_t \subseteq Z$, we define a \emph{shadow restriction of $\TT $ to $B_t$} as a shadow of $\TT|_{B_t}$, and we denote it by
 $\TT|_{B_t}^s$.
 \end{definition}

\begin{definition}\label{def:equivalent} Two $(Z,t)$-supertrees $\mathfrak{T}=(T,\varphi, \psi, \rho)$ and $\mathfrak{T}'=(T',\varphi', \psi', \rho')$ are \emph{equivalent}, and we denote it by  $\mathfrak{T}\simeq \mathfrak{T}'$, if  there exists an isomorphism $\alpha$ from $T$ to $T'$ such that
\begin{itemize}
\item[$\bullet$]  $\forall v \in Z$, $\forall a\in \varphi(v)$, $\alpha(a)\in\varphi'(v)$,
\item[$\bullet$] $\forall e \in E(T)$, $\psi(e) = \psi'(\alpha(e))$, and \item[$\bullet$] $\forall v \in Z$, $\alpha(\rho(v)) = \rho'(v)$.
\end{itemize}
\end{definition}

Every  node $t$ of $(\mathsf{T},\mathcal{B})$ is associated with a set $\Rcal_t$ of pairs  $(\mathfrak{T},\gamma)$, called \emph{colored shadow $B_t$-supertrees}, where $\mathfrak{T}=(T,\varphi,\psi,\rho)$ is a shadow $B_t$-supertree and $\gamma:V(T) \rightarrow 2^{\dotsss{k}}$ is the so-called \emph{coloring function}.
The dynamic programming algorithm will maintain the following invariant:

\begin{invariant} \label{inv:dp-minor}
{A colored shadow $B_t$-supertree $(\mathfrak{T}=(T,\varphi, \psi, \rho),\gamma)$ belongs to $\Rcal_t$ if and only if there exists a valid $V_t$-supertree $\mathfrak{T}
_\ps=(T_\ps,\varphi_\ps, \psi_\ps, \rho_\ps)$ such that}

\begin{itemize}
\item[\emph{\textbf{(1)}}] $\mathfrak{T}\simeq \mathfrak{T}_\ps|_{B_t}^s$,
\item[\emph{\textbf{(2)}}] for every $a \in V(T)$, a color $i \in \gamma(a)$ if and only if there exists $ u \in V_t$ with $c(u)=i$  such that $a \in \varphi_{\ps}(u)$, and
\item[\emph{\textbf{(3)}}] for every $z \in S(T)$ with neighbors $x$ and $y$ in $V(T)$, a color $i \in \gamma(z)$ if there exists $u \in V_t$ with $c(u)=i$ and $x,y \not \in \varphi_\ps(u)$ such that
  the unique path between $x$ and $y$ in $T_\ps$ uses at least one vertex of $\varphi_\ps(u)$.
\end{itemize}
\end{invariant}

Intuitively, condition \textbf{(2)} of Invariant~\ref{inv:dp-minor} guarantees that for every vertex $v\in V(T)$, we can recover the set of trees for which $v$ has already appeared in a vertex-model of a vertex of $V_t\setminus B_t$.  On the other hand, condition \textbf{(3)} of Invariant~\ref{inv:dp-minor} is useful for the following reason. When a vertex is forgotten in the tree-decomposition, we need to keep track of its ``trace'', in the sense that the colors given to the corresponding shadow vertex guarantee that the algorithm will  construct vertex-models appropriately.
If $\gamma$ is a coloring function satisfying conditions \textbf{(2)} and \textbf{(3)}, we say that $\gamma$ is \emph{consistent} with $\mathfrak{T}_\ps$.

\smallskip

 For $Z=\emptyset$, we denote by $\oslash$ the unique colored shadow $(Z,t)$-supertree. From the above description, it follows that the collection $\mathcal{T}$ is compatible  if and only if $\oslash \in \Rcal_{t_r}$. Indeed, for $t = t_r$ we have that $B_{t_r} = \emptyset$ and $V_{t_r} = V_D$. In that case, the only condition imposed by Invariant~\ref{inv:dp-minor} is the existence of a valid $V_D$-supertree. Then, by Definition~\ref{def:supertree}, the existence of such a supertree is equivalent to the existence of a compatible supertree $\widehat{T}$ of $\mathcal{T}$ in which the vertex-models and edge-models are given by the functions $\varphi$ and $\psi$, respectively. Finally, note that the first condition of Definition~\ref{def:supertree}, namely that $|\widehat{T}| \leq |B_{t_r}| + |V_{t_r}|  = |V_D|$, is not a restriction on the set of solutions, as we may clearly assume that the size of a compatible supertree is always at most the size of the display graph.






\paragraph{\textbf{\emph{Description of the dynamic programming algorithm}}.} Let $(\mathsf{T},\mathcal{B})$ be a nice tree-decomposition of the display graph $D$ of $\mathcal{T}$. We proceed to describe how to compute the set $\Rcal_t$  for every node $t\in\mathsf{T}$. For that, we will assume inductively that, for every descendant $t'$ of $t$, we have at hand the set $\Rcal_{t'}$ that has been correctly built.  We distinguish several cases depending on the type of node $t$:


\begin{figure}[t]
\vspace{-.1cm}
\begin{center}
\vspace{.25cm}
 \scalebox{.85}{\begin{tikzpicture}

\def\x{7}
\def\y{0}

\capt{-1+\x,-4+\y}{(iii)};
\node[vertex,fill=black,label={0:$x$}, inner sep=0pt,minimum size=6pt] (x) at (\x,\y) {};
\outb{x}

\draw (0+\x,-0.5+\y) -- +(0,-0.5);
\draw (0+\x,-1+\y) -- +(0.1,0.15);
\draw (0+\x,-1+\y) -- +(-0.1,0.15);

\def\x{7}
\def\y{-2}
\node[vertex,fill=black,label={0:$x$}, inner sep=0pt,minimum size=6pt] (x) at (\x,\y) {};

\node[vertex,fill=black,label={0:$a$}, inner sep=0pt,minimum size=6pt] (a) at (\x,\y-1) {};
\draw (x) --(a);
\node[circle,draw=black,fill=white, inner sep=0pt,minimum size=6pt,label={0:$s$}] at (\x,\y-1/2) {};
\outb{x}



\def\x{9.5}
\def\y{0}
\capt{0+\x,-4+\y}{(iv)};
\node[vertex,fill=black,label={180:$x$}, inner sep=0pt,minimum size=6pt] (x) at (\x,\y) {};
\node[circle,draw=black,fill=white, inner sep=0pt,minimum size=6pt, label={0:$y$}] (y) at (\x+2,\y) {};
\draw (x) -- (y);


\outb{x}
\outb{y}

\draw (1+\x,-0.5+\y) -- +(0,-0.5);
\draw (1+\x,-1+\y) -- +(0.1,0.15);
\draw (1+\x,-1+\y) -- +(-0.1,0.15);

\def\x{9.5}
\def\y{-2}
\node[vertex,fill=black,label={180:$x$}, inner sep=0pt,minimum size=6pt] (x) at (\x,\y) {};
\node[circle,draw=black,fill=white, inner sep=0pt,minimum size=6pt, label={0:$y$}] (y) at (\x+2,\y) {};
\node[vertex,fill=black,label={90:$b$}, inner sep=0pt,minimum size=6pt] (b) at (\x+1,\y) {};
\node[vertex,fill=black,label={0:$a$}, inner sep=0pt,minimum size=6pt] (a) at (\x+1,\y-1) {};
\draw (x) -- (y);
\draw (a) -- (b);
\outb{x}
\outb{y}

\node[circle,draw=black,fill=white, inner sep=0pt,minimum size=6pt,label={-90:$s_1$}] at (\x+0.5,\y) {};
\node[circle,draw=black,fill=white, inner sep=0pt,minimum size=6pt,label={0:$s_2$}] at (\x+1,\y-0.5) {};

\def\x{2.5}
\def\y{0}
\capt{0+\x,-4+\y}{(ii)};
\node[vertex,fill=black,label={180:$x$}, inner sep=0pt,minimum size=6pt] (x) at (\x,\y) {};
\node[circle,draw=black,fill=white, inner sep=0pt,minimum size=6pt,label={0:$y$}] (y) at (\x+2,\y) {};
\draw (x) -- (y);
\outb{x}
\outb{y}



\draw (1+\x,-0.5+\y) -- +(0,-0.5);
\draw (1+\x,-1+\y) -- +(0.1,0.15);
\draw (1+\x,-1+\y) -- +(-0.1,0.15);




\def\x{2.5}
\def\y{-2}
\node[vertex,fill=black,label={180:$x$}, inner sep=0pt,minimum size=6pt] (x) at (\x,\y) {};
\node[circle,draw=black,fill=white, inner sep=0pt,minimum size=6pt,label={0:$y$}] (y) at (\x+2,\y) {};
\node[vertex,fill=black,label={-90:$a$}, inner sep=0pt,minimum size=6pt] (a) at (\x+1.34,\y) {};
\draw (x) -- (y);

\outb{x}
\outb{y}

\node[circle,draw=black,fill=white, inner sep=0pt,minimum size=6pt,label={-90:$s$}] at (\x+0.66,\y) {};


\def\x{0}
\def\y{0}
\capt{-1+\x,-4+\y}{(i)};
\node[vertex,fill=black,label={0:$a$}, inner sep=0pt,minimum size=6pt] (x) at (\x,\y) {};
\outb{x}

\draw (0+\x,-0.5+\y) -- +(0,-0.5);
\draw (0+\x,-1+\y) -- +(0.1,0.15);
\draw (0+\x,-1+\y) -- +(-0.1,0.15);


\def\x{0}


\def\y{-2}
\node[vertex,fill=black,label={0:$a$}, inner sep=0pt,minimum size=6pt] (x) at (\x,\y) {};
\outb{x}


\end{tikzpicture}
 }
\end{center}
\vspace{-.3cm}
\caption{The four possible cases (i-iv) in the dynamic programming algorithm. The configurations above correspond to $T'$, while the ones below correspond to $T$. Full dots correspond to vertices in $O(T)$, the other ones being in $S(T)$.\label{fig:minor}}
\end{figure}


\begin{enumerate}
\item {\bf $t$ is a leaf with $B_t=\{v\}$:} $\Rcal_{t}= \{ ((T,\varphi, \psi, \rho),\gamma) \}$, where $T$ is a tree with only one vertex $a$, $\rho(v) = a$, $\varphi(v) = \{a\}$, $\psi : \es \rightarrow 2^{\dotsss{k}}$, and $\gamma(a) = \{c(v)\}$.

\medskip
\item {\bf $t$ is an introduce-vertex node such that the introduced vertex $v$  is unlabeled:}
For every element $(\TT' = (T',\varphi',\psi',\rho'),\gamma')$ of $\Rcal_{t'}$,
we add  to $\Rcal_t$ the elements  of the form $(\TT = (T,\varphi,\psi,\rho),\gamma)$ that can be built according to one of the following four cases. For all of them, we define the vertex-representative function such that  $\rho(v)= a$ for some vertex $a \in V(T)$, and  for every $u \in B_{t'}$, $\rho(u) = \rho'(u)$. The different cases depend on this vertex $a$.


\medskip
\begin{itemize}
\item[(i)] {\bf $\rho(v) = a$ such that  $a \in V(T')$ and $c(v) \not \in \gamma'(a)$.} See Figure~\ref{fig:minor}(i) for an example.  We define $T = T'$.  Let us define  $\varphi$, $\psi$, and $\gamma$.
\begin{itemize}
\item \emph{Definition of the vertex-model function}: $T[\varphi(v)]$ is connected, contains $a$,
  and for every $z \in \varphi(v)$, $c(v) \not \in \gamma'(z)$. For every $u \in B_{t'}$, $\varphi(u) = \varphi'(u)$.
\item \emph{Definition of the edge-model function}: For every $e \in E(T)$, $\psi(e) = \psi'(e)$.
\item\emph{ Definition of the coloring function}: For every $z \in V(T)$, $\gamma(z) = \gamma'(z) \cup \{c(v) \mid  z \in \varphi(v)\}$.
\end{itemize}

\medskip
\item[(ii)] \textbf{$\rho(v)=a$  and $a$ subdivides an edge $\{x,y\}$ of $T'$ with $c(v) \not \in \psi'(\{x,y\})$}. See Figure~\ref{fig:minor}(ii) for an example.
Since $T'$ is a shadow tree, assume w.l.o.g. that $x \in O(T')$ and $y \in S(T')$.
Then $T$ is obtained from $T'$ by removing the edge $\{x,y\}$, adding two vertices $a\in O(T)$ and $s \in S(T)$ and three edges $\{x,s\}$, $\{s,a\}$, and $\{a,y\}$. Let us define  $\varphi$, $\psi$, and $\gamma$.
\begin{itemize}
\item \emph{Definition of the vertex-model function}: $T[\varphi(v)]$ is connected, contains $a$,
  and for each $z \in \varphi(v)$, $c(v) \not \in \gamma'(z)$.
  For each $u \in B_{t'}$, $T[\varphi(u)]$ is connected,  $\varphi'(u) \subseteq \varphi(u) \subseteq \varphi'(u) \cup \{a\} \cup S(T)$, and
 if $u$ is unlabeled, then $\varphi(u) = \varphi'(u)$.
For each $u,u' \in B_{t}$ with $c(u) = c(u')$, $\varphi(u) \cap \varphi(u') = \es$.

\item \emph{Definition of the edge-model function}: For each $e \in E(T) \sm \{\{x,s\}, \{s,a\},\{a,y\}\}$, $\psi(e) = \psi'(e)$. Also,
  $\psi(\{x,s\}) = \psi(\{s,a\}) = \psi(\{a,y\}) = \psi'(\{x,y\})$.

\item \emph{Definition of the coloring function}: For each $z \in O(T) \sm \{a\}$, $\gamma(z) = \gamma'(z) \cup \{c(v) \mid z \in \varphi(v)\}$.  $\gamma(a) = \{i \mid \exists u \in B_t : c(u) = i$ and $ a \in \varphi(u) \} \cup \psi'(\{x,y\})$.
For each $z \in S(T')$,   $\gamma(z) = \gamma'(z) \cup \{i \mid \exists u \in B_t : c(u) = i$ and $ z \in \varphi(u) \}$. Finally, $\gamma(s) = \{i \mid \exists u \in B_t : c(u) = i$ and $s \in \varphi(u) \} \cup \psi'(\{x,y\})$.

\end{itemize}

\medskip
\item[(iii)] \textbf{$\rho(v)=a$  with $a \notin V(T')$ and $a$ is connected to a vertex $x \in V(T')$}. See Figure~\ref{fig:minor}(iii) for an example.
 $T$ is obtained from $T'$ by adding two vertices $a\in O(T)$ and $s \in S(T)$ and two edges $\{a,s\}$ and $\{s,x\}$. Let us define $\varphi$, $\psi$, and $\gamma$.
\begin{itemize}
\item \emph{Definition of the vertex-model function}:
$T[\varphi(v)]$ is connected, contains $a$, and for each $z \in \varphi(v)$, $c(v) \not \in \gamma'(z)$.
For each $u \in B_{t'}$, $T[\varphi(u)]$ is connected, $\varphi'(u) \subseteq \varphi(u) \subseteq \varphi'(u) \cup \{a\} \cup S(T)$, and
 if $u$ is unlabeled, then $\varphi(u) = \varphi'(u)$.
For each $u,u' \in B_{t}$ with $c(u) = c(u')$, $\varphi(u) \cap \varphi(u') = \es$.

\item \emph{Definition of the edge-model function}:
For each $e \in E(T) \sm \{\{a,s\}, \{s,x\}\}$, $\psi(e) = \psi'(e)$, and
  $\psi(\{a,s\}) = \psi(\{s,x\}) = \es$.

\item \emph{Definition of the coloring function}:
  For each $z \in V(T) \sm \{a,s\}$, $\gamma(z) = \gamma'(z) \cup \{c(v) \mid z \in \varphi(v)\}$.
  For each $z \in \{a,s\}$, $\gamma(z) = \{i \mid \exists u \in B_t : c(u) = i$ and $ z \in \varphi(u) \}$.
\end{itemize}

\medskip
\item[(iv)] \textbf{$\rho(v)=a$ with $a \notin V(T')$ and $a$ subdivides an edge $\{x,y\}$ of $T'$}. See Figure~\ref{fig:minor}(iv) for an example. Again, we may assume that $x \in O(T')$ and $y \in S(T')$.
Then $T$ is obtained from $T'$ by removing the edge $\{x,y\}$,
adding four vertices $a,b \in O(T)$ and $s_1,s_2 \in S(T)$, and five edges $\{x,s_1\}$, $\{s_1,b\}$, $\{b,y\}$, $\{a,s_2\}$, and $\{s_2,b\}$. Let us define $\varphi$, $\psi$, and $\gamma$.
\begin{itemize}
\item \emph{Definition of the vertex-model function}:
$T[\varphi(v)]$ is connected, contains $a$ and, for every $z \in \varphi(v)$, $c(v) \not \in \gamma'(z)$.
For each $u \in B_{t'}$, $T[\varphi(u)]$ is connected,  $\varphi'(u) \subseteq \varphi(u) \subseteq \varphi'(u) \cup \{a,b\} \cup S(T)$, and
 if $u$ is unlabeled, then $\varphi(u) = \varphi'(u)$.
For each $u,u' \in B_{t}$ with $c(u) = c(u')$, $\varphi(u) \cap \varphi(u') = \es$.

\item \emph{Definition of the edge-model function}:
For each edge $e \in E(T) \sm \{\{a,s_2\}, \{s_2,b\}, \{x,s_1\},\{s_1,b\}, \{b,y\}\}$, $\psi(e) = \psi'(e)$.
$\psi(\{x,s_1\}) = \psi(\{s_1,b\}) = \psi(\{b,y\}) = \psi'(\{x,y\})$, and
$\psi(\{a,s_2\}) = \psi(\{s_2,b\}) = \es$.

\item \emph{Definition of the coloring function}:
For every $z \in O(T) \sm \{a,b\}$, $\gamma(z) = \gamma'(z) \cup \{c(v) \mid z \in \varphi(v)\}$.
For every $z \in \{a,s_2\}$, $\gamma(z) = \{i  \mid \exists u \in B_t : c(u) = i$ and $ z \in \varphi(u) \}$.
For every $z  \in \{b,s_1\}$, $\gamma(z) = \{i \mid \exists u \in B_t : c(u) = i$ and $ z \in \varphi(u) \} \cup \psi'(\{x,y\})$.
For every $z \in S(T')$,   $\gamma(z) = \gamma'(z) \cup \{i \mid \exists u \in B_t : c(u) = i$ and $ z \in \varphi(u) \}$.
\end{itemize}
\end{itemize}

\medskip
\item {\bf $t$ is an introduce-vertex node such that the introduced vertex $v$  is labeled:} This case is very similar to Case~2 but, as vertex $v$ is a leaf, only Case~2(iii) and Case~2(iv) can be applied. In both cases, we further impose that  $\varphi(v) = \{a\}$ and $\gamma(v)=\{i\in [k]\mid v\in L(T_i), T_i\in\mathcal{T}\}$.

\medskip
\item {\bf $t$ in an introduce-edge node for an edge $\{v,w\}$ with $c(\{v,w\}) = i$:}
Let $(\TT' = (T',\varphi',\psi',\rho'),\gamma')$ be an element of $\Rcal_{t'}$ such that there exist $a \in \varphi'(v)$ and $b \in \varphi'(w)$ such that $\{a,b\} \in E(T)$ and $ i \not \in \psi'(\{a,b\})$.
We construct $(\TT = (T,\varphi,\psi,\rho),\gamma)$ as an element of $\Rcal_t$ as follows:
 $T = T'$.
For every $v \in B_t$, $\varphi(v) = \varphi'(v)$.
For every $e \in E(T) \sm \{\{a,b\}\}$, $\psi(e) = \psi'(e)$.
$\psi(\{a,b\}) = \psi'(\{a,b\}) \cup \{i\}$.
For every $v \in V(T)$, $\gamma(v) = \gamma'(v)$.

\medskip
\item
{\bf $t$ is a forget-vertex node for a vertex $v$:} Let $(\TT' = (T',\varphi',\psi',\rho'),\gamma')$ be an element of $\Rcal_{t'}$.
We construct $(\TT = (T,\varphi,\psi,\rho),\gamma)$ as an element of $\Rcal_t$ as follows: $\TT = \TT'|^z_{B_{t'}}$.
For every $a \in O(T)$, $\gamma(a) = \gamma'(a)$.
For every $z \in S(T)$, if $x$ and $y$ are the neighbors of $z$ in $T$, then $\gamma(z) = \{i \mid \exists a \in V(T')$ on the path between $x$ and $y$ in $T' : (i \in \gamma'(a))$ and $(\forall u \in B_{t}: a \not \in \varphi'(u))\}$.

\medskip
\item
{\bf $t$ is a join node:}
Let $(\TT' = (T,\varphi,\psi',\rho),\gamma')$ be an element of $\Rcal_{t'}$ and
let $(\TT'' = (T,\varphi,\psi'',\rho),\gamma'')$ be an element of $\Rcal_{t''}$ such that
for every $z \in V(T)$, $\gamma'(z) \cap \gamma''(z) = \es$ and
for every $e \in E(T)$, $\psi'(z) \cap \psi''(z) = \es$. We construct $(\TT = (T,\varphi,\psi,\rho),\gamma)$ as an element of $\Rcal_t$ as follows:
For every $e \in E(T)$, $\psi(e) = \psi'(e) \cup \psi''(e)$, and
 for every $z \in V(T)$, $\gamma(z) = \gamma'(z) \cup \gamma''(z)$.

\end{enumerate}





