\documentclass[envcountset]{llncs}
\usepackage{amsmath}
\usepackage{amssymb}
\usepackage{enumerate}
\usepackage{graphicx} 
\usepackage{array}
\usepackage{arydshln}


\usepackage{tikz}
\usetikzlibrary{positioning}
\usetikzlibrary{arrows}

\newcommand{\problemFont}[1]{\protect\ensuremath{\mathsf{#1}}}
\newcolumntype{L}[1]{>{\raggedright\let\newline\\\arraybackslash\hspace{0pt}}m{#1}}
\newcolumntype{C}[1]{>{\centering\let\newline\\\arraybackslash\hspace{0pt}}m{#1}}
\newcolumntype{R}[1]{>{\raggedleft\let\newline\\\arraybackslash\hspace{0pt}}m{#1}}


\newcommand{\PSPACE}{\problemFont{PSPACE}}
\newcommand{\NP}{\problemFont{NP}}
\newcommand{\PTIME}{\problemFont{PTIME}}
\newcommand{\si}{\sigma}
\newcommand{\Si}{\Sigma}
\newcommand{\sub}{\subseteq}
\newcommand{\pr}{\mathrm{pr}}
\newcommand{\tuple}[1]{\vec{#1}}
\newcommand {\indep}[3] {#2 ~\bot_{#1}~ #3}
\newcommand {\indepc}[2] {#1 ~\bot~ #2}
\newcommand {\indepa}[3] {#2 ~\bot^{\mathfrak a}_{#1}~#3}
\newcommand {\indepca}[2] {#1 ~\bot^{\mathfrak a}~ #2}
\newcommand {\join}[2] {*(#1, \ldots , #2)}
\newcommand {\bijoin}[2] {*(#1, #2)}
\newcommand{\len}{\textrm{len}}
\newcommand{\Dom}{\textrm{Dom}}
\newcommand{\Ran}{\textrm{Ran}}
\newcommand{\Var}{\problemFont{Var}}
\newcommand{\Fr}{\textrm{Fr}}
\newcommand{\R}{\mathbb{R}}
\newcommand{\C}{\mathbb{C}}
\newcommand{\Q}{\mathbb{Q}}
\newcommand{\N}{\mathbb{N}}
\newcommand{\No}{\mathbb{N}_0}
\newcommand{\Z}{\mathbb{Z}}
\newcommand{\Po}{\mathcal{P}}
\newcommand{\I}{\mathcal{I}}
\newcommand{\M}{\mathcal{M}}
\newcommand{\Lo}{\mathcal{L}}
\newcommand{\on}{\exists}
\newcommand{\ja}{\wedge}
\newcommand{\tai}{\vee}
\newcommand{\yli}{\overline}
\newcommand{\diam}{\operatorname{diam}}
\renewcommand{\a}{\alpha}
\renewcommand{\b}{\beta}
\newcommand{\g}{\gamma}
\newcommand{\la}{\lambda}
\newcommand{\ep}{\epsilon}
\newcommand{\de}{\delta}
\def\cF{{\cal F}}\def\dep{=\!\!}
\newcommand{\bow}[1]{\bowtie\hspace{-.6mm}(#1)}
\newcommand{\bo}{\bowtie}
\newcommand{\ap}{\approx}
\newcommand{\at}{\problemFont{Att}}



\newcommand{\ind}[1]{\FO (\bot_{\rm c})({#1}\mbox{\rm-ind})}
\newcommand{\indNR}[1]{\FO (\bot)({#1}\mbox{\rm-ind})}
\newcommand{\inc}[1]{\FO (\subseteq)({#1}\mbox{\rm-inc})}
\newcommand{\incforall}[1]{\FO (\subseteq) ({#1}\forall)}
\newcommand{\indforall}[1]{\FO (\bot_{\rm c}) ({#1}\forall)}
\newcommand{\indNRforall}[1]{\FO (\bot) ({#1}\forall)}
\newcommand{\indincforall}[1]{\FO (\bot_{\rm c}),\subseteq) ({#1}\forall)}
\newcommand{\indNRincforall}[1]{\FO (\bot,\subseteq) ({#1}\forall)}
\newcommand{\indlogic}{\FO (\bot_{\rm c})}
\newcommand{\inclogic}{\FO (\subseteq)}
\newcommand{\indNRlogic}{\FO (\bot)}
\newcommand{\indNRinclogic}{\FO (\bot,\subseteq)}
\newcommand{\indinclogic}{\FO (\bot_{\rm c},\subseteq)}
\newcommand{\deplogic}{\FO (\dep(\ldots ))}
\newcommand{\y}[1]{\overleftarrow{#1}}
\newcommand{\vdasn}{\vdash^{d}}
\newcommand{\vdast}{\vdash^{d^*}}
\newcommand{\der}{\Si \vdasn }
\newcommand{\dert}{\Si \vdast }
\newcommand{\res}{\hspace{-.5mm}\upharpoonright\hspace{-.5mm}}
\newcommand\re[2]{{\left.\kern-\nulldelimiterspace #1 \vphantom{|} \right|_{#2} }}
\newcommand{\Val}{\problemFont{Val}}
\newcommand{\chase}[1]{\overline{(#1)}}
\newcommand{\ochase}[1]{\textrm{chase}\overline{(#1)}}
\newcommand{\id}{\mathrm{id}}
\def\cF{{\cal F}}\def\dep{=\!\!}
\newcommand{\FO}{{\rm FO}}
\newcommand{\mA}{{\mathfrak A}}
\newcommand{\ESO}{{\rm ESO}}


\spnewtheorem{ex}{Example}{\bfseries}{\rmfamily}



\pagestyle{plain}




\begin{document}



\title{Reasoning about embedded dependencies\\ using inclusion dependencies}
\author{Miika Hannula}
\institute{University of Helsinki, Department of Mathematics and Statistics,
P.O. Box 68, 00014 Helsinki, Finland \email{miika.hannula@helsinki.fi}}
\maketitle
\

\begin{abstract}
The implication problem for the class of embedded dependencies is undecidable. However, this does not imply lackness of a proof procedure as exemplified by the chase algorithm. In this paper we present a complete axiomatization of embedded dependencies that is based on the chase and uses  inclusion dependencies and implicit existential quantification in the intermediate steps of deductions. \end{abstract}

\begin{keywords}
axiomatization, chase, implication problem, dependence logic, embedded dependency, tuple generating dependency, equality generating dependency, inclusion dependency
\end{keywords}


\section{Introduction}
Embedded dependencies generalize the concept database dependencies within the framework of first-order logic. Their implication is undecidable but however recursively enumerable, thus enabling complete axiomatizations. A standard example of such a proof procedure is the chase that was invented in the late 1970s \cite{aho79,maier79}, and then soon  extended to  equality and tuple generating dependencies \cite{beeri84}. In this paper we present an axiomatization for the class of embedded dependencies that simulates the chase at the logical level 
using inclusion dependencies. In particular, completeness of the rules is obtained by constructing deductions in which all the intermediate steps are inclusion dependencies, except for the first and the last step. These inclusion dependencies consist of attributes of which some are new, i.e., such that they are not allowed to appear at any earlier stage of the deduction. 

As a background example, consider the combined class of functional and inclusion dependencies. It is well known that the corresponding implication problem is undecidable, lacking hence finite axiomatization \cite{chandra85,mitchell83}.  In these situations, one strategy has been to search for axiomatizations within a more general class of   dependencies, and partly for this reason many different dependency notions were introduced in the 1980s. For instance, a textbook on dependency theory from 1991 considers more than  80 different dependency classes \cite{thalheim91}. In \cite{mitchell83b} Mitchell proposed  another strategy by presenting an axiomatization of functional and inclusion dependencies using a notion of new attributes which are to be thought of as implicitly existentially quantified. In this paper we take an analogous approach, and present an axiomatization for embedded dependencies where new attributes correspond to new  values obtained from an associated chasing sequence. These attributes are implicitly existentially quantified in the sense of \emph{team semantics}, that is, a semantic framework that has \emph{teams}, i.e.,  sets of assignments, as its underlying concept \cite{hodges97}. Team semantics is compositionally applicable to logics that extend first-order logic with various database dependencies \cite{galliani12,vaananen07}.  In this setting, \emph{inclusion logic}, i.e., first-order logic with additional inclusion dependencies, captures positive greatest fixed-point logic and hence all  recognizable classes of finite, ordered models \cite{gallhella13,immerman86,vardi82}. Therefore, inclusion dependencies with new attributes can be  thought of as existentially quantified inclusion logic formulae which in turn translate into greatest fixed-point logic. Moreover, all existentially quantified dependencies that appear in deductions translate into existential second-order logic. This may in part enable succinct intermediate  steps in contrast to axiomatic systems that simulate the chase by composing first-order definable dependencies. 

The methods described in this paper generalize the axiomatization of conditional independence and inclusion dependencies presented in \cite{hannula14}. It is also worth noting that extending relations with new attributes reminds of algebraic dependencies, that are, typed embedded dependencies defined in algebraic terms. The complete axiomatization of algebraic dependencies presented in \cite{yannakakis82} involves also an extension schema that introduces new copies of attributes.









\section{Preliminaries}
For two sets  and , we write  to denote their union, and for two sequences , we write  to denote their concatenation. For a sequence  and a mapping , we write  for . We denote by  the identity function and by  the function that maps a sequence to its th projection. 
For a function  and , we write  for the restriction of  to , and for a set of mappings , we write  for .

We start by fixing two countably infinite sets  and , the first denoting possible values of relations and the second  attributes. For notational convenience, we will assume that . For , a \emph{tuple} over  is a  mapping , and a \emph{relation} over  is a set of tuples over . We may sometimes write  to denote that  is a relation over . Values of a relation  over  are denoted by , i.e., . Let  be a \emph{valuation}, i.e., a mapping . Then for a tuple , we write , and for a relation , . A valuation  \emph{embeds}  a relation  (a tuple ) to  if  (). Since we are usually interested only valuations of a relation, we say that  is a \emph{valuation on }. For a valuation  on , we say that  is an \emph{extension} of  to another relation  if  is a valuation on  such that it agrees with  on values of .

\emph{Embedded dependencies} (ed's) can be written using first-order logic in the following way.
\begin{definition}[Embedded dependency]
Embedded dependency is a first-order sentence of the form

where  and
\begin{itemize}
\item  is a (possibly empty) conjunction of relational atoms using all of the variables ;
\item  is a conjunction of relational and equality atoms using all of the variables ; 
\item there are no equality atoms in  involving existentially quantified variables.
\end{itemize}

\end{definition}
If at most one relation symbol occurs in an ed, then we say that the ed is \emph{unirelational}, and otherwise it is \emph{multirelational}. An ed is called \emph{typed} if there is an assignment of variables to column positions such that variables in relation atoms occur only in their assigned position, and each equality atom involves a pair of variables assigned to the same position. Otherwise we say that an ed is \emph{untyped}. If  contains only one atom, then we say that the ed is \emph{single-head}, and otherwise it is \emph{multi-head}.
A single-head ed where  is an equality is called an \emph{equality generating dependency} (egd). If  is a conjunction of relational atoms, then the ed is called a \emph{tuple generating dependency} (tgd). For notational simplicity, we restrict attention to unirelational ed's. It is easy to se that any ed is equivalent to a set of tgd's and egd's, and hence we restrict attention to ed's that belong to either of these subclasses. 


The following alternative tableau presentation for egd's and tgd's are used in this paper. \begin{definition}
Let  and  be finite relations over , and . Then  and  are an egd and a tgd over , respectively, with the below satisfaction relation for a relation   over :
\begin{itemize}
\item   for all valuations  such that , it holds that  .
\item    for all valuations  on  such that , there is an extension  of  to  such that . 
\end{itemize}
\end{definition}
Sometimes we write  to denote that  is a dependency over . 
If  or  is a singleton, then we may omit the set braces in the notation, e.g., write  instead of . 

We also extend valuations to dependencies. For an egd  we write , and for a tgd   we write . Moreover,  if  is a valuation, then  and .


\begin{ex}
Consider the relation  and the tgd's  and  obtained from Fig. \ref{kuv}.\footnote{In a tableau presentation of a dependency , the distinct values of  are sometimes denoted by blank cells.}
We notice that there are two valuations on   that embed  to , namely  and . Then  since  and  embed  into , witnessed by tuples  and , respectively. We also notice that  since, although  embeds  into , no extension of  does the same.
\end{ex}

\begin{figure}[h]
\begin{center}
r=ABCs_001 2s_130 1s_22 30s_31 43\sigma_1= ABCtxy zt'x yuzx\sigma_2= ABCtxy zt'x yvzaxv'a\caption{\label{kuv}}



\end{center}
\end{figure}



Next we define inclusion dependencies which are examples of possibly untyped tgd's. 
\begin{definition}[Inclusion dependency]
Let  and  be (not necessarily distinct) tuples of attributes. Then  is an \emph{inclusion dependency} (ind) over  with the following semantic rule for a relation  over :

\end{definition}
The axiomatization presented in the next section involves inclusion dependencies that introduce new attributes. These attributes are here interpreted as existentially quantified in \emph{lax team semantics} sense \cite{galliani12}:

where  and  is the mapping that agrees with  everywhere except that it maps  to . Interestingly, inclusion logic formulae with this concept of existential quantification can be characterized with positive greatest fixed-point logic formulae (see Theorem 15 in \cite{gallhella13}).











\section{Axiomatization}\label{axiomatization}
In this section we present an axiomatization for the class of all embedded dependencies. The axiomatization contains an identity rule and three rules for the chase. We also involve conjunction in the language and therefore incorporate its usual introduction and elimination rules in the definition.  
Regarding the equalities that appear in the rules, note that both  and   indicate that the values of  and  coincide in each row. Therefore, we use  to denote ind's of either form. For a tgd (an egd) , we say that  is \emph{distinct} if it appears at most once as a value in . Namely, 
\begin{itemize}
\item for a tgd ,  is distinct if for all  and , if , then  and ;
\item for an egd ,  is distinct if  and for all  and , if , then  and .

\end{itemize}
Lastly, note that in the following  rules we  assume that values can appear as attributes and vice versa.



\begin{definition}\label{axioms}
In addition to the below rules we adopt the usual introduction and elimination rules for conjunction. In the last three rules, we assume that  is a sequence listing the attributes of .
\\
\begin{itemize}







\item[EE] Equality Exchange: 
where  is an ind and  is obtained from  by replacing any number of occurrences of  by  and any number of occurrences of  by .








\item[CS] Chase Start: 

where ,   consists of \emph{new attributes}, and  consists of \emph{distinct values}. 

\item[CR] Chase Rule: 



where tgd:  is a valuation that it is 1-1 on , and  is a \emph{new attribute} for .

\item[CT] Chase Termination: 





where  , , and  consists of \emph{distinct values}. Moreover, tgd:  is a mapping   that is the identity on , and egd: .
\end{itemize}
\end{definition}


For a dependency  over , we let , and for a set of dependencies , we let  .

\begin{definition}\label{dedu}
A deduction from  is a sequence  such that:
\begin{enumerate}
\item Each  is either an element of , an instance of [CS], or follows from one or more formulae of  by one of the rules presented above.
\item For each , if  is new in , then , and otherwise .
\end{enumerate}
We say that  is provable from , written , if there is a deduction  from  with  and such that no attributes in  are new in . \end{definition}

 We will also use the following rules that are derivable from [EE]:\\
\begin{itemize}
\item[ES] Equality Symmetry: 
\item[ET] Equality Transitivity: 




\end{itemize}
One may find the chase rules  slightly convoluted at first sight. However, the ideas behind the rules are relatively simple as illustrated in the following examples. 






\begin{ex}[Chase Start]
Let  be as in Figure \ref{3}, for  and .
\begin{figure}[h]
\center\scalebox{1.2}[1.1]{
ABCxyzt_0xy zt_1x yu_0xyz\quad 
ABCt_0xy zt_1x yu_1zx
\quad 

ABCt_0xy zt_1x yu_2zvu_3vz
}\caption{\label{3}}
\end{figure}
Then

is an instance of [CS].  Here   are interpreted either as  values or as new attributes. By the latter we intuitively mean that any relation  can be extended to some   such that . For instance, one can define  where  is the following SPJR  query

where  refers to (S)election,  to (P)rojection,  to (J)oin, and  to (R)ename operator. Then  is a relation over  such that its restriction to  lists all  for which there exist  such that  and . Let  be as in Figure \ref{3}. 
Now,

Hence proving  reduces to showing that .
\end{ex}





\begin{ex}[Chase Rule]
 Assume 

where  is as in Fig. \ref{3}, for . Then, interpreting  as , one can derive with one application of [CR]   
 from \eqref{eq2}. Note that in \eqref{eq4}  is interpreted as a \emph{new} attribute, and the idea is that any relation   satisfying \eqref{eq2} and with   can be extended to a relation  satisfying \eqref{eq4}  by introducing  suitable values for .
\end{ex}


\begin{ex}[Chase Termination]
Assume

where  is as in Fig. \ref{3}, for  and .
Then, letting , one can derive  as in Fig. \ref{3} from  \eqref{eq3} with one application of [CT].





\end{ex}








\section{Soundness Theorem}
In this section we show that the axiomatization presented in the previous section is sound.  First note that the next lemma  follows from the definitions of egd's, tgd's and ind's.
\begin{lemma}\label{clear}
Let  be a dependency over , and let  and  be relations over supersets of  and with . Then .
\end{lemma}
Then we prove the following lemma  which implies soundness of the axioms. For attribute sets  with  and a relation  over , we say that a relation  over  is an extension of  to  if . Recall from equation \ref{lax} that exactly such extensions are used in the existential quantification of lax team semantics.
\begin{lemma}\label{soundlem}
Let  be a relation over  such that , and  let  be a deduction from . Then  there exists an extension  of  to  such that .
\end{lemma}
\begin{proof}
We prove the claim by induction on . We denote by  the set . Assuming the claim for , we first find an extension  of  to  such that . If  is obtained by an application of a conjunction or some ind rule, then it is easy to see that we may choose . Hence, it suffices to consider the cases where  is obtained by using one of the chase rules. Due to Lemma \ref{clear}, it suffices to find an extension  of  to  such that . In the following cases,  denotes a sequence listing the attributes of .


\paragraph{\textbf{Case [CS].}} Assume that  is obtained by [CS] and is of the form 
where ,  consists of \emph{new attributes} and  of \emph{distinct values}. Let  be an extension of  to , where 
 We claim that . Consider the first conjunct of , and let  be a valuation on  such that . Then  is is a valuation on  such that , i.e.,  for some . Since  consists of \emph{distinct values} and thus , we may define  as an extension of  with , for . Then , and therefore .


Consider then , for , and let .
By the definition,  for some valuation  on  such that   , and hence we obtain that  for some . Therefore, .



\paragraph{\textbf{Case [CR].}} Assume that  is of the form (i)  or (ii) , and is obtained by [CR] from 

\begin{enumerate}[(i)]
\item 
\item 
\end{enumerate}
where in case (ii)  is a valuation on  such that it is 1-1 on   and  is a \emph{new attribute} for .  Let . Since ,   we first obtain that . \begin{enumerate}[(i)]
\item  Since  we find a mapping   such that , for . Since  is 1-1 on , we  can now define  as the relation obtained from  by extending each  with  for . Then for each ,  , and hence  we obtain that 
\item  It suffices to show that . Since  by , this follows immediately.
\end{enumerate}

\paragraph{\textbf{Case [CT].}} Assume that  is of the form (i)  or (ii)  and  is obtained by [CT] from 
\begin{enumerate}[(i)]
\item  where   is a mapping   that is the identity on ,
\item  where .
\end{enumerate}
Moreover, in both cases , ,  and  consists of \emph{distinct values}. It suffices to show that , so let  be a valuation on  such that . Since  consists of disctinct values,  can be extended to a valuation  on  such that . Since , there is an extension  of  to attributes in  such that . Hence, we obtain that . Let then  be such that it agrees with  on .
\begin{enumerate}[(i)]

\item  Since ,  we obtain that . Moreover, we notice that  on . 

\item Since  , we obtain that . Then  since  . 

\end{enumerate}
Hence, in both cases we obtain that . This concludes the [CT] case and the proof.\qed


   

\end{proof}

Using the previous lemma,  soundness of the rules follows.

\begin{theorem}
Let  be a finite set of egd's and tgd's over . Then  if .
\end{theorem}
\begin{proof}
Let  be a relation such that , and assume that  is a deduction from  where  contains no attributes that appear as new in .  If , then by Lemma   \ref{soundlem}  we find an extension  of  to  such that . Then using Lemma \ref{clear} we obtain that .\qed
\end{proof}



\section{Chase Revisited}\label{chasing}
In this section we define the chase for the class of egd's and tgd's. The chase algorithm was generalized to typed egd's and tgd's in \cite{beeri84}, and here we present the chase using notation similar to that in \cite{abiteboul95}. First let us assume, for notational convenience,  that there is a total, well-founded order  on the set , e.g.,  for . Let  be a set of egd's and tgd's over . A \emph{chasing sequence} of  over  is a (possibly infinite) sequence  where 
, and
 is obtained from , with , according to either of the following rules.

Let  be of the form , and  suppose that there is a valuation  on  such that  but . Then  (and ) can be applied to  as follows:
\begin{itemize}
\item \textbf{egd rule:}  Let  where  is the identity everywhere except that it maps  to  if , and  to  if .
\end{itemize}
Let  be of the form , and suppose that there is a valuation  on  such that , but there exists no extension  of  to  such that . Then   can be applied to  as follows:
\begin{itemize}
\item \textbf{tgd rule:}  List all  that have the above property, and for each  choose a \emph{distinct extension} to , i.e., an extension  to  such that each variable in  is assigned a distinct new value greater than any value in . Moreover, no new value is assigned by two  where . Then we let .
\end{itemize}

Construction of a chasing sequence is restricted with the following two conditions:
\begin{enumerate}[(i)]
\item Whenever an egd is applied, it is applied repeatedly until it is no longer applicable.
\item No dependency is starved, i.e., each dependency that is applicable infinitely many times is applied infinitely many times.
\end{enumerate}
Let  be a chasing sequence of  over . Due to the possibility of applying egd's, a chasing sequence may not be monotone with respect to . Hence, depending on whether  is a tgd or an egd, we define
\begin{itemize}
\item egd: ,
\item tgd: ,
\end{itemize}
where  and  is  for  such that  for all . Note that ``newer" values introduced by the tgd rule are always greater than the ``older" ones, and values may only be replaced  with smaller ones. Hence, no value can change infinitely often, and therefore  is always well defined and non-empty.

We also associate each chasing sequence with the following descending valuations , for . We let  ,  if  is obtained by an application of the egd rule where , and  otherwise. We then define , i.e.,  if  such that  for all . Then we obtain that 


A dependency  is \emph{trivial} if
\begin{itemize}
\item  is of the form , or
\item  is of the form  and there is a valuation  on  such that  is the identity on  and .
\end{itemize}


It is well-known that the chase algorithm captures unrestricted implication of dependencies. The proof of the following proposition is hence located in Appendix.
\begin{proposition}\label{chase}
Let  be a set of egd's and tgd's over  . Then the following are equivalent:
\begin{enumerate}[(i)]
\item ,
\item there is a chasing sequence  of  over  such that   is trivial,
\item there is a chasing sequence  of  over  such that   is trivial, for some .
\end{enumerate}
\end{proposition}





\section{Completeness Theorem}
In this section we show that the rules presented in Definition \ref{axioms} are complete for the implication problem of embedded dependencies. Let us first illustrate the use of the axioms in the following simple example.
\begin{ex}
Consider the implication problem  where  are illustrated in Fig. \ref{A}, e.g.,  where  consists of the top two rows of  and  is the bottom row. Note that  and  are embedded multivalued dependencies of the form  and , respectively, and  is a functional dependency of the form .
\begin{figure}[h]
\begin{center}
\si=ABCDa_0b_0 c_0d_0a_0b_1 c_1d_1a_0 b_0c_1d_2\si'= ABCDa_0b_0 c_0d_0a_1b_1 c_0d_1d_0=d_1\tau= ABCDa_0b_0 c_0d_0a_0b_1c_1d_1a_0 b_0c_1d_1\caption{\label{A}}



\end{center}
\end{figure}
It is easy to see that the implication holds, and this can be also verified by a chasing sequence  of  over  where  is trivial. In the chasing sequence,  and  is the result of applying  to . For this, note that there exists two valuations on   that embed  to  but has no extension that embeds  into . These valuations are the identity and the function  that swaps the values of the top and bottom row of . Then  is obtained by adding to   and  where  and 
are distinct extensions of  and  to , e.g.,  also on  and  maps  to . Also,  is the result of applying  to  two times, i.e.,  is obtained from  by replacing  with  and  with . Clearly  is trivial, and hence we obtain the claim by Proposition \ref{chase}.

\begin{figure}[h]
\begin{center}
\tau_0= ABCDa_0b_0 c_0d_0a_0b_1c_1d_1a_0 b_0c_1d_1\tau_1=ABCDa_0b_0 c_0d_0a_0b_1 c_1d_1a_0b_0 c_1d_2a_0b_1 c_0d_3a_0 b_0c_1d_1\tau_2=ABCDa_0b_0 c_0d_0a_0b_1 c_1d_1a_0b_0 c_1d_1a_0b_1 c_0d_0a_0 b_0c_1d_1\caption{\label{B}}



\end{center}
\end{figure}
This procedure can now be simulated with our axioms as follows. First, with one application of [CS] we derive 
 
where , , and  is a set of values that are interpreted as new attributes. Here  and , for , and  are interpreted as  distinct values.  is  illustrated in Fig. \ref{C} where all the distinct values are hidden.
\begin{figure}[h]
\center\scalebox{1.2}[1.1]{
ABCDa_0b_0b_1c_0c_1d_0d_1ta_0b_0 c_0d_0t'a_0b_1 c_1d_1\ida_0b_0b_1c_0c_1d_0d_1}\caption{\label{C}}
\end{figure}
Now with one application of [CR], letting , we derive  from 

Note that in this step,  is interpreted as a new attribute. Let then  be the valuation that is the identity on , and otherwise maps , , and . We notice that 
 and . Hence, we may derive with one application  of [CR] , i.e.,  from 

Then we apply [EE] and derive  from

Finally, we may apply [CT] and derive  from




\end{ex}




The following lemma shows that the above technique extends to all chasing sequences. The proof is straightforward and located in Appendix. \begin{lemma}\label{complem}
Let  be a chasing sequence of  over , where    is a finite set of egd's and tgd's over , let  be a sequence listing the attributes of , let  and , and let . Then there exists a deduction from , with  attributes from , listing the following dependencies:
\begin{enumerate}[(i)]
\item 
where , , and  consists of distinct values,
\item , for each application of   and    to , for ,

\item , for  where  .
\end{enumerate}
\end{lemma}

With the lemma, we can now show completeness.

\begin{theorem}
Let  be a finite set of egd's and tgd's over . Then  .
\end{theorem}
\begin{proof}
Assume that , and let  be a sequence listing . Then by Proposition \ref{chase} there is a chasing sequence  of  over  such that  is trivial for some . Let  be a deduction from  obtained by Lemma \ref{complem}, and let  and .

Assume first that  is an egd of the form . Then  is  where . Now, either  is , or the equality  (or its reverse) is listed in  by item (ii). Hence, using repeatedly [ES,ET] we may further on derive . Since  is derivable analogously, we therefore obtain  by [ES,ET]. Then with one application of [CT], we  derive  from  where . Note that the  of the correct form is listed in  by item (i) of Lemma \ref{complem}.


Assume then that  is a tgd of the form , and let . Then  is , and there is a valuation  on  such that  is the identity on  and .  Let  . Then  and by item (iii) of Lemma \ref{complem} we obtain that  is listed in .
For , we  have then two cases :
\begin{itemize}
\item If , then we first notice that  is  since  . Also we notice that the equality  can be derived analogously to the egd case. 
\item If , then  since by the definition of the chase  is the identity on . 

\end{itemize}
Now, letting  be the mapping  which is the identity on  and agrees with  on , we can by the previous reasoning and using repeatedly [EE]  derive  from . Finally, we can then with one application of [CT] derive  from 
 \qed

\end{proof}








\section{Typed dependencies}
Consider then the class of typed embedded dependencies. In this setting [CS] and [CT] can be replaced with rules that involve only embedded join dependencies (ejd's) and inclusion dependencies. We define ejd's over tuples of attributes as follows.
\begin{definition}\label{joinatom}
Let  be tuples of attributes listing , respectively, and let . Then  is an \emph{embedded join dependency} with the semantic rule
\begin{itemize}
\item  if and only if .
\end{itemize}

\end{definition}
The two alternative rules for the chase are now the following. We call a relation  typed if none of its values appears in two distinct columns. 










\begin{itemize}
\item[CS*] Chase Start: 
 where  is a typed relation and  is a set of \emph{new attributes}.



\item[CT*] Chase Termination: 


where tgd:  is a mapping   that is the identity on , and egd: .
\end{itemize}

The first rule is sound for typed dependencies since, for arbitrary  with ,  an instance of [CS*] is satisfied by  where  is the SPJR query

where  is the rename operator and . However, a counter example  for soundness can be easily constructed for untyped dependencies. If  and  are the relations illustrated in Fig. \ref{loppu}, then  no extension  of  to   satisfies .
\begin{figure}[h]
\begin{center}
T= ABtxy t'yx r=ABs01 \caption{\label{loppu}}



\end{center}
\end{figure}



 Soundness of [CT*] is obtained analogously to that of [CT]. Also, completeness is obtained by deriving exactly in the same way as in the general case,    (in the tgd case) or   (in the egd case)  from .  Let us then write  if  is deduced from  in the sense of Definition \ref{dedu} and using  rules [EE,CS*,CR,CT*] together with elimination and introduction of conjunction. Then we obtain the following theorem.
\begin{theorem}
Let  be a finite set of typed egd's and tgd's over . Then .
\end{theorem}












\section*{Acknowledgement}
The author was supported by grant 264917 of the Academy of Finland.
\bibliographystyle{splncs}
\bibliography{biblio}
\newpage
\section*{Appendix}

\begin{proof}(Proposition \ref{chase})
Let  be a set of egd's and tgd's over . The direction 
 is clear because it suffices to choose  such that all the relevant tuples and values remain fixed in  for . We show  and .

: Assume that there is a chasing sequence  of  over  such that   is non-trivial. We claim that  and . Let   denote . Assume first that  and assume to the contrary that  is a valuation such that  but . Then there exists  such that   for all , contradicting the assumption that no dependency is starved in the chase. 

Assume that , and assume that  is a valuation such that , and let  be such that   for all . Then there is an extension  of  to  such that  for some , where we define . Note that   for all , and  hence there exists  such that  for all .
Since ,   is the identity on , and hence we obtain that .

Finally, we show that . Analogously to the previous case we find a valuation  such that . If  is of the form , then we obtain that  is . Since  is non-trivial,  and  must  be two distinct values. Hence,   witnesses . 

Assume then that  is of the form . Then analogously   and  for some . Also note that by the construction  is the identity on . Now, if there is an extension  of  to  such that , then  is trivial. Hence  is a witness of .

: Let  be a chasing sequence of  over , where  is trivial, and let  (or ) denote  (or ). Assume that , and let  be a valuation on  to .  Using the chase construction rules and the assumption it is easy to show  inductively that for all  there is an extension  of  to  such that 
\begin{enumerate}[(i)]
\item ,
\item  .
\end{enumerate} Assume first that  is of the form , and hence . Then by the induction claim we obtain that . Assume that  is of the form , and let  be a valuation such that  and  is the identity on . Then  where, by the induction claim,  is  on . Hence, we obtain that  in both cases. This concludes the proof.\qed



\end{proof}





\begin{proof}[Lemma \ref{complem}]
W.l.o.g. we may assume that no attribute of  appears as a value in the  chasing sequence, i.e., . We show the claim by induction on . 
\paragraph{\textbf{The base case.}} First it suffices to deduce by one application of [CS] 
 where  is of the form described in (i).

\paragraph{\textbf{The inductive step.}} Assuming the claim for , we next show the claim for .  Assume first that  is obtained from  by using the egd rule for    over a valuation  on   such that  and . Then  where  is the identity everywhere except that it maps, say  to . 
By the induction assumption, it now suffices to consider  (ii) and (iii) only in the cases that associate with the construction of . 
\begin{itemize}
\item[(ii)]   The  equality  can be derived with one application of [CR], since , for all , have been deduced by the assumption.
\item[(iii)] Let , and let  be such that . If , then  has been derived by the induction assumption. Otherwise,  for some . Now  using repeatedly [EE] to  and , we obtain .
\end{itemize}


Assume then that  is obtained from  by using the tgd rule for  .  W.l.o.g. we may assume that there is only one valuation   on  with the property that  embeds  to , but no extension of  to   embeds  to . Let  be the distinct extension associated with this step, i.e.,  is an extension of  to  such that each variable in  is assigned a distinct new value greater than any value appearing in . By the induction assumption, none of these new values appear in the deduction. Hence, by the assumption we may with one application of [CR] from  deduce  where all the new values are interpreted as new attributes. Since , this concludes item (iii) and thus the proof.\qed

\end{proof}

\end{document}