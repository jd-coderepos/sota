\begin{figure*}[t!]
\begin{center}
\includegraphics[width=1.0\linewidth]{figures/reconstructions_multiresolution_1024.jpg}
\end{center}
\vspace{-5mm}
   \caption{\textbf{FFHQ reconstructions.} Reconstructions of unseen images with StyleALAE trained on FFHQ~\cite{Karras2019} at .}
\label{fig:recffhq}
\vspace{-4mm}
\end{figure*}
\begin{table}[t!]
\centering
\newcolumntype{x}{>{\centering\arraybackslash\hspace{0pt}}p{11.2mm}}
\resizebox{0.7\columnwidth}{!}{\footnotesize{
\begin{tabular}{|l@{\hspace{1mm}}|c|x|x|}
\hline
  \textbf{Method}                    & \textbf{FFHQ}     & \textbf{LSUN Bedroom}      \\\hline\hline
  StyleGAN~\cite{Karras2019}                             & 4.40              & 2.65             \\\hline PGGAN~\cite{karras2017progressive} & -                 & 8.34             \\\hline IntroVAE~\cite{Huang2018}          & -                 & 8.84             \\\hline Pioneer~\cite{heljakka2018pioneer}                            & -                 & 18.39             \\\hline Balanced Pioneer~\cite{heljakka2019towards}                   & -                 & 17.89             \\\hline StyleALAE Generation                    & 13.09             & 17.13             \\\hline StyleALAE Reconstruction                & 16.52             & 15.92   \\\hline \end{tabular}
}
}
\caption{\textbf{FID scores.} FID scores (lower is better) measured on FFHQ~\cite{Karras2019} and LSUN Bedroom~\cite{yu2015lsun}.
}
\label{tab:fid}
\vspace{-4mm}
\end{table}
\subsection{Learning style representations}

\textbf{FFHQ.} We evaluate StyleALAE with the FFHQ~\cite{Karras2019} dataset. It is very recent and consists of 70000 images of people faces aligned and cropped at resolution of . In contrast to~\cite{Karras2019}, we split FFHQ into a training set of 60000 images and a testing set of 10000 images. We do so in order to measure the reconstruction quality for which we need images that were not used during training.

We implemented our approach with PyTorch. Most of the experiments were conducted on a machine with 4 GPU Titan X, but for training the models at resolution  we used a server with 8 GPU Titan RTX. We trained StyleALAE for 147 epochs, 18 of which were spent at resolution . Starting from resolution  we grew StyleALAE up to . When growing to a new resolution level we used k training samples during the transition, and another k samples for training stabilization. Once reached the maximum resolution of , we continued training for M images. Thus, the total training time measured in images was M. In contrast, the total training time for StyleGAN~\cite{Karras2019} was M images, and M of them were used at resolution . At the same resolution we trained StyleALAE with only M images, so, 15 times less.

\begin{table}[t!]
\centering
\newcolumntype{x}{>{\centering\arraybackslash\hspace{0pt}}p{11.2mm}}
\resizebox{0.7\columnwidth}{!}{\footnotesize{
\begin{tabular}{|l@{\hspace{1mm}}c|x|x|}
\hline
  \multirow{2}{*}{\textbf{Method}}  &           & \multicolumn{2}{c|}{\textbf{Path length}} \\
                                    &           & \textbf{full}     & \textbf{end}      \\\hline\hline
  StyleGAN                          &      & 412.0             & 415.3             \\\hline StyleGAN no mixing                &      & 200.5             & 160.6             \\\hline StyleGAN                          &      & 231.5             & 182.1             \\\hline StyleALAE                              &      & 300.5             & 292.0             \\\hline StyleALAE                              &      & \textbf{134.5}    & \textbf{103.4}    \\\hline \end{tabular}
}
}
\caption{\textbf{PPL.} Perceptual path lengths on FFHQ measured in the  and the  spaces (lower is better).
}
\label{tab:disentangle}
\vspace{-4mm}
\end{table}
\begin{table}[b!]
\vspace{-4mm}
  \centering
  \newcolumntype{x}{>{\centering\arraybackslash\hspace{0pt}}p{11.2mm}}
\resizebox{0.7\columnwidth}{!}{\footnotesize{
\begin{tabular}{|l@{\hspace{1mm}}|c|x|x|x|x|}
\hline
    & \textbf{FID}  & \textbf{PPL full}  \\\hline\hline
  PGGAN~\cite{karras2017progressive}  &  &   \\\hline GLOW~\cite{Kingma2018}  &  &   \\\hline PIONEER~\cite{heljakka2018pioneer}  &  & 155.2 \\\hline Balanced PIONEER~\cite{heljakka2019towards} &   &  \\\hline StyleALAE (ours) & 19.21 &  \\\hline \end{tabular}}
}
  \caption{Comparison of FID and PPL scores for CelebA-HQ images at  (lower is better). FID is based on 50,000 generated samples compared to training samples.}
    \label{t-celeba}
\end{table}


Table~\ref{tab:fid} reports the FID score~\cite{heusel2017gans} for generations and reconstructions. Source images for reconstructions are from the test set and were not used during training. The scores of StyleALAE are higher, and we regard the large training time difference between StyleALAE and StyleGAN (M vs M) as the likely cause of the  discrepancy.
\begin{figure}[t!]
\begin{center}
\includegraphics[width=1.0\linewidth]{figures/uncurated-ffhq.jpg}
\end{center}
\vspace{-4mm}
\caption{\textbf{FFHQ generations.} Generations with StyleALAE trained on FFHQ~\cite{Karras2019} at .}
\vspace{-1mm}
\label{fig:genffhq}
\end{figure}

Table~\ref{tab:disentangle} reports the perceptual path length (PPL)~\cite{Karras2019} of SyleALAE. This is a measurement of the degree of disentanglement of representations. We compute the values for representations in the  and the  space, where StyleALAE is trained with style mixing in both cases. The StyleGAN score measured in  corresponds to a traditional network, and in  for a style-based one. We see that the PPL drops from  to , indicating that   is perceptually more linear than , thus less entangled. Also, note that for our models the PPL is lower, despite the higher FID scores.

Figure~\ref{fig:genffhq} shows a random collection of generations obtained from
StyleALAE. Figure~\ref{fig:recffhq} instead shows a collection of reconstructions. In Figure~\ref{fig:stylemix} instead, we repeat the style mixing experiment in~\cite{Karras2019}, but with real images as sources and destinations for style combinations. We note that the original images are faces of celebrities that we downloaded from the internet. Therefore, they are not part of FFHQ, and come from a different distribution. Indeed, FFHQ is made of face images obtained from Flickr.com depicting non-celebrity people. Often the faces do not wear any makeup, neither have the images been altered (e.g., with Photoshop). Moreover, the imaging conditions of the FFHQ acquisitions are very different from typical photoshoot stages, where professional equipment is used. Despite this change of image statistics, we observe that StyleALAE works effectively on both reconstruction and mixing.


\textbf{LSUN.} We evaluated StyleALAE with LSUN Bedroom~\cite{yu2015lsun}.  Figure~\ref{fig:genbed} shows generations and reconstructions from unseen images during training. Table~\ref{tab:fid} reports the FID scores on the generations and the reconstructions.
\begin{figure}[t!]
\begin{center}




\includegraphics[width=\linewidth]{figures/uncurated-bedroom.jpg}
\includegraphics[width=\linewidth]{figures/reconstructions_multiresolution_256_bedroom.jpg}

\end{center}
\vspace{-5mm}
\caption{\textbf{LSUN generations and reconstructions.} Generations (first row), and reconstructions using StyleALAE trained on LSUN Bedroom~\cite{yu2015lsun} at resolution .}
\label{fig:genbed}
\vspace{-4mm}
\end{figure}










\textbf{CelebA-HQ.} CelebA-HQ~\cite{karras2017progressive} is an improved subset of CelebA~\cite{liu2015deep} consisting of 30000 images at resolution . We follow~\cite{heljakka2018pioneer, heljakka2019towards, Kingma2018, karras2017progressive} and use CelebA-HQ downscaled to  with training/testing split of 27000/3000. Table~\ref{t-celeba} reports the FID and PPL scores, and Figure~\ref{f-ablation} compares StyleALE reconstructions of unseen faces with two other approaches.

\begin{figure}[t]
\begin{tabular}{c}
\includegraphics[width=0.23\linewidth]{figures/pioneer/120_orig.jpg}
\includegraphics[width=0.23\linewidth]{figures/pioneer/9792_orig.jpg}
\includegraphics[width=0.23\linewidth]{figures/pioneer/9794_orig.jpg}
\includegraphics[width=0.23\linewidth]{figures/pioneer/9795_orig.jpg}\\
\midrule
\includegraphics[width=0.23\linewidth]{figures/pioneer/120_alae.jpg}
\includegraphics[width=0.23\linewidth]{figures/pioneer/9792_alae.jpg}
\includegraphics[width=0.23\linewidth]{figures/pioneer/9794_alae.jpg}
\includegraphics[width=0.23\linewidth]{figures/pioneer/9795_alae.jpg}\\
\includegraphics[width=0.23\linewidth]{figures/pioneer/120_pine.jpg}
\includegraphics[width=0.23\linewidth]{figures/pioneer/9792_pine.jpg}
\includegraphics[width=0.23\linewidth]{figures/pioneer/9794_pine.jpg}
\includegraphics[width=0.23\linewidth]{figures/pioneer/9795_pine.jpg}\\
\includegraphics[width=0.23\linewidth]{figures/pioneer/120_pine_old.jpg}
\includegraphics[width=0.23\linewidth]{figures/pioneer/9792_pine_old.jpg}
\includegraphics[width=0.23\linewidth]{figures/pioneer/9794_pine_old.jpg}
\includegraphics[width=0.23\linewidth]{figures/pioneer/9795_pine_old.jpg}
\end{tabular}
\vspace{-3mm}
\caption{\textbf{CelebA-HQ reconstructions.} CelebA-HQ reconstructions of unseen samples at resolution . Top row: real images. Second row: StyleALAE. Third row: Balanced PIONEER~\cite{heljakka2019towards}. Last row: PIONEER~\cite{heljakka2018pioneer}. StyleALAE reconstructions look sharper and less distorted.}
\vspace{-4mm}
\label{f-ablation}
\end{figure}

  









\newcommand{\h}{0mm}
\newcommand{\hh}{0mm}
\newcommand{\hhh}{0mm}
\newcommand{\vv}{0mm}
\newcommand{\vvv}{0mm}
\newcommand{\ext}{jpg}  \newcommand{\extt}{jpg} \renewcommand{\h}{27.4mm}
\renewcommand{\hh}{2.5mm}
\renewcommand{\vv}{\vspace*{-0.35mm}}
\begin{figure*}[t]
\centering
\begin{minipage}[c]{0.972\linewidth}
\makebox[\hh][c]{}\hfill \makebox[\h][c]{\textbf{\normalsize{\ Destination set}}}\hfill \rotatebox[origin=l]{90}{\makebox[0mm][l]{\hspace*{0.05\linewidth}\textbf{\normalsize{\raisebox{.5mm}[0mm][0mm]{Source set}}}}}\hfill \newcommand{\varA}{var639}\includegraphics[height=\h]{figures/source_0.jpg}\newcommand{\varB}{var701}\includegraphics[height=\h]{figures/source_1.jpg}\newcommand{\varC}{var687}\includegraphics[height=\h]{figures/source_2.jpg}\newcommand{\varD}{var615}\includegraphics[height=\h]{figures/source_3.jpg}\newcommand{\varE}{var2268}\includegraphics[height=\h]{figures/source_4.jpg}\vspace*{-3.5mm}\\
\begin{tikzpicture}\draw (0,0) -- (\linewidth,0);\end{tikzpicture}\vspace*{-0.5mm}\\\makebox[\hh]{\rotatebox[origin=l]{90}{\makebox[\h][c]{\hspace{-0.332\linewidth}\normalsize{Coarse styles from Source set}}}}\hspace{0.5mm}\newcommand{\seed}{seed888}
\includegraphics[height=\h]{figures/dst_coarse_0.jpg}\hfill \includegraphics[height=\h]{figures/rec_coarse_0_0.jpg}\includegraphics[height=\h]{figures/rec_coarse_0_1.jpg}\includegraphics[height=\h]{figures/rec_coarse_0_2.jpg}\includegraphics[height=\h]{figures/rec_coarse_0_3.jpg}\includegraphics[height=\h]{figures/rec_coarse_0_4.jpg}\vv\\
\makebox[\hh][c]{}\hspace{0.5mm}\renewcommand{\seed}{seed829}
\includegraphics[height=\h]{figures/dst_coarse_1.jpg}\hfill \includegraphics[height=\h]{figures/rec_coarse_1_0.jpg}\includegraphics[height=\h]{figures/rec_coarse_1_1.jpg}\includegraphics[height=\h]{figures/rec_coarse_1_2.jpg}\includegraphics[height=\h]{figures/rec_coarse_1_3.jpg}\includegraphics[height=\h]{figures/rec_coarse_1_4.jpg}\vv\\
\makebox[\hh][c]{}\hspace{0.5mm}\renewcommand{\seed}{seed1898}
\includegraphics[height=\h]{figures/dst_coarse_2.jpg}\hfill \includegraphics[height=\h]{figures/rec_coarse_2_0.jpg}\includegraphics[height=\h]{figures/rec_coarse_2_1.jpg}\includegraphics[height=\h]{figures/rec_coarse_2_2.jpg}\includegraphics[height=\h]{figures/rec_coarse_2_3.jpg}\includegraphics[height=\h]{figures/rec_coarse_2_4.jpg}\vspace*{-3.5mm}\\
\begin{tikzpicture}\draw (0,0) -- (\linewidth,0);\end{tikzpicture}\vspace*{-0.5mm}\\\makebox[\hh]{\rotatebox[origin=l]{90}{\makebox[\h][c]{\hspace{-\h}\normalsize{Middle styles from Source set}}}}\hspace{0.5mm}\renewcommand{\seed}{seed1733}
\includegraphics[height=\h]{figures/dst_coarse_3.jpg}\hfill \includegraphics[height=\h]{figures/rec_coarse_3_0.jpg}\includegraphics[height=\h]{figures/rec_coarse_3_1.jpg}\includegraphics[height=\h]{figures/rec_coarse_3_2.jpg}\includegraphics[height=\h]{figures/rec_coarse_3_3.jpg}\includegraphics[height=\h]{figures/rec_coarse_3_4.jpg}\vv\\
\makebox[\hh][c]{}\hspace{0.5mm}\renewcommand{\seed}{seed1614}
\includegraphics[height=\h]{figures/dst_coarse_4.jpg}\hfill \includegraphics[height=\h]{figures/rec_coarse_4_0.jpg}\includegraphics[height=\h]{figures/rec_coarse_4_1.jpg}\includegraphics[height=\h]{figures/rec_coarse_4_2.jpg}\includegraphics[height=\h]{figures/rec_coarse_4_3.jpg}\includegraphics[height=\h]{figures/rec_coarse_4_4.jpg}\vspace*{-3.5mm}\\
\begin{tikzpicture}\draw (0,0) -- (\linewidth,0);\end{tikzpicture}\vspace*{-0.5mm}\\\makebox[\hh]{\rotatebox[origin=l]{90}{\makebox[\h][c]{\hspace{-1mm}\normalsize{Fine from Source}}}}\hspace{0.5mm}\renewcommand{\seed}{seed845}
\includegraphics[height=\h]{figures/dst_coarse_5.jpg}\hfill \raisebox{0mm}[0mm][0mm]{\makebox[0mm]{\hspace*{-1.5mm}\begin{tikzpicture}\draw (0,0mm) -- (0,-196.2mm);\end{tikzpicture}}}\includegraphics[height=\h]{figures/rec_coarse_5_0.jpg}\includegraphics[height=\h]{figures/rec_coarse_5_1.jpg}\includegraphics[height=\h]{figures/rec_coarse_5_2.jpg}\includegraphics[height=\h]{figures/rec_coarse_5_3.jpg}\includegraphics[height=\h]{figures/rec_coarse_5_4.jpg}\\
\end{minipage}
\caption{\label{fig:stylemix}Two sets of real images were picked to form the Source set and the Destination set. The rest of the images were generated by copying specified subset of styles from the Source set into the Destination set. This experiment repeats the one from~\cite{Karras2019}, but with real images. Copying the coarse styles brings high-level aspects such as pose, general hair style, and face shape from Source set, while all colors (eyes, hair, lighting) and finer facial features resemble the Destination set. Instead, if we copy middle styles from the Source set, we inherit smaller scale facial features like hair style, eyes open/closed from Source, while the pose, and general face shape from Destination are preserved. Finally, copying the fine styles from the Source set brings mainly the color scheme and microstructure.
\vspace*{-1mm}
}
\end{figure*}










