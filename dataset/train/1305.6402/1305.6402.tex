\documentclass[5p,authoryear,times]{elsarticle}

\usepackage{IEEEtrantools}
\usepackage[cmex10]{amsmath}
\usepackage{amsthm}
\usepackage{amssymb}
\usepackage[tight,footnotesize]{subfigure}
\usepackage{amsfonts}
\usepackage{epstopdf}

\hyphenpenalty 9000
\exhyphenpenalty 5000

\newtheorem{thm}{Theorem}
\newtheorem{lem}[thm]{Lemma}
\newtheorem{proposition}[thm]{Proposition}
\newtheorem{corollary}[thm]{Corollary}
\newtheorem{assumption}{Assumption}
\newtheorem{definition}{Definition}
\newtheorem{rmk}{Remark}
\newtheorem{notation}{Notation}

\newcommand{\eg}{e.g.}

\raggedbottom
\begin{document}

\title{From Parametric Model-based Optimization\\to robust PID Gain Scheduling}
\author[rvt]{Minh H.T. Nguyen \corref{cor1}}
\ead{tuanminh@nus.edu.sg}

\author[rvt]{K.K. Tan}
\ead{kktan@nus.edu.sg}

\cortext[cor1]{Corresponding author}
\address[rvt]{National University of Singapore, Department of Electrical and Computer Engineering,\\3 Engineering Drive 3, Singapore 117576}

\begin{abstract}
In chemical process applications, model predictive control (MPC) effectively deals with input and state constraints during transient operations. However, industrial PID controllers directly manipulates the actuators, so they play the key role in small perturbation robustness. This paper considers the problem of augmenting the commonplace PID with the constraint handling and optimization functionalities of MPC. First, we review the MPC framework, which employs a linear feedback gain in its unconstrained region. This linear gain can be any preexisting multi-loop PID design, or based on the two stabilizing PI/PID designs for multivariable systems proposed in the paper. The resulting controller is a feedforward PID mapping, a straightforward form without the need of tuning PID to fit an optimal input. The parametrized solution of MPC under constraints further leverages a familiar PID gain scheduling structure. Steady state robustness is achieved along with the PID design so that additional robustness analysis is avoided.


\end{abstract}

\begin{keyword}
Robust tracking \sep constrained linear systems \sep model predictive control \sep PID gain scheduling.
\end{keyword}
\maketitle

\section{Introduction}\label{intro}

Multilevel control attracts intensive research as a systematic tool for control of real plants with respect to high-level target while adhering to the local constraints \citep{Tat08Advanced}. The upper levels are usually concerned with plant-wide steady state objectives with low rate sampling. The lower levels address fast dynamic control. There is a mature trend of applying advanced optimization packages to fill the gap between these two layers. Well-known industrial examples such as AspenOne and RHMPC use MPC as the core optimizer to deal with constraints \citep{Qin03survey,Fro06Model}. MPC is a constraint-handling optimization method where the core idea is based on the receding horizon control. At each sampling time, the current plant output/state is measured, and an optimal input is derived to minimize a performance index subject to state and input constraints. This desired inputs are sent to PID controllers to directly manipulate the actuators. These PIDs must be tuned to minimize the mismatch with the updated optimal input at each sampling step. The first objective of the paper aims to bypass this two-phase complication through direct optimization of the PID gains.

Currently, there are two approaches of MPC, using either \emph{online} implementation \citep{May00Constrained} for slow processes or \emph{offline} implementation \citep{Bem02explicit} for fast processes. The former control approach solves in real time an optimization problem, thus it is more flexible to system design changes. The latter approach solves the same problem offline for all feasible states, and obtains the optimal control law in real time by searching the current state over feasible regions. This scheme, named parametric MPC, can effectively facilitate a PID gain scheduling implementation. The resulting PID controller will deal with constraints by changing gains upon the transition of active constraint regions, not at each time step. This is the second and main objective: to develop a practical implementation of parametric MPC.

The PID realization of MPC can be achieved with its robustness property intact. In fact, a great number of research methods have carefully addressed the robustness of MPC for perturbations both along the trajectory (robust performance) and at steady state (robust stability). Polytopic uncertainty model is discussed in \citet{Gri03} with LMI and in \citet{Bem03Min,Nag04Open} where min-max solutions are formed; bounded disturbances addressed by tube-based MPC is proposed in \citet{Alv08,Mar11Stochastic}. The tradeoff lies in the complexity of the solutions. In this note, we are keen on observing the robust stability provided by the simple PID form of the proposed solution.

In the literature, many finite-horizon optimal PID designs for constrained multivariable systems have been attempted to deliver a systematic PID tuning. In \citet{Mor03}, the velocity form of PID prohibited the variable gain structure, thus a fixed PID gain must be used across the prediction horizon. As shown in \citet{Camacho03Model,Aro08,Sat12} the GPC-based PID results apply to the plants approximated by a first or second order model, thus limiting their applications to multivariable plants. The solution in \citet{Di10Model} partially solves the problem, but the two controllers MPC and PID must operate in parallel. A flexible framework for optimal PIDs is still under ongoing research.

Collectively through the two mentioned objectives, this paper seeks to improve the MPC-based PID scheme to further close the gap between MPC optimization and PID controllers. In Section II, we formulate the tracking problem and analyze the controllability and observability of the augmented system. In Section III, we describes the MPC formula and shows that either a new or existing multi-loop unconstrained PID designs can be adopted into the framework. For convenience, two methods are provided to calculate the PI/PID gains at the operating point so that the closed loop system is stable. The first method applies LQR on the PI state while the latter leads to linear matrix inequalities (LMI) with the size proportional to the number of tracked outputs. Section IV applies this PID design on the piecewise affine (PWA) solution of MPC, which suggests a distributed PID gain scheduling framework to deal with constraints. Fig. \ref{fig1} shows the involved levels within the plantwide structure.

\subsection*{Notation}
The operators  are the integral and differential terms. The notation  denotes positive definiteness. ,  and  denote the state, estimated state and state error;  and  denote the inputs for tracking and regulating problems, respectively. Subscript  indicates matrix/vector component and  the prediction step, superscript  is the critical region index.  is an identity matrix of order . 

\begin{figure}[t]
\centering
\includegraphics[width=3.2in]{ctrlDiag1-eps-converted-to.pdf}
\caption{Optimization control with multi-layers.}
\label{fig1}
\end{figure}
\section{Preliminaries}\label{preli}
To obtain a linear feedback involving proportional-integral-differential gains, it is necessary to form a system state that contains the corresponding variables. Provided that is the case, an optimal linear feedback gain is also an optimal PID gain. This section introduces the augmented PI/PID-state systems and covers the analysis of their controllability and observability. 

\subsection{Plant Model}
Consider a linear time-invariant system 
\begin{IEEEeqnarray}{rCl}
x(k+1)&=& Ax(k) + Bu(k)\nonumber \\
v(k)&=& C_vx(k)\nonumber \\
y(k) &=& Cx(k).\label{mdl}
\end{IEEEeqnarray}
subject to the constraint

In \eqref{mdl}, ,  and  are the state, input, tracked output and measured output. Assume  is controllable and  is observable;  having full row rank;  are appropriate matrices defining the state and input constraints. 


The plant model \eqref{mdl} is augmented with an integral of the tracked output  to ensure zero offset during the steady state. The following PI-state model is used
\begin{IEEEeqnarray}{rCl}
\begin{bmatrix}x(k+1)\\ \sum{v(k+1)}\end{bmatrix}&=&
\begin{bmatrix}A& 0\\C_v& I_q\end{bmatrix}
\begin{bmatrix}x(k)\\ \sum{v(k)}\end{bmatrix}+
\begin{bmatrix}B\\ 0\end{bmatrix}u(k)\nonumber \\
y(k) &=& Cx(k).\label{augmdl}
\end{IEEEeqnarray}
In special cases,  requires a full-state tracking while  expects only output tracking.

\begin{proposition}
The PI-augmented system \eqref{augmdl} is detectable. Furthermore, it is controllable if and only if (A,B) is controllable and

\end{proposition}

\begin{proof}
The Hautus condition for observability is


The condition \eqref{hauObsv} does not hold only at , but the unobservable integrating state can be controlled to decay to a constant so the system is detectable.

Similarly, \eqref{hauCtrl} follows directly from Hautus controllability where only the case of  is to check.
\end{proof}

By addition of the differential term, the PID-state system presents as
\begin{IEEEeqnarray}{rCl}
\begin{bmatrix}x(k+1)\\ \sum{v(k+1)}\\ \Delta v(k+1)\end{bmatrix}&=&
\begin{bmatrix}A& 0& 0\\ C_v& I_q& 0\\ C_v(A-I_n)& 0& 0\end{bmatrix}
\begin{bmatrix}x(k)\\ \sum{v(k)}\\ \Delta v(k)\end{bmatrix}+\begin{bmatrix}B\\ 0\\ C_vB\end{bmatrix}u(k)\nonumber \\
y(k) &=& Cx(k).\label{augmdl2}
\end{IEEEeqnarray}
This PID-augmented system is detectable and stabilizable. The proof is similar to Proposition \ref{hauCtrl}.

\begin{rmk}The number of tracked variables is presumed less than or equal to the number of manipulated variables ( for PI case and  for PID case); the other case was well treated in \citet{Mae09Linear}.
\end{rmk}

The objective is to design a finite-horizon optimal control based on the augmented system \eqref{augmdl} or \eqref{augmdl2} so that  tracks a piece-wise constant reference.

\subsection{Observer Design}
From the system detectability, an observer can make use of the system \eqref{mdl} to estimate the current state, and simply calculate the integral and differential state through a sum of the estimated  and its difference.

Since  is observable, the observer is designed as
\begin{IEEEeqnarray}{rCl}
\hat{x}(k)&=&A\hat{x}(k-1)+Bu(k-1)\nonumber\\
&&+L_x[-y(k-1)+C\hat{x}(k-1)]\nonumber\\
\sum{\hat{v}(k)}&=& \sum{\hat{v}(k-1)}+C_v\hat{x}(k-1)\nonumber\\
&&+C_vL_x[-y(k-1)+C\hat{x}(k-1)]\nonumber\\
\Delta \hat{v}(k) &=& C_v(\hat{x}(k)-\hat{x}(k-1))
\end{IEEEeqnarray}
It is only necessary to design the observer gain  as  so that . This automatically leads to  being stable. The integral estimation error is not required to decay to zero, but a steady state because  means .

\section{Controller Design}\label{MPCPID}
\subsection{MPC tracking structure}\label{MPC}
This section will outline the general MPC controller design for a state space model that results in PI/PID control implementation fulfilling the constraints.

Consider the linear system with constraints . Define the operating points  and the deviation variables
\begin{IEEEeqnarray}{rCl}
\tilde{z}(k)&=&z_s-z(k)\nonumber \\
\tilde{u}(k)&=&u_s-u(k),
\label{dev}
\end{IEEEeqnarray}
\begin{IEEEeqnarray}{rCl}
\text{then\qquad}\tilde{z}(k+1)&=&A_m\tilde{z}(k)+B_m\tilde{u}(k)\nonumber\\
\tilde{y}(k) &=& C_m\tilde{z}(k).
\label{devmdl}
\end{IEEEeqnarray}

The finite-horizon quadratic optimal control problem is posed as 
\begin{IEEEeqnarray}{lCl}
V_N^o(\tilde{z}_0,\tilde{U}&&)=\underset{\tilde{U}}{\operatorname{min.}}\,\tilde{z}_N^TP\tilde{z}_N\nonumber\\
&&\qquad+\sum_{k=0}^{N-1}({\tilde{z}_k}^TC_m^TQC_m\tilde{z}_k+\tilde{u}_k^TR\tilde{u}_k)\label{MPClaw}\\
subj.\,to \, && \tilde{z}_k\in X, \tilde{u}_k\in U \quad \forall k\in {0,...,N-1},\nonumber\\
&&\tilde{z}_0\in X_0,\,\tilde{z}_N\in X_f,\nonumber \\
&& \tilde{z}_{k+1}= A_m\tilde{z}_k+ B_m\tilde{u}_k,\, \nonumber
\end{IEEEeqnarray}
where . Here  are the weighting matrices,  is detectable;  is the terminal penalty matrix.  are the initial feasible set and the terminal constraint set. Note that  are translated constraints from \eqref{cons} through the transformation in \eqref{dev}. By the receding horizon policy, only  is applied to the plant.

\begin{assumption}
The state and input constraints are not active for . Also,  contains the origin.
\end{assumption}

The optimizer  stabilizes \eqref{devmdl} if the value function  corresponds to a local Lyapunov function  within the terminal set . In addition, the decay rate of that Lyapunov function must be larger than the stage cost \citep{May00Constrained}. Under this setup, any admissible  is steered to a level set of  (and so ) within N steps, after which convergence and stability of the origin follows. In other words,  is stable at  for .

Therefore, given the state and input weighting matrices , one would want to first compute an unconstrained stabilizing feedback  and its Lyapunov function  that satisfy
\begin{IEEEeqnarray}{rCl}
V_f(\tilde{z})&=& \tilde{z}^TP\tilde{z}\geq 0,\nonumber\\
\Delta V_f(\tilde{z})&=& \tilde{z}^TA_K^TPA_K\tilde{z}- \tilde{z}^TP\tilde{z}\nonumber \\
&\leq &-\tilde{z}^TQ\tilde{z}-\tilde{z}^TK^TRK\tilde{z},\, \forall \tilde{z}\in X_f,\label{conMPC}
\end{IEEEeqnarray}
where . The other ingredients of MPC formula are then determined as follows.
\begin{itemize}
	\item  is the maximal positively invariant polyhedron of  with respect to . As commented in \citet{Rawlings09Model}, if  is ellipsoidal, the problem is no longer a quadratic program but a convex program but can be solved with available softwares.
	\item  is the N-step stabilizable set of the system \eqref{MPClaw} with respect to .  is a trade-off value between the complexity of MPC problem and a larger set  (i.e. larger initial error ).
	\item  is chosen as the solution of the equality in \eqref{conMPC}, the unique positive-definite solution of 
a discrete Lyapunov equation once  is known \citep{Gri05Stabilizing}.
\end{itemize}
 can be calculated analytically using the method detailed in \citet{Bla99Set,Ale06}. 

A popular choice for  is obtained from the LQR gain with weighting matrices  \citep{Chm96constrained, Sco98Constrained}. However, in this note, it is left as a general stabilizing gain  that will be computed in the next section.

\subsection{Computation of Stabilizing PI/PID}\label{PIDsec}
This session describes a method to compute an unconstrained feedback gain  that is used to reconstruct the MPC formula \eqref{MPClaw}. It is because this gain would result in PI/PID controllers, as shown in the following theorem. For the general case, let .

\begin{thm}\label{theo1}
A control law  implements PID control on the system state  which ensures robust tracking for .
\end{thm}
\begin{proof}
Because  is the augmented state error, the control law is written as
\begin{IEEEeqnarray}{rCl}
\tilde{u}(k)&=& K\tilde{z}(k)\nonumber \\
&=& K_1\tilde{x}(k) + K_2\sum{\tilde{v}(k)}+ K_3\Delta\tilde{v}(k)\nonumber\\
&=& K_1\tilde{x}(k)+K_2C_v\sum{\tilde{x}(k)}+K_3C_v\Delta\tilde{x}(k).\IEEEeqnarraynumspace
\label{PID}
\end{IEEEeqnarray}
Since , there are  P controllers and  PID controllers. In particular, PID control is applied to the state variables which influence the tracked output , so they are robust against disturbances.
\end{proof}

Let  be the augmented model and state of \eqref{augmdl}.


\subsubsection{PI Controller ()}
For this case, it is essential to obtain the feedback gain for . The PI control can be formulated by applying LQR to the error model of \eqref{augmdl}
to produce a PI control law . From here simply take  and use \eqref{conMPC} to apply the MPC formula.

\subsubsection{PID Controller}
To get a non-trivial differential gain , one can treat the differential term as an output feedback of the system \eqref{augmdl}. Define  where . Then
\begin{IEEEeqnarray}{rCl}
\bar{x}(k+1)&=& \bar{A}\bar{x}(k)+\bar{B}u(k)\nonumber \\
\phi(k) &=& \bar{C}\bar{x}(k)=\begin{bmatrix}I_n&0\\ 0& I_q\\ C_v(A-I_n)& 0\end{bmatrix}\bar{x}(k).
\label{sofmdl}
\end{IEEEeqnarray}

Design of static output feedback (SOF)  for the discrete time system above has been investigated in \citet{Gar03Robust,Bar05Static,Don07Static,He08Output} which use LMI conditions. There exists more outputs than inputs in this case, so we present a simple solution to determine  in Theorem \ref{SOF} \citep{Bar05Static}. In that work, the solution can be extended to the  design, but the detail is omitted here for simplicity (refer to Remark \ref{rmk3}).





\begin{thm}\label{SOF}
System \eqref{sofmdl} is stabilizable by a static output feedback if there exist a symmetric positive definite matrix  and a positive scalar  such that

is satisfied. Furthermore, the SOF gain  can be obtained by solving

\end{thm}

Conditions \eqref{LMI1}, \eqref{LMI2} can be solved as two LMI problems. Once we have found a stabilizing output feedback  or equivalently , it can be rewritten in the PID form \cite{Zhe02design} as
\begin{IEEEeqnarray}{rCl}
\tilde{u}(k)&=&F_1\tilde{x}^T+F_2\sum{\tilde{v}^T}+F_3\Delta \tilde{v}+F_3C_vB\tilde{u}(k)\\
\text{so\ }\tilde{u}(k)&=&(I_n+F_3C_vB)^{-1}[F_1\tilde{x}(k)^T+F_2\sum{\tilde{v}(k)}+F_3\Delta \tilde{v}(k)]\nonumber\\
&=&K_{PID}\tilde{z}(k),
\end{IEEEeqnarray}
where . The invertibility of matrix  is a necessary condition to render .

The MPC formula takes ,
\begin{IEEEeqnarray}{l}\label{Am}
A_m=\begin{bmatrix}A& 0& 0\\ C_v& I_n& 0\\ C_v(A-I_n)& 0& 0\end{bmatrix}, B_m=\begin{bmatrix}B\\ 0\\C_vB\end{bmatrix}u(k)\IEEEeqnarraynumspace
\end{IEEEeqnarray}
and  to apply into \eqref{conMPC}.

\begin{rmk} \label{rmk2} Applying LQR directly to PID state for system \eqref{Am} will not result in a PID controller. In fact, since  is no longer full rank, the optimal input  depends only on the first two components of , so it is not a full PID but a PI gain. However, we realize that increasing the weight on  of  does reduce the overshoot and enhance the disturbance response of .
\end{rmk}

\begin{rmk} \label{rmk3} The PID design for multivariable systems used in this paper is not unique. It is possible to use other techniques such as \citet{Dic09parameter,Soy03Fast,Tos09Robust} to derive a robust PID gain before applying it into MPC.
\end{rmk}


\section{From parametric MPC to PID gain scheduling controllers}\label{fastMPC}

The result from Section III holds when it is applied to either an \emph{online} or \emph{offline} MPC formulation. In this section, we particularly use parametric MPC (offline) to demonstrate the PID gain scheduling realization.

\subsection{Parametric MPC}
Observe that the problem \eqref{MPClaw} minimizes a convex value function subject to a convex constraint set. We have the following definition
\begin{definition}[Critical Region]
A critical region is defined as the set of parameters  for which the same set of constraints is active at the optimum .
\end{definition}

In other words, if the constraints in \eqref{MPClaw} is presented as  and  is an associated set of row index,
\begin{IEEEeqnarray}{l}
CR_A =\{\tilde{z}\in X_0\,|\,G_i\bar{U}^0=S_i\tilde{z}+W_i\text{ for all } i\in A\}\IEEEeqnarraynumspace
\label{eq:}
\end{IEEEeqnarray}

In \citet{Bao02,Ton03algorithm}, it is shown that these critical regions are a finite number of closed, non-overlapped polyhedra and they covers completely . Since , the same properties apply for . Theorem \ref{theo2} states the key result (see \citet{Bem02explicit}).

\begin{thm}[Parametric solution of MPC]
\label{theo2}
The optimal control law , obtained as a solution of \eqref{MPClaw} is continuous and piecewise affine on the polyhedra

where the polyhedral sets  are a partition of the feasible set .
\end{thm}

\subsection*{Tracking for piecewise constant setpoint}
Recall the admissible set  the MPC controller can stabilize depends on the linearized model  and control horizon . Tracking of a new setpoint can be done by increasing  based on the new model so that a jump in reference  is feasible within  steps. 

\begin{figure}[!t]
\centering
\includegraphics[width=3.0in]{feas-eps-converted-to.pdf}
\caption{Feasibility check of new setpoint .}
\label{feas}
\end{figure}

In the case of fixed , Corollary \ref{reffsb} states the necessary and sufficient condition for a new feasible setpoint

\begin{corollary}\label{reffsb}
With a fixed-horizon proposed controller, a change in setpoint  is feasible if and only if .
\end{corollary}
\begin{proof}
The proof can be inferred from Fig \ref{feas}. If  is out of the maximal admissible region  constructed around , it is impossible to drive the current error  to zero with the existing controller.
\end{proof}

Corollary \ref{reffsb} suggests a way to detect if a new setpoint is feasible so that the local optimization for steady state target can recalculate  early before the infeasibility happens. One can use a single model and treat the model mismatch at a different operating point as disturbance, but generally  still needs to be rebuilt through \eqref{dev} because the constraints change with setpoint relocation.



\subsection{PID Gain Scheduling Design}
The optimal input of MPC is applied for regions outside . When  reaches , the system will be stabilized by the pure gain . Therefore, one practical way to design PID for constrained systems is designing a PID gain for its unconstrained region, which has been accomplished in Section \ref{MPCPID}, and applying these settings on the MPC formulation \eqref{MPClaw}.

\begin{figure}[t]
\centering
\includegraphics[width=3.6in]{ctrlDiag3_1-eps-converted-to.pdf}
\caption{Proposed PID gain scheduling structure.}
\label{ctrlDiag3}
\end{figure}

Fig. \ref{ctrlDiag3} shows a series of PIDs plus a single feedforward vector where the controller gains are determined from \eqref{mulctrl}. Each of the PIDs is fully flexible (might contain only P or PI components) and have its own look-up gain scheduling for different partition indexes. At each time step, the proposed scheme would look for the region in which the augmented error  lies in. This search engine would broadcast the region index  to the PID network. The feedforward term associated with region  is added to compensate the active constraints. Non-zero tracking accounts for the addition of steady state input and recovers the input delivered to the plant.

\begin{rmk}
As seen from Fig. \ref{ctrlDiag3}, the PID network consists of one-to-one mappings between each state variable error of the original state  and an input. This fact results from equation \eqref{PID}.
\end{rmk}



\section{Example}



\begin{figure}[t]
\centering
\subfigure{
\label{x1}
\includegraphics[width=\columnwidth]{X_1-eps-converted-to.pdf}}
\subfigure{
\label{u1}
\includegraphics[width=\columnwidth]{X_2-eps-converted-to.pdf}}
\subfigure{
\label{x2}
\includegraphics[width=\columnwidth]{U_1-eps-converted-to.pdf}}
\subfigure{
\label{u2}
\includegraphics[width=\columnwidth]{U_2-eps-converted-to.pdf}}
\caption{State responses and control inputs under disturbances at transient and steady-state.}
\label{stateinput}
\end{figure}

The proposed control design was illustrated in the following example, generalized from \citet{Bem02explicit} with two inputs. Consider a continuous stirred-tank reactor model
\begin{IEEEeqnarray}{rCl}
A&=&\begin{bmatrix}0.7326& -0.0861\\0.1722& 0.9909\end{bmatrix}, B=\begin{bmatrix}0.0609& 0\\ 0& 0.0064\end{bmatrix},\nonumber\\
C&=&\begin{bmatrix}1& 0\\ 0& 1\end{bmatrix},\, X=\left\{x\in \mathbb{R}^2|\begin{bmatrix}-0.5\\ -0.5\end{bmatrix}\leq x \leq \begin{bmatrix}1.5\\ 2.5\end{bmatrix}\right\},\nonumber\\
U&=&\left\{u\in \mathbb{R} | \begin{bmatrix}-2\\ -2\end{bmatrix}\leq u \leq \begin{bmatrix}2\\ 2\end{bmatrix} \right\}.
\end{IEEEeqnarray}The task was to track the level 1  with the reference . To observe the robustness of tested controllers, the disturbances  (impulse),  (additive) within an active constrained region at  and   (additive) at steady state  were introduced.

The three following controllers were compared: simple parametric MPC (I), the whole state tracking with full PI (II) and -tracking with partial PID (III). The prediction horizon (also control horizon in this case) is chosen as . 

Tuning weighting matrices for PID control had been discussed in \citet{Ngu11Enhanced}. For PI,  and , ; for PID , , . MATLAB LMI solver was used to obtain the unconstrained PID gain for case III, and Multiparametric toolbox \citep{mpt} was applied to obtain the gains under critical regions.

The unconstrained gain  in the three cases were 
\begin{IEEEeqnarray}{lCl}
K_I &=& \begin{bmatrix}4.501& 3.792\\0.711& 2.160\end{bmatrix},\nonumber\\
K_{II} &=& \begin{bmatrix}5.792& 6.353& 0.289& 0.469 \\1.103& 7.094& -0.579& 0.326\end{bmatrix},\nonumber \\
K_{III} &=&  \begin{bmatrix}0.493& 1.399& 0.139& -0.392\\3.014&  18.545& 1.765& -0.766 \end{bmatrix},\IEEEeqnarraynumspace
\end{IEEEeqnarray}
and they resulted in control laws with  critical regions, respectively.

\begin{figure}[!ht]
\centering
\subfigure[]{\label{state1}
\includegraphics[width=\columnwidth,height=2.0in]{S_1-eps-converted-to.pdf}}
\subfigure[]{\label{state2}
\includegraphics[width=\columnwidth,height=2.0in]{S_2-eps-converted-to.pdf}}
\subfigure[]{
\label{state3}
\includegraphics[width=\columnwidth,height=2.0in]{S_3-eps-converted-to.pdf}}
\caption{Controller partitions projected on subspace  and the state trajectory with (a) Scheme I, (b) Scheme II (cut at ) and (c) Scheme III (cut at ).}
\label{state}
\end{figure}

From the state response in Fig. \ref{stateinput}, we saw that the scheme I could not negate the additive disturbance happened either at an active constraint region or at steady state. It resulted in offset  and , respectively. The scheme II could track both the state variables but with significant overshoot due to the regulation of  back to 0. That effect could be removed by tracking it to a constant (as a tuning parameter), but ignored in this example for simplicity. The scheme III tracked  as required, and successfully forced the disturbance effect into . The tracking under setpoint change and disturbance rejection also happened faster than scheme II. We stressed that all the three schemes were able to deal with the state and input constraints ,  during transient stage because of the feedforward term  in the parametric MPC law.

Fig. \ref{state} gave another perspective of the result. Provided that the impulse disturbance did not excite the current state out of the feasible region , it was feasible to find an optimal input for all the three schemes. Secondly, scheme II hit on the outer constraint  and took a long time to recover. Indeed, an integral windup happened at this upper output bound. The scheme III showed the full PID potential. It is known that the proportional-integral deals with the present and past behavior of the plant. The differential term predicts the plant behavior and can be used to stabilize the plant faster. This was in line with Remark \ref{rmk2}. The trajectory quickly returned to the origin in both cases of setpoint change and additive disturbance. Lastly, while scheme II regulated the state error back to the origin, scheme III only drove it to the axis  as expected. It meant only  PIDs and  Ps were needed to track  outputs.

In conclusion, it is observed that as long as the disturbance does not drive the equilibrium outside of the unconstrained region, output tracking using the integral state variables remains feasible. The robust stability during transient stage is inherent through the PID form. The robust stability around setpoint only concerns the PID control design described in Section \ref{PIDsec}, which can be improved further by  approaches as stated in Remark \ref{rmk3}. Overall, extension to integral and differential terms is the natural to perform tracking control.

\section{Conclusion and Future work}

As it was never emphasized enough, the link between MPC and a robust linear controller at equilibrium is revisited in this paper. We modify the linear controller to be capable of offset-free tracking. The resultant control architecture is a PID gain scheduling network with a feedforward part to deal with state and input constraints. A simple test for setpoint tracking feasibility is also discussed. Finally, the example results show that the robustness stability of the proposed method is inherent within the PI/PID structure when disturbances arrives.



\bibliographystyle{elsarticle-harv}
\bibliography{D:/Dropbox/Research/referencelist}
\end{document}