


\documentclass[runningheads]{llncs}

\usepackage{amssymb}
\setcounter{tocdepth}{3}
\usepackage{graphicx}

\usepackage{url}



\newcommand\nc{\newcommand}

\newtheorem{cor}[theorem]{Corollary}
\newtheorem{prop}[theorem]{Proposition}
\newtheorem{defn}[theorem]{Definition}
\newtheorem{conj}[theorem]{Conjecture}
\newtheorem{ques}[theorem]{Question}
\newtheorem{prob}[theorem]{Problem}
\newtheorem{fact}[theorem]{Fact}

\def\boxit#1{\vbox{\hrule\hbox{\vrule\kern4pt
  \vbox{\kern1pt#1\kern1pt}
\kern2pt\vrule}\hrule}}

\nc{\crl}[2]{\begin{corollary}\label{crl:#1} #2 \end{corollary}}
\nc{\dfn}[2]{\begin{definition}\label{def:#1} #2 \end{definition}}
\nc{\lem}[2]{\begin{lemma}\label{lem:#1} #2 \end{lemma}}
\nc{\prp}[2]{\begin{proposition}\label{prp:#1} #2
\end{proposition}}
\nc{\thm}[2]{\begin{theorem}\label{thm:#1} #2\end{theorem}}
\nc{\fac}[2]{\begin{fact}\label{fact:#1} #2 \end{fact}}

\nc{\eqn}[2]{}


\nc{\fig}[4]{\begin{figure}[h]
\begin{center}
\includegraphics[width=#2\textwidth]{#4}
\end{center}
\caption{#3}\label{fig:#1}
\end{figure}}



\nc{\tbl}[3]{\begin{table}[hbt] #3 \caption{#2} \label{tab:#1}
\end{table}}

\nc{\refc}[1]{Corollary~\ref{crl:#1}}
\nc{\refd}[1]{Definition~\ref{def:#1}}
\nc{\reff}[1]{Figure~\ref{fig:#1}}
\nc{\refl}[1]{Lemma~\ref{lem:#1}}
\nc{\refp}[1]{Proposition~\ref{prp:#1}}
\nc{\reft}[1]{Theorem~\ref{thm:#1}}
\nc{\refe}[1]{(\ref{eqn:#1})}
\nc{\reftb}[1]{Table~\ref{tab:#1}}

\nc{\reffc}[1]{Fact~\ref{fact:#1}}





\begin{document}





\title{New Branching Rules: Improvements on Independent Set and Vertex Cover in Sparse Graphs\footnote{The paper was presented at the 2nd
annual meeting of asian association for algorithms and computation
(AAAC 2009), April 11-12, 2009, Hangzhou, China.}}

\titlerunning{Independent Set and Vertex Cover in Sparse graphs}

\author{Mingyu Xiao}
\authorrunning{}




\institute{School of Computer Science and Engineering\\
University of Electronic Science and Technology of China \\
Chengdu 610054, CHINA\\
Email: {\tt myxiao@gmail.com}}


\toctitle{Maximum Independent Set} \tocauthor{} \maketitle


\begin{abstract}
We present an -time algorithm for finding a maximum
independent set in an -vertex graph with degree bounded by ,
which improves the previously known algorithm of running time
 by Bourgeois, Escoffier and Paschos [IWPEC 2008].
We also present an -time algorithm to decide if a
graph with degree bounded by  has a vertex cover of size ,
which improves the previously known algorithm of running time
 by Chen, Kanj and Xia [ISAAC 2003].

Two new branching techniques, \emph{branching on a bottle} and
\emph{branching on a -cycle}, are introduced, which help us to
design simple and fast algorithms for the maximum independent set
and minimum vertex cover problems and avoid tedious branching
rules.


\vspace{5mm}\noindent {\bf Key words.} \ \ Graph Algorithm,
Independent Set, Vertex Cover, Sparse Graph
\end{abstract}


\section{Introduction}\label{sec_intr}
The \emph{maximum independent set} problem (MIS), to find a
maximum set of vertices in a graph such that there is no edge
between any two vertices in the set, is one of the basic NP-hard
optimization problems and has been well studied in the literature,
in particular in the line of research on worst-case analysis of
algorithms for NP-hard optimization problems. In 1977, Tarjan and
Trojanowski~\cite{Tarjan:IS} published the first algorithm for
this problem, which runs in  time and polynomial
space. Later, the running time was improved to 
by Jian~\cite{Jian:Is}. Robson~\cite{Robson:IS} obtained an
-time polynomial-space algorithm and an
-time exponential-space algorithm. In a technical
report~\cite{Robson:IS_1}, Robson also claimed better running
times. Recently, Fomin et al.~\cite{Fomin:is} got a simple
-time polynomial-space algorithm by using the
``Measure and Conquer" method. There is also a considerable amount
of contributions to the maximum independent set problem in sparse
graphs, especially in degree-
graphs~\cite{Beigel:is},\cite{Chen:labeled3vc},\cite{xiao:IS3},\cite{Bourgeois:3IS}.
We summarize the results on low-degree graphs as well as general
graphs in Table~\ref{table1}.



\begin{table}[htp]\label{table1}
\begin{tabular}{|l|l|l|l|}\hline
\textbf{Authors} & \textbf{Running times} &\textbf{References}& \textbf{Notes}\\
\hline \hline Tarjan \& Trojanowski & for MIS&
1977~\cite{Tarjan:IS}&: number of vertices\\ \hline Jian &  for MIS& 1986~\cite{Jian:Is}&\\
\hline Robson & for MIS &1986~\cite{Robson:IS}&Exponential space\\
\hline
 Beigel &  for MIS
& 1999~\cite{Beigel:is}&: number of edges\\
& for -MIS & &-MIS: MIS in degree- graphs\\
 \hline Robson & for MIS &2001~\cite{Robson:IS_1}&Partially computer-generated\\
 \hline Chen et al. &  for -MIS & 2003~\cite{Chen:labeled3vc}&\\
\hline Xiao et al. &  for -MIS & 2005~\cite{xiao:IS3}&Published in Chinese\\
\hline Fomin et al. &  for MIS & 2006~\cite{Fomin:is}&\\
\hline Fomin \& H\o ie &  for -MIS & 2006~\cite{Fomin:cubicgraph}&\\
\hline F{\"u}rer &  for -MIS & 2006~\cite{Furer:ISsparse}&\\
\hline Razgon &  for -MIS & 2006~\cite{Razgon:3IS}&\\
\hline Bourgeois et al. &  for -MIS &
2008~\cite{Bourgeois:3IS} &\\
\hline Xiao & for -MIS& This paper & \\
\hline
\end{tabular}
\vspace{3mm} \caption{Exact algorithms for the maximum independent
set problem}
\end{table}
In the literature, there are several methods of designing
algorithms for finding maximum independent sets in graphs. One
method is to find a minimum vertex cover (a set of vertices such
that each edge in the graph has at least one endpoint in the set),
and then to get a maximum independent set by taking all the
remaining vertices, such as the algorithms presented
in~\cite{Chen:labeled3vc},\cite{VC2005}. In this kind of
algorithms, the dominating part of the running time is the running
time for finding a minimum vertex cover. Another method is based
on the search tree method. We will use a branch-and-reduce
paradigm. We choose a parameter, such as the number of vertices or
edges or others, as a measure of the size of the problem. When the
parameter is zero or a negative number, the problem can be solved
in polynomial time. We branch on the current graph  into serval
graphs , ,  such that the parameter 
of graph  is less than the parameter  of graph 
(), and a maximum independent set in  can be
found in polynomial time if a maximum independent set in each of
the  graphs , ,  is known. By this
method, we can build up a search tree, and the exponential part of
the running time of the algorithm is corresponding to the size of
the search tree. The running time analysis leads to a linear
recurrence for each node in the search tree that can be solved by
using standard techniques. Let  denote the worst-case size
of the search tree when the parameter of graph  is , then we
get recurrence relation . Solving
the recurrence, we get , where  is the largest
root of the function . As for the
measure (the parameter ), a natural one is the number of
vertices or edges in the graph. Most previous algorithms for the
maximum independent set problem are analyzed by using the number
of vertices as a
measure~\cite{Tarjan:IS},\cite{Jian:Is},\cite{Robson:IS},\cite{Fomin:is}.
The number of edges is considered in Beigel's
algorithm~\cite{Beigel:is}. There are also some other measures.
Xiao et al.~\cite{xiao:IS3} used the number of degree- vertices
as a measure to analyze algorithms and got an -time
algorithm for MIS in degree- graphs. Unfortunately, that paper
was published in Chinese. Recently, Razgon~\cite{Razgon:3IS} also
got an -time algorithm for MIS in degree- graphs
by measuring the number of degree- vertices. But the two
algorithms are totally different. F{\"u}rer~\cite{Furer:ISsparse}
designed an algorithm for MIS in degree- graphs by tackling
, where  is the number of edges and  the number of
vertices. Based upon a refined branching with respect to
F{\"u}rer's algorithm, Bourgeois et al.~\cite{Bourgeois:3IS} got
the current best algorithm for MIS in degree- graphs with
running time . In this paper, we still use the
number of degree- vertices as a measure to analyze our
algorithm. Based on two new branching rules, \emph{branching on a
bottle} and \emph{branching on a -cycle}, we design an even
faster algorithm for MIS in degree- graphs, which runs in
 time. Our algorithm is simple and does not contain
many branching rules. Furthermore, it can be used to solve the
\emph{-vertex cover} problem (to decide if the graph has a
vertex cover of size ) in degree- graphs in 
time, which improves the previously known result of
 by Chen et al.~\cite{Chen:labeled3vc}.



\section{Preliminaries}
We shall try to be consistent in using the following notation. The
number of vertices in a graph will be denoted by  and the
number of degree- vertices (vertices of degree  will
also be counted with a weight) by . For a vertex  in a
graph,  is the degree of ,  the set of all
neighbors of ,  the set of vertices with
distance at most  from , and  the set of vertices
with distance exactly  from . We say edge  is incident on
a vertex set , if at least one endpoint of  is in . In
our algorithm, when we remove a set of vertices, we also remove
all the edges that are incident on it. Throughout the paper we use
a modified  notation that suppresses all polynomially bounded
factors. For two functions  and , we write  if , where  is a
polynomial.


Our algorithms are based on the branch-and-reduce paradigm. We
will first apply some reduction rules to reduce the size of
instances of the problem. Then we apply some branching rules to
branch on the graph by including some vertices in the independent
set or excluding some vertices from the independent set. In each
branch, we will get a maximum independent set problem in a graph
with a smaller measure. Next, we introduce the reduction rules and
branching rules that will be used in our algorithms.

\subsection{Reduction Rules}
There are several standard preprocesses to reduce the size of
instances of the problem. \emph{Folding a degree- or degree-
vertex} and \emph{removing a dominated vertex} are frequently used
rules. Besides these reduction rules, we still need to reduce some
other local structures called \emph{- structure},
\emph{- structure} and \emph{- structure}.

\vspace{2mm}\noindent \textbf{Folding a degree-1 vertex}\\
\emph{Folding a degree-1 vertex  means removing  and 
from the graph, where  is the unique neighbor of .}

\vspace{2mm}\noindent \textbf{Folding a degree-2 vertex}\\
\emph{Folding a degree- vertex  (with two neighbors  and
) means\\
 removing ,  and  from the
graph, when  and  are adjacent.\\
 removing ,  and  from the graph and introducing a
new vertex  that is adjacent to all neighbors of  and  in
 except the removed vertex , when  and  are
nonadjacent.}

\vspace{0mm}Please refer to \reff{folding} for an illustration of
the operation in case  of folding a degree- vertex. Let
 denote the size of a maximum independent set of graph
 and  the graph after folding a degree- or
degree- vertex . Then we have the following
lemma.\lem{fold_1}{For any degree- or degree- vertex  in
graph , \vspace{-2mm} }
\vspace{-4mm}\fig{folding}{1}{Illustrations of folding
operations}{folding}

The correctness of folding a degree- or degree- vertex has
been discussed in many pervious papers. In fact, general folding
rules are known in the literature, which can deal with a vertex of
degree  or a set of independent
vertices~\cite{VC2005},\cite{Fomin:is}. In this paper, we still
need to fold the following three local structures called -
structure, - structure and - structure.

Let  and  be two independent degree- vertices, if they
have three common neighbors  and , then we say that the
five vertices compose a \emph{- structure} (see
\reff{folding}), and denote it by -. Let 
be a degree- vertex, and  and  two adjacent vertices of
degree . If , then we say that
the six vertices  compose a \emph{-
structure} (see \reff{folding}), and denote it by
-. Let  and  be three independent
vertices of degree  , if they have exact four neighbors
 and , then we say that the seven vertices compose a
\emph{- structure}, and denote it by
-.

\vspace{2mm}\noindent \textbf{Folding a - structure, - structure or - structure}\\
\emph{Let - be a - structure or - structure or
- structure. Folding - means\\
 removing  from the
graph, when  is not an independent set.\\
 removing  from the graph and introducing a new
vertex  that is adjacent to all neighbors of vertices in 
except the removed vertices, when  is an independent
set.}


\lem{fold_2}{If graph  has a - structure or -
structure, then
 where  is the graph
after folding a - structure or - structure in .

If graph  has a - structure, then
 where  is the graph
after folding a - structure in . }

A degree- vertex can be regarded as a - structure
according to our definitions. In fact, a degree- vertex,
- structure and - structure are special cases
described in Lemma  in \cite{VC2005}. The - structure
is for the first time being introduced. The correctness of folding
an - structure (a - structure, - structure,
- structure or - structure) follows from this
observation: When  is not an independent set, there is a
maximum independent set that contains  (or two independent
vertices in , when - is a - structure). When  is
an independent set, there is a maximum independent set that
contains either  or  (or two independent vertices in ,
when - is a - structure). We ignore the detailed proof
here.

\vspace{2mm}\noindent \textbf{Dominance}\\
\emph{If there are two vertices  and  such that
, we say  dominates .} \lem{}{If vertex
 is dominated by any other vertex in graph , then
\vspace{-2mm} }

\dfn{}{A graph is called a \emph{reduced graph}, if it has no
degree- vertex, degree- vertex, dominated vertex, -
structure, - structure or - structure.}

\subsection{Branching Rules}

Next we introduce two branching techniques, \emph{branching on a
bottle} and \emph{branching on a -cycle}, which are simple and
obvious, but can  avoid tedious branching rules in the description
of the algorithms.

Let  be a degree- vertex, and  the three neighbors of
. If two neighbors of , say  and , are adjacent, then
we say that the four vertices compose a \emph{bottle} and denote
it by --.

\lem{bottlebranch}{Let -- be a bottle in graph ,
then there is a maximum independent set  in  such that
either  or .} \proof{If  is not in a maximum
independent set, we can directly remove  from the graph. In the
remaining graph  becomes a degree- vertex and the two
neighbors of it are adjacent. In this case, there is a maximum
independent set that contains .}

Based on \refl{bottlebranch}, we get the following branching rule.

\vspace{2mm}\noindent \textbf{Branching on a bottle}\\
\emph{Branching on a bottle -- means branching by
either including  in the independent set or including  in
the independent set.}

\noindent\textbf{Note. }In fact, we can fold a bottle by using the
general folding rule mentioned in~\cite{Fomin:is} (also
in~\cite{Beigel:is}), but this folding rule is helpless for our
analysis, especially when the three neighbors of the degree-
vertex are high-degree vertices.

\vspace{2mm}Let  and  be four vertices in graph , if
 has four edges , ,  and , then we say that
 is a \emph{-cycle} in .

\lem{cyclebranch}{Let  be a -cycle in graph , then for
any independent set  in , either  or .}\proof{Since any independent set contains at most  vertices
in a -cycle and the two vertices can not be adjacent, we know
the lemma holds.}

Based on \refl{cyclebranch}, we get the following branching rule.

\vspace{2mm}\noindent \textbf{Branching on a -cycle}\\
\emph{Branching on a -cycle  means branching by either
excluding  and  from the independent set or excluding 
and  from the independent set.}


\section{A Simple Algorithm}\label{alg}

Our algorithm for the maximum independent set problem is described
in Figure~\ref{mis3}. It works as follows. If the graph has a
component of at most  vertices, we find a maximum independent
set in this component directly (\textbf{Step~}). If the graph
has a degree- or degree- vertex, we fold it in
\textbf{Step~}. If the graph has a dominated vertex, we remove
it in \textbf{Step~}. If the graph has a - structure or
- structure or - structure, we fold it in
\textbf{Step~} and \textbf{Step~}. When the graph can not be
reduced, we apply our branching rules. If there is a bottle, we
branch on a bottle (\textbf{Step~}). Else if there is a
-cycle, we branch on a -cycle (\textbf{Step~}). Else in
\textbf{Step~}, we greedily select a vertex of maximum degree
and branch on it by including it in the independent set or
excluding it from the independent set.

 \vspace{-0mm}\begin{figure}[!htbp] \setbox4=\vbox{\hsize28pc
\noindent\strut
\begin{quote}
\vspace*{-2mm} \footnotesize {\bf } \\

\textbf{Input}: A graph . \\
\textbf{Output}: The size of a maximum independent set in .\\

\begin{enumerate}
\item \textbf{If} \{ has a component  of at most 
vertices\}, \textbf{return} , where  is the size of
a minimum independent set in . \item \textbf{Else if}
\{:  or \}, \textbf{return}
. \item \textbf{Else if} \{:
\}, \textbf{return} . \item
\textbf{Else if} \{there is a - structure or -
structure\}, \textbf{return} .\item \textbf{Else
if} \{there is a - structure\}, \textbf{return}
. \item \textbf{Else if} \{there is a bottle
--\}, \textbf{return} . \item \textbf{Else if}\{there is a -cycle
\}, \textbf{return} .
 \item \textbf{Else,}
pick up a vertex  of maximum degree, and \textbf{return}
.

\end{enumerate}

\vspace{3mm}\textbf{Note}: With a few modifications, the algorithm
can provide a maximum independent set.
\end{quote} \vspace*{-5mm} \strut}   \vspace*{-9mm}
\caption{The Algorithm } \label{mis3}
\end{figure}

 \section{The Analysis}
To analyze the time complexity of our algorithm, we will consider
recurrence relations related to parameter , the number of
degree- vertices (vertices of degree  will also be
counted with a weight) in the corresponding graph. When , the
graph has only degree-, degree- and degree- vertices and
the maximum independent set problem can be solved in linear time.
We use  to denote the worst-case size of the search tree in
our algorithm when the parameter of the graph is . In our
algorithm, it is possible to create a vertex of degree 
when folding. We will regard a degree- () vertex as a
combination of  degree- vertices and count  in
parameter . Then when a degree- vertex is removed, parameter
 will be reduced by . When an edge incident on a
degree- vertex is removed, parameter  will be reduced by
. In the remaining of the paper, when we say a graph has 
degree- vertices, it does not mean that the graph really has
exactly  vertices of degree . In fact, all the vertices of
degree  are counted. Next, we analyze how much  can be
reduced in each step of our algorithm.

\lem{}{After folding a degree- or degree- vertex, parameter
 will not increase.}

\lem{folding}{Let  be a graph having no degree- or
degree- vertex, then after folding a - structure or
- structure or - structure, or removing a dominated
vertex in , parameter  will be reduced by at least .}
\proof{In each case, a degree- vertex is removed (or an even
better case occurs), and then we can further reduce  by 
from  neighbors of the vertex. Totally  will be reduced by
at least .}


\lem{d1}{Let  be a connected graph. If  has at least 
degree- vertices and  vertices of degree  (a
degree- () vertex will be regarded as  degree-
vertices), then after iteratively folding degree- vertices
until the graph has no degree- vertex, parameter  will be
reduced by at least . }

\proof{Let  be the set of vertices of degree
 in the remaining graph after iteratively folding
degree- vertices (The lemma obviously holds, when ). Assume there are  edges between  and
. After removing , we can reduce  degree- vertices
from . We will prove that there are at least  degree-
vertices in . To prove that, we first construct a new graph
 from  by contracting  into a single vertex  and
remove all self-loops incident on it (keeping parallel edges).
Then we only need to prove that except vertex ,  has at
least  degree- vertices.

Since all the  degree- vertices of  are in , 
has at least  degree- vertices, where  when  is
a degree- vertex and  when  is not a degree-
vertex. Note that a tree with  degree- vertices has at
least  degree- vertices. We know that  has least
 degree- vertices ( is a connected graph). We
consider the following three cases. Case 1: . For this case,
 is a degree- vertex and . Then  has at least
 degree- vertices. Case 2: . For this case, 
is a degree- vertex and , and  still has at least
 degree- vertices. Case 3: . For this case, 
is a degree- vertex and . Excepting  degree-
vertices counted from , there are sill 
degree- vertices.

Therefore, after removing ,  will be reduced by at least
.}

\crl{degree1}{Let  be a graph having not any component of a
path. If  has any degree- vertex, then we can reduce  by
at least  by iteratively folding degree- vertices. If 
has exactly  degree- vertices, then we can reduce  by at
least  by iteratively folding degree- vertices.}


\lem{3db}{Let  be a reduced graph and  a degree- vertex
in . Then not any degree- vertex or component of a -path
or component of a -path is created after removing .}
\proof{If a degree- vertex  is created, then  has a
- structure -. If a -path  is
created, then there is a - structure -.
If a -path  is created, then there is a - structure
-.}


\lem{3degree}{Let  be a connected reduced graph of more than
 vertices and  a degree- vertex in . Then after
removing , parameter  will be reduced by at least .
Furthermore, if each -cycle in  contains at least one vertex
of degree , then after removing , parameter  will
be reduced by at least .}

\proof{There is at most one edge with both endpoints in ,
otherwise  will dominate a neighbor of it. Therefore, there are
at least four edges between  and . If
,  will be reduced by  directly after
removing . If , it is impossible to create a
component of a -path () after removing . By
\refl{d1} and \refc{degree1} and \refl{3db} we know that
eventually  will be reduced by at least .

Next, we assume that in each -cycle in  there is a vertex of
degree . We distinguish the following two cases.
\textbf{Case~}: All vertices in  are degree- vertices.
In this case, none pair of vertices in  are adjacent and
there are exactly six edges between  and , which
means at most  degree- vertices will be created after
removing . It is impossible to create a component of a path
after removing  (Obviously, no path of length  will
be created. \refl{3db} shows no path of length  will be
created. If a -path is created, then the graph  has only 
vertices). So by \refc{degree1}, if a component with  or 
degree- vertices is created after removing , we can
further reduce  by  or  by further reducing degree-
vertices in the component. If a component with  degree-
vertices is created, then the component also contains at least 
degree- vertices, otherwise the only possibility of the
component is that it has  vertices: a degree- vertex
adjacent with three degree- vertices, which also implies a
contradiction--the graph  has only  vertices. By \refl{d1},
we still can further reduce  by least . In any case, totally
we can reduce  by at least .
 \textbf{Case~}: There is a vertex of degree
 in . Then there are at least five edges between
 and  (Note that there is at most one edge with both
endpoints in ). By \refl{d1} and \refl{3db} we know that 
will be reduced by at least . }

\lem{4degree}{Let  be a connected reduced graph of more than
 vertices and  a vertex of degree  in . Then after
removing , parameter  will be reduced by at least . }

\proof{The lemma obviously holds when  is a vertex of degree
 or a degree- vertex with . Now we
assume  is a degree- vertex and .
\textbf{Case~}: . In this case, after removing
,  is reduced by at least , or  is reduced by
at least  and the only vertex in  becomes a
degree- vertex (Note that there are at least  edges
between  and ). In the later case, we can reduce 
by at least  by folding degree- vertices.\textbf{ Case~}:
. The two vertices  are adjacent,
otherwise there is a - structure -.
Then after removing , at most one of  and  becomes a
degree- vertex, otherwise  has only  vertices. Therefore,
we also can reduce  by at least  from .
\textbf{Case~}: . If one vertex in  is a
vertex of degree , then the lemma holds. Otherwise, all
vertices in  are degree- vertices, and then the number of
edges between  and  is  or  or . If no
degree- vertex is created after removing , then  will
be reduced by at least  directly (Note that it is impossible
to create two degree- vertices, and when one degree- vertex
is created, there are at least three edges between  and
 that are incident on the other two vertices in ).
If some degree- vertices are created but no path component is
created, then we can further reduce  by at least  by
\refl{d1}. The difficult case occurs when a path component is
created. The path can only be a -path or -path. If it is a
-path, then the graph has only  vertices. Therefore the path
is a -path. If one vertex in the path is a vertex of degree
 in , then after removing ,  is reduced by at
least  directly. If the two vertices in the path are
degree- vertices in , then there are at least two edges
between  and  that are incident on the third vertex
 in . So after removing ,  is reduced by at
least , or  is reduced by at least  and 
becomes a degree- vertex, folding which will further reduce 
by at least . We have checked all the cases and then finished
the proof. }


\lem{bottle}{Let  be a connected reduced graph of more than 
vertices.  If  has a bottle, then algorithm  will
branch on a bottle with recurrence relation \eqn{8-8_1}{C(r)\leq
2C(r-8),} where  is the worst-case size of the search tree
in our algorithm.

Moreover, if each -cycle in  contains at least one vertex of
degree , then   will branch on a bottle with
recurrence relation \eqn{10-10_1}{C(r)\leq 2C(r-10).} }

\proof{Let the bottle called by our algorithm be
--. Our algorithm will branch by either removing
 or . By \refl{3degree} and \refl{4degree}, we get
\refe{8-8_1} and \refe{10-10_1} directly. }


\lem{4cycle}{Let  be a connected bottle-free reduced graph of
more than  vertices. If  has a -cycle, then algorithm
 will branch on a -cycle with recurrence relation
\eqn{8-8}{C(r)\leq 2C(r-8).} Moreover, if each -cycle or
-cycle in  contains at least one vertex of degree ,
then   will branch on a -cycle with recurrence relation
\eqn{10-10}{C(r)\leq 2C(r-10).}}

\proof{Let the -cycle called by our algorithm be . Our
algorithm will branch by removing either  or 
from the graph. We look at the branch where  is removed
(It is the same to ). Since none of the four vertices is
dominated by others, each of the four vertices will be adjacent to
a vertex different from the four vertices. If after removing , no degree- vertex is created, then we can reduce  by
 directly in this branch. If some degree- vertices are
created, then our algorithm will fold one, say , in the next
step. Obviously,  is a degree- vertex in the original graph.
The operation of removing  and then folding  is
equivalent to the removing of . We can reduce  by at
least  by \refl{3degree}. Therefore, we get \refe{8-8}.

Next, we prove \refe{10-10}. There is at least one vertex of
degree , say , in the -cycle. We distinguish the
following three cases. \textbf{Case~}: There is only one vertex
of degree  in the -cycle. No matter we remove 
or , at least one degree- will be created. As
discussed above, after further folding a degree- vertex, we can
reduce  by at least  in each branch by \refl{3degree}. Then
we get \refe{10-10}. \textbf{Case~}: There are exactly two
vertices of degree  in the -cycle and the two vertices
are adjacent to each other in the -cycle (the two vertices are
not  or ). Without loss of generality, we can
assume the two vertices of degree  are  and . In the
branch where  is removed,  becomes a degree-
vertex. In the branch where  is removed,  becomes a
degree- vertex. Then in each branch we will remove  for
some degree- vertex  in . We still can get \refe{10-10}.
 \textbf{Case~}: There are exactly two
vertices of degree  in the -cycle and the two vertices
are a pair of opposite vertices in the -cycle. Then the two
vertices of degree  are  and  (We have assumed that
 is a vertex of degree ). It is easy to see that after
removing ,  will be reduced by at least . In the
branch where  is removed, some degree- vertices are
created (at least  and ). Then in this branch we will remove
 for some degree- vertex  in , where  has two
vertices of degree   and . Since  has no bottle,
there is not any edge with both endpoints in . Therefore,
there are at least  edges between  and . If no
degree- vertex is created after removing  (\refl{3db}
also shows that no degree- vertex will be created), then  is
reduced by  directly. If some degree- vertices are
created, the case becomes complicated. In fact, as we do in the
proof of \refl{3degree} we can prove that no component of less
than  degree- vertices will be created. By  \refl{d1}, we
know that  will also be reduced by at least  in this
branch. We get \eqn{8-13}{C(r)\leq C(r-8)+C(r-13).}
\textbf{Case~}: There are exactly three vertices of degree
 in the -cycle. Without loss of generality, we assume
the remaining degree- vertex is . After removing ,
 will be reduced by at least . After removing ,
 will be reduced by at least . We get \eqn{10-12}{C(r)\leq
C(r-10)+C(r-12).} \textbf{Case~}: All the four vertices in the
cycle are vertices of degree . It is clear that  will be
reduced by at least  in each branch. We also get \refe{10-10}.
Since \refe{10-10} covers \refe{8-13} and \refe{10-12}, we know
that the lemma holds. }




\lem{}{Let  be a connected reduced graph of more than 
vertices that has no bottle or -cycle. If  has a vertex of
degree , then algorithm  will branch on a vertex of
maximum degree with recurrence relation \eqn{6-14}{C(r)\leq
C(r-6)+C(r-14).} }

\proof{Our algorithm will select a vertex  of maximum degree
and branch on it by excluding it from the independent set or
including it in the independent set. In the former branch,  is
removed and  decreases by at least . In the latter
branch,  is removed. Since  has no bottle or -cycle,
there are at least  vertices in . Then in this branch,
 will be reduced by at least . Therefore, we get
\refe{6-14}.}


\lem{3dv}{Let  be a connected reduced graph of more than 
vertices that has no bottle or -cycle. If  is also a
-regular graph, then algorithm  can branch with
recurrence relation \eqn{10-14-14}{C(r)\leq C(r-10)+2C(r-14).}}

\proof{Our algorithm will select a degree- vertex and branch on
it. Since  is -regular graph that has no -cycle or
-cycle, there are exactly  vertices in . In the
branch where  is removed,  degree- vertices are
reduced. So we can branch with recurrence relation
\eqn{10-4}{C(r)\leq C(r-10)+Q(r-4),} where  is some
function corresponding to the size of the branch where  is
removed. Next, we focus on refining analysis of .

In the branch where  is removed,  nonadjacent degree-
vertices are created. Our algorithm will fold the three degree-
vertices in the next step. Let  be the resulted graph. Then
 has exactly  degree- vertices (Note that the original
graph has no -cycle or -cycle. It is impossible to create a
degree- vertex after folding a degree- vertex),  and each
-cycle or -cycle in the current graph contains at least one
degree- vertex. If  has a bottle or -cycle, we can
branch with  by \refl{bottle} and
\refl{4cycle}. If  has no bottle or -cycle, we will branch
on a degree- vertex . We further distinguish three
different cases. \textbf{Case~}: The other two degree-
vertices are adjacent to . In this case, we have
 (the three degree- vertices may form a
triangle). In the branch where  is removed,  is reduced by
at least , and in the branch where  is removed,  is
reduced by at least . We get .
\textbf{Case~}: There is only one degree- vertex adjacent to
. Since there is no bottle and -cycle, we get
. In the branch where  is removed,  is
reduced by . We also get .
\textbf{Case~}: There is no degree- vertex adjacent to .
We will branch on  with \refe{6-14} directly, and in the
branch where  is removed, some other degree- vertices are
left. We can further branch with \refe{6-14} at least. Then we get
.

The worst case is that after branching with \refe{10-4} we branch
with \refe{10-10_1} or \refe{10-10}, in which we get , as claimed in the lemma. }

Among all the cases in our algorithm, the worst running time
corresponds to recurrence relation \refe{10-14-14}. Since
 satisfies \refe{10-14-14}, we get

\thm{}{Algorithm  can find a minimum independent set in a
degree- graph in  time.}


\section{Improvement on -Vertex Cover}
Given a graph  and a parameter , the \emph{-vertex cover}
problem is to decide if  has a vertex cover of size at most
. The -vertex cover problem is one of the most extensively
studied problems in the area of Parameterized Algorithms. In this
section, we show that the -vertex cover problem can be solved
in  time, which improves
 the previously known result of  by Chen \emph{et~al.}~\cite{Chen:labeled3vc}.

Nemhauser and Trotter~\cite{Nemhauser:VCkernel} proved the
following theorem:

\prp{}{For a graph  with  vertices and  edges, we
can compute two disjoint vertex sets  in
 time, such that

 Every minimum vertex cover in induced subgraph  plus
 forms a minimum vertex cover of .

 A minimum vertex cover of  contains at least  vertices. }

Our simple algorithm works as follows. Given an instance 
of the -vertex cover problem in degree- graphs, we first use
Nemhauser and Trotter's algorithm to construct  and . If
, then  does not have a vertex cover of size at most
. Else we use our algorithm presented in Section~\ref{alg} to
find a maximum independent set  in  in
 time. Then  is a
minimum vertex cover of . If , then  does not
have a vertex cover of size at most . Else  is
satisfied vertex cover.

\thm{}{The -vertex cover problem in degree- graphs can be
solved in  time.}









\section{Concluding Remarks}\label{conclusion}
In this paper, we have presented a simple -time
algorithm for the minimum independent set problem in degree-
graphs and a simple -time algorithm for the
-vertex cover problem in degree- graphs. Both algorithms
improve previously known algorithms.

Unlike most previous algorithms, our algorithms do not contain
many branching rules. We use two new branching techniques, called
branching on a bottle and branching on a -cycle, to avoid
tedious examinations of the local structures. In fact, new
branching rules catch the structural properties of small cycles in
graphs, which make our algorithms simple and practical. It is easy
to see that many previous algorithms can apply these two new
branching rules to simplify the description and analysis.

Our algorithm for the maximum independent set problem is analyzed
by measuring the number of degree- vertices. We have checked
that our algorithm  can also be analyzed by measuring
parameter  to get the same running time bound, where  is
the number of edges,  the number of vertices, and  the
number of tree components in the graph. Readers may note that the
algorithms presented by F{\"u}rer~\cite{Furer:ISsparse} and
Bourgeois \emph{et al.}~\cite{Bourgeois:3IS} are analyzed by
measuring . In fact, their algorithms also need to consider
the tree components created in the graphs and they have a separate
section to analyze them. We guess that considering the tree
components in the parameter may lead to a clearer analysis.








\bibliographystyle{splncs}
\begin{thebibliography}{10}

\bibitem{Tarjan:IS}
Tarjan, R., Trojanowski, A.:
\newblock Finding a maximum independent set.
\newblock SIAM Journal on Computing \textbf{6}(3) (1977)  537--546

\bibitem{Jian:Is}
T.Jian:
\newblock An  algorithm for solving maximum independent set
  problem.
\newblock IEEE Transactions on Computers \textbf{35}(9) (1986)  847--851

\bibitem{Robson:IS}
Roboson, J.:
\newblock Algorithms for maximum independent sets.
\newblock Journal of Algorithms \textbf{7}(3) (1986)  425--440

\bibitem{Robson:IS_1}
Roboson, J.:
\newblock Finding a maximum independent set in time .
\newblock {T}echnical {R}eport 1251-01, LaBRI, Univsersite Bordeaux I (2001)

\bibitem{Fomin:is}
Fomin, F.V., Grandoni, F., Kratsch, D.:
\newblock Measure and conquer: a simple  independent set
  algorithm.
\newblock In: SODA, ACM Press (2006)  18--25

\bibitem{Beigel:is}
Beigel, R.:
\newblock Finding maximum independent sets in sparse and general graphs.
\newblock In: Proceedings of the 10th annual ACM-SIAM symposium on discrete
  algorithms (SODA 1999). (1999)  856--857

\bibitem{Chen:labeled3vc}
Chen, J., Kanj, I.A., Xia, G.:
\newblock Labeled search trees and amortized analysis: Improved upper bounds
  for {NP}-hard problems.
\newblock Algorithmica \textbf{43}(4) (2005)  245--273 A preliminary version
  appeared in ISAAC 2003.

\bibitem{xiao:IS3}
Xiao, M.Y., Chen, J.E., Han, X.L.:
\newblock Improvement on vertex cover and independent set problems for
  low-degree graphs.
\newblock Chinese Journal of Computers \textbf{28}(2) (2005)  153--160

\bibitem{Bourgeois:3IS}
Bourgeois, N., Escoffier, B., Paschos, V.T.:
\newblock An  exact algorithm for max independe6nt set in
  sparse graphs.
\newblock In Grohe, M., Niedermeier, R., eds.: IWPEC. Volume 5018 of Lecture
  Notes in Computer Science., Springer (2008)  55--65

\bibitem{Fomin:cubicgraph}
Fomin, F.V., H{\o}ie, K.:
\newblock Pathwidth of cubic graphs and exact algorithms.
\newblock Inf. Process. Lett. \textbf{97}(5) (2006)  191--196

\bibitem{Furer:ISsparse}
F{\"u}rer, M.:
\newblock A faster algorithm for finding maximum independent sets in sparse
  graphs.
\newblock In Correa, J.R., Hevia, A., Kiwi, M.A., eds.: LATIN. Volume 3887 of
  Lecture Notes in Computer Science., Springer (2006)  491--501

\bibitem{Razgon:3IS}
Razgon, I.:
\newblock A faster solving of the maximum independent set problem for graphs
  with maximal degree 3.
\newblock In Broersma, H., Dantchev, S.S., 0002, M.J., Szeider, S., eds.: ACiD.
  Volume~7 of Texts in Algorithmics., King's College, London (2006)  131--142

\bibitem{VC2005}
Chen, J., Kanj, I., Xia, G.:
\newblock Simplicity is beauty: Improved upper bounds for vertex cover.
\newblock Technical Report TR05-008, School of CTI, DePaul University (2005)

\bibitem{Nemhauser:VCkernel}
Nemhauser, G.L., Trotter, L.E.:
\newblock Vertex packings: Structural properties and algorithms.
\newblock Mathematical Programming \textbf{8}(1) (1975)  232--248

\end{thebibliography}




\end{document}
