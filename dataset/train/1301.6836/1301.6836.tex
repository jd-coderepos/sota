\documentclass[letter]{ieice}
\usepackage[dvips]{graphicx}


\setcounter{page}{1}


\field{}
\title{Towards Interactive Object-Oriented Programming}
\authorlist{\authorentry{Keehang KWON}{n}{labelA}
\authorentry{Kyunghwan Park}{n}{labelB}
\authorentry{Mi-Young Park}{m}{labelA}
}
\affiliate[labelA]{The authors are with Computer Eng., DongA Univ., Korea.
\{khkwon,openmp\}@dau.ac.kr}
\affiliate[labelB]{The author is with Computer Eng., DongA Univ. khpark@dau.ac.kr.
The corresponding author.}






\newenvironment{describe}{\begin{list}{}{\setlength\leftmargin{80pt}}\setlength\labelsep{10pt}\setlength\labelwidth{70pt}}{\end{list}}

\newenvironment{flag}{\begin{list}{\makebox[20pt]{\hss\enspace}}
                                  {\labelwidth 20pt}}{\end{list}}





\newenvironment{numberedlist}
{\begin{list}{\makebox[20pt]{\hss(\arabic{itemno})\enspace}}
             {\usecounter{itemno}\labelwidth 20pt}}{\end{list}}

\newenvironment{alphabetlist}
{\begin{list}{\makebox[20pt]{\hss(\alph{itemno1})\enspace}}
             {\usecounter{itemno1}\labelwidth 20pt}}{\end{list}}

\newenvironment{romanlist}
{\begin{list}{\makebox[20pt]{\hss(\roman{itemno2})\enspace}}
             {\usecounter{itemno2}\labelwidth 20pt}}{\end{list}}

\newcounter{itemno}

\newcounter{itemno1}

\newcounter{itemno2}
\newcounter{lemma}
\newcounter{exno}

\newcounter{defno}






\newenvironment{defn}{\refstepcounter{defno}\medskip \noindent {\bf
Definition \thedefno.\ }}{\medskip}

\newenvironment{ex}{\refstepcounter{exno}\medskip \noindent {\bf
Example \theexno.\ }}{\medskip}














\newcommand{\sep}{\;\vert\;}

\newcommand{\ra}{\rightarrow}
\newcommand{\app}{\ }
\newcommand{\appt}{\ }
\newcommand{\tup}[1]{\langle\nobreak#1\nobreak\rangle}

\newcommand{\hu}{{\cal H}^+}
\newcommand{\Free}{{\cal F}}
\newcommand{\oprove}{\vdash\kern-.6em\lower.7ex\hbox{}\,}
\newcommand{\true}{\top}

\newcommand{\Dscr}{{\cal D}}
\newcommand{\Pscr}{{\cal P}}
\newcommand{\Gscr}{{\cal G}}
\newcommand{\Fscr}{{\cal F}}
\newcommand{\Vscr}{{\cal V}}
\newcommand{\Uscr}{{\cal U}}


\newcommand{\all}{\forall}
\newcommand{\some}{\exists}
\newcommand{\lambdax}[1]{\lambda #1\,}
\newcommand{\somex}[1]{\some#1\,}
\newcommand\allx[1]{\all#1\,}

\newcommand{\subs}[3]{[#1/#2]#3}
\newcommand{\rep}[3]{S^{#2}_{#1}{#3}}
\newcommand{\ie}{{\em i.e.}}
\newcommand{\eg}{{\em e.g.}}

\newcommand{\lbotr}{-R}
\newcommand{\ldbotr}{\bot\mbox{\rm -R}}

\newsavebox{\lpartfig}
\newsavebox{\rpartfig}




\newenvironment{exmple}{
 \begingroup \begin{tabbing} \hspace{2em}\= \hspace{3em}\= \hspace{3em}\=
\hspace{3em}\= \hspace{3em}\= \hspace{3em}\= \kill}{
 \end{tabbing}\endgroup}
\newenvironment{example2}{
 \begingroup \begin{tabbing} \hspace{8em}\= \hspace{2em}\= \hspace{2em}\=
\hspace{10em}\= \hspace{2em}\= \hspace{2em}\= \hspace{2em}\= \kill}{
 \end{tabbing}\endgroup}

\newenvironment{example}{
\begingroup  \begin{tabbing} \hspace{2em}\= \hspace{3em}\= \hspace{3em}\=
\hspace{3em}\= \hspace{3em}\= \hspace{3em}\= \hspace{3em}\= \hspace{3em}\=
\hspace{3em}\= \hspace{3em}\= \hspace{3em}\= \hspace{3em}\= \kill}{
 \end{tabbing} \endgroup }


\newcommand{\sand}{sand} \newcommand{\pand}{pand} \newcommand{\cor}{cor} 

\newcommand{\lb}{\langle}
\newcommand{\rb}{\rangle}
\newcommand{\pr}{prov}
\newcommand{\prG}{intp}
\newcommand{\prSG}{intp_E}
\newcommand{\intp}{intp_o}
\newcommand{\prove}{exec} \newcommand{\np}{invalid} \newcommand{\Ra}{\supset}
\newcommand{\add}{\sqcup} \newcommand{\adc}{\&} \newcommand{\Cscr}{{\cal C}}
\newcommand{\seqweb}{SProlog}
\newcommand{\sprog}{{SProlog}}

\newtheorem{theorem}[lemma]{Theorem}

\newtheorem{proposition}[lemma]{Proposition}

\newtheorem{corollary}[lemma]{Corollary}
\newenvironment{proof}
     {\begin{trivlist}\item[]{\it Proof. }}{\\* \hspace*{\fill} \end{trivlist}}

\newcommand{\seqand}{\prec}
\newcommand{\seqor}{\cup}
\newcommand{\seqandq}[2]{\prec_{#1}^{#2}}
\newcommand{\parandq}[2]{\land_{#1}^{#2}}
\newcommand{\exq}[2]{\exists_{#1}^{#2}}
\newcommand{\ext}{intp_G}
\renewcommand{\seqor}{:}



\begin{document}
\maketitle
\begin{summary}
To represent interactive objects,  we propose
 a choice-disjunctive declaration statement of the form
 where  are the (procedure or field) declaration statements within
 a class.
This statement has the
following semantics: request the user to choose one between  and  when an
object of this class is created.
  This statement is useful for representing interactive objects that require
interactions with the user.
\end{summary}
\begin{keywords}
interactions, object-oriented, computability logic.
\end{keywords}


\section{Introduction}\label{sec:intro}

Interactive  programming \cite{Lyn96,Rol10} is an important modern trend in information technology.
 Despite much popularity, object-oriented languages \cite{Avi03,Jos12,Jos08} have
traditionally lacked mechanisms  for representing interactive objects.
For example, an object like a lottery ticket is in a superposition state of
several possible values and require further interactions with the environment to determine
their final value.

To represent interactive objects,  we propose to adopt a choice-disjunctive
operator in computability logic \cite{Jap03,Jap08}. To be specific,
we allow, within a class definition,
 a choice-disjunctive declaration statement of the form
.
This statement has the
following semantics: request the user to choose one between   and  when
an object is created.
  This statement  is useful for representing interactive objects. For example,
a lottery ticket, declared as 0 \add\ value = \, indicates that it
has two possible values, nothing or one million dollars, and its final value will
be determined by the environment (or the user).




The remainder of this paper is structured as follows. We describe the new language 
 in
the next section. In Section \ref{sec:modules}, we
present some examples.
Section~\ref{sec:conc} concludes the paper.


\section{The Language}\label{sec:logic}

The language is a subset of the core (untyped) Java
 with some extensions. It is described
by - and -formulas given by the syntax rules below:
\begin{exmple}
\> \>    \\   \\
\> \>  \\

\end{exmple}
\noindent
In the rules above,   is an object name,  is a field name,  is an expression, and
  represents a procedure (or a method) of the form .
The notation  in  denotes an assignment statement.

A -formula  is called a class definition. The notation
 in  denotes a field  with an initial
value . The notation
 in  denotes a procedure declaration where  is called a procedure body.
The notation  denotes a conjunction of two -formulas.

In the transition system to be considered, -formulas will function as the
main program (or procedure bodies), and a set of tuples  where  is an object
name and  is a -formula will constitute  a program.

 We will  present an operational
semantics for this language via a proof theory. The rules  are formalized by means of what
it means to
execute the main task  from a program .
These rules in fact depend on the top-level
constructor in the expression,  a property known as
uniform provability\cite{MNPS91}. Below the notation  denotes
 but with the  tuple being distinguished
(marked for backchaining). Note that execution  alternates between
two phases: the goal-reduction phase (one  without a distinguished tuple)
and the backchaining phase (one with a distinguished tuple).
The notation  denotes the following: execute  and execute
 sequentially. It is considered a success if both executions succeed.
The notation   represent an association of  with every field or procedure
name appearing in .
For example, if  is , then   represents  .

\begin{defn}\label{def:semantics}
Let  be an object name, let  be a main task and let  be a program.
Then the notion of   executing  successfully and producing a new
program --  --
 is defined as follows:
\begin{numberedlist}

\item     if
  . \% matching procedure for  is found

\item     if   . \% argument passing

 \item     if   . \% looking for the procedure  in .

\item     if   . \% looking for  the procedure  in 


\item     if    and
. \% a procedure call in object 



\item   where
 is obtained from  by first evaluating  to  and updating
the value of the field  to  in the object .


\item    if   sand \\
  .


\item   where  is obtained from  by
first removing choice-disjunctions and then by initializing its fields. \% object creation






\end{numberedlist}
\end{defn}

\noindent
If  has no derivation, then the machine returns the failure.
In the above, the rules (1) to (4) deal with the backchaining phase, whereas the rules (5) to (8) deal
with the goal reduction phase.
Our operational semantics is a standard one appearing in most textbooks.
Only the rule (8) is a novel feature.



\section{Examples}\label{sec:modules}

Imagine Temple University charges \3,000 for employees.
An example of this class is provided by the
following program:


\begin{exmple}
 \\
 \\
\\
3000\  \hspace{10em} else\ tuition = \\\
\\
; \\
;\\

\end{exmple}
\noindent In the above, creating a TempleU object via the  construct
 basically proceeds as follows: the machine asks the user ``are you an employee?''.
If the user answers yes by choosing the left disjunct,  will be initialized to
 and the machine will eventually print \employeefalse5000 for its tuition.
Our language thus makes it possible to customize the amount for tuition via
interaction with the user.



\section{Conclusion}\label{sec:conc}

In this paper, we have considered an extension to the core Java with
disjunctive statements within a class definition. This extension allows statements of
the form    where  are statements.
These statements are
 particularly useful for representing interactive objects.




\section{Acknowledgements}

This work  was supported by Dong-A University Research Fund.


\bibliographystyle{ieicetr}\begin{thebibliography}{1}


\bibitem{Jap03}
G.~Japaridze, ``Introduction to computability logic'', Annals  of Pure and
 Applied  Logic, vol.123, pp.1--99, 2003.

\bibitem{Jap08}
G.~Japaridze,   ``Sequential operators in computability logic'',
 Information and Computation, vol.206, No.12, pp.1443-1475, 2008.

\bibitem{MNPS91}
D.~Miller, G.~Nadathur, F.~Pfenning, and A.~Scedrov, ``Uniform proofs as a
  foundation for logic programming,'' Annals of Pure and Applied Logic, vol.51,
  pp.125--157, 1991.

\bibitem{Lyn96}
Lynn Andrea Stein, ``Interactive programming: revolutionizing introductory computer science," ACM Comput. Surv. 28, 4es, Article No. 103, December 1996.

\bibitem{Rol10}
Roly Perera, ``First-order interactive programming," 12th international conference on Practical Aspects of Declarative Languages (PADL'10),
Springer-Verlag, Berlin, Heidelberg, pp. 186-200, Jan. 2010.

\bibitem{Avi03}
Avinash C. Kak, Programming with Objects: A Comparative Presentation of Object-Oriented Programming with C++ and Java,
John Wiley \& Sons, Inc., New York, NY, USA, 2003.

\bibitem{Jos12}
Joseph Albahari, and Ben Albanhari, C\# 5.0 Pocket Reference, O'Reilly Media, 224 pages, May 2012.

\bibitem{Jos08}
Joshua Bloch, Effective Java, Second Edition, Addison-Wesley, 346 pages, May 2008.

\end{thebibliography}



\end{document}
