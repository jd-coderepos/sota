\documentclass{llncs}
\usepackage{graphicx}
\usepackage{amsmath}
\usepackage{amssymb}
\usepackage{amsmath}
\usepackage{amssymb}

\newcommand{\s}{\{0,1\}}
\newcommand{\tx}{\textsf}
\pagestyle{plain}

\newtheorem{notat}{Notation}
\newtheorem{obs}{Observation}

\title{Balanced permutations Even-Mansour ciphers}
\author{Shoni Gilboa \inst {1}  \and Shay Gueron \inst {2, 3} \and Mridul Nandi \inst {4}}
\institute{
The Open University of Israel, Raanana 43107, Israel
\and
University of Haifa, Israel
\and
Intel Corporation, Israel Development Center, Israel
\and
Indian Statistical Institute, Kolkata
}

\begin{document}
\maketitle
\centerline{\today}

\begin{abstract}
The -rounds Even-Mansour block cipher is a generalization of the well known Even-Mansour block cipher to  iterations. 
Attacks on this construction were described by Nikoli{\'c} et al. and Dinur et al., for .
These attacks are only marginally better than brute force, but are based on an interesting observation (due to Nikoli{\'c} et al.): for a ``typical'' permutation , the distribution of   is not uniform.
This naturally raises the following question. Call permutations for which the distribution of  is uniform ``balanced''. 
Is there a sufficiently large family of balanced permutations, and what is the security of the resulting Even-Mansour block cipher?

We show how to generate families of balanced permutations from the Luby-Rackoff construction, and use them to define a -bit block cipher from the -rounds Even-Mansour scheme.
We prove that this cipher is indistinguishable from a random permutation of , for any adversary who has oracle access to the public permutations and to an encryption/decryption oracle, as long as the number of queries is . As a practical example, we discuss the properties and the performance of a -bit block cipher that is based on our construction, and uses AES as the public permutation.
\end{abstract}

{\small
\begin{quote}
\textbf{Keywords:} Even-Mansour, block-cipher, Luby-Rackoff
\end{quote}}

{\small
\begin{quote}
\textbf{Mathematics Subject Classification:} 94A60
\end{quote}}

\section{Introduction}\label{intro}
The -rounds Even-Mansour (EM) block cipher, suggested by Bogdanov et al. \cite{BKLSST}, encrypts an -bit plaintext  by 

where  are secret keys and  are publicly known permutations, which are selected uniformly and independently at random, from the set of permutations of . 
The confidentiality of the EM cipher is achieved even though the permutations  are made public. 
For , (\ref{eq:def_of_EM}) reduces to the classical Even-Mansour construction \cite{EM}.  

As a practical example, Bogdanov et al. defined the -bit block cipher AES, which is an instantiation of the -rounds EM cipher where the two public permutations are AES with two publicly known ``arbitrary'' keys (they chose the binary digits of the constant ). The complexity of the best (meet-in-the-middle) attack they showed uses  cipher revaluations. Consequently, they conjectured that AES offers -bit security.

Understanding the security of the EM cipher has been the topic of extended research. 
First, Even and Mansour \cite{EM} proved, for , that an adversary needs to make  oracle queries before he can decrypt a new message with high success probability. 
Daemen \cite{Daemen} showed that this bound is tight, by demonstrating a chosen-plaintext key-recovery attack after  evaluations of  and the encryption oracle.
Bogdanov et al. \cite{BKLSST}, showed, for the -rounds EM cipher, , that an adversary who sees only  chosen plaintext-ciphertext pairs cannot distinguish the encryption oracle from a random permutation of .
This result has been recently improved by Chen and Steinberger \cite{CS}, superseding intermediate progress made by Steinberger \cite{Steinberger} and by Lampe, Patarin and Seurin \cite{LPS}. They showed that for every , an adversary needs  chosen plaintext-ciphertext pairs before he can distinguish the -rounds EM cipher from a random permutation of . This bound is tight, by Bogdanov et al.'s \cite{BKLSST} distinguishing attack after  queries.

Nikoli{\'c} et al. \cite{NWW} demonstrated a chosen-plaintext key-recovery attack on the single key variant () of the -rounds EM cipher. Subsequently, Dinur et al. \cite{DDKS} produced additional key-recovery attacks on various other EM variants. 
All the attack in \cite{NWW} and \cite{DDKS} are only slightly better than a brute force approach. For example, the attack (\cite{DDKS}) on the single key variant of the -rounds EM cipher has time complexity , and the attack (\cite{DDKS}) on AES (with three different keys) has complexity of  
(still better than Bogdanov et al. \cite{BKLSST}, thus enough to invalidate their that AES has  security). 

The above attacks are based on the astute observation, made in \cite{NWW}, that for a "typical" permutation  of , the distribution of  over uniformly chosen  is not uniform. Currently, this observation yields only weak attacks, but the unveiled asymmetry may have the potential to lead to stronger results. 

This motivates the following question. 
Call a permutation  of  ``balanced'' if the distribution of , 
over uniformly chosen , is uniform. Can we construct a block cipher based on balanced permutations?
We point out that, a priori, it is not even clear that there exists a family of such permutations, that is large enough to support a block cipher construction. 

In this work, we show how to generate a large family of balanced permutations of , by observing that a -rounds Luby-Rackoff construction with any two identical \emph{permutations} of , always yields a balanced permutation (of ). We use these permutations in an EM setup 
(illustrated in Figure \ref{2rEM}, top panel), to construct a block cipher with block size of  bits. 
Note that in this EM setup, the permutations  are not 
chosen uniformly at random from the set of all permutations of . They are selected from a particular subset of the permutations of , and defined via a random choice of two 
permutations of , as the paper describes. 

For the security of the resulting  bits block cipher, we would ideally
like to maintain the security of the EM cipher (on blocks of  bits ). This would be guaranteed if we replaced the random permutation in the EM cipher, with an indifferentiable block cipher (as defined in \cite{MRH}). However, the balanced permutations we use in the EM construction are -rounds Luby-Rackoff permutations, and it was shown in \cite{CPS} that even the -rounds Luby-Rackoff construction does not satisfy indifferentiability. Therefore, it is reasonable to expect weaker security properties in our cipher. 
Indeed, we demonstrate a distinguishing  (not a key recovery) 
attack that uses  queries. On the other hand, we prove that a smaller number of chosen plaintext-ciphertext queries is not enough to distinguish the block cipher from a random permutation of . 

We comment that the combination of EM and Luby-Rackoff constructions have already been used and analyzed. 
Gentry and Ramzan \cite{GR} showed that the internal permutation of the Even-Mansour construction for -bits block size, can be securely replaced by a -rounds Luby-Rackoff scheme with public round functions. They proved that the resulting construction is secure up to  queries. 
Lampe and Seurin \cite{LS} discuss -rounds Luby-Rackoff 
constructions where the round functions are of the form ,  is a public random function, and  is a (secret) round key. For an even number of rounds, this can be seen as a 
-rounds EM construction, where the permutations are -rounds 
Luby-Rackoff permutations. 
They show that this construction is secure up to  queries, where  for non-adaptive chosen-plaintext adversaries, and  for adaptive chosen-plaintext and ciphertext adversaries.
These works bare some similarities to ours, but the new feature in our construction is the emergence of balanced permutations. 

The paper is organized as follows. In Section \ref{sec:Balanced} we discuss 
balanced permutations and balanced permutations EM ciphers. Section \ref{sec:prelim} provides general background for the security analysis given in Section \ref{sec:proof}. In Section \ref{sec:attack}, we demonstrate the distinguishing attack. 
A practical use of our construction is a -bit block cipher is based on AES. Section \ref{sec:practical} defines this cipher and discusses its performance characteristics. We conclude with a discussion in Section \ref{sec:discussion}.


\section{Balanced permutations and balanced permutations EM ciphers}
\label{sec:Balanced}

\subsection{Balanced permutations}

\begin{definition}[Balanced permutation\footnote{Also known as ``orthomorphism'' in the mathematical literature}] Let  be a permutation of . Define the function  by , for every . We say that  is a balanced permutation if  is also a permutation.
\end{definition}

\begin{example}
Let  be a matrix such that both  and  are invertible. Define  by . Then  is a balance permutation of .
One such matrix is defined by  for ,  and  for all other .
\end{example}

\begin{example}
Let  be an element of  such that . Identify  with , so that field addition corresponds to bitwise XOR. The field's multiplication is denoted by . The function  is a balanced permutation of .
Note that this example is actually a special case of the previous one.
\end{example}
The balanced permutations provided by the above examples are a small family of permutations, and moreover are all linear. We now give a recipe for generating a large family of balanced permutations, by employing the Feistel construction that turns any function  to a permutation of .

Let us use the following notation. For a string , denote the string of its first  bits by , and the string of its last  bits by . Denote the concatenation of two strings  (in this order) by . We have the following identities:


\begin{definition}[Luby-Rackoff permutations]
Let  be a function. Let  be the Luby-Rackoff (a.k.a Feistel) permutation

For every  and  functions , we define the -rounds Luby-Rackoff permutation to be the composition

Since we use here extensively the special case , we denote it by .
\end{definition}

The following proposition shows that when  is, itself, a permutation, then  is a balanced permutation.

\begin{proposition}
Let  be a permutation of . Then, the -rounds Luby-Rackoff permutation, , is a balanced permutation of .
\label{2_Feistel_Rounds}
\end{proposition}

\begin{proof} Denote . Observe first that

Therefore,

Assume that  such that , i.e.,

Then,  and .
Since (by assumption)  is one-to-one,  and , it follows that
.
We established that  implies  which concludes the proof.
\end{proof}
Figure \ref{2rFeistel} shows an illustration of -rounds Luby-Rackoff (balanced) permutation.

\begin{figure}[ht!]
\centering
\includegraphics[width=100mm]{2rFeistel.jpg}
\caption{
The figure shows a function from  to , based on two Feistel rounds with a function . For any function , this construction is a permutation
of , denoted . We call it a ``-rounds Luby-Rackoff permutation''.
Proposition \ref{2_Feistel_Rounds} shows that if  itself is a {\it permutation} of , then
 is a balanced permutation of .}
\label{2rFeistel}
\end{figure}

\subsection{Balanced permutations EM ciphers}

\begin{definition}[-rounds balanced permutations EM ciphers (BPEM)]
Let  and  be integers. Let  be  strings in . Let ,,,  be  permutations of . Their associated -rounds Luby-Rackoff (balanced) permutations (of ) are , respectively. The -rounds balanced permutations EM (BPEM) block cipher is  defined as

(where  is defined by \eqref{eq:def_of_EM}).
It encrypts -bit blocks with an -rounds EM cipher with the keys , where the  permutations  (of ) are set to be  , respectively.
\end{definition}

The use of the -rounds BPEM cipher for encryption (and decryption) starts with an initialization step, where the permutations  are selected uniformly and independently, at random from the set of permutations of . After they are selected, they can be made public. Subsequently, per session/message, the secret keys  are selected uniformly and independently, at random, from . Figure \ref{2rEM} illustrates a -rounds BPEM cipher 
, which is the focus of this paper.

\begin{remark}
\label{rem:linear}
The -rounds EM cipher is not necessarily secure with {\it any} choice of balanced permutations as . For example, it can be easily broken when using the linear balanced permutations shown in Examples 1 and 2.
\end{remark}

\begin{remark}\label{rem:2LR}
In our construction, the permutations  are not random permutations. Therefore, the security analysis of the ``classical" EM does not apply, and the resulting cipher (BPEM) may not be secure. Indeed, it is easy to see that the -round BPEM does not provide confidentiality. For any plaintexts , we have, by \eqref{feistel2},

Therefore, by \eqref{eq:def_of_BPEM}, \eqref{eq:def_of_EM} and \eqref{feistel2},

It follows that if, e.g.,  then
 which means that the ciphertexts leak out information on .
This also implies that the -rounds BPEM cipher must be used with  to have any hope for achieving security.
\end{remark}

\begin{remark}\label{rem:immune}
By construction,  () is immune
against any attack that tries to leverage the non-uniformity of  (including \cite{NWW} and \cite{DDKS})). Obviously, this does not guarantee it is secure (as indicated in Remark \ref{rem:linear}).
\end{remark}

In Section \ref{sec:proof} we prove that the -round BPEM cipher is indistinguishable from a random permutation.

\begin{figure}[ht!]
\centering
\includegraphics[width=0.50\textwidth]{EM_0_New1.jpg}
\includegraphics[width=0.99\textwidth]{EM_1_New1.jpg}
\vspace{-1.5cm}
\caption{The -rounds balanced permutations EM (BPEM) cipher operates on blocks of size  bits.
The permutations  and  are balanced permutations of , defined as -rounds Luby-Rackoff permutations.  and  are two (public) permutations of .
Each of  is a -bit secret key. See explanation in the text.}
\label{2rEM}
\end{figure}

\subsection{Equivalent representation of BPEM in terms of LR}\label{sec:BPEM_LR}
In this section we show that 2-rounds BPEM can be viewed as a ``keyed"\footnote{By ``keyed" we mean that each function used in the Luby-Rackoff construction is selected from a family of functions indexed by a key.} Luby-Rackoff cipher (in fact, the -rounds BPEM has a similar representation for every ).
\begin{notat}\label{notat:f_oplus_K}
Given a function  and a key  we denote  by , namely

\end{notat}
\begin{lemma}\label{lem:BPEM=LR}
Let  and let  be two permutations of . Then,

where

\end{lemma}
\begin{proof}
 For every function ,  and  we have, by \eqref{feistel},

and hence

In particular

and then

Therefore, by \eqref{eq:def_of_BPEM} and \eqref{eq:def_of_EM},

\end{proof}

\section{Security preliminaries and definitions}
\label{sec:prelim}
Let  be an oracle adversary which interacts with one or more oracles.
Suppose that  and  are two oracles (or a tuple of oracles) with same domain and range spaces. We define the distinguishing advantage of  distinguishing  and  as

The maximum advantage  over all adversaries with complexity  (which includes query, time complexities etc.) is denoted by . When we consider computationally unbounded adversaries (which is done in this paper), the time and memory parameters are not present and so we only consider query complexities.
In the case of a single oracle,  is the number of queries, and in the case of a tuple of oracles,  would be of the form  where  denotes the number of queries to the  oracle.
While we define security advantages of , we usually choose  to be an ideal candidate, such as the random permutation  or a random function.
The PRP-advantage of  against a keyed construction  is . The maximum PRP-advantage with query complexity  is denoted as .

In this paper, we always assume that queries to an oracle  are allowed in both directions, i.e., to  as well. We denote

The SPRP-advantage of a keyed construction  (where the adversary has access to both the encryption  and its decryption ) is defined by 
When a construction  is based on one or more ideal permutations or random permutations  and a key , we define SPRP-advantage of a distinguisher , in presence of ideal candidates,
as  where  is sampled independently of . We denote the maximum advantage by  which we call SPRP-advantage in the ideal model.
The complexity parameters of the above advantages depend on the number of oracles, and will be explicitly declared in specific instances.

We state two simple observations on the distinguishing advantages for oracles (we skip the proofs of these observations, as these are straightforward).
\begin{obs}\label{obs1}
If ,  and  are three independent oracles, then

\end{obs}
\begin{obs}\label{obs2}
If  is an oracle construction, then (by using standard reduction)

(where  is the number of queries to , needed to simulate one query to the construction ).
\end{obs}

Note that in the Observation \ref{obs2}, we do not need to assume any kind of independence of the oracles. Analogous observations, up to obvious changes, hold for the case where  are tuples of oracles.

\subsection{Coefficient-H technique}
Patarin's coefficient-H technique \cite{Patarin1} (see also \cite{Patarin3}) is a tool for showing an upper bound for the distinguishing advantage. Here is the basic result of the technique.

\begin{theorem}[Patarin \cite{Patarin1}]
\label{fact:H}
Let  and  be two oracle algorithms with domain  and range . Suppose there exist a set  and  such that the following conditions hold:
\begin{enumerate}
\item
For all , , ,

(the above probabilities are called interpolation probabilities).
\item
For all  making at most  queries to ,  where ,  and  denote the   query and response of  to .
\end{enumerate}
Then,

\end{theorem}

The above result can be applied for more than one oracle. In such cases we split the parameter  into  where  denotes the maximum number of queries to the  oracle. Moreover, if we have an oracle  and its inverse  then the interpolation probability for both  and  can be simply expressed through the interpolation probability of  only. For example, if an adversary makes a query  to  and obtains the response , we can write .
Therefore, under the conditions of Theorem \ref{fact:H} we also have .

\subsection{Known related results}\label{subsection3_2}

\subsubsection{The security of Even-Mansour cipher} 
It is known that the Even-Mansour cipher  is SPRP secure in the ideal model, in the following sense:  . The same is true for the single key variant .
In Section \ref{sec:proof}, we provide (using Patarin's coefficient-H technique) a simple proof of this result (Lemma \ref{lemma2}) and a more general result (Lemma \ref{lemma3}).

\subsubsection{The security of Luby-Rackoff encryption}
The -rounds Luby-Rackoff construction is PRP secure and the -rounds Luby-Rackoff construction is SPRP secure ,
when the underlying functions  are PRP's (or PRF's).
We use the following quantified version of the SPRP security of the -rounds case.
\begin{theorem}[Piret \cite{Piret}]
\label{Piret}
Let  be four independent random permutations of
, and let  be a random permutation of .
Then,  is SPRP  secure in the following sense:

\end{theorem}

The above bound  is tight (see \cite{TP}).
In the proof of Theorem \ref{thm:indistinguish_1key_1perm} we use the following, more general, result.
\begin{theorem}[Nandi \cite{Nandi}]
\label{Nandi}
Let , and let   be a sequence of numbers from  such that . Let  be  independent random permutations of , and let  be a random permutation of . Then,  is  SPRP  secure in the following sense:

\end{theorem}

\section{Security analysis of our construction}\label{sec:proof}

\subsection{Security analysis of tuples of single key -round  EM cipher}

\begin{notat}
Let .
We use  to indicate the existence of a collision, i.e., that
 for some . Otherwise, we write , and say that the tuple  is block-wise distinct. Given a function  and a tuple  we define 
For positive integers , denote

\end{notat}

\begin{obs}
For every ,  and a uniform random permutation  on ,

More generally, let  be independent uniform random permutations over  then for every block-wise distinct tuples ,  we have

\end{obs}

Now we show that for a random permutation  of  and a uniformly chosen , the permutation  (single keyed -round EM, see Notation \ref{notat:f_oplus_K}) is SPRP secure in the ideal model.
\begin{lemma}
\label{lemma2}
Let  and  be independent random permutations of . Then

\end{lemma}
\begin{proof} We use Patarin's coefficient H-technique. We take the set of bad views  to be the empty set.
We need to show that for every tuples , ,

where .
With no loss of generality we may assume that each of the tuples  is block-wise distinct. Then, by \eqref{eq:rp-general},

We say that a key  is ``good'' if   and  for all , . In other words, for a good key all the inputs (outputs) of  (in the  computation) are block-wise distinct. Thus, for any given good key ,

By a simple counting argument, the number of good keys is at least , i.e., the probability that a randomly chosen key is good, is at least , where . Therefore, we have

and the result follows by Theorem \ref{fact:H}.
\end{proof}

Now, we extend Lemma \ref{lemma2} to a tuple  of single key -round EM encryptions, where some keys and permutations can be repeated.
\begin{lemma}\label{lemma3}
Let  be independent random permutations of  and  be chosen uniformly and independently from . We write  to denote . Let  and   be a sequence of elements from  and , respectively, such that 's are distinct. Then, for any  (specifying the maximum number of queries for each permutation) we have

where

and  for every .
\end{lemma}
\begin{proof}
The proof is similar to the proof of Lemma \ref{lemma2}. Let , , , , be block-wise distinct tuples. From \eqref{eq:rp-general}, we have that

We say that a tuple of keys  is ``bad'' if one of the following holds:
\begin{enumerate}
 \item
There are , ,
 such that
,
, and
.
 \item
There are , ,

such that
.
 \item
There are , ,
 such that
, , and
.
 \item
There are , ,

such that
.
\end{enumerate}
Note that there are at most 
cases in the first and in the third items, and at most
 cases in the second and fourth items.

If a key tuple is not bad, we say that it is a ``good'' key tuple.
As in the proof of Lemma \ref{lemma2},
for a good key tuple all the inputs (outputs) of each permutation are distinct.
Thus, given a good tuple of keys , it is easy to see that

It now remains to bound the probability that a random key tuple is bad. This can happen with one of the cases listed in items 1-4 where each case has probability  to occur. Hence, the probability that a random key tuple is bad, is at most
, and the probability that a random key tuple is good is therefore at least . The result follows by Theorem \ref{fact:H}.
\end{proof}

\subsection{Main theorems}
\begin{theorem}\label{thm:indistinguish}
Consider the BPEM cipher  where the (secret) keys
 are selected uniformly and independently at random.
Let  be the maximum number of queries to the encryption/decryption oracle, and let  be the maximum numbers of queries to the public permutations  and , respectively. Then,

\end{theorem}

\begin{proof}
By Lemma \ref{lem:BPEM=LR}, we know that our BPEM construction is same as 
where  are defined via \eqref{matrix1} by . The matrix in \eqref{matrix1} is lower triangular with non-zero diagonal, and hence non-singular. Thus, the ``new" keys  are also distributed uniformly and independently. As  are independent of all the ``ingredients" of  , it suffices to prove our result without the keys  and .

Let  be random permutations of  and let  be a random permutation of , all are independent of each other and independent of
).
Denote   and  .
By Observation \ref{obs2} and Lemma \ref{lemma3}, we have\footnote{
Note that each query to the oracle construction  translates to four queries - one to each permutation
 , }

Finally, by applying the triangle inequality,  Observation \ref{obs1} and Theorem \ref{Piret}, the SPRP-advantage in the ideal model is

\end{proof}

The same argument can be used to show a similar bound for the single permutation -rounds BPEM cipher.
\begin{theorem}
\label{thm:indistinguish_1perm}
Consider the single permutation BPEM cipher  where the (secret) keys
 are selected uniformly and independently at random.
Let  be the maximum number of queries to the encryption/decryption oracle, and let  be the maximum number  of queries to the public permutation . Then,

\end{theorem}
\begin{remark}
The difference in the bounds we received in Theorems \ref{thm:indistinguish} and \ref{thm:indistinguish_1perm}  are due only to the difference in the value of  in the application of Lemma \ref{lemma3}.
\end{remark}

We also comment that the same bounds hold in the single key variants. By \eqref{matrix1} we have

where
\begin{center}
.
\end{center}

For both constructions, the ``new" keys  are no longer independent, so we need to generalize lemma \ref{lemma3} as stated below.

\begin{lemma}\label{lemma4}
Let  be independent random permutations of  and  be chosen uniformly and independently from .
We write  to denote .
Let  be a sequence of elements from .
Let  be a binary matrix of size , with no zero rows, satisfying the following condition: for every  such that , the  and  rows of  are distinct.
Let , for every  .

Then, for any  (specifying the maximum number of queries) we have
 where

and  is as defined in Lemma \ref{lemma3}.
\end{lemma}

We skip the proof of this lemma as it is similar to that of Lemma \ref{lemma3}.
Similarly to the proof of Theorem \ref{thm:indistinguish} (while using Lemma \ref{lemma4} instead of Lemma \ref{lemma3}), we can obtain the following bound.
\begin{theorem}
\label{thm:indistinguish_1key}
Consider the single key BPEM cipher  where the (secret) key
 is selected uniformly at random.
Let  be the maximum number of queries to the encryption/decryption oracle, and let  be the maximum numbers of queries to the public permutations  and , respectively. Then, 
\end{theorem}

Finally, similarly to the proof of Theorem \ref{thm:indistinguish_1perm} (while using Lemma \ref{lemma4} instead of Lemma \ref{lemma3}, and using Theorem \ref{Nandi} instead of Theorem \ref{Piret}), we obtain the following bound.
\begin{theorem}
\label{thm:indistinguish_1key_1perm}
Consider the single key single permutation BPEM cipher  where the (secret) key
 is selected uniformly at random.
Let  be the maximum number of queries to the encryption/decryption oracle, and let  be the maximum number  of queries to the public permutation . Then,

\end{theorem}

\section{A distinguishing attack on BPEM}\label{sec:attack}

In this section we describe a distinguishing attack on BPEM that uses  queries. This is the same attack as the one described in 
\cite[Section 3.2]{TP} for the -rounds Luby-Rackoff with internal permutations, not at all surprising, since we showed (in Section \ref{sec:BPEM_LR}) that BPEM can be viewed as a -rounds Luby-Rackoff with internal (keyed) permutations. Nevertheless, for the sake of completeness, we describe and analyze the attack in this BPEM terminology.
We will use the following technical lemma.
\begin{lemma}\label{attack_lemma}
If  such that

then .
\end{lemma}
\begin{proof}
Denote

By \eqref{eq:def_of_BPEM} and \eqref{eq:def_of_EM} we have that

hence, by \eqref{feistel2},

Similarly

Therefore we get from \eqref{attack_lemma_given} that , hence, since  is injective, . Therefore, using \eqref{feistel2} again,

hence

Since  is injective we get that

hence .
\end{proof}
\begin{proposition}\label{propos:attack}
Consider the BPEM cipher  with arbitrary (secret) keys
.
Let  be the maximum number of queries to the encryption oracle. Then,

\end{proposition}
\begin{remark}
Note that Proposition \ref{propos:attack} implies that the adversary advantage becomes non-negligible for .
\end{remark}
\begin{proof}
 Fix an -bit string  and  distinct -bit strings . We query the encryption oracle for the plaintexts , and let  be the corresponding ciphertexts. We now search for collisions between the  -bit strings .
By Lemma \ref{attack_lemma} there will be no collision if the oracle encrypts using . By contrast, if the oracle encrypts by applying a randomly chosen permutation of  then the probability there is no collision is at most

\end{proof}

\section{A practical constructions of a -bit cipher}
\label{sec:practical}

In this section, we demonstrate a practical construction of a -bit block cipher based on the -rounds BPEM cipher, where the underlying permutation is AES.

\begin{definition}[: a -bit block cipher]
Let  and  be two -bit keys and let  be three -bit secret keys (assume  are selected uniformly and independently at random). Let the permutations  and  be the AES encryption using  and  as the AES key, respectively. \\
The 256-bit block cipher  is defined as the associated instantiation of the -rounds BPEM cipher . \\
Usage of :

\begin{itemize}
\item []
 and  are determined during the setup phase, and can be made public (e.g., sent from the sender to the receiver as an IV).
\item []
 are selected per encryption session.
\end{itemize}
The single key EM256AES is the special case where a single value  and a single value  are selected uniformly and independently at random, and the EM256AES cipher uses  and .
\end{definition}

Hereafter, we use the single key EM256AES. To establish security properties for , we make the standard assumption about AES with a secret key that is selected (uniformly at random): an adversary has negligible advantage in distinguishing AES from a random permutation of  even after seeing a (very) large number of plaintext-ciphertext pairs (i.e., the assumption is that AES satisfies its design goals (\cite{NIST_request}, Section 4).
This assumption is certainly reasonable if the number of blocks that are encrypted with the same keys is limited to be much smaller than .
Therefore, in our context, we can consider assigning the randomly selected key  at setup time, as an approximation for a random selection of the permutation .
Under this assumption, we can rely on the result of Theorem \ref{thm:indistinguish_1key_1perm} for the security of .

\subsubsection{ efficiency:}
An encryption session between two parties requires exchanging a -bit secret key and transmitting a -bit IV (). One key (and IV) can be used for  blocks as long as we keep . \\
Computing one (-bit) ciphertext involves  AES computations (with the  as the AES key) plus a few much cheaper XOR operations. Let us assume that the encryption is executed on a platform that has the capability of computing AES at some level of performance. If the  encryption (decryption) is done in a serial mode, we can estimate the encryption rate to be roughly half the rate of AES (serial) computation on that platform ( AES operations per one -bit block). Similarly, if the  encryption is done in a parallelized mode, we can estimate the throughput to be roughly half the throughput of AES.

\subsubsection{ performance:}
To test the actual performance of , and validate our predictions, we coded an optimized implementation of \\
. Its performance is reported here. \\
The performance was measured on an Intel� Core� i7-4700MQ (microarchitecture Codename Haswell) where the enhancements (Intel� Turbo Boost Technology, Intel� Hyper-Threading Technology, and Enhanced Intel Speedstep� Technology) were disabled. The code used the AES instructions (AES-NI) that are available on such modern processors. \\
On this platform, we point out the following baseline: the performance of AES (-bit key) in a parallelized mode (CTR) is  C/B, and in a serial mode (CBC) it is  cycles per byte (C/B hereafter). \\
The measured performance of our  implementation was  C/B for the parallel mode, and  C/B for the serial mode. The measured performance clearly matches the predictions. \\
It is also interesting to compare the performance of  to another -bit cipher. To this end, we prepared an implementation of Rijndael256 cipher \cite{Rijndael256}
\footnote[2]{AES is based on the Rijndael block cipher. While AES standardizes only a  block size, the Rijndael definitions support both -bit and -bit blocks}. For details on how to code Rijndael256 with AES-NI, see \cite{Gueron_WP}). Rijndael256 (in ECB mode) turned out to be much slower than , performing at  C/B.

\section{Discussion}\label{sec:discussion}

In this work, we showed how to construct a large family of balanced permutations, and analyzed the resulting new variation, BPEM, of the EM cipher. 

The resulting -bit block cipher is obtained by using a permutation of  as a primitive.
The computational cost of encrypting (decrypting) one -bit block is  evaluations of a permutation of  (plus a relatively small overhead).
Note that this makes BPEM readily useful in practice, for example to define a -bit cipher, because ``good'' permutations of  are available. We demonstrated the specific cipher , which is based on AES, and showed that its throughput is (only) half the throughput of AES (and  times faster than Rijndael256).

A variation on the way by which BPEM can be used, would make it a tweakable -bit block cipher. Here, the public IV (=) can be associated with each encrypted block as an identifier, to be viewed as the tweak. The implementation would switch this tweak for each block. To randomize the keys for the (public) permutations, an additional encryption (using some secret key) is necessary.

The expression of the advantage in Theorem \ref{thm:indistinguish} behaves linearly with the number of queries to the public permutations, and quadratically with the number of queries to the encryption/decryption oracle.
This reflects the intuition that the essential limitations on the number of adversary queries should be on the encryption/decryption invocations, while weaker (or perhaps no) limitations should be imposed on the number of queries to the public permutations. 
It also suggests the following protocol, where the secret keys are changed more frequently than the random permutations.
{\it 
Choose the public permutations for a period of, say,  blocks, divided into  sessions of  blocks. Change the secret keys every session.}
This way, the relevant information on the block cipher, from a specific choice of keys, is limited to a session, while the adversary can accumulate relevant information from replies to the public permutations across sessions. Therefore,  is limited to , while  is limited to . Theorem \ref{thm:indistinguish} guarantees that this usage is secure.


\begin{thebibliography}{99}

\bibitem{NIST_request}
 \textendash, Announcing request for candidate algorithm nominations for the Advanced Encryption Standard (AES),
http://csrc.nist.gov/CryptoToolkit/aes/pre-round1/aes\textunderscore 9709.htm (1997).

\bibitem{BKLSST}
A. Bogdanov, L. R. Knudsen, G. Leander, F-X Standaert, J. P. Steinberger,\ and\ E. Tischhauser, Key-alternating ciphers in a provable setting: encryption using a small number of public permutations (extended abstract), in {\it Advances in cryptology---EUROCRYPT 2012}, Lecture Notes in Comput. Sci., 7237, Springer, Heidelberg, 45--62 (2012).

\bibitem{CS}
S. Chen\ and\  J. Steinberger, Tight security bounds for key-alternating ciphers, in {\it Advances in cryptology---EUROCRYPT 2014}, 327--350, Lecture Notes in Comput. Sci., 8441, Springer, Heidelberg, 2014.

\bibitem{CPS} 
J.-S. Coron, J. Patarin, Y. Seurin, The random oracle model and the ideal cipher model are equivalent, in {\it Advances in cryptology---CRYPTO 2008}, 1--20, Lecture Notes in Comput. Sci., 5157, Springer, Berlin. 

\bibitem{Daemen}
J. Daemen, Limitations of the Even-Mansour construction, in
{\it Advances in cryptology---ASIACRYPT '91}, Lecture Notes in Computer Science, 739, Springer, Berlin, (H. Imai, R. L. Rivest, T. Matsumoto, editors) 495-498 (1993).

\bibitem{Rijndael256}
J. Daemen, V. Rijmen, AES Proposal: Rijndael (National Institute of Standards and Technology),
http://csrc.nist.gov/archive/aes/rijndael/Rijndael-ammended.pdf (1999).

\bibitem{DDKS}
I. Dinur, O. Dunkelman, N. Keller, A. Shamir, Key Recovery Attacks on 3-round Even-Mansour, 8-step LED-128, and full AES, http://eprint.iacr.org/2013/391 (2013).

\bibitem{EM}
S. Even, Y. Mansour, A construction of a cipher from a single pseudorandom permutation, J. Cryptology {\bf 10} (1997), no.~3, 151--161.

\bibitem{GR} 
C. Gentry, Z. Ramzan, Eliminating random permutation oracles in the Even-Mansour cipher, in {\it Advances in cryptology---ASIACRYPT 2004}, 32--47, Lecture Notes in Comput. Sci., 3329, Springer, Berlin.

\bibitem{Gueron_WP}
S. Gueron, Intel Advanced Encryption Standard (AES) Instructions Set (Rev 3.01),
http://software.intel.com/sites/default/files/article/165683/aes-wp-2012-09-22-v01.pdf (2014).

\bibitem{LS}
R. Lampe, Y. Seurin, 
Security Analysis of Key-Alternating Feistel Ciphers,
in {\it Fast Software Encryption --- FSE 2014}, 243--264, Lecture Notes in Comput. Sci., 8540, (2014).

\bibitem{LPS}
R. Lampe, J. Patarin, Y. Seurin, An Asymptotically Tight Security Analysis of the Iterated Even-Mansour
Cipher, in {\it Advances in cryptology---ASIACRYPT 2012}, 278-295, Lecture Notes in Comput. Sci., 7658, (2012).

\bibitem{MRH}
U. Maurer, R. Renner, C. Holenstein, Indifferentiability, impossibility results on reductions, and applications to the random oracle methodology, in {\it Theory of cryptography}, 21--39, Lecture Notes in Comput. Sci., 2951, Springer, Berlin.

\bibitem{Nandi}
M. Nandi, The characterization of Luby-Rackoff and its optimum single-key variants, in {\it Progress in cryptology---INDOCRYPT 2010}, 82--97, Lecture Notes in Comput. Sci., 6498, Springer, Berlin.

\bibitem{NWW}
I. Nikoli{\'c}, L. Wang, S. Wu, Cryptanalysis of Round-Reduced LED. In {\it FSE}, 2013. To appear in Lecture Notes in Computer Science.

\bibitem{Patarin1}
J. Patarin, {\it \'Etude des g\'en\'erateurs de permutations pseudo-al\'eatoires bas\'es sur le sch\'ema du D.E.S}, INRIA, Rocquencourt, 1991.

\bibitem{Patarin3}
J. Patarin, Luby-Rackoff: 7 rounds are enough for  security, in {\it Advances in cryptology---CRYPTO 2003}, 513--529, Lecture Notes in Comput. Sci., 2729, Springer, Berlin.

\bibitem{Piret}
G. Piret, Luby-Rackoff revisited: on the use of permutations as inner functions of a Feistel scheme, Des. Codes Cryptogr. {\bf 39} (2006), no.~2, 233--245.

\bibitem{Steinberger}
J. P. Steinberger,
Improved Security Bounds for Key-Alternating Ciphers via Hellinger Distance,
IACR Cryptology ePrint Archive 2012: 481 (2012).

\bibitem{TP}
J. Treger, J. Patarin, Generic attacks on Feistel networks with internal permutations, in {\it Progress in cryptology---AFRICACRYPT 2009}, 41--59, Lecture Notes in Comput. Sci., 5580, Springer, Berlin. 


\end{thebibliography}

\end{document}
