\documentclass[11pt,a4paper]{article}
\usepackage[hyperref]{style/emnlp2020}
\usepackage{times}
\usepackage{latexsym}
\renewcommand{\UrlFont}{\ttfamily\small}

\usepackage{microtype}

\aclfinalcopy \def\aclpaperid{361} 

\setlength\titlebox{5cm}

\title{Understanding tables with intermediate pre-training}

\author{Julian Martin Eisenschlos, Syrine Krichene, Thomas M{\"u}ller \\ \\
  Google Research, Z{\"u}rich \\
  \texttt{\{eisenjulian,syrinekrichene,thomasmueller\}@google.com}}

\date{}





\usepackage{amsthm}



\usepackage{framed}
\usepackage{mdwlist}
\usepackage{colortbl}
\usepackage{xcolor}
\usepackage{nicefrac}
\usepackage{booktabs}
\usepackage{amsfonts}
\usepackage[T1]{fontenc}
\usepackage{bold-extra}
\usepackage{amsmath}
\usepackage{amssymb}
\usepackage{bm}
\usepackage{graphicx}
\usepackage{mathtools}
\usepackage{multirow}
\usepackage{multicol}
\usepackage[normalem]{ulem}
\usepackage{lipsum}
\usepackage{float}
\usepackage{ifthen}
\usepackage[ruled,vlined]{algorithm2e}

\usepackage{latexsym,comment}
\usepackage{tikz}
\usepackage{xspace}
\usepackage{xr}
\usetikzlibrary{shapes,arrows}

\newcommand{\breakalign}{\right. \nonumber \\ & \left. \hspace{2cm}}

\newcommand{\feat}[1]{{\small \texttt{#1}}}
\newcommand{\act}[1]{{\small \texttt{#1}}}
\newcommand{\ngram}[0]{-gram}
\newcommand{\topic}[1]{\underline{#1}}
\newcommand{\gem}[1]{\mbox{\textsc{gem}}}
\newcommand{\abr}[1]{\textsc{#1}}
\newcommand{\camelabr}[2]{{\small #1}{\textsc{#2}}}
\newcommand{\abrcamel}[2]{{\textsc #1}{\small{#2}}}
\newcommand{\grammar}[1]{{\color{red} #1}}
\newcommand{\explain}[2]{\underbrace{#2}_{\mbox{\footnotesize{#1}}}}
\newcommand{\dir}[1]{\mbox{Dir}(#1)}
\newcommand{\bet}[1]{\mbox{Beta}(#1)}
\newcommand{\py}[1]{\mbox{\textsc{py}}(#1)}
\newcommand{\td}[2]{\mbox{\textsc{TreeDist}}_{#1} \left( #2 \right)}
\newcommand{\yield}[1]{\mbox{\textsc{Yield}} \left( #1 \right)}
\newcommand{\mult}[1]{\mbox{Mult}( #1)}
\newcommand{\wn}{\textsc{WordNet}}
\newcommand{\twentynews}{\textsc{20news}}
\newcommand{\g}{\, | \,}
\newcommand{\Gam}[1]{\Gamma \left( \textstyle #1 \right)}
\newcommand{\LG}[1]{\log \Gamma \left( \textstyle #1 \right)}
\newcommand{\Pois}[1]{\mbox{Poisson}(#1)}
\newcommand{\pcfg}[3]{#1_{#2 \rightarrow #3}}
\newcommand{\grule}[2]{#1 \rightarrow #2}
\newcommand{\kl}[2]{D_{\mbox{\textsc{KL}}} \left( #1 \,||\, #2 \right)}

\newcommand{\digambig}[1]{\Psi \left( #1 \right) }
\newcommand{\digam}[1]{\Psi \left( \textstyle #1 \right) }
\newcommand{\ddigam}[1]{\Psi' \left( \textstyle #1 \right) }

\DeclareMathOperator*{\argmax}{arg\,max}
\DeclareMathOperator*{\argmin}{arg\,min}
\newcommand{\bmat}[1]{\text{\textbf{#1}}}
\newcommand{\bvec}[1]{\boldsymbol{#1}}

\newcommand{\emaillink}[1]{ {\small \href{mailto://#1}{\texttt{#1}}}}
\newcommand{\smallemaillink}[2]{ {\small \href{mailto://#2}{\texttt{#1}}}}

\newcommand{\ch}[1]{\begin{CJK*}{UTF8}{gbsn}#1\end{CJK*}}

\newcommand{\e}[2]{\mathbb{E}_{#1}\left[ #2 \right] }
\newcommand{\h}[2]{\mathbb{H}_{#1}\left[ #2 \right] }
\newcommand{\ind}[1]{\mathds{1}\left[ #1 \right] }
\newcommand{\ex}[1]{\mbox{exp}\left\{ #1\right\} }
\newcommand{\D}[2]{\frac{\partial #1}{\partial #2}}
\newcommand{\elbo}{\mathcal{L}}

\newcommand{\hidetext}[1]{}
\newcommand{\ignore}[1]{}

\newcommand{\todo}[1]{\textcolor{red}{{\bf TODO: #1}}}

\newif\ifcomment\commenttrue

\newif\ifinfotabs\infotabsfalse

\newif\ifsubscripterror\subscripterrortrue

\ifcomment
\newcommand{\pinaforecomment}[3]{\colorbox{#1}{\parbox{.8\linewidth}{#2: #3}}}
\else
\newcommand{\pinaforecomment}[3]{}
\fi

\ifinfotabs
\newcommand{\infotabstext}[1]{#1}
\else
\newcommand{\infotabstext}[1]{}
\fi

\ifsubscripterror
\newcommand{\err}[1]{\textsubscript{~#1}}
\else
\newcommand{\err}[1]{  #1}
\fi


\newcommand{\juliancomment}[1]{\pinaforecomment{yellow}{Julian}{#1}}
\newcommand{\thomascomment}[1]{\pinaforecomment{lightblue}{Thomas}{#1}}
\newcommand{\syrinecomment}[1]{\pinaforecomment{orange}{Syrine}{#1}}
\newcommand{\yasemincomment}[1]{\pinaforecomment{cyan}{Yasemin}{#1}}
\newcommand{\jbgcomment}[1]{\pinaforecomment{red}{JBG}{#1}}
\newcommand{\reviewercomment}[1]{\pinaforecomment{blue}{Reviewer}{#1}}
\newcommand{\jszcomment}[1]{\pinaforecomment{green}{JSG}{#1}}
\newcommand{\halcomment}[1]{\pinaforecomment{magenta!20}{Hal}{#1}}

\newcommand{\smalltt}[1]{ {\tt \small #1 }}
\newcommand{\smallurl}[1]{ \begin{tiny}\url{#1}\end{tiny}}
\newenvironment{compactenum}{ \begin{enumerate*} \small }{ \end{enumerate*} }

\definecolor{lightblue}{HTML}{3cc7ea}
\definecolor{CUgold}{HTML}{CFB87C}
\definecolor{grey}{rgb}{0.95,0.95,0.95}
\definecolor{ceil}{rgb}{0.57, 0.63, 0.81}
\definecolor{UMDred}{HTML}{ed1c24}
\definecolor{UMDyellow}{HTML}{ffc20e}
\definecolor{darkgreen}{HTML}{008f00}


\newcommand{\qb}[0]{Quizbowl}
\newcommand{\qa}[0]{\abr{qa}}
\newcommand{\triviaqa}{\camelabr{Trivia}{qa}}
\newcommand{\qblink}{\abrcamel{qb}{Link}}
\newcommand{\qanta}{\textsc{qanta}}
\newcommand{\muse}{\textsc{muse}}
\newcommand{\squad}{\textsc{sq}{\small u}\textsc{ad}}
\newcommand{\elmo}{\textsc{elm}{\small o}}
\newcommand{\sqa}{\textsc{SQA}\xspace}
\newcommand{\tabfact}{\textsc{TabFact}\xspace}
\newcommand{\boolq}{\textsc{BoolQ}\xspace}
\newcommand{\infotabs}{\textsc{InfoTabs}\xspace}

\newcommand{\editout}[1]{{\color{red} \sout{#1}}}
\newcommand{\editin}[1]{{\color{darkgreen} \textbf{#1}}}

\newcommand{\plus}[1]{{\color{darkgreen} \textbf{#1}}}
\newcommand{\minus}[1]{{\color{red} \textbf{#1}}}

\newcommand{\cls}{\texttt{[CLS]}}
\newcommand{\sep}{\texttt{[SEP]}}
\newcommand\ours{\textsc{OURS}\xspace}
\newcommand\sql{\textsc{SQL}\xspace}
\newcommand\sota{state-of-the-art\xspace}
\newcommand{\tapas}{\textsc{TAPAS}\xspace}
\newcommand{\bert}{\textsc{BERT}\xspace}
\newcommand{\roberta}{\textsc{RoBERTa}\xspace}
\newcommand{\tablebert}{\textsc{Table-BERT}\xspace}
\newcommand{\lfc}{\textsc{LogicalFactChecker}\xspace}
\newcommand{\masklm}{\textsc{Mask-LM}\xspace}
\newcommand{\latexfile}[1]{\input{2020_emnlp_entablement/sections/#1}}
\newcommand{\figfile}[1]{2020_emnlp_entablement/figures/#1}
\newcommand{\tabfile}[1]{\input{2020_emnlp_entablement/tables/#1}}
\newcommand{\autofig}[1]{2020_emnlp_entablement/auto_fig/#1}
\newcommand\la{}  \newcommand\ra{}
\newcommand{\bftab}{\fontseries{b}\selectfont}

\theoremstyle{definition}
\newtheorem{thm}{Theorem}
\newtheorem{defn}[thm]{Definition}

\newcommand{\errordescr}{Error margins are estimated as half the interquartile range}

\newcommand{\pruningtable}[1]{
\resizebox{0.8\columnwidth}{!}{
\begin{tabular}{lcll}
\toprule
\textbf{Method} & \textbf{PT Size} & \textbf{FT Size} & \multicolumn{1}{l}{\textbf{Val}} \\ 
\midrule
\tablebert{} &  & 512\ifthenelse{\equal{#1}{full}}{}{\footnotemark} & 66.1 \\ 
\midrule
\ours & 512 & 512 & 78.3\err{0.2} \\
      & 256 & 512 &	78.6\err{0.3} \\
      & 128 & 512 &	77.5\err{0.3} \\
\midrule
\ours{} - HEL  & 128  & 512 & 76.7\err{0.4} \\
               & 128  & 256 & 76.3\err{0.1} \\
               & 128  & 128 & 71.0\err{0.3} \\
\midrule
\ours{} - HEM &  256 & 512 & 78.8\err{0.3} \\
              &  256 & 256 & 78.1\err{0.1} \\
              &  128 & 512 & 78.2\err{0.4} \\
              &  128 & 256 & 77.0\err{0.2} \\
\ifthenelse{\equal{#1}{full}}{
    &  128 & 128 & 72.7\err{0.2} \\
\midrule
\ours - W2V	& 128 & 512 & 77.7\err{0.3} \\
	        & 128 & 256 & 76.0\err{0.2} \\
	        & 128 & 128 & 70.6\err{0.3} \\
\midrule
\ours - IWF & 128 & 512 & 77.9\err{0.2} \\
            & 128 & 256 & 77.2\err{0.1} \\
            & 128 & 128 & 72.7\err{0.3} \\
\midrule            
\ours - CHAR & 128 & 512 & 77.5\err{0.2} \\
             & 128 & 256 & 74.8\err{0.1} \\
             & 128 & 128 & 68.7\err{0.0} \\
}{
    &  128 & 128 & 72.7\err{0.2} \\
}
\bottomrule
\end{tabular}}\ifthenelse{\equal{#1}{full}}{
\caption{Accuracy of different pruning methods: The heuristic entity linking (HEL) \cite{2019TabFactA}, Heuristic exact match (HEM), word-to-vec (W2V), inverse word frequency (IWF), character ngram (CHAR) at different pre-training (PT) and fine-tuning (FT) sizes. \errordescr.}
}{
\caption{Accuracy of column pruning methods, that reduce input length for faster training and prediction: The heuristic entity linking (HEL) \cite{2019TabFactA} and Heuristic exact match (HEM) at various pre-training (PT) and fine-tuning (FT) sizes. HEM out-performs HEL on all input sizes, and in the faster case (128) out-performs \tablebert by 6.6 points. Accuracy with size 256 is 0.7 points behind the full input size. \errordescr.}
}
} 

\makeatletter
\newcommand*{\addFileDependency}[1]{\typeout{(#1)}
  \@addtofilelist{#1}
  \IfFileExists{#1}{}{\typeout{No file #1.}}
}
\makeatother
 
\newcommand*{\myexternaldocument}[1]{\externaldocument{#1}\addFileDependency{#1.tex}\addFileDependency{#1.aux}} 
\begin{document}
\maketitle

\begin{abstract}
    

Table entailment, the binary classification task of finding if a sentence is supported or refuted by the content of a table, requires parsing language and table structure as well as numerical and discrete reasoning.
While there is extensive work on textual entailment, table entailment is less well studied.
We adapt \tapas~\cite{herzig-2020}, a table-based \bert model, to recognize entailment. 
Motivated by the benefits of data augmentation, we create a balanced dataset of millions of automatically created training examples which are learned in an intermediate step prior to fine-tuning.
This new data is not only useful for table entailment, but also for \sqa~\cite{iyyer-etal-2017-search}, a sequential table QA task.
To be able to use long examples as input of \bert~models, we evaluate table pruning techniques as a pre-processing step to drastically improve the training and prediction efficiency at a moderate drop in accuracy.
The different methods set the new \sota on the \tabfact~\cite{2019TabFactA} and \sqa~datasets. \end{abstract}


\section{Introduction}
\label{sec:intro}

\begin{figure*}[!t]
\centering
\begin{tabular}{llllll}
\textbf{Rank} &	\textbf{Player} & \textbf{Country} & \textbf{Earnings} & \textbf{Events} & \textbf{Wins} \\
\hline
1	&Greg Norman    &	Australia    &	1,654,959&	16&	3 \\
2	&Billy Mayfair  &	United States&	1,543,192&	28&	2 \\
3	&Lee Janzen     &	United States&	1,378,966&	28&	3 \\
4	&Corey Pavin    &	United States&	1,340,079&	22&	2 \\
5	&Steve Elkington&	Australia    &	1,254,352&	21&	2 \\
\end{tabular}\\
\vspace{12pt}
\begin{tabular}{ll}
{\small \emph{Entailed:}} &{\small Greg Norman and Steve Elkington are from the same country. }\\
                         &{\small Greg Norman and Lee Janzen both have 3 wins.}\\

{\small \emph{Refuted:}}  &{\small Greg Norman is from the US and Steve Elkington is from Australia. }\\
                         &{\small Greg Norman and Billy Mayfair tie in rank.}\\

{\small \emph{Counterfactual:}} & {\small \editin{Greg Norman} has the highest earnings.}\\
                                & {\small \editout{Steve Elkington} has the highest earnings.}\\
{\small\emph{Synthetic:}}&{\small  is less than wins when Player is Lee Janzen.}\\
&{\small The sum of Earnings when Country is Australia is .}\\
\end{tabular}\caption{A \tabfact{} table with real statements\footnotemark and counterfactual and synthetic examples.}
\label{fig:example}
\end{figure*}

Textual entailment~\cite{dagan2005pascal}, also known as natural language inference~\cite{bowman-etal-2015-large}, is a core natural language processing (\abr{nlp}) task.
It can predict effectiveness of reading comprehension~\cite{dagan-10}, which argues that it can form the foundation of many other \abr{nlp} tasks, and is a useful neural pre-training task~\cite{subramanian2018learning,conneau-etal-2017-supervised}.

Textual entailment is well studied, but many relevant data sources are structured or semi-structured: 
health data both worldwide and personal, fitness trackers, stock markets, and sport statistics.
While some information needs can be anticipated by hand-crafted templates, user queries are often surprising, and having models that can reason and parse that structure can have a great impact in real world applications~\cite{Khashabi19, aristo}.

A recent example is \tabfact~\cite{2019TabFactA}, a dataset of statements that are either entailed or refuted by tables from Wikipedia (Figure~\ref{fig:example}).
Because solving these entailment problems requires sophisticated reasoning and higher-order operations like , averaging, or comparing, human accuracy remains substantially (18 points) ahead of the best models~\cite{zhong2020logicalfactchecker}.

The current models are dominated by semantic parsing approaches that attempt to create logical forms from weak supervision.
We, on the other hand, follow \citet{herzig-2020} and \citet{2019TabFactA} and encode the tables with \bert{}-based models to directly predict the entailment decision.
But while \bert{} models for text have been scrutinized and optimized for how to best pre-train and represent \emph{textual} data, the same attention has not been applied to tabular data, limiting the effectiveness in this setting.
This paper addresses these shortcomings using \emph{intermediate task} pre-training~\cite{pruksachatkun-2020-intermediate-task}, creating efficient data representations, and applying these improvements to the tabular entailment task. 

Our methods are tested on the English language, mainly due to the availability of the end task resources. However, we believe that the proposed solutions could be applied in other languages where a pre-training corpus of text and tables is available, such as the Wikipedia datasets.







Our main contributions are the following: 

i) We introduce two \emph{intermediate} pre-training tasks, which are learned from a trained \masklm{}~model, one based on synthetic and the other from counterfactual statements.
The first one generates a sentence by sampling from a set of logical expressions that filter, combine and compare the information on the table, which is required in table entailment (e.g., knowing that Gerald Ford is taller than the average president requires summing all presidents and dividing by the number of presidents).
The second one corrupts sentences about tables appearing on Wikipedia by swapping entities for plausible alternatives.
Examples of the two tasks can be seen in Figure \ref{fig:example}. 
The procedure is described in detail in section \ref{sec:methods}.

ii)
We demonstrate column pruning to be an effective means of lowering computational cost at minor drops in accuracy, doubling the inference speed at the cost of less than one accuracy point.

iii) Using the pre-training tasks, we set the new \sota on \tabfact{} out-performing previous models by  points when using a \bert{}-base model and  points for a \bert{}-large model. The procedure is data efficient and can get comparable accuracies to previous approaches when using only 10\% of the data.
We perform a detailed analysis of the improvements in Section \ref{sec:analysis}.
Finally, we show that our method improves the \sota on a question answering task (\sqa{}) by 4 points.

We release the pre-training checkpoints, data generation and training code at \href{https://github.com/google-research/tapas}{github.com/google-research/tapas}.
\footnotetext{Based on table \texttt{2-14611590-3.html} with light edits.}


 \section{Model}
\label{sec:model}



We use a model architecture derived from \bert{} and add additional embeddings to encode the table structure, following the approach of \citet{herzig-2020} to encode the input. 

The statement and table in a pair are tokenized into word pieces and concatenated using the standard \texttt{[CLS]} and \texttt{[SEP]} tokens in between. The table is flattened row by row and no additional separator is added between the cells or rows.

Six types of learnable input embeddings are added together as shown in Appendix \ref{sec:apx-input}. 
\textbf{Token embeddings}, \textbf{position embeddings} and \textbf{segment embeddings} are analogous to the ones used in standard \bert{}. 
Additionally we follow \citet{herzig-2020} and use \textbf{column and row embeddings} which encode the two dimensional position of the cell that the token corresponds to and
\textbf{rank embeddings} for numeric columns that encode the numeric rank of the cell with respect to the column, and provide a simple way for the model to know how a row is ranked according to a specific column.

Recall that the bi-directional self-attention mechanism in transformers is unaware of order, which motivates the usage of positional and segment embeddings for text in \bert{}, and generalizes naturally to column and row embeddings when processing tables, in the -dimensional case.

Let  and  represent the sentence and table respectively and  and  be their corresponding input embeddings. 
The sequence  is passed through a transformer~\cite{vaswani-2017} denoted  and a contextual representation is obtained for every token. We model the probability of entailment  with a single hidden layer neural network computed from the output of the \cls ~token:

where the middle layer has the same size as the hidden dimension and uses a \texttt{tanh} activation and the final layer uses a \texttt{sigmoid} activation.





 \section{Methods}
\label{sec:methods}

The use of challenging pre-training tasks has been successful in improving downstream accuracy~\cite{Clark2020ELECTRA:}.
One clear caveat of the method adopted in \citet{herzig-2020} which attempts to fill in the blanks of sentences and cells in the table is that not much understanding of the table in relation with the sentence is needed. 

With that in mind, we propose two tasks that require sentence-table reasoning and feature complex operations performed on the table and entities grounded in sentences in non-trivial forms.

We discuss two methods to create pre-training data that lead to stronger table entailment models.
Both methods create statements for existing Wikipedia tables\footnote{Extracted from a Wikipedia dump from 12-2019.}. 
We extract all tables that have at least two columns, a header row and two data rows.
We recursively split tables row-wise into the upper and lower half until they have at most 50 cells.
This way we obtain 3.7 million tables.

\subsection{Counterfactual Statements}

Motivated by work on counterfactually-augmented data ~\cite{Kaushik2020Learning, gardner-2020},
we propose an automated and scalable method to get table entailments from Wikipedia and, for each such positive examples, create a minimally differing refuted example.
For this pair to be useful we want that their truth value can be predicted from the associated table but not without it.

The tables and sentences are extracted from Wikipedia as follows:
We use the page title, description, section title, text and caption. 
We also use all sentences on Wikipedia that link to the table's page and mentions at least one page (entity) that is also mentioned in the table.
Then these snippets are split into sentences using the NLTK~\cite{loper-02} implementation of Punkt~\cite{kiss-06}.
For each relevant sentence we create one positive and one negative statement.

Consider the table in Figure \ref{fig:example} and the sentence \emph{`[Greg Norman] is [Australian].'} (Square brackets indicate mention boundaries.).
A mention\footnote{We annotate numbers and dates in the table and sentence with a simple parser and rely on the Wikipedia mention annotations (anchors) for identifying entities.} is a potential \textbf{focus mention} if the same entity or value is also mentioned in the table. In our example, \emph{Greg Norman} and \emph{Australian} are potential focus mentions.
Given a focus mention (\emph{Greg Norman}) we define all the mentions that occur in the same column (but do not refer to the same entity) as the \textbf{replacement mentions} (e.g., \emph{Billy Mayfair}, \emph{Lee Janzen}, \dots). We expect to create a false statement if we replace the focus mention with a replacement mention (e.g., \emph{`Billy Mayfair is Australian.'}), but there is no guarantee it will be actually false.

We call a mention of an entity that occurs in the same row as the focus entity a \textbf{supporting mention}, because it increases the chance that we falsify the statement by replacing the focus entity. In our example, \emph{Australian} would be a supporting mention for \emph{Greg Norman} (and vice versa). If we find a supporting mention we restrict the replacement candidates to the ones that have a different value. In the example, we would not use \emph{Steve Elkington} since his row also refers to Australia.

Some replacements can lead to ungrammatical statements that a model could use to identify the negative statements,
so we found it is useful to also replace the entity in the original positive sentence from Wikipedia with the mention from the table.\footnote{
 Consider that if \emph{Australian} is our focus and we replace it with \emph{United States} we get `\emph{Greg Norman is United States.}'.} \\
We also introduce a simple type system for entities (named entity, date, cardinal number and ordinal number) and only replace entities of the same type.
Short sentences having less than  tokens not counting the mention, are filtered out.

Using this approach we extract 4.1 million counterfactual pairs of which 546 thousand do have a supporting mention and the remaining do not.

We evaluated 100 random examples manually and found that the percentage of negative statements that are false and can be refuted by the table is 82\% when they have a supporting mention and 22\% otherwise.
Despite this low value we still found the examples without supporting mention to improve accuracy on the end tasks (Appendix \ref{sec:apx-table-pruning}).

\subsection{Synthetic Statements}

\begin{figure}[!t]
\centering
\begin{tabular}{rl}
\la statement\ra  & \la expr\ra \la compare\ra \la expr\ra \\
\la expr\ra  & \la select\ra ~when \la where\ra ~| \\ & \la select\ra \\
\la select\ra  &  \la column\ra ~| \\ & the \la aggr\ra ~of \la column\ra ~| \\ & the count \\
\la where\ra  & \la column\ra \la compare\ra \la value\ra ~|\\ & \la where\ra ~and \la where\ra \\
\la aggr\ra & first | last | \\ & lowest | greatest | \\ & sum | average | range \\
\la compare\ra  & is | \\ & is greater than | \\ & is less than \\
\la value\ra   & \la string\ra ~| \la number\ra
\end{tabular}
\caption{Grammar of synthetic phrases. \la column\ra ~is the set of column names in the table. We also generate constant expressions by replacing expressions with their values. Aggregations are defined in Table \ref{table:aggregations}.}
\label{fig:grammar}
\end{figure}

\begin{table}[!t]
\centering
\resizebox{1\columnwidth}{!}{
\begin{tabular}{ll}
\textbf{Name}        & \textbf{Result} \\ 
\hline
first                & the value in C with the lowest row index. \\
last                 & the value in C with the highest row index. \\
greatest             & the value in C with the highest numeric value. \\
lowest               & the value in C with the lowest numeric value. \\
sum                  & The sum of all the numeric values.\\
average              & The average of all the numeric values.\\
range                & The difference between greatest and lowest.\\
\hline
\end{tabular}
\vspace{-12pt}
}
\caption{Aggregations used in synthetic statements, where  are the column values. When  is empty or a singleton, it results in an error. Numeric functions also fail if any of their values is non-numeric.}
\label{table:aggregations}
\end{table}

Motivated by previous work~\cite{Geva2020InjectingNR}, we propose a synthetic data generation method to improve the handling of numerical operations and comparisons.
We build a table-dependent \emph{statement} that compares two simplified SQL-like expressions.
We define the (probabilistic) \emph{context-free grammar} shown in Figure \ref{fig:grammar}.
Synthetic statements are sampled from the CFG.
We constrain the  \la select\ra~values of the left and right expression to be either both \emph{the count} or to have the same value for \la column\ra.
This guarantees that the domains of both expressions are comparable.
\la value\ra~is chosen as at random from the respective column.
A statement is redrawn if it yields an error (see Table \ref{table:aggregations}).

With probability 0.5 we replace one of both expressions by the values it evaluates to.
In the example given in figure \ref{fig:example}, ``[The [sum] of [Earnings]] when [[Country] [is] [Australia]]'' is an \la expr\ra~that can be replaced by the constant value .

We set  to  in all our experiments. Everything else is sampled uniformly. 
For each Wikipedia table we generate a positive and a negative statement which yields M pairs.


\subsection{Table pruning}
\label{sec:pruning}
Some input examples from \tabfact{} can be too long for \bert-based models. We evaluate table pruning techniques as a pre-processing step to select relevant columns that respect the input example length limits.
As described in section~\ref{sec:model}, an example is built by concatenating the statement with the flattened table. 
For large tables the example length can exceed the capacity limit of the transformer.

The \tapas{} model handles this by shrinking the text in cells. A token selection algorithm loops over the cells. For each cell it starts by selecting the first token, then the second and so on until the maximal length is reached. 
Unless stated otherwise we use the same approach.
Crucially, selecting only relevant columns would allow longer examples to fit without discarding potentially relevant tokens.

\textbf{Heuristic entity linking (HEL)} is used as a baseline. It is the table pruning used in \tablebert{}~\cite{2019TabFactA}. 
The algorithm aligns spans in statement to the columns
by extracting the longest character n-gram that matches a cell. The span matches represent linked entities. 
Each entity in the statement can be linked to only one column.
We use the provided entity linking statements data\footnote{\href{https://github.com/wenhuchen/Table-Fact-Checking/blob/master/tokenized_data}{github.com/wenhuchen/Table-Fact-Checking/\\blob/master/tokenized\_data}}.
We run the \tapas{} algorithm on top of the input data to limit the input size.

We propose a different method that tries to retain as many columns as possible.
In our method, the columns are ranked by a relevance score and added in order of decreasing relevance.
Columns that exceed the maximum input length are skipped.
The algorithm is detailed in Appendix \ref{sec:apx-table-pruning}.
\textbf{Heuristic exact match (HEM)} computes the Jaccard coefficient between the statement and each column.
Let  be the set of tokens in the statement  and  the tokens in column , with  the set of columns. Then the score between the statement and column is given by .

We also experimented with approaches based on word2vec \cite{mikolov-13}, character overlap and TF-IDF.
Generally, they produced worse results than HEM.
Details are shown in Appendix~\ref{sec:apx-table-pruning}.

 \section{Experimental Setup}
\label{sec:experiments}

In all experiments, we start with the public \tapas{} checkpoint,\footnote{\href{https://github.com/google-research/tapas}{github.com/google-research/tapas}} train an entailment model on the data from Section \ref{sec:methods} and then fine-tune on the end task (\tabfact or \sqa). We report the median accuracy values over  pre-training and  fine-tuning runs ( runs in total). 
We estimate the error margin as half the \emph{interquartile range}, that is half the difference between the 25th and 75th percentiles.
The hyper-parameters, how we chose them, hardware and other information to reproduce our experiments are explained in detail in Appendix~\ref{sec:apx-repro}.

The training time depends on the sequence length used. For a \bert{}-Base model it takes around  minutes using  tokens and it scales almost linearly up to . For our pre-training tasks, we explore multiple lengths and how they trade-off speed for downstream results.

\subsection{Datasets}

We evaluate our model on the recently released \tabfact~dataset \cite{2019TabFactA}. 
The tables are extracted from Wikipedia and the sentences written by crowd workers in two batches. The first batch consisted of \textbf{simple} sentences, that instructed the writers to refer to a single row in the table. The second one, created \textbf{complex} sentences by asking writers to use information from multiple rows.

In both cases, crowd workers initially created only positive (entailed) pairs, and in a subsequent annotation job, the sentences were copied and edited into negative ones, with instructions of avoiding simple negations. Finally, there was a third verification step to filter out bad rewrites.
The final count is . The split sizes are given in Appendix ~\ref{sec:apx-dataset-stats}.
An example of a table and the sentences is shown in Figure~\ref{fig:example}. 
We use the standard \tabfact split and the official accuracy metric.

\infotabstext{
\infotabs{}~\cite{gupta2020infotabs} is another recently released table entailment dataset. It differs from \tabfact in that it only uses Wikipedia info-boxes. 
}

We also use the \sqa{} \cite{iyyer-etal-2017-search} dataset for pre-training (following \citet{herzig-2020}) and for testing if our pre-training is useful for related tasks.
\sqa is a question answering dataset that was created by asking crowd workers to split a compositional subset of WikiTableQuestions \cite{pasupat2015compositional} into multiple referential questions.
The dataset consists of 6,066 sequences ( question per sequence on average).
We use the standard split and official evaluation script.

\subsection{Baselines}

\citet{2019TabFactA} present two models, \tablebert and the Latent Program Algorithm (LPA), that yield similar accuracy on the \tabfact data.

LPA tries to predict a latent program that is then executed to verify if the statement is correct or false. The search over programs is restricted using lexical heuristics. Each program and sentence is encoded with an independent transformer model and then a linear layer gives a relevance score to the pair. The model is trained with weak supervision where programs that give the correct binary answer are considered positive and the rest negative.

\tablebert is a \bert{}-base model that similar to our approach directly predicts the truth value of the statement.
However, the model does not use special embeddings to encode the table structure but relies on a template approach to format the table as natural language. The table is mapped into a single sequence of the form: \emph{``Row 1 Rank is 1; the Player is Greg Norman; ... . Row 2 ...''}.
The model is also not pre-trained on table data.

\lfc~\cite{zhong2020logicalfactchecker} is another transformer-based model that given a candidate logical expression, combines contextual embeddings of program, sentence and table, with a tree-RNN~\cite{socher-etal-2013-parsing} to encode the parse tree of the expression. The programs are obtained through either LPA or an LSTM generator (Seq2Action).
 \section{Results}


\begin{table*}[tb]
\small
\centering
\resizebox{2\columnwidth}{!}{
\begin{tabular}{llllllll} 
\toprule
\multicolumn{3}{l}{\textbf{Model}} & \multicolumn{1}{l}{\textbf{Val}} & \multicolumn{1}{l}{\textbf{Test}} & \textbf{Test\textsubscript{simple}} & \textbf{Test\textsubscript{complex}} & \textbf{Test\textsubscript{small}}  \\ 
\midrule
\multicolumn{3}{l}{\bert~classifier w/o Table}                  & 50.9                    & 50.5 & 51.0 &      50.1               & 50.4        \\ 
\midrule
\multicolumn{3}{l}{\tablebert-Horizontal-T+F-Template}  & 66.1  & 65.1 & 79.1 &  58.2 & 68.1      \\ 
\multicolumn{3}{l}{LPA-Ranking w/ Discriminator (Caption)}  & 65.1                    & 65.3                 &  78.7 &    58.5      & 68.9         \\ 
\multicolumn{3}{l}{\lfc~(program from LPA)}         & 71.7 & 71.6 & 85.5 &64.8 &74.2 \\
\multicolumn{3}{l}{\lfc~(program from Seq2Action)}  & 71.8 & 71.7 & 85.4 &65.1 & 74.3\\
\midrule
\ours &Base & \masklm{}                            & 69.6\err{4.4} & 69.9\err{3.8} & 82.0\err{5.9} & 63.9\err{2.8} & 72.2\err{4.7} \\
\ours &Base & SQA                                  & 74.9\err{0.2} & 74.6\err{0.2} & 87.2\err{0.2} & 68.4\err{0.4} & 77.3\err{0.3} \\
\ours &Base & Counterfactual                       & 75.5\err{0.5} & 75.2\err{0.4} & 87.8\err{0.4} & 68.9\err{0.5} & 77.4\err{0.3} \\
\ours &Base & Synthetic                            & 77.6\err{0.2} & 77.9\err{0.3} & 89.7\err{0.4} & 72.0\err{0.2} & 80.4\err{0.2} \\
\ours &Base & Counterfactual + Synthetic           & {\bftab 78.6}\err{0.3} & {\bftab 78.5}\err{0.3} & {\bftab 90.5}\err{0.4} & {\bftab 72.5}\err{0.3} & {\bftab 81.0}\err{0.3} \\
\midrule
\ours &Large & Counterfactual + Synthetic          & {\bftab 81.0}\err{0.1} & {\bftab 81.0}\err{0.1} & {\bftab 92.3}\err{0.3} & {\bftab 75.6}\err{0.1} & {\bftab 83.9}\err{0.3} \\
\midrule
\multicolumn{3}{l}{Human Performance}              &  -                      &      -  &  - & -      & 92.1         \\
\bottomrule
\end{tabular}
}
\caption{
The \tabfact~results. Baseline and human results are taken from \citet{2019TabFactA} and \citet{zhong2020logicalfactchecker}. 
The best \bert{}-base model while comparable in parameters out-performs \tablebert by more than 12 points.
Pre-training with counterfactual and synthetic data gives an accuracy 8 points higher than only using \masklm{} and more than 3 points higher than using \sqa.
Both counterfactual and synthetic data out-perform pre-training with a \masklm{} objective and \sqa. Joining the two datasets gives an additional improvement. 
\errordescr.}
\label{tab:results}
\end{table*}

\paragraph{\tabfact{}}

In Table \ref{tab:results} we find that our approach outperforms the previous \sota{} on \tabfact{} by more than  points (Base) or more than  points (Large).
A model initialized only with the public \tapas{}~\masklm{}~ checkpoint is behind \sota{} by 2 points ( vs ).
If we train on the counterfactual data, it out-performs \lfc{} and reaches  test accuracy (), slightly above using \sqa{}. 
Only using the synthetic data is better (),
and when using both datasets it achieves .
Switching from \bert{}-Base to Large improves the accuracy by another  points.
The improvements are consistent across all test sets.





\paragraph{Zero-Shot Accuracy and low resource regimes}

The pre-trained models are in principle already complete table entailment predictors.
Therefore it is interesting to look at their accuracy on the \tabfact{} evaluation set before fine-tuning them. 
We find that the best model trained on all the pre-training data is only two points behind the fully trained \tablebert ( vs ).
This relatively good accuracy mostly stems from the counterfactual data.

When looking at \textbf{low data regimes} in Figure \ref{fig:result-train-size} we find that
pre-training on \sqa or our artificial data consistently leads to better results than just training with the \masklm{} objective.
The models with synthetic pre-training data start out-performing \tablebert when using  of the training set.
The setup with all the data is consistently better than the others and synthetic and counterfactual are both better than \sqa.

\begin{figure}[!t]
\includegraphics[width=1\linewidth]{2020_emnlp_entablement/figures/size_plot.pdf}
\caption{Results for training on a subset of the data. Counterfactual + Synthetic (C+S) consistently out-performs only Counterfactual (C) or Synthetic (S), which in turn out-perform pre-training on \sqa. C+S and S surpass \tablebert{} at 5\% (around 4,500) of examples, C and \sqa at 10\%.
C+S is comparable with \lfc when using 10\% of the data. }
\label{fig:result-train-size}
\end{figure}


\infotabstext{
\paragraph{\infotabs{}} 

Results are shown in table~\ref{tab:infotabs_results}.

\begin{table}[!t]
\begin{center}
\resizebox{1.0\columnwidth}{!}{
\begin{tabular}{llrrrr}
\toprule
\textbf{Model} & \textbf{Size} & \textbf{Val} & \bm{} & \bm{} & \bm{} \\
\midrule
\citet{gupta2020infotabs}
& \bert base w/o Table & 62.69 & 63.45 & 49.65 & 50.45 \\
& \bert     base & 63.67 & 64.04 & 53.59 & 49.05\\
& \roberta  base & 68.06 & 66.70 & 56.87 & 55.26\\
& \roberta large & 77.61 & 75.06 & 69.02 & 64.61\\
\midrule
\masklm{}     & \bert base & 69.50 &  69.78 &  57.06 & 55.22 \\
Synthetic     & \bert base & 69.94 &  69.94 &  56.83 & 54.72 \\
Counterfactual& \bert base & 70.11 &  70.44 &  57.56 & 54.83 \\
Counterfactual + Synthetic
& \bert     base & 71.22 &  71.39 &  58.89 & 56.11 \\
& \bert     large& 74.44 &  74.83 & 64.11 & 61.89 \\  \bottomrule
\end{tabular}}
\end{center}
\caption{Results on \infotabs{}.  is a standard held-out test set,  is an adversarial set created by perturbing the statements in ,  contains out-of-domain tables.}
\label{tab:infotabs_results}
\end{table}
}

\paragraph{\sqa{}} 

Our pre-training data also improves the accuracy on a QA task.
On \sqa{}~\cite{iyyer-etal-2017-search} a model pre-trained on the synthetic entailment data outperforms one pre-trained on the \masklm{}~task alone (Table \ref{tab:sqa_results}).
Our best \bert{}~Base model out-peforms the \bert{}-Large model of \citet{herzig-2020} and a \bert{}-Large model trained on our data improves the previous \sota{} by 4 points on average question and sequence accuracy. See dev results and error bars in Appendix \ref{sec:sqa_dev_results}.

\begin{table}[!t]
\begin{center}
\resizebox{1.0\columnwidth}{!}{
\begin{tabular}{lllllll}
\toprule
\textbf{Data} & \textbf{Size} & \textbf{ALL} & \textbf{SEQ}\\
\midrule
  \citet{iyyer-etal-2017-search}    &                    & 44.7  & 12.8\\
  \citet{mueller-2019}              &                    & 55.1  & 28.1\\
\citet{herzig-2020}                 &  Large             & 67.2  & 40.4 \\
\midrule
\masklm{}                        & Base  & 64.0\err{0.2} & 34.6\err{0.0} \\
Counterfactual                   & Base  & 65.0\err{0.5} & 36.5\err{0.6} \\
Synthetic                        & Base  & 67.4\err{0.2} & 39.8\err{0.4} \\
Counterf. + Synthetic            & Base  & 67.9\err{0.3} & 40.5\err{0.7} \\
\midrule
Counterf. + Synthetic            & Large & {\bftab 71.0}\err{0.4} & {\bftab 44.8}\err{0.8} &\\
\bottomrule
\end{tabular}}
\end{center}
\caption{
\sqa{} test results. ALL is the average question accuracy and SEQ the sequence accuracy.
Both counterfactual and synthetic data out-perform the \masklm{} objective. 
Our \emph{Large} model outperforms the \masklm{} model by almost 4 points on both metrics.
Our best \emph{Base} model is comparable to the previous \sota{}.
\errordescr.
}
\label{tab:sqa_results}
\end{table}

\paragraph{Efficiency} 

As discussed in Section \ref{sec:pruning} and Appendix \ref{sec:apx-train-time}, we can increase the model efficiency by reducing the input length.
By pruning the input of the \tabfact{} data we can improve training as well as inference time.
We compare pruning with the heuristic entity linking (HEL) \cite{2019TabFactA} and heuristic exact match (HEM) to different target lengths. 
We also studied other pruning methods, the results are reported in Appendix \ref{sec:apx-table-pruning}.
In Table \ref{tab:result-pruning} we find that HEM consistently outperforms HEL.
The best model at length , while twice as fast to train (and apply), is only  points behind the best full length model.
Even the model with length , while using a much shorter length, out-performs \tablebert{} by more than  points.

Given a pre-trained \masklm{} model our training consists of training on the artificial pre-training data and then fine-tuning on \tabfact{}.
We can therefore improve the training time by pre-training with shorter input sizes.
Table \ref{tab:result-pruning} shows that  and  give similar accuracy while the results for  are about  point lower.





\begin{table}[!t]
\centering
\pruningtable{short}
\label{tab:result-pruning}
\end{table}
\footnotetext{Not explicitly mentioned in the paper but implied by the batch size given (6) and the defaults in the code.}
 \section{Analysis}
\label{sec:analysis}

\paragraph{Salient Groups}

\begin{figure*}[!t]
\small
\centering
\begin{tabular}{l|p{6cm}p{6cm}}
\textbf{Group} &	\textbf{Consistently Better} & \textbf{Persisting Errors} \\
\hline
\textbf{Aggregations} & 
Choi Moon - Sik played in Seoul three times in total. &
The total number of bronze medals were half of the total number of medals.
\\
\textbf{Superlatives} & 
Mapiu school has the highest roll in the state authority. &
Carlos Moya won the most tournaments with two wins.
\\
\textbf{Comparatives} & 
Bernard Holsey has 3 more yards than Angel Rubio. &
In 1982, the Kansas City Chiefs played more away games than home games.
\\
\textbf{Negations}    & 
The Warriors were not the home team at the game on 11-24-2006. &
Dean Semmens is not one of the four players born after 1981.
\end{tabular}\caption{
On the left column we show examples that our model gets correct for most runs and that \masklm gets wrong for most runs. The right column shows examples that the model continues to make mistakes on. Many of those include deeper chains of reasoning or more complex numeric operations.}
\label{fig:errors}
\vspace{-3ex}
\end{figure*}

\begin{table}[!t]
\small
\centering
\resizebox{1.0\columnwidth}{!}{
\begin{tabular}{lr|rr|rrr} \toprule			
    &               &  \multicolumn{2}{c|}{\textbf{C+S}} & \multicolumn{3}{c}{\textbf{\masklm}} \\
	& \textbf{Size}	&	\textbf{Acc}	&	\textbf{ER}	&	\textbf{Acc}	&	\bm{}	&	\bm{}	\\
\midrule	
\textbf{Validation}	&  &  &  &  &  &	 	\\
\midrule
\textbf{Superlatives}  &   &   &    &   &   &    \\
\textbf{Aggregations}  &   &   &    &   &    &    \\
\textbf{Comparatives}  &   &   &    &   &    &    \\
\textbf{Negations}     &    &   &    &   &   &    \\
\midrule
\textbf{Multiple of the above}  &    &   &    &   &    &    \\
\textbf{Other}         &   &   &    &   &    &    \\
\bottomrule \end{tabular}
}
\caption{Comparing accuracy and total error rate (ER) for counterfactual and synthetic (C+S) and \masklm. 
Groups are derived from word heuristics.
The error rate in each group is taken with respect to the full set.
Negations and superlatives show the highest relative gains.
}
\vspace{-3ex}
\label{tab:slices}
\end{table}


To obtain detailed information of the improvements of our approach, we manually annotated  random examples with the complex operations needed to answer them. 
We found  salient groups: \textbf{Aggregations}, \textbf{superlatives}, \textbf{comparatives} and \textbf{negations}, and sort pairs into these groups via keywords in the text. To make the groups exclusive, we add a fifth case when more than one operation is needed. The accuracy of the heuristics was validated through further manual inspection of  samples per group. The trigger words of each group are described in Appendix \ref{sec:slice_words}.


For each group within the validation set, we look at the difference in accuracy between different models. We also look at how the total error rate is divided among the groups as a way to guide the focus on pre-training tasks and modeling. The error rate defined in this way measures potential accuracy gains if all the errors in a group  were fixed: .

Among the groups, the intermediate task data improve \textbf{superlatives} ( error reduction) and \textbf{negations} () most (Table \ref{tab:slices}).
For example, we see that the accuracy is higher for \textbf{superlatives} than the for the overall validation set. 


In Figure~\ref{fig:errors} we show examples in every group where our model is correct on the majority of the cases (across  trials), and the \masklm baseline is not. We also show examples that continue to produce errors after our pre-training. Many examples in this last group require multi-hop reasoning or complex numerical operations.

\paragraph{Model Agreement}

\begin{figure}[!t]
\includegraphics[width=1\linewidth]{2020_emnlp_entablement/figures/agreement_plot.pdf}
\caption{
Frequency of the number of models that give the correct answer, out of 9 runs.
Better pre-training leads to more consistency across models.
The ratio of samples answered correctly by all models is  for \masklm but  for Synthetic + Counterfactual.
}
\label{fig:agreement}
\end{figure}

Similar to other complex binary classification datasets such as \boolq{}~\cite{clark-etal-2019-boolq}, for \tabfact{} one may question whether models are guessing the right answer.
To detect the magnitude of this issue we look at  independent runs of each variant and analyze how many of them agree on the correct answer. 
Figure \ref{fig:agreement} shows that while for \masklm only for  of the examples all models agree on the right answer, it goes up to  when using using the counterfactual and synthetic pre-training. This suggests that the amount of guessing decreases substantially. 

 \section{Related Work}
\label{sec:related}

\paragraph{Logic-free Semantic Parsing}
Recently, methods that skip creating logical forms and generate answers directly have been used successfully for semantic parsing \cite{mueller-2019}.
In this group, \tapas{}~\cite{herzig-2020} uses special learned embeddings to encode row/column index and numerical order and pretrains a \masklm model on a large corpus of text and tables co-occurring on Wikipedia articles. 
Importantly, \emph{next sentence prediction} 
from \citet{devlin-19}, 
which in this context amounts to detecting whether the table and the sentence appear in the same article, was not found to be effective. Our hypothesis is that the task was not hard enough to provide a training signal.
We build on top of the \tapas~model and propose harder and more effective pre-training tasks to achieve strong performance on the \tabfact~dataset.


\paragraph{Entailment tasks}
Recognizing entailment has a long history in NLP~\cite{dagan-10}. Recently, the text to text framework has been expanded to incorporate structured data, like knowledge graphs \cite{vlachos-riedel-2015-identification}, tables \cite{jo2019aggchecker,gupta2020infotabs} or images \cite{suhr2017corpus,suhr2019corpus}.
The large-scale \tabfact~dataset~\cite{2019TabFactA} is one such example. Among the top performing models in the task is a BERT based model, acting on a flattened versioned of the table using textual templates to make the tables resemble natural text. Our approach has two key improvements: the usage of special embeddings, as introduced in \citet{herzig-2020}, and our novel \emph{counterfactual and synthetic pre-training} (Section~\ref{sec:methods}).

\paragraph{Pre-training objectives}
\emph{Next Sentence Prediction} (NSP) was introduced in \citet{devlin-19}, but follow-up work such as \citet{liu2019} identified that it did not contribute to model performance in some tasks. 
Other studies have found that application specific self-supervised pre-training objectives can improve performance of \masklm~models. One examples of such an objective is the \emph{Inverse Cloze Task} (ICT) \cite{lee-19}, that uses in-batch negatives and a two-tower dot-product similarity metric. 
\citet{Chang2020Pre-training} further expands on this idea and uses hyperlinks in Wikipedia as a weak label for topic overlap.

\paragraph{Intermediate Pre-training}
Language model fine-tuning \cite{howard-2018-universal} also know as domain adaptive pre-training \cite{gururangan-2020-dont-stop} has been studied as a way to handle covariate shift.
Our work is closer to intermediate task fine-tuning \cite{pruksachatkun-2020-intermediate-task} 
where one tries to teach the model \emph{higher-level abilities}.
Similarly we try to improve the discrete and numeric reasoning capabilities of the model.

\paragraph{Counterfactual data generation}
The most similar approach to ours appears in \citet{Xiong2020Pretrained}, replacing entities in Wikipedia by others with the same type for a \masklm~model objective. We, on the one hand, take advantage of other rows in the table to produce plausible negatives, and also replace dates and numbers.
Recently, \citet{Kaushik2020Learning,gardner-2020} have shown that exposing models to pairs of examples which are similar but have different labels can help to improve generalization, in some sense our \emph{Counterfactual} task is a heuristic version of this, that does not rely on manual annotation. \citet{sellam-2020-BLEURT} use perturbations of Wikipedia sentences for intermediate pre-training of a learned metric for text generation.

\paragraph{Numeric reasoning}
Numeric reasoning in Natural Language processing has been recognized as an
important part in entailment models \cite{sammons2010ask} and reading comprehension \cite{ran2019numnet}.
\citet{wallace-etal-2019-nlp} studied the capacity of different models on understanding numerical operations and show that \bert{}-based model still have headroom. This motivates the use of the synthetic generation approach to improve numerical reasoning in our model. 

\paragraph{Synthetic data generation} Synthetic data has been used to improve learning in \abr{nlp} tasks \cite{alberti-etal-2019-synthetic,lewis-etal-2019-unsupervised,wu-etal-2016-bilingually, leonandya-etal-2019-fast}. In semantic parsing for example \cite{wang-etal-2015-building, iyer-17, dbpal}, templates are used to bootstrap models that map text to logical forms or SQL.
\citet{salvatore-etal-2019-logical} use synthetic data generated from logical forms to evaluate the performance of textual entailment models (e.g., \bert{}).
\citet{geiger-2019-posing} use synthetic data to create \emph{fair} evaluation sets for natural language inference.
\citet{Geva2020InjectingNR} show the importance of injecting numerical reasoning via generated data into the model to solve reading comprehension tasks. They propose different templates for generating synthetic numerical examples. In our work we use a method that is better suited for tables and to the entailment task, and is arguably simpler.



 \section{Conclusion}
\label{sec:conclusion}

We introduced two pre-training tasks, counterfactual and synthetic, to obtain \sota results on the \tabfact\cite{2019TabFactA} entailment task on tabular data.
We adapted the \bert-based architecture of \tapas\cite{herzig-2020} to binary classification and showed that pre-training on both tasks yields substantial improvements on \tabfact but also on a QA dataset, \sqa~\cite{iyyer-etal-2017-search}, even with only a subset of the training data. 

We ran a study on column selection methods to speed-up training and inference.
We found that we can speed up the model by a factor of 2 at a moderate drop in accuracy ( point)
and by a factor of 4 at a larger drop but still with higher accuracy than previous approaches.

We characterized the complex operations required for table entailment to guide future research in this topic.
Our code and models will be open-sourced.

\section*{Acknowledgments}
We would like to thank Jordan Boyd-Graber, Yasemin Altun, Emily Pitler, Benjamin Boerschinger, William Cohen, Jonathan Herzig, Slav Petrov, and
the anonymous reviewers for their time, constructive
feedback, useful comments and suggestions about this work.  
\bibliographystyle{style/acl_natbib}
\bibliography{bib/journal-full,bib/jbg}

\clearpage

\appendix\section*{Appendix}



We provide details on our experimental setup and hyper-parameter tuning in Section \ref{sec:apx-repro}.
Section \ref{sec:apx-input} and \ref{sec:apx-dataset-stats} give additional information on model and the \tabfact{}~dataset.
We give details and results regarding our column pruning approach in Section~\ref{sec:apx-table-pruning-descr}.
Full results for \sqa are displayed in Section ~\ref{sec:sqa_dev_results}. Section ~\ref{sec:apx-table-pruning} shows the accuracy on the pre-training tasks held-out sets.
Section \ref{sec:slice_words} contains the trigger words used for identifying the salient groups in the analysis section.

\section{Reproducibility}
\label{sec:apx-repro}

\subsection{Hyper-Parameter Search}

The hyper-parameters are optimized using a black box Bayesian optimizer similar to Google Vizier~\cite{vizier} which looked at validation accuracy after  steps only, in order to prevent over-fitting and use resources effectively. 
The ranges used were a learning rate from  to , dropout probabilities from  to  and warm-up ratio from  to . We used  runs and kept the median values for the top  trials.

In order to show the impact of the number of trials in the expected validation results, we follow~\citet{AAAI1816669} and \citet{dodge-19}. Given that we used Bayesian optimization instead of random search, we applied the \emph{bootstrap} method to estimate mean and variance of the max validation accuracy at  steps for different number of trials. From trial  to  we noted an increase  of  in accuracy and a standard deviation that decreases from  to .

\subsection{Hyper-Parameters}

We use the same hyper-parameters for pre-training and fine-tuning.
For pre-training, the input length is 256 and 512 for fine-tuning if not stated otherwise.
We use  training steps, a \textbf{learning rate of } and a \textbf{warm-up ratio of 0.05}.
We disable the attention dropout in \bert{} but use a \textbf{hidden dropout probability of 0.07 }.
Finally, we use an Adam optimizer with weight decay with the same configuration as \bert{}.

For \sqa we do not use any search algorithm and use the same model and the same hyper-parameters as the ones used in~\citet{herzig-2020}.
The only difference is that we start the fine-tuning from a checkpoint trained on our intermediate pre-training entailment task.

\subsection{Number of Parameters}

The number of parameters is the same as for \bert{}:  for base models and  for Large models.

\subsection{Training Time}
\label{sec:apx-train-time}

We train all our models on \textbf{Cloud TPUs V3}. The input length has a big impact on the processing speed of the batches and thus on the overall training time and training cost.
For a \bert{}-Base model during training, we can process approximately 8700 examples per second at input length 128, 5100 at input length 256 and 2600 at input length 512.
This corresponds to training times of approx. \textbf{78 minutes, 133 minutes and 262 minutes}, respectively.

A \bert{}-Large model  processes approximately 800 examples per second at length 512 and takes \textbf{14 hours} to train. 

\section{Model}
\label{sec:apx-input}

For illustrative purposes, we include the input representation using the  types of embeddings, as depicted by \citet{herzig-2020}.

\begin{figure}
    \centering
    \scalebox{0.9}{
    \includegraphics[width=0.95\linewidth]{2020_emnlp_entablement/figures/BERT_input_example.pdf}
    }
    \caption{Input representation for model.}
    \label{fig:apx-input}
\end{figure}

\section{Dataset}
\label{sec:apx-dataset-stats}

Statistics of the \tabfact dataset can be found in table \ref{tab:apx-dataset-stats}.

\begin{table}[H]
\begin{center}
\scalebox{0.9}{
\begin{tabular}{llll}
\toprule
   & \textbf{Statements} & \textbf{Tables} \\
\midrule
\textbf{Train}      & 92,283 & 13,182 \\
\textbf{Val}        & 12,792 & 1,696 \\
\textbf{Test}       & 12,779 & 1,695  \\
\midrule
\textbf{Total}      & 118,275 & 16,573 \\
\midrule
\textbf{Simple}     & 50,244 & 9,189 \\
\textbf{Complex}    & 68,031 & 7,392 \\
\bottomrule

\end{tabular}
}
\end{center}
\caption{\tabfact~ dataset statistics.}
\label{tab:apx-dataset-stats}
\end{table}

\begin{figure}[H]
    \scalebox{1.0}{
    \includegraphics[width=1\linewidth]{2020_emnlp_entablement/figures/examples_distribution.pdf}
    }
    \caption{Input length histogram for \tabfact{} validation dataset when tokenized with \bert{}~tokenizer.}
    \label{fig:examples_distribution}
\end{figure}

\section{Columns selection algorithm}
\label{sec:apx-table-pruning-descr}

Let  be the function that computes the number of tokens given a text using the \bert{}~tokenizer,  the tokenized statement text,  the text of the column . We denote the columns as  ordered by their scores  where  is the number of columns. Let  be the maximum number of tokens. 
Then the cost of the column must verify the following condition.

where  is the set of retained columns at the iteration .  is added to the condition as two special tokens are added to the input: .
If a current column  doesn't respect the condition then the column is not selected. Whether or not the column is retained, the algorithm continues and verifies if the next column can fit. It follows  contains the maximum number of columns that can fit under  by respecting the columns scoring order.

There is a number of heuristic pruning approaches we have experimented with. Results are given in \ref{tab:apx-result-pruning-full}.

\paragraph{Word2Vec (W2V)} uses a publicly available word2vec \cite{mikolov-13} model\footnote{\url{https://tfhub.dev/google/tf2-preview/gnews-swivel-20dim/1}} to extract one embedding for each token. Let  be the set of tokens in the statement and  the set of tokens in a column. The cosine similarity for each pair is given by 

where  represents the embedding of the token .
For a given column token  we define the relevance with respect to the statement as the average similarity to every token:

Where  is a threshold that helps to remove noise from unrelated word embeddings. We set  to .
We experimented with  and  as other aggregation function but found the average to perform best.
The final score between the statement  and the column  is given by 


\paragraph{Term frequency–inverse document frequency (IWF)} Scores the columns' tokens proportional to the word frequency in the statement and offset by the word frequency computed over all the tables and statements from the training set.

Where  is how often the token  occurs in the statement , and  is the frequency of  in a word count list.
The final score of a column  is given by 


\paragraph{Character N-gram (CHAR)} Scores columns by character overlap with the statement. This method looks for sub-list of word’s characters in the statement. The length of the characters' list has a minimum and maximum length allowed. In the experiments we use  and  as minimum and maximum length. Let  be the set of all the overlapping characters' lengths.  
The scoring for each column is given by 



\begin{table}[H]
\centering
\pruningtable{full}
\label{tab:apx-result-pruning-full}
\end{table}

\section{\sqa{}}
\label{sec:sqa_dev_results}



\begin{table*}[tb]
\begin{center}
\resizebox{2.0\columnwidth}{!}{
\begin{tabular}{llcccccccccc}
\toprule
 & & \multicolumn{2}{c}{\textbf{ALL}} & \multicolumn{2}{c}{\textbf{SEQ}} & \multicolumn{2}{c}{\textbf{Q1}} & \multicolumn{2}{c}{\textbf{Q2}} & \multicolumn{2}{c}{\textbf{Q3}} \\
\textbf{Data} & \textbf{Size} & Dev & Test & Dev & Test & Dev & Test & Dev & Test & Dev & Test \\
\midrule
\masklm{}                        & Base  & 60.0\err{0.3} & 64.0\err{0.2} & 35.3\err{0.7} & 34.6\err{0.0} & 72.4\err{0.4} & 79.2\err{0.6} & 59.7\err{0.4} & 61.2\err{0.4} & 50.5\err{1.1} & 55.6\err{0.7} \\
Counterfactual                   & Base  & 63.2\err{0.7} & 65.0\err{0.5} & 39.3\err{0.6} & 36.5\err{0.6} & 74.7\err{0.3} & 78.4\err{0.4} & 63.8\err{1.2} & 63.7\err{0.3} & 52.4\err{0.7} & 57.5\err{0.7} \\
Synthetic                        & Base  & 64.1\err{0.4} & 67.4\err{0.2} & 41.6\err{0.8} & 39.8\err{0.4} & 75.3\err{0.7} & 79.3\err{0.1} & 64.4\err{0.6} & 66.2\err{0.2} & 55.8\err{0.7} & 60.2\err{0.6} \\
Counterfactual + Synthetic       & Base  & 64.5\err{0.2} & 67.9\err{0.3} & 40.2\err{0.4} & 40.5\err{0.7} & 75.6\err{0.3} & 79.3\err{0.3} & 65.3\err{0.6} & 67.0\err{0.3} & 55.4\err{0.5} & 61.1\err{0.9} \\
\midrule
Counterfactual + Synthetic       & Large & 68.0\err{0.2} & 71.0\err{0.4} & 45.8\err{0.3} & 44.8\err{0.8} & 77.7\err{0.6} & 80.9\err{0.5} & 68.8\err{0.4} & 70.6\err{0.3} & 59.6\err{0.5} & 64.0\err{0.3} \\
\bottomrule
\end{tabular}}
\end{center}
\caption{\sqa{} dev (first fold) and test results. ALL is the average question accuracy, SEQ the sequence accuracy, and QX, the accuracy of the X'th question in a sequence. We show the median over 9 trials, and errors are estimated with half the interquartile range .}
\label{tab:sqa_dev_results}
\end{table*}

Table \ref{tab:sqa_dev_results} shows the accuracy on the first development fold and the test set. As for the main results, the error margins displayed are half the interquartile range over 9 runs, which is half the difference between the first and third quartile. This range contains half of the runs and provides a measure of robustness.

\section{Pre-Training Data}
\label{sec:apx-table-pruning}

When training on the pre-training data we hold out approximately 1\% of the data for testing how well the model is solving the pre-training task.
In Table \ref{tab:apx-result-data}, we compare the test pre-training accuracy on synthetic and counterfactual data to models that are only trained on the statements to understand whether there is considerable bias in the data.
Both datasets have some bias (i.e. the accuracy without table is higher than 50\%.). Still there is a sufficient enough gap between training with and without tables so that the data is still useful. 

The synthetic data can be solved almost perfectly whereas for the counterfactual data we only reach an accuracy of 84.3\%.
This is expected as there is no guarantee that the model has enough information to decide whether a statement is true or false for the counterfactual examples.

\begin{table}[H]
\centering
\resizebox{1\columnwidth}{!}{
\begin{tabular}{llccc} 
\toprule
\textbf{Data} & \textbf{Model} & \textbf{PT Size} & \multicolumn{1}{l}{\textbf{Val}} & \multicolumn{1}{l}{\textbf{Val}} \\ 
\midrule
Counterfactual           & base & 128 &       & 82.0 \\
Counterfactual w/o table  & base & 128 &       & 76.0 \\
\midrule
Synthetic             & base & 128 & 94.3 \\
Synthetic w/o table    & base & 128 & 77.8 \\
\midrule
Synthetic + Counterfactual & base & 128 &  93.7 & 79.3 \\
                        & base & 256 &  98.0 & 83.9 \\
                        & base & 512 &  98.4 & 84.3 \\
\midrule
Synthetic + Counterfactual & large & 128 & 94.3 & 81.0 \\
                        & large & 256 & 98.5 & 86.8 \\
                        & large & 512 & 98.9 & 87.3 \\
\bottomrule
\end{tabular}}
\caption{Accuracy on synthetic (Val) and counterfactual held-out sets (Val) of the pre-traininig data.}
\label{tab:apx-result-data}
\end{table}

In table~\ref{tab:apx-supported} we show the ablation results when removing the counterfactual statements that lack a supporting entity, that is a second entity that appears in both the table and sentence. This increases the probability that our generated negative pairs are incorrect, but it also discards 7 out of 8 examples, which ends up hurting the results.

\begin{table}[H]
\centering
\resizebox{1\columnwidth}{!}{
\begin{tabular}{llccc} 
\toprule
\textbf{Data} & \textbf{Val}\\ 
Synthetic                           & 77.6 \\
\midrule
Counterfactual                      & 75.5 \\
Counterfactual + Synthetic          & 78.6 \\
\midrule
Counterfactual (only supported)              & 73.6 \\
Counterfactual (only supported) + Synthetic  & 77.1 \\
\bottomrule
\end{tabular}}
\caption{Comparisons of training on counterfactual data with and without statements that don't have support mentions.}
\label{tab:apx-supported}
\end{table}

\section{Salient Groups Definition}
\label{sec:slice_words}

In table~\ref{tab:slice_words} we show the words that are used as markers to define each of the groups. We first identified manually the operations that were most often needed to solve the task and found relevant words linked with each group. The heuristic was validated by manually inspecting 50 samples from each group and observing higher than 90\% accuracy.

\begin{table}[H]
\centering
\resizebox{1.0\columnwidth}{!}{
\begin{tabular}{l|l}
\toprule													
\textbf{Slice}	&	\textbf{Words}	\\
\midrule													
\textbf{Aggregations} & total, count, average, sum,\\& amount, there, only \\
\textbf{Superlatives} & first, highest, best,\\& newest, most, greatest, latest, \\& biggest and their opposites \\
\textbf{Comparatives} & than, less, more, better, \\ & worse, higher, lower, shorter, same \\
\textbf{Negations}    & not, any, none, no, never \\
\bottomrule
\end{tabular}
}
\caption{Trigger words for different groups.}
\label{tab:slice_words}
\end{table} 
\end{document}
