\documentstyle[11pt,epsfig]{article}

\setlength{\textheight}{8.8in}
\setlength{\textwidth}{6.5in}
\setlength{\evensidemargin}{-0.18in}
\setlength{\oddsidemargin}{-0.18in}
\setlength{\headheight}{0in}
\setlength{\headsep}{10pt}
\setlength{\topsep}{0in}
\setlength{\topmargin}{0.0in}
\setlength{\itemsep}{0in}      \renewcommand{\baselinestretch}{1.2}
\parskip=0.080in

\newcommand {\ignore} [1] {}

\newtheorem{theorem}{Theorem}[section]
\newtheorem{lemma}[theorem]{Lemma}
\newtheorem{fact}[theorem]{Fact}
\newtheorem{corollary}[theorem]{Corollary}
\newtheorem{definition}{Definition}[section]
\newtheorem{proposition}[theorem]{Proposition}
\newtheorem{observation}[theorem]{Observation}
\newtheorem{claim}[theorem]{Claim}
\newtheorem{assumption}[theorem]{Assumption}
\newtheorem{notation}[theorem]{Notation}
\newenvironment{proof}{\noindent{\bf Proof:\/}}{\hfill \vskip 0.1in}
\newenvironment{proofsp}{\noindent{\bf Proof}}{\hfill \vskip 0.1in}
\pagenumbering{arabic}

\begin{document}

\date{}

\title{Approximating minimum-cost edge-covers of crossing biset-families\thanks{Part of this paper 
appeared in the preliminary version \cite{N}.}}

\author{
Zeev Nutov \\
\small The Open University of Israel \\
\small {\tt nutov@openu.ac.il}
}

\maketitle

\begin{abstract}
An ordered pair  of subsets of a groundset  is called a {\em biset} if ;
 is the {\em co-biset} of .
Two bisets  
{\em intersect} if  and 
{\em cross} if both  and .
The intersection and the union of two bisets  is defined by
 and .
A biset-family  is {\em crossing (intersecting)} if 
 
for any  that cross (intersect). 
A directed edge covers a biset  if  it goes from  to .
We consider the problem of covering a crossing biset-family  by a minimum-cost set of directed edges.
While for intersecting , a standard primal-dual algorithm computes an optimal solution,
the approximability of the case of crossing  is not
yet understood, as it includes several NP-hard problems, for which a poly-logarithmic approximation
was discovered only recently or is not known.
Let us say that a biset-family  is {\em -regular} if 
 for any  
with  that intersect.
In this paper we obtain an -approximation algorithm for arbitrary crossing ;
if in addition both  and the family of co-bisets of  are -regular, 
our ratios are:
 if  for all ,
and 
 
if  for all .
Using these generic algorithms, we derive for some network design problems 
the following approximation ratios:
 for {\sf -Connected Subgraph},  and 
 
for {\sf Subset -Connected Subgraph} when all edges with positive cost have their endnodes in the subset.  
\end{abstract}

\section{Introduction}

\subsection{Problem definition and main results}

Following \cite{F-R}, an ordered pair  of subsets of a groundset  is called a {\em biset} 
if ;  is the {\em inner part} and  is the {\em outer part} of ,
and  is the {\em boundary} of . 
The {\em co-biset} of  is the biset .
Any set  can be considered as a biset  with . 

\begin{definition} \label{d:1}
Two bisets  on  
{\em intersect} if  and are {\em disjoint} otherwise;
 {\em cross} if  and .
The intersection and the union of bisets  is defined by
 and .
We say that a biset-family  is: 
\begin{itemize}
\item
{\em crossing (intersecting)} if  
for any  that cross (intersect).
\item
{\em -regular} if  
for any  with  that intersect.
\end{itemize}
\end{definition}

A biset  is {\em proper} if  are both nonempty.
All biset-families in this paper are assumed to contain only proper bisets.
A directed edge  leaves/covers a (proper) biset  if it goes from  to .
An edge-set/graph  is an {\em edge-cover} of  
if every biset in  is covered by some edge in .
We consider the following generic problem.

\vspace{0.1cm}

\begin{center} 
\fbox{
\begin{minipage}{0.965\textwidth}
\noindent
{\sf Biset-Family Edge-Cover}  \\ 
{\em Instance:} \ A directed graph  with edge-costs  and a biset-family  on . \\
{\em Objective:}  Find a minimum-cost edge-cover  of .
\end{minipage}
}
\end{center}

\vspace{0.1cm}

Given two bisets  we write  and 
say that  contains  if 
or if  and ; similarly,  and  
properly contains  if  or if  and .

\begin{definition}
A biset  is a {\em core} of a biset-family ,
or an {\em -core} for short, 
if  contains no biset in , namely,
if  is an inclusion-minimal member of .
Let  denote the family of -cores and let 
 denote the number of -cores.
\end{definition}

In the {\sf Biset-Family Edge-Cover} problem,  may not be given explicitly,
and a polynomial in  implementation of our algorithms requires that certain queries related 
to  can be answered in polynomial time.
Given an edge set  on , the {\em residual family} 
of  consists of all members of  that are uncovered by the edges of .
It is known that if  is crossing or intersecting, so is , for any .
The {\em co-family} of  is the biset-family 
 of co-bisets of the bisets in .
It is easy to see that  is crossing if, and only if, its co-family is crossing,
and that  covers  if, and only if, the reverse edge-set of  covers the co-family 
of . We assume that for any edge set  on  and any  we are able to compute 
in polynomial time the cores of the biset-family 
 
and also the cores of its co-family, or to determine that  is empty.
In specific graph problems we consider, this can be implemented in polynomial time using the 
Ford-Fulkerson Max-Flow Min-Cut Algorithm; we omit the somewhat standard implementation details.

For intersecting , {\sf Biset-Family Edge-Cover} can be solved in polynomial time by a 
standard primal-dual algorithm; in fact, even a more general problem of covering an intersecting supermodular 
biset-function by a digraph can also be solved in polynomial time \cite{F-R}. 
However, the case of crossing  includes the min-cost {\sf -Connectivity Augmentation} problem
which is NP-hard, and its approxi\-mability is not yet understood.
Given a biset-family  let .
It is known that any crossing set-family  
(namely, a crossing biset-family  with ) 
is decomposable into two intersecting families 
 and 
, where  is arbitrary,
such that an edge-set  covers  if, and only if,  covers  
and the reverse edge-set of  covers .
This implies ratio  for {\sf Biset-Family Edge-Cover} for crossing 
with .
In a similar way we can decompose any crossing  into 
 intersecting biset-families. 
This implies ratio  for {\sf Biset-Family Edge-Cover} with crossing . 
Using ideas from \cite{RW,KN2,FL,N,N-subs}, we give approximation algorithms with logarithmic ratios.

For an edge-set or a graph  and a biset  on 
let  denote the set of edges in  covering . 
Let  denote the optimal value of an LP-relaxation for covering
a biset-family , namely,


Let  denote the th harmonic number.
Our main result is the following.

\vspace{0.2cm}

\begin{theorem} \label{t:FL} 
{\sf Biset-Family Edge-Cover} with crossing  (and a directed graph ) 
admits a polynomial time algorithm that computes a solution of cost at most
 where: 
\begin{itemize}
\item[{\em (i)}]
 for arbitrary crossing . 
\item[{\em (ii)}]

if both  and the co-family of  are -regular 
and if  for all .
\item[{\em (iii)}]
 if both  
and the co-family of  are -regular and if .
\end{itemize}
\end{theorem}

We note that Theorem~\ref{t:FL} easily extends to the undirected case,
with a loss of a factor of  in the approximation ratio.

\subsection{Related work and applications}

A directed/undirected graph is {\em -connected} if there are
 internally-disjoint paths from every its node to the other.
A fundamental problem in network design is the following: 

\vspace{0.1cm}


\begin{center} 
\fbox{
\begin{minipage}{0.965\textwidth}
\noindent
{\sf -Connected Subgraph} \\
{\em Instance:}
\ A graph  with edge-costs  and an integer . \\
{\em Objective:}
Find a minimum cost -connected spanning subgraph of .
\end{minipage}
}
\end{center}

\vspace{0.1cm}

We refer the reader to \cite{N,FL,KN-sur} for a history of the problem.
Let the {\sf -Connectivity Augmentation} problem be the restriction of 
{\sf -Connected Subgraph} to instances in which  contains an -connected spanning 
subgraph  of cost zero, and we seek to increase at minimum cost
the connectivity of  from  to . 
The following statement is known, and its parts were implicitly proved in \cite{FJ} and \cite{RW}, see also \cite{KN2}.

\begin{proposition} \label{p:RW}
{\sf -Connectivity Augmentation} is a particular case of {\sf Biset-Family Edge-Cover} with crossing 
, such that  for all , 
and such that both  and the co-family of  are -regular.
Furthermore, if the latter problem admits a polynomial time algorithm that computes a solution
of cost , then {\sf -Connected Subgraph}
admits a polynomial time algorithm that computes a solution 
of cost .
In particular, if  is increasing in  then the cost of the solution
computed , where 

is the optimal value of a natural LP-relaxation for the problem.
\end{proposition}

Thus part~(ii) of Theorem~\ref{t:FL} implies the following result from \cite{N}.

\begin{corollary} [\cite{N}] \label{c:aug}
{\sf -Connectivity Augmentation} admits a polynomial time algorithm 
that computes a solution of cost 
;
the approximation ratio is , unless .
The problem also admits a polynomial time algorithm that computes a solution of cost
.
{\sf -Connected Subgraph} admits an 
-approximation algorithm;
the ratio is , unless .
The problem also admits a polynomial time algorithm that computes a solution of cost
.
\end{corollary}

Now let us consider the following known generalization of the {\sf -Connected Subgraph} problem.
Let us say that a subset  of nodes of a directed/undirected graph is -connected if in the graph 
there are  internally-disjoint paths from every node in  to any other node in .

\vspace{0.1cm}

\begin{center} 
\fbox{
\begin{minipage}{0.965\textwidth}
\noindent
{\sf Subset -Connected Subgraph} \\
{\em Instance:}
\ A graph  with edge-costs , , and an integer . \\
{\em Objective:}
Find a minimum cost subgraph of  in which  is -connected.
\end{minipage}
}
\end{center}

\vspace{0.1cm}

Note that the {\sf -Connected Subgraph} problem is a particular case of 
{\sf Subset -Connected Subgraph} when .
Let the {\sf Subset -Connectivity Augmentation} problem be the restriction of 
{\sf Subset -Connected Subgraph} to instances in which  contains a 
subgraph  of cost zero such that  is -connected in ,
and we seek to increase at minimum cost
the connectivity of  from  to . 
When the costs are arbitrary, {\sf Subset -Connected Subgraph} is unlikely to admit 
a polylogarithmic approximation \cite{KKL} (see also \cite{LN} for a simpler proof).
The currently best known ratio for this problem for  is
,
where  for undirected graphs and  for directed graphs, and
 is the ratio for the rooted version of the problem \cite{L-subs,N-subs}; 
currently,  for undirected graphs \cite{N-focs}, and 
 for directed graphs. For  the best ratio is .
We consider the version of the problem when every edge with positive cost has its both endnodes
in . Then a similar statement to the one in Proposition~\ref{p:RW} applies, 
except that  is a biset-family on  and  for all .
Furthermore, when , then by applying  times an approximation algorithm for
the rooted version of the problem, we can reduce the number of cores to ;
such a procedure is described in \cite{L-subs,N-subs}.
The rooted version when every edge has its tail in  admits a polynomial time algorithm \cite{F-R}.
Thus parts (i) and (iii) of Theorem~\ref{t:FL} imply the following.

\begin{corollary} \label{t:aug'}
In the case when every edge with positive cost has its both endnodes in ,
{\sf Subset -Connectivity Augmentation} admit a polynomial time algorithm that computes a solution 
of cost , where 

and .
{\sf Subset -Connected Subgraph} admits an -approximation algorithm.
\end{corollary}

Cheriyan and Laekhanukit \cite{ChL} considered the following directed edge-connectivity problem,
that gene\-ra\-lizes the {\sf Subset -Connected Subgraph} problem.
Given two disjoin subsets  in a graph , we say that  is -edge-outconnected from  to ,
or that  is --edge-connected,
if  has  edge-disjoint -paths for every . 

\begin{center} 
\fbox{
\begin{minipage}{0.965\textwidth}
\noindent
{\sf --Edge-Connected Subgraph} \\
{\em Instance:}
\ A directed graph  with edge-costs , disjoint subsets , 
and an integer . \\
{\em Objective:}
Find a minimum cost -edge-outconnected from  to  subgraph of .
\end{minipage}
}
\end{center}

In the so called ``standard version'' of the problem we have .
Let the {\sf --Edge-Connectivity Augmentation} problem be the restriction of 
{\sf --Edge-Connected Subgraph} to instances in which  contains a
subgraph  of cost zero such that  is --connected in ,
and we seek to increase at minimum cost the -connectivity from  to . 

Let us say that two sets  {\em -cross} if .
A set-family  is {\em -crossing} 
if  for any  that -cross.
Generalizing the algorithm of Fackaroenphol and Laekhanukit \cite{FL} for the 
{\sf -Connected Subgraph} problem, 
Cheriyan and Laekhanukit \cite{ChL} gave an approximation
algorithm with ratio  
for the standard version of the {\sf --Connectivity Augmentation} problem.
They also implicitly proved that it is a particular case of the {\sf Set-Family Edge-Cover} problem 
with , , and -crossing .
Our algorithm in Theorem~\ref{t:FL}(i) easily extends to the problem 
of covering an -crossing family by a minimum-cost edge-set. 
Here we preferred the biset-family setting for simplicity of exposition, and since
the concept of -regularity is not a natural one for -crossing families.
Furthermore, the case of an -crossing set family  
is reduced to the case of a crossing biset-family ,
where for every set  there is a biset  in ;
it is not hard to verify that if  is an -crossing family then  is 
a crossing biset-family, and that an edge from  to  covers a set  if, and only if, it covers . 
Thus from Theorem~\ref{t:FL}(i) we have the following generalization of the result of \cite{ChL}. 

\begin{corollary}
{\sf Set-Family Edge-Cover} with -crossing  and  
admits a polynomial time algorithm that computes a solution of cost 
. 
\end{corollary}


\section{Proof of Theorem~\ref{t:FL}}

\subsection{Proof of part~(i)}

Recall that we assume that for any edge set  on  and any  we are able to compute 
in polynomial time the cores of the biset-family 
 
and also the cores of its co-family, or to determine that  is empty.
Note that if  is crossing, then  has a unique core, and the 
co-family of  also has a unique core.

 \begin{lemma} \label{l:poly}
A crossing biset-family  has at most  cores and they can be computed in polynomial time.
\end{lemma}
\begin{proof}
For every ordered pair of nodes  we compute the core  
of the biset-family , 
if  is non-empty.
Then  consists from the inclusion-minimal members (cores) 
of the biset-family . 
\end{proof}

\begin{definition}
Given a biset-family  and a core  of  
let  denote the family 
of the bisets in  that contain  and contain no other core of . 
\end{definition}

The following statement can be easily verified. 

\begin{claim} \label{c:iterative-merging}
Let  be a biset-family and  a set of directed edges on .
If for some ,
 covers  and covers no core distinct from , 
then  and 
.
Furthermore, if  is intersection-closed and  covers  
for every -core , 
then every -core contains at least two -cores, 
and thus .
\end{claim}

\begin{lemma} \label{l:HH}
Let  be distinct cores of a crossing biset-family  
and let  and  . 
Then  do not cross. Consequently, no edge covers both .
\end{lemma}
\begin{proof}
Suppose to the contrary that  and  cross.
Then .
Thus  contains some -core .
We cannot have  as  and  is a core, and
we cannot have       as , ,
and  is a core. This gives a contradiction. 
The second statement follows from the observation
an edge covers two bisets  simultaneously if, and only if, it goes from 
 to , and hence  cross.
\end{proof}

\begin{lemma} \label{l:H}
Let  be a core of a crossing biset-family , 
and let . 
If  cross then .
\end{lemma}
\begin{proof}
Since  is crossing, .
Since 
and since , it follows that 
and that .
It remains to prove that  contains no core distinct from .
Suppose to the contrary that  contains a core  distinct from .
Since , none of  contains .
This implies that  cross or  cross,
so  or .
This contradicts that  is a core.
\end{proof}

\begin{lemma} \label{l:intersecting}
Let  be a crossing biset-family on  and let .
Then the co-family 
of  is intersecting, and its cores can be found in polynomial time.
\end{lemma}
\begin{proof}
Let  be the co-bisets of ,
respectively. Suppose that  intersect.
Then  cross, hence ,
by Lemma~\ref{l:H}. The co-bisets of  and  are
 and ,
hence .
This implies that  is an intersecting biset family. 
Now we show how to find the cores of  in polynomial time.
For an -core  let 
 be the set of all edges 
from  to . 
Let . 
We claim that .
To see this, note that for any : \\ 
(i) \  covers all bisets in  that contain ; 
(ii)   does not cover any biset in , by Lemma~\ref{l:HH}. 
Now choose , and for every  compute the core  
of the co-family of . 
The -cores are the inclusion-minimal members of the family . 
\end{proof}

\begin{corollary} \label{c:S-greedy}
{\sf Biset-family Edge-Cover} with crossing 
admits a polynomial time algorithm that given a core  
computes an -cover  of cost 
.
Moreover, , and thus
there exists  such that .
\end{corollary}
\begin{proof}
By Lemma~\ref{l:intersecting}, the co-family 
 of  is intersecting.
Thus, after reversing the edges in , we can apply a standard primal-dual algorithm to compute an 
edge-cover of  of cost ; 
 is the reverse edge set of this cover.
This primal-dual algorithm can be implemented in polynomial time if the cores of 
can be found in polynomial time, which is possible by Lemma~\ref{l:intersecting}.
The second statement of the lemma follows from Lemma~\ref{l:HH}.
\end{proof}

Now we finish the proof of part~(i) of Theorem~\ref{t:FL}.
The algorithm start with .
At iteration , it finds  and  with 
 and 
, and adds  to ; 
such  exists and can be found in polynomial time by 
Lemma~\ref{l:poly},
Claim~\ref{c:iterative-merging},
Lemma~\ref{l:HH}, and
Corollary~\ref{c:S-greedy}.
At each iteration  decreases by , by Claim~\ref{c:iterative-merging}.
Thus at the end of iteration  we have .
Consequently, .
Thus at the end of the algorithm, 
.

\subsection{Proof of part~(ii)} \label{s:main}

The following concept plays a central role in the proof of part~(ii) of Theorem~\ref{t:FL}.

\begin{definition} \label{d:q}
A biset-family  is {\em intersection-closed} if 
for any intersecting .
An intersection-closed biset-family  is {\em -semi-intersecting} if 
 for all , and if 
 for any intersecting  with .
\end{definition}



\begin{fact} \label{f:q}
If a biset-family  is -regular, then the subfamily 
 of  is -semi-intersecting 
(and in particular, is intersection closed) for any .
\end{fact}


The following statement is straightforward.

\begin{lemma} \label{l:HC}
Let  be an intersection-closed biset-family.
If  and  intersect, then .
Thus the -cores are pairwise disjoint.
\end{lemma}

Let .
We will give an algorithm that computes an -cover of cost at most
.
To cover the entire , we apply this algorithm twice: 
once on  and once on the ''reversed'' instance with the biset-family being the co-family 
 of  and with the reverse edge-set  of ;
after a solution  to the reversed instance is computed, we return the reversed edge-set of .
The union of the two partial solutions computed covers the entire ,
and has cost as stated in part~(ii) of Theorem~\ref{t:FL}.

The algorithm that computes an -cover of cost at most 
 is as follows.
Start with , and iteratively, until , find and add to 
a cover  of cost 
of all families  of -cores, as in Corollary~\ref{c:S-greedy}.
By Lemma \ref{l:HC} and Claim~\ref{c:iterative-merging}, 
after step  we have  for every -core .
Hence the number of iterations is at most . 
As at every iteration we add to  an edge set of cost , the total
cost of the -cover computed is .

In the next section we will prove the following theorem, which is our main technical result,
and which we believe is of independent interest.

\begin{theorem} \label{t:q-int}
{\sf Biset-Family Edge-Cover} with -semi-intersecting biset-family  admits a polynomial time algorithm 
that computes an edge-set  such that  
and . 
\end{theorem}

Note that for  any intersecting biset-family is -semi-intersecting, hence 
-semi-intersecting biset-families generalize intersecting biset-families, 
and then the algorithm in Theorem~\ref{t:q-int} computes an optimal -cover of cost .

Part~(ii) of Theorem~\ref{t:FL} follows from Theorem~\ref{t:q-int} and part~(i).
Compute an edge set  as in Theorem~\ref{t:q-int} and then compute 
a cover  of the residual family  using the algorithm from part~(i). 
The cost of the -cover  computed is bounded by


\subsection{Proof of part~(iii)}

Note that the algorithm from part~(ii), if applied on an arbitrary crossing -regular 
biset-family  with , 
returns an edge set  of costs ,
such that the residual family  has the following property: 
the size of the inner part of every biset
in  or in the co-family of  is larger than .
Consequently, to prove part~(iii) of Theorem~\ref{t:FL} it is sufficient to prove
the following.

\begin{lemma} \label{l:q}
{\sf Biset-Family Edge-Cover} with crossing 
admits a polynomial time algorithm that computes an edge-cover 
of  of cost at most 
,
provided that  holds
for every biset  that belongs to  or to the co-family of .
\end{lemma}

In the rest of this section we prove Lemma~\ref{l:q}. The key observation is the following.

\begin{lemma} \label{l:lov}
Let  be a crossing biset-family on  with 
for every biset  in  or in the co-family of .
Then there exists  of size 
 
such that  for every .
\end{lemma}
\begin{proof}
Consider the hypergraph 
of the inner parts of the cores of .
We combine the following two observations.  
\begin{itemize}
\item[(i)]
The function  on  defined by  for all  is a fractional 
hitting-set of  (namely  for all )
of value . 
\item[(ii)]
The maximum degree in the hypergraph  is at most .
\end{itemize}
Given observations (i) and (ii), the greedy algorithm 
computes a subset  as stated.
Observation~(i) follows from the assumption that  for all .
We prove (ii). Since  is crossing, the members of  
are pairwise non-crossing. Thus if  is a set of cores
which inner parts contain the same element , then the sets in 
 are pairwise disjoint.
As each of these sets is an inner part of a biset in the co-family of ,
the number of such sets is at most . Observation~(ii) follows.
\end{proof}

Lemma~\ref{l:q} easily follows from Lemma~\ref{l:lov}.
Note that if  is crossing, then 
for every  the co-family of the biset-family 
 is intersecting.
Thus given an instance of {\sf Biset-Family Edge-Cover} and ,
we can compute in polynomial time an edge-cover  of  of cost
.
Now let  be as in Lemma~\ref{l:lov}. For every  we compute an edge-cover 
 of  as above, and return . 
This concludes the proof of Lemma~\ref{l:q} and thus also the proof of part~(iii)
of Theorem~\ref{t:FL} is now complete.

\section{Proof of Theorem~\ref{t:q-int}} \label{s:q-int}

\begin{definition}
For biset-families  and  on  let 

A biset-family  is {\em weakly-intersecting} if  is 
an intersecting biset-family for every .
\end{definition}

Clearly, any -semi-intersecting biset-family is weakly-intersecting.
Note that if  
and if the members of  are pairwise disjoint, 
then  is an intersecting biset-family, if  is weakly-intersecting.
We will prove the following refinement of Theorem~\ref{t:q-int}.

\begin{lemma} \label{l:P12}
{\sf Biset-Family Edge-Cover} with weakly-intersecting biset-family  admits a polynomial time algorithm 
that computes a sub-family  of pairwise disjoint bisets  
and an edge-set  such that the following two properties hold.
\begin{description}
\item[Property 1:] 
For any -core  the following holds:
\begin{itemize}
\item[{\em (i)}]
If  and  intersect then  intersects 
no -core distinct from . 
\item[{\em (ii)}]
The union  of  and the bisets in  intersecting with 
is not in .
\end{itemize}
\item[Property 2:] 
 is an optimal cover of , thus
. 
\end{description}
\end{lemma}

To see that Lemma~\ref{l:P12} implies Theorem~\ref{t:q-int},
it is sufficient to show that if  is -semi-intersecting, then 
Property~1 implies .
Let  be the inner part of .
Note that the sets  are pairwise disjoint;
this is since the members of  are pairwise disjoint, the -cores are pairwise disjoint, 
and since by Property~1(i) every  intersects at most one -core.
Now observe that if  is -semi-intersecting then , 
since , by Property~1(ii).
Consequently, ,
and Lemma~\ref{l:P12} implies Theorem~\ref{t:q-int}.

In the rest of this section we prove Lemma~\ref{l:P12}.
The algorithm is a variation of a standard primal-dual algorithm
for covering an {\em intersecting} biset-family, and it only needs that the 
-cores can be computed in polynomial time. 
The dual LP of the LP-relaxation {\bf (P)} from the Introduction is:

Given a partial solution  to {\bf (D)}, an edge  is {\em tight}
if the inequality in {\bf (D)} that corresponds to  holds with equality.
The algorithm produces an edge set , a sub-family  of ,
and a dual solution  to {\bf (D)}, such that the following holds: 
 covers ,  is a feasible solution to {\bf (D)},
and (the characteristic vector of)  and  satisfy the complementary slackness conditions: 

\noindent
e; \\
.

{\bf Phase~1} starts with  and  and applies a sequence of iterations.
At each iteration we choose some -core  and do the following:
\begin{enumerate}
\item
Add  to  and exclude from  the bisets contained in .
\item
Raise (possibly by zero) the dual variable corresponding to , 
until some edge  covering  becomes tight, and add  to .
\end{enumerate}

Phase~1 terminates when , namely, when  covers . 

\vspace{0.2cm}

{\bf Phase~2} applies on  ``reverse delete'' like the family  
is the one we want to cover, which means the following.
Let , where  was added after . 
For  downto , we delete  from  if  still covers 
the family .
This can be implemented in polynomial time as follows. 
When an edge  is checked for deletion,
 is covered by  if, and only if, 
no  contains an -core.   
At the end of the algorithm,  is output.

\vspace{0.2cm}

Summarizing, the algorithm is a variation of a standard primal-dual algorithm 
for intersecting biset-families, with the following changes.
\begin{enumerate}
\item
Unlike a standard primal-dual algorithm in which {\em all} the dual variables corresponding to 
cores are raised uniformly, we raise the dual variable of only {\em one} core.
\item
The algorithm maintains a biset-family .
In each iteration, we add to  the corresponding tight -core ,
and exclude from  all the bisets contained in , if any.
\item
While at Phase~1 the algorithm intends to cover the entire biset-family ,
Phase~2 (reverse-delete) is applied like the family  is the one that we want to cover.
\end{enumerate}

Using Lemma~\ref{l:poly}, it is easy to see that the algorithm can be implemented in polynomial time,
under the oracles assumed. Using Lemma~\ref{l:HC} it is also easy to see the following.

\begin{claim} \label{c:feasible}
During the algorithm the following holds:
the members of  are pairwise disjoint,
every edge in  has its tail in the inner part of some , and
 for every .
\end{claim}

Retrospectively, it turns out that our algorithm coincides with some run of an almost 
standard primal-dual algorithm that intends to cover the intersecting biset-family . 
The difference is in item 2 above, as we raise the dual variable of only one core;
it is also possible to raise all the dual variables corresponding to cores,
but this makes the analysis more complicated.
We will show that this modified algorithm still computes an optimal solution.
This will ensure Property~2 in Lemma~\ref{l:P12}; we give a formal proof of Property~2 after
proving that Property~1 holds for  at the end of the algorithm.

The following statement is easily verified, see Figure~\ref{f:sets}.

\begin {figure} 
\centering 
\epsfbox{sets.eps}
   \caption {(a) All types of edges that can cover  or . 
             (b) An -core  and the members of  intersecting .
            }
   \label{f:sets}
\end {figure}

\begin{fact} \label{f:XY}
Let  be bisets and let  be an edge. 
\begin{itemize}
\item[{\em (i)}]
If  covers  or  then  covers  or .
\item[{\em (ii)}]
If  covers  and has tail in  then  covers . 
\item[{\em (iii)}]
If  covers both  and  then  covers both  and .
\end{itemize}
\end{fact}

Let  be the set of edges stored in  at the end of Phase~1.
Then  covers . Note that every edge in  
has its tail in the inner part of some , and that 
 for every .
Thus the following statement implies that Property~1 holds for  after Phase~2, 
even if  is only intersection-closed.

\begin{lemma}
Let  be an edge-cover of an intersection-closed biset-family  and let
 be a sub-family of  of pairwise disjoint bisets 
such that every edge in  has its tail in the inner part of some  and
such that  for every .
Let  be a cover of  and let  be an -core.
Then the following holds (see Figure~\ref{f:sets}(b)):
\begin{itemize}
\item[{\em (i)}]
If  and  intersect then ;
thus  intersects no -core distinct from .
\item[{\em (ii)}]
, where  is the union of  
and the bisets in  intersecting with ;
thus .
\end{itemize}
\end{lemma}
\begin{proof}
We prove (i). Let  and  intersect.  
Note that , since  is intersection-closed.
Hence .
Let .
Then  covers  or , by Fact~\ref{f:XY}(i).
But  does not cover , since , and since  does not 
cover . Hence  covers . 
Consequently,  for any ,
hence .
This implies that the tail of  is in , and it cannot be in the inner part of 
any other -core, since the -cores are pairwise disjoint.
Thus  intersects no -core distinct from .

We prove (ii). 
Suppose to the contrary that there is .
Let  be the biset whose inner part contains the tail of .
Note that .
Let  and 
let  be the union of  and the bisets in 
that intersect . Note that .
Hence the tail of  lies in  and  covers .
Thus  covers , by Fact~\ref{f:XY}(ii).
Applying the same argument on the bisets  we obtain that  covers .
Hence , as  is the only edge in  that covers .
By (i),  covers .
Consequently,  covers both  and ,
and hence  covers both  and , by Fact~\ref{f:XY}(iii).
However , contradicting that .
\end{proof}

For Property~2 it is sufficient to prove the following.

\begin{lemma} If  is weakly-intersecting, then at the end of the algorithm the following holds: 
 co\-vers the biset-family ,
 is a feasible solution to {\em \bf (D)} for , 
and  satisfy the complementary slackness conditions;
hence  and  are optimal solutions.
\end{lemma}
\begin{proof}
It is clear that  covers , 
and that during the algorithm  remains a feasible solution to {\bf (D)}.
Since only tight edges enter , after Phase~1 the {\em Primal Complementary Slackness Conditions} hold for .
So, the only part that requires proof is that after Phase~2, the 
{\em Dual Complementary Slackness Conditions} hold for  and .

\vspace{0.1cm}

\noindent
{\em  Claim:}
{\em Consider an arbitrary  with  and an edge .
Then there exists  such that  and 
 for some .}

\vspace{0.1cm}

\noindent
{\em Proof:}
Such  can be chosen as any member of 
which becomes uncovered if we delete (instead of keeping)
 at the reverse delete step when  was considered for deletion;
note that the algorithm decided to keep , hence such  exists.
Moreover, since the edges were deleted in the reverse order,
. 
Obviously,  and  and  intersect.
Finally, to see that  note that: 
(i) at any iteration before  was added,  was uncovered;
(ii) since , there was an iteration before  was added 
at which  was an -core. Hence , by Lemma~\ref{l:HC}. 
\hfill 

Now assume to the contrary that there is  with  
such that there are , .
Let  such that .
Let  be bisets for  as in the Claim above. 
so ,  and 
.
In particular,  intersect and .
Thus , since  
and since  is weakly intersecting.
Consequently, there is an edge . 
This implies that  or , by Fact~\ref{f:XY}(i).
Consequently,  or .
Since the tail of each one of  is in , so is the tail of .
The head of  is in , since  covers .
We conclude that . This is a contradiction since 
, , and . 
\end{proof}

The proof of Lemma~\ref{l:P12}, and thus also of Theorem \ref{t:q-int} is complete.

\section{Concluding remarks and open problems}
\ignore{-----------------
Cheriyan and Laekhanukit \cite{ChL} considered the following generalization
of several min-cost connectivity problems.
In the {\sf - Connectivity Augmentation} problem
we are given a --connected graph 
(namely, in  are  edge-disjoint -paths for every ) 
and an edge-set  with costs. 
The goal is to find a minimum-cost edge set  such that the graph
 is --connected.
They gave an -approximation algorithm for the case when 
every positive edge has tail in  and head in .
This version includes the version of the {\sf Subset -Connected Subgraph} problem considered in this paper.
Note however that unless , our ratio for this problem is better,
because we use the -regularity property, which does not have a natural extension 
to the {\sf - Connectivity Augmentation} problem.

The following problem includes the problem considered by Cheriyan and Laekhanu\-kit \cite{ChL},
and generalizes the {\sf Biset-Family Edge-Cover} problem with crossing  
(see the reduction in \cite{FJ}).
Given a directed bipartite graph  with costs on the edges
and all edges from  to , and a {\em set-family}  on ,  
find a minimum-cost edge-set  that covers ;
here we say that  is {\em -crossing} if 
for any  with  and 
.
Our algorithm in Theorem~\ref{t:FL}(i) easily extends to the problem 
of covering an -crossing family by a minimum-cost edge-set, as is shown in \cite{N-ST}. 
Here we preferred the biset-family setting for simplicity of exposition, and since
the concept of -regularity is not a natural one for -crossing families.
We note also that Theorem~\ref{t:FL} easily extends to undirected edge-covers 
with a loss of a factor of  in the approximation ratio.
-----------------------------}
The main open question is whether the {\sf -Connectivity Augmentation} 
problem admits a constant ratio approximation algorithm also for large values of ,
say . In this context we mention four papers.
In \cite{LN} is is shown that for values of  close to , 
the approximability of the {\sf -Connectivity Augmentation} problem is the same 
for directed and undirected graphs, up to a factor of . 
Therefore, one should not expect to obtain a constant ratio for undirected graphs only. 
On the positive side, Frank and Jordan \cite{FJ} showed that for directed graphs, {\sf -Connected Subgraph} 
can be solved in polynomial time when the input graph is complete and the costs are in .
For arbitrary costs however, there are two negative results.
In \cite{RWe} Ravi and Williamson gave an example showing that the approximation
ratio of a standard primal-dual algorithm that intends to edge-cover 
the biset-family  as in Fact~\ref{f:q} has approximation ratio .
This does not exclude that some other variation of the primal-dual algorithm, that relies on concepts from \cite{FJ}
has a constant ratio for crossing biset-families.
However recently, Aazami, Cheriyan, and Laekhanukit \cite{ACL} showed that the standard iterative 
rounding method that is based on a standard LP-relaxation for {\sf -Connectivity Augmentation}
(thus intends to edge-cover a crossing -regular biset-family) has approximation ratio . 



\begin{thebibliography}{10}

\bibitem{ACL}
A.~Aazami, J.~Cheriyan, and B.~Laekhanukit.
\newblock A bad example for the iterative rounding method for mincost
  -connected spanning subgraphs.
\newblock Manuscript.

\bibitem{ChL}
J.~Cheriyan and B.~Laekhanukit.
\newblock Approximation algorithms for minimum-cost -{} connected
  digraphs.
\newblock Manuscript, 2010.

\bibitem{FL}
J.~Fackharoenphol and B.~Laekhanukit.
\newblock An -approximation algorithm for the -vertex
  connected subgraph problem.
\newblock In {\em STOC}, pages 153--158, 2008.

\bibitem{F-R}
A.~Frank.
\newblock Rooted -connections in digraphs.
\newblock {\em Discrete Applied Math.}, 157(6):1242--1254, 2009.

\bibitem{FJ}
A.~Frank and T.~Jord\'{a}n.
\newblock Minimal edge-coverings of pairs of sets.
\newblock {\em J. of Comb. Theory B}, 65:73--110, 1995.

\bibitem{KKL}
G.~Kortsarz, R.~Krauthgamer, and J.~R. Lee.
\newblock Hard\-ness of approximation for vertex-connectivity net\-work design
  problems.
\newblock {\em {SIAM} Journal on Computing}, 33(3):704--720, 2004.

\bibitem{KN2}
G.~Kortsarz and Z.~Nutov.
\newblock Approximating -node connected subgraphs via critical graphs.
\newblock {\em {SIAM} Journal on Computing}, 35(1):247--257, 2005.

\bibitem{KN-sur}
G.~Kortsarz and Z.~Nutov.
\newblock Approximating minimum-cost connectivity problems.
\newblock In T.~F. Gonzalez, editor, {\em {\em Chapter 58 in} Approximation
  Algorithms and Metaheuristics}. Chapman \& Hall/CRC, 2007.

\bibitem{L-subs}
B.~Laekhanukit.
\newblock An improved approximation algorithm for minimum-cost subset
  -connectivity.
\newblock In {\em ICALP}, pages 13--24, 2011.

\bibitem{LN}
Y.~Lando and Z.~Nutov.
\newblock Inapproximability of survivable networks.
\newblock {\em Theortical Computer Science}, 410(21-23):2122--2125, 2009.

\bibitem{N-focs}
Z.~Nutov.
\newblock Approximating minimum cost connectivity problems via uncrossable
  bifamilies.
\newblock Manuscript 2010. Preliminary version in FOCS 2009, pages 417-426.

\bibitem{N}
Z.~Nutov.
\newblock An almost {}-approximation for -connected subgraphs.
\newblock In {\em SODA}, pages 922--931, 2009.

\bibitem{N-subs}
Z.~Nutov.
\newblock Approximating subset -connectivity problems.
\newblock In {\em WAOA}, pages 9--20, 2011.

\bibitem{RW}
R.~Ravi and D.~P. Williamson.
\newblock An approximation algorithm for minimum-cost vertex-connectivity
  problems.
\newblock {\em Algorithmica}, 18:21--43, 1997.

\bibitem{RWe}
R.~Ravi and D.~P. Williamson.
\newblock Erratum:~an~approxi\-mation algorithm for minimum-cost
  vertex-connectivity problems.
\newblock {\em Algorithmica}, 34(1):98--107, 2002.

\end{thebibliography}


\end{document}
