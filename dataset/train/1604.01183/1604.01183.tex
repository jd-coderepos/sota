\documentclass[11pt]{article}   \usepackage[letterpaper,hmargin=2.1cm,vmargin=3cm]{geometry}
\usepackage{graphicx}
\usepackage{amsmath}            \usepackage{amssymb}            \usepackage{amsthm}             \usepackage{amsfonts}           \usepackage{url}                \usepackage{color}              \usepackage{multirow}           \usepackage{cite}               \usepackage{hyperref}           \hypersetup{colorlinks=true,linkcolor=[rgb]{0.75,0,0},citecolor=[rgb]{0,0,0.75}}\usepackage{epstopdf}           \usepackage{thmtools}			\usepackage{thm-restate}
\declaretheorem[name=Theorem]{theorem}
\declaretheorem[name=Lemma,numberwithin=section]{lemma}
\def\proof{\par\noindent{\bf Proof\/}:}
\def\endproof{\hfill\hbox{\sqcap}\par\bigskip}
\def\qedsymbol{\hfill\hbox{\sqcap}}
\newcommand{\floor}[1]{\left\lfloor #1\right\rfloor}
\newcommand{\ceil}[1]{\left\lceil #1\right\rceil}
\newcommand{\ang}[1]{\langle #1\rangle}
\newcommand{\RE}{\mathbb{R}}    \newcommand{\ZZ}{\mathbb{Z}}    \newcommand{\eps}{\varepsilon}  \newcommand{\ST}{\,:\,}         \newcommand{\sq}{\square}
\newcommand{\SoftOh}{\widetilde{O}}
\newcommand{\SoftOmega}{\widetilde{\Omega}}
\newcommand{\OO}[1]{O\kern-2pt\left(#1\right)}  \newcommand{\half}[1]{\frac{#1}{2}}
\newcommand{\inv}[1]{\frac{1}{#1}}
\newcommand{\alg}{\textrm{SplitReduce}}
\newcommand{\diam}{\mathrm{diam}}
\newcommand{\radius}{\mathrm{radius}}
\newcommand{\vol}{\mathrm{vol}}
\newcommand{\area}{\mathrm{area}}
\newcommand{\conv}{\mathrm{conv}}
\newcommand{\Vor}{\mathrm{Vor}}
\newcommand{\etal}{\textit{et al.}}
\renewcommand{\P}{\kern+1pt}    \newcommand{\N}{\kern-2pt}      \newcommand{\NN}{\kern-4pt}     \newcommand{\polar}[1]{\mathrm{polar}(#1)}            \newcommand{\polarX}[2]{\mathrm{polar}_{#1}(#2)}      

\begin{document}

\title{Approximate Polytope Membership Queries\thanks{Preliminary results of this paper appeared in ``Approximate Polytope Membership Queries'', in
Proc. ACM Sympos. Theory Comput. (STOC), 2011, 579--586 and ``Polytope Approximation and the Mahler Volume'', in Proc. ACM-SIAM Sympos. Discrete Algorithms (SODA), 2012, 29--42.}
}

\author{Sunil Arya\thanks{Research supported by the Research Grants Council of Hong Kong, China under project numbers 610108 and 16200014.}\\
		Department of Computer Science and Engineering \\
		Hong Kong University of Science and Technology \\
		Clear Water Bay, Kowloon, Hong Kong\\
		arya@cse.ust.hk \\
		\and
	Guilherme D. da Fonseca\thanks{Research supported by CNPq and FAPERJ grants.}\\
		Universit\'e Clermont Auvergne and LIMOS \\
		Clermont-Ferrand, France\\
		fonseca@isima.fr \\
		\and
	David M. Mount\thanks{Research supported by NSF grant CCF--1618866.}\\
		Department of Computer Science and \\
		Institute for Advanced Computer Studies \\
		University of Maryland \\
		College Park, Maryland 20742 \\
		mount@cs.umd.edu \\
}

\date{}

\maketitle

\begin{abstract}
In the polytope membership problem, a convex polytope  in  is given, and the objective is to preprocess  into a data structure so that, given any query point , it is possible to determine efficiently whether . We consider this problem in an approximate setting. Given an approximation parameter , the query can be answered either way if the distance from  to 's boundary is at most  times 's diameter. We assume that the dimension  is fixed, and  is presented as the intersection of  halfspaces. Previous solutions to approximate polytope membership were based on straightforward applications of classic polytope approximation techniques by Dudley (1974) and Bentley {\etal} (1982). The former is optimal in the worst-case with respect to space, and the latter is optimal with respect to query time. 
 
We present four main results. First, we show how to combine the two above techniques to obtain a simple space-time trade-off. Second, we present an algorithm that dramatically improves this trade-off. In particular, for any constant , this data structure achieves query time roughly  and space roughly . We do not know whether this space bound is tight, but our third result shows that there is a convex body such that our algorithm achieves a space of at least . Our fourth result shows that it is possible to reduce approximate Euclidean nearest neighbor searching to approximate polytope membership queries. Combined with the above results, this provides significant improvements to the best known space-time trade-offs for approximate nearest neighbor searching in . For example, we show that it is possible to achieve a query time of roughly  with space roughly , thus reducing by half the exponent in the space bound. 
\end{abstract}


\textbf{Keywords:}
Polytope membership, nearest-neighbor searching, geometric retrieval, space-time trade-offs, approximation algorithms, convex approximation, Mahler volume.

\section{Introduction} \label{sec:intro}


Convex polytopes are key structures in many areas of mathematics and computation. In this paper, we consider a fundamental search problem related to convex polytopes. Let  denote a convex body in , that is, a closed, convex set of bounded diameter that has a nonempty interior. We assume that  is presented as the intersection of  closed halfspaces. (Our results generally hold for any representation that satisfies the access primitives given at the start of Section~\ref{sec:split-reduce}.) The \emph{polytope membership problem} is that of preprocessing  so that it is possible to determine efficiently whether a given query point  lies within . Throughout, we assume that the dimension  is a fixed constant that is at least . 

It follows from standard results in projective duality that polytope membership is equivalent to answering halfspace emptiness queries for a set of  points in . In dimension , it is possible to build a data structure of linear size that can answer such queries in logarithmic time~\cite{textbook}. In higher dimensions, however, all known exact data structures with roughly linear space have a query time
of \footnote{Throughout, we use  and  as variants of  and , respectively, that ignore logarithmic factors. We use ``'' to denote base-2 logarithm.}
\cite{Mat92}, which is unacceptably high for many applications. Polytope membership is a special case of polytope intersection queries \cite{ChD87,DoK83,BaL15}. Barba and Langerman \cite{BaL15} showed that for any fixed , it is possible to preprocess polytopes in  so that given two such polytopes that have been translated and rotated, it can be determined whether they intersect each other in time that is logarithmic in their total combinatorial complexity. However, the preprocessing time and space grow as the combinatorial complexity of the polytope raised to the power .

The lack of efficient exact solutions motivates the question of whether polytope membership queries can be answered approximately. Let  be a positive real parameter, and let  denote 's diameter. Given a query point , an \emph{-approximate polytope membership query} returns a positive result if , a negative result if the distance from  to its closest point in  is greater than , and it may return either result otherwise (see Figure~\ref{fig:apxmembership}(a)). Polytope membership queries, both exact and approximate, arise in many application areas, such as linear-programming and ray-shooting queries~\cite{lp-Chan, lp-Chan2, lp-Ramos, lp-Mat, lp-Mat2}, nearest neighbor searching and the computation of extreme points~\cite{hull-Chan,Clarkson-ANN}, collision detection~\cite{collision2}, and machine learning~\cite{SVM}. 

\begin{figure}[htbp]
  \centerline{\includegraphics[scale=0.40]{Figs/apxmembership}}
  \caption{Approximate polytope membership: (a) problem formulation, (b) outer -approximation.}
  \label{fig:apxmembership}
\end{figure}


Existing solutions to approximate polytope membership queries have been based on straightforward applications of classic polytope approximation techniques. We say that a polytope  is an \emph{outer -approximation} of  if , and the Hausdorff distance between  and  is at most  (see Figure~\ref{fig:apxmembership}(b)). An \emph{inner -approximation} is defined similarly but with . Dudley~\cite{Dudley} showed that there exists an outer -approximating polytope for any bounded convex body in  formed by the intersection of  halfspaces, and Bronshteyn and Ivanov~\cite{BrIv} proved an analogous bound on the number of vertices needed to obtain an inner -approximation. Both bounds are known to be asymptotically tight in the worst case (see, e.g., \cite{Bro08}). These results have been applied to a number of problems, for example, the construction of coresets~\cite{kernel-survey}. By checking that a given query point lies within each of the halfspaces of Dudley's approximation, -approximate polytope membership queries can be answered with space and query time of .

\medskip

The principal contribution of this paper is to show that it is possible to achieve nontrivial space-time trade-offs for approximate polytope membership. In order to motivate our methods, in Section~\ref{sec:prelim} we present a simple space-time trade-off (stated in the following theorem), which is based on a straightforward combination of the approximations of Dudley~\cite{Dudley} and Bentley~{\etal}~\cite{BFP}. Throughout, we will treat  and  as asymptotic quantities, while the dimension  is a constant.

\begin{theorem}[Simple Trade-off] \label{thm:simple-trade-off}
Given a convex polytope  in , a positive approximation parameter , and a real parameter , there is a data structure for -approximate polytope membership queries that achieves

The constant factors in the space and query time depend only on  (not on , , or ). 
\end{theorem}


We will strengthen this trade-off significantly in Sections~\ref{sec:split-reduce} and~\ref{sec:firstbound}. We will show that it is possible to build a data structure with  space that allows polytope membership queries to be answered in roughly  time, thus reducing the exponent in the query time of Theorem~\ref{thm:simple-trade-off} (for ) by . Further, we will show that by iterating a suitable generalization of this construction, we can obtain the following trade-offs.

\begin{theorem} \label{thm:membership-ub}
Given a convex polytope  in , an approximation parameter , and a real constant , there is a data structure for  -approximate polytope membership queries that achieves

The constant factors in the space and query time depend only on  and  (not on  or ).
\end{theorem}


The above space bound is a simplification, and the exact bound is given in Lemma~\ref{lem:trade-off-ub}. Both bounds are piecewise linear in  (with breakpoints at powers of two), but the bounds of Lemma~\ref{lem:trade-off-ub} are continuous as a function of . The resulting space-time trade-off is illustrated below in Figure~\ref{fig:trade-offs}(a). (The plot reflects the more accurate bounds.) 

The above theorem is intentionally presented in a purely existential form. This is because our construction algorithm assumes the existence of a procedure that computes an -approximating polytope whose number of bounding hyperplanes is at most a constant factor larger than optimal. Unfortunately, we know of no efficient solution to this problem. In Lemma~\ref{lem:preproc-time} we will show that, if the input polytope is expressed as the intersection of  halfspaces, it is possible to build such a structure in time , such that the space and query times of the above theorem increase by an additional factor of .

Note that, in contrast to many complexity bounds in the area of convex approximation, which hold only in the limit as  approaches zero (see, e.g.,~\cite{Gru93,Bor00}), Theorems~\ref{thm:simple-trade-off} and~\ref{thm:membership-ub} hold for any positive . The data structure of Theorem~\ref{thm:membership-ub} is quite simple. It is based on a quadtree subdivision of space in which each cell is repeatedly subdivided until the combinatorial complexity of the approximating polytope within the cell is small enough to achieve the desired query time. 

We do not know whether the upper bounds presented in Theorem~\ref{thm:membership-ub} are tight for our algorithm. In Section~\ref{sec:lb}, we establish the following lower bound on the trade-off achieved by this algorithm.

\begin{theorem} \label{thm:lb}
In any fixed dimension  and for any constant , there exists a polytope such that for all sufficiently small positive , the data structure described in Theorem~\ref{thm:membership-ub} when generated to achieve query time  has space

\end{theorem}


Although  is not an asymptotic quantity, for the sake of comparing the upper and lower bounds, let us imagine that it is. For roughly the same query time, the  dependencies appearing in the exponents of the upper bounds on space are  for Theorem~\ref{thm:simple-trade-off},  for Theorem~\ref{thm:membership-ub}, and the lower bound of Theorem~\ref{thm:lb} is roughly . The trade-offs provided in these theorems are illustrated in Figure~\ref{fig:trade-offs}(a).


\begin{figure}[htbp]
	\centerline{\begin{tabular}{ccc}
		  \includegraphics[width=6.0cm]{Figs/memtradeoff} & \hspace*{0cm} &
		  \includegraphics[width=6.0cm]{Figs/anntradeoff} \\
		  (a) & \hspace{0cm} & (b)
	\end{tabular}}
  \caption{The multiplicative factors in the exponent of the  terms for (a) polytope membership queries and (b) approximate nearest neighbor (ANN) queries. Each point  represents a term of  for storage and  for query time, where the  term does not depend on .}
  \label{fig:trade-offs}
\end{figure}


\medskip

The second major contribution of this paper is to demonstrate that our trade-offs for approximate polytope membership queries imply significant improvements to the best known space-time trade-offs for approximate nearest neighbor searching (ANN). We are given a set  of  points in . Given any , an \emph{-approximate nearest neighbor} of  is any point of  whose distance from  is at most  times the distance to 's closest point in . The objective is to preprocess  in order to answer such queries efficiently. Data structures for approximate nearest neighbor searching (in fixed dimensions) have been proposed by several authors~\cite{Chan-ANN02, DGK01, HP-AVD, Clarkson-ANN, SSS06}. The best space-time trade-offs~\cite{AVD-JACM} have query times roughly  with storage roughly , for  (see the dashed line in Figure~\ref{fig:trade-offs}(b)). 

These results are based on a data structure called an \emph{approximate Voronoi diagram} (or AVD). In general, a data structure for approximate nearest neighbor searching is said to be in the \emph{AVD model} if it has the general form of decomposition of space (generally a covering) by hyperrectangles of bounded aspect ratio, each of which is associated with a set of representative points. Given any hyperrectangle that contains the query point, at least one of these representatives is an -approximate nearest neighbor of the query point~\cite{AVD-JACM}. The AVD model is of interest because it is possible to prove lower bounds on the performance of such a data structure. In particular, the lower bounds proved in \cite{AVD-JACM} are shown in the dotted curve in Figure~\ref{fig:trade-offs}(b). By violating the AVD model, small additional improvements were obtained in~\cite{proximity}.

Our improvements to ANN searching are given in the following theorem.

\begin{theorem} \label{thm:ann-ub}
Let  be a real parameter,  be a real constant, and  be a set of  points in . There is a data structure in the AVD model for approximate nearest neighbor searching that achieves

The constant factors in the space and query time depend only on  and  (not on ). 
\end{theorem}


The above space bound is a simplification of the more accurate bound given in Lemma~\ref{lem:ann-ub}. (Also see the remarks following the proof of this lemma for further minor improvements achievable by forgoing the AVD model.) As before, both bounds are piecewise linear in  (with breakpoints at powers of two), but the bounds of Lemma~\ref{lem:ann-ub} are continuous as a function of . The resulting space-time trade-off is illustrated in Figure~\ref{fig:trade-offs}(b). (The plot reflects the more accurate bounds of Lemma~\ref{lem:ann-ub}.)

As an example of the strength of the improvement that this offers, observe that in order for the existing AVD-based results to yield a query time of  the required space would be roughly . The exponent in the space bound is nearly twice that given by Theorem~\ref{thm:ann-ub}, which arises by setting . The connection between the polytope membership problem and ANN has been noted before by Clarkson~\cite{Clarkson-ANN}. Unlike Clarkson's, our results hold for point sets with arbitrary aspect ratios.



\medskip

Our data structure is based on a simple quadtree-based decomposition of space. Let  denote the desired query time. We begin by preconditioning  so that it is fat and has at most unit diameter. We then employ a quadtree that hierarchically subdivides space into hypercube cells. The decomposition stops whenever we can declare that a cell is either entirely inside or outside of , or (if it intersects 's boundary) it is locally approximable by at most  halfspaces. This procedure, called {\alg} is presented in Section~\ref{sec:split-reduce}. Queries are answered by descending the quadtree to determine the leaf cell containing the query point, and (if not inside or outside) testing whether the query point lies within the approximating halfspaces. 

Although the algorithm itself is very simple, the analysis of its space requirements is quite involved. In Section~\ref{sec:firstbound}, we begin with a simple analysis, which shows that it is possible to obtain a significant improvement over the Dudley-based approach (in particular, reducing the exponent in the query time by  half with no increase in space). While this simple analysis introduces a number of useful ideas, it is neither tight nor does it provide space-time trade-offs. 

Our final analysis requires a deeper understanding of the local structure of the convex body's boundary. In Section~\ref{sec:cap} we introduce local surface patches of 's boundary, called -dual caps. We relate the data structure's space requirements to the existence of a low cardinality hitting set of the dual caps. We present a two-pronged strategy for generating such a hitting set, one focused on dual caps of large surface area (intuitively corresponding to boundary regions of low curvature) and the other focused on dual caps of small surface area (corresponding to boundary regions of high curvature). We show that simple random sampling suffices to hit dual caps of high surface area, and so the challenge is to hit the dual caps of low surface area. To do this, we show that dual caps of low surface area generate Voronoi patches on a hypersphere enclosing  of large surface area. We refer to this result as the \emph{area-product bound}, which is stated in Lemma~\ref{lem:dual-basic}. This admits a strategy based on sampling points randomly on this hypersphere, and then projecting them back to their nearest neighbor on the surface of . 

The area-product bound is proved with the aid of a classical concept from the theory of convexity, called the \emph{Mahler volume}~\cite{BoMi,Santalo}. The Mahler volume of a convex body is a dimensionless quantity that involves the product of the body's volume and the volume of its polar body. We demonstrate that dual caps and their Voronoi patches exhibit a similar polar relationship. The proof of the area-product bound is quite technical and is deferred to Section~\ref{sec:proof}. 

Armed with the area-product bound, in Section~\ref{sec:membership} we establish our final bound on the space-time trade-offs of {\alg}, which culminates in the proof of Theorem~\ref{thm:membership-ub}. In Section~\ref{sec:preproc} we present details on how the data structure is built and discuss preprocessing time. In Section~\ref{sec:lb} we establish the lower bound result, which is stated in Theorem~\ref{thm:lb}. 

Finally, in Section~\ref{sec:ann} we show how these results can be applied to improve the performance of approximate nearest neighbor searching in Euclidean space. It is well known that (exact) nearest neighbor searching can be reduced to vertical ray shooting to a polyhedron that results by lifting points in dimension  to tangent hyperplanes for a paraboloid in dimension  \cite{ray-shooting-NN,edels}. We show how to combine approximate vertical ray shooting (based on approximate polytope membership) with approximate Voronoi diagrams to establish Theorem~\ref{thm:ann-ub}.

\section{Preliminaries} \label{sec:prelim}


Throughout, we will use asymptotic notation to eliminate constant factors. In particular, for any positive real , let  denote a quantity that is at most , for some constant . Define  and  analogously. We will sometimes introduce constants within a local context (e.g., within the statement of single lemma). To simplify notation, we will often use the same symbol ``'' to denote such generic constants. Recall that we use ``'' to denote the base-2 logarithm. We will use ``'' when the base does not matter. Some of our search algorithms involve integer grids, and for these we assume a model of computation that supports integer division. 

Let  denote a full-dimensional convex body in , and let  denote its boundary. For concreteness, we assume that  is represented as the intersection of  closed halfspaces. Our data structure can generally be applied to any representation that supports access primitives (i)--(iii) given at the start of Section~\ref{sec:split-reduce}.

\subsection{Absolute and Relative Approximations} \label{sec:prelim-abs}


Earlier, we defined approximation relative to 's diameter, but it will be convenient to define the approximation error in absolute terms. Given a positive real , define  to be set of points that lie within Euclidean distance  of . We say that a polytope  is an \emph{absolute -approximation} of a convex body  if 

When we wish to make the distinction clear, we refer to the definition in the introduction as a \emph{relative approximation}. Henceforth, unless otherwise stated approximations are in the absolute sense.

In order to reduce the general approximation problem into a more convenient absolute form, we will transform  into a ``fattened'' body of bounded diameter. Given a parameter , we say that a convex body  is \emph{-fat} if there exist concentric Euclidean balls  and , such that , and . We say that  is \emph{fat} if it is -fat for a constant  (possibly depending on , but not on  or ). The following lemma shows that  can be fattened without significantly altering the approximation parameter. Let  denote the -dimensional axis-aligned hypercube of unit diameter centered at the origin. When  is clear, we refer to this as .

\begin{lemma} \label{lem:fat}
Given a convex body  in  and , there exists an affine transformation  such that  is -fat and . If  is an absolute -approximation of , then  is a relative -approximation of .
\end{lemma}


We omit the proof of this lemma for now, since it is subsumed by Lemma~\ref{lem:precondition-1} below. Our approach will be to map  to , set , and then apply an absolute -approximation algorithm to  (or more accurately, to the result of applying  to each of 's defining halfspaces). Since  is within a constant factor of , the asymptotic complexity bounds that we will prove for the absolute case will apply to the original (relative) approximation problem case as well.


\subsection{Concepts from Quadtrees} \label{sec:prelim-quadtree}


By the above reduction, it suffices to consider the problem of computing an absolute -approxi\-mation to a fat convex body  that lies within . Our construction will be based on a quadtree decomposition of . More formally, we define a \emph{quadtree cell} by the following well known recursive decomposition.  is a quadtree cell, and given any quadtree cell , each of the  hypercubes that result by bisecting each of 's sides by an axis-orthogonal hyperplane is also a quadtree cell. A cell  that results from subdividing  is a \emph{child} of . Clearly, the child's diameter is half that of its parent. The subdivision process defines a -ary tree whose nodes are quadtree cells, and whose leaves are cells that are not subdivided.

It will be useful to define a notion of approximation that is local to a quadtree cell . An obvious definition would be to approximate . The problem with this is that a point  that is close to  need not be close to  (see Figure~\ref{fig:extended-region}(a)). To remedy this we say that a polytope  is an \emph{-approximation of  within}  if

(see Figure~\ref{fig:extended-region}(b)). This definition implies that for any query point , we can correctly answer -approximate polytope membership queries with respect to  by checking whether . We do not care what happens outside of , and indeed  may even be unbounded.

\begin{figure}[htbp]
  \centerline{\includegraphics[scale=0.40]{Figs/extended-region}}
  \caption{(a) and (b): -approximation of  within  and (c): .}
  \label{fig:extended-region}
\end{figure}


As we shall see later, computing an -approximation of  within a quadtree cell , will generally require that we consider  in a region that extends slightly beyond . We define  to be the portion of  that lies within distance  of  (see Figure~\ref{fig:extended-region}(c)). Because  will be of particular interest, we use  as a shorthand for .

In order to apply constructions on quadtree cells of various sizes, it will be convenient to transform all such constructions into a common form. Given a quadtree cell , we define \emph{standardization} to be the application of an affine transformation that uniformly scales and translates space so that  is aligned with the standard quadtree cell . We transform  using this same transformation, and apply the same scale factor to . Although we assume that the input body is contained within , after standardization, the transformed image of  need not be contained within .

\subsection{Polarity and the Mahler Volume} \label{sec:prelim-polar}


Some of our analysis will involve the well known concept of \emph{polarity}. Let us recall some general facts (see, e.g., Eggleston~\cite{Egg58}). Given vectors , let  denote their inner product, and let  denote 's Euclidean length. Given a convex body  define its \emph{polar} to be the convex set

If  contains the origin then  is bounded. Given ,  is simply the closed halfspace that contains the origin whose bounding hyperplane is orthogonal to  and at distance  from the origin (on the same side of the origin as ). The polar has the inclusion-reversing property that  lies within  if and only if  lies within . We may equivalently define  as the intersection of , for all . 

Generally, given , define 

It is easy to see that for any ,  is the closed halfspace at distance  (see Figure~\ref{fig:polar}(a)). Thus,  is a uniform scaling of  by a factor of . In particular, if  is a Euclidean ball of radius  centered at the origin, then  is a concentric ball of radius .

\begin{figure}[htbp]
  \centerline{\includegraphics[scale=0.40]{Figs/polar}}
  \caption{The generalized polar transform and polar body.}
  \label{fig:polar}
\end{figure}


An important concept related to polarity is the \emph{Mahler volume}, which is defined to be the product of the volumes of a convex body and its polar. There is a large literature on the Mahler volume, mostly for centrally symmetric bodies. Later in the paper we will make use of the following bound on the Mahler volume for arbitrary convex bodies (see, e.g., Kuperberg~\cite{Kuperberg}). Given a convex body  in , let  denote its volume, or more formally, its -dimensional Hausdorff measure.

\begin{lemma}[Mahler volume] \label{lem:mahler}
There is a constant  depending only on , such that given a convex body  in , . More generally, given , .
\end{lemma}


\subsection{Simple Approximation Trade-off} \label{sec:prelim-simple-approx}


Before presenting our results, it will be illuminating to see how to obtain simple data structures for approximate polytope membership by combining two existing approximation methods. Let us begin by describing Dudley's approximation. Assuming that  is contained within , let  denote the -dimensional sphere of radius  centered at the origin, which we call the \emph{Dudley hypersphere}. (The value  is not critical; any sufficiently large constant suffices.) For , a set  of points on  is said to be \emph{-dense} if every point of  lies within distance  of some point of . Let  be a -dense set of points on  (see Figure~\ref{fig:dudley-bentley}(a)). By a simple packing argument there exists such a set of cardinality . For each point , let  be its nearest point on 's boundary. For each such point , consider the halfspace containing  that is defined by the supporting hyperplane passing through  that is orthogonal to the line segment . Dudley shows that the intersection of these halfspaces is an outer -approximation of . We can answer approximate membership queries by testing whether  lies within all these halfspaces (by brute force). This approach takes  query time and space.

\begin{figure}[htbp]
  \centerline{\includegraphics[scale=0.33]{Figs/simple-tradeoff}}
  \caption{The -approximations of (a)~Dudley and (b)~Bentley {\etal}, and (c) the simple trade-off. (Not drawn to scale.)}
  \label{fig:dudley-bentley}
\end{figure}


An alternative solution is related to a grid-based approximation by Bentley~{\etal}~\cite{BFP}. Again, we assume that  is contained within . For the sake of illustration, let us think of the -th coordinate axis as pointing upwards. Partition the upper facet of  into a -dimensional square grid with cells of diameter . A packing argument implies that the number of cells is . Extend each of these cells downwards to form a subdivision of  into vertical columns (see Figure~\ref{fig:dudley-bentley}(b)). Trim each column at the highest and lowest points at which it intersects . Together, these trimmed columns define a collection of hyperrectangles whose union contains . The resulting data structure has  space. Given a query point , in  time we can determine the vertical column containing  (assuming a model of computation that supports integer division), and we then test whether  lies within the trimmed column. In contrast to the method based on Dudley's construction, this method provides a better query time of  but with higher space of .

It is possible to combine these two solutions into a simple unified approach that achieves a trade-off between space and query time. Given a parameter , where , subdivide  into a grid of hypercube cells each of diameter . For each cell  that intersects 's boundary, apply Dudley's approximation to this portion of the polytope. By a straightforward packing argument, the number of grid cells that intersect 's boundary is  (see, for example, Lemma~{3} of \cite{ARS}). We apply standardization to  (thus mapping  to  and scaling  to ) and apply Dudley's construction. By Dudley's results, the number of halfspaces needed to approximate  within  is . To answer a query, in  time we determine which hypercube of the grid contains the query point (assuming a model of computation that supports integer division). We then apply brute-force search to determine whether the query point lies within all the associated halfspaces in time . The query time is dominated by this latter term. The space is dominated by the total number of halfspaces, which is . If we express  in terms of a parameter , where , then Theorem~\ref{thm:simple-trade-off} follows as an immediate consequence. Note that the resulting trade-off interpolates nicely between the two extremes for .

\section{The Data Structure and Construction} \label{sec:split-reduce}


In this section we show how to improve the approach from the previous section by replacing the grid with a quadtree. The data structure is constructed by the recursive procedure, called {\alg}, whose inputs consist of a convex body  and a quadtree cell . We are also given the approximation parameter , and a parameter  that controls the query time. Although we assume that  is presented as the intersection of  halfspaces, this procedure can be applied to any representation that supports the following access primitives:
\begin{enumerate}
\item[(i)] Determine whether  is disjoint from .

\item[(ii)] Determine whether  is contained within .

\item[(iii)] Determine whether there exists a set of at most  halfspaces whose intersection -approximates  within , and if so generate such a set.
\end{enumerate}


Recall that we assume that  has been transformed so it is -fat and lies within  (the hypercube of unit diameter centered at the origin). The data structure is built by the call . In general,  checks whether any of the above access primitives returns a positive result, and if so it terminates the decomposition and  is declared a \emph{leaf cell}. Otherwise, it makes a recursive call on the children of  (see Figure~\ref{fig:split-reduce}(a)). On termination, each leaf cell is labeled as either ``inside'' or ``outside'' or is associated with a set of at most  approximating halfspaces (see Figure~\ref{fig:split-reduce}(b)).

\begin{figure}[htbp]
  \centerline{\includegraphics[scale=0.40]{Figs/split-reduce}}
  \caption{(a) Cases arising in {\alg} for  and (b) the final subdivision.}
  \label{fig:split-reduce}
\end{figure}


\newcommand{\stepout}{1}  \newcommand{\stepin}{2}
\newcommand{\stepapx}{3}
\newcommand{\steprecur}{4}
\noindent{\alg}:
  \begin{enumerate}
\setlength{\itemsep}{-0.5ex}\setlength{\parsep}{0pt}\item[(\stepout)] If ,  label  as ``outside.''
	
	\item[(\stepin)] If , label  as ``inside.''
  
	\item[(\stepapx)] If there exists a set at most  halfspaces whose intersection provides an -approximation to  within , associate  with such a set  of minimum size.
	
  \item[(\steprecur)] Otherwise, split  into  quadtree cells and recursively invoke {\alg} on each. 
\end{enumerate}


For the sake of our space-time trade-offs, we will usually assume that  is reasonably large, say, . Under our assumption that , steps (\stepout) and (\stepin) are not needed, since it is possible to -approximate any cell satisfying these conditions with a single halfspace. This assumption on the value of  is mainly a convenience to simplify the formulas of our mathematical analysis. Observe that even if , the procedure will terminate and provide a correct answer once the cell diameter falls below .

It is easy to see that the recursion must terminate as soon as  (since, irrespective of whether it intersects , any such cell can be labeled either as ``inside'' or ``outside''). Of course, it may terminate much sooner. Since  is of unit diameter, it follows that the height of the quadtree is . The total space used by the data structure is the sum of the space needed to store the quadtree and the space needed to store the approximating halfspaces for the cells that intersect 's boundary. Our next lemma shows that if the query time is sufficiently large, the latter quantity dominates the space asymptotically. For each leaf cell  generated by Step~(\stepapx), define , and define  for all the other leaf cells.

\begin{lemma} \label{lem:total-space}
Given a convex body  in . If  is , then the total space of the data structure produced by  on  for query time  is asymptotically dominated by the sum of  over all the leaf cells  that intersect 's boundary.
\end{lemma}


\begin{proof}
Let  denote the quadtree produced by running {\alg} on . As mentioned above,  is of height . By our hypothesis that  is , there exists a constant  such that . Let  denote the set of leaves of  that intersect the boundary of , and let  denote the internal nodes of  that have the property that all their children are leaves. (These are the lowest internal nodes of the tree.) Let  denote the sum of  over all .

The fact that each node  was subdivided by {\alg} implies that more than  halfspaces are needed to approximate  within 's cell. Therefore, the children of  that intersect 's boundary together require at least  halfspaces. In addition to , each quadtree leaf  can (implicitly) contribute its  bounding hyperplanes to the approximation. Therefore,  hyperplanes suffice to approximate  in all the cells of , implying that . Since , we have , and thus . 

Each internal node of  is either in  or is an ancestor of a node in . Thus, the total number of internal nodes of  is at most . Since each internal node of a quadtree has  children, the total number of nodes in the tree, excluding the root, is at most

Each internal node of  and each leaf node that does not intersect 's boundary contributes only a constant amount to the total space. Therefore, the space contribution of the nodes other than those of  is at most a constant factor larger than the total number of nodes of , which we have shown is . Therefore, the total space is , as desired.
\end{proof}


A query is answered by performing a point location in the quadtree to determine the leaf cell containing the query point. If the leaf cell is not labeled as being ``inside'' or ``outside'', we test whether the query point lies within all the associated halfspaces, and if so, we declare the point to be inside . Otherwise it is declared to be outside. Clearly, the query time is .

The algorithm is correct provided that the set of halfspaces  computed in Step~(\stepapx) defines any -approximation of  within , but our analysis of the data structure's space requirements below (see the proof of Lemma~\ref{lem:restricted-dudley}) relies on the assumption that the size of  is within a constant factor of the minimum number of halfspaces of any -approximating polytope within . Unfortunately, we know of no constant-factor approximation to the problem of computing such a polytope. Thus, strictly speaking, the bounds stated in Theorem~\ref{thm:membership-ub} are purely existential. In Section~\ref{sec:preproc} we will show that through a straightforward modification of the greedy set-cover heuristic, it is possible to compute an approximation in which the number of defining halfspaces exceeds the optimum (for slightly smaller approximation parameter) by a factor of at most . From the following result it follows that this increases our space and query time bounds by .

\begin{lemma} \label{lem:weak-membership}
Given any  and any constant , if the number of halfspaces of  computed in Step~(\stepapx) of {\alg} is within a factor  of the minimum number of facets of any -approximating polytope within , then Theorems~\ref{thm:membership-ub} and~\ref{thm:ann-ub} hold but with the asymptotic space and query time bounds larger by a factor of .
\end{lemma}


\begin{proof}
Let us refer to the hypothesized version of {\alg} whose Step~(\stepapx) is suboptimal as . Consider an execution of  using  as the desired query time and  as the approximation parameter, and let us compare this to an execution of {\alg} using  and , respectively. Since  is a constant, the asymptotic dependencies on  are unaffected, and therefore the space and query times stated in Theorems~\ref{thm:membership-ub} and~\ref{thm:ann-ub} apply without modification to the execution of {\alg}. In this execution, if the subdivision declares some quadtree cell  to be a leaf, then  halfspaces suffice to -approximate  within , and so by our hypothesis in the corresponding execution of , Step~(\stepapx) returns at most  halfspaces, implying this execution also declares  to be a leaf. Therefore, the tree generated by  is a subtree of the tree generated by {\alg}, but each leaf node may contain up to a factor of  more halfspaces. Thus, the asymptotic space and query time bounds for  are larger than those of  by this same factor.
\end{proof}


\section{Simple Upper Bound} \label{sec:firstbound}


In this section, we present a simple upper bound of  on the storage of the data structure obtained by the {\alg} algorithm for any given query time . The tools developed in this section will be useful for the more comprehensive upper bounds, which will be presented in subsequent sections.

Throughout this section we do not necessarily assume that  has been scaled to lie within  and may generally be much larger. Recall that  denotes a hypersphere of radius  centered at the origin. Let  denote a surface patch of  that lies within . Let  denote the set of points exterior to  whose closest point on  lies within . We refer to the surface patch  (the points of  whose closest point on  lies within ) as the \emph{Voronoi patch} of . Voronoi patches are related to Dudley's construction. In particular, a sample point  from Dudley's construction generates a supporting halfspace at a point of  if and only if . The following two lemmas are straightforward adaptations of Dudley's analysis~\cite{Dudley}. The first is just a restatement of Dudley's result.

\begin{lemma} \label{lem:Dudley1}
Given a convex body  in  that lies within  and , there exists an -approximating polytope  bounded by at most  facets, where  is a constant depending only on .
\end{lemma}


The second lemma is a technical result that is implicit in Dudley's analysis. Given two points , let  denote the segment between them, and let  denote the Euclidean length of this segment.

\begin{lemma} \label{lem:Dudley}
Let  be a convex body, let , and let  and  be two points of  such that . Let  and  be the points on  that are closest to  and , respectively. If  is within unit distance of the origin, then
\begin{enumerate}
\item[]  and

\item[] the supporting hyperplane at  orthogonal to the segment  intersects segment  at distance less than  from  (see Figure~\ref{fig:dudley-lemma}).
\end{enumerate}
\end{lemma}


\begin{figure}[htbp]
  \centerline{\includegraphics[scale=0.40]{Figs/dudley-lemma}}
  \caption{Lemma~\ref{lem:Dudley}.}
  \label{fig:dudley-lemma}
\end{figure}


The following lemma is an extension of Dudley's results, which allows us to bound the complexity of an -approximation of  within a quadtree cell . Recall from Section~\ref{sec:prelim-quadtree} that  denotes portion of  that lies within distance  of . 

\begin{lemma} \label{lem:restricted-dudley}
Let  be a convex body,  be a quadtree cell that intersects , and . Let  denote a set of -dense points on the Dudley sphere . Then  (see Figure~\ref{fig:restricted-dudley}(a)).
\end{lemma}


\begin{proof}
We construct an approximating polytope  by the following local variant of Dudley's construction. For each point , let  be its nearest point on the boundary of . (Note that .) For each point , take the supporting halfspace to  passing through  that is orthogonal to the segment . Let  be the (possibly unbounded) intersection of these halfspaces.

\begin{figure}[htbp]
  \centerline{\includegraphics[scale=0.40]{Figs/restricted-dudley}}
  \caption{Lemma~\ref{lem:restricted-dudley}. (Not drawn to scale.)}
  \label{fig:restricted-dudley}
\end{figure}


First, we show that  is nonempty. Consider any point  on . Let  denote any point of  whose closest point on  is . By definition of , there is a point  whose distance from  is at most . Letting  denote 's closest point on , by Lemma~\ref{lem:Dudley}(i), . Thus,  lies within , which implies that . It follows that  is bounded by at least one halfspace.


We now show that  is an (outer) -approximation of  within . Since  is defined by supporting hyperplanes,  is contained within . Consider any  that is at distance greater than  from . It suffices to show that , that is, there exists a bounding hyperplane for  that separates  from . Let  denote the point of  that is closest to  (see Figure~\ref{fig:restricted-dudley}(b)). Note that  is constrained to lie within , and hence this may not be the closest point to  on . By continuity, there must be a point on the segment  that is at distance exactly  from , which we denote by . Since  is convex, this segment is contained in , and, hence, so is . 

Let  be the point on  that is closest to . (Note that  need not lie within .) Because  is centered at the origin, 's distance from the origin is at most . Let  denote the point of intersection with the Dudley hypersphere  of the ray emanating from  and passing through . Let  be a point of  that lies within distance  of , and let  be its closest point on . By Lemma~\ref{lem:Dudley}(i) , and (ii) the supporting hyperplane  at  orthogonal to the segment  intersects segment  at distance less than  from . Thus,  separates  from , and therefore it separates  from .

To complete the proof that , it suffices to show that  is indeed included in our construction of . By the triangle inequality and our assumption that , the distance from  to  is at most

It follows that , and so  is included in the construction of . By our hypothesis that the set  constructed in Step~(\stepapx) of  is the minimum-sized set of halfspaces needed to -approximate  within , we have . (Note that this works even if  is an ``inside'' cell that intersects 's boundary. In such a case  by definition, and as argued above,  is nonempty.) This completes the proof.
\end{proof}


Next, we prove a useful technical lemma, which bounds the total complexity of a set of leaves whose cells are of a given minimum size. Recalling the definition of  from the previous lemma, we may assume that .

\begin{lemma} \label{lem:space-bound}
Let  be a convex body in , let , and let  denote a set of disjoint quadtree cells contained within  such that each intersects  and is of diameter . Then .
\end{lemma}


\begin{proof}
By applying Lemma~\ref{lem:restricted-dudley} to each  we have

Since , to complete the proof it suffices to show that each  lies within  for at most constant number of . To see this, let  be the point on  that is closest to . Since each cell  has size at least , by disjointness and a packing argument it follows that at most a constant number (depending on dimension) of such cells can lie within distance  of , which establishes the claim.
\end{proof}


Combining the above results, we obtain the main result of this section.

\begin{lemma} \label{lem:firstboundunit}
Let  be a convex body in  and . The output of  for  has total space .
\end{lemma}


\begin{proof}
Let  be the constant of Lemma~\ref{lem:Dudley1}, and define . We may assume that , for otherwise  and clearly  will not generate more than a constant number of cells. 

Let  denote the quadtree produced by the algorithm, and let  denote the set of leaf cells of  that intersect the boundary of . Recall from Lemma~\ref{lem:total-space} that the data structure's total space is asymptotically bounded by the sum of  for all . Thus, it suffices to prove that


Towards this end, we first prove a lower bound on the size of any leaf cell . We assert that the cell  associated with any internal node has diameter at least . It will then follow that each leaf cell has diameter at least . Suppose to the contrary that . Recall the standardization transformation from Section~\ref{sec:prelim-quadtree}, which maps  to  and scales  to at least . Let us denote this value by . Since , we have . By applying Lemma~\ref{lem:Dudley1} to the transformed body (with  playing the role of ), it follows that the polytope  can be -approximated by a polytope  defined by the intersection of at most 
 
halfspaces. Since , it is easy to see that  is an -approximation of  within . Since , the termination condition of our algorithm implies that such a cell is not further subdivided, contradicting our hypothesis that this is an internal node. Therefore, the cells of  satisfy the conditions of Lemma~\ref{lem:space-bound}. The desired bound follows by applying this lemma.
\end{proof}


It is useful to contrast this with the Dudley-based approach described in Section~\ref{sec:prelim-simple-approx}. For , we obtain the same  space in each case, but the exponent in the query time of {\alg} is only half that of the Dudley-based approach. Later, in Lemma~\ref{lem:baseunit}, we will present a more refined analysis showing that it is possible to reduce this further, achieving a query time of only .

It will be useful in later sections to generalize the above lemma to quadtree cells of arbitrary size. By a direct application of standardization, we obtain the following.

\begin{lemma} \label{lem:firstbound}
Let  be a convex body in ,  be a quadtree cell contained within , and let . The output of the call  for  has total space .
\end{lemma}


\section{Dual Caps and Approximation} \label{sec:cap}


The bounds proved in the previous section apply to query times . In Section~\ref{sec:membership} we will show how to obtain good space bounds for smaller query times. This will involve analyzing the local geometry about small boundary patches of the convex body. In this section, we introduce the principal geometric underpinnings that will be needed for this more refined analysis. In particular, we discuss the concepts of dual caps and restricted dual caps and their role in polytope approximation.

Although we do not assume that  is smooth, it will simplify the presentation to imagine that each boundary point has a unique supporting hyperplane and a unique normal vector. To achieve this, we employ an \emph{augmented representation} of the boundary points of . In particular, each boundary point  will be expressed as a pair , where  is a supporting hyperplane at . We will often refer to  as . When  is clear from context or unimportant, we avoid explicit reference to it.

We first observe that computing an outer -approximation of a convex body  by halfspaces can be reduced to a hitting-set problem. Consider any point  that is external to  at distance  from its boundary, and let  denote the augmented boundary point consisting of the closest point  to  and the supporting hyperplane through  that is orthogonal to the segment  (see Figure~\ref{fig:dual-cap}(a)). We define the \emph{-dual cap} of , denoted , to be the set of augmented boundary points  such that the supporting hyperplane  through  intersects the closed line segment . (Equivalently, these are the points of  that are visible to .) 

\begin{figure}[htbp]
  \centerline{\includegraphics[scale=0.38]{Figs/dual-cap}}
  \caption{(a) Dual caps, (b) restricted dual caps, and (c) the Voronoi patch of a dual cap.}
  \label{fig:dual-cap}
\end{figure}


Any outer -approximation of  by halfspaces must contain at least one halfspace that separates  from , and this can be achieved by including  for any pair  within . A set of augmented points  is said to be an \emph{-hitting set} for  if for every , . It follows directly that the intersection of the supporting halfspaces for any -hitting set is an outer -approximation of . This observation will be formalized within our quadtree-based context in our next lemma. Before stating the lemma, we need to introduce one additional concept. In order to approximate  within a given quadtree cell , we are interested only in the geometry of 's boundary that lies close to . For this reason, it will be desirable to limit the diameter of dual caps. Given , let  denote the closed Euclidean ball of radius  centered at . Define the \emph{-restricted dual cap}, denoted , to be the intersection of  with  (see Figure~\ref{fig:dual-cap}(b)).

\begin{lemma} \label{lem:patchouter}
Let  be a convex body,  be a quadtree cell that intersects , and . Let  be any set of augmented points on  that hits the set of all -restricted -dual caps whose defining point is in  (see Figure~\ref{fig:patchouter}(a)). Then there is a polytope  defined as the intersection of  halfspaces that -approximates  within .
\end{lemma}


\begin{proof}
Let  be the polytope defined by the intersection of the supporting halfspaces associated with each augmented point of  (see Figure~\ref{fig:patchouter}(b)). Clearly, . Consider any point  that is at distance greater than  from . It suffices to show that , that is, there exists a bounding hyperplane for  that separates  from . 

\begin{figure}[htbp]
  \centerline{\includegraphics[scale=0.40]{Figs/patchouter}}
  \caption{Lemma~\ref{lem:patchouter}.}
  \label{fig:patchouter}
\end{figure}


We apply a similar argument to the one that we used in the proof of Lemma~\ref{lem:restricted-dudley}. Consider any  that is at distance greater than  from  (see Figure~\ref{fig:patchouter}(c)). It suffices to show that there exists a bounding hyperplane for  that separates  from . Let  denote the point of  that is closest to . By continuity, there must be a point on the segment  that is at distance exactly  from , which we denote by . Since  is convex, this segment must be contained in , and, hence, so is . 

Let  be the point on  that is closest to . (In our figure , but generally  need not lie within .) Since , we have . It follows that . Therefore, there exists an augmented point  that hits the -restricted -dual cap defined by  (whose apex is at ). The supporting hyperplane  separates  (and therefore ) from , as desired.
\end{proof}


Our analysis of the space bounds of {\alg} is based on the combined sizes of the -hitting sets for  within each quadtree cell . Dudley's construction can be viewed as one method of computing -hitting sets. Unfortunately, Dudley's construction does not lead to the best bounds because it tends to over-sample in regions of very low or very high curvature. Our analysis will be based on a more refined, area-based approach to bounding the sizes of hitting sets. The key geometric observation is that the product of the areas of any -dual cap and its associated Voronoi patch on the Dudley sphere  must be large. Intuitively, if the surface area of an -dual cap is small, then the total curvature of the patch must be high, and so the associated Voronoi patch must have relatively large area (see Figure~\ref{fig:dual-cap}(c)). More precisely, we show that (under certain conditions) the product of the areas of an -dual cap and its Voronoi patch is . This result is stated formally in Lemma~\ref{lem:dual-basic} below. Given a -dimensional manifold, let  denote its -dimensional Hausdorff measure. Given a convex body  in , we use  as a shorthand for . 

\begin{lemma}[Area-Product Bound] \label{lem:dual-basic}
Let  be a convex body in , let . Consider a pair , where  and  is a supporting hyperplane passing through . Let  denote the -restricted -dual cap whose defining point is . If  is fat and of diameter at least , there exists a constant  (depending only on ) such that if  lies within a unit ball centered at the origin, then .
\end{lemma}


The proof of the lemma is quite technical and will be deferred to Section~\ref{sec:proof}. The geometric basis of the proof involves the Mahler volume, which was introduced in Section~\ref{sec:prelim-polar}. The bound stated in the lemma holds if  is -fat for any  in the interval  under the assumption that  does not depend on . In particular, the proof will reveal that .

We will exploit this observation to demonstrate the existence of smaller -hitting sets than those given by Dudley's construction. We will hit the restricted -dual caps that have large surface area by sampling points randomly on the boundary of , and we will hit those with small surface area by sampling points randomly on the Dudley hypersphere and then selecting their nearest neighbors on . In order to prove that such a random sampling strategy works to stab all the dual caps, we need to establish bounds on the VC-dimension of an appropriate range space based on dual caps. This is not surprising given that dual caps and restricted dual caps are defined by a constant number of parameters. The result is stated in the following lemma. The proof involves a straightforward application of basic geometric principles and appears in the appendix.

\begin{restatable}{lemma}{VCLemmaStmt}\label{lem:VC}
Let  be a convex body in  that lies within , and let  and  be positive real parameters. The following range spaces  have constant VC-dimension (where the constant depends only on ):
\begin{enumerate}
	\setlength{\itemsep}{-0.5ex}\setlength{\parsep}{0pt}\item[]  and  is the set of -dual caps.
	\item[]  and  is the set of Voronoi patches of the -dual caps.
	\item[]  and  is the set of -restricted -dual caps.
	\item[]  and  is the set of Voronoi patches of the -restricted -dual caps.
\end{enumerate}
\end{restatable}


In the next section we will exploit this result to establish the existence of small -nets for these range spaces. Note that the range spaces defined in this lemma are defined over , a domain of infinite cardinality. However, for our purposes, it suffices to consider dual caps and restricted dual caps whose defining points are drawn from any sufficiently dense set of points on  (depending on ), and therefore the domains of the range spaces can be treated as finite sets.

\section{Final Upper Bound} \label{sec:membership}


In this section, we use the tools developed in Sections~\ref{sec:firstbound} and~\ref{sec:cap} to obtain better upper bounds for approximate polytope membership. In particular, we present a proof of Theorem~\ref{thm:membership-ub}. We will first show how to apply the area-based techniques described in the previous section to improve the simple upper bound from Lemma~\ref{lem:firstboundunit} at the low-space end of the trade-off spectrum. (This will be presented in Lemma~\ref{lem:baseunit}.) We will then apply this improvement repeatedly in an inductive manner to establish trade-offs throughout the spectrum. For technical reasons, many of the lemmas of this section assume constant upper bounds on the value of . There is no loss of generality in doing so, since it is easy to show that if  is bounded below by any fixed constant, the asymptotic space and query times of {\alg} are both .

Throughout this section, recall that  is the portion of  that is within distance  of , and . Also, define . We will assume that , for otherwise it is trivial to compute an -approximation of constant size. Our first result establishes an area-based bound on the number of halfspaces needed to approximate  within a quadtree cell .

\begin{lemma} \label{lem:Guilherme}
Let  be a fat convex body in , let , and let  be a quadtree cell that intersects . Letting  denote the constant of Lemma~\ref{lem:dual-basic}, define

There is a polytope  defined as the intersection of  halfspaces that is an -approximation of  within .
\end{lemma}


\begin{proof}
Letting  and , we can express the value of  more succinctly as . First, we assert that . To see this, we consider two cases. First, if  lies entirely within distance  of , then , which implies that . Since  is fat and by our assumption that , it follows that . Therefore, . On the other hand, if some part of  lies at distance greater than  from ,  is a boundary patch of  of diameter . Since both  and  are fat, it follows that . By convexity, as we go from a boundary patch on  to its Voronoi cell on , distances cannot decrease. Therefore , and again we have . Through a minor adjustment to constant factor  in 's definition, we may assume that .

By Lemma~\ref{lem:patchouter}, in order to show the existence of an -approximating polytope  for  within , it suffices to show that it is possible to hit all -restricted -dual caps whose defining point lies in  (not to be confused with ) using  points. To do this, we distinguish between two types of such restricted dual caps. A restricted dual cap  is of \emph{type~1}, if , and otherwise it is of \emph{type~2}.

By assertions~(3) and~(4) of Lemma~\ref{lem:VC}, we know that -restricted -dual caps and their Voronoi patches both have constant VC-dimension. The VC-dimension is no larger if we restrict the domain of the range space. Therefore, by standard machinery (see, e.g., \cite{probabilistic}) we can build a -net for any restriction of these range spaces of size  each by random sampling. 

For type-1 dual caps, consider the restriction  of the range space given in Lemma~\ref{lem:VC}(3). Let  denote a -net. Consider any type-1 dual cap . Since 's defining point lies within  and it is -restricted, it lies entirely within . Thus, we have

Therefore  contains at least one point of . It follows that  hits all type-1 dual caps.

For type-2 dual caps, let us consider the restriction  of the range space of Lemma~\ref{lem:VC}(4). Let  denote a -net. Because  and ,  lies within a ball centered at the origin of radius . Given any type-2 dual cap  whose defining (augmented) point lies in , we may apply Lemma~\ref{lem:dual-basic} to obtain

As before, since 's defining point lies within , . From this we have

Therefore  contains at least one point of , implying that  hits the Voronoi patches of all type-2 dual caps. For each point of , we select its nearest neighbor on , obtaining a set . It follows directly that the set  hits all type-2 dual caps. Therefore, the union  forms the desired set of size  that hits all -restricted -dual caps whose defining point lies within .
\end{proof}


In order to establish our storage bounds, we analyze the behavior of the algorithm at a particular level of the decomposition. Given the query-time parameter , recall that we stop the subdivision process in  if the number of hyperplanes needed to approximate  within  falls below . Also recall that  denotes the number of approximating halfspaces associated with . Let us consider the state of the subdivision process when the cell sizes reach roughly . Cells that have stopped subdividing by this point are ``good,'' since we can bound the total space requirements for all such cells by appealing to Lemma~\ref{lem:space-bound}. For the remaining ``bad'' cells, we will bound their space requirements on a cell-by-cell basis by using the simple upper bound from Lemma~\ref{lem:firstbound}. For our approach to work well, it is crucial to obtain a good bound on the number of such bad cells. We exploit the area bound of Lemma~\ref{lem:Guilherme} for this purpose. Whenever {\alg} subdivides a cell of size , we can infer that more than  hyperplanes are required to approximate  within this cell. Since the portion of  lying within this cell is small, the area of its Voronoi patch on the Dudley sphere must be large. A packing argument applied on the Dudley sphere will be used to limit the number of these bad cells.

In order to formalize the notion of good and bad cells, let  denote the quadtree produced by , and let  denote the subtree of  induced by cells of diameter at least . For the remainder of this section, let  denote the (good) leaf cells of  that are not subdivided further by the algorithm, and let  be the remaining (bad) leaf cells of . The cells of  and  are all of diameter . Each cell in  can be approximated using at most  halfspaces, and those in  require more. In our next lemma, we bound the total number of approximating halfspaces over all the good leaf cells and the total number of bad leaf cells. 

\begin{lemma} \label{lem:aux2}
Let  be a fat convex body in , let . Let  denote the quadtree produced by , for , and let  and  be as defined above. Then 
\begin{enumerate}
\item[] ,

\item[] .
\end{enumerate}
\end{lemma}


\begin{proof}
Because the cells of  are disjoint and each is of diameter , assertion~(i) follows as a direct consequence of Lemma~\ref{lem:space-bound}. Thus, it remains to prove assertion~(ii). Let  be any cell of . Since any child of a cell of  is of diameter smaller than  and 's diameter is twice this, we have . Recall that . Also, let  and  denote the values of  and , respectively, from the proof of Lemma~\ref{lem:Guilherme}, when applied to . 

Because  and  involves a boundary patch of  that intersects  and includes an additional expansion by distance , it follows that this boundary patch has diameter . Therefore, . By applying Lemma~\ref{lem:Guilherme} (and recalling the constant  from Lemma~\ref{lem:dual-basic}), we have , where

In Lemma~\ref{lem:Guilherme} we showed that (after a suitable adjustment to ), we have . Since  is subdivided further, we know that , which implies that . Because , by simple manipulations we have . By combining this with the upper bound on  from above, we obtain , which yields the lower bound


As shown in the proof of Lemma~\ref{lem:space-bound}, given any set of disjoint quadtree cells of diameter  a point of  can be in  for at most a constant number of these cells. Since the quadtree cells of  satisfy these conditions,

Combining this with our lower bound, we have

Since  is a hypersphere of constant radius, its area is bounded, and assertion~(ii) follows immediately.
\end{proof}


Recall that we showed in Lemma~\ref{lem:firstboundunit} that it is possible to answer approximate membership queries in  time using space . By using the above lemma, we show next that we can improve this to achieving query time roughly  for the same space.

\begin{lemma} \label{lem:baseunit}
Let  be a fat convex body in , and let . For , the output of  has total space .
\end{lemma}


\begin{proof}
Let  denote the quadtree produced by the algorithm. By Lemma~\ref{lem:total-space}, the data structure's total space is dominated by the space needed to store the hyperplanes in the leaf cells. Thus, it suffices to show that the sum of  over all leaf cells  of  is . Let , , and  be as defined just prior to Lemma~\ref{lem:aux2}. By Lemma~\ref{lem:aux2}(i), the total contribution of  for all cells in  is . So, it suffices to bound the contribution due to .

Let  be any cell of . Recall from the proof of Lemma~\ref{lem:aux2} that . Since , it follows that . Because , we have . By Lemma~\ref{lem:firstbound}, the output of  has total space at most 

By Lemma~\ref{lem:aux2}(ii), . Since , we have . Summing up the space contributions of all , the total space for these cells is 

as desired. 
\end{proof}


In order to extend the space-time trade-off to other query times, we will apply the previous result as the basis case in an induction argument. The induction will be controlled by a parameter , which we assume to be a constant. The proof is rather technical, but it involves a straightforward application of the earlier results of this section.

\begin{lemma} \label{lem:trade-off-ub}
Let  be a fat convex body in , and let . Let  be a real-valued constant. For , the output of  has total space

\end{lemma}


\begin{proof}
Define , which implies that , and . Expressed as a function of , the desired space bound can be expressed as

for a constant  (depending on  but not on ).

We begin exactly as in the proof of the previous lemma. Let  denote the quadtree produced by the algorithm, and by Lemma~\ref{lem:total-space}, it suffices to bound the sum of  over all leaf cells of . Given , , and  defined prior to Lemma~\ref{lem:aux2}, the space contribution due to the cells of  is . To see that this satisfies our space bound, observe that since  and , we have

Therefore, the total contribution of  for all cells in  is

which matches the desired bound given in Eq.~(\ref{eq:alt-bound}).

It remains to bound the contribution to the space of the cells of . We do this by induction on . For the basis case , we have . Therefore . By applying Lemma~\ref{lem:baseunit}, the total space of the data structure (which includes the contribution of ) is . It follows from Eq.~(\ref{eq:ind-bound}) (for the case ) that this satisfies our storage bound.

For the induction step, we assume that the lemma holds for  (that is, ), and our objective is to prove it for . It will be convenient to express the induction hypothesis in a form that holds for an arbitrary quadtree cell . By applying standardization to  (thus mapping  to  and scaling  to ), the induction hypothesis states that for

there is a constant  such that the output of  has total space at most


Let  be any cell of . In the proof of Lemma~\ref{lem:aux2} we showed that . By the bound on  from the statement of this lemma, we have

implying that  satisfies Eq.~(\ref{eq:time-hyp}). If  is at most , we may apply the induction hypothesis, yielding the space bound given in Eq.~(\ref{eq:sp-hyp}). Since , this can be simplified to


By combining Lemma~\ref{lem:aux2}(ii) with the lower bound on  given in the statement of this lemma, the number of cells in  satisfies

The total contribution to the space by the cells of  is the product of the space requirements for each cell of , given in Eq.~(\ref{eq:sp-bound}), and the number of such cells, given in Eq.~(\ref{eq:l2-size}). There exists a constant  (depending on  but not on ) such the total space is at most

On the other hand, if  exceeds , then since  it follows that  is , and we can adjust to  to satisfy this bound. In either case, we achieve the bound in Eq.~(\ref{eq:alt-bound}). 
\end{proof}


Observe that the exponent in the space bound in the preceding lemma is a piecewise linear function in , whose breakpoints coincide with powers of two. It is easily verified that the exponent is a continuous function of . (In particular, observe that , where .)

\medskip

We can now present the proof of Theorem~\ref{thm:membership-ub}. Recall that  is a convex polytope in . By Lemma~\ref{lem:fat}, we can precondition  so that it is -fat and is contained within , thus allowing us to approximate  absolutely. Also, if , we set . (Both of these changes result in a constant factor decrease to , which will not affect the asymptotic bounds.) We then set  and invoke . Let  denote the resulting data structure. Given the preconditioning of  and the alteration of , we may apply Lemma~\ref{lem:trade-off-ub} to show that the total space for  is 

Using the fact that , this is

which matches the space bound of Theorem~\ref{thm:membership-ub}.

Recall that a query is answered by locating the leaf node of  that contains the query point, followed by an inspection of the (at most)  halfspaces stored in this leaf node. By our remarks following the presentation of {\alg},  is of height , which implies that the query time is dominated by the value of . This completes the proof of Theorem~\ref{thm:membership-ub}.

\section{Preprocessing} \label{sec:preproc}


Our principal focus so far has been in establishing the existence of trade-offs between space and query time, without considering how to construct the data structure. In this section we discuss preprocessing issues. We first discuss the preconditioning of  as described in Lemma~\ref{lem:fat} and then discuss the implementation of the access primitives (i)--(iii) needed for {\alg} as presented at the start of Section~\ref{sec:split-reduce}. We assume that the input convex body  is presented as the intersection of a set  of  halfspaces in . Throughout, let  denote an arbitrary quadtree cell.

Let  denote the query-time parameter in {\alg}. As observed in Section~\ref{sec:split-reduce}, under our assumption that , Steps~{\stepout} and {\stepin} are not needed, since we can rely entirely on Step~{\stepapx}, and therefore access primitives~(i) and~(ii) are not needed.\footnote{If we wished to we could implement access primitive~(i) in linear time by linear programming. Also, by testing the membership of each of 's vertices in , we could implement a stronger version of access primitive~(ii), namely that of determining whether  (as opposed to ).}
The remainder of this section will be focused on preconditioning (Section~\ref{sec:precondition}) and the implementation of access primitive~(iii), which locally approximates  within  (Section~\ref{sec:apx-cover}).

\subsection{Preconditioning} \label{sec:precondition}


Recall that we assume that  is a (full-dimensional) convex polytope in  that is presented as the intersection of a set of  closed halfspaces. Also recall that  is the axis-aligned hypercube of unit diameter that is centered at the origin. Our objective is to precondition  by computing an affine transformation that both fattens  and maps it to lie within .  has a side length of , and therefore it contains a ball of radius  centered at the origin. Let  denote this ball, and let  denote its radius. For , let  denote the concentric ball of radius . We say that a polytope is in \emph{-canonical position} if it is nested between  and  (see Figure~\ref{fig:canonical}). Clearly, a polytope that is in canonical position is contained within  and is -fat. The following lemma shows that  can be efficiently mapped into this form, and furthermore an absolute approximation to the transformed body can be easily mapped to a relative approximation of . (Lemma~\ref{lem:fat} of Section~\ref{sec:prelim-abs} follows as an immediate consequence of this.) Such fattening operations are commonplace in geometric approximation algorithms (see, e.g., \cite{AHV-coreset,Chan-coreset,HP-book,BHP-bbox}), and we employ the standard approach based on minimum enclosing volumes, the John Ellipsoid in particular. 

\begin{figure}[htbp]
  \centerline{\includegraphics[scale=0.40]{Figs/canonical}}
  \caption{A polytope in -canonical position.}
  \label{fig:canonical}
\end{figure}


\begin{lemma} \label{lem:precondition-1}
Let  be a convex polytope in  defined as the intersection of a set  of  halfspaces, and let . There is an algorithm that, given  and , in  time computes an affine transformation  that maps  into -canonical position, such that if  is an absolute -approximation of , then  is a relative -approximation of .
\end{lemma}


\begin{proof}
Chazelle and Matou\v{s}ek~\cite{ChM96} show that in any fixed dimension, there exists an  time algorithm that, given a convex polytope  presented as the intersection of  halfspaces, computes an ellipsoid  of maximum volume contained within , also known as the \emph{John Ellipsoid}~\cite{Bal97}. (At the expense of an increase in the constant factors, we can apply the simpler construction by Har-Peled and Barequet~\cite{BHP-bbox}.) Since  is fixed, in constant time we can compute an affine transformation  that maps  to the ball  (that is, ). (Because  is full dimensional, both  and its inverse  are well-defined.) It is well known from John's Theorem (see, e.g., \cite{Bal97}) that  is contained within a uniform scaling of  by a factor of , which we denote by . Therefore, we have , which implies that .

Let . Because  maps an ellipse of diameter  to a ball of diameter , it follows that  maps any vector  to a vector of length at least . (The principal axis aligned with 's diameter is scaled by exactly , and all other principal axes, which form a basis for the space, are scaled by at least this much.) Therefore, . 

If  is any absolute -approximation to , then by definition 

To show that  is a relative -approximation to , observe first that by applying  to the first inclusion of Eq.~(\ref{eqn:inclusion}), we have . Also, by the second inclusion of Eq.~(\ref{eqn:inclusion}) we know that for any , there exists  such that the vector  is of length at most . Therefore, 

We conclude that

and therefore  is a relative -approximation of .
\end{proof}


In order to make subsequent processing more efficient, we adapt a standard coreset construction to reduce the number of halfspaces to a function depending only on  and . The process will involve some further scaling, which will slightly modify the parameters.

\begin{lemma} \label{lem:precondition-2}
Let  be a convex polytope in  defined as the intersection of a set  of  halfspaces, and let . There is an algorithm that, given  and , in  time computes an affine transformation  and a subset  of size  such that: 
\begin{enumerate}
\setlength{\itemsep}{-0.5ex}\setlength{\parsep}{0pt}\item[] applying  to the intersection of  results in a convex polytope  that is in -canonical position;

\item[] furthermore, if  is an absolute -approximation of , then  is a relative -approximation of .
\end{enumerate}
\end{lemma}


\begin{proof}
Given , we begin by computing the transformation  of Lemma~\ref{lem:precondition-1} in  time. Let  denote the resulting polytope, which is in -canonical position (see Figure~\ref{fig:precondition-2}(a)).

Given a set  of points , the \emph{extent measure} associates each unit vector  with the minimum distance between two hyperplanes orthogonal to  that contain  between them (see Figure~\ref{fig:coreset}(a)). More formally, define  (recalling that  denotes inner product). A subset  is said to be an \emph{-coreset} for the extent measure if for all unit vectors ,  (see Figure~\ref{fig:coreset}(b)). Agarwal {\etal}~\cite{AHV-coreset} showed that, given a set of  points in , it is possible to construct an -coreset for the extent measure of size . We will employ an improvement of this result due to Chan, who presented an algorithm to compute such a coreset in  time \cite{Chan-coreset}.

\begin{figure}[htbp]
  \centerline{\includegraphics[scale=0.40]{Figs/coreset}}
  \caption{(a) The extent measure  and (b) a coreset.}
  \label{fig:coreset}
\end{figure}


Since  is in -canonical position, we have . Let  denote the set of  points in  that result by applying the polar transformation (see Section~\ref{sec:prelim-polar}) to each hyperplane of . It follows from the definition of the polar transformation that  (see Figure~\ref{fig:precondition-2}(b)). As mentioned in Section~\ref{sec:prelim-polar}, the polar transformation maps an origin-centered ball of radius  to a ball of radius . Thus,  is nested between an inner ball of radius  and an outer ball of radius . Given  and , we can easily compute the set  in  time. Let . We then apply Chan's algorithm to compute an -coreset  in time  (see Figure~\ref{fig:precondition-2}(c)). Let  be the subset of  that results by taking the polar duals of the points of , and let  be the convex body that results from intersecting these halfspaces (see Figure~\ref{fig:precondition-2}(d)).

\begin{figure}[htbp]
  \centerline{\includegraphics[scale=0.40]{Figs/precondition-2}}
  \caption{Proof of Lemma~\ref{lem:precondition-2}. (Not drawn to scale.)}
  \label{fig:precondition-2}
\end{figure}


Clearly,  and . We assert that the Hausdorff distance between  and  is at most . To prove this, we apply an observation due to Chan \cite{Chan-coreset}. Define the \emph{one-sided extent measure}, denoted  to be . (This is the distance from the origin to 's closest supporting hyperplane orthogonal to and on the same side as .) In Observation~{1.4} of \cite{Chan-coreset} Chan shows that if  is an -coreset for the extent measure, then . Given the nesting properties of  and the fact that  is a unit vector we have 

Therefore, , and with Chan's observation this yields . Under our assumption that , we have , and so .

Let  be the point that determines  (see Figure~\ref{fig:precondition-2}(c)). Treating  as a vector, its polar dual is a hyperplane, which we denote by  (see Figure~\ref{fig:precondition-2}(d)). It follows directly from the definition of polarity that if we shoot a bullet from the origin parallel to  until it hits , the length of the resulting segment is . (To see this, observe that , and so  lies on , for .) Analogously, letting  denote the point that determines , the length of the segment parallel to  that hits the associated polar dual hyperplane  is of length . Applying this to every unit vector in , the Hausdorff distance between  and  is at most the supremum over all unit vectors of 

which establishes our assertion.

Therefore, if  is any absolute -approximation to , then by the triangle inequality (applied to the Hausdorff distance)  is an absolute  approximation to . By Lemma~\ref{lem:precondition-1},  is a relative -approximation of .

We are almost done, but the canonical-position condition fails, because  need not lie within  of radius  (even though  does). Since the Hausdorff distance between  and  is at most , it follows that  lies within . The simple fix is to apply a uniform scaling of space by a factor of . In particular, define  to be the composition of  with such a scaling transformation, and apply the above coreset construction to  transformed by , but now with  scaled accordingly to . The resulting transformed and reduced polytope is nested between  and , and so it is in -canonical position. Also, if  is an absolute -approximation of the transformed and reduced polytope, then  is a relative -approximation of , as desired.
\end{proof}


\subsection{Efficient Local Approximations} \label{sec:apx-cover}


Next, we consider the implementation of access primitive~(iii), which given a convex body  in -canonical position, a quadtree cell , and query-time , determines whether there exist  halfspaces whose intersection -approximates  within . The space and query times stated in Theorem~\ref{thm:membership-ub} are based on the assumption that the number of bounding halfspaces of this local approximating polytope is within a constant factor of optimal. However, we know of no efficient algorithm that can achieve this. In this section we show how to efficiently implement Step~(\stepapx) of {\alg} approximately in the sense that the number of halfspaces in the approximation exceeds the optimum (for a slightly smaller approximation parameter) by a factor of . As shown in Lemma~\ref{lem:weak-membership}, this will lead to an increase in the space and query times stated in Theorem~\ref{thm:membership-ub} by a factor of only .

A natural approach would be to adapt Clarkson's algorithm for polytope approximation \cite{Clarkson-polytope}. There are a few messy technical issues involved with such an adaptation. (For example, Clarkson's algorithm applies to the convex hull of a set of points, rather than the intersection of halfspaces.) Since we do not require the strong approximation bounds provided by Clarkson's algorithm, we will instead present a simple direct solution based on a reduction to the set-cover problem. Our approach is to construct a set system where the point set consists of a dense set of points of spacing  that covers the portion of  that is external to , for a suitable constant . We associate each bounding halfspace of  with the set of grid points that lie \emph{outside} of this halfspace. We will show that the halfspaces associated with a minimum set cover for this system produces the desired local approximation. We use the greedy set cover heuristic to construct this cover.

Recall that  denotes the set of points that lie within Euclidean distance  of . In order to avoid the complexities of determining whether a point lies outside of , it will suffice for our purposes to perform the simpler test of whether a point lie outside a scaled copy of .

\begin{lemma} \label{lem:scaled-growth}
For  and , let  be a polytope in  that is in -canonical position, and let . Then 

\end{lemma}


\begin{proof}
By definition of -canonical position, the distance of every point of  to the origin lies between  and  (see Figure~\ref{fig:scaled-growth}(a)). To prove the first inclusion (), notice that the distance between any supporting hyperplane of  and its parallel supporting hyperplane of  is at least

The Minkowski sum  is equivalent to translating every supporting hyperplane of  away from the origin by distance  and intersecting the associated (infinite) set of halfspaces. If we apply this to just the (finitely many) defining hyperplanes of , we obtain a polytope that contains , and therefore  (see Figure~\ref{fig:scaled-growth}(b) and~(c)). 

\begin{figure}[htbp]
  \centerline{\includegraphics[scale=0.40]{Figs/scaled-growth}}
  \caption{Proof of Lemma~\ref{lem:scaled-growth}.}
  \label{fig:scaled-growth}
\end{figure}


To establish the second inclusion (), observe that each point  is in 1--1 correspondence with the point . Since  is within distance  of the origin,  is within distance  of . This implies that , as desired.
\end{proof}


While access primitive~(iii) does not place any restrictions on the halfspaces used when computing an -approximation to  within , when the query point  lies outside of , it may be useful to add further restrictions. In particular, when the query point lies outside of , it is desirable to obtain a \emph{witness} to nonmembership in the form of a bounding halfspace of  that does not contain . (This will be exploited in Section~\ref{sec:ann} in the reduction of approximate nearest neighbor searching to approximate polytope membership. The witness hyperplane is used to identify the approximate nearest neighbor.) To achieve this, we would like to use bounding halfspaces from the original polytope in our approximation. By a simple application of Carath\'{e}odory's Theorem, we can show that we sacrifice only a constant factor by adding this restriction. The following is a straightforward generalization of Lemma~{3.1} from Mitchell and Suri \cite{MS}. 

\begin{lemma} \label{lem:discrete-apx}
Let  be a convex polytope in  defined as the intersection of a set  of halfspaces, and let  be a quadtree cell. If there exists an -approximation of  within  bounded by  halfspaces, then there exists a subset of  of size at most  that -approximates  within . 
\end{lemma}


\begin{proof}
The cases  and  are both trivial, so let us assume that the boundary of  intersects . Let  be an -approximation of  within  that is bounded by  halfspaces, and let  be any one of these halfspaces. By the definition of  such an approximation, . We may assume that  is a supporting halfspace of , for otherwise we can translate it until it is. Let  be any vertex of  on 's boundary. Let  denote the union of  and 's bounding halfspaces. By Carath\'{e}odory's Theorem there exist  halfspaces from  such that the complement of  is contained within the union of the complement of these  halfspaces. If we replace  with these  halfspaces, the resulting polytope still -approximates  within . 

After repeating this for each of the  halfspaces bounding , we obtain an -approximating polytope by a subset of at most  halfspaces of . Let  be the result of removing from this polytope all the halfspaces that are not in  (and hence must bound ). We have

and therefore  is the desired -approximation.
\end{proof}


We are now in a position to present our set-cover-based local approximation. This is a bi-criteria approximation since it is suboptimal with respect to both the number of bounding halfspaces and the approximation parameter.

\begin{lemma} \label{lem:apx-cover}
For  and , let  be a polytope in  in -canonical position that is given as the intersection of a set  of  halfspaces. Let  be a quadtree cell. In  time, it is possible to compute a subset  such that  
\begin{enumerate}
\setlength{\itemsep}{-0.5ex}\setlength{\parsep}{0pt}\item[] The intersection of the halfspaces of  is an -approximation of  within 

\item[] If  denotes the minimum number of halfspaces needed to -approximate  within , then  is .
\end{enumerate}
\end{lemma}


\begin{proof}
First, we may assume without loss of generality that . Otherwise, setting  will certainly satisfy (i) and will only affect the constant factors in the asymptotic bounds of claim (ii) and the construction time. Define . By the above assumption, we have

Let  and let . By applying Lemma~\ref{lem:scaled-growth} but with  taking on the values  and , respectively, we have

(see Figure~\ref{fig:apx-cover}(a)). Let  and let  denote the vertices of a hypercube grid of diameter . Let  be the set of grid points that lie within  but outside of , that is, . Since , the resulting set is of size , and hence it can be computed in time , by testing each grid point against each halfspace of . Because , we have .

\begin{figure}[htbp]
  \centerline{\includegraphics[scale=0.40]{Figs/apx-cover}}
  \caption{Proof of Lemma~\ref{lem:apx-cover}.}
  \label{fig:apx-cover}
\end{figure}


Next, we define a set system to model the approximation process. For each  we define a subset  as follows. First, let  denote the corresponding bounding halfspace of the scaled body  (see Figure~\ref{fig:apx-cover}(b)). Define  to be the subset of points of  that lie outside of . Consider a set system consisting of the points of  and the sets  for all . Since every point of  lies outside of , and hence outside of , together these sets cover . The resulting collection of sets has total cardinality . 

Consider any set cover  of the resulting set system. Let  denote the polyhedron that results by intersecting the halfspaces  whose associated set  is included in this cover. (Note that the sets  are based on the halfspaces bounding the scaled body  while  is based on the halfspaces bounding the original body .) We assert that  -approximates  within . It suffices to show that for any point ,  is not in . First, observe that for such a point , all the vertices of the grid cell in which it lies are within distance  of . Therefore, by the triangle inequality, each such vertex is at distance at least  from . Since by Eq.~\eqref{eq:apx-cover}, , these vertices are all exterior to , which implies that they are all members of . Let  be any of these vertices. Since  is a cover, there exists a halfspace  such that  is in the cover and contains this point. This implies that  lies outside the associated halfspace  (see Figure~\ref{fig:apx-cover}(c)). Because  is in -canonical position, the minimum distance between 's bounding hyperplane and the origin is at least . Therefore the distance between any point in  to any point exterior to  is at least

It follows by the triangle inequality that  is exterior to , and therefore it lies outside of , as desired.

Let  denote a set cover that results by running the greedy heuristic \cite{CLRS} on the aforementioned set system. By standard results on the greedy heuristic, the size of the resulting cover exceeds that of an optimal cover by a factor of at most .  can be computed in time that is proportional to the total cardinality of the sets of the set system, which is . Let  denote the associated set of halfspaces, and let  denote the intersection of these halfspaces. By the above remarks,  is an -approximation to  within , which establishes claim~(i).

To establish~(ii), consider a -approximation of  within  that is bounded by the minimum number  of halfspaces. By Lemma~\ref{lem:discrete-apx} there exists such an approximation that uses only the bounding halfspaces of , such that the number of halfspaces is larger by a factor of at most . Let  denote this approximation, and let  denote its bounding halfspaces. By Eq.~\eqref{eq:apx-cover}, we have . Let . Clearly, . Therefore, every point of  lies outside of . It follows that the sets  associated with the halfspaces  that bound  form a set cover of  within our system. Letting  denote this cover, we have , as desired.
\end{proof}


We can now present the main result of this section, which summarizes the preprocessing time.

\begin{lemma} \label{lem:preproc-time}
Given a full-dimensional convex polytope  in  defined as the intersection of a set of  halfspaces, approximation parameter , and query time parameter , there is an algorithm that runs in time  for some constant  (which does not depend on ) that constructs a data structure satisfying Theorem~\ref{thm:membership-ub} but with an additional factor of  in both the space and query times.
\end{lemma}


\begin{proof}
Given 's bounding halfspaces, we apply Lemma~\ref{lem:precondition-2}. In  time we obtain a polytope , such that  is in -canonical position for .  is bounded by a subset  of halfspaces of size , and the problem of computing a relative -approximation of  reduces to the problem of computing an absolute -approximation of , where . 

Ideally, we would like to invoke {\alg} on  using  as the approximation parameter and  as the query time parameter. Since we do not know how to determine minimum-sized convex approximations efficiently, we will need to relax our expectations. For any quadtree cell  generated by {\alg}, we apply Lemma~\ref{lem:apx-cover} on the set  of halfspaces. By claim~(i) of this lemma, after  time, a subset  can be computed that is an -approximation of  within . Irrespective of the choice of the query time, the maximum number of quadtree cells generated by {\alg} is , and therefore (after preconditioning) the overall running time of {\alg} is . Combined with the  time for preconditioning, the algorithm's overall running time is , where .

Let . By Lemma~\ref{lem:apx-cover}(ii) the number of halfspaces in  is within a factor of  of the size of the minimum-sized -approximation of  within . Since  for a constant , Lemma~\ref{lem:weak-membership} implies that the conclusions of Theorem~\ref{thm:membership-ub} hold but with an additional factor of  in both the space and query times.
\end{proof}

\section{Lower Bound} \label{sec:lb}


In this section, we establish lower bounds on the space-time trade-offs obtained by {\alg} for polytope membership. In particular, we will prove Theorem~\ref{thm:lb} from Section~\ref{sec:intro}. Our approach is similar to the lower bound proof of~\cite{AVD-JACM}. (Note that this is a lower bound on the performance of {\alg}, not on the problem complexity. It applies to the stronger existential version of the algorithm.) It is based on analyzing the performance of the algorithm on a particular convex body, a generalized hypercylinder that is curved in  dimensions and flat in  dimensions. We select the value of  that produces the best lower bound on the storage as a function of , , and . Throughout, we use the term -approximation in the absolute sense, as defined in Section~\ref{sec:prelim-abs}. 

As mentioned earlier, it is well known that  facets are required to -approximate a Euclidean ball of unit radius (see, e.g., \cite{Bro08}), and this holds for any polytope that that is sufficiently close to a ball in terms of Hausdorff distance. The following utility lemma generalizes this observation to different diameters. The proof is straightforward, but for the sake of completeness we include its proof in the appendix.

\begin{restatable}{lemma}{BallLemmaStmt}\label{lem:ball}
Let  and  be real parameters, where . There exists a constant  and a polytope  in  of diameter at most  such that any outer -approximation of  requires at least  facets.
\end{restatable}


Intuitively, in order to produce a polytope that is hard to approximate, it should have high curvature. If the curvature is high in all dimensions, however, the polytope will have a small surface area, and this will make it easier to approximate. Our approach is to consider polytopes based on generalized cylinders, which have constant curvature in some dimensions but are flat in others. Our next lemma introduces such a cylindrical polytope where the number of curved dimensions has been carefully chosen to maximize the space needed by our algorithm for a given query time. Theorem~\ref{thm:lb} is an immediate consequence.

\begin{lemma} \label{lem:cyl}
There exists a polytope  in  such that for all sufficiently small positive  (depending on  and ) and  , the output of  on  has total space

\end{lemma}


\begin{proof}
To start, as a function of , we wish to compute an integer dimension  in order to apply Lemma~\ref{lem:ball}. Define reals ,  and . We observe first that

Let , implying that . (Although we do not include the derivation here,  has been chosen to produce the best lower bound, but since it is not necessarily an integer,  is obtained by rounding to a nearby integer.) Since  and , we have .

Let  denote the constant of Lemma~\ref{lem:ball}, and let . By our assumptions about  and , we have  and . It follows that for all sufficiently small , . Let  denote the linear subspace spanned by the first  coordinate axes. We apply Lemma~\ref{lem:ball} in  for this value of . The resulting polytope  (lying in ) has the property that the number of facets of any -approximation is at least

We can bound 's diameter by observing that for all sufficiently small 

(Here we made use of the fact that for all sufficiently small , the quantity raised to power of  is greater than .) Letting , we obtain

Since , for all sufficiently small , we have . Therefore,  can be enclosed within .

Returning to , consider an infinite polyhedral hypercylinder whose ``axis'' is the -dimensional orthogonal complement of , and whose ``cross-section'' (i.e., intersection with any -dimensional hyperplane parallel to ) is . Define the polytope  to be the truncated cylinder obtained by intersecting the infinite hypercylinder with hypercube  (see Figure~\ref{fig:cylinder}(a)). Let  denote the output of  for , , and . We will show that 's total space satisfies the bound given in the lemma's statement. To do this, let  denote any set of points placed on 's axis such that the distance between each pair of points is at least . (In the degenerate case where  the axis is -dimensional and  degenerates to a single point.) By a simple packing argument, there exists such a set having  points.

\begin{figure}[htbp]
  \centerline{\includegraphics[scale=0.40]{Figs/cylinder}}
  \caption{Lemma~\ref{lem:cyl} for  and .}
  \label{fig:cylinder}
\end{figure}


For any , let  denote the cross-section of  passing through  (see Figure~\ref{fig:cylinder}(b)). Consider the set of leaf cells of  that intersect . By applying Lemma~\ref{lem:ball} to the -dimensional hyperplane on which  lies, it follows that these cells together must contain at least  halfspaces. We count the contributions of these cells by classifying them into two types. We say that a leaf cell of  is \emph{large} if its side length is at least , and otherwise it is \emph{small}. By a simple packing argument, the number of large leaf cells intersecting  is at most . Since each leaf cell contains at most  halfspaces, the large leaf cells can together contain at most  halfspaces. 

Therefore, the small leaf cells intersecting  together contain at least  halfspaces. Because the points of  are separated from each other by distance at least , which is strictly larger than the diameter of any small leaf cell, each small leaf cell can intersect  for at most one . Therefore, the total space contribution of all the small leaf cells for all points of  is at least . Let . 's total space can be asymptotically bounded from below as

Clearly, . Recall that . Then, 's total space is asymptotically bounded from below as

Let  denote this exponent. In order to complete the proof, we provide a lower bound on . We use the fact that , apply the definitions of , , and , and straightforward manipulations to obtain

Substituting this value for the exponent in Eq.~(\ref{eq:expo}) completes the proof.
\end{proof}


\section{Approximate Nearest Neighbor Searching} \label{sec:ann}


In this section, we present a reduction from approximate nearest neighbor searching to approximate polytope membership, which will allow us to prove Theorem~\ref{thm:ann-ub} from Section~\ref{sec:intro}. Our reduction will involve the following additional assumptions regarding the implementation of {\alg}. First, (as in Section~\ref{sec:preproc}) we assume that  is presented as the intersection of  halfspaces. Second, we assume that a leaf node is labeled as ``inside'' only if it lies entirely within  (as opposed to lying within  as described in {\alg}). Third, we assume that leaf cells that store halfspaces use only bounding halfspaces of .

Clearly, these assumptions do not affect the data structure's correctness. We assert that they do not affect the data structure's asymptotic query time or space bounds. Regarding the second assumption, observe that for any cell  that lies within ,  can be -approximated within  using a single halfspace (any halfspace that contains  suffices). Regarding the third assumption, recall that Lemma~\ref{lem:discrete-apx} shows that we may assume that the approximating halfspaces for each node are drawn from the input halfspaces at the expense of a constant factor increase in the query time.

The reduction from approximate nearest neighbor searching to approximate polytope membership is based on the approximate Voronoi diagram (AVD) construction from~\cite{AVD-JACM}. The AVD employs a height balanced variant of a quadtree, a balanced box decomposition (BBD) tree~\cite{ARS} to be precise. Each cell of a BBD tree corresponds to the set theoretic difference of two quadtree cells, an \emph{outer box} and an optional \emph{inner box}. Each leaf cell of the tree stores a set of \emph{representative points} with the property that for any query point  lying within this cell, at least one of these representatives is an -nearest neighbor of . A query is answered by locating the leaf cell that contains the query point and then determining the nearest representative from this cell (by brute force). The AVD's space is dominated by the total number of representatives over all the leaf cells. The query time is the height of the tree plus the number of representatives in the leaf cell. A data structure for nearest neighbor searching is said to be in the \emph{AVD model} if it has this general form, that is, a covering of the query region by hyperrectangles of bounded aspect ratio, each of which is associated with a set of representative points~\cite{AVD-JACM}. Lower bounds on the performance of any data structure in the AVD model were given in~\cite{AVD-JACM}.

The reader need not be familiar with the details of the AVD data structure. The next lemma encapsulates the important technical information needed for our reduction. It follows easily from the proofs of Lemmas~{6.1} and~{8.1} in~\cite{AVD-JACM}. Given a cell  in a BBD tree, let  denote the ball of radius  whose center coincides with the center of 's outer box (see Figure~\ref{fig:separation}(a)). Given a Euclidean ball  of radius  and positive , let  denote the ball concentric with  of radius .

\begin{lemma} \label{lem:avd}
Let  be a real parameter and  be a set of  points in . It is possible to construct a BBD tree  with  nodes, where each leaf cell  stores a subset  satisfying the following properties:

\begin{enumerate}
\setlength{\itemsep}{-0.5ex}\setlength{\parsep}{0pt}\item[] For any point  in , one of the points in  is an -approximate nearest neighbor of .

\item[] At most one point of  is contained in the ball , and the remaining points of  are contained in  for some constant  (which depends on the dimension).

\item[] The total number of representative points over all the leaf cells of  is .
\end{enumerate}

Moreover, it is possible to compute the tree  and the sets  for all the leaf cells in total time , and the cell that contains a query point can be located in time .
\end{lemma}


\begin{figure}[htbp]
  \centerline{\includegraphics[scale=0.40]{Figs/separation}}
  \caption{Approximate nearest neighbor searching: (a) Lemma~\ref{lem:avd} (black points are members of ), (b) the lifting transformation. (Note that the figure is not drawn to scale, and the paraboloid in (b) has been translated to aid legibility.)}
  \label{fig:separation}
\end{figure}


In the AVD data structure of \cite{AVD-JACM} the closest representative point to a query point is determined by brute-force enumeration of the elements of . We consider whether it is possible search them more efficiently by reduction to polytope approximation. The following lemma explains how to connect Lemma~\ref{lem:avd} with approximate polytope membership queries. Our construction uses the well known \emph{lifting transformation}~\cite{ray-shooting-NN,edels}. Let  denote the coordinates of , and let us think of st coordinate axis as being directed vertically upwards. Let  denote the paraboloid . Given a point , let  denote the vertical projection of  onto  (see Figure~\ref{fig:separation}(b)), and let  denote the hyperplane tangent to  at . That is, the points of  satisfy . Given , let  denote the point on  hit by a vertical ray shot downwards from . A straightforward consequence of the definition of  is that the squared distance between  and  in  is equal to the length of this vertical segment, that is, . 

This suggests the following approach to computing the closest representative point through vertical ray shooting. Consider the (unbounded) convex polyhedron that results by taking the upper envelope of the hyperplanes  associated with the lifted representatives. Given the query point , a ray shot vertically downward from  hits some facet of this polyhedron. It follows from the above remarks, that the representative associated with this hyperplane is the closest to . We can simulate ray shooting by applying polytope membership queries in concert with binary search. Of course, some care will be needed to map this problem into our context, which assumes a bounded polytope and approximation.

\begin{lemma} \label{lem:mini-reduction}
Let  be a real parameter and consider a quadtree cell  and a set of representative points  as in Lemma~\ref{lem:avd}. Given a data structure for -approximate polytope membership in -dimensional space with query time  and space , it is possible to preprocess  into an ANN data structure for query points in  with query time  and space .
\end{lemma}


\begin{proof}
Since at most one point of  is contained in , the corresponding point may be inspected separately without increasing the complexity bounds. Therefore, we may assume that all points of  are contained in . 

Although we assume that the errors in polytope membership are absolute (because of standardization), errors in approximate nearest neighbor searching are relative. That is, a point  is an -approximate nearest neighbor of  if , where  is 's true nearest neighbor. Because errors are relative, we may assume that space has been translated and uniformly scaled so that  is mapped to , the hypercube of unit diameter centered at the origin in . As a result,  is mapped to a ball of radius . It follows that the distance from any point of  to any point of  is greater than . Therefore, an absolute error of  implies a relative error of at most .

In order to reduce nearest neighbor searching among the points of  to polytope membership, let  denote the upper envelope, that is, the intersection of the upper halfspaces, of the hyperplanes , for all  (the shaded region in Figure~\ref{fig:separation}(b)). As mentioned above, the facet of  hit by shooting a ray vertically downward from  corresponds to the closest point of  to . 

Since the upper envelope is unbounded, we first compute a bounded convex polytope on which to perform approximate membership queries. Because the query points lie in , we are only interested in the portion of  that projects vertically onto . Given that the distance of any point  to the origin is at most , it follows that the portion of  of interest fits within an axis-aligned -dimensional hypercube of constant diameter that is centered at the origin. Let  denote such a hypercube, let , and let . We invoke {\alg} to construct an -approximate membership data structure for . (More formally, we first scale  into standard form, and we scale  by the same factor. We then apply {\alg} with the scaled value of . Since  is of constant diameter, the scale factor will also be a constant, and therefore only the constant factors in the analysis will be affected. We then apply an inverse scaling to obtain the desired -approximating polytope for .)

We simulate the ray shooting process by a binary search to locate the contact point approximately. Consider the vertical segment formed by intersecting  with the vertical line passing through . The upper endpoint of this segment is clearly inside  and its lower endpoint is outside. We repeatedly split the segment at its midpoint, perform an approximate polytope membership query, and retain the subsegment whose upper endpoint is (approximately) inside  and whose lower endpoint is (approximately) outside. We terminate the search when the length of the segment falls below . Since  is of constant diameter, the search terminates after  membership queries. Let us denote the endpoints of this final segment as  (upper) and  (lower).

Recall our assumption that cells are labeled by {\alg} as ``inside'' or ``outside'' only if they lie entirely inside or outside , respectively. It follows that as we traverse the cells that intersect the segment  from top to bottom, we cannot transition directly from an ``inside'' cell to an ``outside'' cell. Therefore, at least one of these cells must contain a set of representative hyperplanes. Let  denote the hyperplane having the topmost intersection with the vertical ray. We return  as the approximate nearest neighbor (see Figure~\ref{fig:paraboloid}). It is easy to see that this algorithm satisfies the desired time and space bounds. 

\begin{figure}[htbp]
  \centerline{\includegraphics[scale=0.40]{Figs/paraboloid}}
  \caption{Proof of Lemma~\ref{lem:mini-reduction}. (Not drawn to scale.)}
  \label{fig:paraboloid}
\end{figure}


All that remains is to establish correctness, by showing that  is indeed an -approximate nearest neighbor of . In order to do this, let  be 's true nearest neighbor in . Due to the nature of the binary search,  lies within distance  of . (Note that it might lie within .) Thus, the distance from  to the upper halfspace bounded by  is at most . By the triangle inequality, the distance from  to this halfspace is at most . Since  is 's true nearest neighbor,  lies on , and so the hyperplane  separates  from . This implies that the distance from  to  is also not greater than . 

We claim that the vertical distance from  to  is at most . To see why, recall that  lies within a ball of radius  centered at the origin. This implies that  cannot be too steep, that is, the angle formed between 's normal vector and the vertical axis can be bounded away from  by a constant. By basic linear algebra, it can be shown that the ratio of the vertical and orthogonal distances of any point to  is bounded above by . Therefore, we have , as desired. 

Because  is the witness produced by the algorithm,  separates  from , which implies that  lies above . Thus, we have . Therefore,

By the lifting transformation, we have , and combining this with the fact that , we have

Therefore,  is an -approximate nearest neighbor of , which completes the proof.
\end{proof}


The above lemma shows how to apply approximate polytope membership to efficiently answer approximate nearest neighbor queries within each cell of the AVD. To obtain a complete data structure for approximate nearest neighbor searching we apply this to every leaf cell of the AVD.

\begin{lemma} \label{lem:reduction}
Let  be a real parameter and  be a set of  points in . Given a data structure for approximate polytope membership in -dimensional space with query time at most  and storage , it is possible to preprocess  into an ANN data structure with query time  and space

\end{lemma}


\begin{proof}
Following Lemma~\ref{lem:avd}, construct a BBD-tree , and for each leaf cell  of , construct the set of representative points . For each leaf cell such that , simply store the set  and answer the corresponding queries by brute force. For the nodes with , use the construction from Lemma~\ref{lem:mini-reduction}. 

To answer an ANN query we search the AVD of Lemma~\ref{lem:avd} to find the leaf cell containing the query point and then apply Lemma~\ref{lem:mini-reduction}. Thus, the query time is 

To bound the total space, observe from Lemma~\ref{lem:avd}(iii) that the total number of representative points is . Thus, by a simple counting argument, the number of leaf cells with more than  representatives is . Therefore, the total space of the data structure is .
\end{proof}


Because of its reliance on binary search, the generic reduction given in Lemmas~\ref{lem:mini-reduction} and~\ref{lem:reduction} is not formally in the AVD model. Recall that the AVD model is important because lower bounds have been established in this model~\cite{AVD-JACM}, and thus these lower bounds do not apply here. However, by sacrificing generality and a factor of  in the space bound, we can exploit the properties of {\alg} to obtain a data structure that is in the AVD model.

\begin{lemma} \label{lem:avd-model}
Let  be a real parameter and  be a set of  points in . Given a split-reduce data structure for approximate polytope membership in -dimensional space with query time at most  and storage , it is possible to preprocess  into an ANN data structure in the AVD model with query time  and space

\end{lemma}


\begin{proof}
As in Lemma~\ref{lem:reduction}, construct a BBD-tree , and for each leaf cell  of , construct the set of representative points . We may assume that , since otherwise we just use the points of  as the representatives. In order to handle query points lying within , we apply Lemma~\ref{lem:mini-reduction}, where queries are answered using the tree produced by {\alg}. Let  denote the resulting tree. We exploit the fact that the {\alg} data structure associates a collection of hyperplanes with each leaf cell of , and by the nature of our reduction, each of these hyperplanes corresponds to a lifted point of . These lifted points will play the role of nearest neighbor representatives. Intuitively, our approach is to ``undo'' the lifting transformation by projecting the leaf cells of  vertically from  down to  and then building a -dimensional AVD structure based on this projection. 

The projection of the cells of  onto  naturally defines a quadtree subdivision of , which we denote by  (see Figure~\ref{fig:avd-model}(a)). For each leaf cell  of , let  denote the infinite vertical cylinder in  whose cross section is  (see Figure~\ref{fig:avd-model}(b)). Because  is a leaf, any leaf cell of  that intersects this cylinder projects onto a hypercube that contains . 

\begin{figure}[htbp]
  \centerline{\includegraphics[scale=0.40]{Figs/avd-model}}
  \caption{Producing an ANN data structure in the AVD model.}
  \label{fig:avd-model}
\end{figure}


Recall the lifted polytope  of Lemma~\ref{lem:mini-reduction}. For each leaf cell of  that contains a point whose vertical distance from  is at most , we create a representative point corresponding to each of the hyperplanes that {\alg} associates with this leaf cell. We denote the resulting collection of representatives by . These are the only hyperplanes that are relevant to the binary search of Lemma~\ref{lem:mini-reduction}, and therefore one of them will provide the final witness in the binary search (the point  in the proof of Lemma~\ref{lem:mini-reduction}). This implies that  constitutes a valid representative set for -approximate nearest neighbor searching for any query point that lies in . Thus, the resulting data structure is a valid AVD structure. 

In order to bound the query time we recall some of the observations made in the proof of Lemma~\ref{lem:mini-reduction}. Since  is contained within a hypercube of constant diameter centered at the origin, the absolute slopes of the hyperplanes of the approximating polytope are bounded above by some constant. Recall that the leaf cells of  that contribute a point to  have side lengths at least as large as that of . By the same reasoning used in Lemma~{3} of \cite{ARS}, the number of such quadtree leaf cells that can intersect  is bounded by a constant, which we denote by . (This constant depends on the dimension  and the largest possible slope.) Therefore, the total number of cells contributing a representative to  is at most . Since each cell contributes at most  representatives, the total number of representatives associated with any leaf cell of  is at most .

The bound on the total space is complicated by the fact that a large cell that intersects  may overlap the columns of many small leaf cells, and hence a large cell's representatives may be replicated many times. Let  denote the set of internal nodes of  all of whose children are leaves. We encountered this set earlier in the proof of Lemma~\ref{lem:total-space}. As we saw in that earlier lemma, because each node of  was split by {\alg}, it follows that each such cell requires more than  halfspaces to approximate , and thus, the children of  together require at least as many representatives. Therefore we have . Reasoning as we did in Lemma~\ref{lem:total-space}, every internal node of  is either in  or is an ancestor of a node in . Thus, the number of internal nodes is at most . Since every internal node has  children, the total number of nodes in  is at most . Clearly, the number of leaf cells of  can be no larger. As we saw in the previous paragraph, each leaf cell of  is associated with at most  representatives. Since the tree is of height , the total number of representatives over all these cells is at most

By Lemma~\ref{lem:avd}(iii), the total number of representatives in  is . By a counting argument, the number of leaf cells with more than  representatives is . Therefore, the total space is

as desired.
\end{proof}


By combining this with Theorem~\ref{thm:membership-ub} (applying the more accurate space bounds from Lemma~\ref{lem:trade-off-ub}) we obtain the main result of this section.

\begin{lemma} \label{lem:ann-ub}
Let  be a real parameter,  be a real constant, and  be a set of  points in . There is a data structure in the AVD model for approximate nearest neighbor searching that achieves

The constant factors in the space and query time depend only on  and  (not on ). At the expense of increasing the query time and space by a factor of  it is possible to construct the data structure in time , for some constant  (that does not depend on  or ).
\end{lemma}


\begin{proof}
Given  and , we first observe that if , we may set , since this will only affect the constant factors in the asymptotic bounds. We consider two cases based on the value of . 

If , we will apply Theorem~\ref{thm:membership-ub} with the values of  and  of the theorem set to  and , respectively. The theorem states that there is a data structure that achieves query time 

and space

Letting  and  denote the quantities of Eqs.~\eqref{eq:ann-ub-time} and~\eqref{eq:ann-ub-space}, respectively, we apply Lemma~\ref{lem:avd-model} to obtain a data structure in the AVD-model with query time  and space

as desired.

Otherwise, if , we apply Theorem~\ref{thm:membership-ub} (but using the more accurate space bounds from Lemma~\ref{lem:trade-off-ub}) in dimension  and with trade-off parameter . (Observe that , as required by Theorem~\ref{thm:membership-ub} and Lemma~\ref{lem:trade-off-ub}.) This yields an approximate polytope membership data structure with query time  and space

By Lemma~\ref{lem:avd-model} this implies the existence of a data structure in the AVD-model with the desired query time of  and space

Since , we may ignore the ``'' term in the inner parenthetical factor. After some simplification we obtain the desired space bound of


The preprocessing involves first computing the AVD, which by Lemma~\ref{lem:avd} takes  time. For each of the  leaf cells  of the AVD, we apply {\alg} in dimension  to its associated set  of representatives. By Lemma~\ref{lem:preproc-time} this takes  time, where , and  is a constant that does not depend on . Summing over all the leaf cells of the AVD and recalling that the total number of representatives is , it follows that the total preprocessing time is on the order of

where , as desired. Because of the reliance on approximate set cover in the processing of Lemma~\ref{lem:preproc-time}, the query time and space are larger by a factor of .
\end{proof}


Note that the above proof uses the AVD-based reduction given in Lemma~\ref{lem:avd-model}. If instead we had used Lemma~\ref{lem:reduction}, we would obtain a slight improvement in the space, by a factor of , at the loss of having a data structure in the AVD model. By the simple observation that , the above space bound for the  case simplifies to , and this establishes Theorem~\ref{thm:ann-ub}.

\section{Proof of the Area-Product Bound} \label{sec:proof}


In this section, we present lower bounds for the product of the area of (restricted) -dual caps and the associated Voronoi patches, and in particular, we present a proof of Lemma~\ref{lem:dual-basic}, which appeared at the end of Section~\ref{sec:cap}.

We begin by recalling some notation. We are given a convex body  in , and a pair  where  and  is a supporting hyperplane passing through , such that  lies within a unit ball centered at the origin. Also recall that  denotes the point lying at distance  from  in the direction of the outward normal orthogonal to  at .  denotes the Dudley hypersphere, which is centered at the origin and is of radius . For , let  be any hyperplane that is parallel to  and translated away from  by distance . (This is illustrated in Figure~\ref{fig:area-bound-setup}. Note that the figures of this section are not drawn to scale.) To simplify our descriptions, we consider the directed line segment from  to  to be ``vertically downwards,'' so that the hyperplanes  and  are ``horizontal'' with  above .

\begin{figure}[htbp]
  \centerline{\includegraphics[scale=0.40]{Figs/area-bound-setup}}
  \caption{Definitions of ,  and .}
  \label{fig:area-bound-setup}
\end{figure}


Recall that the \emph{-dual cap} defined by , denoted , is the portion of  that is visible from  (see Figure~\ref{fig:voronoi}(a)). Also, recall that  consists of the points that are exterior to  whose closest point on  lies within . Define the \emph{base} of , denoted , to be the intersection of  with the convex hull of . 

For , recall that the \emph{-restricted -dual cap} defined by , denoted , is , where  is the Euclidean ball of radius  centered at  (see Figure~\ref{fig:voronoi}(b)). As before,  is the set of points that are exterior to  whose closest point on  lies within . Also, the \emph{-restricted base}, denoted  is .

Our objective in this section is to establish bounds on the product of the area of a -restricted -dual cap and its Voronoi patch on the Dudley hypersphere. It will be easier to start with hyperplane patches on  and then generalize to spherical patches on . The main result of this section is given in the following lemma. Part~(ii) is equivalent to Lemma~\ref{lem:dual-basic}, which is our main objective. Part~(i) is a useful intermediate result.

\begin{figure}[htbp]
  \centerline{\includegraphics[scale=0.38]{Figs/voronoi}}
  \caption{Dual caps, bases, and Voronoi regions for the (a) unrestricted and (b) restricted cases.}
  \label{fig:voronoi}
\end{figure}


\begin{lemma} \label{lem:dual}
Let  be a convex body in , and let  and . There are constants  and  (depending only on ) such that for any point :
\begin{enumerate}
\item[]  Given any , .

\item[] If  is fat and has diameter at least , and  lies within a unit ball centered at the origin, then .
\end{enumerate}
\end{lemma}


This lemma holds generally for any , but it suffices for our purposes to consider the restricted case of . Note that the additional assumptions on fatness and diameter of part~(ii) are necessary for establishing a lower bound. If  is not fat or not of sufficiently large diameter, then  can be arbitrarily small. Since the Dudley hypersphere is bounded, it would not be possible to establish any lower bound on the product of their areas. 

The remainder of this section is devoted to proving this lemma. Because  will be fixed throughout, in order to simplify the notation, we will drop references to . For example, we will use , , , , and  in place of , , , , and , respectively.

Since it will be useful to relate sets on  with sets on , we observe that each of these hyperplanes can be consistently identified with  by endowing them with parallel coordinate frames, one centered at  (for ) and one centered at 's orthogonal projection onto . Thus, a point on  and its vertical projection onto  have the same coordinates.

We start by proving Lemma~\ref{lem:dual}(i). Since the value of  will be fixed throughout this part of the proof, we refer to  simply as . Let  denote the origin of 's coordinate system (the vertical projection of  onto ). (See Figure~\ref{fig:area-bound-setup}.) In order to exploit Lemma~\ref{lem:mahler} on the Mahler volume, rather than considering  directly, we will find it convenient to instead analyze the polar dual of the base . Using the aforementioned coordinate frame, we can think of  as a body in . For , consider the generalized polar of the dual base, , which we can think of as a convex subset of . Because  contains the origin of  (namely, ), it follows directly that  is bounded, convex, and also contains the origin of  (namely, ). In order to obtain a lower bound on , we will first show that  is a subset of  and then derive a lower bound on . The first assertion is established by the following lemma.

\begin{lemma} \label{lem:dual-subset}
Given the preconditions of Lemma~\ref{lem:dual} and , we have .
\end{lemma}


The proof is rather technical and involves a reduction to the problem in 2-dimensional space. Before giving the proof, it will help to provide some intuition regarding the relationship between  and the polar of . 

For the sake of simplicity, let us consider just the 2-dimensional setting. Let  denote a point of tangency on  with respect to  (see Figure~\ref{fig:dual-subset-intuition}), and let  be the intersection of the line segment  with . Shoot a ray from  perpendicular to  until it intersects . Let  denote this intersection point. Since  is convex, all the points on the segment  have their nearest neighbor on the portion of  between  and , that is, they all lie within . Observe that if we translate this perpendicular line so that it emanates from  instead of , it will hit  at a point  that is closer to . Therefore, the segment  also lies within . Let  denote the distance between  and . By similar triangles, it is easy to see that the length of  is . Since ,  lies within , where . Because  and , we have . This observation generalizes readily to higher dimensions, and it follows that . We will show how to generalize this intuition to higher dimensions and the -restricted setting.

\begin{figure}[htbp]
  \centerline{\includegraphics[scale=0.40]{Figs/dual-subset-intuition}}
  \caption{The relationship between  and the polar of .}
  \label{fig:dual-subset-intuition}
\end{figure}


For any  let  denote its nearest neighbor on . In order to prove Lemma~\ref{lem:dual-subset}, it suffices to show that if  (implying that ), then . By our assumption that  lies below  it follows that  lies on the ``lower surface'' of  (meaning that a vertical ray directed downwards from  does not intersect the interior of ). Since  is not in the restricted cap, we know that either  or .

It will simplify the analysis to reduce the problem to a 2-dimensional setting. Consider the plane  that contains the points , , and . (Note that these points are not collinear.) Let  be the point of tangency on  with respect to  that lies on the same side as  (see Figure~\ref{fig:dual-subset}(a)). Let  be the intersection of the line segment  with . We may identify  with  by imposing a coordinate system on  where the origin is at , the -axis is directed upwards away from  and the -axis is parallel to the vector from  to . Given a point , let  and  denote its coordinates relative to this coordinate system. Further, if , let  denote the slope of the (unique) supporting line on  passing through . Note that  need not lie on . Let  be the orthogonal projection of  onto . Observe that , and therefore . By our choice of coordinate system and assumptions about orientations, the coordinates of , , and the slopes  and  are all nonnegative quantities. 

\begin{figure}[htbp]
  \centerline{\includegraphics[scale=0.40]{Figs/dual-subset}}
  \caption{The reduction to the plane .}
  \label{fig:dual-subset}
\end{figure}


The point  lies on the base  of 's unrestricted dual cap. By employing our coordinate system on , we can identify  with a vector in  (emanating from ). If , then  contributes a bounding halfspace to . This halfspace is bounded by a hyperplane that is orthogonal to  and lies at distance  from the origin. Let us think of this halfspace, which we denote by , as lying on  (see Figure~\ref{fig:dual-subset}(a)). Recalling that , the distance of 's bounding hyperplane to the origin  is . On the other hand, if , then  lies outside the restricted base. In this case 's subvector of length  lies on the boundary of the restricted base and contributes to  a halfspace whose bounding hyperplane is at distance  from the origin. Recalling that , this is equal to . Thus, in either case,  is bounded by a halfspace whose defining hyperplane is orthogonal to  and lies at distance  from the origin. This hyperplane intersects the horizontal line  at some point  that lies to the right of  at distance  (see Figure~\ref{fig:dual-subset}(b)). 

Because the hyperplane passing through  is orthogonal to , in order to show that , it suffices to show that  does not lie within , which is equivalent to showing that . We have thus reduced the problem to a two-dimensional setting.

Recall that  is the closest point to  on . We assert that  is also the closest point to  on . The reason is that the squared distance from  to any point on  can be expressed as the sum of the squared distance from  to this point and the squared distance from  to . Since the latter quantity is the same for all points on , the closest point to  is also the closest point to . From basic properties of convexity, it follows that the line  is orthogonal to the support line passing through  on . Therefore, the slope of  (in 's coordinate system) is , and in particular we have . Since  and  are separated by at least unit distance (with  above ), we have , and so . 

Thus, to complete the proof of Lemma~\ref{lem:dual-subset}, it suffices to show that if  then . We first establish two useful technical results. These results will be applied in a context where  lies within the unrestricted dual cap but outside the restricted dual cap. That is, when  but . The first result shows that if  is sufficiently small, the slope of the line  is at most unity. The second shows that if  is sufficiently large, the slope of  is not much smaller than the slope of 's supporting line.

\begin{restatable}{lemma}{DualSubsetHelperStmt}\label{lem:dual-subset-helper}
Given the preconditions of Lemma~\ref{lem:dual} and the aforementioned 2-dimensional reduction, and given  and  as introduced above, where  and :
\begin{enumerate}
\item[] if , then , and

\item[] if , then .
\end{enumerate}
\end{restatable}


The proof involves simple geometry and is given in the appendix.

\medskip

We are now in position to complete the proof of Lemma~\ref{lem:dual-subset}. Recall that our objective is to show that if  then , where . We consider two cases, depending on . First, if , then . Since the line  has slope  and , we have . We consider two subcases. If , then we have , as desired. On the other hand, if , then  is inside the unrestricted cap . Since by our hypothesis,  is not in the restricted cap, it must be that , that is, . By Lemma~\ref{lem:dual-subset-helper}(i), we have . Therefore, , which implies that . Therefore, , as desired.

For the second case, assume that . In this case . As before, we consider two subcases. If , then by convexity , and so , as desired. On the other hand, if , then since  lies within the unrestricted cap, we may infer that . By Lemma~\ref{lem:dual-subset-helper}(ii), we have . Because the support line at  passes below the origin, we also have . Therefore . This completes the proof of Lemma~\ref{lem:dual-subset}.

\medskip

Because it is easier to deal with flat objects than curved ones, before returning to the proof of Lemma~\ref{lem:dual}(i), we show that the area of the restricted dual cap is, up to a constant factor, bounded below by the area of its base. This result is straightforward for unrestricted caps, since it is easy to show that the base is contained within the orthogonal projection of the dual cap onto . However, restriction complicates the analysis. The proof involves a technical geometric argument and is presented in the appendix.

\begin{restatable}{lemma}{BaseCapAreaStmt}\label{lem:base-cap-area}
Given the preconditions of Lemma~\ref{lem:dual}, it follows that .
\end{restatable}


\medskip

We are now ready to prove Lemma~\ref{lem:dual}(i). Recall that . As observed earlier,  is a scaled copy of  by a factor of , and therefore (since these are -dimensional bodies) we have . By applying Lemma~\ref{lem:dual-subset}, we have

By Lemma~\ref{lem:base-cap-area}, , and therefore

We now apply the Mahler-volume bound. By Lemma~\ref{lem:mahler} (in ), there exists a constant  (depending only on ) such that . Therefore,

Selecting any  establishes Lemma~\ref{lem:dual}(i).

\bigskip

Next, let us establish Lemma~\ref{lem:dual}(ii). Recall that we assume that  is fat and of diameter at least . In particular, let us assume that  is -fat, where  is a constant independent of  and  that lies in the interval . (As a result of Lemma~\ref{lem:fat}, we may assume that  is  when applying this result.)

It is natural to try to generalize the approach used in part~(i). First, we would show that 

and then we would apply the Mahler-volume bound to yield a lower bound on the product . A problem arises, however, if  is not smooth. In particular, if some portion of the boundary of  in 's vicinity is nearly vertical, then the boundary of  can be arbitrarily close to the origin (namely ), implying that  cannot be bounded, and hence its area can be arbitrarily large. This was not an issue in part~(i), because  is also unbounded. But since  is bounded,  cannot be arbitrarily large. We will remedy this by smoothing  by taking its Minkowski sum with a small Euclidean ball of radius . We shall see (in the proof of Lemma~\ref{lem:smooth-to-polar}) that this allows us to constrain the area of . This smoothing operation requires us to adapt many of the prior results of this section to this new context.

To construct the smoothed body, for the remainder of this section define , and let  (see Figure~\ref{fig:smooth}(a)). Recall that  denotes the supporting hyperplane at  and  is the point at distance  from  in the direction orthogonal to . As before, for the sake of illustration, let us assume that  is vertically below . Let  be the midpoint of the segment . Clearly, , and the parallel hyperplane  passing through  is a supporting hyperplane for .

\begin{figure}[htbp]
  \centerline{\includegraphics[scale=0.40]{Figs/smooth}}
  \caption{The smoothed body .}
  \label{fig:smooth}
\end{figure}


Let us also define the dual base in this smoothed context. Define  to be the intersection of  and . Let , and define the restricted base  to be the intersection of  and a ball of radius  centered at  (see Figure~\ref{fig:smooth}(b)). Our analysis will be based on  and , as opposed to  and . Our first objective will be to show that the area of  is not significantly larger than that of . As before, we endow  and  with parallel coordinate frames whose origins are located at  and , respectively. Then we can think of  and  as convex sets in . The following lemma relates these two bodies.

\begin{restatable}{lemma}{SmoothAreaStmt}\label{lem:smooth-area}
Given a convex body  that is -fat and of diameter at least  and given  and  as defined above, there exists a constant  (depending on  and the dimension ) such that .
\end{restatable}


The proof is rather technical, but it involves simple geometric reasoning. It is given in the appendix.

Recall that  consists of the set of points on the sphere  whose closest point on  lies within the restricted dual cap . Let  denote the corresponding restricted dual cap for , that is, the set of points of  that are visible from  and lie within the ball . Our analysis will be based on establishing a lower bound on the area of . The following lemma shows that this will provide a lower bound on the area of .

\begin{lemma} \label{lem:smooth-voronoi}
Given the preconditions of Lemma~\ref{lem:dual}(ii), .
\end{lemma}


\begin{proof}
We will prove the stronger result that . We begin by observing that both  and  lie within the ball bounded by the Dudley hypersphere . To see this, recall that by the conditions of Lemma~\ref{lem:dual}(ii),  lies within unit distance of the origin, and so by the triangle inequality  lies within distance  of the origin. The points of  and  lie within distances  and  of  and , respectively. Therefore, the distance of any point of  or  from the origin is at most . As shown in the proof of Lemma~\ref{lem:smooth-area}, , and therefore this distance is at most . Therefore, both caps lie within . 

Consider any point . It suffices to show that . Let  be the closest point to  on . By convexity,  is the closest point of  on  if and only if the supporting hyperplane at , denoted , is orthogonal to the line .  Let  be the closest point to  on . By basic properties of Minkowski sums, the segment  is orthogonal to both the supporting hyperplanes  and  at  and , respectively (see Figure~\ref{fig:smooth-voronoi}). It follows that all three points ,  and  are collinear, and  is orthogonal to the segment . Therefore,  is the closest point to  on .

\begin{figure}[htbp]
  \centerline{\includegraphics[scale=0.40]{Figs/smooth-voronoi}}
  \caption{Proof of Lemma~\ref{lem:smooth-voronoi}.}
  \label{fig:smooth-voronoi}
\end{figure}


Since , we have , which means that  is visible from  and  lies within distance  of . Because  and  are parallel, it follows that  is also visible from . Therefore,  lies in 's unrestricted dual cap . To prove that  lies within the restricted cap , it suffices to show that . We apply a standard result from convexity theory which states that for any convex body  and two points  and  that are exterior to , if  and  are their respective nearest neighbors on , then  (see, e.g., Lemma~{4.3} in \cite{Dudley}). Clearly, this applies in our situation, and so , which implies that . In summary, the closest point to  on  lies within , implying that , as desired.
\end{proof}


Before completing the proof of Lemma~\ref{lem:dual}(ii), we exploit the smoothness of  to establish a relationship between the areas of  and , where the polar radius  is suitably modified for the smoothed context. This is given in the next lemma. The proof involves basic geometric reasoning and is given in the appendix.

\begin{restatable}{lemma}{SmoothToPolarStmt}\label{lem:smooth-to-polar}
Given the preconditions of Lemma~\ref{lem:dual}(ii) and , we have .
\end{restatable}


\bigskip

We are now ready to prove Lemma~\ref{lem:dual}(ii). Recall that . By Lemmas~\ref{lem:smooth-voronoi} and~\ref{lem:smooth-to-polar}, we have

As observed earlier,  is a scaled copy of  by a factor of , and therefore (since these are -dimensional bodies) we have

By Lemma~\ref{lem:base-cap-area}, . Also, by Lemma~\ref{lem:smooth-area} there is a constant  (depending on the fatness parameter  and ) such that . Therefore, we have

Combining these, we obtain

By applying Lemma~\ref{lem:mahler} (in ) to , there exists a constant  (depending on ) such that

Therefore, we have

Selecting any  establishes Lemma~\ref{lem:dual}(ii). This concludes our proof of the area bounds. 

\section{Concluding Remarks}


In this paper we have presented an efficient data structure for determining approximately whether a given query point lies within a convex body. Our solution is based on a simple and natural quadtree-based algorithm, called {\alg}. Our principal technical contribution has been an analysis of the space-time trade-offs for this algorithm. These are the first nontrivial space-time trade-offs for this problem. We do not know whether this analysis is tight, but we presented a lower bound example that demonstrates the limits of possible improvements. We also demonstrated the value of approximate polytope membership by showing that our data structure can be combined with an AVD data structure to produce significant improvements to the space-time trade-offs of approximate nearest neighbor searching in Euclidean space.

Our analysis of the trade-offs involved a combination of a number of novel techniques, which may be of broader interest. One notable example is the application of the Mahler volume as a means of analyzing the local structure of a convex body through consideration of both its primal and dual representations. This resulted in an efficient two-pronged sampling strategy for computing hitting sets of low cardinality for -dual caps. The Mahler volume has also been applied in \cite{AFM12} to derive an optimal area-sensitive bound on the number of facets needed to approximate a convex body.

This work provokes a number of questions for further research. The first involves extending approximate polytope membership queries to other approximate query problems involving convex bodies. For example, in Section~\ref{sec:ann} we showed how to reduce approximate nearest neighbor searching in dimension  to vertical ray shooting queries in dimension . However, the polytope involved had a very restricted structure. It would be interesting to know whether there is a data structure exhibiting similar trade-offs for answering approximate ray-shooting queries for general convex bodies. Another example is answering approximate linear-programming queries, where a convex body is preprocessed, and the problem is to determine an  extreme point of the body approximately in a given query direction. A further generalization of this would be to extend the work of Barba and Langerman \cite{BaL15} to an approximate setting. It particular, is it possible to preprocess convex bodies so that given two such bodies that have been translated and rotated, it can be decided efficiently whether they intersect each other approximately.

Our result on approximate nearest neighbor searching relies on the lifting transformation to reduce the problem to approximate polytope membership. As a consequence, this approach is applicable only to Euclidean distances. This raises the question of whether there exists a more direct route to approximate nearest neighbor searching that achieves similar space-time improvements and yet avoids reliance on lifting. For example, Arya and Chan \cite{ArC14} have presented improvements to approximate nearest-neighbor searching that do not involve lifting. This raises the hope that generalizations to other norms may be possible. While their focus was different from ours (for example, space-time trade-offs are not considered), their results are inferior to our best bounds. These better bounds arise explicitly from concepts like the Mahler volume, which are only applicable in the context of convex approximation, and hence they rely crucially on lifting. A major challenge is whether it is possible to bypass this intermediate step in order to obtain analogous improvements for approximate nearest neighbor searching.

\section{Acknowledgments}


The authors would like to thank the anonymous reviewers for the journal version of the paper for their many insightful comments.

\subsection*{Note Added in Proof}


After the original submission of this paper, the authors have discovered a new approach to polytope membership that achieves query time  with storage of only ~\cite{AFM17}. As a consequence, it is possible to answer -approximate nearest neighbor queries for a set of  points in  time with storage of only . While these new results surpass the results of this paper theoretically, the data structure presented there involves significantly larger constant factors and lacks the simplicity and practicality of the approach described here.



\bibliographystyle{abbrv}
\bibliography{shortcuts,polytope-journal}



\appendix

\section{Omitted Proofs}


{\VCLemmaStmt*}


\begin{proof}
We may assume that  contains the origin, since this clearly does not affect the VC-dimension of these range spaces. To prove~(1), consider the set of augmented points  (recalling that  is any supporting hyperplane passing through ). Under the polar transformation (recall Section~\ref{sec:prelim-polar}), these supporting hyperplanes are mapped to a set of points that form the boundary of , which is a convex body. Consider any point  that is external to  (see Figure~\ref{fig:cap-duality}(a)). If we treat  as a vector,  is a hyperplane that intersects  (see Figure~\ref{fig:cap-duality}(b)). 

\begin{figure}[htbp]
  \centerline{\includegraphics[scale=0.40]{Figs/cap-duality}}
  \caption{Proof of Lemma~\ref{lem:VC}.}
  \label{fig:cap-duality}
\end{figure}


Consider the \emph{cap} of  induced by , which we define to be the points of the boundary of  that are separated from the origin by . By the inclusion-reversing properties of the polar, the points of this cap are in 1--1 correspondence with the supporting hyperplanes of  that separate  from . It follows that the set of dual caps of  is equivalent, through polarity, to the set of caps of . The range space of caps is equivalent to the range space of halfspaces, which is known to have constant VC-dimension, and therefore the VC-dimension of dual caps is equal.

To establish~(2), consider the function that maps each point  to its closest augmented point , where  is chosen to be orthogonal to the line segment . This function is a bijection (in fact, a homeomorphism) from the points of  to the augmented points of . This function induces a 1--1 correspondence between the set of -dual caps of  (in fact, any set of surface patches) and the Voronoi patches associated with these dual caps. Therefore,  and  have the same VC-dimension.
 
To establish~(3) and~(4), observe that each range from  (resp., ) results from intersecting a range of  (resp., ) and a Euclidean ball. It is well known (see, e.g.,~\cite{Mat02}) that a range space that results by taking the intersection of ranges from two range spaces of constant VC-dimension has itself constant VC-dimension.
\end{proof}


{\BallLemmaStmt*}


\begin{proof}
Consider a -dimensional ball  of radius , and let  be any polytope that is an outer -approximation of , that is, . Since ,  is of diameter at most . We will show that  satisfies the conditions of the lemma.

Since the Hausdorff distance is a metric, it follows from the triangle inequality that any polytope  that is an -approximation of  is a -approximation of . To complete the proof, it suffices to show that any -approximation of  has at least the desired number of facets. To do this, define , and let  denote a set of points on  such that the distance between any two points of  is at least . By a simple packing argument, for a suitable constant  there exists such a set of size at least . It is easy to see that the function that maps each point of  to its closest point on  induces a 1--1 correspondence between these two sets. Let  be the points corresponding to  on .

\begin{figure}[htbp]
	\centerline{\includegraphics[scale=0.40]{Figs/pythag}}
	\caption{Proof of Lemma~\ref{lem:ball}.}
	\label{fig:pythag}
\end{figure}


We assert that the points of  lie on distinct facets of . To see this, suppose to the contrary that two points  were on the same facet of  (see Figure~\ref{fig:pythag}). Then the line segment  lies on . Because  is a -approximation of ,  and  are both within distance  of 's center. Therefore, by the Pythagorean theorem and the fact that , it follows that the distance from the midpoint of the line segment  to 's center is at most


This implies that the line segment  intersects 's interior, which contradicts the hypothesis that  is an outer approximation. Therefore,  must have at least  facets, and this completes the proof.
\end{proof} 


{\DualSubsetHelperStmt*}


\begin{proof}
To prove part~(i), observe first that, by the preconditions of Lemma~\ref{lem:dual},  and , and so . Since the supporting line passing through  passes on or below the origin, we have . (It may be helpful to refer to Figure~\ref{fig:dual-subset}.) Since , by convexity we have . Combining this, we have , as desired.

To prove part~(ii), observe that because  is a point of tangency with respect to ,  lies on or above the line . This implies that . Since  is positive, . Because , we know that , which implies that . If , then

By our bound on , we have , implying that , as desired. 

On the other hand, if , then 

The definition of  and our bounds on  imply that , and so , which completes the proof of~(ii).
\end{proof}


{\BaseCapAreaStmt*}


\begin{proof}
Let  denote the orthogonal projection of  onto . Since the area of the orthogonal projection of a set cannot exceed the area of the original set, . Given a set  in real space, let  denote the set . We assert that  (using the aforementioned coordinate system on  whose origin is at ). Since  lies in , from this assertion we will have

from which the result will follow.

It remains to prove the assertion. Consider any , and recall that  denotes the ball of radius  centered at . By definition of , , and so . It suffices to show that , that is, there exists a point  whose projection onto  is . Note that this is trivially true if , and so we may assume that . Under this assumption, consider the unique plane  defined by the points , , and . We will impose the same coordinate system on  as in the proof of Lemma~\ref{lem:dual-subset}, with  as origin and  on the negative -axis. Henceforth, we restrict our attention to this plane.

We may assume by symmetry that  lies in the positive -halfplane. By definition of , there exists a point  on the lower surface of  such that the line segment  passes through . Let  be the point of  whose orthogonal projection onto  equals  (see Figure~\ref{fig:base-cap-area}). Note that  exists along the portion of  between  and , and therefore . Recall that  and  denote the slopes of the support lines at  and , respectively. By basic properties of convexity, we have .

\begin{figure}[htbp]
  \centerline{\includegraphics[scale=0.40]{Figs/base-cap-area}}
  \caption{Proof of Lemma~\ref{lem:base-cap-area}.}
  \label{fig:base-cap-area}
\end{figure}


We consider two cases, based on the location of . First, if , then  (see Figure~\ref{fig:base-cap-area}(a)). This implies that . Clearly,  is star-shaped with respect to , and therefore .

On the other hand, if , then  (see Figure~\ref{fig:base-cap-area}(b)). Because  lies on the line , we have . Also, since  and , we have . Combined with the fact that , this yields . Let  be the point of  whose orthogonal projection onto  is . By convexity, we have  and . Because , we also have . Thus, we obtain

and therefore , which implies that , as desired.
\end{proof}


{\SmoothAreaStmt*}
\begin{proof}
Recall that . The proof is based on two assertions: 
\begin{enumerate}
\setlength{\itemsep}{-0.5ex}\setlength{\parsep}{0pt}\item[(1)]  contains a -dimensional Euclidean ball of radius .

\item[(2)]  (treating  and  as subsets of ).
\end{enumerate}
To see why this suffices, observe that by (1), if we scale  by a factor of  about the center of this ball, the scaled copy contains . (To see why, observe that each supporting hyperplane of the original body is at distance at least  from the ball's center, and so the scaled body has a parallel supporting hyperplane at distance at least  from the original supporting hyperplane.) Scaling increases 's area by a factor of . By (2),  is contained within this scaled copy, and therefore its area cannot be larger. The result follows by setting .

We first prove assertion (1). Our approach will be to show that  contains a ball that is sufficiently large and sufficiently close to  that it contributes a ball of the desired radius to . Since  is -fat, there exist two concentric balls,  and , whose radii differ by a factor of  such that  (see Figure~\ref{fig:smooth-area-1}(a)). Let  and  denote the radii of  and , respectively, and let  denote the distance of their common center to .  is the natural candidate for the desired ball, but it may be too far from  to contribute to  (due to restriction). Since ,  does not lie entirely within a ball of radius  centered at . Let us scale space uniformly about  so that  just barely fits within this ball. Call the scaled body , and let  and  denote the scaled copies of  and , respectively (see Figure~\ref{fig:smooth-area-1}(b)). We will show that  is the desired ball. 

\begin{figure}[htbp]
  \centerline{\includegraphics[scale=0.40]{Figs/smooth-area-1}}
  \caption{Proof of assertion (1) of Lemma~\ref{lem:smooth-area}.}
  \label{fig:smooth-area-1}
\end{figure}


Since the scale factor is not greater than unity, . Let  denote the radius of , and let   denote its center's distance to . Because the center of  lies within , we have

which implies that . Because the scaling is uniform, it follows that . Another consequence of scaling is that

which implies that . Let  denote the -dimensional horizontal diametrical disk of radius  that lies within  (see Figure~\ref{fig:smooth-area-1}(b)). It is easy to verify that if we project  centrally towards  onto , the result is a -dimensional ball of radius , which we denote by . Clearly  lies within the unrestricted dual-cap base. As a result of scaling, both  and  lie within a ball of radius  centered at . Since , we have , and therefore  lies within the restricted base, that is . By the above inequalities, the radius of  is

which establishes assertion~(1).

Next, we prove assertion~(2). Consider any point . By definition of  (and thinking of  as a vector in ),  is naturally associated with a point  by shooting a ray from  through  until it hits  (see Figure~\ref{fig:smooth-area-2}). Since , there exists a unique point  that is at distance  from . Let  denote the vector . Similarly,  is associated with a point  located where the line segment  intersects . (The point  lies on the base of the unrestricted dual cap , but not necessarily on the restricted base .) 

Now, thinking of  and  as vectors in , we claim (i) that there exists a scalar  such that  lies within distance  of , and (ii)  is of length at most . From (i) and the fact that  is star-shaped with respect to the origin (namely ) it follows that . From (ii), we have . From these two claims we conclude that each point  is within distance  of a point in , which implies assertion~(2). The remainder of the proof involves proving these two claims.

\begin{figure}[htbp]
  \centerline{\includegraphics[scale=0.40]{Figs/smooth-area-2}}
  \caption{Proof of assertion (2) of Lemma~\ref{lem:smooth-area}.}
  \label{fig:smooth-area-2}
\end{figure}


To establish claim~(i), let us consider a coordinate frame whose origin is , whose th coordinate axis points vertically upwards, and whose other  coordinate vectors are taken from 's coordinate frame. We may express any point  as a pair , where  and  is the vertical distance from  to . Define the transformation  that projects a point  centrally towards  until it intersects . Also define an analogous transformation  that projects a point  onto . It is easy to verify that

Since  and by definition of , we have

Since , we have , and since , we have . It follows that 

Therefore, for some scalars  and , we have

Since  and , it follows that  lies within distance  of , which establishes claim~(i). 

To establish claim~(ii), observe that because  it lies within distance  of the origin on  (namely, ). Therefore,  lies within distance  of the origin on  (namely, ). Since  and , it is easy to verify that . Since  is the origin, this implies that  which establishes claim~(ii) and completes the proof.
\end{proof}


{\SmoothToPolarStmt*}
\begin{proof}
We begin by showing that  contains a -dimensional Euclidean ball (centered at ) of radius . To see this, observe that because ,  contains a ball  of radius  that has  on its boundary (see Figure~\ref{fig:dual-2}(a)). By similar triangles it is easy to show that the hyperplane  intersects the ``ice cream cone'' shaped structure  in a -dimensional Euclidean ball of radius .  This ball clearly lies within the unrestricted dual base, and since , we have , and so this ball also lies within the restricted dual base, .

\begin{figure}[htbp]
  \centerline{\includegraphics[scale=0.40]{Figs/dual-2}}
  \caption{Proof of Lemma~\ref{lem:smooth-to-polar}.}
  \label{fig:dual-2}
\end{figure}


For any , recall that  is the hyperplane that is parallel to  and at distance  below . Let  denote the hyperplane that is at unit distance below . Let  denote the vertical projection of  onto . Recall that we endow  and  with parallel coordinate frames with origins at  and , respectively. As a consequence of the above observation, for each vector  of length  in , there is a halfspace bounding  that is orthogonal to  and lies at distance  from the origin. Thus,  (when viewed as a subset of ) is contained within a -dimensional unit ball centered at  (see Figure~\ref{fig:dual-2}(b)).

Let  denote the semi-infinite generalized cylinder whose horizontal cross section is , whose upper surface lies on , and which extends vertically downwards  (see Figure~\ref{fig:dual-2}(b)). Lemma~\ref{lem:dual-subset} (applied now to , ,  and ) implies that . Since this applies not only to  but to any hyperplane lying below , it follows that .

We will show that the orthogonal projection of  onto  is equal to . It suffices to show that the base of  (which lies on ) lies entirely within . We can express any point  on 's base as the sum of two perpendicular vectors , where  is horizontal and  is vertical. Since  lies within unit distance of the origin,  lies below  at distance , and  lies at unit distance below , we have . As observed above, every point of  lies within unit distance of , and since  lies within unit distance of the origin, we have . Therefore, the squared distance from  to the origin is

which implies that  lies within the sphere  of radius . Therefore, 's base lies entirely within .

Since , and since the area of the orthogonal projection of a set cannot be larger than the area of the original set, we have

as desired.
\end{proof}
\end{document}
