\documentclass[12pt]{article}
\usepackage{amsmath}
\usepackage{amssymb}
\usepackage{dsfont}
\usepackage[ruled]{algorithm2e}
\usepackage{fullpage}
\usepackage{pstricks}
\usepackage{amsthm}
\def\abs#1{{|\,#1\,|}}
\def\mirror{{\operatorname{Mir}}}
\def\id{{\operatorname{Idt}}}
\def\tr{{\operatorname{Trn}}}
\def\mtt#1{{\tt #1}}
\newtheorem{theorem}{Theorem}
\newtheorem{proposition}[theorem]{Proposition}
\newtheorem{example}[theorem]{Example}
\newtheorem{lemma}[theorem]{Lemma}


\title{Pseudo-Power Avoidance}
\author{Ehsan Chiniforooshan \and Lila Kari \and Zhi Xu}
\date{The University of Western Ontario, \\
Department of Computer Science, \\
Middlesex College, London, Ontario, Canada, N6A 5B7 \\
{\tt \{ehsan,lila,zhi\underline{ }xu\}@csd.uwo.ca} \\
\medskip \today}
\begin{document}
\maketitle


\begin{abstract}
Repetition avoidance has been studied since Thue's work. In this
paper, we considered another type of repetition, which is called
pseudo-power. This concept is inspired by Watson-Crick
complementarity in DNA sequence and is defined over an antimorphic
involution . We first classify the alphabet  and the
antimorphic involution , under which there exists sufficiently
long pseudo-th-power-free words. Then we present algorithms to
test whether a finite word  is pseudo-th-power-free.
\end{abstract}


\section{Introduction}\label{section:introduction}
Let  be an alphabet. The set of finite words and infinite
words over  are denoted by  and ,
respectively. Word  is called a \emph{factor} of  if 
for some words  and . A nonempty word  is called a
\emph{square} if  can be written as  for some
, and is called a \emph{cube} if  can be written as
 for some . For example, the English word
``murmur'' is a square. More generally, for an integer , a
nonempty word  is called a \emph{th power} if  for some
. A word  is called \emph{square-free}
(respectively, \emph{cube-free}, \emph{th-power-free}) if 
does not contain any square (respectively, cube, th power) as a
factor. In the early 1900's, Thue showed in a series of papers
examples of square-free infinite words over  letters and 
letters respectively and cube-free infinite word over binary
alphabet \cite{Thue1906,Thue1912} (see \cite{Berstel1995} for
English translation of Thue's work). In 1921, Morse \cite{Morse1921}
independently discovered Thue's construction. In 1957, Leech
\cite{Leech1957} showed another construction of square-free infinite
word, which is generated by morphism. In 1995, Yu \cite{Yu1995}
showed square-free infinite words that cannot be generated by any
morphism.


Let  be the total number of letter  that appear in the
word , and let  for
. A nonempty word  is called an \emph{abelian
square} if  is in the form  such that
 for each letter . For example, the
English word ``teammate'' is a abelian square. Analogously,
 is called an \emph{abelian cube} if , where
 for each , and
called an \emph{abelian th power} if , where
integer ,  for  and
. A word  is called \emph{abelian-square-free}
(respectively, \emph{abelian-cube-free},
\emph{abelian-th-power-free}) if  contains no abelian square
(respectively, abelian cube, abelian th power) as a factor. In
1957, Erd\"os \cite{Erdos1957} asked whether there exists an
abelian-square-free infinite word. The construction of such words
were given by Pleasants \cite{Pleasants1970} in 1970 over 
letters and by Ker\"anen \cite{Keranen1992} in 1992 over 
letters. Most recently, Ker\"anen \cite{Keranen2009} presented a
great many new abelian-square-free infinite words. In 1979, Dekking
\cite{Dekking1979} discussed abelian-th-power-free infinite words
for .


The discussion on th power is related to biology. Repeats of
certain segments in human DNA sequences may predict certain disease
\cite{Mirkin2007}. A variation on the th power, called pseudo
th power, is defined by antimorphic involutions, which are
generalizations of the famous Watson-Crick complementarity function
in biology. Other concepts in combinatorics on words have been
generalized in the setting of antimorphic involutions, such as
pseudo-primitivity \cite{Czeizler&Kari&Seki2009} and
pseudo-palindrome \cite{Kari&Mahalingam2009}. A variation on the
concept of the pseudo th power has also appeared in tiling
problems (see \cite{Beauquier&Nivat1991,Braquelaire&Vialard1999}),
where the function involved is a morphism other than an
antimorphism.


In this paper, we will discuss the word that does not contain any
pseudo th power as a factor. Here the pseudo power is defined by
antimorphic involutions. In Section~\ref{section:freeword}, we will
introduce the concept of pseudo-power-free words and discuss the
existence of such words over different setting of alphabet and
antimorphic involutions. In Section~\ref{section:decision}, we will
discuss the algorithms for the decision problem of
pseudo-power-freeness. At the end, we will summarize the results and
present open problems.



\section{Pseudo-power-free infinite words}\label{section:freeword}
Without loss of generality, in the following discussion we always
assume the letters are . The empty word is
denoted by . A function  is
called an \emph{involution} if  for all
, and called an \emph{antimorphism} (respectively,
\emph{morphism}) if  (respectively,
). We call  an
\emph{antimorphic involution} if  is both an involution and
an antimorphism. For example, the classic Watson-Crick
complementarity in biology is an antimorphic involution over four
letters. The \emph{mirror image} is the function of reversing a word
as defined by . Then the
mirror image over any alphabet is also an antimorphic involution. A
\emph{transposition}  is a morphism that is defined by
. One can verify that the mirror image and any
transposition commute, and two transpositions  and
 commute for . By definitions,
every antimorphic involution on the alphabet is a permutation of the
letters. Furthermore, we have the following proposition.


\begin{proposition}\label{prop:decomposition}
Let  be an antimorphic involution over the alphabet
. Then  can be uniquely written as the composition
of transpositions with a mirror image

up to changing the order of composition, where  for
 and .
\end{proposition}
\begin{proof}
First we show that  is a single letter for any letter
. By definitions, , so . For
any , since , we have
 and . Hence
 for any .

Now we prove the existence of the decomposition in
Eq.~(\ref{equationdecomp}) for  by induction on the size of
the alphabet. If , then  and
. So  and
Eq.~(\ref{equationdecomp}) holds. If , then either
 or . One can verify that
 when  and
 when .
Suppose Eq.~(\ref{equationdecomp}) holds for any .
For , either  or
 for some . If
, then by induction hypothesis the restriction
of  on alphabet  can be written as
. So
 in this
case. If , then . By induction
hypothesis, the restriction of  on alphabet
 can be written as
. So
.
Therefore, Eq.~(\ref{equationdecomp}) holds.

To show the uniqueness of the form in Eq.~(\ref{equationdecomp}), we
first notice that every transposition is the inverse of itself.
Assume there is another decomposition
,
where  for  and . Then
 and thus there is some .
Since  for , the order of composition in
Eq.~(\ref{equationdecomp}) can be arbitrarily changed. Hence we have

Continuing this procedure, it follows that
. By
the construction of , however, we know  for
. So , since otherwise .
Therefore,  and
.
\end{proof}


For any antimorphic involution  over , we define
  
The following proposition follows directly from
Proposition~\ref{prop:decomposition}.


\begin{proposition}\label{prop:id+2tr=absigma}
Let  be an antimorphic involution over the alphabet
. Then  can be written as the composition of
 distinct transpositions with a mirror image, and

\end{proposition}
\begin{proof}
By the proof of Proposition~\ref{prop:decomposition},  can
be written as
,
where  for . Then
 and
. So 
and Eq.~(\ref{equationid2tr}) holds.
\end{proof}


For integer  and antimorphism , we call word  a
\emph{pseudo th power} (with respect to ) if  can be
written as , where either  or
 for . In particular, we call
pseudo nd power by \emph{pseudo square}, and pseudo rd power
by \emph{pseudo cube}. For example, over the alphabet  (here we use the conventional symbols
instead of ) with respect to the
Watson-Crick complementarity (),  is a pseudo square and  is a pseudo cube. By definitions, every
pseudo th power is an abelian th power with respect to the
mirror image, and every th power is a pseudo th power (with
respect to any antimorphic involution on the same alphabet). A word
 is called \emph{pseudo-th-power-free} (respectively,
\emph{pseudo-square-free}, \emph{pseudo-cube-free}) if  cannot be
written as  where  is a pseudo th power (respectively,
pseudo square, pseudo cube).


In the remaining of this section, we will discuss the following
problem: is there a pseudo-th-power-free infinite word over
 with respect to ? The discussion on
pseudo-power-free words is related to power-free words and
abelian-power-free words in the sense of the following lemmas.


\begin{lemma}\label{lemma:lemma0}
If  is the minimal size of alphabet over which there is a
th-power-free infinite word, then
\begin{itemize}
  \item[(1)] there is no pseudo-th-power-free infinite word over  or less letters;
  and
  \item[(2)] there is a pseudo-th-power-free infinite word over  letters
  with respect to , where .
\end{itemize}
\end{lemma}
\begin{proof}
(1) If  is the minimal size of alphabet over which there is a
th-power-free infinite word, then for any  there is an
integer  such that any word of length greater than  over 
letters contains a th-power. Since a th-power is a
pseudo-th-power (with respect to any antimorphic involution), any
word of length greater than  over  letters contains a
pseudo-th-power with respect to .

(2) Let  be an antimorphic involution on  letters such
that . We choose
 such that . Then
there is an infinite word  over  such that  is
th-power-free. Now we claim that  is also
pseudo-th-power-free over  letters with respect to .
Suppose  contains a pseudo-th-power. Then , where either  or  for . For any , by definition,  and
. So  and
thus  is not a word over  for all . Hence  for  and  is
a normal th-power, which contradicts the fact that  is
th-power-free. Therefore,  is pseudo-th-power-free with
respect to .
\end{proof}


\begin{lemma}\label{lemma:lemma1}
If there is an abelian-th-power-free infinite word over 
letters, then there is a pseudo-th-power-free infinite word over
 letters with respect to arbitrary antimorphic involution
.
\end{lemma}
\begin{proof}
By Eq.~(\ref{equationid2tr}) in
Proposition~\ref{prop:id+2tr=absigma}, it follows that
 and thus
. We choose
 such that
. Then there is an infinite word  over 
such that  is abelian-th-power-free. Now we claim that  is
also pseudo-th-power-free over  letters with respect to
.

Suppose  contains a pseudo-th-power, where
either  or  for . Then
either  is a word over  or  contains at least
one letter from . If , then
. So  is an abelian
th power, which contradicts the fact that  is
abelian-th-power-free. Otherwise,  contains at least one
letter from , say . Then since  and
, we have
. So
 for , and thus  contains a th power,
which again contradicts the fact that  is
abelian-th-power-free.

Therefore,  is pseudo-th-power-free with respect to .
\end{proof}


\begin{lemma}\label{lemma:lemma2}
If there is a pseudo-th-power-free infinite word over 
with respect to , then there is a pseudo-th-power-free
infinite word over  with respect to , where
 and
.
\end{lemma}
\begin{proof}
We choose  such that
. Since
,
we can choose

such that . Define
, and
define antimorphic involution  by  for . Then
,
 and
. So  and
 are identical up to renaming of the letters. There is a
word  over  such that  is pseudo-th-power-free
with respect to . We claim  is also
pseudo-th-power-free over  with respect to .

Suppose  contains a pseudo-th-power, where
either  or  for . Then
either  is a word over  or  contains at
least one letter from .
In the former case,  and thus 
contains a pseudo-th-power with respect to , which
contradicts to the pseudo-th-power-freeness of . In the latter
case, we assume  contains
. One can verify
that , so  for all
. Hence  contains a th-power , which again contradicts the pseudo-th-power-freeness of
.

Therefore,  is pseudo-th-power-free over  with
respect to .
\end{proof}


\subsection{Pseudo-square-free infinite words}
First we consider pseudo-square-free infinite words. Since every
binary word of length greater than  contains squares, there is no
square-free infinite word over  letters. By Ker\"anen's
construction of abelian-square-free infinite words, there exist
pseudo-square-free infinite words over  letters with respect to
the mirror image. Furthermore, we have the following result.


\begin{proposition}\label{prop:pseudoleech}
For -letter-alphabet, pseudo-square-free infinite word exists
with respect to mirror image and does not exist with any other
antimorphic involution.
\end{proposition}
\begin{proof}
There are two kinds of antimorphic involutions over  letters:
 is either the mirror image or a transposition composed with
the mirror image.

Suppose  is the mirror image. Then the following morphism
 given by Leech \cite{Leech1957} preserves square-freeness and
presents an infinite word , which is
pseudo-square-free with respect to the mirror image:

To see  contains no pseudo square, first we
observe that  contains no square. If  contains pseudo square
of the form  such that , then
 contains a square of length  in the middle.
Since  is square free,  does not contain pseudo square.

Now suppose  is a transposition composed with the mirror
image. Without loss of generality, we assume
. We prove no pseudo-square-free
infinite word exists in this setting. Suppose  is a
pseudo-square-free infinite word. A pseudo-square-free word (with
respect to ) cannot contains . So  is either in
 or in
. If we omit all symbol  in
, the new infinite word should also be pseudo-square-free. But
the new word is over  and every binary infinite
word contains pseudo squares, which is a contradiction. So there is
no pseudo-square-free infinite word over  letters (with respect
to a transposition composed with the mirror image).
\end{proof}


There is another way to show the non-existence of pseudo-square-free
infinite word over  letters with respect to . We
use computer to find the longest pseudo-square-free word, if any.
Starting from empty word , if a word is
pseudo-square-free, then we extend the word by adding a new letter
 at the end; otherwise, we do back-tracking and try the next
letter. In other words, we do a depth-first-search in a labeled
tree, where each node presents a finite word. Such tree is called
\emph{trie} and application of similar technique has been appeared
in the literatures (for example, see \cite{Shallit2004}). In the
case of  letters and , the tree
is finite and all pseudo-square-free words are enumerated. There are
in total  nodes, including  leaves. The tree is of depth 
and one of the longest pseudo-square-free words is
.


\begin{theorem}
There is no pseudo-square-free infinite word over  letters for
, and there is a pseudo-square-free infinite word over 
letters for , with respect to arbitrary antimorphic
involution. For , the existence of pseudo-square-free
infinite words depends on the antimorphic involution.
\end{theorem}
\begin{proof}
(1) Since  is the minimal size of alphabet over which there is a
square-free infinite word, by Lemma~\ref{lemma:lemma0}, there is no
pseudo-square-free infinite word over  letters for  and
there is pseudo-square-free infinite word with respect to 
for .

(2) Since there exists an abelian-square-free infinite word over 
letters, by Lemma~\ref{lemma:lemma1}, there is a pseudo-square-free
infinite word over  or more letters.

(3) By Proposition~\ref{prop:pseudoleech}, over  letters, there
is a pseudo-square-free infinite word with respect to mirror image,
where , and there is no
pseudo-square-free infinite word with respect to other antimorphic
involution. So by Lemma~\ref{lemma:lemma2}, there is a
pseudo-square-free infinite word with respect to  such that
.

The result is summarized in Table~\ref{table:pseudosquare}, where
the subscription presents which situation of the (1), (2), (3) the
case falls in.
\end{proof}

\begin{table}
\centering \caption{The existence of pseudo-square-free infinite
words}
\begin{tabular}{c|p{5.5ex}p{5.5ex}p{5.5ex}p{5.5ex}p{5.5ex}p{5.5ex}p{5.5ex}p{5.5ex}}
  \hline
 &  &  &  &  &  &  &  &  \\
  \hline
   &  &  &  &  &  &  &  &  \\
   &  &  &  &  &  &  &  &  \\
   &  &  &  & {\small open} &  &  &  &  \\
   &  &  &  &  &  &  &  &  \\
   &  &  &  &  &  &  &  &  \\
  \hline
\end{tabular}\label{table:pseudosquare}
\end{table}


It is still a open problem that whether there exists a
pseudo-square-free infinite word over  letters with respect to
antimorphic involution , where . Experimental
computation shows that there are long pseudo-square-free words in
the setting, but we don't have proof of the existence of arbitrarily
long pseudo-square-free words. But if there are, they must satisfy
certain conditions as in the following proposition.


\begin{proposition}
 is an pseudo-square-free infinite word over  letters with
respect to , where
, if and
only if  and  is
square free.
\end{proposition}
\begin{proof}
``''.
Since  is pseudo-square-free,  is square free. Furthermore,
any word in  is a pseudo square, so the letters in
 must appear alternatively from  and . Hence
.

``''.
Suppose  contains a pseudo square. Since  is square free, it
must be the case that . By the definition of
antimorphism involution, the last letter of  and the first letter
of  are either both from  or both from ,
which contradicts the fact .
\end{proof}


\subsection{Other pseudo-power-free infinite words}
Now we consider pseudo-cube-free infinite words. By Dekking's
construction of abelian-cube-free infinite words, there exist
pseudo-cube-free infinite words over  letters with respect to the
mirror image. The case over unary alphabet is trivial. For binary
alphabet, we have the following result.


\begin{proposition}\label{prop:pseudocube}
Pseudo-cube-free infinite word does not exist over binary alphabet
with any antimorphic involution.
\end{proposition}
\begin{proof}
There are two kinds of antimorphic involutions over binary alphabet:
we have either  or
.

Suppose . Again, we use computer to find the longest
pseudo-cube-free word, if any. Starting from empty word ,
if a word is pseudo-cube-free, then we extend the word by appending
; otherwise, we do back-tracking and try the next letter. The
resulted depth-first-search tree is finite. There are in total 
nodes, including  leaves. The tree is of depth  and one of
the longest pseudo-cube-free words is
.

Suppose . Similarly, we verified
by computer that there are only finitely many pseudo-cube-free
words. There are in total  nodes, including  leaves. The tree
is of depth  and one of the longest pseudo-cube-free words is
. In fact, any word in this setting is a pseudo power.
\end{proof}


\begin{proposition}\label{prop:pseudodekking}
There is a pseudo-cube-free infinite word over  letters with any
antimorphic involution.
\end{proposition}
\begin{proof}
There are two kinds of antimorphic involutions over  letters:
 is either the mirror image or a transposition composed with
the mirror image.

Suppose  is the mirror image. The following morphism 
given by Dekking \cite{Dekking1979} presents an abelian-cube-free
infinite word  over  letters, which is also
pseudo-cube-free:

So there is a pseudo-cube-free infinite word 
over  letters with respect to mirror image.

Now suppose  is a transposition composed with the mirror
image. Without loss of generality, we assume
. Consider the following morphism:

One can verify that the word

is the Thue-Morse sequence \cite{Thue1912} with letter 
inserted between every two consecutive letters. Now we prove that
 is pseudo-cube-free. Suppose  contains a
pseudo-cube  with . Either
the last letter of  or the first letter of  is ,
but not both. Since , we have
. So . By the same reason, .
Then the length of  must be even. Otherwise, either the
first letter of  or the first letter of  is , but
not both, and thus we have . Now since
 is even, we can omit the letter
 from each word and get new words  such
that  and  is a factor of the
Thue-Morse sequence, which contradicts the fact that Thue-Morse
sequence is cube-free. Therefore,  is
pseudo-cube-free with respect to  over
 letters.
\end{proof}


\begin{theorem}
There is no pseudo-cube-free infinite word over  letters for
, and there is a pseudo-cube-free infinite word over 
letters for , with respect to arbitrary antimorphic
involution.
\end{theorem}
\begin{proof}
(1) Since  is the minimal size of alphabet over which there is a
cube-free infinite word, by Lemma~\ref{lemma:lemma0}, there is no
pseudo-cube-free infinite word over  letters for  and
there is pseudo-cube-free infinite word with respect to  for
.

(2) There is an abelian-cube-free infinite word 
over  letters. By Lemma~\ref{lemma:lemma1}, there is a
pseudo-cube-free infinite word over  or more letters.

(3) By Proposition~\ref{prop:pseudocube}, there is no
pseudo-cube-free infinite word over binary alphabet. By
Proposition~\ref{prop:pseudodekking}, there is a pseudo-cube-free
infinite word over  letters. In particular, there is a
pseudo-cube-free infinite word over  letters with respect to
mirror image, where . So
by Lemma~\ref{lemma:lemma2}, there is a pseudo-cube-free infinite
word with respect to  such that
.

The result is summarized in Table~\ref{table:pseudocube}, where the
subscription presents which situation of the (1), (2), (3) the case
falls in.
\end{proof}

\begin{table}
\centering \caption{The existence of pseudo-cube-free infinite
words}
\begin{tabular}{c|p{5.5ex}p{5.5ex}p{5.5ex}p{5.5ex}p{5.5ex}p{5.5ex}p{5.5ex}p{5.5ex}}
  \hline
 &  &  &  &  &  &  &  &  \\
  \hline
   &  &  &  &  &  &  &  &  \\
   &  &  &  &  &  &  &  &  \\
   &  &  &  &  &  &  &  &  \\
   &  &  &  &  &  &  &  &  \\
   &  &  &  &  &  &  &  &  \\
  \hline
\end{tabular}\label{table:pseudocube}
\end{table}


We now discuss other pseudo-th-power-free infinite words with
. Every word over a single letter is a power. So the unary
case is trivial and no X-free infinite word exists for X being
either th-power, or abelian-th-power, or pseudo-th-power.


\begin{theorem}
For any integer , there is no pseudo-th-power-free
infinite word over  letters for , and there is a
pseudo-th-power-free infinite word over  letters for , with respect to arbitrary antimorphic involution. For binary
alphabet, the existence of pseudo-th-power-free infinite words
depends on the antimorphic involution.
\end{theorem}
\begin{proof}
(1) There is no pseudo-power-free infinite words over unary
alphabet. Since there is a th-power-free infinite word over
binary alphabet, by Lemma~\ref{lemma:lemma0}, there is a
pseudo-th-power-free infinite word with respect to  for
.

(2) There exists abelian-th-power-free infinite word over binary
alphabet, such as the following construction by Dekking
\cite{Dekking1979}  where

So there exists an abelian-th-power-free infinite word  over
binary alphabet for any integer . By
Lemma~\ref{lemma:lemma1}, there is a pseudo-th-power-free
infinite word over  or more letters.

(3) That infinite word  is also a
pseudo-th-power-free infinite word for  over binary
alphabet with respect to the mirror image, where
. So by
Lemma~\ref{lemma:lemma2}, there is a pseudo-th-power-free
infinite word for  with respect to  such that
. If
 over binary alphabet, then any
word is a pseudo power.

The result is summarized in Table~\ref{table:pseudoforth}, where the
subscription presents which situation of the (1), (2), (3) the case
falls in.
\end{proof}

\begin{table}
\centering \caption{The existence of pseudo-th-power-free
infinite words for integer }
\begin{tabular}{c|p{5.5ex}p{5.5ex}p{5.5ex}p{5.5ex}p{5.5ex}p{5.5ex}p{5.5ex}p{5.5ex}}
  \hline
 &  &  &  &  &  &  &  &  \\
  \hline
   &  &  &  &  &  &  &  &  \\
   &  &  &  &  &  &  &  &  \\
   &  &  &  &  &  &  &  &  \\
   &  &  &  &  &  &  &  &  \\
   &  &  &  &  &  &  &  &  \\
  \hline
\end{tabular}\label{table:pseudoforth}
\end{table}



\section{Testing pseudo-power-freeness of words}\label{section:decision}
In this section, we will discuss the following problem: given a
finite word  and integer , does  contain any
pseudo-th-power as a factor? First, we will discuss the general
algorithm for arbitrary .


\subsection{General algorithm for arbitrary th pseudo-power}
The na\"\i ve algorithm runs in  time to decide whether 
contains any pseudo-th-power as a factor. The idea is that we
check each possible candidate factors  of  to see whether 
is a pseudo-th-power. There are  factors and check
whether a word is a pseudo th power can be done with 
comparisons of letters.


    Here we describe an -time algorithm to decide weather an input string  of length 
    contains a -th pseudo-power of a word or not.
Our algorithm has three steps: in the first step, it constructs an  zero-one matrix  such that
    
Then, using , the algorithm constructs a set of binary strings
    
Having 's, it is easy to find a pseudo-th-power, if there exists any.
    \begin{lemma}\label{lem:two-to-k}
      Given  and  as inputs, there is an algorithm with
      time linear to  that finds {\em all} pseudo-th-powers in .
    \end{lemma}
\begin{proof}
      In linear time, for all , we will break the string  into  strings  such that  consists of the characters at positions  in
      .
This can be done trivially in linear-time.

      Now, observe that there is a pseudo-th-power in  starting at position  of length
       if and
      only if  has  consecutive s starting at position
      .
    \hfill

    Our method is summarized in Algorithm~\ref{alg:main}.

    \begin{algorithm}
      \label{alg:main}
      \caption{\textsc{Pseudo-Power-Freeness}}
      Initial  to a zero matrix\;
      \For{all  and  such that } {
        \lIf{} {}\;
        \lIf{} {}\;
      }
      \SetVline
      \nl\For{}{
        \SetNoline
        \For{all  and  such that } {
          \lIf{ and } {}\;
          \lIf{ and } {}\;
        }
      }
      \SetVline
      \nl\For{all  such that } {
        \SetNoline
        \;
        \For{}{
          Let  be distinct integers such that \;
          \lIf{
            for all }{\;}
          \lIf{
            for all }{\;}
        }
      }
      \SetVline
      \nl\For{all  such that } {
        \SetNoline
        Break  into  strings  as described in
        Lemma~\ref{lem:two-to-k}\;
        \lIf{there exists  such that  contains  consecutive s}
        {\Return NO\;}
      }

      \Return YES\;
    \end{algorithm}

  \begin{theorem}
    Algorithm~\ref{alg:main} runs in time  and returns YES if and only if  does not have any pseudo-th-power as
    a substring.
  \end{theorem}
  {\noindent\bf Proof.}
    The running time of block 1 and 2 is  and block 3 runs in  as explained in
    Lemma~\ref{lem:two-to-k}.

    As for the correctness, it is enough to show that, after block 1 and 2, the matrix  and the set of strings
     have the values that the are supposed to have; i.e.
    
    and
    
    To prove that  holds, we use induction on .
For   holds because of the initialization before block 1.
Assuming that  holds for , it is easy to see that  holds for :
     if and only if  and .

    Proving  is similar to proving .

    For , note that
    \begin{enumerate}
      \item  is a pseudo-square if and only if  or .

      \item  if and only if
         for all
        , where  are distinct integers such that .

      \item  if and only if
         for all
        , where  are distinct integers such that
        .\qedhere
    \end{enumerate}
\end{proof}


In the following subsection, we consider, for fixed small ,
whether a given word  is pseudo-th-power-free.


\subsection{Testing pseudo-square-freeness}
\begin{theorem}
To decide whether a word  contains a pseudo-square as a factor
can be done in linear time.
\end{theorem}
\begin{proof}
Let . A word  contains a pseudo-square if and only if
 contains a square or a word of the form .

To check whether  contains a square can be done in linear time.
There are a few works in the literatures on testing square-freeness
in linear time \cite{Crochemore1983,Main&Lorentz1985}.

To check whether  contains a word of the form , it is
enough to check whether  contains a word  for a letter
. To see this, if w contains , then let  be the
right-most letter of  and  contains ; for the other
direction, word  itself is a pseudo-square.

\begin{algorithm}
  \SetLine \SetKw{KwFrom}{from} \SetKw{KwBreak}{break}  \SetKw{KwContinue}{continue}
  \linesnumbered
  \KwIn{a word .}
  \KwOut{``YES'' if  is pseudo-square-free; ``NO'' otherwise.}
  \lIf{ contains a square}{\Return{``NO''}}\;
  \For{ \KwFrom  \KwTo }{
    \lIf{}{\Return{``NO''}}\;
  }
  \Return{``YES''}\;
  \caption{Decide whether  is pseudo-square-free in linear time}
  \label{figure:square}
\end{algorithm}

The algorithm is illustrated in Algorithm~\ref{figure:square}. It is
obvious that the algorithm is linear.
\end{proof}


\subsection{Testing pseudo-cube-freeness}
Before we show a cubic time algorithm for the pseudo-cube-freeness
of a word, we first introduce some concepts. Let  be
a finite word over  and let  be an antimorphic
involution with the same alphabet . A \emph{right minimal
periodic}  of  is a vector and is defined by
  
and similarly a \emph{left minimal periodic}  of
 is defined by
  
For example, when , we have
 and
. A
\emph{centralized maximal pseudo-palindrome}  of
 (with respect to ) is a vector and is defined by
  
For example, when  and , we have
. The left-most and right-most elements
of  are always .


\begin{lemma}\label{lemma:rmplmp}
For any fixed integer , the right (respectively, left) minimal
periodic  (respectively, ) of word  can be computed
in linear time .
\end{lemma}
There is an algorithm to compute , the shortest square
starting at each position, in linear time \cite{Kosaraju1994} by
using suffix tree. Since vector  can be obtained by first
computing  and then reversing , vector
 can also be computed in linear time.


\begin{lemma}\label{lemma:cmp}
The centralized maximal pseudo-palindrome  can be computed in
linear time .
\end{lemma}
Lemma~\ref{lemma:cmp} has been proved in the book
\cite[page~197--198]{Gusfield1997}, which claimed all the maximal
even-length palindromes can be found in linear time.


Now we are ready to show a quadratic time algorithm to test the
pseudo-cube-freeness of a given word  of length . By
definition, a pseudo-cube is in one of the following form ,
, , and . In order to check whether
 contains any pseudo-cube, we check each of the four cases.


To check whether  contains any word of the form  can be done
in linear time. Word  contains a cube if and only if one of the
maximal repetition in  has exponent , and there is linear
algorithm \cite{Kolpakov&Kucherov1999} to find all the maximal
repetitions. So this case can be checked in  time.


To check whether  contains any word of the form , we
check whether there is a pair of factors  and
 that overlap in the sense that . By the definitions of  and , we only need to check
for each position  whether . this can be
done in  time when all  are already computed. The
case for  is similar.


To check whether  contains any word of the form , we
check whether there is a pair of factors
 and  that
overlap in the sense that  and .
By the definition of , we check for each pair of indices
 with  whether  and .
This can be done in  time when  is already known.


\begin{algorithm}
  \SetLine \SetKw{KwFrom}{from}
  \linesnumbered
  \KwIn{a word .}
  \KwOut{``YES'' if  is pseudo-cube-free; ``NO'' otherwise.}
  compute , , \;
  \lIf(\tcp{The case }){ contains a cube}{\Return{``NO''}}
  \For{ \KwFrom  \KwTo }{
    \lIf(\tcp{The case }){}{\Return{``NO''}}
    \lIf(\tcp{The case }){}{\Return{``NO''}}
    \For{ \KwFrom  \KwTo }{
      \lIf(\tcp{The case }){}{\Return{``NO''}}}
  }
  \Return{``YES''}\;
  \caption{Decide whether  is pseudo-cube-free in linear time}
  \label{figure:cube}
\end{algorithm}
The completed algorithm is given in Algorithm~\ref{figure:cube}. So
we have the following theorem.
\begin{theorem}
To decide whether a word  contains a pseudo-cube as a factor can
be done in quadratic time.
\end{theorem}
\begin{proof}
Let . Algorithm~\ref{figure:cube} checks the
pseudo-cube-freeness of  in quadratic time. By
Lemma~\ref{lemma:rmplmp} and Lemma~\ref{lemma:cmp}, the computation
of  in line~1 can be done in  times.
Line~2 can be done in  time. Line~3--10 can be done in
 time. So the algorithm runs in  time.

Now we prove the correctness of the algorithm. First, we prove that
if the algorithm returns ``NO'', then  contains a pseudo cube. If
the algorithm return at line~2, then  contains a cube of the form
, which is also a pseudo cube. Suppose the algorithm return at
line~4. Let  and . Then  and the
word  is of the form , which is a
pseudo cube. Suppose the algorithm return at line~5. Let
 and . Then  and the word
 is of the form , which is a pseudo
cube. Suppose the algorithm return at line~7. Then the word
 is of the form , which is a pseudo
cube.

Now, we prove that if  contains a pseudo cube, then the algorithm
returns ``NO''. If  contains a pseudo cube of the form ,
then the algorithm returns at line~2. Suppose
. Then  and
. So the algorithm returns at line~5
for , (although the detected pseudo cube
 may be different from
.) The case  is
similar. Suppose . Then
 and . So the
algorithm returns at line~7 for  and .
\end{proof}



\section{Conclusion}\label{section:conclusion}
In this paper, we discussed the existence of infinite words that do
not contain pseudo-th-power. For alphabet size , there is
no pseudo-square-free infinite words and for alphabet size ,
there exist pseudo-square-free infinite words. For other alphabet
size, the existence of pseudo-square-free infinite words depends on
the antimorphic involution . For alphabet size , there
is no pseudo-cube-free infinite words and for alphabet size , there exist pseudo-cube-free infinite words. For alphabet size
, there exist pseudo-th-power-free infinite words for any
integer . For binary alphabet and any integer ,
the existence of pseudo-th-power-free infinite words depends on
the antimorphic involution .


We also discussed the algorithm for testing whether a given word 
is pseudo-th-power-free. For arbitrary th pseudo-power, there
is a -time algorithm to find all th pseudo-power in
, where . For , there is a -time algorithm
for testing pseudo-square-freeness of word  of length . For
, there is a -time algorithm for testing
pseudo-cube-freeness of word  of length .



\section*{Acknowledgement}
The authors wish to thank Dr. Shinnosuke Seki and Professor Lucian
Ilie for very helpful discussion and comments on the draft of this
paper.


\begin{thebibliography}{10}

\bibitem{Beauquier&Nivat1991}
D.~Beauquier and M.~Nivat.
\newblock On translating one polyomino to tile the plane.
\newblock {\em Discrete Comput. Geom.}, 6(1):575--592, 1991.

\bibitem{Berstel1995}
J.~Berstel.
\newblock {\em {Axel Thue}'s Papers on Repetitions in Words: a Translation}.
\newblock Number~20 in Publications du Laboratoire de Combinatoire et
  d'Informatique Math\'ematique. Universit\'e du Qu\'ebec \`a Montr\'eal, 1995.

\bibitem{Braquelaire&Vialard1999}
{J.-P.} Braquelaire and A.~Vialard.
\newblock Euclidean paths: A new representation of boundary of discrete
  regions.
\newblock {\em Graphical Models and Image Processing}, 61(1):16--43, 1999.

\bibitem{Crochemore1983}
M.~Crochemore.
\newblock Recherche lin\'eaire d'un carr\'e dans un mot.
\newblock {\em Comptes Rendus Acad. Sci. Paris S\'er. I}, 296:781--784, 1983.

\bibitem{Czeizler&Kari&Seki2009}
E.~Czeizler, L.~Kari, and S.~Seki.
\newblock On a special class of primitive words.
\newblock preprint, 2009.

\bibitem{Dekking1979}
F.~M. Dekking.
\newblock Strongly non-repetitive sequences and progression-free sets.
\newblock {\em J. Combin. Theory Ser. A}, 27(2):181--185, 1979.

\bibitem{Erdos1957}
P.~{Erd\"os}.
\newblock Some unsolved problems.
\newblock {\em Michigan Math. J.}, 4(3):291--300, 1957.

\bibitem{Gusfield1997}
D.~Gusfield.
\newblock {\em Algorithms on strings, trees, and sequences: computer science
  and computational biology}.
\newblock Cambridge University Press, 1997.

\bibitem{Kari&Mahalingam2009}
L.~Kari and K.~Mahalingam.
\newblock {Watson-Crick} palindromes in {DNA} computing.
\newblock {\em Nat. Comput.}, DOI:10.1007/s11047-009-9131-2, 2009.

\bibitem{Keranen1992}
V.~{Ker\"anen}.
\newblock Abelian squares are avoidable on 4 letters.
\newblock In W.~Kuich, editor, {\em "Proc. 19th Int'l Conf. on Automata,
  Languages, and Programming (ICALP '92)"}, pages 41--52. Springer-Verlag,
  1992.

\bibitem{Keranen2009}
V.~{Ker\"anen}.
\newblock A powerful abelian square-free substitution over 4 letters.
\newblock {\em Theoret. Comput. Sci.}, 410:38--40, 2009.

\bibitem{Kolpakov&Kucherov1999}
R.~Kolpakov and G.~Kucherov.
\newblock Finding maximal repetitions in a word in linear time.
\newblock In {\em "Proc. 40th Ann. Symp. Found. Comput. Sci. (FOCS '99)"},
  pages 596--604. IEEE Computer Society, 1999.

\bibitem{Kosaraju1994}
S.~R. Kosaraju.
\newblock Computation of squares in a string.
\newblock In M.~Crochemore and D.~Gusfield, editors, {\em "Proc. 5th
  Combinatorial Pattern Matching"}, pages 146--150. Springer Verlag, 1994.

\bibitem{Leech1957}
J.~Leech.
\newblock A problem on strings of beads.
\newblock {\em Math. Gaz.}, 41(338):277--278, 1957.

\bibitem{Main&Lorentz1985}
M.~Main and R.~Lorentz.
\newblock Linear time recognition of square free strings.
\newblock In A.~Apostolico and Z.~Galil, editors, {\em Combinatorial Algorithms
  on Words}, pages 272--278. Springer Verlag, 1985.

\bibitem{Mirkin2007}
S.~M. Mirkin.
\newblock Expandable dna repeats and human disease.
\newblock {\em Nature}, 447(7147):932--940, 2007.

\bibitem{Morse1921}
H.~M. Morse.
\newblock Recurrent geodesics on a surface of negative curvature.
\newblock {\em Trans. Amer. Math. Soc.}, 22(1):84--100, 1921.

\bibitem{Pleasants1970}
P.~A.~B. Pleasants.
\newblock Non-repetitive sequences.
\newblock {\em Math. Proc. Cambridge Philos. Soc.}, 68(2):267--274, 1970.

\bibitem{Shallit2004}
J.~Shallit.
\newblock Simultaneous avoidance of large squares and fractional powers in
  infinite binary words.
\newblock {\em Int'l. J. Found. Comput. Sci.}, 15:317--327, 2004.

\bibitem{Thue1906}
A.~Thue.
\newblock {\"U}ber unendliche {Zeichenreihen}.
\newblock {\em Norske Vid. Selsk. Skr. I. Mat.-Nat. Kl.}, (7):1--22, 1906.

\bibitem{Thue1912}
A.~Thue.
\newblock {\"U}ber die gegenseitige {Lage} gleicher {Teile} gewisser
  {Zeichenreihen}.
\newblock {\em Norske Vid. Selsk. Skr. I. Mat.-Nat. Kl.}, (1):1--67, 1912.

\bibitem{Yu1995}
X.~Yu.
\newblock A new solution for {Thue}'s problem.
\newblock {\em Inform. Process. Lett.}, 54:187--191, 1995.

\end{thebibliography}
\end{document}
