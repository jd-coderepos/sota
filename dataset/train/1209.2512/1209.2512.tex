\documentclass[11pt]{article}

\usepackage{latexsym,ifthen,epsfig}
\usepackage[latin1]{inputenc}
\usepackage{amsmath}

\newcommand{\join}{\text{\textcircled{{\footnotesize 1}}}}
\newcommand{\cojoin}{\text{\textcircled{{\footnotesize 0}}}}

\newcommand{\peo}{perfect elimination ordering }
\newcommand{\qed}{\hfill }
\newcommand{\proof}{\noindent {\bf Proof. }}
\newcommand{\NP}{\ensuremath{\mathbb{NP}}}
\newcommand{\ad}{\sim}
\newcommand{\noad}{\not\sim}
\newcommand{\0}{\text{ has a co-join to }}
\newcommand{\1}{\text{ has a join to }}

\newtheorem{defi}{Definition}
\newtheorem{theo}{Theorem}
\newtheorem{lemm}{Lemma}
\newtheorem{coro}{Corollary}
\newtheorem{cons}{Consequence}[section]
\newtheorem{clai}{Claim}[section]
\newtheorem{prop}{Proposition}
\newtheorem{obse}{Observation}

\def\inst#1{}
\title{Maximum Weight Independent Sets in Odd-Hole-Free Graphs Without Dart or Without Bull} 

\author{
Andreas Brandst\"adt\thanks{Institut f\"ur Informatik, Universit\"at Rostock, 
D-18051 Rostock, Germany. E-mail: ab@informatik.uni-rostock.de} 
\and
Raffaele Mosca\thanks{Dipartimento di Economia, Universit\'a degli Studi ``G. d'Annunzio'', Pescara 65121, Italy. 
E-mail: r.mosca@unich.it}}
\date{}
\textheight=22cm
\oddsidemargin=20pt
\textwidth=15cm

\begin{document}

\maketitle

\begin{abstract} 
The Maximum Weight Independent Set (MWIS) Problem on graphs with vertex weights asks for a set of pairwise nonadjacent vertices of maximum total weight. Being one of the most investigated and most important problems on graphs, it is well known to be NP-complete and hard to approximate. 
The complexity of MWIS is open for hole-free graphs (i.e., graphs without induced subgraphs isomorphic to a chordless cycle of length at least five). By applying clique separator decomposition as well as modular decomposition, we obtain polynomial time solutions of MWIS for odd-hole- and dart-free graphs as well as for odd-hole- and bull-free graphs ({\em dart} and {\em bull} have five vertices, say , and dart has edges , while bull has edges 
). If the graphs are hole-free instead of odd-hole-free then stronger structural results and better time bounds are obtained.
\end{abstract}

\noindent{\em Keywords}: Maximum weight independent set; clique separators; modular decomposition; polynomial time algorithm; 
hole-free graphs; dart-free graphs; bull-free graphs.

\section{Introduction}

The Maximum Weight Independent Set (MWIS) Problem on a given finite undirected simple graph with vertex weights asks for a set of pairwise nonadjacent vertices of maximum total weight. Being one of the most investigated and most important problems on graphs, it is well known to be NP-complete and hard to approximate. It is solvable in polynomial time on various graph classes while it remains NP-complete on some others. Its complexity is open for hole-free graphs (i.e., graphs without induced subgraphs isomorphic to a chordless cycle of length at least five). Recently, the following subclasses of hole-free graphs were studied: 
\begin{itemize} 
\item[(i)] hole-free graphs without induced diamond and in general without induced paraglider \cite{BraGiaMaf2012}; 
\item[(ii)] hole-free graphs without induced co-chair \cite{BraGia2012};
\item[(iii)] hole-free graphs without induced dart \cite{BasChaKar2012}.  
\end{itemize} 

In \cite{BraGiaMaf2012}, it was shown that the atoms of the graphs in the class (i) are weakly chordal or specific graphs, and in \cite{BraGia2012}, 
it was shown that the prime atoms of the graphs in the class (ii) are nearly weakly chordal, from which polynomial time algorithms for MWIS on these graphs follow.  

Using the approach of \cite{BraGia2012,BraGiaMaf2012}, Basavaraju, Chandran and Karthick in \cite{BasChaKar2012} showed that the MWIS problem can be solved in polynomial time for hole- and dart-free graphs by reducing it first to hole-, dart-, and gem-free graphs (which are hole- and paraglider-free). We extend previous results and show:
\begin{itemize}
\item[(i)] odd-hole- and dart-free graphs are nearly perfect, and (hole, dart)-free atoms are nearly weakly chordal. 
\item[(ii)] odd-hole- and bull-free prime graphs are nearly perfect, and (hole, bull)-free prime graphs are nearly weakly chordal. 
\item[(iii)] hole- and dart-free graphs as well as hole- and bull-free graphs have nice structure properties; MWIS for hole- and bull-free graphs can be solved in time . 
\item[(iv)] MWIS for - and bull-free graphs can be solved in time . 
\end{itemize}

The results in (i) and (ii) are based on the Strong Perfect Graph Theorem and imply that the MWIS problem is solvable in polynomial time for odd-hole- and dart-free graphs as well as for odd-hole- and bull-free graphs. 

Actually, Lemma 1 in \cite{BraGia2012} implies that also (odd-hole,co-chair)-free prime atoms are nearly perfect (which implies polynomial time for the MWIS problem on (odd-hole,co-chair)-free graphs). This fact was not explicitely mentioned in \cite{BraGia2012}. 

\section{Some Basic Notions and Results}  

\subsection{Basic Notions}  

For any missing notation or reference let us refer to \cite{BraLeSpi1999}. Let  be a finite undirected graph which is simple (i.e., without self-loops and multiple edges) with vertex set  and edge set . 
Let  and . 
For a vertex , let  denote the {\em open neighborhood} of  and let  denote the {\em closed neighborhood} of . We also say that for vertices ,  and  {\em see each other} ({\em miss each other}, respectively) if  (, respectively). 
Let  denote the {\em anti-neighborhood} of .  

For any vertex set , let  and  as well as .
For any nonempty vertex subset  with , let  (the set of {\em contact vertices} of  and ).

For any vertex set  let  be the subgraph of  induced by . Let  denote the {\em complement graph of }, also denoted as co-. Pairs  are also called {\em co-edges} of . 

For any disjoint vertex sets  of , let us say that  has a {\em join} (a {\em co-join}, respectively) to  if each vertex of  sees each vertex of  (misses each vertex of , respectively). If  and  has a join to , then let us say that  {\em dominates}  or is {\em universal for }.

 is the induced path with  vertices and  edges.  is the induced cycle with  vertices and  edges. A {\em hole} is  with . An 
{\em odd hole} is  with . An {\em anti-hole} is the complement graph of a hole. 
A {\em diamond} (or ) is formed by vertices , and edges . A {\em gem} is a one-vertex extension of a diamond, which can be obtained by adding  a dominating vertex to a .
A {\em dart} has five vertices , and edges , i.e., it consists of a diamond plus a degree-one vertex being adjacent to one of the degree-3 vertices of the diamond. A {\em bull} has five vertices  and edges . See Figure \ref{dart} for most of these specific graphs. 

\begin{figure}
  \begin{center}
    \epsfig{file=dartbull.eps}
    \caption{diamond, dart, bull, paraglider, and co-}
    \label{dart}
  \end{center}
\end{figure}

For any graph , let us say that  is {\em -free} if  contains no induced subgraph isomorphic to . A class of graphs is {\em hereditary} if it is closed under taking induced subgraphs. A graph is {\em chordal} it it is hole-free and -free. A graph is {\em weakly chordal} if it is hole- and anti-hole-free.
{\em Perfect graphs} play a crucial role in algorithmic graph theory, and the Strong Perfect Graph Theorem (see Theorem \ref{SPGT}) characterized them in terms of forbidden odd holes and odd anti-holes. It is well known that chordal graphs are weakly chordal, and weakly chordal graphs are perfect (see e.g. \cite{BraLeSpi1999} for this and the many facets of these graphs). 

An {\em independent set} of  is a subset of pairwise non-adjacent vertices of . A  of  is a set of pairwise adjacent vertices of .

\subsection{Techniques for the Maximum Weight Independent Set Problem}
 
The Maximum Independent Set (MIS) Problem asks for an independent vertex set of maximum cardinality. If  is a real-valued function on  then the Maximum Weight Independent Set (MWIS) Problem asks for an independent set of maximum total weight. Let  denote the maximum weight of an independent vertex set in .

A subset  is a  (or ) in  if  has more connected components than . A {\em clique cutset} is a cutset which is a clique. An  in  is an induced subgraph of  without clique cutset. More generally, a graph is an  if it has no clique cutset. A famous divide-and-conquer approach by using clique separators is described in \cite{Tarjan1985,White1984}. A consequence is the following:

\begin{theo}[\cite{Tarjan1985,White1984}]\label{MWISatomred}
If for a hereditary graph class , the MWIS problem is solvable in polynomial time for the atoms of  then the MWIS problem is solvable in polynomial time on graph class .
\end{theo}

Let  denote a (hereditary) graph property. A graph  is {\em nearly } if for all , the subgraph  has property . For short, we say that for every vertex, the anti-neighborhood has property . For example,  is {\em nearly weakly chordal} if the anti-neighborhood of every vertex is weakly chordal. Obviously the following holds:

\begin{obse}

\end{obse}

Thus, we obtain: 

\begin{coro}\label{nearlyPiMWIS}
Whenever MWIS can be solved in time  on a hereditary graph class with property , it can be solved in time  on nearly  graphs. 
\end{coro}

For example, since the MWIS problem can be solved in time  for weakly chordal graphs \cite{SpiSri1995}, it can be solved in time  for nearly weakly chordal graphs.

In Section \ref{holebullfr} we make use of modular decomposition. We say that a vertex  {\em distinguishes two vertices } if  sees  and misses . A subset  of vertices is a {\em module in } if no vertex  distinguishes two vertices . A module is {\em trivial} if it is either the empty set or  or one-elementary. A graph is {\em prime} if all its modules are trivial. We will use primality for solving MWIS on ((odd-)hole,bull)-free graphs. It is well known that MWIS can be solved in time  bottom-up along the modular decomposition tree for a hereditary graph class  if MWIS can be solved in time  for the prime graphs of ; the modular decomposition tree of a given graph can be determined in linear time \cite{McCSpi1999}. 
    
\begin{theo}\label{MWISprimered}
If for a hereditary graph class , the MWIS problem is solvable in time , , for the prime graphs of  then the MWIS problem is solvable in time  on graph class .
\end{theo} 

\subsection{Basic Results on Some Classes of Perfect and Other Graphs}

Subsequently, we use the following results and facts:

\begin{theo}[Strong Perfect Graph Theorem \cite{ChuRobSeyTho2006}]\label{SPGT}
A graph is perfect if and only if it is odd-hole-free and odd-anti-hole-free.
\end{theo}

\begin{theo}\label{basicrecprop}
The following graph classes can be recognized in polynomial time:
\begin{enumerate}
\item[] weakly chordal graphs 
\item[] hole-free graphs  
\item[] perfect graphs .
\end{enumerate}
\end{theo}
 
Recognition of odd-hole-free graphs is open, though recognition of odd-hole-free graphs with cliques of bounded size can be done in polynomial time \cite{ConCorLiuVusZam}.
 
\begin{theo}\label{basicMWISprop}
The MWIS problem for can be solved in polynomial time for the following graph classes:
\begin{enumerate}
\item[] weakly chordal graphs ; the time bound is .
\item[] perfect graphs .
\item[] hole-free graphs with no banner , with no paraglider , with no co-chair  and with no dart \cite{BasChaKar2012}.
\end{enumerate}
\end{theo} 

The MWIS problem for hole-free graphs (and for odd-hole-free graphs) is open; it is open even for the subclass of ()-free graphs. 

Finally let us mention a recent paper \cite{Chudn2011} introducing many structural properties for bull-free graphs (see also \cite{DeFMafPor1997} for perfect bull-free graphs) and the papers \cite{BraHoaLe2003,DeSSas1993} focussing on efficiently solving MWIS for (bull,chair)-free graphs. 

\section{Structure and MWIS for (Odd-)Hole- and Dart-Free Graphs}

In this section, we show that odd-hole- and dart-free graphs are nearly perfect and that hole- and dart-free atoms are nearly weakly chordal. This is the main result of the section and implies polynomial time for MWIS on both graph classes (with better time bound for hole- and dart-free graphs). We first collect some properties of dart-free graphs and deal with (odd-hole,dart)-free graphs.. 

\subsection{Structure and MWIS for (Odd-Hole,Dart)-Free Graphs}

\begin{prop}\label{prop:1}
Let  be a dart-free graph and let . If  and vertices  induce a  in  then  does not see all three vertices 
. 
\end{prop}

\noindent
{\bf Proof.} Otherwise, since  sees a vertex  (recall that ), such vertices  together with  and  would induce a dart. \qed

\medskip

Subsequently, when dealing with cycles of length  as well as their complements, index arithmetic is done modulo .

\begin{lemm}\label{lemm:1}
Let  be a dart-free graph containing an anti-hole , , say , with vertices  and co-edges  for , such that , and let . Then the following hold:
\begin{enumerate}
\item[] If  sees vertex  for some index , then  sees  and .
\item[]  is even, and either  or 
.
\end{enumerate}
\end{lemm}

\noindent
{\bf Proof.} (i): Assume without loss of generality that  sees . First, we show that  sees either  or .
Assume to the contrary that  sees neither  nor . Then to avoid that  induce a dart,  sees , and then by Proposition \ref{prop:1},  misses  (since ). Similarly by symmetry one has that  sees , and  misses . If , then  misses , a contradiction to Proposition \ref{prop:1}. If , then  sees , and then  induce a dart, a contradiction. Then  sees either  or .

Then let us assume without loss of generality that  sees , so by Proposition \ref{prop:1},  misses  and  but then  sees  too, otherwise  induce a dart.

\medskip

\noindent
(ii): By Proposition \ref{prop:1}, no vertex of  dominates . Thus, if  is odd, then by an iterated application of statement (i), one has that each vertex  dominates , a contradiction to Proposition \ref{prop:1}; if  is even then by statement (i) the last part of statement (ii) follows.      
\qed

\medskip

\begin{coro}\label{dartfreenearlycoC7free}
Dart-free graphs are nearly -free for .
\end{coro}

Theorem \ref{SPGT}, Lemma \ref{lemm:1} and Corollary \ref{dartfreenearlycoC7free} imply the following result:

\begin{theo}
Odd-hole,dart-free graphs are nearly perfect.
\end{theo}

\noindent
{\bf Proof.} Let  be an (odd hole, dart)-free graph. To prove the assertion it is sufficient to recall Theorem \ref{SPGT} and to 
show that if  has a odd anti-hole  with at least 7 vertices, then  which follows by contradiction from Lemma \ref{lemm:1} (ii).  
\qed

\medskip

Theorem \ref{basicMWISprop} (ii) implies:

\begin{coro}
The MWIS problem is solvable in polynomial time for odd-hole,dart-free graphs.
\end{coro}

\subsection{Structure and MWIS for (Hole,Dart)-Free Graphs}

For (hole,dart)-free graphs, some stronger properties can be shown if one additionally excludes clique cutsets: 

\begin{lemm}\label{lemm:co-c7}
Hole, dart-free atoms are nearly -free for .
\end{lemm}

\noindent
{\bf Proof.} Assume to the contrary that there is a vertex  in  such that the anti-neighborhood  of  contains an induced , say , for , with vertices  and co-edges  for  (index arithmetic modulo ). Clearly  and  is a cutset for . Let  be the connected component of the anti-neighborhood  of  containing . Then since  is an atom, there exist two vertices, say , such that: 
\begin{itemize}
\item[(a)]  misses , and 
\item[(b)] both  and  see some vertex of . 
\end{itemize}

Then by Lemma \ref{lemm:1}, let us distinguish between the following two cases, which are exhaustive by symmetry:

\begin{enumerate}
\item If  and , then  induce a dart, a contradiction.
\item If  and , then  induce a ; on the other hand, let  be a shortest path form  to  in ; then the subgraph induced by  is a hole, a contradiction.   
\end{enumerate}
This finally shows Lemma \ref{lemm:co-c7}.
\qed

\medskip

The proof of the subsequent Lemma \ref{lemm:co-c6} is similar to the one of Theorem 2 in \cite{BraGia2012}; in particular, the parts which are exactly the same (i.e., those which did not require the assumption co-chair-freeness) are reported in their original form.

\begin{lemm}\label{lemm:co-c6}
Hole, dart-free atoms are nearly -free.
\end{lemm}

\noindent
{\bf Proof.}
Assume to the contrary that there is a vertex  in  such that its anti-neighborhood  contains an induced , say , with vertices  such that  is a clique left(),  is a clique right(), and  and  are the edges between left() and right() (the {\em matching edges of A}). Let  denote the neighbors of  which see exactly  vertices in , , and let  denote the neighbors of  which see a vertex in the connected component  of the anti-neighborhood of  containing . Note that  depends on  but for short (and in order to avoid confusion with anti-neighborhoods) we write  instead of .

Let also  denote the vertices of  which see exactly  and  in  and similarly for some other cases. We first collect some simple properties.

\begin{clai}\label{clai:3}

\end{clai}

\noindent
{\em Proof of} Claim \ref{clai:3}. Follows by Proposition \ref{prop:1}. 


\begin{clai}\label{clai:4}
If  then  sees the two vertices of a matching edge in .
\end{clai}

\noindent
{\em Proof of} Claim \ref{clai:4}. Assume not; then  is either adjacent to two vertices in left() (right() respectively), say  sees , in which case  induce a dart, or  is adjacent to two nonadjacent vertices, say  sees  in which case  induce a . 

\begin{clai}\label{clai:5}
If  then  has a join to  or  has a join to .
\end{clai}

\noindent
{\em Proof of} Claim \ref{clai:5}. Let us consider the following cases which are exhaustive by symmetry. If  sees three vertices of  inducing a , then one has a contradiction to Proposition \ref{prop:1}. If  sees two vertices of left(), say , and a vertex of right() missing , that is vertex , then  induce a dart. 


\medskip

Since  is hole-free, we have:

\begin{clai}\label{clai:7}
For all  with  missing , we have  or vice versa. Moreover,  and  have a common neighbor in .
\end{clai}

\begin{clai}\label{clai:8}
If  then  and vice versa.
\end{clai}

\noindent
{\em Proof of} Claim \ref{clai:8}. Assume not; let  and , say  and  by Claims \ref{clai:4} and \ref{clai:5}. Then since  induce no ,  misses  but now, by Claim \ref{clai:7}, the neighborhoods of  and  must be comparable - contradiction. 


\begin{clai}\label{clai:9}
At most one of  is nonempty.
\end{clai}

\noindent
{\em Proof of} Claim \ref{clai:9}. Assume not; without loss of generality, let  and . Then by Claim \ref{clai:7},  sees  but now  induce a . 


\begin{clai}\label{clai:10}
The set  of neighbors of all  in  is a clique.
\end{clai}

\noindent
{\em Proof of} Claim \ref{clai:10}. Assume not; let without loss of generality  see  and  see . Then since  do not induce a ,  misses  but now by Claim \ref{clai:7},  and  must be comparable - contradiction. 


\begin{clai}\label{clai:11}
No vertex in  sees a vertex in , i.e., .
\end{clai}

\noindent
{\em Proof of} Claim \ref{clai:11}. Assume not; let  and first assume that , say  with  seeing . If  sees , then  induce a dart, and if  sees  then  induce a . The other cases are symmetric. This shows Claim \ref{clai:11}. 


\begin{clai}\label{clai:12}
 is a clique.
\end{clai}

\noindent
{\em Proof of} Claim \ref{clai:12}. First note that  since  is -free. If there are  with  missing  then  induce a dart. Then  is a clique. The fact that  is a clique is shown analogously. 


\medskip

Now we conclude that in any case, we get a clique separator between  and some vertex in  (which finally contradicts to the assumption that  is an atom):

\medskip

\noindent
{\bf Case 1.} . 

\medskip

Then by Claim \ref{clai:8}, .
First suppose that . We claim that  is a clique separator: Recall that by Claim \ref{clai:10},  is a clique, by Claim \ref{clai:12},  is a clique, and by Claims \ref{clai:7} and \ref{clai:11}, every  sees every  (note that if  then , and the case that  is one of the matching edges is impossible if ). Obviously,  is a separator between  and a nonempty part of .

Now suppose that . Then  is a clique separator (in this case, possibly  and ).

\medskip

\noindent
{\bf Case 2.} . 

\medskip

Then by Claim \ref{clai:8}, , and by Claim \ref{clai:9}, at most one of the sets , ,  is nonempty, say  and . Then  is a clique separator (note that in this case, by Claims \ref{clai:7} and \ref{clai:11}, ).

\medskip
\noindent
{\bf Case 3.} . 

\medskip

Then  since  is connected, and again by Claim \ref{clai:10},  is a clique separator.
This finishes the proof of Lemma \ref{lemm:co-c6}.             
\qed

\medskip

By Lemmas \ref{lemm:co-c7} and \ref{lemm:co-c6}, one obtains the following result:

\begin{theo}\label{holedartnearlywc}
Hole, dart-free atoms are nearly weakly chordal.
\end{theo}

By Theorem \ref{basicMWISprop} (i), we obtain:

\begin{coro}
The MWIS problem is solvable in polynomial time for hole, dart-free graphs.
\end{coro}

The corresponding result in \cite{BasChaKar2012} gives a time bound of  for the MWIS problem on hole, dart-free graphs. This result is based on Theorem 4 in \cite{BasChaKar2012} showing that (hole,dart)-free atoms are nearly gem-free. Their time bound is better than the obvious time bound resulting from Theorem \ref{holedartnearlywc} and Theorem \ref{basicMWISprop} (ii). However, more can be said about the structure of (hole,dart)-free graphs: Lemma 2 in \cite{Brand2004} implies that prime dart- and gem-free graphs are diamond-free. This is slightly better than Lemma 1 in \cite{BasChaKar2012} saying that prime dart- and gem-free graphs are paraglider-free.  

Moreover, we need the following result:

\begin{theo}[\cite{BerBraGiaMaf2012}]\label{holediamondcoC6freeatoms}   
Hole,diamond-free atoms that contain no induced  are either a clique or chordal bipartite. 
\end{theo}

Theorem \ref{holediamondcoC6freeatoms} and Lemma \ref{lemm:co-c6} imply: 

\begin{coro}\label{holedartfrstructure}
Hole, dart-free prime atoms are either nearly a clique or nearly chordal bipartite. 
\end{coro}

This does not improve the  time bound of \cite{BasChaKar2012} but gives more structural insight as asked for in the last paragraph of \cite{BasChaKar2012}.  
 
\section{Structure and MWIS for (Odd-)Hole- and Bull-Free Graphs}\label{holebullfr}

In this section, we show that prime odd-hole- and bull-free graphs are nearly perfect and that prime hole- and bull-free atoms are nearly weakly chordal. This is the main result of the section and implies polynomial time for MWIS on both graph classes (with better time bound for (hole, bull)-free graphs). We first collect some properties of prime bull-free graphs. 

\medskip

Let  be a prime graph with at least 7 vertices, and suppose that  contains a , say , for some . Since  is prime,  is not a module, and thus, there is a vertex  distinguishing vertices , say  and . Since  is connected, we can assume without loss of generality that  distinguishes an edge  in . Let  and for , let  result by adding a distinguishing vertex  to , that is, . As before, we can assume without loss of generality that  distinguishes an edge in . This defines a strictly increasing sequence of induced subgraphs of . Since  is finite, there is a largest  such that  exists.    

\begin{lemm}\label{lemm:bull}
If  is a connected bull-free graph containing some induced subgraph , , as defined above, with nonempty anti-neighborhood , and  then  has a join to . 
\end{lemm}

\noindent 
{\bf Proof.}
We show Lemma \ref{lemm:bull} by induction on . For , the proof goes as follows. Let  and let  be a neighbor of . To prove the assertion, let us show that if  sees a vertex  for some index , then  sees also  and  (index arithmetic modulo ); by iterating this argument the assertion follows.

Then let us assume without loss of generality that  sees , and show that  sees  and . To avoid that  induce a bull,  sees either  or  or . Moreover, by a symmetry argument, one has that  sees either  or  or . Also, if  sees  and , then  sees  and , otherwise  or  induce a bull. All the previous facts imply that  sees either  or ; without loss of generality, say  sees . Then  sees , otherwise  induce a bull. Then  sees , otherwise  induce a bull. This completes the proof for . 

\medskip

Now, as the induction hypothesis, assume that the assertion is true for values at most , and let us show that it holds for . Let  be any vertex distinguishing an edge  in , say  and , and let . 
Suppose that , and without loss of generality choose a vertex  such that the distance between  and  is 2. Now, also  holds, and the distance between  and  is at least 2. If the distance is 2 then let  be a neighbor of . Thus,  misses  and, by the induction hypothesis, any neighbor  of  has a join to . Since ,  holds. Since  is bull-free,  is not a bull, and since  misses  and  sees , it follows that  and thus,  has a join to . 

In the other case, assume that the distance between  and  is at least 3, and recall that the distance between  and  is 2; thus, let  be a common neighbor of  and . Note that in this case,  misses , and now by the induction hypothesis,  must have a join to  which is a contradiction.  
This shows Lemma \ref{lemm:bull}. 
\qed

\begin{lemm}\label{lemm:bullnearlycoCkfr}
Prime bull-free graphs are nearly -free for any .
\end{lemm}

\noindent 
{\bf Proof.} Suppose that there is a vertex  such that  contains a , say , for some . Without loss of generality, let  be in distance 2 to , and let . Thus, by definition and by Lemma \ref{lemm:bull},  is contained in . Now let , , and  be defined as above. We claim that  misses : Clearly  misses . Suppose that  misses  and sees ; recall that  results by adding a distinguishing vertex  to ; say, for some edge  in ,  sees  and misses . Now if  misses  and sees  then , and now, since , by Lemma \ref{lemm:bull},  should be universal for  - a contradiction.     

Thus, by Lemma \ref{lemm:bull}, it follows that  is a module in  which is a contradiction to the assumption that  is prime.   
\qed

\medskip

By the Strong Perfect Graph Theorem and by the definition of weakly chordal graphs, it follows:

\begin{coro}\label{holebullfrnearlywc}
Prime odd-hole,bull-free graphs are nearly perfect, and prime hole,bull-free graphs are nearly weakly chordal.
\end{coro}

By Theorem \ref{MWISprimered}, Corollary \ref{nearlyPiMWIS} and Theorem \ref{basicMWISprop}, we obtain: 

\begin{coro}\label{holebullfrMWIS}
For odd-hole,bull-free graphs, the MWIS problem is solvable in polynomial time. 
\end{coro}

The time bound for (hole,bull)-free graphs is .

\medskip

Note that Corollary \ref{holebullfrMWIS} concerning (odd-hole,bull)-free graphs is close to the MWIS result implied by the structure result of De Simone \cite{DeSim1993}, i.e., MWIS is polynomial for graphs with no odd apples and no six specific induced subgraphs, five of which contain a bull. It is also well known that MWIS can be solved in polynomial time for - and bull-free graphs since such graphs do not contain any of the forbidden subgraphs in \cite{DeSim1993}.
However, for - and bull-free graphs, we can say more; the proof of Lemma \ref{lemm:bull} can be adapted to similar cases as we describe subsequently.   

\section{MWIS for - and Bull-Free Graphs Revisited}\label{P5bullfr}

\begin{lemm}\label{lemm:bullnearlyC5housefr}
Prime - and bull-free graphs are nearly -free and nearly house-free.
\end{lemm}
 
\noindent 
{\bf Proof.}
We recursively define a sequence , , of increasing subgraphs for which  is nonempty,  and  being a neighbor of  in the same way as before Lemma \ref{lemm:bull}; only  is different. For showing that prime - and bull-free graphs are nearly -free,  is a , say with vertices  and edges  (index arithmetic modulo 5), and for showing that prime - and bull-free graphs are nearly house-free,  is a house, say with vertices  such that  induce a  and  sees  and . 
  
We show Lemma \ref{lemm:bullnearlyC5housefr} by induction on . Let  be a prime (,bull)-free graph. For , we first consider the case when  is a . We need the following notion: 

For a subgraph  of , a vertex  is an {\em -vertex of } if  sees exactly  vertices in . Obviously,  has no 1-vertex since  is -free, and  has no 2-vertex since two consecutive neighbors lead to a bull, and two nonconsecutive neighbors lead to . Similarly, if for a 3-vertex, not all neighbors are consecutive, we get a bull, and for consecutive neighbors, we get a . Any 4-vertex leads to a bull. Thus, any vertex outside  seeing  is universal for . Now with the same inductive arguments as in the proof of Lemma \ref{lemm:bull},
we show that any vertex  has a join to . With the same arguments as in Lemma \ref{lemm:bullnearlycoCkfr}, we obtain that  is nearly -free. 

\medskip
Now, let  be a house as described above.  

\begin{enumerate}
\item For a 1-vertex , we obtain a  if  is adjacent to  or  or a bull if  is adjacent to  or . 

\item For a 2-vertex , there are two cases: 

\begin{enumerate}
\item  has two neighbors in the :  
\begin{enumerate}
\item if  sees  and  then  induce a bull, 
\item if  sees  and  then  induce a ,      
\item if  sees  and  then  induce a bull, and
\item if  sees  and  then  induce a bull. 
\end{enumerate}

\item  has one neighbor in the  and sees :
\begin{enumerate}
\item if  sees  then  induce a bull, and 
\item if  sees  then  induce a bull.      
\end{enumerate}
\end{enumerate}

\item For a 3-vertex , there are two cases: 

\begin{enumerate}
\item  has three neighbors in the :  
\begin{enumerate}
\item if  sees ,  and  then  induce a bull, and 
\item if  sees ,  and  then  induce a bull.      
\end{enumerate}

\item  has two neighbors in the  and sees :
\begin{enumerate}
\item if  sees  and  then  induce a bull, 
\item if  sees  and  then  induce a bull,      
\item if  sees  and  then  induce a bull, and 
\item if  sees  and  then  induce a bull.          
\end{enumerate}
\end{enumerate}

\item For a 4-vertex , there are two cases: 

\begin{enumerate}
\item If  has four neighbors in the  then  induce a bull. 

\item Otherwise, if  has three neighbors in the  and sees  then:
\begin{enumerate}
\item if  sees ,  and  then  induce a bull, and  
\item if  sees ,  and  then  induce a bull.      
\end{enumerate}
\end{enumerate} 
 
\end{enumerate}
 
Thus,  is universal for , and with the same arguments as above, we obtain that  is nearly house-free.
\qed  

\medskip

A bipartite graph  with color classes  and  is a {\em bipartite chain graph} if the neighborhoods of the vertices of one color class form an increasing sequence with respect to set inclusion. It is well known that these graphs can be recognized in linear time, have bounded clique-width, and MWIS can be solved in linear time for them. From a result by Fouquet \cite{Fouqu1993}, it follows: 
 
\begin{coro}\label{P5bullfrnearlychain}
Prime ,bull-free graphs are nearly bipartite chain graphs or nearly co-bipartite chain graphs. 
\end{coro}

\begin{coro}\label{P5bullfrMWIS}
For ,bull-free graphs, the MWIS problem is solvable in time . 
\end{coro}

A polynomial time algorithm for MWIS on (,dart)-free graphs follows from \cite{Mosca2004}. 

\section{Conclusion}

The class of hole-free graphs is closely related to many well-studied graphs classes, such as chordal graphs, weakly chordal graphs and perfect graphs. However the complexity of the MWIS problem for hole-free graphs is open. Following a recent line of research \cite{BraGia2012,BraGiaMaf2012}, we show that MWIS can be solved in polynomial time for odd-hole- and dart-free graphs as well as for odd-hole- and bull-free graphs. For this purpose, the main structural results of this paper are the following (which might also be useful in other contexts):  

\begin{enumerate}
\item 
\begin{enumerate}
\item Dart-free graphs are nearly -free for . 
\item Hole- and dart-free atoms are nearly -free for .
\item Consequently, odd-hole- and dart-free graphs are nearly perfect, and hole- and dart-free atoms are nearly weakly chordal.  
\end{enumerate}

\item 
\begin{enumerate}
\item Prime bull-free graphs are nearly -free for any .
\item Consequently, prime odd-hole- and bull-free graphs are nearly perfect, and prime hole- and bull-free atoms are nearly weakly chordal. 
\end{enumerate}

\item Using the approach for bull-free graphs, we show that prime (,bull)-free graphs are nearly bipartite chain graphs or nearly co-bipartite chain graphs; this leads to a better time bound for MWIS on (,bull)-free graphs.  
\end{enumerate}

The results on dart-free graphs imply that MWIS is solvable in polynomial time for odd-hole- and dart-free graphs, by finally reducing the problem to perfect graphs. Note that (hole, dart)-free graphs can be recognized in polynomial time since recognition of hole-free graphs can be done in polynomial time by \cite{Spinrad1991} (see also \cite{ChvFonSunZem2002} for recognizing dart-free perfect graphs in polynomial time, though in general perfect graphs can be recognized in polynomial time by \cite{ChuCorLiuSeyVus2005,CorLiuVus2003}).

\medskip

The results on bull-free graphs allow to solve MWIS in polynomial time for odd-hole- and bull-free graphs, again by finally reducing the problem to perfect graphs. Note that (hole, bull)-free graphs can be recognized in polynomial time since recognition of hole-free graphs can be done in polynomial time by \cite{Spinrad1991}. 

\medskip

Let us conclude by a remark on the class of (hole,gem)-free graphs. It seems that the situation for (hole,gem)-free graphs is more complicated than for (hole,dart)-free and (hole,bull)-free graphs. Clearly, by the Strong Perfect Graph Theorem, (hole,gem)-free graphs are perfect, and thus, the MWIS problem is solvable in polynomial time for (hole,gem)-free graphs. However, we would like to find a direct combinatorial algorithm with a good time bound as in the other cases. Therefore we conclude with the following: 

\medskip

\noindent
{\bf Open Problem.} What is a good time bound for the MWIS problem for (hole,gem)-free graphs?




\begin{footnotesize}
\renewcommand{\baselinestretch}{0.4}

\begin{thebibliography}{99}

\bibitem{BasChaKar2012}
    M.~Basavaraju, L.S.~Chandran, and T.~Karthick,
    Maximum Weight Independent Sets in Hole- and Dart-Free Graphs,
    {\sl Discrete Applied Math.} 160 (2012) 2364-2369.

\bibitem{BerBorHeg2000}
    A.~Berry, J.-P.~Bordat, and P.~Heggernes, 
    Recognizing weakly triangulated graphs by edge separability,  
    {\em Nordic J. of Computing}, 7 (2000) 164-177.

\bibitem{BerBraGiaMaf2012}
    A.~Berry, A.~Brandst\"adt, V.~Giakoumakis, and F.~Maffray, 
    Efficiently recognizing, decomposing and triangulating hole- and diamond-free graphs,
    manuscript 2012.
    
\bibitem{Brand2004}
    A.~Brandst\"adt, 
    (,diamond)-free graphs revisited: structure and linear time optimization,
    {\sl Discrete Applied Math.} 138 (2004) 13-27.
    
\bibitem{BraGia2012}
    A.~Brandst\"adt and V.~Giakoumakis,
    Maximum Weight Independent Sets in hole- and co-chair-free graphs,
    {\sl Information Processing Letters} 112 (2012) 67-71.

\bibitem{BraGiaMaf2012}
    A.~Brandst\"adt, V.~Giakoumakis, and F.~Maffray, 
    Clique separator decomposition of hole-free and diamond-free graphs and algorithmic conseguences,
    {\sl Discrete Applied Math.} 160 (2012) 471-478.

\bibitem{BraHoaLe2003}
    A.~Brandst\"adt, C.T.~Ho\`ang, and V.B.~Le, 
    Stability number of bull- and chair-free graphs revisited,
    {\sl Discrete Applied Math.} 131 (2003) 39-50.

\bibitem{BraKleLozMos2010}
    A.~Brandst\"adt, T.~Klembt, V.V.~Lozin, and R.~Mosca,
    On independent vertex sets in subclasses of apple-free graphs,
    {\sl Algorithmica} 56 (2010) 383-393.

\bibitem{BraLeSpi1999}
    A.~Brandst\"adt, V.B.~Le, and J.P.~Spinrad,
    Graph Classes: A Survey,
    {\sl SIAM Monographs on Discrete Math. Appl., Vol.} 3,
    SIAM, Philadelphia (1999).

\bibitem{Chudn2011}
    M.~Chudnovsky, 
    The structure of bull-free graphs II and III. A summary, 
    {\sl J. of Combinatorial Theory Ser. B} to appear (available online).

\bibitem{ChuCorLiuSeyVus2005}
    M.~Chudnovsky, G.~Cornu\'ejols, X.~Liu, P.~Seymour, and K.~Vuskovi\v c,
    Recognizing Berge graphs, 
    {\sl Combinatorica} 25 (2005) 143-186.

\bibitem{ChuRobSeyTho2006}
    M.~Chudnovsky, N.~Robertson, P.~Seymour, and R.~Thomas,
    The strong perfect graph theorem, 
    {\sl Ann. Math.} 164 (2006) 51-229.

\bibitem{ChvFonSunZem2002}
    V.~Chv\'atal, J.~Fonlupt, L.~Sun, and A.~Zemirline,
    Recognizing dart-free perfect graphs,
    {\sl SIAM J. Comput.} 31 (2002) 1315-1338.

\bibitem{ConCorLiuVusZam}
    M.~Conforti, G.~Cornu\'ejols, X.~Liu, K.~Vu\v skovic, and G.~Zambelli,
    Odd hole recognition in graphs of bounded clique size,
    {\sl SIAM J. Discrete Math.} 20 (2006) 42-48.

\bibitem{CorLiuVus2003}
    G.~Cornu\'ejols, X.~Liu, and K.~Vu\v skovic,
    A polynomial algorithm for recognizing perfect graphs,
    Proc. 44th Ann. IEEE Sympos. on Foundations of Computer Science FOCS 2003, 20-27.

\bibitem{DeSim1993}
    C.~De~Simone,
    On the vertex packing problem,
    {\sl Graphs and Combinatorics} 9 (1993) 19-30.

\bibitem{DeSSas1993}
    C.~De~Simone and A.~Sassano,  
    Stability number of bull- and chair-free graphs,
    {\sl Discrete Applied Math.} 9 (1993) 121-129.

\bibitem{DeFMafPor1997}
    C.M.H.~De~Figueiredo, F.~Maffray, and O.~Porto,
    On the structure of bull-free perfect graphs,
    {\sl Graphs and Combinatorics} 13 (1997) 31-55.

\bibitem{EscSri1995}
    E.M.~Eschen and R.~Sritharan,
    A characterization of some graph classes with no long holes,
    {\sl J. Combinatorial Theory Ser. B} 65 (1995) 156-162.

\bibitem{FriHedJac1998}
    G.H.~Fricke, S.T.~Hedetniemi, and D.P.~Jacobs,
    Independence and irredundance in -regular graphs,
    {\sl Ars Combinatoria} 49 (1998) 271-279.

\bibitem{Fouqu1993}
    J.-L.~Fouquet,
    A decomposition for a class of -free graphs,
    {\sl Discrete Math.} 121 (1993) 75-83.




\bibitem{GroLovSch1984}
    M.~Gr\"otschel, L.~Lov\'asz, and A.~Schrijver,
    Polynomial algorithms for perfect graphs,
    {\sl Annals of Discrete Math.} 21 (1984) 325-356.
 
\bibitem{HaySpiSri2007}
    R.B.~Hayward, J.B.~Spinrad, R.~Sritharan,
    Improved algorithms for weakly chordal graphs,
    {\sl ACM Transactions on Algorithms} 3 (2007).

\bibitem{McCSpi1999} 
    R.M.~McConnell, J.P.~Spinrad,
    Modular decomposition and transitive orientation,
    {\sl Discrete Math.} 201 (1999) 189-241.

\bibitem{Mosca2004} 
    R.~Mosca, 
    Soem results on maximum stable sets in certain -free graphs,
    {\sl Discrete Applied Math.} 132 (2004) 175-183.


\bibitem{Spinrad1991}
    J.P.~Spinrad,
    Finding large holes,
    {\sl Information Processing Letters} 39 (1991) 227-229.

\bibitem{SpiSri1995}
    J.P.~Spinrad, R.~Sritharan,
    Algorithms for weakly triangulated graphs,
    {\sl Discrete Applied Math} 59 (1995) 181-191.

\bibitem{Tarjan1985}
    R.E.~Tarjan,
    Decomposition by clique separators,
    {\sl Discrete Math.} 55 (1985) 221-232.

\bibitem{White1984}
    S.H.~Whitesides,
    A method for solving certain graph recognition and optimization
    problems, with applications to perfect graphs, in: Berge, C.
    and V. Chv\'atal (eds),
    {\sl Topics on perfect graphs},
    North-Holland, Amsterdam, 1984.

\end{thebibliography}

\end{footnotesize}

\end{document}
