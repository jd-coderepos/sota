\documentclass[conference]{IEEEtran}


\usepackage{color}
\usepackage{algorithm,algorithmicx,algcompatible, cite}
\usepackage{verbatim, subfigure}
\usepackage{array, calc, multicol}
\usepackage{xspace}
\usepackage{graphics, epstopdf, graphicx, epsfig, psfrag, epsf, subfigure}
\usepackage{amssymb, amsfonts, amsmath, times}
\usepackage{multicol,balance, verbatim}
\usepackage{textcomp}








\newcommand{\ie}{{\em i.e., }}
\newcommand{\Ie}{{\em I.e., }}
\newcommand{\eg}{{\em e.g., }}
\newcommand{\Eg}{{\em E.g., }}

\newtheorem{theorem}{Theorem}
\newtheorem{lemma}[theorem]{Lemma}
\newtheorem{proposition}[theorem]{Proposition}
\newtheorem{corollary}[theorem]{Corollary}
\newtheorem{definition}{Definition}
\newtheorem{example}{Example}
\newtheorem{problem}{Problem}


\DeclareGraphicsExtensions{.eps, .pdf, .png, .jpg}   \newcommand{\Jset}{\mathcal{J}}
\newcommand{\Sset}{\mathcal{S}}
\newcommand{\Kset}{\mathcal{K}}
\newcommand{\Aset}{\mathcal{A}}
\newcommand{\Pset}{\mathcal{P}}
\newcommand{\Cset}{\mathcal{C}}
\newcommand{\Nset}{\mathcal{N}}
\newcommand{\Hset}{\mathcal{H}}
\newcommand{\Iset}{\mathcal{I}}
\newcommand{\Lset}{\mathcal{L}}
\newcommand{\Zset}{\mathcal{Z}}
\newcommand{\Qset}{\mathcal{Q}}
\newcommand{\Uset}{\mathcal{U}}

\DeclareMathOperator*{\argmax}{arg\,max}

\IEEEoverridecommandlockouts

\makeatletter
\newcommand{\oset}[2]{{\mathop{#2}\limits^{\vbox to -.5\ex@{\kern-\tw@\ex@
\hbox{\scriptsize #1}\vss}}}}
\makeatother




\begin{document}

\title{Flow Control and Scheduling for Shared FIFO Queues over Wireless Networks \vspace{-5pt}  }



\author{
\authorblockN{Shanyu Zhou}
\authorblockA{University of Illinois at Chicago \\
  \tt szhou45@uic.edu}
 \and
 \authorblockN{Hulya Seferoglu}
\authorblockA{University of Illinois at Chicago \\
 \tt hulya@uic.edu}
 \and
\authorblockN{Erdem Koyuncu}
\authorblockA{University of California, Irvine\\
 \tt  ekoyuncu@uci.edu}
}

\maketitle





{}\vspace{-35pt}{}

\allowdisplaybreaks


\begin{abstract}
We investigate the performance of First-In, First-Out (FIFO) queues over wireless networks. We characterize the stability region of a general scenario where an arbitrary number of FIFO queues, which are served by a wireless medium, are shared by an arbitrary number of flows. In general, the stability region of this system is non-convex. Thus, we develop a convex inner-bound on the stability region, which is provably tight in certain cases. The convexity of the inner bound allows us to develop a resource allocation scheme; . Based on the structure of , we develop a stochastic flow control and scheduling algorithm; . We show that  achieves optimal operating point in the convex inner bound. Simulation results show that our algorithms significantly improve the throughput of wireless networks with FIFO queues, as compared to the well-known queue-based flow control and max-weight scheduling. 
\end{abstract}

\section{Introduction}\label{sec:intro}

The recent growth in mobile and media-rich applications continuously increases the demand for wireless bandwidth, and puts a strain on wireless networks \cite{cisco_index}, \cite{ericsson_report}. This dramatic increase in demand poses a challenge for current wireless networks, and calls for new network control mechanisms that make better use of scarce wireless resources. Furthermore, most existing, especially low-cost, wireless devices have a relatively rigid architecture with limited processing power and energy storage capacities that are not compatible with the needs of existing theoretical network control algorithms. One important problem, and the focus of this paper, is that low-cost wireless interface cards are built using First-In, First-Out (FIFO) queueing structure, which is not compatible with the per-flow queueing requirements of the optimal network control schemes such as backpressure routing and sheduling \cite{tass_eph1}.


The backpressure routing and scheduling paradigm has emerged from the pioneering work \cite{tass_eph1}, \cite{tass_eph2}, which showed that, in wireless networks where nodes route and schedule packets based on queue backlogs, one can stabilize the queues for any feasible traffic. It has also been shown that backpressure can be combined with flow control to provide utility-optimal operation \cite{neely_mod}. Yet, backpressure routing and scheduling require each node in the network to construct per-flow queues. The following example demonstrates the operation of backpressure. 

\begin{example}
Let us consider a canonical example in Fig. \ref{fig:example_fifo_v1}(a), where a transmitter node , and two receiver nodes ,  form a one-hop downlink topology. There are two flows with arrival rates  and  destined to nodes  and , respectively. The throughput optimal backpressure scheduling scheme, also known as max-weight scheduling, assumes the availability of
per-flow queues  and  as seen in Fig. \ref{fig:example_fifo_v1}(a), and 
makes a transmission decision at each transmission opportunity based on queue backlogs, \ie  and . In particular, the max-weight scheduling algorithm determines , and transmits from queue . It was shown in \cite{tass_eph1}, \cite{tass_eph2} that if the arrival rates  and  are inside the stability region of the wireless network, the max-weight scheduling algorithm stabilizes the queues. On the other hand, in some devices, per-flow queues cannot be constructed. In such a scenario, a FIFO queue, say  is shared by flows  and  as shown in Fig.~\ref{fig:example_fifo_v1}(b), and the packets are served from  in a FIFO manner.
\hfill 
\end{example}


\begin{figure}[t!]
\vspace{-20pt}
\centering
\subfigure[Per-flow queues]{ \label{fig:intro_example_a} \scalebox{.5}{\includegraphics{FIFO_example_fig_a.eps}} }
\subfigure[FIFO queue] {\label{fig:intro_example_b} \scalebox{.5}{\includegraphics{FIFO_example_fig_b.eps}} }
\vspace{-5pt}
\caption{Queueing structure of one-hop downlink topology with (a) per-flow queues, and (b) a FIFO queue. } \label{fig:example_fifo_v1}
\vspace{-15pt}
\end{figure}

Constructing per-flow queues may not be feasible in some devices especially at the link layer due to rigid architecture, and one FIFO queue is usually shared by multiple flows. For example, although current WiFi-based devices have more than one hardware queue \cite{madwifi_chipsets}, their numbers are restricted (up to 12 queues according to the list in \cite{madwifi_chipsets}), while the number of flows passing through a wireless device could be significantly higher. Also, multiple queues in the wireless devices are mainly constructed for prioritized traffic such as voice, video, etc., which further limits their usage as per-flow queues. 
On the other hand, constructing per-flow queues may not be preferable in some other devices such as sensors or home appliances for which maintaining and handling per-flow queues could introduce too much processing and energy overhead. Thus, some devices, either due to rigid architecture or limited processing power and energy capacities, inevitably use shared FIFO queues, which makes the understanding of the behavior of FIFO queues over wireless networks very crucial.


{\em Example 1 - continued:}
Let us consider Fig.~\ref{fig:example_fifo_v1} again. When a FIFO queue is used instead of per-flow queues, the well-known head-of-line (HOL) blocking phenomenon occurs. As an example, suppose that at transmission instant , the links  and  are at ``ON'' and ``OFF'' states, respectively. In this case, a packet from  can be transmitted if per-flow queues are constructed. Yet, in FIFO case, if HOL packet in  belongs to flow , no packet can be transmitted and wireless resources are wasted.
\hfill 

Although HOL blocking in FIFO queues is a well-known problem, achievable throughput with FIFO queues in a wireless network is generally not known. In particular, stability region of a wireless network with FIFO queues as well as resource allocation schemes to achieve optimal operating points in the stability region are still open problems.


In this work, we investigate FIFO queues over wireless networks. We consider a wireless network model presented in Fig.~\ref{fig:main-example} with multiple FIFO queues that are in the same transmission and interference range. (Note that this scenario is getting increasing interest in practice in the context of device-to-device and cooperative networks \cite{microcast}.) 
Our first step towards understanding the performance of FIFO queues in such a setup is to characterize the stability region of the network. Then, based on the structure of the stability region, we develop efficient resource allocation algorithms; {\em Deterministic FIFO-Control} () and {\em Queue-Based FIFO-Control} (). The following are the key contributions of this work:
\begin{itemize}
\item We characterize the stability region of a general scenario where an arbitrary number of FIFO queues are shared by an arbitrary number of flows.
\item The stability region of the FIFO queueing system under investigation is non-convex. Thus, we develop a convex inner-bound on the stability region, which is provably tight for certain operating points.
\item We develop a resource allocation scheme; , and a queue-based stochastic flow control and scheduling algorithm; . We show that  achieves optimal operating point in the convex inner bound.
\item We evaluate our schemes via simulations for multiple FIFO queues and flows. The simulation results show that our algorithms significantly improve the throughput as compared to the well-known queue-based flow control and max-weight scheduling schemes.
\end{itemize}

The structure of the rest of the paper is as follows. Section~\ref{sec:system} gives an overview of the system model. Section~\ref{sec:stability_region} characterizes the stability region with FIFO queues. Section~\ref{sec:oFC_qFC} presents our resource allocation algorithms;  and . Section~\ref{sec:performance} presents simulation results. Section~\ref{sec:related} presents related work. Section~\ref{sec:conclusion} concludes the paper.


\section{System Model}\label{sec:system}
{\em Wireless Network Setup:}
We consider a wireless network model presented in Fig.~\ref{fig:main-example} with  FIFO queues. Let  be the set of FIFO queues,  be the th FIFO queue, and  be the set of flows passing through . Also, let  and  denote the cardinalities of sets  and , respectively. We assume in our analysis that time is slotted, and  refers to the beginning of slot .

\begin{figure}
\vspace{5pt}
\centering
\scalebox{.55}{\includegraphics{FIFO_general_fig.eps}}
\vspace{-5pt}
\caption{The wireless network model that we consider in this paper.  FIFO queues share a wireless medium, where the th FIFO queue,   carries  flows towards their respective receiver nodes. The arrival rate of the th flow passing through the th queue is .}
\label{fig:main-example}
\vspace{-15pt}
\end{figure}

{\em Flow Rates:}  Each flow passing through  and destined for node  is generated according to an arrival process  at time slot . The arrivals are i.i.d. over the time slots such that for every  and , we have  and , where  denotes the expected value.

{\em Channel Model:} In our setup in Fig.~\ref{fig:main-example}, as we mentioned earlier, we assume that all FIFO queues are in the same transmission and interference range, \ie only one FIFO queue could be served by a shared wireless medium at time . On the other hand, a channel state from a FIFO queue to a receiver node may vary. In particular, at slot ,  is the channel state vector, where  is the state of the link at time  from the th queue  to receiver node  such that .  The link state  takes values from the set  according to a probability distribution which is i.i.d. over time slots. If , packets can be transmitted to receiver node  with rate . We assume, for the sake of simplicity in this paper, that , and  packet can be transmitted at time slot  if . If , no packets are transmitted. The  and  probabilities of  are  and , respectively. Note that  only determines the channel state; \ie the actual transmission opportunity from  depends on the HOL packet as explained next.


{\em Queue Structure and Evolution:}  Suppose that the Head-of-Line (HOL) packet of  at time  is . The HOL packet together with the channel state defines the state of . In particular, let  be the state of  at time  such that . The state of  is , \ie  if  at time . Otherwise, . We define            as the set of the states of all FIFO queues.

Let us now consider the evolution of the HOL packet. If the state of queue  is  at time , \ie , the HOL packet can be transmitted (depending on the scheduling policy). If we assume that HOL packet is transmitted according to the scheduling policy, then a new packet is placed in the HOL position in . The probability that this new HOL packet belongs to the th flow is  and it depends on the arrival rates via .

Now, we can consider the evolution of . At time ,  packets arrive to , and  packets are served according to the FIFO manner. Thus, queue size  evolves according to the following dynamics.
 
Note that  depends on the states of the queues;  at time , which characterize the stability region of the wireless network.  Note that  depends on arrival rates of flows to each FIFO queue; \ie  as well as the - probability of each link, \ie . In the next section, by taking into account  and , we characterize the stability region of the wireless network.


\section{Stability Region} \label{sec:stability_region}
In this section, our goal is to characterize the stability region of a wireless network where an arbitrary number of FIFO queues are served by a wireless medium. We first begin with the single-queue case shown in Fig.~\ref{fig:FIFO_one_queue_fig} to convey our approach for a canonical scenario, then we extend our stability region analysis for arbitrary number of FIFO queues and flows.

\begin{figure}
\centering
\subfigure[FIFO queue]{ \label{fig:FIFO_one_queue_fig_a} \scalebox{.6}{\includegraphics{FIFO_one_queue_fig.eps}} } \hspace{5pt}
\subfigure[Stability region]{ \label{fig:FIFO_one_queue_fig_b} \scalebox{.6}{\includegraphics{FIFO_example_stareg_one_xmit_perflow_fifo.eps}} }
\vspace{-10pt}
\caption{(a) Single-FIFO queue;  is shared by  flows. (b) Stability region of a single-FIFO queue as well as per-flow queues with two flows.}
\label{fig:FIFO_one_queue_fig}
\vspace{-5pt}
\end{figure}

\subsection{Single-FIFO Queue} \label{sec:stability_single_queue}
We study the special case of a single FIFO queue  where  with . For this special case, we thus drop the queue index  from the notation in Section \ref{sec:system} for brevity. In other words, we write  instead of ,  instead of , and so on. Our main result in this context is then the following theorem.
\begin{theorem}\label{theorem1}
For a FIFO queue  shared by      flows, if the channel states  and arrival rates  are i.i.d. over time slots, the stability region  includes all arrival rates satisfying
 In other words, the stability region of the single-FIFO queue system is  (\ref{eq:stab_one_queue}).
\end{theorem}
{\em Proof:} The state of the FIFO queue  takes values from  depending the HOL packet and the states of the wireless links. Now, let us take a closer look at the FIFO states. The  state occurs if for some  we have  and . Let  be the state that  and . We denote the probability of  as . Also, let  be the state that FIFO queue is at  state for some HOL packet. The state  happens precisely when the channel corresponding to the HOL packet is in the  state. Therefore, the probability of  is .


Having defined the queue state probabilities, we can observe that the packets from the FIFO queue could be served only at state . It is also clear that the sum of the arrival rates to the queue  should be less than the service rate, which is . Noting that we assumed , we conclude that .

Let us now calculate  and  using a Markov chain with states;  and . We first show that the state transition probability from  to  is , where . Since we consider only one FIFO queue, when the queue is at state , the HOL packet is always transmitted. The new HOL packet in the next state will belong to the th flow with probability , and  with probability . Therefore, the state transition probability from  to  is , as claimed.

The probability of moving from state  to  is  as we can move to the unblocking state  from the blocking state  if the channel is  (with probability .). On the other hand, staying in the blocking state  is the  probability of the channel . Thus, . Note that the expressions for  and  do not involve the quantity . The reason is that  is the blocking state, so when we move from  to another state (or staying at state ), the HOL packet is not transmitted and does not change (because  at state ).

For any given  with , the state transition probability from  to  is . This follows since it is not possible to move from a blocking state to another (the HOL packet cannot be transmitted.). Finally, the probability of staying at state  is  as the condition  should be satisfied. The state transition probabilities are as shown in Fig.~\ref{fig:markov_chain_single_fifo}.

\begin{figure}
\vspace{5pt}
\centering
\scalebox{.45}{\includegraphics{single_queue_markov_chain.eps}}
\caption{Markov chain for the single-FIFO queue system shown in Fig~\ref{fig:FIFO_one_queue_fig_a}.}
\label{fig:markov_chain_single_fifo}
\vspace{-5pt}
\end{figure}

Now that we know the state transition probabilities of our Markov chain, we can calculate the balance equations, and these yield . The calculations are provided in the following. 

Let . In the steady the state, the following set of equations are satisfied for the Markov Chain shown in Fig.~\ref{fig:markov_chain_single_fifo}.  
 
If we combine the th equation in (\ref{eq:ststmatrix}), which is , and the fact that , we have


We can then obtain  which is equivalent to (\ref{eq:stab_one_queue}). This concludes the proof.
\hfill 

\begin{example}
Now suppose that single-FIFO queue  is shared by two flows with rates  and . According to Theorem~\ref{theorem1}, the arrival rates should satisfy  for stability. This stability region is shown in Fig.~\ref{fig:FIFO_one_queue_fig}(b). In the same figure, we also show the stability region of per-flow queues, \cite{neely_book}. As seen, the FIFO stability region is smaller as compared to per-flow capacity region. Yet, we still need flow control and scheduling algorithms to achieve the optimal operating point in this stability region. This issue will be discussed later in Section~\ref{sec:oFC_qFC}.
\hfill 
\end{example}

\subsection{Arbitrary Number of Queues and Flows}
We now consider a wireless network with arbitrary number of FIFO queues and flows as shown in Fig.~\ref{fig:main-example}. 
The main challenge in this setup is that packet scheduling decisions affect the stability region. 
For example, if both  and  in Fig.~\ref{fig:main-example} are at  state, a decision about which queue to be served should be made. This decision affects future transmission opportunities from the queues, hence the stability region.

In this paper, we consider a scheduling policy where the packet transmission probability of each queue depends only on the queue states. In other words, if the state of the FIFO queues is , a packet from queue  is transmitted with probability . We call this scheduling policy the {\em queue-state} policy.
Note that as     is the transmission probability from queue , we have the obvious constraint
 
Our main result is then the following theorem.
\begin{theorem} \label{theorem2}
For a wireless network with  FIFO queues, if a queue-state policy  is employed, then the stability region consists of the flow rates that satisfy
 
where 

\end{theorem}
{\em Proof:} The proof is provided in Appendix A. \hfill 



The stability region of a FIFO queue system with  FIFO queues served by a wireless medium is characterized by    (\ref{eq:lamdba_nk}), (\ref{eq:sum_tau})                 .


\begin{example}
Now let us consider two FIFO queues  and  which are shared by three flows with rates; , , and  (Fig.~\ref{fig:two_queue}(a)). According to Theorem~\ref{theorem2}, the stability region  should include arrival rates satisfying inequalities in (\ref{eq:lamdba_nk}) and (\ref{eq:sum_tau}). In this example, with two queues and three flows, these inequalities are equivalent to
 with    , and   . The stability region corresponding to these inequalities is the region below the surface  in Fig.~\ref{fig:two_queue}(b). \footnote{Note that the time sharing argument to convexify the stability region does not apply to this scenario, because the non-convexity comes from the relationship among the arrival rates instead of the service rates from the FIFO queues. Thus, the centralized time-sharing for the arrival rates is not practical.}
\hfill 
\end{example}

In general, we wish to find the optimal operating points on the boundary of the stability region . However, the stability region may not be convex for arbitrary number of queues and flows. Developing a convex inner bound on the stability region is crucial for developing efficient resource allocation algorithms for wireless networks with FIFO queues. We thus next propose a convex inner bound on the stability region.

\begin{figure}
\vspace{-10pt}
\centering
\subfigure[Two FIFO Queues]{ \label{fig:FIFO_two_queue_fig_a} \scalebox{.40}{\includegraphics{FIFO_two_queues_three_flows.eps}} }
\subfigure[Stability Region]{ \label{fig:FIFO_two_queue_fig_b} \scalebox{.28}{\includegraphics{stab_reg_two_queues_three_flows_v4.eps}} }
\vspace{-5pt}
\caption{(a) Two FIFO queues;  and  are shared by two and one flows, respectively. (b) Three dimensional stability region with ,  and  for the two-FIFO queues scenario shown in (a) when , , and .}
\label{fig:two_queue}
\vspace{-15pt}
\end{figure}


\subsection{A Convex Inner Bound on the Stability Region:} \label{sec:stability_single_innerBound}
Let us consider a flow with arrival rate  to the FIFO queue . If there are no other flows and queues in the network, then the arrival rate should satisfy  according to Theorem~\ref{theorem2}. In this formulation,  is the total amount of wireless resources that should be allocated to transmit  the flow with rate . For multiple-flow, single-FIFO case, the stability region is . Similar to the single-flow case,  term is the amount of wireless resources that should be allocated to the th flow. Finally, for the general stability region for arbitrary number of queues and flows, let us consider (\ref{eq:lamdba_nk}) again. Assuming , we can write  from (\ref{eq:lamdba_nk}) as;
 which, assuming that ,  is equivalent to
 Intuitively speaking, the right hand side of (\ref{eq:gec_v2}) corresponds to the amount of wireless resources that is allocated to the th queue . Thus, similar to the single-FIFO queue, we can consider that  term corresponds to the amount of wireless resources that should be allocated to the th flow.


Our key point while developing an inner bound on the stability region is to provide rate fairness across competing flows in each FIFO queue. Since each flow requires  amount of wireless resources; it is intuitive to have the following equality ,  to fairly allocate wireless resources across flows. More generally, we define a function ,  where , and we develop a stability region for  instead of . The role of the exponent  is to provide  flexibility to the targeted fairness. For example, if we want to allocate more resources to flows with better channels, then  should be larger. 

Now, by the definition of  , we have the equivalent form
 of (\ref{eq:lamdba_nk}), where . As seen, (\ref{eq:an}) is a convex function of . 
Thus, we can define the region  (\ref{eq:an}), (\ref{eq:sum_tau}),  , which is clearly an inner bound on the actual stability region . 
Despite the fact that  is only inner bound on , for some operating points, \ie at the intersection of  ,  lines, the two stability regions ( and ) coincide. Thus, for some utility functions, optimal operating points in both  and  coincide.
In the next section, we develop resource allocation schemes;  and  that achieve utility optimal operating points in .



\section{Flow Control and Scheduling} \label{sec:oFC_qFC}
In this section, we develop resource allocation schemes; {\em deterministic FIFO-Control} (), and a {\em queue-based FIFO control} (). 

In general, our goal is to solve the optimization problem
 and to find the corresponding optimal rates,  where  is a concave utility function assigned to flow with rate . 
Although the objective function  in (\ref{eq:main_opt}) is concave, the optimization domain  (\ie the stability region) may not be convex. Thus, we convert this problem to a convex optimization problem based on the structure of the inner bound we have developed in Section~\ref{sec:stability_single_innerBound}. 
In particular, setting , the problem in (\ref{eq:main_opt}) reduces to     , . This is our deterministic FIFO-control scheme;  and expressed explicitly as;

\underline{Deterministic FIFO-Control ():}
 Note that  optimizes  and . After the optimal values are determined, 
packets are inserted into the FIFO queue  depending on  and served from the FIFO queue  depending on . 

Although  gives us optimal operating points in the stability region; , it is a centralized solution, and its adaptation to varying wireless channel conditions is limited. Thus, we also develop a more practical and queue-based FIFO-control scheme , next. 

\underline{Queue-Based FIFO-Control ():}
\begin{itemize} 
\item {\em Flow Control:} At every slot , the flow controller attached to the FIFO queue  determines  according to; 
 where  is a large positive number, and  is a positive value larger than the maximum outgoing rate from FIFO queue  (which is  as we assume that the maximum outgoing rate from a queue is 1 packet per slot). After  is determined according to (\ref{eq:flow_control}),  is set as    . Then,  packets from the th flow are inserted in . 
\item {\em Scheduling:} At slot , the scheduling algorithm determines the FIFO queue from which a packet is transmitted according to;
 After  is determined, the outgoing traffic rate from queue  is set to       , and  packets (which is 1 or 0 in our case) are transmitted from . 
\end{itemize}

Thus, the queue dynamics change according to (\ref{eq:queue_Qn}) and based on (\ref{eq:flow_control}) and (\ref{eq:scheduling}). Such queue dynamics lead to the following result. 
\begin{theorem} \label{theorem_lyap}
If the channel states are i.i.d. over time slots, the traffic arrival rates are controlled by the rate control algorithm in (\ref{eq:flow_control}), and the FIFO queues are served by the scheduling algorithm in (\ref{eq:scheduling}), then the admitted flow rates converge to the utility optimal operating point in the stability region  with increasing .
\end{theorem}
{\em Proof:} The proof is provided in Appendix B.
\hfill 



\section{Performance Evaluation}\label{sec:performance}
In this section, we evaluate our  and  algorithms as compared to the baselines; (i) {\em optimal} solution, and (ii) {\em max-weight} algorithm for different number of FIFO queues and flows. Next, we briefly explain our baselines. 

\subsection{Baselines}
The {\em optimal} solution is a solution to (\ref{eq:main_opt}), and we compared  and  with the {\em optimal} solution for some scenarios where the stability region  is convex. On the other hand, {\em max-weight} algorithm is a queue-based flow control and max-weight scheduling scheme. Our baseline {\em max-weight} algorithm mimics the structure of the solution provided in \cite{neely_mod}, and it is summarized briefly in the following. 

\underline{{\em Max-weight} for FIFO:}
\begin{itemize} 
\item {\em Flow Control:} At every time slot , the flow controller attached to the FIFO queue  determines  according to;
 where  and  are positive large constants similar to (\ref{eq:flow_control}), and  is the number of packets that belong to the th flow in queue . 
\item {\em Scheduling:} At slot , the scheduling algorithm determines the FIFO queue from which a packet is transmitted according to; 
 After     is determined, a packet from the queue  is transmitted if      ; no packet is transmitted, otherwise. 
\end{itemize}
Next, we present our simulation results for single and multiple FIFO queues. 


\subsection{Single-FIFO Queue}
In this section, we consider  a single FIFO queue . Similar to Section~{\ref{sec:stability_single_queue}}, we drop the queue index  from the notation for brevity. In other words, we write  instead of ,  instead of , and so on. 

Fig.~\ref{fig:sim_1} presents simulation results for a single queue and two flows for , , and . Fig.~\ref{fig:sim_1}(a) shows per-flow rates;  and  when  is increasing. As seen,  is the same for all algorithms; optimal, , and . This also holds for . These results show that our algorithms  and  are as good as the optimal solution, and achieve the optimal operating points in  in this scenario. The simulations results also show that our algorithms reduce the second flow rate  when  increases while  and  do not change. This means that our algorithms do not penalize a flow (flow 1) when the channel of another competing flow (flow 2) deteriorates, which shows the effectiveness of our algorithms to provide fairness. 

Fig.~\ref{fig:sim_1}(b) shows the total rate  versus  for the same setup. As seen, our algorithms improves throughput over max-weight significantly. This is expected as our algorithms are designed to reduce the HOL blocking and to allocate wireless resources fairly among multiple flows. 
\begin{figure}
\centering
\subfigure[Per-flow rates]{ \scalebox{.45}{\includegraphics{one_queue_two_flows_dFC_qFC.eps}} } \hspace{-20pt}
\subfigure[Total rate]{ \scalebox{.45}{\includegraphics{one_queue_two_flows_dFC_qFC_mw.eps}} }
\vspace{-10pt}
\caption{Single-FIFO queue shared by two flows when , , and . (a) Per-flow rates vs. . (b) Total flow rate vs. . }
\label{fig:sim_1}
\vspace{-5pt}
\end{figure}

Fig.~\ref{fig:sim_2} shows simulation results for a single queue shared by multiple flows. In this setup,  is selected randomly between , , . The simulations are repeated for 1000 different seeds, and the average values are reported.  Fig.~\ref{fig:sim_2}(a) shows average flow rate versus number of flows for our algorithms as well as max-weight. As seen,  and  are as good as the optimal solution, and they improve over max-weight significantly. Fig.~\ref{fig:sim_2}(b) shows the same simulation results, but reports the improvement of  over max-weight. This figure shows that the improvement of our algorithms increases with increasing number of flows. Indeed, the improvement is up to 100\% when , which is significant. The improvement is higher for large number of flows, because our algorithm allocates resources to the flows based on the quality of their channels and reduces the flow rate for the flows with bad channel conditions. However, max-weight does not have such a mechanism, and when there are more flows in the system, the probability of having a flow with bad channel condition increases, which reduces the overall throughput.

\begin{figure}
\centering
\subfigure[Flow rates]{ \scalebox{.45}{\includegraphics{one_queue_average_rate_vs_number_of_flows.eps}} } \hspace{-20pt}
\subfigure[Throughput improvement]{ \scalebox{.45}{\includegraphics{one_queue_average_rate_improvement_vs_number_of_flows.eps}} }
\vspace{-10pt}
\caption{Single-FIFO queue shared by multiple flows.  is selected randomly between , , and . (a) Average flow rate versus number of flows. (b) Percentage of throughput improvement of  over max-weight. }
\label{fig:sim_2}
\vspace{-5pt}
\end{figure}

\subsection{Two-FIFO Queues}
In this section, we consider two FIFO queues  and . There are four flows in the system and each queue carries two flows, \ie  carries flows with rates ,  and  carries flows with rates , . 

Fig.~(\ref{fig:sim_3})(a) shows the total flow rate versus  for the scenario of two-FIFO queues with four flows when , , , , and  utility is employed, \ie . (We do not present the results of the optimal solution as the stability region  is not convex in this scenario.) As seen,  and  have the same performance and improve over max-weight. The improvement increases with increasing  as  and  penalize flows with bad channel conditions more when  increases, which increases the total throughput.   

Fig.~(\ref{fig:sim_3})(b) shows the total rate versus  for two-FIFO queues with four flows when  and . As seen,  and  improve significantly over max-weight. Furthermore, they achieve almost maximum achievable rate  all the time. The reason is that  and  penalizes the queues with with bad channels. For example, when , the total rate is , because they allocate all the resources to  and  as there is no point to allocate those resources to  and  since their channels are always . On the other hand, max-weight does not arrange the flow and queue service rates based on the channel conditions, so the total rate reduces to  when , \ie it is not possible to transmit any packets when max-weight is employed in this scenario.



\begin{figure}
\centering
\subfigure{\scalebox{.45}{\includegraphics{two_queues_four_flows_rate_vs_beta.eps}}}
\subfigure{\scalebox{.45}{\includegraphics{two_queues_four_flows_rate_vs_pn2_pm2.eps}}}
\vspace{-10pt}
\caption{Two FIFO queues with four flows. (a) Total flow rate versus  when , , , , and  utility is employed, \ie . (b) Total rate versus  when  and . }
\label{fig:sim_3}
\vspace{-5pt}
\end{figure}

Fig.~\ref{fig:sim_4} further demonstrates how our algorithms treat flows with bad channel conditions. In particular, Fig.~\ref{fig:sim_4} presents per-flow rate versus  for the scenario of two-FIFO queues with four flows when  and  for (a)  and  and (b) max-weight. As seen, when  increases,  decreases in Fig.~\ref{fig:sim_4}(a) since its channel is getting worse. Yet, this does not affect the other flows. In fact,  even increases as more resources are allocated to it when  increases. On the other hand, both  and  decrease with increasing  in max-weight (Fig.~\ref{fig:sim_4}(b)). This is not fair, because  decreases with increasing  although its channel is always  as . In the same scenario (Fig.~\ref{fig:sim_4}(b)), the rates of the th queue ( and ) increase with increasing  as they use available resource opportunistically. This makes the total rate the same for , , and max-weight. Yet, as we discussed, max-weight is not fair to flow  in this scenario. 



\begin{figure}
\centering
\subfigure[ and ]{ \scalebox{.45}{\includegraphics{two_queues_four_flows_rate_vs_pm2_oFC_qFC.eps}} } \hspace{-20pt}
\subfigure[Max-weight]{ \scalebox{.45}{\includegraphics{two_queues_four_flows_rate_vs_pm2_max_weight.eps}} }
\vspace{-10pt}
\caption{Per-flow rates versus  for the scenario of two-FIFO queues with four flows when  and . (a)  and . (b) Max-weight.}
\label{fig:sim_4}
\vspace{-5pt}
\end{figure}


\section{Related Work}\label{sec:related}
In this work, our goal is to understand FIFO queues in wireless networks and develop efficient flow control and scheduling policies for such a setup. In the seminal paper \cite{Karol87}, the authors analyze FIFO queues in an input queued switch. They show that the use of FIFO queues in that context limits the throughput to approximately 58\% of the maximum achievable throughput. However, in the context of wireless networks, similar results are in general not known.

Backpressure routing and scheduling framework has emer-ged from the pioneering work \cite{tass_eph1,tass_eph2}, which has generated a lot of research interest \cite{neely_book}; especially for wireless ad-hoc networks \cite{tass3,kahale,andrews,neely_mod_pow,stolyar_greedy,liu_stolyar}. Furthermore, it has been shown that backpressure can be combined with flow control to provide utility-optimal operation guarantee \cite{neely_mod,stolyar_greedy}. Such previous work mainly considered per-flow queues. However, FIFO queueing structure, which is the focus of this paper, is not compatible with the per-flow queueing requirements of these routing and scheduling schemes.

The strengths of backpressure-based network control have recently received increasing interest in terms of practical implementation. Multi-path TCP scheme is implemented over wireless mesh networks in \cite{horizon} for routing and scheduling packets using a backpressure based heuristic. At the link layer, \cite{DiffQ,umut_stolyar,sridharan2} propose, analyze, and evaluate link layer backpressure-based implementations with queue prioritization and congestion window size adjustment. Backpressure is implemented over sensor networks \cite{routing_wtht_routes} and wireless multi-hop networks \cite{xpress}. In these schemes, either last-in, first-out queueing is employed \cite{routing_wtht_routes} or link layer FIFO queues are strictly controlled \cite{xpress} to reduce the number of packets in the FIFO queues, hence HOL blocking.

In backpressure, each node constructs per-flow queues. There is some work in the literature to stretch this necessity. For example, \cite{pkt_by_pkt_adap_rout}, \cite{locbui} propose using real per-link and virtual per-flow queues. Such a method reduces the number of queues required in each node, and reduces the delay, but it still needs to construct per-link queues. Similarly, \cite{diffmax} constructs per-link queues in the link layer, and schedule packets according to FIFO rule from these queues. Such a setup is different than ours as per-link queues do not introduce HOL blocking. 

The main differences in our work are: (i) we consider FIFO queues shared by multiple flows where HOL blocking occurs as each flow is transmitted over a possibly different wireless link, (ii) we characterize the stability region of a general scenario where an arbitrary number of FIFO queues, which are served by a wireless medium, are shared by an arbitrary number of flows, and (iii) we develop efficient resource allocation schemes to exploit achievable rate in such a setup. 


\section{Conclusion}\label{sec:conclusion}
We investigated the performance of FIFO queues over wireless networks and characterized the stability region of this system for arbitrary number of FIFO queues and flows. We developed inner bound on the stability region, and developed resource allocation schemes;  and , which achieve optimal operating point in the convex inner bound. Simulation results show that our algorithms significantly improve throughput in a wireless network with FIFO queues as compared to the well-known queue-based flow control and max-weight scheduling schemes.


\bibliographystyle{IEEEtran}
\begin{thebibliography}{}

\bibitem{cisco_index} Cisco Visual Networking Index: Global Mobile Data Traffic Forecast Update, 2010 - 2015.

\bibitem{ericsson_report} Ericsson Mobility Report, November 2013.

\bibitem{tass_eph1} L. Tassiulas and A. Ephremides, ``Stability properties of constrained queueing systems and scheduling policies for maximum throughput in multihop radio networks,'' {\em IEEE Trans. Autom. Control}, vol. 37, no. 12, pp. 1936--1948, Dec. 1992.

\bibitem{tass_eph2} L. Tassiulas and A. Ephremides, ``Dynamic server allocation to parallel queues with randomly varying connectivity,'' {\em IEEE Trans. Inf. Theory}, vol. 39, no. 2, pp. 466--478, Mar. 1993.

\bibitem{neely_mod} M. J. Neely, E. Modiano, and C. Li, ``Fairness and optimal stochastic control for heterogeneous networks,'' {\em IEEE/ACM Trans. Net.}, vol. 16, no. 2, pp. 396--409, Apr. 2008.

\bibitem{madwifi_chipsets} http://madwifi-project.org/wiki/Chipsets. 

\bibitem{microcast} L.~Keller, A.~Le, B.~Cici, H.~Seferoglu, C.~Fragouli, A.~Markopoulou, ``MicroCast: Cooperative Video Streaming on Smartphones,'' {\em ACM MobiSys}, June 2012.



\bibitem{Karol87} M. J. Karol, M. G. Hluchyj, and S. P. Morgan, ``Input versus output queueing on a space-division packet switch,'' {\em IEEE Trans. Commun.}, vol. 35, no. 12, pp. 1347--1356, Dec. 1987.

\bibitem{neely_book} M. J. Neely, \emph{Stochastic Network Optimization with Application to Communication and Queueing Systems}, Morgan \& Claypool, 2010.

\bibitem{tass3} L. Tassiulas, ``Scheduling and performance limits of networks with constantly changing topology,'' {\em IEEE Trans. Inf. Theory}, vol. 43, no. 3, pp. 1067--1073, May 1997.

\bibitem{kahale} N. Kahale and P. E. Wright, ``Dynamic global packet routing in wireless networks,'' {\em IEEE INFOCOM}, Apr. 1997.

\bibitem{andrews} M. Andrews, K. Kumaran, K. Ramanan, A. Stolyar, P. Whiting, and R. Vijaykumar, ``Providing quality of service over a shared wireless link,'' {\em IEEE Commun. Mag.}, vol. 39, no. 2, pp. 150--154, Feb. 2001.

\bibitem{neely_mod_pow} M. J. Neely, E. Modiano, and C. E. Rohrs, ``Dynamic power allocation and routing for time varying wireless networks,'' {\em IEEE J. Select. Areas Commun.}, vol. 23, no. 1, pp. 89--103, Jan. 2005.

\bibitem{stolyar_greedy} A. L. Stolyar, ``Greedy primal dual algorithm for dynamic resource allocation in complex networks,'' {\em Queuing Systems}, vol. 54, no. 3, pp. 203--220, 2006.

\bibitem{liu_stolyar} J. Liu, A. L. Stolyar, M. Chiang, and H. V. Poor, ``Queue backpressure random access in multihop wireless networks: optimality and stability,'' {\em IEEE Trans. Inf. Theory}, vol. 55, no. 9, pp. 4087--4098, Sept. 2009.

\bibitem{horizon} B. Radunovic, C. Gkantsidis, D. Gunawardena, and P. Key, ``Horizon: balancing TCP over multiple paths in wireless mesh network,'' {\em ACM MobiCom}, Sept. 2008.

\bibitem{DiffQ} A. Warrier, S. Janakiraman, S. Ha, I. Rhee, ``DiffQ: practical differential backlog congestion control for wireless networks,'' {\em IEEE INFOCOM}, Apr. 2009.

\bibitem{umut_stolyar} U. Akyol, M. Andrews, P. Gupta, J. Hobby, I. Saniee, and A. Stolyar, ``Joint scheduling and congestion control in mobile ad-hoc networks,'' {\em IEEE INFOCOM}, Apr. 2008.

\bibitem{sridharan2} A. Sridharan, S. Moeller, B. Krishnamachari, ``Making distributed rate control using Lyapunov drifts a reality in wireless sensor networks,'' {\em IEEE WiOpt}, Apr. 2008.

\bibitem{routing_wtht_routes} S. Moeller, A. Sridharan, B. Krishnamachari, and O. Gnawali, ``Routing without routes: the backpressure collection protocol,'' {\em ACM IPSN}, Apr. 2010.

\bibitem{xpress} R. Laufer, T. Salonidis, H. Lundgren, and P. L. Guyadec, ``XPRESS: a cross-layer backpressure architecture for wireless multi-hop networks,'' {\em ACM MobiCom}, Sept. 2011.
    
\bibitem{pkt_by_pkt_adap_rout} E.~Athanasopoulou, L.~X.~Bui, T.~Ji, R.~Srikant, and A.~Stolyar, ``Backpressure-based packet-by-packet adaptive routing in communication networks,'' {\em IEEE/ACM Trans. Net.}, vol. 21, no. 1, pp. 244--257, Feb. 2013.

\bibitem{locbui} L.~X.~Bui, R.~Srikant, and A.~Stolyar, ``A novel architecture for reduction of delay and queueing structure complexity in the back-pressure algorithm,'' {\em IEEE/ACM Trans. Net.}, vol. 19, no. 6, pp. 1597--1609, Dec. 2011.

\bibitem{diffmax} H.~Seferoglu, E.~Modiano, ``Diff-Max: Separation of Routing and Scheduling in Backpressure-Based Wireless Networks,'' {\em IEEE INFOCOM}, Apr. 2013.

\end{thebibliography}



\section*{Appendix A: Proof of Theorem~\ref{theorem2}}
In this section, we provide a proof of Theorem~\ref{theorem2} for arbitrary number of FIFO queues and flows. 
Let us first consider , which should satisfy the following inequality. 
 where  is the probability that the states of the queues are  and , which is required as we can transmit a packet from the th flow only when the HOL packet belongs to the th flow. 
In this equation, we can calculate  as 
 
 Thus, we have
 Now, we should calculate . 

We claim that                 . To prove this claim, we should show, without loosing generality, that the following conditions hold. 
 We can calculate the conditional probabilities in the left hand side of the conditions; C1, C2, , CN in (\ref{eq:app_cond_prob}) by using a Markov chain. For C1, we can write a state transition probability of going from state  to  as    , which is equal to . \Ie     . Similarly, if we write the state transition probabilities for the other conditions C2, , CN, we have             . Therefore, in all Markov chains we can create for C1, C2, , CN, we have the same transition probabilities, so we have       . This proves our claim that           . 

Now that we have shown that            holds, (\ref{eq:states_v1}) is expressed as
 
which leads to 
 Now, we should calculate  in (\ref{eq:states_v4}). The state transition diagram for the states ,  and for the th queue is shown in Fig.~\ref{fig:app_Km_flows}. 
\begin{figure}
\vspace{5pt}
\centering
\scalebox{.45}{\includegraphics{markov_chain_Km_flows.eps}} 
\vspace{-5pt}
\caption{The state transition diagram for the states ,  and for the th queue. Note that this state transition diagram only shows a subset of state transitions for clarity.}
\label{fig:app_Km_flows}
\vspace{-5pt}
\end{figure}
We can write the global balance equations for the state  as 
 which is expressed as
 which leads to 
 Similarly, the global balance equations for state  leads to 
 From (\ref{eq:app_state_mk}) and (\ref{eq:app_state_ml}), we have 
 Thus, we have 
 Since  should be satisfied, we have 
 When (\ref{eq:app_prob_Hm_k}) is substituted in (\ref{eq:states_v4}), we have 
 Since , we have 
 When we substitute (\ref{eq:states_v6}) into (\ref{eq:appA_1}), we have (\ref{eq:lamdba_nk}). This concludes the proof. 



\section*{Appendix B: Proof of Theorem~\ref{theorem_lyap}}
Let define a Lyapunov function as; , and the Lyapunov drift as; , where . Then, the Lyapunov drift is expressed as; 
  

Note that we have, from Eq.~(\ref{eq:queue_Qn}) and the assumption  that,
 Using Eq.~(\ref{eq:appB_queue_Qn}) in Eq.~(\ref{eq:appB_drift1}), and using the fact that , we have 
 which is expressed as
 There always exist a finite and positive  satisfying; . Thus, Eq.~(\ref{eq:appB_drift2}) is expressed as; 
 Note that if the flow arrival rates  are inside the capacity region , then the minimizing the right hand side of the drift inequality in Eq.~(\ref{eq:appB_drift3}) corresponds to the scheduling part of  in Eq.~(\ref{eq:scheduling}). 

Now, let us consider again the stability region constraint in Eq.~(\ref{eq:appA_1}), which is   , and expressed as; 
 which is equal to
 Since , we have


  Let . Then, Eq.~(\ref{eq:appB_gec1}) is expressed as; 
  There exists a small positive value  satisfying
  Thus, we can find a randomized policy satisfying 


Now, let us consider Eq.~(\ref{eq:appB_drift3}) again, which is expressed as; 
 We minimize the right hand side of Eq.~(\ref{eq:appB_drift3}), so the following inequality satisfies; 
 where  and  are the solutions of a randomized policy. Incorporating Eq.~(\ref{eq:appB_randEp}) in Eq.~(\ref{eq:appB_drift4}), we have
 The time average of Eq.~(\ref{eq:appB_drift5}) leads to
 
 This concludes that the time average of the queues are bounded if the arrival rates are inside the capacity region . 

Now, let us focus on the original claim of Theorem~\ref{theorem2}. Let us consider a drift+penalty function as; 
 Since we set , we have
 Note that minimizing the right hand side of Eq.~(\ref{eq:appB_dpp1}) corresponds to the flow control and scheduling algorithms of  in Eq.~(\ref{eq:flow_control}) and Eq.~(\ref{eq:scheduling}), respectively. Since there exists a randomized policy satisfying Eq.~(\ref{eq:appB_randEp}), Eq.~(\ref{eq:appB_dpp1}) is expressed as
 where  is the maximum time average of the sum utility function that can be achieved by any control policy that stabilizes the system. Then, the time average of Eq.~(\ref{eq:appB_dpp2}) becomes 
 Now, let us first consider the stability of the queues. If both sides of Eq.~(\ref{eq:appB_dpp3}) is divided by  and the terms are arranged, we have
 Since the right hand side is a positive finite value, this concludes that the time averages of the total queue sizes are bounded. 

Now, let us consider the optimality. If both sides of Eq.~(\ref{eq:appB_dpp3}) are divided by , we have
 By arranging the terms, we have 
 Since  is positive for any , we have
 which leads to
 This proves that the admitted flow rates converge to the utility optimal operating point with increasing . This concludes the proof. 


\end{document}
