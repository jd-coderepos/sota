\documentclass[10pt]{article}
\usepackage{hyperref}
\usepackage{color,graphicx}
\pagestyle{plain}

\begin{document}

 
\begin{center}

\textbf{\LARGE On arc-disjoint Hamiltonian cycles in De Bruijn graphs}

\bigskip{\large \href{http://www.ms.sapientia.ro/~kasa}{Z. K\'asa}}

\href{http://www.emte.ro}{Sapientia Hungarian University of Transylvania}

email: \texttt{kasa@ms.sapientia.ro}

\bigskip \emph{Presented at:} \\
\href{http://riesz.math.klte.hu/~macs/abstracts.pdf}{5th Joint Conference on Mathematics and Computer Science, Debrecen, June 9-12, 2004} \\
 \emph{and  published in Hungarian:}\\ De Bruijn-gr\'afok mint h\'al\'azati modellek (De Bruijn Graphs as Network Model), \href{http://www.emt.ro/downloads/muszaki_szemle/msz43.pdf}{M\H{u}szaki Szemle/Technical Review, 43 (2008) 3-6} 
\end{center}

\bigskip
{\small
\noindent\textbf{Abstract.} We give two equivalent formulations of a conjecture [2,4] on the number of arc-disjoint Hamiltonian cycles in De Bruijn graphs. 
}

\bigskip\noindent
A De Bruijn word  of type  for a given  and  is a word over an alphabet with  letters, containing all  -length words exactly once. The length of such a word is . 
For example if , then  0012202110 is a De Bruijn word of type .

\medskip\noindent
For a -letter alphabet  the De Bruijn graph  is defined as:\\

with \\
 {}   \quad the set of vertices
\\
 {}  the set of directed arcs
\\
 there is an arc from vertex {} to  vertex {}  if 
{} 
and this arc is denoted by {}. 
      
\medskip\noindent
In the De Bruijn graph  a path (i. e. a walk with distinct vertices)
  \   \  
\ldots \  \   
corresponds to an -length word 
 {},
 which 
is obtained by maximal overlapping of the neighboring vertices. 

\begin{figure}[t]
\centering\includegraphics[scale=0.6]{fig1.pdf}
\label{br22}\caption{De Bruijn graph } 
\end{figure}



\medskip\noindent
In  the path {, , } corresponds to the 
word {.}  

\medskip\noindent
Every maximal length path in the graph  (which is a
Hamiltonian one) corresponds to a De Bruijn word.      


\bigskip\noindent{\textcolor{red}{Algorithm to generate all De Bruijn words}}
\\
 the letters of an alphabet, with the values:  for , 
 a vector to store the letters of a De Bruijn word, 
 a vector to store the states of words.   represents the value of the number  in base  . 

\begin{figure}[t]
\centering\includegraphics[scale=0.6]{fig2.pdf}
\label{br23}\caption{De Bruijn graph } 
\end{figure}

\medskip\noindent Initially:  for all , 
 and  for other indices. 
The call of the procedure: DeBruijn (S,k+1,k,B).

\begin{tabbing}
\textbf{{procedure}} DeBruijn ()\\
\textbf{{for}} \=  \textbf{{to}}  \textbf{{do}}\\
             \> \\
             \>  \\
             \> \textbf{{if}}  \=\textbf{{then}} \= \\
             \>                     \>              \> DeBruijn()\\
             \>                     \>              \> \\
             \>                     \>\textbf{{else}} \= \textbf{{if}}  \=         \\ 
            \> \> \> \> \textbf{{then}} \= write (S)\\
             \> \> \> \>                                           \>  exit for\\
             \>         \>            \>               \> \textbf{{endif}}  \\          
             \> \textbf{{endif}}\\
\textbf{{endfor}} \\
\textbf{{endprocedure}}            
\end{tabbing}

\begin{figure}[t]
\centering\includegraphics[scale=0.8]{fig3.pdf}
\label{br32}\caption{De Bruijn graph } 
\end{figure}



\medskip\noindent 
In the directed graph  there always exists an Eulerian circuit because
it is connected and  all its
vertices have the same indegree and outdegree . An Eulerian
circuit in   is a Hamiltonian path in  (which always can
be continued in a Hamiltonian cycle).		 



\medskip\noindent\textbf{{Conjecture 1.}} [2, 4]

\noindent\textcolor{red}{In the De Bruijn graph  for  and  the number of arc-disjoint Hamiltonian cycles is .}

\medskip\noindent
Let us define a morphism  on words over an alphabet :

It is easy to see that  for any .

\medskip\noindent\textbf{Example:} 

Let be a word  on the alphabet .

 



\medskip\noindent
From a De Bruijn word we obtian a Hamiltonian cycle. Let an  be such a Hamiltonian cycle. Using  for  we will obtain De Bruijn words corresponding to the Hamiltonian cycles . 


 
\medskip\noindent\textbf{{Conjecture 2.}}

\noindent\textcolor{red}{In the De Bruijn graph  for {} there exists a Hamiltonian cycle {} such that the Hamiltonian cycles  , , \ldots  (obtained from {} by using the morphisms , )), together with  are  arc-disjoint Hamiltonian cycles.}


\medskip\noindent 
{}\\
 {0011220210}, \\   {0022110120}\\
{} \\
 {00010021011022202012111221200}, \\
 {00020012022011101021222112100}
\\
{}\\   {00102113230331220}, \\ 
 {00203221310112330}, \\
 {00301332120223110}  
\\
{} \\
  {00102112041422430332313440},\\  
 {00203223012133140443424110},  \\
  {00304334023244210114131220}, \\
  {00401441034311320221242330}


\medskip\noindent
Two words  and  are equivalent if  for some  

\medskip\noindent 
If , , then  because, applying  to  we obtain


\medskip\noindent
In the set of all De Bruijn words over an alphabet  this equivalence relation will introduce a partition of De Bruijn words in equivalent classes.


\medskip\noindent
\textbf{{An equivalent assertion to Conjecture 2.}}

\medskip\noindent\textcolor{red}{In the De Bruijn graph {} for {} there exist the Hamiltonian cycles  , , , \ldots  such that they correspond to De Bruijn words , , , \ldots  from the same equivalence class.}


\begin{figure}
\centering\includegraphics[scale=0.5]{fig4.pdf}
\caption{Arc-disjoint Hamiltoniain cycles in De Bruijn graph }
\end{figure}

\newpage\section*{References}
          
\quad\ 1. M. C. Anisiu, Z. Bl\'azsik. Z. K\'asa, \textit{Maximal complexity of finite words}, 
Pure Math. and Appl. Vol. \textbf{13} (2002) No. 1--2, pp. 39--48.

\smallskip
      2. J. Bond, A. Iv\'anyi,  \textit{Modelling of interconnection
networks using De Bruijn graphs}, Third Conference of Program Designer, 
Ed. A. Iv\'anyi, Budapest, 1987,  75--87. 
{\scriptsize \url{http://www.acta.sapientia.ro/PD/Third-conference-of-program-designers.pdf}}

\smallskip
      3. N. G. de Bruijn, \textit{A combinatorial problem}, 
Nederl. Akad. Wetensch. Proc. 49 (1946), 758--764.

\smallskip
      4. Sophie Gire, 
\emph{R\'eseaux d'interconnexion},  Option Ma\^{\i}trise Informatique, 1996--97,
{\scriptsize {http://fastnet.univ-brest.fr/gire/COURS/OPTIONRESEAUX/node48.html}}

\smallskip
      5. A. Iv\'anyi, M. Horv\'ath, \textit{Perfect sequences}, ICAI'2004, Eger, January 28, 2004.

\smallskip
      6. M. Lothaire, \textit{Combinatorics on words}, 
Addison-Wesley, Reading, MA, 1983.

\smallskip
      7. M. H. Martin, \textit{A problem in arrangements}, 
Bull. A.M.S. 40 (1934), 859--864.


\end{document}
