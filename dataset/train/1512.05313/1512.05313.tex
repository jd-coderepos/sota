\documentclass{CSML}

\def\dOi{11(4:22)2015}
\lmcsheading {\dOi}
{1--43}
{}
{}
{Apr.~15, 2015}
{Dec.~31, 2015}
{}

\ACMCCS{[{\bf Theory of computation}]: Logic---Constructive mathematics}
\keywords{Classical realizability, Lambda-mu calculus, Bar-recursion, Axiom of
choice}

\usepackage[T2A,T1]{fontenc}
\usepackage{amsmath}
\renewcommand{\ldots}{...}
\usepackage{amsthm}
\newtheorem{definition}{Definition}
\newtheorem*{definition*}{Definition}
\newtheorem{lemma}{Lemma}
\newtheorem*{lemma*}{Lemma}
\newtheorem{theorem}{Theorem}
\newtheorem*{theorem*}{Theorem}
\usepackage{amssymb}
\usepackage{cmll}
\usepackage{bm}
\usepackage{stmaryrd}
\usepackage[all]{xy}
\usepackage{bussproofs}
\usepackage{expl3}
\usepackage{hyperref}
\ExplSyntaxOn
\cs_generate_variant:Nn\tl_if_empty:nTF{f}
\newcommand*\ifpresent[3]{\tl_if_empty:fTF{#1}{#3}{#2}}
\ExplSyntaxOff
\newcommand*\AXM[1]{\AxiomC{}}
\newcommand*\UIM[1]{\UnaryInfC{}}
\newcommand*\BIM[1]{\BinaryInfC{}}
\newcommand*\TIM[1]{\TrinaryInfC{}}
\newcommand*\RLM[1]{\RightLabel{}}
\newcommand*\DP\DisplayProof
\newcommand*\Def{\mathrel{\overset{\Delta}{=}}}
\newcommand*\GramDef{\quad\mathrel{::=}\quad}
\newcommand*\Entails{\mathrel{\vdash}}
\newcommand*\Derives{\mathrel{|\mkern-8.75mu\sim}}
\newcommand*\BarSep{\mathrel{|}}
\newcommand*\Sequent[3]{#1\Entails#2\ifpresent{#3}{\BarSep}{}#3}
\newcommand*\FV[1]{\text{FV}\left(#1\right)}
\newcommand*\SetSuch[2]{{\left\{#1\ifpresent{#2}{\!\;\middle|\;#2}{}\right\}}}
\newcommand*\SortBase\iota
\newcommand*\SortTo\to
\newcommand*\omicron{o}
\newcommand*\SortA{\sigma}
\newcommand*\SortB{\tau}
\newcommand*\SortC{\nu}
\newcommand*\SortD{\rho}
\newcommand*\LogSortedTerm[2]{#1^{#2}}
\newcommand*\LogTermA{t}
\newcommand*\LogTermB{u}
\newcommand*\LogTermC{v}
\newcommand*\LogTermD{w}
\newcommand*\LogVarA{x}
\newcommand*\LogVarB{y}
\newcommand*\LogVarC{z}
\newcommand*\LogVarD{u}
\newcommand*\LogVarE{v}
\newcommand*\LogVarF{w}
\newcommand*\LogConst[1]{\mathsf{#1}}
\newcommand*\LogConstA{\LogConst{c}}
\newcommand*\LogConstB{\LogConst{d}}
\newcommand*\LogConstC{\LogConst{e}}
\newcommand*\LogConstD{\LogConst{f}}
\newcommand*\LogNeg[1]{{#1^-}}
\newcommand*\LogPos[1]{{#1^+}}
\newcommand*\LogImp{\mathbin{\Rightarrow}}
\newcommand*\LogAnd{\mathbin{\wedge}}
\newcommand*\LogBot\bot
\newcommand*\LogRel[1]{\llparenthesis#1\rrparenthesis}
\newcommand*\LogRelForm[1]{{#1^\mathrm{r}}}
\newcommand*\LogForallRel{\forall^\mathrm{r}}
\newcommand*\LogExistsRel{\exists^\mathrm{r}}
\newcommand*\LogFormA{A}
\newcommand*\LogFormB{B}
\newcommand*\LogFormC{C}
\newcommand*\LogFormD{D}
\newcommand*\LogAxioms{\mathcal{A}x}
\newcommand*\LogPredA{P}
\newcommand*\LogPredB{Q}
\newcommand*\LogPredC{R}
\newcommand*\LogPredD{S}
\newcommand*\LogProofA{\mathfrak{p}}
\newcommand*\LogProofB{\mathfrak{q}}
\newcommand*\LogProofC{\mathfrak{r}}
\newcommand*\LogProofD{\mathfrak{s}}
\newcommand*\LogSubst[1]{\left\{#1\right\}}
\newcommand*\LogRuleAxConcl[3]{\Sequent{#1\ifpresent{#1}{,}{}#3}{#3}{#2}}
\newcommand*\LogRuleAx[3]{\AXM{}\UIM{\LogRuleAxConcl{#1}{#2}{#3}}\DP}
\newcommand*\LogRuleAxSigConcl[3]{\Sequent{#1}{#3}{#2}}
\newcommand*\LogRuleAxSig[3]{\AXM{}\RLM{\left(#3\in\LogAxioms\right)}\UIM{\LogRuleAxSigConcl{#1}{#2}{#3}}\DP}
\newcommand*\LogRuleImpIntroFirst[4]{\Sequent{#1\ifpresent{#1}{,}{}#3}{#4}{#2}}
\newcommand*\LogRuleImpIntroConcl[4]{\Sequent{#1}{#3\LogImp#4}{#2}}
\newcommand*\LogRuleImpIntro[4]{\AXM{\LogRuleImpIntroFirst{#1}{#2}{#3}{#4}}\UIM{\LogRuleImpIntroConcl{#1}{#2}{#3}{#4}}\DP}
\newcommand*\LogRuleImpIntroTree[5]{\AXM{#1}\UIM{\LogRuleImpIntroFirst{#2}{#3}{#4}{#5}}\UIM{\LogRuleImpIntroConcl{#2}{#3}{#4}{#5}}\DP}
\newcommand*\LogRuleImpElimFirst[4]{\Sequent{#1}{#3\LogImp#4}{#2}}
\newcommand*\LogRuleImpElimSecond[4]{\Sequent{#1}{#3}{#2}}
\newcommand*\LogRuleImpElimConcl[4]{\Sequent{#1}{#4}{#2}}
\newcommand*\LogRuleImpElim[4]{\AXM{\LogRuleImpElimFirst{#1}{#2}{#3}{#4}}\AXM{\LogRuleImpElimSecond{#1}{#2}{#3}{#4}}\BIM{\LogRuleImpElimConcl{#1}{#2}{#3}{#4}}\DP}
\newcommand*\LogRuleImpElimTree[6]{\AXM{#1}\UIM{\LogRuleImpElimFirst{#3}{#4}{#5}{#6}}\AXM{#2}\UIM{\LogRuleImpElimSecond{#3}{#4}{#5}{#6}}\BIM{\LogRuleImpElimConcl{#3}{#4}{#5}{#6}}\DP}
\newcommand*\LogRuleAndIntroFirst[4]{\Sequent{#1}{#3}{#2}}
\newcommand*\LogRuleAndIntroSecond[4]{\Sequent{#1}{#4}{#2}}
\newcommand*\LogRuleAndIntroConcl[4]{\Sequent{#1}{#3\LogAnd#4}{#2}}
\newcommand*\LogRuleAndIntro[4]{\AXM{\LogRuleAndIntroFirst{#1}{#2}{#3}{#4}}\AXM{\LogRuleAndIntroSecond{#1}{#2}{#3}{#4}}\BIM{\LogRuleAndIntroConcl{#1}{#2}{#3}{#4}}\DP}
\newcommand*\LogRuleAndIntroTree[6]{\AXM{#1}\UIM{\LogRuleAndIntroFirst{#3}{#4}{#5}{#6}}\AXM{#2}\UIM{\LogRuleAndIntroSecond{#3}{#4}{#5}{#6}}\BIM{\LogRuleAndIntroConcl{#3}{#4}{#5}{#6}}\DP}
\newcommand*\LogRuleAndElimFirst[3]{\Sequent{#1}{#3_1\LogAnd#3_2}{#2}}
\newcommand*\LogRuleAndElimConcl[3]{\Sequent{#1}{#3_i}{#2}}
\newcommand*\LogRuleAndElim[3]{\AXM{\LogRuleAndElimFirst{#1}{#2}{#3}}\UIM{\LogRuleAndElimConcl{#1}{#2}{#3}}\DP}
\newcommand*\LogRuleAndElimTree[4]{\AXM{#1}\UIM{\LogRuleAndElimFirst{#2}{#3}{#4}}\UIM{\LogRuleAndElimConcl{#2}{#3}{#4}}\DP}
\newcommand*\LogRuleAndElimLFirst[4]{\Sequent{#1}{#3\LogAnd#4}{#2}}
\newcommand*\LogRuleAndElimLConcl[4]{\Sequent{#1}{#3}{#2}}
\newcommand*\LogRuleAndElimL[4]{\AXM{\LogRuleAndElimLFirst{#1}{#2}{#3}{#4}}\UIM{\LogRuleAndElimLConcl{#1}{#2}{#3}{#4}}\DP}
\newcommand*\LogRuleAndElimLTree[5]{\AXM{#1}\UIM{\LogRuleAndElimLFirst{#2}{#3}{#4}{#5}}\UIM{\LogRuleAndElimLConcl{#2}{#3}{#4}{#5}}\DP}
\newcommand*\LogRuleAndElimRFirst[4]{\Sequent{#1}{#3\LogAnd#4}{#2}}
\newcommand*\LogRuleAndElimRConcl[4]{\Sequent{#1}{#4}{#2}}
\newcommand*\LogRuleAndElimR[4]{\AXM{\LogRuleAndElimRFirst{#1}{#2}{#3}{#4}}\UIM{\LogRuleAndElimRConcl{#1}{#2}{#3}{#4}}\DP}
\newcommand*\LogRuleAndElimRTree[5]{\AXM{#1}\UIM{\LogRuleAndElimRFirst{#2}{#3}{#4}{#5}}\UIM{\LogRuleAndElimRConcl{#2}{#3}{#4}{#5}}\DP}
\newcommand*\LogRuleForallIntroFirst[4]{\Sequent{#1}{#4}{#2}}
\newcommand*\LogRuleForallIntroConcl[4]{\Sequent{#1}{\forall#3#4}{#2}}
\newcommand*\LogRuleForallIntro[4]{\AXM{\LogRuleForallIntroFirst{#1}{#2}{#3}{#4}}\RLM{\left(#3\notin\FV{#1,#2}\right)}\UIM{\LogRuleForallIntroConcl{#1}{#2}{#3}{#4}}\DP}
\newcommand*\LogRuleForallIntroTree[5]{\AXM{#1}\UIM{\LogRuleForallIntroFirst{#2}{#3}{#4}{#5}}\RLM{\left(#4\notin\FV{#2,#3}\right)}\UIM{\LogRuleForallIntroConcl{#2}{#3}{#4}{#5}}\DP}
\newcommand*\LogRuleForallElimFirst[5]{\Sequent{#1}{\forall#3#5}{#2}}
\newcommand*\LogRuleForallElimConcl[5]{\Sequent{#1}{#5\LogSubst{#4/#3}}{#2}}
\newcommand*\LogRuleForallElim[5]{\AXM{\LogRuleForallElimFirst{#1}{#2}{#3}{#4}{#5}}\UIM{\LogRuleForallElimConcl{#1}{#2}{#3}{#4}{#5}}\DP}
\newcommand*\LogRuleForallElimTree[6]{\AXM{#1}\UIM{\LogRuleForallElimFirst{#2}{#3}{#4}{#5}{#6}}\UIM{\LogRuleForallElimConcl{#2}{#3}{#4}{#5}{#6}}\DP}
\newcommand*\LogRuleBotIntroFirst[3]{\Sequent{#1}{#3}{#3\ifpresent{#2}{,}{}#2}}
\newcommand*\LogRuleBotIntroConcl[3]{\Sequent{#1}{\LogBot}{#3\ifpresent{#2}{,}{}#2}}
\newcommand*\LogRuleBotIntro[3]{\AXM{\LogRuleBotIntroFirst{#1}{#2}{#3}}\UIM{\LogRuleBotIntroConcl{#1}{#2}{#3}}\DP}
\newcommand*\LogRuleBotIntroTree[4]{\AXM{#1}\UIM{\LogRuleBotIntroFirst{#2}{#3}{#4}}\UIM{\LogRuleBotIntroConcl{#2}{#3}{#4}}\DP}
\newcommand*\LogRuleBotElimFirst[3]{\Sequent{#1}{\LogBot}{#3\ifpresent{#2}{,}{}#2}}
\newcommand*\LogRuleBotElimConcl[3]{\Sequent{#1}{#3}{#2}}
\newcommand*\LogRuleBotElim[3]{\AXM{\LogRuleBotElimFirst{#1}{#2}{#3}}\UIM{\LogRuleBotElimConcl{#1}{#2}{#3}}\DP}
\newcommand*\LogRuleBotElimTree[4]{\AXM{#1}\UIM{\LogRuleBotElimFirst{#2}{#3}{#4}}\UIM{\LogRuleBotElimConcl{#2}{#3}{#4}}\DP}
\newcommand*\LogRuleForallRelIntroFirst[4]{\Sequent{#1\ifpresent{#1}{,}{}#3}{#4}{#2}}
\newcommand*\LogRuleForallRelIntroConcl[4]{\Sequent{#1}{\LogForallRel#3#4}{#2}}
\newcommand*\LogRuleForallRelIntro[4]{\AXM{\LogRuleForallRelIntroFirst{#1}{#2}{#3}{#4}}\RLM{\left(#3\notin\FV{#1,#2}\right)}\UIM{\LogRuleForallRelIntroConcl{#1}{#2}{#3}{#4}}\DP}
\newcommand*\LogRuleForallRelIntroTree[5]{\AXM{#1}\UIM{\LogRuleForallRelIntroFirst{#2}{#3}{#4}{#5}}\RLM{\left(#3\notin\FV{#1,#2}\right)}\UIM{\LogRuleForallRelIntroConcl{#2}{#3}{#4}{#5}}\DP}
\newcommand*\LogRuleForallRelElimFirst[5]{\Sequent{#1}{\LogForallRel#3#5}{#2}}
\newcommand*\LogRuleForallRelElimConcl[5]{\Sequent{#1}{#5\LogSubst{#4/#3}}{#2}}
\newcommand*\LogRuleForallRelElim[5]{\AXM{\LogRuleForallRelElimFirst{#1}{#2}{#3}{#4}{#5}}\RLM{\left(\FV{#4}\subseteq\Gamma\right)}\UIM{\LogRuleForallRelElimConcl{#1}{#2}{#3}{#4}{#5}}\DP}
\newcommand*\LogRuleForallRelElimTree[6]{\AXM{#1}\UIM{\LogRuleForallRelElimFirst{#2}{#3}{#4}{#5}{#6}}\RLM{\left(\FV{#4}\subseteq\Gamma\right)}\UIM{\LogRuleForallRelElimConcl{#2}{#3}{#4}{#5}{#6}}\DP}
\newcommand*\ModInterp[2]{#2^#1}
\newcommand*\ModM{\mathcal{M}}
\newcommand*\ModElemA{a}
\newcommand*\ModElemB{b}
\newcommand*\ModElemC{c}
\newcommand*\ModElemD{d}
\newcommand*\ModMInterp[1]{{#1}^\ModM}
\newcommand*\Models\vDash
\newcommand*\LmSortBot0
\newcommand*\LmSortBase{X}
\newcommand*\LmSortExtract{Z}
\newcommand*\LmSortTimes\times
\newcommand*\LmSortTo\to
\newcommand*\LmSortA{T}
\newcommand*\LmSortB{U}
\newcommand*\LmSortC{V}
\newcommand*\LmSortD{W}
\newcommand*\LmConsts{\mathcal{C}st}
\newcommand*\LmTerm[2]{#1\mathrel{:}#2}
\newcommand*\LmTermA{M}
\newcommand*\LmTermB{N}
\newcommand*\LmTermC{P}
\newcommand*\LmTermD{Q}
\newcommand*\LmVarA{x}
\newcommand*\LmVarB{y}
\newcommand*\LmVarC{z}
\newcommand*\LmVarD{u}
\newcommand*\LmVarE{v}
\newcommand*\LmVarF{w}
\newcommand*\LmMVarA\alpha
\newcommand*\LmMVarB\beta
\newcommand*\LmMVarC\beta
\newcommand*\LmMVarD\delta
\newcommand*\LmConst[1]{\mathsf{#1}}
\newcommand*\LmConstA{\LmConst{c}}
\newcommand*\LmConstB{\LmConst{d}}
\newcommand*\LmConstC{\LmConst{e}}
\newcommand*\LmConstD{\LmConst{f}}
\newcommand*\LmPair[2]{\left\langle#1,#2\right\rangle}
\newcommand*\LmProj{\pi}
\newcommand*\LmSubst[1]{\left\{#1\right\}}
\newcommand*\Lam{\lambda^{R\times+}}
\newcommand*\LamTypeTo\to
\newcommand*\LamTypeTimes\times
\newcommand*\LamTypePlus{+}
\newcommand*\LamTypeUnit{1}
\newcommand*\LamTypeR{R}
\newcommand*\LamTypeA{T}
\newcommand*\LamTypeB{U}
\newcommand*\LamTypeC{V}
\newcommand*\LamTypeD{W}
\newcommand*\LamTermA{M}
\newcommand*\LamTermB{N}
\newcommand*\LamTermC{P}
\newcommand*\LamTermD{Q}
\newcommand*\LamVarA{k}
\newcommand*\LamVarB{l}
\newcommand*\LamVarC{m}
\newcommand*\LamVarD{n}
\newcommand*\LamSubst[1]{\left\{#1\right\}}
\newcommand*\LamCPS[1]{\raisebox{0pt}[.9\height]{#1}}
\newcommand*\LamIn{\mathsf{in}}
\newcommand*\LamCase[4]{\mathsf{case}\,#1\left\{\LamIn_1\,#2\mapsto#3\BarSep\LamIn_2\,#2\mapsto#4\right\}}
\newcommand*\LamCaseBlock[4]{\mathsf{case}\,#1\left\{\begin{gathered}\LamIn_1\,#2\mapsto#3\\\LamIn_2\,#2\mapsto#4\end{gathered}\right\}}
\newcommand*\LamUnit{*}
\newcommand*\LmRuleAxConcl[4]{\Sequent{#1\ifpresent{#1}{,}{}\LmTerm{#3}{#4}}{\LmTerm{#3}{#4}}{#2}}
\newcommand*\LmRuleAx[4]{\AXM{}\UIM{\LmRuleAxConcl{#1}{#2}{#3}{#4}}\DP}
\newcommand*\LmRuleAxSigConcl[4]{\Sequent{#1}{\LmTerm{\LmConst{#3}}{#4}}{#2}}
\newcommand*\LmRuleAxSig[4]{\AXM{}\RLM{(\LmTerm{\LmConst{#3}}{#4}\,\in\,\LmConsts)}\UIM{\LmRuleAxSigConcl{#1}{#2}{#3}{#4}}\DP}
\newcommand*\LmRuleImpIntroFirst[6]{\Sequent{#1\ifpresent{#1}{,}{}\LmTerm{#3}{#4}}{\LmTerm{#5}{#6}}{#2}}
\newcommand*\LmRuleImpIntroConcl[6]{\Sequent{#1}{\LmTerm{\lambda#3.#5}{#4\LmSortTo#6}}{#2}}
\newcommand*\LmRuleImpIntro[6]{\AXM{\LmRuleImpIntroFirst{#1}{#2}{#3}{#4}{#5}{#6}}\UIM{\LmRuleImpIntroConcl{#1}{#2}{#3}{#4}{#5}{#6}}\DP}
\newcommand*\LmRuleImpIntroTree[7]{\AXM{#1}\UIM{\LmRuleImpIntroFirst{#2}{#3}{#4}{#5}{#6}{#7}}\UIM{\LmRuleImpIntroConcl{#2}{#3}{#4}{#5}{#6}{#7}}\DP}
\newcommand*\LmRuleImpElimFirst[6]{\Sequent{#1}{\LmTerm{#3}{#6\LmSortTo#4}}{#2}}
\newcommand*\LmRuleImpElimSecond[6]{\Sequent{#1}{\LmTerm{#5}{#6}}{#2}}
\newcommand*\LmRuleImpElimConcl[6]{\Sequent{#1}{\LmTerm{#3#5}{#4}}{#2}}
\newcommand*\LmRuleImpElim[6]{\AXM{\LmRuleImpElimFirst{#1}{#2}{#3}{#4}{#5}{#6}}\AXM{\LmRuleImpElimSecond{#1}{#2}{#3}{#4}{#5}{#6}}\BIM{\LmRuleImpElimConcl{#1}{#2}{#3}{#4}{#5}{#6}}\DP}
\newcommand*\LmRuleImpElimTree[8]{\AXM{#1}\UIM{\LmRuleImpElimFirst{#3}{#4}{#5}{#6}{#7}{#8}}\AXM{#2}\UIM{\LmRuleImpElimSecond{#3}{#4}{#5}{#6}{#7}{#8}}\BIM{\LmRuleImpElimConcl{#3}{#4}{#5}{#6}{#7}{#8}}\DP}
\newcommand*\LmRuleAndIntroFirst[6]{\Sequent{#1}{\LmTerm{#3}{#4}}{#2}}
\newcommand*\LmRuleAndIntroSecond[6]{\Sequent{#1}{\LmTerm{#5}{#6}}{#2}}
\newcommand*\LmRuleAndIntroConcl[6]{\Sequent{#1}{\LmTerm{\LmPair{#3}{#5}}{#4\times#6}}{#2}}
\newcommand*\LmRuleAndIntro[6]{\AXM{\LmRuleAndIntroFirst{#1}{#2}{#3}{#4}{#5}{#6}}\AXM{\LmRuleAndIntroSecond{#1}{#2}{#3}{#4}{#5}{#6}}\BIM{\LmRuleAndIntroConcl{#1}{#2}{#3}{#4}{#5}{#6}}\DP}
\newcommand*\LmRuleAndIntroTree[8]{\AXM{#1}\UIM{\LmRuleAndIntroFirst{#3}{#4}{#5}{#6}{#7}{#8}}\AXM{#2}\UIM{\LmRuleAndIntroSecond{#3}{#4}{#5}{#6}{#7}{#8}}\BIM{\LmRuleAndIntroConcl{#3}{#4}{#5}{#6}{#7}{#8}}\DP}
\newcommand*\LmRuleAndElimFirst[4]{\Sequent{#1}{\LmTerm{#3}{#4_1\LmSortTimes#4_2}}{#2}}
\newcommand*\LmRuleAndElimConcl[4]{\Sequent{#1}{\LmTerm{\LmProj_i\,#3}{#4_i}}{#2}}
\newcommand*\LmRuleAndElim[4]{\AXM{\LmRuleAndElimFirst{#1}{#2}{#3}{#4}}\UIM{\LmRuleAndElimConcl{#1}{#2}{#3}{#4}}\DP}
\newcommand*\LmRuleAndElimTree[5]{\AXM{#1}\UIM{\LmRuleAndElimFirst{#2}{#3}{#4}{#5}}\UIM{\LmRuleAndElimConcl{#2}{#3}{#4}{#5}}\DP}
\newcommand*\LmRuleAndElimLFirst[5]{\Sequent{#1}{\LmTerm{#3}{#4\LmSortTimes#5}}{#2}}
\newcommand*\LmRuleAndElimLConcl[5]{\Sequent{#1}{\LmTerm{\LmProj_1\,#3}{#4}}{#2}}
\newcommand*\LmRuleAndElimL[5]{\AXM{\LmRuleAndElimLFirst{#1}{#2}{#3}{#4}{#5}}\UIM{\LmRuleAndElimLConcl{#1}{#2}{#3}{#4}{#5}}\DP}
\newcommand*\LmRuleAndElimLTree[6]{\AXM{#1}\UIM{\LmRuleAndElimLFirst{#2}{#3}{#4}{#5}{#6}}\UIM{\LmRuleAndElimLConcl{#2}{#3}{#4}{#5}{#6}}\DP}
\newcommand*\LmRuleAndElimRFirst[5]{\Sequent{#1}{\LmTerm{#3}{#4\LmSortTimes#5}}{#2}}
\newcommand*\LmRuleAndElimRConcl[5]{\Sequent{#1}{\LmTerm{\LmProj_2\,#3}{#5}}{#2}}
\newcommand*\LmRuleAndElimR[5]{\AXM{\LmRuleAndElimRFirst{#1}{#2}{#3}{#4}{#5}}\UIM{\LmRuleAndElimRConcl{#1}{#2}{#3}{#4}{#5}}\DP}
\newcommand*\LmRuleAndElimRTree[6]{\AXM{#1}\UIM{\LmRuleAndElimRFirst{#2}{#3}{#4}{#5}{#6}}\UIM{\LmRuleAndElimRConcl{#2}{#3}{#4}{#5}{#6}}\DP}
\newcommand*\LmRuleBotIntroFirst[5]{\Sequent{#1}{\LmTerm{#4}{#5}}{\LmTerm{#3}{#5}}\ifpresent{#2}{,}{}#2}
\newcommand*\LmRuleBotIntroConcl[5]{\Sequent{#1}{\LmTerm{[#3]#4}{\LmSortBot}}{\LmTerm{#3}{#5}}\ifpresent{#2}{,}{}#2}
\newcommand*\LmRuleBotIntro[5]{\AXM{\LmRuleBotIntroFirst{#1}{#2}{#3}{#4}{#5}}\UIM{\LmRuleBotIntroConcl{#1}{#2}{#3}{#4}{#5}}\DP}
\newcommand*\LmRuleBotIntroTree[6]{\AXM{#1}\UIM{\LmRuleBotIntroFirst{#2}{#3}{#4}{#5}{#6}}\UIM{\LmRuleBotIntroConcl{#2}{#3}{#4}{#5}{#6}}\DP}
\newcommand*\LmRuleBotElimFirst[5]{\Sequent{#1}{\LmTerm{#4}{\LmSortBot}}{\LmTerm{#3}{#5}\ifpresent{#2}{,}{}#2}}
\newcommand*\LmRuleBotElimConcl[5]{\Sequent{#1}{\LmTerm{\mu#3.#4}{#5}}{#2}}
\newcommand*\LmRuleBotElim[5]{\AXM{\LmRuleBotElimFirst{#1}{#2}{#3}{#4}{#5}}\UIM{\LmRuleBotElimConcl{#1}{#2}{#3}{#4}{#5}}\DP}
\newcommand*\LmRuleBotElimTree[6]{\AXM{#1}\UIM{\LmRuleBotElimFirst{#2}{#3}{#4}{#5}{#6}}\UIM{\LmRuleBotElimConcl{#2}{#3}{#4}{#5}{#6}}\DP}
\newcommand*\LamRuleOrIntroFirst[3]{\Sequent{#1}{\LmTerm{#2}{#3_i}}{}}
\newcommand*\LamRuleOrIntroConcl[3]{\Sequent{#1}{\LmTerm{\LamIn_i\,#2}{#3_1\LamTypePlus#3_2}}{}}
\newcommand*\LamRuleOrIntro[3]{\AXM{\LamRuleOrIntroFirst{#1}{#2}{#3}}\UIM{\LamRuleOrIntroConcl{#1}{#2}{#3}}\DP}
\newcommand*\LamRuleOrIntroTree[4]{\AXM{#1}\UIM{\LamRuleOrIntroFirst{#2}{#3}{#4}}\UIM{\LamRuleOrIntroConcl{#2}{#3}{#4}}\DP}
\newcommand*\LamRuleOrIntroLFirst[4]{\Sequent{#1}{\LmTerm{#2}{#3}}{}}
\newcommand*\LamRuleOrIntroLConcl[4]{\Sequent{#1}{\LmTerm{\LamIn_1\,#2}{#3\LamTypePlus#4}}{}}
\newcommand*\LamRuleOrIntroL[4]{\AXM{\LamRuleOrIntroLFirst{#1}{#2}{#3}{#4}}\UIM{\LamRuleOrIntroLConcl{#1}{#2}{#3}{#4}}\DP}
\newcommand*\LamRuleOrIntroLTree[5]{\AXM{#1}\UIM{\LamRuleOrIntroLFirst{#2}{#3}{#4}{#5}}\UIM{\LamRuleOrIntroLConcl{#2}{#3}{#4}{#5}}\DP}
\newcommand*\LamRuleOrIntroRFirst[4]{\Sequent{#1}{\LmTerm{#2}{#4}}{}}
\newcommand*\LamRuleOrIntroRConcl[4]{\Sequent{#1}{\LmTerm{\LamIn_2\,#2}{#3\LamTypePlus#4}}{}}
\newcommand*\LamRuleOrIntroR[4]{\AXM{\LamRuleOrIntroRFirst{#1}{#2}{#3}{#4}}\UIM{\LamRuleOrIntroRConcl{#1}{#2}{#3}{#4}}\DP}
\newcommand*\LamRuleOrIntroRTree[5]{\AXM{#1}\UIM{\LamRuleOrIntroRFirst{#2}{#3}{#4}{#5}}\UIM{\LamRuleOrIntroRConcl{#2}{#3}{#4}{#5}}\DP}
\newcommand*\LamRuleOrElimFirst[6]{\Sequent{#1}{\LmTerm{#3}{#5_1\LamTypePlus#5_2}}{}}
\newcommand*\LamRuleOrElimSecond[6]{\Sequent{#1\ifpresent{#1}{,}{}\LmTerm{#2}{#5_1}}{\LmTerm{#4_1}{#6}}{}}
\newcommand*\LamRuleOrElimThird[6]{\Sequent{#1\ifpresent{#1}{,}{}\LmTerm{#2}{#5_2}}{\LmTerm{#4_2}{#6}}{}}
\newcommand*\LamRuleOrElimConcl[6]{\Sequent{#1}{\LmTerm{\LamCase{#3}{#2}{#4_1}{#4_2}}{#6}}{}}
\newcommand*\LamRuleOrElim[6]{\AXM{\LamRuleOrElimFirst{#1}{#2}{#3}{#4}{#5}{#6}}\AXM{\LamRuleOrElimSecond{#1}{#2}{#3}{#4}{#5}{#6}}\AXM{\LamRuleOrElimThird{#1}{#2}{#3}{#4}{#5}{#6}}\TIM{\LamRuleOrElimConcl{#1}{#2}{#3}{#4}{#5}{#6}}\DP}
\newcommand*\LamRuleOrElimTree[9]{\AXM{#1}\UIM{\LamRuleOrElimFirst{#4}{#5}{#6}{#7}{#8}{#9}}\AXM{#2}\UIM{\LamRuleOrElimSecond{#4}{#5}{#6}{#7}{#8}{#9}}\AXM{#3}\UIM{\LamRuleOrElimThird{#4}{#5}{#6}{#7}{#8}{#9}}\TIM{\LamRuleOrElimConcl{#4}{#5}{#6}{#7}{#8}{#9}}\DP}
\newcommand*\LamRuleTopConcl[1]{\Sequent{#1}{\LmTerm{\LamUnit}{\LamTypeUnit}}{}}
\newcommand*\LamRuleTop[1]{\AXM{}\UIM{\LamRuleTopConcl{#1}}\DP}
\newcommand*\LmInterpForm[1]{{#1}^*}
\newcommand*\LmInterpProof[1]{{#1}^*}
\newcommand*\LmInterpAxiom[1]{M_{#1}}
\newcommand*\CatC{\mathcal{C}}
\newcommand*\CatObj[1]{Ob\left(#1\right)}
\newcommand*\CatObjA{A}
\newcommand*\CatObjB{B}
\newcommand*\CatObjC{C}
\newcommand*\CatObjD{D}
\newcommand*\CatR{R}
\newcommand*\CatRC{\CatExp{\CatR}{\CatC}}
\newcommand*\CatTimes\times
\newcommand*\CatPlus{+}
\newcommand*\CatExp[2]{#1^{#2}}
\newcommand*\CatPar\parr
\newcommand*\CatRCHomA\phi
\newcommand*\CatRCHomB\psi
\newcommand*\CatRCHomC\zeta
\newcommand*\CatRCHomD\xi
\newcommand*\CatRCHomE\varphi
\newcommand*\CatCHomA\mcyrzh
\newcommand*\CatCHomB\mcyri
\newcommand*\CatCHomC\mcyrl
\newcommand*\CatCHomD\mcyrch
\newcommand*\CatCHomE\mcyrd
\newcommand*\CatId[1]{\mathbf{Id}_{#1}}
\newcommand*\CatPair[2]{{\mathbf{pair}\bm{\left(}#1,#2\bm{\right)}}}
\newcommand*\CatEval{\mathbf{ev}}
\newcommand*\CatLambda{\bm{\Lambda}}
\newcommand*\CatTerm{\mathbf{1}}
\newcommand*\CatTermMorph[1]{\mathbf{1}_{#1}}
\newcommand*\CatInit{\mathbf{0}}
\newcommand*\CatInterpSortNeg[1]{{\left\llbracket#1\right\rrbracket}}
\newcommand*\CatInterpSort[1]{{\left[#1\right]}}
\newcommand*\CatInterpTerm[1]{\left[#1\right]}
\newcommand*\RealValNeg[1]{\left\|#1\right\|}
\newcommand*\RealVal[1]{\left|#1\right|}
\newcommand*\RealBot{{\bot\mkern-11mu\bot}}
\newcommand*\HA{{H\!A}}
\newcommand*\HAom{{\HA^\omega}}
\newcommand*\PA{{P\!A}}
\newcommand*\PAom{{\PA^\omega}}
\newcommand*\CA{{C\!A}}
\newcommand*\CAom{{\CA^\omega}}
\newcommand*\CASort\iota
\newcommand*\CASortList[1]{{#1}^\diamond}
\newcommand*\CALogs{\LogConst{s}}
\newcommand*\CALogsSort[3]{\left(#1\SortTo#2\SortTo#3\right)\SortTo\left(#1\SortTo#2\right)\SortTo#1\SortTo#3}
\newcommand*\CALogk{\LogConst{k}}
\newcommand*\CALogkSort[2]{#1\SortTo#2\SortTo#1}
\newcommand*\CALogZ{\LogConst{0}}
\newcommand*\CALogZSort{\CASort}
\newcommand*\CALogS{\LogConst{S}}
\newcommand*\CALogSSort{\CASort\SortTo\CASort}
\newcommand*\CALogrec{\LogConst{rec}}
\newcommand*\CALogrecSort[1]{#1\SortTo\left(\CASort\SortTo#1\SortTo#1\right)\SortTo\CASort\SortTo#1}
\newcommand*\CAAxName[1]{{\bm{\scriptstyle(#1)}}}
\newcommand*\CAReflName{\CAAxName{refl}}
\newcommand*\CARefl[1]{\forall\LogSortedTerm{\LogVarA}{#1}\left(\LogVarA=_#1\LogVarA\right)}
\newcommand*\CAReflNoType{\forall\LogVarA\left(\LogVarA=\LogVarA\right)}
\newcommand*\CAReflTerm{\lambda\LmVarA.\LmVarA}
\newcommand*\CAReflSort{\LmSortBot\LmSortTo\LmSortBot}
\newcommand*\CALeibName{\CAAxName{Leib}}
\newcommand*\CALeib[3]{\forall\LogSortedTerm{\vec{\LogVarC}}{\vec{#3}}\,\forall\LogSortedTerm{\LogVarA}{#1}\,\forall\LogSortedTerm{\LogVarB}{#1}\left(\neg#2\LogImp#2\LogSubst{\LogVarB/\LogVarA}\LogImp\LogVarA\neq_#1\LogVarB\right)}
\newcommand*\CALeibNoType[1]{\forall\vec{\LogVarC}\,\forall\LogVarA\,\forall\LogVarB\left(\neg#1\LogImp#1\LogSubst{\LogVarB/\LogVarA}\LogImp\LogVarA\neq\LogVarB\right)}
\newcommand*\CALeibTerm{\lambda\LmVarA.\LmVarA}
\newcommand*\CALeibSort[1]{\left(\LmInterpForm{#1}\LmSortTo\LmSortBot\right)\LmSortTo\LmInterpForm{#1}\LmSortTo\LmSortBot}
\newcommand*\CAdefsName{\CAAxName{\Delta s}}
\newcommand*\CAdefs[3]{\forall\LogSortedTerm{\LogVarA}{#1\SortTo#2\SortTo#3}\forall\LogSortedTerm{\LogVarB}{#1\SortTo#2}\forall\LogSortedTerm{\LogVarC}{#1}\left(\CALogs\,\LogVarA\,\LogVarB\,\LogVarC=_#3\LogVarA\,\LogVarC\left(\LogVarB\,\LogVarC\right)\right)}
\newcommand*\CAdefsNoType{\forall\LogVarA\,\forall\LogVarB\,\forall\LogVarC\;\;\CALogs\,\LogVarA\,\LogVarB\,\LogVarC=\LogVarA\,\LogVarC\left(\LogVarB\,\LogVarC\right)}
\newcommand*\CAdefsTerm{\lambda\LmVarA.\LmVarA}
\newcommand*\CAdefsSort{\LmSortBot\LmSortTo\LmSortBot}
\newcommand*\CAdefkName{\CAAxName{\Delta k}}
\newcommand*\CAdefk[2]{\forall\LogSortedTerm{\LogVarA}{#1}\forall\LogSortedTerm{\LogVarB}{#2}\left(\CALogk\,\LogVarA\,\LogVarB=_#1\LogVarA\right)}
\newcommand*\CAdefkNoType{\forall\LogVarA\,\forall\LogVarB\;\;\CALogk\,\LogVarA\,\LogVarB=\LogVarA}
\newcommand*\CAdefkTerm{\lambda\LmVarA.\LmVarA}
\newcommand*\CAdefkSort{\LmSortBot\LmSortTo\LmSortBot}
\newcommand*\CAdefrecZName{\CAAxName{\Delta rec0}}
\newcommand*\CAdefrecZ[1]{\forall\LogSortedTerm{\LogVarA}{#1}\forall\LogSortedTerm{\LogVarB}{\CASort\SortTo#1\SortTo#1}\left(\CALogrec\,\LogVarA\,\LogVarB\,\CALogZ=_#1\LogVarA\right)}
\newcommand*\CAdefrecZNoType{\forall\LogVarA\,\forall\LogVarB\;\;\CALogrec\,\LogVarA\,\LogVarB\,\CALogZ=\LogVarA}
\newcommand*\CAdefrecZTerm{\lambda\LmVarA.\LmVarA}
\newcommand*\CAdefrecZSort{\LmSortBot\LmSortTo\LmSortBot}
\newcommand*\CAdefrecSName{\CAAxName{\Delta recS}}
\newcommand*\CAdefrecS[1]{\forall\LogSortedTerm{\LogVarA}{#1}\forall\LogSortedTerm{\LogVarB}{\CASort\SortTo#1\SortTo#1}\forall\LogSortedTerm{\LogVarC}{\CASort}\left(\CALogrec\,\LogVarA\,\LogVarB\left(\CALogS\,\LogVarC\right)=_#1\LogVarB\,\LogVarC\left(\CALogrec\,\LogVarA\,\LogVarB\,\LogVarC\right)\right)}
\newcommand*\CAdefrecSNoType{\forall\LogVarA\,\forall\LogVarB\,\forall\LogVarC\;\;\CALogrec\,\LogVarA\,\LogVarB\left(\CALogS\,\LogVarC\right)=\LogVarB\,\LogVarC\left(\CALogrec\,\LogVarA\,\LogVarB\,\LogVarC\right)}
\newcommand*\CAdefrecSTerm{\lambda\LmVarA.\LmVarA}
\newcommand*\CAdefrecSSort{\LmSortBot\LmSortTo\LmSortBot}
\newcommand*\CASnZName{\CAAxName{Snz}}
\newcommand*\CASnZ{\forall\LogSortedTerm{\LogVarA}{\CASort}\left(\CALogS\,\LogVarA\neq_\CASort\CALogZ\right)}
\newcommand*\CASnZNoType{\forall\LogVarA\;\;\CALogS\,\LogVarA\neq\CALogZ}
\newcommand*\CASnZTerm{\CALmom}
\newcommand*\CASnZSort{\LmSortBot}
\newcommand*\CAindUnrelName{\CAAxName{ind}}
\newcommand*\CAindUnrel[1]{\forall\LogSortedTerm{\vec{\LogVarB}}{\vec{\sigma}}\left(#1\LogSubst{\CALogZ/\LogVarA}\LogImp\forall\LogSortedTerm{\LogVarA}{\CASort}\left(#1\LogImp#1\LogSubst{\CALogS\,\LogVarA/\LogVarA}\right)\LogImp\forall\LogSortedTerm{\LogVarA}{\CASort}#1\right)}
\newcommand*\CAindUnrelNoType[1]{\forall\vec{\LogVarB}\left(#1\LogSubst{\CALogZ/\LogVarA}\LogImp\forall\LogVarA\left(#1\LogImp#1\LogSubst{\CALogS\,\LogVarA/\LogVarA}\right)\LogImp\forall\LogVarA\,#1\right)}
\newcommand*\CAindName{\CAAxName{ind^r}}
\newcommand*\CAind[2]{\forall\LogSortedTerm{\vec{\LogVarB}}{\vec{#2}}\left(#1\LogSubst{\CALogZ/\LogVarA}\LogImp\LogForallRel\LogSortedTerm{\LogVarA}{\CASort}\left(#1\LogImp#1\LogSubst{\CALogS\,\LogVarA/\LogVarA}\right)\LogImp\LogForallRel\LogSortedTerm{\LogVarA}{\CASort}#1\right)}
\newcommand*\CAindNoType[1]{\forall\vec{\LogVarB}\left(#1\LogSubst{\CALogZ/\LogVarA}\LogImp\LogForallRel\LogVarA\left(#1\LogImp#1\LogSubst{\CALogS\,\LogVarA/\LogVarA}\right)\LogImp\LogForallRel\LogVarA\,#1\right)}
\newcommand*\CAindTerm\CALmrec
\newcommand*\CAindSort[1]{\LmInterpForm{#1}\LmSortTo\left(\CALmnSort\LmSortTo\LmInterpForm{#1}\LmSortTo\LmInterpForm{#1}\right)\LmSortTo\CALmnSort\LmSortTo\LmInterpForm{#1}}
\newcommand*\CADCUnrelName{\CAAxName{DC}}
\newcommand*\CADCUnrel[3]{\forall\LogSortedTerm{\vec{\LogVarD}}{\vec{#3}}\left(\forall\LogSortedTerm{\LogVarA}{\CASort}\,\forall\LogSortedTerm{\LogVarB}{#1}\,\exists\LogSortedTerm{\LogVarC}{#1}\,#2\LogSubst{\LogVarA,\LogVarB,\LogVarC}\;\;\LogImp\;\;\exists\LogSortedTerm{\LogVarE}{\CASort\SortTo#1}\,\forall\LogSortedTerm{\LogVarA}{\CASort}\,#2\LogSubst{\LogVarA,\LogVarE\,\LogVarA,\LogVarE\left(\CALogS\,\LogVarA\right)}\right)}
\newcommand*\CADCUnrelNoType[1]{\forall\vec{\LogVarD}\left(\forall\LogVarA\,\forall\LogVarB\,\exists\LogVarC\,#1\LogSubst{\LogVarA,\LogVarB,\LogVarC}\;\;\LogImp\;\;\exists\LogVarE\,\forall\LogVarA\,#1\LogSubst{\LogVarA,\LogVarE\,\LogVarA,\LogVarE\left(\CALogS\,\LogVarA\right)}\right)}
\newcommand*\CADCName{\CAAxName{DC^r}}
\newcommand*\CADC[3]{\forall\LogSortedTerm{\vec{\LogVarD}}{\vec{#3}}\left(\LogForallRel\LogSortedTerm{\LogVarA}{\CASort}\,\LogForallRel\LogSortedTerm{\LogVarB}{#1}\left(\forall\LogSortedTerm{\LogVarC}{#1}\,\neg#2\LogSubst{\LogVarA,\LogVarB,\LogVarC}\LogImp\forall\LogSortedTerm{\LogVarA'}{\CASort}\,#2\LogSubst{\LogVarA',\LogVarB,\LogVarB}\right)\;\LogImp\;\exists\LogSortedTerm{\LogVarE}{\CASort\SortTo#1}\,\LogForallRel\LogSortedTerm{\LogVarA}{\CASort}\,#2\LogSubst{\LogVarA,\LogVarE\,\LogVarA,\LogVarE\left(\CALogS\,\LogVarA\right)}\right)}
\newcommand*\CADCNoType[1]{\forall\vec{\LogVarD}\left(\LogForallRel\LogVarA\,\LogForallRel\LogVarB\left(\forall\LogVarC\,\neg#1\LogSubst{\LogVarA,\LogVarB,\LogVarC}\LogImp\forall\LogVarA'\,#1\LogSubst{\LogVarA',\LogVarB,\LogVarB}\right)\;\;\LogImp\;\;\exists\LogVarE\,\LogForallRel\LogVarA\,#1\LogSubst{\LogVarA,\LogVarE\,\LogVarA,\LogVarE\left(\CALogS\,\LogVarA\right)}\right)}
\newcommand*\CADCTerm{\lambda\LmVarA\LmVarB.\CALmbarrec\left(\lambda\LmVarE.\LmVarA\,\CALmlen{\LmVarE}\left(\CALmifz\,\CALmlen{\LmVarE}\,\CALmn{0}\left(\LmProj_1\,\CALmind{\LmVarE}{\CALmpred\,\CALmlen{\LmVarE}}\right)\right)\right)\LmVarB\,\CALmnil}
\newcommand*\CADCSort[2]{\left(\CALmnSort\LmSortTo#1\LmSortTo\left(\LmInterpForm{#2}\LmSortTo\LmSortBot\right)\LmSortTo\LmInterpForm{#2}\right)\LmSortTo\left(\left(\CALmnSort\LmSortTo\LmInterpForm{#2}\right)\LmSortTo\LmSortBot\right)\LmSortTo\LmSortBot}
\newcommand*\CARels[3]{\LogRel{\LogSortedTerm{\CALogs}{\CALogsSort{#1}{#2}{#3}}}}
\newcommand*\CARelsNoType{\LogRel{\CALogs}}
\newcommand*\CARelsTerm{\lambda\LmVarA\LmVarB\LmVarC.\LmVarA\,\LmVarC\left(\LmVarB\,\LmVarC\right)}
\newcommand*\CARelsSort[3]{\left(#1\LmSortTo#2\LmSortTo#3\right)\LmSortTo\left(#1\LmSortTo#2\right)\LmSortTo#1\LmSortTo#3}
\newcommand*\CARelk[2]{\LogRel{\LogSortedTerm{\CALogk}{\CALogkSort{#1}{#2}}}}
\newcommand*\CARelkNoType{\LogRel{\CALogk}}
\newcommand*\CARelkTerm{\lambda\LmVarA\LmVarB.\LmVarA}
\newcommand*\CARelkSort[2]{#1\LmSortTo#2\LmSortTo#1}
\newcommand*\CARelZ{\LogRel{\LogSortedTerm{\CALogZ}{\CALogZSort}}}
\newcommand*\CARelZNoType{\LogRel{\CALogZ}}
\newcommand*\CARelZTerm{\CALmn{0}}
\newcommand*\CARelZSort{\CALmnSort}
\newcommand*\CARelS{\LogRel{\LogSortedTerm{\CALogS}{\CALogSSort}}}
\newcommand*\CARelSNoType{\LogRel{\CALogS}}
\newcommand*\CARelSTerm{\CALmsucc}
\newcommand*\CARelSSort{\CALmnSort\LmSortTo\CALmnSort}
\newcommand*\CARelrec[1]{\LogRel{\LogSortedTerm{\CALogrec}{\CALogrecSort{#1}}}}
\newcommand*\CARelrecNoType{\LogRel{\CALogrec}}
\newcommand*\CARelrecTerm{\CALmrec}
\newcommand*\CARelrecSort[1]{#1\LmSortTo\left(\CALmnSort\LmSortTo#1\LmSortTo#1\right)\LmSortTo\CALmnSort\LmSortTo#1}
\newcommand*\CALmn[1]{\LmConst{\overline{#1}}}
\newcommand*\CALmnSort{I}
\newcommand*\CALmom{\LmConst{\Omega}}
\newcommand*\CALmomSort\LmSortBot
\newcommand*\CALmsucc{\LmConst{succ}}
\newcommand*\CALmsuccSort{\CALmnSort\LmSortTo\CALmnSort}
\newcommand*\CALmrec{\LmConst{rec}}
\newcommand*\CALmrecSort[1]{#1\LmSortTo\left(\CALmnSort\LmSortTo#1\LmSortTo#1\right)\LmSortTo\CALmnSort\LmSortTo#1}
\newcommand*\CALmfix{\LmConst{Y}}
\newcommand*\CALmfixSort[1]{\left(#1\LmSortTo#1\right)\LmSortTo#1}
\newcommand*\CALmbarrec{\LmConst{barrec}}
\newcommand*\CALmbarrecSort[2]{\left(\CASortList{#1}\LmSortTo\left(#1\LmSortTo#2\right)\LmSortTo#1\right)\LmSortTo\left(\left(\CALmnSort\LmSortTo#1\right)\LmSortTo#2\right)\LmSortTo\CASortList{#1}\LmSortTo#2}
\newcommand*\CALmifz{\LmConst{if_0}}
\newcommand*\CALmifzSort[1]{\CALmnSort\LmSortTo#1\LmSortTo#1}
\newcommand*\CALmpred{\LmConst{pred}}
\newcommand*\CALmpredSort{\CALmnSort\LmSortTo\CALmnSort}
\newcommand*\CALmsub{\LmConst{sub}}
\newcommand*\CALmsubSort{\CALmnSort\LmSortTo\CALmnSort\LmSortTo\CALmnSort}
\newcommand*\CALmifl{\LmConst{if_<}}
\newcommand*\CALmiflSort[1]{\CALmnSort\LmSortTo\CALmnSort\LmSortTo#1\LmSortTo#1\LmSortTo#1}
\newcommand*\CALmife{\LmConst{if_=}}
\newcommand*\CALmifeSort[1]{\CALmnSort\LmSortTo\CALmnSort\LmSortTo#1\LmSortTo#1\LmSortTo#1}
\newcommand*\CALmconcat{\mathbin{@}}
\newcommand*\CALmextend{\mathbin{*}}
\newcommand*\CALmnil{\epsilon}
\newcommand*\CALmlen[1]{\left|#1\right|}
\newcommand*\CALmind[2]{#1\,\downharpoonright\mkern-5mu#2\mkern-5mu\downharpoonleft}
\DeclareSymbolFont{cyrillic}{T2A}{cmr}{m}{n}
\DeclareMathSymbol{\MCYRA}{\mathalpha}{cyrillic}{192}
\DeclareMathSymbol{\mcyra}{\mathalpha}{cyrillic}{224}
\DeclareMathSymbol{\MCYRB}{\mathalpha}{cyrillic}{193}
\DeclareMathSymbol{\mcyrb}{\mathalpha}{cyrillic}{225}
\DeclareMathSymbol{\MCYRV}{\mathalpha}{cyrillic}{194}
\DeclareMathSymbol{\mcyrv}{\mathalpha}{cyrillic}{226}
\DeclareMathSymbol{\MCYRG}{\mathalpha}{cyrillic}{195}
\DeclareMathSymbol{\mcyrg}{\mathalpha}{cyrillic}{227}
\DeclareMathSymbol{\MCYRD}{\mathalpha}{cyrillic}{196}
\DeclareMathSymbol{\mcyrd}{\mathalpha}{cyrillic}{228}
\DeclareMathSymbol{\MCYRE}{\mathalpha}{cyrillic}{197}
\DeclareMathSymbol{\mcyre}{\mathalpha}{cyrillic}{229}
\DeclareMathSymbol{\MCYRZH}{\mathalpha}{cyrillic}{198}
\DeclareMathSymbol{\mcyrzh}{\mathalpha}{cyrillic}{230}
\DeclareMathSymbol{\MCYRZ}{\mathalpha}{cyrillic}{199}
\DeclareMathSymbol{\mcyrz}{\mathalpha}{cyrillic}{231}
\DeclareMathSymbol{\MCYRI}{\mathalpha}{cyrillic}{200}
\DeclareMathSymbol{\mcyri}{\mathalpha}{cyrillic}{232}
\DeclareMathSymbol{\MCYRISHRT}{\mathalpha}{cyrillic}{201}
\DeclareMathSymbol{\mcyrishrt}{\mathalpha}{cyrillic}{233}
\DeclareMathSymbol{\MCYRK}{\mathalpha}{cyrillic}{202}
\DeclareMathSymbol{\mcyrk}{\mathalpha}{cyrillic}{234}
\DeclareMathSymbol{\MCYRL}{\mathalpha}{cyrillic}{203}
\DeclareMathSymbol{\mcyrl}{\mathalpha}{cyrillic}{235}
\DeclareMathSymbol{\MCYRM}{\mathalpha}{cyrillic}{204}
\DeclareMathSymbol{\mcyrm}{\mathalpha}{cyrillic}{236}
\DeclareMathSymbol{\MCYRN}{\mathalpha}{cyrillic}{205}
\DeclareMathSymbol{\mcyrn}{\mathalpha}{cyrillic}{237}
\DeclareMathSymbol{\MCYRO}{\mathalpha}{cyrillic}{206}
\DeclareMathSymbol{\mcyro}{\mathalpha}{cyrillic}{238}
\DeclareMathSymbol{\MCYRP}{\mathalpha}{cyrillic}{207}
\DeclareMathSymbol{\mcyrp}{\mathalpha}{cyrillic}{239}
\DeclareMathSymbol{\MCYRR}{\mathalpha}{cyrillic}{208}
\DeclareMathSymbol{\mcyrr}{\mathalpha}{cyrillic}{240}
\DeclareMathSymbol{\MCYRS}{\mathalpha}{cyrillic}{209}
\DeclareMathSymbol{\mcyrs}{\mathalpha}{cyrillic}{241}
\DeclareMathSymbol{\MCYRT}{\mathalpha}{cyrillic}{210}
\DeclareMathSymbol{\mcyrt}{\mathalpha}{cyrillic}{242}
\DeclareMathSymbol{\MCYRU}{\mathalpha}{cyrillic}{211}
\DeclareMathSymbol{\mcyru}{\mathalpha}{cyrillic}{243}
\DeclareMathSymbol{\MCYRF}{\mathalpha}{cyrillic}{212}
\DeclareMathSymbol{\mcyrf}{\mathalpha}{cyrillic}{244}
\DeclareMathSymbol{\MCYRH}{\mathalpha}{cyrillic}{213}
\DeclareMathSymbol{\mcyrh}{\mathalpha}{cyrillic}{245}
\DeclareMathSymbol{\MCYRC}{\mathalpha}{cyrillic}{214}
\DeclareMathSymbol{\mcyrc}{\mathalpha}{cyrillic}{246}
\DeclareMathSymbol{\MCYRCH}{\mathalpha}{cyrillic}{215}
\DeclareMathSymbol{\mcyrch}{\mathalpha}{cyrillic}{247}
\DeclareMathSymbol{\MCYRSH}{\mathalpha}{cyrillic}{216}
\DeclareMathSymbol{\mcyrsh}{\mathalpha}{cyrillic}{248}
\DeclareMathSymbol{\MCYRSHCH}{\mathalpha}{cyrillic}{217}
\DeclareMathSymbol{\mcyrshch}{\mathalpha}{cyrillic}{249}
\DeclareMathSymbol{\MCYRHRDSN}{\mathalpha}{cyrillic}{218}
\DeclareMathSymbol{\mcyrhrdsn}{\mathalpha}{cyrillic}{250}
\DeclareMathSymbol{\MCYRERY}{\mathalpha}{cyrillic}{219}
\DeclareMathSymbol{\mcyrery}{\mathalpha}{cyrillic}{251}
\DeclareMathSymbol{\MCYRSFTSN}{\mathalpha}{cyrillic}{220}
\DeclareMathSymbol{\mcyrsftsn}{\mathalpha}{cyrillic}{252}
\DeclareMathSymbol{\MCYREREV}{\mathalpha}{cyrillic}{221}
\DeclareMathSymbol{\mcyrerev}{\mathalpha}{cyrillic}{253}
\DeclareMathSymbol{\MCYRYU}{\mathalpha}{cyrillic}{222}
\DeclareMathSymbol{\mcyryu}{\mathalpha}{cyrillic}{254}
\DeclareMathSymbol{\MCYRYA}{\mathalpha}{cyrillic}{223}
\DeclareMathSymbol{\mcyrya}{\mathalpha}{cyrillic}{255}
\DeclareMathSymbol{\MCYRGUP}{\mathalpha}{cyrillic}{128}
\DeclareMathSymbol{\mcyrgup}{\mathalpha}{cyrillic}{160}
\DeclareMathSymbol{\MCYRGHCRS}{\mathalpha}{cyrillic}{129}
\DeclareMathSymbol{\mcyrghcrs}{\mathalpha}{cyrillic}{161}
\DeclareMathSymbol{\MCYRDJE}{\mathalpha}{cyrillic}{130}
\DeclareMathSymbol{\mcyrdje}{\mathalpha}{cyrillic}{162}
\DeclareMathSymbol{\MCYRTSHE}{\mathalpha}{cyrillic}{131}
\DeclareMathSymbol{\mcyrtshe}{\mathalpha}{cyrillic}{163}
\DeclareMathSymbol{\MCYRSHHA}{\mathalpha}{cyrillic}{132}
\DeclareMathSymbol{\mcyrshha}{\mathalpha}{cyrillic}{164}
\DeclareMathSymbol{\MCYRZHDSC}{\mathalpha}{cyrillic}{133}
\DeclareMathSymbol{\mcyrzhdsc}{\mathalpha}{cyrillic}{165}
\DeclareMathSymbol{\MCYRZDSC}{\mathalpha}{cyrillic}{134}
\DeclareMathSymbol{\mcyrzdsc}{\mathalpha}{cyrillic}{166}
\DeclareMathSymbol{\MCYRLJE}{\mathalpha}{cyrillic}{135}
\DeclareMathSymbol{\mcyrlje}{\mathalpha}{cyrillic}{167}
\DeclareMathSymbol{\MCYRYI}{\mathalpha}{cyrillic}{136}
\DeclareMathSymbol{\mcyryi}{\mathalpha}{cyrillic}{168}
\DeclareMathSymbol{\MCYRKDSC}{\mathalpha}{cyrillic}{137}
\DeclareMathSymbol{\mcyrkdsc}{\mathalpha}{cyrillic}{169}
\DeclareMathSymbol{\MCYRKBEAK}{\mathalpha}{cyrillic}{138}
\DeclareMathSymbol{\mcyrkbeak}{\mathalpha}{cyrillic}{170}
\DeclareMathSymbol{\MCYRKVCRS}{\mathalpha}{cyrillic}{139}
\DeclareMathSymbol{\mcyrkvcrs}{\mathalpha}{cyrillic}{171}
\DeclareMathSymbol{\MCYRAE}{\mathalpha}{cyrillic}{140}
\DeclareMathSymbol{\mcyrae}{\mathalpha}{cyrillic}{172}
\DeclareMathSymbol{\MCYRNDSC}{\mathalpha}{cyrillic}{141}
\DeclareMathSymbol{\mcyrndsc}{\mathalpha}{cyrillic}{173}
\DeclareMathSymbol{\MCYRNG}{\mathalpha}{cyrillic}{142}
\DeclareMathSymbol{\mcyrng}{\mathalpha}{cyrillic}{174}
\DeclareMathSymbol{\MCYRDZE}{\mathalpha}{cyrillic}{143}
\DeclareMathSymbol{\mcyrdze}{\mathalpha}{cyrillic}{175}
\DeclareMathSymbol{\MCYROTLD}{\mathalpha}{cyrillic}{144}
\DeclareMathSymbol{\mcyrotld}{\mathalpha}{cyrillic}{176}
\DeclareMathSymbol{\MCYRSDSC}{\mathalpha}{cyrillic}{145}
\DeclareMathSymbol{\mcyrsdsc}{\mathalpha}{cyrillic}{177}
\DeclareMathSymbol{\MCYRUSHRT}{\mathalpha}{cyrillic}{146}
\DeclareMathSymbol{\mcyrushrt}{\mathalpha}{cyrillic}{178}
\DeclareMathSymbol{\MCYRY}{\mathalpha}{cyrillic}{147}
\DeclareMathSymbol{\mcyry}{\mathalpha}{cyrillic}{179}
\DeclareMathSymbol{\MCYRYHCRS}{\mathalpha}{cyrillic}{148}
\DeclareMathSymbol{\mcyryhcrs}{\mathalpha}{cyrillic}{180}
\DeclareMathSymbol{\MCYRHDSC}{\mathalpha}{cyrillic}{149}
\DeclareMathSymbol{\mcyrhdsc}{\mathalpha}{cyrillic}{181}
\DeclareMathSymbol{\MCYRDZHE}{\mathalpha}{cyrillic}{150}
\DeclareMathSymbol{\mcyrdzhe}{\mathalpha}{cyrillic}{182}
\DeclareMathSymbol{\MCYRCHVCRS}{\mathalpha}{cyrillic}{151}
\DeclareMathSymbol{\mcyrchvcrs}{\mathalpha}{cyrillic}{183}
\DeclareMathSymbol{\MCYRCHRDSC}{\mathalpha}{cyrillic}{152}
\DeclareMathSymbol{\mcyrchrdsc}{\mathalpha}{cyrillic}{184}
\DeclareMathSymbol{\MCYRIE}{\mathalpha}{cyrillic}{153}
\DeclareMathSymbol{\mcyrie}{\mathalpha}{cyrillic}{185}
\DeclareMathSymbol{\MCYRSCHWA}{\mathalpha}{cyrillic}{154}
\DeclareMathSymbol{\mcyrschwa}{\mathalpha}{cyrillic}{186}
\DeclareMathSymbol{\MCYRNJE}{\mathalpha}{cyrillic}{155}
\DeclareMathSymbol{\mcyrnje}{\mathalpha}{cyrillic}{187}
\DeclareMathSymbol{\MCYRYO}{\mathalpha}{cyrillic}{156}
\DeclareMathSymbol{\mcyryo}{\mathalpha}{cyrillic}{188}
\DeclareMathSymbol{\MCYRII}{\mathalpha}{cyrillic}{73}
\DeclareMathSymbol{\mcyrii}{\mathalpha}{cyrillic}{105}
\DeclareMathSymbol{\MCYRJE}{\mathalpha}{cyrillic}{74}
\DeclareMathSymbol{\mcyrje}{\mathalpha}{cyrillic}{106}
\DeclareMathSymbol{\MCYRQ}{\mathalpha}{cyrillic}{81}
\DeclareMathSymbol{\mcyrq}{\mathalpha}{cyrillic}{113}
\DeclareMathSymbol{\MCYRW}{\mathalpha}{cyrillic}{87}
\DeclareMathSymbol{\mcyrw}{\mathalpha}{cyrillic}{119}
\begin{document}
\title{Typed realizability for first-order classical analysis}
\author{Valentin Blot}
\address{Department of Computer Science, University of Bath, United Kingdom}
\email{v.blot@bath.ac.uk}
\thanks{Research supported by the UK EPSRC grant EP/K037633/1.}
\begin{abstract}
We describe a realizability framework for classical first-order logic in which realizers live in (a model of) typed -calculus. This allows a direct interpretation of classical proofs, avoiding the usual negative translation to intuitionistic logic. We prove that the usual terms of G\"odel's system T realize the axioms of Peano arithmetic, and that under some assumptions on the computational model, the bar recursion operator realizes the axiom of dependent choice. We also perform a proper analysis of relativization, which allows for less technical proofs of adequacy. Extraction of algorithms from proofs of  formulas relies on a novel implementation of Friedman's trick exploiting the control possibilities of the language. This allows to have extracted programs with simpler types than in the case of negative translation followed by intuitionistic realizability.
\end{abstract}
\maketitle
\section*{Introduction}
Realizability is a mean of formalizing the Brouwer-Heyting-Kolmogorov constructive interpretation of logic. To each formula is associated a set of programs, its realizers, which contain computational information about the formula. While the Curry-Howard isomorphism draws a correspondence between proofs in a certain logical system and programs in a suitable typed programming language, realizability takes a different approach and defines a realizer as a program which behaves like a proof. First, this allows to give computational content to the axioms of a theory, and second this allows to choose more freely the programming language, independently from the logical system. It is even possible to consider untyped programming languages, as was the case in the first realizability model from Kleene~\cite{Kleene}, in which a realizer may be any recursive function. It is however still possible to consider a typed language, as did Kreisel in his modified realizability model~\cite{Kreisel}. Both models from Kleene and Kreisel gave computational interpretation to Heyting arithmetic, the intuitionistic variant of Peano arithmetic.\par
G\"odel's negative translation~\cite{GodelNegative} allowed for the first interpretations of classical logic. Indeed, this translation from classical to intuitionistic logic, when followed by an interpretation of intuitionistic proofs, gives a computational interpretation to classical logic. This is what we call the indirect interpretation. This interpretation can easily be extended to arithmetic, and therefore allows to get computational interpretations of Peano arithmetic from the models of Kleene and Kreisel. Much later, Griffin discovered in~\cite{GriffinControl} that the \texttt{call/cc} operator of the functional language Scheme could be typed with a classical principle: the law of Peirce. This opened the possibility to what we call a direct interpretation of classical logic, using programming languages with control features. Following this path, Parigot defined in~\cite{ParigotLambdaMu} the -calculus, a language for Gentzen's classical sequent calculus extending the Curry-Howard isomorphism to classical logic. Selinger axiomatized the universal categorical model of -calculus in~\cite{SelingerControl}. On another side, Krivine considered untyped -calculus extended with the \texttt{call/cc} operator to give a realizability interpretation to classical second-order Zermelo-Fr\ae nkel set theory~\cite{KrivineZF}, later extended to handle the axiom of dependent choice~\cite{KrivineDependent,KrivinePanoramas}.\par
In this paper we define a realizability model for first-order classical logic which, contrary to Krivine's and similarly to Kreisel's (but for classical logic), uses typed programs as realizers. We interpret our proofs in the language PCF, which is a combination of the functional Turing-complete language PCF with the control features of call-by-name -calculus.  As in Krivine's model, we associate to each formula a truth value and a falsity value which are orthogonal to each other, but we perform a fine analysis of relativization and introduce positive predicates for which only the truth value is required and prove it to be correct through a suitable restriction on the proofs in our logical system. We validate our model by proving that the usual terms of G\"odel's system T realize the axioms of Peano arithmetic. We also implement Friedman's trick through the use of an external -variable rather than through the replacement of the  formula by an existential statement, which allows for a simpler and more effective interpretation. This variable is also used to define the orthogonality relation between our truth and falsity values.\par
Interpreting the axiom of dependent choice in a classical setting is much more complicated than interpreting arithmetic. Spector defined in~\cite{Spector} the bar recursor and used it in G\"odel's Dialectica interpretation~\cite{GodelDialectica} (a computational interpretation similar to realizability) to interpret the axiom of countable choice, and therefore the axiom schema of specification. This operator was later studied in~\cite{KohlenbachThesis}, and a more uniform version was used in~\cite{BerardiBezemCoquand} to give an indirect realizability interpretation of countable choice. A version with an implicit termination condition was later defined in~\cite{BergerOlivaChoice} and used to interpret, still in an indirect realizability setting, the double-negation shift principle and therefore the axiom of dependent choice.\par
We prove here that under some assumptions on our model of PCF, the bar recursion operator of~\cite{BergerOlivaChoice} realizes the axiom of dependent choice in our direct interpretation of classical logic. Our implementation of Friedman's trick then allows us to obtain an extraction result on  formulas provable in classical analysis (Peano arithmetic + the axiom of dependent choice).
\section{Logic}
\label{logic}
We define in this section the logical system under which we will work through the article. First, we define the general case of classical multisorted first-order logic (handling classical reasoning by the use of multi-conclusioned sequents), then we describe the case of logics with equality, the case of Peano arithmetic and its extension with the axiom of dependent choice, and finally we recall some basic definitions about models of classical logic.
\subsection{Classical multisorted first-order logic}
The logical framework we use is multisorted first-order logic, where the sorts are fixed to be the types of simply typed -calculus (see e.g.~\cite{TroelstraVanDalenConstructivism}). We build from a set of base sorts  the set of sorts:

which are used for the individuals of the logic. We fix a set of sorted individual constants (ranged over by ) from which we build the set of individuals of the logic:

We also fix a set of sorted predicates (ranged over by ) from which we define the formulas of the logic:

The set of sorts, individuals and formulas of the logic is parameterized by a signature:
\begin{defi}
A signature  is a set of base sorts together with a set of sorted constant individuals and a set of sorted predicates.
\end{defi}
Negation is defined as . We choose to have only negative connectives, since the interpretation of our logic in categories of continuations defined in section~\ref{LmInterp} is based on a negative call-by-name continuation-passing-style translation. The positive connectives are defined from the negative ones:  and . It is well-known that with this coding of positive connectives with negative ones, a formula  is provable in our system if and only if the formula obtained by replacing every  with  in  is provable in its intuitionistic restriction. In section~\ref{RelPred} we will also introduce the notion of negative basic predicates, which are those for which  is valid under the realizability interpretation. We perform a more detailed comparison between our system and the usual ones in section~\ref{usualtheories}.\par
Since one of the goals of realizability is to provide a computational interpretation of theories beyond pure first-order logic, our model is dependent upon the particular set of axioms under consideration:
\begin{defi}
A theory on a given signature  is a set  of closed formulas (axioms) written in the language defined by .
\end{defi}
We work in a variant of natural deduction in sequent style, so the interpretation of classical proofs in -calculus is as direct as possible. In this setting, a context  or  is a finite unordered sequence of formulas and a sequent is of the form:

The formula  on the right is the formula that is being worked on, and this presentation is again chosen to have an easy interpretation in -calculus. The above sequent should be interpreted as: the conjunction of the formulas of  implies the disjunction of  and of the formulas of . If  is empty we simply write . The set of derivable sequents of a given theory is defined from  using the rules of Figure~\ref{LogicRules}.
\begin{figure}
5pt]
\LogRuleImpIntro{\Gamma}{\Delta}{\LogFormA}{\LogFormB}
\qquad
\LogRuleImpElim{\Gamma}{\Delta}{\LogFormA}{\LogFormB}
\5pt]
\LogRuleForallIntro{\Gamma}{\Delta}{\LogSortedTerm{\LogVarA}{\SortA}}{\LogFormA}
\qquad
\LogRuleForallElim{\Gamma}{\Delta}{\LogSortedTerm{\LogVarA}{\SortA}}{\LogSortedTerm{\LogTermA}{\SortA}}{\LogFormA}
\Sequent{}{\LogBot\LogImp\LogFormA}{}\qquad\Sequent{}{\neg\left(\neg\LogFormA\right)\LogImp\LogFormA}{}\qquad\Sequent{}{\left(\left(\LogFormA\LogImp\LogFormB\right)\LogImp\LogFormA\right)\LogImp\LogFormA}{}\AXM{\Sequent{\Gamma}{\LogFormA}{\Delta}}\RLM{\Gamma\subseteq\Gamma',\Delta\subseteq\Delta'}\UIM{\Sequent{\Gamma'}{\LogFormA}{\Delta'}}\DP\AXM{\Sequent{\Gamma,\LogFormA,\LogFormA}{\LogFormB}{\Delta}}\UIM{\Sequent{\Gamma,\LogFormA}{\LogFormA\LogImp\LogFormB}{\Delta}}\AXM{}\UIM{\Sequent{\Gamma,\LogFormA}{\LogFormA}{\Delta}}\BIM{\Sequent{\Gamma,\LogFormA}{\LogFormB}{\Delta}}\DP\qquad\qquad\qquad\AXM{\Sequent{\Gamma}{\LogFormA}{\LogFormB,\LogFormB,\Delta}}\RLM{\LogFormB,\LogFormB,\Delta\subseteq\LogFormA,\LogFormB,\LogFormB,\Delta}\UIM{\Sequent{\Gamma}{\LogFormA}{\LogFormA,\LogFormB,\LogFormB,\Delta}}\UIM{\Sequent{\Gamma}{\LogBot}{\LogFormA,\LogFormB,\LogFormB,\Delta}}\UIM{\Sequent{\Gamma}{\LogFormB}{\LogFormA,\LogFormB,\Delta}}\UIM{\Sequent{\Gamma}{\LogBot}{\LogFormA,\LogFormB,\Delta}}\UIM{\Sequent{\Gamma}{\LogFormA}{\LogFormB,\Delta}}\DP\LogAxioms\Derives\LogFormA\CAReflName\quad\CARefl{\SortA}\qquad\CALeibName\quad\CALeib{\SortA}{\LogFormA}{\SortB}\forall\LogVarA\,\forall\LogVarB\left(\LogVarA=\LogVarB\LogImp\forall\LogVarC\left(\LogVarC\,\LogVarA=\LogVarC\,\LogVarB\right)\right)\qquad\qquad\forall\LogVarA\,\forall\LogVarB\left(\LogVarA=\LogVarB\LogImp\forall\LogVarC\left(\LogVarA\,\LogVarC=\LogVarB\,\LogVarC\right)\right)
\begin{aligned}
\CALogs&\quad\text{of sort}\quad\CALogsSort{\SortA}{\SortB}{\SortC}\quad&\quad\CALogk&\quad\text{of sort}\quad\CALogkSort{\SortA}{\SortB}\\
\CALogZ&\quad\text{of sort}\quad\CALogZSort\quad&\quad\CALogS&\quad\text{of sort}\quad\CALogSSort
\end{aligned}\\
\CALogrec\quad\text{of sort}\quad\CALogrecSort{\SortA}
\LogSortedTerm{\LogTermA}{\SortA}\neq\LogSortedTerm{\LogTermB}{\SortA}
\CAReflName\;\;\;&\CARefl{\SortA}&\CALeibName\;\;\;&\CALeib{\SortA}{\LogFormA}{\SortB}\\
\CASnZName\;\;\;&\CASnZNoType&\CAindUnrelName\;\;\;&\CAindUnrelNoType{A}\\
\CAdefsName\;\;\;&\CAdefsNoType&\CAdefkName\;\;\;&\CAdefkNoType\\
\CAdefrecZName\;\;\;&\CAdefrecZNoType&\CAdefrecSName\;\;\;&\CAdefrecSNoType
\forall\LogVarA\,\forall\LogVarB\left(\CALogS\,\LogVarA=\CALogS\,\LogVarB\LogImp\LogVarA=\LogVarB\right)\forall\LogVarA\,\forall\LogVarB\left(\forall\LogVarC\left(\LogVarC\,\LogVarA=\LogVarC\,\LogVarB\right)\LogImp\LogVarA=\LogVarB\right)\xymatrix@!0{
&\PA_=\ar@{-}[rr]&&\HA_=\\
\PA\ar@{-}[ur]\ar@{-}[rr]&&\HA\ar@{-}[ur]\\
&\PA^\omega_=\ar@{-}[uu]\ar@{-}[rr]&&\HA^\omega_=\ar@{-}[uu]\\
\PA^\omega\ar@{-}[uu]\ar@{-}[ur]\ar@{-}[rr]&&\HA^\omega\ar@{-}[uu]\ar@{-}[ur]
}\forall\LogSortedTerm{\vec{\LogVarC}}{\vec{\SortB}}\forall\LogSortedTerm{\LogVarA}{\SortA}\forall\LogSortedTerm{\LogVarB}{\SortA}\left(\LogVarA=\LogVarB\LogImp\LogFormA\LogImp\LogFormA\LogSubst{\LogVarB/\LogVarA}\right)\forall\LogSortedTerm{\vec{\LogVarC}}{\vec{\SortB}}\forall\LogSortedTerm{\LogVarA}{\SortA}\forall\LogSortedTerm{\LogVarB}{\SortA}\left(\neg\LogFormA\LogImp\LogFormA\LogSubst{\LogVarB/\LogVarA}\LogImp\LogVarA\neq\LogVarB\right)\forall\LogSortedTerm{\LogVarA}{\CASort}\forall\LogSortedTerm{\LogVarB}{\CASort}\left(\CALogS\,\LogVarA=\CALogS\,\LogVarB\LogImp\LogVarA=\LogVarB\right)\forall\LogSortedTerm{\LogVarA}{\CASort}\forall\LogSortedTerm{\LogVarB}{\CASort}\left(\LogVarA\neq\LogVarB\LogImp\CALogS\,\LogVarA\neq\CALogS\,\LogVarB\right)\CADCUnrelName\quad\CADCUnrel{\SortA}{\LogFormA}{\SortB}\forall\LogSortedTerm{\vec{\LogVarD}}{\vec{\SortB}}\left(\forall\LogSortedTerm{\LogVarA}{\CASort}\,\exists\LogSortedTerm{\LogVarB}{\SortA}\,\LogFormB\;\;\LogImp\;\;\exists\LogSortedTerm{\LogVarE}{\CASort\SortTo\SortA}\,\forall\LogSortedTerm{\LogVarA}{\CASort}\,\LogFormB\LogSubst{\LogVarE\,\LogVarA/\LogVarB}\right)\forall\LogSortedTerm{\vec{\LogVarD}}{\vec{\SortB}}\left(\forall\LogSortedTerm{\LogVarA}{\CASort}\,\forall\LogSortedTerm{\LogVarB}{\SortA}\,\exists\LogSortedTerm{\LogVarC}{\SortA}\,\LogFormA\LogSubst{\LogVarA,\LogVarB,\LogVarC}\;\;\LogImp\;\;\forall\LogSortedTerm{\LogVarF}{\SortA}\exists\LogSortedTerm{\LogVarE}{\CASort\SortTo\SortA}\left(\LogVarE\,\CALogZ=\LogVarF\LogAnd\forall\LogSortedTerm{\LogVarA}{\CASort}\,\LogFormA\LogSubst{\LogVarA,\LogVarE\,\LogVarA,\LogVarE\left(\CALogS\,\LogVarA\right)}\right)\right)\ModM\Models\LogFormA\ModMInterp{\neq_\SortA}\quad\Def\quad\SetSuch{\left(\ModElemA,\ModElemB\right)\in\ModMInterp{\SortA}\times\ModMInterp{\SortA}}{\ModElemA\neq\ModElemB}\LmSortA,\LmSortB\GramDef\LmSortBase\BarSep\LmSortA\LmSortTo\LmSortB\BarSep\LmSortA\LmSortTimes\LmSortB\BarSep\LmSortBot\Sequent{\LmTerm{\LmVarA_1}{\LmSortA_1},\ldots,\LmTerm{\LmVarA_n}{\LmSortA_n}}{\LmTerm{\LmTermA}{\LmSortA}}{\LmTerm{\LmMVarA_1}{\LmSortB_1},\ldots,\LmTerm{\LmMVarA_m}{\LmSortB_m}}\Sequent{\LmTerm{\LmVarA_1}{\LmSortA_1},\ldots,\LmTerm{\LmVarA_n}{\LmSortA_n}}{\LmTerm{\LmTermA}{\LmSortA}}{}
\LmRuleAx{\LmTerm{\vec{\LmVarA}}{\vec{\LmSortA}}}{\LmTerm{\vec{\LmMVarA}}{\vec{\LmSortB}}}{\LmVarA}{\LmSortA}
\qquad
\LmRuleAxSig{\LmTerm{\vec{\LmVarA}}{\vec{\LmSortA}}}{\LmTerm{\vec{\LmMVarA}}{\vec{\LmSortB}}}{\LmConstA}{\LmSortA}
\5pt]
\LmRuleAndIntro{\LmTerm{\vec{\LmVarA}}{\vec{\LmSortA}}}{\LmTerm{\vec{\LmMVarA}}{\vec{\LmSortB}}}{\LmTermA}{\LmSortA}{\LmTermB}{\LmSortB}
\qquad
\LmRuleAndElim{\LmTerm{\vec{\LmVarA}}{\vec{\LmSortA}}}{\LmTerm{\vec{\LmMVarA}}{\vec{\LmSortB}}}{\LmTermA}{\LmSortA}
\
\caption{Typing rules of -calculus}
\label{TypingRules}
\end{figure}
The left and right weakening rules are admissible in that type system, and we use them without explicitly mentioning it. Here are some examples of derivable typing judgments.
5pt]
\Sequent{}{\LmTerm{\lambda\LmVarB.\mu\LmMVarA.[\LmMVarA]\,\LmVarB\left(\lambda\LmVarA.\mu\LmMVarB.[\LmMVarA]\,\LmVarA\right)}{\left(\left(\LmSortA\LmSortTo\LmSortB\right)\LmSortTo\LmSortA\right)\LmSortTo\LmSortA}}{}
\LmInterpForm{\LogBot}=\LmSortBot\qquad\LmInterpForm{\left(\LogFormA\LogImp\LogFormB\right)}=\LmInterpForm{\LogFormA}\LmSortTo\LmInterpForm{\LogFormB}\qquad\LmInterpForm{\left(\LogFormA\LogAnd\LogFormB\right)}=\LmInterpForm{\LogFormA}\LmSortTimes\LmInterpForm{\LogFormB}\qquad\LmInterpForm{\left(\forall\LogSortedTerm{\LogVarA}{\SortA}\,\LogFormA\right)}=\LmInterpForm{\LogFormA}
\LmInterpProof{\left(\LogRuleAx{\Gamma}{\Delta}{\LogFormA}\right)}&=\LmRuleAx{\LmInterpForm{\Gamma}}{\LmInterpForm{\Delta}}{\LmVarA}{\LmInterpForm{\LogFormA}}\10pt]
\LmInterpProof{\left(\LogRuleImpIntro{\Gamma}{\Delta}{\LogFormA}{\LogFormB}\right)}&=\LmRuleImpIntro{\LmInterpForm{\Gamma}}{\LmInterpForm{\Delta}}{\LmVarA}{\LmInterpForm{\LogFormA}}{\LmTermA}{\LmInterpForm{\LogFormB}}\10pt]
\LmInterpProof{\left(\LogRuleAndIntro{\Gamma}{\Delta}{\LogFormA}{\LogFormB}\right)}&=\LmRuleAndIntro{\LmInterpForm{\Gamma}}{\LmInterpForm{\Delta}}{\LmTermA}{\LmInterpForm{\LogFormA}}{\LmTermB}{\LmInterpForm{\LogFormB}}\10pt]
\LmInterpProof{\left(\LogRuleForallIntro{\Gamma}{\Delta}{\LogSortedTerm{\LogVarA}{\SortA}}{\LogFormA}\right)}&=\AXM{\Sequent{\LmInterpForm{\Gamma}}{\LmTerm{\LmTermA}{\LmInterpForm{\LogFormA}}}{\LmInterpForm{\Delta}}}\DP\10pt]
\LmInterpProof{\left(\LogRuleBotIntro{\Gamma}{\Delta}{\LogFormA}\right)}&=\LmRuleBotIntro{\LmInterpForm{\Gamma}}{\LmInterpForm{\Delta}}{\LmMVarA}{\LmTermA}{\LmInterpForm{\LogFormA}}\
\caption{Interpretation of classical proofs in -calculus}
\label{ProofInterp}
\end{figure}
\subsubsection{\texorpdfstring{}{lambda-mu} theories}
\label{lmtheo}
We follow~\cite{SelingerControl} and define -calculus as an equational theory. We only consider here the case of call-by-name semantics. The axioms of the call-by-name -calculus are given in Figure~\ref{LambdaMuAxioms}, where in each equation the two terms are typed with the same type.
\begin{figure}


\caption{Axioms of the call-by-name -calculus}
\label{LambdaMuAxioms}
\end{figure}
In the equations ,  and ,  is a placeholder for the term coming after , i.e.  is obtained by replacing in  all the subterms of the form  with . From these axioms  we define the notion of  theory:
\begin{defi}[ theory]
A  theory is a set of equations between typed terms of the same type (with free variables of the same type) which contains the axioms of call-by-name -calculus and is a congruence (contextually-closed equivalence relation).
\end{defi}
We use here a slightly different set of axioms from that of Selinger~\cite{SelingerControl}. Our  and  equations are the  and  equations of Selinger, and we replace his  (which is  if ) with our . However, the two systems are equivalent:
\begin{lem}
Under the contextual closures of  and  (the  and  of Selinger), the equation  of Selinger is equivalent to .
\end{lem}
\proof
Suppose  holds, let  and  be a -variable of type . Using the contextual closure of , we have , and again by  we get . Then by contextual closure , which is equal to  using , since  does not appear free in .\par
Conversely, suppose  holds, let ,  be a -variable of type  and  be a -variable of type  which does not appear free in . Using  we have , so by contextual closure , so by  we get , but  since  does not appear free in , and we get finally .
\qed
Replacing Selinger's  allows us to have a more symmetric calculus, with a set of three equations for each type constructor, the  equations representing the transmission of the context of a  to the subterms .
\subsubsection{\texorpdfstring{}{mu}PCF}
\label{muPCF}
We give here an example of a  theory called PCF, which will also be the language in which we will interpret Peano arithmetic and the axiom of choice. Programming language for Computable Functions (PCF) is a functional programming language described by Plotkin in~\cite{PlotkinPCF}. It is based on Scott's Logic for Computable Functions (LCF), which was presented in~\cite{ScottLCF}. The language contains constants for natural numbers and general recursion. It is probably the simplest example of a Turing-complete higher-order language. Here we consider an extension of PCF to primitively handle control operators, by presenting PCF as a  theory. In \cite{OngStewartControl} the authors define a call-by-value semantics for -calculus, and they illustrate it with PCF, a call-by-value version of PCF with control. Later on, Laird defines in \cite{LairdThesis} its call-by-name version, which is the version we use here. The choice of this language is justified by our will to get computational content directly from classical proofs.\par
Our version of PCF has only one base type for natural numbers: , and the constants are:

It will also be useful to have a canonical term on each type so we define:

This term represents non-termination, or ``undefined''. The equations of PCF given in Figure~\ref{muPCFEq} are standard and include the interactions of the constants with the  operator.
\begin{figure}

\caption{Equations of PCF}
\label{muPCFEq}
\end{figure}
We can finally define PCF as the  theory generated by these equations:
\begin{defi}[PCF]
The  theory PCF is the smallest  theory containing the equations of Figure~\ref{muPCFEq}.
\end{defi}
\subsection{Categories of continuations}
\label{catcont}
Categories of continuations are to call-by-name -calculus what cartesian closed categories are to -calculus, in the sense that if we fix a signature, there is a one-to-one correspondence between  theories and categories of continuations together with an interpretation of the signature. Ong defines  categories in~\cite{OngClassicalProofs} by reformulating the syntax of -calculus in categorical terms. Later on, Hofmann and Streicher proved the soundness and completeness of categories of continuations with respect to -calculus, providing the first abstract version of -categories. Finally, Selinger axiomatized these in~\cite{SelingerControl} under the name of control categories, proving that they are equivalent (for a suitable notion of equivalence based on weak functors) to categories of continuations, and therefore sound and complete with respect to call-by-name -calculus. Moreover, he proved that the categorical dual of control categories are sound and complete with respect to call-by-value -calculus. Here we are only interested in the call-by-name version, and we use the model of categories of continuations.
\begin{defi}[Category of continuations]
Let  be a distributive category, that is, a category with finite products and coproducts such that the canonical distributivity morphisms from  to  are isomorphisms (which implies that the morphism from  to  is also an isomorphism), and let  be a fixed object such that all exponentials  for  exist. Then the full subcategory  of  consisting of the objects  for  is called a category of continuations.
\end{defi}
We will differentiate morphisms in  and  by writing , \footnote{Pronounced as in   (Doctor Zhivago).}  for morphisms in  and , , ,  for morphisms in . As observed in~\cite{LafontReusStreicher}, a category of continuations  is in particular a cartesian closed category:
\begin{lem}
If  is a category of continuations, then  defines a terminal object and  defines a cartesian product of  and , so  is cartesian. Moreover, if  and , then  defines an exponential in  of  by . Consequently,  is cartesian closed, the exponential of  by  being .
\end{lem}
Be careful that we have two (isomorphic) terminal objects, one in  and one in , and two (isomorphic) products of  and , again one in  and one in . To avoid confusion and without loss of generality we will suppose that they are equal:  and . For the same reason, we also suppose .
\subsubsection{Classical disjunction in categories of continuations.} We could have added a primitive connective  for the disjunction in the logic, with the following rules:

and then interpret these logical rules by adding the following typing rules to -calculus:

These rules are present in~\cite{SelingerControl}, however we choose here to keep things simple and stick to the usual -calculus without disjunction types. Nevertheless we still use the binoidal functor  in categories of continuations to interpret multi-conclusioned sequents. Following~\cite{SelingerControl} we write:  for the interpretation of the classical disjunction between  and ,  for the interpretation of the right weakening rule and  for the interpretation of the right contraction rule.\par
Another interesting fact about categories of continuations is that we can define a functor from  to  which maps  to , and  to  which is the currying of:

The morphism  corresponds to  (where  is the unique morphism from  to the terminal object ) and the morphism  corresponds to  (remember that ). Also, in particular, if  in , then  in  (and in ). Through this functor, the cocartesian structure of  translates to the cartesian structure of .
\subsubsection{Interpretation of \texorpdfstring{}{lambda-mu}-calculus in categories of continuations}
\label{CatInterp}
We describe here the interpretation of call-by-name -calculus in a category of continuations as defined in~\cite{SelingerControl}. We fix a signature of -calculus and a category of continuations . To each type  of -calculus we associate an object  of continuations of type , and an object  of computations of type . The objects  where  is a base type of the signature are parameters of the interpretation, and we define inductively:

We have in particular  where  is the cartesian product in  defined above, , using the definition of the exponential in  given above, and .\par
Once we have an interpretation of types in , we define the interpretation of typed -terms such that a term:

is interpreted as a morphism in :

where  associates to the left and  associates to the right. In order to do that, we suppose given for each constant  of the signature a morphism in :

which is again a parameter of the interpretation. These parameters are summarized in the following definition:
\begin{defi}[Interpretation]
Given a signature and a category of continuations , an interpretation of -calculus is given by an object  for each base type  of the signature and a morphism  in  for each constant  of the signature.
\end{defi}
We now have all necessary material to interpret every typed -term as a morphism in . The interpretation of typed -terms is almost identical to the interpretation of -calculus in a cartesian closed category (since as shown in the previous section,  is cartesian closed). The first difference is that we must be able to carry over the -context, so we want to build from  a morphism . The second difference is that in order to interpret the introduction rules for  and , we also need to have canonical morphisms from  to  and from  to . These requirements are axiomatized in~\cite{SelingerControl}, to which we refer for the full definition of the interpretation, and the proof that the axioms of call-by-name -calculus are sound under this interpretation.\par
A model of a  theory is then a sound interpretation:
\begin{defi}[Model of a  theory]
\label{LambdaMuModel}
A model of a  theory is a category of continuations together with an interpretation for the  signature such that every equation  of the theory is true in the model: .
\end{defi}
If the  theory is generated from a given set of equations, then any interpretation satisfying these equations is a model of the  theory.\par
Since call-by-name -calculus is the internal language of categories of continuations (as shown in~\cite{SelingerControl}), we can apply -calculus constructs on morphisms of  through the use of -terms with parameters in . Therefore, we will also drop the interpretation brackets for terms. For example, if  and , then , where formally  is the term with parameters .
\subsubsection{Connection with the call-by-name CPS translation of \texorpdfstring{}{lambda-mu}-calculus}
\label{cps}
Another interesting thing about the interpretation of -calculus into a category of continuations is the exact correspondence with the interpretation of its call-by-name CPS translation in the underlying cartesian ``-closed'' category, as stressed in~\cite{StreicherContinuation}. The target of such a translation is a simply-typed -calculus  with product and sum types, a particular base type , and the function types being restricted to . This particular -calculus can be interpreted in  by interpreting the product type as the product in , the sum type as the coproduct, and using the fact that function types are of the form , so having all exponentials  in  is enough.\par
To be more precise,  has one base type  for each base type  of -calculus and another particular base type . From these we build the types:

The arrow types are syntactically restricted to be of the form . We map every type  of -calculus to a type  of  (each base type  being obviously mapped to ) as follows:

 also has one constant  for each constant  of the source language. We also suppose given for each -variable  of the source language a variable  in the target language, and for each -variable  of the source language a variable  in the target language. A typed -term:

will then be translated to a typed -term:

Before defining the translation, we give the typing rules of  in Figure~\ref{LamRules}.
\begin{figure}
5pt]
\LmRuleImpIntro{\LmTerm{\vec{\LamVarA}}{\vec{\LamTypeA}}}{}{\LamVarA}{\LamTypeA}{\LamTermA}{\LamTypeR}
\qquad
\LmRuleImpElim{\LmTerm{\vec{\LamVarA}}{\vec{\LamTypeA}}}{}{\LamTermA}{\LamTypeR}{\LamTermB}{\LamTypeA}
\5pt]
\LamRuleOrIntro{\Gamma}{\LamTermA}{\LamTypeA}
\qquad
\LamRuleOrElim{\Gamma}{\LamVarA}{\LamTermA}{\LamTermB}{\LamTypeA}{\LamTypeB}

\LamCPS{\lambda\LmVarA.\LmTermA}&=\lambda\LamVarA.\left(\lambda\LamCPS{\LmVarA}.\LamCPS{\LmTermA}\right)\left(\LmProj_1\,\LamVarA\right)\left(\LmProj_2\,\LamVarA\right)&\LamCPS{\LmTermA\,\LmTermB}&=\lambda\LamVarA.\LamCPS{\LmTermA}\left\langle\LamCPS{\LmTermB},\LamVarA\right\rangle\\
\LamCPS{\LmPair{\LmTermA}{\LmTermB}}&=\lambda\LamVarA.\LamCase{\LamVarA}{\LamVarB}{\LamCPS{\LmTermA}\,\LamVarB}{\LamCPS{\LmTermB}\,\LamVarB}&\LamCPS{\LmProj_i\,\LmTermA}&=\lambda\LamVarA.\LamCPS{\LmTermA}\,\LamIn_i\,\LamVarA\\
\LamCPS{\mu\LmMVarA.\LmTermA}&=\lambda\LamCPS{\LmMVarA}.\LamCPS{\LmTermA}\,\LamUnit&\LamCPS{\left[\LmMVarA\right]\,\LmTermA}&=\lambda\LamVarA.\LamCPS{\LmTermA}\,\LamCPS{\LmMVarA}
\CatInterpSortNeg{\LamCPS{\LmSortBase}}=\CatInterpSortNeg{\LmSortBase}\;\quad\CatInterpSortNeg{\LamTypeA\LamTypeTo\LamTypeR}=\CatExp{\CatR}{\CatInterpSortNeg{\LamTypeA}}\;\quad\CatInterpSortNeg{\LamTypeA\LamTypeTimes\LamTypeB}=\CatInterpSortNeg{\LamTypeA}\LamTypeTimes\CatInterpSortNeg{\LamTypeB}\;\quad\CatInterpSortNeg{\LamTypeUnit}=\CatTerm\;\quad\CatInterpSortNeg{\LamTypeA\LamTypePlus\LamTypeB}=\CatInterpSortNeg{\LamTypeA}\CatPlus\CatInterpSortNeg{\LamTypeB}\LamCPS{\LmTermA}:\CatExp{\CatR}{\CatInterpSortNeg{\LamCPS{\LmSortA_1}}}\CatTimes\ldots\CatTimes\CatExp{\CatR}{\CatInterpSortNeg{\LamCPS{\LmSortA_n}}}\CatTimes\CatInterpSortNeg{\LamCPS{\LmSortB_1}}\CatTimes\ldots\CatTimes\CatInterpSortNeg{\LamCPS{\LmSortB_m}}\to\CatExp{\CatR}{\CatInterpSortNeg{\LamCPS{\LmSortA}}}\LamCPS{\LmTermA}:\CatExp{\CatR}{\CatInterpSortNeg{\LmSortA_1}}\CatTimes\ldots\CatTimes\CatExp{\CatR}{\CatInterpSortNeg{\LmSortA_n}}\CatTimes\CatInterpSortNeg{\LmSortB_1}\CatTimes\ldots\CatTimes\CatInterpSortNeg{\LmSortB_m}\to\CatExp{\CatR}{\CatInterpSortNeg{\LmSortA}}\CatLambda\left(\LamCPS{\LmTermA}\right):\CatExp{\CatR}{\CatInterpSortNeg{\LmSortA_1}}\CatTimes\ldots\CatTimes\CatExp{\CatR}{\CatInterpSortNeg{\LmSortA_n}}\to\CatExp{\CatR}{\CatInterpSortNeg{\LmSortA}\CatTimes\CatInterpSortNeg{\LmSortB_1}\CatTimes\ldots\CatTimes\CatInterpSortNeg{\LmSortB_m}}\CatLambda\left(\LamCPS{\LmTermA}\right)=\LmTermA{2}
\left(\beta_\LamTypeTo^\lambda\right)\quad&\left(\lambda\LamVarA.\LamTermA\right)\LamTermB=\LamTermA\LamSubst{\LamTermB/\LamVarA}
\qquad&
\left(\eta_\LamTypeTo^\lambda\right)\quad&\lambda\LamVarA.\LamTermA\,\LamVarA=\LamTermA\quad\left(\LamVarA\notin\FV{\LamTermA}\right)
\\
\left(\beta_\LamTypeTimes^\lambda\right)\quad&\LmProj_i\LmPair{\LamTermA_1}{\LamTermA_2}=\LamTermA_i
\qquad&
\left(\eta_\LamTypeTimes^\lambda\right)\quad&\LmPair{\LmProj_1\,\LamTermA}{\LmProj_2\,\LamTermA}=\LamTermA
\\
\left(\beta_\LamTypePlus^\lambda\right)\quad&\LamCaseBlock{\left(\LamIn_i\,\LamTermA\right)}{\LamVarA}{\LamTermB_1}{\LamTermB_2}=\LamTermB_i\LamSubst{\LamTermA/\LamVarA}
\qquad&
\left(\eta_\LamTypePlus^\lambda\right)\quad&\LamCaseBlock{\LamTermA}{\LamVarA}{\LamIn_1\,\LamVarA}{\LamIn_2\,\LamVarA}=\LamTermA
\\
\quad&
\qquad&
\left(\eta_\LamTypeUnit^\lambda\right)\quad&\LamUnit=\LamTermA

\left[\LmPair{\LamCPS{\CatRCHomA}}{\CatCHomA}\right]\,\CatRCHomB&=\left[\CatCHomA\right]\,\CatRCHomB\,\CatRCHomA&\text{if }&\left\{\begin{gathered}\CatRCHomA:\prod_{j\in J}\CatExp{\CatR}{\CatObjA_j}\to\CatExp{\CatR}{\CatObjB}\CatPar\left(\bigparr_{k\in K}\CatExp{\CatR}{\CatObjD_k}\right)\\\CatRCHomB:\prod_{j\in J}\CatExp{\CatR}{\CatObjA_j}\to\CatExp{\CatR}{\CatExp{\CatR}{\CatObjB}\CatTimes\CatObjC}\CatPar\left(\bigparr_{k\in K}\CatExp{\CatR}{\CatObjD_k}\right)\\\CatCHomA:\left(\prod_{j\in J}\CatExp{\CatR}{\CatObjA_j}\right)\CatTimes\left(\prod_{k\in K}\CatObjD_k\right)\to\CatObjC\end{gathered}\right.\\
\left[\LamUnit\right]\,\CatRCHomA&=\CatRCHomA&\text{if }&\CatRCHomA:\prod_{j\in J}\CatExp{\CatR}{\CatObjA_j}\to\CatExp{\CatR}{\CatTerm}\CatPar\left(\bigparr_{k\in K}\CatExp{\CatR}{\CatObjD_k}\right)\\
\left[\LamIn_i\,\CatCHomA\right]\,\CatRCHomA&=\left[\CatCHomA\right]\,\LmProj_i\,\CatRCHomA&\text{if }&\left\{\begin{gathered}\CatRCHomA:\prod_{j\in J}\CatExp{\CatR}{\CatObjA_j}\to\left(\CatExp{\CatR}{\CatObjB_1}\CatTimes\CatExp{\CatR}{\CatObjB_2}\right)\CatPar\left(\bigparr_{k\in K}\CatExp{\CatR}{\CatObjD_k}\right)\\\CatCHomA:\left(\prod_{j\in J}\CatExp{\CatR}{\CatObjA_j}\right)\CatTimes\left(\prod_{k\in K}\CatObjD_k\right)\to\CatObjB_i\end{gathered}\right.
\LogRel{\LogSortedTerm{.}{\SortBase}}\LogRel{\LogSortedTerm{\LogTermA}{\SortA\SortTo\SortB}}\Def\forall\LogSortedTerm{\LogVarA}{\SortA}\,\LogRel{\LogSortedTerm{\LogVarA}{\SortA}}\LogImp\LogRel{\LogSortedTerm{\LogTermA\,\LogVarA}{\SortB}}\qquad\text{(so in particular }\LmInterpForm{\LogRel{\LogSortedTerm{.}{\SortA\SortTo\SortB}}}=\LmInterpForm{\LogRel{\LogSortedTerm{.}{\SortA}}}\LmSortTo\LmInterpForm{\LogRel{\LogSortedTerm{.}{\SortB}}}\text{)}\LogForallRel\LogSortedTerm{\LogVarA}{\SortA}\,\LogFormA\Def\forall\LogSortedTerm{\LogVarA}{\SortA}\left(\LogRel{\LogSortedTerm{\LogVarA}{\SortA}}\LogImp\LogFormA\right)\qquad\LogExistsRel\LogSortedTerm{\LogVarA}{\SortA}\,\LogFormA\Def\neg\LogForallRel\LogSortedTerm{\LogVarA}{\SortA}\,\neg\LogFormA\LogRelForm{\LogPredA\left(\LogSortedTerm{\LogTermA_1}{\SortA_1},\ldots,\LogSortedTerm{\LogTermA_n}{\SortA_n}\right)}\Def\LogPredA\left(\LogSortedTerm{\LogTermA_1}{\SortA_1},\ldots,\LogSortedTerm{\LogTermA_n}{\SortA_n}\right)\qquad\LogRelForm{\LogBot}\Def\LogBot\\\LogRelForm{\left(\LogFormA\LogImp\LogFormB\right)}\Def\LogRelForm{\LogFormA}\LogImp\LogRelForm{\LogFormB}\qquad\LogRelForm{\left(\LogFormA\LogAnd\LogFormB\right)}\Def\LogRelForm{\LogFormA}\LogAnd\LogRelForm{\LogFormB}\qquad\LogRelForm{\left(\forall\LogSortedTerm{\LogVarA}{\SortA}\,\LogFormA\right)}\Def\LogForallRel\LogSortedTerm{\LogVarA}{\SortA}\,\LogRelForm{\LogFormA}
\LogNeg{\LogFormA},\LogNeg{\LogFormB}&\GramDef\LogNeg{\LogPredA}\left(\LogSortedTerm{\LogTermA_1}{\SortA_1},\ldots,\LogSortedTerm{\LogTermA_n}{\SortA_n}\right)\BarSep\LogBot\BarSep\LogFormA\LogImp\LogNeg{\LogFormB}\BarSep\LogNeg{\LogFormA}\LogAnd\LogNeg{\LogFormB}\BarSep\forall\LogSortedTerm{\LogVarA}{\SortA}\LogNeg{\LogFormA}\\
\LogPos{\LogFormA},\LogPos{\LogFormB}&\GramDef\LogPos{\LogPredA}\left(\LogSortedTerm{\LogTermA_1}{\SortA_1},\ldots,\LogSortedTerm{\LogTermA_n}{\SortA_n}\right)\BarSep\LogFormA\LogImp\LogPos{\LogFormB}\BarSep\LogPos{\LogFormA}\LogAnd\LogFormB\BarSep\LogFormA\LogAnd\LogPos{\LogFormB}\BarSep\forall\LogSortedTerm{\LogVarA}{\SortA}\LogPos{\LogFormA}
\Sequent{\Gamma}{\LogFormA}{\Delta}\LmInterpForm{\LogNeg{\LogPredA}}\Def\LmSortBot\LogAxioms\Derives\LogFormA\qquad\Longrightarrow\qquad\LogRelForm{\LogAxioms}\Derives\LogRelForm{\LogFormA}\Sequent{\Gamma}{\LogFormA}{\LogNeg{\Delta}}\Sequent{\LogRel{\LogSortedTerm{\vec{\LogVarA}}{\vec{\SortA}}},\LogRelForm{\Gamma}}{\LogRelForm{\LogFormA}}{\LogRelForm{\LogNeg{\Delta}}}\LogRuleAx{\Gamma}{\LogNeg{\Delta}}{\LogFormA}\;\rightsquigarrow\;\LogRuleAx{\LogRel{\LogSortedTerm{\vec{\LogVarA}}{\vec{\SortA}}},\LogRelForm{\Gamma}}{\LogRelForm{\LogNeg{\Delta}}}{\LogRelForm{\LogFormA}}
\LogRuleBotIntro{\Gamma}{\LogNeg{\Delta}}{\LogNeg{\LogFormA}}&\;\;\rightsquigarrow\;\;\LogRuleBotIntro{\LogRel{\LogSortedTerm{\vec{\LogVarA}}{\vec{\SortA}}},\LogRelForm{\Gamma}}{\LogRelForm{\LogNeg{\Delta}}}{\LogRelForm{\LogNeg{\LogFormA}}}\5pt]
\LogRuleImpIntro{\Gamma}{\LogNeg{\Delta}}{\LogFormA}{\LogFormB}&\;\;\rightsquigarrow\;\;\LogRuleImpIntro{\LogRel{\LogSortedTerm{\vec{\LogVarA}}{\vec{\SortA}}},\LogRelForm{\Gamma}}{\LogRelForm{\LogNeg{\Delta}}}{\LogRelForm{\LogFormA}}{\LogRelForm{\LogFormB}}\5pt]
\LogRuleAndIntro{\Gamma}{\LogNeg{\Delta}}{\LogFormA}{\LogFormB}&\;\;\rightsquigarrow\;\;\LogRuleAndIntro{\LogRel{\LogSortedTerm{\vec{\LogVarA}}{\vec{\SortA}}},\LogRelForm{\Gamma}}{\LogRelForm{\LogNeg{\Delta}}}{\LogRelForm{\LogFormA}}{\LogRelForm{\LogFormB}}\
\caption{Relativization of the rules for logical connectives}
\label{RelatRules}
\end{figure}
where in the cases of introduction of implication and conjunction, the two premises can get the same relativized variables on the left by applying the (admissible) left weakening rule when necessary. The translation of the introduction of universal quantification is given by:

and preserves the fact that all the free variables of a sequent are relativized in the context. Finally, in order to translate the elimination of the universal quantification we must prove that we can lift relativization to all constructs on individuals. This can be obtained using the hypothesis of the lemma requiring that for every individual constant  of , . Indeed, we have the following lemma:
\begin{lem}
\label{relindiv}
Suppose that for every individual constant  of , . Let  be an individual of the logic with free variables . We have:

\end{lem}
\proof
We prove it by induction on . If  is some , then this is an assumption of the lemma. If  is a variable , then , which is trivially derivable, and if  is , then we get  and  from the induction hypotheses, so we obtain  and .
\qed
Now we can describe the translation of the elimination rule of the universal quantifier:

where we can suppose without loss of generality that  (using the admissible left weakening rule if necessary), and  is easily derivable using  (with ) from lemma~\ref{relindiv}.\par
Finally, the negativeness of the formulas in the right-hand context is preserved through the relativization, thanks to lemma~\ref{RelNeg}.
\qed
\subsubsection{Relativized \texorpdfstring{}{PAomega}: \texorpdfstring{}{PAomegar}}
We now define the relativized version  of , to which we apply lemma~\ref{RelTheo}. First,  is , that is  augmented with the positive predicate symbol . The axioms of  are those of  (unrelativized) where the induction scheme is replaced with:

plus the axioms  and . Notice that  is different from  (the parameters  are not relativized). In order to use lemma~\ref{RelTheo} we need the following lemmas:
\begin{lem}

\end{lem}
\proof
The formulas  and  are provable using the axioms of equality and respectively  and . Indeed,  is the following formula:

which is equivalent in first-order logic to:

using  and  this is equivalent to:

and since  and  are the following formulas:

we can instantiate these with ,  and  to obtain a proof of . The case of  is similar.\par
Similarly,  is provable using the axioms of equality, ,  and .  is equivalent in first-order logic to:

if we instantiate  with  it is sufficient to prove:

which is equivalent, using ,  and , to:

which is an instance of  with the formula .\par
Moreover, these proofs of ,  and  respect the condition of having only negative formulas in the right-hand context, since this context is empty (which means that the proofs are valid in minimal logic).
\qed
\begin{lem}
For any , .
\end{lem}
\proof
For , , ,  and  it follows from the fact that the formula  is derivable in first-order logic. For  it comes from this and the fact that the following formula:

is an instance of . Here again, the right-hand context is empty so the proofs are correct.
\qed
\begin{lem}
For every sort  on , there is a closed individual .
\end{lem}
\proof
 from section~\ref{PeanoTheory} is such a term.
\qed
Using these three lemmas, it follows from lemma~\ref{RelTheo} that for any closed formula  on the signature :

\subsubsection{Interpreting \texorpdfstring{}{PAomegar} in system T + \texorpdfstring{}{omega}}
\label{LmInterpPA}
System T was introduced by G\"odel in~\cite{GodelDialectica} in order to give a consistency proof of Heyting arithmetic (and therefore of Peano arithmetic by double-negation translation). This system can be equivalently formulated as a system of primitive recursive functionals, which is an extension of primitive recursive functions to higher types. It is strictly more powerful than primitive recursion, since for example the Ackermann's function is expressible in system T.\par
System T has one base type  for natural numbers, product and function types, constants for  and successor, and a recursion operator of type  for any type . Restricting the type of the recursor to  gives back the usual primitive recursive functions.\par
Since PCF contains constants for every natural number, successor, predecessor and general recursion, it is easy to encode system T in it. Indeed, if we define:

Then it is easy to derive:

and in order to prove that it implements G\"odel's recursor we must prove that it satisfies the corresponding equations:
\begin{lem}
Let  and . We have:

and for any :

\end{lem}
\proof
This follows easily from the definition of  and the equations of PCF for ,  and .
\qed
Therefore in the following we will consider system T as a subsystem of PCF. We will also use the constant  to interpret the fact that  is not a successor. In order to interpret  in system T + , we first fix the interpretation of the relativization predicate:

The inequality predicate being a negative one, it is interpreted as . Finally, we provide a term  of type  for each  in Figure~\ref{AxiomTrans}, where  is a formula with free variables among .
\begin{figure}



\caption{Interpretation of the axioms of }
\label{AxiomTrans}
\end{figure}
\subsubsection{Relativized axiom of choice}
\label{RelCAom}
As we did for , we define the relativized version  of . First, the signature is the same as : it is  augmented with a positive predicate symbol . In this section,  denotes a formula over  with free variables among . For clarity, we write  instead of . The axioms of  are those of  plus the following version of dependent choice:

where  is of the shape . The formula  in  sould be understood as an abbreviation, and the axiom is schematic in . This version is quite different from  which is:

where  is a formula over  with free variables among , for which we use again the notation . First, some quantifications have been unrelativized, but another important difference is that we replaced  with . This change is in the spirit of~\cite{EscardoOlivaPeirce} and will allow an easier realizability interpretation in section~\ref{ChoiceAdequacy}. In order to use lemma~\ref{RelTheo}, we need to prove that  is derivable in . We will actually prove that the instance of  with:

implies  in first-order logic. This is indeed an instance of  since  is of the shape . Since  is derivable, it is sufficient to prove the following lemma:
\begin{lem}
The following formula over the signature :

where , is provable in the logical system with relativization defined in section~\ref{RelPred}.
\end{lem}
\proof
We do this by proving the following two sequents:

\begin{itemize}
\item For the first one, we suppose , ,  and , and we want to prove .  is an hypothesis and we deduce  from  (which is valid since  is a negative formula by lemma~\ref{RelNeg} and therefore  is derivable) by applying the hypothesis  with  and , so the only thing left to prove is . We derive it from  and , which amounts to proving the following sequent:

which is immediate
\item The second sequent can be rewritten as:

Therefore, it is sufficient to prove:

which is after unfolding some definitions:

and this sequent is indeed provable in our logical system.\qed\end{itemize}
Thanks to this lemma, we can now apply lemma~\ref{RelTheo} to get for any  on the signature :

\subsubsection{Interpreting the axiom of choice with bar recursion}
\label{barrecursion}
Bar recursion is an operator which can be seen as recursion on well-founded trees. It was first introduced by Spector in~\cite{Spector} to extend G\"odel's Dialectica interpretation to Heyting arithmetic augmented with the axiom of countable choice. This operator was studied in~\cite{KohlenbachThesis}, and a more uniform operator which is very similar to bar recursion was introduced in~\cite{BerardiBezemCoquand} and used in a realizability setting. A version in which the well-foundedness of trees is implicit was proposed in~\cite{BergerOlivaChoice} under the name of modified bar recursion, and we use this version here. For a comparison between these different forms of bar recursion and other similar principles we refer the reader to~\cite{PowellThesis,PowellEquivalence}.\par
In this section, we first encode lists and list operators in PCF. Then we define the modified bar recursion operator of~\cite{BergerOlivaChoice} that we will use to provide our computational interpretation of the axiom of dependent choice.
\paragraph{Encoding lists in PCF}
In order to define bar recursion, we first need to encode lists and operations on lists in PCF. We choose to represent a list by a natural number (the size of the list) together with a (partial) function on natural numbers. We define the type of lists and notations for the size of a list, the empty list, and the access to a particular element:

If  and , then it is easy to prove:

In order to define extensions of lists, we need subtraction on natural numbers  and tests of equality  and strict ordering , which we can define in PCF. These operators satisfy the following equations for :

We are now able to define the extension of a list by a single element:

as expected, if  and  then ,  and . Finally, we define infinite extension of a list with a constant element:

We can derive from  and  that  and:

\paragraph{The bar recursion operator}
\label{barrec}
We have now all necessary material to define formally the bar recursion operator:

Bar recursor can be typed as expected:

And it verifies indeed the equation:

\paragraph{Interpreting dependent choice using bar recursion}
We use here the bar-recursion operator to provide the term  interpreting the axiom of dependent choice. Remember that  is:

where  is a formula over  with free variables among  and which is of the shape . As in the previous section, we write  instead of . The type of ,  is:

In order to define  we make an informal reasoning. Suppose  is a witness of:

and  is a witness of:

We want to build from this, using , a witness of . We will use the following instance of :

The idea now is that  will build a sequence of witnesses of . The first argument represents the recursive step. If we have an element  which represents the sequence of witnesses already computed, then  computes the next element of , given its length and last element. Here we have two cases, the first one is when , so we must initialize the sequence with an element of type  that we can choose arbitrarily. Since  is a sort of the logic, we can write it as  so:

and we define this arbitrary element to be . In the following, we will simply write this term as , leaving the type implicit. In the second case, the last element of  is . Therefore, we provide  with  and . The  is because the last element of  is a witness of  where  is the length of , so since  is of shape ,  is a witness of . The first argument is then:

The second argument represents the behavior if we have an infinite sequence of witnesses . In that case we simply provide  with this argument , so the third argument is just . Finally, the last argument of  is the initial sequence of witnesses, that is the empty sequence . We have then:

The interpretation of  is therefore defined as:

\subsection{The realizability relation}
\subsubsection{Negative translation and orthogonality}
\label{NegTrans}
The first realizability models for classical logic were obtained by combining G\"odel's negative translation with intuitionistic realizability~\cite{BerardiBezemCoquand,BergerOlivaChoice,KohlenbachProofTheory}. G\"odel's negative translation from Peano arithmetic  (equivalent to ) to  (see section~\ref{usualtheories}) maps a formula  to  by prefixing inductively all the positive connectives and atomic predicates of  with a double negation. It holds that if , then , and the proof of this relies on the fact that for every axiom  of , . Therefore, a realizability model for  can be obtained from a realizability model for  using G\"odel's negative translation. Concerning the extraction of witnesses, if , then  from which we easily get . While in usual intuitionistic realizability the formula  has no realizer so the model is sound, Friedman's trick is to allow  to have realizers. If we then take the realizers of  to be the same as those of , then combining the proof of  with the identity gives a realizer of , and therefore the witness.\par
In Krivine's~\cite{KrivinePanoramas} classical realizability models this double step (negative translation + intuitionistic realizability) is avoided through the use of orthogonality in system F. In these models there is a set of terms  and a set of stacks . Each formula has a set of realizers (the truth value, subset of ) and a set of counter-realizers (the falsity value, subset of ). The falsity values are primitive, an orthogonality relation is defined between  and , and the truth values are defined as the orthogonals of the falsity values, so they are orthogonally closed.\par
Here we work in a typed setting so we must choose the types of realizers and counter-realizers so they can interact. Given a formula , the set of realizers of  would normally be a set of morphisms in a category of continuations  from the terminal object  to the interpretation of : , and under the duality between terms and contexts of -calculus, a natural choice for the counter-realizers of  is a set of morphisms in  from  to . Then we can combine a potential realizer of  with a potential counter-realizer using the evaluation morphism  so we obtain a morphism from  to . The potential realizer and counter-realizer are orthogonal to each other or not, depending on the result. Therefore, in order to define a non-trivial orthogonality relation, there must be at least two morphisms from  to , and if we want to perform extraction this homset has to be isomorphic to the set of values we want to extract. In realizability for arithmetic, this is usually done by choosing an object  which is the same as the interpretation of natural numbers, however this choice has some drawbacks. Indeed, some computational models can be naturally seen as categories of continuations for a given , and this  may not be isomorphic to the object of natural numbers. Take for example the model of Hyland-Ong games~\cite{HO}. We know that by relaxing the well-bracketing condition we obtain a fully abstract model of PCF~\cite{LairdControl}. It is therefore not a surprise that the same game model is a category of continuations , and it turns out that in this category of continuations, the object  is the one-move arena. The key point for this is that the arena of natural numbers is the exponential of the one-move arena by the countable product of one-move arenas, details can be found in~\cite{BlotThesis}, section 4.4. But since there is only one strategy on the one-move arena (the empty strategy), we cannot easily define a non-trivial orthogonality relation.\par
Therefore we choose here a different approach and rely on Friedman's trick directly in the definition of the realizability relation: our orthogonality relation relies on an artificially added output channel. Formally we add a -variable in the process of interpreting logic in -calculus. A proof:

is now translated to a -term:

where  is some fixed base type, by applying the (admissible) rule of right weakening of -calculus. The -variable  is, intuitively, a continuation variable which can be used by a realizer or a counter-realizer to stop computation and give an answer. Apart from its use in the definition of the realizability relation, this feature will also be used in the proof of the extraction result, for which  will be instantiated with the type of natural numbers. After translating the proof  to a -term , we interpret it in  as a morphism:

In the particular case of empty contexts,  is a morphism in , and we therefore choose the potential realizers of a closed formula  to be such morphisms. Similarly we choose the potential counter-realizers of  to be morphisms in . As explained in section~\ref{CatInterp} and by taking the convention that the potential realizers have a free -variable  of type , we use the syntax of -calculus (and possibly  and ) to manipulate these. We also substitute morphisms of  for -variables, as in section~\ref{interaction}, so if  is a potential realizer, and if  is a potential counter-realizer, then , since  and , and therefore . Since the object  can (and will) be chosen larger than , we can now define when  is orthogonal to  depending on . Typically, in the model of unbracketed games,  is the one-move arena and  is the countable product of the one-move arena, so  is indeed the usual arena of natural numbers.\par
The choice of having a separate -variable instead of choosing a big enough object  can also provide a simpler interpretation of proofs in . We give a comparison of our interpretation of the identity proof of  in arithmetic with the usual interpretation in figure~\ref{Comparison}, where we take  to be the unbracketed games model.
\begin{figure}
10pt]
\hline
\text{arena:}
&
\vcenter{\xymatrix@!0@R=30pt@C=15pt
{
&&&&&&q\\
&&q'\ar@{-}[urrrr]&&0\ar@{-}[urr]&1\ar@{-}[ur]&\ldots&n\ar@{-}[ul]&\ldots\\
0'\ar@{-}[urr]&1'\ar@{-}[ur]&\ldots&n'\ar@{-}[ul]&\ldots
}}
&
\vcenter{\xymatrix@!0@R=30pt@C=15pt
{
&&&&q\\
q'\ar@{-}[urrrr]&&0_1\ar@{-}[urr]&1_1\ar@{-}[ur]&\ldots&n_1\ar@{-}[ul]&\ldots\\
}}\
\caption{Comparison of our interpretation with the usual one}
\label{Comparison}
\end{figure}
We can see that with the usual interpretation where  is the object of natural numbers, this proof is interpreted as a strategy from natural numbers to natural numbers. However, in our interpretation the same proof is interpreted as a strategy on the arena  where  is the one-move arena and  is the arena of natural numbers (indeed, the  operation on arenas is the merge of roots). This simplification doesn't happen in every model however. For example, in Scott domains, the natural choice of taking the singleton set for  leads to a degenerated model, since  is then isomorphic to  for any domain . In order to get a non-degenerated model, we would need to choose for  a bigger domain, and therefore we would lose the benefit of our simpler interpretation.\par
In~\cite{BlotRibaBarRec} the goals of orthogonally-defined realizability and extraction were achieved similarly by taking realizers of a formula  to be interpretations of closed terms of type , but also interpreting the atomic formulas and  with the base type  instead of the empty type , which led to unnecessary complex types for the realizers.\par
To summarize, in order to have the benefit of the simpler interpretation, our categorical model  needs to have an object  which is significantly simpler than , and  should be large enough to interpret natural numbers. This rules out Scott domains, but unbracketed games models do have this property. It would be interesting to find other examples of such models, the main candidates being Laird's bistable biorders~\cite{LairdBistable} and Berry and Curien's sequential algorithms~\cite{BerryCurienSequential}, which are both fully abstract models of PCF.
\subsubsection{Truth values, falsity values}
\label{OrthoReal}
We fix a first-order signature  and a -structure . We also fix a corresponding  signature and a type  in this signature for each positive predicate  of , as in section~\ref{LmInterp}. We interpret -calculus in a category of continuations  as in section~\ref{CatInterp} and we use the syntax of -calculus to describe strategies of , so we omit the interpretation brackets. All the -terms that we write from now on are to be understood as morphisms in .\par
In order to build our realizability relation by orthogonality, and later on to perform extraction on  formulas using Friedman's trick, our model is parameterized with a set

which is, intuitively, the set of ``correct'' values that can be output through the variable .\par
We now define the set of realizers of a formula, that we call its truth value. We fix for each positive predicate  of  and each  a set of morphisms . The extension of this to every closed formula  on  with parameters in  is given in Figure~\ref{TruthValues},
\begin{figure}


\caption{Truth values}
\label{TruthValues}
\end{figure}
so . Remark that contrary to~\cite{KrivinePanoramas}, we do not define the truth values as the orthogonals of the falsity values. In our logical system, only some predicates are negative, and therefore only some formulas are negative. In the realizability interpretation, only the negative formulas are given a falsity value and their truth values will be proved to be orthogonal to their falsity values. For the other formulas the truth value is primitive and may not be bi-orthogonally closed.\par
For every closed negative formula  with parameters in  the falsity value of  is given in Figure~\ref{FalsityValues},
\begin{figure}


\caption{Falsity values}
\label{FalsityValues}
\end{figure}
so . As explained in section~\ref{cps}, we use here the syntax of  to manipulate morphisms in . We define now an orthogonality relation between  and : if  and , then , so we define:

The following lemma states that the truth values are indeed the orthogonals of the falsity values defined above for negative formulas:
\begin{lem}
\label{orthogonal}
For every closed negative formula  with parameters in :

\end{lem}
\proof
We prove this result by induction on the structure of the formula:
\begin{itemize}
\item: if  then:

from which we conclude since .
\item: if  then the result is immediate, and otherwise the proof is the same as for .
\item: if , , , then , therefore:

and finally:

\item: if ,  and , then , therefore:

and finally:

\item: if , then:

\end{itemize}
\subsection{Adequacy}
\subsubsection{Adequacy for first-order logic}
We will now state the adequacy lemma, which states the soundness of our realizability interpretation with respect to first-order classical logic. It is interesting to remark that the only cases which depends on the orthogonality relation are those of introduction and elimination of the  formula. It is not much of a surprise, since these rules are the ones that make our proof system classical. Most of the other cases are straightforward, though some care must be taken for the  rules.\par
In order to prove the adequacy lemma, we suppose that the interpretations of the terms associated to the axioms are realizers of these axioms: for every , . The adequacy lemma is then as follows:
\begin{lem}
\label{adequacyLemma}
Suppose  is a proof of  with , so:

then for any , we have:

In particular if  is a closed formula, 
\end{lem}
\proof
By induction on the proof tree:
\begin{itemize}
\item If  is the identity rule then  is:

Let , ,  and . Then we have:

\item If  is the axiom rule then  is:

Let ,  and . Then we have:

since  is closed and  by assumption.
\item If  ends with an introduction of  then  is:

Let ,  and . Then for any  we have:

by induction hypothesis, since . Therefore:

\item If  ends with an elimination of  then  is:

Let ,  and . Then we have:

since by induction hypothesis:

\item If  ends with an introduction of  then  is:

Let ,  and . Then we have:

since by induction hypothesis:

\item If  ends with an elimination of  then  is:

Let ,  and . Then we have:

since by induction hypothesis:

\item If  ends with an introduction of  then  is:

Let ,  and . Then for any ,  and  (since ), so by induction hypothesis:

Therefore .
\item If  ends with an elimination of  then  is:

Let ,  and . By induction hypothesis:

so taking  we get:

and since  and  we get:

\item If  ends with an introduction of  then  is:

Let , ,  and . Then we have:

since by induction hypothesis:

so  by lemma~\ref{orthogonal}, and .
\item If  ends with an elimination of  then  is:

Let ,  and . Then for any  we have:

since by induction hypothesis, . Therefore:
 and so by lemma~\ref{orthogonal}:


\end{itemize}
\subsubsection{Adequacy for Peano arithmetic}
\label{OrthoRealPeano}
We fix now the theory to be  on  (the inequality predicate being negative, and the relativization predicate being positive), and the structure  to be a model of , the inequality predicate being interpreted in  as in section~\ref{models}. For simplicity we write  for  where . The  signature is that of PCF (see section~\ref{muPCF}) and we interpret the two predicates as in sections~\ref{RelPred} and~\ref{LmInterpPA}:

and the axioms as in section~\ref{LmInterpPA}. We suppose that the category of continuations  is a model of PCF (see definition~\ref{LambdaMuModel}). Since we want to extract algorithms on natural numbers, we fix .\par
The realizability value for the relativization predicate is:

First, all equalities which are true in the model are trivially realized:
\begin{lem}
Let  and  be first-order terms with .

\end{lem}
\proof
Let . Since , we have , so:

Therefore, for any :

and so:

Therefore, since  is a model of  we have immediately the following results:

The non-confusion axiom and Leibniz scheme are easy:
\begin{lem}

\end{lem}
\proof
Since  if , it is sufficient to prove that for any , , which is true since  so . Let now , ,  and . If , then  and therefore . Otherwise,  so  and by definition of  we get . Moreover, since ,  and . Finally we get .
\qed
The adequacy for the induction axiom scheme is as follows:
\begin{lem}

\end{lem}
\proof
Let , , . Since  if , we can use induction to prove that for any , :
\begin{itemize}
\item: 
\item: . Since , we get:

and since by induction hypothesis we have  we get:

\end{itemize}
The last axioms are the relativization ones:
\begin{lem}

\end{lem}
\proof
The first one is immediate, since . For the second one, remember that:

Since  if , it is sufficient to prove that for any , . Since , it follows from .
\qed
\subsubsection{Adequacy for the axiom of choice}
\label{ChoiceAdequacy}
When it comes to the axiom of countable choice , the usual route of negative translation followed by intuitionistic realizability becomes much more difficult. Indeed,  is not provable in intuitionistic logic, therefore the path described in section~\ref{NegTrans} cannot be followed as-is and an intuitionistic realizer of  must be provided. In~\cite{BerardiBezemCoquand}, a variant of bar recursion was used to realize , while in~\cite{BergerOlivaChoice} the principle of double negation shift  was realized using bar recursion (see~\ref{barrecursion}). With this principle, it becomes possible to derive  in intuitionistic logic, and since  is realized in intuitionistic models by the identity, one obtains a realizer of .\par
We follow here a different approach, since our realizability model is for classical logic, and we prove that bar recursion realizes the axiom of dependent choice  in our classical model. The negative translation of proofs corresponds to the continuation-passing-style translation on terms and the semantics of -calculus in a category of continuations  corresponds to the semantics of its CPS-translation into  in the cartesian ``-closed'' category  as stated in section~\ref{cps}, so in order to compare more closely our model with the usual indirect realizability interpretation, one would need to define a realizability relation for intuitionistic logic directly in . Our choice of working in the category of continuations  is mainly motivated by the existence of computational models with a natural structure of category of continuations (as the unbracketed Hyland-Ong games, see section~\ref{NegTrans}).\par
We suppose now that  satisfies , and we use the variant of the bar recursion operator defined in section~\ref{barrecursion} to realize the axiom of dependent choice . First, we need to make some assumptions on our category of continuations. The first one is a continuity requirement:
\begin{defi}[Continuity]
\label{continuity}
If , , then there exists  such that for any :

\end{defi}\medskip

\noindent This requirement is satisfied in particular by games models since in that case  is a cpo-enriched category in which the base types  and  are interpreted as flat domains. Note however that the full set-theoretic model doesn't satisfy this, since we can for example consider a function which gives  if the input sequence is the constant  sequence, and  otherwise.\par
The second requirement states that we can construct a function from any sequence of elements:
\begin{defi}[Sequence internalization]
\label{sequence}
If  is a sequence of morphisms in the homset , then there exists a morphism  such that for any , .
\end{defi}
In particular, all functions on natural numbers must exist in the model, even the uncomputable ones. It is in particular true in the games models, but it is of course not satisfied by the term model of PCF, and this is the main motivation for working in a model of PCF rather than directly with the syntactic language.\par
Assuming  satisfies these two assumptions, we now prove that the interpretation of  realizes :
\begin{lem}
Let .

where  denotes a formula over  with free variables among  and which is of the shape . For clarity, we write  instead of .
\end{lem}
\proof
Recall that we use the bar recursor:

To simplify notations we define:

so  is:

Let now  and write . We use again the notation . Let:

be such that:

We then have to prove:

For conciseness we write:

so we must prove that . With our notations, for any  we have:

The following iteration lemma (the proof of which is deferred to the end of the section) is the heart of the adequacy:
\begin{lem}
Let  and  be such that:

where  is the term from section~\ref{PeanoTheory}. There exists  and  such that:
\smallskip
\end{lem}

\noindent In order to prove , we use a reductio-ad-absurdum and suppose . By iterating the lemma, we build sequences  in  and  in  such that:

Since  satisfies , we can build  such that for every , , and using the second assumption on  (definition~\ref{sequence}) we also build  such that for every , . We now prove that:

Since  if , it is sufficient to prove that for , , but since ,  and , this is immediate. Now, since:

we get , so . We use now the first assumption on  (definition~\ref{continuity}), so there is some  such that any morphism  which satisfies:

is such that . If we write , this is verified in particular for:

so we get:

and finally:

from which we get our contradiction.
\qed
Here is the proof of the iteration lemma:
\proof
Write . We have:

Using ,  and  we can build some  such that for any , , and for any , . Since , we have in particular:

therefore:

Since  if , there must be some  such that:

If  then we get:

which contradicts the hypothesis of the lemma. Therefore,  and:

First, since  we have , and therefore:

We prove now that  by distinguishing cases:
\begin{itemize}
\item:  (indeed,  by an easy induction on )
\item:  since:

\end{itemize}
Therefore we have:

and so:

which means that there exists some  such that:

and so there exists  such that  and:


\subsection{Extraction}
Since the Friedman translation is directly built in the realizability interpretation, the extraction result is an easy consequence of the definitions. We show that from any -formula provable in  we can extract a computable witness in . Note that in the extraction lemma, the equality  is at any type:
\begin{lem}
From a proof of  in , one can extract a -term  such that for any  and , there is some  such that:

\end{lem}
\proof
Since  and  are just encodings, we actually have a proof of . First, we can easily turn it into a proof of , and we get by relativization a proof  in  of . The adequacy lemma then tells us:

If  and , we get . Let fix now:

We prove that . For that let , and let prove that . There are two cases:
\begin{itemize}
\item: in that case, , so:

and  so 
\item: in that case, , so:

and therefore 
\end{itemize}
We get , so . This means that  is some  (so ). Finally, we have the claimed result with .
\qed
Similarly to what was done in~\cite{BlotRibaBarRec}, we can define an operatonal semantics for PCF and adapt the techniques of e.g.~\cite{AmadioCurien} to prove computational adequacy with respect to a non-degenerated , i.e. if  in PCF, then for any :

Using computational adequacy and the extraction lemma in the particular case of , we get that for any , there is some  such that  and .
\section*{Conclusion}
Realizability interpretation of classical logic with control operators is a recent field in which much of the contributions take place in Krivine's untyped setting. We defined here a realizability model in which proofs are interpreted in a category of continuations, which is the universal model of typed -calculus. The duality between terms and contexts, which appears in Krivine's work as a duality between terms and stacks, is reflected here by the duality in a category of continuations  between  and .\par
The direct interpretations of classical logic can be seen as a way to avoid G\"odel's negative translation on formulas, by using programming languages with an operational semantics which corresponds to that of their CPS-translation to -calculus. Similarly, the choice of having a free -variable  in our realizers may be seen as a way to avoid Friedman's -translation. Indeed, in Friedman's original work, the translation is obtained by replacing each basic predicate  with the disjunction . In classical sequent calculus, the right-hand context is meant to be interpreted as a disjunction, so adding a fixed -variable to this context corresponds to applying Friedman's translation on programs instead of proofs. In many implementations of Friedman's trick, the formula  is not added as a disjunction to every predicate, but instead the  formula is replaced everywhere with . Our implementation of Friedman's translation however allows simpler interpretations of the realizers in some models. Indeed, the continuation passing hidden behind the usual implementations is abstracted with the help of -calculus constructs. Rather than adding a continuation on top of every interpretation of , we only add one continuation on top of the whole realizer. This simplicity of the interpretation requires some properties on the model, and in particular it rules out Scott domains. However, all these properties are satisfied in unbracketed games, and it would be interesting to find other models satisfying these. The natural candidates for this are bistable biorders~\cite{LairdBistable} and sequential algorithms~\cite{BerryCurienSequential}, but these are not the only ones. Coherent spaces may also satisfy these properties if we choose carefully the object of natural numbers.\par
The -variable  is also exploited in the definition of the realizability interpretation, to parameterize the orthogonality relation between realizers and counter-realizers. However, contrary to usual interpretations of classical logic, we consider positive predicates, for which no counter-realizers are defined. Through a suitable restriction on proofs we preserve the adequacy of the interpretation. Roughly, it amounts to forbid classical reasoning on the positive formulas, while keeping it in the general case. This introduction of positive predicates in a classical, negative setting is mainly motivated by the decomposition of the relativized universal quantifier into a uniform quantifier and a relativization predicate, the relativization predicate being a positive one.\par
We proved that the usual terms of G\"odel's system T realize the axioms of Peano arithmetic in our direct setting, without CPS-translation. This is not a surprise since all these axioms do imply their negative translation intuitionistically. However, when it comes to the axiom of choice, this is not as easy and we prove, as was done in~\cite{BlotRibaBarRec}, that the bar recursion operator realizes it in a direct interpretation. Interestingly, this bar recursor also gives computational content to the double-negation shift principle in an intuitionistic setting, and that same principle allows to derive intuitionistically the negative translation of the axiom of choice from the axiom of choice.\par
Finally, we validate our model by proving an extraction result for  formulas, relying once again on the -variable  which allows the orthogonality relation to be a parameter of the model.
\subsection*{Acknowledgments}I thank the reviewers for their careful reading of my work, which led to constructive comments and suggestions.
\begin{thebibliography}{00000}

\bibitem[AC98]{AmadioCurien}
Roberto Amadio and Pierre-Louis Curien.
\newblock {\em {Domains and Lambda-Calculi}}, volume~46 of {\em Cambridge
  Tracts in Theoretical Computer Science}.
\newblock Cambridge University Press, 1998.

\bibitem[AHS07]{ArilaHerbelinSabryContinuations}
Zena Ariola, Hugo Herbelin, and Amr Sabry.
\newblock {A proof-theoretic foundation of abortive continuations}.
\newblock {\em Higher-Order and Symbolic Computation}, 20(4):403--429, 2007.

\bibitem[BBC98]{BerardiBezemCoquand}
Stefano Berardi, Marc Bezem, and Thierry Coquand.
\newblock {On the Computational Content of the Axiom of Choice}.
\newblock {\em Journal of Symbolic Logic}, 63(2):600--622, 1998.

\bibitem[BC82]{BerryCurienSequential}
G{\'e}rard Berry and Pierre-Louis Curien.
\newblock {Sequential Algorithms on Concrete Data Structures}.
\newblock {\em Theoretical Computer Science}, 20(3):265--321, 1982.

\bibitem[Blo14]{BlotThesis}
Valentin Blot.
\newblock {\em {Game semantics and realizability for classical logic}}.
\newblock PhD thesis, \'Ecole Normale Sup\'erieure de Lyon, 2014.

\bibitem[BO05]{BergerOlivaChoice}
Ulrich Berger and Paulo Oliva.
\newblock {Modified bar recursion and classical dependent choice}.
\newblock In {\em Logic Colloquium '01, Proceedings of the Annual European
  Summer Meeting of the Association for Symbolic Logic}, volume~20 of {\em
  Lecture Notes in Logic}, pages 89--107. A K Peters, Ltd., 2005.

\bibitem[BR13]{BlotRibaBarRec}
Valentin Blot and Colin Riba.
\newblock {On Bar Recursion and Choice in a Classical Setting}.
\newblock In {\em 11th Asian Symposium on Programming Languages and Systems},
  volume 8301 of {\em Lecture Notes in Computer Science}, pages 349--364.
  Springer, 2013.

\bibitem[CH00]{CurienHerbelinDuality}
Pierre-Louis Curien and Hugo Herbelin.
\newblock {The duality of computation}.
\newblock In {\em 5th International Conference on Functional Programming},
  pages 233--243. {ACM} Press, 2000.

\bibitem[dG94]{DeGrooteLambdaMu}
Philippe de~Groote.
\newblock {On the Relation between the Lambda-Mu-Calculus and the Syntactic
  Theory of Sequential Control}.
\newblock In {\em 5th International Conference on Logic Programming and
  Automated Reasoning}, volume 822 of {\em Lecture Notes in Computer Science},
  pages 31--43. Springer, 1994.

\bibitem[DP01]{DavidPyBohm}
Ren{\'e} David and Walter Py.
\newblock {\mbox{-}calculus and B{\"o}hm's theorem}.
\newblock {\em Journal of Symbolic Logic}, 66(1):407--413, 2001.

\bibitem[EO12]{EscardoOlivaPeirce}
Mart\'\i n~H\"otzel Escard\'o and Paulo Oliva.
\newblock {The Peirce translation}.
\newblock {\em Annals of Pure and Applied Logic}, 163(6):681--692, 2012.

\bibitem[Gir91]{GirardLC}
Jean{-}Yves Girard.
\newblock {A New Constructive Logic: Classical Logic}.
\newblock {\em Mathematical Structures in Computer Science}, 1(3):255--296,
  1991.

\bibitem[Gir93]{GirardLU}
Jean{-}Yves Girard.
\newblock {On the Unity of Logic}.
\newblock {\em Annals of Pure and Applied Logic}, 59(3):201--217, 1993.

\bibitem[G{\"o}d33]{GodelNegative}
Kurt G{\"o}del.
\newblock {Zur intuitionistischen Arithmetik und Zahlentheorie}.
\newblock {\em Ergebnisse eines mathematischen Kolloquiums}, 4:34--38, 1933.

\bibitem[G{\"o}d58]{GodelDialectica}
Kurt G{\"o}del.
\newblock {{\"U}ber eine bisher noch nicht ben{\"u}tzte Erweiterung des finiten
  Standpunktes}.
\newblock {\em Dialectica}, 12(3-4):280--287, 1958.

\bibitem[Gri90]{GriffinControl}
Timothy Griffin.
\newblock {A Formulae-as-Types Notion of Control}.
\newblock In {\em 17th Symposium on Principles of Programming Languages}, pages
  47--58. {ACM} Press, 1990.

\bibitem[Her95]{HerbelinPhD}
Hugo Herbelin.
\newblock {\em {S\'equents qu'on calcule: de l'interpr{\'e}tation du calcul des
  s{\'e}quents comme calcul de \mbox{-}termes et comme calcul de
  strat{\'e}gies gagnantes}}.
\newblock PhD thesis, Universit\'e Paris 7, 1995.

\bibitem[HO00]{HO}
Martin Hyland and Luke Ong.
\newblock {On Full Abstraction for PCF: I, II, and III}.
\newblock {\em Information and Computation}, 163(2):285--408, 2000.

\bibitem[HS02]{StreicherContinuation}
Martin Hofmann and Thomas Streicher.
\newblock {Completeness of Continuation Models for
  \mbox{-}Calculus}.
\newblock {\em Information and Computation}, 179(2):332--355, 2002.

\bibitem[HS09]{HerbelinSaurinLambdaMu}
Hugo Herbelin and Alexis Saurin.
\newblock {\mbox{-}calculus and \mbox{-}calculus: a
  Capital Difference}.
\newblock http://hal.inria.fr/inria-00524942, 2009.

\bibitem[Kle45]{Kleene}
Stephen~Cole Kleene.
\newblock {On the Interpretation of Intuitionistic Number Theory}.
\newblock {\em Journal of Symbolic Logic}, 10(4):109--124, 1945.

\bibitem[Koh90]{KohlenbachThesis}
Ulrich Kohlenbach.
\newblock {\em {Theory of majorizable and continuous functionals and their use
  for the extraction of bounds from non-constructive proofs: effective moduli
  of uniqueness for best approximations from ineffective proofs of
  uniqueness}}.
\newblock PhD thesis, Goethe Universit{\"a}t Frankfurt, 1990.

\bibitem[Koh08]{KohlenbachProofTheory}
Ulrich Kohlenbach.
\newblock {\em {Applied Proof Theory: Proof Interpretations and their Use in
  Mathematics}}.
\newblock Springer Monographs in Mathematics. Springer, 2008.

\bibitem[Kre59]{Kreisel}
Georg Kreisel.
\newblock {Interpretation of analysis by means of constructive functionals of
  finite types}.
\newblock In {\em Constructivity in mathematics: Proceedings of the colloquium
  held at Amsterdam, 1957}, Studies in Logic and the Foundations of
  Mathematics, pages 101--128. North-Holland Publishing Company, 1959.

\bibitem[Kri94]{KrivineStorage}
Jean-Louis Krivine.
\newblock {A General Storage Theorem for Integers in Call-by-Name
  lambda-Calculus}.
\newblock {\em Theoretical Computer Science}, 129(1):79--94, 1994.

\bibitem[Kri01]{KrivineZF}
Jean-Louis Krivine.
\newblock {Typed lambda-calculus in classical Zermelo-Fr\ae nkel set theory}.
\newblock {\em Archive for Mathematical Logic}, 40(3):189--205, 2001.

\bibitem[Kri03]{KrivineDependent}
Jean-Louis Krivine.
\newblock {Dependent choice, `quote' and the clock}.
\newblock {\em Theoretical Computer Science}, 308(1--3):259--276, 2003.

\bibitem[Kri09]{KrivinePanoramas}
Jean-Louis Krivine.
\newblock {Realizability in classical logic}.
\newblock {\em Panoramas et synth\`eses}, 27:197--229, 2009.

\bibitem[Lai97]{LairdControl}
James Laird.
\newblock {Full Abstraction for Functional Languages with Control}.
\newblock In {\em 12th Annual {IEEE} Symposium on Logic in Computer Science},
  pages 58--67. {IEEE} Computer Society, 1997.

\bibitem[Lai99]{LairdThesis}
James Laird.
\newblock {\em {A semantic analysis of control}}.
\newblock PhD thesis, University of Edinburgh, 1999.

\bibitem[Lai07]{LairdBistable}
James Laird.
\newblock {Bistable Biorders: A Sequential Domain Theory}.
\newblock {\em Logical Methods in Computer Science}, 3(2), 2007.

\bibitem[LRS93]{LafontReusStreicher}
Yves Lafont, Bernhard Reus, and Thomas Streicher.
\newblock {Continuations Semantics or Expressing Implication by Negation}.
\newblock Technical Report 93-21, Ludwig-Maximilians-Universit{\"a}t,
  M{\"u}nchen, 1993.

\bibitem[Miq11]{MiquelWitness}
Alexandre Miquel.
\newblock {Existential witness extraction in classical realizability and via a
  negative translation}.
\newblock {\em Logical Methods in Computer Science}, 7(2), 2011.

\bibitem[Ong96]{OngClassicalProofs}
Luke Ong.
\newblock {A Semantic View of Classical Proofs: Type-Theoretic, Categorical,
  and Denotational Characterizations (Preliminary Extended Abstract)}.
\newblock In {\em 11th Annual {IEEE} Symposium on Logic in Computer Science},
  pages 230--241. {IEEE} Computer Society, 1996.

\bibitem[OS97]{OngStewartControl}
Luke Ong and Charles Stewart.
\newblock {A Curry-Howard Foundation for Functional Computation with Control}.
\newblock In {\em 24th Symposium on Principles of Programming Languages}, pages
  215--227. {ACM} Press, 1997.

\bibitem[OS08]{OlivaStreicher}
Paulo Oliva and Thomas Streicher.
\newblock {On Krivine's Realizability Interpretation of Classical Second-Order
  Arithmetic}.
\newblock {\em Fundamenta Informaticae}, 84(2):207--220, 2008.

\bibitem[Par92]{ParigotLambdaMu}
Michel Parigot.
\newblock {\mbox{-}Calculus: An Algorithmic Interpretation of
  Classical Natural Deduction}.
\newblock In {\em 3rd International Conference on Logic Programming and
  Automated Reasoning}, volume 624 of {\em Lecture Notes in Computer Science},
  pages 190--201. Springer, 1992.

\bibitem[Plo77]{PlotkinPCF}
Gordon Plotkin.
\newblock {LCF Considered as a Programming Language}.
\newblock {\em Theoretical Computer Science}, 5(3):223--255, 1977.

\bibitem[Pow13]{PowellThesis}
Thomas Powell.
\newblock {\em {On bar recursive interpretations of analysis}}.
\newblock PhD thesis, Queen Mary University of London, 2013.

\bibitem[Pow14]{PowellEquivalence}
Thomas Powell.
\newblock {The equivalence of bar recursion and open recursion}.
\newblock {\em Annals of Pure and Applied Logic}, 165(11):1727--1754, 2014.

\bibitem[Sau05]{SaurinLambdaMu}
Alexis Saurin.
\newblock {Separation with Streams in the \mbox{-}calculus}.
\newblock In {\em 20th {IEEE} Symposium on Logic in Computer Science}, pages
  356--365. {IEEE} Computer Society, 2005.

\bibitem[Sco93]{ScottLCF}
Dana Scott.
\newblock {A Type-Theoretical Alternative to ISWIM, CUCH, OWHY}.
\newblock {\em Theoretical Computer Science}, 121(1-2):411--440, 1993.

\bibitem[Sel01]{SelingerControl}
Peter Selinger.
\newblock {Control categories and duality: on the categorical semantics of the
   calculus}.
\newblock {\em Mathematical Structures in Computer Science}, 11(2):207--260,
  2001.

\bibitem[Spe62]{Spector}
Clifford Spector.
\newblock {Provably recursive functionals of analysis: a consistency proof of
  analysis by an extension of principles in current intuitionistic
  mathematics}.
\newblock In {\em Recursive Function Theory: Proceedings of Symposia in Pure
  Mathematics}, volume~5, pages 1--27. American Mathematical Society, 1962.

\bibitem[TVD88]{TroelstraVanDalenConstructivism}
Anne~Sjerp Troelstra and Dirk Van~Dalen.
\newblock {\em {Constructivism in mathematics: an introduction}}, volume 121,
  123 of {\em Studies in logic and the foundations of mathematics}.
\newblock Elsevier, 1988.

\end{thebibliography}
\end{document}