


\documentclass[11pt]{article} 
\usepackage{algorithmic}
\usepackage{algorithm}
\usepackage{times} 
\usepackage{amsfonts}
\usepackage{amsmath} 
\usepackage{latexsym} 
\usepackage{epsfig}
\usepackage{latexsym} 
\usepackage{verbatim}
\begin{document}
 
\newtheorem{theorem}{Theorem}
\newtheorem{corollary}[theorem]{Corollary}
\newtheorem{prop}[theorem]{Proposition} 
\newtheorem{problem}[theorem]{Problem}
\newtheorem{lemma}[theorem]{Lemma} 
\newtheorem{remark}[theorem]{Remark}
\newtheorem{observation}[theorem]{Observation}
\newtheorem{defin}{Definition} 
\newtheorem{example}{Example}
\newtheorem{conj}{Conjecture} 
\newenvironment{proof}{{\bf Proof:}}{\hfill} 
\newcommand{\PR}{\noindent {\bf Proof:\ }} \def\EPR{\hfill \linebreak\vskip.5mm} 

\def\Pol{{\sf Pol}} 
\def\mPol{{\sf MPol}} 
\def\Polo{{\sf Pol}_1} 
\def\PPol{{\sf pPol\;}} 
\def\Inv{{\sf Inv}}
\def\mInv{{\sf MInv}} 
\def\Clo{{\sf Clo}\;} 
\def\Con{{\sf Con}} 
\def\concom{{\sf Concom}\;} 
\def\End{{\sf End}\;}
\def\Sub{{\sf Sub}} 
\def\Im{{\sf Im}} 
\def\Ker{{\sf Ker}\;} 
\def\H{{\sf H}}
\def\S{{\sf S}} 
\def\D{{\sf P}} 
\def\I{{\sf I}} 
\def\Var{{\sf var}} 
\def\PVar{{\sf pvar}} 
\def\fin#1{{#1}_{\rm fin}}
\def\P{{\sf P}} 
\def\Pfin{{\sf P_{\rm fin}}} 
\def\Id{{\sf Id}}
\def\R{{\rm R}} 
\def\F{{\rm F}} 
\def\Term{{\sf Term}}
\def\var#1{{\sf var}(#1)} 
\def\Sg#1{{\sf Sg}(#1)} 
\def\Sgo#1{{\sf Sg}_{\mathrm{old}}(#1)} 
\def\Sgn#1{{\sf Sg}_{\mathrm{new}}(#1)} 
\def\Sgg#1#2{{\sf Sg}_{#1}(#2)} 
\def\Cg#1{{\sf Cg}(#1)}
\def\Cgg#1#2{{\sf Cg}_{#1}(#2)} 
\def\tol{{\sf tol}} 
\def\rbcomp#1{{\sf rbcomp}(#1)}
  
\let\cd=\cdot 
\let\eq=\equiv 
\let\op=\oplus 
\let\omn=\ominus
\let\meet=\wedge 
\let\join=\vee 
\let\tm=\times
\def\ldiv{\mathbin{\backslash}} 
\def\rdiv{\mathbin/}
  
\def\typ{{\sf typ}} 
\def\zz{{\un 0}} 
\def\zo{{\un 1}}
\def\one{{\bf1}} 
\def\two{{\bf2}} 
\def\three{{\bf3}}
\def\four{{\bf4}} 
\def\five{{\bf5}}
\def\pq#1{(\zz_{#1},\mu_{#1})}
  
\let\wh=\widehat 
\def\ox{\ov x} 
\def\oy{\ov y} 
\def\oz{\ov z}
\def\of{\ov f} 
\def\oa{\ov a} 
\def\ob{\ov b} 
\def\oc{\ov c}
\def\od{\ov d} 
\def\oob{\ov{\ov b}} 
\def\rx{{\rm x}}
\def\rf{{\rm f}} 
\def\rrm{{\rm m}} 
\let\un=\underline
\let\ov=\overline 
\let\cc=\circ 
\let\rb=\diamond 
\def\ta{{\tilde a}} 
\def\tz{{\tilde z}}
  


\def\zZ{{\mathbb Z}} 
\def\B{{\mathcal B}} 
\def\P{{\mathcal P}}
\def\zL{{\mathbb L}} 
\def\zD{{\mathbb D}}
 \def\zE{{\mathbb E}}
\def\zG{{\mathbb G}} 
\def\zA{{\mathbb A}} 
\def\zB{{\mathbb B}}
\def\zC{{\mathbb C}} 
\def\zM{{\mathbb M}} 
\def\zR{{\mathbb R}}
\def\zS{{\mathbb S}} 
\def\zT{{\mathbb T}} 
\def\zN{{\mathbb N}}
\def\zQ{{\mathbb Q}} 
\def\zW{{\mathbb W}} 
\def\bK{{\bf K}}
\def\C{{\bf C}} 
\def\M{{\bf M}} 
\def\E{{\bf E}} 
\def\N{{\bf N}}
\def\O{{\bf O}} 
\def\bN{{\bf N}} 
\def\bX{{\bf X}} 
\def\GF{{\rm GF}} 
\def\cC{{\mathcal C}} 
\def\cA{{\mathcal A}}
\def\cB{{\mathcal B}} 
\def\cD{{\mathcal D}} 
\def\cE{{\mathcal E}} 
\def\cF{{\mathcal F}} 
\def\cG{{\mathcal G}} 
\def\cH{{\mathcal H}}
\def\cI{{\mathcal I}} 
\def\cL{{\mathcal L}} 
\def\cP{{\mathcal
P}} \def\cR{{\mathcal R}} 
\def\cRY{{\mathcal RY}}
\def\cS{{\mathcal S}} 
\def\cT{{\mathcal T}} 
\def\oB{{\ov B}}
\def\oC{{\ov C}} 
\def\ooB{{\ov{\ov B}}} 
\def\ozB{{\ov{\zB}}}
\def\ozD{{\ov{\zD}}} 
\def\ozG{{\ov{\zG}}}
\def\tcA{{\widetilde\cA}} 
\def\tcC{{\widetilde\cC}}
\def\tcF{{\widetilde\cF}} 
\def\tcI{{\widetilde\cI}}
\def\tB{{\widetilde B}} 
\def\tC{{\widetilde C}}
\def\tD{{\widetilde D}} 
\def\ttB{{\widetilde{\widetilde B}}}
\def\ttC{{\widetilde{\widetilde C}}}
\def\tba{{\tilde\ba}} 
\def\ttba{{\tilde{\tilde\ba}}}
\def\tbb{{\tilde\bb}} 
\def\ttbb{{\tilde{\tilde\bb}}}
\def\tbc{{\tilde\bc}} 
\def\tbd{{\tilde\bd}}
\def\tbe{{\tilde\be}} 
\def\tbt{{\tilde\bt}}
\def\tbu{{\tilde\bu}} 
\def\tbv{{\tilde\bv}}
\def\tbw{{\tilde\bw}} 
\def\tdl{{\tilde\dl}} 
\def\ocP{{\ov\cP}}
\def\tzA{{\widetilde\zA}} 
\def\tzC{{\widetilde\zC}}
\def\new{{\mbox{\footnotesize new}}}
\def\old{{\mbox{\footnotesize old}}}
\def\prev{{\mbox{\footnotesize prev}}}
\def\oo{{\mbox{\sf\footnotesize o}}}
\def\pp{{\mbox{\sf\footnotesize p}}}
\def\nn{{\mbox{\sf\footnotesize n}}} 
\def\oR{{\ov R}}
\def\bA{\mathbf{R}}
  


\def\gA{{\mathfrak A}} 
\def\gV{{\mathfrak V}} 
\def\gS{{\mathfrak S}} 
\def\gK{{\mathfrak K}} 
\def\gH{{\mathfrak H}}
  
\def\ba{{\bf a}} 
\def\bb{{\bf b}} 
\def\bc{{\bf c}} 
\def\bd{{\bf d}} 
\def\be{{\bf e}} 
\def\bbf{{\bf f}} 
\def\bg{{\bf g}}
\def\bh{{\bf h}}
\def\bi{{\bf i}} 
\def\bm{{\bf m}} 
\def\bo{{\bf o}} 
\def\bp{{\bf p}} 
\def\bs{{\bf s}} 
\def\bu{{\bf u}} 
\def\bt{{\bf t}} 
\def\bv{{\bf v}} 
\def\bx{{\bf x}}
\def\by{{\bf y}} 
\def\bw{{\bf w}} 
\def\bz{{\bf z}}
\def\ga{{\mathfrak a}} 
\def\oal{{\ov\al}} 
\def\obeta{{\ov\beta}}
\def\ogm{{\ov\gm}} 
\def\oep{{\ov\varepsilon}}
\def\oeta{{\ov\eta}} 
\def\oth{{\ov\th}} 
\def\ovm{{\ov\mu}}
\def\ozero{{\ov0}}
  


\def\CCSP{\hbox{\rm c-CSP}} 
\def\CSP{{\rm CSP}} 
\def\NCSP{{\rm \#CSP}} 
\def\mCSP{{\rm MCSP}} 
\def\FP{{\rm FP}} 
\def\PTIME{{\bf PTIME}} 
\def\GS{\hbox{()}} 
\def\ry{\hbox{\rm r+y}}
\def\rb{\hbox{\rm r+b}} 
\def\Gr#1{{\mathrm{Gr}(#1)}}
\def\Grp#1{{\mathrm{Gr'}(#1)}} 
\def\Grpr#1{{\mathrm{Gr''}(#1)}}
\def\Scc#1{{\mathrm{Scc}(#1)}} 
\def\rel{R} 
\def\relo{Q}
\def\rela{S} 
\def\dep{\mathsf{dep}}
\def\Filt{\mathrm{Ft}}
\def\Filts{\mathrm{Fts}} 
\def\Agr{}
\def\Al{\mathrm{Alg}}
\def\Sig{\mathrm{Sig}}
\def\strat{\mathsf{strat}}
\def\relmax{\mathsf{relmax}}
\def\srelmax{\mathsf{srelmax}}
\def\Meet{\mathsf{Meet}}
\def\amax{\mathsf{amax}}
\def\as{\mathsf{as}}
\def\star{\hbox{}}
\def\bmal{{\mathbf m}}
\def\Af{\mathsf{Af}}
\let\sqq=\sqsubseteq



\let\sse=\subseteq 
\def\ang#1{\langle #1 \rangle}
\def\angg#1{\left\langle #1 \right\rangle}
\def\dang#1{\ang{\ang{#1}}} 
\def\vc#1#2{#1 _1\zd #1 _{#2}}
\def\tms#1#2{#1 _1\tm\dots\tm #1 _{#2}}
\def\zd{,\ldots,} 
\let\bks=\backslash 
\def\red#1{\vrule height7pt depth3pt width.4pt
\lower3pt\hbox{}}
\def\fac#1{/\lower2pt\hbox{}}
\def\me{\stackrel{\mu}{\eq}} 
\def\nme{\stackrel{\mu}{\not\eq}}
\def\eqc#1{\stackrel{#1}{\eq}} 
\def\cl#1#2{\arraycolsep0pt
\left(\begin{array}{c} #1\\ #2 \end{array}\right)}
\def\cll#1#2#3{\arraycolsep0pt \left(\begin{array}{c} #1\\ #2\\
#3 \end{array}\right)} 
\def\clll#1#2#3#4{\arraycolsep0pt
\left(\begin{array}{c} #1\\ #2\\ #3\\ #4 \end{array}\right)}
\def\cllll#1#2#3#4#5#6{ \left(\begin{array}{c} #1\\ #2\\ #3\\
#4\\ #5\\ #6 \end{array}\right)} 
\def\pr{{\rm pr}}
\let\upr=\uparrow 
\def\ua#1{\hskip-1.7mm\uparrow^{#1}}
\def\sua#1{\hskip-0.2mm\scriptsize\uparrow^{#1}} 
\def\lcm{{\rm lcm}} 
\def\perm#1#2#3{\left(\begin{array}{ccc} 1&2&3\\ #1&#2&#3
\end{array}\right)} 
\def\w{} 
\let\ex=\exists
\def\NS{{\sc (No-G-Set)}} 
\def\lev{{\sf lev}}
\let\rle=\sqsubseteq 
\def\ryle{\le_{ry}} 
\def\ryprec{\le_{ry}}
\def\os{\mbox{[}} 
\def\zs{\mbox{]}}
\def\link{{\sf link}}
\def\solv{\stackrel{s}{\sim}} 
\def\mal{\mathbf{m}}
\def\precs{\prec_{as}}



\def\lb{}  
  
\def\ar{\hbox{ar}} 
\def\Im{{\sf Im}\;} 
\def\deg{{\sf deg}}
\def\id{{\rm id}}
  
\let\al=\alpha 
\let\gm=\gamma 
\let\dl=\delta 
\let\ve=\varepsilon
\let\ld=\lambda 
\let\om=\omega 
\let\vf=\varphi 
\let\vr=\varrho
\let\th=\theta 
\let\sg=\sigma 
\let\Gm=\Gamma 
\let\Dl=\Delta
  


\font\tengoth=eufm10 scaled 1200 
\font\sixgoth=eufm6
\def\goth{\fam12} 
\textfont12=\tengoth 
\scriptfont12=\sixgoth
\scriptscriptfont12=\sixgoth 
\font\tenbur=msbm10
\font\eightbur=msbm8 
\def\bur{\fam13} 
\textfont11=\tenbur
\scriptfont11=\eightbur 
\scriptscriptfont11=\eightbur
\font\twelvebur=msbm10 scaled 1200 
\textfont13=\twelvebur
\scriptfont13=\tenbur 
\scriptscriptfont13=\eightbur
\mathchardef\nat="0B4E 
\mathchardef\eps="0D3F
  
  
\title{Graphs of finite algebras, edges, and connectivity}
\author{Andrei A.\ Bulatov\\ 
} 
\date{} 
\maketitle
\begin{abstract}
We refine and advance the study of the local structure of idempotent finite algebras started
in [A.Bulatov, \emph{The Graph of a Relational Structure and Constraint Satisfaction Problems}, LICS, 2004]. We introduce a 
graph-like structure on an arbitrary finite idempotent algebra omitting type \one. We
show that this graph is connected, its edges can be classified into 3 types corresponding to
the local behavior (semilattice, majority, or affine) of certain term operations, and that
the structure of the algebra can be `improved' without introducing type \one\ by choosing
an appropriate reduct of the original algebra. Then we refine this structure demonstrating 
that the edges of the graph of an algebra can be made `thin', that is, there are term 
operations that behave very similar to semilattice, majority, or affine operations on 
2-element subsets of the algebra. Finally, we prove certain connectivity properties of the refined
structures.

This research is motivated by the study of the Constraint Satisfaction Problem, although
the problem itself does not really show up in this paper.
\end{abstract}

\section{Introduction}\label{sec:introduction}

The study of the Constraint Satisfaction Problem (CSP) and especially the Dichotomy 
Conjecture triggered a wave of research in universal algebra, as it turns out that the
algebraic approach to the CSP developed in \cite{Bulatov05:classifying,Jeavons97:closure}
is the most prolific one in this area. These developments have led to a number of strong results
about the CSP, see, e.g., \cite{Barto11:conservative,Barto14:local,Barto12:near,Bulatov06:3-element,Bulatov11:conservative,Bulatov14:conservative,Bulatov06:simple,Idziak10:few}. However, successful application of the algebraic approach also requires new results 
about the structure of finite algebras. Two ways to describe this structure have been 
proposed. One is based on absorption properties \cite{Barto15:constraint,Barto12:absorbing}
and has led not only to new results on the CSP, but also to significant developments in
universal algebra itself.

In this paper we refine and advance the alternative approach originally developed in 
\cite{Bulatov04:graph,Bulatov11:conjecture,Bulatov08:recent}, which is based on the 
local structure of finite algebras. This 
approach identifies subalgebras or factors of an algebra having `good' term operations,
that is, operations of one of the three types: semilattice, majority, or affine. It then explores 
the graph or hypergraph formed by such subalgebras, and exploits its connectivity properties.
In a nutshell, this method stems from  the early study of the CSP over so called conservative 
algebras \cite{Bulatov11:conservative}, and has led to a much simpler proof of the dichotomy
conjecture for conservative algebras \cite{Bulatov16:conservative} and to a characterization of 
CSPs solvable by consistency algorithms \cite{Bulatov09:bounded}. In spite of these applications
the original methods suffers from a number of drawbacks that make its use difficult. In the
present paper we refine many of the constructions and fix the deficiencies of the original
method. As in \cite{Bulatov04:graph,Bulatov08:recent} an edge is a pair of elements 
such that there is a factor algebra of the subalgebra generated by  that has an
operation which is semilattice, majority, or affine on the blocks containing ; this
operation determines the type of edge . In this paper we allow edges to have more 
than one type if there are several factors witnessing different types. The main difference
from the previous results is the introduction of oriented \emph{thin} majority and afiine edges.
An edge  is said to thin if there is a term operation that is semilattice on , or 
there is a term operation that satisfies the identities of a majority or affine term (say, in
variables ) on , but only when  and . Oriented thin edges
allow us to prove a stronger version of the connectivity of the graph related to an algbera. 
This updated approach makes it possible to give a much simpler proof of the 
result of \cite{Bulatov09:bounded} (see also \cite{Barto14:local}), however, this is a subject
of subsequent papers. 


\section{Preliminaries}\label{sec:preliminaries}

In terminology and notation we follow the standard texts on universal algebra
\cite{Burris81:universal,Mckenzie87:algebras}. We also assume familiarity with
the basics of the tame congruence theory \cite{Hobby88:structure}. All algebras in this paper
are assumed to be finite, idempotent, and omitting type \one.

Algebras will be denoted by , etc. The subalgebra of an algebra  generated by 
a set  is denoted , or if  is clear from the context simply by
. The set of term operations of algebra  is denoted by .
Subalgebras of direct products are often considered as relations. An element
(a tuple) of  is denoted in boldface, say, , and its th component
is referred to as , that is, . The set  will be denoted 
by . For , say, , , by  we denote
the -tuple , and for  by 
we denote the set . If  or  we write 
 rather than . The tuple  and relation  are called
the \emph{projections} of  and  on . A subalgebra (a relation) 
of  is said to be a \emph{subdirect product} of  if 
for every . For a congruence  of  and , by  we denote 
the -block containing , and by  the factor algebra modulo .
For , the congruence generated by  will be denoted by 
or just . By  we denote the least (i.e.\ the equality relation),
and the greatest (i.e.\ the total relation) congruence of , respectively. Again, we 
often simplify this notation to .



\section{Graph: Thick edges}\label{sec:thick}


\subsection{The three types of edges}\label{sec:three-types}

Let  be an algebra with universe .
We introduce graph   as follows. The vertex set
is the set . A pair  of vertices is a \emph{edge} if and only if
there exists a congruence  of  and a term operation of  such 
that either  is an affine 
operation on , or  is a semilattice operation on
, or  is a majority operation on
. 

If there exists a congruence and a term operation of  such that  
is a semilattice operation on  then  is said to have the
{\em semilattice type}. An edge  is of {\em majority type} if there are 
a congruence  and  (a term operation of , respectively)
 such that  is a majority operation on . Finally,  
has the {\em affine type} if there are a congruence  and  
(a term operation of , respectively) such that  is an affine operation 
on .  In all cases we say that congruence  \emph{witnesses}
the type of edge .

Note that, for every edge  of , there is the associated pair 
 from the factor structure. We will need both of these types of pairs 
and will sometimes call  a {\em thick} edge (see Fig.~1). The 
smallest congruence certifying the type of an edge  will be denoted by . 
\begin{figure}[t]
\centerline{\includegraphics{edges.eps}}
\caption{Edges and thick edges}
\end{figure}

Note also that a pair  may have more than one type witnessed by different 
congruences . Sometimes we need a stricter version of type. A pair 
is \emph{strictly semilattice} if it is semilattice;  is said to be \emph{strictly
majority}, if it is majority but not semilattice. Finally, pair  is said to be 
\emph{strictly affine} if it is affine, but not semilattice or majority.


\subsection{General connectivity}\label{sec:general-connectivity}

\begin{theorem}\label{the:connectedness}
If an idempotent algebra  omits type \one, then  is connected
for every subalgebra of .
\end{theorem}

Let  be an idempotent algebra. Recall that a {\em
tolerance} of  is a binary reflexive and symmetric
relation compatible with . The transitive closure of a tolerance is a
congruence of . In particular, if  is simple then the transitive
closure of every its tolerance different from the equality relation is
the total relation. If a tolerance satisfies this condition then we
say that it is {\em connected}. Let  be a tolerance. A set
 maximal with respect of inclusion and such that  is said to be a {\em class} of . We will need the following 
simple observation.
\begin{lemma}\label{lem:tol-class}
Every class of a tolerance of an idempotent algebra is a subalgebra.
\end{lemma}


Let  be a hypergraph. A {\em path} in 
is sequence  of hyperedges such that , 
for . The hypergraph  is said to be {\em connected} if,
for any , there is a path  such that ,
.

Clearly, the universe of an algebra  along with the family of all
its proper subalgebras forms a hypergraph denoted by
. Lemma~\ref{lem:tol-class} implies that, for a simple
idempotent algebra , the hypergraph  is connected unless 
 is tolerance free. In the latter case it can be disconnected.

If  is a congruence of a finite algebra  and  is a
compatible binary relation, then the {\em -closure} of  is
defined to be . A relation equal to its
-closure is said to be {\em -closed}. If  is a prime
quotient of , then the {\em basic tolerance for}  (see
\cite{Hobby88:structure}, Chapter~5) is the -closure of the
relation  is an -trace 
if , and it is the
-closure of the compatible relation generated by
 is an -trace if
. The basic tolerance is the
smallest -closed tolerance  of  such that
. 

Let  is a prime quotient of . An {\em
-quasi-order} is a compatible reflexive and
transitive relation  such that , and the
transitive closure of  is . The quotient
 is said to be {\em orderable} if there exists an
-quasi-order. By Theorem~5.26 of
\cite{Hobby88:structure},  is 
orderable if and only if .

Recall that an element  of an algebra 
is said to be {\em absorbing} if whenever  is an
-ary term operation of  such that  depends on  and
, then . A congruence  of 
 is said to be {\em skew} if it is the kernel of no projection
mapping of  onto its factors. If  is a simple idempotent
algebra, then the result of \cite{Kearnes96:idempotent} states that
one of the following holds: (a)  is term equivalent to a module;
(b)  has an absorbing element; or (c)  has no skew
congruence. 

We also need the following easy observation.
\begin{lemma}\label{lem:binary-tol}
Let  be an -ary compatible relation on  such that, for
any , . Then, for any ,
the relation  there are  such that  is a tolerance of .
\end{lemma}

Tolerance of the form  will be called \emph{link tolerance}, or
\emph{th link tolerance}
\begin{prop}\label{pro:simple}
Let  be a simple idempotent algebra.\2mm]
(2) If  and  has a proper tolerance,
then  is connected.\2mm]
(4) If ,  and  is tolerance free,
then either  are connected in , or there is a binary
term operation  or a ternary term operation  such that  
is a semilattice operation on , or  is a majority operation
on .
\end{prop}

\begin{proof}
(1) By Theorem~5.26 of \cite{Hobby88:structure}, there exists
-quasi-order  on , which is, clearly, just a
compatible partial order. Let  be such that  implies  or . We claim that  is a subalgebra of
. Indeed, for any term operation  of  and any
, we have . Finally, it follows from Lemma~5.24(3) and Theorem~5.26(2) that
 is connected.\2mm]
(3) Follows from the results of \cite{Kearnes96:idempotent}.\2mm]
Consider the relation  generated by . By the
assumption made,  is not the graph of a bijective mapping. By
Lemma~\ref{lem:binary-tol},  are tolerances of 
different from the equality relation. Thus, they are the total
relation. Therefore, there is  such that
. If both  are proper
subalgebras of , then  are connected in
. Otherwise, let, say, . Since
 and  is idempotent,  for any
. In particular, . This means that there is a
binary term operation  such that , as required.
\medskip

\noindent
{\sc Case 2.} There is an automorphism  of  such that
 and .\2mm]
Let  and . Since , this
relation is compatible. Clearly,  if and only if
. Notice that . Since
 and  is tolerance free, every pair 
is a trace. Therefore, there is a polynomial operation  with
 and, hence, there is a term operation  such
that . For this operation we have

\medskip

Next we show that  cannot be the equality relation. Suppose
for contradiction that it is. Then the relation
 there is  such that
 is a congruence of . It
cannot be a skew congruence, hence, it is kernel of the projection of
 onto one of its factors. Without loss of generality let
. This means that, for any
 and any , we have
. However, , a contradiction. The same
argument applies when .

Thus,  is the total relation, and there is  such
that  which implies . 
\medskip

\noindent
{\sc Claim 2.} For any , the tuple
.\
\lefteqn{g\left(\cll aba,\cll baa,\cll cda\right)=\cll{g(a,b,c)}{g(b,a,d)}a}\\
&=&\cll{c'}{\vf(g(\vf^{-1}(b),\vf^{-1}(a),\vf^{-1}(d)}a\\
&=&\cll{c'}{\vf(g(a,b,\vf^{-1}(d)}a=
\cll{c'}{\vf(\vf^{-1}(d'))}a=\cll{c'}{d'}a\in\rel. 

h\left(\cll aab,\cll ccd\right)=\cll bba,
1mm]
(1)  omits type \one.\1mm]
(3) If  is strict majority and  is sm-connected, then
 is sm-connected.
\end{theorem}

We prove Theorem~\ref{the:adding} by induction on the `structure' of the algebra.
The base case of this induction is given by strictly simple algebras.
Recall that a simple algebra whose proper subalgebras are all
1-element is said to be {\em strictly simple}. We need the description
of finite idempotent strictly simple algebras given in
\cite{Szendrei90:surjective}. 

Let  be a permutation group acting on a set . By  we
denote the set of operations on  preserving each
relation of the form  where
, and  denotes the set of idempotent
members of .

Let  be a finite dimensional
vector space over a finite field ,   the group of
translations 
, and  the endomorphism ring
of . Then one can consider  as a module over
. This module is denoted by .

Finally, let  denote the set of all
operations preserving the relation

where  is some fixed element of ,
and let .
\begin{theorem}[\cite{Szendrei90:surjective}] \label{str}
A finite strictly simple idempotent algebra  is 
term equivalent to one of the following algebras:

(a)  for a permutation group  on 
such that every nonidentity member of  has at most one
fixed point;

(b)  for
some vector space  over a finite field ;

(c)  for some  (),
some element  and some permutation group  acting on 
such that  is the unique fixed point of every nonidentity
member of~;

(d)  where  and  contains a semilattice operation;

(e) a two-element algebra with an empty set of basic operations.
\end{theorem}
It can be easily shown (see e.g.\ \cite{Bulatov05:classifying}) that
in case (c)  has a term {\em zero-multiplication} operation, that
a binary operation  such that  whenever .

\begin{proof}[of Theorem~\ref{the:adding}.]
Let  be an edge of semilattice type and  is a term operation
such that  is a semilattice operation on
. We will omit index
 everywhere it does not lead to a confusion. Let
 where  is the set of binary term operations  of
 such that  on  is either a projection or
equals . The subalgebra of  generated by a set
 will be denoted by , while the subalgebra of
 generated by the same set will be denote by . In
general, .
\medskip

\noindent
{\em Claim 1.}  can be chosen to satisfy the identity
.\
h(x,y)=f(\underbrace{x,f(x,\ldots f(x}_{\mbox{\footnotesize 
times}},y)\ldots)).
2mm]
In this case,  and . If 
 then there exists a term operation  of
 such that . As is easily seen, the
operation  equals  on
; hence, it belongs to
. However, , a contradiction with
the assumption made. Thus, . 

Then there is a term operation  of  which is either an affine 
or majority or semilattice operation on . The operation
 
in the first two cases or
 in the latter case belong to  and is an
affine or majority or semilattice operation on  respectively.
\medskip

\noindent
{\sc Case 1.B.}  is a 2-element
semilattice.\2mm] 
The operation  on  has the form 
and either  or  is invertible. Suppose that  is invertible
and  for a certain . Then set 

Since  and  are idempotent,  on  and
 on . 

Then, as in Case~1.A we show that
. Therefore,  is a
strictly simple algebra. If  is 2-element then we get one of the
previous cases. Otherwise,  either has a zero-multiplication
operation  or it is of the form  for a certain
permutation group . In the former case,  belongs
to  and is a zero-multiplication operation on . In the latter 
case,  has an operation which is either a semilattice or majority 
operation on . Arguing as above we get an operation of
 which is semilattice or majority on  respectively.
\medskip

\noindent
{\sc Case 1.D.}  has a zero-multiplication
operation .\2mm] 
If there is no automorphism  in  such that  and
, then, by Proposition~\ref{pro:simple},  has a term
operation  which is a semilattice operation on . So, let
us suppose that there is an automorphism swapping  and . 

If  has no
operation which is semilattice on  then we are
done. Otherwise, let  be a term operation of  semilattice on
 and  a term operation of  majority on
. If one of  is a semilattice
operation on  then we proceed as before. Otherwise,  is a
projection on ; without loss of generality let it be the first
projection. Then  equals  on 
 and is a projection on . We complete the proof as
before. 
\medskip

Now, suppose that the claim proved for all proper subalgebras of
. We consider two cases.
\medskip

\noindent
{\sc Case 1.1.} There is a maximal congruence  of
 such that  is commutative on .\2mm]
If, for every pair , where , and  is a term operation of ,
the subalgebra  of  is a
proper subalgebra of 
, then  is connected and, therefore  are connected 
by induction hypothesis. Indeed, if
, then
. Therefore,
.

Suppose that, for a certain  and a ternary term
operation  of , we have
. Then, for any
, there is a term operation  of  such that
. Consider
. We have
, hence, . On the
other hand, , where ,
. Thus, .

The elements  and  are connected by a path 
in . Thus, it is enough to show that if  is
connected by edges of semilattice, majority or affine type in ,
then so is . We may assume  is an edge. Let  be a 
maximal congruence of
 witnessing that it is an edge and  a maximal
congruence of  containing . 

If  is affine or 2-element we proceed in
the same way as in the base case of induction. If
 then
 are connected by a chain of 2-element
subalgebras, and the result follows from induction hypothesis.

So, suppose that . If
 is connected then we are done by
induction hypothesis. Otherwise we use
Proposition~\ref{pro:simple}. If there is no automorphism  of 
such that  and 
, then, by Proposition~\ref{pro:simple},
 has a term operation  which is a semilattice operation on
. So, let us suppose that there is an
automorphism swapping  and .  

If  has no operation which is semilattice on
 then we are 
done. Otherwise, let  be a term operation of  semilattice on
 and  a term operation of 
majority on 
. If one of 
is a semilattice operation on  then we
proceed as before. Otherwise,  is a projection on
; without loss of
generality let it be the first projection. Then
 equals  on
 and is a projection on
. We complete the proof as before.
\medskip

Now let  be of majority type and  a term operation
such that  is a majority operation on
. Let
 where  is the set of binary and ternary term
operations  of  such that  on  is either a
projection or equals . As before, the subalgebra of
 generated by a set 
 will be denoted by , while the subalgebra of
 generated by the same set will be denote by . In
general, .
\medskip

\noindent
{\em Claim 2.}  can be chosen to satisfy the identity
.\
h(x,y,z)&=&m(\underbrace{x,m(x,\ldots m(x}_{\mbox{\footnotesize 
times}},y,z),m(x,y,z)\ldots)),\\
& & m(x,\ldots,y,z),m(x,y,z)\ldots))).
2mm]
In this case,  and . If 
 then there exists a term operation  of
 such that . As is easily seen, the
operation  equals  on
; hence, it belongs to
. However, , a contradiction with
the assumption made. Thus, . 

Then there is a term operation  of  which is either an affine 
or majority or semilattice operation on . The operation
 
in the first two cases or
 in the latter case belong to  and is an
affine or majority or semilattice operation on  respectively.
\medskip

\noindent
{\sc Case 2.B.}  is a 2-element
semilattice.\2mm] 
The operation  on  has the form
 and either  or  or  is
invertible. Suppose that  is invertible and  for a certain
. Then set 

We have,  on  and
 on . Let  be the
characteristics of the ring . We set

For the operation  we have
 on  and 
 on .

Then, as in Case~2.A we show that
 (by substituting
). Therefore,  is a 
strictly simple algebra. If  is 2-element then we get one of the
previous cases. Otherwise,  either has a zero-multiplication
operation  or it is of the form  for a certain
permutation group . In the former case,
 belongs 
to  and is a zero-multiplication operation on . In the latter 
case,  has an operation which is either a semilattice or majority 
operation on . Arguing as above we get an operation of
 which is semilattice or majority on  respectively.
\medskip

\noindent
{\sc Case 2.D.}  has a zero-multiplication
operation .\2mm] 
If there is no automorphism  in  such that  and
, then, by Proposition~\ref{pro:simple},  has a term
operation  which is a semilattice operation on . So, let
us suppose that there is an automorphism swapping  and . 

If  has no
operation which is semilattice on  then we are
done. Otherwise, let  be a term operation of  semilattice on
 and  a term operation of  majority on
. If  can be chosen such that it is a majority
operation on  then we
proceed as before. Otherwise,  is either a projection or minority
or 2/3-minority operation on
. In the two latter case
one of  is the first projection on
 and the second
projection on ; let it be . 
Then  equals  on
 and is a projection on
. We complete the proof as before.
\medskip

Now, suppose that the claim proved for all proper subalgebras of
. We consider two cases.
\medskip

\noindent
{\sc Case 2.1.} There is a maximal congruence  of
 such that  in .\
&& px+(q+r)y=qx+(p+r)y,\\
&& px+(q+r)y=rx+(p+q)y,\\
&& qx+(p+r)y=rx+(p+q)y.

(p-q)x=(p-q)y,\ (p-r)x=(p-r)y,\ (q-r)x=(q-r)y.

m(x,m(x,y,y),m(x,y,y))&=&m(x,y,y)\\
(p+2p^2)x+4p^2y &=& px+2py\\
2p^2x+(4p^2-2p)y &=& 0.

g(h(h(c,d),d),d)=g(h(c,d),d)\in\Sgn{d,h(c,d)},

D&=&\{(m(c',d',d'),m(d',c',d')),
(m(c',d',d'),m(d',d',c')),(m(d',d',c'),\\
& & \ \ \ m(d',c',d'))\mid c',d'\in\zC\}
2mm]
If, for every pair , where  and  is a term 
operation of ,
the subalgebra  of  is a
proper subalgebra of , then  is connected
and, therefore  are connected  by induction hypothesis. Indeed, if
, then
. Therefore,
.

Suppose that, for a certain  and a ternary term
operation  of , we have
. Then, for any
, there is a term operation of  such that
. Without loss of generality we may assume
that  for certain
. Consider
. We have
, hence, . On the
other hand, . Thus, .

The elements  and  are connected by a path 
in . Thus, it is enough to show that if
 is 
connected by edges of semilattice, majority or affine type in ,
then so is . Assume  is an edge. Let  be a maximal congruence of
 witnessing that it is an edge and  a maximal
congruence of  containing . 

If  is affine or 2-element we proceed in
the same way as in the base case of induction. If
 then
 are connected by a chain of 2-element
subalgebras, and the result follows from induction hypothesis.

So, suppose that . If  is connected then
we are done by induction hypothesis. Otherwise we use
Proposition~\ref{pro:simple}. 
\end{proof}

\subsection{Unified operations}\label{sec:unified}

To conclude this section we prove that the polymorphisms (or term operations) 
certifying the strict type of edges can be significantly unifying (cf.\ Proposition~2
from \cite{Bulatov03:conservative}).
\begin{theorem}\label{the:uniform}
Let  be an idempotent algebra. There are term operations 
of  such that 
\begin{description}
\item
 is a semilattice
operation if  is a strict semilattice edge, it is the first projection if  is a 
strict majority or affine edge;
\item
 is a majority
operation if  is a strict majority edge, it is the first
projection if  is a strict affine edge, and
 if  is 
strict semilattice;
\item
 is an affine operation
operation if  is a strict affine edge, it is the first
projection if  is a strict majority edge, and
 if  is strict semilattice.
\end{description}
\end{theorem}

\begin{proof}
Show first that there is an operation  that is semilattice on
each semilattice edge. Let  be the list of all semilattice edges in the graph 
. To avoid clumsy notation we shall denote the operation
,  simply by . Let also
 be the list of term operations of the algebra such that
 is a semilattice operation. Notice that every binary
idempotent operation on a 2-element set is either a projection or a
semilattice operation, and every binary operation of a module can be
represented in the form . Since each  is idempotent,
for any ,  is either a projection, or a semilattice
operation. We prove by induction, that the operation  constructed
via the following rules is a semilattice operation on :
\begin{itemize}
\item
;
\item
.
\end{itemize}

The base case of induction,  holds by the choice of
. Suppose that  satisfies the required conditions. If
 is a projection, say, ,
then 

that is a semilattice operation on . Let , and  a
semilattice operation such that . Then

hence,  is again a semilattice operation. 

Thus, for each edge ,  is a semilattice operation if 
is red and either a semilattice operation or a projection or 
otherwise. However, if  is not red, then the subalgebra with the
universe  has no semilattice operation, therefore,  is
a projection or  whenever  is yellow or blue. Arguing
as in the previous section, one can transform  such that it
become a projection on blue edges. Finally, 
it is easy to check that  satisfies the conditions of the
proposition.

Now let ,  be the lists of all yellow and all blue
    edges respectively, and ,  the lists of 
    term operations of the algebra such that  is an
    affine operation, and  is the minority
    operation. Notice first, that since neither
 nor  has a 
    term semilattice operation, every their binary term operation is 
    either a projection or, for blue edges an operation of the form . 
Therefore, for any ,
    ,
    , and 
    ,, . This means that the operations  are of one of the
    following types: a projection, the minority operation, the
    majority operation, a 2/3-{\em minority operation}, that is an
    operation satisfying the equalities ,
     or similar. 

First we prove by induction that for every  there is an
operation  which is majority on  for . The
operation  gives the base case of induction. Let us assume
that  is already found. If  is the majority
operation, set . Otherwise, it is either a projection, or
a 2/3-minority operation, or the minority operation. In all these case
its variables can be permuted such that
. Then the operation
 satisfies the conditions ,
and  for all . It is not hard to see that the
operation

satisfies the required conditions. 

Further, consider the operation . Its restriction ,
, is either a projection, or the minority operation. If
 is an operation , then using the
methods of the previous section we can derive
an operation  such that  for all , and . The operation 
is majority on , , a projection on . Therefore,
 can be assumed to be a projection for all . Then for the operation  

we have

Finally, to make  acting correctly on red edges we set 

The operation  is as required. 

Next we show that for any  there is  such that
 is an affine operation for . As usual,
 gives the base case of induction. If  is obtained,
then if  is an affine operation then set
. Otherwise, . One of
the coefficients is invertible, let  is invertible and
. Then set 

we have  and  for
. Furthermore, : ,
 for . If  is invertible then repeating the
procedure above we get  which an affine operation on all
the , . Otherwise,  is invertible, therefore,
applying the same procedure to  we get an operation
 such that  and  for
. Then to obtain the required operation we set
. 

Finally, set ,

and

As is easily seen  satisfies the conditions required.
\end{proof}


\section{Thin edges}\label{sec:thin}

We start with an observation that operations  identified in 
Theorem~\ref{the:uniform} can be assumed to satisfy certain identities.

\begin{lemma}\label{lem:fgh-identities}
Operations  found in Theorem~\ref{the:uniform} can be chosen such that
\begin{itemize}
\item[1.]
 for all ;
\item[2.]
 for all ;
\item[3.]
 for all .
\end{itemize}
\end{lemma}

\begin{proof}
1. Let  for . We need to show that  can be chosen such 
that . Clearly, this can be done by substituting  
 times. It remains to show that every function  obtained inductively 
from  and  is a replacement for . 
That is, for any semilattice edge , where  witnesses that  
is a semilattice edge,

By induction we have 


2. Let  be the operation that is majority on all strict majority edges, and . 
We need to show that  can be chosen such that . Clearly, this 
can be done by substituting   times. It remains to show 
that every function  obtained inductively from  and 
 is a replacement for . That is, for any 
strict majority edge , where  witnesses that  is a 
majority edge,

By induction we have 


3. Let  for . The goal is to find  such that 
 for 
all  and all . Clearly, this can be done by substituting  
 times. It remains to show that every function  obtained 
inductively from  and  
is a replacement for . That is, for any affine edge , 
where  witnesses that  is an affine edge,

By induction we have 

\end{proof}

\subsection{Semilattice edges}\label{sec:thin-semilattice}

In this section we focus on (strict) semilattice edges of the graph . Note
first that if one fix a term operation  such that  is a semilattice
operation on every thick semilattice edge of , then one can define
an orientation of every semilattice edge. A semilattice edge  is oriented from  
to  if . 
Clearly, the orientation strongly depends on 
the choice of the term operation . The graph  oriented
according to a term operation  will be denoted by . We
then can define {\em semilattice-connected} and {\em strongly semilattice-connected}
components of . We will also use the natural order on the
set of strongly semilattice-connected components of : for
components ,  if there is a directed path in 
consisting of semilattice edges and connecting a vertex from  with a vertex
from . 

We shall now improve the choice of operation  and restrict the kind of
semilattice edges we will use later.
First we show that those semilattice edges  for which  is not
the equality relation can be thrown out of the graph  such that
the graph remains connected. Therefore, we can assume that every semilattice
edge  is such that  is a semilattice operation on . 
 
\begin{prop}\label{pro:thin-thick}
Let  be a finite algebra omitting type \one,  a binary term operation 
semilattice on every (thick) semilattice edge and such that , 
and  the subgraph of  obtained by omitting semilattice 
edges  such that  is not the equality relation. Then  
is connected. Moreover, if  is s-connected then  
is semilattice-connected. If  is sm-connected then  is
sm-connected. 
\end{prop}

\begin{proof}
Firstly, by Theorem~\ref{the:adding} we may assume that every thick
semilattice edge of  is a subalgebra.
It suffices to show that for any semilattice edge  (in ),
the veritces  are connected (s-connected or sm-connected) in .
So, assume  and  is a semilattice edge in .
We proceed by induction on order ideals of the lattice  of subalgebras 
of . The base case 
of induction, when  is strictly simple is obvious,
because  and there is a semilattice operation on
this algebra.

Let  be the maximal congruence of  witnessing that 
 is an edge. Let , then . By the induction
hypothesis  is connected (s-connected,sm-connected) with  in
. Therefore, we may assume . If 
 then we are done, because  and
 are connected inside  and  and  are
connected inside . Otherwise there is a term operation 
 such that . Then, for the operation
, we have

Thus, there is a semilattice operation on , hence
. 
\end{proof}

The graph  oriented according to a binary term operation  will be
denoted by . Semilattice edges  such that  is
the equality relation will be called \emph{thin semilattice edges}.

Using Proposition~\ref{pro:thin-thick} we are able to impose more
restrictions on the term operation .
\begin{prop}\label{pro:good-operation}
Let  be a finite algebra omitting type \one. There is a binary
term operation  of  such that  is a semilattice operation on
every thick semilattice edge of  and, for any , either
 or the pair  is a semilattice edge of .
\end{prop}

\begin{proof}
Let  be a binary term operation such that  is semilattice on every 
semilattice edge and . Let  be such that
, and set  and 
for .

\medskip

{\sc Claim 1.} 
For any , .

\smallskip

Indeed, , and for any 


Let . Then , and there is 
with .

\medskip

{\sc Claim 2.}
.

\smallskip

Since , there is a term operation
 such that . Let . For this operation we have


This means that  is a semilattice edge, and the congruence witnessing
it is the equality relation. By the choice of , it is a
semilattice operation on any such pair.

Let  be the maximal among the numbers chosen as before
in Claim 2 for all pairs  with . Let ,
and  for . Let also
.

\medskip

{\sc Claim 3.} For any , either , or the
pair , where  is a semilattice edge witnessed by the
equality relation.

\smallskip

If  then it is straightforward that .
Suppose . We proceed by induction. Since
, where the  are constructed as before,
by Claim 2 . This gives the base case
of induction. Suppose . Then

Claim 3 is proved.

\smallskip

To complete the proof it suffices to check that  is a
semilattice operation on every (thick) semilattice edge of .
However, this is straightforward from the construction of
.
\end{proof}

It will be convenient for us to denote binary operation  that satisfies the 
conditions of Theorem~\ref{the:uniform}, Lemma~\ref{lem:fgh-identities}(1),
and Proposition~\ref{pro:good-operation} by , that is, to write 
or just  for . The fact that  is a thin semilattice edge we will
also denote by . In other words,  if and only if .

\begin{lemma}\label{lem:sl-thick-thin}
Let  be a thick semilattice edge,  the congruence of  that witnesses 
this, and . Then there is  such that  is a thin semilattice 
edge.
\end{lemma}

\begin{proof}
By Proposition~\ref{pro:good-operation}  or . Since  
the former option is impossible. Therefore  is a thin semilattice edge.
\end{proof}


\subsection{Thin majority edges}\label{sec:thin-majority}

Here we introduce thin majority edges in a way similar to thin semilattice edges,
although in a weaker sense.

\begin{lemma}\label{lem:thin-majority}
Let  be an algebra,  a majority edge in it, and  the congruence of 
 witnessing that. Then there is  and a ternary term operation 
 of  such that .
\end{lemma}
\marginpar{Check the `strict' business}

\begin{proof}
Suppose that  is such that  is minimal among all subalgebras  
for , and such that . Such an element exists by 
Lemma~\ref{lem:fgh-identities}. Consider the ternary relation  
generated by . Applying  to these tuples we get 
 for some . Since , say, , 

Again, as , using  we get .
\end{proof}

A majority edge satisying the conditions of Lemma~\ref{lem:thin-majority} will be
called a \emph{thin majority edge}. More precisely, a pair  is called a thin 
majority edge if (a) it is a majority edge, (b) for any , 
, (c) , and (d) there exists a ternary term operation 
 such that .  The operation  from 
Theorem~\ref{the:uniform} does not have to satisfy any specific conditions on 
the set , except what follows from its definition. Also, thin majority edges
are directed, since  in Lemma~\ref{lem:thin-majority} occur asymmetrically.

\begin{corollary}\label{cor:thin-majority}
For any strict majority edge , where  is a witnessing congruence, there is 
 such that  is a thin majority edge.
\end{corollary}

We now consider the interaction of term operations on thin edges in different 
similar algebras.

\begin{lemma}\label{lem:thin-majority-triple}
Let  be similar idempotent algebras all omitting type \one.
Let , , and  be thin majority edges in , 
witnessed by congruences , respectively. Then there is an operation 
 such that , , .
\end{lemma}

\begin{proof}
Let  be the subalgebra of  generated by 
. Since  satisfies condition
(c) of the definition of thin majority edges,

and . By condition (b)
, in particular, there is a term operation  such that 
. Then 

and . Again by condition (b) , in particular, 
there is a term operation  such that . Then 

The result follows.
\end{proof}

\begin{lemma}\label{lem:majority-sl}
Let  be similar idempotent algebras all omitting type \one.
Let  be a thin majority edge in , witnessed by congruences , and 
 in . Then there is an binary operation  such that  and 
.
\end{lemma}

\begin{proof}
Let  be the subalgebra of 
 generated by . Since  satisfies condition (c) of
the definition of thin majority edges,

as  is the first projection on semilattice edges. Therefore  satisfies 
the conditions.
\end{proof}


\subsection{Thin affine edges}\label{sec:thin-affine}

\begin{lemma}\label{lem:thin-affine}
Let  be an algebra,  an affine edge in it, and  the congruence of 
 witnessing that. Then there is  and a ternary term operation 
 of  such that  .
\end{lemma}

\begin{proof}
Suppose  satisfy the conditions of the lemma. 
Let , by Lemma~\ref{lem:fgh-identities}(3) . 
As is easily seen, there is  that satisfies this condition and such that for any 
 where  is the restriction of  on , it holds
.

Consider relation  generated by pairs . Since , 

and . By the assumption , in particular, . 
The result follows.
\end{proof}

Similar to the majority case, an affine edge satisying the conditions of 
Lemma~\ref{lem:thin-affine} will be called a \emph{thin affine edge}. More precisely, 
a pair  is called a thin majority edge if (a) it is an affine edge, (b) for any 
, , (c) , and (d) there exists a ternary 
term operation  such that .  The operation  from 
Theorem~\ref{the:uniform} does not have to satisfy any specific conditions on 
the set , except what follows from its definition. Also, thin affine edges
are directed, since  in Lemma~\ref{lem:thin-affine} occur asymmetrically.

\begin{corollary}\label{cor:thin-affine}
For any affine edge , where  is a witnessing congruence, there is  
such that  is a thin affine edge.
\end{corollary}

\begin{lemma}\label{lem:thin-affine-pair}
Let  be similar idempotent algebras all omitting type \one.
Let  and  be thin affine edges in , witnessed by congruences 
, respectively. Then there is an operation  such that  
and .
\end{lemma}

\begin{proof}
Let  be the subalgebra of  generated by . 
By condition (c) of the definition of thin affine edges,

and . By condition (b) , in particular, 
. The result follows.
\end{proof}

\begin{lemma}\label{lem:affine-sl}
Let  be similar idempotent algebras all omitting type \one.
Let  be a thin affine edge in , witnessed by congruences , and  in . Then there is an operation  such that  and .
\end{lemma}

\begin{proof}
Let  be the subalgebra of  generated by . 
By condition (c) of the definition of thin affine edges,

as  on semilattice edges. The result follows.
\end{proof}

\begin{lemma}\label{lem:affine-maj}
Let  be a thin affine edge in , witnessed by congruences , and  
is a thin majority edge in . Then there is a binary operation  such that 
 and .
\end{lemma}

\begin{proof}
Let  be similar idempotent algebras all omitting type \one.
Let  be the subalgebra of  generated by . 
By condition (c) of the definition of thin majority edges,

where , as  is the first projection on . 
Then as , we get . The result follows.
\end{proof}



\section{Connectivity}\label{sec:connectivity}

Let  be an algebra omitting type \one. A \emph{path} in  is a sequence 
 such that  is a thin edge for all  (note that thin 
edges are always assumed to be directed). We will 
distinguish paths of several types depending on what types of edges are allowed. 
If  for  then the path is called a \emph{semilattice} or 
\emph{s-path}. If for every  either  or  is a thin
affine edge then the path is called \emph{affine-semilattice} or \emph{as-path}. 
Similarly, if only semilattice and thin majority edges are allowed we have a 
\emph{semilattice-majority} or \emph{sm-path}. 
We say that  is \emph{connected} to , , if there is a path 
. If this path is semilattice (aftine-semilattice, 
semilattice-majority) then  is said to be \emph{s-connected} (or \emph{as-connected},
or \emph{sm-connected}) to . We denote this by  (for s-connectivity),
 and  for as- and sm-connectivity, respectively.

Let  denote the digraph whose nodes are the elements of , and
the arcs are the thin semilattice edges. The strongly connected component of 
containing  will be denoted by . The set of strongly connected 
components of  are ordered in the natural way (if  then ), 
the elements belonging to maximal ones will be called \emph{maximal}, and
the set of all maximal elements from  by . In a similar way
we construct the graph  by including all the thin semilattice and 
affine edges. The strongly connected component of  containing
 will be denoted by . A maximal strongly connected component 
of this graph is called an \emph{as-component}, an element from an as-component 
is called \emph{as-maximal}, and the set of all as-maximal elements is denoted by 
.

In this section we show that all maximal elements are connected to each other.
The undirected connectivity easily follows from the definitions, so the challenge 
is to prove directed connectivity, as defined above. We start with an auxiliary lemma.

Let  be a relation. Recall that , ,  denotes
the \emph{link} tolerance 



\begin{lemma}\label{lem:going-maximal}
Let  be simple, , and  a
subdirect square of . Let also  be the tolerance defined
by . If
 is a connected tolerance then there is a sequence  such that ,  is maximal, , and if  is such that  then
 can also be chosen maximal. 
\end{lemma}

\begin{proof}
We start with any sequence , 
connecting  and . Such a sequence exists because  is a
connected tolerance. We prove by induction on .  
The base case of induction is obvious by the choice
of . Suppose  is maximal. Let  be
a semilattice path and  a maximal element. Let also
 be extensions of the  and
 for . Then for each , , we construct the sequence  and for
each , , the sequence , where 

Then observing that

we get that  for
any . Note also that  is a maximal element.
Continuing in a similar way we also can guarantee that  is a
maximal element. 

This process replaces  with a maximal element. However,  is also
replaced with another element, and we need to restore the connection of  with 
the preceding elements. Since  is maximal and , these two 
elements belong to the same maximal component. Therefore, there is a semilattice path
. We now proceed as before. Let  be
extensions of the . Then for each , , we construct
sequence  and for each , ,
sequence , where 

Then observing that

we get that  for any
. Note also that ,  is a maximal
element, and  belongs to the same maximal component as .  
\end{proof}

\begin{prop}\label{pro:as-connectivity}
Let . Then  is connected to .
\end{prop}

\begin{proof}
We prove the proposition by induction on the size of  through a sequence of claims.

\smallskip

{\sc Claim 1.}
 can be assumed to be .

\smallskip

If , , then we are done by the induction hypothesis. 
Suppose they are not and let  be such that  and . 
By the induction hypothesis  is connected to . 
As ,  is connected to . It remains to show that 
 is connected to . This, however, follows straightforwardly from the assumption 
that  is maximal, and therefore , and so  in .

\smallskip

{\sc Claim 2.}
 can be assumed simple.

\smallskip

Suppose   is not simple and  is its maximal congruence. Let . By the 
induction hypothesis  is connected to , that is, there is a sequence  such that  or  is a 
thin affine or majority edge in . We will choose some , where  is 
viewed as a subalgebra of , such that  is connected to  in . 
Set .

Depending on whether  is a semilattice, affine, or majority edge, use 
Lemma~\ref{lem:sl-thick-thin}, \ref{lem:thin-affine}, or~\ref{lem:thin-majority} to
choose  such that  is a thin edge.


It remains to show that  is connected to . Since  is maximal, it suffices 
to take elements  maximal in  and such that  and 
. Then  is connected to  by the induction hypothesis, and  
is connected to  in , as .

\smallskip

{\sc Claim 3.}
 can be assumed equal to  for any , .

\smallskip

If  for some , , then by the 
induction hypothesis  is connected to  for some , . 
Therefore  is also connected to .

\smallskip

Let  be the binary relation generated by  and . We consider two cases.

\smallskip

{\sc Case 1.}
 is not the graph of a mapping, or, in other words, there is no automorphism of  that 
maps  to  and  to .

\smallskip

Consider the tolerance e induced by  on . Since  is simple and  is not the graph of a mapping,  is a connected  tolerance. There are again two options.

\smallskip

{\sc Subcase 1a.}
For every  the set .

\smallskip

There are  such that , , and 
 for every . By 
Lemma~\ref{lem:going-maximal} the 's can be chosen maximal, 
and therefore for every  we can choose  which is 
maximal in . For each  choose  with  
and . By the induction hypothesis  is connected to . Then clearly 
 is connected to , and, as  is maximal and ,  is 
connected to .

\smallskip

{\sc Subcase 1b.}
There is  such that .

\smallskip

By Lemma~\ref{lem:going-maximal} there are , , and a maximal 
element  such that . Since , we have 
. Thus,  can be assumed maximal. We have therefore 
. If both  and  are proper 
subalgebras of , then proceed as in Subcase 1a. Otherwise suppose 
. This means , and, in particular, . 
Therefore there is a binary term operation  such that , that 
is . Since both elements are maximal, , implying they are connected.

\smallskip

{\sc Case 2.}
 is the graph of a mapping, or, in other words, there is an automorphism of 
 that maps  to  and  to .

\smallskip

There are two cases to consider.

\smallskip

{\sc Subcase 2a.}
There is no nonmaximal element  or  for any , . 

\smallskip

If there is a maximal  such that  for some nonmaximal  
(that is, ), then 
by Case~1 and Subcase~2b  is connected to  and  is connected to .
Suppose all elements in  are maximal. 
By Theorem~\ref{the:connectedness} and Proposition~\ref{pro:thin-thick} there are 
 such that for any  either  is an 
affine or majority edge (not a thin edge), or , or . 
In the latter two cases , and therefore there is a semilattice 
path from  to . We need to show that if  is an affine 
or majority edge then  is connected 
to . Let  be a congruence of  witnessing that 
 is an affine or majority edge. By Lemmas~\ref{lem:thin-majority} 
and~\ref{lem:thin-affine} there is  such that  is a
thin edge. Then take  such that  
and . By the induction hypothesis  is connected 
to . Finally, as all elements in  are maximal,  is connected with 
 in  with a semilattice path.

\smallskip

{\sc Subcase 2b.}
There is a nonmaximal element  with  or  for some  
or . In particular, this happens whenever there is a nonmaximal element 
 with  or .

\smallskip

Note first that we may assume that, for any , there is an 
automorphism that sends  to  and  to , as otherwise we are in the
conditions of Case~1. Recall that we also assume  
. Because of this and the automorphism swapping  and , 
without loss of generality we may assume that there is nonmaximal . 
Consider .

If , consider the relation  generated by . Since  
is not maximal,  cannot be the graph of an automorphism. Therefore  
induces a nontrivial tolerance on  that, in particular, connects  and , and 
we either complete as in Case~1, or show that , which is a contradiction, as 
 in this case.

If , take  and such that . By the 
induction hypothesis  is connected to . Now let  such that 
. It remains to show that  is connected to . If 
, the result follows by the induction hypothesis. If there 
is no automorphism that swaps  and , we argue as in Case~1. So, let 
 and there is an automorphism swapping  and .

Elements  are said to be \emph{v-connected} if there is  such 
that  and . The result follows from the next statement.

\smallskip

{\sc Claim 4.}
If  are v-connected and there is an automorphism of  that swaps  
and , then they are connected.

\smallskip

Let . The \emph{s-distance} from  to  is the length of the 
shortest semilattice path from  to . The s-distance from  to  is 
the shortest s-distance from  to an element from . The \emph{depth} 
of an element  is the greatest s-distance to a maximal  
component, denoted . We prove the Claim by induction on the size of 
 and , provided , .

If , , in particular, if , then there is a binary term 
operation  such that . Let . Since there is an automorphism 
swapping  and ,  and . 
Set . We have

Since  and  are automorphic,  is a majority operation on . 
Therefore,  and  are connected.


Suppose the Claim is proved for all algebras and pairs of elements v-connected 
through an element of depth less than . Let  
and , and  or . We may assume 
 is a nonmaximal element and consider . As in 
Case~1a if  then there is no automorphism swapping  
and . Then we consider relation  generated by 
. We can show that  and  are connected in this case.  

Suppose . Let  be such that , let also 
 be such that . By the induction hypothesis  is connected 
to , and therefore to . Also,  and  are v-connected through , 
and . If , we conclude by the induction 
hypothesis of the proposition. If , by the induction hypothesis of 
Claim~4,  is connected to .
\end{proof}

\bibliographystyle{plain}
\begin{thebibliography}{10}

\bibitem{Barto11:conservative}
Libor Barto.
\newblock The dichotomy for conservative constraint satisfaction problems
  revisited.
\newblock In {\em {LICS}}, pages 301--310, 2011.

\bibitem{Barto15:constraint}
Libor Barto.
\newblock The constraint satisfaction problem and universal algebra.
\newblock {\em The Bulletin of Symbolic Logic}, 21(3):319--337, 2015.

\bibitem{Barto12:absorbing}
Libor Barto and Marcin Kozik.
\newblock Absorbing subalgebras, cyclic terms, and the constraint satisfaction
  problem.
\newblock {\em Logical Methods in Computer Science}, 8(1), 2012.

\bibitem{Barto14:local}
Libor Barto and Marcin Kozik.
\newblock Constraint satisfaction problems solvable by local consistency
  methods.
\newblock {\em J. {ACM}}, 61(1):3:1--3:19, 2014.

\bibitem{Barto12:near}
Libor Barto, Marcin Kozik, and Ross Willard.
\newblock Near unanimity constraints have bounded pathwidth duality.
\newblock In {\em Proceedings of the 27th Annual {IEEE} Symposium on Logic in
  Computer Science, {LICS} 2012, Dubrovnik, Croatia, June 25-28, 2012}, pages
  125--134, 2012.

\bibitem{Bulatov03:conservative}
Andrei~A. Bulatov.
\newblock Tractable conservative constraint satisfaction problems.
\newblock In {\em 18th {IEEE} Symposium on Logic in Computer Science {(LICS}
  2003), 22-25 June 2003, Ottawa, Canada, Proceedings}, page 321, 2003.

\bibitem{Bulatov04:graph}
Andrei~A. Bulatov.
\newblock A graph of a relational structure and constraint satisfaction
  problems.
\newblock In {\em {LICS}}, pages 448--457, 2004.

\bibitem{Bulatov06:3-element}
Andrei~A. Bulatov.
\newblock A dichotomy theorem for constraint satisfaction problems on a
  3-element set.
\newblock {\em J. {ACM}}, 53(1):66--120, 2006.

\bibitem{Bulatov09:bounded}
Andrei~A. Bulatov.
\newblock Bounded relational width.
\newblock available at {\tt
  https://www.cs.sfu.ca/~abulatov/papers/relwidth.pdf}, 2009.

\bibitem{Bulatov11:conservative}
Andrei~A. Bulatov.
\newblock Complexity of conservative constraint satisfaction problems.
\newblock {\em {ACM} Trans. Comput. Log.}, 12(4):24, 2011.

\bibitem{Bulatov11:conjecture}
Andrei~A. Bulatov.
\newblock On the {CSP} dichotomy conjecture.
\newblock In {\em Computer Science - Theory and Applications - 6th
  International Computer Science Symposium in Russia, {CSR} 2011, St.
  Petersburg, Russia, June 14-18, 2011. Proceedings}, pages 331--344, 2011.

\bibitem{Bulatov14:conservative}
Andrei~A. Bulatov.
\newblock Conservative constraint satisfaction re-revisited.
\newblock {\em CoRR}, abs/1408.3690, 2014.

\bibitem{Bulatov16:conservative}
Andrei~A. Bulatov.
\newblock Conservative constraint satisfaction re-revisited.
\newblock {\em Journal of Computer and System Sciences}, 82(2):347--356, 2016.

\bibitem{Bulatov06:simple}
Andrei~A. Bulatov and V{\'{\i}}ctor Dalmau.
\newblock A simple algorithm for mal'tsev constraints.
\newblock {\em {SIAM} J. Comput.}, 36(1):16--27, 2006.

\bibitem{Bulatov05:classifying}
Andrei~A. Bulatov, Peter Jeavons, and Andrei~A. Krokhin.
\newblock Classifying the complexity of constraints using finite algebras.
\newblock {\em {SIAM} J. Comput.}, 34(3):720--742, 2005.

\bibitem{Bulatov08:recent}
Andrei~A. Bulatov and Matthew Valeriote.
\newblock Recent results on the algebraic approach to the {CSP}.
\newblock In {\em Complexity of Constraints - An Overview of Current Research
  Themes [Result of a Dagstuhl Seminar].}, pages 68--92, 2008.

\bibitem{Burris81:universal}
S.~Burris and H.P. Sankappanavar.
\newblock {\em A course in universal algebra}, volume~78 of {\em Graduate Texts
  in Mathematics}.
\newblock Springer-Verlag, New York-Berlin, 1981.

\bibitem{Hobby88:structure}
D.~Hobby and R.N. McKenzie.
\newblock {\em The Structure of Finite Algebras}, volume~76 of {\em
  Contemporary Mathematics}.
\newblock American Mathematical Society, Providence, R.I., 1988.

\bibitem{Idziak10:few}
Pawel~M. Idziak, Petar Markovic, Ralph McKenzie, Matthew Valeriote, and Ross
  Willard.
\newblock Tractability and learnability arising from algebras with few
  subpowers.
\newblock {\em {SIAM} J. Comput.}, 39(7):3023--3037, 2010.

\bibitem{Jeavons97:closure}
Peter Jeavons, David~A. Cohen, and Marc Gyssens.
\newblock Closure properties of constraints.
\newblock {\em J. {ACM}}, 44(4):527--548, 1997.

\bibitem{Kearnes96:idempotent}
K.~Kearnes.
\newblock Idempotent simple algebras.
\newblock In {\em Logic and algebra (Pontignano, 1994)}, volume 180 of {\em
  Lecture Notes in Pure and Appl. Math.}, pages 529--572. Dekker, New York,
  1996.

\bibitem{Mckenzie87:algebras}
R.N. McKenzie, G.~McNulty, and W.~Taylor.
\newblock {\em Algebras, Lattices, Varieties, I}.
\newblock Wadsworth--Brooks/Cole, Monterey, California, 1987.

\bibitem{Szendrei90:surjective}
A.~Szendrei.
\newblock Simple surjective algebras having no proper subalgebras.
\newblock {\em Journal of the Australian Mathematical Society (Series A)},
  48:434--454, 1990.

\end{thebibliography}


\end{document}