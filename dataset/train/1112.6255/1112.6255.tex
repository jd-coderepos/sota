\documentclass[11pt]{article}
\usepackage{fullpage} 

\usepackage[all=normal,bibliography=tight,bibnotes=tight]{savetrees}


\usepackage{amssymb,comment}
\usepackage{amsmath}
\usepackage{xspace}
\usepackage{tikz}
\usepackage{pgflibrarysnakes}
\usetikzlibrary{snakes}
\usepackage{times}
\usepackage{amstext}
\usepackage{calc}
\usepackage{amsopn}
\usepackage[noend]{algorithmic}
\usepackage[boxed]{algorithm}
\usepackage{eucal}
\usepackage{latexsym}
\usepackage{tabularx}

\usepackage{amsthm}

\newtheorem{theorem}{Theorem}
\newtheorem{corollary}[theorem]{Corollary}
\newtheorem{proposition}[theorem]{Proposition}
\newtheorem{lemma}[theorem]{Lemma}


\theoremstyle{definition}
\newtheorem{definition}[theorem]{Definition}
\newtheorem{remark}[theorem]{Remark}
\newtheorem{reduction}{Reduction}
\newtheorem{step}{Step}{\bf}{\rm}


\newcommand{\Z}{\ensuremath{\mathbb{Z}}}
\newcommand{\N}{\ensuremath{\mathbb{N}}}

\floatname{algorithm}{Algorithm}
\renewcommand{\algorithmicrequire}{\textbf{Function}}

\newcommand{\heading}[1]{
  \medskip
  \noindent \textbf{#1}
}

\newcommand{\tudu}[1]{{\bf{TODO: #1}}}

\newcommand{\defproblemu}[4]{
\vspace{1mm}
\noindent\fbox{
  \begin{minipage}{\textwidth}
  \begin{tabular*}{\textwidth}{@{\extracolsep{\fill}}lr} #1 & {\bf{Parameter:}} #3 \\ \end{tabular*}
  {\bf{Input:}} #2  \\
  {\bf{Question:}} #4
  \end{minipage}
  }
\vspace{1mm}
}

\newcommand{\defproblemugoal}[4]{
\vspace{1mm}
\noindent\fbox{
  \begin{minipage}{\textwidth}
  \begin{tabular*}{\textwidth}{@{\extracolsep{\fill}}lr} #1 & {\bf{Parameter:}} #3 \\ \end{tabular*}
  {\bf{Input:}} #2  \\
  {\bf{Goal:}} #4
  \end{minipage}
  }
\vspace{1mm}
}

\newcommand{\gfvs}{{\sc{Group Feedback Vertex Set}}\xspace}
\newcommand{\gfvsshort}{{\sc{GFVS}}\xspace}
\newcommand{\compgfvs}{{\sc{Compression Group Feedback Vertex Set}}\xspace}
\newcommand{\cgfvsshort}{{\sc{C-GFVS}}\xspace}
\newcommand{\fvsname}{{\sc{Feedback Vertex Set}}}
\newcommand{\sfvs}{{\sc{Subset Feedback Vertex Set}}\xspace}
\newcommand{\esfvs}{{\sc{Edge Subset Feedback Vertex Set}}\xspace}
\newcommand{\esfvsshort}{{\sc{ESFVS}}\xspace}
\newcommand{\oct}{{\sc{Odd Cycle Transversal}}\xspace}
\newcommand{\octshort}{{\sc{OCT}}\xspace}
\newcommand{\fvsshort}{{\sc{FVS}}}
\newcommand{\mwc}{{\sc{Multiway Cut}}\xspace}
\newcommand{\mwcshort}{{\sc{MWC}}\xspace}
\newcommand{\asatname}{{\sc{Almost 2-SAT}}\xspace}
\newcommand{\Ohstar}{\ensuremath{O^\ast}}

\begin{document}

  \date{}

\author{
  Marek Cygan \thanks{
    Institute of Informatics, University of Warsaw, Poland,
      \texttt{cygan@mimuw.edu.pl}
  }
  \and
  Marcin Pilipczuk\thanks{
    Institute of Informatics, University of Warsaw, Poland,
      \texttt{malcin@mimuw.edu.pl}
  }
  \and
  Micha\l{} Pilipczuk \thanks{
    Department of Informatics, University of Bergen, Norway, \texttt{michal.pilipczuk@ii.uib.no}
  }
}

  \title{On group feedback vertex set parameterized by the size of the cutset}

  \maketitle

\begin{abstract}
We study the parameterized complexity of
a robust generalization of the classical \fvsname{} problem,
namely the \gfvs{} problem; we are given a graph  with edges labeled
with group elements, and the goal is to compute the smallest set of vertices that
hits all cycles of  that evaluate to a non-null element of the group.
This problem generalizes not only \fvsname{}, but also \sfvs, \mwc{} and \oct{}.
Completing the results of Guillemot [Discr. Opt. 2011], we provide a fixed-parameter
algorithm for the parameterization by the size of the cutset only.
Our algorithm works even if the group is given as a polynomial-time oracle.
\end{abstract}

\section{Introduction}

\begin{comment}
Many real-life computational tasks can be modeled as combinatorial problems,
allowing rigorous theoretical analysis of possible algorithmic approaches.
One of the most important such abstractions is the well-known \fvsname{} (\fvsshort{})
problem, where, given an undirected graph , we seek for a minimum set of vertices
that hits all cycles of .
Another examples are \mwc{}
(\mwcshort{}: separate each pair from a given set of terminals in a graph
with a minimum cutset) or \oct{} (\octshort{}: make a graph bipartite by a minimum number
of vertex deletions).

Although all three aforementioned problems are NP-complete
even in such restricted cases as planar graphs, they turn out to be fixed-parameter
tractable when parameterized by the solution size. Recall that a fixed-parameter algorithm
solves an instance of size  with parameter  in time  for a computable
function  and a constant  independent of . Thus, the instances are 
solvable efficiently for small values of the solution size.
\end{comment}

The parameterized complexity is an approach for tackling NP-hard problems by designing algorithms that perform well, when the instance is in some sense simple; its difficulty is measured by an integer, called the {\emph{parameter}}, additionally appended to the input. Formally, we say that a problem is {\emph{fixed-parameter tractable}} (FPT), if it admits an algorithm that given input of length  and parameter , resolves the task in time , where  is some computable function and  is a constant independent of the parameter.

The search for fixed-parameter algorithms led to the development of a number of new techniques and gave valuable insight into structures of many classes of NP-hard problems. Among them, there is a family of so-called {\emph{graph cut}} problems, where the goal is to delete as few as possible edges or vertices (depending on the variant) in order to make a graph satisfy a global separation requirement. This class is perhaps best represented by the classical \fvsname{} problem (\fvsshort{}) where, given an undirected graph , we seek for a minimum set of vertices
that hits all cycles of . Another examples are \mwc{} (\mwcshort{}: separate each pair from a given set of terminals in a graph
with a minimum cutset) or \oct{} (\octshort{}: make a graph bipartite by a minimum number
of vertex deletions).

The research on the aforementioned problems had a great impact on the development
of parameterized complexity. 
The long line of research concerning parameterized algorithms for \fvsshort{} 
contains 
\cite{fvs:4krand,fvs1,fvs:3.83k,fvs:5k,fvs7,fvs2,fvs3,guo:fvs,fvs6,fvs5},
leading to an algorithm working in  time \cite{fvs:3k}.
The search for a polynomial kernel for \fvsshort{} lead to surprising applications
of deep combinatorial results such as the Gallai's theorem \cite{fvs:quadratic-kernel},
which has also been found useful in designing FPT algorithms \cite{sfvs}.
While investigating the graph cut problems such as \mwcshort{}, M\'arx \cite{marx:cut}
introduced the {\em{important separator}} technique,
which turned out to be very robust, and is now the key ingredient
in parameterized algorithms for various problems such as variants
of \fvsshort{}~\cite{directed-fvs,sfvs} or {\sc{Almost 2-SAT}} \cite{almost2sat-fpt}.
Moreover, the recent developments on \mwcshort{} show applicability of
linear programming in parameterized complexity, leading to the fastest currently known
algorithms not only for \mwcshort{}, but also \asatname{} and \octshort{}
\cite{nmc:2k,saket:lp}. Last but not least, the research on the \octshort{} problem resulted in the introduction of iterative compression, a simple yet powerful technique for designing parameterized algorithms \cite{reed:ic}.

\paragraph{Considered problem.} In this paper we study a robust generalization of the \fvsshort{} problem, namely \gfvs{}\footnote{In this paper, we follow the notation of Guillemot \cite{guillemot-journal}.}.
Let  be a finite (not necessarily abelian) group, with unit element .
We use the multiplicative convention for denoting the group operation.

\begin{definition}
For a finite group , a directed graph  and
a labeling function , we call  
a {\em -labeled} graph iff for each arc  we have 
and .
\end{definition}

We somehow abuse the notation and by  denote
the -labeled graph  with vertices of  removed,
even though formally  has in its domain arcs that do not exist in .

For a path  we denote .
Similarly, for a cycle  we denote .
We call a cycle  a {\em non-null} cycle, iff .
Observe that if the group  is non-abelian, then it may happen that cyclic shifts of the same cycle yield different elements of the group; nevertheless, the notion of a non-null cycle is well-defined, as either all of them are equal to  or none of them.

\begin{lemma}
Let  be a cycle in a -labeled graph .
If , then .
\end{lemma}

\begin{proof}
Let  and . 
We have that  iff  and the lemma follows.
\end{proof}

In the \gfvs problem we want to hit all non-null cycles in a -labeled graph using at most  vertices.

\defproblemu{\gfvs (\gfvsshort{})}{A -labeled graph  and an integer .}{}{Does there exist a set  of at most  vertices, such that there is no non-null cycle in ?}

As observed in~\cite{guillemot-journal}, for a graph excluding a non-null cycle we can define a consistent labeling.

\begin{definition}
\label{def:consistent}
For a -labeled graph  we call  a {\em consistent labeling} iff
for each arc  we have .
\end{definition}

\begin{lemma}[\cite{guillemot-journal}]
\label{lem:consistent}
A -labeled graph  has a consistent labeling iff it does not contain a non-null cycle.
\end{lemma}

Note that when analyzing the complexity of the \gfvsshort{} problem, it is important how the group  is represented.
In~\cite{guillemot-journal} it is assumed that  is given via its multiplication table
as a part of the input.
In this paper we assume a more general model, where operations in  are computed
by an oracle in polynomial time.
More precisely, we assume that the oracle can multiply two elements, return an inverse of an element, provide the neutral element , or check whether two elements are equal.

As noted in \cite{stefan:new}, \gfvsshort{} subsumes not only the classical \fvsshort{} problem,
but also \octshort{} (with ) and \mwcshort{} (with  being an arbitrary group of size not smaller than the number of terminals).
We note that if  is given in the oracle model, \gfvs subsumes also \esfvs, which is equivalent to \sfvs~\cite{sfvs}.

\defproblemu{\esfvs (\esfvsshort{})}{An undirected graph , a set  and an integer .}{}{Does there exist a set  of at most  vertices,
  such that in  there are no cycles with at least one edge from ?}

\begin{lemma}\label{lem:sfvs}
Given an \esfvsshort{} instance , one can in polynomial time construct an equivalent \gfvsshort{} instance  with group .
\end{lemma}
\begin{proof}
To construct the new \gfvsshort instance, create the graph  by replacing each edge of  with arcs in both direction,
 keep the parameter , take  and construct a -labeling  by setting
 any  linearly independent values of  for  and  for .
Clearly, this construction can be done in polynomial time and the operations on the group  can be performed
by a polynomial-time oracle.
\end{proof}

We note that the \gfvs problem was also studied from the graph theoretical point of view, as, in addition to the aforementioned
reductions, it also subsumes the setting of Mader's -paths theorem \cite{maria:gfvs,kebab:gfvs}.
In particular, Kawarabayashi and Wollan proved the Erd\"{o}s-P\'{o}sa property for non-null cycles in highly connected graphs,
generalizing a list of previous results \cite{kebab:gfvs}.

The study of parameterized complexity of \gfvsshort{} was initiated by Guillemot \cite{guillemot-journal}, who presented
a fixed-parameter algorithm for \gfvsshort{} parameterized by 
running in time\footnote{The  notation suppresses terms polynomial in the input size.}
.
When parameterized by , Guillemot showed a fixed-parameter algorithm for
the easier edge-deletion variant of \gfvsshort{}, running in time .
Very recently, Kratsch and Wahlstr\"{o}m presented a randomized kernelization algorithm
that reduces the size of a \gfvsshort{} instance to  \cite{stefan:new}.

The main purpose of studying the \gfvsshort{} problem is to find the common points in the fixed-parameter algorithms for problems it generalizes. Precisely this approach has been presented by Guillemot in \cite{guillemot-journal}, where at the base of the algorithm lies a subroutine that solves a very general version of \mwc{}. When reducing various graph cut problems to \gfvsshort{}, usually the size of the group depends on the number of distinguished vertices in the instance, as in Lemma~\ref{lem:sfvs}. Hence, the usage of the general  algorithm of Guillemot unfortunately incorporates this parameter in the running time. It appears that by a more refined combinatorial analysis, usually one can get rid of this dependence; this is the case both in \sfvs~\cite{sfvs} and in \mwc~\cite{nmc:2k,saket:lp}. This suggests that the phenomenon can be, in fact, more general.

\paragraph{Our result and techniques.}
Our main result is a fixed-parameter algorithm for \gfvsshort{} parameterized by the
size of the cutset only.
\begin{theorem}
\label{thm:main}
\gfvs can be solved in  time and polynomial space.
\end{theorem}
Our algorithm uses a similar approach as described by Kratsch and Wahlstr\"{o}m
in \cite{stefan:new}: in each step of iterative compression, when we are given
a solution  of size , we guess the values of a consistent labeling on the
vertices of , and reduce the problem to \mwc{}. However, by a straightforward
application of this approach we obtain  time complexity.
To reduce the dependency on , we carefully analyze the structure of a solution,
provide a few reduction rules in a spirit of the ones used in the recent algorithm
for \sfvs{} \cite{sfvs} and, finally, for each vertex of 
we reduce the number of choices for a value of a consistent labeling to polynomial in .
Therefore, the number of reasonable consistent labelings of  is bounded
by  and we can afford solving a \mwc{} instance for each such labeling. 

Note that the bound on the running time of our algorithm matches the currently best known algorithm for \sfvs{} \cite{sfvs}. Therefore, we obtain the same running time as in \cite{sfvs} by applying a much more general framework.

In the \gfvs problem definition in~\cite{guillemot-journal}
a set of forbidden vertices 
is additionally given as a part of the input.
Observe that one can easily gadget such vertices by replacing each forbidden vertex by a clique
of size  labeled with ; therefore, for the sake of simplicity we assume
that all the vertices are allowed.

\section{Preliminaries}

\paragraph{Notation.} We use standard graph notation.
For a graph~, by~ and~ we denote its vertex and edge sets, respectively.
In case of a directed graph , we denote the arc set of  by .
For~, its neighborhood~ is defined as~, and~ is the closed neighborhood of~.
We extend this notation to subsets of vertices:~ and~.
For a set~ by~ we denote the subgraph of~ induced by~.
For a set~ of vertices or edges of~, by~ we denote the graph
with the vertices or edges of~ removed; in case of vertex removal, we remove
also all the incident edges.





\section{Algorithm}

In this section we prove Theorem \ref{thm:main}.
We proceed with a standard application of the iterative compression technique
in Section \ref{sec:alg:1}.
In each step of the iterative compression, we solve a \compgfvs{} problem, where
we are given a solution  of size a bit too large ---  --- and we are to find a new
solution disjoint with it.
We first prepare the \compgfvs{} instance by {\em{untangling}} it in Section \ref{sec:alg:2},
  in the same manner as it is done in the kernelization algorithm of \cite{stefan:new}.
The main step of the algorithm is done in Section \ref{sec:alg:3},
 where we provide a set of reduction rules that enable us for each vertex
 to limit the number of choices for a value of a consistent labeling on 
to polynomial in . Finally, we iterate over all 
remaining labelings of  and, for each labeling, reduce the instance to
\mwc{} (Section \ref{sec:alg:4}).


\subsection{Iterative compression}
\label{sec:alg:1}

The first step in the proof of Theorem~\ref{thm:main} is a standard technique in the design
of parameterized algorithms, that is, iterative compression, introduced by Reed et al.~\cite{reed:ic}.
Iterative compassion was also the first step of the parameterized algorithm for \sfvs~\cite{sfvs}.

We define a {\em compression problem}, where the input additionally contains a feasible 
solution , and we are asked whether there exists a solution of size at most 
which is disjoint with . 

\defproblemugoal{\compgfvs (\cgfvsshort{})}{A -labeled graph , an integer  and a set , such that  has no non-null cycle.}{}{Find a set  of at most  vertices, such that there is no non-null cycle in  or return NO, if such a set does not exist.}

In Section~\ref{sec:alg:2} we prove the following lemma providing a parameterized algorithm for \compgfvs.

\begin{lemma}
\label{lem:compression}
\compgfvs can be solved in  time and polynomial space.
\end{lemma}

Armed with the aforementioned result, we can easily prove Theorem~\ref{thm:main}.

\begin{proof}[Proof of Theorem~\ref{thm:main}]
In the iterative compression approach we start with an empty solution for an empty graph, and in each of the  steps we add a single vertex both to a feasible solution and to the graph; we use Lemma~\ref{lem:compression} to compress the feasible solution after guessing which vertices of the solution of size at most  should not be removed.

Formally, for a given instance  let .
For  define  (in particular )
and let  be the function  restricted to the set
of arcs .
Initially we set , which is a solution to the graph .
For each  we set , which is a feasible solution
to  of size at most .
If , then we set  and continue the inductive process.
Otherwise, if , we guess by trying all possibilities, a subset of vertices
 that is not removed in a solution of size  to 
and use Lemma~\ref{lem:compression} for the instance .
If for each set  the algorithm from Lemma~\ref{lem:compression} returns NO, 
then there is no solution for  and, consequently,
there is no solution for .
However, if for some  the algorithm from Lemma~\ref{lem:compression}
returns a set  of size smaller than , then we set .
Since ,
the set  is a solution of size at most  for the instance .

Finally, we observe that since , the set 
is a solution for the initial instance  of \gfvs. The claimed bound on running time follows from the observation that  for each of polynomially many steps.
\end{proof}

At this point a reader might wonder why we do not add an assumption  to the \cgfvsshort problem
definition and parameterize the problem solely by . 
The reason for this is that in Section~\ref{sec:alg:3} we will solve the \cgfvsshort problem
recursively, sometimes decreasing the value of  without decreasing the size of ,
and to always work with a feasible instance of the \cgfvsshort problem we avoid adding the 
assumption to the problem definition.

\subsection{Untangling}
\label{sec:alg:2}

In order to prove Lemma~\ref{lem:compression} we use the concept of {\em untangling},
previously used by Kratsch and Wahlstr\"om~\cite{stefan:new}.
We transform an instance of \cgfvsshort to ensure that each arc 
with both endpoints in  is labeled  by .

\begin{definition}
We call an instance  of \cgfvsshort\ {\em untangled}, iff for each arc  such that 
we have .

Moreover, by {\em untangling} a labeling  around vertex  with a group element  we mean changing the labeling to , such that for , we have

\end{definition}

\begin{lemma}
\label{lem:relabel}
Let  be a -labeled graph,  be a vertex of  and let  be a group element.
For any subset of vertices  the graph  contains a non-null cycle iff  contains a non-null cycle, where  is
the labeling  untangled around the vertex  with a group element~.
\end{lemma}

\begin{proof}
The lemma follows from the fact that for any cycle  in  we have .
\end{proof}

In Section~\ref{sec:alg:3} we prove the following lemma.

\begin{lemma}
\label{lem:untangled}
\compgfvs for untangled instances can be solved in  time and polynomial space.
\end{lemma}

Having Lemmata~\ref{lem:relabel} and \ref{lem:untangled} we can prove Lemma~\ref{lem:compression}.

\begin{proof}[Proof of Lemma~\ref{lem:compression}]
Let  be an instance of \cgfvsshort.
Since  has no non-null cycle, by Lemma~\ref{lem:consistent}
there is a consistent labeling  of .

Let  be a result of untangling  around each
vertex  with . 
Note that, by associativity of , the order in which we untangle subsequent vertices does not matter.
After all the untangling operations, for an arc ,
such that , we have .
Therefore, by Lemma~\ref{lem:relabel} the instance 
is an untangled instance of \cgfvsshort, which is a YES-instance
iff  is a YES-instance.
Consequently, we can use Lemma~\ref{lem:untangled} and the claim follows.
\end{proof}

\subsection{Fixing a labeling on }
\label{sec:alg:3}

In this section we prove Lemma~\ref{lem:untangled} using the following lemma, which we prove in Section~\ref{sec:alg:4}.

\begin{lemma}
\label{lem:fixed}
Let  be an untangled instances of \cgfvsshort.
There is an algorithm which for a given
function ,
finds a set  of size at most ,
such that there exists a consistent labeling 
 of ,
where ,
or checks that such a set  does not exist; the algorithm works in  time and uses polynomial space.
\end{lemma}

We could try all 
possible assignments  and use the algorithm from Lemma~\ref{lem:fixed}.
Unfortunately, since  is not our parameter we cannot 
iterate over all such assignments. Therefore, the goal of
this section is to show that after some preprocessing,
it is enough to consider only  assignments ; together with Lemma~\ref{lem:fixed} this suffices to prove Lemma~\ref{lem:untangled}.

\begin{definition}
Let  be an untangled instance of \cgfvsshort,
let  be a vertex in  and by  denote 
the set .

By a flow graph , we denote the undirected
graph , where 
and .
\end{definition}

Less formally, in the flow graph we take the underlying undirected graph of 
and add a vertex for each group element , 
that is a group element for which there exists an arc from  to  
labeled with  by .
A vertex  is adjacent to all the vertices of 
for which there exists an arc going from , labeled with  by .


\begin{lemma}
\label{red:1}
Let  be an untangled instance of \cgfvsshort.
Let  be the flow graph  for some .
If for some vertex , 
in  there are at least  paths from  to  that are vertex disjoint apart from ,
then  belongs to every solution of \cgfvsshort.
\end{lemma}

\begin{proof}
Let us assume, that  is not a part of a solution ,
where .
Then there at least  out of the  paths from  to  remain in .
These two paths are vertex disjoint apart from , so they correspond to a non-null cycle in , a contradiction.
\end{proof}

\begin{definition}
For an untangled instance  of \cgfvsshort by an {\em external path}
we denote any path  beginning and ending in , but with all internal vertices belonging to .
Moreover, for two distinct vertices  by  we denote the set of all elements ,
for which there exists an external path  from  to  with .
\end{definition}

\begin{lemma}
\label{lem:many_paths}
Let  be an untangled instance of \cgfvsshort.
If for each  and 
there are at most  vertex disjoint paths from 
to  in 
and for some , , we have , 
then there is no solution for .
\end{lemma}

\begin{proof}
Let us assume that  is a solution for .
Let  be a set of external paths from  to ,
containing exactly one path  for each  with .
Note that the only arcs with non-null labels in  are possibly the first and the last arc.

By the pigeon-hole principle, there exists a vertex ,
which belongs to at least  paths in ,
since otherwise there would be at least two paths in 
disjoint with , creating a non-null cycle disjoint with . This cycle is not necessarily simple; however, if it is non-null, then it contains a simple non-null subcycle that is also disjoint with .

Consider a connected component  of  to which  belongs.
Observe that there exists a vertex 
that has at least  incident arcs
going to  with pairwise different labels in ,
since otherwise  would belong to at most  paths in .

Let  be the flow graph 
and let  be the set of labels of arcs going from  to ; recall that .
Since there is no non-null cycle in , 
we infer that in , 
no two vertices of  belong to the same connected component.
Moreover, as  is connected in , for each  there exists a path  with endpoints  and  in .
Let  be the closest to  vertex from  on the path . As  and ,
there exists  such that  for at least  elements .
By the definition of the vertices  and the fact that there are no two vertices of  in the same connected component of ,
the subpaths of  from  to  for all  with  are vertex disjoint apart from .
As there are at least  of them, we have a contradiction.
\end{proof}

We are now ready to prove Lemma~\ref{lem:untangled} given Lemma \ref{lem:fixed}.

\begin{proof}[Proof of Lemma~\ref{lem:untangled}]
If there exists a vertex , satisfying the properties of Lemma~\ref{red:1},
we can assume that it has to be a part of the solution; therefore,
we can remove the vertex from the graph and solve the problem
for decremented parameter value.
Hence, we assume that for each  and ,
there are at most  vertex disjoint paths from  to  in .
We note that one can compute the number of such vertex disjoint paths in polynomial time, using 
a maximum flow algorithm.

By Lemma~\ref{lem:many_paths}, if there is a pair of vertices  with , 
we know that there is no solution.
Observe, that one can easily verify the cardinality of , since the only
non-null label arcs on paths contributing to  are the first and the last one,
and we can iterate over all such arcs and check whether their endpoints are in the same connected component in .
Clearly, this can be done in polynomial time.

Knowing that the sets  have sizes bounded by a function of ,
we can enumerate all the reasonable labelings of .
For the sake of analysis let  be an auxiliary undirected graph,
where two vertices of  are adjacent, when they are connected by an external path
in , for some fixed solution .
Let  be any spanning forest of .
Since  has at most  edges, we can guess ,
by trying at most  possibilities.
Let us assume, that we have guessed  correctly.
Observe that for any two vertices ,
belonging to two different connected components of ,
there is no path between  and  in .
Therefore, there exists a consistent labeling of ,
which labels an arbitrary fixed vertex from each connected component of  with .
For all other vertices of  we use the fact that if we have already fixed a value ,
then for each external path corresponding to an edge  of , there are at most  possible 
values of , since .
Hence, we can exhaustively try  labelings  of ,
and use Lemma~\ref{lem:fixed} for each of them.
\end{proof}

\subsection{Reduction to Multiway Cut}
\label{sec:alg:4}

In this section, we prove Lemma~\ref{lem:fixed},
by a reduction to \mwc.
A similar reduction was also used recently by Kratsch and Wahlstr\"om
in the kernelization algorithm for \gfvs parameterized by  with constant ~\cite{stefan:new}.
Currently the fastest FPT algorithm for \mwc is due to Cygan et al.~\cite{nmc:2k},
and it solves the problem in  time and polynomial space.

\defproblemugoal{\mwc}{An undirected graph , a set of terminals , and a positive integer .}{}
{Find a set ,
  such that  and no pair of terminals from the set  is contained in one connected component of the graph ,
  or return NO if such a set  does not exist.}

\begin{proof}[Proof of Lemma~\ref{lem:fixed}]
Firstly, we check whether the given function  satisfies ,
for each arc , since otherwise there is no set  we are looking for.

Given a -labeled graph ,
a set , an integer , and a function ,
we create an undirected graph .
As the vertex set, we set  and .
Note that in the set  there exactly these elements of ,
which are potential values of a consistent labeling
of  that matches  on .
As the edge set, we set .
We show that  is a YES-instance of \mwc iff 
there exists a set , such that there exists a consistent labeling  of  with .

Let  be solution for . We define a consistent labeling  of .
For  we set . 
For , if  is reachable from a terminal  in , we set .
If  is not reachable from any terminal in , we set .
Since each arc in  is labeled  by , 
and each vertex in  is reachable from at most one terminal in ,
 is a consistent labeling of .

Let  be a set of vertices of , ,
such that there is a consistent labeling  of ,
where .
By the definition of edges between  and  in ,
each vertex of  is reachable from at most one terminal in ,
since otherwise  would not be a consistent labeling of .
Therefore,  is a solution for .

We can now apply the algorithm for \mwc of~\cite{nmc:2k} to the instance  in order to conclude the proof.
\end{proof}

\section{Conclusions and open problems}

We have shown a relatively simple fixed-parameter algorithm
for \gfvs{} running in time . Our algorithm
works even in a robust oracle model, that allows us to generalize
the recent algorithm for \sfvs{} \cite{sfvs} within the same complexity bound.

We would like to note that if we represent group elements
by strings consisting  and  for  (formally, we perform the computations in the free group over generators corresponding to the arcs of the graph), then after slight modifications
of our algorithm we can solve the \gfvs problem
even for infinite groups for which the word problem, i.e., the problem of checking whether results of two sequences of multiplications are equal, is polynomial-time solvable. The lengths of representations of group elements created during the computation can be bounded linearly in the size of the input graph. Therefore, if a group admits a polynomial-time algorithm solving the word problem, then we can use this algorithm as the oracle.

Both our algorithm and the algorithm for \sfvs{} of \cite{sfvs} seems
hard to speed up to time complexity .
Can these problems be solved in  time, or can we prove that such a result
would violate Exponential Time Hypothesis?

\vskip 0.5cm

\noindent{\bf{Acknowledgements.}} We thank Stefan Kratsch and Magnus Wahlstr\"om for inspiring discussions on graph separation problems and for drawing
our attention to the \gfvs{} problem.

\bibliographystyle{plain}
\bibliography{group-fvs}

\newpage

\end{document}
