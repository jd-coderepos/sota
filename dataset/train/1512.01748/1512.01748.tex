\subsection{Another ADMM}\label{sec:otherorder}
\subsubsection{Algorithm design}
In this part, we provide another ADMM algorithm, which considers different constraints compared with the algorithm we previously proposed, \textit{i.e.}, in X \textbf{update} we consider the rank constraint while in Y \textbf{update} we consider the convex constraint.

We denote ,  and the augmented function with dual variable  as

Following the ADMM procedure, we will have the following three updates in each iteration.
\begin{itemize}
\item  \textbf{update}:

\item  \textbf{update}:

\item  \textbf{update}:

\end{itemize}

\paragraph{The Subproblems}
By similar arguments, the optimization problem in  update is equivalent to
\label{prob:ADMM}
\min_{X}&& \|X-\frac{1}{1+\frac{\rho}{2}}\left(\hat{X}+\frac{\rho}{2}(Y^k-U^k)\right)\|_F^2
\nonumber\\
\textrm{s.t.}
& & \text{rank}(X)\leq K.\nonumber

which is to find a low-rank matrix to approximate  and can be solved by truncated SVD.

The optimization problem in  update is equivalent to
\label{prob:ADMM_Y}
 \min_{\mathbf{Y}}&& \frac{\rho}{2}\|\mathbf{X}^{k+1}-\mathbf{Y}+\mathbf{U}^k\|_F^2
\nonumber\\
\textrm{s.t.}
& & g(\mathbf{Y})\leq 0.\nonumber


The problem is convex and it can be viewed as a projection of  onto the convex set . It can be solved efficiently or even have closed-form solutions.

\subsubsection{Convergence analysis}
The theoretical analysis is also based on the assumption that the dual variable will converge. With the same argument in Section~\ref{sec:convergence}, the value of primal variable is iteratively updated by the function , where
 \textit{i.e.},  An interpretation of the update is provide in Fig~\ref{fig:X_update}.
\begin{figure}
  \centering
\includegraphics[width=0.5\columnwidth]{X_update.eps}\\
  \caption{An illustration of  update}\label{fig:X_update}
\end{figure}



If the feasible approximation region is convex, the update rule will converge with , but unfortunately, the set  is non-convex and the update is not guaranteed to converge. In the following, we provide some simple necessary conditions for the convergence of the primal variable.
\iffalse
\paragaph{A \textit{trivial} Sufficient Condition}

A trivial sufficient condition for the convergence of  is given in Proposition~\ref{proposition:contractmapping}

\begin{proposition}\label{proposition:contractmapping}
If the function  is a contractive mapping, then it has a fixed point  and the  update will converge to .
\end{proposition}

This proposition can be proved by the Banach fixed point theorem. However, this sufficient condition is not so useful because it is difficult to guarantee or even check whether  is contractive or not. Moreover, the author believes that this condition is hard to satisfy because this function is inherently doing projection onto a non-convex set, which is non-contractive \footnote{On the other hand, if we are doing projection onto a convex set, we will be sure that the function is strictly contractive and the update will converge.}.

\subsubsection{A \textit{useful} Necessary Condition}
\fi
Firstly, if  converges to , then  is a fixed point of  \footnote{This result depends on the continuity of  at , which requires a formal proof. We conjecture that  is continuous at the points where it is well-defined.}, as shown in Proposition~\ref{proposition:continuity}.

\begin{proposition}\label{proposition:continuity}.
For any sequence  converging to  we will have

\end{proposition}



We can see that if  converges to , then  is a fixed point of . We define a set  as 
and provide a necessary condition for the convergence of  in Lemma~\ref{lemma:necessarycondition}.

\begin{lemma}\label{lemma:necessarycondition}
Under the condition that the dual variable  converges to , if  converges to , then .
\end{lemma}


Lemma~\ref{lemma:necessarycondition} is useful in the sense that, with the knowledge of , we can restrict our attention to . On one hand, if we observe that the dual variable converges, we do not need to keep updating since the only possible limiting points are in ; on the other hand, if we find  is empty, then we can say that the primal variable will not converge and there is no need to spend more effort on updating.


\paragraph{the Fixed Points of }

It turns out that the fixed points of  are highly related to the singular value decomposition of , which we will show in this section.

To warm up, we firstly provide one fixed point in Corollary~\ref{corrollary:one_fixed_point}.
\begin{corollary}\label{corrollary:one_fixed_point}
If  is the optimal solution of  by truncated SVD, then .
\end{corollary}

\begin{proof}
Let , we can obtain that the optimal solution is . Then,

which is the singular value decomposition of . Then the value of  is .

The proof is completed.
\end{proof}

We note that even though  is a fixed point of , it may not be a stationary point of  update because  may not satisfy .


Next, we characterize all possible fixed points of  in Corollary~\ref{corollary:all_fixed_points}.

\begin{corollary}\label{corollary:all_fixed_points}
Suppose  admits the singular value decomposition of  with , and  is a subset of  with size . If for any  and  we can have , then  is a fixed point of . On the other hand, if  is a fixed point, then there must exists such a subset , such that .
\end{corollary}
\begin{proof}
Suppose  is a fixed point of  and  admits the singular value decomposition of . Then .
By substituting into  we can have

Thus 
However, to have the standard SVD of , we need to permutate the diagonal elements of  to make them decreasing and the corresponding columns of .

However, the diagonal elements of  are decreasing, which establishes the results in this corollary.
\end{proof}

We provide some remarks regarding this corollary.
\begin{itemize}
\item By Corollary~\ref{corollary:all_fixed_points}, it is easy to see that Corollary~\ref{corrollary:one_fixed_point} is true.
\item The number of fixed points is upper bounded by  because we need to choose  singular values to form one fixed point of .
\item A larger value of  (meaning a larger ) will make the condition  more difficult to satisfy and the possible fixed points will be fewer.
\end{itemize}











