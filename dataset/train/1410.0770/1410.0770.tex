\documentclass[]{elsarticle}
\usepackage{amsmath,amssymb,amsthm}
\usepackage{graphicx}
\usepackage{a4wide}
\usepackage[ruled, vlined, linesnumbered]{algorithm2e}

\usepackage{microtype}

\newcommand{\NP}{\textsf{NP}}

\newtheorem{theorem}{Theorem}[]
\newtheorem{lemma}[theorem]{Lemma}
\newtheorem{corollary}[theorem]{Corollary}
\newtheorem{proposition}[theorem]{Proposition}
\newtheorem{observation}[theorem]{Observation}

\newtheorem{claim}{Claim}
\newtheorem{definition}{Definition}
\def\R{{\mathbb{R}}}
\def\Q{{\mathbb{Q}}}
\def\Z{{\mathbb{Z}}}
\def\N{{\mathbb{N}}}
\def\mwis{{\sc{MWIS }}}
\def\minwed{{\sc{Min-WED }}}
\def\maxwed{{\sc{Max-WED }}}
\def\kcg{{\sc{KCG }}}

\begin{document}
\begin{frontmatter}
\title{Polynomial-time Algorithms for Weighted Efficient Domination Problems in AT-free Graphs and Dually Chordal Graphs}

\author[ROSTOCK]{Andreas Brandst\"adt}
\ead{ab@informatik.uni-rostock.de}

\author[IAM,FAMNIT]{Pavel Fi\v cur}
\ead{pavel.ficur@upr.si}

\author[KENT]{Arne Leitert}
\ead{aleitert@cs.kent.edu}

\author[IAM,FAMNIT]{Martin Milani\v c\corref{cor}}
\ead{martin.milanic@upr.si}

\address[ROSTOCK]{Institut f\"ur Informatik, Universit\"at Rostock, Germany}
\address[IAM]{University of Primorska, UP IAM, Muzejski trg 2, SI6000 Koper, Slovenia}
\address[FAMNIT]{University of Primorska, UP FAMNIT, Glagolja\v ska 8, SI6000 Koper, Slovenia}
\address[KENT]{Kent State University, Department of Computer Science, Kent, Ohio 44242, USA}

\cortext[cor]{Corresponding author}

\begin{abstract}
An efficient dominating set (or perfect code) in a graph is a set of vertices the closed neighborhoods of which partition the vertex set of the graph.
The minimum weight efficient domination problem is the problem of finding an efficient dominating set of minimum weight in a given vertex-weighted graph;
the maximum weight efficient domination problem is defined similarly.
We develop a framework for solving the weighted efficient domination problems based on a reduction to the maximum weight independent set problem in the square of the input graph. 
Using this approach, we improve on several previous results from the literature by deriving polynomial-time algorithms for the weighted efficient domination problems in the classes of dually chordal and AT-free graphs. 
In particular, this answers a question by Lu and Tang regarding the complexity of the minimum weight efficient domination problem in strongly chordal graphs.
\end{abstract}

\begin{keyword}
perfect code \sep efficient domination \sep weighted efficient domination\sep maximum weight independent set problem
\sep dually chordal graph \sep cocomparability graph \sep AT-free graph  \sep polynomial-time algorithm
\end{keyword}
\end{frontmatter}

\noindent
\textbf{Notice:} This is the author's version of a work that was accepted for publication in Information Processing Letters. Changes resulting from the publishing process, such as peer review, editing, corrections, structural formatting, and other quality control mechanisms may not be reflected in this document. Changes may have been made to this work since it was submitted for publication. A definitive version was subsequently published in Information Processing Letters 115 (2015) 256-262, DOI: 10.1016/j.ipl.2014.09.024.

\section{Introduction}

The concept of an efficient dominating set in a graph was introduced by Biggs~\cite{Biggs} as a generalization of the notion of a perfect error-correcting code in coding theory. Given a (simple, finite, undirected) graph , we say that a vertex \emph{dominates} itself and each of its neighbors. 
An \emph{efficient dominating set} in  is a subset of vertices  such that every vertex  is dominated by precisely one vertex from . 
Efficient domination has several interesting applications in coding theory and resource allocation of parallel processing systems~\cite{Biggs, LLT97, LS90}. The notion of an efficient dominating set appeared in the literature under various other names such as: \emph{perfect code}, \emph{-perfect code}, \emph{independent perfect dominating set}, and \emph{perfect dominating set}. 
Note, however, that the name \emph{perfect dominating set} has also been used in the literature to denote a subset of vertices  such that every vertex  is dominated by precisely one vertex from . 
See~\cite{LT02} for a nice historical overview of the notion of efficient dominating set.

A graph is \emph{efficiently dominatable} if it contains an efficient dominating set.
All paths are efficiently dominatable, and a cycle  on  vertices is efficiently dominatable if and only if  is a multiple of .
Bange \emph{et al.}~\cite{BBS88} showed that if a graph  has an efficient dominating set, then all efficient dominating sets of  have the same cardinality, which equals the minimum cardinality of a dominating set of . 
The efficient domination (ED) problem consists in determining whether the input graph is efficiently dominatable. 
The ED problem is \NP-complete even for restricted graph classes such as planar cubic graphs~\cite{Kratochvil91}, bipartite graphs~\cite{YL96}, planar bipartite graphs~\cite{LT02}, chordal bipartite graphs~\cite{LT02}, chordal graphs~\cite{YL96}, and line graphs of planar bipartite graphs of maximum degree three~\cite{BHN}.
On the other hand, the ED problem is polynomial for several graph classes, including trees~\cite{BBS88,FH91}, block graphs~\cite{YL96}, interval graphs~\cite{CL94, CRC95, KE00, KMM95}, circular-arc graphs~\cite{CL94,KE00}, cocomparability graphs~\cite{C97,CRC95}, bipartite permutation graphs~\cite{LT02}, permutation graphs~\cite{LLT97}, distance-hereditary graphs~\cite{LT02}, trapezoid graphs~\cite{LLT97,L98}, split graphs~\cite{CL93}, dually chordal graphs~\cite{BLR}, AT-free graphs~\cite{BLR,BKM99}, and hereditary efficiently dominatable graphs~\cite{Milanic}.
The efficient domination problem has also been studied from a parameterized point of view, see, e.g.,~\cite{Cesati,Guo-Niedermeier}.

In this paper, we consider two weighted versions of the ED problem.
In its decision form, the minimization version of problems can be stated as follows:

\medskip
\begin{center}
\fbox{\parbox{0.85\linewidth}{\noindent
{\sc Minimum Weight Efficient Dominating Set (Min-WED)}\.8ex]
\begin{tabular*}{.93\textwidth}{rl}
\emph{Input:} & A tuple  consisting of a graph , \\ & vertex weights , and an integer .\\
\emph{Question:} & Does  contain an independent set  such that ?
\end{tabular*}
}}
\end{center}
\medskip

This problem, together with its unweighted version, is a classical \NP-complete problem, which remains hard even for several restricted graph classes such as triangle-free graphs~\cite{Pol74}, -free graphs \cite{Min80} and planar graphs of maximum degree at most three~\cite{GJ77}.
On the other hand, the MWIS problem has been shown to admit polynomial solutions in several graph classes, such as claw-free graphs~\cite{Min80,NT01,OPS08}, and their generalization fork-free graphs~\cite{LM08}, perfect graphs~\cite{GLS84}, and AT-free graphs~\cite{BKM99}.

The following simple but useful observation from~\cite{BLR} establishes a connection between efficient dominating sets in a graph  and independent sets in its square.

\begin{observation}[\cite{BLR}]\label{lem:eds-wis}
Let  be a graph, and let  be a vertex weight function for  defined by  for all . 
Then, the following statements are equivalent for every :
\begin{enumerate}
  \item[(i)]  is an efficient dominating set in .
  \item[(ii)]  is a maximum-weight independent set in  such that .
\end{enumerate}
\end{observation}

The above observation (as well as a slightly more general observation from~\cite{Milanic}) has an immediate algorithmic corollary, reducing the ED problem to the MWIS problem in the square of the input graph. 
In~\cite{BLR,Milanic}, the exact time complexity of such a reduction was not specified; we do this in the following proposition.
Given a graph , we denote by  its encoding length.

\begin{proposition}[\cite{BLR,Milanic}]\label{prop:1}
Let  be a graph class for which the \mwis problem is solvable in time  on squares of graphs from .
Then, the ED problem for  is solvable in time \hbox{}.
\end{proposition}

\begin{proof}
It is easy to see that the square  can be computed in time  using matrix multiplication. 
Alternatively, assuming that  is given by adjacency lists, computing  can be done in time , as follows. 
First, we fix an ordering  of  and order the adjacency lists with respect to .
This can be done in linear time, see, e.g.,~\cite[p.~]{Gol04}.
Then, we create a new copy of these lists. For every vertex , we process its neighbors in order and for each of them, say , we merge the current adjacency list of  with the adjacency list of  (removing duplicates).
Since the lists are ordered, each such merging step can be done in  time.
Hence, the overall time complexity of computing  is

where the last equality holds since .
This justifies the time complexities given in Proposition~\ref{prop:1}.
\end{proof}

We extend the approach of Brandst\"adt {\it et al.}~\cite{BLR} and Milani\v c~\cite{Milanic} to the weighted versions of the ED problem based on a suggestion made in~\cite{Leitert}.
We have the following

\begin{theorem}\label{thm:wed-mwis}
Let  be a graph class for which the \mwis problem is solvable in time  on squares of graphs from .
Then, the \minwed and \maxwed problems are solvable on graphs in  in time .
\end{theorem}


The proof of this theorem will rely on two auxiliary lemmas.
The first one shows how the problem can be reduced to the case of non-negative weights.

\begin{lemma}\label{lem:wed-mwis}
Let  be an instance to the \minwed problem such that .
Suppose that  has an efficient dominating set, and let  be the common cardinality of all efficient dominating sets of .
For all , let , and let .
Then,   is a yes instance to the \minwed problem if and only if  is a yes instance to the \minwed problem.
In addition, .
\end{lemma}

\begin{proof}
Recall that if a graph  has an efficient dominating set, then all efficient dominating sets of  have the same cardinality, which equals the minimum cardinality of a dominating set of . 
This implies that if we replace each  with , then the weight of each efficient dominating set will increase by exactly , where  is the common cardinality of all efficient dominating sets of .
Consequently,  is a yes instance if and only if  is a yes instance.
The fact that  is clear.
\end{proof}

The second lemma deals with the case of non-negative weights.

\begin{lemma}\label{lem:wed-mwis-2}
Let  be an instance to the \minwed problem such that .
Let , let , for all , and .
Then,   is a yes instance to the \minwed problem if and only if  is a yes instance to the \mwis problem.
\end{lemma}

\begin{proof}
On the one hand, if  is an efficient dominating set in  such that , then, by Observation~\ref{lem:eds-wis},  is an independent set in  such that .
Therefore, 

and  is a yes instance to the \mwis problem.

On the other hand, let  be an independent set in  with .
We claim that  and .
Since  is an independent set in , it is an independent set in  such that the closed neighborhoods (in ) of its elements are pairwise disjoint. 
Therefore, 
Conversely,  implies  since .
We thus have .
By Observation~\ref{lem:eds-wis},  is an efficient dominating set in .
The inequality  is equivalent to
 
which implies .
Thus,  is a yes instance to the \minwed problem, as claimed.
\end{proof}

\begin{proof}[Proof of Theorem~\ref{thm:wed-mwis}]
Consider an instance  with  and  to the \minwed problem.
Let .
If , we transform the instance, using Lemma~\ref{lem:wed-mwis} to an equivalent instance  to the \minwed problem with non-negative weights and .
The time complexity of this step is dominated by solving the ED problem in , which, by Proposition~\ref{prop:1}, is of the order \hbox{}.

To solve the \minwed problem on , we apply Lemma~\ref{lem:wed-mwis-2}. 
The corresponding instance  to the \mwis problem can be computed in time .

Summarizing, in order to solve the \minwed problem, we only need to compute  and solve at most two instances of the \mwis problem on .

The \maxwed problem can be reduced in time  to the \minwed problem, since a tuple  is a yes instance to the \maxwed problem if and only if  is a yes instance to the \minwed problem.
Thus the result follows.
\end{proof}


\section{New Polynomial Cases of the Weighted Efficient Domination Problems}

In this section, we exploit the approach developed in Section~\ref{sec:wed-mwis} and develop new polynomial results for the \minwed and \maxwed problems.


\subsection{Dually chordal graphs}\label{sec:dually-chordal}


In~\cite{LT02}, Lu and Tang wrote that {\it `` it would be of interest to know whether or not there is a polynomial-time algorithm to solve the weighted efficient domination problem on} () {\it strongly chordal graphs.''}
Recently, Brandst\"adt \emph{et al.}~\cite{BLR} gave a linear-time algorithm for the efficient domination problem in the class of dually chordal graphs.
The algorithm is a modification of Frank's algorithm (Algorithm~\ref{algo:mwisChordal} below), which solves the \mwis problem for chordal graphs in linear time~\cite{Frank}. 
The modification allows to find a maximum weight independent set of~ in linear time if  is dually chordal, without computing its square.

\begin{algorithm}
\caption{\cite{Frank} Algorithm to find a maximum weight independent set in chordal graphs.}\label{algo:mwisChordal}

\KwIn{A chordal graph  with  and a vertex weight function~.}
\KwOut{A maximum weight independent set  of .}

Find a perfect elimination ordering  of  and set . \label{line:mwisChorFindPEO}

\For{ \KwTo }
{
    If , mark  and set  for all vertices .
}

\For{}
{
    If  is marked, set  and unmark all .
}

\Return{}
\end{algorithm}


We will make a similar modification of Frank's algorithm so that the obtained algorithm (Algorithm~\ref{algo:EDdc} below) solves the \minwed problem for dually chordal graphs. 
The modification contains two major changes. 
First, we add a preprocessing step, transforming the given \minwed instance (with possibly negative weights) to an instance of the \mwis problem. 
Second, we rewrite Algorithm~\ref{algo:mwisChordal} to find a maximum weight independent set of the square of the given graph. 
By Lemma~\ref{lem:wed-mwis-2}, this will be a solution to the \minwed problem.

For the first step, we use Lemma~\ref{lem:wed-mwis}. 
For the second, we need the two following lemmas.

\begin{lemma}[\cite{BCD98}]\label{lem:MNO_linear}
    A maximum neighborhood ordering of  which simultaneously is a perfect elimination ordering of  can be found in linear time.
\end{lemma}

Additionally to finding a maximum neighborhood ordering , the algorithm in \cite{BCD98} also computes a maximum neighbor  for each vertex  such that for all  no vertex  is its own maximum neighbor (). 
This is necessary for the next lemma.

\begin{lemma}\label{lem:ijInE_iff_mjInE2}
Let  be a graph with  and a maximum neighborhood ordering  where for all ,  is a maximum neighbor of  with . 
If  and , then .
\end{lemma}

\begin{proof}
 Vertex  is adjacent in  to  (). 
Thus, the distance between  and  is at most . 
So  and  are also adjacent in  ().

 Vertices  and  are adjacent in  ().
If , then .
Now assume that .
Then vertices  and  have a common neighbor  in , choose the rightmost such vertex (that is,
the one maximizing the value of ).
We distinguish between two cases:
\begin{enumerate}[(i)]
    \item .
    If , then  so we may assume that .
    By definition  is adjacent to all neighbors of  in . 
    This includes .

    \item . In this case any maximum neighbor  of  satisfies  (since  but ),  and . 
    In particular,  is a common neighbor of  and . 
    Since  for some , this contradicts the choice of .
\end{enumerate}
\end{proof}

\begin{algorithm}[ht]
\label{algo:EDdc}
\caption{A linear time algorithm for the \minwed problem in dually chordal graphs}

\KwIn{An instance  of the \minwed problem where  is dually chordal graph.}
\KwOut{An efficient dominating set  in  with , if one exists, {\sc No}, otherwise.}

\BlankLine
{

     \nllabel{line:posWeightsStart}

    \If{}
    {
        Solve the efficient domination problem on .\nllabel{line:compgamma}

        \If{ has no efficient dominating set}
        {
            \Return{\sc No};
        }
        \Else
        {
            Let  be the common cardinality of all efficient dominating sets of .
        }

        For all  set 

        
        \nllabel{line:posWeightsEnd}
    }

    , , . \nllabel{line:compMk}

\ForAll{\nllabel{line2} }
    {
         Set  and .\nllabel{line:compOmega}

        Set  unmarked and not blocked.\nllabel{line4}
    }

    Find a maximum neighborhood ordering 
    of  which simultaneously is a perfect elimination ordering of ,
    with the corresponding maximum
    neighbors  where  for . \nllabel{line:compMNO}

    \For{ \KwTo \nllabel{line6}}
    {
        For all  set \nllabel{line7}.

        \If{\nllabel{line8} }
        {
             Mark  and set .\nllabel{line9}
        }
    }

    \For{\nllabel{line10} }
    {
        \If{ is marked and  is not blocked\nllabel{line11} }
        {
             Set  and block all .\nllabel{line:compMWISend}
        }
    }

    \If{\nllabel{line13} }
    {
        \Return{};\nllabel{line14}
    }
    \Else{\nllabel{line15}
    \Return{\sc No};\nllabel{line16}
    }
}
\end{algorithm}

\begin{theorem}\label{thm:meed}
Algorithm~\ref{algo:EDdc} solves the \minwed problem for dually chordal graphs in linear time.
\end{theorem}

\begin{proof}

To get Algorithm~\ref{algo:EDdc} we extended the Algorithm~\ref{algo:mwisChordal} by lines~\ref{line:posWeightsStart}--\ref{line:posWeightsEnd} to ensure non-negative weights. 
Additionally, we added the calculation for~ and~ (line~\ref{line:compMk}), and defined  now as  instead of  (line~\ref{line:compOmega}). 
Therefore, based on Lemma~\ref{lem:wed-mwis}, we first created an instance  for the \minwed problem with non-negative weights, and then by Lemma~\ref{lem:wed-mwis-2} an instance  for the \mwis problem on  (with ).

Next, the \mwis instance is solved in the lines~\ref{line:compMNO}--\ref{line:compMWISend} in linear time.
To achieve this, we rewrite Algorithm~\ref{algo:mwisChordal} to work on the dually chordal graph instead of its chordal square.

Based on Lemma~\ref{lem:MNO_linear}, line~\ref{line:compMNO} of Algorithm~\ref{algo:EDdc} computes a perfect elimination ordering of .

Let  and  be adjacent in  and . Now Lemma~\ref{lem:ijInE_iff_mjInE2} allows to modify the two loops. 
For the first loop (line~\ref{line6}), there is an extra vertex weight . 
When  is marked, instead of decrementing the weights  of the neighbors of  (and their neighbors),  of 's maximum neighbour~ is incremented by .
Now before comparing  to ,  is decremented by  for all . 
Because  is adjacent to  (Lemma~\ref{lem:ijInE_iff_mjInE2}), this ensures that each time the weight~ is compared to , it has the same value as it would have in Algorithm~\ref{algo:mwisChordal}.

For the second loop (line~\ref{line10}) the argumentation works similarly. 
After selecting a vertex~ (i.e., ), all its neighbors in  are blocked. 
Thus by Lemma~\ref{lem:ijInE_iff_mjInE2}, if a vertex~ is adjacent in  to a selected vertex, then the maximum neighbor  is blocked.

It follows that after line~\ref{line:compMWISend},  is a solution to the instance  for the \mwis problem on  created earlier.
    
Thus, the set~ is a solution for the given \minwed instance if and only if  (lines~\ref{line13}--\ref{line16}).


\medskip
Solving the efficient domination problem (line~\ref{line:compgamma}) and finding a maximum neighborhood ordering (line~\ref{line:compMNO}) can be done in linear time \cite{BCD98,BLR}. 
Also the overall runtime of lines~\ref{line6}--\ref{line:compMWISend} is bounded by the number of vertices and edges in . 
Thus, Algorithm~\ref{algo:EDdc} runs in linear time.
\end{proof}

As explained at the end of the proof of Theorem~\ref{thm:wed-mwis}, the \maxwed problem can be reduced in time  to the \minwed problem by negating the weights. 
Therefore, the \maxwed problem can also be solved in linear time on dually chordal graphs.

\subsection{AT-free graphs}
\label{sec:AT-free}

We now consider the class of AT-free graphs. 
Recall the following result due to Chang {\it et al.}~\cite{CHK03}.

\begin{theorem}[\cite{CHK03}]\label{ATGk}
Every proper power of an AT-free graph is a cocomparability graph.
\end{theorem}

Therefore, by Theorem~\ref{thm:wed-mwis}, a polynomial time algorithm for the \minwed and \maxwed problems in AT-free graphs will follow if the \mwis problem is solvable in polynomial time in the class of cocomparability graphs.
The \mwis problem in the class of cocomparability graphs is equivalent to the maximum weight clique problem in the class of comparability graphs, for which the following result is known.

\begin{theorem}[\cite{Gol04,MS99}]
The maximum weight clique problem can be solved in linear time on comparability graphs.
\end{theorem}

The algorithm given by Golumbic~\cite{Gol04} requires a transitive orientation of a comparability graph  as input. 
Such an orientation can be found in linear time using methods of McConnell and Spinrad~\cite{MS99}. Consequently:

\begin{theorem}\label{MWIS:cocomp}
The \mwis problem on instances  such that  is a cocomparability graph can be solved in time .
\end{theorem}


Theorems~\ref{thm:wed-mwis}, \ref{ATGk} and~\ref{MWIS:cocomp} imply the following result.

\begin{theorem}\label{thm:AT-free}
The \minwed and \maxwed problems are solvable in the class of AT-free graphs in time .
\end{theorem}

\begin{proof}
Let  be the class of AT-free graphs, and let  with . By Theorem~\ref{ATGk}, the square of  is cocomparability.
By Theorem~\ref{MWIS:cocomp}, the \mwis problem is solvable on  in time . Thus, Theorem~\ref{thm:wed-mwis} implies that in the class of AT-free graphs, the \maxwed and \minwed problems are solvable in time
.
\end{proof}

Theorem~\ref{thm:AT-free} generalizes the polynomial-time algorithms for the efficient dominating set problem in the class of AT-free graphs by Brandst\"adt \emph{et al.}~\cite{BLR} and by Broersma \emph{et al.}~\cite{BKM99}.
Both papers~\cite{BLR,BKM99} solve the unweighted version of the problem in time , while we give an algorithm of complexity  to solve the more general, weighted versions of the problem.
The polynomial time solvability implied by Theorem~\ref{thm:AT-free} can also be seen as a common extension of the polynomial-time solvability of the \minwed problem on interval graphs~\cite{CL94}, cocomparability graphs~\cite{CRC95}, and permutation graphs~\cite{LLT97}, all subclasses of AT-free graphs.


\subsection*{Acknowledgements}

We are grateful to the two anonymous referees for their helpful comments. 
M.M.~is supported in part by ``Agencija za raziskovalno dejavnost Republike Slovenije'', research program P-- and research projects J-, J-, and J-.

\bibliographystyle{abbrv}
\begin{thebibliography}{10}
\bibitem{BBS88}
D. W. Bange, A. E. Barkauskas and P. J. Slater, Efficient dominating sets in graphs, in R.
D. Ringeisen and F. S. Roberts (Eds.): Applications of Discrete Mathematics, pages
189--199, SIAM, Philadelphia, PA, 1988.

\bibitem{Biggs}
N. Biggs, Perfect codes in graphs, J.~Combin.~Theory, Ser. B, 15 (1973) 288--296.


\bibitem{BCD98}
A.~Brandst\"adt, V.~Chepoi and F.F.~Dragan,
The algorithmic use of hypertree
structure and maximum neighbourhood orderings, Discrete Applied Math. 82
(1998) 43--77.

\bibitem{BDCV}
A.~Brandst\"adt, F.F.~Dragan, V.~Chepoi and V.I.~Voloshin,
Dually Chordal Graphs, SIAM J. Discrete Math. 11 (1998) 437--455.

\bibitem{BHN}
A.~Brandst\"adt, C.~Hundt and R.~Nevries, Efficient edge domination on hole-free graphs in polynomial time,
In A.~L\'opez-Ortiz (Ed.): Proceedings LATIN 2010: Theoretical Informatics, 9th Latin American Symposium,
Lecture Notes in Computer Science 6034, pp.~650--661, 2010.

\bibitem{BLV99}
A. Brandst\"adt, V.B. Le and J.P. Spinrad, Graph Classes: A Survey, SIAM Monographs on Discrete Math. Appl., Vol. 3, SIAM, Philadelphia, 1999.

\bibitem{BLR}
A.~Brandst\"adt, A.~Leitert and D.~Rautenbach, Efficient dominating and edge dominating sets
for graphs and hypergraphs. In Proceedings of ISAAC 2012, Lecture Notes in Computer Science 7676 (2012) 267--277.



\bibitem{BKM99} H.~Broersma, T.~Kloks, D.~Kratsch and H.~M\"uller, 
Independent sets in asteroidal triple-free
graphs, SIAM J.~Discrete Math. 12 (1999) 276--287.

\bibitem{Cesati}
M. Cesati, Perfect code is W[1]-complete,
Inf. Process. Lett. 81 (2002) 163--168.

\bibitem{C97}
M.S. Chang, Weighted domination of cocomparability graphs, Discrete Appl. Math. 80 (1997)
135--148.

\bibitem{CHK03}
J.-M. Chang, C.-W. Ho and M.-T. Ko,
Powers of asteroidal triple-free graphs with applications,
Ars Combinatoria 67 (2003)  161--173.

\bibitem{CL93}
M-S. Chang and Y.-C. Liu,
Polynomial algorithm for the weighted perfect domination problems on chordal
graphs and split graphs, Inform. Process. Lett. 48 (1993) 205--210.

\bibitem{CL94}
M-S. Chang and Y.-C. Liu,
Polynomial algorithms for weighted perfect domination problems on interval and circular-arc graphs,
J. Inf. Sci. Eng. 11 (1994) 549--568.

\bibitem{CRC95}
G. J. Chang, C. Pandu Rangan and S. R. Coorg,
 Weighted independent perfect domination on cocomparability graphs,
 Discrete Applied Math. 63 (1995) 215--222.

\bibitem{COS97}
D.G. Corneil, S. Olariu and L. Stewart, Asteroidal triple-free graphs, SIAM J. Discrete
Math. 10 (1997) 399--430.

\bibitem{CMR00}
B.~Courcelle, J.A.~Makowsky and U.~Rotics, Linear Time Solvable
Optimization Problems on Graphs of Bounded Clique-Width, Theory
Comput.~Systems 33 (2000) 125--150.

\bibitem{FH91}
M.R. Fellows and M.N. Hoover, Perfect domination, Australas. J. Combin. 3 (1991) 141--150.

\bibitem{Frank}
A. Frank,
Some polynomial algorithms for certain graphs and hypergraphs.
Proc. 5th Br. comb. Conf., Aberdeen 1975, Congr. Numer. XV (1976) 211--226.

\bibitem{GJ77} M.G.~Garey and D.S.~Johnson, The rectilinear Steiner
tree problem is NP-complete, SIAM J.~Appl.~Math. 32 (1977)
826--834.

\bibitem{Gol04}
M.C. Golumbic, Algorithmic Graph Theory and Perfect Graphs, Second Edition. North Holland, 2004.

\bibitem{graphclasses}
Information System on Graph Classes and their Inclusions (ISGCI)
by H.N. de Ridder et al.~2001--{2014}.
http://www.graphclasses.org.

\bibitem{GLS84} M.~Gr\"otschel, L.~Lov\'asz and A.~Schrijver, Polynomial algorithms for
perfect graphs. Topics on perfect graphs, 325--356, North-Holland
Math. Stud., 88, North-Holland, Amsterdam, 1984.

\bibitem{Guo-Niedermeier}
J. Guo and R. Niedermeier,
Linear problem kernels for NP-hard problems on planar graphs,
In: L. Arge et al. (Eds.): Proceedings of ICALP 2007, Lecture Notes in Computer Science 4596, pp. 375–386, 2007.

\bibitem{KE00}
W.F. Klostermeyer and E.M. Eschen,
Perfect codes and independent dominating sets,
Congr. Numer. 142 (2000) 7--28.

\bibitem{Kratochvil91}
J. Kratochv\'il, Perfect codes in general graphs, Rozpravy \v Ceskoslovensk\'e Akad. V\v ed \v Rada
Mat. P\v r\'irod V\v d 7 (Akademia, Praha, 1991).

\bibitem{KMM95}
J. Kratochv\'il, P. Manuel and M. Miller, Generalized domination in chordal graphs,
Nordic J. Comput. 2 (1995) 41--50.

\bibitem{Leitert}
A. Leitert, Das Dominating Induced Matching Problem f\"ur azyklische Hypergraphen,
Diploma Thesis, University of Rostock, Germany, 2012.

\bibitem{LLT97}
Y. D. Liang, C.-L. Lu and C.-Y. Tang. Efficient domination on permutation graphs
and trapezoid graphs. In: T. Jiang and D. T. Lee (Eds.): Proceedings of the Third COCOON conference,
Lecture Notes in Computer Science 1276, pp.~232--241, 1997.

\bibitem{L98}
Y. L. Lin, Fast algorithms for independent domination and efficient domination in trapezoid graphs,
In: K.-Y. Chwa and O. H. Ibarra (Eds.): Proceedings of ISAAC'98, Lecture Notes in Computer Science 1533, pp.~267--275, 1998.

\bibitem{LS90}
M. Livingston and Q.F. Stout, Perfect dominating sets, Congr. Numer. 79 (1990) 187--203.

\bibitem{LM08}
V.~Lozin and M.~Milani\v c, A polynomial algorithm to
find an independent set of maximum weight in a fork-free graph,
J.~Discrete Algorithms (2008) 595--604.

\bibitem{LT02} C. L. Lu and C. Y. Tang,
Weighted efficient domination problem on some perfect graphs,
Discrete Appl. Math. 117 (2002)  163--182.

\bibitem{MS99}
R.M.~McConnell and J.P.~Spinrad,
Modular decomposition and transitive orientation,
Discrete Math. 201 (1999) 189--241.

\bibitem{Milanic}
M. Milani\v c, Hereditary efficiently dominatable graphs,
J. Graph Theory 73 (2013) 400--424.

\bibitem{Min80}
G.J.~Minty, On maximal independent sets of vertices in claw-free
graphs.~J.~Combin.~Theory Ser.~B 28 (1980) 284--304.

\bibitem{NT01}
D.~Nakamura and A.~Tamura, A revision of Minty's algorithm for finding a
maximum weight stable set of a claw-free graph, J.~Oper.~Res.~Soc.~Japan 44
(2001) 194--204.

\bibitem{OPS08}
G.~Oriolo, U.~Pietropaoli and G.~Stauffer, A new algorithm for the
maximum weighted stable set problem in claw-free graphs,
Proceedings IPCO 2008, Bertinoro,
LNCS 5035 (2008) 96--107.

\bibitem{Pol74}
S.~Poljak, A note on stable sets and coloring of graphs,
Comment.~Math.~Univ.~Carolinae 15 (1974) 307--309.

\bibitem{Williams}
V.~V.~Williams,
Multiplying matrices faster than Coppersmith-Winograd, STOC '12,
Proceedings of the 44th symposium on Theory of Computing, pages 887--898, ACM, New York, NY, USA.

\bibitem{Y92}
C.-C. Yen, Algorithmic aspects of perfect domination, Ph.D.~Thesis, Institute of Computer Science, National Tsing
Hua University, Taiwan, ROC, 1992.

 \bibitem{YL96}
C.-C. Yen and R. C. T. Lee,
The weighted perfect domination problem and its variants,
Discrete Applied Math. 66 (1996) 147--160.
\end{thebibliography}
\end{document}
