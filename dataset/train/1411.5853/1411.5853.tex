\documentclass[12pt,a4paper]{article}
\usepackage{times}
\usepackage{codams,float,amssymb}
\usepackage[leqno]{amsmath}
\usepackage[dvipdf]{graphicx}
\floatstyle{ruled}
\newfloat{displaycode}{htp}{alg}[section]
\floatname{displaycode}{Eiffel code}
\usepackage[bookmarks,bookmarksopen,
          pdftitle={"An accl not a crcl"},
          pdfauthor={Colm \'O D\'unlaing}]{hyperref}
\renewcommand{\theequation}{\arabic{equation}}
\newtheorem{definition}[theorem]{Definition}
\newtheorem{lemma}[theorem]{Lemma}
\newtheorem{remark}[theorem]{Remark}
\newtheorem{notation}[theorem]{Notation}
\newtheorem{proposition}[theorem]{Proposition}
\newtheorem{corollary}[theorem]{Corollary}
\newtheorem{assumption}[theorem]{Assumption}
\newtheorem{conjecture}[theorem]{Conjecture}
\newtheorem{specification}[theorem]{Specification}
\newcommand{\redstar}{{\overset{*}{\rightarrow}}}
\newcommand{\mapstar}{{\overset{*}{\mapsto}}}
\newcommand{\expstar}{{\overset{*}{\leftarrow}}}
\newcommand{\thuecong}{{\overset{*}{\leftrightarrow}}}
\newcommand{\irr}{{\text{\rm irr}}}
\newcommand{\pres}{\,
{\vrule height5pt width.5pt depth0pt
\overset{\vrule height.5pt width10pt depth0pt}
{\vrule height0pt width0pt depth0pt}
\vrule height5pt width.5pt depth0pt}\,}
\newcommand{\presstar}{{\overset{*}{\pres}}}
\numberwithin{equation}{section}
\newcommand{\vor}{{\rm Vor}}
\newcommand{\csta}{C_{\text{\rm stable}}}
\newcommand{\cexp}{C_{\text{\rm expand}}}
\newcommand{\range}{\text{\rm range}}
\newcommand{\domain}{\text{\rm domain}}
\newcommand{\kernel}{\text{\rm kernel}}
\newcommand{\lcm}{\text{\rm lcm}}
\newcommand{\mymod}{\text{\rm mod \ }}
\newcommand{\pos}{{\text{\rm pos}}}
\renewcommand{\neg}{{\text{\rm neg}}}

\newcommand{\be}{}
\newcommand{\hb}{\hfil\break}
\def\cent{\hbox{\rm\rlap/c}}
\newcommand{\fp}{{\text{\rm FP}}}
\DeclareGraphicsExtensions{.eps}
\title{An ACCL which is not a CRCL}
\author{Colm \'O D\'unlaing\thanks{e-mail: odunlain@maths.tcd.ie.
Mathematics department website: http://www.maths.tcd.ie.}\\
{\em Mathematics, Trinity College, Dublin 2, Ireland}}


\begin{document}

\maketitle
\begin{abstract}
It is fairly easy to show that every regular set is
an almost-confluent congruential language (ACCL), and it
is known [\ref{dkrw}] that every regular set
is a Church-Rosser congruential language (CRCL).
Whether there exists an ACCL, which is not a CRCL, seems to remain
an open question.  In this note we present one such ACCL.
\end{abstract}

\section{Introduction}

 denotes the set of `strings' over an alphabet
 ---  can be any finite set;
strings over  are finite sequences drawn from .
 is a  monoid (with identity , the empty string)
under string concatenation.
The length of a string  is denoted  ().
If  and  then

\be
|x|_a
\ee
is the number of occurrences of  in , so
\be
\sum_{a \in \Sigma} |x|_a = |x|.
\ee


\begin{definition}
\label{def: thue congruence}
A {\em Thue system} over a finite alphabet 
is a set of ordered pairs  of strings
in .
In this note only finite Thue systems are considered.


If  is a Thue system, then we call the pairs
 in  its {\em rules}, sometimes written
.

A {\em congruence} on  (or any semigroup) is
an equivalence relation  such that
for all ,
\be
x\equiv y\implies uxv \equiv uyv
\ee
The equivalence classes can be multiplied and
thus there is a quotient monoid
\be
\Sigma^* / \equiv .
\ee
If  is a congruence and  a string, we write
\be
[x]_\equiv
\ee
for the congruence class of  modulo .
\end{definition}

\noindent
Given , we write
\be
x \leftrightarrow_T y
\ee
if there exist strings , such
that , , and
either  or .


This relation is symmetric, and its
reflexive transitive closure
\be
{\thuecong}_T
\ee
is a congruence on .
The notation for  congruence class is simplified as follows.
\be
[x]_T ~~=\text{\rm (def)}~~ [x]_{{\thuecong}_T}.
\ee

Emphasis is placed on the relative lengths of strings
in rules of .

\noindent
If  and in addition , , or
, respectively, write
\be
x \to_T y,\quad\text{or}\quad x \mapsto_T y,
\quad\text{or}\quad x {\pres}_T y,
\ee
respectively.

Since the relation  is
symmetric, we can assume that for any
,
\be
|u| \geq |w|
\ee

\begin{definition}
\label{def: redex}
When , so
, we call  the {\em redex} and  the {\em reduct}.
\end{definition}


\begin{definition}
A Thue system  is, respectively, {\rm (i)} Church-Rosser,
{\rm (ii)} almost confluent, {\rm (iii)} preperfect,
(see {\rm [\ref{bo}]}), if whenever ,

\begin{itemize}
\item [{\rm (i)}] 
there exists a string  such that  and
;
\item [{\rm (ii)}] there exist strings  and  such that
,
,
and ;
\item [{\rm (iii)}]
there exists a string  such that
 and .
\end{itemize}
\end{definition}

\begin{definition}
\label{def: irr}
If  is a Church-Rosser Thue system, then for any string
, every string  in  reduces (modulo ) to
the same irreducible string; we call this string
\be
\irr_T(x) .
\ee
\end{definition}

The word problem for Church-Rosser systems is in linear
time, and for the other two kinds it is PSPACE complete;
testing for the Church-Rosser property is tractable;
testing for almost confluence is in PSPACE;
it is undecidable whether a Thue system is preperfect [\ref{bo}].

\begin{definition}
\label{def: congruential language}
A language  is {\em congruential} if there exists
a congruence  and a finite set of strings


If the congruence is generated by a Thue system,
i.e., it is  for some finite Thue system ,
and  is, respectively, Church-Rosser, or almost confluent,
or preperfect, then  is a Church-Rosser, or almost confluent,
or preperfect congruential language: {\em CRCL, ACCL}, or {\em PPCL}.
\end{definition}

An interesting and  old result is that every regular set is
an ACCL.  It can be shown as follows:
if  is a regular set then there exists
a finite monoid  and a homomorphism from
 to  such that  is a union
of  for suitable  in .
But this partition 
\be
\{h^{-1}(g): ~ g \in M\}
\ee
can also be realised by a finite almost-confluent
system, namely: let  be the maximal length of minimal
strings in this partition (a  string is minimal
if whenever , ).
Then the system
\be
S =\{(x,y):\quad x,y\in \Sigma^*,~
|x| \leq N+1,~ x{\thuecong}_T y,~ y ~ \text{minimal} \}
\ee
is almost confluent and its congruence classes
coincide with the inverse images , as required.

A long-standing open problem was
whether every regular set is a CRCL:  it was
settled in the affirmative a few years ago
[\ref{dkrw}].

That left open the unlikely possibility
that every ACCL is a CRCL.  This note shows
the contrary.

The analysis in this paper is simple and direct.
In fact, the problem is not susceptible to more
sophisticated methods.  As noted in [\ref{ods}],
Kolmogorov-complexity-based analyses showing
palindromes not to be Church-Rosser\footnote{Church-Rosser
languages are a much richer class of languages than
Church-Rosser congruential.} also shows them not
to be almost confluent.  Indeed, in [\ref{ods}] we were only able to
show that they are `preperfect languages'.

All Church-Rosser monoids  are  [\ref{squier},\ref{cohen}].
On the other hand, if one inspects
the group furnished by Squier [\ref{squier}],
which is not , it has an obvious presentation
as a monoid, but the presentation again turns out to be
preperfect rather than almost confluent.

Book's reduction machine [\ref{bo}]
can be used with almost-confluent Thue systems,
from which is follows that ACCLs are linear time recognisable.
The word problem for
an almost confluent Thue system is PSPACE-complete, but
(as is easy to show) if the
system presents a {\em group} then the word problem is linear time.
So there are few complexity-based arguments separating
ACCLs from CRCLs.

\section{An ACCL which is not a CRCL}

We shall introduce an almost confluent Thue system over
a 4-letter alphabet
, and
an involution
\be
a \mapsto c \mapsto a,\quad
b\mapsto d \mapsto b
\ee
or
\be
\overline{a} = c, \overline{c} = a,
\overline{b} = d, \overline{d} = b.
\ee

\noindent Any string in 
can and will be written using .
\begin{definition}
We call  {\em positive} and  (i.e.,
) {\em negative}.

Given a string  over ,


the number of occurrences of positive and negative
letters in .
\end{definition}

Let
\be
h: \Sigma^* \to \IZ
\ee (the additive group of integers) denote the following map:
\be
h(x) ~=~
|x|_\pos - |x|_\neg .
\ee
This is a homomorphism, and
\be
a\mapsto 1, b \mapsto 1,
\overline{a}\mapsto -1, \overline{b}\mapsto -1.
\ee
Let  be the Thue system


\noindent
The map 
preserves both sides of each rule in , and
therefore induces a homomorphism
\be
\Sigma^*/ {\thuecong}_S \to \IZ.
\ee

For the rest of this paper, we assume that strings
are written in terms of .

\begin{definition}
Given a string , the string
 is defined as
\be
\tilde{x} = \overline{a_k} \, \overline{a_{k-1}} \ldots \overline{a_1}.
\ee
\end{definition}
Clearly  and
.

\begin{definition}
A string  is {\em mixed} if it contains both positive
( or ) and negative ( or ) letters.
Else it is {\em unmixed}.  Unmixed strings can be
empty, positive, or negative, in the obvious sense.
\end{definition}

If  is mixed, then it contains an adjacent pair of
positive and negative letters which can be reduced
(modulo ).  Thus mixed strings are reducible.
Unmixed strings are irreducible.

Thus every string  can be reduced to a positive
or negative string.
If  is positive then .  If 
is negative then .

\begin{lemma}
If  and  are both positive strings,
or both negative, and , then
.\qed
\end{lemma}

\begin{corollary}
 is almost confluent and  induces an
isomorphism of  with .
\end{corollary}

{\bf Proof.}
Suppose .

Reduce  and  (modulo ) to irreducible
strings  and .  Then ,
and  and  are unmixed.

If , then .
If , then  and  are entirely positive,
, and .

Similarly if .

We have shown that if  then there exist irreducible
strings  and  such that ,
, and .

In particular, .
Conversely, as has been noted, if  then
:
 induces
an isomorphism of  with its image, .

Finally, if , then ,
so there exist strings  so
\be
x {\redstar}_S x' {\presstar}_S y' 
{\overset{*}{\leftarrow}}_S y
\ee
so  is almost confluent.\qed


\begin{definition}
\be
L = [\lambda]_S = h^{-1}(0).
\ee
This is our candidate for a non-CRCL.
\end{definition}

\begin{corollary}
 is an ACCL.\qed
\end{corollary}

\begin{theorem}
\label{thm: main}
 is not a CRCL.
\end{theorem}

We prove this by contradiction. Otherwise there exists
a Church-Rosser Thue system  and a list  of irreducible strings
\be
u_1,\ldots, u_n
\ee
in  such that
\be
\takeanumber
\tag{\thetheorem}
\label{eq: union}
L = [\lambda]_S = [u_1]_T \cup \ldots \cup [u_n]_T
\ee
or equivalently
\be
x \in L \iff \irr_T(x) \in \{u_1, \ldots, u_n\}.
\ee

Associated with  and the strings , we define the
following constants:
\begin{definition}
\label{def: QR}
\be
Q = \max_{(\ell,r)\in T} |\ell|
\quad\text{and}\quad
R = \max_{1\leq j\leq n}|u_j|_\neg.
\ee
( is the maximum length of redexes in .)
\end{definition}

\begin{lemma}
\label{lem: T refines}
If such a Thue system  exists, then 
refines  (in the sense
that ).
\end{lemma}

{\bf Proof.}  It is enough to show that whenever
\be
x \to_T y,
\ee
. Clearly
\be
x\tilde{x} \to_T y\tilde{x}
\ee
But , which is a union of
congruence class modulo , so .
Then .
But , so
, as required.\qed

\begin{corollary}
If  is unmixed, then  is irreducible (modulo ).
\end{corollary}

{\bf Proof:}  is irreducible (modulo ) and  refines
.\qed

\begin{lemma}
\label{lem: reduced by Q}
Suppose that  where  is unmixed (and ).
Then  can be factored as  where  is  unmixed and 
(\ref{def: QR}).
\end{lemma}

{\bf Proof.} The redex in  cannot be entirely in
 since  is irreducible.  Therefore the redex
is in  where  (possibly ).
Setting ,  is a suffix of ,
 is unmixed, and .\qed

\begin{lemma}
\label{lem: balanced reduction}
Suppose . Then 
and .
\end{lemma}

{\bf Proof}
Since , ,
so the number of positive and negative letters is reduced
by the same amount, namely, .\qed

\begin{corollary}
\label{cor: rightmost k}
For any positive integer , if  is positive
of  length  (\ref{def: QR}), then
for ,
\be
y\quad\text{\rm and}\quad \irr_T(u_i y)
\ee
agree on their rightmost  letters.
\end{corollary}

{\bf Proof.}
Lemma \ref{lem: reduced by Q} can be extended inductively
so that if  is reduced  times, then
the reduced string agrees with  on their rightmost
 letters.  By Lemma \ref{lem: balanced reduction},
 can be reduced at most  times.
But  and ,
so  and  agree on their rightmost
 letters; and .\qed

\hb

{\bf Proof of} Theorem \ref{thm: main}.
Let  and let  be a positive
string of length .  For any positive string
 of the same length as , .

Let  (noting that ).
For any positive string  with , 
so .  But
, so
 and  for some .
Therefore  and
. But ,
so, for every positive  with ,
\be
\takeanumber
\tag{\thetheorem}
\label{eq: mismatch class count}
[u_iy]_T = [x u_j]_T
\ee
for some .
Let  be an enumeration of all positive strings 
of length  which agree with  on their first 
letters.  There are  such strings. By Corollary
\ref{cor: rightmost k}, for each string ,
\be
y_q \quad\text{\rm and}\quad  \irr_T(u_i y_q)
\ee
agree on their rightmost
 letters.  The irreducible strings belong to
different congruence classes. Therefore there are
at least  congruence classes fitting the left-hand
side of equation \ref{eq: mismatch class count}, and there
are at most  classes matching the right-hand side.
Since , we have  a contradiction:  is not
a CRCL.\qed

\section{Acknowledgement}
The author is grateful to Friedrich Otto
for some corrections and helpful suggestions.

\section{References}
\label{references} \begin{enumerate}
\item
\label{bo}
Ronald V.\ Book and Friedrich Otto (1993). {\em String-rewriting
systems.} Springer texts and monographs in computer science.
\item
\label{cohen}
Daniel E.\ Cohen (1997). String rewriting and homology of monoids.
{\em Math.\ Structures in Computer Science \bf 7:3}, 207--240.
\item
\label{dkrw}
Volker Diekert, Manfred Kufleitner, Klaus Reinhardt, and
Tobias  Walter (2012).
Regular languages are Church-Rosser congruential.
{\em Proc.\ 39th.\ ICALP II, Springer LNCS 7392}, 177--188.
\item
\label{ods}
Colm \'O D\'unlaing and Natalie Schluter (2010).
A shorter proof that palindromes are not a Church-Rosser
language, with extensions to almost-confluent and preperfect Thue systems.
{\em Theoretical Computer Science \bf 411}, 677--690.
\item
\label{squier}
Craig C.\ Squier (1987). Word problems and a homological finiteness
condition for monoids. {\em J. Pure and Applied Algebra \bf 49}, 201--217.
\end{enumerate}
\end{document}
