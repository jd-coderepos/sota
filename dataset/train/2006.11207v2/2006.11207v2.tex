\documentclass{article}





     \usepackage[nonatbib]{neurips_2020}

\usepackage[utf8]{inputenc} \usepackage[T1]{fontenc}    \usepackage{hyperref}       \usepackage{url}            \usepackage{booktabs}       \usepackage{amsfonts}       \usepackage{nicefrac}       \usepackage{microtype}      

\title{Domain Generalization via Stylization}


\author{David S.~Hippocampus\thanks{Use footnote for providing further information
    about author (webpage, alternative address)---\emph{not} for acknowledging
    funding agencies.} \\
  Department of Computer Science\\
  Cranberry-Lemon University\\
  Pittsburgh, PA 15213 \\
  \texttt{hippo@cs.cranberry-lemon.edu} \\
}

\begin{document}
\maketitle


\section{Introduction}

\begin{itemize}
    \item Domain Generalization aims to solve one of the central goals of computer vision, learning good features that generalize across data sets
    \item Domain Adaptation attempt to solve some of these problems with unlabeled target data
    \item In addition this problem is out of the realm of standard statistical generalization as a function class that generalizes to a particular data producing distribution has no guarantee for other related distributions 
    \item Domain Generalization aims to solve the harder problem of using multiple sources and no target data
    \item This more closely mimics the environment of the real world
    \item A number of previous approaches have been proposed to attempt to solve this problem, varying from meta learning based approaches all the way to adversarial alignment based methods
    \item We introduce a conceptually simple method that at its core considers which biases a model needs to be able to transfer to arbitrary visual data
\end{itemize}

\section{Contributions}
\begin{itemize}
    \item Introduce a conceptually simple especially as compared to previous methods method for domain generalization
    \item Introduce a stochastic stylization based augmentation which achieves state of the art results on a number of common DG datasets
    \item Show that the method is effective even when only applied with in Domain data
    \item Conjecture that this relates to the over reliance of the model on low level texture information which doesn't generalize well to different distributions
\end{itemize}

\section{Todo}
\begin{enumerate}
    \item Run multiple models over the same dataset to see how stylization effects different models, does this follow the shape texture bias curve?
    \item Test Domain Adaptation methods vs stylization
    \item Shape vs. texture bias and figure 4. of shape bias paper
    \item Use the method from the adversarial examples paper to generate a dataset with only adversarial examples and see how stylization helps
\end{enumerate}











































































\end{document}
