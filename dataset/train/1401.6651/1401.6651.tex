
\documentclass[journal,a4paper,12pt,onecolumn]{IEEEtran}
\usepackage{amssymb}
\usepackage{amsmath}
\usepackage{amsfonts}
\usepackage{graphicx}
\usepackage{subfigure}

\setcounter{MaxMatrixCols}{10}

\newtheorem{theorem}{Theorem}
\newtheorem{acknowledgement}[theorem]{Acknowledgement}
\newtheorem{algorithm}[theorem]{Algorithm}
\newtheorem{axiom}[theorem]{Axiom}
\newtheorem{case}[theorem]{Case}
\newtheorem{claim}[theorem]{Claim}
\newtheorem{conclusion}[theorem]{Conclusion}
\newtheorem{condition}[theorem]{Condition}
\newtheorem{conjecture}[theorem]{Conjecture}
\newtheorem{corollary}[theorem]{Corollary}
\newtheorem{criterion}[theorem]{Criterion}
\newtheorem{definition}[theorem]{Definition}
\newtheorem{example}[theorem]{Example}
\newtheorem{exercise}[theorem]{Exercise}
\newtheorem{lemma}[theorem]{Lemma}
\newtheorem{notation}[theorem]{Notation}
\newtheorem{problem}[theorem]{Problem}
\newtheorem{proposition}[theorem]{Proposition}
\newtheorem{remark}[theorem]{Remark}
\newtheorem{solution}[theorem]{Solution}
\newtheorem{summary}[theorem]{Summary}

\begin{document}

\title{On Near-controllability, Nearly-controllable Subspaces, and
Near-controllability Index of a Class of Discrete-time Bilinear Systems: A
Root Locus Approach\thanks{This work was supported by the China Postdoctoral Science Foundation funded
project and the National Natural Science Foundation of China.}}
\author{Lin Tie\thanks{The author is with the School of Automation Science and Electrical
Engineering, Beihang University (Beijing University of Aeronautics and
Astronautics), 100191, Beijing, P. R. China. \textit{E-mail address}:
tielinllc@gmail.com.}}
\maketitle

\begin{abstract}
This paper studies near-controllability of a class of discrete-time bilinear
systems via a root locus approach. A necessary and sufficient criterion for
the systems to be nearly controllable is given. In particular, by using the
root locus approach, the control inputs which achieve the state transition
for the nearly controllable systems can be computed. Furthermore, for the
non-nearly controllable systems, nearly-controllable subspaces are derived
and near-controllability index is defined. Accordingly, the controllability
properties of such class of discrete-time bilinear systems are fully
characterized. Finally, examples are provided to demonstrate the results of
the paper.
\end{abstract}

\begin{keywords}
discrete-time bilinear systems, near-controllability, nearly-controllable
subspaces, near-controllability index, root locus approach.
\end{keywords}

\section{Introduction}

Given the capability of modeling a large number of processes in the real
world, bilinear systems have received considerable attention [1-4].
Furthermore, bilinear systems are thought to be simpler and better
understood than most other nonlinear systems. Such systems have particular
advantages on structural properties, optimization, identification, and
control. The relaxed version of a linear switched system is actually a
bilinear system [5]. Owing to both the practical and theoretical importance,
bilinear systems have been a hot topic in the literature of nonlinear
systems over decades.



Controllability is one fundamental concept in mathematical control theory.
It was identified in the early 1960s, and then the theory of controllability
for linear systems based on the state space description was systematically
established [6,7]. During nearly the same period, controllability of
nonlinear systems also was considered. Since the 1970s, Lie algebra methods
and other powerful tools of differentiable manifold theory have been
developed to study controllability of nonlinear systems [8-11]. Today,
controllability has played an essential role in the development and
application of mathematical control theory. There are many different kinds
of definitions on controllability, such as local controllability, global
controllability, approximate controllability, positive controllability, and
null controllability. Roughly speaking, controllability is defined as the
ability of a system that the system can be steered from an arbitrary initial
state to an arbitrary terminal state under the action of admissible
controls. It is an important as well as fundamental\ property of control
systems and is of great practical relevance.



In bilinear system theory, controllability is one of the main research
topics. This is particularly true for the continuous-time case. More
specifically, controllability of continuous-time bilinear systems has been
extensively investigated profiting from the Lie algebra methods. Various
Lie-algebraic criteria on controllability of continuous-time bilinear
systems were obtained in the literature, which have been summarized and
updated in a recent monograph [4] on bilinear systems. However, for
discrete-time bilinear systems, the controllability results are rather
sparse compared with their continuous-time counterparts. Most of the works
on controllability of discrete-time bilinear systems were done in the 1970s
[12-15], dealing with systems of the formwhere , , and . \textit{System (1) is said to be controllable if, for any }\textit{\ in }\textit{\ (}\textit{),\ there exist a positive integer }\textit{\ and a finite control sequence }\textit{\ (}\textit{) such that the system can be steered from }\textit{\ to }\textit{\ at }. In particular, [12] gave a sufficient condition
for controllability of system (1), which requires, at least, \ is similar
to an orthogonal matrix. [13] studied controllability of system (1) under
the assumption of rank and presented necessary as well as sufficient
conditions; based on the work in [13], [15] improved these conditions by
raising necessary and sufficient ones. It can be seen that, for
controllability of discrete-time bilinear systems, only specific subclasses
have been considered, while most cases remain unsolved. Since the middle
1980s, few work has been reported until the 2000s. One of the reasons
leading to such a few results on controllability of discrete-time bilinear
systems is that the controllability problems are quite difficult to deal
with due to the systems' nonlinearity. Another one is that, for
discrete-time nonlinear systems, semigroups tend to appear so that less
algebraic structure of the systems is available [16]. Recently, [17]
continued the study on controllability of system (1) and obtained necessary
and sufficient conditions for the case of , where it was shown that
the system can be controllable only if its dimension is no greater than two.
Nevertheless, an uncontrollable system can still have a large \textit{controllable region}\footnote{A controllable region is a region in  on which the system is controllable. Namely, for any  in
this region, there exist control inputs that steer the system from  to
.}. This has first been proved for system (1) with , i.e.which is uncontrollable with dimension greater than two [17]. Indeed, if 
has only real eigenvalues that are nonzero and pairwise distinct, then the
system (2) has a large controllable region, which nearly covers the whole
space, and is nearly controllable.



If we only use \textquotedblleft uncontrollable\textquotedblright\ to
describe a system which is not controllable according to the general
controllability definition, we may miss some valuable properties of it.
Near-controllability is thus introduced to describe those systems that are
uncontrollable but have a very large controllable region. Nearly
controllable systems exist widely in nonlinear systems. This property was
first defined and was demonstrated on system (2) in [18], and it was then
generalized to system (1) with  in [19] and to both continuous-time and
discrete-time nonlinear systems that are not necessarily bilinear in [20].
The definition of near-controllability, which was first given in [18], has
now been updated in [20]. \textit{A continuous-time system }\textit{\
(discrete-time system }\textit{) is said to be nearly controllable if,
for any }\textit{\ and any }\textit{, there exist a piecewise
continuous control }\textit{\ and }\textit{\ (a
finite control sequence }\textit{, }\textit{, where }\textit{\ is a positive integer) such that the system
can be steered from }\textit{\ to }\textit{\ at some }\textit{\ (}\textit{), where }\textit{\
and }\textit{\ are two sets of zero Lebesgue measure in }. If we let , then the
near-controllability definition degenerates to the general controllability
definition. Thus, near-controllability includes the notion of
controllability and may better characterize the controllability properties
of nonlinear systems. However, most of the existing works on
near-controllability are reported for discrete-time bilinear systems and the
study on this topic is just at the beginning.



In this paper, we continue the research on near-controllability of system
(2) with  having real eigenvalues only. We improve the sufficient
condition for near-controllability of the system (2) in [18] by giving a
necessary and sufficient one. In particular, we apply a new approach this
time to prove near-controllability. That is, a root locus approach is
proposed in this paper. Compared with the technique used in [18,19] which is
based on the implicit function theorem, by the root locus approach we can
not only improve the obtained result on near-controllability, but also
compute the required control inputs that achieve the state transition\footnote{For nonlinear systems, it is, in general, hard to compute the control inputs
to achieve state transition even if controllability has been proved. In this
paper, we obtain both the controllability and \textquotedblleft control
computability\textquotedblright\ (the ability of computing the required
control inputs for the transition of any given initial and terminal states).}. We thus present a useful algorithm. Furthermore, inspired by the state
space description for linear systems, we also consider the non-nearly
controllable systems. It is well known that if a linear time-invariant
system is uncontrollable, then the state space can be decomposed as a direct
sum of a controllable subspace and an uncontrollable subspace. For the
non-nearly controllable systems, we derive nearly-controllable subspaces and
define near-controllability index by using the improved near-controllability
result, which shows that the non-nearly controllable systems can be
controllable on the nearly-controllable subspaces. The near-controllability
index is used to determine the largest nearly-controllable subspaces that a
non-nearly controllable system has. In summary, the controllability
properties of the system (2) are fully characterized. Finally, we provide
examples to illustrate the conceptions and the results of this paper.



\section{A Root Locus Approach to Near-controllability}

In this section, we propose a root locus approach and apply it to prove
near-controllability of the system (2). The idea of the root locus approach
is to use the root locus theory to achieve the state transition, including
transferring any initial state to itself and to a state close to it. More
importantly, the control inputs are computable in these steps. Then, we use
the matrix theory by establishing a transition matrix and some algebraic
techniques to prove near-controllability. The proof steps are similar to
those in [18,19], but the factor that plays the key role has changed. That
is, the implicit function theorem has been replaced by the root locus
approach.



\begin{lemma}
If  has only nonzero and real eigenvalues, then there exist
nonzero real numbers  such thatif and only if the dimension of the largest Jordan block in the Jordan
canonical form of  is no greater than two. Further, if  not only
satisfies the necessary and sufficient condition but also is cyclic\footnote{A matrix is said to be cyclic if its characteristic polynomial is equal to
its minimal polynomial. Namely, only one Jordan block exists for each
eigenvalue in the Jordan canonical form of the matrix.}, then there exist  nonzero real and pairwise distinct numbers  such that (3) holds, where  is the number of the distinct eigenvalues
of .
\end{lemma}

\begin{proof}
We put the proof of necessity in Appendix and only prove sufficiency here.
Assume first that  has the following cyclic formwhere  is already in Jordan canonical form since a nonsingular
transformation to  does not affect the proof and  are nonzero real and pairwise distinct. In addition, .
We now show that the equationadmits a solution of nonzero real . To this end, it will be
proved that the reciprocals of  are on the root loci of the
characteristic equation of a closed transfer function related to 's
eigenvalues. Multiplying both sides of equation (5) bywe havefrom which\footnote{Throughout this paper, such kind of expression  is used to represent the summation
\par
for simplicity. There is no meaning of division in the expression.}Writing (7) through two groups of equations yieldsBy adding the following \ constraintsto the second group of equations in (8) and the two constraintsto all equations in (8), where  are chosen
such thatwe can put the above 
equations into the matrix formFrom Lemma 15 in Appendix, we obtainIf we denote  bythen, from the Vi\`{e}te's formulas, \ are the roots of the followingth-degree equationand are thus on the root loci of the characteristic equation of the closed
loop transfer function has \ real poles only and has no
zero. By condition (9), in the  real poles,  is the largest pole of  and is a single pole;  are double poles of ;  and  are two single poles of 
between two double poles in .
Therefore, we can see from the root locus theory [21] that, as  increases
from  to , all root loci of  that start
at the  real poles will first move on the real axis.
That is to say, we can always choose a  such that 
has  \textit{nonzero real} and \textit{pairwise distinct} roots. Indeed, by making  small enough,  are perturbed away from 
and hence are not only nonzero real but also pairwise distinct. Then, the
reciprocals of the nonzero real and pairwise distinct roots are the real
numbers that satisfy equation (5) since  satisfy the equations in (8), (7), and (6).

Finally, if  is noncyclic, we can choose the largest cyclic main
submatrix of , named , and there exist real numbers 
such thatfrom the above analysis. By noting the fact thatimpliesone can easily verify that eq. (10) still works if  is replaced by .
\end{proof}



By using Lemma 1, we can transfer an arbitrary state to itself and improve
Theorem 2 obtained in [18]. Moreover, by the root locus approach, we can
compute the required control inputs to achieve the state transition.



\begin{theorem}
Consider that  in (2) has only real eigenvalues. Then, the system (2) is
nearly controllable if and only if  is nonsingular, cyclic, and does not
have a Jordan block with dimension greater than two in its Jordan canonical
form.
\end{theorem}

\begin{proof}
The proof of necessity is put in Appendix. For sufficiency, we still assume
that  is of the Jordan canonical form given in (4) without loss of
generality. According to Lemma 1, there exist  nonzero
real and pairwise distinct values 
such thatWe next prove that, for almost any  in , we can construct control inputs such that  can be transferred
to an arbitrary state which is close to . We first prove the existence
of such control inputs, which is similar to what we have done in [18]. We
then show how to compute such control inputs by applying the root locus
approach, which is concluded in the final step of Algorithm 5 (to shorten
the proof we do not write it here).

Consider the following functionwhere  and . Apparently,  is a zero of . Using Lemma 1 in [18] yieldswhere  is the Vandermonde determinant and 
denotes the determinant of a matrix. Thus, for anythe Jacobian determinant (12) does not vanish at . According to the implicit function
theorem, there exist two open neighborhoods, namedrespectively, such thatwhere  \ and , i.e.Then, by (11) we haveThis means  can be transferred to any state that is close enough to . From 's structure and PBH test [22],that is a hypersurface in  and separates  into  open orthants (which are the same as those in (12) in
[19]). We now prove that the system (2) is controllable on each of the  open orthants. For any two states  in one orthant, we
establish the \textit{transition} \textit{matrix}It can be seen that  and all the
eigenvalues of  are positive since 
belong to the same orthant. Furthermore,Therefore, we can choose a positive integer  such that  is sufficiently close to  and
hence can be reached from \ according to (13). That is, there exist
control inputs 
such thatwhere . Note that  and  commute with each other. Applying  groups of  yieldsThat is, controllability on each of the  open orthants has been
proved. The rest is to \textquotedblleft connect\textquotedblright\ the  open orthants, i.e. to prove that the system is controllable on the
union of the  open orthants. One can readily finish this by using
Lemma 5 in [18] and following the arguments used in [18] (pp. 2856-2857,
from eq. (33) to eq. (37)), where the only explanation we should make here
is that, although  are double eigenvalues
of , the fact does affect connecting\ the  open orthants. Indeed,
just by replacing the subscript  by  in equations from (33) to (37) in
[18], we can complete the proof.

So far, we have proved that the system (2) is controllable onRecalling the near-controllability definition, we have thatand the system (2) is nearly controllable since 
are two sets of zero Lebesgue measure in .
\end{proof}



\begin{remark}
Lemma 1 is important to obtain the stronger result on near-controllability
of the system (2). One can see that Lemma 2 in [18] is a special case of
Lemma 1 and Theorem 2 in [18] is a special case of Theorem 2 also. In fact,
[18] was focusing on the case of B being diagonalizable, while the case
dealt with in this paper is more general. Moreover, the \textit{transition}
matrix  with the root locus approach will make it
possible to compute the required control inputs that achieve the state
transition.
\end{remark}



\begin{remark}
One can further prove that  in the proof of Theorem 2 can be . That is, for any  in (15) and any ,  can be transferred to . See the corresponding part in
the proof of Theorem 1 in [19] (from the last equation in p. 653 to the end
of the proof) for reference.
\end{remark}



By using the root locus approach, an algorithm is given to compute the
required control inputs that steer the nearly controllable system (2) from
one state to another, which both belong to (15).



\begin{algorithm}
Steps on computing control inputs for given initial and terminal states:
\end{algorithm}

\begin{itemize}
\item 1. Transform  into the Jordan canonical form as given in (4) by a
nonsingular matrix . The initial and terminal states  are
thus transformed into , respectively.
\end{itemize}

\begin{itemize}
\item 2. Find the control inputs that transfer  to a state 
which belongs to the same orthant as  belongs to (Lemma 5 in [18]
will be helpful and its proof includes the details on how to find such
control inputs).

\item 3. Get the transition matrix  for  from (14).

\item 4. Choose  that satisfy (9).

\item 5. Choose a positive integer  and compute .

\item 6. Obtain the root loci of the characteristic equation of the
following closed loop transfer function

where

with  denoting
the th entry of  and \ increases from  to . If any of the root loci
leaves the real axis directly at the pole, return to the former step and
choose another integer  greater than the previous one. Otherwise, choose
a suitable  such that the roots of  are all real.
Then, the reciprocals of the real roots are the control inputs that transfer
 to .  groups
of such control inputs together with the control inputs that transfer 
to  are the desired ones which steer the nearly controllable system
(2) from  to .
\end{itemize}



\emph{An Explanation to Step 6 of Algorithm 5.} Consider the equationFrom the proof of Theorem 2, we know that if  is large enough, then eq.
(18) admits a real solution of . Eq.
(18) is equivalent to the matrix equationLetAs shown for deriving eq. (7) in the proof of Lemma 1, we can obtainwhich can be written via equations as those in (8). Using  chosen in Step 4 and introducing the same constraints
as those for (8), we can deduceBy Lemma 15,where  are given in (17). Then, with (19) we
have that  are the roots of the followingth-degree
equationwhich is equivalent toThus,  are on the root loci of the characteristic equation of the closed
loop transfer function given in (16).



An example will be provided in Section 4 to show the effectiveness of
Algorithm 5.



\begin{remark}
By using the similar idea, one can try to derive an algorithm for computing
the control inputs to achieve the state transition for the nearly
controllable bilinear systems studied in [19]. Furthermore, although the
result obtained in [19], namely Theorem 1 in [19], seems similar to Theorem
2, it cannot yield Theorem 2 (vice versa). To see this, for  of
the system (2) that belong to (15), from Theorem 1 in [19] we can have  such thatwhere  satisfies the conditions in Theorem 2 and   are the corresponding initial and terminal states,
respectively. This impliesUnfortunately, we can transfer  to  but not to  since we do not have  either from Theorem 1 in [19] or
from Theorem 2.
\end{remark}



\begin{remark}
Note that the opposite numbers of 's eigenvalues are poles of  given in (16), which are all real so that it is possible for the
root loci of  to first move on the real axis.
However, if  has complex eigenvalues, then  may have
complex poles and some of the root loci of  will not
start at the real axis. In such a case, it is rather difficult to ensure
that all the root loci can have the same moment moving on the real axis,
then the real control inputs that achieve the state transition cannot be
obtained. Therefore, for near-controllability of system (2) with  having
complex eigenvalues, we need to develop the proposed root locus approach or
to find a new method.
\end{remark}



\section{Nearly-controllable subspaces and Near-controllability index}

Consider the systemwhere , , and . Since  has a Jordan block with dimension greater
than two, system (20) is non-nearly controllable according to Theorem 2.
Nevertheless, consider system (20) on regionIt can be seen that the system is invariant on (21), i.e.where  and \ is the main submatrix of \ by taking the
entries of  in both rows \ and columns . From Theorem 2,
subsystem (22) is nearly controllable. More specifically, it is controllable
onTherefore, system (20) is controllable onwhich is a two-dimensional region in . Similarly, one can deduce that the one-dimensional regionis also a region on which system (20) is controllable.



In this section, we study the non-nearly controllable systems. It is well
known that if a linear time-invariant system is uncontrollable, then the
state space can be decomposed as a direct sum of a controllable subspace and
an uncontrollable subspace. On the controllable subspace the linear system
is controllable. Furthermore, the controllable subspace corresponds to a
controllable linear subsystem. Inspired by these facts, we will derive
nearly-controllable subspaces of the system (2) by using Theorem 2.
Actually, the regions in (23) and (24) are nearly-controllable subspaces of
system (20). We do not use \textquotedblleft controllable
subspace\textquotedblright\ here since a subspace is in general a linear
space, while a nearly-controllable subspace does not contain the zero
element (the origin is an isolated state of system (1) that once it is
reached, then the system cannot be steered away).



\begin{definition}
A nearly-controllable subspace of the system (2) is a controllable region of
the system (2) that is derived from the corresponding nearly controllable
subsystem of the system (2) on the region.
\end{definition}



By the following Lemma, we will show how to obtain a nearly-controllable
subspace.



\begin{lemma}
Consider the systemwhere , , , and . Then, any state in a controllable region of system
(25), which contains more than one state, has the property that only the
first two entries can be nonzero.
\end{lemma}



The proof of this lemma can be found in Appendix.



\begin{theorem}
Consider the system (2) with  in Jordan canonical formwithout loss of generality, where  if  and  is the Jordan matrix associated with
eigenvalue  for . Let  be the dimension of 's largest main submatrix that is nonsingular, cyclic, and does not have
a Jordan block with dimension greater than two. Then, the system (2) has an -dimensional nearly-controllable subspace for .
\end{theorem}

\begin{proof}
Choose any main submatrix  of  that is
nonsingular, cyclic, and does not have a Jordan block with dimension greater
than two, where  is required to correspond to the first or the second
row (and column) of a Jordan block in  in view of Lemma 9 for  and if some  corresponds to the second row (and column) of a
Jordan block, then  corresponds to the first row (and column) of
the same one. Consider the system (2) onThen, the system can be rewritten aswhereandDue to the chosen , subsystem (26) is nearly
controllable by Theorem 2. Thus, the system (2) is controllable on the -dimensional regionwhereThat is, region (27) is an -dimensional nearly-controllable subspace of
the system (2).
\end{proof}



\begin{remark}
From the proof of Theorem 10, in order to obtain a nearly-controllable
subspace of the system (2), the main submatrix  of  that the nearly-controllable subspace corresponds to must be chosen to be
nonsingular, cyclic, and have no Jordan block with dimension greater than
two, where  is required to correspond to the first or the second row
(and column) of a Jordan block in  for  and if some 
corresponds to the second row (and column) of a Jordan block, then 
corresponds to the first row (and column) of the same one.
\end{remark}



\begin{definition}
\ in Theorem 10 is called the near-controllability index of the system
(2).
\end{definition}



If , then the system (2) is nearly controllable. Otherwise, we can have
the nearly-controllable subspaces with dimension from  to . In
particular, the state transition on every nearly-controllable subspace can
also be achieved through Algorithm 5. Additionally, even for the nearly
controllable system (2), it has nearly-controllable subspaces in the removed
region of (15) (i.e. ) and (15) can be regarded as an -dimensional nearly-controllable subspace. An example is given in the next
section.



\section{Examples}

\begin{example}
Consider the systemwhere  and . Given  . Find the control inputs such that  is transferred to .
\end{example}



We now apply Algorithm 5 to compute the required control inputs. \textbf{Step 1}: let  whereThen,and  . From Theorem 2, system (29) is nearly controllable (so is
system (28)) and is controllable onthat consists of four open orthants. \textbf{Step 2}: since 
belong to different orthants, let . It followswhich is in the orthant that  belongs to. \textbf{Step 3}: obtain
from (14) the transition matrix\textbf{Step 4}: choose  in view of (9).
\textbf{Step 5}: choose . Then,\textbf{Step 6}: by (17)ConsiderBy Matlab the root loci of  are shown in the
following figure,
\begin{figure}[h!]
\begin{center}
\includegraphics[scale=0.808,trim=0 0 0 0]{Fig.1.eps} \
v_{1}\approx -0.770,\text{ }v_{2}\approx -0.643,\text{ }v_{3}\approx -0.452,\text{ }v_{4}\approx 0.250,\text{ }v_{5}\approx 0.439,\text{ }v_{6}\approx
0.612,\text{ }v_{7}\approx 6.650.

x\left( k+1\right) =\left( I+u\left( k\right) B\right) x\left( k\right)
=\left( I+u\left( k\right) \left[
\begin{array}{cccc}
\lambda _{1} & 1 & 0 & 0 \\
0 & \lambda _{1} & 1 & 0 \\
0 & 0 & \lambda _{1} & 0 \\
0 & 0 & 0 & \lambda _{2}\end{array}\right] \right) x\left( k\right)

&&\left\{ \xi =\left[
\begin{array}{cccc}
\xi _{1} & \xi _{2} & 0 & \xi _{4}\end{array}\right] ^{T}\right\} \setminus \left\{ \xi \left\vert \text{ }\left\vert
\begin{array}{ccc}
\xi _{1,2,4} & B_{1,2,4}\xi _{1,2,4} & B_{1,2,4}^{2}\xi _{1,2,4}\end{array}\right\vert =0\right. \right\}  \notag \\
&=&\left\{ \xi =\left[
\begin{array}{cccc}
\xi _{1} & \xi _{2} & 0 & \xi _{4}\end{array}\right] ^{T}\left\vert \text{ }\xi _{2}\xi _{4}\neq 0\right. \right\}

\left\{ \xi =\left[
\begin{array}{cccc}
0 & \xi _{2} & \xi _{3} & \xi _{4}\end{array}\right] ^{T}\left\vert \text{ }\xi _{3}\xi _{4}\neq 0\right. \right\}

\left\{ \xi =\left[
\begin{array}{cccc}
\xi _{1} & \xi _{2} & 0 & 0\end{array}\right] ^{T}\text{ }\left\vert \text{ }\xi _{2}\neq 0\right. \right\} ,\text{
}\left\{ \xi =\left[
\begin{array}{cccc}
\xi _{1} & 0 & 0 & \xi _{4}\end{array}\right] ^{T}\text{ }\left\vert \text{ }\xi _{1}\xi _{4}\neq 0\right. \right\}

\left\{ \xi =\left[
\begin{array}{cccc}
\xi _{1} & 0 & 0 & 0\end{array}\right] ^{T}\text{ }\left\vert \text{ }\xi _{1}\neq 0\right. \right\} ,\text{
}\left\{ \xi =\left[
\begin{array}{cccc}
0 & 0 & 0 & \xi _{4}\end{array}\right] ^{T}\text{ }\left\vert \text{ }\xi _{4}\neq 0\right. \right\}

C=\left[
\begin{array}{ccc}
\lambda _{1}^{2m+2} & \cdots & \lambda _{1} \\
\left( 2m+2\right) \lambda _{1}^{2m+1} & \cdots & 1 \\
\vdots & \vdots & \vdots \\
\lambda _{m}^{2m+2} & \cdots & \lambda _{m} \\
\left( 2m+2\right) \lambda _{m}^{2m+1} & \cdots & 1 \\
\lambda _{m+1}^{2m+2} & \cdots & \lambda _{m+1} \\
\lambda _{m+2}^{2m+2} & \cdots & \lambda _{m+2}\end{array}\right] ,\text{ }d=\left[
\begin{array}{c}
-\lambda _{1}^{2m+3} \\
-\left( 2m+3\right) \lambda _{1}^{2m+2} \\
\vdots \\
-\lambda _{m}^{2m+3} \\
-\left( 2m+3\right) \lambda _{m}^{2m+2} \\
-\lambda _{m+1}^{2m+3} \\
-\lambda _{m+2}^{2m+3}\end{array}\right]

Cz=d

z=C^{-1}d=\left[
\begin{array}{c}
\left( -1\right) \left( 2\overset{m}{\underset{i=1}{\sum }}\lambda
_{i}+\lambda _{m+1}+\lambda _{m+2}\right) \\
\overset{}{\underset{}{\vdots }} \\
\left( -1\right) ^{2m+2}\lambda _{m+1}\lambda _{m+2}\overset{m}{\underset{i=1}{\prod }}\lambda _{i}^{2}\end{array}\right] .  \notag

\lambda _{i}^{2m+3}+z_{1}\lambda _{i}^{2m+2}+\cdots +z_{2m+2}\lambda _{i}=0

\left( 2m+3\right) \lambda _{i}^{2m+2}+z_{1}\left( 2m+2\right) \lambda
_{i}^{2m+1}+\cdots +z_{2m+2}=0

\lambda _{i}^{2m+2}+z_{1}\lambda _{i}^{2m+1}+\cdots +z_{2m+2}=0

&&\left( 2m+3\right) \lambda _{i}^{2m+2}+z_{1}\left( 2m+2\right) \lambda
_{i}^{2m+1}+\cdots +z_{2m+2}-\left( \lambda _{i}^{2m+2}+z_{1}\lambda
_{i}^{2m+1}+\cdots +z_{2m+2}\right) \\
&=&\left( 2m+2\right) \lambda _{i}^{2m+2}+z_{1}\left( 2m+1\right) \lambda
_{i}^{2m+1}+\cdots +z_{2m+1}\lambda _{i}=0 \\
&\Rightarrow &\left( 2m+2\right) \lambda _{i}^{2m+1}+z_{1}\left( 2m+1\right)
\lambda _{i}^{2m}+\cdots +z_{2m+1}=0

s^{2m+2}+z_{1}s^{2m+1}+\cdots +z_{2m+2}=0\text{,}

z_{1} &=&\left( -1\right) \left( 2\overset{m}{\underset{i=1}{\sum }}\lambda
_{i}+\lambda _{m+1}+\lambda _{m+2}\right) , \\
&&\vdots \\
z_{2m+2} &=&\left( -1\right) ^{2m+2}\lambda _{m+1}\lambda _{m+2}\overset{m}{\underset{i=1}{\prod }}\lambda _{i}^{2}.

B=\left[
\begin{array}{cccc}
\lambda & 1 &  &  \\
& \lambda & \ddots &  \\
&  & \ddots & 1 \\
&  &  & \lambda \end{array}\right] \in
\mathbb{R}
^{N\times N}

\left( \frac{1}{u\left( L\right) }I+B\right) \left( \frac{1}{u\left(
L-1\right) }I+B\right) \cdots \left( \frac{1}{u\left( 1\right) }I+B\right)
\left( \frac{1}{u\left( 0\right) }I+B\right) =\overset{L}{\underset{k=0}{\prod }}\frac{1}{u\left( k\right) }I.

B^{L+1}+\overset{L}{\underset{k=0}{\sum }}\frac{1}{u\left( k\right) }B^{L}+\cdots +\overset{L}{\underset{k=0}{\sum }}\frac{u\left( k\right) }{\overset{2m+2}{\underset{j=0}{\prod }}u\left( j\right) }B &=&0\Rightarrow
\notag \\
B^{L}+\overset{L}{\underset{k=0}{\sum }}\frac{1}{u\left( k\right) }B^{L-1}+\cdots +\overset{L}{\underset{k=0}{\sum }}\frac{u\left( k\right) }{\overset{2m+2}{\underset{j=0}{\prod }}u\left( j\right) }I &=&0,

\left( w\left( L\right) I+\bar{B}\right) \left( w\left( L-1\right) I+\bar{B}\right) \cdots \left( w\left( 1\right) I+\bar{B}\right) \left( w\left(
0\right) I+\bar{B}\right) =\overset{L}{\underset{k=0}{\prod }}\left( w\left(
k\right) -\lambda \right) I

\bar{B}=\left[
\begin{array}{cccc}
0 & 1 &  &  \\
& 0 & \ddots &  \\
&  & \ddots & 1 \\
&  &  & 0\end{array}\right] .  \notag

& \bar{B}^{L+1}+\overset{L}{\underset{k=0}{\sum }}w\left( k\right) \bar{B}^{L}+\cdots +\overset{L}{\underset{k=0}{\sum }}\frac{\overset{L}{\underset{j=0}{\prod }}w\left( j\right) }{w\left( k\right) }\bar{B}+\overset{L}{\underset{k=0}{\prod }}w\left( k\right) I  \notag \\
& =\sum w\left( k_{0}\right) w\left( k_{1}\right) \cdots w\left(
k_{L-N+1}\right) \bar{B}^{N-1}+\sum w\left( k_{0}\right) w\left(
k_{1}\right) \cdots w\left( k_{L-N+2}\right) \bar{B}^{N-2}+  \notag \\
& \cdots +\overset{L}{\underset{k=0}{\sum }}\frac{\overset{L}{\underset{j=0}{\prod }}w\left( j\right) }{w\left( k\right) }\bar{B}+\overset{L}{\underset{k=0}{\prod }}w\left( k\right) I  \notag \\
& =\overset{L}{\underset{k=0}{\prod }}\left( w\left( k\right) -\lambda
\right) I

\sum w\left( k_{0}\right) w\left( k_{1}\right) \cdots w\left(
k_{L-N+1}\right) &=&0,  \notag \\
&&\vdots  \notag \\
\overset{L}{\underset{k=0}{\sum }}\frac{\overset{L}{\underset{j=0}{\prod }}w\left( j\right) }{w\left( k\right) } &=&0,  \notag \\
\overset{L}{\underset{k=0}{\prod }}w\left( k\right) &=&\overset{L}{\underset{k=0}{\prod }}\left( w\left( k\right) -\lambda \right) .

\overset{L}{\underset{k=0}{\sum }}w\left( k\right) &=&\left( -1\right) c_{1},
\notag \\
&&\underset{}{\vdots }  \notag \\
\sum w\left( k_{0}\right) w\left( k_{1}\right) \cdots w\left( k_{L-N}\right)
&=&\left( -1\right) ^{L-N+1}c_{L-N+1}.

\overset{L}{\underset{k=0}{\prod }}\left( w\left( k\right) -\lambda \right)
&=&\overset{L}{\underset{k=0}{\prod }}w\left( k\right) +\overset{L}{\underset {k=0}{\sum }}\frac{\overset{L}{\underset{j=0}{\prod }}w\left( j\right) }{w\left( k\right) }\left( -\lambda \right) +\cdots +\overset{L}{\underset{k=0}{\sum }}w\left( k\right) \left( -\lambda \right) ^{L}+\left( -\lambda
\right) ^{L+1} \\
&=&\overset{L}{\underset{k=0}{\prod }}w\left( k\right) +\left( -1\right)
^{L-N+1}c_{L-N+1}\left( -\lambda \right) ^{N}+\cdots +\left( -1\right)
c_{1}\left( -\lambda \right) ^{L}+\left( -\lambda \right) ^{L+1} \\
&=&\overset{L}{\underset{k=0}{\prod }}w\left( k\right) +\left( -1\right)
^{L+1}\left( c_{L-N+1}\lambda ^{N}+\cdots +c_{1}\lambda ^{L}+\lambda
^{L+1}\right) =\overset{L}{\underset{k=0}{\prod }}w\left( k\right) .

c_{L-N+1}\lambda ^{N}+\cdots +c_{1}\lambda ^{L}+\lambda ^{L+1}
&=&0\Rightarrow  \notag \\
c_{L-N+1} &=&-\left( \lambda ^{L-N+1}+c_{1}\lambda ^{L-N}+\cdots
+c_{L-N}\lambda \right) .

\overset{L}{\underset{k=0}{\prod }}w\left( k\right) =\left( -1\right)
^{L+1}c.

s^{L+1}+c_{1}s^{L}+\cdots +c_{L-N}s^{N+1}-\left( \lambda
^{L-N+1}+c_{1}\lambda ^{L-N}+\cdots +c_{L-N}\lambda \right) s^{N}+c=0.

B=\left[
\begin{array}{cc}
\ast & \ast \\
\mathbf{0} & 0\end{array}\right]

\left[
\begin{array}{ccccc}
\ast & \ast &  &  &  \\
& \lambda &  &  &  \\
&  & \ast & \ast &  \\
&  &  & \lambda &  \\
&  &  &  & \ast \end{array}\right]

&&\overset{L}{\underset{k=0}{\prod }}\left( I+u\left( k\right) B\right) \left[
\begin{array}{ccccc}
\cdots & \xi _{i} & \cdots & \xi _{j} & \cdots \end{array}\right] ^{T} \\
&=&\left[
\begin{array}{ccccc}
\ast & \ast &  &  &  \\
& \overset{L}{\underset{k=0}{\prod }}\left( 1+u\left( k\right) \lambda
\right) &  &  &  \\
&  & \ast & \ast &  \\
&  &  & \overset{L}{\underset{k=0}{\prod }}\left( 1+u\left( k\right) \lambda
\right) &  \\
&  &  &  & \ast \end{array}\right] \left[
\begin{array}{c}
\vdots \\
\xi _{i} \\
\vdots \\
\xi _{j} \\
\vdots \end{array}\right] \\
&=&\left[
\begin{array}{ccccc}
\cdots & \overset{L}{\underset{k=0}{\prod }}\left( 1+u\left( k\right)
\lambda \right) \xi _{i} & \cdots & \overset{L}{\underset{k=0}{\prod }}\left( 1+u\left( k\right) \lambda \right) \xi _{j} & \cdots \end{array}\right] ^{T}\triangleq \left[
\begin{array}{ccccc}
\cdots & \eta _{i} & \cdots & \eta _{j} & \cdots \end{array}\right] ^{T},

\left\{ \xi =\left[
\begin{array}{cccc}
\xi _{1} & 0 & \cdots & 0\end{array}\right] ^{T}\left\vert \text{ }\xi _{1}\neq 0\right. \right\} ,\text{ }\left\{ \xi =\left[
\begin{array}{ccccc}
\xi _{1} & \xi _{2} & 0 & \cdots & 0\end{array}\right] ^{T}\left\vert \text{ }\xi _{2}\neq 0\right. \right\}

\bar{\xi}=\left[
\begin{array}{cccccc}
\xi _{1} & \cdots & \xi _{j} & 0 & \cdots & 0\end{array}\right] ^{T}

\prod\limits_{k=0}^{L_{1}+L_{2}+1}\left( I+u\left( k\right) B\right) \bar{\xi }=\prod\limits_{k=L_{1}+1}^{L_{1}+L_{2}+1}\left( I+u\left( k\right) B\right)
\eta =\prod\limits_{k=0}^{L_{2}}\left( I+v\left( k\right) B\right) \eta =\bar{\xi}

\prod\limits_{k=0}^{L_{1}+L_{2}+1}\left( I+u\left( k\right) B\right) \left[
\begin{array}{cccccc}
\xi _{1} & \cdots & \xi _{j} & 0 & \cdots & 0\end{array}\right] ^{T}=\left[
\begin{array}{cccccc}
\xi _{1} & \cdots & \xi _{j} & 0 & \cdots & 0\end{array}\right] ^{T}.

\prod\limits_{k=0}^{L_{1}+L_{2}+1}\left( I+u\left( k\right) B_{1,\ldots
,j}\right) \left[
\begin{array}{ccc}
\xi _{1} & \cdots & \xi _{j}\end{array}\right] ^{T}=\left[
\begin{array}{cccc}
\Pi & \Sigma _{1} & \cdots & \Sigma _{j-1} \\
& \ddots & \ddots & \vdots \\
&  & \Pi & \Sigma _{1} \\
&  &  & \Pi \end{array}\right] \left[
\begin{array}{c}
\xi _{1} \\
\vdots \\
\xi _{j}\end{array}\right] =\left[
\begin{array}{c}
\xi _{1} \\
\vdots \\
\xi _{j}\end{array}\right]

\Pi =\prod\limits_{k=0}^{L_{1}+L_{2}+1}\left( 1+u\left( k\right) \lambda
\right) ,\text{ }\Sigma _{i}=\frac{1}{i!}\frac{d^{i}\prod \limits_{k=0}^{L_{1}+L_{2}+1}\left( 1+u\left( k\right) y\right) }{dy^{i}}\left\vert
\begin{array}{c}
\\
_{y=\lambda }\end{array}\right. \text{ for }i=1,\ldots ,j-1.

\Pi \xi _{1}+\Sigma _{1}\xi _{2}+\cdots +\Sigma _{j-1}\xi _{j} &=&\xi _{1},
\\
&&\vdots \\
\Pi \xi _{j-1}+\Sigma _{1}\xi _{j} &=&\xi _{j-1}, \\
\Pi \xi _{j} &=&\xi _{j}.

\Pi =1,\text{ }\Sigma _{1}=0,\ldots ,\Sigma _{j-1}=0.

\prod\limits_{k=0}^{L_{1}+L_{2}+1}\left( I+u\left( k\right) B_{1,\ldots
,j}\right) =I.
However, since , there do not exist such nonzero control inputs to
satisfy (42) according to Lemma 1. Therefore, system (25) can only have the
states of which the latter  entries are all zero in its
controllable regions that contain more than one state.
\end{proof}

\end{document}
