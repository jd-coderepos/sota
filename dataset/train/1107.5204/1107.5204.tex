\documentclass[letterpaper,11pt]{article}

\usepackage[T1]{fontenc}
\usepackage[letterpaper,top=1in,bottom=1in,left=1in,right=1in]{geometry}



\usepackage[cmex10]{amsmath}
\usepackage{amsthm}
\usepackage{amssymb,latexsym}
\usepackage[usenames]{color}
\usepackage{graphicx}
\usepackage{psfrag}
\usepackage{xspace}
\usepackage{comment}
\usepackage{multirow}




\usepackage{comment}
\usepackage{xspace}


\pagestyle{plain}

\newcommand{\sep}{,\xspace}

\newcommand{\incircle}{\textsf{Incircle}\xspace}
\newcommand{\orthoff}{ortho-45\xspace}
\newcommand{\vor}{Voronoi\xspace}
\newcommand{\rfl}{\mathcal{R}\xspace}
\newcommand{\rfx}[1]{\mathcal{R}(#1)\xspace}
\newcommand{\EE}{\mathbb{E}\xspace}

\newcommand{\ppp}{\xspace}
\newcommand{\pps}{\xspace}
\newcommand{\pss}{\xspace}
\newcommand{\sss}{\xspace}

\newcommand{\pppp}{\xspace}
\newcommand{\ppsp}{\xspace}
\newcommand{\pssp}{\xspace}
\newcommand{\sssp}{\xspace}

\newcommand{\ppps}{\xspace}
\newcommand{\ppss}{\xspace}
\newcommand{\psss}{\xspace}
\newcommand{\ssss}{\xspace}

\newenvironment{red}{\color{red}}{\color{black}}
\newenvironment{smallvmatrix}{\left|\begin{smallmatrix}}{\end{smallmatrix}\right|}

\newtheorem{theorem}{Theorem}
\newtheorem{prop}[theorem]{Proposition}
\newtheorem{lemma}[theorem]{Lemma}

\begin{document}
\title{Analysis of the \textsf{Incircle} predicate for the Euclidean\\
  Voronoi diagram of axes-aligned line segments}




\author{Manos N. Kamarianakis\hfil{}
Menelaos I. Karavelas\5pt]
\it{}Department of Applied Mathematics,
\it{}University of Crete\\
\it{}GR-714 09 Heraklion, Greece\5pt]
{\texttt{mkaravel@tem.uoc.gr, manos@tem.uoc.gr}}\5pt]
    \begin{tabular}{|c|c|c|c|c|}\hline
      &\ppps & \ppss & \psss & \ssss
      \\\hline\hline
      General \cite{b-ecvdl-96}&8&24&32&40\\
      Axes-aligned \cite{b-ecvdl-96}&6&12&4&2\\
      Axes-aligned [this paper]&6&6&4&2\\\hline
    \end{tabular}
  \end{center}
  \caption{Maximum algebraic degrees for the eight types of the
    \incircle predicate according to: \cite{b-ecvdl-96} for the general
    and the axes-aligned segments case, and
    this paper. Top/Bottom table: the query object is a
    point/segment.}
  \label{tbl:algdeg}
\end{table}


In Sections \ref{sec:ppp}-\ref{sec:pss} we analyze, in more or less detail,
all eight possible configurations for the \incircle predicate, and
show how we can reduce the algebraic degrees for the \pps case from 8
and 12 to 6. This is done by means of three key ingredients:
(1) we formulate the \incircle predicate as an algebraic problem of
the following form: we compute a linear polynomial  and
a quadratic polynomial , such that the result of
the \incircle predicate is the sign of  evaluated at a specific
root of ,
(2) for the \pps and \pss cases, we express the \incircle
predicate as a difference of distances, instead of as a difference of
squares of distances, and
(3) we reduce the \ppsp case to the \ppps case.
Regarding the first ingredient, we describe in the following
subsection how we can do better than finding the appropriate root of
 and substitute it in  (this is essentially what is done
in \cite{b-ecvdl-96}). Regarding the second and third ingredients we
postpone the discussion until the corresponding sections.
There is one final tool that we will be very useful in order to
simplify our analysis: in order to reduce our case analysis we make
extensive use of the reflection transformation through the line
; see Subsection \ref{sec:reflection} for the details.


\subsection{Evaluation of the sign of  at a specific
  root of }\label{sec:lqsolve}

Let  and  be a linear and a
quadratic polynomial, respectively, such that  has non-negative
discriminant. Let the algebraic degrees of , , , 
and  be , , , , and
, respectively. We are interested in the sign of
, where  is one of the two roots 
of . In our analysis below we will assume, without loss of
generality that .

The obvious approach is to solve for  and substitute into the
equation of . Let  be the discriminant of 
. Then , which in turn
yields . Computing
the sign of  is dominated, with respect to the algebraic
degree of the quantities involved, by the computation of the 
sign of . Evaluating the sign
of this quantity amounts to evaluating the sign of
, which is of algebraic degree
.

Observe now that evaluating the sign of  is equivalent to
evaluating the sign of , and possibly the sign of
, where  stands for the root of
.
Indeed, if , we immediately know that 
if , or that  if . If
, we need to additionally evaluate the sign of
. If , we know that
, which implies that , whereas if
, we have , which gives
. Finally, if , we still need to evaluate the
sign of . If , then ,
and thus  if , and  if
. Similarly, if , then ,
and thus  if , and  if
. There is one last case to consider:
. Given that , this can happen only if
, in which case we deduce .
Since , evaluating
the sign of  means evaluating the sign of an algebraic
expression of degree . Moreover,
; hence, evaluating the sign of
 reduces to evaluating the signs of  and
, the degrees of which are  and ,
respectively. 

Notice that the latter among the two approaches described above is never
worse than the first one; in fact, if  it gives a lower
maximum algebraic degree. We summarize this observation in the
following lemma.

\begin{lemma}\label{lemma:lqsolve}
  Let , , and ,
  , be a linear and quadratic polynomial, respectively,
  such that the discriminant of  is non-negative. If the
  algebraic degrees of , , ,  and  be
  , , , , and
  , respectively, then we can evaluate the sign of
  , where  is a specific root of , using
  expressions of maximum algebraic degree .
\end{lemma}

\begin{comment}
\item Resultants \\
Define , we can evaluate  by evaluating 
. We get that  
are the roots of , where

According to Vieta's formulas, we get that
,
and . Assume , 
the signs of  and  are sufficient to evaluate the 
sign of . 

To distinguish whether  and  or the opposite, 
we evaluate

hence ; If  we 
get , whereas if  we get , 
for .

This method involves examining quantities of maximum 
algebraic degree .
\end{enumerate}
\end{comment}


\subsection{Reflection transformation}
\label{sec:reflection}

Let  denote the reflection transformation
through the line .  maps a point  to the
point . 
The reflection transformation preserves circles and line segments and
is inclusion preserving. This immediately implies that, given a \vor
circle  defined by three sites ,  and ,
and a query point ,  lies inside, on, or outside the \vor circle
 if and only if  lies inside, on, or outside
the \vor circle  (cf. Fig. 
\ref{fig:transform} for the case where  and  are points and
 is a segment). Hence,
.
Notice that reflection reverses the orientation of a circle, which is
why we consider the \vor circle 
instead of the \vor circle .
The same principle applies in the case where the query object is a
line segment :
.

As a final note, the reflection transformation  maps an -axis
parallel segment to a -axis parallel segment, and vice versa. This
property will be used, in the sections that follow, to reduce the
analysis and computation of the \incircle predicate, where one of the
's is -axis parallel, to the case where one of the 's is
-axis parallel.


\begin{figure}[!h]
  \begin{center}
    \includegraphics[width=0.7\textwidth]{transform} 
  \end{center}
  \caption{ is equivalent to
    , where
     stands for the image of  under the reflection
    transformation through the line .}
  \label{fig:transform}
\end{figure} 



\section{The \ppp case}\label{sec:ppp}

As of this section, we discuss and analyze the \incircle predicate for
each of the four possible configurations for the \vor circle. We start
with the case where the \vor circle is defined by three points ,
 and .

\subsection{The query object is a point}\label{sec:pppp}
This is the well known \incircle predicate for four points , ,
 and , where  is the query point, and it amounts to the
computation of the sign of the determinant

Its algebraic degree is clearly 4. 




\subsection{The query object is a segment}\label{sec:ppps}

Let  be the query segment.
In this case, we must first check that relative position of  and
 with respect to  using , . If at least one of  and  lies inside , we
clearly have . 

Otherwise, we have to examine if the segment  intersects with
. This is equivalent to point-locating the points  and
 in the arrangement of the lines ,  and
 if  is -axis parallel or, ,  and  if 
 is -axis parallel, where ,  (resp.,
 ,) are the extremal points of  in the
direction of the -axis (resp., -axis). In fact, the case where
the segment  is -axis parallel can be reduced to the case where
the query segment is -axis parallel by noting that 

(see Section \ref{sec:reflection}). We will
therefore restrict our analysis to the case where  is -axis
parallel. 

\begin{figure}[b]
  \begin{center}
    \includegraphics[width=0.7\textwidth]{china4} 
  \end{center}
  \caption{Relative positions of the -axis aligned query segment
     with respect to the lines , , .}
  \label{fig:ppps}
\end{figure}

We first determine if  lies outside the band delimited by the lines
 and ; in this case we immediately get
. Otherwise, if  lies inside the band
(resp.,  lies on either  or ), we check the
relative positions of  and  against the line ; the
segment  intersects (resp., is tangent to)  if and only
if  and  lie on different sides of the line . 

In order to determine the relative position of  with respect to the lines 
 and , we will evaluate a quadratic -polynomial 
that vanishes at  and : let 
be this polynomial.
Having computed this polynomial,  if and only
if ,
 if and only if ,
and, finally,  if and only if .

To evaluate such a polynomial, we first observe that every point 
on  satisfies . Expanding the four-point \incircle determinant in terms of , we end
up with a quadratic polynomial  for
, where


For a fixed value  of , the roots of 
are the points of intersection of the line  with the \vor
circle .  has no real roots if
, has two distinct roots if
 and has a double root if
. In the last case, the discriminant
 of  has to
vanish. Now consider the discriminant as a polynomial of . Clearly,
 is a quadratic -polynomial, with a strictly
negative, since the points ,  and  are not
collinear. Moreover,  vanishes for
, hence it may serve as the quadratic
polynomial  we were aiming for.
More specifically,  where, 
, , , and


In an analogous manner, we can evaluate a quadratic -polynomial
that vanishes at  and , which we call . More
precisely, , where
,  and .
In order to determine the relative position of   and  with
respect to the line , we use the fact that
. Hence, using the
fact that , checking on which side of  lies point ,
for , amounts to determining the sign
.

The algebraic degrees of , , , , and  are 4,
3, 2, 3, and 3, respectively. Therefore, the algebraic degrees of
, , , , , and  are 4, 5, 6, 4, 5, and 6,
respectively. This implies that the algebraic degree of  is 6,
while the algebraic degree of , , is 5. We,
thus, conclude that we can answer the \incircle predicate in the \ppps
case by evaluating expressions of maximum algebraic degree 6.




\section{The \sss case}\label{sec:sss}

In this section we consider the case where the \vor circle is defined
by three axis-aligned segments ,  and . In order for the
circle  to be well defined, exactly two of these segments
must parallel to each other, while the third perpendicular to the
other two.
Given that , we
can assume without loss of generality that the first two segments are
parallel to each other, and thus the third is perpendicular to the
first two. Hence we only have to consider two cases:
(1) ,  are -axis parallel and  is -axis parallel,
and
(2) ,  are -axis parallel and  is -axis parallel.
In fact the second case can be reduced to the first one by noting that

(see Section \ref{sec:reflection}).
We shall, therefore, assume that ,  are -axis parallel and 
 is -axis parallel.

\begin{comment}
\begin{figure}[b]
  \begin{center}
    \includegraphics[width=0.17\textwidth]{china12}
  \end{center}
  \caption{The Voronoi circle defined by the axes-aligned segments
    ,  and .}
  \label{fig:sss}
\end{figure}  
\end{comment}

\subsection{The query object is a point}\label{sec:sssp}
  
Let  be the query point. Since the center  of  lies
on the bisector of the lines  and , and the
radius  of the circle is the distance of  from either
 or  (i.e., half the distance of the two lines),
we have 


To answer the \incircle predicate for , we first examine if  and
 lie on the same side with respect to the lines ,
 and . If this is not the case, we immediately
conclude that . Otherwise we must compare 
the distance  of  from  against the \vor radius
. More precisely: ,
where ,
which is an algebraic expression of degree 2 in the input
quantities. Given that the sideness tests for  against the lines
,  and  are of degree 1, we conclude
that answering the \incircle predicate in the \sssp case amounts to
computing the signs of expressions of algebraic degree at most 2.



\subsection{The query object is a segment}\label{sec:ssss}

Let  be the query segment. We first determine if the endpoints 
and/or  of  lie inside , in which case we
immediately get . Otherwise, we must
consider the orientation of  and make the appropriate checks.

Assume first that  is -axis parallel. We first check if  is
inside the band  delimited by the lines  and
. If  lies outside , we immediately get that
. Otherwise, we have to determine the
relative positions of  and  with respect to the line ,
where , by evaluating the signs  
of  and . If  lies inside  (resp., on the
boundary  of ,  intersects (resp., is tangent to)
, if and only if  and  lie on different sides of
the line , i.e., if and only if
. Determining if  lies inside  amounts
to computing the signs of  and , which are degree 1
quantities. The quantities  and  are also of degree
1, which implies that we can answer the \incircle predicate in this
case using quantities of algebraic degree up to 2 (the algebraic
degree needed to evaluate , 
dominates the degrees of all other quantities to be evaluated).

Consider now the case where  is -axis parallel. We first need
to check if the line , intersects with . To do
this we need to evaluate the sign of quantity , where
 is given by \eqref{equ:lll-vc}. Computing the signs of
 and , we may express  as a
polynomial expression in the input quantities; its algebraic degree is,
clearly, 1. If ,  does not intersect
, and we immediately get .
Otherwise, if  (resp., )
 either intersects with (resp. either is tangent to) the
\vor circle or does not intersect the \vor circle at all. To
distinguish between these two cases we have to determine if the points
 and  lie on different sides of the line : the segment
 intersects with (resp., is tangent to) the \vor circle
 if and only if . Since
 (see rel. \eqref{equ:lll-vc}), determining
the signs  and  amounts to computing the
sign of quantities of algebraic degree 1. As in the case where  is
-axis parallel, the algebraic degree for evaluating the \incircle
predicate is dominated by the algebraic degree for evaluating
, , which is 2.




\section{The generic approach for the evaluation of the \incircle
  predicate in the \pps and \pss cases}\label{sec:generic}

In this section we present our approach for evaluating the
\incircle predicate in a generic manner. The approach presented is
applicable when the \vor circle is defined by at least one point and
at least one segment, i.e., we can treat the cases \pps and \pss.

Let  be the center of the Voronoi circle defined by the sites
, , , that touches the sites ,  and  in
that order when we traverse the \vor circle in the counterclockwise
sense.
As already stated, we want to evaluate the \incircle predicate for a
query point or a query line segment with respect to this circle.
To do this we compute a quadratic polynomial  that 
vanishes at , while using geometric considerations and the
requirement on the orientation of the \vor circle, we can determine
which of the roots  of  corresponds to . 
Regarding , the situation is entirely symmetric. We also
compute a quadratic polynomial  that vanishes at  and, as
for , we can determine which of the two roots  of
 corresponds to . Moreover, in all cases  and 
are linearly dependent, which means that we may express  as
, where ,
 and  are polynomials in the input quantities.

\subsection{The query site is a point}\label{sec:pxsp}

Let  be the query point. Since at least one of ,  and
 is a point , determining the \incircle predicate amounts to
evaluating the sign of the quantity
.
Replacing , using the relation
, and gathering
the terms of , we get
, where
 and
.
If , the we can immediately evaluate the \incircle predicate by
evaluating the signs of  and . Otherwise, deciding the
\incircle predicate reduces to evaluating the sign of , as well
as the sign of , evaluated at a specific known root of a
quadratic polynomial  (it is the root of 
that corresponds to ). This is exactly the problem we analyzed in
Subsection \ref{sec:lqsolve}.

\begin{comment}
with roots  (this happens in the
\pps and \pss cases), evaluating the sign
of  amounts to evaluating the signs of  and
, where .
If , we have
, otherwise, if 
(resp., ), we have to check the sign of
. 
If , we get 
(resp. ), while if , we get
 (resp. ).  

The signs of  and  are equal to

and 

\end{comment}

Let us now analyze the algebraic degrees of the expressions above. As
we will see in the upcoming sections (see Sections \ref{sec:pps} and
\ref{sec:pss}),  is a homogeneous polynomial in terms of its
algebraic degree. Letting  the algebraic degree of ,
the algebraic degrees of  and  become  and
. Let also  be the algebraic degree of
. In our context, the algebraic degree of  is
always one more that the degree of , i.e., it is
, whereas the algebraic degree of  is always
equal to that of . This implies that the algebraic
degrees of  and  are  and
, respectively. Applying Lemma \ref{lemma:lqsolve}, we
conclude that we can resolve resolve the \incircle predicate using
expressions of maximum algebraic degree
.

\begin{comment}
Hence, the 
algebraic degree of the quantity  is
. Similarly, the
algebraic degree of the quantity  is
. Since the algebraic
degree of  is always greater or equal
to that of , we conclude the following:

\begin{lemma}\label{incircle:xxxp}
  Answering the  predicate, when at least
  one of the sites ,  or  is a point, amounts to
  computing the sign of a polynomial expressions of algebraic degree
  at most .
\end{lemma}
\end{comment}

\subsection{The query site is a segment}\label{sec:pxss}

Let  be the query segment. The first step is to compute
 and, if needed,
. If at least one  and  lies inside
the \vor circle , we get
. Otherwise, we need to determine if the
line  intersects . If  does not
intersect the \vor circle, we have . If
 intersects the \vor circle we have to check if  and 
lie on the same or opposite sides of the line  that
goes through the \vor center  and is perpendicular to
. Notice that since  is axes-aligned, the line
 is either the line  or the line .
Since at least one of ,  and  is a segment ,
answering the \incircle predicate is equivalent to comparing the
distance of  from the line  to the segment : 
We can assume without loss of generality that  is
-axis parallel, since, otherwise we can reduce
 to
  (see Section
\ref{sec:reflection}), in which case  is -axis parallel.
Let us now examine and analyze the right-hand side difference of
\eqref{equ:incircle-xxss}.

Assume first that the segment  is -axis parallel. In this case the
equation of  is , and, hence,
. Recall that  is a specific root of a
quadratic polynomial . Therefore, determining the sign of
 reduces to evaluating the sign of  and .
Let  be this polynomial, and let ,
,  be the algebraic degrees of ,  and
, respectively (as for ,  is a homogeneous polynomial).
Consider now the case where  is -axis parallel. The equation of
 is , and, hence, . As in
the -axis parallel case,  is a specific known root of the 
quadratic polynomial , and determining the sign of 
amounts to evaluating the sign of  and . 
Last but not least, since the segment  is -axis parallel,
. As before, we can determine the sign of 
by evaluating the signs of  and .

Having made the above observations, we conclude that, if  is
-axis parallel,

where  and  are given in the following table.
\begin{center}
\begin{tabular}{|c|c|c|c|}\hline 
    &&&\\\hline\hline
    \multirow{2}{*}{}&&&\\\cline{2-4}
    &&&\\\hline
    \multirow{2}{*}{}&&&\\\cline{2-4}
    &&&\\\hline 
  \end{tabular}
\end{center}
Clearly, if  we have
. Otherwise,
given that  is a root of , evaluating
 can be done using the analysis in
Subsection \ref{sec:lqsolve}. Since the algebraic degrees of  and
 are 0 and 1, respectively, we deduce, by Lemma
\ref{lemma:lqsolve}, that we can resolve the \incircle predicate using
expressions of algebraic degree at most
.

For the case where  is -parallel we use the fact that
. Using this
linear dependence between  and , we get

where  and  are given in the following table.
\begin{center}
\begin{tabular}{|c|c|c|c|}\hline 
    &&&\\\hline\hline
    \multirow{2}{*}{}
    &&&\\\cline{2-4}
    &&&\\\hline
    \multirow{2}{*}{}
    &&&\\\cline{2-4}
    &&&\\\hline 
  \end{tabular}
\end{center}
If ,
.
Otherwise, given that  is a known root of ,
determining the sign of  can be done as in Subsection
\ref{sec:lqsolve}. As in the previous subsection, we let
 be the algebraic degree of  (and also of
), which means that the degree of  is
. Hence, the algebraic degree of  is
, whereas that of  is
. By Lemma
\ref{lemma:lqsolve}, in order to evaluate the sign  we
need to compute the signs of expressions of algebraic degree at most
.


\begin{comment}
Having made the above observations, we conclude that, if  is
-axis parallel,
,
where  and  are given in Table \ref{table:QSisXpar}.
Since  is a root of , determining the sign
of  is equivalent to determining the signs of
 and .
Evaluating  and  at , we get

Since the algebraic degrees of  and  are 0 and 1,
respectively, we deduce that determining the sign of
, or equivalently the sign of
, reduces to determining the sign of
a quantity of algebraic degree . Analogously, the
algebraic degree of  is , which suggests
that computing the sign of  requires the
computation of the signs of algebraic expressions of degrees up to
.

\begin{table}[ht]
\begin{center}
\begin{tabular}{|c|c|c|c|}
\hline 
 & 
& \multicolumn{2}{c|}{} \\ 
\hline 
\multirow{2}{*}{} &  &  &  \\ \cline{2-4}
&  &  &   \\ \hline
\multirow{2}{*}{} &   &  &  \\ \cline{2-4}
&  &  &  \\
\hline 
\end{tabular} 
\end{center}
\caption{Expression for , when  is
  -axis parallel.}\label{table:QSisXpar}
\end{table}


For the case where  is -parallel we use the fact that
. Using this
linear dependence between  and , we get
,
where  and  are given in Table \ref{table:QSisYpar}.
Given that  is a known root of ,
determining the sign of  is equivalent to determining the
signs of  and  ,  
Replacing  at  and  we get

Let  and  be the algebraic degrees of  and ,
respectively; as in the previous subsection, the algebraic degree of
 is . The algebraic degrees of  and  are
 and , respectively (strictly speaking, the algebraic
degree of  is , which becomes  given that
). Hence, the algebraic degrees of the quantities
 and  become
 and
, respectively. Summarizing:

\begin{lemma}\label{incircle:xxxl}
  Answering the  predicate, when at least
  one of the sites ,  or  is a segment, amounts to
  computing the sign of a polynomial expressions of algebraic degree
  at most .
\end{lemma}


\begin{table}[t]
\begin{center}
\begin{tabular}{|c|c|c|c|}
\hline 
 &  &
\multicolumn{2}{c|}{} \\ 
\hline 
\multirow{2}{*}{}
&  & &  \\ \cline{2-4}
&  & &   \\ \hline
\multirow{2}{*}{}
&   & &  \\ \cline{2-4}
&  & &  \\
\hline 
\end{tabular} 
\end{center}
\caption{Expression for , when  is
  -axis parallel.}\label{table:QSisYpar} 
\end{table}
\end{comment}


As we mentioned at the beginning of this subsection, if
, we need to check the
position of  and  with respect to the either line  (if
 is -axis parallel), or the line  (if  is -axis
parallel). To check the position of , , against the
line , we simply have to compute the signs of  and
. The algebraic degrees of these quantities are 
and , respectively. In a symmetric manner, to check the
position of , , against the line , we
simply have to compute the signs of  and . The
algebraic degrees of these quantities are  and
, respectively. Notice that in both cases for the
orientation of , the algebraic degree of the quantities whose sign
needs to be evaluated to resolve the \incircle predicate are never
greater than those computed above for evaluating
.
Recalling that, in order to evaluate , the
first step is to evaluate , and, if needed,
, we conclude that in order to evaluate the
\incircle predicate when the query object is a segment we need to
compute the sign of polynomial expressions of algebraic degree
at most .



\section{The \pps case}\label{sec:pps}

Let  and  be the two points and  be the segment defining the
\vor circle. Without loss of generality we may assume that  is
-axis parallel, since otherwise we can reduce 
to , as described in
Section \ref{sec:reflection}.

\subsection{The query object is a point}\label{sec:ppsp}

\begin{figure}[!b]
  \begin{center}
    \includegraphics[width=0.4\textwidth]{china13a}\hfil \includegraphics[width=0.4\textwidth]{china13b}\
\incircle(A,B,CD,Q)=\begin{cases}
    -\incircle(A,B,Q,CD),
    &\text{if\ }\textsf{Orientation}(A,B,Q)>0\\
    \textcolor{white}{-}\incircle(B,A,Q,CD),
    &\text{if\ }\textsf{Orientation}(A,B,Q)<0
  \end{cases}
5pt]
\begin{tabular}{|c|c|}
\hline
Relative positions of ,  & Root of  of interest\\
\hline \hline
 &  \\ \hline 
 &  \\ \hline 
\end{tabular}
\end{center}
The degrees of , , , ,  and 
are 1, 2, 3, 2, 3 and 4, respectively. Furthermore, the
degrees of ,  and  are 1, 2 and 1,
respectively. Applying the analysis in Subsection \ref{sec:pxss}
(where , ), we deduce that we
can answer the \incircle predicate using expressions of algebraic
maximum algebraic degree .

\begin{comment}
\begin{table}[t]
\begin{center}
\begin{tabular}{|c|c|}
\hline
Relative positions of ,  and  & Root of  of interest\\
\hline \hline
  & \\ \hline
  & \\ \hline
  & \\ \hline
 & \\ \hline
\end{tabular}
\end{center}
\caption{The possible relative positions of ,  and  and the
  corresponding root of  of interest (assuming
  ).}\label{Table:xPPS}
\end{table}

\begin{table}[t]
\begin{center}
\begin{tabular}{|c|c|}
\hline
Relative positions of ,  & Root of  of interest\\
\hline \hline
 &  \\ \hline 
 &  \\ \hline 
\end{tabular}
\end{center}
\caption{The possible relative positions of ,  and the
  corresponding root of  of interest (assuming
  ).}\label{Table:yPPS}
\end{table}
\end{comment}

For the special case , we easily get
 and , where
,
.
In this case, if  is -axis parallel, we need to determine the
sign of the quantity
, or, equivalently, the sign
of the quantity , which is of algebraic
degree 2. If  is -axis parallel, we need to evaluate the sign
of the quantity , or,
equivalently, the sign of the quantity
, which is also of algebraic degree
2. Given, that the algebraic degree for the \ppsp case
is 6 (see previous subsection), 
we conclude that we can answer the \incircle predicate in the \ppss
case by computing the signs of expressions of algebraic degree at most
6.





\section{The \pss case}\label{sec:pss}

\subsection{The query object is a point}\label{sec:pssp}

In this section we consider the case where the \vor circle is defined
by two segments, a point and the query object is a point. Let ,
 and  be the point and the two segments defining the \vor
circle and let  be the query point. Since each of ,  may be
-axis or -axis parallel we have four cases to consider:
(1)  and  are -axis parallel,
(2)  and  are -axis parallel,
(3)  is -axis parallel and  is -axis parallel, and
(4)  is -axis parallel and  is -axis parallel.
However, Cases (2) and (4) reduce to Cases (1) and (4), respectively,
by simply performing a reflection transformation through the line
 (see Section \ref{sec:reflection}). More precisely, in both
cases we have
. Thus,
for Case (2),  and  are -axis parallel, while,
for Case (4),  is -axis parallel and  is
-axis parallel. Therefore it suffices to consider Cases (1) and (3).
In what follows, we follow the generic procedure described in
Subsection \ref{sec:pxsp}, and refer to the notation introduced
there.

\begin{figure}[t]
  \begin{center}
    \begin{minipage}{0.442\textwidth}
\hspace*{1mm}\includegraphics[width=0.9\textwidth]{china14}\vspace*{2mm}
    \end{minipage}\begin{minipage}{0.44\textwidth}
\hspace*{5mm}\includegraphics[width=0.85\textwidth]{china7}
    \end{minipage}
  \end{center}
  \caption{Voronoi circle defined by the point  and the line
    segments  and . Left: ,  are -axis
    parallel. Right:  is -axis parallel and  is -axis
    parallel.}
  \label{fig:pss}
\end{figure}  




\subsubsection{ and  are -axis parallel}\label{sec:pssp-parallel}

We first notice that if  does not lie inside the band 
delimited by the  and , it cannot be inside the
\vor circle . This can be easily checked by evaluating the
signs of  and , which are quantities of algebraic
degree 1. Suppose now that  is inside  and notice that  has
to lie in  in order for the \vor circle  to exist. 

Let  be the center of . The -coordinate of  is,
trivially, , whereas the radius  of
the \vor circle is equal to . Given that
 is a point on , we have that . Using
the expressions for  and , we deduce that  is a root
of the polynomial , where
 and .
If  are the two roots of , the root that
corresponds to  is given in the table below (see also
Fig. \ref{fig:pss}(left)).
\begin{center}
\begin{tabular}{|c|c|}
\hline
Relative positions of  and  & Root of  of interest\\
\hline \hline
&\\\hline
&\\\hline
\end{tabular}
\end{center}
Moreover, in this case we have ,
 and . Therefore, the algebraic degrees
involved in the evaluation of the \incircle predicate are
. As per Subsection \ref{sec:pxsp},
the  predicate can be evaluated using algebraic
expressions of maximum degree .

\begin{comment}
\begin{table}[ht]
\begin{center}
\begin{tabular}{|c|c|}
\hline
Relative positions of  and  & Root of  of interest\\
\hline \hline
&\\\hline
&\\\hline
\end{tabular}
\end{center}
\caption{Root of  that corresponds to  as a function of the
  relative positions of  and ;  is assumed to be -axis
  parallel.}
\label{Table:PSSP}
\end{table}
\end{comment}

\subsubsection{ is -axis parallel and  is -axis parallel}
\label{sec:pssp-vertical}

The lines  and  subdivide the plane into
four quadrants , ,  and .
The bisector of  and  is the line
 with equation , whereas the bisector of
 and  is the line  with equation .

The center  of the \vor circle  lies on both the
bisector of  and , as well as on the parabola
that is at equal distance from  and ; the equation of
the latter is:

Assuming that  lies in , the bisector of 
and  is . Substituting  in terms of ,
using the equation of , we deduce that the -coordinate
 of  is a root of the quadratic polynomial
, where
, and
.
Similarly, if  lies in ,  is a root of the
quadratic polynomial , where
,
.
If  are the two roots of , the root that
corresponds to  is the same as in the case where  is -axis
parallel.
Moreover, in this case we have ,
, , if , and
, , , if
. In both cases, the algebraic degrees
involved in the evaluation of the \incircle predicate are
. Again, as per Subsection \ref{sec:pxsp},
the  predicate can be evaluated using algebraic
expressions of maximum degree .







\subsection{The query object is a segment}\label{sec:psss}

Let  be the query segment, while the \vor
circle is defined by the point  and the segments  and . Let
 be the center of the \vor circle. As in the previous
subsection, it suffices to consider the cases where, either both 
and  are -axis parallel, or  is -axis parallel and 
is -axis parallel. Recall that, in both cases, we have shown that
 is always a root of a quadratic polynomial ,
where the algebraic degrees of  and  are 1 and 2, respectively.


\subsubsection{ and  are -axis parallel}\label{sec:psss-parallel}

If  is also -axis parallel we first need to determine if 
lies inside the band  delimited by  and
. This is easily done by checking if  lies inside ,
which in turn means checking the signs of  and , as
described in the previous subsection. Clearly, if  is not inside
the band , then .
Assume now that  lies inside . The first step is to evaluate
the  and, if necessary,
. If  or
, then we immediately know that
. Otherwise, we simply need to determine on
which side of the line   and  lie:  intersects the
\vor circle  if and only if  and  lie on different
sides of . Determining the side of  on which the point
, , lies is equivalent to computing the sign of the
difference . This, in turn, reduces to computing the
signs of the expressions  and , which are
expressions of algebraic degree 2 and 1, respectively.

In the case where  is -axis parallel, we proceed according to
the generic approach presented in Subsection \ref{sec:pxss}. In this
case , i.e., ,
 and . Moreover,  is a linear
polynomial , thus the algebraic degrees of 
and , , are  and ,
respectively. By applying the analysis of Subsection \ref{sec:pxss},
with , 
we conclude that we can answer the \incircle predicate by evaluating
the signs of expressions of algebraic degree at most
.


\subsubsection{ is -axis parallel and  is -axis
  parallel}\label{sec:psss-vertical}

For the purposes of resolving this case, we are going to follow the
analysis of Subsection \ref{sec:pxss}. In the previous subsection
we argued that in this case the center  of the \vor
circle  lies on the intersection of the parabola with
equation \eqref{equ:parabola-A-CD} and either the line 
(if ) or the line  (if
). Solving in terms of  we deduce that  is
a root of the quadratic polynomial , where
,
,
if , whereas
,
,
if .
Notice that in both cases the algebraic degrees of  and  are
1 and 2, respectively. Furthermore, if  are the two roots
of , the root of  of interest is given in the following
table (see also Fig. \ref{fig:pss}(right)).
\begin{center}
\begin{tabular}{|c|c|}
\hline
Relative positions of  and  & Root of  of interest\\
\hline \hline
&\\\hline
&\\\hline
\end{tabular}
\end{center}
Finally, as already 
described in the previous subsection, in this case we have ,
, , if , and
, , , if .
We are now ready to apply the analysis of Subsection \ref{sec:pxss},
with . We thus conclude that 
the predicate  can be evaluated using algebraic
quantities of degree at most .

\begin{comment}
\begin{table}[t]
\begin{center}
\begin{tabular}{|c|c|}
\hline
Relative positions of  and  & Root of  of interest\\
\hline \hline
&\\\hline
&\\\hline
\end{tabular}
\end{center}
\caption{Root of  that corresponds to  as a function of the
  relative positions of  and ; here  is assumed -axis
  parallel, while  is -axis parallel.}
\label{Table:PSSSxy}
\end{table}
\end{comment}





















\section{Conclusion and future work}\label{sec:concl}

In this paper we have studied the \incircle predicate involved in the
computation of the Euclidean \vor diagram for axes-aligned line
segments. We have described in detail, and in a self-contained manner,
how to evaluate this predicate. We have shown that we can always
resolve it using polynomial expressions in the input quantities that
are of maximum algebraic degree 6.

Our analysis is thus far theoretical. We would like to implement the
approach presented in this paper and compare it against the generic
implementation in CGAL \cite{cgal:k-sdg2-10b}.
Finally, we would like to study 
the rest of the predicates involved in the computation of the \vor
diagram, as well as consider the ortho-45 case, i.e., the case
where the segments are allowed to lie on lines parallel to the lines
 and .





\section*{Acknowledgments}
\noindent
Work partially supported by the FP7-REGPOT-2009-1 project ``Archimedes
Center for Modeling, Analysis and Computation''.


\bibliographystyle{plain}
\bibliography{biblio,svd-vd04}

\end{document}
