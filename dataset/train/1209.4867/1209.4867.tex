

\section{Experimental Evaluation}
\label{sec:eval}

We now present results from extensive simulations of \hsys and \ksys. We
first describe our experimental setup, and then we present the simulation
results.
\subsection{Experimental setup}
\label{sec:setup}
We built simulators for both \hsys and \ksys in Java. Each simulator
includes the basic lookup mechanism of the network,\footnote{We use  ``network'' to indicate the underlying DHT,
  i.e. Chord or Kad.} the A-Boost reputation tree for each node,
collaborative boosting, a model for node churn, and attacker models
specific to the network (see Section~\ref{sec:model:attack}).




\paragraphX{Setup for \hsys.}
All our simulations for \hsys were run for networks with 1000
nodes.\footnote{Larger networks can be simulated, but take an
  impractically long time to finish since each node in the network must
  build up reputation information through lookups.} In our experiments
we use a redundancy of 10 as suggested for regular Halo with 1000 nodes
in the network. 


\paragraphX{Setup for \ksys.}  
Most of our simulations for \ksys were run for networks with 
nodes. The largest simulation was run for 100,000 nodes. In Kad and
\ksys, we initialize the system with  lookups per node to populate
the -buckets. The difference between collaborative boosting and
A-boost is in the selection process of the  nodes returned by an
intermediate queried node during every step of a lookup process, as
described in Section~\ref{sec:kad-reds}.
Also as discussed in Section~\ref{sec:kad-reds}, we impose an attacker
model on Kad that includes routing table pollution, since routing tables
are populated dynamically, and we modify the bucket replacement policy
to replace low reputation nodes.
Kad has inherent redundancy controlled by
parameters  and  as described in Section~\ref{kad}. We
used =10 and =7 redundancy for most of the simulations.


\paragraphX{Churn.} To evaluate how the network handles node churn, we
add and remove nodes probabilistically after each lookup (which are
treated as atomic operations). The probability of a given node joining
or leaving after a given lookup is set based on the intended churn rate
for that simulation run. For example, in a simulation with 
nodes, a colluding fraction of ,  training lookups, and
a churn rate of  over the whole simulation (i.e., in a network
with 1000 nodes, on average 250 nodes leave the network and 250 new
nodes join the network over the course of the simulation),  the
probabilities for a single node are ,
calculated as

where the  stems from the fact that only honest nodes do training
lookups.



\paragraphX{Nodes chosen for lookups.} For all simulations, all honest
nodes are selected in a random permutation as querying nodes. For A-Boost
and collaborative boosting, this helps to build the reputation trees of
all the honest nodes. For all \sys simulations, nodes use
\emph{deterministic maximum score} as described in
Section~\ref{sec:reds:dynamic} to select fingers for routing.

\paragraphX{Shared reputation.} When shared reputation is used, all
honest nodes are trained and queried similarly, with the addition that
nodes gather shared reputation information from joint knuckles when
evaluating fingers. Shared reputation is only available in \hsys, as
described in Section~\ref{sec:reds:sReDS}.


\paragraphX{Sampling.}  For some of the results in this section, we used
a {\em continuous simulation mode}, in which the network is sampled at
regular intervals as the simulation progresses. This allowed us to
monitor the evolution of the failure rate as nodes learn more
information about the network and as nodes join and leave the
network. To achieve this, we conduct alternating phases of  {\em
  training lookups}, during which reputation scores are set, and 
{\em probing lookups}, during which the failure rate of lookups was
recorded. Probing can be thought of as taking a snapshot of the state of
the network. One set of training lookups and probing lookups is a
\emph{slot}.
We then took the failure rate achieved in the steady-state as the final
result. Since continuous simulations show changes over time, they
represent a single (often very long) simulation run.

In other (non-continuous) simulations, we simply run a long training
phase and then a single probing phase at the end. Each data point in
these graphs corresponds to an average value with standard error bars
from  different instantiations of the DHT, where we typically set
.





\subsection{Long-term performance under churn}
\label{sec:eval-churn}

\begin{figure}[t]
\centering
\includegraphics[width=\scalefac\columnwidth]{graphs/failure_rate_graph_cont_c02_ma_c}
\caption{{\bf Halo-ReDS (Continuous).} The evolution of the failure
   rate. The failure rate appears to stabilize around a steady-state
   value.}    \label{fig:churn_c02}
\end{figure}

One issue with reputation in a DHT is that as nodes join and leave the
network (i.e., under node churn), reputation information becomes
stale. 
We also seek to determine whether \sys performance reaches a reliable
steady state as churn continues to affect the system.


To these ends, we ran \hsys experiments in continuous simulation mode in
which all nodes were replaced on average once every 800 lookups (160
slots), i.e., by the time a node has done 800 lookups, on average all
nodes in the network have been replaced once. We then let this
simulation continue for altogether 20,000 lookups per node (4,000 slots,
20 million total lookups), and calculate the average failure rate over
the latter half of the simulation (i.e., the latter 10,000 lookups or
2,000 slots) to get a \emph{steady state} value for the failure rate.



Figure~\ref{fig:churn_c02} shows the results for  and attack
rate .  The failure rate quickly drops from a relatively high
rate of  early on, when no reputation information is available, to
less than . As churn sets in, the failure rate increases until it
reaches a steady state of around , indicating that removing
reputation information for fingers that leave the network to a large
extent solves the problem of stale reputation, limiting the effect of
node churn on \sys.



\begin{figure}[t]
  \centering \subfigure[Attack rate . A-Boost is unable to
    improve over regular Halo due to node churn. Collaborative boosting,
    however, performs significantly better throughout.]{
    \label{fig:failure_rate_a1.0}
  \includegraphics[width=\scalefac\columnwidth]{graphs/failure_rate_graph_4000s_a10_c}}
  \subfigure[Attack rate . A-Boost now performs worse than
    regular Halo, while collaborative boosting again performs
    significantly better than either regular Halo or A-Boost.]{
    \label{fig:failure_rate_0.5}
    \includegraphics[width=\scalefac\columnwidth]{graphs/failure_rate_graph_4000s_a05_c}}
\caption{{\bf Halo.} These graphs compare the failure rates
    for different Halo algorithms and colluding fractions.}
  \label{fig:failure_rate}
\end{figure}

\subsection{Comparison of different Halo schemes.}
\label{sec:eval-schemes}

We now compare the failure rates of different schemes (regular Halo,
A-Boost, and collaborative boosting) for different colluding rates using
non-continuous simulation runs. Figure~\ref{fig:failure_rate_a1.0} shows
the failure rate for regular Halo, A-Boost, and collaborative boosting
for an attack rate of . Due to node churn, A-Boost performs
roughly the same as regular Halo.
Collaborative boosting, on the other hand, significantly reduces the
failure rate, down to 1.5\% for a colluding fraction of 20\%, an
improvement of around 79\% over regular Halo and 73\% over A-Boost.


The results for an attack rate of  are shown in
Figure~\ref{fig:failure_rate_0.5}. Note that in such a scenario, there
is only half as much information available to the reputation system
about attackers. The failure rate of regular Halo is now approximately
half of what it was for an attack rate of , because only half of
all lookups are attacked. A-Boost is now performing worse than regular
Halo. Collaborative boosting, on the other hand, is again performing
much better than regular Halo and A-Boost, reducing the failure rate by
40\% and 50\% compared to regular Halo and A-Boost, respectively.






A-Boost suffers under churn because it cannot easily adapt to changes
beyond its fingers. The reputation trees of nodes that did not see the
change in their fingers' fingers (and further down the tree) do not
account for these churn events.
In collaborative boosting, however, nodes can rely on their fingers to
adapt to changes further down the lookup paths. This greatly improves
the speed at which lookup paths are modified to address churn events.



\subsection{Performance of \ksys.}
\label{sec:eval-kad-perf}


\begin{figure}[t]
 \centering 
  \includegraphics[width=\scalefac\columnwidth]{graphs/pollution} 
\caption{{\bf Kad (Continuous).} Evolution of routing table pollution
  over the lifetime of an experiment with  and . Regular
  Kad faces increasing pollution over time, while \ksys identifies and
  removes malicious fingers from -buckets.}
\label{fig:kad_pollution}
\end{figure}


We now turn towards evaluating our design for \ksys. Our primary
concerns in \ksys are whether routing table pollution can be overcome,
the general effectiveness of the design under different redundancy
parameters, and the performance of A-Boost and collaborative boosting
under churn.

\paragraphX{Routing Table Pollution.} We find that routing table
pollution is the most critical factor in Kad and \ksys
performance. Thus, we first seek to understand the extent of routing
table pollution these systems. To this end, we perform a continuous time
simulation with 10,000 nodes under  churn. Each of the nodes
performs 100 lookups in order to populate the -buckets and build the
reputation system. We divide the simulation time into 400 slots of 2500
lookups each.


As discussed in Section~\ref{sec:kad-reds}, attacker nodes are
attempting to get their own nodes into as many routing tables as
possible. In \ksys, however, attacker nodes with low reputation scores
will have little chance to be selected as the next hop contacts in
future lookups and will eventually be kicked out of many
-buckets. Figure~\ref{fig:kad_pollution} shows the pollution of
routing tables over the training period for both regular Kad and
\ksys. We see that the reputation system is very effective, leading to a
decreasing rate of pollution of routing tables with \ksys, compared with
increasing pollution rates in Kad.

\begin{figure}[t!]
\centering
  \subfigure[Kad.]{
    \label{fig:kad_f_areg}
    \includegraphics[width=\scalefac\columnwidth]{graphs/A3_A5_vanilla}}
  \subfigure[\ksys, collaborative boosting. Note that the -axis is
    from 0-45\%]{
    \label{fig:kad_f_a}
    \includegraphics[width=\scalefac\columnwidth]{graphs/A3_A5}}
\caption{{\bf Kad (No Churn).} Failure rates with , , and
   varying  and . Higher redundancy () improves both
   systems, but \ksys is much better than Kad for .}
      \label{fig:kad_f_graphs}
\end{figure}

\begin{figure}[t!]
\centering
  \subfigure[Attack rate of .]{
    \label{fig:kad_churn_a100}
    \includegraphics[width=\scalefac\columnwidth]{graphs/a100-churn25}}
 \subfigure[Attack rate of .]{
    \label{fig:kad_churn_a0}
    \includegraphics[width=\scalefac\columnwidth]{graphs/a0-churn25}}
\caption{{\bf Kad.} Failure rates of different \ksys schemes for
   different colluding fractions . Both A-Boost and Collaborative are
   much more effective than Kad.}
      \label{fig:kad_churn_graphs}
\end{figure}



\paragraphX{Redundancy.} We also explore the performance of \ksys in
extensive non-continuous simulations. First, we break down the
performance in detail without churn. Figure~\ref{fig:kad_f_graphs} shows
the effect of redundancy on regular Kad without churn.
We use an attack rate of . Figure~\ref{fig:kad_f_areg} shows how
the failure rate decreases as the redundancy increases. However, even
with a high redundancy of , the failure rate when  is
over 21\%. Collaborative boosting dramatically improves the
system. Figure~\ref{fig:kad_f_a} shows failure rates for \ksys. With a
lower redundancy of  and , the failure rate when
 is less than 1\%. When , , and , Kad
has a  failure rate, while \ksys has a  failure rate, a 95\%
decrease.



\paragraphX{Comparison Under Churn.} We now compare the performance of
Kad and both collaborative and A-boost versions of \ksys under churn. 
We present results for  churn, and  and 
redundancy. Figure~\ref{fig:kad_churn_a100} shows the failure rate of
regular Kad, A-Boost, and collaborative boosting for an attack rate of
. Performances of A-Boost and collaborative are almost the same
(within the margin of error), significantly reducing the failure rate
of Kad down to 4-5\% from 54\% for . By comparing
Figure~\ref{fig:kad_f_a} and Figure~\ref{fig:kad_churn_a100} we note
that, the performance of Kad-\sys with =10 and =7 degrades
from 2-3\% failure rate to 4-5\% due to the 25\% churn. 


The results for an attack ratio of  are shown in
Figure~\ref{fig:kad_churn_a0}. In our attack model
(Section~\ref{sec:kad-reds}), the attackers are always trying to
pollute the routing tables even when they are not attacking.
Figure~\ref{fig:kad_churn_a0} shows that failure rate slightly increases
for 0\% attack rate compared to the failure rate for 100\% attack
rate. We will discuss more about our attack effectiveness in
Section~\ref{sec:overall_effectiveness}.


Over all scenarios we tested, we find that performance is improved by at
least 93.4\% with \ksys compared to regular Kad.





\subsection{Overall attack effectiveness.}
\label{sec:overall_effectiveness}
In this experiment, we show the \emph{overall attack effectiveness} when
honest nodes use 
collaborative boosting. The overall attack effectiveness is the maximum
continuous failure rate that the attacker can achieve when his nodes use
a consistent attack rate, i.e. without jumping to 100\% attack rate for
evaluation. This allows us to identify the best that the attacker could
do consistently over time.


\paragraphX{Halo.} We first performed the experiment for both regular
Halo and A-Boost. In regular Halo, the attack effectiveness grows
linearly with the attack rate as expected. In A-Boost, which is
generally about as effective as regular Halo under churn, the results
are quite similar. For  attack rate, both systems had less than
1\% failure rate for 5\% to 20\%. For  and 20\%, the
failure rate is between 3.5\% to 4.0\%, and for  and 20\%,
the failure rate is between 6.9\% to 7.5\%.

\begin{figure}[t!]
  \centering 
\includegraphics[width=\scalefac\columnwidth]{graphs/graph_effective_collaborative_4000s_c}
  \caption{{\bf \hsys.} The failure rate for different continuous attack
    rates, with . Attacker effectiveness peaks at less than
    .}
  \label{fig:overall_collaborative} 
\end{figure}



Figure~\ref{fig:overall_collaborative} shows the overall attack
effectiveness that the attacker can achieve against collaborative
boosting. We see that increasing the attack rate up to a point results
in more lookup failures. Beyond a certain attack rate, e.g., at 60\% for
a colluding fraction of , the overall failure rate goes back
down. Specifically, for , the effectiveness peaks at a 
failure rate, for  at , for  at , and
for  at a tiny fraction of a percent. Comparing the overall
effectiveness of collaborative boosting to A-Boost and regular Halo,
\sys reduces effective failure rates by up to 70\% for ,
and by up to 80\% for .
The reason for the peak in effectiveness is that with lower attack
rates, fewer lookups are being subverted, while with higher attack
rates, the malicious nodes are more easily detected by the reputation
system and are no longer used.

Despite the ability of attackers to operate at a peak rate, we note that
for colluding fractions of 20\% and below, no matter what rate the
attackers attack with, \emph{collaborative boosting limits their
  effectiveness to below 2.1\%}.









\begin{figure}[t!]
\centering
  \subfigure[Kad.]{
    \label{fig:kad_churn_areg}
    \includegraphics[width=\scalefac\columnwidth]{graphs/a-reg}}
 \subfigure[\ksys. -axis ranges from 0-8\%.]{
    \label{fig:kad_churn_a}
    \includegraphics[width=\scalefac\columnwidth]{graphs/a}}
\caption{{\bf Kad.} Overall attack effectiveness with different attack
   rates. Attackers perform best by attacking with low attack rates. }
\label{fig:kad_churn_ar_graphs}
\end{figure}

\paragraphX{Kad.} We similarly examine overall attack effectiveness in
Kad. Figure~\ref{fig:kad_churn_ar_graphs} shows our simulation
results. Kad has very different results
(Figure~\ref{fig:kad_churn_areg}) from Halo due to routing table
pollution in the attacker model.
In particular, for Kad with colluding fraction , the failure
rate is 80\% when the attack rate is  and drops to 54\% when
. The high failure rate with no manipulation of lookups ()
is due to the effectiveness of routing table pollution against Kad. As
the attack rate increases, routing table pollution decreases, which
leads to the drop in failure rates.
This occurs because, when the attacker manipulates lookups, the results
returned by attacker nodes are always attacker nodes and are generally
further away from the target than those returned by honest nodes. So
whenever both attacker nodes and honest nodes are responding to a
lookup, the attacker nodes not only fail to manipulate the lookup result
but also do not appear in the later stages of the lookup process. This
reduces the malicious nodes' chances of being opportunistically added to
-buckets.

In comparison, Figure~\ref{fig:kad_churn_a} shows how effective \ksys is
to counter the advanced route subversion attack described in
Section~\ref{sec:kad-reds}. The attackers gain a slight advantage with
lower attack rates. The reputation system, however, severely limits the
growth of the average failure rate by curbing routing table
pollution. \ksys maintains a failure rate of no more than 5.1\% for all
attack rates with  or less, which is a 93\% improvement over
Kad.






\subsection{Comparison of subsearch failure reasons}

Next, we analyzed how and when subsearches (these are the redundant
lookups that together form the overall search) fail under the different
algorithms. A more indirect way of comparing different algorithms and
showing how the algorithms improve the failure rate is by looking at why
searches fail.

Figure~\ref{fig:failure_graphs} shows the failure reasons for regular
Halo in Figure~\ref{fig:failure_regular} and collaborative boosting in
Figure~\ref{fig:failure_collaborative}. For both regular Halo and
collaborative boosting there are certain failures which cannot be
avoided.  For example, a knuckle may return the wrong successor node,
as this is an inherent limitation of Chord's structure, where a knuckle
for a particular exponential offset simply does not exist (25\% of the
time)~\cite{halo}. Other failures, however, can be reduced using
reputation information. The main failures for both regular Halo and
collaborative boosting are: a lookup hits a bad node in the lookup
path, the starting node for a knuckle-search was already a colluder, or
the knuckle that was found was a colluding node.  For regular Halo as
shown in Figure~\ref{fig:failure_regular}, the most important failure
reason is that a lookup hits a colluder node in the lookup path, which
subverts the search. As the colluding fraction increases, the number of
subsearches that fail because of a bad node in the path increases as
well.  For collaborative boosting as shown in
Figure~\ref{fig:failure_collaborative}, it is clearly visible that
collaborative boosting routes around bad nodes and thus has a much
smaller percentage of subsearches that fail due to a colluding node in
the lookup path. The percentage increases as the colluding fraction
increases, but it stays at a much smaller percentage compared to
regular Halo. Likewise, collaborative boosting is also more successful
at picking good nodes as starting nodes by exploiting reputation
information.  This shows that reputation information is indeed useful
in avoiding bad nodes in the lookup paths, by sending lookups through
parts of the Chord network which are known to be reliable.

\begin{figure}[t!]
\centering \subfigure[{\bf Regular Halo.} As the colluding fraction
  increases, more subsearches fail because of a bad node in the lookup
  path.]{
    \label{fig:failure_regular}
    \includegraphics[width=\scalefac\columnwidth]{graphs/failure_graph_regular}}
\hspace{0.5cm} \subfigure[{\bf A-Boost.} Compared to regular Halo,
  A-Boost has fewer failures due to a bad node in the path, showing that
  it is indeed routing around malicious nodes.]{
    \label{fig:failure_aboost}
    \includegraphics[width=\scalefac\columnwidth]{graphs/failure_graph_aboost_max}}
\subfigure[{\bf Collaborative boosting.} Even when the colluding
  fraction increases, failures due to a bad node in the lookup path
  increase much more slowly than in regular Halo or A-Boost.]{
    \label{fig:failure_collaborative}
    \includegraphics[width=\scalefac\columnwidth]{graphs/failure_graph_collaborative}}
 \caption{{\bf Halo Failures.} What percentage of subsearches fail and
   for what reasons.}
      \label{fig:failure_graphs}
\end{figure}

\subsection{Effectiveness of shared reputation}\label{sec:results-shared}
In these experiments, we explore whether 
sharing reputation values can help lower the failure rate. We again
simulate malicious nodes attacking at a rate of 1.0, for different
fractions of colluding nodes, and see how the failure rate evolves
We also look at different ways of calculating shared reputation:
average, median, and the \sharedrep algorithm described in
Section~\ref{sec:reds:sReDS}.
The values returned by malicious nodes are furthermore calculated to
maximize the probability that the value is accepted by the requesting
node, by taking into account the shared reputation algorithm used. This
maximizes the attackers' advantage.


Figure~\ref{fig:shared0.1} shows the results for 10\% colluding
nodes. We see that while there is a slight difference in the
convergence speed at the beginning of the simulation (median and
\sharedrep shared reputation converging slightly faster), the difference
is not statistically significant. For a higher fraction of
colluding nodes of 20\% and for attack rates lower than  (not
shown), shared reputation fails to improve the failure rate over
collaborative in any scenario.

\begin{figure}[t!]
\centering 
\includegraphics[width=\scalefac\columnwidth]{graphs/TDSC_gnuplot/shared_rep_c01_c}
\caption{Shared reputation () at best performs about the same
   than normal collaborative mode (within the margin of error).}
      \label{fig:shared0.1}
\end{figure}


















\paragraphX{New Nodes.} Next we study whether new nodes joining the
system can benefit from shared reputation.
Intuitively, shared reputation should be particularly useful for new
nodes joining an existing Chord network. A new node does not have any
reputation information yet, so asking other nodes in the network for
shared reputation helps a node to make routing decisions until it has
collected enough observations by itself. In this experiment, we test the
failure rate of nodes newly added to the Chord network, where those
nodes rely solely on shared reputation and collaborative boosting.



The results do not bear out this intuition,
however. Figure~\ref{fig:newnodes} shows that using shared reputation
for newly added nodes does not improve the failure rate, and can even
worsen the failure rate for example for colluding fractions of
. This can be explained by noting that using collaborative
boosting necessarily decreases (and cannot increase) the failure rate,
as each node only operates according to its first-hand observations. In
shared reputation, however, nodes become susceptible to slandering and
self-promotion of malicious nodes.






















