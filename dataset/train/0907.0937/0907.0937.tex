\documentclass[aop,noinfoline]{imsart} 
\usepackage{amssymb,amsmath,amsfonts}
\usepackage{mathrsfs}
\usepackage{graphicx}
\usepackage{pstricks,pst-node}
\usepackage{pst-all}
\setattribute{journal}{name}{}


\newtheorem{theorem}{Theorem}[section]
\newtheorem{proposition}[theorem]{Proposition}
\newtheorem{lemma}[theorem]{Lemma}
\newtheorem{example}[theorem]{Example}
\newtheorem{definition}[theorem]{Definition}


\newtheorem{corollary}[theorem]{Corollary}
\newtheorem{claim}[theorem]{Claim}
\newtheorem{conjecture}[theorem]{Conjecture}
\renewcommand{\Box}{{\vrule width0.6ex height1em depth0cm}}
\newenvironment{proof}{\noindent{\bf Proof:}}{\hfill \Box}
\newenvironment{sketchofproof}{\noindent{\bf Sketch of proof:}}{\hfill
\Box}
\newenvironment{newproof}[1]{\noindent{\bf Proof of #1:}\
}{\medskip\par}
\newenvironment{Example}{\begin{example} \rm}{\end{example}}
\newenvironment{remark}{\noindent{\bf Remark:}\ }{\medskip\par}
\newenvironment{notation}{\noindent{\bf Notation:}\ }{\medskip\par}






\def\build#1_#2^#3{\mathrel{\mathop{\kern 0 pt#1}\limits_{#2}^{#3}}}

\newcommand{\onetwo}{(1,2)}
\newcommand{\sat}{\mathsf{SAT}}
\newcommand{\threesat}{\mbox{\sf 3-SAT}}
\newcommand{\qsat}{\mathsf{
QSAT}}
\newcommand{\unsat}{\mathsf{UNSAT}}
\newcommand{\onetwoqsat}{\mbox{\sf (1,2)-QSAT}}
\newcommand{\twosat}{\mbox{\sf 2-SAT}}
\newcommand{\negate}[1]{\overline{#1}}
\newcommand{\sign}{\epsilon}
\newcommand{\atomoflit}[1]{|#1|}
\newcommand{\emptyclause}{\square}
\newcommand{\onetwoqcnf}{\mbox{\sf \onetwo-QCNF}}
\newcommand{\twocnf}{\mbox{\sf 2-CNF}}
\newcommand{\mncn}{\mbox{\sf (1,2)-}F((m,n), cn)}

\newcommand{\deronetwoQres}{\vdash_{\mathrm{\onetwo-Q-res}}}
\newcommand{\quantifierprefix}[1]{\mbox{}}
\newcommand{\clauseset}[1]{\mbox{}}
\newcommand{\onetwoqresolvent}{\mbox{\sf \onetwo-Q-resolvent}}
\newcommand{\onetwoqres}{\mbox{\sf \onetwo-Q-res}}

\newcommand{\coNP}{\mathsf{coNP}}
\newcommand{\conp}{\mathsf{coNP}}
\newcommand{\ptime}{\mathsf{P}}
\newcommand{\pspace}{\mathsf{PSPACE}}
\newcommand{\falsum}{\bot}
\newcommand{\verum}{\top}



\newcommand{\Prob}{\mathsf{Pr}}
\newcommand{\calS}{{\mathcal{S}}}
\newcommand{\PP}{\mathbb{P}}
\newcommand{\NN}{\mathbb{N}}
\newcommand{\EE}{\mathbb{E}}

\newcommand{\pmc}{\mathbb{P}_{m,c}}
\newcommand{\pmplusonec}{\mathbb{P}_{m+1,c}}
\newcommand{\pmLu}{\mathbb{P}_{m,c}}
\newcommand{\pmLr}{\PP_{m,L}^\mathcal{R}}
\newcommand{\pmp}{\PP_{m,p}}

\newcommand{\eps}{\varepsilon}

\renewcommand{\hat}{\widehat}

\newcommand{\DELETE}[1]{}

\begin{document}

\begin{frontmatter}


\title{The threshold for random } \thanks{This work has been
supported by EGIDE
    10632SE, \"OAD Amad\'ee 2/2006 and ACI NIM 202. Preliminary
    versions of this article appeared in \cite{CreignouDER-08} and \cite{CreignouDER-09}}


\runtitle{The threshold for random }

\begin{aug}
\author{\fnms{Nadia} \snm{Creignou}},
\author{\fnms{Herv\'e} \snm{Daud\'e}},
\author{\fnms{Uwe} \snm{Egly}}
\and
\author{\fnms{Rapha\"el}  \snm{Rossignol}\corref{}\ead[label=e1]{raphael.rossignol@math.u-psud.fr}}
\runauthor{N. Creignou, H. Daud\'e, U. Egly and R. Rossignol}
\affiliation{ Universit\'{e}
   d'Aix-Marseille 2, Universit\'{e}
   d'Aix-Marseille 1, Technische Universit\"at
   Wien, and Universit\'e Paris Sud}


\address{Nadia Creignou\\
Universit\'{e} d'Aix-Marseille 2\\ Laboratoire
   d'Informatique Fondamentale\\ 163 avenue de Luminy F-13288 Marseille\\ France}
\address{ Herv\'e Daud\'e\\ Universit\'{e}
   d'Aix-Marseille 1\\ Laboratoire d'Analyse\\ Topologie et
   Probabilit\'es\\ 
   Chateau Gombert F-13453 Marseille\\ France}
\address{Uwe Egly \\ Institut f\"ur
   Informationsysteme 184/3\\ Technische Universit\"at
   Wien\\
   Favoritenstrasse 9-11\\ A-1040 Wien\\ Austria}
\address{Rapha\"el
  Rossignol \\ Universit\'{e} de
   Paris 11\\
   D\'epartement de Math\'ematiques, B\^{a}timent 425\\
   F-91405 Orsay Cedex\\ France\\
\printead{e1}}
\end{aug}


\date{\today}
\maketitle

\begin{abstract}

The  problem is the quantified version of the  problem.
We show the existence of a threshold effect for the phase transition
associated with the satisfiability of random quantified extended 2-CNF
formulas. We consider boolean CNF formulas of the form , where  has  variables,  has 
variables and each clause in  has one literal from  and
two from . For such formulas, we show that the threshold
phenomenon is controlled by the ratio between the number of clauses
and the number  of existential variables. Then we give the exact
location of the associated critical ratio .  Indeed, we prove
that  is a decreasing function of , where  is
the limiting value of  when  tends to infinity. 



\end{abstract}


\begin{keyword}[class=AMS]
\kwd{68R01}
\kwd{60C05}
\kwd{05A16}
\end{keyword}

\begin{keyword}
\kwd{Random quantified formulas}
\kwd{satisfiability}
\kwd{phase transition}
\kwd{sharp threshold}
\end{keyword}

\end{frontmatter}






\section{Introduction}\label{sec:introduction}




A significant tool for SAT research has been the study of random
instances.  It has stimulated fruitful interactions among the areas
of artificial intelligence, theoretical computer science,
mathematics and statistical physics. Recently there has been a
growth of interest in a powerful generalization of the Boolean
satisfiability, namely the satisfiability of Quantified Boolean
formulas, QBFs.  Compared to the well-known propositional
formulas, QBFs permit both universal and existential quantifiers
over Boolean variables. Thus QBFs allow  the modelling of problems
having higher complexity than SAT, ranging in the polynomial
hierarchy up to PSPACE. These problems include problems from the
areas of verification, knowledge representation and logic (see,
e.g., \cite{Egly00c}).




 Models for
generating random instances of QBF have been proposed \cite{GentW-99,ChenI-05}.
Problems for which one can combine practical experiments with
 theoretical studies are natural candidates for first investigations
 \cite{CreignouDE-07}. In this paper, we focus on a certain subclass
 of closed quantified Boolean formulas, which can be seen as
 quantified extended -formulas. These formulas bear
 similarities with -formulas, whose random instances have
 been extensively studied in the literature (see, e.g.,
 \cite{ChvatalR-92, Goerdt-96, Verhoeven-99, Bollobasetal01,
   Vega-01}). At the same time, the introduction of quantifiers
 increases the complexity and requires additional parameters for the
 generation of random instances.  More precisely, we are interested in
 closed formulas in conjunctive normal form (CNF) having two quantifier
 blocks, namely in formulas of the type , where  and  denote distinct sets of variables,
 and  is a conjunction of 3-clauses, each of which
 containing exactly one universal literal and two existential
 ones. Evaluating the truth value of such formulas is known to be
 -complete \cite{FloegelKKB-90}. In order to generate random
 instances we have to introduce several parameters.  The first one is
 the pair  that specifies the number of variables in each
 quantifier block, i.e., in  and .  The second one is , the number of clauses. We shall study the
 probability that a formula drawn at random uniformly out of this set
 of formulas evaluates to true as  tends to infinity. We will
 denote by  this probability. Thus, we are interested in


Let us recall that the transition from satisfiability to
unsatisfiability for random  formulas is sharp. Indeed, there
is a \emph{ critical value} (or a \emph{threshold}\/) of the ratio of
the number of clauses to the number of variables, above which the
likelihood of a random -formula being satisfiable vanishes as
 tends to infinity, and below which it goes to 1. Moreover, this
critical value is known to be  (see \cite{ChvatalR-92, Goerdt-96}).

On the one hand observe that, when , a -formula
with , clauses can   be seen as the conjunction of 
two independent -formulas (each of which corresponds to an
assignment to the universal variable  and has on average 
clauses).  On the other hand,   when  is large enough, a random
-formula with   clauses has
essentially strictly distinct universal literals, and then behaves
as an existential -formula. Thus,  we can easily prove
that the transition between satisfiability and unsatisfiability
for  random  -formulas occurs when  is  between 1
and 2.  Our main contribution is to identify the scale for  (as
a function of ) at which an  intermediate and original regime
can be observed, . Moreover,  at
this specific scale in developing further the techniques used by
Chv\`atal and Reed  \cite{ChvatalR-92}, and Goerdt
\cite{Goerdt-96},  we get the precise location of the threshold as
a function of .  Our main result is:



\begin{theorem}\label{thm:main}
 For any ,
 there exists    such that:

  \begin{itemize}
  \item if , then , \\
  \item if , then . \\
  \end{itemize}
  Moreover, the critical ratio    is given by 
   



\end{theorem}

 Figure \ref{fig:thresholdevolution} shows the evolution of the critical ratio
 as a function of . 




\begin{figure}[htbp]
 \begin{center}
\psset{xunit=0.5,yunit=1.5}
\begin{pspicture}(-0.5,-0.3)(20,2.5)
 \psaxes[labels=y,ticks=all,tickstyle=top,ticksize=2pt]{->}(0,0)(-0.5,-0.3)(20,2.5)
\psecurve[linecolor=magenta](-0.1,2)(0,2)(1.4,2)(1.444, 1.999874517)(2.371800000, 1.812582003)(3.299600000, 1.681618364)(4.227400000, 1.593855092)(5.155200000, 1.531000997)(6.083000000, 1.483521353)(7.010800000, 1.446188463)(7.938600000, 1.415921174)(8.866400000, 1.390787054)(9.794200000, 1.369511004)(10.72200000, 1.351216012)(11.64980000, 1.335277810)(12.57760000, 1.321239086)(13.50540000, 1.308756571)(14.43320000, 1.297567155)(15.36100000, 1.287465485)(16.28880000, 1.278288729)(17.21660000, 1.269905955)(18.14440000, 1.262210571)(19.07220000, 1.255114831)(20.00000000, 1.248545784)
  \psline[linestyle=dashed](0,1)(19,1)
  \psline[linestyle=dashed](1.444,0)(1.444,2)
   \rput(1.444,-0.2){}
   \rput(20,-0.2){}
   \rput(5,-0.2){}
   \rput(10,-0.2){}
    \rput(15,-0.2){}
 \rput(-1,2.5){}
\label{fig:thresholdevolution}
\end{pspicture}
\caption{Evolution of the critical ratio values.}
\end{center}
\end{figure}




The paper is organized as follows. In Section \ref{sec:complexity} we
examine the complexity of deciding the truth value of a
-formula. In order to make the paper self-contained, we
give there an alternative proof of the -completeness of this
problem.  In Section \ref{sec:truth_char} we characterize the truth of
-formulas. We introduce specific substructures,
comparable to the ones introduced by Chv\`atal and Reed in
\cite{ChvatalR-92}: we define \textit{pure bicycles}, which are
necessary to ensure the falsity of a -formula, and
\textit{pure snakes}, whose appearance is sufficient to ensure the
falsity. In Section \ref{subsec:enum} we give some enumerative results
concerning pure bicycles and snakes, which will be useful for
determining the location of the threshold.  In Section
\ref{sec:transition} we present the probabilistic model and we  give first
estimates for the location of the threshold. In Section \ref{sec:main} we prove our main
result, Theorem \ref{thm:main}. Finally, Section \ref{sec:technical}
contains the proof of a technical proposition.



\section{ The complexity of  }\label{sec:complexity}



 A \emph{literal}\/ is a propositional variable or its negation. The
\emph{atom}\/ of a literal  is the variable  if  is  or . Literals are said to be \emph{strictly distinct } when their corresponding atoms are pairwise different.
 A \emph{clause}\/ is a finite disjunction of
literals.   A formula is in \emph{conjunctive
normal
  form}\/ (CNF) if it is a conjunction of clauses.
A formula is in -CNF, if any clause consists of exactly 
literals.  Here we are interested in quantified propositional 
formulas of the form

where , and , and
 is a -CNF formula, with exactly one universal
and two existential literals in each clause.  We will call such
formulas \onetwoqcnf{}s. These formulas can be considered as
quantified extended  formulas, because deleting the only
universal literal in each clause and removing the then superfluous
-quantifiers result in an existentially quantified
conjunction of binary clauses. 

A truth assignment for the existential (resp. universal) variables,  (resp.  is a Boolean function  (resp.  which can be extended   to literals by  



A \onetwoqcnf{} formula is {\it true} (or {\it satisfiable}) if for
every assignment  to the variables , there exists an assignment to
the variables  such that  is true under this assignment.
The exhaustive algorithm which consists in deciding whether for
all assignment to the variables , there exists an assignment to
the variables  such that  is true provides a first
upper bound for the worst case complexity. Indeed, since the
satisfiability of a  formula can be decided in linear time
\cite{AspvallPT-79}, the evaluation of the formula  can be performed in time , where  is the number of universal variables and
 denotes the size of . Observe that,
if  is of the order of , then it
provides a polynomial time algorithm. 

In its full generality the problem  is much harder as stated in the
following theorem. This theorem
was proved originally   in \cite{FloegelKKB-90}. In
order to make the paper self-contained,  we give here an
alternative proof.




 \begin{theorem} {\rm \cite{FloegelKKB-90}}\label{thm:complexity}
    The  evaluation problem  is -complete.
 \end{theorem}
 
\begin{proof}
To show membership in , guess a vector of truth values  corresponding to . Replace in
 all free occurrences of any  by
, remove  from the clauses and delete clauses with
. The resulting formula is a -QCNF formula, whose
unsatisfiability (i.e. falsity) can be decided in linear time
(see \cite{AspvallPT-79} for the details).
\medskip



\noindent It remains to be shown that the problem is -hard.
We show this by a polynomial-time computable reduction from the
satisfiability problem for 3-CNF formulas.

Consider such a formula

over the variables   where each  is a
disjunction of exactly three literals ,  and
. We construct   ,   a
\onetwoqcnf\ formula.
Then we show that

The reduction is as follows. We first choose  variables , all of which are different from the variables  occurring in . We take any minimally
unsatisfiable -CNF formula with  clause, e.g.,
 where




For each clause 
occurring in , we define



Let  be a new variable, i.e.,  is different from the
ones in  and . Then

Obviously, the reduction is polynomial-time
computable.

\medskip

We next prove (\ref{eq:sat-iff-false}).  Observe that the formula
resulting from  by any instantiation of the 's
is a conjunction of  clauses (maybe with repetitions) from
. Therefore, since  is minimally unsatisfiable, this
formula will be unsatisfiable if and only if every clause from
 occurs.

\noindent : Suppose  is satisfiable. Take
an arbitrary truth assignment , which satisfies  .
Then, for all , there is (at least) one , such that . In the formula , replace all free occurrences of  by 
for  and  by . Observe that, whenever  in  (for some ) is true, we get  after  simplification. Therefore, in the existential -CNF
formula obtained after simplification it remains the clause
 and at least one copy of each clause  for every
 (the one resulting from , for which
). Therefore, this formula is unsatisfiable, thus
proving that  is false.




\noindent : Suppose  is false. Then,
there is a vector of truth values 
corresponding to , such that the -QCNF
formula obtained  by replacing all  occurrences of any  by
 is unsatisfiable. Since  is
minimally unsatisfiable, and according to the remark above, this
means that this resulting formula contains at least one copy of
each . This copy can only come from a clause 
for some . Hence, we can deduce that the
assignment  for   sets the literal
 to true, and thus satisfies the clause .
Hence, this assignment satisfies the formula .
\end{proof}




\section{Truth value  of  -formulas}\label{sec:truth_char}

 
\subsection{Pure subformulas}\label{subsec:pure}

Let us first introduce a notion of purity over sets of universal literals that will be of use to characterize the truth value  of -formulas.

\begin{definition}
A (multi-)set of literals is \emph{pure} if it does not contain both a variable  and its negation . 
By extension, we call a  -formula, , \emph{pure} if the set of universal literals occurring in  is pure.
\end{definition}

\begin{proposition}\label{prop:truth_char}
 A  -formula is false if and only if it contains a false pure subformula.
\end{proposition}
\begin{proof}
 One direction is obvious. Suppose that the -formula  is false. Then, there is an assignment  to the universal variables  such that for all assignment to ,  evaluates to false. Consider the subformula of  obtained in keeping only the clauses for which the universal literal is assigned  by , and deleting the other ones. This subformula is pure (it cannot contain both a clause with a universal variable  and another with  since either  or  is assigned  by ), and is false by the choice of . 
\end{proof}
\medskip

Now observe that the truth value of a pure  -formula  is the same as the truth value of the existential  formula  obtained in removing the universal literal in each clause and then deleting the universal quantifiers. Therefore, we can appeal to the work of Chv\`atal and Reed \cite{ChvatalR-92} in order to identify substructures that are sufficient (respectively, necessary) to ensure falsity.
On the one hand Chv\`atal and Reed exhibited elementary unsatisfiable -formulas, called \emph{snakes}. On the other hand they identified extremal substructures, called \emph{bicycles}, that appear  in any unsatisfiable -formula. Thus, we can define \emph{ pure snakes} and \emph{pure bicycles}.
\begin{definition}
  A pure \emph{snake}  of length , with ,  is a set
  of   clauses  which have the following
  structure: there is a sequence of  
 
strictly distinct existential literals , and a pure sequence of  universal literals   such
that, for every ,  with .
\end{definition}

\begin{definition}
  A \emph{ pure bicycle} of length , is a set of   clauses  which have the following structure:
there is a sequence of  
strictly distinct existential literals , and a pure sequence of  universal literals   such
that, for  , ,  and   with literals  and  chosen from  with . 
\end{definition}

Thus, we get the following proposition.
\begin{proposition}\label{prop:certificates}\ 

 \begin{itemize}
 \item Every -formula that contains a pure snake is false.
\item Every   -formula that is false, contains a pure bicycle.
\end{itemize}
\end{proposition}

\subsection{Enumerative results}\label{subsec:enum}

\begin{proposition}\label{prop:uppercertificatespure}
 Let  be the number of universal variables and let  be the number of existential variables we can choose from. 
  
   \begin{itemize} 
  \item The number of snakes of length  is 

    where
  
  with  denoting the Stirling number of the second kind,
  and 
  
  .

\item Given a pure snake  of length . For every ,  let  denote the number  of pure snakes  of
  length  such that  and  share exactly   clauses.  Then for   

 and for  

hold.

\item The number of bicycles of length    is

\end{itemize}

\end{proposition}
\begin{proof}
Given a literal , let  denote its underlying variable. Observe that a snake of length  contains  distinct variables.  Moreover, every variable  appearing in a snake occurs exactly twice (once positively and once negatively), except for  which occurs four times (twice positively and twice negatively). This special variable will be called the \emph{double point} of the snake. A snake can be   described by a (circular) sequence of existential literals  (with ), together with the corresponding pure sequence of universal literals . 

Choosing a snake  of length  comes down to choose a sequence of   strictly distinct
literals ,   and then   choose the pure sequence of  universal literals  (they are not necessarily distinct but no literal can be the
  complement of another).
  Let  be the number of pure sequences of literals of length
  , having a set of  variables from which the literals can be
  built.  Let us recall that  is the number of
  applications from a set of  elements onto a set of  elements.
  A pure sequence of literals of length  is obtained by exactly
  one sequence of choices of the following choosing process.
  \begin{enumerate}
  \item Choose the number  of different variables occurring in the
    sequence.
  \item Choose the  variables.
  \item For each such variable, choose whether it occurs positively or
    negatively.
  \item Choose their places in the sequence.
  \end{enumerate}
This gives the announced number of snakes.
\medskip

Given a pure snake  of length . Let  be the
number of pure snakes  of length   such  that  and  share exactly  
clauses. If , this number can be decomposed as
 where  is the number of  pure snakes  such  that  and  share exactly   clauses and  variables. In the rest of the proof, for more readability we omit the subscripts  in , thus writing . Now we are looking for  upper bounds on the .

Let us note that the intersection of  and  can be read on the (circular) sequence of literals
 , where .  In order to get  clauses and  variables in common, one has to choose  blocks of consecutive literals in this sequence. 
We make a case distinction according to whether the two snakes   and  have the same double point or not.
\begin{itemize}
 \item  denotes the  number of pure snakes  of length  such
that   and  share exactly   clauses and  variables, and have the same double point ,
\item  denotes the  number of pure snakes  of length  such
that   and  share exactly   clauses and  variables, and do not have    the same double point.\\
\end{itemize}
Thus   

Let us  first consider 
Observe that in the special case when  (only one block), and  and  have the same double point, then  is necessarily equal to or larger than . Therefore,  
 
In the general case, to count , we
perform the following sequence of choices :

\begin{tabular}{ll}
&  the intersection  such that it has  clauses and  variables,\\
& the sequence of strictly distinct existential literals that are  in \\
&  the places of the  blocks of  among the literals chosen in ,\\
&  the universal literals occurring in  the clauses of .
\end{tabular}

\emph{Step (i)}. To build the intersection , we choose 
literals  in the sequence representing  . They represent the first and last literals of the
 blocks of . The first literal is chosen after or at . 
To
define completely the intersection, we need to know whether this first
literal is the beginning or the end of a block, so we get
at most  possible choices.

\emph{Step (ii)}. Notice that    is the double point of . So, it remains only to choose a sequence of   strictly distinct literals.
Thus,
we have at most  possible choices.

\emph{Step (iii)}.
 We need to choose how the   blocks will be plugged among the ``remaining literals''  chosen in
. 
This
leads to at most  possible choices.

\emph{Step (iv)}. There are  universal literals to choose, and they must be chosen in
a pure way. So, there are at most  choices.

Thus, since    we obtain that   for 

The enumeration of  differs from the one of  only at step (ii). Indeed, when  does not have  as a double point, at step (ii) we have first to choose  a sequence of   strictly distinct literals (thus having  determined the  variables occurring in ), and then choose one of these    variables as the double point. Hence, we have at most  choices.
Thus, we get  for   and 

Then, equation   follows from (\ref{eqn:boundN0}), (\ref{eqn:boundN1})and  (\ref{eqn:boundN2}) while    follows from (\ref{eqn:boundN1}) and (\ref{eqn:boundN2}).\\
 
 The enumeration of bicycles is similar to the one of snakes. We just have to choose in addition  and  among  such that   . This explains  the  extra factor, , in (\ref{eqn:nbBicycles})

\end{proof}






\section{Location of the   transition for
}\label{sec:transition}


We consider formulas  built on  universal variables and  existential variables. Thus we have 
 different clauses at hand.  
 We may  establish our result in considering random formulas obtained by taking
each one of the  possible clauses
  independently from the others with probability . Let , 
it is well known, see for instance  \cite[Sections 1.4 and 1.5]{JansonLR-00},  that the threshold obtained in this model translates to the
model alluded to in the introduction -- in which , distinct clauses are picked
uniformly at random among all the  possible choices --,  when .
Thus, from now on we shall always suppose that
, and we continue to denote by  the probability
that a random formula in this model is satisfiable. We are
interested in studying 
as a function of the parameters  and . Any value of  such
that  (resp.\ such that  gives a
lower (resp.\ upper) bound for the threshold effect associated to
the phase transition.\\

 Let us recall that the  property exhibits a sharp transition, with a
critical value equal to 1 (see \cite{ChvatalR-92} and \cite{Goerdt-96}). From
this result it is easy to deduce that the phase transition from satisfiability
to unsatisfiability for  
 \onetwoqcnf\ formulas occurs
when . 
 
\begin{proposition}\label{prop:first_estimates}
  Let  be any sequence of integers.
  \begin{itemize}
  \item If  then .\\
  \item If  then .
  \end{itemize}

\end{proposition}
\begin{proof}
Let  be  a random -formula.
 Let us consider , the
 formula obtained from  by setting all the variables
   to \emph{true} and omitting all quantifiers. If
   is satisfiable, then so is . Notice that  can be
  obtained by picking independently each possible 2-clause with
  probability
   Thus
  the average number of clauses in  is equal to   It follows from the threshold of 2-SAT
  \cite{ChvatalR-92,Goerdt-96} that  is unsatisfiable with
  probability tending to 1 if . Thus, the same holds for .

  Now, we look at the existential part of the formula, . Observe
  that if  is satisfiable, then so is .  In , each of the
   2-clauses appears independently with probability
  
  Therefore,
  the threshold of 2-SAT tells us that when , the formula 
  is satisfiable with probability tending to one. The same holds
  for .
\end{proof}
\medskip

\iffalse

\subsection{Empirical Results }\label{subsec:experiments}
As stated in the introduction one of the motivation for studying  the  problem is that it can bear practical experiments at a reasonable scale. Such experiments can confirm theoretical results and also guide further investigations.

All experiments reported here  have
been conducted according to the same scheme, which can be
described with the help of  Figure \ref{fig:threshold}. One
experiment consisted in generating at random (in drawing uniformly
and independently)  formulas for given values of 
universal variables and  existential variables, with a ratio
``number of clauses/number of existential variables'' varying from
0.85 to 1.2 in steps of 0.05. In Figure~\ref{fig:threshold}, the
values for  and  are 5000, 10000, 20000 and 40000. For each
of the chosen values of ratio, a sample of 1000 formulas have been
studied using the QBF solver QuBE \cite{Giunchiglia01a}, thus
computing the truth value of each formula. The proportion of true
(or satisfiable) instances for each considered value of ratio has
been plotted in Figure~\ref{fig:threshold}. 

First, experiments suggest that there is indeed a sharp transition for  and confirm  that the critical ratio is between
1 and 2. Moreover, the value of the threshold seems to be a decreasing function of . Two extremal cases, when  large (  of the same order as ) and  when  is small ( constant)  are illustrated here. 
The experimental results shown in Figure \ref{fig:threshold}
show  that, when , the transition between satisfiability
and unsatisfiability occurs when the ratio of the number of
clauses to the number of existential variables, , is equal to
1. It also  shows that if  is constant,
, then the transition occurs at .  



\begin{figure}[!h]
\begin{tabular}{cc}
\includegraphics[scale=0.5]{res-1-2-nk-nk.ps} 
& \includegraphics[scale=0.5]{res-1-2-2-nk.ps} \\
\end{tabular}
\caption{ when  or : threshold  at  or   .}
  \label{fig:threshold}
\end{figure}


Moreover, the
experiments reported in Figure~\ref{fig:curves-moving} indicate
that an intermediate regime, with a transition occurring 
between 1 and 2, can also be observed.




\begin{figure}[th]
  \centering
  \includegraphics[scale=0.9]{res-1-2-moving.ps}
  \caption{ when  is varying.}
  \label{fig:curves-moving}
\end{figure}




The curves
shown in Figure \ref{fig:curves-moving} suggest that a good
candidate to look at is when  is of logarithmic order with
respect to . Indeed, each of the curves in this figure
corresponds to  for some value
, respectively for ,  and . In
Figure \ref{fig:running-time} we present three curves for
 and . Besides the usual
``satisfiability curves'', we have added a fourth curve showing the
average solving time for each of the 1000 problems in the set
having a specific ratio of clauses and existential variables. As
expected one recognizes a pattern of the form ``easy-hard-less
hard''. The problems hardest to solve occur around the crossing
point of the three curves.



\begin{figure}[th]
  \centering
 \includegraphics[scale=0.9]{res-1-2-15-8-m-n.ps}
  \caption{Average running time of the evaluation process.}
  \label{fig:running-time}
\end{figure}
\medskip

\fi


\section{Proof of the main result} \label{sec:main}

\subsection{General inequalities}
Let  and  be respectively the
number of pure bicycles and pure snakes of length    in a random
\onetwoqcnf  \ formula.     Les us recall that in such a formula, each
clause is chosen with probability . Hence, if   and 
denote   the average number of  bicycles and snakes of length  in
a random \onetwoqcnf  \ formula,    we get from (\ref{eqn:nbSnakes}),
(\ref{eqn:cms}) and (\ref{eqn:nbBicycles}) the following two equations:
    
    
    In order to prove that  is the critical value for the (decreasing) satisfiability  property for  -formulas, we will use  two  sequences of inequalities. The first one follows from    Proposition \ref{prop:certificates} and Markov
inequality applied on the number of bicycles. We have 

The second one is obtained in   considering  the number of snakes. Proposition
\ref{prop:certificates} and  a general exponential inequality given in
\cite[Theorem 2.18  ii)]{JansonLR-00} show that   for any  


Finally,  recall that we can  suppose that , according to  Proposition \ref{prop:first_estimates}. 

\subsection{When the critical ratio is equal to }\label{subsec:c=2}

Let us start with a proposition which enables to control the mean
number of bicycles for any  in .

\begin{proposition}\label{prop:bicyclebound}
 For any , the following statements hold when   tends to  infinity 
\begin{itemize}
\item 
if   then  


\item  if   with  then   
\end{itemize}
\end{proposition}
\begin{proof}
Let us recall that the coefficient  occurring in  is  the number of pure sequences of literals of length
  , when we have    variables from which the literals can be
  built.   Note that  is bounded
from above by  times the number of applications
from  to . Therefore,

From (\ref{Meanbicycles}),  it follows that  if    then  . Thus 

If ,  then (\ref{eq:majostirlingtriviale}) gives  When  and , standard computations show
that

Hence  we get 

The proof  of  Proposition \ref{prop:bicyclebound}  is now an easy consequence of  (\ref{Majsnake1}) and (\ref{Majsnake2}). 
 \end{proof}




 Theorem \ref{thm:main} when  follows from Proposition
 \ref{prop:bicyclebound}, inequality (\ref{eqn:first_moment})  and
 Proposition \ref{prop:first_estimates}.


In the sequel, we consider the case where , with . 

\subsection{The critical ratio as a function of }


 The main  difficulty when dealing with  and  is to handle the coefficient  given in Proposition \ref{prop:uppercertificatespure} 

    
    First, let us denote for   
         
From (\ref{MeanSnakes}) and (\ref{Meanbicycles}), the behavior of  and  is clearly governed by the  coefficients . Indeed,  since  we get      
         

Second, we will need   better bounds than the one given in (\ref{eq:majostirlingtriviale}). We will use  well-known estimates for binomial coefficients.  If , then the following inequalities hold:

Then, from  \cite{Temme-93}, we have the following bounds
for Stirling numbers of the second kind. There exist  
 and  such that, for , the following inequalities hold:
 
where  is a function of  defined implicitly for  by
 , and for  by .
The conventions are that  and .

By using these precise results, already used in \cite{DuboisB-97} and \cite{CreignouDE-07},  it appears that the behaviour of the coefficients  and so the one of the average number of snakes or bicycles, is  governed by a continuous function of several real variables. From (\ref{keyexpression}), (\ref{binomialbounds}) and (\ref{Stirlingbounds})  we obtain: 
\begin{proposition}\label{thebig} There exist  and  such that   for any  , for every positive  integers  and  such that     :  
  
 where  is the continuous  function  on   defined for  by
 
 with  and  
  \end{proposition}

  Recall that we have taken . Observe
  that the second part of Proposition \ref{prop:bicyclebound} together with
  (\ref{Meanbicycles}) indicates that  long  snakes, and similarly
  long bicycles, of length , have asymptotically no chance
  to appear when  and .  Therefore, in our study we will  focus  on snakes  of length proportional to .    Hence, let us
set , . The following result will point out for each , the values of  and  that contribute the most to the average number. Indeed we  will prove the following central result :
\begin{proposition}\label{prop:function g}
 Let  , and  for any  let  be the following domain   
 
 The function  
 defined by (\ref{eqn:def-of-g})  has a  global maximum on , given by its unique stationarity point in . More precisely 

with

     \ 


Moreover, for any domain  such that  then 


\end{proposition}
The proof of this result is rather technical, so we postpone it to the next section.\\

Now we can prove  Theorem \ref{thm:main} when . In other words that,  when  
, the critical ratio  is  the unique root of   For this,  we will use  two  corollaries of Proposition \ref{thebig} and Proposition \ref{prop:function g}.
\begin{corollary}\label{lowerbound}
Let  and  be such that .  Then, as  tends to infinity 
 
\end{corollary}
\begin{proof} From Proposition \ref {prop:bicyclebound}, we have  Then, from (\ref {keyrelation}), the upper bound (\ref {minMaj}) and (\ref{Themax}) we get 
 
 with . Therefore,
  
\end{proof}\medskip

With (\ref{eqn:first_moment}), this corollary  proves that,  when  and   for any , we have  . \smallskip

In considering (\ref{eqn:Janson}) with ,  it will follow easily from the  corollary  given below
that, when ,  for any  we have
. This will end the proof of Theorem \ref{thm:main}. Note that the coefficients  appearing in the following corollary are the ones defined in Proposition \ref{prop:uppercertificatespure}.
\begin{corollary}\label{upperbound}
Let  and  be such that , and let
. Then there exist
,  and  such that   


and 

\end{corollary}
\begin{proof} 
From  (\ref{Thepasmax}) in Proposition \ref{prop:function g}, we first choose  such that 
 
Again in using (\ref {keyrelation}) and   the lower bound in (\ref{minMaj}), we can find  such that for  

As , the first assertion is proved.


Then, with ,   from (\ref{majo1N}) and  (\ref{majo2N})  we get first for  

and second for  

At last, in using (\ref{minMaj}) with  and with our choice for  we obtain 




\end{proof}



 



\section{Proof of Proposition \ref{prop:function g}}\label{sec:technical}


Let us recall that  for any  and  ,  we consider the  domain   
  for the function  given from (\ref{eqn:def-of-g}) by  

   
 
 
 with   defined implicitly  when  by 
 
 In the sequel, we shall write  for  and  for .\\
 
  Proposition \ref{prop:function g} tells us that  has a  strict and global maximum on   which is equal to  with
 The proof of   Proposition \ref{prop:function g}  follows from  the  following claim :


\begin{claim}
\label{onevariableproof} For any  and  ,
 
 \begin{enumerate}
\item  for every fixed  with , the function 
 is strictly concave on  with a strict  maximum at 

\item the function   is strictly concave on  with a  maximum at  , then  with 
     
\end{enumerate}
\end{claim}

\begin{proof} 
 For the first point of this claim we compute, from (\ref{eqn-of-g}) and (\ref{x0}), the partial derivatives of  with respect to . We get
  
      With  (\ref{x0}) we  first   observe that  
      
Then

let . The function  is
decreasing with . Hence,  and

From the second identity in (\ref{partialderivatives}), (\ref{eqn:first_ineq}) and (\ref{eqn:second_ineq}) we conclude that . The  strict concavity of  follows.
Then the first identity in (\ref{partialderivatives}) and  (\ref{x0}) give the expected formula for the unique extremum, indeed we obtain 
  

\medskip

 For the second point of the claim, observe that with  (\ref{eqn-of-g}) we have :

 thus from (\ref{gammabeta}) we obtain 
  
 where for any  , .
   is strictly concave on  and reaches its maximum at . From (\ref{g-in-gammabeta}) with  we get 
. Then, with (\ref{gammabeta}) we obtain   

At last, observe that  , so 
        and   give the coordinates of the unique stationarity point of , that is the unique solution of
  .
 \\


\end{proof}



 \section{Conclusion}
 



We have performed an extensive study of a natural and expressive
quantified problem, . We have proved the existence of
a sharp phase transition from satisfiability to unsatisfiability
for -formulas and we have given the exact location of the
threshold. The obtained results have several interesting features.
The  parameter  , which is the number of universal variables,
controls the worst-case computational complexity of the problem
(which is ranging from linear time solvable to -complete),
as well as the typical behavior of random instances.    When 
is small, there is a sharp threshold at . On the other side,
when  is large enough, actually   when , there is a
sharp threshold at : the analysis is similar, and in fact
easier, to what we have done for pure snakes in Section
\ref{sec:main}, in considering snakes with strictly distinct
universal variables, as shown in \cite{CreignouDER-08}. This fact should be compared to the fact
that the threshold location  for  goes to 1 when  goes to infinity. More
importantly, an original regime is observed when . Using counting arguments on pure bicycles,
which are the seed of unsatisfiability, and on pure snakes, which
are special minimally false formulas, we got respectively a lower
and an upper bound for the threshold. It turns out that these two
bounds coincide, thus giving the exact location of the threshold
as a function of . 

A challenging question would be to determine the scaling window around
 and get precise information on the typical contradictory
cycles that occur in random formulas inside this window.  

\iffalse

Experiments, which make use of a current QBF solver, have played an
important role in our study. Indeed, they have been carried out at a
scale large enough in order to give a useful intuition on the
behavior of random instances. However they have also
provided evidence that the asymptotical behavior of random quantified
instances is difficult to reach.
In particular experiments that would aim at measuring the width of the
transition   from satisfiability to unsatisfiability (see e.g.,
\cite{Wilson-02}) seem to be out of scope.
This makes the study of the sharpness of the  transition for 
(as performed in \cite{Bollobasetal01} for ) a challenging task.



\fi



\begin{thebibliography}{10}

\bibitem{AspvallPT-79}
B.~Aspvall, M.~F. Plass, and R.~E. Tarjan.
\newblock A linear-time algorithm for testing the truth of certain quantified
  boolean formulas.
\newblock {\em Information Processing Letters}, 8(3):121--123, 1979.

\bibitem{Bollobasetal01}
B.~Bollob{\'a}s, C.~Borgs, J.T. Chayes, J.H. Kim, and D.B. Wilson.
\newblock The scaling window of the 2-{SAT} transition.
\newblock {\em Random Structures and Algorithms}, 18(3):201--256, 2001.

\bibitem{ChenI-05}
H.~Chen and Y.~Interian.
\newblock A model for generating random quantified boolean formulas.
\newblock In {\em Proceedings of the 19th International joint Conference on
  Artificial Intelligence (IJCAI 2005)}, pages 66--71, 2005.

\bibitem{ChvatalR-92}
V.~Chv\'atal and B.~Reed.
\newblock Mick gets some (the odds are on his side).
\newblock In {\em Proceedings of the 33rd Annual Symposium on Foundations of
  Computer Science (FOCS 92)}, pages 620--627, 1992.

\bibitem{CreignouDE-07}
N.~Creignou, H.~Daud\'e, and U.~Egly.
\newblock Phase transition for random quantified {XOR}-formulas.
\newblock {\em Journal of Artificial Intelligence Research}, 19(1):1--18, 2007.

\bibitem{CreignouDER-08}
N.~Creignou, H.~Daud\'e, U.~Egly, and R.~Rossignol.
\newblock New results on the phase transition for random quantified {B}oolean
  formulas.
\newblock {\em Proceedings of the 11th
  International Conference on Theory and Applications of Satisfiability Testing
  (SAT 2008)}, volume 4996, pages 34--47. Lecture Notes in Computer Science,
  2008.

\bibitem{CreignouDER-09}
N.~Creignou, H.~Daud\'e, U.~Egly, and R.~Rossignol.
\newblock (1,2)-{QSAT}: A good candidate for understanding phase transitions
  mechanisms.
\newblock {\em Proceedings of the 12th
  International Conference on Theory and Applications of Satisfiability Testing
  (SAT 2009)}, volume 5584, pages 363--376. Lecture Notes in Computer Science,
  2009.

\bibitem{Vega-01}
W.~Fernandez de~la Vega.
\newblock Random 2-{SAT}: results and problems.
\newblock {\em Theoretical Computer Science}, 265(1-2):131--146, 2001.

\bibitem{DuboisB-97}
O.~Dubois and Y.~Boufkhad.
\newblock A general upper bound for the satisfiability threshold of random
  r-{SAT} formulae.
\newblock {\em Journal of Algorithms}, 24(2):395--420, 1997.

\bibitem{Egly00c}
U.~Egly, T.~Eiter, H.~Tompits, and S.~Woltran.
\newblock {S}olving {A}dvanced {R}easoning {T}asks {U}sing {Q}uantified
  {B}oolean {F}ormulas.
\newblock In {\em Proceedings of the 17th National Conference on Artificial
  Intelligence and the 12th Innovative Applications of Artificial Intelligence
  Conference (AAAI/IAAI 2000)}, pages 417--422. AAAI Press / MIT Press, 2000.

\bibitem{FloegelKKB-90}
A.~Fl{\"o}gel, M.~Karpinski, and H.~Kleine B{\"u}ning.
\newblock Subclasses of quantified {B}oolean formulas.
\newblock In {\em Proceedings of the 4th Workshop on Computer Science Logic
  (CSL 90)}, pages 145--155, 1990.

\bibitem{GentW-99}
I.P. Gent and T.~Walsh.
\newblock Beyond {NP}: the {QSAT} phase transition.
\newblock In {\em Proceedings of AAAI-99}, 1999.

\bibitem{Goerdt-96}
A.~Goerdt.
\newblock A threshold for unsatisfiability.
\newblock {\em Journal of of Computer and System Sciences}, 53(3):469--486,
  1996.

\bibitem{JansonLR-00}
S.~Janson, T.~Luczack, and A.~Rucinski.
\newblock {\em Random graphs}.
\newblock John Wiley, 2000.

\bibitem{Temme-93}
N.M. Temme.
\newblock Asymptotic estimates of {S}tirling numbers.
\newblock {\em Stud. appl. Math.}, 89:223--243, 1993.

\bibitem{Verhoeven-99}
Y.~Verhoeven.
\newblock Random 2-{SAT} and unsatisfiability.
\newblock {\em Information Processing Letters}, 72(3-4):119--123, 1999.

\end{thebibliography}

\end{document}
