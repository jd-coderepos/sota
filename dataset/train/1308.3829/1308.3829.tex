\documentclass{article}
\usepackage{graphicx}
\usepackage{comment}
\usepackage{amssymb} 
\usepackage{amsmath}
\newtheorem{theorem}{Theorem}
\newtheorem{lemma}{Lemma}
\newtheorem{proposition}{Proposition}
\newtheorem{corollary}{Corollary}
\newtheorem{example}{Example}
\date{}


\title{On OBDDs for CNFs of bounded treewidth}
\author{Igor Razgon\\ Department of Computer Science and Information Systems,\\ Birkbeck, University of London\\
        igor@dcs.bbk.ac.uk}
\begin{document}
\maketitle
\begin{abstract}
Knowledge compilation is a rewriting approach to propositional knowledge representation.
The `knowledge base' is initially represented as a {\sc cnf} for which many important types of
queries are {\sc np}-hard to answer. Therefore, the {\sc cnf} is compiled into another
representation for which the minimal requirement is that the clausal entailment query
(can the given partial assignment be extended to a complete satisfying assignment?)
can be answered in a polynomial time \cite{DerMar}. Such transformation can result in exponential
blow up of the representation size. A possible way to circumvent this issue is to
identify a structural parameter of the input {\sc cnf} such that the resulting transformation
is exponential in this parameter and polynomial in the number of variables.
A notable result in this direction is an  upper bound on the size of
Decomposable Negation Normal Form ({\sc dnnf}) \cite{DarwicheJACM}, where  is the
number of variables of the given CNF and  is the treewidth of its primal graph. Quite recently
this upper bound has been shown to hold for Sentential Decision Diagrams ({\sc sdd}) \cite{SDD}, 
a subclass of {\sc dnnf} that can be considered as a generalization of the famous Ordered Binary Decision Diagrams
({\sc obdd}) and shares with the {\sc obdd} the key nice features (e.g. poly-time equivalence testing). Under the 
treewidth parameterization, the best known upper bound for an {\sc obdd} is  \cite{VardiTWD}. 
A natural question is whether, similarly to {\sc sdd}, a fixed parameter upper bound holds for {\sc obdd}.

We provide a negative answer to the above question. In particular, for every fixed , 
we demonstrate an infinite class of {\sc cnf}s of the primal graph treewidth at most  for which the 
{\sc obdd} size is , essentially matching the upper bound of \cite{VardiTWD}.
This result establishes a \emph{parameterized} separation of {\sc obdd} from {\sc sdd}. We further show that
the considered class of instances can be transformed into one for which the {\sc obdd} size is at least 
and the {\sc sdd} size is  thus separating {\sc obdd} from {\sc sdd} in the \emph{classical sense}. 


We also provide a more optimistic version of the  upper bound for the {\sc obdd} showing
that it in fact holds when  is the treewidth of the incidence graph of the given {\sc cnf}.
\begin{comment}
Thus together with the upper bound of \cite{VardiTWD},
the results of this paper provide a complete classification of the expressive power of {\sc obdd} for
{\sc cnf}s parameterized by the treewidth of their primal and incidence graphs.
\end{comment} 
\begin{comment}
In \cite{VardiTWD}, it is shown that a CNF of  variables and the treewidth
 of its primal graph can be compiled into an Ordered Binary Decision Diagram (OBDD)
of size . In this paper we show that there is an infinite class of CNFs for 
which the OBDD size is at least  thus essentially matching the above upper bound.
To the best of our knowledge, this is the first result ruling the possibility of fixed-parameter
tractable size OBDDs for CNFs of bounded treewidth. We also show that the  upper bound
holds for the case where  is the treewidth of the incidence graph of the given {\sc cnf}.
\end{comment}
\end{abstract}

\section{Introduction}
Knowledge compilation is a rewriting approach to propositional knowledge representation.
The `knowledge base' is initially represented as a {\sc cnf} or even as a Boolean circuit. 
For these representations many important types of queries are {\sc np}-hard to answer.
Therefore, the initial representation is compiled into another one 
for which the minimal requirement is that the clausal entailment query
(can the given partial assignment be extended to a complete satisfying assignment?)
can be answered in a polynomial time \cite{DerMar}. Such transformation can result in exponential
blow up of the representation size. A possible way to circumvent this issue is to
identify a structural parameter of the input {\sc cnf} such that the resulting transformation
is exponential in this parameter and polynomial in the number of variables.
A notable result in this direction is an  upper bound on the size of
Decomposable Negation Normal Form ({\sc dnnf}) \cite{DarwicheJACM}, where  is the
number of variables of the given {\sc cnf} and  is the treewidth of its primal graph.
Quite recently, the same upper bound has been shown to hold for Sentential Decision Diagrams ({\sc sdd}) \cite{SDD},
a subclass of {\sc dnnf}  
that can be seen as a generalization
of the famous Ordered Binary Decision Diagrams ({\sc obdd}) and 
shares with the {\sc obdd} the key nice features (e.g. poly-time equivalence testing). It is known that a {\sc cnf} of treewidth
 can be compiled into an {\sc obdd} of size  \cite{VardiTWD}. A natural question is whether 
{\sc obdd}, similarly to {\sc sdd}, admits a fixed-parameter upper bound of form  for some constant . 



In this paper we provide a negative answer to this question. In particular, we demonstrate an infinite class of
{\sc cnf}s of the primal graph treewidth at most  for which the {\sc obdd} size is at least
 where  is a function exponentially small in . In other words, we show
that the {\sc obdd} size of these {\sc cnf}s is  for every fixed .
This result provides a \emph{parameterized} separation from {\sc sdd} and essentially matches the upper bound of \cite{VardiTWD}. 
In fact, this result shows impossibility of not only a fixed-parameter upper bound, but also 
of a sublinear dependence on  in the base of the exponent or even of 
an exponent  for some large constant . Moreover, a corollary of this result is that there is
an infinite class of instances (obtained, roughly speaking, by setting ) on which the {\sc obdd} 
size is at least , while the {\sc sdd} size is  thus separating {\sc obdd} 
from {\sc sdd} in the \emph{classical sense}.


Our second result is `strengthening' of the upper bound  of \cite{VardiTWD} by showing that it
holds if  is the treewidth of the \emph{incidence} graph of the given {\sc cnf} thus extending the 
upper bound to the case of sparse {\sc cnf}s with large clauses. 

\begin{comment}
Taking into account that  lower bound presented in this paper applies when  is 
the treewidth of the incidence
graph (the treewidth of the incidence graph is at most the treewidth of the primal graph plus one),
we conclude that, together with the upper bound of \cite{VardiTWD},
the results of this paper provide a \emph{complete classification} of the expressive power of {\sc obdd} for
{\sc cnf}s parameterized by the treewidth of their primal and incidence graphs. 
\end{comment}

\begin{comment}
In \cite{VardiTWD}, Ferrara, Pan, and Vardi showed that for any {\sc cnf} with 
variables and treewidth  of the primal graph can be compiled into an Ordered Binary 
Decision Diagram ({\sc obdd}) of size
. A natural question is whether there is a Fixed-Parameter upper bound on the
size of the {\sc obdd}, i.e. the one represented in the form  where 
is some exponential (or even superexponential) function and  is a constant independent
on . In this paper we prove that this is impossible. In particular, we show 
that for each  there is a function 
and an infinite class of {\sc cnf}s with treewidth at most  of the 
primal graph such that for each element  of this class the size of the smallest
possible {\sc obdd} computing  is at least .
This result shows impossibility of not only a fixed-parameter upper bound, but also 
of a sublinear dependence on  in the base of the exponent or even of 
an exponent  for some large constant . On the positive side, we show that 
the  upper bound holds for the case where  is the treewidth of the 
incidence graph of the given {\sc cnf}. To the best of our knowledge the lower bound reported 
in this paper is the first one ruling out the possibility of fixed-parameter size 
{\sc obdd}s for {\sc cnf}s with a fixed treewidth. 
\end{comment}

\begin{comment}
Our motivation to consider these questions comes from the knowledge compilation perspective.
Indeed, according to \cite{DerMar}, {\sc obdd} can be thought as a special case of {\sc dnnf}
and stronger formalisms such as {\sc dnnf} \cite{DarwicheJACM} and {\sc sdd} \cite{SDD} are known to be size 
{\sc fpt} parameterized by the treewidht of the input {\sc cnf}. Therefore, in order to completely
classify the subclasses w.r.t this parameterization, it is important to understand it regarding
{\sc obdd}. To the best of our knowledge the lower bound reported in this paper is the first one 
ruling out the possibility of fixed-parameter size 
{\sc obdd}s for {\sc cnf}s with a fixed treewidth. 
\end{comment}


In order to obtain the parameterized lower bound, we introduce a notion of \emph{matching width} of a graph
and prove that if a {\sc cnf}  of the considered class
has matching width  of the primal graph then for any ordering of the variables of 
there is a prefix  such that the number of distinct functions that can be obtained 
from  by assigning the variables of  is at least .
This will immediately imply that any {\sc obdd} realizing
 will have at least  nodes. Finally we will prove that the matching width of the considered
{\sc cnf}s is . Substituting this lower bound instead  will get the
desired lower bound for the {\sc obdd} size. 

Similarly to the case of primal graph, the upper bound is obtained by showing that if \emph{pathwidth}
of the incidence graph of the given {\sc cnf} is at most  then this {\sc cnf} can be compiled into 
an {\sc obdd} of size . Then the  upper bound is obtained using a well known relation
 between the treewidth and the pathwidth of the given graph. The approach to obtain the 
 bound is similar to \cite{VardiTWD}: variables are ordered 'along' the path decomposition
and it is observed that the for each prefix the number of functions caused by assigning the 'previous'
variables is . The technical difference is that in our case the bags of the path decomposition
include clauses and this circumstance must be taken into account. 

The proposed results contribute to a large body of existing results concerning the space complexity
of {\sc obdd}s. To begin with, there are many results concerning the complexity of {\sc obdd}s for 
\emph{particular} classes of Boolean functions, see e.g. the book \cite{WegBook} and the survey \cite{WegSurvey}. 
The space complexity of {\sc obdd} remains polynomial if parameterized by the treewidth of a \emph{circuit} representing the given 
function \cite{OBDDTWJha}, however the dependence on the treewidth becomes double exponential. 
A fixed-parameter upper bound can be achieved if tree of {\sc obdd}s is used instead of a single
{\sc obdd} \cite{McMillan94,SubbaTree}. In the complexity theory the {\sc obdd} is classified
as the \emph{oblivious read-once branching program}, see the book \cite{Yukna} for the results concerning the
complexity of branching programs on particular classes of formulas

The proposed lower bound also contributes to the understanding of relationship between {\sc obdd}
and {\sc sdd}. Other results in this direction are \cite{SDDvsOBDD} showing an exponential separation
between {\sc sdd} and {\sc obdd} based on the same order of variables (the order of variables for
{\sc sdd} is defined as the order of visiting the corresponding nodes of the underlying \emph{vtree}
by a left-right tree traversal algorithm) and \cite{DynSDD} empirically showing that conceptually similar 
heuristics produce {\sc sdd}s orders of magnitude smaller than {\sc obdd}s. 

\begin{comment}
{\sc obdd} is a way of representation of Boolean functions that 
is widely used in model checking \cite{OBDD} due to the possibility to efficiently answer a number of 
important queries regarding the properties of these functions. There are many results concerning the space complexity 
of {\sc obdd} representing \emph{particular}
classes of formulas, see e.g. the book \cite{WegBook} and the survey \cite{WegSurvey}. 
In the complexity theory the {\sc obdd} is classified
as the \emph{oblivious read-once branching program}, see the book \cite{Yukna} for the results concerning the
complexity of branching programs on particular classes of formulas. 
\end{comment}
\begin{comment}
It is shown in \cite{VardiTWD} that if the primal graph of a {\sc cnf} has the pathwidth  then there is
an {\sc obdd} of size . In fact the upper bound for the {\sc obdd} size parameterized by the 
treewidth has been obtained by substituting into  the well known upper bound  
for  where  is the treewidth of the primal graph.
\end{comment}
\begin{comment}
The space complexity of {\sc obdd} remains polynomial if parameterized by the treewidth of a \emph{circuit} representing the given 
function \cite{OBDDTWJha}. {\sc obdd} can be thought as a \emph{good} representation of Boolean functions, i.e. the one
on which it takes a poly-time to test if the given partial truth assignment  can be extended to a truth assignment to all
the variables so that the given function becomes true (we borrowed the term `good representation' from \cite{GoodSat}).
There are a number of powerful good representation of which {\sc obdd} can thought as a special case whose space complexity
is fixed-parameter being parameterized by the treewidth of the initial representation. 
For example, a Boolean circuit of treewidth  can be translated into a fixed-parameter size of tree  of {\sc obdd}s 
\cite{McMillan94,SubbaTree}. An {\sc obdd} can be seen as a special case of a decomposable negation normal form ({\sc dnnf})
which has fixed parameter size for functions representable by {\sc cnf}s of treewidth  of the primal graph.
A generalization of this result has been provided in \cite{RazPet} showing that the {\sc dnnf} in fact admits a fixed-parameter
size representation for \emph{circuits} of bounded treewidth and even of bounded cliquewidth. The fixed-parameter size of {\sc dnnf} for bounded width {\sc cnf}s is
preserved for sentential decision diagrams ({\sc sdd}) \cite{SDD} that can be thought as a representation between {\sc dnnf} and {\sc obdd}.
This context implies an additional motivation of the lower bound proposed in this paper because it is fact a \emph{parameterized separation}
of {\sc obdd} from its more powerful generalizations.
\end{comment}
\begin{comment}
To keep the representation size fixed-parameter in the treewidth size, {\sc obdd} has been upgraded
to the tree of {\sc obdd}s \cite{McMillan94,SubbaTree}. 
\end{comment}
\begin{comment}
In addition to the  upper bound mentioned above, \cite{VardiTWD} also considers the case where some variables of the given 
{\sc cnf} are existentially quantified and proves a lower bound of  where  is an exponential function. 
This result shows that the
operation of projection (variable elimination) causes exponential blow-up in the {\sc obdd} size. 
It is interesting to note that for a more general class {\sc dnnf} the projection operation effectively
does not increase the representation size at all \cite{DarwicheJACM}. In other words, if we consider an {\sc obdd}
as a {\sc dnnf} \cite{DerMar} and apply the projection operation as specified in \cite{DarwicheJACM}, the 
resulting representation will not necessarily be an {\sc obdd}.
\end{comment}

The rest of the paper is structured as follows.
The next section introduces the necessary background. 
The section after that proves the lower bound, the proofs of auxiliary statements 
are provided in the two following sections. 
Then follows the section presenting the
upper bound for the parameterization
by the treewidth of the incidence graph. 



\section{Preliminaries} \label{prelim}
The structure of this section is the following. First, we introduce
notational conventions. Then we define the {\sc obdd} and specify the approach
we use to prove the lower bound. Next, we introduce terminology related
to {\sc cnf}s. Finally, we define the notion of treewidth.

In this paper by a \emph{set of literals} we mean one that does not
contain an occurrence of a variable and its negation.
For a set  of literals we denote by  the set of variables
whose literals occur in . If  is a Boolean function
or its representation by a {\sc cnf} or {\sc obdd}, we denote by 
the set of variables of . A truth assignment to  on which 
is true is called a \emph{satisfying assignment} of . A set  of literals
represents the truth assignment to  where variables occurring
positively in  (i.e. whose literals in  are positive) are assigned with 
and the variables occurring negatively are assigned with .
We denote by  a function whose set of satisfying assignments consists of 
such that  is a satisfying assignment of . We call  a \emph{subfunction}
of . In other words, a Boolean function  is a subfunction of a Boolean function
 is  can be obtained from  by giving a truth assignment to a subset of variables of .

An {\sc obdd}  representing a Boolean function  is a directed acyclic graph ({\sc dag}) with one root and two leaves
labelled by  and . 
The internal nodes are labelled with variables of . There is a fixed permutation  of 
(that is, elements of  are linearly ordered according to )
so that the vertices along any path from the root to a leaf are labelled with variables according to this order. 
Each internal vertex is associated with  leaving edges labelled with  and .
Each path  from the root of  is called a \emph{computational path} and is associated with truth assignment to
the variables labelling all the vertices but the last one. In particular, each variable is assigned with the value labelling
the edge of the path that leaves the corresponding vertex. We denote by  the assignment associated with the 
computational path . The set of all  where  is a computational path ending at the  leaf is precisely
the set of satisfying assignments of . 

\begin{figure}[h]
\centering 
\includegraphics[height=5cm]{KROBDD.pdf}
\caption{An {\sc obdd} for 
under permutation }
\label{OBDDPic}
\end{figure}

Figure \ref{OBDDPic} shows an {\sc obdd} for the function 
under the permutation . Consider the path . 
Then . 

\begin{comment}
Consider an arbitrary internal vertex of , not necessarily the root and let
 be the induced subgraph of  (retaining the labeling on vertices and edges) consisting of all vertices reachable
from . It is not hard to see that  is an {\sc obdd}. Moreover, let  be a path from the root of  to .
Then it is not hard to see that the function  represented by  if . This observation implies
a corollary leading to a widely used approach to establishing lower bounds on {\sc obdds}, 
\end{comment}

In order to obtain the lower bound on the {\sc obdd} size we use a standard approach
of counting subfunctions. See \cite{WegBook} for examples of application of this approach.
This approach is based on the following statement.

\begin{proposition} \label{paths}
Let  be a Boolean function on a set  of variables and let  be a permutation
of . Partition  into a prefix 
and a suffix  and suppose that the number of distinct subfunctions of  obtained by giving
truth assignments to all the variables of  is at least . Then an {\sc obdd} of  with
the underlying order  contains at least  nodes. 
\end{proposition}

The standard way to utilize Proposition \ref{paths} is to show that for \emph{any} permutation
 of  there is a partition of  into a prefix  and a suffix  such that
the instantiation of variables of  results in at least  different subfunctions.
Then Proposition \ref{paths} immediately implies that  is a lower bound on the size of {\sc obdd}
for \emph{any} underlying order.

\begin{comment}
\begin{proposition} \label{paths}
Let  and  be  computational paths of  such that  Then 
 and  have distinct final vertices. 
\end{proposition}

It immediately follows from Proposition \ref{paths} that if there are computational paths  
such that  are pairwise distinct then  has at least  vertices. 
If  is superpolynomial compared
to the number of variables  then this means that a superpolynomial lower bound has been established.
\footnote{This is a slight modification of the approach used in \cite{OBDDComp} (Lemma 2). 
The set  is called in \cite{OBDDComp}
a \emph{fooling set} of the considered {\sc obdd}.}
\end{comment}

Given a {\sc cnf} , its \emph{primal graph} has the set of vertices corresponding to the variables of .
Two vertices are adjacent if and only if there is a clause of  where the
corresponding variables both occur. In the \emph{incidence graph}
of  the vertices are partitioned into those corresponding to the variables of  and those corresponding to its
clauses. A variable vertex is adjacent to a clause vertex if and only if the corresponding variable occurs in the
corresponding clause. 

Given a graph , its \emph{tree decomposition} is a pair  where  
is a tree and  is a set of bags  corresponding to the vertices  of .
Each  is a subset of  and the bags obey the rules of \emph{union} (that is, ),
\emph{containment} (that is, for each  there is  such that ),
and \emph{connectedness} (that is for each , the set of all  such that  induces a subtree of ).
The \emph{width} of  is the size of the largest bag minus one. The treewidth of  is the smallest width of a tree
decomposition of . If  is a path then we use the respective notions of \emph{path decomposition} and \emph{pathwidth}. 


\begin{figure} [h]
\centering
\includegraphics[height=5cm]{KRTWD.pdf}
\caption{A graph and its tree decomposition}
\label{TWDPic}
\end{figure}

Figure \ref{TWDPic} shows a graph and its tree decomposition.
The width of this tree decomposition is  since the size of the largest bag
is .

\section{The lower bound} \label{lbmain}
\begin{comment}
\begin{theorem}
Given a parameter  there is a family
 of CNFs such that each 
has  variables and  treewidth of the primal graph
and the smallest OBDD realizing each  is of size .
\end{theorem}
\end{comment}
In this section, given two integers  and  we define a class of {\sc cnf}s, 
roughly speaking, based on complete binary trees of height  where each node is associated 
with a clique of size . Then we prove that the treewidth of the primal graphs of {\sc cnf}s of this
class is linearly bounded by . Further on, we state the main technical theorem (proven in the next
section) that claims that the smallest {\sc obdd} size for {\sc cnf}s of this class exponentially depends
on . Finally, we re-interpret this lower bound in terms of the number of variables and the treewidth to get
the lower bound announced in the Introduction.


Let  be a graph. A \emph{graph based} {\sc cnf} denoted by  is defined as follows. 
The set of variables consists of variables  for each 
and variables  for each . 
The set of clauses consists of clauses 
for each . In other words, the variables of  correspond to the vertices
and edges of . The clauses correspond to the edges of . 

\begin{comment}
\footnote{In other words, each edge corresponds
to exactly one variable for which the order of indices in the subscript does not matter. 
The same is true regarding the clauses.}.
\end{comment}



\begin{comment}
Next, we define a notion of \emph{clique} graph.
Let  be a graph. Then a graph denoted by 
is defined as follows. Associate each vertex of  with a clique
of size . For each edge  of  make all the vertices
of the cliques associated with  and  mutually adjacent.
\end{comment}

Denote by  a complete binary tree of height .
Let  be the graph obtained from  by associating each vertex
with a clique of size  and, for each edge  of , making all the vertices
of the cliques associated with  and  mutually adjacent.
Denote  by . 

\begin{figure}[h]
\centering
\includegraphics[height=5cm]{KRPATTERN.pdf}
\caption{ and }
\label{TWPattern}
\end{figure}

Figure \ref{TWPattern} shows  and .
To avoid shading the picture of  with many edges,
the cliques corresponding to the vertices of  are marked
by circles and the bold edges between the circles mean that
that there are edges between all pairs of vertices of the corresponding 
cliques. 

\begin{lemma} \label{twf}
The treewidth of the primal graph of  is at least  at most . 
In fact, for , this treewidth is exactly . 
\end{lemma}

{\bf Proof.}
The primal graph of  can be obtained from 
by adding one vertex  for each edge  of  and
making this vertex adjacent to the ends of .

The lower bound follows from existence of a clique of size  in .
Indeed, in any tree decomposition of , there is a bag containing
all the vertices of such a clique \cite{BodlaenderM93}. Consequently, the
width of any tree decomposition is at least . In fact if 
then  has a clique of size  created by cliques of two adjacent
nodes. Hence, due to the same argumentation, the treewidth of 
is at least  for . 

For the upper bound,
consider the following tree decomposition  of .
 is just . We look upon  as a rooted tree, the centre of
 being the root. The bag  of each node  contains the clique of
 corresponding to . In addition, if  is not the root vertex
then  also contains the clique corresponding to the parent of . 
Observe that  satisfies the connectivity property.
Indeed, each vertex appears in the bag corresponding to its `own' clique and 
the cliques of its children. Clearly, the set of nodes corresponding to the bags 
induce a connected subgraph. The rest of the tree decomposition properties can
be verified straightforwardly. We conclude that  is indeed a tree
decomposition of .

In order to `upgrade' , add  new adjacent vertices 
to each vertex of . These vertices will correspond to the edges of cliques associated
with the respective nodes of . In addition, add  new adjacent vertices to each non-root vertex of . 
These vertices will correspond to the edges between the clique associated with the corresponding node of 
and the clique of its parent.
The bag of each new vertex will contain , corresponding to the edge  associated with this bag, 
plus the ends of . A direct inspection shows that this is indeed a tree decomposition
of the primal graph of  and that the size of each bag is at most .

Notice that for  the lower and upper bounds coincide, thus allowing to state
the treewidth precisely. 


The following is the main technical result whose proof is given in the next section. 
\begin{theorem} \label{maintheor}
The size of {\sc obdd} computing  is at least .
\end{theorem}


The following corollary reformulates the lower bound in terms of the number of variables of  and .

\begin{corollary} \label{reform1}
Let  be the number of variables of .
Then the size of {\sc obdd} computing  is at least
 where 
\end{corollary}

{\bf Proof.}
Recall that  has  nodes.
For each node  of ,  has  variables corresponding to the
vertices of the clique of  plus  variables corresponding to the edges
of this clique. In addition, if  is a non-root node then it is associated with 
variables connecting the clique of  with the clique of its parent. Thus each node of
 is associated with at most  variables and hence the total number of
variables .
Thus . According to Theorem \ref{maintheor}, the
size of an {\sc obdd} computing  is at least
 as required. 

\begin{comment}
Let us reformulate the statement of Theorem \ref{maintheor} in terms of the number
of variables of  and the treewidth of its primal graph, having in mind the bounds
on the treewidth as in Lemma \ref{twf}.

First of all, taking into account that , where  is the treewidth of the primal graph of ,
the lower bound can be seen as . Next, let  be the number of variables of .
Then, it is not hard to observe that .
Replacing  this way in , we obtain 
 as a lower bound on the 
{\sc obdd} size for . Clearly, if we consider  as a constant, this lower bound can be seen
as . 
\end{comment}

\begin{comment}
Let
 be the number of nodes of . Taking into account that , 
. Further on, according to Lemma \ref{twf}, 
where  is the treewidth of the primal graph of . Hence the lower bound
can be replaced by . Finally  where  is the number of 
variables. It is not hard to see that for a sufficiently big  compared to  (causing a sufficiently big 
) the bound  can be rewritten as .
Theorem \ref{maintheor} requires  to be even. However, for an odd ,
the theorem implies an obvious bound of ,
where  can be included into  starting from a sufficiently large . 

In order to prove the main result of the paper we need to look at the above statement
from a little bit different angle of finding a lower bound for the \emph{given} fixed treewidth .
If  is odd just take  and the above reasoning directly applies.
Otherwise, take  and in the resulting formula replace  by  which would cost 
a factor that can be included into  denominator. 
\end{comment}

Now we are ready to state the parameterized lower bound for {\sc obdd}s. 

\begin{corollary} \label{paramlower}
There is a function  such that for each 
there is an infinite sequence of {\sc cnf}s  of treewidth at most 
of their primal graphs 
such that for each  the size of {\sc obdd} 
computing it is at least  where  is the number of variables of . 
Put it differently, for each fixed , there is a class of {\sc cnf}s
of treewidth at most  of the primal graph for which the {\sc obdd} size is . 
\end{corollary}

{\bf Proof.}
For an odd , consider the {\sc cnf}s  for all 
and for an even , consider the {\sc cnf}s  for all .
By Lemma \ref{twf}, the treewidth of the primal graph of  is at 
most  and of  at most . Thus the treewidth requirement is
satisfied regarding these classes. 

By Corollary \ref{reform1}, the {\sc obdd} size is lower-bounded by
 for the former class
and by  for the latter class.
Observe that  is a lower bound for both
these lower bound. Hence, the corollary follows by assuming 
.

\begin{comment}
Observe that for an even  the primal graph treewidth of  is 
and that the above argumentation still applies. Indeed, since , it is 
legitimate to represent the lower bound as . Further on, in the inequality
that follows, the occurrence of  in the denominator (as an upper bound of
the actual treewidth) even strengthens this inequality.
\end{comment}
 

Corollary \ref{paramlower} establishes \emph{parameterized} separation between
{\sc obdd} and {\sc sdd}. The next corollary shows that essentially the same method
can be used to separate {\sc obdd} and {\sc sdd} in the \emph{classical sense}.

\begin{corollary}
There is an infinite family of functions for which the smallest OBDDs
are of size  while there are SDDs of size .
\end{corollary}

{\bf Proof}
Consider functions .
Let us compute the number  of variables of . 
Following the calculation as in Corollary \ref{reform1},
we observe that


Denote  by  and  by .
Then 
. 

It follows from \eqref{maineq} that for a sufficiently large ,
 and hence .
Then it follows from Theorem \ref{maintheor} that for a sufficiently
large , an {\sc obdd} for  is of size at least 
.

On the other hand, it follows from \eqref{maineq} 
that for a sufficiently large , .
Thus, according to \cite{SDD}, the size of {\sc sdd} for  is bounded by
, confirming the required separation.


\begin{comment}
Using this equality lower and upper bounds on 
in terms of  that hold for sufficiently large .
Then it follows from Theorem \ref{maintheor} that for a sufficiently
large , an OBDD for  is of size at least 
.

On the other hand, it follows .


The lower bound together with Theorem \ref{maintheor} imply that for a sufficiently
large , an OBDD for  is of size at least 
.

By Lemma \ref{twf}, the treewidth of the primal graph of  is at most
 which is at most  by the above upper bound.
Thus, according to \cite{SDD}, the size of {\sc sdd} for  is bounded by
, as required. 
\end{comment}

\section{Proof of Theorem \ref{maintheor}} \label{lbaux1}
The plan of the proof is the following. We introduce the notion of
matching width of a graph. Then we provide two statements regarding this
notion. The first statement (Lemma \ref{cltreemt}) claims a linear in  lower bound 
for the matching width of graphs  underlying the considered class
 (the proof of the lemma is provided in the next section).
The second statement (Lemma \ref{manyass}) claims that if a graph  has a matching width 
then any permutation of the variables of  can be partitioned into 
a suffix and a prefix so that there are at least  subfunctions of 
resulting from instantiation of variables of the prefix. The proof of Lemma \ref{manyass}
constitutes the essential part of this section. 
Finally, we provide a proof of Theorem \ref{maintheor}.
In this proof we notice that according to the approach 
outlined in the Preliminaries section, Lemma \ref{manyass} together
with Proposition \ref{paths} implies that the size of an {\sc obdd} of 
is at least . Taking  as  and substituting the lower
bound claimed by Lemma \ref{cltreemt}, we obtain the desired lower bound for
. 

The \emph{matching width} is defined as follows.
Let  be a \emph{permutation} of the set  of vertices of a graph
. 
Let  be a \emph{prefix} of  (i.e. all vertices of  are ordered after
). Let us call the matching width of ,
the largest \emph{matching} (that is, a set of edges not having common ends)
consisting of the edges between  and 
(we take the liberty to use sequences as sets, the correct use will be always clear
from the context). Further on, the matching width of  is the largest matching
width of a prefix of . Finally the matching width of , denoted by , is the smallest 
matching width of a permutation of . 

\begin{example}
Consider a path of  vertices  so that  is adjacent to  for
. The matching width of permutation  is  since between any suffix
and prefix there is only one edge. However, the matching width of the permutation 
 is  as witnessed by the partition
 and . Since the matching width of a graph is determined
by the permutation having the smallest matching width, and, since the graph has edges, there cannot be
a permutation of matching width , we conclude that the matching width of this graph is .
\end{example}

\begin{lemma} \label{cltreemt}
For any ,
the matching width of  is at least .
\end{lemma}

The proof of Lemma \ref{cltreemt} is provided is the next section.

{\bf Remark.} The above definition of matching width is a special 
case of a more general notion of \emph{maximum matching width} as defined
in \cite{VaThesis}. In particular our notion of matching width can be seen
as a variant of maximum matching width of \cite{VaThesis} 
where the tree  involved in the definition is a caterpillar. 

\begin{comment}
Then, applying Lemma 4.2.4 of \cite{VaThesis} to this case, one
can observe that , where  is the pathwidth of .
It is well known that the pathwidth of  is at least  \cite{Bod98}. Furthermore, since
all the vertices of a clique get into the same bag in any path decomposition, this
lower bound is immediately extended to  for , immediately implying
a bound slightly above  for . 
\end{comment}

We are now showing that for {\sc cnf}s of form , a large matching width
of  is sufficient for establishing a strong lower bound. 

\begin{lemma} \label{manyass}
Let  be a graph having matching width .
Denote  by . Then any permutation 
of  has a prefix  such that 
there are at least  different functions of form 
such that  is a truth assignment to the variables of .
\end{lemma}


{\bf Proof.}
Let us partition  into sets  of variables corresponding to
the vertices of  and  of variables corresponding to the edges of 
. Let  be the permutation of  ordered in the way as they are ordered
in . Let  be a prefix of  \emph{witnessing} the matching width  of .
(Recall that the matching width of  is at least the matching width of .)
The word `witnessing' in this context means that there is a matching
 between  and . 
Let  be the prefix of  ending with the last element of .
Thus the variables  corresponding to  belong to
 while the variables  corresponding to 
do not. We denote the set of clauses  by .


In the rest of the proof we essentially show that  different assignments to variables
 produce  different subfunctions of  thus confirming the
lemma. Roughly speaking, this is done by showing that by a careful fixing the
assignments to \emph{the rest} of the variables of  we can achieve the
effect that an assignment to  does not `influence' an assignment to  for
. As a result no two assignments to  can have the same
effect on  and this guarantees that desired large set of subfunctions.




We start from defining a set of  assignments for which we then claim that any two assignments
induce two distinct subfunctions of .
In particular, let  be the set of all assignments
to the variables of  that assign the variables   (of course,
those of them that belong to ) with 
and the rest of variables except  with . 
It is easy to see by construction that  is in a natural one-to-one correspondence
with the set of possible assignments to . In particular,
each  corresponds to the assignment  to 
contained in it. Indeed, the assignments of the rest of the variables are fixed
in  by construction. It follows that the size of  is .


We are going to show that for any distinct , ,
confirming the lemma. Due to the correspondence established above, we can specify  such that
 and  assign  with distinct values. Assume w.l.o.g. that 
 is assigned with  by  and with  by . 
Observe that  does not have a satisfying assignment including  and 
assigning both  and  with . Indeed, as a result, 
the clause  is falsified. 
We are going to show that both  and  can be assigned 
with  in a satisfying assignment of  including .
Indeed, assign all the variables of  with  and see that the resulting 
assignment together with  satisfies all the clauses of .
Indeed, if a clause 
does not belong to  then  is assigned with  
(by construction, the only `edge' variables assigned by  are , that is 
those that occur in the clauses of ) . Furthermore, for any clause
 of  such that ,  is assigned with
. Finally  is assigned with  by . It follows that indeed all the clauses of  are satisfied. 

Assume that . Then, by the reasoning as above,  has a satisfying assignment 
including  while  does not implying that .
Otherwise, if , it is assigned with  in both  and , by construction.
It follows that  has a satisfying assignment including  while  does not. It follows 
again that .


{\bf Remark.} Notice the role of variables  in the proof of Lemma \ref{manyass}.
They allow the values of  to \emph{not influence} the values of  for 
and thus keep the number of different subfunctions up to the desired bound. Due to the same reason, it
is important that the edges  constitute a \emph{matching}, i.e. have disjoint
ends. 

{\bf Proof of Theorem \ref{maintheor}}
Lemma \ref{manyass} combined with Proposition \ref{paths}
says that if  has matching width at least  then 
for any permutation of  the corresponding {\sc obdd}
has at least  nodes. In other words,  is a lower bound
on the {\sc obdd} size for . Taking 
and hence  and substituting  for  according 
to Lemma \ref{cltreemt}, we obtain a lower bound of  on the
{\sc obdd} size of , as required.  

\begin{comment}
According to Lemma \ref{manyass}, for any permutation of 
 of  we can find a prefix  and a suffix
 so that the assignments to  produce at least 
subfuncto

According to Lemma \ref{cltreemt}, the matching width
of  is at least .
Let  be an {\sc obdd} computing .
Let  be the order of  explored by .
Let  be a prefix of  as specified by Lemma \ref{manyass}
and let  be a set of  assignments to  such that
for any , .

Each assignment  corresponds to a computational path of . 
Due to Proposition \ref{paths}, the final nodes of the computational paths are pairwise distinct. It follows that  has at least  nodes as required. 
\end{comment}



\section{Proof of Lemma \ref{cltreemt}} \label{lbaux2}
This section is organized as follows. First, we introduce the 
notion of \emph{induced permutation}. Then we provide proof
of Lemma \ref{cltreemt} for . After that, we outline how
to upgrade this special case to a complete proof. Finally, we provide
the complete proof. Note that the proof of the special case 
and the following outline are \emph{technically} redundant. 
However, the reader may find them useful as they provide a 
\emph{sketch} reflecting the proof idea. 

The notion of \emph{induced permutation} is defined as follows.
Let  be a permutation of elements of a set 
and let . Then  induces a permutation  of 
where the elements of  are ordered exactly as they are ordered in .
For example, let  and let  be the subset of even numbers
of . Let . Then .

{\bf Proof of the special case of Lemma \ref{cltreemt} for }
We are going to prove that for an odd , the matching width of  is at least
. For an even  we can simply take a subgraph of 
isomorphic to  (it is not hard to see that the matching width of a graph is not 
less than the matching width of its subgraph).

The proof goes by induction on . For , this is clear,
so consider the case . Imagine  rooted in the natural way, the root being its centre.
Then  has  grandchildren, the subtree rooted by each of them being
. Denote these grandchildren by .
Let  be any permutation of the vertices of . This permutation induces
respective permutations  of vertices of 
being ordered exactly as in . By the induction assumption, we know that
each of  can be partitioned into a prefix and a suffix
so that the edges between the prefix and the suffix induce graph having
matching of size at least . Each of these prefixes naturally corresponds
to the prefix of  ending with the same vertex. Since  are 
pairwise disjoint, this correspondence supplies  \emph{distinct} prefixes 
 of .
Moreover, for each  we know that the graph  induced by the edges between the
vertices of  and the rest of the vertices has a matching of size  consisting 
\emph{only} of the edges of . In order to `upgrade' this
matching by  and hence to reach the required size of , all we need to show is that
in an least one  there is an edge both ends are not vertices of  and hence this
edge can be safely added to the matching. 

At this point we make a notational assumption that does not lead to loss of generality 
and is convenient for the further exposition. By construction, 
are linearly ordered by containment and we assume w.l.o.g. that the ordering is by the increasing 
order of the subscript, that is . 
We claim that the upgrade to the matching as specified above is possible for .

Indeed, observe that  is a connected graph. Thus all we need to show
is that at least one vertex of  gets into  and at least one
vertex of  gets outside , that is in .

For the former, recall that  and that by construction, 
contains  vertices of  being a subgraph of .
Thus we conclude that  contains vertices of 
For the latter, observe that since , . Furthermore, by construction,  contains
 vertices of  being a subgraph of . 
Thus we conclude that  contains vertices of  as well,
thus finishing the proof. 

A proof for the general case of Lemma \ref{cltreemt} proceeds by induction on  similarly
to the special case above. Of course we need to keep in mind that instead of nodes of 
we have cliques of size . The consequence of this substitution is that at the inductive step
of moving from  to  we can increase the matching width by  rather than by 
as above. The auxiliary Lemma \ref{kmatching} allows us to demonstrate the possibility of this
upgrade essentially in the same way as we did for : we just show that the considered prefix 
and suffix of the given permutation both contain at least  vertices outside the grandchild
serving the part of the matching guaranteed by the induction assumption. 

\begin{lemma} \label{kmatching}
Let  be a tree with at least  nodes and let  be a positive integer.
Let  be a graph obtained from  by associating
with each vertex of  a clique of an arbitrary size 
and making the vertices of cliques associated with adjacent vertices
of  mutually adjacent. Let  standing for 'white'
and 'black' be a partition of  such that 
and . Then  has a matching of size  formed by edges
with one white and one black end.
\end{lemma}

{\bf Proof.}
The proof is by induction
on the number of nodes of . It is clearly true when there are  nodes.
Assume that the tree has  nodes and let  be a leaf of 
and  be its only neighbour. 

Let  be the size of the clique  associated with  in .
Assume w.l.o.g. that . Denote 
by . Clearly, the  vertices of  can be matched with the
vertices of . If , we are done. Next, 
if , then the lemma follows by induction assumption
applied on .

Consider the remaining possibility where  for some
. Observe that . Indeed, the total number of vertices of 
 is  so,  will imply , a contradiction.

Let  be the clique associated 
with the neighbour  of . It follows from our assumption that 
because at most  vertices of  can be black. Match  vertices
of  with vertices of  (this is possible due to our assumption that
). Match  unmatched vertices of 
 (there are  unmatched vertices of  and we have just shown
that ) with  vertices of . We are in the situation where
in  there are at least  vertices of , at least
 vertices of  and the size of each associated clique is clearly at 
least . Hence, the lemma follows by the induction assumption.


\begin{comment}
Let  and  be two adjacent vertices of  and let  and 
be the vertices of the corresponding cliques of . Assume that the number of both
white and black vertices of  is at least . Then the required
matching clearly exists. Indeed, take an arbitrary subset 
of white vertices of  and an arbitrary subset 
of black vertices. The edges  form the required
matching. 

We assume the above situation does not happen.
It follows that for any edge  of  either the number of white vertices
in the corresponding clique 
is smaller than  (and, since ,
the number of black vertices is larger than ) or the number of black vertices is
smaller than  (and the number of white vertices is larger than ). Let us call
edges of  of the first type \emph{white} edges and the edges of the second type \emph{black}
edges. 

Consider first the situation where all the edges are white. Assume first that
in each clique corresponding to a vertex of , the number of white vertices
is smaller than or equal to the number of black vertices. Then for each clique match the white
vertices of this clique with arbitrary black vertices of the same clique. Clearly,
all the white vertices will be matched. Since the number of white vertices is at least
, the size of the matching will be at least . Next, assume that there is exactly
one vertex clique  where the number of white vertices is larger than the number of black vertices.
Observe that to satisfy the assumptions of the lemma,  must have at least  vertices. 
Since  is connected,  is adjacent to
a vertex clique  and, according to our assumption the number of white vertices in 
is smaller than the number of black vertices. Since  induce a clique of , we can match
the white vertices of  to arbitrary black vertices of  and, for the rest of
cliques, do as in the previous case. Again, since all
the white vertices are matched, we obtain a matching of size at least .

Assume now that there are at least two vertices  and  of  whose corresponding cliques
 and  of  contain more white vertices than black ones. Observe that  and 
are not adjacent because otherwise  is a black edge in contradiction to our assumption.
Assume first that there are two different vertices  and  of  being neighbors of 
 and , respectively and let  and  be the cliques of  corresponding to 
and , respectively. Since  and  induce cliques of ,
we can match all the white vertices of  with the black ones and do the same regarding
. Since the number of white vertices in each of  is more than , the size of the 
resulting matching will be larger than . If the  and  do not exist then there is vertex
 which is the only neighbor of both  and  in . Let  be the corresponding clique
of  in . Denote by  the number 
of white vertices of , respectively. Denote  by . Since the edge between  and 
is white one, . On the other hand, since , . In  match white vertices with black
vertices so that all the black vertices of  are taken to the matching 
(this is possible since in  there are less black vertices than white ones)
plus additional black vertices of .
As a result there will be  (in fact more than ) unmatched black vertices in . These vertices can be matched with  white 
vertices of  existing since  constituting a matching of size . Thus we have established the
matching in the case where all the edges are white ones. The case where all the edges are black ones follows by symmetry.

Consider now the situation where  has both white and black edges. Since  is connected, its line graph is
connected as well. Consequently there are a white and a black edge that are adjacent i.e. have a common end. 
Let  and  be such white and black edge, respectively. Let  be the respective
vertex cliques of . For each  denote by  and  the respective number of white and black vertices 
of . It follows from our assumption that  and that . 
Assume that  and . Then the desired matching of size  can be obtained by matching
the white vertices of  with the black vertices of  and  and the black vertices of  with
white vertices of  and . Otherwise, assume, for example, that , that is 
implying that . It follows that , a contradiction to being  a white edge.
It remains to assume that  causing contradiction by symmetric reasoning. 
\end{comment}

{\bf Proof of Lemma \ref{cltreemt}.}
We prove that for an odd , the matching width of  is at least
. For an even , it will be enough to consider a subgraph of
 being isomorphic to .
The proof is by induction on .
Assume first that . Then the lemma holds according to
Lemma \ref{kmatching}.

For , let us view  as a rooted tree with its centre  being the 
root. Let  be the  subtrees of  rooted by the `grandchildren'
of . Let  be the subgraphs of  `corresponding' to
. That is, each  is a subgraph of  
induced by (the vertices of) cliques associated with the vertices
of . It is not hard to see that each  is isomorphic to  and each
 is isomorphic to  and that  are pairwise disjoint. 

\begin{comment}
Let us call  the -\emph{clique-tree} of .
Let , be the subgraphs of  that are 
 clique-trees of the subtrees of  rooted by the grandchildren 
of .
\end{comment}

\begin{comment}
In order to continue, we need the notion of \emph{induced permutation}. 
In particular, let  be a permutation of elements of a set 
and let . Then  induces a permutation  of 
where the elements of  are ordered exactly as they are ordered in .
\end{comment}

Let  be an arbitrary permutation of .
Let  be the respective permutations of 
induced by . By the induction assumption for each  there is a prefix
 such that the edges of  with one end in  and the other end in
 induce a graph having matching of size at least .
Let  be the last vertices of , respectively.
Assume w.l.o.g. that these vertices occur in  in exactly this order.
Let  be the prefix of  with final vertex . We are going to show that
the subgraph of  induced by the edges between  and 
has matching of size at least . In fact, as specified above, we already have 
matching of size  if we confine ourself to the edges between 
and . Thus, it only remains to show the existence of matching
of size  in the subgraph of  induced by the edges between
 and . 
Observe that  is a partition of vertices of .
Therefore, it is sufficient to show that  and  and then
the existence of the desired matching of size  will follow from Lemma \ref{kmatching}.

Due to our assumption that  precedes  in , it follows that  is contained in .
Moreover, since  and  are disjoint,  is disjoint with  and hence .
Recall that by the induction assumption, the vertices of  serve as ends
of a matching of size  with no two vertices sharing the same edge of the matching.
That is . Since  by assumption, we conclude that  and
hence .

The proof that  is symmetrical. By our assumption,  precedes  is  and
hence  is contained in  and due to the disjointness of  and
,  is in fact contained in . That 
is derived analogously to the proof that . 

\begin{comment}
Let  be an arbitrary ordering of vertices of .
Let  be the orderings of , respectively,
whose order is as in . By the induction assumption, the matching width
of each  is at least . Let  be the prefixes of 
, witnessing the matching width of the respective sequences. 
Let  be the prefixes of  with the last vertex of
each  is as the last vertex of . We assume w.l.o.g. that the prefixes
 are ordered by their length as they listed.
In fact, since  are pairwise disjoint, the last vertices of 
 are pairwise distinct. Consequently, 
. We are going to show
that there is matching  of size at least  created by edges between 
and . Taking into account the existence of 
matching  of size at least  created by edges between  and 
and that  and  are disjoint by definition,  is a matching of size at least
 created between  and  as required by the lemma.

Denote  by . Notice that . Furthermore, observe that 
is a sequence of vertices of  which is isomorphic to
. We are going to show that  and  and then
the desired claim will immediately follow by applying Lemma \ref{kmatching} to 
and viewing  and  as white and black vertices, respectively.

Observe that  contains . Indeed,  and 
 is disjoint with . Hence . By the induction
assumption as above,  contains ends of edges of the matching of size , hence 
is clearly of size at least . Finally, observe that  contains .  
Indeed, since , . Taking into account
that ,
it follows that . Since  is disjoint with
, . Arguing as for , we observe that 
and hence  are of size at least  as required. 
\end{comment}

\section{{\sc obdd}s parameterized by the treewidth of the incidence graph} \label{twinc}
Recall that the incidence graph of the given {\sc cnf}  has the set of vertices
corresponding to its variables and clauses and a variable vertex is adjacent to a clause vertex if and 
only if the corresponding variable occurs in the corresponding clause. 
\begin{comment}
It is not hard to see that the treewidth of the incidence graph 
of the given {\sc cnf}  is at most as the treewidth of the primal graph plus one. Indeed, for each clause  of , 
there is a clique in the primal graph of . Hence any tree decomposition of this primal graph has a bag 
containing all the variables of . Consequently, we just create a new leaf connected to the node of this bag 
and put into this leaf the clause  and all its variables. The resulting structure is a tree decomposition
of the incidence graph of  of width one more than the considered tree decomposition of the primal graph
of . It immediately follows that the  lower bound on the size of {\sc obdd} 
parameterized by the treewidth of the primal graph applies to the parameterization by the treewidth of the incidence graph.
\end{comment}
The upper bound of \cite{VardiTWD} does not straightforwardly apply to the 
case of incidence graphs because there are classes of {\sc cnf}s having constant treewidth of the 
incidence graph and unbounded treewidth of the primal graph. Indeed, consider, for example a {\sc cnf}
with one large clause. Nevertheless, we show in this section that the  upper bound on the
size of {\sc obdd} holds if  is the treewidth of the incidence graph of the considered {\sc cnf}.
\begin{comment}
Thus, together with the upper bound of \cite{VardiTWD}, the results of this paper provide a 
\emph{complete classification} of the expressive power of {\sc obdd} for {\sc cnf}s of bounded
treewidth of the primal or incidence graph.
\end{comment}
\begin{comment}
The treewidth of incidence
graph is known to be a `stronger' parameter than the treewidth of the primal graph in the following
sense.  On the one hand, there are classes 
of {\sc cnf}s (e.g. containing one big clause) with a small treewidth of the incidence graph and 
an unbounded treewidth of the primal graph and, on the other hand, 
In this section, since there are {\sc cnf}s with an 
\end{comment}

As in \cite{VardiTWD}, we show that if  is the pathwidth of the
\emph{incidence graph}  of the given {\sc cnf}  then the function of 
can be realized by an {\sc obdd} of size  implying (through the ) the
 upper bound where  is the treewidth of . The resulting {\sc obdd} is seen 
as a {\sc dag} whose nodes are partitioned into layers, each layer consisting of nodes labelled
by the same variable. The main technical lemma shows that under the right permutation of variables
the nodes of each layer correspond to  subfunctions of . Consequently, 
nodes per layer are sufficient, which in turn, immediately implies the desired upper bound.

\begin{comment}
This upper bound is stronger than 
the one provided in \cite{VardiTWD} for the primal graph because, on the one hand, there are classes 
of {\sc cnf}s (e.g. containing one big clause) with a small treewidth of the incidence graph and 
an unbounded treewidth of the primal graph and, on the other hand, the treewidth of the incidence graph 
is at most as the treewidth of the primal graph plus one. Indeed, for each clause  there 
is a bag of the tree decomposition of the primal graph 
containing all the variables of . Hence, we just create a new leaf connected to the node of this bag
and put into this leaf the clause  and all its variables. Due to the latter reason the lower bound
proved in the previous section holds for the parameterization by the treewidth of the incidence graph.
\end{comment}
\begin{comment}
We also show that the  upper bound cannot be extended to the case where the function to be
realized as an {\sc obdd} is represented by a circuit of width  due to existence of a class of circuit
for which the equivalent {\sc obdd} is of size at least  for some constant . It is interesting
to notice that  a well-known Tseitin transformation converts the given circuit  into a {\sc cnf}  with 
at most a linear treewidth increase (see e.g. \cite{RazPet}). Consequently, 
can be compiled into an {\sc obdd} of size . However,  contains variables that do not occur
in  and it is the elimination of these variables that causes the exponential gap between 
and . The exponential explosion of the {\sc obdd} size caused by elimination of variables
has been observed earlier in \cite{VardiTWD}. However, the combination of the results of this section shows
that a more \emph{specific} elimination of variables, namely of those that are introduced by the Tseitin
transformation explodes the {\sc obdd} size as well.
\end{comment}

Let us start from fixing the notation. 
Let  be a {\sc cnf} and  be its incidence graph, 
whose nodes are  (corresponding to the variables of )
and  (corresponding to the clauses of ) and  and adjacent
to  if and only if  occurs in  (for the sake of brevity, we identify 
the vertices of  with the corresponding variables and clauses). 
Let  be a path decomposition of .  Fix an end
vertex of  and enumerate the vertices of  along the path starting from
this fixed vertex. Let  be the enumeration.
For each , let  be the smallest  such that .
We call a linear ordering  of  such  whenever
 an ordering \emph{respecting} . 

Now we are ready to prove the main technical lemma.

\begin{lemma} \label{layersize}
Let  be an ordering respecting .
Let  be a prefix of . Then the number of distinct
 such that  is an assignment to  is at most
 where  is the width of .
\end{lemma} 

{\bf Proof.} 
Let  be the last variable of . Denote  by . 
We assume w.l.o.g. that all the clauses of  are pairwise distinct
and hence identify a {\sc cnf} with its set of clauses. 
Partition  into three sets of clauses: , consisting of those that 
appear in some  for  and do not appear in ; , 
consisting of those that appear in  and  consisting of those
that appear in  for some  and do not appear in .
Observe that this is indeed a partition of clauses.
Indeed, otherwise  as all other possibilities
contradict the definition of the sets . 
Then due to the connectedness property of ,
either  or .
However, both these possibilities contradict the definition of  and .
We conclude that  indeed partition the clauses of . For a visual 
justification of their disjointness, see Figure \ref{FPFCFFPic}.

\begin{figure}[h] 
\centering
\includegraphics[height=6cm]{FPFCFF.pdf}
\caption{Black circles correspond to vertices  of . Clauses of 
and  cannot belong to  by definition. Suppose that a clause  belongs 
to . Then  belongs to a bag of a vertex \emph{above}  and to a bag of a vertex \emph{below} .
By the connectivity property,  must belong to , a contradiction.}
\label{FPFCFFPic}
\end{figure}

Denote by  the set of all functions  such that  is an assignment
to . Denote by , ,  the analogous sets regarding
, , and , respectively. 

Let us compute the sizes of the latter  sets. Let  be a clause of . 
By definition  is a subset of variables appearing in the bags 
for . By definition, these variables are ordered \emph{before} . It follows
that  and hence any assignment to  either satisfies or
falsifies . Consequently  is either  or .

It is not hard to see that  is obtained from  by removal of all the clauses
that are satisfied by  and removal of the occurrences of  from the rest of the 
clauses. It follows that if  and  have the same set of satisfied 
clauses then  in other words,  is completely determined by
a set of satisfied clauses. Hence  is bounded above by the number of 
subsets of clauses of , i.e. it is at most  where  is the number of 
clauses of .

Finally let . It is not hard to see that for an assignment 
to ,  is completely determined by the subset of  assigning the variables
of . Therefore, the number of distinct functions  is at most
as the number of distinct assignments to , which is  where .

Let  be an assignment on . It is not hard to see that 
. If  then . Otherwise,
 and hence . In other words,  is either false
or there are  and  such that .
That is . 

We claim that  implying the lemma. Indeed, the clauses of  all belong to
 by definition. Observe that  as well.
Indeed, let . Since  is either  or ordered before ,
there must be  such that . On the other hand,
by definition of , there must be  such that .
By the connectedness property . Since  and  are clearly
disjoint being a set of `clause vertices' and a set of `variable vertices', 
the size of their union is the sum of their sizes and the size of their union cannot be larger
that , as required. 

The upper bound can now be formally stated. 

\begin{theorem}
Let  be a {\sc cnf} with  variables and the pathwidth  of its incidence graph. 
Then  can be compiled into an {\sc obdd} of size . 
\end{theorem}

{\bf Proof.}
In fact we prove that the  upper bound holds even for \emph{uniform} {\sc obdd}s
where each path from the root to a leaf includes \emph{all} the variables. 
Notice that the uniformity is not required by the definition of the {\sc obdd}, only
the order of variables along a computational path is essential. For instance, the {\sc obdd}
shown in Figure \ref{OBDDPic} is not uniform. 

Let  be an ordering respecting  as above. Let  be a smallest possible uniform {\sc obdd} of  
with  being the underlying ordering.  It is well known that the subgraph of  induced by any internal node 
and all the vertices reachable from  (the labels on vertices and edges are retained) is an {\sc obdd}
whose function is  where  is an arbitrary path from the root to  
(recall that  denotes the assignment associated with ). 
Moreover, the minimality of  implies that all the nodes marked with the same variable represent distinct functions.
Indeed, if there are  nodes representing the same function then one of them can be removed, with the
in-edges of the removed node becoming the in-edges of another node associated with the same function and with possible
removal of some nodes that become not reachable from the root. This produces another uniform {\sc obdd} implementing
the same function and having a smaller size in contradiction to the minimality of .

\begin{comment}
If for each variable  the number of nodes labeled by
 is at most  then we are done. Otherwise, let  be a variable such that there are
more than  nodes labeled with . Let  be the prefix of  including all
the variables occurring before . Assume that the prefix is nonempty. Then, as described
in the Preliminaries section, for each node  labeled with  there is an assignment  to 
such that the subgraph of  reachable from  is an {\sc obdd} realizing function . 
Since the number of nodes labeled with  is above  and the number of the respective functions
is at most  according to Lemma \ref{layersize}, there are two nodes  and  labeled with 
such that the subgraphs rooted by them realize the same function. It follows that  and  can be
contracted into a single node producing an {\sc obdd} smaller than  in contradiction to its minimality.
If  is the first variables then there is only one node labeled with , namely the root.

Since there are  variables, the number of nodes labeled with variables is at most  and 
thus (taking into account that the rest of nodes are just  leaves), the claimed upper bound holds.
\end{comment}

By construction the function of a node labelled with a variable  of  is a subfunction of
 obtained by an assignment to the variables preceding  in . According to Lemma \ref{layersize}
the number of such subfunctions is . Since distinct nodes labelled by  are associated
with distinct subfunctions, there are  nodes labelled by . Multiplying this by the number
 of variables of , we obtain the desired  bound on the number of nodes of . 


\begin{corollary}
A {\sc cnf} with  variables and having treewidth  can be compiled into an {\sc obdd}
of size .
\end{corollary}

We close this section with discussion of yet another parameter of {\sc cnf}s, introduced in \cite{HuDar}, 
whose fixed value guarantees a linear size {\sc obdd}. 
In \cite{HuDar} this parameter has not been given a name so, let us name it \emph{combined width}.
Let  be a linear ordering on variables of
the given {\sc cnf} . For each variable  in this ordering we define the \emph{cutwidth} of  (w.r.t. to )
as the number of clauses with one variable being either  or ordered before  and one variable ordered after  in . 
Further on, we define the \emph{pathwidth} of  (w.r.t. to ) as the number of variables that are either  or ordered
before  that occur in clauses having at least one occurrence of a variable ordered after . 
The combined width of  is the minimum of the cutwidth and the pathwdith of . The combined width
of  is the maximum over all the combined widths of the variables. Finally, the combined width of  
is the minimum of combined widths of all possible orders of the variables of . It is shown in \cite{HuDar}
that a {\sc cnf} of combined width  can be complied into an {\sc obdd} of size . 

The combined width of  is a mixture of two parameters of the primal graph of : the cutwidth
(maximum cutwidth of a variable in the given permutation taken minimum over all permutations)
and the pathwidth. Moreover, the combined width is not just their minimum but can in fact be 
much smaller than both cutwidth and pathwidth. Consider for example a {\sc cnf}
 where  and  are {\sc cnf}s defined as follows.
  and 
We assume that the variables of  are disjoint with the variables of  and that 
can be arbitrarily large. The primal graph of  has a large cutwidth. Indeed, for any ordering
of variables of  there is a subset  of  of size at least  that are
either all smaller than  or all larger than . Specify a variable  that is a 'median' of 
according to the considered order. Then the cutwidth of this variable will be about .
Furthermore, the pathwidth of the primal graph of  is large because this graph is just one big clique.
On the other hand, the combined width of  and  is small. Indeed, order the variables as follows:
. Then the pathwidth index of the first  variables is  and hence
the combined width will be at most  as well. Further, the cutwidth of the last  variable is  and hence
the combined width of these variables is  as well. Thus the combined width of this order is  and hence
the combined width of  is at most  which is clearly much smaller than the minimum of the 
pathwdith and the cutwidth of  (determined by the respective connected components of the primal graph 
of ). We leave the relationship between the incidence graph treewidth and the combined width as an open question. 



\begin{thebibliography}{10}

\bibitem{BodlaenderM93}
Hans~L. Bodlaender and Rolf~H. M{\"o}hring.
\newblock The pathwidth and treewidth of cographs.
\newblock {\em SIAM J. Discrete Math.}, 6(2):181--188, 1993.

\bibitem{DynSDD}
Arthur Choi and Adnan Darwiche.
\newblock Dynamic minimization of sentential decision diagrams.
\newblock In {\em AAAI}, 2013.

\bibitem{DarwicheJACM}
Adnan Darwiche.
\newblock Decomposable negation normal form.
\newblock {\em J. ACM}, 48(4):608--647, 2001.

\bibitem{SDD}
Adnan Darwiche.
\newblock {S}{D}{D}: A new canonical representation of propositional knowledge
  bases.
\newblock In {\em IJCAI}, pages 819--826, 2011.

\bibitem{DerMar}
Adnan Darwiche and Pierre Marquis.
\newblock A knowledge compilation map.
\newblock {\em J. Artif. Intell. Res. (JAIR)}, 17:229--264, 2002.

\bibitem{VardiTWD}
Andrea Ferrara, Guoqiang Pan, and Moshe~Y. Vardi.
\newblock Treewidth in verification: Local vs. global.
\newblock In {\em LPAR}, pages 489--503, 2005.

\bibitem{HuDar}
Jinbo Huang and Adnan Darwiche.
\newblock Using dpll for efficient obdd construction.
\newblock In {\em SAT}, 2004.

\bibitem{OBDDTWJha}
Abhay~Kumar Jha and Dan Suciu.
\newblock On the tractability of query compilation and bounded treewidth.
\newblock In {\em ICDT}, pages 249--261, 2012.

\bibitem{Yukna}
Stasys Jukna.
\newblock {\em Boolean Function Complexity: Advances and Frontiers}.
\newblock Springer-Verlag, 2012.

\bibitem{McMillan94}
Kenneth~L. McMillan.
\newblock Hierarchical representations of discrete functions, with application
  to model checking.
\newblock In {\em CAV}, pages 41--54, 1994.

\bibitem{SubbaTree}
Sathiamoorthy Subbarayan, Lucas Bordeaux, and Youssef Hamadi.
\newblock Knowledge compilation properties of tree-of-{B}{D}{D}s.
\newblock In {\em AAAI}, pages 502--507, 2007.

\bibitem{VaThesis}
Martin Vatshelle.
\newblock {\em New width parameters of graphs}.
\newblock PhD thesis, Department of Informatics, University of Bergen, 2012.

\bibitem{WegBook}
Ingo Wegener.
\newblock {\em Branching Programs and Binary Decision Diagrams}.
\newblock SIAM, 2000.

\bibitem{WegSurvey}
Ingo Wegener.
\newblock Bdds--design, analysis, complexity, and applications.
\newblock {\em Discrete Applied Mathematics}, 138(1-2):229--251, 2004.

\bibitem{SDDvsOBDD}
Yexiang Xue, Arthur Choi, and Adnan Darwiche.
\newblock Basing decisions on sentences in decision diagrams.
\newblock In {\em AAAI}, 2012.

\end{thebibliography}
\end{document}
