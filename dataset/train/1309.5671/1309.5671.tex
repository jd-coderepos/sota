\mysection{A Generic Protocol}\label{sec:bevs}

Specifications~\ref{alg:active} and~\ref{alg:passive} contain
internal variables and transitions
that need to be refined for an executable implementation.
We start with refining active replication.
\multiconsensus\
(Specification~\ref{alg:bevs}) refines active replication and contains
no internal variables or transitions.
As previously, the \texttt{invoke} and \texttt{response} transitions (and corresponding variables) have been
omitted---they are the same as in Specification~\ref{alg:replsvc}. Transitions
\texttt{propose} and \texttt{update} of Specification~\ref{alg:active} have been omitted for the same reason.
In this section we explain how and why \multiconsensus\ works.




\mysubsection{Certifiers and {\round}s}

\multiconsensus\ has two basic building blocks:
\begin{itemize}\item A static set of  processes called \emph{certifiers}.
A minority of these may crash.  So for tolerating at most 
failures, we require that  holds.
\item An unbounded number of \emph{{\round}}s.
\end{itemize}

\noindent
For ease of reference, Table~\ref{tab:translate}
contains a translation between terminology used in this paper
and those found in the papers describing the protocols under consideration.

In each round, a consensus protocol assigns to at most one certifier the role
of \emph{\sequencer}.
The {\sequencer} of a round can certify at most one command for each slot.
The other certifiers can copy the {\sequencer}, certifying the same
command for the same slot and {\round}.
Note that if two certifiers certify a command in the same
slot and the same {\round}, it must be the same command.
Moreover, a certifier cannot retract a certification.
Once a majority of certifiers certify
the command within a {\round}, the command is \emph{decided}
(and because certifications cannot be retracted the command will remain
decided thereafter).
In Section~\ref{sec:recovery} we show why
two {\round}s cannot decide different commands for the same slot.

Each {\round} has a {\round}-id that uniquely identifies the {\round}.
{\Round}s are totally ordered by their {\round}-ids.
A {\round} is in one of three modes: \emph{pending}, \emph{operational},
or \emph{wedged}.
One {\round} is the first {\round} (it has the smallest {\round}-id), and initially
only that {\round} is operational.  Other {\round}s start out pending.
The two possible transitions on the mode of a {\round} are as follows:
\begin{enumerate}\item A pending {\round} can become operational only if all {\round}s with
lower {\round}-id are wedged;
\item A pending or operational {\round} can become wedged under any
circumstance.
\end{enumerate}

\noindent
This implies that at any time at most one {\round} is operational and that
wedged {\round}s can never become unwedged.

\begin{comment}
An operational {\round} has exactly one certifier that is designated
as \emph{coordinator}---this certifier handles the recovery part of the protocol (Section~\ref{sec:recovery}).
\end{comment}

\mysubsection{Tracking Progress}\label{sec:progind}

In Specification~\ref{alg:bevs}, each certifier  maintains a
\emph{progress indicator}  for each , defined as:

\begin{definition}{Progress Indicator}
A progress indicator is a pair 
where  is the identifier of a {\round} and  is a proposed command
or~, satisfying:

\begin{itemize}\item If , then the progress indicator guarantees that no {\round}
with an id less than  can ever decide, or have decided, a proposal
for the slot.
\item If , then the progress indicator guarantees that
if a {\round} with id  such that  decides
(or has decided) a proposal  for the slot, then .
\item Given two progress indicators  and
 for the same slot, if neither  nor
 equals , then .
\end{itemize}
\end{definition}

\noindent
We define a total ordering  on
progress indicators for the same slot as follows:
 iff
\begin{itemize}\item ; or
\item .
\end{itemize}

\noindent
At any certifier, the progress indicator for a slot is monotonically
non-decreasing.





\begin{figure}[t]
\begin{center}

 \begin{algorithm}[H]

\caption{\label{alg:bevs} \multiconsensus}
\footnotesize{
\begin{distribalgo}[0]

\BLANK

\INDENT{\var ~ }
   \STATE 
\ENDINDENT

\BLANK

\INDENT{\init: }
	\STATE 
	\STATE 
\ENDINDENT

\BLANK
\BLANK

\INDENT{\outtransition\ :}
\INDENT{\precond:}
  \STATE 
\STATE 
  \STATE 
\ENDINDENT
\INDENT{\action:}
  \STATE 
  \STATE 
\ENDINDENT
\ENDINDENT

\BLANK

\INDENT{\outtransition\ :}
\INDENT{\precond:}
  \STATE 
  \STATE 
\ENDINDENT
\INDENT{\action:}
  \STATE 
  \STATE 
\ENDINDENT
\ENDINDENT



\BLANK

\INDENT{\outtransition\ :}
\INDENT{\precond:}
  \STATE \\
       \hspace{0.05in} 
  \STATE 
\ENDINDENT
\INDENT{\action:}
  \STATE 
\ENDINDENT
\ENDINDENT

\BLANK

\INDENT{\outtransition\ :}
\INDENT{\precond:}
  \STATE 
\ENDINDENT
\INDENT{\action:}
  \STATE 
  \STATE \\
       \hspace{.08in} 
\ENDINDENT
\ENDINDENT

\BLANK

\INDENT{\outtransition\ :}
\INDENT{\precond:}
  \STATE 
  \STATE 
\ENDINDENT
\INDENT{\action:}
  \INDENT{}
  	\STATE 
	\STATE 
    \INDENT{\textbf{if}  \textbf{then}}
	  \STATE 
    \ENDINDENT
  \ENDINDENT
  \STATE 
\ENDINDENT
\ENDINDENT

\end{distribalgo}
}
\end{algorithm}
\end{center}
\end{figure}

\floatname{algorithm}{Specification}
\renewcommand{\thealgorithm}{\arabic{algorithm}}


\begin{comment}




\begin{figure*}[t]
\begin{tabular}{c@{\hspace{0cm}}c}
 \begin{minipage}[t]{.5\textwidth}

 \begin{algorithm}[H]

\caption{\label{alg:bevs_po} \multiconsensus\ + Prefix Ordering}
\footnotesize{
\begin{distribalgo}[0]



\INDENT{\var ~ }
   \STATE 
\ENDINDENT

\BLANK

\INDENT{\init: }
	\STATE 
	\STATE 
\ENDINDENT

\BLANK

\INDENT{\outtransition\ :}
\INDENT{\precond:}
  \STATE 
  \STATE 
\STATE 
\ENDINDENT
\INDENT{\action:}
  \STATE 
  \STATE 
\ENDINDENT
\ENDINDENT

\BLANK

\INDENT{\outtransition\ :}
\INDENT{\precond:}
  \STATE 
  \STATE 
  \STATE 
\ENDINDENT
\INDENT{\action:}
  \STATE 
  \STATE 
\ENDINDENT
\ENDINDENT

\end{distribalgo}
}
\end{algorithm}
\end{minipage}

&

\begin{minipage}[t]{.6\textwidth}

\setcounter{algorithm}{\value{algorithm}-1}
\floatname{algorithm}{\textcolor{white}{Specification}}

\begin{algorithm}[H]
\renewcommand{\thealgorithm}{}
\caption{\textcolor{white}{\multiconsensus}}
\footnotesize{
\begin{distribalgo}[0]

\INDENT{\outtransition\ :}
\INDENT{\precond:}
  \STATE 
  \STATE 
\ENDINDENT
\INDENT{\action:}
  \STATE 
\ENDINDENT
\ENDINDENT

\BLANK
\vspace{.09cm}

\INDENT{\outtransition\ :}
\INDENT{\precond:}
  \STATE 
\ENDINDENT
\INDENT{\action:}
  \STATE 
  \STATE 
\ENDINDENT
\ENDINDENT

\BLANK

\INDENT{\outtransition\ :}
\INDENT{\precond:}
  \STATE 
  \STATE 
\ENDINDENT
\INDENT{\action:}
  \INDENT{}
  	\STATE 
	\STATE 
         \STATE 
  \ENDINDENT
  \STATE 
\ENDINDENT
\ENDINDENT

\end{distribalgo}
}
\end{algorithm}
\end{minipage}
\end{tabular}
\end{figure*}
\end{comment}

\mysubsection{Normal case processing}\label{sec:bev-normal}

Each certifier  \emph{supports}
exactly one {\round}-id , initially~, the {\round}-id of the
first {\round}.
The \emph{normal case} holds when a majority of certifiers support
the same {\round}-id, and one of these certifiers is {\sequencer} (its  flag is set to \texttt{true}).

Transition 
is performed when {\sequencer}  certifies command  for the given slot
and {\round}.
The condition 
holds only if no command can be decided in this slot
by a {\round} with an id lower than .
The transition requires that  is the lowest empty slot of the
sequencer.
If the transition is performed,
 updates  to reflect that if a command is
decided in its {\round}, then it must be command .
{\Sequencer}  also notifies all other certifiers by adding
 to set 
(modeling a broadcast to the certifiers).

A certifier that receives such a message checks if the message
contains the same {\round}-id that it is currently supporting and that
the progress indicator in the message exceeds its own progress
indicator for the same slot.  If so, then the certifier updates its
own progress indicator and certifies the proposed command (transition
).


The 
transition at  is enabled if a majority of
certifiers in the same {\round} have certified  in .
If so, the command is decided and, as explained in the next section,
all replicas that undergo the \observedecision\ transition
for this slot will decide on the same command.

\mysubsection{Recovery}\label{sec:recovery}

In this section, we show how \multiconsensus\ deals with failures.
The reason for having an unbounded number of {\round}s is to achieve liveness.
When an operational {\round} is no longer certifying proposals,
perhaps because its {\sequencer} has crashed or is slow,
the round can be
wedged and a {\round} with a higher {\round}-id can become operational.

Modes of {\round}s are implemented as follows:
A certifier  can transition to supporting a new {\round}-id
 and \emph{prospective sequencer} 
(transition ).
This transition can only increase .
Precondition  ensures that a certifier supports a {\round} and a prospective sequencer for the round at most once.
The transition sends the certifier's \emph{snapshot} by adding it to the
set .
A snapshot is a four-tuple  containing
the certifier's identifier, its current {\round}-id, the identifier of , and
the certifier's list of progress indicators.
Note that a certifier can send at most one snapshot for each {\round}.

{\Round}  with sequencer  is operational,
by definition, if a majority of certifiers support  and added
 to the set \textit{snapshots}.
Clearly, the majority requirement guarantees that there cannot be two
{\round}s that are simultaneously operational, nor can there be operational
{\round}s that do not have exactly one sequencer.
Certifiers that support  can no longer certify commands
in {\round}s prior to .
Consequently, if a majority of certifiers support a {\round}-id
larger than , then all {\round}s with an id of  or lower
are wedged.

\begin{figure*}
\begin{center}

\end{center}

\caption{\label{fig:refine} {\small Relation between the 
variable of Specification~\ref{alg:bevs} and the
 variable of Specification~\ref{alg:active}.
Here  is the number of certifiers.}}

\end{figure*}

\begin{comment}
With \multiconsensus, the coordinator
of a {\round} is also its sequencer.  We shall see that this is not the case with VSR (Section~\ref{sec:implementation}).
\end{comment}


Transition 
is enabled at  if
the set  contains snapshots for  and sequencer 
from a majority of certifiers.
The sequencer
helps to ensure that the {\round} does not
decide commands inconsistent with prior {\round}s using the snapshots
it has collected.
For each slot, sequencer  determines the maximum
progress indicator  for the slot in the
snapshots contained in .  It then sets its own progress
indicator for the slot to .
It is easy to see that .

We argue that  satisfies the definition
of progress indicator in Section~\ref{sec:progind}.  All certifiers
in  support  and form a majority.  Thus, it is not
possible for any {\round} between  and  to decide a command
because none of these certifiers can certify a command in those
{\round}s.  There are two cases:

\begin{itemize}\item If , no command can be
decided before , so no command can be decided before .
Hence,  is a correct progress indicator.
\item If , then if a command is decided by
 or a {\round} prior to , it must be .
Since no command can be decided by {\round}s between  and ,
 is a correct progress indicator.
\end{itemize}



\noindent
The sequencer sets its  flag upon recovery.  As a result, it
is enabled to propose new commands. Normal case for the {\round} begins
and holds as long as a majority of certifiers support the corresponding
{\round}-id.


\mysubsection{Refinement Mapping}

\multiconsensus\ refines active replication.
We first show how the internal variables of Specification~\ref{alg:active} are
derived from the variables in Specification~\ref{alg:bevs}.
Predicate  holds if there exists any {\round} that has decided
the command (Section~\ref{sec:recovery} argues why all {\round}s of a given slot can only decide the same command); otherwise .
This is captured formally in
Fig.~\ref{fig:refine}, where  is a tuple .

The transition 
 of \multiconsensus\ always corresponds to a stutter in active replication.
The  transition
of Specification~\ref{alg:active} is performed
when, for the first time, a majority of certifiers in some
{\round}  have certified command 
in slot , that is, when
the last certifier  in the majority performs transition
.
The \texttt{\observedecision} transition of \multiconsensus\ corresponds exactly
to the \texttt{learn} transition of active replication.  Both the \texttt{support{\Round}} and \texttt{recover} transitions are
stutters with respect to Specification~\ref{alg:active} as they
do not affect any of its state variables.


\mysubsection{Passive Replication}\label{sec:bev-prefix}

Section~\ref{sec:ap-formal} showed how in passive replication a state
update from a particular primary can only be decided in a slot if it corresponds to applying an operation
on the state decided in the previous slot.
We called this property prefix ordering.
However, Specification~\ref{alg:bevs} does not satisfy prefix ordering
because any proposal can be decided in a slot, in particular one that
does not correspond to a state decided in the prior slot.
Thus \multiconsensus\ does not refine Passive Replication.
One way of implementing prefix ordering would be for the primary to
wait with proposing a command for a slot until it knows the decisions
for all prior slots.
Doing so would be slow.

A better solution is to refine \multiconsensus\ to obtain a
specification that also refines Passive Replication and satisfies
prefix ordering.  We call this specification \multiconsensus-PO.
\multiconsensus-PO guarantees that each decision is the result
of an operation applied to the state decided in the prior slot
(except for the first slot).
We complete the refinement by adding two preconditions:
\begin{enumerate}
\item[(i)]
In \multiconsensus-PO, slots have to be decided in order.
To guarantee this, we have each certifier certify commands
in order by adding the following
precondition to transition \texttt{certify}:
 .
Thus if, in some round, there exists a majority of certifiers that have
certified a command in , there also exists a majority of certifiers
that have certified a command in the prior slot.

\item[(ii)]
To guarantee that a decision in  is based on the state
decided in the prior slot, we add the following precondition to transition \texttt{certifySeq}:
.
This works because by the properties of progress indicators, if a
command has been or will be decided in round  or a prior round,
it is the command in .
Therefore, if the sequencer's proposal for  is decided in round , it is the command in .
If the primary and the sequencer are co-located, as they usually are,
this condition is satisfied automatically as the primary computes
states in order.
\end{enumerate}

Also, \multiconsensus-PO inherits transitions \texttt{invoke} and \texttt{response} from Specification~\ref{alg:replsvc} as well as transitions
\texttt{propose}, \texttt{update}, and \texttt{resetShadow} from Specification~\ref{alg:passive}.  The variables contained in these transitions are inherited as well.

The passive replication protocols that we consider in this paper,
VSR and Zab,
share the following design decision in the recovery procedure: The
{\sequencer} broadcasts a single message containing its entire
snapshot rather than sending separate certifications for each slot.
Certifiers wait for this comprehensive snapshot, and overwrite their
own snapshot with it, before they certify new commands in this {\round}.
As a result, at a certifier all  slots have the same {\round}
identifier, and can thus be maintained as a separate variable.  

\subsubsection{Refinement Mappings}
\label{sec:MCPO:refinement}

Below, we present a refinement between \multiconsensus-PO\ and \multiconsensus\ and then show that \multiconsensus-PO refines Passive Replication.

\subsubsection{Refining \multiconsensus}

Showing the existence of a refinement between \multiconsensus-PO\
and \multiconsensus\ is straightforward.  Transitions and variables
inherited from passive replication are mapped to variables and
transitions of \multiconsensus\ in the same way as they are mapped
from passive replication to active replication (note that these
variables and transitions are mapped to variables and transitions
of \multiconsensus\ that are themselves inherited from active
replication).  Since \multiconsensus-PO\ only adds constraints to
the transitions that are specific to \multiconsensus, the refinement
exists.

\subsubsection{Refining Passive Replication}
Passive replication has a single internal variable, ,
that is derived from  as in Fig.~\ref{fig:refine},
that is, for any slot ,  holds if a majority
of replicas have certified  for slot .

Transitions \texttt{certifySeq}, \texttt{support{\Round}}, and
\texttt{recover} are stutters in passive replication, while an
\texttt{\observedecision} transition of \multiconsensus-PO
corresponds to a \texttt{learn} transition in
Specification~\ref{alg:passive}.

Similarly to the refinement between active replication and
\multiconsensus, transition  is performed when, for the first
time, a majority of certifiers in some {\round}  have certified
command  in slot , that is, when the last certifier
 in the majority performs transition .  

In this case however, we
must additionally show that when the last certifier of the majority
undergoes the \texttt{certify} transition, the following condition
holds in Specification~\ref{alg:passive}: 
and .  

The first
part of the condition holds since certifiers only certify commands
that have been proposed.  The second part of the condition holds
for the following reason.  From constraint (ii) of \multiconsensus-PO, the sequencer only certifies
a command if its corresponding \textit{oldState} equals the \textit{newState} of the command it stores in the previous slot.
From the properties of progress indicators, if , then if a command is decided
in an earlier round than , then it must be .  From constraint (i) (commands are certified in order), if a command is decided for , then
 decided on a command previously.  Consequently,  .

