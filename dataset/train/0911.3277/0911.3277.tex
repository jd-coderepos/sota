\documentclass{eptcs}
\providecommand{\event}{Axel Legay (Ed.): International Workshop on Verification of Infinite-State Systems (INFINITY)} \providecommand{\volume}{13}   \providecommand{\anno}{2009}   \providecommand{\firstpage}{1} \providecommand{\eid}{1}       \usepackage{breakurl}        \usepackage{macro}
\usepackage{picinpar}
\usepackage{epsfig}





\title{Automated Predicate Abstraction for \\ Real-Time Models}

\author{Bahareh Badban\thanks{Corresponding author} \qquad\qquad Stefan Leue
\institute{Department of Computer and Information Science, University of
  Konstanz, Germany}
\email{\quad Bahareh.Badban@uni-konstanz.de \quad\qquad Stefan.Leue@uni-konstanz.de}
\and
Jan-Georg Smaus
\institute{Institut f\"ur Informatik, Universit{\"a}t Freiburg, Germany}
\email{Smaus@informatik.uni-freiburg.de}
}
\def\titlerunning{Automated Predicate Abstraction for Real-Time Models}\def\authorrunning{B. Badban \& S. Leue \& J.G. Smaus}
\begin{document}
\maketitle
\thispagestyle{empty}
\setcounter{page}{\firstpage}


\paragraph{Introduction}
Model checking has been widely successful in validating and debugging hardware
designs and communication protocols. However, state-space explosion is an
intrinsic problem which limits the applicability of model checking tools. To
overcome this limitation software model checkers have suggested different
approaches, among which abstraction methods have been highly esteemed.  modern
techniques.  Among others, predicate abstraction is a prominent technique which
has been widely used in modern model checking.  This technique has been shown
to enhance the effectiveness of the reachability computation technique in
infinite-state systems.
In this technique an infinite-state system is represented abstractly by a
finite-state system, where states of the abstract model correspond to the
truth valuations of a chosen set of atomic predicates.
Predicate abstraction was first introduced in~\cite{Graf97} as a method for
automatically determining invariant properties of infinite-state systems.
This technique involves abstracting a concrete transition system using a set
of formulas called {\em predicates} which usually denote some state properties
of the concrete system.

The practical applicability of predicate abstraction is impeded by two problems.
First, predicates need to be provided manually~\cite{Lahiri06b,Das02}. 
This means that the selection of appropriate abstraction predicates is based
on a user-driven trial-and-error process. The high degree of user intervention
also stands in the way of a seamless integration into practical software 
development processes. Second, very often the abstraction is too coarse in order
to allow relevant system properties to be verified. This calls for abstraction 
refinement~\cite{Colon98}, often following a counterexample guided abstraction refinement
scheme~\cite{clarke00guided,Ball02}.
Real time models are one example of systems with a large state space as time
adds much complexity to the system. In this event, recently there have been
increasing number of research to provide a means for the abstraction of such
models.  It is the objective of this paper to provide support for an automated
predicate abstraction technique for concurrent dense real-time models
according to the timed automaton model of~\cite{alur94}.  We propose a method
to generate an efficient set of predicates than a manual, ad-hoc process would
be able to provide.  We use the results from our recent work~\cite{BaLeSm09}
to analyze the behavior of the system under verification to discover its local
state invariants and to remove transitions that can never be traversed.
We then describe a method to compute a predicate abstraction based on these
state invariants. We use information regarding the control state labels as
well as the newly computed invariants in the considered control states when
determining the abstraction predicates.  We have developed a prototype tool
that implements the invariant determination.  Work is under way to also
implement the computation of
a predicate abstraction based on our proposed method. We plan to embed 
our approach into a comprehensive abstraction and refinement methodology
for timed automata.

\paragraph{Related Work.}
An interactive method for
predicate abstraction of real-time systems where a set of predicates called
{\em basis} is provided by the user is presented in~\cite{Colon98}. 
The manual choice of the abstraction basis depends on the user's understanding
of the system.  The work presented in~\cite{Moller02,Sorea:FTRTFT04} proposes an abstraction method
which is based on identifying a set of predicates that is fine enough to
distinguish between any two clock regions and which creates a strongly preserving
abstraction of the system. 
The basis predicates are discovered by spurious paths obtained
through model-checking of the system. Also, in this approach the choice of
the original set of predicates relies on the user's understanding of the system, 
as well as on the counterexample generation experiments.
To the best of our knowledge, at the time of writing, there
has been no research done on {\em automatically} 
generating invariants (predicates) for dense real time models,
which will be the central contribution of our paper.

In the functional setting the CEGAR methodology based on the seminal 
paper~\cite{clarke00guided}
has been rather influential in the development of hard- and software 
verification methodologies, e.g.,~\cite{Ball02}.
Abstraction predicate discovery based on the analysis of 
spurious counterexamples is at the heart of the work 
in~\cite{Das02}.
The approaches presented in~\cite{Jhala05,millan06} and  
in~\cite{Henzinger04} use interpolation to detect feasibility of an 
abstract trace. \cite{millan03} introduces a
proof-based automatic predicate abstraction.



\section{Preliminary Definitions and our Previous Results}
\label{sec.ta}


\paragraph{Timed Automata.}
To have this article self-contained we need to briefly explain some of the
results in~\cite{BaLeSm09}.  A {\em timed automaton}
~\cite{bluebook,alur94}.consists of a finite state automaton together with a
finite set of clock variables, simply called {\em clocks}, and a finite set of
integer variables.  In the notation we distinguish clock and integer variables
only where necessary.  Clocks are non-negative real valued variables which all
increase at the same speed, while integers change only when there is an
explicit assignment.  Initially, all clocks are set to .  A clock may be
reset, but afterwards it immediately starts running again.  The finite state
automaton describes the system {\em control} states of the system, which are
referred to as \emph{locations}, as well as its transitions between locations.
A {\em state} or configuration of the system has the form  where 
is the current control location and  is a valuation function which assigns
to each its current value.  For , we denote by  a
valuation that assigns to each clock  the value , i.e., it
increases the value of all clocks by , while the integer variables remain
unchanged.
 denotes the set of (clock or integer) {\em constraints}  for a set
   of clock variables. Each  is of the form  where , and , called {\em term}, is
  either a variable in  or a linear integer expression, which is an
  expression of the form  where the  are
  integer variables and  and  are integer constants.\footnote{The
    restriction to integers does not constitute a loss of generality
    \cite[Section 4.1]{alur94}.}  We usually write  for . By  we denote the set of all clock variables appearing in .
Formally, a timed automaton  is a tuple  where 
\begin{itemize}
\item  is a finite set of {\em (control) locations}.  is the initial location.
\item  is a finite set of labels, called {\em events} or {\em channels}.  
\item  is a finite set of variables. 
\item 
 assigns to each location in  some constraint in .   
\item  represents
  {\em discrete} transitions.
\end{itemize}


The constraint associated with each location  is called its {\em
  invariant}, denoted . We later refer to these invariants as the {\em
  original} invariants. Time can pass in a control location  only as long
as  remains , i.e.  must hold whenever the current location
is .
The semantics of a nondeterministic timed automaton  is defined by a {\em transition
  system} . 
States or configurations of  are
pairs , where  is a control location of  and  is a
valuation over  which satisfies , i.e. . 
 is an {\em initial} state of  if  is the initial location.

\noindent
{\em Transitions.} 
For each transition system the system state changes by:\begin{itemize}

\item {\em Delay transitions}, denoted by , which allow time  to 
elapse. The value of all clocks is increased by  leading to the
transition .
\footnote{Recall that the integer variables remain unchanged.} 
This transition can take place only when the invariant of location  is
satisfied along the transition, i.e.\ .

\item {\em Discrete transitions}, denoted by , which enable a transition. A transition
   is {\em enabled} when the current clock valuation satisfies
  .  When  is executed, all variables, except those which are
  reset, remain unchanged. This results in the transition  where  is an event,  is a guard
  and  is a reset.
\end{itemize}
An {\em execution} of a system is a possibly infinite sequence of
states 
 where each pair of two consecutive states corresponds to
either a discrete or a delay transition.









\paragraph{Creating New Invariants by .}
Here, we explain briefly the  algorithm from ~\cite{BaLeSm09}. This
algorithm strengthens the given original invariants in each control location
by analysing the incoming discrete transitions to that specific control
location; It also reduces the size of the model by pruning away those
transitions which can never be traversed.
The input of the  algorithm is a timed automaton , the output is
's pruned version together with a set of new invariants for .

A discrete transition  is called {\em idle}
if it can never be enabled.  Amongst other reasons, a transition can be idle
when the constraint over the transition is unsatisfiable, or when the
valuation function obtained from the transition does not satisfy the invariant
of the target location, which means that .  For
instance, if  is the discrete transition  where  is an invariant in location , then
this transition is idle since the constraint  is never satisfied.

At each control location ,  first collects the set  of
all the original invariants, and then accumulates all its incoming transitions
in .  The idle transitions within these sets are identified and
are deleted from the model.


For each non-idle  in  the algorithm next computes all
propagated constraints into .  Since  may also have some original
invariant, the new invariant, i.e. , is the conjunction of the
original invariant and all of the previously computed imposed constraints on
.  Computing  may render some of the outgoing transitions of
 idle.  Therefore, the algorithm next checks all outgoing transitions of
 for idleness again.  It then removes all transitions detected as being
idle.
Two timed automata  and  are {\em equivalent}, denoted ,
if they differ only in some idle transitions.
\begin{theorem}
\label{theo.inv}
 always terminates. It also satisfies the following properties:
\begin{itemize} 
\item if  then .
\item If , then , for each reachable
  state  in .  In other words,  is invariant in .
\end{itemize}
\end{theorem}



\paragraph{Networks of Timed Automata.}

 can also be used to treat networks of timed automata in which several
parallel automata synchronize with one another via synchronous message passing.
Transitions associated with emitting or receiving a message of type  are labeled
with  or , respectively. 
The intuitive semantics of a synchronous message passing is such that the message sending and the 
message receiving primitives are blocking and executed in a rendez-vous manner.

Formally, the semantics of this kind of synchronization is defined as follows.
Let  be a parallel composition of 
timed automata , denoted by , where
 for each  and
for each two non-equal  and  .  For  we have
, ,
and  for . The initial location is denoted by .
A state of the network is a configuration  where
  is a configuration in  and  for each  and .
 denotes the replacement of  by  in ,
which is .
Delay transition in this systems is defined as before. Other transitions are:
\begin{itemize}
\item {\em Discrete transitions:} If 
 then  is a discrete
transition in the network model if  for  and
 for .
\item {\em Synchronization transitions:} If 
 and   
then  is a discrete
transition in the network model if  for  and , and
 for .
\end{itemize}

We first run the \cipm\ algorithm over each automaton individually. We then compose the
pruned automata to obtain a pruned network. Conjuncting the newly generated
invariants within the individual automata yields new invariants for the network:
\begin{theorem}
\label{theo.parallel}
Assume  is a network of timed automata where
 for each , and . Then we will have  and  is invariant in .
\end{theorem}








\begin{example}
\label{example1} Figures~\ref{fig.automata1} and~\ref{fig.automata41} show an example of a
timed automaton  in~\cite{Moller02,Sorea:FTRTFT04}, also the outcome of applying \cipm\ on it.


\begin{figure}
\begin{minipage}[h]{0.4\linewidth}
\center
\begin{picture}(0,0)\includegraphics{automata1.pstex}\end{picture}\setlength{\unitlength}{1243sp}\begingroup\makeatletter\ifx\SetFigFont\undefined \gdef\SetFigFont#1#2#3#4#5{\reset@font\fontsize{#1}{#2pt}\fontfamily{#3}\fontseries{#4}\fontshape{#5}\selectfont}\fi\endgroup \begin{picture}(5700,5006)(3091,-5544)
\put(5671,-781){\makebox(0,0)[lb]{\smash{{\SetFigFont{6}{7.2}{\rmdefault}{\bfdefault}{\updefault}{\color[rgb]{0,0,0}}}}}}
\put(4771,-1906){\makebox(0,0)[lb]{\smash{{\SetFigFont{6}{7.2}{\rmdefault}{\bfdefault}{\updefault}{\color[rgb]{0,0,0}}}}}}
\put(6211,-4786){\makebox(0,0)[lb]{\smash{{\SetFigFont{6}{7.2}{\rmdefault}{\bfdefault}{\updefault}{\color[rgb]{0,0,0}}}}}}
\put(3826,-5011){\makebox(0,0)[lb]{\smash{{\SetFigFont{6}{7.2}{\rmdefault}{\bfdefault}{\updefault}}}}}
\put(3916,-1951){\makebox(0,0)[lb]{\smash{{\SetFigFont{6}{7.2}{\rmdefault}{\bfdefault}{\updefault}}}}}
\put(8776,-5056){\makebox(0,0)[lb]{\smash{{\SetFigFont{6}{7.2}{\rmdefault}{\bfdefault}{\updefault}}}}}
\put(6256,-3481){\makebox(0,0)[lb]{\smash{{\SetFigFont{6}{7.2}{\rmdefault}{\bfdefault}{\updefault}{\color[rgb]{0,0,0}}}}}}
\put(5131,-3526){\makebox(0,0)[lb]{\smash{{\SetFigFont{6}{7.2}{\rmdefault}{\bfdefault}{\updefault}{\color[rgb]{0,0,0}}}}}}
\put(3106,-3481){\makebox(0,0)[lb]{\smash{{\SetFigFont{6}{7.2}{\rmdefault}{\bfdefault}{\updefault}{\color[rgb]{0,0,0}}}}}}
\put(7741,-5146){\makebox(0,0)[lb]{\smash{{\SetFigFont{6}{7.2}{\rmdefault}{\bfdefault}{\updefault}{\color[rgb]{0,0,0}}}}}}
\end{picture} \caption{Example from \cite{Moller02}\label{fig.automata1}}
\end{minipage} 
\begin{minipage}[h]{0.4\linewidth}
\center
\begin{picture}(0,0)\includegraphics{automata41.pstex}\end{picture}\setlength{\unitlength}{1285sp}\begingroup\makeatletter\ifx\SetFigFont\undefined \gdef\SetFigFont#1#2#3#4#5{\reset@font\fontsize{#1}{#2pt}\fontfamily{#3}\fontseries{#4}\fontshape{#5}\selectfont}\fi\endgroup \begin{picture}(6422,4967)(11026,-5589)
\put(13501,-826){\makebox(0,0)[lb]{\smash{{\SetFigFont{6}{7.2}{\rmdefault}{\bfdefault}{\updefault}{\color[rgb]{0,0,0}}}}}}
\put(12466,-5056){\makebox(0,0)[lb]{\smash{{\SetFigFont{6}{7.2}{\rmdefault}{\bfdefault}{\updefault}}}}}
\put(11656,-5056){\makebox(0,0)[lb]{\smash{{\SetFigFont{6}{7.2}{\rmdefault}{\bfdefault}{\updefault}{\color[rgb]{0,0,0}}}}}}
\put(16741,-5011){\makebox(0,0)[lb]{\smash{{\SetFigFont{6}{7.2}{\rmdefault}{\bfdefault}{\updefault}}}}}
\put(12961,-3571){\makebox(0,0)[lb]{\smash{{\SetFigFont{6}{7.2}{\rmdefault}{\bfdefault}{\updefault}{\color[rgb]{0,0,0}}}}}}
\put(14086,-3571){\makebox(0,0)[lb]{\smash{{\SetFigFont{6}{7.2}{\rmdefault}{\bfdefault}{\updefault}{\color[rgb]{0,0,0}}}}}}
\put(11026,-3526){\makebox(0,0)[lb]{\smash{{\SetFigFont{6}{7.2}{\rmdefault}{\bfdefault}{\updefault}{\color[rgb]{0,0,0}}}}}}
\put(11746,-1996){\makebox(0,0)[lb]{\smash{{\SetFigFont{6}{7.2}{\rmdefault}{\bfdefault}{\updefault}}}}}
\put(12466,-1951){\makebox(0,0)[lb]{\smash{{\SetFigFont{6}{7.2}{\rmdefault}{\bfdefault}{\updefault}{\color[rgb]{0,0,0}}}}}}
\put(15526,-5146){\makebox(0,0)[lb]{\smash{{\SetFigFont{6}{7.2}{\rmdefault}{\bfdefault}{\updefault}}}}}
\end{picture} \caption{After applying \cipm \label{fig.automata41}} \vspace{-2ex}
\end{minipage} 
\end{figure} 
\end{example}






\begin{example}
\label{ex.synch}
The example depicted in Figure~\ref{fig.synch} includes synchronization.
Running the \cipm\ algorithm on  would result in the automaton 
depicted in Figure~\ref{fig.synch-end}. The algorithm would not change .
However the parallel composition of  and  would lead to the
parallel automata in Figure~\ref{fig.synch-end}. This is because by
Theorem~\ref{theo.inv}  and according to the
definition of synchronization transitions .  As the
figure depicts any configuration of the form  for  or
 is unreachable in .  Therefore, according to
Theorem~\ref{theo.parallel} any such configuration is also unreachable in
.






\begin{figure}[htb]
\begin{minipage}[h]{0.4\linewidth}
\center
\resizebox{55mm}{!}{\begin{picture}(0,0)\includegraphics{synch.pstex}\end{picture}\setlength{\unitlength}{1202sp}\begingroup\makeatletter\ifx\SetFigFont\undefined \gdef\SetFigFont#1#2#3#4#5{\reset@font\fontsize{#1}{#2pt}\fontfamily{#3}\fontseries{#4}\fontshape{#5}\selectfont}\fi\endgroup \begin{picture}(10133,9055)(3376,-9632)
\put(5671,-781){\makebox(0,0)[lb]{\smash{{\SetFigFont{6}{7.2}{\rmdefault}{\bfdefault}{\updefault}{\color[rgb]{0,0,0}}}}}}
\put(6886,-6901){\makebox(0,0)[lb]{\smash{{\SetFigFont{8}{9.6}{\rmdefault}{\bfdefault}{\updefault}{\color[rgb]{0,0,0}}}}}}
\put(12556,-3526){\makebox(0,0)[lb]{\smash{{\SetFigFont{8}{9.6}{\rmdefault}{\bfdefault}{\updefault}{\color[rgb]{0,0,0}}}}}}
\put(9406,-3526){\makebox(0,0)[lb]{\smash{{\SetFigFont{8}{9.6}{\rmdefault}{\bfdefault}{\updefault}{\color[rgb]{0,0,0}}}}}}
\put(7426,-2086){\makebox(0,0)[lb]{\smash{{\SetFigFont{8}{9.6}{\rmdefault}{\bfdefault}{\updefault}{\color[rgb]{0,0,0}}}}}}
\put(6256,-3481){\makebox(0,0)[lb]{\smash{{\SetFigFont{6}{7.2}{\rmdefault}{\bfdefault}{\updefault}{\color[rgb]{0,0,0}}}}}}
\put(8401,-6751){\makebox(0,0)[lb]{\smash{{\SetFigFont{6}{7.2}{\rmdefault}{\bfdefault}{\updefault}{\color[rgb]{0,0,0}}}}}}
\put(6106,-5281){\makebox(0,0)[lb]{\smash{{\SetFigFont{6}{7.2}{\rmdefault}{\bfdefault}{\updefault}{\color[rgb]{0,0,0}}}}}}
\put(6061,-4846){\makebox(0,0)[lb]{\smash{{\SetFigFont{6}{7.2}{\rmdefault}{\bfdefault}{\updefault}{\color[rgb]{0,0,0}}}}}}
\put(7696,-8206){\makebox(0,0)[lb]{\smash{{\SetFigFont{6}{7.2}{\rmdefault}{\bfdefault}{\updefault}{\color[rgb]{0,0,0}}}}}}
\put(3691,-1996){\makebox(0,0)[lb]{\smash{{\SetFigFont{6}{7.2}{\rmdefault}{\bfdefault}{\updefault}{\color[rgb]{0,0,0}}}}}}
\put(3736,-8026){\makebox(0,0)[lb]{\smash{{\SetFigFont{6}{7.2}{\rmdefault}{\bfdefault}{\updefault}{\color[rgb]{0,0,0}}}}}}
\put(8911,-8161){\makebox(0,0)[lb]{\smash{{\SetFigFont{6}{7.2}{\rmdefault}{\bfdefault}{\updefault}{\color[rgb]{0,0,0}}}}}}
\put(5941,-9556){\makebox(0,0)[lb]{\smash{{\SetFigFont{6}{7.2}{\rmdefault}{\bfdefault}{\updefault}{\color[rgb]{0,0,0}}}}}}
\put(4951,-3166){\makebox(0,0)[lb]{\smash{{\SetFigFont{6}{7.2}{\rmdefault}{\bfdefault}{\updefault}{\color[rgb]{0,0,0}}}}}}
\put(8326,-4381){\makebox(0,0)[lb]{\smash{{\SetFigFont{6}{7.2}{\rmdefault}{\bfdefault}{\updefault}{\color[rgb]{0,0,0}}}}}}
\put(3691,-4921){\makebox(0,0)[lb]{\smash{{\SetFigFont{6}{7.2}{\rmdefault}{\bfdefault}{\updefault}{\color[rgb]{0,0,0}}}}}}
\put(10936,-9556){\makebox(0,0)[lb]{\smash{{\SetFigFont{6}{7.2}{\rmdefault}{\bfdefault}{\updefault}{\color[rgb]{0,0,0}}}}}}
\put(12241,-5326){\makebox(0,0)[lb]{\smash{{\SetFigFont{6}{7.2}{\rmdefault}{\bfdefault}{\updefault}{\color[rgb]{0,0,0}}}}}}
\put(12241,-1951){\makebox(0,0)[lb]{\smash{{\SetFigFont{6}{7.2}{\rmdefault}{\bfdefault}{\updefault}{\color[rgb]{0,0,0}}}}}}
\put(4681,-1951){\makebox(0,0)[lb]{\smash{{\SetFigFont{6}{7.2}{\rmdefault}{\bfdefault}{\updefault}{\color[rgb]{0,0,0}}}}}}
\put(3376,-3886){\makebox(0,0)[lb]{\smash{{\SetFigFont{6}{7.2}{\rmdefault}{\bfdefault}{\updefault}{\color[rgb]{0,0,0}}}}}}
\end{picture} }
\caption{Parallel composition.~\label{fig.synch}} 
\end{minipage} 
\begin{minipage}[h]{0.7\linewidth}
\center
\resizebox{55mm}{!}{\begin{picture}(0,0)\includegraphics{synch-end.pstex}\end{picture}\setlength{\unitlength}{1202sp}\begingroup\makeatletter\ifx\SetFigFont\undefined \gdef\SetFigFont#1#2#3#4#5{\reset@font\fontsize{#1}{#2pt}\fontfamily{#3}\fontseries{#4}\fontshape{#5}\selectfont}\fi\endgroup \begin{picture}(9678,9010)(3151,-9587)
\put(5671,-781){\makebox(0,0)[lb]{\smash{{\SetFigFont{6}{7.2}{\rmdefault}{\bfdefault}{\updefault}{\color[rgb]{0,0,0}}}}}}
\put(12091,-5011){\makebox(0,0)[lb]{\smash{{\SetFigFont{6}{7.2}{\rmdefault}{\bfdefault}{\updefault}{\color[rgb]{0,0,0}}}}}}
\put(6256,-3481){\makebox(0,0)[lb]{\smash{{\SetFigFont{6}{7.2}{\rmdefault}{\bfdefault}{\updefault}{\color[rgb]{0,0,0}}}}}}
\put(6106,-5281){\makebox(0,0)[lb]{\smash{{\SetFigFont{6}{7.2}{\rmdefault}{\bfdefault}{\updefault}{\color[rgb]{0,0,0}}}}}}
\put(6061,-4846){\makebox(0,0)[lb]{\smash{{\SetFigFont{6}{7.2}{\rmdefault}{\bfdefault}{\updefault}{\color[rgb]{0,0,0}}}}}}
\put(4591,-5056){\makebox(0,0)[lb]{\smash{{\SetFigFont{6}{7.2}{\rmdefault}{\bfdefault}{\updefault}{\color[rgb]{0,0,0}}}}}}
\put(7696,-8161){\makebox(0,0)[lb]{\smash{{\SetFigFont{6}{7.2}{\rmdefault}{\bfdefault}{\updefault}{\color[rgb]{0,0,0}}}}}}
\put(3646,-2041){\makebox(0,0)[lb]{\smash{{\SetFigFont{6}{7.2}{\rmdefault}{\bfdefault}{\updefault}{\color[rgb]{0,0,0}}}}}}
\put(3646,-4966){\makebox(0,0)[lb]{\smash{{\SetFigFont{6}{7.2}{\rmdefault}{\bfdefault}{\updefault}{\color[rgb]{0,0,0}}}}}}
\put(4906,-9511){\makebox(0,0)[lb]{\smash{{\SetFigFont{6}{7.2}{\rmdefault}{\bfdefault}{\updefault}{\color[rgb]{0,0,0}}}}}}
\put(8956,-7846){\makebox(0,0)[lb]{\smash{{\SetFigFont{6}{7.2}{\rmdefault}{\bfdefault}{\updefault}{\color[rgb]{0,0,0}}}}}}
\put(3676,-7801){\makebox(0,0)[lb]{\smash{{\SetFigFont{6}{7.2}{\rmdefault}{\bfdefault}{\updefault}{\color[rgb]{0,0,0}}}}}}
\put(8716,-4561){\makebox(0,0)[lb]{\smash{{\SetFigFont{6}{7.2}{\rmdefault}{\bfdefault}{\updefault}{\color[rgb]{0,0,0}}}}}}
\put(10966,-9466){\makebox(0,0)[lb]{\smash{{\SetFigFont{6}{7.2}{\rmdefault}{\bfdefault}{\updefault}{\color[rgb]{0,0,0}}}}}}
\put(11971,-2026){\makebox(0,0)[lb]{\smash{{\SetFigFont{6}{7.2}{\rmdefault}{\bfdefault}{\updefault}{\color[rgb]{0,0,0}}}}}}
\put(4501,-3211){\makebox(0,0)[lb]{\smash{{\SetFigFont{6}{7.2}{\rmdefault}{\bfdefault}{\updefault}{\color[rgb]{0,0,0}}}}}}
\put(3151,-3841){\makebox(0,0)[lb]{\smash{{\SetFigFont{6}{7.2}{\rmdefault}{\bfdefault}{\updefault}{\color[rgb]{0,0,0}}}}}}
\put(4681,-1951){\makebox(0,0)[lb]{\smash{{\SetFigFont{6}{7.2}{\rmdefault}{\bfdefault}{\updefault}{\color[rgb]{0,0,0}}}}}}
\put(4591,-8116){\makebox(0,0)[lb]{\smash{{\SetFigFont{6}{7.2}{\rmdefault}{\bfdefault}{\updefault}{\color[rgb]{0,0,0}}}}}}
\put(7741,-5056){\makebox(0,0)[lb]{\smash{{\SetFigFont{6}{7.2}{\rmdefault}{\bfdefault}{\updefault}{\color[rgb]{0,0,0}}}}}}
\end{picture} }
\caption{After applying \cipm .~\label{fig.synch-end}}
\end{minipage} 
\end{figure} 

\end{example}










\section{Predicate Abstraction, New Results and the Ongoing Work}
\label{sec.predicate-abst}


In this section, we introduce a method for using the invariants generated by
 \cipm\ in order to build an over-approximating {\em predicate abstraction} of the 
original timed automaton. 
We consider the abstract states not as Boolean vectors over the designated
set of abstraction predicates, but rather as {\em pairs} of control locations and
conjuncted, positive or negative predicates.  
In the sequel we will explain this in more detail.

A {\em cube}  over ,
called a {\em minterm} in~\cite{Lahiri06},
is a conjunction 
 
over the elements of  and their negations, 
i.e.\, each  is equivalent to either  or
its negation .
For example  is a cube over
.   denotes the set of all cubes over
. In the sequel we assume that  for a real time model
 , and our intention is to explain how to generate a predicate
 abstraction for . Without loss of generality, in the remainder of the paper 
we use  for .  
 
  
\begin{definition}[States of . ]
\label{def.abst}
The set  is a
collection of all invariants . Our predicate abstraction over
, denoted , is  a finite state automaton where states are
pairs like  for . 
\end{definition}

Spurious counterexamples when searching in the abstract state space are often due 
to invariant violations in the concrete model.
In order to reduce the risk of generating spurious counterexamples 
we associate with each control location  its invariant
as generated by \cipm . 
These invariants are gathered in .
We first pair up each control location to its own invariant. Then we
add the rest of the cubes from  to the pair. 
During construction of the abstraction each configuration  from the concrete
model is abstracted to a abstract state in which  holds.

Let us consider  as the set of all cubes over  which are
satisfiable in conjunction with the predicates in : 
 



\noindent
For each  we denote by  the abstract state 
.
 abstracts all configurations  in the concrete
model  whose valuation  satisfies , i.e.\ .

\begin{example}
Let us continue with the first example (Figure~\ref{fig.automata41}).
According to the example, we have ,
, 
 and hence, . 
We use  to denote the invariant corresponding to the location , therefore:



Some of these combinations are unsatisfiable, for instance .  
After removing such combinations and eliminating the
'' symbol, for simplicity, we obtain: , , and .
As illustrated in Figure~\ref{fig.abst1} these three sets build an abstract
model  which consists of six states for example like , 
. As we shall see later
on, the dashed line in this figure identifies unreachable states.
\end{example}



\begin{definition}[Transitions of ]{\em
In  we execute a transition from a state  to a state
 only when one of the following conditions holds in the concrete
model :
\begin{itemize}
\item there are two valuations  and  and a non-idle transition  where  and
, or
\item  is identical to , and there is a delay transition  for some valuation  such
that  and .
\end{itemize}}
\end{definition}

\noindent
Let  denote the set of all successor states of  in
, then with respect to definition above:

Recall that  is a discrete and  is a delay transition.

Since  is an abstraction of , each of its transitions should 
have a counterpart in the original model . This means that whenever 
, there must exist a non-idle transition from
at least one of the corresponding concrete states of  to that of . 
Such a transition needs to satisfy all the invariants of the source location and also
all the invariants of the target location. Also if there is a reset for some variable, the new
value of the respective variable should satisfy the invariant of the target location:

\begin{lemma}
\label{lem.trans}
Assume that  is an abstraction of  with respect to some set of
predicates .  There is a transition from  to  in
, i.e. , if and only if 
one of the conditions below holds:
\begin{enumerate}
\item there are two clock valuations  and , and a non-idle transition 
 in the concrete model such that:
  \begin{enumerate}
   \item  and .
   \item if  then  is satisfiable, 
   \item if  then  is satisfiable, 
   \item if  then  is satisfiable, 
   \item for all variables , .
  \end{enumerate}
\item  and  where
   and .
\end{enumerate}
\end{lemma} 
The next theorem shows that in order to establish a predicate abstraction for the
original concrete model  it is enough to do so for the pruned
equivalent version obtained from an application of the  algorithm:

\begin{theorem}
\label{theo.approx}
If , then . \vspace{-0ex}
\end{theorem}

\noindent
The cube  has caused two
different abstract states in Figure~\ref{fig.abst1}. This is because  and  are invariants of 
 and , respectly, and therefore coupled with them in the abstract model.
The dashed line in this figure depicts the set of unreachable
abstract states of the first example. These states are unreachable since they correspond to some 
unreachable concrete states in  (cf. Lemma~\ref{lem.trans}).
Using Lemma~\ref{lem.trans} to compute the transitions in the abstract model, 
one would obtain Figure~\ref{fig.abst3} as the initial predicate abstraction
of . For instance from  there is a transition to
 because the transition  fullfils  Lemma~\ref{lem.trans}.



\begin{figure}
\begin{minipage}[h]{0.5\linewidth}
\center
\begin{picture}(0,0)\includegraphics{abst1-new.pstex}\end{picture}\setlength{\unitlength}{1036sp}\begingroup\makeatletter\ifx\SetFigFont\undefined \gdef\SetFigFont#1#2#3#4#5{\reset@font\fontsize{#1}{#2pt}\fontfamily{#3}\fontseries{#4}\fontshape{#5}\selectfont}\fi\endgroup \begin{picture}(7825,6160)(1241,-9756)
\put(2104,-8807){\makebox(0,0)[lb]{\smash{{\SetFigFont{6}{7.2}{\rmdefault}{\mddefault}{\updefault}{\color[rgb]{0,0,0}}}}}}
\put(6645,-6067){\makebox(0,0)[lb]{\smash{{\SetFigFont{5}{6.0}{\rmdefault}{\mddefault}{\updefault}}}}}
\put(6435,-6577){\makebox(0,0)[lb]{\smash{{\SetFigFont{6}{7.2}{\rmdefault}{\mddefault}{\updefault}{\color[rgb]{0,0,0}}}}}}
\put(2389,-6104){\makebox(0,0)[lb]{\smash{{\SetFigFont{5}{6.0}{\rmdefault}{\mddefault}{\updefault}}}}}
\put(2179,-6614){\makebox(0,0)[lb]{\smash{{\SetFigFont{6}{7.2}{\rmdefault}{\mddefault}{\updefault}{\color[rgb]{0,0,0}}}}}}
\put(6387,-4614){\makebox(0,0)[lb]{\smash{{\SetFigFont{6}{7.2}{\rmdefault}{\mddefault}{\updefault}{\color[rgb]{0,0,0}}}}}}
\put(6597,-4104){\makebox(0,0)[lb]{\smash{{\SetFigFont{5}{6.0}{\rmdefault}{\mddefault}{\updefault}}}}}
\put(2341,-4141){\makebox(0,0)[lb]{\smash{{\SetFigFont{5}{6.0}{\rmdefault}{\mddefault}{\updefault}}}}}
\put(2131,-4651){\makebox(0,0)[lb]{\smash{{\SetFigFont{6}{7.2}{\rmdefault}{\mddefault}{\updefault}{\color[rgb]{0,0,0}}}}}}
\put(2314,-8297){\makebox(0,0)[lb]{\smash{{\SetFigFont{5}{6.0}{\rmdefault}{\mddefault}{\updefault}}}}}
\put(6360,-8770){\makebox(0,0)[lb]{\smash{{\SetFigFont{6}{7.2}{\rmdefault}{\mddefault}{\updefault}{\color[rgb]{0,0,0}}}}}}
\put(6570,-8260){\makebox(0,0)[lb]{\smash{{\SetFigFont{5}{6.0}{\rmdefault}{\mddefault}{\updefault}}}}}
\end{picture} \caption{The states of ~\label{fig.abst1}} \vspace{-2ex}
\end{minipage} 
\begin{minipage}[h]{0.5\linewidth}
\center
\begin{picture}(0,0)\includegraphics{abst3.pstex}\end{picture}\setlength{\unitlength}{1036sp}\begingroup\makeatletter\ifx\SetFigFont\undefined \gdef\SetFigFont#1#2#3#4#5{\reset@font\fontsize{#1}{#2pt}\fontfamily{#3}\fontseries{#4}\fontshape{#5}\selectfont}\fi\endgroup \begin{picture}(6576,6531)(1371,-8043)
\put(2431,-3016){\makebox(0,0)[lb]{\smash{{\SetFigFont{5}{6.0}{\rmdefault}{\mddefault}{\updefault}{\color[rgb]{0,0,.69}}}}}}
\put(6687,-2979){\makebox(0,0)[lb]{\smash{{\SetFigFont{5}{6.0}{\rmdefault}{\mddefault}{\updefault}{\color[rgb]{0,0,.69}}}}}}
\put(6645,-6067){\makebox(0,0)[lb]{\smash{{\SetFigFont{5}{6.0}{\rmdefault}{\mddefault}{\updefault}{\color[rgb]{0,0,.69}}}}}}
\put(6435,-6577){\makebox(0,0)[lb]{\smash{{\SetFigFont{6}{7.2}{\rmdefault}{\mddefault}{\updefault}{\color[rgb]{0,0,0}}}}}}
\put(2389,-6104){\makebox(0,0)[lb]{\smash{{\SetFigFont{5}{6.0}{\rmdefault}{\mddefault}{\updefault}{\color[rgb]{0,0,.69}}}}}}
\put(2179,-6614){\makebox(0,0)[lb]{\smash{{\SetFigFont{6}{7.2}{\rmdefault}{\mddefault}{\updefault}{\color[rgb]{0,0,0}}}}}}
\put(2221,-3526){\makebox(0,0)[lb]{\smash{{\SetFigFont{6}{7.2}{\rmdefault}{\mddefault}{\updefault}{\color[rgb]{0,0,0}}}}}}
\put(6477,-3489){\makebox(0,0)[lb]{\smash{{\SetFigFont{6}{7.2}{\rmdefault}{\mddefault}{\updefault}{\color[rgb]{0,0,0}}}}}}
\end{picture} \caption{~\label{fig.abst3}, predicate abstraction of . } 
\end{minipage} 

\end{figure} 



In the following we give a simple succinctness analysis of our approach: 
Each timed automaton has a finite number of control locations, .
We associate with each location  at most  abstract
states. This way the number of the abstract states is at most  in the worst case. In the example depicted in
Figure~\ref{fig.abst1}, this number is .
By pruning the original model using \cipm\ and also with respect to
Lemma~\ref{lem.trans} this number reduces to  abstract states, see Figure~\ref{fig.abst3}.
With neither detecting the idle transitions nor pairing the control locations
with their invariants, in the abstraction facet, one would have gotten
 abstract states where  is the number of distinguished
satisfiable cubes and  is
the number of control locations. This number would have even raised to  abstract states if no satisfiability check on the cubes was done. 
\vspace{-3.8ex}










\bibliographystyle{plain}
\bibliography{bib}

\end{document}
