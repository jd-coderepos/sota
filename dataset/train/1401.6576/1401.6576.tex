\documentclass[submission,hidelink]{dmtcs-episciences}






\usepackage[round]{natbib}

\usepackage[utf8]{inputenc}
\hypersetup{pdfborder = {0 0 0}
}

\usepackage{subfigure}

\usepackage{amsthm}

\usepackage{stmaryrd}
\usepackage{mathdots}
\usepackage{float}
\usepackage{multirow}
\usepackage{here}
\usepackage{ae,aeguill}
\usepackage{fancyhdr}
\usepackage{amssymb}
\usepackage{wrapfig}
\usepackage{amsmath}
\usepackage{color}
\usepackage{amsfonts}
\usepackage[toc,page]{appendix}
\usepackage{fancybox}
\usepackage{moreverb}
\usepackage{pgf}
\usepackage{tabularx}
\usepackage{tikz}
\usepackage{MnSymbol}

\usetikzlibrary{calc,automata,arrows,chains,matrix,positioning,scopes}


\makeatletter
\pagestyle{plain}
\newenvironment{Proof}{\begin{proof}}{\qed\end{proof}}
\newcommand{\ptvi}{\vrule height 10pt depth 4 pt width 0 pt}
\newtheorem{theorem}{Theorem}
\newtheorem{proposition}[theorem]{Proposition}
\newtheorem{corollary}[theorem]{Corollary}
\newtheorem{example}{Example}
\newtheorem{definition}[theorem]{Definition}
\newtheorem{lemma}[theorem]{Lemma}
\newtheorem{remark}{Remark}

\newcommand{\WF}{K}
\newcommand{\WFp}[1]{\textrm{K}_{#1}}
\newcommand{\WFpp}[1]{\textrm{F}_{#1}}

\newcommand{\Zd}{\Z_d}
\newcommand{\Zn}{\Z_n}
\newcommand{\Cl}[2]{\textrm{C}\ell_{#1}(#2)}
\newcommand{\Z}{{\mathbb Z}}     \newcommand{\N}{{\mathbb N}}
\renewcommand{\leq}{\leqslant}
\renewcommand{\geq}{\geqslant}
\newcommand{\D}{\mathbf{D}}
\newcommand{\K}{\mathbf{K}}


\newcommand{\A}{\mathbf{A}}
\newcommand{\R}{\mathbf{R}}
\renewcommand{\L}{\mathbf{L}}
\newcommand{\equations}[1]{{\llbracket #1\rrbracket}}
\newcommand{\upset}{\uparrow \!}
\newcommand{\DA}{\mathbf{DA}}
\newcommand{\QDA}{\mathbf{QDA}}
\newcommand{\QLDA}{\mathbf{QLDA}}
\newcommand{\LDA}{\mathbf{LDA}}

\newcommand{\QA}{\mathbf{QA}}
\newcommand{\cQA}{\mathcal{QA}}
\newcommand{\QV}{\mathbf{QV}}
\newcommand{\stst}[1]{{#1_{\textrm{st}}}}
\newcommand{\QLV}{\mathbf{QLV}}
\newcommand{\LV}{\mathbf{LV}}
\newcommand{\lV}{\mathbf{\ell V}}

\newcommand{\V}{\mathbf{V}}
\newcommand{\gV}{\mathbf{gV}}
\newcommand{\Com}{\mathbf{Com}}
\newcommand{\cd}{\textrm{cd}}
\newcommand{\J}{\mathbf{J}}
\newcommand{\Jun}{\mathbf{J}_{\mathbf{1}}}
\newcommand{\QJun}{\mathbf{QJ}_{\mathbf{1}}}
\newcommand{\QJ}{\mathbf{QJ}}
\newcommand{\QLJ}{\mathbf{QLJ}}
\newcommand{\LJ}{\mathbf{LJ}}
\newcommand{\cD}{\mathcal{D}}
\newcommand{\cV}{\mathcal{V}}
\newcommand{\cW}{\mathcal{W}}
\newcommand{\cL}{\mathcal{L}}
\newcommand{\cP}{\mathcal{P}}
\newcommand{\cS}{\mathcal{S}}
\newcommand{\cU}{\mathcal{U}}
\newcommand{\cC}{\mathcal{C}}
\newcommand{\cF}{\mathcal{F}}
\newcommand{\cA}{\mathcal{A}}
\newcommand{\cQDA}{\mathcal{QDA}}
\newcommand{\cDA}{\mathcal{DA}}
\newcommand{\cB}{\mathcal B}
\newcommand{\cQV}{\mathcal{QV}}
\newcommand{\cQJun}{\mathcal{QJ}_1}
\newcommand{\cJun}{\mathcal{J}_1}
\newcommand{\FO}{\mathbf{FO}}
\newcommand{\MSO}{\mathbf{MSO}}
\newcommand{\BS}{\mathbf{{\cal B}\Sigma}}
\newcommand{\X}{\mathbf{X}}
\newcommand{\LI}{\mathbf{LI}}
\newcommand{\Y}{\mathbf{Y}}
\newcommand{\F}{\mathbf{F}}
\newcommand{\cc}{\mathcal{C}_0}
\newcommand{\Ae}{A_d}
\newcommand{\Reg}{\mathrm{Reg}}
\newcommand{\MOD}{\mathrm{MOD}}
\newcommand{\loc}{\mathrm{LOC}}
\newcommand{\dloc}{\mathrm{DLOC}}
\newcommand{\dlocl}{\mathrm{DLOC}_{\mathcal{L}}}
\newcommand{\dlocr}{\mathrm{DLOC}_{\mathcal{R}}}
\newcommand{\enpp}{{\odot}}
\newcommand{\enps}{{\circledast}}

\newcommand{\MODV}{\mathbf{MOD}}
\newcommand{\Frag}{\mathbf {F}}
\newcommand{\FOMOD}{\FO[<, \MOD]}
\newcommand{\FOMODd}{\FO^2[<, \MOD_d]}
\newcommand{\T}{\mathbf{TL}}
\newcommand{\Fsigmod}{\cF[\sigma,\MOD]}
\newcommand{\Fsig}{\cF[\sigma]}
\newcommand{\Fsigmodn}{\cF[\sigma, \MOD^n]}
\newcommand{\Fsigmodd}{\cF[\sigma, \MOD^d]}
\newcommand{\EF}{Ehrenfeucht-Fraïssé }
\newcommand{\Rst}{\mathop{\mathcal{R}_{st}}}
\newcommand{\Lst}{\mathop{\mathcal{L}_{st}}}
\newcommand{\Hst}{\mathop{\mathcal{H}_{st}}}
\newcommand{\timder}[2]{C_{\psi_#1}(#2)}
\newcommand{\timderp}[2]{C_{#1}(#2)}

\newcommand{\Fs}{\cF[\sigma]}

\newcommand{\rank}{}
\newcommand{\inv}{^{-1}}

\newcommand{\Ob}{}
\newcommand{\Cnote}[1]{\marginpar{\color{red}{C:\ #1}}}
\newcommand{\Ccomment}[1]{{\color{red}C:\ #1}\\}
\newcommand{\Lnote}[1]{\marginpar{\color{blue}{L:\ #1}}}
\newcommand{\Lcomment}[1]{{\color{blue}{L:\ #1}}\\}


\newenvironment{itemize2}[1]{\begin{list}{--}{\setlength{\leftmargin}{0cm}}#1}{\end{list}}

\newenvironment{conditions}
{\begin{list}{\rm (\theenumi)}{\noindent \usecounter{enumi}\setlength{\topsep}{2pt}\setlength{\partopsep}{0pt}\setlength{\itemsep}{2pt}\setlength{\parsep}{0pt}\setlength{\leftmargin}{2.5em}\setlength{\labelwidth}{1.5em}\setlength{\labelsep}{0.5em}\setlength{\listparindent}{0pt}\setlength{\itemindent}{0pt}}}{\end{list}}

\newenvironment{bulitem}
{\begin{list}{\rm }{\noindent \usecounter{enumi}\setlength{\topsep}{2pt}\setlength{\partopsep}{0pt}\setlength{\itemsep}{2pt}\setlength{\parsep}{0pt}\setlength{\leftmargin}{2.5em}\setlength{\labelwidth}{1.5em}\setlength{\labelsep}{0.5em}\setlength{\listparindent}{0pt}\setlength{\itemindent}{0pt}}}{\end{list}}

\newcommand{\Arr}{\textrm{Arr}}
\newcommand\nostarr[2]{\hspace{2pt}\raisebox{#1pt}{}\hspace{4pt}}
\newcommand\nostar[1]{\nostarr{2}{#1}}
\newcommand\withstarr[2]{\setbox3=\hbox{}\hspace{-\wd3}
\hspace{-3pt}\raisebox{8pt}{\smash{}}\nostarr{#1}{#2}}
\newcommand\withstar[1]{\withstarr{2}{#1}}
\newcommand{\tvi}{\vrule height 12pt depth 5 pt width 0 pt}
\newcommand{\lowtvi}{\vrule height 0pt depth 6 pt width 0 pt}

\newcommand{\Rdec}{-decomposition }
\setlength{\fboxsep}{1.5pt}

\DeclareFontFamily{U}{mathx}{\hyphenchar\font45}
\DeclareFontShape{U}{mathx}{m}{n}{
      <5> <6> <7> <8> <9> <10>
      <10.95> <12> <14.4> <17.28> <20.74> <24.88>
      mathx10
      }{}
\DeclareSymbolFont{mathx}{U}{mathx}{m}{n}

\newcommand{\tinf}[1]{\mathcal{I}_E(#1)}
\newcommand{\ACOM}{\mathbf{ACom}}


\author[L. Dartois, C. Paperman]{Luc Dartois\affiliationmark{1,2}
  \and Charles Paperman\affiliationmark{3}}
\title[Adding modular predicates]{Adding modular predicates to first-order fragments}
\affiliation{
LIF, Aix-Marseille Université, France\\
  Université Libre de Bruxelles, Belgium\\
  Warsaw University, Poland}
\keywords{First order logic, automata theory, semigroup, modular predicates}

\date{\today}



\begin{document}
\maketitle

\begin{abstract}
We investigate the decidability of the definability problem for fragments of first order logic over finite words enriched with modular predicates.
Our approach aims toward the most generic statements that we could achieve, which successfully covers the quantifier alternation hierarchy of first order logic and some of its fragments.
 We obtain that deciding this problem for each level of the alternation hierarchy of
 both first order logic and its two-variable fragment when equipped with all regular numerical predicates
 is not harder than deciding it for the corresponding level
 equipped with only the linear order and the successor. For two-variable fragments we also treat the case of the signature containing
 only the order
 and modular predicates.

Relying on some recent results, this proves the decidability for each level of the alternation hierarchy of the two-variable first order fragment
while in the case of the first order logic the question remains open for levels greater than two.

The main ingredients of the proofs are syntactic transformations of first order formulas as well as the algebraic framework of finite categories.
\end{abstract}



\section{Introduction}

The equivalence between regular languages and automata~\citep{RabinScott59}
 as well as monadic second order
logic~\citep{Buchi60} and finite monoids~\citep{Nerode59} was the start of a
domain of research that is still active today.
	In this article, we are  interested in the logic on finite words, and more precisely the
	question we address is the \emph{definability problem} for fragments of logic.
	Fragments of logic are defined as sets of monadic second order formulas satisfying some restrictions, and are
	equipped with a set of predicates called a \emph{signature}.
	Then the definability problem of a fragment of logic  consists in deciding
	 if
	a regular language can be defined by a formula of .

	This question has already been considered and solved in many cases where the signature
	contains only the predicate , which denotes the linear order over the positions of the word. For instance,
	a celebrated result by~\cite{Schutzenberger65} and~\cite{MP71}
	gave an effective algebraic characterization of languages definable by first order formulas.
	The decidability has often been achieved through
	algebraic means, showing a deep connection between algebraic and logical properties of
	a given regular language.
	This is the approach privileged in this article.

	We investigate the question of the behaviour of the decidability of some fragments
	when their signature is enriched with \emph{modular predicates}.
	These predicates allow to specify the congruence of the position of a variable modulo an integer.
	They form with the order and the local predicates the set of \emph{regular numerical predicates}.
	These predicates are exactly the formulas of monadic second order logic
	without letter predicates. Intuitively they correspond to the maximal
	class of \emph{numerical predicates} that can enrich the signature of a fragment of , while keeping the definable languages regular.
	This question was already considered in the case of first order logic () by~\cite{Bar92} and one
	of its fragments, the formulas without quantifier alternation, by~\cite{Pel92}.

	The enrichment by regular numerical predicates arose in the context of the
	\emph{Straubing's conjectures}~\citep{Straubing94}. Roughly speaking, these conjectures state that
	deciding the definability  of a regular language in a fragment of enriched logic corresponds to
	 deciding its circuit complexity.
	It is known~\citep{Pel92,Straubing94} that
	an enrichment of the classical fragments by regular numerical predicates is equivalent to an enrichment by the signature
	, where  denotes the \emph{local predicates}
	and  the \emph{modular predicates}.
	A first step toward  the study of fragments of logic with
	these predicates was initiated by~\cite{Straubing85}.
	He obtained  that
	adding the local predicates preserves the decidability for a large number of fragments. As a corollary
	of this work, Straubing obtained that the decidability of the alternation
	hierarchy of first order logic () equipped with  reduces to the
	decidability of the simpler one .
More recently,~\cite{KL13} proved the decidability of the
	alternation hierarchy of the two-variable first order
	fragment () equipped with  by extending the recent results by~\cite{KS12} and~\cite{KW12} on the
	decidability of this hierarchy with~.

	In this context, the case of modular predicates is poorly understood.
	The study of this enrichment
	was first considered for first order logic by~\cite{Bar92}, and had been extended to the first
	level of its alternation hierarchy with the successor predicate by~\cite{Pel92},
	and later without it by~\cite{CPS06b}.
		The enrichment by a finite set of modular predicate was considered by~\cite{EI03}.
	Finally, the authors
	provided a characterization of the two-variable first order logic over the signature~~\citep{DP13}.
In this paper, we focus on the enrichment by all regular predicates as well as the question of the enrichment
	by modular predicates only. This latter one surprisingly turns out to be more intricate.

	To study this enrichment in a generic setting, we offer a definition of fragment as a set of formulas satisfying some syntactic properties.
	This allows for some generic proofs instead of a one by one situation.
	The main applications of our theorems are then the \emph{quantifier alternations} hierarchies of the first order logic
	and its two-variable counterpart.
	Our main results state that for both of these hierarchies, the decidability of each level
	equipped with regular numerical predicates reduces to decidability of the same level with the signature .
	Then by using
	the recent decidability result of~\cite{KL13}, as well as the decidability of 
	by~\cite{PZ14}, we deduce that the fragments  and , for any positive , as well as  are decidable.
	Our settings also reproves known results and apply to fragments of first order with small signatures.


	\paragraph{Proofs methods.}
	The proofs of the main results can be decomposed in two major steps.
	 The first part is rather classical and shows that the information given by a finite number of modular predicates can be put into the alphabet and thus we can reduce the problem to a question on the fragment over a bigger alphabet.
	The second part is dedicated to finding a systematic way to select,
	for a given regular language and a fragment, a finite number of
	modular predicates that can serve as witnesses of its definability.
This is done through the use of the algebraic
	framework of varieties, using two mains approaches.
	The first one uses finite categories and the global of a variety, while the second one
	introduces a new  notion for
	varieties of semigroups that we call the infinitely testable property.
	Under some assumptions, we show that this property
	allows us to find such a witness set for modular predicates.

	\paragraph{Organization of the paper.}
	The next section is dedicated to the basic logical and algebraic definitions, and the main applications of our results to logic are presented in Section~\ref{Section:Main}.
	 Then Section~\ref{Section:Finite} deals with adding a finite number of modular predicates.
	 This is done through an easy reduction to adding predicates modulo a given congruence.
	 In Section~\ref{Section:Delay}, we then deal with the delay problem, which can be quickly stated as computing a finite set of congruences that can serve as a witness for the definability problem of a language.
	 More specifically, we first introduce the framework of categories as an extension of the monoids theory,
	 and use it to prove a delay for differents classes of fragments.
	 In Subsections~\ref{Subsection:Local} and~\ref{Subsection:FiniteRank}, we rely on an algebraic description of the global of a variety, which is a variety of finite categories.
	 Then Subsection~\ref{SsSection:InfTest} solves the delay for a class of fragments satisfying a given algebraic property, the so-called \emph{infinitely testable property}.

\section{Preliminary definitions}\label{Section:Defs}
	\subsection{Languages and Logic}
	We consider the monadic second order logic on finite words  as usual
	(see~\cite{Straubing94} for example).
	We denote by  an \emph{alphabet} and by  a \emph{letter} of .
	A word  over an alphabet  is a set of labelled positions ordered from  to , where  is an integer denoting the \emph{length} of .
	The set of words over  is denoted  and a subset  of  is called a \emph{language}.
	We also denote by  the set of non-empty words.
	A language is said to be \emph{defined} by a formula if it corresponds exactly
	to the set of words that satisfy this formula.
	It is said to be \emph{regular} if it is defined by a  formula.
	When syntactic restrictions are applied to ,
	one defines fragments of logic that characterize subclasses of regular languages.
	The most well-known fragment is probably the first order logic,
	whose expressive power was characterized
	thanks to the results of~\cite{MP71} and \cite{Schutzenberger65}.
	The first order logic itself gave birth to its own zoo of fragments.
	These were defined using syntactical restrictions such as limiting the number of variables,
	or by enrichment of its signature.
	A fragment  with signature  will  be denoted  and will refer to the formulas as well as the class of languages it
	defines.

	We first define the different signatures that will appear through this paper,
	and then formally define the quantifier alternation hierarchies, as they form the main focus of the applications of our theorems.



\paragraph{Signatures.}
	We are interested in regular numerical predicates, which are numerical predicates that can only define regular languages.
	Simultaneously,~\cite{Straubing94} and~\cite{Pel92}  defined three sets of regular numerical predicates that can be used as a base for all the regular numerical predicates.
	The first set is the singleton order  which is a binary predicate corresponding to the natural order on the positions of the input word.
	The second set is  and is called the \emph{local predicates}.
	The predicates  and  are unary predicates that are satisfied respectively on the first and last positions.
	The predicate , the -\emph{successor}, is a binary predicate satisfied if the second variable quantifies the
	-successor of the first one.

	\begin{example}
		The formula 
		defines the regular language .
	\end{example}
	We alternatively use the \emph{descriptive local predicates}.
	These predicates are of the form
	 (resp. , ) for , holding if the position at  (resp. , ) is labelled by an .
	\begin{example}
		The previous formula can be rewrite by the following quantifier-free formula: .
	\end{example}
	Most of the time, both \emph{descriptive} and classical local predicates provides the same expressive power. However the descriptive
	predicates are proved to be more convenient for abstract fragments since they don't bound two variables together. For the sake of
	simplicity we will denote in the following by  this class of descriptive local predicates. This notation is justified thanks to
	the close relation between descriptive local predicates and the successor function.
	Also note that the presence or absence of the equality predicate is important since  is strictly less expressive than
	.

	Finally, we define, for each positive integer , the \emph{modular predicates on }, denoted , as the set, for , of predicates
 which are unary predicates satisfied if the position quantified by  is congruent to  modulo , and the predicates
	 which are constants holding if the length of the input word is congruent to .
	We denote by  the union of the classes , for any positive .
	\begin{example}
		The language  is defined by the formula:
		
	\end{example}

The signatures that we will consider for our fragments are unions of these three sets of regular numerical predicates, and will always contain the letter predicates.
Abusing notations, we will also write .


\paragraph{Fragments.}
\
A \emph{fragment of logic}  with signature  is a set of closed formulas of 
	that contains the quantifier-free formulas and that is
	closed
	under the following operations :
	\begin{description}
	\item[\textit{Conjunction}] If  and  are formulas of ,
				then  is also a formula of .
	\item[\textit{Disjunction}] If  and  are formulas of ,
				then  is also a formula of .
\item[\textit{Quantifier-free substitutions}] If  is a formula of  and  a quantifier-free subformula of  with free variables , then any formula obtained by replacing  by another quantifier-free formula with the same set of free variables is also in .
	\end{description}

	If  is a fragment of logic and  is a class of predicates, then the \emph{enrichment} of  by  is denoted by  and corresponds to the closure of  under the quantifier-free substitutions, where predicates range over the signature .
As a closed formula defines a language, a fragment of logic defines a class of languages. Abusing notations, we will denote by  a
	fragment of logic, as well as the class of languages it recognizes.
	It is worth noting that~\cite{KL12} defined another notion of fragment of logic as sets of
	formulas closed under some syntactical substitutions ensuring algebraic characterisation of the fragment.

	The fragment  is the subclass of formulas of  using only two symbols of variables which can be reused (see Example~\ref{FO2:ex}).
	Here, the class of languages defined by  is strictly contained in  and  (see~\cite{TW98,DP13}).
	\begin{example}\label{FO2:ex}
	The language  can be described by the first order formula
	
	This formula uses three variables  and . However, by reusing  we get an equivalent formula that uses only two variables:
	
	\end{example}



\paragraph{Alternation hierarchies.}
	Given a first order formula, one can compute a prenex normal form using the De Morgan's laws.
	We define the \emph{quantifier alternation depth} of a formula as the number of blocks  of quantifiers  and  in its prenex normal form.
	For example, the formula  has a quantifier depth of .
	It describes the language .
	Then given a signature  and a positive integer , we denote by  the set of prenex normal formulas of  whose quantifier depth is smaller or equal to .
 They form the levels of the \emph{quantifier alternation hierarchy} over .

	When  is reduced to , this hierarchy is called the Straubing-Th\'erien hierarchy~\citep{Straubing81,Th81}.
	Only the first~\citep{SI75} and second~\citep{PZ14} levels are known to be decidable.
	For , this hierarchy is called the Dot-Depth hierarchy~\citep{BC71}.
	The decidability of each level reduces to the decidability of the corresponding
	level of the Straubing-Th\'erien hierarchy~\citep{Straubing85}.
	In both cases, the hierarchies are known to be strict, and cover all Star-Free languages.
	In this article, we also consider the alternation hierarchy of .
	To define formally the number of alternations of a formula,
	we cannot rely on the prenex normal form since the construction increases the number of variables.
	In particular, remark that  is equivalent to  which
	is a subclass of ~\citep{DGK08}.
	 That said, the number
	of alternations is still a relevant parameter that could be defined as follows:
	Consider the parse tree naturally associated to a formula. For instance,
	\eqref{eq1} has  as a root  and the atomic formulas as the leaves.
	In a two-variable first order formula we count the maximal number of alternations appearing on a branch, i.e. between the
	root and a leaf,
	once the negations have been pushed on to the leaves. A more precise definition
	can be found in~\cite{IW09}. We denote by 
	the formulas of  that have at most  quantifier alternations.
	The hierarchy induced by  is known to be strict~\citep{IW09} and its definability
	problem is decidable~\citep{KS12,KW12}.
	Note that the hierarchy  is also known to be decidable~\citep{KL13}.

	\textbf{Remark:} The classes of formulas  and 
	as well as each level of the alternation hierarchies are fragments of  as defined previously.

	\subsection{Varieties of languages, monoids and semigroups}
We quickly present here the fundamental notions used by the article and refer the reader to the book of~\cite{Pin97a} for a detailed approach.
A (finite) \emph{semigroup} is a finite set equipped with an associative internal law.
A semigroup with a neutral element for this law is called a \emph{monoid}.
Recall that a semigroup  \emph{divides} another semigroup  if  is a quotient of a
subsemigroup of .
This defines a partial order on finite semigroups.
Given a finite semigroup , an element  of  is \emph{idempotent} if .
We denote by  the set of idempotents of .
For any element  of , there exists a positive integer  such that  is idempotent. We call this element the \emph{idempotent power} of  and denote it by .
One can check that the application  is well defined.

A semigroup  recognizes a language  over an alphabet  via a \emph{morphism}
.
Given a regular language , we can compute its \emph{syntactic semigroup} as the smallest semigroup that recognizes , in the sense of division.
A subset  of  is an \emph{ideal} if the sets  and  are both included in .
A (pseudo-)\emph{variety} of semigroups (resp. monoids) is a non empty class of finite semigroups (resp. monoids) closed under
division and finite product.
Finally, a local monoid of  is a monoid of the form  where  is an idempotent of .

A fragment of logic is \emph{characterized} by a variety if they recognize the same languages.
By extension, a variety  will also refer to the class of languages it recognizes.
The most famous example is the equality ~\citep{MP71,Schutzenberger65}, where  denotes the class of aperiodic semigroups, which are
finite semigroups that are not divided by any group.
As for , the definability problem for a fragment of logic has often been solved thanks to an algebraic characterization (\cite{SI75,Th81,TW98} for example).
This decidability is sometimes obtained through \emph{profinite equations}.
We refer the reader to~\cite{PIN09} for a survey on the profinite background. The algebraic characterisations of most the fragments that we consider are given in Figure~\ref{Tableau:Alg}.
\begin{figure}[h]
\begin{tabular}{|c|c|c|}
\hline\tvi
Fragment & Variety & Equations \\
\hline\tvi
 &   &  \\
\hline\tvi
 &  & ,  \\
\hline\tvi
&  &  \\
\hline\tvi
 &  & ,  \\
\hline\tvi
 &  & \\
\hline\tvi
 &  & See Example~\ref{FinRank-Ex} \\
\hline


\end{tabular}
\centering
\caption{Algebraic characterisations}\label{Tableau:Alg}
\end{figure}

\paragraph{Stability index.}
One important tool to study modular predicates is the stability index.
For a monoid morphism , the set  is an element of the powerset
monoid of .
As such it has an idempotent power. The \emph{stability index} of a morphism
is the 	least positive integer 
such that . This set forms a subsemigroup called
the \emph{stable semigroup} of
.
The set  is called the \emph{stable monoid}
of .
The stable monoid (resp. semigroup) of a regular language is the stable monoid (resp. semigroup) of its syntactic morphism.



\section{Adding finitely many modular predicates}\label{Section:Finite}

We consider here the question of adding the modular predicates associated to a finite set of congruences.
First, let us remark that if
 is a fragment of logic and  and  are two positive integers,
then . This can be proved by some basic arithmetic reasoning and some quantifier-free substitutions.
Then as a formula only uses a finite number of modular predicates, for any language of , there exists an integer  such that it belongs to .
The consequence is that adding a finite set of modular predicates is equivalent to adding the predicates relating to one specific congruence.
The remainder of this section deals with this question.



\subsection{Alphabet enriched by modular counting}

In order to deal with modular predicates, we now define enriched modular alphabets. These notions come naturally in the context of \emph{wreath product} and instantiated for instance in~\cite{DP13,DP15}.
We now fix a positive integer  and an alphabet . Let  be the cyclic group of order .
	\begin{definition}[Enriched alphabet]
		We call the set 
		the \emph{enriched alphabet} of ,
		and we denote by  the projection defined by 
		for each .
		For example,
		the word  is an enriched word of  for .
		We say that  is the \emph{underlying word} of .
	\end{definition}
	\begin{definition}[Well-formed words]
		A word  of  is \emph{well-formed} if
		for , . We denote by  the set of
		all well-formed words of .
We also note  the set of well-formed factor such that the first letter is labelled by  and the last by .

		For any , let  be the function defined
		for any word 
		by .
		We simply denote  by  and the word  is called the \emph{well-formed word attached} to u.
	\end{definition}

		Note that the restriction of  to the set of well-formed words is one-to-one.
		For instance, the enriched word  is a well-formed word for .
		It is the unique well-formed word having the word
	 as underlying word. Finally, given a language , we write .

\subsection{A first transfer theorem}
Using the enriched alphabet and the well-formed words, the next theorem links a fragment with its enrichment by congruences modulo one integer.
It transfers the expressiveness of modular predicates to the alphabet.
An aware reader could notice that it is very similar to the wreath product principle of varieties. It is in fact not a coincidence, since
this operation matches with a wreath product by the \emph{length-multiplying} variety  (see~\cite{CPS06b} for more details).
\begin{theorem}\label{semimod}
			Let  be a fragment of logic,
			 a regular language and  a positive integer.
			Then the following properties are equivalent:
			\begin{conditions}
				\item\label{thmsemimod:1}  is definable by a formula of ,

				\item\label{thmsemimod:3} there exists
				      some languages  of
				       over  such that:
			\end{conditions}
\end{theorem}
		To prove the result,
 		we need an auxiliary result which gives a decomposition of the language defined by a formula into smaller pieces.

	\begin{lemma}\label{formNorm}
	Let  be a fragment of logic and  a formula of .
	Then there exists  formulas  of  that do not contain any predicate 
	and such that 
	Moreover, we have:  
	\end{lemma}
	\begin{proof}
	For , we define the formula  to be the formula  where we replaced
		every predicate  by \emph{true} and every  with  by \emph{false}.
		One should notice that, by definition of a fragment, the formulas
		 are in .
		We can conclude the proof since the formula 
		recognizes the language .
	\end{proof}


		\begin{proof}[of theorem~\ref{semimod}]
		Let  be a formula of . Then  belongs to  for some .
		Using Lemma \ref{formNorm}, we know it is sufficient to consider a formula
		 without any length predicate.
		We transform it into a formula  by doing the following transformation:
		
		The resulting formula
		 is in  and .
		Conversely, we transform a formula  of  into a formula  of  by
		replacing every predicate  in  by .
		We also get .
	\end{proof}

	The previous theorem provides a semantic counterpart to the action of adding modular predicates to a fragment of logic. In the case where
	the fragment is \emph{expressive enough}, this counterpart provides a transfer of decidability, as stated in the next
	corollary.
	\begin{corollary}[The transfer result]\label{Cor:transfer}
		Let  be a fragment of logic. If  is decidable and if both  and 
		are
		definable in
		, then
		 is decidable.
	\end{corollary}
	\begin{proof}
	The result comes from the fact that if  is definable, then using modular predicates the languages  are definable in .
	If furthermore we can define the language of well-formed words, then item~\ref{thmsemimod:3} of Theorem~\ref{semimod} is equivalent to the language  being definable in  over the enriched alphabet.
	This language being computable from , we get decidability.
	\end{proof}
	\textbf{Remark:} Corollary~\ref{Cor:transfer} applies to fragments , , when  and  contains either  or the order. It
	also applies to fragments , ,  or  when
	 is contained in .

\section{Main results}\label{Section:Main}

As stated in the previous section, any language defined by a fragment with modular predicates can be done so with a formula using only congruences to one specific integer.
In fact, there exists an infinite number of such witnesses.
The remaining of the article is dedicated to the problem of deciding one witness, given a language.
We call it the \emph{delay problem} and can be explicitly stated as follows:\\
\noindent\textbf{The delay question:}
Given a regular language  and a fragment , is it possible to compute an integer  such that  belongs to  \emph{if, and only if}, it belongs to ?\\

\noindent
Remark that such an integer  could depend of  and . The denomination stems from the Delay Theorem of~\cite{Straubing85} that solves a similar question for the enrichment
by the
successor predicate.
Section~\ref{Section:Delay} is devoted to solve the delay problem for different classes of varieties.
It relies heavily on algebraic notions, in particular the framework of categories.
We present here the main applications to fragments of logic, which are summed up in Figure~\ref{TableauFinal}.
This figure does not include decidability of the smaller fragments of  equipped with modular predicates:  (Theorem~\ref{Main:local}),  (Theorem~\ref{Main:local2}),  (Theorem~\ref{Main:finiterank}) and  (Theorem~\ref{Main:InfTest}).
\newcommand{\newresult}{\multirow{2}{60pt}{\textbf{New result}}}
\begin{figure}

\centering

\begin{tabular}{|c|c|c|c|}
\hline
\tvi &  &  &  \\
\hline 
\tvi \multirow{2}{60pt}{ } & \cite{SI75} & \multirow{2}{110pt}{\cite{CPS06b}}& \multirow{2}{100pt}{ \cite{MPT00}} \\
&\cite{TO82}&&\\
 \hline
\tvi \multirow{2}{60pt}{ } & \cite{KS12} &  \newresult &  \newresult\\
\tvi & \cite{KW12} & & \\
 \hline
 
 \tvi \multirow{2}{60pt}{\hfil } &  \multirow{2}{120pt}{\cite{TW98}} & \multirow{2}{130pt}{\cite{DP13}} &\newresult \\
 \tvi & & & \\
 \hline
 
  \tvi \multirow{2}{60pt}{\hfil } &  \multirow{2}{110pt}{\cite{PZ14}} & \newresult & \newresult \\
  \tvi & & & \\
 \hline
 
 \tvi \multirow{2}{60pt}{\hfil } & \multirow{2}{40pt}{Open }& Reduces to  & Reduces to \\
\tvi & & \textbf{New result}& \textbf{New result} \\
\hline 

 \tvi \multirow{2}{60pt}{\hfil } & \cite{MP71} &   \multirow{2}{75pt}{\cite{Straubing94}}  &  \multirow{2}{110pt}{\cite{Bar92}} \\
\tvi & \cite{Schutzenberger65} &  & \\
\hline 
\end{tabular}
\caption{Decidability results of first-order fragments}\label{TableauFinal}
\end{figure}


 
The first decidability results comes from the local property.
Although it does not bring many new results, mainly reproving~\cite{Bar92} and~\cite{DP13}, it gives a unified proof for these fragments.
\emph{Local} varieties have a particular role in the previous work of Straubing, where they are identified as varieties that \emph{behave}
gently compared toward . In the context of modular predicates, they also have this good property that allows us to state a
fairly generic statement under this assumption. A formal definition of locality can be found in Section~\ref{Subsection:Local}.
\begin{theorem}[Local case, for monoids varieties]\label{Main:local}
	Let  be a fragment equivalent to a local variety .
Now let  be a regular language and  its stability index, then the following statements are equivalent.
\begin{itemize}
	\item   belongs to .
	\item   belongs to .
	\item  the stable monoid of  belongs to .
\end{itemize}
Furthermore, if  is decidable, then so is .
\end{theorem}
Example of interest includes  ,  or , which are equivalent to ,  and  respectively.
The locality of  and  can be found in the article of~\cite{Tilson}, the locality of  is slightly more intricate (see~\cite{Almeida96}).


When the initial variety is local, we can nest our approach with the one with the successor predicates. It is no longer needed to
use the intricate framework of categories since in this case, we can apply Corollary~\ref{Cor:transfer} to slightly simplify the question.
\begin{theorem}[Local case, for semigroups varieties]\label{Main:local2}
Let  be a fragment corresponding to a local variety 
Now let  be a regular language and  its stability index, then the following statements are equivalent.
\begin{itemize}
	\item   belongs to .
	\item   belongs to .
	\item  the local monoids of the stable semigroups belongs to .
\end{itemize}
Furthermore, if  is decidable, then so is .
\end{theorem}
This theorem is a consequence of Proposition~\ref{prop:QLV}.
Note that both  and  fall into the scope of this theorem. In the case of full first order
logic, the successor predicate being definable with the order, the expressiveness remains unchanged. The reduction to logic of these results can be found in Subsection~\ref{SsSection:InfTest} for Theorem~\ref{Main:local2} and Subsection~\ref{Subsection:Local} for Theorem~\ref{Main:local}. Note that this provides decidability.


A generalized approach of the previous results brings fresh ones, although we fail to obtain a delay independent from the fragment.
We need to assume some properties on the \emph{varieties of categories} generated by the initial variety. In particular, we assume
that the \emph{path-equations} of the so called \emph{global} of a variety use a bounded number of vertices. Under this assumption we successfully compute a delay.
\begin{theorem}[Finite rank case]\label{Main:finiterank}
Let  be a fragment corresponding to a variety  of rank .
Now let  be a regular language and  its stability index, then the following statements are equivalent.
\begin{itemize}
	\item   belongs to .
	\item   belongs to .
\end{itemize}
Furthermore, if  is decidable, then so is .
\end{theorem}
Example of application of this theorem include  which is known to be equivalent to the variety of rank  of aperiodic and
commutative monoids, as well as the alternation hierarchy of  whose  level is of rank . This approach is detailed in Section~\ref{Subsection:FiniteRank}. In those cases, this last theorem also provides decidability by reducing to decidability of
the fragment with the successor predicate.

Finally, the next theorem provides a delay for all fragments containing the successor predicates. In particular,
it reduces the decidability of  to the decidability of  providing decidability for the fragment
 and a reduction of the decidability of  to the decidability of , which itself reduces
to decidability of  thanks to~\cite{Straubing85}.
\begin{theorem}[Infinitely testable case]\label{Main:InfTest}
Let  be a fragment corresponding to a variety  which is not a variety of groups.
Now let  be a regular language and  its stability index, then the following statements are equivalent.
\begin{itemize}
	\item   belongs to .
	\item   belongs to .
\end{itemize}
Furthermore, if  is decidable, then so does .
\end{theorem}
The condition that  is not equivalent to a group variety is necessary to apply the simplification of Corollary~\ref{Cor:transfer}.
However, in the case where  is indeed a variety of groups, then both  and  are decidable
since varieties of groups are known to be local as variety of monoids but seems intricate when both  and  are in the signature since groups are not local as varieties of semigroups (for instance see book~\cite[page 104]{RS09}).
The proof of this last theorem is given in Subsection~\ref{SsSection:InfTest}.

\section{Solving the Delay problem}\label{Section:Delay}



This section is devoted to solve the delay question for different classes of varieties.

We first present the framework of finite categories as well as some known results, and use it to
reduce the combinatoric characterisation of Theorem~\ref{semimod} to the decidability of the global of a variety, an algebraic notion from the framework of finite categories.


The remainder of the section then uses this characterisation to solve the delay question for different classes of varieties.
The first case is the simplest one of local varieties, where we get a clear characterisation of .
The second case, the finite rank, is a generalisation of the local case, where an algebraic characterisation of the global is known.
Finally, the last case solves the delay for a class of varieties where little is known about the global.
It is the class of varieties of semigroups \emph{expressive enough} and satisfying an extra property:
the \emph{infinitely testable property}, which is a new notion.


\subsection{A derived category theorem}
\paragraph{Finite categories: a short introduction.}

		In this section, we present the theory of finite categories, as an extension of finite monoids.
		Informally, a category can be seen as a partial monoid where only some products are allowed.
		Nonetheless, notions from monoids can be correctly lifted, and we will consider varieties of categories.
		The framework of variety of categories has been successful to obtain algebraic characterizations of
		\emph{wreath products} of varieties~\citep{Tilson}.
		For example, the enrichment by modular predicates can be seen as a wreath product by a variety of morphisms.
		This comes from an adapted version of the Wreath Product Principle, that is evoked by~\cite{CPS06b}.
		We chose not to focus on this, since it would require to introduce additional definitions and proofs that are not necessary and would burden the article.


		\emph{A graph } is a set of \emph{objects} denoted 
		such that for any couple of objects , we associate a set  of \emph{arrows} from  to .
		Two arrows	 are coterminal if there exists  such that . They are consecutive if there exists
		 such that  and .  An arrow  is a loop from
		 if . A  composition law associates to each pairs of consecutive arrows,  an arrow .
		This law is said to be associative if for any consecutive arrows  we have .

		A \emph{category}  is a graph
		with an associative composition law and containing
		for each object  an identity denoted .
		Thus the set of loops around a given object, equipped with the composition law, forms a monoid,
		called the \emph{local monoid} of that object.
		Note that the terminology of local monoids of a category clashes with the terminology of local monoids of a semigroup.
		In fact, the two coincide when we consider the idempotent category of a semigroup, which is defined later.

		Here we only consider categories as
		a generalization of finite monoids, since a monoid can be viewed as a
		one-object category.
		A \emph{morphism of categories}  is an application
		 and for each pairs
		of object , an application  such that
		\begin{conditions}
			\item for any consecutive arrows  we have ,
			\item for any , .
		\end{conditions}

		A \emph{division} of categories 
		is given by a mapping ,
		and for each pair of objects  and , by a relation 
		such that
			\begin{conditions}
			\item  for consecutive arrows ,
			\item  for any arrow ,
			\item  for any object  of .
			\end{conditions}
	We remark that the inverse of an onto morphism of categories is a division of a categories (but the converse is not
	true).
	Then a \emph{variety of categories} is a class of categories closed under direct product and division.

	\begin{definition}
		Given a variety of monoids , the \emph{global} of , denoted , is the class of all categories that divide a monoid of , when seen as a one-object category.
\end{definition}

	\noindent\textbf{Remark:} Since the division of categories is a partial order and a variety is closed under product, the class of categories  is closed by division and by product, and it is therefore a variety of categories.

	\begin{definition}[Consolidated semigroup, consolidated stamp]
			Let  be a finite category and 
			the set of arrows of . We denote by  the semigroup
			defined on the set
			
			with   for any , , and for ,
			
	\end{definition}
	The following proposition is a well-known result stating that the membership of a category
	in  reduces to the membership of  is the variety is \emph{expressive enough}. This is a
	\emph{category} version of Corollary~\ref{Cor:transfer} which means that the membership of a language to an expressive enough fragment enriched with a finite set of modular predicates reduces to the membership of a different language to the fragment without them.
\paragraph{Background: the local predicates and derived category for locally testable language.}
	\
		In this section, we recall some known results that we will be using in the remainder of the article and give some intuitions about their significance.
		We first give the definition of the derived category for definite languages and provide
		the delay theorem of~\cite{Straubing85} as well as its improvement
		by~\cite{Tilson}.
			Let  be a semigroup,  an integer and  the canonical semigroup morphism
			of .
			The \emph{-derived category} of  with respect to definite languages, denoted , is
			the category with  as set of objects,
			and the arrows from  to  are the elements  of  such
			that there exists a word  that  and  the
			suffix of size  of  is equal to .
			The -derived category with respect to definite languages, of a regular language ,
			denoted , is the category .
		Finally we also introduce the \emph{idempotents' category} of a semigroup , denoted by
		 and defined by~\cite{Tilson} as follows. Its set of objects are
		the idempotents of . And for  and  two idempotents, we set .
		We do not recall the definition of the wreath product of a variety  by , denoted by
		.
		However, as our only use of this product is given by the following theorem, an unfamiliar reader can take
		the following theorem as a definition.

		\begin{theorem}[Delay theorem for definite languages]\label{thm:delaydefinite}
			Let  be a variety and  a semigroup. The following conditions are equivalent.
			\begin{conditions}
				\item The semigroup  belongs to .
				\item There exists an integer  such that  belongs to .
				\item For ,  belongs to .
				\item The category  belongs to .
			\end{conditions}
		\end{theorem}
		For sufficiently expressive fragments, the operation of adding the local predicates corresponds to mapping the equivalent variety  to .
		In fact, it will not be the case only if the fragment cannot use these predicates properly.
		In all cases, it is equivalent to adding the descriptive local predicates defined in Section~\ref{Section:Defs}.
The proof of the following proposition follows the proof of Theorem~\ref{semimod}, by using an adapted
		notion of enriched alphabet. We omit the proof, that could be find in~\cite{PapermanPhd}.
	\begin{proposition}\label{Prop:localpredfragment}
			Let  be a
			fragment of logic equivalent to a variety  and
			 a regular language with  as syntactic semigroup.
			The following conditions are equivalent.
			\begin{conditions}
\item { is definable in .}
				\item  belongs to .
				\item  belongs to .
			\end{conditions}
		\end{proposition}
	\paragraph{The derived category relatively to modular languages.}
		Following the preceding paragraph, we give the definition of the \emph{derived category} adapted to
		modular languages which was largely inspired by the article of~\cite{CPS06b}.

Let  be a morphism and  an integer.
			The \emph{-derived category} of , denoted , is
			the category with  as set of objects,
			and the arrows from  to  are the elements  of  such
			that there exists a word  satisfying  and .
			The -derived category of a regular language , denoted , is the category .
			The following lemma is a straightforward consequence of the definition that will be of some use.
		\begin{lemma}\label{monLocaux}
		Let  be a positive integer, and  be a regular language of stability index .
		Then the local monoids of  are isomorphic to .
		In particular, the local monoids of  are isomorphic to the stable monoid of .
		\end{lemma}


		\begin{example}\label{Excatder}
		The -derived category of the language  is given below.
		Let  be its syntactic morphism and  its stable monoid. Its stability index is .
		\begin{figure}[h]
\begin{tikzpicture}[scale=0.65,->,>=latex',shorten >=1pt,node distance=1.2cm, every loop/.style={looseness=5}]
\node [state](q_0){};
\node [state](q_1) at () {};
\node [state](q_2) at (){};
\node [state](q_3) at (){};
\node (S) at () {};
\node (SA) at () {};
\node (SA2) at () {};
\node (SA3) at () {for , 
};
   \path[->]
   (q_0) edge [bend left = 25]  node[fill = white] {} (q_1)
   (q_1) edge [bend left = 25] node[fill = white] {} (q_0)
   (q_0) edge  [bend left = 25]node[fill = white] {} (q_3)
   (q_3) edge [bend left = 25]  node[fill = white] {} (q_0)
   (q_2) edge [bend left = 25]  node[fill = white] {} (q_3)
   (q_3) edge [bend left = 25]  node[fill = white] {} (q_2)
   (q_2) edge  [bend left = 25]	 node[fill = white] {} (q_1)
   (q_1) edge [bend left = 25]  node[fill = white] {} (q_2);
 
   \path[<->]
   (q_1) edge  node[pos = 0.5] {} (q_3)
   (q_0) edge  node[pos = 0.5, fill = white] {} (q_2) ;

  
   \path[->]
   (q_0) edge  [in = 60, out = 120, loop] node[fill = white]  {} ()
   (q_2) edge  [in = -60, out = -120, loop] node[fill = white]  {} ()

   (q_1) edge  [in = 30, out = -30, loop] node[fill = white]  {} ()
   (q_3) edge  [in = 210, out = 160, loop] node[fill = white]  {} ();    
     
  
\end{tikzpicture}

\end{figure}
 		\end{example}


		\begin{proposition}\label{prop:division_cat}
		Let  be a regular language.
		For any , if  divide , then   divides .
		\end{proposition}
\begin{proof}
	Let  be regular language
	and  be integers such that  divides . We define the relation .
			The object application  is defined by 
			for any . Let  be an arrow of . By definition,
			there exists  such that
			 and .
			Let  and . Then, since  divides ,
			. Thus,
			 the arrow
			 is in .
			We define .
			The application  is a morphism and for any
			, we have
			. Therefore,
			 defines a division from  to .
\end{proof}

		The derived category theorem was originally proved by~\cite{Tilson} for varieties of monoids and semigroups. Unfortunately the case of
		 modular languages can not be dealt with the framework of Tilson since they do not
		 form a variety of language. However it has been extended to \emph{length-multiplying} varieties in the PhD thesis of~\cite{ChaubPhD}. Since this work is only available in french, we provide a proof inspired by the work of Chaubard, but adapted to our framework.


		 \begin{theorem}	\label{thm:derivedcategory}
			Let  be a fragment of logic equivalent to a variety of monoids ,
			 a regular language and  a positive integer.
			Then the following properties are equivalent:
			\begin{conditions}
				\item\label{derivedcategory:1}  is definable by a formula of ,

				\item\label{derivedcategory:3} there exists
				      some languages  of
				       over  such that:
				\item\label{derivedcategory:2} the category  belongs to .
			\end{conditions}
\end{theorem}
		\begin{proof}
			The equivalence between the two first points is obtained directly by Theorem~\ref{semimod}. We only
			prove the equivalence between~\eqref{derivedcategory:2} and~\eqref{derivedcategory:3}.
			As always, we denote by  the syntactic morphism of  and  its
			accepting set.
\begin{conditions}
				\item[:]
				Assume that  belongs to . By definition, it means that there exists a division of categories
				, where  is a monoid of  seen as a one object category.
				We need to define some appropriate languages  for .
				To this end,
				we construct an \emph{adequate} morphism from  to . \\
Let then  be defined by 
				where  is any element in .
			  For , let  and
				. Because
				 is in , these languages are all in .

				It remains to verify that these languages satisfy the hypothesis. This is equivalent to check that
				for all 
				
				Let  be a well-formed word of , by construction of ,
				we have
				
				Therefore, we have
				
				Furthermore, since  is a division, for all ,
				 for all  and thus .

				\item[:]
					Let  be languages of  as stated by~\eqref{derivedcategory:3}.
					Then each of them is definable by a monoid of , and since varieties are closed by product, there exists a morphism  that recognizes them all, with .
					We now prove that  divides .
					Let   be defined by
					
					The application  satisfies the first three axioms of a division of categories.
					\begin{conditions}
						\item We have  for any  of .
						\item Let  be an arrow of .
						By definition, there exists  in  such that .
						Let . By definition,  and thus
						.
						\item Let  and  be two arrows in  and
						, . By hypothesis, there exists  and
						 such that
						 and , and such that  and
						. Then,  belongs to  since
						,  and .
					\end{conditions}
					Unfortunately, it could happen that  does not satisfy the last condition.
					Without detailing, this is due to the fact that
					some elements of the syntactic congruence of  might merge when appearing at some specific congruences, leading to non empty intersection of images of arrows.
					In the following, we use the idea that for any pair of elements there exists a congruence that separates them by definition of the syntactic congruence.

					Thus, we now introduce a \emph{twisted product} of , denoted by 
					and formally define it by
					
					Because  is a product of  by it self  times, it satisfies immediately the
					first
					three axioms of a division of categories. We now prove that  is a division by proving the separation axiom.
					\begin{conditions}
						\setcounter{enumi}{3}
						\item Let  be two distinct elements of  such that
						 and  are arrows of . We first
						prove that there exists  satisfying 
						and such that 
						and then conclude by using .
						Let  and  in  such that  and .

						Since , and by definition of , the syntactic monoid of ,
						we can assume that there exists  such that 
						if and only if .
						Let  and .
						We remark that  and  are
						arrows   for .
						Without loss of generality, we assume  to be in , the other case being symmetrical.
						By hypothesis, we have the following:
						
						However
						

						Since  is recognized by  through the morphism  we have
						
						To conclude, it suffices to notice that
						
						where , ,  and . Since both  and
						 are nonempty, we conclude that .
						We proved that for every arrow  and  in , there exists
						  and such that .
						Therefore, we obtain that  
						for every coterminal arrows  and  in , which concludes the proof.

					\end{conditions}



			\end{conditions}
		\end{proof}


\subsection{local case}\label{Subsection:Local}

	For any variety , we define  to be the class of morphisms (-variety of morphisms to be precise,
	see the article of~\cite{PS05} for more details) whose stable
	monoid is in .
	Following the article of \cite{Tilson}, we denote by
	 the variety of categories whose local monoids are all in .
	A variety of monoids  is said to be \emph{local} if .
	The next theorem makes explicit the link between  and .

	\begin{theorem}\label{QVlV}
	Let  be a variety and  a regular language of  of
	stability index . The following properties are equivalent:
	\begin{conditions}
	\item\label{qv}  is recognized by a morphism in ,
	\item\label{dlv} there exists an integer  such that  is in ,
	\item\label{slv}  is in .
	\end{conditions}
	\end{theorem}
	\begin{proof}
\\
	. If  is recognized by a stamp in , then its syntactic stamp is also in 
			and its stable monoid
			is in . But, thanks to Lemma \ref{monLocaux},
			the local monoids of   belong to , and thus  is in .\\
	. Is obvious.\\
	. Suppose that  is in .
			Then the local monoids of ,
			which are isomorphic to  by Lemma \ref{monLocaux}, belong to .
			Thus , which is a submonoid of , also
			belongs to . Finally, by definition of the stability index, the monoid
			 is in  and thus
			 is in .
	\end{proof}

	Observe that any monoid of , viewed as a one-object category,
	belongs to . Therefore by definition of , any category of 
	divides a category of , and thus .
	The varieties satisfying  are exactly the  local varieties.
	Combining this with Theorem~\ref{thm:derivedcategory} and
	 since the stability index and the stable monoid of
	 a given regular language are computable, one gets the following corollary.
	 \begin{corollary}
		Let  be a fragment equivalent to a local variety . Then  is
		decidable  if and only if  is decidable. Furthermore, the fragment  is
		equivalent to .
	\end{corollary}

		Adding modular predicates does not always coincide with the  operation.
		A counterexample is the variety , which is known to be nonlocal.
		\cite{CPS06b} proved the decidability of , using the
		characterization of  given by~\cite{Kna83} (see Figure~\ref{gJ}).
		Using this characterization, we can prove that the language ,
		whose stable monoid is in  does not satisfy Knast's equation since
		
		It is therefore not definable in  (see Example~\ref{Excatder}).
\begin{figure}[H]
\centering
\begin{tikzpicture}[->,>=latex',shorten >=1pt,node distance=1.2cm,  every loop/.style={looseness=5}]
\node [state](q_0){};
\node [state](q_1) at () {};
   \path[->]
   (q_0) edge [bend left = 25]  node[fill = white] {} (q_1)
   (q_1) edge [bend left = 25] node[fill = white] {} (q_0);

\node (equa) at () {};
\end{tikzpicture}
\caption{Path equation of  by Knast. }\label{gJ}
\end{figure}


\subsection{Finite rank}\label{Subsection:FiniteRank}


Although the local property gives a nice algebraic characterisation, it only applies to a few varieties.
Nonetheless, we can still obtain a delay when the global is well-understood.
To be more precise, we now prove a delay for varieties where equations for the global are known.
As the global is a variety of categories, we first extend the framework of profinite equations to categories.
Note finally that this is the only case where we obtain a delay that is greater than the stability index.
The main applications on fragments of logic are given in Corollary~\ref{FinRank-CorDecid}.

\paragraph{Path equations}
The theory of profinite equation of varieties of monoids extends naturally to path equations on graphs,
		characterising varieties of categories.
The complexity of a variety of categories is given by its rank,
		which is the minimal size required to describe the variety in terms of path equations.
Let  be a graph and  the set of arrows of . Then 
			is the set of words on  such that for all
			,  and  are consecutive arrows.
			 is named the free category on .
Let  and  be coterminal paths of . Then
			
			where  is a category morphism and  a finite category.
			We define  which is an ultrametric distance  on .
			The completion of  for this metric is called the profinite free category on  and is denoted by .
			The following proposition is very standard in the framework of (pseudo-)varieties of monoids and categories.
\begin{proposition}
			Let  be a graph,  a finite category and  a morphism of categories.
			Then, there exists a unique continuous function  that extends .
			Furthermore, for any , there exists  such that .
		\end{proposition}

Let  be a graph and  coterminal profinite paths.
			We say that the finite category  satisfy the equation  if for any morphism
			, we have .
\begin{theorem}[Tilson]
			Every non trivial variety of finite categories is defined by a set of equations. \end{theorem}
		\begin{definition}[Rank of a variety]\label{def:rank}
			We say that a variety of monoids  has a rank   if its global is defined by a set of bounded path
			equations with at most  vertices. If  has a finite rank, we denote by
			 the minimal  such that  has a rank .
		\end{definition}
		We remark that the varieties of rank one are exactly the local ones.
		Furthermore, most of the known fragments of logic are equivalent to a variety of finite rank. The question remains however
		open in some cases, as for instance for the levels of the dot-depth hierarchy.
\begin{example}\label{FinRank-Ex}
			We now  give several varieties where equations for the global are known.
			\begin{enumerate}
				\item Several varieties are known to be local. For instance, the variety of semilattice monoids ,
				the variety ,
				the variety of aperiodic monoids .
				\item The variety of commutative monoids  . The variety of categories
				 is defined below and thus is of rank .	

\begin{tikzpicture}[->,>=latex',shorten >=1pt,node distance=0.5cm, every loop/.style={looseness=5}]
\node [state](q_0){};
\node [state](q_1) at () {};
\node  at () { };
   \path[->]
   (q_0) edge [bend left = 25]  node[fill = white] {} (q_1)
   (q_1) edge [bend left = 25] node[fill = white] {} (q_0);  
\end{tikzpicture}
 
				\item A recent algebraic description of
				the languages definable by formulas of 
				was established in~\cite{KS12,KW12}.
				 In the subsequent we will denote by  the equivalent variety of monoids.
				This result was extended to  in~\cite{KL12}. From this latter result we derive
				the following description of , giving a rank of at most .
			\end{enumerate}
							\begin{figure}[h!]
\begin{tikzpicture}[->,>=latex',shorten >=1pt,node distance=1.2cm, every loop/.style={looseness=5}]
\node [state](q_0){};
\node [state](q_1) at () {};
\node  at () {We define  and };
\node  at () {} ;
\node  at () {} ;
\node  at () { satisfies the equation } ;


\node  [state](q_2) at () { };
\node  [state](q_3) at () { };
\node  [state](q_4) at () { };
\node  [state](q_5) at () { };
\node (p2) at (){}; 
\node (p3) at (){}; 
\node (pm) at (){}; 
\node (qm) at (){};
\node (p4) at (){}; 
\node (p22) at (){}; 
\node (p33) at (){}; 


   \path[->]
   (q_0) edge [bend left = -25]  node[fill = white] {} (q_1)
   (q_1) edge [bend left = -25] node[fill = white] {} (q_0)
   (q_1) edge[bend left = -25]  node[fill = white] {} (q_3)
   (q_3) edge [bend left = -25] node[fill = white] {} (q_2)
   (q_5) edge [bend left = -25] node[fill = white] {} (q_4)
   (q_2) edge [bend left =- 25] node[fill = white] {} (q_0)
   (q_3) edge  (p3)
   (p2) edge  (q_2)
   (q_4) edge  (pm)
   (qm) edge  (q_5)
 ;




;

\end{tikzpicture}
\caption{An equation for .}\label{Fig:gVk}
\end{figure} 
		\end{example}

	\begin{theorem}[A  Delay Theorem for finite rank varieties]\label{delay-FinRank}
			Let  be a fragment equivalent to a variety  rank .
			A language  belongs to  if and only if  belongs to
			
		\end{theorem}

		\begin{proof}
	First notice that since the if condition is trivial, we only need to prove the only if implication.
	Remark now that if  then the variety is local and we know that we can restrict to congruence modulo the stability index.
	For the rest of the proof we assume that .\\
	Let now  be such that .
	Without loss of generality, we assume that  is greater than .
	Indeed if , we consider . Then by Proposition~\ref{prop:division_cat}  divides  and thus also belongs to . \\
	 So in the remainder of the proof we will assume that .
	 Since  we know that .
		Then  satisfies every path equation  defining .
	The goal of this proof is to show if  does not satisfy a path equation defining , then  cannot satisfy it either.

		So assume that there exists a path equation  of rank  defining  that is not satisfied by .
		Then, there exists a category morphism 
		such that .
		We define  the set of objects of  that have a preimage by , and
		
		 the set of arrows that have a preimage by .
		Notice that .

		We will construct a category morphism 
		such that .
		In order to do that, we define a map 
		such that for all  in ,  is an arrow of .

		\begin{lemma}
		There exists a smallest integer  such that
		 .
		\end{lemma}
		\begin{proof}
		As the size of  is , the size of  is at most .
		Then the maximal distance between two consecutive vertices of 
		is at least .
		\end{proof}

		We define  as follow :
		
		The idea behind this is that  if  appears before the gap and  if  appears after it.
		Then each arrow from  will either appear directly as it does for  if it does not go over the gap, and since the gap is of size , we will be able to pump the arrows that go over it.

		\begin{lemma}\label{lemma:theta_arrow}
		For any arrow  of ,  is an arrow of .
		\end{lemma}
		\begin{proof}
		Let  be an arrow of . Then there exists a word  such that
		 and .
		We now distinguish the cases depending on the length of .
		\begin{itemize}
		\item If , then we know, by definition of the stability index, that for any positive integer , there exists a word  such that
		,  and .
		Then as  preserves the congruence modulo ,
		
		is an arrow of .
		\item If , then we have to treat several subcases:
			\begin{itemize}
				\item If  and , then .
				Thus  is an arrow of .
				\item If  and ,
		 then as  has a size smaller than , we have  and . Consequently  and
		  		 is an arrow of .
		  		\item If  and , then
		  		.
		  		So  .
		  		The same word  labels an arrow from  to  and thus
		  		 is an arrow of .
		  		\item Finally, the case where   and  cannot happen since
		  		it implies that  and , and that
		  		 which contradicts the  hypothesis.

			\end{itemize}
		\end{itemize}
		\end{proof}

		We now define a new morphism . We proceed as follow:
		\begin{itemize2}
				\item First we define  to be .
				\item We now have to define  on arrows.
				Let  be an arrow of  and .
				We set . This is well defined thanks
				to Lemma~\ref{lemma:theta_arrow}.
		\end{itemize2}
		\begin{lemma}\label{lemma:psi}
				Let  be a path in . If , then .
		\end{lemma}
		\begin{proof}
			Let  such that  and . Therefore,
			. However, since for all 
			, we have
			 and
			
		\end{proof}

		Recall that .
		Then we can find  co-terminal paths of  such that ,
		 , 
		and .
		 We set  with  for any  and
		 with  for any .
		To conclude we show that  which is absurd. Indeed, if 
		and , then  since  separates  and . Furthermore, by Lemma~\ref{lemma:psi},
		 we also have  and  in .
		 Finally  and thus  does not satisfy , holding a contradiction.

		\end{proof}

		Combining the previous theorem with the decidable path equations given in Example~\ref{FinRank-Ex} yields the following corollaries.

		\begin{corollary}\label{FinRank-CorDecid}
		Given a regular language  of stability index  and an integer . We have the following results.
		\begin{itemize}
			\item   belongs to  if, and only if, it belongs to
		.
			\item  belongs to  if, and only if, it belongs to	 .
		\end{itemize}

		As the corresponding global varieties are decidable, we get that the fragments  and  for any  are decidable.
		\end{corollary}





\subsection{Infinitely testable case}\label{SsSection:InfTest}
	In this Section, we present the \emph{infinitely testable} property.
	We then prove that for any expressive enough fragment equipped with all regular predicates, this property holds, leading to a delay.
	In fact, Proposition~\ref{Prop:VD} proves that, given Proposition~\ref{Prop:localpredfragment}, as soon as a fragment contains the local predicates, it will be infinitely testable.
	Theorem~\ref{thm:delayinftest} then proves that a delay can be computed in this latter case.
	Informally, a variety is infinitely testable if the membership of a language
	to the variety only depends on words \emph{long enough}.
	\paragraph{Definition.}

	Given a semigroup , the \emph{idempotents' ideal} of , denoted ,
	is the ideal of  generated by its idempotents, i.e. , where  denotes the set of idempotents of .
	Note also that given a morphism
	 , it is the semigroup
	of all elements of  having an infinite number of preimages by .
	An aware reader could notice that  is the set of all elements of  that
	are -below an idempotent.
	A variety of semigroups  is said to be \emph{infinitely testable}
	if the membership of a semigroup to  is equivalent to
	the membership of its idempotents' ideal.
	Informally, a variety is infinitely testable if its membership
	can be reduced to an algebraic condition on the idempotents' ideal.
	By extension, we say that a fragment of logic is infinitely testable if
	it is characterized by an infinitely testable variety.
	\begin{example}
	The fragment  is equivalent to the
	aperiodic and commutative variety . This fragment is also described by the equations
	 and . This fragment is not infinitely
	testable. For instance the language equal to the singleton  has a trivial idempotents' ideal while it is
	not
	definable in .
	\end{example}


\begin{example}\label{Prop:FO1-InfTest}
	The fragment  is equivalent to the languages whose syntactic semigroup belongs to
	the variety: ~\citep[Theorem VI.3.1]{Straubing94}.
	This fragment is also described by the profinite equation
	
	We now show that it is an infinitely testable fragment.
	Let  be a regular language and  its syntactic semigroup.
	We simply prove that if the equation~\eqref{acomli} is not satisfied by , then it is not satisfied by
	.
	Suppose that there exists  such that the  equation~\eqref{acomli} is not satisfied.
	Then by setting:
		
	All new variables belong to  and they also fail to satisfy~\eqref{acomli}.
\end{example}

	In fact, the approach given in the last example can be generalised to any variety of the form .
	This is proved by Proposition~\ref{Prop:VD}.



\begin{proposition}\label{Prop:VD}
	 Let  be a variety.
		The variety  is infinitely testable.
\end{proposition}
	\begin{proof}
		Let  be a regular language with  its syntactic morphism and 
		its syntactic semigroup.
		Using Theorem~\ref{thm:delaydefinite}, we have that  belongs to  if and only if
		 belongs to . To conclude, we just notice that by definition, , and therefore
		 is infinitely testable.
	\end{proof}











We finally prove here that if a fragment is equivalent to a variety whose global is infinitely testable, then we can effectively compute a delay, which furthermore is independent from the fragment.
For varieties of the form , this also gives the decidability thanks to Proposition~\ref{Prop:VD} and
Corollary~\ref{Cor:transfer}.

\begin{theorem}\label{thm:delayinftest}
	Let  be a fragment equivalent to a variety  which is not a group variety and
	let  be a language of stability index . Then  belongs to  if and only if
	 belongs to .
\end{theorem}
\begin{proof}
	First by Theorem~\ref{semimod}, a language  belongs to  if and only if there exists
	,  in  such that
	
	Because  is a fragment which is a variety of monoids but not a group variety, the
	language  and  belongs to .
	 We recall that  for any .
	Thus, for ,   belongs to  and we have the equality
	
	Therefore,  belongs to  if and only if there exists  such that
	 belongs to .
		Thus, it suffices to prove that
		if  is definable in , then
		 is in  as well.
		We set  and   the syntactic morphisms
		of  and  respectively.

		\noindent\textbf{Claim.} The semigroup  divides .\\

		Before proving this claim, let us remark that since a variety of semigroups is closed by division,
		this claim ends the proof. Since
		if  belongs to  then  belongs to  and therefore
		 belongs to  as well.
		By division,  belongs to , and thanks to Proposition~\ref{Prop:VD},  belongs to . Finally, we deduce that 
		belongs to .

		We now aim to construct a division from  to .
		This is done through the enriched alphabet.
		We introduce the following projection
		
		and  the language of \emph{well-formed factors},
		 which is the set of well-formed words that do not necessarily start by a letter .
Note that
		.
		Let us remark also that the image of a word not in  (resp. )  by  (resp. )
		has
		an absorbing zero as image by  (resp. ).
		This zero being idempotent, it belongs to  (resp. ).
		Finally, if two words of  have the same image by , then
		they have the same length modulo  and their first (and consequently last) letters have the same enrichment.

		Consider then  a non-zero element of .
		We show that
		
		Since  belongs to , there exists a word 
of  of length greater than  in the preimage of .
		And since  by definition of the stability index,
		for any  there exists a word  of  of length greater than  such that
		 and , since
		.
		Then for  sufficiently large, there exists a word  in ,
		such that  belongs to .
		Note that by taking  as a multiple of , we obtain a word  such that
		.
		Thus for each element , we can choose such an element, that we denote .
		This justifies the definition of the following function:
		
		We conclude by proving that  is an injective morphism, and thus
		 is a subsemigroup of .
	\begin{bulitem}
		\item[The application  is a morphism.] Let . We show that
		.
		First, we can assume without loss of generality that  and .
		We remark that since ,
		the concatenated word  is well-formed if, and only if,
		 is well-formed too.
		If .Then,  have
		a well-formed preimage
		and  is well-formed.
		Then as  and  are syntactically equivalent with respect to
		both  and ,
		 meaning that .

		Now if , then either  has no well-formed preimage or
		 is a zero for .
		In the latter case, then  according to the previous point.
		If  has no well-formed preimage, then  is not well-formed
		and consequently .
		\item[The application  is injective.]
		Let  be such that . Without loss of generality, we  assume that .
		Necessarily, there exist  such that  if, and only if, .
		Let  and  be words from the preimage of  and  respectively.
		Then there exists two words  and  such that
		 if, and only if, . Therefore, we have  and  is injective.
	\end{bulitem}
	\end{proof}
\\

The following proposition deals with fragments which are not varieties of groups. Varieties of groups are notoriously ill behaving with respect to their global. Indeed ~\cite{Auinger10} exhibited a variety of group  such that
 is undecidable (as a variety of \emph{semigroupoids}). However, for a local variety  which is not a variety of groups, the
variety of semigroups  is local, as proved in~\cite{PapermanPhd}. Since this article does not deal with the framework
of varieties of \emph{semigroupoids}, we provide a self contain proof extracted from this latter result.

\begin{proposition}\label{prop:QLV}
		Let  be a
		 fragment equivalent to a local variety  which is not a variety of groups.
		Then 
		is equivalent to .
\end{proposition}
\begin{proof}
First we remark that since  is equivalent to a local variety, by
Proposition~\ref{Prop:localpredfragment}, and by definition of locality,
  is equivalent to . Furthermore, since  is not a variety of groups,
  belongs to . Therefore, we obtain the following:
 

	\noindent{\textbf{Claim:}}  belongs to  if and only if
	 belongs to .

	We now prove both implications of the claim. In the following  will be the syntactic semigroup
	of  and  the one of .
	\begin{itemize}
	\item Assume that  belongs to .
	 Let . We remark that 
	is a semigroup. Therefore, the set  is a subsemigroup of . Since  belongs to ,
	the semigroup  belongs to  as well.
	 Remark now that  is a quotient of the product of  and the syntactic semigroup of
	. Since the image of  in the syntactic monoid of  is the stable semigroup of  and
	 the image of  in the syntactic
	semigroup of  is trivial, we can conclude as  is isomorphic to the
	stable semigroup of .

		\item Assume that  belongs to , and we denote by  its stable semigroup.
		By hypothesis,
		  is in . One can remark that since  is not a variety of groups, it contains the semigroup
		  (equipped with the integer multiplication).
		 Therefore, the semigroup , obtained by adding an
		absorbing element, also belongs to . Indeed, it divides .

		 We now have to show that  is in  as well.
		Let  be an idempotent of . First, if  is the zero of , then
		. Otherwise,  is the image of a well-formed factor  that starts by a letter of
		the form  and ends by a letter of the form  with .
		We denote by		 the image of  by the syntactic morphism of . This element
		is idempotent and, therefore, belongs to .
		We conclude by noting that
		the local monoid   is a quotient of .
	\end{itemize}
\end{proof}


\section{Conclusion}\label{Section:Concl}
In this paper, we studied the definability problem for fragments of logic enriched with the modular predicates.
We presented a generic approach that gives the decidability of this problem in many cases, while the main applications are to the alternation hierarchies of the first order logic and its two variables counterpart.

The global approach is divided in two steps.
The first one relies entirely on logic.
We prove that adding a finite set of modular predicates preserves the decidability, given that the fragment is expressive enough.

The second part, which we call the delay problem, consists in deciding which finite set of modular predicates should be added to express a given regular language.
This is the most intricate part of the paper.
While unable to solve this question for any given fragment, we were able to reduce, following some known results, this question to a decidability question on the global of a fragment, a variety of categories.
Then decidability was obtained for many fragments, using different approaches.
They can be sorted in two cases.
The first case is when the global is understood and finitely describable.
Then we are able to decide a delay depending on the stability index and the said description.
The second case is when the fragment is expressive enough to handle the modular predicates.
This happens in particular if the fragment contains the local predicates and can use them extensively.

The main applications of these results are given in Figure~\ref{TableauFinal}, mainly on the levels of the quantifier alternation hierarchies, although this approach can be used on other fragments that satisfy the same hypotheses, such as the fragment .

An interesting fact is that while the stability index often serves as a valid delay, this is still open whether this would hold for varieties of rank greater than two.

The question of solving the adding of modular predicate in a general setting seems achievable, although the more natural question would be to solve the decidability of the semidirect product by .
While we avoided this characterisation as it served no purpose in our approach, an aware reader could have noticed that Theorem~\ref{thm:derivedcategory}
proves that adding modular predicates is algebraically equivalent to a semidirect product by the length-multiplying variety of morphisms .
Then our question reduces to whether this semidirect product preserves decidability.
\cite{Auinger10} proved that the semidirect product in general does not preserve decidability, but the problem is still open for the case of .

\bibliographystyle{abbrvnat}
\bibliography{biblio}

\end{document}
