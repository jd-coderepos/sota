

In this section, notions of similarity and bisimilarity for $\LOP$ are introduced, in
the spirit of Abramsky's work on applicative bisimulation~\cite{Abramsky-90}. Definitionally, this
consists in seeing $\LOP$'s operational semantics as a labelled Markov chain, then giving
the Larsen and Skou's notion of (bi)simulation for it. States will be terms, while labels
will be of two kinds: one can either \emph{evaluate} a term, obtaining (a distribution of) values, 
or \emph{apply} a term to a value.

The resulting bisimulation (probabilistic applicative bisimulation) 
will be shown to be a congruence, thus included in probabilistic context equivalence.
This will be done by a non-trivial generalization of Howe's technique~\cite{Howe-96}, which is
a well-known methodology to get congruence results in presence of higher-order functions, but
which has not been applied to probabilistic calculi so far. 

Formalizing probabilistic applicative bisimulation requires some care.
As usual, two values $\abstr{\varone}{\termone}$ and $\abstr{\varone}{\termtwo}$ are defined
to be bisimilar if for every $\termthree$, $\subst{\termone}{\varone}{\termthree}$
and $\subst{\termtwo}{\varone}{\termthree}$ are themselves bisimilar. But how if
we rather want to compare two arbitrary closed terms $\termone$ and $\termtwo$?
The simplest solution consists in following Larsen and Skou and stipulate that
every equivalence class of $\val$ modulo bisimulation is attributed the same
measure by both $\sem{\termone}$ and $\sem{\termtwo}$. Values
are thus treated in two different ways (they are both terms and values), and this is the reason why each of
them corresponds to \emph{two} states in the underlying Markov chain.


\begin{definition}
\label{d:multisort}
$\LOP$ can be seen as a multisorted labelled Markov chain
$(\LOPp{\emptyset}\uplus\val,\LOPp{\emptyset}\uplus\{\evlabel\},\translop)$
that we denote with $\cbn{\LOP}$. Labels are either closed terms, which model
parameter passing, or $\evlabel$, that models evaluation. 
Please observe that the states of the labelled Markov chain we have 
just defined are elements of the disjoint union $\LOPp{\emptyset}\uplus\val$. 
Two distinct states correspond to the
same value $\valone$, and to avoid ambiguities, we call the second one
(i.e. the one coming from $\val$) a \emph{distinguished value}.  When we
want to insist on the fact that a value $\abstr{\varone}{\termone}$ is
distinguished, we indicate it with $\clabstr{\varone}{\termone}$. We
define the transition probability matrix $\translop$ as follows:
\begin{varitemize}
\item 
  For every term $\termone$ and for every 
  distinguished value $\clabstr{\varone}{\termtwo}$,
  $$
  \translop(\termone, \evlabel,\clabstr{\varone}{\termtwo})\defi\sem{\termone}(\clabstr{\varone}{\termtwo});
  $$
\item 
  For every term $\termone$ and for every distinguished value $\clabstr{\varone}{\termtwo}$,
  $$
  \translop(\clabstr{\varone}{\termtwo}, \termone,
  \subst{\termtwo}{\varone}{\termone})\defi 1;
  $$
\item
  In all other cases, $\translop$ returns $0$.
\end{varitemize}
\end{definition}
Terms seen as states  only interact with the
environment by performing $\evlabel$, while distinguished values 
only take other closed terms as parameters. 

Simulation and bisimulation relations can be defined for $\cbn{\LOP}$ as for any 
labelled Markov chain. Even if, strictly speaking, these are binary relations on 
$\LOPp{\emptyset}\uplus\val$, we often see them just as their restrictions to 
$\LOPp{\emptyset}$. 
Formally, a \emph{probabilistic applicative bisimulation} (a
$\pabn$) is simply a probabilistic bisimulation on $\cbn{\LOP}$. This way one can 
define \emph{probabilistic applicative bisimilarity}, which is denoted $\cbnpab$.
Similarly for \emph{probabilistic applicative simulation} ($\pasn$) and \emph{probabilistic
applicative similarity}, denoted $\cbnpas$.

\begin{remark}[Early vs. Late]\label{r:el} 
  Technically, the distinction between terms and values in
  Definition~\ref{d:multisort} means that our
  bisimulation is in \emph{late} style. In bisimulations for
  value-passing concurrent languages, ``late'' indicates the explicit
  manipulation of functions in the clause for input actions: functions are
  chosen first, and only later, the input value received is taken into
  account~\cite{SaWabook}. Late-style is used in contraposition to 
  \emph{early} style, where the order of quantifiers is exchanged, so that the
  choice of functions may depend on the specific input value received.
  In our setting, adopting an early style would mean having transitions such as
  $\lambda x. M \arr{N} M \sub N x$, and then setting up a probabilistic
  bisimulation on top of the resulting transition system. 
  We leave for future work a study of the comparison between the two
  styles. In this paper, we stick to the late style because easier to deal with,
  especially under Howe's technique.
  Previous works on applicative
  bisimulation for nondeterministic functions also focus on the late
  approach~\cite{Ong93,PittsSurvey}.
\end{remark} 

\begin{remark}
\label{r:pabfirenze} 
Defining applicative bisimulation in terms of multisorted labelled Markov
chains has the advantage of recasting the definition in a familiar
framework; most importantly, this formulation will be useful when dealing
with Howe's method. To spell out the explicit operational details of the
definition, a probabilistic applicative bisimulation can be seen as an equivalence
relation $\relone\subseteq \LOPp{\emptyset}\times\LOPp{\emptyset}$ such
that whenever $\termone\mathrel{\relone}\termtwo$:
  \begin{varenumerate}
  \item $\sem{\termone}(\econe \cap
    \val)=\sem{\termtwo}(\econe \cap \val)$, for any
    equivalence class $\econe$ of $\relone$ 
(that
    is, the probability of reaching a value in $\econe$ is the same for the
    two terms);
  \item
    \label{cl:ct} if $\termone$ and $\termtwo$ are values, say $\lambda
    x. P $ and $ \lambda x. Q$, then $P \sub L x \RR Q \sub L x$, for
    all $L \in \LOPp{\emptyset}$.
  \end{varenumerate}
  The special treatment of values, in Clause~\ref{cl:ct}., motivates the
  use of \emph{multisorted} labelled Markov chains in Definition~\ref{d:multisort}.
\end{remark}

As usual, one way to show that any two terms are bisimilar is to prove that one relation
containing the pair in question is a $\pabn$. 
Terms with the same semantics are indistinguishable:
\begin{lemma}\label{lemma:samesem}
  The binary relation $\relone=\{(\termone,\termtwo)\in
  \LOPp{\emptyset}\times\LOPp{\emptyset}\st
  \sem{\termone}=\sem{\termtwo}\}\,\biguplus\,\{(\valone,\valone)\in
  \val\times\val\}$ is a $\pabn$.
\end{lemma}
\begin{proof}
  The fact $\relone$ is an equivalence easily follows from reflexivity,
  symmetry and transitivity of set-theoretic equality. $\relone$ must
  satisfy the following property for closed terms: if
  $\termone\relone\termtwo$, then for every
  $\econe\in\quot{\val}{\relone}$,
  $\translop(\termone,\evlabel,\econe)=\translop(\termtwo,\evlabel,\econe)$. Notice
  that if $\sem{\termone}=\sem{\termtwo}$, then clearly
  $\translop(\termone,\evlabel,\valone)=\translop(\termtwo,\evlabel,\valone)$,
  for every $\valone\in\val$. With the same hypothesis,
  \begin{align*}
    \translop(\termone,\evlabel,\econe)&=\sum_{\valone\in\econe}\translop(\termone,\evlabel,\valone)\\
    &=
    \sum_{\valone\in\econe}\translop(\termtwo,\evlabel,\valone)=\translop(\termtwo,\evlabel,\econe).
  \end{align*}
  Moreover, $\relone$ must satisfy the following property for cloned
  values: if
  $\clabstr{\varone}{\termone}\relone\clabstr{\varone}{\termtwo}$, then for
  every close term $\termthree$ and for every
  $\econe\in\quot{\LOPp{\emptyset}}{\relone}$,
  $\translop(\clabstr{\varone}{\termone},\termthree,\econe)=\translop(\clabstr{\varone}{\termtwo},\termthree,\econe)$. Now,
  the hypothesis
  $\sem{\clabstr{\varone}{\termone}}=\sem{\clabstr{\varone}{\termtwo}}$
  implies $\termone = \termtwo$. Then clearly
  $\translop(\clabstr{\varone}{\termone},\termthree,\termfour)=\translop(\clabstr{\varone}{\termtwo},\termthree,\termfour)$
  for every $\termfour\in\LOPp{\emptyset}$. With the same hypothesis,
  \begin{align*}
    \translop(\clabstr{\varone}{\termone},\termthree,\econe)&=\sum_{\termfour\in\econe}\translop(\clabstr{\varone}{\termone},\termthree,\termfour)\\
    &=
    \sum_{\termfour\in\econe}\translop(\clabstr{\varone}{\termtwo},\termthree,\termfour)=\translop(\clabstr{\varone}{\termtwo},\termthree,\econe).
  \end{align*}
  This concludes the proof.
\end{proof}
Please notice that the previous result yield a nice consequence: for every
$\termone,\,\termtwo\in\LOPp{\emptyset}$,
$(\abstr{\varone}{\termone})\termtwo\cbnpab\subst{\termone}{\varone}{\termtwo}$. Indeed,
Lemma~\ref{lemma:semsumCBN} tells us that the latter terms have the same
semantics.

Conversely, knowing that two terms $\termone$ and
$\termtwo$ are (bi)similar means knowing quite a lot about their
convergence probability:
\begin{lemma}[Adequacy of Bisimulation]\label{lemma:sumsempabCBN}
  If $\termone\cbnpab\termtwo$, then $\sumsem{\termone}=\sumsem{\termtwo}$. Moreover,
  if $\termone\cbnpas\termtwo$, then $\sumsem{\termone}\leq\sumsem{\termtwo}$.
\end{lemma}
\begin{proof}
  \begin{align*}
    \sumsem{\termone}&=\sum_{\econe\in\quot{\val}{\cbnpab}}\translop(\termone,\evlabel,\econe)\\
    &=
    \sum_{\econe\in\quot{\val}{\cbnpab}}\translop(\termtwo,\evlabel,\econe)=\sumsem{\termtwo}.
  \end{align*} 
  And, 
  \begin{align*}
    \sumsem{\termone}&=\translop(\termone,\evlabel,\val)\\
    &\leq\translop(\termtwo,\evlabel,\cbnpas(\val))\\
    &=\translop(\termtwo,\evlabel,\val)=\sumsem{\termtwo}.
  \end{align*}
  This concludes the proof.
\end{proof}
\begin{example}
  Bisimilar terms do not necessarily have the same semantics. After all, this is one
  reason for using bisimulation, and its proof method, as basis to prove
  fine-grained equalities among functions. Let us consider the
  following terms:
  \begin{align*}
        \termone\defi&\;((\lambda x. (x \oplus x))\oplus(\lambda x.x))\oplus\Omega;\\
        \termtwo\defi&\;\Omega\oplus(\lambda x. I x) ;
  \end{align*}                         
  Their semantics                      
  differ, as for every value $\valone$, we have:
  \[
   \begin{array}{rcl}
  \sem{\termone}(\valone) &=& \left\{
    \begin{array}{ll}
      \frac{1}{4} &\mbox{if }\valone \mbox{ is } \lambda x. (x \oplus x) \mbox{ or } \lambda x.x;\\
      0 & \mbox{otherwise};
    \end{array}
    \right.
  \\[10pt]
  \sem{\termtwo}(\valone)& =&  \left\{
    \begin{array}{ll}
      \frac{1}{2} &\mbox{if }\valone \mbox{ is } \lambda x. I x;\\
      0 & \mbox{otherwise}.
    \end{array}
    \right.
    \end{array}
   \]
   Nonetheless,   we can prove
   $\termone\cbnpab\termtwo$. Indeed, $\nu x. (x \oplus x)\cbnpab\nu
   x.x\cbnpab\nu x.I x$ because, for every $L\in\LOPp{\emptyset}$, the three
   terms $L$, $\ps{L}{L}$ and $\app{I}{L}$ all have the same semantics,
   i.e., $\sem{L}$. Now, consider any equivalence class $\econe$ of
   distinguished values modulo $\cbnpab$. If $\econe$ includes the three
   distinguished values above, then
   \[\translop(\termone,
   \evlabel,\econe)=\sum_{\valone\in\econe}\sem{\termone}(\valone) =
   \frac{1}{2} = \sum_{\valone\in\econe}\sem{\termtwo}(\valone) =
   \translop(\termtwo,\evlabel,\econe).
   \]
   Otherwise, $\translop(\termone,
   \evlabel,\econe)= 0 = \translop(\termtwo,\evlabel,\econe)$.
\end{example}
Let us prove the following technical result that, moreover, stipulate that
bisimilar distinguished values are bisimilar values.
\begin{lemma}\label{lemma:lambdaredCBN}
  $\abstr{\varone}{\termone}\cbnpab\abstr{\varone}{\termtwo}$ iff
  $\clabstr{\varone}{\termone}\cbnpab\clabstr{\varone}{\termtwo}$ iff
  $\subst{\termone}{\varone}{\termthree}\cbnpab
  \subst{\termtwo}{\varone}{\termthree}$, for all
  $\termthree\in\LOPp{\emptyset}$.
\end{lemma}
\begin{proof}
  The first double implication is obvious. For that matter, distinguished values are value
  terms. Let us now detail the second double implication.
$(\Rightarrow)$ The fact that $\cbnpab$ is a $\pabn$ implies, by its
  definition, that for every $\termthree\in\LOPp{\emptyset}$ and every
  $\econe\in\quot{\LOPp{\emptyset}}{\cbnpab}$,
  $\translop(\clabstr{\varone}{\termone},\termthree,\econe) =
  \translop(\clabstr{\varone}{\termtwo},\termthree,\econe)$. Suppose then,
  by contradiction, that
  $\subst{\termone}{\varone}{\termthree}\cbn{\not\pab}
  \subst{\termtwo}{\varone}{\termthree}$, for some
  $\termthree\in\LOPp{\emptyset}$. The latter means that, there exists
  $\ectwo\in\quot{\LOPp{\emptyset}}{\cbnpab}$ such that
  $\subst{\termone}{\varone}{\termthree}\in\ectwo$ and
  $\subst{\termtwo}{\varone}{\termthree}\not\in\ectwo$. According to its
  definition, for all $\termfour\in\LOPp{\emptyset}$,
  $\translop(\clabstr{\varone}{\termone},\termthree,\termfour) = 1$ iff
  $\termfour\equiv\subst{\termone}{\varone}{\termthree}$, and
  $\translop(\clabstr{\varone}{\termone},\termthree,\termfour) = 0$
  otherwise. Then, since $\subst{\termone}{\varone}{\termthree}\in\ectwo$,
  we derive $\translop(\clabstr{\varone}{\termone},\termthree,\ectwo) =
  \sum_{\termfour\in\ectwo}\translop(\abstr{\varone}{\termone},\termthree,\termfour)
  \geq
  \translop(\clabstr{\varone}{\termone},\termthree,\subst{\termone}{\varone}{\termthree})
  = 1$, which implies
  $\sum_{\termfour\in\ectwo}\translop(\clabstr{\varone}{\termone},\termthree,\termfour)
  = \translop(\clabstr{\varone}{\termone},\termthree,\ectwo) = 1$. Although
  $\clabstr{\varone}{\termtwo}$ is a distinguished value and the starting
  reasoning we have just made above still holds,
  $\translop(\clabstr{\varone}{\termtwo},\termthree,\ectwo) =
  \sum_{\termfour\in\ectwo}\translop(\clabstr{\varone}{\termtwo},\termthree,\termfour)
  = 0$. We get the latter because there is no $\termfour\in\ectwo$ of the
  form $\subst{\termtwo}{\varone}{\termthree}$ due to the hypothesis that
  $\subst{\termtwo}{\varone}{\termthree}\not\in\ectwo$.

  From the hypothesis on the equivalence class $\ectwo$, i.e.
  $\translop(\clabstr{\varone}{\termone},\termthree,\ectwo) =
  \translop(\clabstr{\varone}{\termtwo},\termthree,\ectwo)$, we derive the
  absurd:
  \begin{align*}
    1 = \translop(\clabstr{\varone}{\termone},\termthree,\ectwo) =
    \translop(\clabstr{\varone}{\termtwo},\termthree,\ectwo) = 0.
  \end{align*}
$(\Leftarrow)$ We need to prove that, for every
  $\termthree\in\LOPp{\emptyset}$ and every
  $\econe\in\quot{\LOPp{\emptyset}}{\cbnpab}$,
  $\translop(\clabstr{\varone}{\termone},\termthree,\econe) =
  \translop(\clabstr{\varone}{\termtwo},\termthree,\econe)$ supposing that
  $\subst{\termone}{\varone}{\termthree}\cbnpab
  \subst{\termtwo}{\varone}{\termthree}$ holds. First of all, let us
  rewrite $\translop(\clabstr{\varone}{\termone},\termthree,\econe)$ and
  $\translop(\clabstr{\varone}{\termtwo},\termthree,\econe)$ as
  $\sum_{\termfour\in\econe}\translop(\clabstr{\varone}{\termone},\termthree,\termfour)$
  and
  $\sum_{\termfour\in\econe}\translop(\clabstr{\varone}{\termtwo},\termthree,\termfour)$
  respectively. Then, from the hypothesis and the same reasoning we have
  made for ($\Rightarrow$), for every $\econe\in\quot{\LOPp{\emptyset}}{\cbnpab}$:
  $$
  \sum_{\termfour\in\econe}\translop(\clabstr{\varone}{\termone},\termthree,\termfour)
  = \left\{
    \begin{array}{ll}
      1 &\mbox{if }\subst{\termone}{\varone}{\termthree}\in\econe;\\
      0 & \mbox{otherwise}
    \end{array}
  \right.  = \left\{
    \begin{array}{ll}
      1 &\mbox{if }\subst{\termtwo}{\varone}{\termthree}\in\econe;\\
      0 & \mbox{otherwise}
    \end{array}
  \right.  =
  \sum_{\termfour\in\econe}\translop(\clabstr{\varone}{\termtwo},\termthree,\termfour)
  $$
  which proves the thesis.
\end{proof}
The same result holds for $\cbnpas$.
\subsection{Probabilistic Applicative Bisimulation is a Congruence}
In this section, we prove that probabilistic applicative bisimulation is indeed
a congruence, and that its non-symmetric sibling is a precongruence. 
The overall structure of the proof is similar to the one by
Howe~\cite{Howe-96}. The main idea consists in defining a way to turn
an arbitrary relation $\relone$ on (possibly open) terms to another one, $\howe{\relone}$,
in such a way that, if $\relone$ satisfies a few simple conditions, then $\howe{\relone}$ is 
a (pre)congruence including $\relone$. The key step, then, is to prove
that $\howe{\relone}$ is indeed a (bi)simulation. In view of Proposition~\ref{prop:pab=pascopas}, 
considering similarity suffices here.

It is here convenient to work with generalizations of relations called
\emph{$\LOP$-relations}, i.e.  sets of triples in the form
$(\vecvarone,\termone,\termtwo)$, where
$\termone,\termtwo\in\LOP(\vecvarone)$.  Thus if a relation has the pair
$(M,N)$ with $\termone,\termtwo\in\LOP(\vecvarone)$, then the corresponding
$\LOP$-relation will include $(\vecvarone,\termone,\termtwo)$.  (Recall
that applicative (bi)similarity is extended to open terms by considering
all closing substitutions.) 
Given any
$\LOP$-relation $\relone$,  we write $\rel{\vecvarone}{\termone}{\relone}{\termtwo}$
if $(\vecvarone,\termone,\termtwo)\in\relone$. A
$\LOP$-relation $\relone$ is said to be \emph{compatible} iff the four
conditions below hold:
\begin{varitemize}
\item[\Comone] $\forall\vecvarone\in\powfin{\setvar}$, $\varone\in\vecvarone$:
  $\rel{\vecvarone}{\varone}{\relone}{\varone}$,
\item[\Comtwo]
  $\forall\vecvarone\in\powfin{\setvar}$,$\forall\varone\in\setvar-\vecvarone$,$\forall\termone,
  \termtwo\in\LOP(\vecvarone\cup\{\varone\})$:
  $\rel{\vecvarone\cup\{\varone\}}{\termone}{\relone}{\termtwo}\Rightarrow
  \rel{\vecvarone}{\abstr{\varone}{\termone}}{\relone}{\abstr{\varone}{\termtwo}}$,
\item[\Comthree]
  $\forall\vecvarone\in\powfin{\setvar}$,$\forall\termone,\termtwo,\termthree,\termfour\in\LOP(\vecvarone)$:
  $\rel{\vecvarone}{\termone}{\relone}{\termtwo}\wedge\rel{\vecvarone}{\termthree}{\relone}{\termfour}\Rightarrow
  \rel{\vecvarone}{\app{\termone}{\termthree}}{\relone}{\app{\termtwo}{\termfour}}$,
\item[\Comfour]
  $\forall\vecvarone\in\powfin{\setvar}$,$\forall\termone,\termtwo,\termthree,\termfour\in\LOP(\vecvarone)$:
  $\rel{\vecvarone}{\termone}{\relone}{\termtwo}\wedge\rel{\vecvarone}{\termthree}{\relone}{\termfour}\Rightarrow
  \rel{\vecvarone}{\ps{\termone}{\termthree}}{\relone}{\ps{\termtwo}{\termfour}}$.
\end{varitemize}
We will often use the following technical results to establish
$\Comthree$ and $\Comfour$ under particular hypothesis.
\begin{lemma}\label{lemma:com3LR}
  Let us consider the properties
  \begin{varitemize}
  \item[\ComthreeL]
    $\forall\vecvarone\in\powfin{\setvar}$,$\forall\termone,\termtwo,\termthree\in\LOP(\vecvarone)$:
    $\rel{\vecvarone}{\termone}{\relone}{\termtwo}\Rightarrow
    \rel{\vecvarone}{\app{\termone}{\termthree}}{\relone}{\app{\termtwo}{\termthree}}$,
  \item[\ComthreeR]
    $\forall\vecvarone\in\powfin{\setvar}$,$\forall\termone,\termtwo,\termthree\in\LOP(\vecvarone)$:
    $\rel{\vecvarone}{\termone}{\relone}{\termtwo}\Rightarrow
    \rel{\vecvarone}{\app{\termthree}{\termone}}{\relone}{\app{\termthree}{\termtwo}}$.
  \end{varitemize}
  If $\relone$ is transitive, then $\ComthreeL$ and $\ComthreeR$ together imply $\Comthree$.
\end{lemma}
\begin{proof}
  Proving $\Comthree$ means to show that the hypothesis
  $\rel{\vecvarone}{\termone}{\relone}{\termtwo}$ and
  $\rel{\vecvarone}{\termthree}{\relone}{\termfour}$ imply
  $\rel{\vecvarone}{\app{\termone}{\termthree}}{\relone}{\app{\termtwo}{\termfour}}$. Using
  $\ComthreeL$ on the first one, with $\termthree$ as steady term, it
  follows
  $\rel{\vecvarone}{\app{\termone}{\termthree}}{\relone}{\app{\termtwo}{\termthree}}$. Similarly,
  using $\ComthreeR$ on the second one, with $\termtwo$ as steady term, it
  follows
  $\rel{\vecvarone}{\app{\termtwo}{\termthree}}{\relone}{\app{\termtwo}{\termfour}}$. Then,
  we conclude by transitivity property of $\relone$.
\end{proof}

\begin{lemma}\label{lemma:com4LR}
  Let us consider the properties
  \begin{varitemize}
  \item[\ComfourL]
    $\forall\vecvarone\in\powfin{\setvar}$,$\forall\termone,\termtwo,\termthree\in\LOP(\vecvarone)$:
    $\rel{\vecvarone}{\termone}{\relone}{\termtwo}\Rightarrow
    \rel{\vecvarone}{\ps{\termone}{\termthree}}{\relone}{\ps{\termtwo}{\termthree}}$,
  \item[\ComfourR]
    $\forall\vecvarone\in\powfin{\setvar}$,$\forall\termone,\termtwo,\termthree\in\LOP(\vecvarone)$:
    $\rel{\vecvarone}{\termone}{\relone}{\termtwo}\Rightarrow
    \rel{\vecvarone}{\ps{\termthree}{\termone}}{\relone}{\ps{\termthree}{\termtwo}}$.
  \end{varitemize}
  If $\relone$ is transitive, then $\ComfourL$ and $\ComfourR$ together imply $\Comfour$.
\end{lemma}
\begin{proof}
  Proving $\Comfour$ means to show that the hypothesis
  $\rel{\vecvarone}{\termone}{\relone}{\termtwo}$ and
  $\rel{\vecvarone}{\termthree}{\relone}{\termfour}$ imply
  $\rel{\vecvarone}{\ps{\termone}{\termthree}}{\relone}{\ps{\termtwo}{\termfour}}$. Using
  $\ComfourL$ on the first one, with $\termthree$ as steady term, it
  follows
  $\rel{\vecvarone}{\ps{\termone}{\termthree}}{\relone}{\ps{\termtwo}{\termthree}}$. Similarly,
  using $\ComfourR$ on the second one, with $\termtwo$ as steady term, it
  follows
  $\rel{\vecvarone}{\ps{\termtwo}{\termthree}}{\relone}{\ps{\termtwo}{\termfour}}$. Then,
  we conclude by transitivity property of $\relone$.
\end{proof}
The notions of an equivalence relation and of a preorder can be straightforwardly generalized to $\LOP$-relations, and
any compatible $\LOP$-relation that is an equivalence relation (respectively, a preorder) is said to be a \emph{congruence} 
(respectively, a \emph{precongruence}).

If bisimilarity is a congruence, then $\ctxone[\termone]$ is bisimilar to $\ctxone[\termtwo]$ whenever
$\termone\cbnpab\termtwo$ and $\ctxone$ is a context. In other words, terms can be replaced by equivalent ones in any context.
This is a crucial sanity-check any notion of equivalence  is expected to pass.

It is well-known that proving bisimulation to be a congruence may be
nontrivial when the underlying language contains higher-order
functions. This is also the case here. Proving $\Comone,\Comtwo$ and
$\Comfour$ just by inspecting the operational semantics of the involved
terms is indeed possible, but the method fails for $\Comthree$, when the
involved contexts contain applications.  In particular, proving $\Comthree$ requires
probabilistic applicative bisimilarity of being stable with respect to
substitution of bisimilar terms, hence not necessarily the
same. In general, a $\LOP$-relation $\relone$ is called (term) \emph{substitutive} if
for all $\vecvarone\in\powfin{\setvar}$, $\varone\in\setvar-\vecvarone$,
$\termone,\termtwo\in\LOPp{\vecvarone\cup\{\varone\}}$ and
$\termthree,\termfour\in\LOPp{\vecvarone}$
\begin{equation}
  \label{eq:SubCBN}
  \rel{\vecvarone\cup\{\varone\}}{\termone}{\relone}{\termtwo}\wedge
  \rel{\vecvarone}{\termthree}{\relone}{\termfour}\Rightarrow
  \rel{\vecvarone}{\subst{\termone}{\varone}{\termthree}}{\relone}{\subst{\termtwo}{\varone}{\termfour}}.
\end{equation}
Note that if $\relone$ is also reflexive, then this implies
\begin{equation}
  \label{eq:CusCBN}
  \rel{\vecvarone\cup\{\varone\}}{\termone}{\relone}{\termtwo}\wedge\termthree\in\LOPp{\vecvarone}\Rightarrow
  \rel{\vecvarone}{\subst{\termone}{\varone}{\termthree}}{\relone}{\subst{\termtwo}{\varone}{\termthree}}.
\end{equation}
We say that $\relone$ is \textit{closed under term-substitution} if it
satisfies (\ref{eq:CusCBN}). Because of the way the open extension of
$\cbnpab$ and $\cbnpas$ are defined, they are closed under
term-substitution.

Unfortunately, directly prove $\cbnpas$ to enjoy such \emph{substitutivity}
property is hard. We will thus proceed indirectly by defining, starting
from $\cbnpas$, a new relation $\howe{\cbnpas}$, called the \emph{Howe's
  lifting} of $\cbnpas$, that has such property by construction and that
can be proved equal to $\cbnpas$.  

Actually, the Howe's lifting of any $\LOP$-relation $\relone$ is the relation $\howe{\relone}$ defined by the
rules in Figure~\ref{fig:howelifting}.
\begin{figure*}
  $$
  \infer[\Howeone]
  {\rel{\vecvarone}{\varone}{\howe{\relone}}{\termone}}
  {\rel{\vecvarone}{\varone}{\relone}{\termone}}
  \qquad
  \infer[\Howetwo]
  {\rel{\vecvarone}{\abstr{\varone}{\termone}}{\howe{\relone}}{\termtwo}}
  {
    \rel{\vecvarone\cup\{\varone\}}{\termone}{\howe{\relone}}{\termthree}
    &&
    \rel{\vecvarone}{\abstr{\varone}{\termthree}}{\relone}{\termtwo}
    &&
    \varone\notin\vecvarone
  }
  $$
  $$
  \infer[\Howethree]
  {\rel{\vecvarone}{\app{\termone}{\termtwo}}{\howe{\relone}}{\termthree}}
  {
    \rel{\vecvarone}{\termone}{\howe{\relone}}{\termfour}
    &&
    \rel{\vecvarone}{\termtwo}{\howe{\relone}}{\termfive}
    &&
    \rel{\vecvarone}{\app{\termfour}{\termfive}}{\relone}{\termthree}
  }
  $$
  $$
  \infer[\Howefour]
  {\rel{\vecvarone}{\ps{\termone}{\termtwo}}{\howe{\relone}}{\termthree}}
  {
    \rel{\vecvarone}{\termone}{\howe{\relone}}{\termfour}
    &&
    \rel{\vecvarone}{\termtwo}{\howe{\relone}}{\termfive}
    &&
    \rel{\vecvarone}{\ps{\termfour}{\termfive}}{\relone}{\termthree}
  }
  $$
\caption{Howe's Lifting for $\LOP$.}\label{fig:howelifting}
\end{figure*}
The reader familiar with Howe's method should have a sense of
\emph{d\'ej\`a vu} here: indeed, this is \emph{precisely} 
the same definition one finds in the realm of \emph{nondeterministic} $\lambda$-calculi. The language of terms, after all,
is the same. This facilitates the first part of the proof. Indeed, one
already knows that if $\relone$ is a preorder,
then $\howe{\relone}$ is compatible
and includes $\relone$, since all these properties are already known (see, e.g.~\cite{PittsSurvey}) and
only depend on the shape of terms and not on their operational semantics. 
\begin{lemma}\label{lemma:howeprop1}
    If $\relone$ is reflexive, then $\howe{\relone}$ is compatible.
  \end{lemma}
  \begin{proof}
    We need to prove that \Comone, \Comtwo,
    \Comthree, and \Comfour\ hold for $\howe{\relone}$:
    \begin{varitemize}
    \item
      Proving \Comone{} means to show:
      $$
      \forall\vecvarone\in\powfin{\setvar},
      \varone\in\vecvarone\Rightarrow\rel{\vecvarone}{\varone}{\howe{\relone}}{\varone}.
      $$
      Since $\relone$ is reflexive, $\forall\vecvarone\in\powfin{\setvar}$,
      $\varone\in\vecvarone\Rightarrow\rel{\vecvarone}{\varone}{\relone}{\varone}$. Thus,
      by \Howeone, we conclude
      $\rel{\vecvarone}{\varone}{\howe{\relone}}{\varone}$. Formally,
      $$
      \infer[\Howeone]
      {\rel{\vecvarone}{\varone}{\howe{\relone}}{\varone}}
      {\infer{\rel{\vecvarone}{\varone}{\relone}{\varone}} {}}
      $$
    \item 
      Proving \Comtwo{} means to show:
      $\forall\vecvarone\in\powfin{\setvar}$,
      $\forall\varone\in\setvar-\vecvarone$, $\forall\termone,
      \termtwo\in\LOP(\vecvarone\cup\{\varone\})$,
      $$
      \rel{\vecvarone\cup\{\varone\}}{\termone}{\howe{\relone}}{\termtwo}\Rightarrow
      \rel{\vecvarone}{\abstr{\varone}{\termone}}{\howe{\relone}}{\abstr{\varone}{\termtwo}}.
      $$
      Since $\relone$ is reflexive, we get
      $\rel{\vecvarone}{\abstr{\varone}{\termtwo}}{\relone}{\abstr{\varone}{\termtwo}}$. Moreover,
      we have
      $\rel{\vecvarone\cup\{\varone\}}{\termone}{\howe{\relone}}{\termtwo}$
      by hypothesis. Thus, by \Howetwo, we conclude
      $\rel{\vecvarone}{\abstr{\varone}{\termone}}{\howe{\relone}}{\abstr{\varone}{\termtwo}}$
      holds. Formally,
      $$
      \infer[\Howetwo]
      {\rel{\vecvarone}{\abstr{\varone}{\termone}}{\howe{\relone}}{\abstr{\varone}{\termtwo}}}
      { \rel{\vecvarone\cup\{\varone\}}{\termone}{\howe{\relone}}{\termtwo}
        &&
        \infer{\rel{\vecvarone}{\abstr{\varone}{\termtwo}}{\relone}{\abstr{\varone}{\termtwo}}}{}
        && \varone\notin\vecvarone }
      $$
   \item
      Proving \Comthree\ means to show:
      $\forall\vecvarone\in\powfin{\setvar}$, $\forall\termone,\termtwo,\termthree,\termfour\in\LOP(\vecvarone)$,
      $$
      \rel{\vecvarone}{\termone}{\howe{\relone}}{\termtwo}\wedge\rel{\vecvarone}{\termthree}{\howe{\relone}}{\termfour}\Rightarrow
      \rel{\vecvarone}{\app{\termone}{\termthree}}{\howe{\relone}}{\app{\termtwo}{\termfour}}.
      $$
      Since $\relone$ is reflexive, we get
      $\rel{\vecvarone}{\app{\termtwo}{\termfour}}{\relone}{\app{\termtwo}{\termfour}}$. Moreover,
      we have $\rel{\vecvarone}{\termone}{\howe{\relone}}{\termtwo}$ and
      $\rel{\vecvarone}{\termthree}{\howe{\relone}}{\termfour}$ by
      hypothesis. Thus, by \Howethree, we conclude
      $\rel{\vecvarone}{\app{\termone}{\termthree}}{\howe{\relone}}{\app{\termtwo}{\termfour}}$
      holds. Formally,
      $$
      \infer[\Howethree]
      {\rel{\vecvarone}{\app{\termone}{\termthree}}{\howe{\relone}}{\app{\termtwo}{\termfour}}}
      { \rel{\vecvarone}{\termone}{\howe{\relone}}{\termtwo} &&
        \rel{\vecvarone}{\termthree}{\howe{\relone}}{\termfour} &&
        \infer{\rel{\vecvarone}{\app{\termtwo}{\termfour}}{\relone}{\app{\termtwo}{\termfour}}}{}}
      $$
    \item
      Proving \Comfour\ means to show:
      $\forall\vecvarone\in\powfin{\setvar}$, $\forall\termone,\termtwo,\termthree,\termfour\in\LOP(\vecvarone)$,
      $$
      \rel{\vecvarone}{\termone}{\howe{\relone}}{\termtwo}\wedge\rel{\vecvarone}{\termthree}{\howe{\relone}}{\termfour}\Rightarrow
      \rel{\vecvarone}{\ps{\termone}{\termthree}}{\howe{\relone}}{\ps{\termtwo}{\termfour}}.$$
      Since $\relone$ is reflexive, we get
      $\rel{\vecvarone}{\ps{\termtwo}{\termfour}}{\relone}{\ps{\termtwo}{\termfour}}$. Moreover,
      we have $\rel{\vecvarone}{\termone}{\howe{\relone}}{\termtwo}$ and
      $\rel{\vecvarone}{\termthree}{\howe{\relone}}{\termfour}$ by
      hypothesis. Thus, by \Howefour, we conclude
      $\rel{\vecvarone}{\ps{\termone}{\termthree}}{\howe{\relone}}{\ps{\termtwo}{\termfour}}$
      holds. Formally,
      $$
      \infer[\Howefour]
      {\rel{\vecvarone}{\ps{\termone}{\termthree}}{\howe{\relone}}{\ps{\termtwo}{\termfour}}}
      { \rel{\vecvarone}{\termone}{\howe{\relone}}{\termtwo} &&
        \rel{\vecvarone}{\termthree}{\howe{\relone}}{\termfour} &&
        \infer{\rel{\vecvarone}{\ps{\termtwo}{\termfour}}{\relone}{\ps{\termtwo}{\termfour}}}{}
      }
      $$
    \end{varitemize}
    This concludes the proof.
\end{proof}
  \begin{lemma}\label{lemma:howeprop2}
    If $\relone$ is transitive, then
    $\rel{\vecvarone}{\termone}{\howe{\relone}}{\termtwo}$ and
    $\rel{\vecvarone}{\termtwo}{\relone}{\termthree}$ imply
    $\rel{\vecvarone}{\termone}{\howe{\relone}}{\termthree}$.
  \end{lemma}
\begin{proof}
  We prove the statement by inspection on the last rule used in the derivation of
  $\rel{\vecvarone}{\termone}{\howe{\relone}}{\termtwo}$, thus on the
  structure of $\termone$.
  \begin{varitemize}
  \item 
    If $\termone$ is a variable, say $\varone \in \vecvarone$, then
$\rel{\vecvarone}{\varone}{\howe{\relone}}{\termtwo}$ holds by
    hypothesis. The last rule used has to be \Howeone.  Thus, we get
    $\rel{\vecvarone}{\varone}{\relone}{\termtwo}$ as additional
    hypothesis. By transitivity of $\relone$, from
    $\rel{\vecvarone}{\varone}{\relone}{\termtwo}$ and
    $\rel{\vecvarone}{\termtwo}{\relone}{\termthree}$ we deduce
    $\rel{\vecvarone}{\varone}{\relone}{\termthree}$. We conclude by
    \Howeone\ on the latter, obtaining
    $\rel{\vecvarone}{\varone}{\howe{\relone}}{\termthree}$,
    i.e. $\rel{\vecvarone}{\termone}{\howe{\relone}}{\termthree}$. Formally,
$$
    \infer[\Howeone]
    {\rel{\vecvarone}{\varone}{\howe{\relone}}{\termthree}}
    {\infer{\rel{\vecvarone}{\varone}{\relone}{\termthree}} {
        \rel{\vecvarone}{\varone}{\relone}{\termtwo} &&
        \rel{\vecvarone}{\termtwo}{\relone}{\termthree} } }
    $$
  \item If $\termone$ is a $\lambda$-abstraction, say
    $\abstr{\varone}{\termfive}$, then $\rel{\vecvarone}{\abstr{\varone}{\termfive}}{\howe{\relone}}{\termtwo}$
    holds by hypothesis. The last rule used has to be \Howetwo. Thus, we
    get
    $\rel{\vecvarone\cup\{\varone\}}{\termfive}{\howe{\relone}}{\termfour}$
    and $\rel{\vecvarone}{\abstr{\varone}{\termfour}}{\relone}{\termtwo}$
    as additional hypothesis. By transitivity of $\relone$, from
    $\rel{\vecvarone}{\abstr{\varone}{\termfour}}{\relone}{\termtwo}$ and
    $\rel{\vecvarone}{\termtwo}{\relone}{\termthree}$ we deduce
    $\rel{\vecvarone}{\abstr{\varone}{\termfour}}{\relone}{\termthree}$. We
    conclude by \Howetwo\ on
    $\rel{\vecvarone\cup\{\varone\}}{\termfive}{\howe{\relone}}{\termfour}$
    and the latter, obtaining
    $\rel{\vecvarone}{\abstr{\varone}{\termfive}}{\howe{\relone}}{\termthree}$,
    i.e. $\rel{\vecvarone}{\termone}{\howe{\relone}}{\termthree}$.
    Formally, $$
    \infer[\Howetwo]
    {\rel{\vecvarone}{\abstr{\varone}{\termfive}}{\howe{\relone}}{\termthree}}
    {
      \rel{\vecvarone\cup\{\varone\}}{\termfive}{\howe{\relone}}{\termfour}
      &&
      \infer{\rel{\vecvarone}{\abstr{\varone}{\termfour}}{\relone}{\termthree}}{
        \rel{\vecvarone}{\abstr{\varone}{\termfour}}{\relone}{\termtwo}
        && \rel{\vecvarone}{\termtwo}{\relone}{\termthree}}}
    $$
  \item If $\termone$ is an application, say $\app{\termsix}{\termseven}$,
    then $\rel{\vecvarone}{\app{\termsix}{\termseven}}{\howe{\relone}}{\termtwo}$
    holds by hypothesis. The last rule used has to be \Howethree. Thus, we
    get $\rel{\vecvarone}{\termsix}{\howe{\relone}}{\termfour}$,
    $\rel{\vecvarone}{\termseven}{\howe{\relone}}{\termfive}$ and
    $\rel{\vecvarone}{\app{\termfour}{\termfive}}{\relone}{\termtwo}$ as
    additional hypothesis. By transitivity of $\relone$, from
    $\rel{\vecvarone}{\app{\termfour}{\termfive}}{\relone}{\termtwo}$ and
    $\rel{\vecvarone}{\termtwo}{\relone}{\termthree}$ we deduce
    $\rel{\vecvarone}{\app{\termfour}{\termfive}}{\relone}{\termthree}$. We
    conclude by \Howethree{} on
    $\rel{\vecvarone}{\termsix}{\howe{\relone}}{\termfour}$,
    $\rel{\vecvarone}{\termseven}{\howe{\relone}}{\termfive}$ and the
    latter, obtaining
    $\rel{\vecvarone}{\app{\termsix}{\termseven}}{\howe{\relone}}{\termthree}$,
    i.e. $\rel{\vecvarone}{\termone}{\howe{\relone}}{\termthree}$.
    Formally, $$
    \infer[\Howethree]
    {\rel{\vecvarone}{\app{\termsix}{\termseven}}{\howe{\relone}}{\termthree}}
    { \rel{\vecvarone}{\termsix}{\howe{\relone}}{\termfour} &&
      \rel{\vecvarone}{\termseven}{\howe{\relone}}{\termfive} &&
      \infer{\rel{\vecvarone}{\app{\termfour}{\termfive}}{\relone}{\termthree}}{
        \rel{\vecvarone}{\app{\termfour}{\termfive}}{\relone}{\termtwo} &&
        \rel{\vecvarone}{\termtwo}{\relone}{\termthree}
      }
    }
    $$
  \item If $\termone$ is a probabilistic sum, say
    $\ps{\termsix}{\termseven}$, then $\rel{\vecvarone}{\ps{\termsix}{\termseven}}{\howe{\relone}}{\termtwo}$
    holds by hypothesis. The last rule used has to be \Howefour. Thus, we
    get $\rel{\vecvarone}{\termsix}{\howe{\relone}}{\termfour}$,
    $\rel{\vecvarone}{\termseven}{\howe{\relone}}{\termfive}$ and
    $\rel{\vecvarone}{\ps{\termfour}{\termfive}}{\relone}{\termtwo}$ as
    additional hypothesis. By transitivity of $\relone$, from
    $\rel{\vecvarone}{\ps{\termfour}{\termfive}}{\relone}{\termtwo}$ and
    $\rel{\vecvarone}{\termtwo}{\relone}{\termthree}$ we deduce
    $\rel{\vecvarone}{\ps{\termfour}{\termfive}}{\relone}{\termthree}$.  We
    conclude by \Howefour\ on
    $\rel{\vecvarone}{\termsix}{\howe{\relone}}{\termfour}$,
    $\rel{\vecvarone}{\termseven}{\howe{\relone}}{\termfive}$ and the
    latter, obtaining
    $\rel{\vecvarone}{\ps{\termsix}{\termseven}}{\howe{\relone}}{\termthree}$,
    i.e. $\rel{\vecvarone}{\termone}{\howe{\relone}}{\termthree}$.
    Formally, $$
    \infer[\Howefour]
    {\rel{\vecvarone}{\ps{\termsix}{\termseven}}{\howe{\relone}}{\termthree}}
    { \rel{\vecvarone}{\termsix}{\howe{\relone}}{\termfour} &&
      \rel{\vecvarone}{\termseven}{\howe{\relone}}{\termfive} &&
      \infer{\rel{\vecvarone}{\ps{\termfour}{\termfive}}{\relone}{\termthree}}{
        \rel{\vecvarone}{\ps{\termfour}{\termfive}}{\relone}{\termtwo}
        && \rel{\vecvarone}{\termtwo}{\relone}{\termthree} } }
    $$
  \end{varitemize}
  This concludes the proof.
\end{proof}
  \begin{lemma}\label{lemma:howeprop3}
    If $\relone$ is reflexive, then
    $\rel{\vecvarone}{\termone}{\relone}{\termtwo}$ implies
    $\rel{\vecvarone}{\termone}{\howe{\relone}}{\termtwo}$.
  \end{lemma}
\begin{proof}
  We will prove it by inspection on the structure of $\termone$.
  \begin{varitemize}
  \item 
    If $\termone$ is a variable, say $\varone \in \vecvarone$, then
$\rel{\vecvarone}{\varone}{\relone}{\termtwo}$ holds by hypothesis. We
    conclude by \Howeone\ on the latter, obtaining
    $\rel{\vecvarone}{\varone}{\howe{\relone}}{\termtwo}$,
    i.e. $\rel{\vecvarone}{\termone}{\howe{\relone}}{\termtwo}$. Formally,
    $$
    \infer[\Howeone]
    {\rel{\vecvarone}{\varone}{\howe{\relone}}{\termtwo}}
    {\rel{\vecvarone}{\varone}{\relone}{\termtwo}}
    $$
  \item If $\termone$ is a $\lambda$-abstraction, say
    $\abstr{\varone}{\termfive}$, then $\rel{\vecvarone}{\abstr{\varone}{\termfive}}{\relone}{\termtwo}$ holds
    by hypothesis. Moreover, since $\relone$ reflexive implies $\howe{\relone}$ compatible,
    $\howe{\relone}$ is reflexive too. Then, from
    $\rel{\vecvarone\cup\{\varone\}}{\termfive}{\howe{\relone}}{\termfive}$
    and $\rel{\vecvarone}{\abstr{\varone}{\termfive}}{\relone}{\termtwo}$
    we conclude, by \Howetwo,
    $\rel{\vecvarone}{\abstr{\varone}{\termfive}}{\howe{\relone}}{\termtwo}$,
    i.e. $\rel{\vecvarone}{\termone}{\howe{\relone}}{\termtwo}$. Formally,
    $$
    \infer[\Howetwo]
    {\rel{\vecvarone}{\abstr{\varone}{\termfive}}{\howe{\relone}}{\termtwo}}
    {
      \rel{\vecvarone\cup\{\varone\}}{\termfive}{\howe{\relone}}{\termfive}
      &&
      \rel{\vecvarone}{\abstr{\varone}{\termfive }}{\relone}{\termtwo}
      && \varone\notin\vecvarone }
    $$
  \item If $\termone$ is an application, say $\app{\termthree}{\termfour}$,
    then $\rel{\vecvarone}{\app{\termthree}{\termfour}}{\relone}{\termtwo}$
    holds by hypothesis. By reflexivity of $\relone$, hence that of
    $\howe{\relone}$ too, we get
    $\rel{\vecvarone}{\termthree}{\howe{\relone}}{\termthree}$ and
    $\rel{\vecvarone}{\termfour}{\howe{\relone}}{\termfour}$. Then, from
    the latter and
    $\rel{\vecvarone}{\app{\termthree}{\termfour}}{\relone}{\termtwo}$ we
    conclude, by \Howethree,
    $\rel{\vecvarone}{\app{\termthree}{\termfour}}{\howe{\relone}}{\termtwo}$,
    i.e. $\rel{\vecvarone}{\termone}{\howe{\relone}}{\termtwo}$. Formally,
    $$
    \infer[\Howethree]
    {\rel{\vecvarone}{\app{\termthree}{\termfour}}{\howe{\relone}}{\termtwo}}
    { \rel{\vecvarone}{\termthree}{\howe{\relone}}{\termthree} &&
      \rel{\vecvarone}{\termfour}{\howe{\relone}}{\termfour} &&
      \rel{\vecvarone}{\app{\termthree}{\termfour}}{\relone}{\termtwo}
    }
    $$
  \item If $\termone$ is a probabilistic sum, say
    $\ps{\termthree}{\termfour}$, then $\rel{\vecvarone}{\ps{\termthree}{\termfour}}{\relone}{\termtwo}$ holds
    by hypothesis. By reflexivity of $\relone$, hence that of
    $\howe{\relone}$ too, we get
    $\rel{\vecvarone}{\termthree}{\howe{\relone}}{\termthree}$ and
    $\rel{\vecvarone}{\termfour}{\howe{\relone}}{\termfour}$. Then, from
    the latter and
    $\rel{\vecvarone}{\ps{\termthree}{\termfour}}{\relone}{\termtwo}$ we
    conclude, by \Howefour,
    $\rel{\vecvarone}{\ps{\termthree}{\termfour}}{\howe{\relone}}{\termtwo}$,
    i.e. $\rel{\vecvarone}{\termone}{\howe{\relone}}{\termtwo}$. Formally,
    $$
    \infer[\Howefour]
    {\rel{\vecvarone}{\ps{\termthree}{\termfour}}{\howe{\relone}}{\termtwo}}
    { \rel{\vecvarone}{\termthree}{\howe{\relone}}{\termthree} &&
      \rel{\vecvarone}{\termfour}{\howe{\relone}}{\termfour} &&
      \rel{\vecvarone}{\ps{\termthree}{\termfour}}{\relone}{\termtwo} }
    $$
  \end{varitemize}
  This concludes the proof.
\end{proof}

Moreover, if $\relone$ is a preorder and closed under term-substitution,
then its lifted relation $\howe{\relone}$ is substitutive. Then,
reflexivity of $\relone$ implies compatibility of $\howe{\relone}$ by
Lemma~\ref{lemma:howeprop1}. It follows $\howe{\relone}$ reflexive too,
hence closed under term-substitution.

\begin{lemma}\label{lemma:closesubsCBN}
  If $\relone$ is reflexive, transitive and closed under term-substitution,
  then $\howe{\relone}$ is (term) substitutive and hence also closed under
  term-substitution.
\end{lemma}
\begin{proof}
  We show that, for all $\vecvarone\in\powfin{\setvar}$,
  $\varone\in\setvar-\vecvarone$,
  $\termone,\,\termtwo\in\LOPp{\vecvarone\cup\{\varone\}}$ and
  $\termthree,\,\termfour\in\LOPp{\vecvarone}$,
  \begin{align*}
    \rel{\vecvarone\cup\{\varone\}}{\termone}{\howe{\relone}}{\termtwo}\wedge
    \rel{\vecvarone}{\termthree}{\howe{\relone}}{\termfour}\Rightarrow
    \rel{\vecvarone}{\subst{\termone}{\varone}{\termthree}}{\howe{\relone}}{\subst{\termtwo}{\varone}{\termfour}}.
  \end{align*}
We prove the latter by induction on the derivation of
  $\rel{\vecvarone\cup\{\varone\}}{\termone}{\howe{\relone}}{\termtwo}$,
  thus on the structure of $\termone$.
  \begin{varitemize}
  \item If $\termone$ is a variable, then either $\termone = \varone$ or
    $\termone\in\vecvarone$. In the latter case, suppose $\termone =
    \vartwo$. Then, by hypothesis, $\rel{\vecvarone\cup\{\varone\}}{\vartwo}{\howe{\relone}}{\termtwo}$
    holds and the only way to deduce it is by rule \Howeone\ from
    $\rel{\vecvarone\cup\{\varone\}}{\vartwo}{\relone}{\termtwo}$. Hence,
    by the fact $\relone$ is closed under term-substitution and
    $\termfour\in\LOPp{\vecvarone}$, we obtain
    $\rel{\vecvarone}{\subst{\vartwo}{\varone}{\termfour}}{\relone}{\subst{\termtwo}{\varone}{\termfour}}$
    which is equivalent to
    $\rel{\vecvarone}{\vartwo}{\relone}{\subst{\termtwo}{\varone}{\termfour}}$. Finally,
    by Lemma~\ref{lemma:howeprop3}, we conclude
    $\rel{\vecvarone}{\vartwo}{\howe{\relone}}{\subst{\termtwo}{\varone}{\termfour}}$
    which is equivalent to
    $\rel{\vecvarone}{\subst{\vartwo}{\varone}{\termthree}}{\howe{\relone}}{\subst{\termtwo}{\varone}{\termfour}}$,
    i.e.
    $\rel{\vecvarone}{\subst{\termone}{\varone}{\termthree}}{\howe{\relone}}{\subst{\termtwo}{\varone}{\termfour}}$
    holds. Otherwise, $\termone = \varone$ and
    $\rel{\vecvarone\cup\{\varone\}}{\varone}{\howe{\relone}}{\termtwo}$
    holds. The only way to deduce the latter is by the rule \Howeone\ from
    $\rel{\vecvarone\cup\{\varone\}}{\varone}{\relone}{\termtwo}$. Hence,
    by the fact $\relone$ is closed under term-substitution and
    $\termfour\in\LOPp{\vecvarone}$, we obtain
    $\rel{\vecvarone}{\subst{\varone}{\varone}{\termfour}}{\relone}{\subst{\termtwo}{\varone}{\termfour}}$
    which is equivalent to
    $\rel{\vecvarone}{\termfour}{\relone}{\subst{\termtwo}{\varone}{\termfour}}$. By
    Lemma~\ref{lemma:howeprop2}, we deduce the following:
    $$
    \infer[]
    {\rel{\vecvarone}{\termthree}{\howe{\relone}}{\subst{\termtwo}{\varone}{\termfour}}}
    { \rel{\vecvarone}{\termthree}{\howe{\relone}}{\termfour} &&
      \rel{\vecvarone}{\termfour}{\relone}{\subst{\termtwo}{\varone}{\termfour}}
    }
    $$
    which is equivalent to
    $\rel{\vecvarone}{\subst{\varone}{\varone}{\termthree}}{\howe{\relone}}{\subst{\termtwo}{\varone}{\termfour}}$. Thus,
    $\rel{\vecvarone}{\subst{\termone}{\varone}{\termthree}}{\howe{\relone}}{\subst{\termtwo}{\varone}{\termfour}}$
    holds.
  \item If $\termone$ is a $\lambda$-abstraction, say
    $\abstr{\vartwo}{\termfive}$, then $\rel{\vecvarone\cup\{\varone\}}{\abstr{\vartwo}{\termfive}}{\howe{\relone}}{\termtwo}$
    holds by hypothesis. The only way to deduce the latter is by rule
    \Howetwo\ as follows:
    $$
    \infer[\Howetwo]
    {\rel{\vecvarone\cup\{\varone\}}{\abstr{\vartwo}{\termfive}}{\howe{\relone}}{\termtwo}}
    {
      \rel{\vecvarone\cup\{\varone,\vartwo\}}{\termfive}{\howe{\relone}}{\termsix}
      &&
      \rel{\vecvarone\cup\{\varone\}}{\abstr{\vartwo}{\termsix}}{\relone}{\termtwo}
      && \varone,\vartwo\notin\vecvarone }
    $$

    Let us denote $\vecvartwo = \vecvarone\cup\{\vartwo\}$. Then, by
    induction hypothesis on
    $\rel{\vecvartwo\cup\{\varone\}}{\termfive}{\howe{\relone}}{\termsix}$,
    we get
    $\rel{\vecvartwo}{\subst{\termfive}{\varone}{\termthree}}{\howe{\relone}}{\subst{\termsix}{\varone}{\termfour}}$. Moreover,
    by the fact $\relone$ is closed under term-substitution and
    $\termfour\in\LOPp{\vecvarone}$, we obtain that
    $\rel{\vecvarone}{\subst{(\abstr{\vartwo}{\termsix})}{\varone}{\termfour}}{\relone}{\subst{\termtwo}{\varone}{\termfour}}$
    holds, i.e.
    $\rel{\vecvarone}{\subst{\abstr{\vartwo}{\termsix}}{\varone}{\termfour}}{\relone}{\subst{\termtwo}{\varone}{\termfour}}$. By \Howetwo, we deduce the following:
    $$
    \infer[\Howetwo]
    {\rel{\vecvarone}{\subst{\abstr{\vartwo}{\termfive}}{\varone}{\termthree}}{\howe{\relone}}{\subst{\termtwo}{\varone}{\termfour}}}
    {
      \rel{\vecvarone\cup\{\vartwo\}}{\subst{\termfive}{\varone}{\termthree}}{\howe{\relone}}{\subst{\termsix}{\varone}{\termfour}}
      &&
      \rel{\vecvarone}{\subst{\abstr{\vartwo}{\termsix}}{\varone}{\termfour}}{\relone}{\subst{\termtwo}{\varone}{\termfour}}
      && \vartwo\notin\vecvarone } 
    $$
    which is equivalent to
    $\rel{\vecvarone}{\subst{(\abstr{\vartwo}{\termfive})}{\varone}{\termthree}}{\howe{\relone}}{\subst{\termtwo}{\varone}{\termfour}}$. Thus,
    $\rel{\vecvarone}{\subst{\termone}{\varone}{\termthree}}{\howe{\relone}}{\subst{\termtwo}{\varone}{\termfour}}$
    holds.
  \item If $\termone$ is an application, say $\app{\termfive}{\termsix}$,
    then $\rel{\vecvarone\cup\{\varone\}}{\app{\termfive}{\termsix}}{\howe{\relone}}{\termtwo}$
    holds by hypothesis. The only way to deduce the latter is by rule
    \Howethree\ as follows:
    $$
    \infer[\Howethree]
    {\rel{\vecvarone\cup\{\varone\}}{\app{\termfive}{\termsix}}{\howe{\relone}}{\termtwo}}
    {
      \rel{\vecvarone\cup\{\varone\}}{\termfive}{\howe{\relone}}{\termfive'}
      &&
      \rel{\vecvarone\cup\{\varone\}}{\termsix}{\howe{\relone}}{\termsix'}
      &&
      \rel{\vecvarone\cup\{\varone\}}{\app{\termfive'}{\termsix'}}{\relone}{\termtwo}
    }
    $$

    By induction hypothesis on
    $\rel{\vecvarone\cup\{\varone\}}{\termfive}{\howe{\relone}}{\termfive'}$
    and
    $\rel{\vecvarone\cup\{\varone\}}{\termsix}{\howe{\relone}}{\termsix'}$,
    we get
    $\rel{\vecvarone}{\subst{\termfive}{\varone}{\termthree}}{\howe{\relone}}{\subst{\termfive'}{\varone}{\termfour}}$
    and
    $\rel{\vecvarone}{\subst{\termsix}{\varone}{\termthree}}{\howe{\relone}}{\subst{\termsix'}{\varone}{\termfour}}$.
    Moreover, by the fact $\relone$ is closed under term-substitution and
    $\termfour\in\LOPp{\vecvarone}$, we obtain that
    $\rel{\vecvarone}{\subst{(\app{\termfive'}{\termsix'})}{\varone}{\termfour}}{\relone}{\subst{\termtwo}{\varone}{\termfour}}$
    holds, i.e.
    $\rel{\vecvarone}{\app{\subst{\termfive'}{\varone}{\termfour}}{\subst{\termsix'}{\varone}{\termfour}}}{\relone}{\subst{\termtwo}{\varone}{\termfour}}$.
By \Howethree, we deduce the following:
    $$
    \infer[\Howethree]
    {\rel{\vecvarone}{\app{\subst{\termfive}{\varone}{\termthree}}{\subst{\termsix}{\varone}{\termthree}}}{\howe{\relone}}{\subst{\termtwo}{\varone}{\termfour}}}
    {
      \rel{\vecvarone}{\subst{\termfive}{\varone}{\termthree}}{\howe{\relone}}{\subst{\termfive'}{\varone}{\termfour}}
      &&
      \rel{\vecvarone}{\subst{\termsix}{\varone}{\termthree}}{\howe{\relone}}{\subst{\termsix'}{\varone}{\termfour}}
      &&
      \rel{\vecvarone}{\app{\subst{\termfive'}{\varone}{\termfour}}{\subst{\termsix'}{\varone}{\termfour}}}{\relone}{\subst{\termtwo}{\varone}{\termfour}}}
    $$
    which is equivalent to
    $\rel{\vecvarone}{\subst{(\app{\termfive}{\termsix})}{\varone}{\termthree}}{\howe{\relone}}{\subst{\termtwo}{\varone}{\termfour}}$. Thus,
    $\rel{\vecvarone}{\subst{\termone}{\varone}{\termthree}}{\howe{\relone}}{\subst{\termtwo}{\varone}{\termfour}}$
    holds.
  \item If $\termone$ is a probabilistic sum, say
    $\ps{\termfive}{\termsix}$, then $\rel{\vecvarone\cup\{\varone\}}{\ps{\termfive}{\termsix}}{\howe{\relone}}{\termtwo}$
    holds by hypothesis. The only way to deduce the latter is by rule
    \Howefour\ as follows:
    $$
    \infer[\Howefour]
    {\rel{\vecvarone\cup\{\varone\}}{\ps{\termfive}{\termsix}}{\howe{\relone}}{\termtwo}}
    {
      \rel{\vecvarone\cup\{\varone\}}{\termfive}{\howe{\relone}}{\termfive'}
      &&
      \rel{\vecvarone\cup\{\varone\}}{\termsix}{\howe{\relone}}{\termsix'}
      &&
      \rel{\vecvarone\cup\{\varone\}}{\ps{\termfive'}{\termsix'}}{\relone}{\termtwo}
    }
    $$
    By induction hypothesis on
    $\rel{\vecvarone\cup\{\varone\}}{\termfive}{\howe{\relone}}{\termfive'}$
    and
    $\rel{\vecvarone\cup\{\varone\}}{\termsix}{\howe{\relone}}{\termsix'}$,
    we get
    $\rel{\vecvarone}{\subst{\termfive}{\varone}{\termthree}}{\howe{\relone}}{\subst{\termfive'}{\varone}{\termfour}}$
    and
    $\rel{\vecvarone}{\subst{\termsix}{\varone}{\termthree}}{\howe{\relone}}{\subst{\termsix'}{\varone}{\termfour}}$. Moreover,
    by the fact $\relone$ is closed under term-substitution and
    $\termfour\in\LOPp{\vecvarone}$, we obtain that
    $\rel{\vecvarone}{\subst{(\ps{\termfive'}{\termsix'})}{\varone}{\termfour}}{\relone}{\subst{\termtwo}{\varone}{\termfour}}$,
    i.e.
    $\rel{\vecvarone}{\ps{\subst{\termfive'}{\varone}{\termfour}}{\subst{\termsix'}{\varone}{\termfour}}}{\relone}{\subst{\termtwo}{\varone}{\termfour}}$. 
By \Howefour, we conclude the following:
    $$
    \infer[\Howefour]
    {\rel{\vecvarone}{\ps{\subst{\termfive}{\varone}{\termthree}}{\subst{\termsix}{\varone}{\termthree}}}{\howe{\relone}}{\subst{\termtwo}{\varone}{\termfour}}}
    {
      \rel{\vecvarone}{\subst{\termfive}{\varone}{\termthree}}{\howe{\relone}}{\subst{\termfive'}{\varone}{\termfour}}
      &&
      \rel{\vecvarone}{\subst{\termsix}{\varone}{\termthree}}{\howe{\relone}}{\subst{\termsix'}{\varone}{\termfour}}
      &&
      \rel{\vecvarone}{\ps{\subst{\termfive'}{\varone}{\termfour}}{\subst{\termsix'}{\varone}{\termfour}}}{\relone}{\subst{\termtwo}{\varone}{\termfour}}
    }
    $$
    which is equivalent to
    $\rel{\vecvarone}{\subst{(\ps{\termfive}{\termsix})}{\varone}{\termthree}}{\howe{\relone}}{\subst{\termtwo}{\varone}{\termfour}}$. Thus,
    $\rel{\vecvarone}{\subst{\termone}{\varone}{\termthree}}{\howe{\relone}}{\subst{\termtwo}{\varone}{\termfour}}$
    holds.
  \end{varitemize}
  This concludes the proof.
\end{proof}
Something is missing, however, before we can conclude that
$\howe{\cbnpas}$ is a precongruence, namely transitivity.
We also follow Howe here building the transitive
closure of a $\LOP$-relation $\relone$ as the relation $\tcrel{\relone}$
defined by the rules in Figure~\ref{fig:transitiveclosure}. 
\begin{figure*}
$$
\infer[\TCone]
{\rel{\vecvarone}{\termone}{\tcrel{\relone}}{\termtwo}}
{\rel{\vecvarone}{\termone}{\relone}{\termtwo}}
$$
$$
\infer[\TCtwo]
{\rel{\vecvarone}{\termone}{\tcrel{\relone}}{\termthree}} {
  \rel{\vecvarone}{\termone}{\tcrel{\relone}}{\termtwo} &&
  \rel{\vecvarone}{\termtwo}{\tcrel{\relone}}{\termthree} }
$$
\caption{Transitive Closure for $\LOP$.}\label{fig:transitiveclosure}
\end{figure*}
Then, it is easy to prove $\tcrel{\relone}$ of being compatible and closed
under term-substitution if $\relone$ is.  
\begin{lemma}\label{lemma:tcrelCom}
  If $\relone$ is compatible, then so is $\tcrel{\relone}$.
\end{lemma}
\begin{proof}
  We need to prove that \Comone, \Comtwo, \Comthree, and \Comfour\ hold for
  $\tcrel{\relone}$:
  \begin{varitemize}
  \item Proving \Comone{} means to show:
    $$
    \forall\vecvarone\in\powfin{\setvar}, \varone\in\vecvarone\Rightarrow
    \rel{\vecvarone}{\varone}{\relone}{\varone}.
    $$
    Since $\relone$ is compatible, therefore reflexive,
    $\rel{\vecvarone}{\varone}{\relone}{\varone}$ holds. Hence
    $\rel{\vecvarone}{\varone}{\tcrel{\relone}}{\varone}$ follows by
    \TCone.
  \item Proving \Comtwo{} means to show:
    $\forall\vecvarone\in\powfin{\setvar}$,
    $\forall\varone\in\setvar-\vecvarone$, $\forall\termone,
    \termtwo\in\LOP(\vecvarone\cup\{\varone\})$,
    $$
    \rel{\vecvarone\cup\{\varone\}}{\termone}{\tcrel{\relone}}{\termtwo}\Rightarrow
    \rel{\vecvarone}{\abstr{\varone}{\termone}}{\tcrel{\relone}}{\abstr{\varone}{\termtwo}}.
    $$
    We prove it by induction on the derivation of
    $\rel{\vecvarone\cup\{\varone\}}{\termone}{\tcrel{\relone}}{\termtwo}$,
    looking at the last rule used. The base case has $\TCone$ as last rule:
    thus, $\rel{\vecvarone\cup\{\varone\}}{\termone}{\relone}{\termtwo}$
    holds. Then, since $\relone$ is compatible, it follows
    $\rel{\vecvarone}{\abstr{\varone}{\termone}}{\relone}{\abstr{\varone}{\termtwo}}$. We
    conclude applying $\TCone$ on the latter, obtaining
    $\rel{\vecvarone}{\abstr{\varone}{\termone}}{\tcrel{\relone}}{\abstr{\varone}{\termtwo}}$. Otherwise,
    if $\TCtwo$ is the last rule used, we get that, for some
    $\termthree\in\LOP(\vecvarone\cup\{\varone\})$,
    $\rel{\vecvarone\cup\{\varone\}}{\termone}{\tcrel{\relone}}{\termthree}$
    and
    $\rel{\vecvarone\cup\{\varone\}}{\termthree}{\tcrel{\relone}}{\termtwo}$
    hold. Then, by induction hypothesis on both of them, we have
    $\rel{\vecvarone}{\abstr{\varone}{\termone}}{\tcrel{\relone}}{\abstr{\varone}{\termthree}}$
    and
    $\rel{\vecvarone}{\abstr{\varone}{\termthree}}{\tcrel{\relone}}{\abstr{\varone}{\termtwo}}$. We
    conclude applying $\TCtwo$ on the latter two, obtaining
    $\rel{\vecvarone}{\abstr{\varone}{\termone}}{\tcrel{\relone}}{\abstr{\varone}{\termtwo}}$.
  \item Proving \Comthree{} means to show:
    $\forall\vecvarone\in\powfin{\setvar}$, $\forall\termone,\termtwo,\termthree,\termfour\in\LOP(\vecvarone)$,
    $$
    \rel{\vecvarone}{\termone}{\tcrel{\relone}}{\termtwo}\ \wedge\
    \rel{\vecvarone}{\termthree}{\tcrel{\relone}}{\termfour}\Rightarrow
    \rel{\vecvarone}{\termone\termthree}{\tcrel{\relone}}{\termtwo\termfour}.
    $$
    Firstly, we prove the following two characterizations:
    \begin{align}
      &\label{equ:com3lp}\forall
      \termone,\termtwo,\termthree,\termfour\in\LOP(\vecvarone).\
      \rel{\vecvarone}{\termone}{\tcrel{\relone}}{\termtwo}\ \wedge \
      \rel{\vecvarone}{\termthree}{\relone}{\termfour}
      \Rightarrow \rel{\vecvarone}{\app{\termone}{\termthree}}{\tcrel{\relone}}{\app{\termtwo}{\termfour}},\\
      &\label{equ:com3rp}\forall
      \termone,\termtwo,\termthree,\termfour\in\LOP(\vecvarone).\
      \rel{\vecvarone}{\termone}{\relone}{\termtwo}\ \wedge \
      \rel{\vecvarone}{\termthree}{\tcrel{\relone}}{\termfour} \Rightarrow
      \rel{\vecvarone}{\app{\termone}{\termthree}}{\tcrel{\relone}}{\app{\termtwo}{\termfour}}.
    \end{align}
    In particular, we only prove (\ref{equ:com3lp}) in details, since
    (\ref{equ:com3rp}) is similarly provable. We prove (\ref{equ:com3lp})
    by induction on the derivation
    $\rel{\vecvarone}{\termone}{\tcrel{\relone}}{\termtwo}$, looking at the
    last rule used.  The base case has $\TCone$ as last rule: we get that
    $\rel{\vecvarone}{\termone}{\relone}{\termtwo}$ holds. Then, using
    $\relone$ compatibility property and
    $\rel{\vecvarone}{\termthree}{\relone}{\termfour}$, it follows
    $\rel{\vecvarone}{\app{\termone}{\termthree}}{\relone}{\app{\termtwo}{\termfour}}$. We
    conclude applying $\TCone$ on the latter, obtaining
    $\rel{\vecvarone}{\app{\termone}{\termthree}}{\tcrel{\relone}}{\app{\termtwo}{\termfour}}$.
    Otherwise, if $\TCtwo$ is the last rule used, we get that, for some
    $\termfive\in\LOP$,
    $\rel{\vecvarone}{\termone}{\tcrel{\relone}}{\termfive}$ and
    $\rel{\vecvarone}{\termfive}{\tcrel{\relone}}{\termtwo}$ hold. Then, by
    induction hypothesis on
    $\rel{\vecvarone}{\termone}{\tcrel{\relone}}{\termfive}$ along with
    $\rel{\vecvarone}{\termthree}{\relone}{\termfour}$, we have
    $\rel{\vecvarone}{\app{\termone}{\termthree}}{\tcrel{\relone}}{\app{\termfive}{\termfour}}$. Then,
    since $\relone$ is compatible and so reflexive too,
    $\rel{\vecvarone}{\termfour}{\relone}{\termfour}$ holds. By induction
    hypothesis on $\rel{\vecvarone}{\termfive}{\tcrel{\relone}}{\termtwo}$
    along with the latter, we get
    $\rel{\vecvarone}{\app{\termfive}{\termfour}}{\tcrel{\relone}}{\app{\termtwo}{\termfour}}$. We
    conclude applying $\TCtwo$ on
    $\rel{\vecvarone}{\app{\termone}{\termthree}}{\tcrel{\relone}}{\app{\termfive}{\termfour}}$
    and
    $\rel{\vecvarone}{\app{\termfive}{\termfour}}{\tcrel{\relone}}{\app{\termtwo}{\termfour}}$,
    obtaining
    $\rel{\vecvarone}{\app{\termone}{\termthree}}{\tcrel{\relone}}{\app{\termtwo}{\termfour}}$.

    Let us focus on the original \Comthree{} statement. We prove it by
    induction on the two derivations
    $\rel{\vecvarone}{\termone}{\tcrel{\relone}}{\termtwo}$ and
    $\rel{\vecvarone}{\termthree}{\tcrel{\relone}}{\termfour}$, which we
    name here as $\pfone$ and $\pftwo$ respectively. Looking at the last
    rules used, there are four possible cases as four are the combinations
    that permit to conclude with $\pfone$ and $\pftwo$:
    \begin{varenumerate}
    \item $\TCone$ for both $\pfone$ and $\pftwo$;
    \item $\TCone$ for $\pfone$ and $\TCtwo$ for $\pftwo$;
    \item $\TCtwo$ for $\pfone$ and $\TCone$ for $\pftwo$;
    \item $\TCtwo$ for both $\pfone$ and $\pftwo$.
    \end{varenumerate}
    Observe now that the first three cases are addressed by
    (\ref{equ:com3lp}) and (\ref{equ:com3rp}). Hence, it remains to prove
    the last case, where both derivations are concluded applying $\TCtwo$
    rule. According to $\TCtwo$ rule definition, we get two additional
    hypothesis from each derivation. In particular, for $\pfone$, we get
    that, for some $\termfive\in\LOP(\vecvarone)$,
    $\rel{\vecvarone}{\termone}{\tcrel{\relone}}{\termfive}$ and
    $\rel{\vecvarone}{\termfive}{\tcrel{\relone}}{\termtwo}$
    hold. Similarly, for $\pftwo$, we get that, for some
    $\termsix\in\LOP(\vecvarone)$,
    $\rel{\vecvarone}{\termthree}{\tcrel{\relone}}{\termsix}$ and
    $\rel{\vecvarone}{\termsix}{\tcrel{\relone}}{\termfour}$ hold. Then, by
    a double induction hypothesis, firstly on
    $\rel{\vecvarone}{\termone}{\tcrel{\relone}}{\termfive}$,
    $\rel{\vecvarone}{\termthree}{\tcrel{\relone}}{\termsix}$ and secondly
    on $\rel{\vecvarone}{\termfive}{\tcrel{\relone}}{\termtwo}$,
    $\rel{\vecvarone}{\termsix}{\tcrel{\relone}}{\termfour}$, we get
    $\rel{\vecvarone}{\app{\termone}{\termthree}}{\tcrel{\relone}}{\app{\termfive}{\termsix}}$
    and
    $\rel{\vecvarone}{\app{\termfive}{\termsix}}{\tcrel{\relone}}{\app{\termtwo}{\termfour}}$
    respectively. We conclude applying $\TCtwo$ on these latter, obtaining
    $\rel{\vecvarone}{\app{\termone}{\termthree}}{\tcrel{\relone}}{\app{\termtwo}{\termfour}}$.
  \item Proving \Comfour{} means to show:
    $\forall\vecvarone\in\powfin{\setvar}$, $\forall\termone,\termtwo,\termthree,\termfour\in\LOP(\vecvarone)$,
    $$
    \rel{\vecvarone}{\termone}{\tcrel{\relone}}{\termtwo}\ \wedge\
    \rel{\vecvarone}{\termthree}{\tcrel{\relone}}{\termfour}\Rightarrow
    \rel{\vecvarone}{\ps{\termone}{\termthree}}{\tcrel{\relone}}{\ps{\termtwo}{\termfour}}.
    $$
    We do not detail the proof since it boils down to that of \Comthree,
    where partial sums play the role of applications.
  \end{varitemize}
  This concludes the proof.
\end{proof}

\begin{lemma}\label{lemma:tcrelCTS}
  If $\relone$ is closed under term-substitution, then so is
  $\tcrel{\relone}$.
\end{lemma}
\begin{proof}
We need to prove $\tcrel{\relone}$ of being closed under
  term-substitution: for all $\vecvarone\in\powfin{\setvar}$,
  $\varone\in\setvar-\vecvarone$,
 $\termone,\termtwo\in\LOPp{\vecvarone\cup\{\varone\}}$ and
  $\termthree,\termfour\in\LOPp{\vecvarone}$,
  $$
  \rel{\vecvarone\cup\{\varone\}}{\termone}{\tcrel{\relone}}{\termtwo}\wedge\termthree\in\LOPp{\vecvarone}\Rightarrow
  \rel{\vecvarone}{\subst{\termone}{\varone}{\termthree}}{\tcrel{\relone}}{\subst{\termtwo}{\varone}{\termthree}}.
  $$
  We prove the latter by induction on the derivation of
  $\rel{\vecvarone\cup\{\varone\}}{\termone}{\tcrel{\relone}}{\termtwo}$,
  looking at the last rule used. The base case has $\TCone$ as last rule:
  we get that
  $\rel{\vecvarone\cup\{\varone\}}{\termone}{\relone}{\termtwo}$
  holds. Then, since $\relone$ is closed under term-substitution, it
  follows
  $\rel{\vecvarone}{\subst{\termone}{\varone}{\termthree}}{\relone}{\subst{\termtwo}{\varone}{\termthree}}$. We
  conclude applying $\TCone$ on the latter, obtaining
  $\rel{\vecvarone}{\subst{\termone}{\varone}{\termthree}}{\tcrel{\relone}}{\subst{\termtwo}{\varone}{\termthree}}$.
  Otherwise, if $\TCtwo$ is the last rule used, we get that, for some
  $\termfour\in\LOPp{\vecvarone\cup\{\varone\}}$,
  $\rel{\vecvarone\cup\{\varone\}}{\termone}{\tcrel{\relone}}{\termfour}$
  and
  $\rel{\vecvarone\cup\{\varone\}}{\termfour}{\tcrel{\relone}}{\termtwo}$
  hold. Then, by induction hypothesis on both of them, we have
  $\rel{\vecvarone}{\subst{\termone}{\varone}{\termthree}}{\tcrel{\relone}}{\subst{\termfour}{\varone}{\termthree}}$
  and
  $\rel{\vecvarone}{\subst{\termfour}{\varone}{\termthree}}{\tcrel{\relone}}{\subst{\termtwo}{\varone}{\termthree}}$. We
  conclude applying $\TCtwo$ on the latter two, obtaining
  $\rel{\vecvarone}{\subst{\termone}{\varone}{\termthree}}{\tcrel{\relone}}{\subst{\termtwo}{\varone}{\termthree}}$.
\end{proof}

It is important to note that the transitive closure of an already Howe's
lifted relation is a preorder if the starting relation is.

\begin{lemma}\label{lemma:tcrelPO}
  If a $\LOP$-relation $\relone$ is a preorder relation, then so is
  $\tcrel{(\howe{\relone})}$.
\end{lemma}
\begin{proof}
  We need to show $\tcrel{(\howe{\relone})}$ of being reflexive and transitive. Of course, being a transitive closure,
  $\tcrel{(\howe{\relone})}$ is a a transitive relation. Moreover, since
  $\relone$ is reflexive, by Lemma~\ref{lemma:howeprop1}, $\howe{\relone}$
  is reflexive too because compatible. Then, by Lemma~\ref{lemma:tcrelCom},
  so is $\tcrel{(\howe{\relone})}$.
\end{proof}
This is just the first half of the story: we also need
to prove that $\tcrel{(\howe{\cbnpas})}$ is  a
simulation. As we already know it is a preorder, the following lemma
gives us the missing bit:
\begin{lemma}[Key Lemma]\label{lemma:keylemma}
  If $\relu{\termone}{\howe{\cbnpas}}{\termtwo}$, then for every
  $\setone\subseteq\LOP(\varone)$ it holds that
  $\sem{\termone}(\abstr{\varone}{\setone})\leq\sem{\termtwo}(\abstr{\varone}{(\howe{\cbnpas}(\setone))})$.
\end{lemma} 
The proof of this lemma is delicate and is discussed in the next section.
From the lemma, using a standard argument we derive the needed
substitutivity results, and ultimately the most important result
of this section.
\begin{theorem}\label{thm:pasprecongrCBN}
  $\cbnpas$ is a precongruence relation for $\LOP$-terms.
\end{theorem}
\begin{proof}
  We prove the result by observing that $\tcrel{(\howe{\cbnpas})}$ is a
  precongruence and by showing that
  $\cbnpas=\tcrel{(\howe{\cbnpas})}$. First of all,
  Lemma~\ref{lemma:howeprop1} and Lemma~\ref{lemma:tcrelCom} ensure that
  $\tcrel{(\howe{\cbnpas})}$ is compatible and Lemma~\ref{lemma:tcrelPO}
  tells us that $\tcrel{(\howe{\cbnpas})}$ is a preorder. As a consequence,
  $\tcrel{(\howe{\cbnpas})}$ is a precongruence.  Consider now the
  inclusion $\cbnpas\subseteq\tcrel{(\howe{\cbnpas})}$.  By
  Lemma~\ref{lemma:howeprop3} and by definition of transitive closure
  operator $\tcrel{(\cdot)}$, it follows that
  $\cbnpas\,\subseteq\,(\howe{\cbnpas})\,\subseteq\,\tcrel{(\howe{\cbnpas})}$.
  We show the converse by proving that $\tcrel{(\howe{\cbnpas})}$ is
  included in a relation $\relone$ that is a call-by-name probabilistic
  applicative simulation, therefore contained in the largest one. In
  particular, since $\tcrel{(\howe{\cbnpas})}$ is closed under
  term-substitution (Lemma~\ref{lemma:closesubsCBN} and
  Lemma~\ref{lemma:tcrelCTS}), it suffices to show the latter only on the
  closed version of terms and cloned values. $\relone$ acts like
  $\tcrel{(\howe{\cbnpas})}$ on terms, while given two cloned values
  $\clabstr{\varone}{\termone}$ and $\clabstr{\varone}{\termtwo}$,
  $(\clabstr{\varone}{\termone})\relone(\clabstr{\varone}{\termtwo})$ iff
  $\termone\tcrel{(\howe{\cbnpas})}\termtwo$. Since we already know that
  $\tcrel{(\howe{\cbnpas})}$ is a preorder (and thus $\relone$ is itself a
  preorder), all that remain to be checked are the following two points:
  \begin{varitemize}
  \item If $\termone\tcrel{(\howe{\cbnpas})}\termtwo$, then for every
    $\setone\subseteq\LOPp{\varone}$ it holds that
    \begin{equation}\label{equ:probsimCBN}
      \translop(\termone,\evlabel,\clabstr{\varone}{\setone})\leq\translop(\termtwo,\evlabel,\relone(\clabstr{\varone}{\setone})).
    \end{equation}
    Let us proceed by induction on the structure of the proof of
    $\termone\tcrel{(\howe{\cbnpas})}\termtwo$:
    \begin{varitemize}
    \item The base case has $\TCone$ as last rule: we get that
      $\rel{\emptyset}{\termone}{\howe{\cbnpas}}{\termtwo}$ holds.  Then,
      in particular by Lemma~\ref{lemma:keylemma},
      \begin{align*}
        \translop(\termone,\evlabel,\clabstr{\varone}{\setone})&=\sem{\termone}(\abstr{\varone}{\setone})\\
        &\leq
        \sem{\termtwo}(\abstr{\varone}{\howe{\cbnpas}(\setone)})\\ &\leq\sem{\termtwo}(\abstr{\varone}{\tcrel{(\howe{\cbnpas})}(\setone)})\\
        &\leq\sem{\termtwo}(\relone(\clabstr{\varone}{\setone}))=\translop(\termtwo,\evlabel,\relone(\clabstr{\varone}{\setone})).
      \end{align*}
    \item If $\TCtwo$ is the last rule used, we obtain that, for some
      $\termfour\in\LOP(\emptyset)$,
      $\rel{\emptyset}{\termone}{\tcrel{(\howe{\cbnpas})}}{\termfour}$ and
      $\rel{\emptyset}{\termfour}{\tcrel{(\howe{\cbnpas})}}{\termtwo}$
      hold. Then, by induction hypothesis, we get
      \begin{align*}
        \translop(\termone,\evlabel,\setone)&\leq\translop(\termfour,\evlabel,\relone(\setone)),\\
        \translop(\termfour,\evlabel,\relone(\setone))&\leq\translop(\termtwo,\evlabel,\relone(\relone(\setone))).
      \end{align*}
      But of course $\relone(\relone(\setone))\subseteq\relone(\setone)$,
      and as a consequence:
      $$
      \translop(\termone,\evlabel,\setone)\leq\translop(\termtwo,\evlabel,\relone(\setone))
      $$
      and (\ref{equ:probsimCBN}) is satisfied.
    \end{varitemize}
  \item If $\termone\tcrel{(\howe{\cbnpas})}\termtwo$, then for every
    $\termthree\in\LOPp{\emptyset}$ and for every
    $\setone\subseteq\LOPp{\emptyset}$ it holds that
    $$
    \translop(\clabstr{\varone}{\termone},\termthree,\setone)\leq\translop(\clabstr{\varone}{\termtwo},\termthree,\relone(\setone)).
    $$
    But if $\termone\tcrel{(\howe{\cbnpas})}\termtwo$, then
    $\subst{\termone}{\varone}{\termthree}\tcrel{(\howe{\cbnpas})}\subst{\termtwo}{\varone}{\termthree}$.
    This is means that whenever
    $\subst{\termone}{\varone}{\termthree}\in\setone$,
    $\subst{\termtwo}{\varone}{\termthree}\in\howe{\cbnpas}(\setone)\subseteq\tcrel{(\howe{\cbnpas})}(\setone)$
    and ultimately
    \begin{align*}
      \translop(\clabstr{\varone}{\termone},\termthree,\setone)&=1\\ 
      &=\translop(\clabstr{\varone}{\termtwo},\termthree,\tcrel{(\howe{\cbnpas})}(\setone))\\
      &=\translop(\clabstr{\varone}{\termtwo},\termthree,\relone(\setone)).
    \end{align*}
    If $\subst{\termone}{\varone}{\termthree}\notin\setone$, on the other
    hand,
    $$
    \translop(\clabstr{\varone}{\termone},\termthree,\setone)=0\leq\translop(\clabstr{\varone}{\termtwo},\termthree,\relone(\setone)).
    $$
  \end{varitemize}
  This concludes the proof.
\end{proof}

\begin{corollary}\label{cor:pabcongrCBN}
  $\cbnpab$ is a congruence relation for $\LOP$-terms.
\end{corollary}
\begin{proof}
  $\cbnpab$ is an equivalence relation by definition, in particular a
  symmetric relation. Since $\cbnpab = \cbnpas \cap \cbnpas^{op}$ by
  Proposition~\ref{prop:pab=pascopas}, $\cbnpab$ is also compatible as a
  consequence of Theorem~\ref{thm:pasprecongrCBN}.
\end{proof}

\subsection{Proof of the Key Lemma}
As we have already said, Lemma~\ref{lemma:keylemma} is indeed a crucial step towards showing that probabilistic 
applicative simulation is a precongruence.
Proving the Key Lemma~\ref{lemma:keylemma} turns out to be much more difficult than for deterministic
or nondeterministic cases. In particular, the case when $\termone$ is an application relies on another technical lemma we
are now going to give, which itself can be proved by tools from linear programming.

The combinatorial problem we will face while proving the Key Lemma can actually be decontextualized and
understood independently. Suppose we have $\natone=3$ non-disjoint sets 
$\setone_1,\setone_2,\setone_3$ whose elements are labelled with real numbers.
As an example, we could be in a situation like the one in Figure~\ref{fig:vennent} (where for the
sake of simplicity only the labels are indicated).
\begin{figure*}
  \centering
  \subfigure[]
  {\includegraphics[scale=0.8]{venndiagram1}\label{fig:vennent}}
  \hspace{20mm}
  \subfigure[]
  {\includegraphics[scale=0.8]{venndiagram2}\label{fig:venndisent}}
  \caption{Disentangling Sets}\label{fig:venn}
\end{figure*}
\newcommand{\meas}[1]{||#1||} 
We fix three real numbers $\probone_1\defi\frac{5}{64}$, $\probone_2\defi\frac{3}{16}$,
$\probone_3\defi\frac{5}{64}$. It is routine to check that for every
$\indsetone\subseteq\{1,2,3\}$ it holds that
$$
\sum_{i\in\indsetone}\probone_i\leq\meas{\bigcup_{i\in\indsetone}\setone_i},
$$
where $\meas{\setone}$ is  the sum of the labels of the elements of $\setone$.
Let us observe that it is of course possible to turn the three sets $\setone_1,\setone_2,\setone_3$
into three disjoint sets $\settwo_1$, $\settwo_2$ and $\settwo_3$ where
each $\settwo_i$ contains (copies of) the elements of $\setone_i$ whose labels, 
however, are obtained by splitting the ones of the original elements. Examples
of those sets are in Figure~\ref{fig:venndisent}: if you superpose the
three sets, you obtain the Venn diagram we started from. Quite remarkably, however,
the examples from Figure~\ref{fig:venn} have an additional property, namely
that for every $i\in\{1,2,3\}$ it holds that $\probone_i\leq\meas{\settwo_i}$.
We now show that finding sets satisfying the properties above is always possible, 
even when $\natone$ is arbitrary.

Suppose $\probone_1,\ldots,\probone_n\in\RRN_{[0,1]}$, and suppose that for
each $\indsetone\subseteq\{1,\ldots,n\}$ a real number
$\realone_\indsetone\in\RRN_{[0,1]}$ is defined such that for every such
$\indsetone$ it holds that
$\sum_{i\in\indsetone}\probone_i\leq\sum_{\indsettwo\cap\indsetone\neq\emptyset}
\realone_\indsettwo\leq 1$. Then $(\{\probone_i\}_{1\leq i\leq
  n},\{\realone_\indsetone\}_{\indsetone\subseteq\{1,\ldots,n\}})$ is said
to be a \emph{probability assignment} for $\{1,\ldots,n\}$.
Is it always possible to ``disentangle'' probability assignments? The
answer is positive. 

The following is a formulation of Max-Flow-Min-Cut Theorem:
\begin{theorem}[Max-Flow-Min-Cut]\label{t:mf-mc}
  For any flow network, the value of the maximum flow is equal to the
  capacity of the minimum cut.
\end{theorem}
\begin{lemma}[Disentangling Probability Assignments]\label{lemma:disentangling}
  Let $\passone\defi(\{\probone_i\}_{1\leq i\leq n},\{\realone_\indsetone\}_{\indsetone\subseteq\{1,\ldots,n\}})$
  be a probability assignment.
  Then for every nonempty $\indsetone\subseteq\{1,\ldots,n\}$ and
  for every $k\in\indsetone$ there is $\realtwo_{k,\indsetone}\in\RRN_{[0,1]}$ such that
  the following conditions all hold:
  \begin{varenumerate}
  \item\label{point:first}
      for every $\indsetone$, it holds that $\sum_{k\in\indsetone}\realtwo_{k,\indsetone}\leq 1$;
    \item\label{point:second}
      for every $k\in\{1,\ldots,n\}$, it holds that $\probone_k\leq\sum_{k\in\indsetone}
      \realtwo_{k,\indsetone}\cdot\realone_\indsetone$.
  \end{varenumerate}
\end{lemma}
\begin{proof}
  For every probability assignment $\passone$, let us define the \emph{flow
    network of $\passone$} as the digraph $\fnet{\passone}$ where:
  \begin{varitemize}
  \item 
    $\vt{\passone}\defi(\pow{\{1,\dots,n\}}-\emptyset)\cup\{s,t\}$, where $s,t$ are a
    distinguished source and target, respectively;
  \item 
    $\ed{\passone}$ is composed by three kinds of edges:
    \begin{varitemize}
    \item $(s,\{i\})$ for every $i\in\{1,\dots,n\}$, with an assigned
      capacity of $\probone_i$;
    \item $(\indsetone,\indsetone\cup\{i\})$, for every nonempty
      $\indsetone\subseteq\{1,\dots,n\}$ and $i\not\in\indsetone$, with an
      assigned capacity of $1$;
    \item $(\indsetone,t)$, for every nonempty
      $\indsetone\subseteq\{1,\dots,n\}$, with an assigned capacity of
      $\realone_\indsetone$.
    \end{varitemize}
  \end{varitemize}
  We prove the following two lemmas on $\nt{\passone}$ which
  together entail the result.
  \begin{varitemize}
  \item
    \begin{lemma}
      If $\nt{\passone}$ admits a flow summing to
      $\sum_{i\in\{1,\dots,n\}}\probone_i$, then the $\realtwo_{k,\indsetone}$
      exist for which conditions \ref{point:first}. and
      \ref{point:second}. hold.
    \end{lemma}
    \begin{proof}
      Let us fix $\probone\defi\sum_{i\in\{1,\dots,n\}}\probone_i$. The idea
      then is to start with a flow of value $\probone$ in input to the source
      $s$, which by hypothesis is admitted by $\nt{\passone}$ and the maximum
      one can get, and split it into portions going to singleton vertices
      $\{i\}$, for every $i\in\indsetone$, each of value
      $\probone_i$. Afterwards, for every other vertex
      $\indsetone\subseteq\{1,\dots,n\}$, values of flows on the incoming
      edges are summed up and then distributed to the outgoing adges as one
      wishes, thanks to conservation property of the flow. Formally, a flow
      ${\flone}:\ed{\probone}\rightarrow\RRN_{[0,1]}$ is turned into a
      function $\overline{\flone}:\ed{\probone}\rightarrow(\RRN_{[0,1]})^n$
      defined as follows:
\begin{varitemize}
      \item For every $i\in\{1,\dots,n\}$,
        $\overline{\flone}_{(s,\{i\})}\defi(0,\dots,\flone_{(s,\{i\})},\dots,0)$,
        where the only possibly nonnull component is exactly the $i$-th;
      \item For every nonempty $\indsetone\subseteq\{1,\dots,n\}$, as soon as
        $\overline{\flone}$ has been defined on all ingoing edges of
        $\indsetone$, we can define it on all its outgoing ones, by just
        splitting each component as we want. This is possible, of course,
        because $\flone$ is a flow and, as such, ingoing and outgoing values
        are the same. More formally, let us fix
        $\overline{\flone}_{(*,\indsetone)}\defi\sum_{\indsetthree\subseteq\{1,\dots,n\}}\overline{\flone}_{(\indsetthree,\indsetone)}$
        and indicate with $\overline{\flone}_{(*,\indsetone),k}$ its $k$-th
        component. Then, for every $i\not\in\indsetone$, we set
        $\overline{\flone}_{(\indsetone,\indsetone\cup\{i\})}\defi(\ratioone_{1,i}\cdot\overline{\flone}_{(*,\indsetone),1},\dots,\ratioone_{n,i}\cdot
        \overline{\flone}_{(*,\indsetone),n})$ where, for every
        $j\in\{1,\dots,n\}$, $\ratioone_{j,i}\in\RRN_{[0,1]}$ are such that
        $\sum_{i\not\in\indsetone}
        \ratioone_{j,i}\cdot\overline{\flone}_{(*,\indsetone),j}=\overline{\flone}_{(*,\indsetone),j}$
        and
        $\sum_{j=1}^n\ratioone_{j,i}\cdot\overline{\flone}_{(*,\indsetone),j}=
        \flone_{(\indsetone,\indsetone\cup\{i\})}$. Of course, a similar
        definition can be given to $\overline{\flone}_{(\indsetone,t)}$, for
        every nonempty $\indsetone\subseteq\{1,\dots,n\}$.
      \end{varitemize}
      Notice that, the way we have just defined $\overline{\flone}$
      guarantees that the sum of all components of $\overline{\flone}_\edone$
      is always equal to $\flone_\edone$, for every
      $\edone\in\ed{\passone}$. Now, for every nonempty
      $\indsetone\subseteq\{1,\dots,n\}$, fix $\realtwo_{k,\indsetone}$ to be
      the ratio $\ratioone_k$ of $\overline{\flone}_{(\indsetone,t)}$; i.e.,
      the $k$-th component of $\overline{\flone}_{(\indsetone,t)}$ (or $0$ if
      the first is itself $0$). On the one hand, for every nonempty
      $\indsetone\subseteq\{1,\dots,n\}$,
      $\sum_{k\in\indsetone}\realtwo_{k,\indsetone}$ is obviously less or
      equal to $1$, hence condition \ref{point:first}. holds. On the other,
      each component of $\overline{\flone}$ is itself a flow, since it
      satisfies the capacity and conservation constraints. Moreover,
      $\nt{\passone}$ is structured in such a way that the $k$-th component
      of $\overline{\flone}_{(\indsetone,t)}$ is $0$ whenever
      $k\not\in\indsetone$. As a consequence, since $\overline{\flone}$
      satisfies the capacity constraint, for every $k\in\{1,\dots,n\}$,
      $$
      \probone_k \leq \sum_{k\in\indsetone}
      \realtwo_{k,\indsetone}\cdot\overline{\flone}_{(\indsetone,t)} \leq
      \sum_{k\in\indsetone} \realtwo_{k,\indsetone}\cdot\realone_\indsetone
      $$
      and so condition \ref{point:second}.  holds too.
    \end{proof}
  \item
    \begin{lemma}
      $\nt{\passone}$ admits a flow summing to
      $\sum_{i\in\indsetone}\probone_i$.
    \end{lemma}
    \begin{proof}
      We prove the result by means of Theorem~\ref{t:mf-mc}. In
      particular, we just prove that the capacity of any cut must be at least
      $\probone\defi\sum_{i\in\{1,\dots,n\}}\probone_i$.
      A cut $(\cutsone,\cuttone)$ is said to be \emph{degenerate}
      if there are $\indsetone\subseteq\{1,\dots,n\}$ and $i\in\{1,\dots,n\}$
      such that $\indsetone\in\cutsone$ and
      $\indsetone\cup\{i\}\in\cuttone$. It is easy to verify that every
      degenerate cut has capacity greater or equal to $1$, thus greater or
      equal to $\probone$. As a consequence, we can just concentrate on
      non-degenerate cuts and prove that all of them have capacity at least
      $\probone$. Given two cuts $\cutone\defi(\cutsone,\cuttone)$ and
      $\cuttwo\defi(\cutstwo,\cutttwo)$, we say that $\cutone\leq\cuttwo$ iff
      $\cutsone\leq\cutstwo$. Then, given $\indsetone\subseteq\{1,\dots,n\}$,
      we call $\indsetone$-cut any cut $(\cutsone,\cuttone)$ such that
      $\bigcup_{\{i\}\in\cutsone}\{i\}=\indsetone$. The \emph{canonical}
      $\indsetone$-cut is the unique $\indsetone$-cut
      $\ccut{\indsetone}\defi(\cutsone,\cuttone)$ such that
      $\cutsone=\{s\}\cup\{\indsettwo\subseteq\{1,\dots,n\}\,|\,\indsettwo\cap\indsetone\neq\emptyset\}$. Please
      observe that, by definition, $\ccut{\indsetone}$ is non-degenerate
      and that the capacity $\ce{\ccut{\indsetone}}$ of $\ccut{\indsetone}$
      is at least $\probone$, because the forward edges in
      $\ccut{\indsetone}$ (those connecting elements of $\cutsone$ to those
      of $\cuttone$) are those going from $s$ to the singletons not in
      $\cutsone$, plus the edges going from any $\indsettwo\in\cutsone$ to
      $t$. The sum of the capacities of such edges are greater or equal to
      $\probone$ by hypothesis.
      We now need to prove the following two lemmas.
      \begin{varitemize}
      \item
        \begin{lemma}\label{l:helper1}
          For every non-degenerate $\indsetone$-cuts $\cutone,\cuttwo$ such
          that $\cutone>\cuttwo$, there is a non-degenerate $\indsetone$-cut
          $\cutthree$ such that $\cutone\geq\cutthree>\cuttwo$ and
          $\ce{\cutthree}\geq\ce{\cuttwo}$.
        \end{lemma}
        \begin{proof}
          Let $\cutone\defi(\cutsone,\cuttone)$ and
          $\cuttwo\defi(\cutstwo,\cutttwo)$. Moreover, let $\indsettwo$ be any
          element of $\cutsone\mathord{\setminus}\cutstwo$. Then, consider
          $\cutthree\defi(\cutstwo\cup\{\indsetthree\subseteq\{1,\dots,n\}\,|\,
          \indsettwo\subseteq\indsetthree\},\cutttwo\mathord{\setminus}\{\indsetthree\subseteq\{1,\dots,n\}\,|\,\indsettwo\subseteq\indsetthree\})$
          and verify that $\cutthree$ is the cut we are looking for. Indeed,
          $\cutthree$ is non-degenerate because it is obtained from $\cuttwo$,
          which is non-degenerate by hypothesis, by adding to it $\indsettwo$
          and all its supersets. Of course, $\cutthree>\cuttwo$. Moreover,
          $\cutone\geq\cutthree$ holds since $\indsettwo\in\cutsone$ and
          $\cutone$ is non-degenerate, which implies $\cutone$ contains all
          supersets of $\indsettwo$ as well. It is also easy to check that
          $\ce{\cutthree}\geq\ce{\cuttwo}$. In fact, in the process of
          constructing $\cutthree$ from $\cuttwo$ we do not lose any forward
          edges coming from $s$, since $\indsettwo$ cannot be a singleton with
          $\cutone$ and $\cuttwo$ both $\indsetone$-cuts, or any other edge
          coming from some element of $\cutstwo$, since $\cuttwo$ is
          non-degenerate.
        \end{proof}
      \item
        \begin{lemma}\label{l:helper2}
          For every non-degenerate $\indsetone$-cuts $\cutone,\cuttwo$ such
          that $\cutone\geq\cuttwo$, $\ce{\cutone}\geq\ce{\cuttwo}$.
        \end{lemma}
        \begin{proof}
          Let $\cutone\defi(\cutsone,\cuttone)$ and
          $\cuttwo\defi(\cutstwo,\cutttwo)$. We prove the result by induction
          on the $n\defi\card{\cutsone}-\card{\cutstwo}$. If $n=0$, then
          $\cutone=\cuttwo$ and the thesis follows. If $n>0$, then
          $\cutone>\cuttwo$ and, by Lemma~\ref{l:helper1}, there is a
          non-degenerate $\indsetone$-cut $\cutthree$ such that
          $\cutone\geq\cutthree>\cuttwo$ and
          $\ce{\cutthree}\geq\ce{\cuttwo}$. By induction hypothesis on
          $\cutone$ and $\cutthree$, it follows that
          $\ce{\cutone}\geq\ce{\cutthree}$. Thus,
          $\ce{\cutone}\geq\ce{\cuttwo}$.
        \end{proof}
      \end{varitemize}
      The two lemmas above permit to conclude. Indeed, for every
      non-degenerate cut $\cuttwo$, there is of course a $\indsetone$ such
      that $\cuttwo$ is a $\indsetone$-cut (possibly with $\indsetone$ as the
      empty set). Then, let us consider the canonical $\ccut{\indsetone}$. On
      the one hand, $\ce{\ccut{\indsetone}}\geq\probone$. On the other, since
      $\ccut{\indsetone}$ is non-degenerate,
      $\ce{\cuttwo}\geq\ce{\ccut{\indsetone}}$ by
      Lemma~\ref{l:helper2}. Hence, $\ce{\cuttwo}\geq\probone$.
    \end{proof}
  \end{varitemize}
  This concludes the main proof.\end{proof}
 
In the coming proof of Lemma~\ref{lemma:keylemma} we will widely, and often
implicitly, use the following technical Lemmas. We denote with
$\clabstr{\varone}\cbnpas(\setone)$ the set of distinguished values
$\{\clabstr{\varone}{\termone}\,|\,\exists\termtwo\in\setone.\,\termtwo\cbnpas\termone\}$.
\begin{lemma}\label{lemma:pascommval}
  For every $\setone\subseteq\LOPp{\varone}$,
  $\cbnpas(\clabstr{\varone}{\setone})=\clabstr{\varone}\cbnpas(\setone)$.
\end{lemma}
\begin{proof}
  \begin{align*}
    \clabstr{\varone}{\termone}\in\cbnpas(\clabstr{\varone}{\setone})&\Leftrightarrow
    \exists\termtwo\in\setone.\,\clabstr{\varone}{\termtwo}\cbnpas\clabstr{\varone}{\termone}\\
    &\Leftrightarrow\exists\termtwo\in\setone.\,\termtwo\cbnpas\termone\\
    &\Leftrightarrow\clabstr{\varone}{\termone}\in\clabstr{\varone}{\cbnpas(\setone)}.
  \end{align*}
  This concludes the proof.
\end{proof}

\begin{lemma}\label{lemma:pascomm}
  If $\termone\cbnpas\termtwo$, then for every $\setone\in\LOPp{\varone}$,
  $\sem{\termone}(\abstr{\varone}{\setone})\leq\sem{\termtwo}(\abstr{\varone}{\cbnpas(\setone)})$.
\end{lemma}
\begin{proof}
  If $\termone\cbnpas\termtwo$, then by definition
  $\sem{\termone}(\clabstr{\varone}{\setone})\leq\sem{\termtwo}(\cbnpas(\clabstr{\varone}{\setone}))$.
  Therefore, by Lemma~\ref{lemma:pascommval},
  $\sem{\termtwo}(\cbnpas(\clabstr{\varone}{\setone}))\leq\sem{\termtwo}(\clabstr{\varone}\cbnpas(\setone))$.
\end{proof}
\begin{remark}\label{r:latetech}
  Throughout the following proof we will implicitly use a routine result
  stating that $\termone\cbnpas\termtwo$ implies
  $\sem{\termone}(\abstr{\varone}{\setone})\leq\sem{\termtwo}(\abstr{\varone}{\mathord{\cbnpas}(\setone)})$,
  for every $\setone\subseteq\LOPp{\varone}$. The property
  needed by the latter is precisely the reason why we have  formulated
  $\LOP$ as a multisorted labelled Markov chain: 
  $\mathord{\cbnpas}(\clabstr{\varone}{\setone})$ consists of distinguished
  values only, and is nothing but
  $\clabstr{\varone}{\mathord{\cbnpas}(\setone)}$.
\end{remark}

\begin{proof}[of Lemma~\ref{lemma:keylemma}]
  This is equivalent to proving that if $\relu{\termone}{\howe{\cbnpas}}{\termtwo}$, then 
  for every $\setone\subseteq\LOP(\varone)$ the following implication holds:
  if $\ibsemn{\termone}{\distone}$, then 
  $\distone(\abstr{\varone}{\setone})\leq\sem{\termtwo}(\abstr{\varone}{(\howe{\cbnpas}(\setone)}))$.
  This is an induction on the structure of the proof of $\ibsemn{\termone}{\distone}$.
  \begin{varitemize}
    \item
      If $\distone=\emdist$, then of course
      $\distone(\abstr{\varone}{\setone})=0\leq\sem{\termtwo}(\abstr{\varone}{\settwo})$ for every $\setone,\settwo\subseteq\LOP(\varone)$.
    \item
      If $\termone$ is a value $\abstr{\varone}{\termthree}$ and $\distone(\abstr{\varone}{\termthree})=1$, then the
      proof of $\relu{\termone}{\howe{\cbnpas}}{\termtwo}$ necessarily ends as follows:
      $$
      \infer
          {\rel{\emcon}{\abstr{\varone}{\termthree}}{\howe{\cbnpas}}{\termtwo}}
          {
          \rel{\{\varone\}}{\termthree}{\howe{\cbnpas}}{\termfour}
          &&
          \rel{\emcon}{\abstr{\varone}{\termfour}}{\cbnpas}{\termtwo}
          }
      $$
      Let $\setone$ be any subset of $\LOP(\varone)$. Now, if
      $\termthree\not\in\setone$, then
      $\distone(\abstr{\varone}{\setone})=0$ and the inequality trivially
      holds. If, on the contrary, $\termthree\in\setone$, then
      $\termfour\in\howe{\cbnpas}(\setone)$. Consider
      $\cbnpas(\termfour)$, the set of terms that are in relation
      with $\termfour$ via $\cbnpas$. 
      We have that for every $\termfive\in\;\cbnpas(\termfour)$, both
      $\rel{\{\varone\}}{\termthree}{\howe{\cbnpas}}{\termfour}$ and
      $\rel{\{\varone\}}{\termfour}{\cbnpas}{\termfive}$ hold, and as a
      consequence $\rel{\{\varone\}}{\termthree}{\howe{\cbnpas}}{\termfive}$ does
      (this is a consequence of a property of $\howe{(\cdot)}$, see~\cite{EV}).  
      In other words, $\cbnpas(\termfour)\subseteq\howe{\cbnpas}(\setone)$.
      But then, by Lemma~\ref{lemma:pascomm},
      $$
      \sem{\termtwo}(\abstr{\varone}{\howe{\cbnpas}(\setone)})\geq
      \sem{\termtwo}(\abstr{\varone}{\cbnpas(\termfour)})\geq
      \sem{\abstr{\varone}{\termfour}}(\abstr{\varone}{\termfour})=1.
      $$
    \item
      If $\termone$ is an application $\termthree\termfour$, then $\ibsemn{\termone}{\distone}$ is obtained
      as follows:
      $$
      \infer
          {\ibsemn{\termthree\termfour}{\sum_{\termfive}}\distthree(\abstr{\varone}{\termfive})\cdot\distfive_{\termfive,\termfour}}
          { \ibsemn{\termthree}{\distthree} &&
            \{\ibsemn{\subst{\termfive}{\varone}{\termfour}}{\distfive_{\termfive,\termfour}}\}_{\termfive,\termfour}
          }
          $$
      Moreover, the proof of
      $\rel{\emptyset}{\termone}{\howe{\cbnpas}}{\termtwo}$ must end as
      follows:
      $$
      \infer
          {\rel{\emcon}{\app{\termthree}{\termfour}}{\howe{\cbnpas}}{\termtwo}}
          { \rel{\emcon}{\termthree}{\howe{\cbnpas}}{\termsix} &&
            \rel{\emcon}{\termfour}{\howe{\cbnpas}}{\termseven} &&
          \rel{\emcon}{\app{\termsix}{\termseven}}{\cbnpas}{\termtwo} }
      $$
      Now, since $\ibsemn{\termthree}{\distthree}$ and
      $\rel{\emcon}{\termthree}{\howe{\cbnpas}}{\termsix}$, by induction
      hypothesis we get that for every $\settwo\subseteq\LOP(\varone)$ it
      holds that
      $\distthree(\abstr{\varone}{\settwo})\leq\sem{\termsix}(\abstr{\varone}{\howe{\cbnpas}(\settwo)})$.
      Let us now take a look at the distribution
      $$
      \distone=\sum_{\termfive}\distthree(\abstr{\varone}{\termfive})\cdot\distfive_{\termfive,\termfour}.
      $$
      Since $\distthree$ is a \emph{finite} distribution, the sum above
      is actually the sum of finitely many summands.  Let the support
      $\supp{\distthree}$ of $\distthree$ be
      $\{\abstr{\varone}{\termfive_1},\ldots,\abstr{\varone}{\termfive_n}\}$. 
      It is now time to put the above into a form that is
      amenable to treatment by Lemma~\ref{lemma:disentangling}. Let us
      consider the $n$ sets
      $\howe{\cbnpas}(\termfive_1),\ldots,\howe{\cbnpas}(\termfive_n)$;
      to each term $\termnine$ in them we can associate the probability
      $\sem{\termsix}(\abstr{\varone}{\termnine})$. We are then in the
      scope of Lemma~\ref{lemma:disentangling}, since by induction
      hypothesis we know that for every $\settwo\subseteq\LOP(\varone)$,
      $$
      \distthree(\abstr{\varone}{\setone})\leq\sem{\termsix}(\abstr{\varone}{\howe{\cbnpas}(\setone)}).
      $$
      We can then conclude that for every 
      \begin{align*}
      \termnine\in\howe{\cbnpas}(\{\termfive_1,\ldots,\termfive_n\})=\bigcup_{1\leq
        i\leq n}\howe{\cbnpas}(\termfive_i)
      \end{align*}
      there are $n$ real numbers $\realone_1^{\termnine,\termsix},\ldots,\realone_n^{\termnine,\termsix}$ such that:
      \begin{align*}
        \sem{\termsix}(\abstr{\varone}{\termnine})&\geq
        \sum_{1\leq i\leq n}\realone_i^{\termnine,\termsix}\qquad\forall\,\termnine\in\bigcup_{1\leq i\leq n}\howe{\cbnpas}(\termfive_i);\\
        \distthree(\abstr{\varone}{\termfive_i})&\leq\sum_{\termnine\in\howe{\cbnpas}(\termfive_i)}\realone_i^{\termnine,\termsix}\qquad\forall\,1\leq i\leq n.
      \end{align*}
      So, we can conclude that
      \begin{align*}
        \distone&\leq\sum_{1\leq i\leq
          n}\left(\sum_{\termnine\in\howe{\cbnpas}(\termfive_i)}\realone_i^{\termnine,\termsix}\right)\cdot
        \distfive_{\termfive_i,\termfour}\\
        &=\sum_{1\leq i\leq
          n}\sum_{\termnine\in\howe{\cbnpas}(\termfive_i)}\realone_i^{\termnine,\termsix}\cdot\distfive_{\termfive_i,\termfour}.
      \end{align*}
      Now, whenever $\termfive_i\howe{\cbnpas}\termnine$ and $\termfour\howe{\cbnpas}\termseven$,
      we know that, by Lemma~\ref{lemma:closesubsCBN}, 
      $\subst{\termfive_i}{\varone}{\termfour}\howe{\cbnpas}\subst{\termnine}{\varone}{\termseven}$. We
      can then apply the inductive hypothesis to the $n$ derivations of
      $\ibsemn{\subst{\termfive_i}{\varone}{\termfour}}{\distfive_{\termfive_i,\termfour}}$,
      obtaining that, for every $\setone\subseteq\LOP(\varone)$,
      {\footnotesize
        \begin{align*}
          &\distone(\abstr{\varone}{\setone})\leq\sum_{1\leq i\leq
            n}\sum_{\termnine\in\howe{\cbnpas}(\termfive_i)}
          \realone_i^{\termnine,\termsix}\cdot\sem{\subst{\termnine}{\varone}{\termseven}}(\abstr{\varone}{\howe{\cbnpas}{(\setone)}})\\
          &\leq\sum_{1\leq i\leq
            n}\sum_{\termnine\in\howe{\cbnpas}(\{\termfive_1,\ldots,\termfive_n\})}\realone_i^{\termnine,\termsix}\cdot
            \sem{\subst{\termnine}{\varone}{\termseven}}(\abstr{\varone}{\howe{\cbnpas}{(\setone)}})\\
          &=\sum_{\termnine\in\howe{\cbnpas}(\{\termfive_1,\ldots,\termfive_n\})}\sum_{1\leq
            i\leq
            n}\realone_i^{\termnine,\termsix}\cdot\sem{\subst{\termnine}{\varone}{\termseven}}(\abstr{\varone}{\howe{\cbnpas}{(\setone)}})\\
          &=\sum_{\termnine\in\howe{\cbnpas}(\{\termfive_1,\ldots,\termfive_n\})}\left(\sum_{1\leq
              i\leq n}\realone_i^{\termnine,\termsix}\right)\cdot\sem{\subst{\termnine}{\varone}{\termseven}}(\abstr{\varone}{\howe{\cbnpas}{(\setone)}})\\
          &\leq\sum_{\termnine\in\howe{\cbnpas}(\{\termfive_1,\ldots,\termfive_n\})}
          \sem{\termsix}(\abstr{\varone}{\termnine})\cdot
          \sem{\subst{\termnine}{\varone}{\termseven}}(\abstr{\varone}{\howe{\cbnpas}{(\setone)}})\\
          &\leq\sum_{\termnine\in\LOP(\varone)}
          \sem{\termsix}(\abstr{\varone}{\termnine})\cdot
          \sem{\subst{\termnine}{\varone}{\termseven}}(\abstr{\varone}{\howe{\cbnpas}{(\setone)}})\\
          &=\sem{\termsix\termseven}(\abstr{\varone}{\howe{\cbnpas}{(\setone)}})\leq
          \sem{\termtwo}(\abstr{\varone}{\cbnpas((\howe{\cbnpas})(\setone))})\\
          &\leq\sem{\termtwo}(\abstr{\varone}{\howe{\cbnpas}(\setone)}),
          \end{align*}
        }
        which is the thesis. 
      \item
        If $\termone$ is a probabilistic sum
        $\ps{\termthree}{\termfour}$, then $\ibsemn{\termone}{\distone}$ is
        obtained as follows:
        $$
        \infer
        {\ibsemn{\ps{\termthree}{\termfour}}{\frac{1}{2}\cdot\distthree
            + \frac{1}{2}\cdot\distfour}} {
          \ibsemn{\termthree}{\distthree} && \ibsemn{\termfour}{\distfour}
        }
        $$
        
        Moreover, the proof of
        $\rel{\emptyset}{\termone}{\howe{\cbnpas}}{\termtwo}$ must end as
        follows:
        $$
        \infer
        {\rel{\emcon}{\ps{\termthree}{\termfour}}{\howe{\cbnpas}}{\termtwo}}
        {
          \rel{\emcon}{\termthree}{\howe{\cbnpas}}{\termsix}
          &&
          \rel{\emcon}{\termfour}{\howe{\cbnpas}}{\termseven}
          &&
          \rel{\emcon}{\ps{\termsix}{\termseven}}{\cbnpas}{\termtwo}
        }
        $$
        Now:
        \begin{varitemize}
        \item
          Since $\ibsemn{\termthree}{\distthree}$ and $\rel{\emcon}{\termthree}{\howe{\cbnpas}}{\termsix}$, by induction hypothesis
          we get that for every $\settwo\subseteq\LOP(\varone)$ it holds that 
          $\distthree(\abstr{\varone}{\settwo})\leq\sem{\termsix}(\abstr{\varone}{\howe{\cbnpas}(\settwo)})$;
        \item
          Similarly, since $\ibsemn{\termfour}{\distfour}$ and $\rel{\emcon}{\termfour}{\howe{\cbnpas}}{\termseven}$, by induction hypothesis
          we get that for every $\settwo\subseteq\LOP(\varone)$ it holds that 
          $\distfour(\abstr{\varone}{\settwo})\leq\sem{\termseven}(\abstr{\varone}{\howe{\cbnpas}(\settwo)})$. 
        \end{varitemize}
        Let us now take a look at the distribution 
        $$
        \distone=\frac{1}{2}\cdot\distthree
        + \frac{1}{2}\cdot\distfour.
        $$
        The idea then is to prove that, for every
        $\setone\subseteq\LOP(\varone)$, it holds
        $\distone(\abstr{\varone}{\setone})\leq\sem{\ps{\termsix}{\termseven}}(\abstr{\varone}{\howe{\cbnpas}(\setone)})$.
        In fact, since $\sem{\ps{\termsix}{\termseven}}(\abstr{\varone}{\howe{\cbnpas}(\setone)})\leq\sem{\termtwo}(\abstr{\varone}{\howe{\cbnpas}(\setone)})$,
        the latter would imply the thesis
        $\distone(\abstr{\varone}{\setone})\leq\sem{\termtwo}(\abstr{\varone}{\howe{\cbnpas}(\setone)})$.
        But by induction hypothesis and
        Lemma~\ref{lemma:semsumCBN}:
        \begin{align*}
          \distone(\abstr{\varone}{\setone})&=\frac{1}{2}\cdot\distthree(\abstr{\varone}{\setone})
          +
          \frac{1}{2}\cdot\distfour(\abstr{\varone}{\setone})\\
          &\leq
          \frac{1}{2}\cdot\sem{\termsix}(\abstr{\varone}{\howe{\cbnpas}(\setone)})
          +
          \frac{1}{2}\cdot\sem{\termseven}(\abstr{\varone}{\howe{\cbnpas}(\setone)})\\
          &=
          \sem{\ps{\termsix}{\termseven}}(\abstr{\varone}{\howe{\cbnpas}(\setone)}).
        \end{align*}
      \end{varitemize}
      This concludes the proof.
\end{proof}

\subsection{Context Equivalence}\label{sec:pabce}
We now formally introduce probabilistic context equivalence and prove it
to be coarser than probabilistic applicative bisimilarity.

\begin{definition}\label{def:context}
  A $\LOP$-term context is a syntax tree with a unique ``hole'' $\ctxhole{\cdot}$, generated as follows:
  $$
  \ctxone,\ctxtwo\in\ctxset \, ::= \, \ctxhole{\cdot}\, |\,
  \abstr{\varone}{\ctxone}\, |\, \app{\ctxone}{\termone}\, |\,
  \app{\termone}{\ctxone}\, |\, \ps{\ctxone}{\termone}\, |\,
  \ps{\termone}{\ctxone}.
  $$
We denote with $\ctxone\ctxhole{\termtwo}$ the $\LOP$-term that results
from filling the hole with a $\LOP$-term $\termtwo$:
\begin{align*}
  \ctxhole{\cdot}\ctxhole{\termtwo} &\defi \termtwo;\\
  (\abstr{\varone}{\ctxone})\ctxhole{\termtwo} &\defi
  \abstr{\varone}{\ctxone\ctxhole{\termtwo}};\\
  (\app{\ctxone}{\termone})\ctxhole{\termtwo} &\defi
  \app{\ctxone\ctxhole{\termtwo}}{\termone};\\
  (\app{\termone}{\ctxone})\ctxhole{\termtwo} &\defi
  \app{\termone}{\ctxone\ctxhole{\termtwo}};\\
  (\ps{\ctxone}{\termone})\ctxhole{\termtwo} &\defi
  \ps{\ctxone\ctxhole{\termtwo}}{\termone};\\
  (\ps{\termone}{\ctxone})\ctxhole{\termtwo} &\defi
  \ps{\termone}{\ctxone\ctxhole{\termtwo}}.
\end{align*}
\end{definition}
We also write $\ctxone\ctxhole{\ctxtwo}$ for the context resulting from
replacing the occurrence of $\ctxhole{\cdot}$ in the syntax tree $\ctxone$
by the tree $\ctxtwo$.

We continue to keep track of free variables by sets $\vecvarone$ of
variables and we inductively define subsets
$\ctxsetp{\vecvarone}{\vecvartwo}$ of contexts by the following rules:
$$
\infer[\Ctxone] {\ctxhole{\cdot}\in\ctxsetp{\vecvarone}{\vecvarone}}{ }
$$

$$
\infer[\Ctxtwo]
{\abstr{\varone}{\ctxone}\in\ctxsetp{\vecvarone}{\vecvartwo}}
{\ctxone\in\ctxsetp{\vecvarone}{\vecvartwo\cup\{\varone\}} &&
  \varone\not\in\vecvartwo}
$$
  
$$
\infer[\Ctxthree]
{\app{\ctxone}{\termone}\in\ctxsetp{\vecvarone}{\vecvartwo}}
{\ctxone\in\ctxsetp{\vecvarone}{\vecvartwo} && \termone\in\LOP(\vecvartwo)}
$$

$$
\infer[\Ctxfour]
{\app{\termone}{\ctxone}\in\ctxsetp{\vecvarone}{\vecvartwo}}
{\termone\in\LOP(\vecvartwo) && \ctxone\in\ctxsetp{\vecvarone}{\vecvartwo}
}
$$

$$
\infer[\Ctxfive]
{\ps{\ctxone}{\termone}\in\ctxsetp{\vecvarone}{\vecvartwo}}
{\ctxone\in\ctxsetp{\vecvarone}{\vecvartwo} && \termone\in\LOP(\vecvartwo)}
$$

$$
\infer[\Ctxsix] {\ps{\termone}{\ctxone}\in\ctxsetp{\vecvarone}{\vecvartwo}}
{\termone\in\LOP(\vecvartwo) && \ctxone\in\ctxsetp{\vecvarone}{\vecvartwo}
}
$$
We use double indexing over $\vecvarone$ and $\vecvartwo$ to indicate the
sets of free variables before and after the filling of the hole by a
term. The two following properties explain this idea.

\begin{lemma}\label{lemma:fillholeterm}
  If $\termone\in\LOPp{\vecvarone}$ and
  $\ctxone\in\ctxsetp{\vecvarone}{\vecvartwo}$, then
  $\ctxone\ctxhole{\termone}\in\LOPp{\vecvartwo}$.
\end{lemma}
\begin{proof}
  By induction on the derivation of
  $\ctxone\in\ctxsetp{\vecvarone}{\vecvartwo}$ from the rules
  $\Ctxone$-$\Ctxsix$.
\end{proof}

\begin{lemma}\label{lemma:fillholectx}
  If $\ctxone\in\ctxsetp{\vecvarone}{\vecvartwo}$ and
  $\ctxtwo\in\ctxsetp{\vecvartwo}{\vecvartwo}$, then
  $\ctxtwo\ctxhole\ctxone\in\ctxsetp{\vecvarone}{\vecvartwo}$.
\end{lemma}
\begin{proof}
  By induction on the derivation of
  $\ctxtwo\in\ctxsetp{\vecvartwo}{\vecvartwo}$ from the rules
  $\Ctxone$-$\Ctxsix$.
\end{proof}
Let us recall here the definition of context preorder and equivalence. 

\begin{definition}
  \label{def:ctxeqCBN}
  The \emph{probabilistic context preorder} with respect to
  call-by-name evaluation is the $\LOP$-relation given by
  $\rel{\vecvarone}{\termone}{\cbnconleq}{\termtwo}$ iff $\forall \, \ctxone\in\ctxsetp{\vecvarone}{\emptyset}$,
  $\ctxone\ctxhole{\termone}\evp{\probone}$ implies
  $\ctxone\ctxhole{\termtwo}\evp{\probtwo}$ with $\probone\leq\probtwo$.
  The $\LOP$-relation of \emph{probabilistic context equivalence},
  denoted $\rel{\vecvarone}{\termone}{\cbnconequiv}{\termtwo}$, holds iff
  $\rel{\vecvarone}{\termone}{\cbnconleq}{\termtwo}$ and
  $\rel{\vecvarone}{\termtwo}{\cbnconleq}{\termone}$ do.
\end{definition}

\begin{lemma}\label{lemma:ctxprecCBN}
  The context preorder $\cbnconleq$ is a precongruence relation.
\end{lemma}
\begin{proof}
  Proving $\cbnconleq$ being a precongruence relation means to prove it
  transitive and compatible.  We start by proving $\cbnconleq$ being
  transitive, that is, for every $\vecvarone\in\powfin{\setvar}$ and for
  every $\termone,\,\termtwo,\,\termthree\in\LOPp{\vecvarone}$,
  $\rel{\vecvarone}{\termone}{\cbnconleq}{\termtwo}$ and
  $\rel{\vecvarone}{\termtwo}{\cbnconleq}{\termthree}$ imply
  $\rel{\vecvarone}{\termone}{\cbnconleq}{\termthree}$. By
  Definition~\ref{def:ctxeqCBN}, the latter boils down to prove that,
  the following hypotheses
  \begin{varitemize}
  \item For every $\ctxone$, $\ctxone\ctxhole{\termone}\evp{\probone}$
    implies $\ctxone\ctxhole{\termtwo}\evp{\probtwo}$, with
    $\probone\leq\probtwo$;
  \item For every $\ctxone$, $\ctxone\ctxhole{\termtwo}\evp{\probone}$
    implies $\ctxone\ctxhole{\termthree}\evp{\probtwo}$, with
    $\probone\leq\probtwo$,
  \item $\ctxtwo\ctxhole{\termone}\evp{\probthree}$
  \end{varitemize}
  imply $\ctxtwo\ctxhole{\termthree}\evp{\probfour}$, with
  $\probthree\leq\probfour$. We can easily apply the first hypothesis when
  $\ctxone$ is just $\ctxtwo$, then the second hypothesis (again with
  $\ctxone$ equal to $\ctxtwo$), and get the thesis.
  We prove $\cbnconleq$ of being a compatible relation starting from
  $\Comtwo$ property because $\Comone$ is trivially valid. In particular,
  we must show that, for every $\vecvarone\in\powfin{\setvar}$, for every
  $\varone\in\setvar-\{\vecvarone\}$ and for every $\termone,\,
  \termtwo\in\LOP(\vecvarone\cup\{\varone\})$, if
  $\rel{\vecvarone\cup\{\varone\}}{\termone}{\cbnconleq}{\termtwo}$ then
  $\rel{\vecvarone}{\abstr{\varone}{\termone}}{\cbnconleq}{\abstr{\varone}{\termtwo}}$.
  By Definition~\ref{def:ctxeqCBN}, the latter boils down to prove that,
  the following hypotheses
  \begin{varitemize}
  \item For every $\ctxone$, $\ctxone\ctxhole{\termone}\evp{\probone}$
    implies $\ctxone\ctxhole{\termtwo}\evp{\probtwo}$, with
    $\probone\leq\probtwo$,
  \item $\ctxtwo\ctxhole{\abstr{\varone}{\termone}}\evp{\probthree}$
  \end{varitemize}
  imply $\ctxtwo\ctxhole{\abstr{\varone}{\termtwo}}\evp{\probfour}$, with
  $\probthree\leq\probfour$. Since
  $\ctxtwo\in\ctxsetp{\vecvarone}{\emptyset}$, let us consider the context
  $\abstr{\varone}{\ctxhole{\cdot}}\in\ctxsetp{\vecvarone\cup\{\varone\}}{\vecvarone}$. Then,
  by Lemma~\ref{lemma:fillholectx}, the context $\ctxthree$ of the form
  $\ctxtwo\ctxhole{\abstr{\varone}{\ctxhole{\cdot}}}$ is in
  $\ctxsetp{\vecvarone\cup\{\varone\}}{\emptyset}$. Please note that, by
  Definition~\ref{def:context}, $\ctxtwo\ctxhole{\abstr{\varone}{\termone}} =
  \ctxthree\ctxhole{\termone}$ and, therefore, the second hypothesis can be
  rewritten as $\ctxthree\ctxhole{\termone}\evp{\probthree}$. Thus,
  it follows that $\ctxthree\ctxhole{\termtwo}\evp{\probfour}$, with
  $\probthree\leq\probfour$.
  Moreover, observe that $\ctxthree\ctxhole{\termtwo}$ is nothing else than
  $\ctxtwo\ctxhole{\abstr{\varone}{\termtwo}}$.  Since we have just proved
  $\cbnconleq$ of being transitive, we prove $\Comthree$ property by
  showing that $\ComthreeL$ and $\ComthreeR$ hold. In fact, recall that by
  Lemma~\ref{lemma:com3LR}, the latter two, together, imply the former. In
  particular, to prove $\ComthreeL$ we must show that, for every
  $\vecvarone\in\powfin{\setvar}$ and for every $\termone,\,
  \termtwo,\,\termthree\in\LOP(\vecvarone)$, if
  $\rel{\vecvarone}{\termone}{\cbnconleq}{\termtwo}$ then
  $\rel{\vecvarone}{\app{\termone}{\termthree}}{\cbnconleq}{\app{\termtwo}{\termthree}}$.
  By Definition~\ref{def:ctxeqCBN}, the latter boils down to prove that,
  the following hypothesis
  \begin{varitemize}
  \item For every $\ctxone$, $\ctxone\ctxhole{\termone}\evp{\probone}$
    implies $\ctxone\ctxhole{\termtwo}\evp{\probtwo}$, with
    $\probone\leq\probtwo$,
  \item $\ctxtwo\ctxhole{\app{\termone}{\termthree}}\evp{\probthree}$
  \end{varitemize}
  imply $\ctxtwo\ctxhole{\app{\termtwo}{\termthree}}\evp{\probfour}$, with
  $\probthree\leq\probfour$. Since
  $\ctxtwo\in\ctxsetp{\vecvarone}{\emptyset}$, let us consider the context
  $\app{\ctxhole{\cdot}}{\termthree}\in\ctxsetp{\vecvarone}{\vecvarone}$. Then,
  by Lemma~\ref{lemma:fillholectx}, the context $\ctxthree$ of the form
  $\ctxtwo\ctxhole{\app{\ctxhole{\cdot}}{\termthree}}$ is in
  $\ctxsetp{\vecvarone}{\emptyset}$. Please note that, by
  Definition~\ref{def:context}, $\ctxtwo\ctxhole{\app{\termone}{\termthree}} =
  \ctxthree\ctxhole{\termone}$ and, therefore, the second hypothesis can be
  rewritten as $\ctxthree\ctxhole{\termone}\evp{\probthree}$.  Thus, it follows that $\ctxthree\ctxhole{\termtwo}\evp{\probfour}$, with
  $\probthree\leq\probfour$. Moreover, observe that $\ctxthree\ctxhole{\termtwo}$ is nothing else than
  $\ctxtwo\ctxhole{\abstr{\varone}{\termtwo}}$.
  We do not detail the proof for $\ComthreeR$ that follows the reasoning
  made for $\ComthreeL$, but considering $\ctxthree$ as the context
  $\ctxtwo\ctxhole{\app{\termthree}{\ctxhole{\cdot}}}$.
  Proving $\Comfour$ follows the same pattern resulted for $\Comthree$. In
  fact, by Lemma~\ref{lemma:com4LR}, $\ComfourL$ and $\ComfourR$ together
  imply $\Comfour$. We do not detail the proofs since they proceed the
  reasoning made for $\ComthreeL$, considering the appropriate context each
  time. This concludes the proof.
\end{proof}

\begin{corollary}
  The context equivalence $\cbnconequiv$ is a congruence relation.
\end{corollary}
\begin{proof}
  Straightforward consequence of the definition
  $\cbnconequiv=\cbnconleq\cap\cbnconleq^{\mathit{op}}$.
\end{proof}

\begin{lemma}\label{lemma:ctxcomprel}
  Let $\relone$ be a compatible $\LOP$-relation. If
  $\rel{\vecvarone}{\termone}{\relone}{\termtwo}$ and
  $\ctxone\in\ctxsetp{\vecvarone}{\vecvartwo}$, then
  $\rel{\vecvartwo}{\ctxone\ctxhole{\termone}}{\relone}{\ctxone\ctxhole{\termtwo}}$.
\end{lemma}
\begin{proof}
  By induction on the derivation of
  $\ctxone\in\ctxsetp{\vecvarone}{\vecvartwo}$:
  \begin{varitemize}
  \item If $\ctxone$ is due to $\Ctxone$ then $\ctxone =
    \ctxhole{\cdot}$. Thus, $\ctxone\ctxhole{\termone} = \termone$,
    $\ctxone\ctxhole{\termtwo} = \termtwo$ and the result trivially holds.
  \item If $\Ctxtwo$ is the last rule used, then $\ctxone =
    \abstr{\varone}{\ctxtwo}$, with
    $\ctxtwo\in\ctxsetp{\vecvarone}{\vecvartwo\cup\{\varone\}}$. By
    induction hypothesis, it holds that
    $\rel{\vecvartwo\cup\{\varone\}}{\ctxtwo\ctxhole{\termone}}{\relone}{\ctxtwo\ctxhole{\termtwo}}$. Since
    $\relone$ is a compatible relation, it follows
    $\rel{\vecvartwo}{\abstr{\varone}{\ctxtwo\ctxhole{\termone}}}{\relone}{\abstr{\varone}{\ctxtwo\ctxhole{\termtwo}}}$,
    that is
    $\rel{\vecvartwo}{\ctxone\ctxhole{\termone}}{\relone}{\ctxone\ctxhole{\termtwo}}$.
  \item If $\Ctxthree$ is the last rule used, then $\ctxone =
    \app{\ctxtwo}{\termthree}$, with
    $\ctxtwo\in\ctxsetp{\vecvarone}{\vecvartwo}$ and
    $\termthree\in\LOP(\vecvartwo)$. By induction hypothesis, it holds that
    $\rel{\vecvartwo}{\ctxtwo\ctxhole{\termone}}{\relone}{\ctxtwo\ctxhole{\termtwo}}$.
    Since $\relone$ is a compatible relation, it follows
    $\rel{\vecvartwo}{\app{\ctxtwo\ctxhole{\termone}}{\termthree}}{\relone}{\app{\ctxtwo\ctxhole{\termtwo}}{\termthree}}$,
    which by definition means
    $\rel{\vecvartwo}{(\app{\ctxtwo}{\termthree})\ctxhole{\termone}}{\relone}{(\app{\ctxtwo}{\termthree})\ctxhole{\termtwo}}$. Hence,
    the result
    $\rel{\vecvartwo}{\ctxone\ctxhole{\termone}}{\relone}{\ctxone\ctxhole{\termtwo}}$
    holds. The case of rule $\Ctxfour$ holds by a similar reasoning.
  \item If $\Ctxfive$ is the last rule used, then $\ctxone =
    \ps{\ctxtwo}{\termthree}$, with
    $\ctxtwo\in\ctxsetp{\vecvarone}{\vecvartwo}$ and
    $\termthree\in\LOP(\vecvartwo)$. By induction hypothesis, it holds that
    $\rel{\vecvartwo}{\ctxtwo\ctxhole{\termone}}{\relone}{\ctxtwo\ctxhole{\termtwo}}$.
    Since $\relone$ is a compatible relation, it follows
    $\rel{\vecvartwo}{\ps{\ctxtwo\ctxhole{\termone}}{\termthree}}{\relone}{\ps{\ctxtwo\ctxhole{\termtwo}}{\termthree}}$,
    which by definition means
    $\rel{\vecvartwo}{(\ps{\ctxtwo}{\termthree})\ctxhole{\termone}}{\relone}{(\ps{\ctxtwo}{\termthree})\ctxhole{\termtwo}}$. Hence,
    the result
    $\rel{\vecvartwo}{\ctxone\ctxhole{\termone}}{\relone}{\ctxone\ctxhole{\termtwo}}$
    holds. The case of rule $\Ctxsix$ holds by a similar reasoning.
  \end{varitemize}
  This concludes the proof.
\end{proof}

\begin{lemma}\label{lemma:ctxbisimCBN}
  If $\rel{\vecvarone}{\termone}{\cbnpab}{\termtwo}$ and
  $\ctxone\in\ctxsetp{\vecvarone}{\vecvartwo}$, then
  $\rel{\vecvartwo}{\ctxone\ctxhole{\termone}}{\cbnpab}{\ctxone\ctxhole{\termtwo}}$.
\end{lemma}
\begin{proof}
  Since $\cbnpab = \cbnpas \cap \cbnpas^{\mathit{op}}$ by
  Proposition~\ref{prop:pab=pascopas},
  $\rel{\vecvarone}{\termone}{\cbnpab}{\termtwo}$ implies
  $\rel{\vecvarone}{\termone}{\cbnpas}{\termtwo}$ and
  $\rel{\vecvarone}{\termtwo}{\cbnpas}{\termone}$. Since, by
  Theorem~\ref{thm:pasprecongrCBN}, $\cbnpas$ is a precongruence hence a
  compatible relation,
  $\rel{\vecvartwo}{\ctxone\ctxhole{\termone}}{\cbnpas}{\ctxone\ctxhole{\termtwo}}$
  and
  $\rel{\vecvartwo}{\ctxone\ctxhole{\termtwo}}{\cbnpas}{\ctxone\ctxhole{\termone}}$
  follow by Lemma~\ref{lemma:ctxcomprel}, i.e. $\rel{\vecvartwo}{\ctxone\ctxhole{\termone}}{\cbnpab}{\ctxone\ctxhole{\termtwo}}$.
\end{proof}
\begin{theorem}\label{thm:pab_ce}
  For all $\vecvarone\in\powfin{\setvar}$ and every $\termone,\,
  \termtwo\in\LOP(\vecvarone)$,
  $\rel{\vecvarone}{\termone}{\cbnpab}{\termtwo}$ implies
  $\rel{\vecvarone}{\termone}{\cbnconequiv}{\termtwo}$.
\end{theorem}
\begin{proof}
  If $\rel{\vecvarone}{\termone}{\cbnpab}{\termtwo}$, then for every
  $\ctxone\in\ctxsetp{\vecvarone}{\emptyset}$,
  $\rel{\emptyset}{\ctxone\ctxhole{\termone}}{\cbnpab}{\ctxone\ctxhole{\termtwo}}$
  follows by Lemma~\ref{lemma:ctxbisimCBN}. By
  Lemma~\ref{lemma:sumsempabCBN}, the latter implies
  $\sumsem{\ctxone\ctxhole{\termone}}=\probone=\sumsem{\ctxone\ctxhole{\termtwo}}$. This
  means in particular that $\ctxone\ctxhole{\termone}\evp{\probone}$ iff
  $\ctxone\ctxhole{\termtwo}\evp{\probone}$, which is equivalent to
  $\rel{\vecvarone}{\termone}{\cbnconequiv}{\termtwo}$ by definition.
\end{proof}
The converse inclusion fails. A counterexample 
is described in the following.
\begin{example}\label{ex:count}
  For $\termone\defi\abstr{\varone}{\ps{\termthree}{\termfour}}$
  and $\termtwo\defi\ps{(\abstr{\varone}{\termthree})}{(\abstr{\varone}{\termfour})}$
  (where $\termthree$ is $\abstr{\vartwo}{\Omega}$ and $\termfour$ is $\abstr{\vartwo}{\abstr{\varthree}{\Omega}}$), 
  we have $\termone\not\cbnsimleq\termtwo$, hence
  $\termone\not\cbnpab\termtwo$, but
  $\termone\cbnconequiv\termtwo$.
\end{example}
We prove that the above two terms are context equivalent by means
of \emph{CIU-equivalence}. This is a relation that can be shown to
coincide with context equivalence by a Context Lemma, itself proved by the
Howe's technique.  See Section~\ref{sec:cfctxeq} and
Section~\ref{sec:ciu-eq} for supplementary details on the above
counterexample.  

\section{Context Free Context Equivalence}\label{sec:cfctxeq}


We present here a way of treating the problem of too concrete
representations of contexts: right now, we cannot basically work up-to
$\alpha$-equivalence classes of contexts. Let us dispense with them
entirely, and work instead with a coinductive characterization of the
context preorder, and equivalence, phrased in terms of $\LOP$-relations.

\begin{definition}\label{def:adequateCBN}
  A $\LOP$-relation $\relone$ is said to be \textit{adequate} if, for every
  $\termone,\,\termtwo\in\LOPp{\emptyset}$,
  $\rel{\emptyset}{\termone}{\relone}{\termtwo}$ implies
  $\termone\evp{\probone}$ and $\termtwo\evp{\probtwo}$, with
  $\probone\leq\probtwo$.
\end{definition}
Let us indicate with $\caset$ the collection of all compatible and adequate
$\LOP$-relations and let
\begin{equation}
  \label{eq:cfctxCBN}
  \cbncfleq \defi \bigcup\,\caset.
\end{equation}
It turns out that the context preorder $\cbnconleq$ is the largest
$\LOP$-relation that is both compatible and adequate, that is $\cbnconleq =
\cbncfleq$. Let us proceed towards a proof for the latter.

\begin{lemma}\label{lemma:cfctxcomp}
  For every $\relone,\reltwo\in\caset$, $\relone\circ\reltwo\in\caset$.
\end{lemma}
\begin{proof}
  We need to show that $\relone\circ\reltwo =
  \{(\termone,\termtwo)\,|\,\exists\,\termthree\in\LOPp{\vecvarone}.\,\rel{\vecvarone}{\termone}{\relone}{\termthree}\,
  \wedge\,\rel{\vecvarone}{\termthree}{\reltwo}{\termtwo}\}$ is a
  compatible and adequate $\LOP$-relation.  Obviously,
  $\relone\circ\reltwo$ is adequate: for every
  $(\termone,\termtwo)\in\relone\circ\reltwo$, there exists a term
  $\termthree$ such that
  $\termone\evp{\probone}\Rightarrow\termthree\evp{\probtwo}\Rightarrow\termtwo\evp{\probthree}$,
  with $\probone\leq\probtwo\leq\probthree$. Then,
  $\termone\evp{\probone}\Rightarrow\termtwo\evp{\probthree}$, with
  $\probone\leq\probthree$.  Note that the identity relation $\id \defi
  \{(\termone,\termone)\,|\,\termone\in\LOPp{\vecvarone}\}$ is in
  $\relone\circ\reltwo$. Then, $\relone\circ\reltwo$ is reflexive and, in
  particular, satisfies compatibility property $\Comone$.  Proving
  $\Comtwo$ means to show that, if
  $\rel{\vecvarone\cup\{\varone\}}{\termone}{(\relone\circ\reltwo)}{\termtwo}$,
  then
  $\rel{\vecvarone}{\abstr{\varone}{\termone}}{(\relone\circ\reltwo)}{\abstr{\varone}{\termtwo}}$. From
  the hypothesis, it follows that there exists a term $\termthree$ such
  that $\rel{\vecvarone\cup\{\varone\}}{\termone}{\relone}{\termthree}$ and
  $\rel{\vecvarone\cup\{\varone\}}{\termthree}{\reltwo}{\termtwo}$. Since
  both $\relone$ and $\reltwo$ are in $\caset$, hence compatible, it holds
  $\rel{\vecvarone}{\abstr{\varone}{\termone}}{\relone}{\abstr{\varone}{\termthree}}$
  and
  $\rel{\vecvarone}{\abstr{\varone}{\termthree}}{\reltwo}{\abstr{\varone}{\termtwo}}$. The
  latter together imply
  $\rel{\vecvarone}{\abstr{\varone}{\termone}}{(\relone\circ\reltwo)}{\abstr{\varone}{\termtwo}}$.
  Proving $\Comthree$ means to show that, if
  $\rel{\vecvarone}{\termone}{(\relone\circ\reltwo)}{\termtwo}$ and
  $\rel{\vecvarone}{\termfive}{(\relone\circ\reltwo)}{\termsix}$, then
  $\rel{\vecvarone}{\app{\termone}{\termfive}}{(\relone\circ\reltwo)}{\app{\termtwo}{\termsix}}$.
  From the hypothesis, it follows that there exist two terms
  $\termthree,\,\termfour$ such that, on the one hand,
  $\rel{\vecvarone}{\termone}{\relone}{\termthree}$ and
  $\rel{\vecvarone}{\termthree}{\reltwo}{\termtwo}$, and on the other hand,
  $\rel{\vecvarone}{\termfive}{\relone}{\termfour}$ and
  $\rel{\vecvarone}{\termfour}{\reltwo}{\termsix}$. Since both $\relone$
  and $\reltwo$ are in $\caset$, hence compatible, it holds:
  $$
  \rel{\vecvarone}{\termone}{\relone}{\termthree}\,\wedge\,\rel{\vecvarone}{\termfive}{\relone}{\termfour}\Rightarrow
  \rel{\vecvarone}{\app{\termone}{\termfive}}{\relone}{\app{\termthree}{\termfour}};
  $$
  $$
  \rel{\vecvarone}{\termthree}{\reltwo}{\termtwo}\,\wedge\,\rel{\vecvarone}{\termfour}{\reltwo}{\termsix}\Rightarrow
  \rel{\vecvarone}{\app{\termthree}{\termfour}}{\reltwo}{\app{\termtwo}{\termsix}}.
  $$
  The two together imply
  $\rel{\vecvarone}{\app{\termone}{\termfive}}{(\relone\circ\reltwo)}{\app{\termtwo}{\termsix}}$.

  Proceeding in the same fashion, one can easily prove property $\Comfour$.
\end{proof}

\begin{lemma}\label{lemma:cfprCBNadeq}
  $\LOP$-relation $\cbncfleq$ is adequate.
\end{lemma}
\begin{proof}
It suffices to note that the property of being \textit{adequate} is
  closed under taking unions of relations. Indeed, if $\relone,\,\reltwo$
  are adequate relations, then it is easy to see that the union
  $\relone\cup\reltwo$ is: for every couple
  $(\termone,\termtwo)\in\relone\cup\reltwo$, either
  $\rel{\vecvarone}{\termone}{\relone}{\termtwo}$ or
  $\rel{\vecvarone}{\termone}{\reltwo}{\termtwo}$. Either way,
  $\termone\evp{\probone}\Rightarrow \termtwo\evp{\probtwo}$, with
  $\probone\leq\probtwo$, implying $\relone\cup\reltwo$ of being adequate.
\end{proof}

\begin{lemma}\label{lemma:cfprCBNprec}
  $\LOP$-relation $\cbncfleq$ is a precongruence.
\end{lemma}
\begin{proof}
  We need to show that $\cbncfleq$ is a transitive and compatible
  relation. By Lemma~\ref{lemma:cfctxcomp},
  $\cbncfleq\circ\cbncfleq\subseteq\cbncfleq$ which implies $\cbncfleq$ of
  being transitive. Let us now prove that $\cbncfleq$ is also compatible.
  Note that the identity relation $\id =
  \{(\termone,\termone)\,|\,\termone\in\LOPp{\vecvarone}\}$ is in $\caset$,
  which implies reflexivity of $\cbncfleq$ and hence, in particular, it
  satisfies property $\Comone$.  It is clear that property $\Comtwo$ is
  closed under taking unions of relations, so that $\cbncfleq$ satisfies
  $\Comtwo$ too. The same is not true for properties $\Comthree$ and
  $\Comfour$. By Lemma~\ref{lemma:com3LR} (respectively,
  Lemma~\ref{lemma:com4LR}), for $\Comthree$ (resp., $\Comfour$) it
  suffices to show that $\cbncfleq$ satisfies $\ComthreeL$ and $\ComthreeR$
  (resp., $\ComfourL$ and $\ComfourR$). This is obvious: contrary to
  $\Comthree$ (resp., $\Comfour$), these properties clearly are closed
  under taking unions of relations.

  This concludes the proof.
\end{proof}

\begin{corollary}
  $\cbncfleq$ is the largest compatible and adequate $\LOP$-relation.
\end{corollary}
\begin{proof}
  Straightforward consequence of Lemma~\ref{lemma:cfprCBNadeq} and
  Lemma~\ref{lemma:cfprCBNprec}.
\end{proof}

\begin{lemma}\label{lemma:ctxeq=cfctxpr}
  $\LOP$-relations $\cbnconleq$ and $\cbncfleq$ coincide.
\end{lemma}
\begin{proof}
  By Definition~\ref{def:ctxeqCBN}, it is immediate that $\cbnconleq$ is
  adequate. Moreover, by Lemma~\ref{lemma:ctxprecCBN}, $\cbnconleq$ is a
  precongruence. Therefore $\cbnconleq\in\caset$ implying
  $\cbnconleq\subseteq\cbncfleq$. Let us prove the converse.
  Since, by Lemma~\ref{lemma:cfprCBNprec}, $\cbncfleq$ is a precongruence
  hence a compatible relation, it holds that, for every
  $\termone,\,\termtwo\in\LOPp{\vecvarone}$ and for every
  $\ctxone\in\ctxsetp{\vecvarone}{\vecvartwo}$,
  $\rel{\vecvarone}{\termone}{\cbncfleq}{\termtwo}$ implies
  $\rel{\vecvartwo}{\ctxone\ctxhole{\termone}}{\cbncfleq}{\ctxone\ctxhole{\termtwo}}$.
  Therefore, for every $\termone,\,\termtwo\in\LOPp{\vecvarone}$ and for
  every $\ctxone\in\ctxsetp{\vecvarone}{\emptyset}$,
  \begin{align*}
    \rel{\vecvarone}{\termone}{\cbncfleq}{\termtwo}&\Rightarrow
    \rel{\emptyset}{\ctxone\ctxhole{\termone}}{\cbncfleq}{\ctxone\ctxhole{\termtwo}}
  \end{align*}
  which implies, by the fact that $\cbncfleq$ is adequate,
  \begin{align*}
    \ctxone\ctxhole{\termone}\evp{\probone}\Rightarrow\ctxone\ctxhole{\termtwo}\evp{\probtwo},\textnormal{
      with }\probone\leq\probtwo
  \end{align*}
  that is, by Definition~\ref{def:ctxeqCBN},
  \begin{align*}
    \rel{\vecvarone}{\termone}{\cbnconleq}{\termtwo}.
  \end{align*}
  This concludes the proof.
\end{proof}

\section{CIU-Equivalence}
\label{sec:ciu-eq}


\renewcommand{\ssemn}[2]{#1\Rightarrow_\mathsf{IN} #2}
\renewcommand{\bsemn}[2]{#1\Downarrow_\mathsf{IN} #2}

CIU-equivalence is a simpler characterization of that kind of program
equivalence we are interested in, i.e., context equivalence. In fact, we
will prove that the two notions coincide. While context equivalence
envisages a quantification over all contexts, CIU-equivalence relaxes such
constraint to a restricted class of contexts without affecting the
associated notion of program equivalence. Such a class of contexts is that
of \emph{evaluation} contexts. In particular, we use a different representation of
evaluation contexts, seeing them as a stack of evaluation frames.

\begin{definition}
  The set of \emph{frame stacks} is given by the following set of rules:
  $$
  \fsone,\fstwo::=\nil\midd\las{\termone}{\fsone}.
  $$
\end{definition}

The set of free variables of a frame stack $\fsone$ can be easily defined
as the union of the variables occurring free in the terms embedded into
it. Given a set of variables $\vecvarone$, define $\stk{\vecvarone}$ as the
set of frame stacks whose free variables are all from $\vecvarone$.
Given a frame stack $\fsone\in\stk{\vecvarone}$ and a term
$\termone\in\LOP(\vecvarone)$, we define the term
$\stktm{\fsone}{\termone}\in\LOP(\vecvarone)$ as follows:
\begin{align*}
  \stktm{\nil}{\termone}&\defi\termone;\\
  \stktm{\las{\termone}{\fsone}}{\termtwo}&\defi\stktm{\fsone}{\termtwo\termone}.
\end{align*}
We now define a binary relation $\cbnfsred$ between pairs of the form
$(\fsone,\termone)$ and \emph{sequences} of pairs in the same form:
\begin{align*}
  (\fsone,\termone\termtwo)&\cbnfsred(\las{\termtwo}{\fsone},\termone);\\
  (\fsone,\ps{\termone}{\termtwo})&\cbnfsred(\fsone,\termone),(\fsone,\termtwo);\\
  (\las{\termone}{\fsone},\abstr{\varone}{\termtwo})&\cbnfsred(\fsone,\subst{\termtwo}{\varone}{\termone}).
\end{align*}
Finally, we define a formal system whose judgments are in the form
$\cbnfscp{\fsone}{\termone}{\probone}$ and whose rules are as follows:
$$
\infer[\Howeredone] {\cbnfscp{\fsone}{\termone}{0}} {}
$$
$$
\infer[\Howeredtwo] {\cbnfscp{\nil}{\valone}{1}} {}
$$
$$
\infer[\Howeredthree]
{\cbnfscp{\fsone}{\termone}{\frac{1}{n}\sum_{i=1}^n\probone_i}}
{(\fsone,\termone)\cbnfsred(\fstwo_1,\termtwo_1),\ldots,(\fstwo_n,\termtwo_n)
  & \cbnfscp{\fstwo_i}{\termtwo_i}{\probone_i}}
$$
The expression $\cbnfssup{\fsone}{\termone}$ stands for the real number
$\sup_{\probone\in\RRN}\cbnfscp{\fsone}{\termone}{\probone}$.  

\begin{lemma}\label{lemma:ciuctxeq}
  For all closed frame stacks $\fsone\in\stk{\emptyset}$ and closed
  $\LOP$-terms $\termone\in\LOPp{\emptyset}$,
  $\cbnfssup{\fsone}{\termone}=\probone$ iff
  $\stktm{\fsone}{\termone}\evp{\probone}$. In particular,
  $\termone\evp{\probone}$ holds iff $\cbnfssup{\nil}{\termone}=\probone$.
\end{lemma}
\begin{proof}
First of all, we recall here that the work of Dal Lago and
  Zorzi~\cite{DalLagoZorzi} provides various call-by-name inductive
  semantics, either big-steps or small-steps, which are all equivalent. 
  Then, the result can be deduced from the following properties:
  \begin{enumerate}
\item For all $\fsone\in\stk{\emptyset}$, if
    $\cbnfscp{\fsone}{\termone}{\probone}$ then
    $\exists\distone.\;\ssemn{\stktm{\fsone}{\termone}}{\distone}$ with
    $\sum\distone=\probone$.
    \begin{proof}
      By induction on the derivation of
      $\cbnfscp{\fsone}{\termone}{\probone}$, looking at the last rule
      used.
      \begin{varitemize}
      \item $\Howeredone$ rule used: $\cbnfscp{\fsone}{\termone}{0}$. Then,
        consider the empty distribution $\distone\defi\emptyset$ and observe
        that $\ssemn{\stktm{\fsone}{\termone}}{\distone}$ by
        $\mathsf{se_{n}}$ rule.
      \item $\Howeredtwo$ rule used: $\cbnfscp{\fsone}{\termone}{1}$
        implies $\fsone=\nil$ and $\termone$ of being a value, say
        $\valone$. Then, consider the distribution $\distone\defi\{\valone^1\}$
        and observe that $\ssemn{\stktm{\nil}{\valone}=\valone}{\distone}$
        by $\mathsf{sv_{n}}$ rule. Of course, $\sum\distone=1=\probone$.
      \item $\Howeredthree$ rule used:
        $\cbnfscp{\fsone}{\termone}{\frac{1}{n}\sum_{i=1}^n\probone_i}$
        obtained from
        $(\fsone,\termone)\cbnfsred(\fstwo_1,\termtwo_1),\ldots,(\fstwo_n,\termtwo_n)$
        and, for every $i\in\{1,\dots,n\}$,
        $\cbnfscp{\fstwo_i}{\termtwo_i}{\probone_i}$. Then, by induction
        hypothesis, there exist $\disttwo_1,\ldots,\disttwo_n$ such that
        $\ssemn{\stktm{\fstwo_i}{\termtwo_i}}{\disttwo_i}$ with
        $\sum\disttwo_i=\probone_i$.
          
        Let us now proceed by cases according to the structure of
        $\termone$.
        \begin{varitemize}
        \item If $\termone=\abstr{\varone}{\termthree}$, then
          $\fsone=\las{\termfour}{\fstwo}$ implying $n=1$,
          $\fstwo_1=\fstwo$ and
          $\termtwo_1=\subst{\termthree}{\varone}{\termfour}$. Then,
          consider the distribution $\distone\defi\disttwo_1$ and observe that
          $\stktm{\fsone}{\termone}=\stktm{\las{\termfour}{\fstwo}}{\abstr{\varone}{\termthree}}=
          \stktm{\fstwo}{\app{(\abstr{\varone}{\termthree})}{\termfour}}\rn
          \stktm{\fstwo}{\subst{\termthree}{\varone}{\termfour}}=\stktm{\fstwo_1}{\termtwo_1}$. Hence,
          $\ssemn{\stktm{\fsone}{\termone}}{\distone}$ by $\mathsf{sm_{n}}$
          rule. Moreover,
          $\sum\distone=\sum\disttwo_1=\probone_1=\frac{1}{n}\sum_{i=1}^n\probone_i=\probone$.
        \item If $\termone=\ps{\termthree}{\termfour}$, then $n=2$,
          $\fstwo_1=\fstwo_2=\fsone$, $\termtwo_1=\termthree$ and
          $\termtwo_2=\termfour$. Then, consider the distribution
          $\distone\defi\sum_{i=1}^2\frac{1}{2}\disttwo_i$ and observe that
          $\stktm{\fsone}{\termone}=\stktm{\fsone}{\ps{\termthree}{\termfour}}\rn
          \stktm{\fsone}{\termthree},\stktm{\fsone}{\termfour}=\stktm{\fstwo_1}{\termtwo_1},\stktm{\fstwo_2}{\termtwo_2}$. Hence,
          $\ssemn{\stktm{\fsone}{\termone}}{\distone}$ by $\mathsf{sm_{n}}$
          rule. Moreover,
          $\sum\distone=\sum\sum_{i=1}^2\frac{1}{2}\disttwo_i=\frac{1}{2}\sum_{i=1}^2\sum\disttwo_i=\frac{1}{2}\sum_{i=1}^2\probone_i=p$.
        \item If $\termone=\app{\termthree}{\termfour}$, then $n=1$,
          $\fstwo_1=\las{\termfour}{\fsone}$ and
          $\termtwo_1=\termthree$. Then, consider the distribution
          $\distone\defi\disttwo_1$ and observe that
          $\ssemn{\stktm{\las{\termfour}{\fsone}}{\termthree}}{\disttwo_1}$
          implies $\ssemn{\stktm{\fsone}{\termone}}{\distone}$. Moreover,
          $\sum\distone=\sum\disttwo_1=\probone_1=
          \frac{1}{n}\sum_{i=1}^n\probone_i=p$.
        \end{varitemize}
      \end{varitemize}


This concludes the proof.
    \end{proof}
  \item For all $\distone$, if $\ssemn{\termone}{\distone}$ then
    $\exists\fsone,\,\termtwo.\;\stktm{\fsone}{\termtwo}=\termone$ and
    $\cbnfscp{\fsone}{\termtwo}{\probone}$ with $\sum\distone=\probone$.
    \begin{proof}
      By induction on the derivation of $\ssemn{\termone}{\distone}$,
      looking at the last rule used. (We refer here to the inductive schema
      of inference rules gave in~\cite{DalLagoZorzi} for small-step
      call-by-name semantics of $\LOP$.)
      \begin{varitemize}
      \item $\mathsf{se_{n}}$ rule used:
        $\ssemn{\termone}{\emptyset}$. Then, for every $\fsone$ and every
        $\termtwo$ such that $\stktm{\fsone}{\termtwo}=\termone$,
        $\cbnfscp{\fsone}{\termtwo}{0}$ by $\Howeredone$ rule. Of course,
        $\sum\distone=0=\probone$.
      \item $\mathsf{sv_{n}}$ rule used: $\termone$ is a value, say
        $\valone$, and $\distone=\{\valone^1\}$ with
        $\ssemn{\valone}{\{\valone^1\}}$. Then, consider $\fsone\defi\nil$ and
        $\termtwo\defi\valone$: by definition
        $\stktm{\fsone}{\termtwo}=\stktm{\nil}{\valone}=\valone=\termone$. By
        $\Howeredtwo$ rule, $\cbnfscp{\nil}{\valone}{1}$ hence
        $\sum\distone=1=\probone$.
      \item $\mathsf{sv_{n}}$ rule used:
        $\ssemn{\termone}{\sum_{i=1}^n\frac{1}{n}\disttwo_i}$ from
        $\termone\rn\termfive_1,\ldots,\termfive_n$ with, for every
        $i\in\{1,\dots,n\}$, $\ssemn{\termfive_i}{\disttwo_i}$. By
        induction hypothesis, for every $i\in\{1,\dots,n\}$, there exist
        $\fstwo_i$ and $\termtwo_i$ such that
        $\stktm{\fstwo_i}{\termtwo_i}=\termfive_i$ and
        $\cbnfscp{\fstwo_i}{\termtwo_i}{\probone_i}$ with
        $\sum\disttwo_i=\probone_i$.

        Let us proceed by cases according to the structure of $\termone$.
        \begin{varitemize}
        \item If $\termone=\app{(\abstr{\varone}{\termthree})}{\termfour}$,
          then $n=1$ and
          $\termfive_1=\subst{\termthree}{\varone}{\termfour}$. Hence,
          consider $\fsone\defi\las{\termfour}{\nil}$ and
          $\termtwo\defi\abstr{\varone}{\termthree}$: by definition,
          $\stktm{\fsone}{\termtwo}=\stktm{\las{\termfour}{\nil}}{\abstr{\varone}{\termthree}}=
          \stktm{\nil}{\app{(\abstr{\varone}{\termthree})}{\termfour}}=\app{(\abstr{\varone}{\termthree})}{\termfour}=
          \termone$. By $\Howeredthree$ rule,
          $(\fsone,\termtwo)=(\las{\termfour}{\nil},\abstr{\varone}{\termthree})\cbnfsred(\nil,\subst{\termthree}{\varone}{\termfour})$
          with, by induction hypothesis result,
          $\cbnfscp{\nil}{\subst{\termthree}{\varone}{\termfour}}{\probone_1}$.
          The latter implies
          $\cbnfscp{\fsone}{\termtwo}{\probone_1}$. Moreover,
          $\sum\distone=\sum\sum_{i=1}^n\frac{1}{n}\disttwo_i=\sum\disttwo_1=\probone_1=p$.
        \item If $\termone=\ps{\termthree}{\termfour}$, then $n=2$,
          $\termfive_1=\termthree$ and $\termfive_2=\termfour$. Hence,
          consider $\fsone\defi\nil$ and $\termtwo\defi\ps{\termthree}{\termfour}$:
          by definition,
          $\stktm{\fsone}{\termtwo}=\stktm{\nil}{\ps{\termthree}{\termfour}}=
          \ps{\termthree}{\termfour}=\termone$. By $\Howeredthree$ rule,
          $(\fsone,\termtwo)=(\nil,\ps{\termthree}{\termfour})\cbnfsred(\nil,\termthree),(\nil,\termfour)$
          with, by induction hypothesis result,
          $\cbnfscp{\nil}{\termthree}{\probone_1}$ and
          $\cbnfscp{\nil}{\termfour}{\probone_2}$. The latter implies
          $\cbnfscp{\fsone}{\termtwo}{\frac{1}{2}\sum_{i=1}^2\probone_i}$. Moreover,
          $\sum\distone=\sum\sum_{i=1}^n\frac{1}{n}\disttwo_i=\sum\sum_{i=1}^n\frac{1}{2}\disttwo_i=
          \frac{1}{2}\sum_{i=1}^2\sum\disttwo_i=\frac{1}{2}\sum_{i=1}^2\probone_i=p$.
        \item If $\termone=\app{\termthree}{\termfour}$ and
          $\termthree\rn\termsix_1,\ldots,\termsix_n$, then
          $\termfive_i=\app{\termsix_i}{\termfour}$ for every
          $i\in\{1,\dots,n\}$. Hence, consider
          $\fsone\defi\las{\termfour}{\nil}$ and $\termtwo\defi\termthree$: by
          definition,
          $\stktm{\fsone}{\termtwo}=\stktm{\las{\termfour}{\nil}}{\termthree}=
          \stktm{\nil}{\app{\termthree}{\termfour}}=\app{\termthree}{\termfour}=
          \termone$. By $\Howeredthree$ rule,
          $(\fsone,\termtwo)=(\las{\termfour}{\nil},\termthree)\cbnfsred(\las{\termfour}{\nil},\termsix_1),\ldots,(\las{\termfour}{\nil},\termsix_n)$
          with, by induction hypothesis result,
          $\cbnfscp{\las{\termfour}{\nil}}{\termsix_i}{\probone_i}$ for
          every $i\in\{1,\dots,n\}$. The latter implies
          $\cbnfscp{\fsone}{\termtwo}{\frac{1}{n}\sum_{i=1}^n\probone_i}$. Moreover,
          $\sum\distone=\sum\sum_{i=1}^n\frac{1}{n}\disttwo_i=
          \frac{1}{n}\sum_{i=1}^n\sum\disttwo_i=\frac{1}{n}\sum_{i=1}^n\probone_i=p$.
        \end{varitemize}
      \end{varitemize}
      This concludes the proof.
    \end{proof}
  \end{enumerate}
  Generally speaking, the two properties above prove the following double
  implication:
  \begin{equation}\label{eq:stacksem}
    \cbnfscp{\fsone}{\termone}{\probone}\;\Longleftrightarrow\;
    \ibsemn{\stktm{\fsone}{\termone}}{\distone} \textnormal{ with }
    \sum\distone=\probone.
  \end{equation}
  Then,
  \begin{align*}
    p=\cbnfssup{\fsone}{\termone}&=\sup_{\probtwo\in\RRN}\cbnfscp{\fsone}{\termone}{\probtwo}
    =\sup_{\ibsemn{\stktm{\fsone}{\termone}}{\distone}}\sum\distone\\&=\sum\sup_{\ibsemn{\stktm{\fsone}{\termone}}{\distone}}\distone=
    \sum\sem{\stktm{\fsone}{\termone}}=\stktm{\fsone}{\termone}\evp{\probone},
  \end{align*}
  which concludes the proof.
\end{proof}

Given $\termone,\termtwo\in\LOP(\emptyset)$, we define
$\termone\cbnciuleq\termtwo$ iff for every $\fsone$,
$\cbnfssup{\fsone}{\termone}\leq\cbnfssup{\fsone}{\termtwo}$. This relation
can be extended to a relation on open terms in the usual way. Moreover, we
stipulate $\termone\cbnciuequiv\termtwo$ iff both
$\termone\cbnciuleq\termtwo$ and $\termtwo\cbnciuleq\termone$.

Since $\cbnciuleq$ is a preorder, proving it to be a precongruence boils
down to show the following implication:
$$
\termone\howe{(\cbnciuleq)}\termtwo\Rightarrow\termone\cbnciuleq\termtwo.
$$
Indeed, the converse implication is a consequence of
Lemma~\ref{lemma:howeprop3} and the obvious reflexivity of $\cbnciuleq$
relation. To do that, we extend Howe's construction to frame stacks in a
natural way:
$$
\infer[\Howestkone] {\nil\howe{\relone}\nil} {}
$$
$$
\infer[\Howestktwo]
{(\las{\termone}{\fsone})\howe{\relone}(\las{\termtwo}{\fstwo})} {
  \rel{\emptyset}{\termone}{\howe{\relone}}{\termtwo} &&
  \fsone\howe{\relone}\fstwo }
$$

\begin{lemma}\label{lemma:betaciueq}
  For every $\vecvarone\in\powfin{\setvar}$, it holds
  $\rel{\vecvarone}{\app{(\abstr{\varone}{\termone})}{\termtwo}}{\cbnciuequiv}{\subst{\termone}{\varone}{\termtwo}}$.
\end{lemma}
\begin{proof}
  We need to show that both
  $\rel{\vecvarone}{\app{(\abstr{\varone}{\termone})}{\termtwo}}{\cbnciuleq}{\subst{\termone}{\varone}{\termtwo}}$
  and
  $\rel{\vecvarone}{\subst{\termone}{\varone}{\termtwo}}{\cbnciuleq}{\app{(\abstr{\varone}{\termone}){\termtwo}}}$
  hold. Since $\cbnciuleq$ is defined on open terms by taking closing
  term-substitutions, it suffices to show the result for close $\LOP$-terms only:
  $\app{(\abstr{\varone}{\termone})}{\termtwo}\cbnciuleq\subst{\termone}{\varone}{\termtwo}$
  and
  $\subst{\termone}{\varone}{\termtwo}\cbnciuleq\app{(\abstr{\varone}{\termone})}{\termtwo}$.

  Let us start with
  $\app{(\abstr{\varone}{\termone})}{\termtwo}\cbnciuleq\subst{\termone}{\varone}{\termtwo}$
  and prove that, for every close frame stack $\fsone$,
  $\cbnfssup{\fsone}{\app{(\abstr{\varone}{\termone})}{\termtwo}}\leq\cbnfssup{\fsone}{\subst{\termone}{\varone}{\termtwo}}$. The
  latter is an obvious consequence of the fact
  that $(\fsone,\app{(\abstr{\varone}{\termone})}{\termtwo})$
reduces to $(\fsone,\subst{\termone}{\varone}{\termtwo})$. Let us look
  into the details distinguishing two cases:
  \begin{varitemize}
  \item If $\fsone=\nil$, then
    $(\fsone,\app{(\abstr{\varone}{\termone})}{\termtwo})\cbnfsred(\las{\termtwo}{\fsone},\abstr{\varone}{\termone})
    \cbnfsred(\fsone,\subst{\termone}{\varone}{\termtwo})$ which implies
    that $\cbnfssup{\fsone}{\app{(\abstr{\varone}{\termone})}{\termtwo}} =
    \sup_{\probone\in\RRN}\cbnfscp{\fsone}{\app{(\abstr{\varone}{\termone})}{\termtwo}}{\probone}
    =
    \sup_{\probone\in\RRN}\cbnfscp{\fsone}{\subst{\termone}{\varone}{\termtwo}}{\probone}
    = \cbnfssup{\fsone}{\subst{\termone}{\varone}{\termtwo}}$.
  \item If $\fsone=\las{\termthree}{\fstwo}$, then we can proceed
    similarly.
  \end{varitemize}
    
  \noindent{}Similarly, to prove the converse,
  $\subst{\termone}{\varone}{\termtwo}\cbnciuleq\app{(\abstr{\varone}{\termone})}{\termtwo}$,
  let us fix $\probone$ as
  $\cbnfscp{\fsone}{\subst{\termone}{\varone}{\termtwo}}{\probone}$ and
  distinguish two cases:
  \begin{varitemize}
  \item If $\fsone=\nil$ and $\probone = 0$, then
    $\cbnfscp{\fsone}{\app{(\abstr{\varone}{\termone})}{\termtwo}}{0}$
    holds too by $\Howeredone$ rule. Otherwise,
    $$
    \infer[\Howeredthree]
    {\cbnfscp{\fsone}{\app{(\abstr{\varone}{\termone})}{\termtwo}}{\probone}
    }
    {(\fsone,\termone)\cbnfsred(\las{\termtwo}{\fsone},\abstr{\varone}{\termone})
      & \infer[\Howeredthree]
      {\cbnfscp{\las{\termtwo}{\fsone}}{\abstr{\varone}{\termone}}{\probone}
      }
      {(\las{\termtwo}{\fsone},\abstr{\varone}{\termone})\cbnfsred(\fsone,\subst{\termone}{\varone}{\termtwo})
        & \cbnfscp{\fsone}{\subst{\termone}{\varone}{\termtwo}}{\probone}}
    }
    $$
    which implies $\cbnfssup{\fsone}{\subst{\termone}{\varone}{\termtwo}} =
    \sup_{\probone\in\RRN}\cbnfscp{\fsone}{\subst{\termone}{\varone}{\termtwo}}{\probone}
    =
    \sup_{\probone\in\RRN}\cbnfscp{\fsone}{\app{(\abstr{\varone}{\termone})}{\termtwo}}{\probone}
    = \cbnfssup{\fsone}{\app{(\abstr{\varone}{\termone})}{\termtwo}}$.
  \item If $\fsone=\las{\termthree}{\fstwo}$, then we can proceed
    similarly.
  \end{varitemize}
  This concludes the proof.
\end{proof}

\begin{lemma}\label{lemma:keyciu}
  For every $\fsone,\fstwo\in\stk{\emptyset}$ and
  $\termone,\termtwo\in\LOP(\emptyset)$, if
  $\fsone\howe{(\cbnciuleq)}\fstwo$ and
  $\termone\howe{(\cbnciuleq)}\termtwo$ and
  $\cbnfscp{\fsone}{\termone}{\probone}$, then
  $\cbnfssup{\fstwo}{\termtwo}\geq\probone$.
\end{lemma}
\begin{proof}
  We go by induction on the structure of the proof of
  $\cbnfscp{\fsone}{\termone}{\probone}$, looking at the last rule used.
  \begin{varitemize}
  \item
    If $\cbnfscp{\fsone}{\termone}{0}$, then trivially $\cbnfssup{\fstwo}{\termtwo}\geq 0$.
  \item If $\fsone=\nil$, $\termone=\abstr{\varone}{\termthree}$ and
    $\probone=1$, then $\fstwo=\nil$ since
    $\fsone\howe{(\cbnciuleq)}\fstwo$. From
    $\termone\howe{(\cbnciuleq)}\termtwo$, it follows that there is
    $\termfour$ with
    $\rel{\varone}{\termthree}{\howe{(\cbnciuleq)}}{\termfour}$ and
    $\rel{\emptyset}{\abstr{\varone}{\termfour}}{\cbnciuleq}{\termtwo}$.
    But the latter implies that $\cbnfssup{\nil}{\termtwo}\geq 1$, which is
    the thesis.
  \item Otherwise, $\Howeredthree$ rule is used and suppose we are in the
    following situation
    $$
    \infer[\Howeredthree] {\cbnfscp{\fsone}{\termone}{\frac{1}{n}\sum_{i=1}^n\probone_i}}
    {(\fsone,\termone)\cbnfsred(\fsthree_1,\termthree_1),\ldots,(\fsthree_n,\termthree_n)
      & \cbnfscp{\fsthree_i}{\termthree_i}{\probone_i}}
    $$
    Let us distinguish the following cases as in definition of $\cbnfsred$:
    \begin{varitemize}
    \item If $\termone=\termfour\termfive$, then $n=1$,
      $\fsthree_1=\las{\termfive}{\fsone}$ and $\termthree_1=\termfour$.
      From $\termone\howe{(\cbnciuleq)}\termtwo$ it follows that there are
      $\termsix,\termseven$ with
      $\rel{\emptyset}{\termfour}{\howe{(\cbnciuleq)}}{\termsix}$,
      $\rel{\emptyset}{\termfive}{\howe{(\cbnciuleq)}}{\termseven}$ and
      $\rel{\emptyset}{\app{\termsix}{\termseven}}{\cbnciuleq}{\termtwo}$.
      But then we can form the following:
      $$
      \infer[\Howestktwo]
      {\rel{\emptyset}{\fsthree_1}{\howe{(\cbnciuleq)}}{\las{\termseven}{\fstwo}}}
      {\rel{\emptyset}{\termfive}{\howe{(\cbnciuleq)}}{\termseven} &
        \rel{\emptyset}{\fsone}{\howe{(\cbnciuleq)}}{\fstwo}}
      $$
      and, by the induction hypothesis, conclude that
      $\cbnfssup{\las{\termseven}{\fstwo}}{\termsix}\geq\probone$. Now
      observe that
      $$
      (\fstwo,\termsix\termseven)\cbnfsred(\las{\termseven}{\fstwo},\termsix),
      $$
      and, as a consequence,
      $\cbnfssup{\fstwo}{\termsix\termseven}\geq\probone$, from which the
      thesis easily follows given that
      $\rel{\emptyset}{\app{\termsix}{\termseven}}{\cbnciuleq}{\termtwo}$.
    \item If $\termone=\ps{\termfour}{\termfive}$, then $n=2$,
      $\fsthree_1=\fsthree_2=\fsone$ and $\termthree_1=\termfour$,
      $\termthree_2=\termfive$. From $\fsone\howe{(\cbnciuleq)}\fstwo$, we
      get that $\fsthree_1\howe{(\cbnciuleq)}\fstwo$ and
      $\fsthree_2\howe{(\cbnciuleq)}\fstwo$. From
      $\termone\howe{(\cbnciuleq)}\termtwo$ it follows that there are
      $\termsix,\termseven$ with
      $\rel{\emptyset}{\termfour}{\howe{(\cbnciuleq)}}{\termsix}$,
      $\rel{\emptyset}{\termfive}{\howe{(\cbnciuleq)}}{\termseven}$ and
      $\rel{\emptyset}{\ps{\termsix}{\termseven}}{\cbnciuleq}{\termtwo}$. Then,
      by a double induction hypothesis, it follows
      $\cbnfssup{\fstwo}{\termsix}\geq\probone$ and
      $\cbnfssup{\fstwo}{\termseven}\geq\probone$. The latter together
      imply $\cbnfssup{\fstwo}{\ps{\termsix}{\termseven}}\geq\probone$,
      from which the thesis easily follows given that
      $\rel{\emptyset}{\ps{\termsix}{\termseven}}{\cbnciuleq}{\termtwo}$.
    \item If $\termone=\abstr{\varone}{\termfour}$, then
      $\fsone=\las{\termfive}{\fsthree}$ because the only case left. Hence
      $n=1$, $\fsthree_1=\fsthree$ and
      $\termthree_1=\subst{\termfour}{\varone}{\termfive}$. From
      $\fsone\howe{(\cbnciuleq)}\fstwo$, we get that
      $\fstwo=\las{\termsix}{\fsfour}$ where
      $\rel{\emptyset}{\termfive}{\howe{(\cbnciuleq)}}{\termsix}$ and
      $\fsthree\howe{(\cbnciuleq)}\fsfour$. From
      $\termone\howe{(\cbnciuleq)}\termtwo$, it follows that for some
      $\termseven$, it holds that
      $\rel{\varone}{\termfour}{\howe{(\cbnciuleq)}}{\termseven}$ and
      $\rel{\emptyset}{\abstr{\varone}{\termseven}}{\cbnciuleq}{\termtwo}$. Now:
      \begin{equation}\label{eq:fsred}
        (\fstwo,\abstr{\varone}{\termseven})=(\las{\termsix}{\fsfour},\abstr{\varone}{\termseven})
        \cbnfsred(\fsfour,\subst{\termseven}{\varone}{\termsix}).
      \end{equation}
      From $\rel{\varone}{\termfour}{\howe{(\cbnciuleq)}}{\termseven}$ and
      $\rel{\emptyset}{\termfive}{\howe{(\cbnciuleq)}}{\termsix}$, by
      substitutivity of $\cbnciuleq$, follow that
      $\rel{\emptyset}{\subst{\termfour}{\varone}{\termfive}}{\howe{(\cbnciuleq)}}{\subst{\termseven}{\varone}{\termsix}}$
      holds. By induction hypothesis, it follows that
      $\cbnfssup{\fsfour}{\subst{\termseven}{\varone}{\termsix}}\geq\probone$. Then,
      from (\ref{eq:fsred}) and
      $\rel{\emptyset}{\abstr{\varone}{\termseven}}{\cbnciuleq}{\termtwo}$,
      the thesis easily follows:
      $$
      \cbnfssup{\fstwo}{\termtwo}\geq\cbnfssup{\fstwo}{\abstr{\varone}{\termseven}}=
      \cbnfssup{\fsfour}{\subst{\termseven}{\varone}{\termsix}}\geq\probone.
      $$
    \end{varitemize}
  \end{varitemize}
  This concludes the proof.
\end{proof}

\begin{theorem}\label{thm:ciupr=ctxpr}
  For all $\vecvarone\in\powfin{\setvar}$ and for all
  $\termone,\,\termtwo\in\LOPp{\vecvarone}$,
  $\rel{\vecvarone}{\termone}{\cbnciuleq}{\termtwo}$ iff
  $\rel{\vecvarone}{\termone}{\cbnconleq}{\termtwo}$.
\end{theorem}
\begin{proof}
  ($\Rightarrow$) Since $\cbnciuleq$ is defined on open terms by taking
  closing term-substitutions, by Lemma~\ref{lemma:closesubsCBN} both it and
  $\howe{(\cbnciuleq)}$ are closed under term-substitution. Then, it
  suffices to show the result for closed $\LOP$-terms: for all
  $\termone,\,\termtwo\in\LOPp{\emptyset}$, if
  $\rel{\emptyset}{\termone}{\cbnciuleq}{\termtwo}$, then
  $\rel{\emptyset}{\termone}{\cbnconleq}{\termtwo}$.  Since $\cbnciuleq$ is
  reflexive, by Lemma~\ref{lemma:howeprop1} follows that
  $\howe{(\cbnciuleq)}$ is compatible, hence reflexive too. Taking $\fstwo
  = \fsone$ in Lemma~\ref{lemma:keyciu}, we conclude that
  $\rel{\emptyset}{\termone}{\howe{(\cbnciuleq)}}{\termtwo}$ implies
  $\rel{\emptyset}{\termone}{\cbnciuleq}{\termtwo}$. As we have remarked
  before the lemma, the latter entails that $\howe{(\cbnciuleq)} =
  \cbnciuleq$ which implies $\cbnciuleq$ of being compatible. Moreover,
  from Lemma~\ref{lemma:ciuctxeq} immediately follows that $\cbnciuleq$ is
  also adequate. Thus, $\cbnciuleq$ is contained in the largest compatible
  adequate $\LOP$-relation, $\cbncfleq$. From
  Lemma~\ref{lemma:ctxeq=cfctxpr} follows that $\cbnciuleq$ is actually
  contained in $\cbnconleq$. In particular, the latter means
  $\rel{\emptyset}{\termone}{\cbnciuleq}{\termtwo}$ implies
  $\rel{\emptyset}{\termone}{\cbnconleq}{\termtwo}$.

  ($\Leftarrow$) First of all, please observe that, since context
  preorder is compatible, if
  $\rel{\emptyset}{\termone}{\cbnconleq}{\termtwo}$ then, for all
  $\fsone\in\stk{\emptyset}$,
  $\rel{\emptyset}{\stktm{\fsone}{\termone}}{\cbnconleq}{\stktm{\fsone}{\termtwo}}$
  by Lemma~\ref{lemma:ctxcomprel}. Then, by adequacy property of
  $\cbnconleq$ and Lemma~\ref{lemma:ciuctxeq}, the latter implies
  $\rel{\emptyset}{\termone}{\cbnciuleq}{\termtwo}$. Ultimately, it holds
  that $\rel{\emptyset}{\termone}{\cbnconleq}{\termtwo}$ implies
  $\rel{\emptyset}{\termone}{\cbnciuleq}{\termtwo}$. Let us take into
  account the general case of open terms.  If
  $\rel{\vecvarone}{\termone}{\cbnconleq}{\termtwo}$, then by compatibility
  property of $\cbnconleq$ it follows
  $\rel{\emptyset}{\abstr{\vecvarone}{\termone}}{\cbnconleq}{\abstr{\vecvarone}{\termtwo}}$
  and hence
  $\rel{\emptyset}{\abstr{\vecvarone}{\termone}}{\cbnciuleq}{\abstr{\vecvarone}{\termtwo}}$. Then,
  from the fact that $\cbnciuleq$ is compatible (as established in
  $(\Rightarrow)$ part of this proof) and Lemma~\ref{lemma:betaciueq}, for
  every suitable $\vectermthree\subseteq\LOPp{\emptyset}$, it holds
  $\rel{\emptyset}{\subst{\termone}{\vecvarone}{\vectermthree}}{\cbnciuleq}{\subst{\termtwo}{\vecvarone}{\vectermthree}}$,
  i.e. $\rel{\vecvarone}{\termone}{\cbnciuleq}{\termtwo}$.
\end{proof}

\begin{corollary}\label{cor:ciueq=ctxeq}
  $\cbnciuequiv$ coincides with $\cbnconequiv$.
\end{corollary}
\begin{proof}
  Straightforward consequence of Theorem~\ref{thm:ciupr=ctxpr}.
\end{proof}

\begin{proposition}
  $\cbnconleq$ and $\cbnsimleq$ do not coincide.
\end{proposition}
\begin{proof}
  We will prove that $\termone\cbnciuleq\termtwo$ but
  $\termone\not\cbnsimleq\termtwo$, where
  \begin{align*}
    \termone&\defi\abstr{\varone}{\abstr{\vartwo}{\ps{\varone}{\vartwo}}};\\
    \termtwo&\defi\ps{(\abstr{\varone}{\abstr{\vartwo}{\varone}})}{(\abstr{\varone}{\abstr{\vartwo}{\vartwo}})}.
  \end{align*}
  $\termone\not\cbnsimleq\termtwo$ can be easily verified, so let us
  concentrate on $\termone\cbnciuleq\termtwo$, and prove that for every
  $\fsone$,
  $\cbnfssup{\fsone}{\termone}\leq\cbnfssup{\fsone}{\termtwo}$. Let us
  distinguish three cases:
  \begin{varitemize}
  \item If $\fsone=\nil$, then $(\fsone,\termone)$ cannot be further
    reduced and
    $(\fsone,\termtwo)\cbnfsred(\fsone,\abstr{\varone}{\abstr{\vartwo}{\varone}}),(\fsone,\abstr{\varone}{\abstr{\vartwo}{\vartwo}})$,
    where the last two pairs cannot be reduced. As a consequence,
    $\cbnfssup{\fsone}{\termone}=0=\cbnfssup{\fsone}{\termtwo}$.
  \item If $\fsone=\las{\termthree}{\fstwo}$, then we can proceed
    similarly.
  \item If $\fsone=\las{\termthree}{\las{\termfour}{\fstwo}}$, then observe
    that
    \begin{align*}
      (\fsone,\termone)&\cbnfsred(\las{\termfour}{\fstwo},\abstr{\vartwo}{\ps{\termthree}{\vartwo}})\cbnfsred(\fstwo,\ps{\termthree}{\termfour})\\
      &\cbnfsred(\fstwo,\termthree),(\fstwo,\termfour);\\
      (\fsone,\termtwo)&\cbnfsred(\fsone,\abstr{\varone}{\abstr{\vartwo}{\varone}}),(\fsone,\abstr{\varone}{\abstr{\vartwo}{\vartwo}});\\
      (\fsone,\abstr{\varone}{\abstr{\vartwo}{\varone}})&\cbnfsred(\las{\termfour}{\fstwo},\abstr{\vartwo}{\termthree})\cbnfsred(\fstwo,\termthree);\\
      (\fsone,\abstr{\varone}{\abstr{\vartwo}{\vartwo}})&\cbnfsred(\las{\termfour}{\fstwo},\abstr{\vartwo}{\vartwo})\cbnfsred(\fstwo,\termfour).\\
    \end{align*}
    As a consequence,
    $$
    \cbnfssup{\fsone}{\termone}=\frac{1}{2}\cbnfssup{\fstwo}{\termthree}+\frac{1}{2}\cbnfssup{\fstwo}{\termfour}=\cbnfssup{\fsone}{\termtwo}.
    $$
  \end{varitemize}
  This concludes the proof.
\end{proof}

\begin{example}
  We consider again the programs from Example~\ref{ex:exp}.  Terms
  \hsk{expone} and \hsk{exptwo} only differ because the former
  performs all probabilistic choices on {natural numbers} obtained by
  applying a function to its argument, while in the latter choices are
  done at the functional level, and the argument to those functions is
  provided only at a later stage.  As a consequence, the two terms are
  not applicative bisimilar, and the reason is akin to that for
  the   inequality of the terms in 
  Example~\ref{ex:count}. 
  In contrast, the bisimilarity between  \hsk{expone} and \hsk{expthree
  k}, where \hsk{k} is any natural number, intuitively holds because
  both 
  \hsk{expone} and \hsk{expthree
    k} evaluate to a single term when fed with a
  function, while they start evolving in a genuinely probabilistic way
  only after the second argument is provided. At
  that point, the two functions evolve in very different ways, but
  their semantics (in the sense of Section~\ref{sect:p}) is 
  the same (cf.,  Lemma~\ref{lemma:samesem}).
  As a bisimulation one can use the equivalence generated by the relation
  \begin{align*}
    &\Bigl(\bigcup_{\hsk{k}}\{( \hsk{expone}, \hsk{expthree k})\}\Bigr) \cup
    \{(M,N)\; | \:  {\sem M} = {\sem N }\}\\
    &\cup \Bigl(\bigcup_L \{ (\lambda \hsk{n}. \hsk{B} \sub L {\tt{f}},
    \lambda \hsk{n}. \hsk{C} \sub L {\tt{f}} )\} \Bigr)
  \end{align*}
  using $\tt{B}$ and $\tt{C}$ for the body of $\hsk{expone}$ and
  $\hsk{expthree}$ respectively.
\end{example}
