

In this section, notions of similarity and bisimilarity for  are introduced, in
the spirit of Abramsky's work on applicative bisimulation~\cite{Abramsky-90}. Definitionally, this
consists in seeing 's operational semantics as a labelled Markov chain, then giving
the Larsen and Skou's notion of (bi)simulation for it. States will be terms, while labels
will be of two kinds: one can either \emph{evaluate} a term, obtaining (a distribution of) values, 
or \emph{apply} a term to a value.

The resulting bisimulation (probabilistic applicative bisimulation) 
will be shown to be a congruence, thus included in probabilistic context equivalence.
This will be done by a non-trivial generalization of Howe's technique~\cite{Howe-96}, which is
a well-known methodology to get congruence results in presence of higher-order functions, but
which has not been applied to probabilistic calculi so far. 

Formalizing probabilistic applicative bisimulation requires some care.
As usual, two values  and  are defined
to be bisimilar if for every , 
and  are themselves bisimilar. But how if
we rather want to compare two arbitrary closed terms  and ?
The simplest solution consists in following Larsen and Skou and stipulate that
every equivalence class of  modulo bisimulation is attributed the same
measure by both  and . Values
are thus treated in two different ways (they are both terms and values), and this is the reason why each of
them corresponds to \emph{two} states in the underlying Markov chain.


\begin{definition}
\label{d:multisort}
 can be seen as a multisorted labelled Markov chain

that we denote with . Labels are either closed terms, which model
parameter passing, or , that models evaluation. 
Please observe that the states of the labelled Markov chain we have 
just defined are elements of the disjoint union . 
Two distinct states correspond to the
same value , and to avoid ambiguities, we call the second one
(i.e. the one coming from ) a \emph{distinguished value}.  When we
want to insist on the fact that a value  is
distinguished, we indicate it with . We
define the transition probability matrix  as follows:
\begin{varitemize}
\item 
  For every term  and for every 
  distinguished value ,
  
\item 
  For every term  and for every distinguished value ,
  
\item
  In all other cases,  returns .
\end{varitemize}
\end{definition}
Terms seen as states  only interact with the
environment by performing , while distinguished values 
only take other closed terms as parameters. 

Simulation and bisimulation relations can be defined for  as for any 
labelled Markov chain. Even if, strictly speaking, these are binary relations on 
, we often see them just as their restrictions to 
. 
Formally, a \emph{probabilistic applicative bisimulation} (a
) is simply a probabilistic bisimulation on . This way one can 
define \emph{probabilistic applicative bisimilarity}, which is denoted .
Similarly for \emph{probabilistic applicative simulation} () and \emph{probabilistic
applicative similarity}, denoted .

\begin{remark}[Early vs. Late]\label{r:el} 
  Technically, the distinction between terms and values in
  Definition~\ref{d:multisort} means that our
  bisimulation is in \emph{late} style. In bisimulations for
  value-passing concurrent languages, ``late'' indicates the explicit
  manipulation of functions in the clause for input actions: functions are
  chosen first, and only later, the input value received is taken into
  account~\cite{SaWabook}. Late-style is used in contraposition to 
  \emph{early} style, where the order of quantifiers is exchanged, so that the
  choice of functions may depend on the specific input value received.
  In our setting, adopting an early style would mean having transitions such as
  , and then setting up a probabilistic
  bisimulation on top of the resulting transition system. 
  We leave for future work a study of the comparison between the two
  styles. In this paper, we stick to the late style because easier to deal with,
  especially under Howe's technique.
  Previous works on applicative
  bisimulation for nondeterministic functions also focus on the late
  approach~\cite{Ong93,PittsSurvey}.
\end{remark} 

\begin{remark}
\label{r:pabfirenze} 
Defining applicative bisimulation in terms of multisorted labelled Markov
chains has the advantage of recasting the definition in a familiar
framework; most importantly, this formulation will be useful when dealing
with Howe's method. To spell out the explicit operational details of the
definition, a probabilistic applicative bisimulation can be seen as an equivalence
relation  such
that whenever :
  \begin{varenumerate}
  \item , for any
    equivalence class  of  
(that
    is, the probability of reaching a value in  is the same for the
    two terms);
  \item
    \label{cl:ct} if  and  are values, say  and , then , for
    all .
  \end{varenumerate}
  The special treatment of values, in Clause~\ref{cl:ct}., motivates the
  use of \emph{multisorted} labelled Markov chains in Definition~\ref{d:multisort}.
\end{remark}

As usual, one way to show that any two terms are bisimilar is to prove that one relation
containing the pair in question is a . 
Terms with the same semantics are indistinguishable:
\begin{lemma}\label{lemma:samesem}
  The binary relation  is a .
\end{lemma}
\begin{proof}
  The fact  is an equivalence easily follows from reflexivity,
  symmetry and transitivity of set-theoretic equality.  must
  satisfy the following property for closed terms: if
  , then for every
  ,
  . Notice
  that if , then clearly
  ,
  for every . With the same hypothesis,
  
  Moreover,  must satisfy the following property for cloned
  values: if
  , then for
  every close term  and for every
  ,
  . Now,
  the hypothesis
  
  implies . Then clearly
  
  for every . With the same hypothesis,
  
  This concludes the proof.
\end{proof}
Please notice that the previous result yield a nice consequence: for every
,
. Indeed,
Lemma~\ref{lemma:semsumCBN} tells us that the latter terms have the same
semantics.

Conversely, knowing that two terms  and
 are (bi)similar means knowing quite a lot about their
convergence probability:
\begin{lemma}[Adequacy of Bisimulation]\label{lemma:sumsempabCBN}
  If , then . Moreover,
  if , then .
\end{lemma}
\begin{proof}
   
  And, 
  
  This concludes the proof.
\end{proof}
\begin{example}
  Bisimilar terms do not necessarily have the same semantics. After all, this is one
  reason for using bisimulation, and its proof method, as basis to prove
  fine-grained equalities among functions. Let us consider the
  following terms:
                           
  Their semantics                      
  differ, as for every value , we have:
  10pt]
  \sem{\termtwo}(\valone)& =&  \left\{
    \begin{array}{ll}
      \frac{1}{2} &\mbox{if }\valone \mbox{ is } \lambda x. I x;\\
      0 & \mbox{otherwise}.
    \end{array}
    \right.
    \end{array}
   \translop(\termone,
   \evlabel,\econe)=\sum_{\valone\in\econe}\sem{\termone}(\valone) =
   \frac{1}{2} = \sum_{\valone\in\econe}\sem{\termtwo}(\valone) =
   \translop(\termtwo,\evlabel,\econe).
   
    1 = \translop(\clabstr{\varone}{\termone},\termthree,\ectwo) =
    \translop(\clabstr{\varone}{\termtwo},\termthree,\ectwo) = 0.
  
  \sum_{\termfour\in\econe}\translop(\clabstr{\varone}{\termone},\termthree,\termfour)
  = \left\{
    \begin{array}{ll}
      1 &\mbox{if }\subst{\termone}{\varone}{\termthree}\in\econe;\\
      0 & \mbox{otherwise}
    \end{array}
  \right.  = \left\{
    \begin{array}{ll}
      1 &\mbox{if }\subst{\termtwo}{\varone}{\termthree}\in\econe;\\
      0 & \mbox{otherwise}
    \end{array}
  \right.  =
  \sum_{\termfour\in\econe}\translop(\clabstr{\varone}{\termtwo},\termthree,\termfour)
  
  \label{eq:SubCBN}
  \rel{\vecvarone\cup\{\varone\}}{\termone}{\relone}{\termtwo}\wedge
  \rel{\vecvarone}{\termthree}{\relone}{\termfour}\Rightarrow
  \rel{\vecvarone}{\subst{\termone}{\varone}{\termthree}}{\relone}{\subst{\termtwo}{\varone}{\termfour}}.

  \label{eq:CusCBN}
  \rel{\vecvarone\cup\{\varone\}}{\termone}{\relone}{\termtwo}\wedge\termthree\in\LOPp{\vecvarone}\Rightarrow
  \rel{\vecvarone}{\subst{\termone}{\varone}{\termthree}}{\relone}{\subst{\termtwo}{\varone}{\termthree}}.

  \infer[\Howeone]
  {\rel{\vecvarone}{\varone}{\howe{\relone}}{\termone}}
  {\rel{\vecvarone}{\varone}{\relone}{\termone}}
  \qquad
  \infer[\Howetwo]
  {\rel{\vecvarone}{\abstr{\varone}{\termone}}{\howe{\relone}}{\termtwo}}
  {
    \rel{\vecvarone\cup\{\varone\}}{\termone}{\howe{\relone}}{\termthree}
    &&
    \rel{\vecvarone}{\abstr{\varone}{\termthree}}{\relone}{\termtwo}
    &&
    \varone\notin\vecvarone
  }
  
  \infer[\Howethree]
  {\rel{\vecvarone}{\app{\termone}{\termtwo}}{\howe{\relone}}{\termthree}}
  {
    \rel{\vecvarone}{\termone}{\howe{\relone}}{\termfour}
    &&
    \rel{\vecvarone}{\termtwo}{\howe{\relone}}{\termfive}
    &&
    \rel{\vecvarone}{\app{\termfour}{\termfive}}{\relone}{\termthree}
  }
  
  \infer[\Howefour]
  {\rel{\vecvarone}{\ps{\termone}{\termtwo}}{\howe{\relone}}{\termthree}}
  {
    \rel{\vecvarone}{\termone}{\howe{\relone}}{\termfour}
    &&
    \rel{\vecvarone}{\termtwo}{\howe{\relone}}{\termfive}
    &&
    \rel{\vecvarone}{\ps{\termfour}{\termfive}}{\relone}{\termthree}
  }
  
      \forall\vecvarone\in\powfin{\setvar},
      \varone\in\vecvarone\Rightarrow\rel{\vecvarone}{\varone}{\howe{\relone}}{\varone}.
      
      \infer[\Howeone]
      {\rel{\vecvarone}{\varone}{\howe{\relone}}{\varone}}
      {\infer{\rel{\vecvarone}{\varone}{\relone}{\varone}} {}}
      
      \rel{\vecvarone\cup\{\varone\}}{\termone}{\howe{\relone}}{\termtwo}\Rightarrow
      \rel{\vecvarone}{\abstr{\varone}{\termone}}{\howe{\relone}}{\abstr{\varone}{\termtwo}}.
      
      \infer[\Howetwo]
      {\rel{\vecvarone}{\abstr{\varone}{\termone}}{\howe{\relone}}{\abstr{\varone}{\termtwo}}}
      { \rel{\vecvarone\cup\{\varone\}}{\termone}{\howe{\relone}}{\termtwo}
        &&
        \infer{\rel{\vecvarone}{\abstr{\varone}{\termtwo}}{\relone}{\abstr{\varone}{\termtwo}}}{}
        && \varone\notin\vecvarone }
      
      \rel{\vecvarone}{\termone}{\howe{\relone}}{\termtwo}\wedge\rel{\vecvarone}{\termthree}{\howe{\relone}}{\termfour}\Rightarrow
      \rel{\vecvarone}{\app{\termone}{\termthree}}{\howe{\relone}}{\app{\termtwo}{\termfour}}.
      
      \infer[\Howethree]
      {\rel{\vecvarone}{\app{\termone}{\termthree}}{\howe{\relone}}{\app{\termtwo}{\termfour}}}
      { \rel{\vecvarone}{\termone}{\howe{\relone}}{\termtwo} &&
        \rel{\vecvarone}{\termthree}{\howe{\relone}}{\termfour} &&
        \infer{\rel{\vecvarone}{\app{\termtwo}{\termfour}}{\relone}{\app{\termtwo}{\termfour}}}{}}
      
      \rel{\vecvarone}{\termone}{\howe{\relone}}{\termtwo}\wedge\rel{\vecvarone}{\termthree}{\howe{\relone}}{\termfour}\Rightarrow
      \rel{\vecvarone}{\ps{\termone}{\termthree}}{\howe{\relone}}{\ps{\termtwo}{\termfour}}.
      \infer[\Howefour]
      {\rel{\vecvarone}{\ps{\termone}{\termthree}}{\howe{\relone}}{\ps{\termtwo}{\termfour}}}
      { \rel{\vecvarone}{\termone}{\howe{\relone}}{\termtwo} &&
        \rel{\vecvarone}{\termthree}{\howe{\relone}}{\termfour} &&
        \infer{\rel{\vecvarone}{\ps{\termtwo}{\termfour}}{\relone}{\ps{\termtwo}{\termfour}}}{}
      }
      
    \infer[\Howeone]
    {\rel{\vecvarone}{\varone}{\howe{\relone}}{\termthree}}
    {\infer{\rel{\vecvarone}{\varone}{\relone}{\termthree}} {
        \rel{\vecvarone}{\varone}{\relone}{\termtwo} &&
        \rel{\vecvarone}{\termtwo}{\relone}{\termthree} } }
    
    \infer[\Howetwo]
    {\rel{\vecvarone}{\abstr{\varone}{\termfive}}{\howe{\relone}}{\termthree}}
    {
      \rel{\vecvarone\cup\{\varone\}}{\termfive}{\howe{\relone}}{\termfour}
      &&
      \infer{\rel{\vecvarone}{\abstr{\varone}{\termfour}}{\relone}{\termthree}}{
        \rel{\vecvarone}{\abstr{\varone}{\termfour}}{\relone}{\termtwo}
        && \rel{\vecvarone}{\termtwo}{\relone}{\termthree}}}
    
    \infer[\Howethree]
    {\rel{\vecvarone}{\app{\termsix}{\termseven}}{\howe{\relone}}{\termthree}}
    { \rel{\vecvarone}{\termsix}{\howe{\relone}}{\termfour} &&
      \rel{\vecvarone}{\termseven}{\howe{\relone}}{\termfive} &&
      \infer{\rel{\vecvarone}{\app{\termfour}{\termfive}}{\relone}{\termthree}}{
        \rel{\vecvarone}{\app{\termfour}{\termfive}}{\relone}{\termtwo} &&
        \rel{\vecvarone}{\termtwo}{\relone}{\termthree}
      }
    }
    
    \infer[\Howefour]
    {\rel{\vecvarone}{\ps{\termsix}{\termseven}}{\howe{\relone}}{\termthree}}
    { \rel{\vecvarone}{\termsix}{\howe{\relone}}{\termfour} &&
      \rel{\vecvarone}{\termseven}{\howe{\relone}}{\termfive} &&
      \infer{\rel{\vecvarone}{\ps{\termfour}{\termfive}}{\relone}{\termthree}}{
        \rel{\vecvarone}{\ps{\termfour}{\termfive}}{\relone}{\termtwo}
        && \rel{\vecvarone}{\termtwo}{\relone}{\termthree} } }
    
    \infer[\Howeone]
    {\rel{\vecvarone}{\varone}{\howe{\relone}}{\termtwo}}
    {\rel{\vecvarone}{\varone}{\relone}{\termtwo}}
    
    \infer[\Howetwo]
    {\rel{\vecvarone}{\abstr{\varone}{\termfive}}{\howe{\relone}}{\termtwo}}
    {
      \rel{\vecvarone\cup\{\varone\}}{\termfive}{\howe{\relone}}{\termfive}
      &&
      \rel{\vecvarone}{\abstr{\varone}{\termfive }}{\relone}{\termtwo}
      && \varone\notin\vecvarone }
    
    \infer[\Howethree]
    {\rel{\vecvarone}{\app{\termthree}{\termfour}}{\howe{\relone}}{\termtwo}}
    { \rel{\vecvarone}{\termthree}{\howe{\relone}}{\termthree} &&
      \rel{\vecvarone}{\termfour}{\howe{\relone}}{\termfour} &&
      \rel{\vecvarone}{\app{\termthree}{\termfour}}{\relone}{\termtwo}
    }
    
    \infer[\Howefour]
    {\rel{\vecvarone}{\ps{\termthree}{\termfour}}{\howe{\relone}}{\termtwo}}
    { \rel{\vecvarone}{\termthree}{\howe{\relone}}{\termthree} &&
      \rel{\vecvarone}{\termfour}{\howe{\relone}}{\termfour} &&
      \rel{\vecvarone}{\ps{\termthree}{\termfour}}{\relone}{\termtwo} }
    
    \rel{\vecvarone\cup\{\varone\}}{\termone}{\howe{\relone}}{\termtwo}\wedge
    \rel{\vecvarone}{\termthree}{\howe{\relone}}{\termfour}\Rightarrow
    \rel{\vecvarone}{\subst{\termone}{\varone}{\termthree}}{\howe{\relone}}{\subst{\termtwo}{\varone}{\termfour}}.
  
    \infer[]
    {\rel{\vecvarone}{\termthree}{\howe{\relone}}{\subst{\termtwo}{\varone}{\termfour}}}
    { \rel{\vecvarone}{\termthree}{\howe{\relone}}{\termfour} &&
      \rel{\vecvarone}{\termfour}{\relone}{\subst{\termtwo}{\varone}{\termfour}}
    }
    
    \infer[\Howetwo]
    {\rel{\vecvarone\cup\{\varone\}}{\abstr{\vartwo}{\termfive}}{\howe{\relone}}{\termtwo}}
    {
      \rel{\vecvarone\cup\{\varone,\vartwo\}}{\termfive}{\howe{\relone}}{\termsix}
      &&
      \rel{\vecvarone\cup\{\varone\}}{\abstr{\vartwo}{\termsix}}{\relone}{\termtwo}
      && \varone,\vartwo\notin\vecvarone }
    
    \infer[\Howetwo]
    {\rel{\vecvarone}{\subst{\abstr{\vartwo}{\termfive}}{\varone}{\termthree}}{\howe{\relone}}{\subst{\termtwo}{\varone}{\termfour}}}
    {
      \rel{\vecvarone\cup\{\vartwo\}}{\subst{\termfive}{\varone}{\termthree}}{\howe{\relone}}{\subst{\termsix}{\varone}{\termfour}}
      &&
      \rel{\vecvarone}{\subst{\abstr{\vartwo}{\termsix}}{\varone}{\termfour}}{\relone}{\subst{\termtwo}{\varone}{\termfour}}
      && \vartwo\notin\vecvarone } 
    
    \infer[\Howethree]
    {\rel{\vecvarone\cup\{\varone\}}{\app{\termfive}{\termsix}}{\howe{\relone}}{\termtwo}}
    {
      \rel{\vecvarone\cup\{\varone\}}{\termfive}{\howe{\relone}}{\termfive'}
      &&
      \rel{\vecvarone\cup\{\varone\}}{\termsix}{\howe{\relone}}{\termsix'}
      &&
      \rel{\vecvarone\cup\{\varone\}}{\app{\termfive'}{\termsix'}}{\relone}{\termtwo}
    }
    
    \infer[\Howethree]
    {\rel{\vecvarone}{\app{\subst{\termfive}{\varone}{\termthree}}{\subst{\termsix}{\varone}{\termthree}}}{\howe{\relone}}{\subst{\termtwo}{\varone}{\termfour}}}
    {
      \rel{\vecvarone}{\subst{\termfive}{\varone}{\termthree}}{\howe{\relone}}{\subst{\termfive'}{\varone}{\termfour}}
      &&
      \rel{\vecvarone}{\subst{\termsix}{\varone}{\termthree}}{\howe{\relone}}{\subst{\termsix'}{\varone}{\termfour}}
      &&
      \rel{\vecvarone}{\app{\subst{\termfive'}{\varone}{\termfour}}{\subst{\termsix'}{\varone}{\termfour}}}{\relone}{\subst{\termtwo}{\varone}{\termfour}}}
    
    \infer[\Howefour]
    {\rel{\vecvarone\cup\{\varone\}}{\ps{\termfive}{\termsix}}{\howe{\relone}}{\termtwo}}
    {
      \rel{\vecvarone\cup\{\varone\}}{\termfive}{\howe{\relone}}{\termfive'}
      &&
      \rel{\vecvarone\cup\{\varone\}}{\termsix}{\howe{\relone}}{\termsix'}
      &&
      \rel{\vecvarone\cup\{\varone\}}{\ps{\termfive'}{\termsix'}}{\relone}{\termtwo}
    }
    
    \infer[\Howefour]
    {\rel{\vecvarone}{\ps{\subst{\termfive}{\varone}{\termthree}}{\subst{\termsix}{\varone}{\termthree}}}{\howe{\relone}}{\subst{\termtwo}{\varone}{\termfour}}}
    {
      \rel{\vecvarone}{\subst{\termfive}{\varone}{\termthree}}{\howe{\relone}}{\subst{\termfive'}{\varone}{\termfour}}
      &&
      \rel{\vecvarone}{\subst{\termsix}{\varone}{\termthree}}{\howe{\relone}}{\subst{\termsix'}{\varone}{\termfour}}
      &&
      \rel{\vecvarone}{\ps{\subst{\termfive'}{\varone}{\termfour}}{\subst{\termsix'}{\varone}{\termfour}}}{\relone}{\subst{\termtwo}{\varone}{\termfour}}
    }
    
\infer[\TCone]
{\rel{\vecvarone}{\termone}{\tcrel{\relone}}{\termtwo}}
{\rel{\vecvarone}{\termone}{\relone}{\termtwo}}

\infer[\TCtwo]
{\rel{\vecvarone}{\termone}{\tcrel{\relone}}{\termthree}} {
  \rel{\vecvarone}{\termone}{\tcrel{\relone}}{\termtwo} &&
  \rel{\vecvarone}{\termtwo}{\tcrel{\relone}}{\termthree} }

    \forall\vecvarone\in\powfin{\setvar}, \varone\in\vecvarone\Rightarrow
    \rel{\vecvarone}{\varone}{\relone}{\varone}.
    
    \rel{\vecvarone\cup\{\varone\}}{\termone}{\tcrel{\relone}}{\termtwo}\Rightarrow
    \rel{\vecvarone}{\abstr{\varone}{\termone}}{\tcrel{\relone}}{\abstr{\varone}{\termtwo}}.
    
    \rel{\vecvarone}{\termone}{\tcrel{\relone}}{\termtwo}\ \wedge\
    \rel{\vecvarone}{\termthree}{\tcrel{\relone}}{\termfour}\Rightarrow
    \rel{\vecvarone}{\termone\termthree}{\tcrel{\relone}}{\termtwo\termfour}.
    
      &\label{equ:com3lp}\forall
      \termone,\termtwo,\termthree,\termfour\in\LOP(\vecvarone).\
      \rel{\vecvarone}{\termone}{\tcrel{\relone}}{\termtwo}\ \wedge \
      \rel{\vecvarone}{\termthree}{\relone}{\termfour}
      \Rightarrow \rel{\vecvarone}{\app{\termone}{\termthree}}{\tcrel{\relone}}{\app{\termtwo}{\termfour}},\\
      &\label{equ:com3rp}\forall
      \termone,\termtwo,\termthree,\termfour\in\LOP(\vecvarone).\
      \rel{\vecvarone}{\termone}{\relone}{\termtwo}\ \wedge \
      \rel{\vecvarone}{\termthree}{\tcrel{\relone}}{\termfour} \Rightarrow
      \rel{\vecvarone}{\app{\termone}{\termthree}}{\tcrel{\relone}}{\app{\termtwo}{\termfour}}.
    
    \rel{\vecvarone}{\termone}{\tcrel{\relone}}{\termtwo}\ \wedge\
    \rel{\vecvarone}{\termthree}{\tcrel{\relone}}{\termfour}\Rightarrow
    \rel{\vecvarone}{\ps{\termone}{\termthree}}{\tcrel{\relone}}{\ps{\termtwo}{\termfour}}.
    
  \rel{\vecvarone\cup\{\varone\}}{\termone}{\tcrel{\relone}}{\termtwo}\wedge\termthree\in\LOPp{\vecvarone}\Rightarrow
  \rel{\vecvarone}{\subst{\termone}{\varone}{\termthree}}{\tcrel{\relone}}{\subst{\termtwo}{\varone}{\termthree}}.
  \label{equ:probsimCBN}
      \translop(\termone,\evlabel,\clabstr{\varone}{\setone})\leq\translop(\termtwo,\evlabel,\relone(\clabstr{\varone}{\setone})).
    
        \translop(\termone,\evlabel,\clabstr{\varone}{\setone})&=\sem{\termone}(\abstr{\varone}{\setone})\\
        &\leq
        \sem{\termtwo}(\abstr{\varone}{\howe{\cbnpas}(\setone)})\\ &\leq\sem{\termtwo}(\abstr{\varone}{\tcrel{(\howe{\cbnpas})}(\setone)})\\
        &\leq\sem{\termtwo}(\relone(\clabstr{\varone}{\setone}))=\translop(\termtwo,\evlabel,\relone(\clabstr{\varone}{\setone})).
      
        \translop(\termone,\evlabel,\setone)&\leq\translop(\termfour,\evlabel,\relone(\setone)),\\
        \translop(\termfour,\evlabel,\relone(\setone))&\leq\translop(\termtwo,\evlabel,\relone(\relone(\setone))).
      
      \translop(\termone,\evlabel,\setone)\leq\translop(\termtwo,\evlabel,\relone(\setone))
      
    \translop(\clabstr{\varone}{\termone},\termthree,\setone)\leq\translop(\clabstr{\varone}{\termtwo},\termthree,\relone(\setone)).
    
      \translop(\clabstr{\varone}{\termone},\termthree,\setone)&=1\\ 
      &=\translop(\clabstr{\varone}{\termtwo},\termthree,\tcrel{(\howe{\cbnpas})}(\setone))\\
      &=\translop(\clabstr{\varone}{\termtwo},\termthree,\relone(\setone)).
    
    \translop(\clabstr{\varone}{\termone},\termthree,\setone)=0\leq\translop(\clabstr{\varone}{\termtwo},\termthree,\relone(\setone)).
    
\sum_{i\in\indsetone}\probone_i\leq\meas{\bigcup_{i\in\indsetone}\setone_i},

      \probone_k \leq \sum_{k\in\indsetone}
      \realtwo_{k,\indsetone}\cdot\overline{\flone}_{(\indsetone,t)} \leq
      \sum_{k\in\indsetone} \realtwo_{k,\indsetone}\cdot\realone_\indsetone
      
    \clabstr{\varone}{\termone}\in\cbnpas(\clabstr{\varone}{\setone})&\Leftrightarrow
    \exists\termtwo\in\setone.\,\clabstr{\varone}{\termtwo}\cbnpas\clabstr{\varone}{\termone}\\
    &\Leftrightarrow\exists\termtwo\in\setone.\,\termtwo\cbnpas\termone\\
    &\Leftrightarrow\clabstr{\varone}{\termone}\in\clabstr{\varone}{\cbnpas(\setone)}.
  
      \infer
          {\rel{\emcon}{\abstr{\varone}{\termthree}}{\howe{\cbnpas}}{\termtwo}}
          {
          \rel{\{\varone\}}{\termthree}{\howe{\cbnpas}}{\termfour}
          &&
          \rel{\emcon}{\abstr{\varone}{\termfour}}{\cbnpas}{\termtwo}
          }
      
      \sem{\termtwo}(\abstr{\varone}{\howe{\cbnpas}(\setone)})\geq
      \sem{\termtwo}(\abstr{\varone}{\cbnpas(\termfour)})\geq
      \sem{\abstr{\varone}{\termfour}}(\abstr{\varone}{\termfour})=1.
      
      \infer
          {\ibsemn{\termthree\termfour}{\sum_{\termfive}}\distthree(\abstr{\varone}{\termfive})\cdot\distfive_{\termfive,\termfour}}
          { \ibsemn{\termthree}{\distthree} &&
            \{\ibsemn{\subst{\termfive}{\varone}{\termfour}}{\distfive_{\termfive,\termfour}}\}_{\termfive,\termfour}
          }
          
      \infer
          {\rel{\emcon}{\app{\termthree}{\termfour}}{\howe{\cbnpas}}{\termtwo}}
          { \rel{\emcon}{\termthree}{\howe{\cbnpas}}{\termsix} &&
            \rel{\emcon}{\termfour}{\howe{\cbnpas}}{\termseven} &&
          \rel{\emcon}{\app{\termsix}{\termseven}}{\cbnpas}{\termtwo} }
      
      \distone=\sum_{\termfive}\distthree(\abstr{\varone}{\termfive})\cdot\distfive_{\termfive,\termfour}.
      
      \distthree(\abstr{\varone}{\setone})\leq\sem{\termsix}(\abstr{\varone}{\howe{\cbnpas}(\setone)}).
      
      \termnine\in\howe{\cbnpas}(\{\termfive_1,\ldots,\termfive_n\})=\bigcup_{1\leq
        i\leq n}\howe{\cbnpas}(\termfive_i)
      
        \sem{\termsix}(\abstr{\varone}{\termnine})&\geq
        \sum_{1\leq i\leq n}\realone_i^{\termnine,\termsix}\qquad\forall\,\termnine\in\bigcup_{1\leq i\leq n}\howe{\cbnpas}(\termfive_i);\\
        \distthree(\abstr{\varone}{\termfive_i})&\leq\sum_{\termnine\in\howe{\cbnpas}(\termfive_i)}\realone_i^{\termnine,\termsix}\qquad\forall\,1\leq i\leq n.
      
        \distone&\leq\sum_{1\leq i\leq
          n}\left(\sum_{\termnine\in\howe{\cbnpas}(\termfive_i)}\realone_i^{\termnine,\termsix}\right)\cdot
        \distfive_{\termfive_i,\termfour}\\
        &=\sum_{1\leq i\leq
          n}\sum_{\termnine\in\howe{\cbnpas}(\termfive_i)}\realone_i^{\termnine,\termsix}\cdot\distfive_{\termfive_i,\termfour}.
      
          &\distone(\abstr{\varone}{\setone})\leq\sum_{1\leq i\leq
            n}\sum_{\termnine\in\howe{\cbnpas}(\termfive_i)}
          \realone_i^{\termnine,\termsix}\cdot\sem{\subst{\termnine}{\varone}{\termseven}}(\abstr{\varone}{\howe{\cbnpas}{(\setone)}})\\
          &\leq\sum_{1\leq i\leq
            n}\sum_{\termnine\in\howe{\cbnpas}(\{\termfive_1,\ldots,\termfive_n\})}\realone_i^{\termnine,\termsix}\cdot
            \sem{\subst{\termnine}{\varone}{\termseven}}(\abstr{\varone}{\howe{\cbnpas}{(\setone)}})\\
          &=\sum_{\termnine\in\howe{\cbnpas}(\{\termfive_1,\ldots,\termfive_n\})}\sum_{1\leq
            i\leq
            n}\realone_i^{\termnine,\termsix}\cdot\sem{\subst{\termnine}{\varone}{\termseven}}(\abstr{\varone}{\howe{\cbnpas}{(\setone)}})\\
          &=\sum_{\termnine\in\howe{\cbnpas}(\{\termfive_1,\ldots,\termfive_n\})}\left(\sum_{1\leq
              i\leq n}\realone_i^{\termnine,\termsix}\right)\cdot\sem{\subst{\termnine}{\varone}{\termseven}}(\abstr{\varone}{\howe{\cbnpas}{(\setone)}})\\
          &\leq\sum_{\termnine\in\howe{\cbnpas}(\{\termfive_1,\ldots,\termfive_n\})}
          \sem{\termsix}(\abstr{\varone}{\termnine})\cdot
          \sem{\subst{\termnine}{\varone}{\termseven}}(\abstr{\varone}{\howe{\cbnpas}{(\setone)}})\\
          &\leq\sum_{\termnine\in\LOP(\varone)}
          \sem{\termsix}(\abstr{\varone}{\termnine})\cdot
          \sem{\subst{\termnine}{\varone}{\termseven}}(\abstr{\varone}{\howe{\cbnpas}{(\setone)}})\\
          &=\sem{\termsix\termseven}(\abstr{\varone}{\howe{\cbnpas}{(\setone)}})\leq
          \sem{\termtwo}(\abstr{\varone}{\cbnpas((\howe{\cbnpas})(\setone))})\\
          &\leq\sem{\termtwo}(\abstr{\varone}{\howe{\cbnpas}(\setone)}),
          
        \infer
        {\ibsemn{\ps{\termthree}{\termfour}}{\frac{1}{2}\cdot\distthree
            + \frac{1}{2}\cdot\distfour}} {
          \ibsemn{\termthree}{\distthree} && \ibsemn{\termfour}{\distfour}
        }
        
        \infer
        {\rel{\emcon}{\ps{\termthree}{\termfour}}{\howe{\cbnpas}}{\termtwo}}
        {
          \rel{\emcon}{\termthree}{\howe{\cbnpas}}{\termsix}
          &&
          \rel{\emcon}{\termfour}{\howe{\cbnpas}}{\termseven}
          &&
          \rel{\emcon}{\ps{\termsix}{\termseven}}{\cbnpas}{\termtwo}
        }
        
        \distone=\frac{1}{2}\cdot\distthree
        + \frac{1}{2}\cdot\distfour.
        
          \distone(\abstr{\varone}{\setone})&=\frac{1}{2}\cdot\distthree(\abstr{\varone}{\setone})
          +
          \frac{1}{2}\cdot\distfour(\abstr{\varone}{\setone})\\
          &\leq
          \frac{1}{2}\cdot\sem{\termsix}(\abstr{\varone}{\howe{\cbnpas}(\setone)})
          +
          \frac{1}{2}\cdot\sem{\termseven}(\abstr{\varone}{\howe{\cbnpas}(\setone)})\\
          &=
          \sem{\ps{\termsix}{\termseven}}(\abstr{\varone}{\howe{\cbnpas}(\setone)}).
        
  \ctxone,\ctxtwo\in\ctxset \, ::= \, \ctxhole{\cdot}\, |\,
  \abstr{\varone}{\ctxone}\, |\, \app{\ctxone}{\termone}\, |\,
  \app{\termone}{\ctxone}\, |\, \ps{\ctxone}{\termone}\, |\,
  \ps{\termone}{\ctxone}.
  
  \ctxhole{\cdot}\ctxhole{\termtwo} &\defi \termtwo;\\
  (\abstr{\varone}{\ctxone})\ctxhole{\termtwo} &\defi
  \abstr{\varone}{\ctxone\ctxhole{\termtwo}};\\
  (\app{\ctxone}{\termone})\ctxhole{\termtwo} &\defi
  \app{\ctxone\ctxhole{\termtwo}}{\termone};\\
  (\app{\termone}{\ctxone})\ctxhole{\termtwo} &\defi
  \app{\termone}{\ctxone\ctxhole{\termtwo}};\\
  (\ps{\ctxone}{\termone})\ctxhole{\termtwo} &\defi
  \ps{\ctxone\ctxhole{\termtwo}}{\termone};\\
  (\ps{\termone}{\ctxone})\ctxhole{\termtwo} &\defi
  \ps{\termone}{\ctxone\ctxhole{\termtwo}}.

\infer[\Ctxone] {\ctxhole{\cdot}\in\ctxsetp{\vecvarone}{\vecvarone}}{ }

\infer[\Ctxtwo]
{\abstr{\varone}{\ctxone}\in\ctxsetp{\vecvarone}{\vecvartwo}}
{\ctxone\in\ctxsetp{\vecvarone}{\vecvartwo\cup\{\varone\}} &&
  \varone\not\in\vecvartwo}

\infer[\Ctxthree]
{\app{\ctxone}{\termone}\in\ctxsetp{\vecvarone}{\vecvartwo}}
{\ctxone\in\ctxsetp{\vecvarone}{\vecvartwo} && \termone\in\LOP(\vecvartwo)}

\infer[\Ctxfour]
{\app{\termone}{\ctxone}\in\ctxsetp{\vecvarone}{\vecvartwo}}
{\termone\in\LOP(\vecvartwo) && \ctxone\in\ctxsetp{\vecvarone}{\vecvartwo}
}

\infer[\Ctxfive]
{\ps{\ctxone}{\termone}\in\ctxsetp{\vecvarone}{\vecvartwo}}
{\ctxone\in\ctxsetp{\vecvarone}{\vecvartwo} && \termone\in\LOP(\vecvartwo)}

\infer[\Ctxsix] {\ps{\termone}{\ctxone}\in\ctxsetp{\vecvarone}{\vecvartwo}}
{\termone\in\LOP(\vecvartwo) && \ctxone\in\ctxsetp{\vecvarone}{\vecvartwo}
}

  \label{eq:cfctxCBN}
  \cbncfleq \defi \bigcup\,\caset.

  \rel{\vecvarone}{\termone}{\relone}{\termthree}\,\wedge\,\rel{\vecvarone}{\termfive}{\relone}{\termfour}\Rightarrow
  \rel{\vecvarone}{\app{\termone}{\termfive}}{\relone}{\app{\termthree}{\termfour}};
  
  \rel{\vecvarone}{\termthree}{\reltwo}{\termtwo}\,\wedge\,\rel{\vecvarone}{\termfour}{\reltwo}{\termsix}\Rightarrow
  \rel{\vecvarone}{\app{\termthree}{\termfour}}{\reltwo}{\app{\termtwo}{\termsix}}.
  
    \rel{\vecvarone}{\termone}{\cbncfleq}{\termtwo}&\Rightarrow
    \rel{\emptyset}{\ctxone\ctxhole{\termone}}{\cbncfleq}{\ctxone\ctxhole{\termtwo}}
  
    \ctxone\ctxhole{\termone}\evp{\probone}\Rightarrow\ctxone\ctxhole{\termtwo}\evp{\probtwo},\textnormal{
      with }\probone\leq\probtwo
  
    \rel{\vecvarone}{\termone}{\cbnconleq}{\termtwo}.
  
  \fsone,\fstwo::=\nil\midd\las{\termone}{\fsone}.
  
  \stktm{\nil}{\termone}&\defi\termone;\\
  \stktm{\las{\termone}{\fsone}}{\termtwo}&\defi\stktm{\fsone}{\termtwo\termone}.

  (\fsone,\termone\termtwo)&\cbnfsred(\las{\termtwo}{\fsone},\termone);\\
  (\fsone,\ps{\termone}{\termtwo})&\cbnfsred(\fsone,\termone),(\fsone,\termtwo);\\
  (\las{\termone}{\fsone},\abstr{\varone}{\termtwo})&\cbnfsred(\fsone,\subst{\termtwo}{\varone}{\termone}).

\infer[\Howeredone] {\cbnfscp{\fsone}{\termone}{0}} {}

\infer[\Howeredtwo] {\cbnfscp{\nil}{\valone}{1}} {}

\infer[\Howeredthree]
{\cbnfscp{\fsone}{\termone}{\frac{1}{n}\sum_{i=1}^n\probone_i}}
{(\fsone,\termone)\cbnfsred(\fstwo_1,\termtwo_1),\ldots,(\fstwo_n,\termtwo_n)
  & \cbnfscp{\fstwo_i}{\termtwo_i}{\probone_i}}
\label{eq:stacksem}
    \cbnfscp{\fsone}{\termone}{\probone}\;\Longleftrightarrow\;
    \ibsemn{\stktm{\fsone}{\termone}}{\distone} \textnormal{ with }
    \sum\distone=\probone.
  
    p=\cbnfssup{\fsone}{\termone}&=\sup_{\probtwo\in\RRN}\cbnfscp{\fsone}{\termone}{\probtwo}
    =\sup_{\ibsemn{\stktm{\fsone}{\termone}}{\distone}}\sum\distone\\&=\sum\sup_{\ibsemn{\stktm{\fsone}{\termone}}{\distone}}\distone=
    \sum\sem{\stktm{\fsone}{\termone}}=\stktm{\fsone}{\termone}\evp{\probone},
  
\termone\howe{(\cbnciuleq)}\termtwo\Rightarrow\termone\cbnciuleq\termtwo.

\infer[\Howestkone] {\nil\howe{\relone}\nil} {}

\infer[\Howestktwo]
{(\las{\termone}{\fsone})\howe{\relone}(\las{\termtwo}{\fstwo})} {
  \rel{\emptyset}{\termone}{\howe{\relone}}{\termtwo} &&
  \fsone\howe{\relone}\fstwo }

    \infer[\Howeredthree]
    {\cbnfscp{\fsone}{\app{(\abstr{\varone}{\termone})}{\termtwo}}{\probone}
    }
    {(\fsone,\termone)\cbnfsred(\las{\termtwo}{\fsone},\abstr{\varone}{\termone})
      & \infer[\Howeredthree]
      {\cbnfscp{\las{\termtwo}{\fsone}}{\abstr{\varone}{\termone}}{\probone}
      }
      {(\las{\termtwo}{\fsone},\abstr{\varone}{\termone})\cbnfsred(\fsone,\subst{\termone}{\varone}{\termtwo})
        & \cbnfscp{\fsone}{\subst{\termone}{\varone}{\termtwo}}{\probone}}
    }
    
    \infer[\Howeredthree] {\cbnfscp{\fsone}{\termone}{\frac{1}{n}\sum_{i=1}^n\probone_i}}
    {(\fsone,\termone)\cbnfsred(\fsthree_1,\termthree_1),\ldots,(\fsthree_n,\termthree_n)
      & \cbnfscp{\fsthree_i}{\termthree_i}{\probone_i}}
    
      \infer[\Howestktwo]
      {\rel{\emptyset}{\fsthree_1}{\howe{(\cbnciuleq)}}{\las{\termseven}{\fstwo}}}
      {\rel{\emptyset}{\termfive}{\howe{(\cbnciuleq)}}{\termseven} &
        \rel{\emptyset}{\fsone}{\howe{(\cbnciuleq)}}{\fstwo}}
      
      (\fstwo,\termsix\termseven)\cbnfsred(\las{\termseven}{\fstwo},\termsix),
      \label{eq:fsred}
        (\fstwo,\abstr{\varone}{\termseven})=(\las{\termsix}{\fsfour},\abstr{\varone}{\termseven})
        \cbnfsred(\fsfour,\subst{\termseven}{\varone}{\termsix}).
      
      \cbnfssup{\fstwo}{\termtwo}\geq\cbnfssup{\fstwo}{\abstr{\varone}{\termseven}}=
      \cbnfssup{\fsfour}{\subst{\termseven}{\varone}{\termsix}}\geq\probone.
      
    \termone&\defi\abstr{\varone}{\abstr{\vartwo}{\ps{\varone}{\vartwo}}};\\
    \termtwo&\defi\ps{(\abstr{\varone}{\abstr{\vartwo}{\varone}})}{(\abstr{\varone}{\abstr{\vartwo}{\vartwo}})}.
  
      (\fsone,\termone)&\cbnfsred(\las{\termfour}{\fstwo},\abstr{\vartwo}{\ps{\termthree}{\vartwo}})\cbnfsred(\fstwo,\ps{\termthree}{\termfour})\\
      &\cbnfsred(\fstwo,\termthree),(\fstwo,\termfour);\\
      (\fsone,\termtwo)&\cbnfsred(\fsone,\abstr{\varone}{\abstr{\vartwo}{\varone}}),(\fsone,\abstr{\varone}{\abstr{\vartwo}{\vartwo}});\\
      (\fsone,\abstr{\varone}{\abstr{\vartwo}{\varone}})&\cbnfsred(\las{\termfour}{\fstwo},\abstr{\vartwo}{\termthree})\cbnfsred(\fstwo,\termthree);\\
      (\fsone,\abstr{\varone}{\abstr{\vartwo}{\vartwo}})&\cbnfsred(\las{\termfour}{\fstwo},\abstr{\vartwo}{\vartwo})\cbnfsred(\fstwo,\termfour).\\
    
    \cbnfssup{\fsone}{\termone}=\frac{1}{2}\cbnfssup{\fstwo}{\termthree}+\frac{1}{2}\cbnfssup{\fstwo}{\termfour}=\cbnfssup{\fsone}{\termtwo}.
    
    &\Bigl(\bigcup_{\hsk{k}}\{( \hsk{expone}, \hsk{expthree k})\}\Bigr) \cup
    \{(M,N)\; | \:  {\sem M} = {\sem N }\}\\
    &\cup \Bigl(\bigcup_L \{ (\lambda \hsk{n}. \hsk{B} \sub L {\tt{f}},
    \lambda \hsk{n}. \hsk{C} \sub L {\tt{f}} )\} \Bigr)
  
  using  and  for the body of  and
   respectively.
\end{example}
