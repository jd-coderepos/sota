\documentclass[journal,12pt,onecolumn]{IEEEtran}
\ifCLASSINFOpdf

\else

\fi

\hyphenation{op-tical net-works semi-conduc-tor}
\usepackage{graphicx}
\usepackage{latexsym,bm}
\usepackage{amsmath}
\usepackage{amssymb}
\usepackage{amsthm}
\usepackage{epstopdf}
\usepackage{cite}
\usepackage{mathrsfs}
\usepackage{multirow}
\usepackage[margin=1.0in]{geometry}\usepackage{booktabs}\usepackage{array}\usepackage{pgfplots} 
\usepackage{pbox}
\usepackage{epsfig}
\usepackage{subcaption}
\pgfplotsset{compat=newest} 
\pgfplotsset{plot coordinates/math parser=false}
\usepackage{algorithm}\usepackage{algpseudocode}\usepackage{slashbox}
\makeatletter
\def\therule{\makebox[\algorithmicindent][l]{\hspace*{.5em}\vrule height .75\baselineskip depth .25\baselineskip}}\newcolumntype{?}{!{\vrule width 1pt}}


\newtoks\therules \therules={}\def\appendto#1#2{\expandafter#1\expandafter{\the#1#2}}\def\gobblefirst#1{#1\expandafter\expandafter\expandafter{\expandafter\@gobble\the#1}}\def\LState{\State\unskip\the\therules}\def\pushindent{\appendto\therules\therule}\def\popindent{\gobblefirst\therules}\def\printindent{\unskip\the\therules}\def\printandpush{\printindent\pushindent}\def\popandprint{\popindent\printindent}

\algdef{SE}[WHILE]{While}{EndWhile}[1]
  {\printandpush\algorithmicwhile\ #1\ \algorithmicdo}
  {\popandprint\algorithmicend\ \algorithmicwhile}\algdef{SE}[FOR]{For}{EndFor}[1]
  {\printandpush\algorithmicfor\ #1\ \algorithmicdo}
  {\popandprint\algorithmicend\ \algorithmicfor}\algdef{S}[FOR]{ForAll}[1]
  {\printindent\algorithmicforall\ #1\ \algorithmicdo}\algdef{SE}[LOOP]{Loop}{EndLoop}
  {\printandpush\algorithmicloop}
  {\popandprint\algorithmicend\ \algorithmicloop}\algdef{SE}[REPEAT]{Repeat}{Until}
  {\printandpush\algorithmicrepeat}[1]
  {\popandprint\algorithmicuntil\ #1}\algdef{SE}[IF]{If}{EndIf}[1]
  {\printandpush\algorithmicif\ #1\ \algorithmicthen}
  {\popandprint\algorithmicend\ \algorithmicif}\algdef{C}[IF]{IF}{ElsIf}[1]
  {\popandprint\pushindent\algorithmicelse\ \algorithmicif\ #1\ \algorithmicthen}\algdef{Ce}[ELSE]{IF}{Else}{EndIf}
  {\popandprint\pushindent\algorithmicelse}\algdef{SE}[PROCEDURE]{Procedure}{EndProcedure}[2]
   {\printandpush\algorithmicprocedure\ \textproc{#1}\ifthenelse{\equal{#2}{}}{}{(#2)}}{\popandprint\algorithmicend\ \algorithmicprocedure}\algdef{SE}[FUNCTION]{Function}{EndFunction}[2]
   {\printandpush\algorithmicfunction\ \textproc{#1}\ifthenelse{\equal{#2}{}}{}{(#2)}}{\popandprint\algorithmicend\ \algorithmicfunction}\makeatother
\DeclareMathAlphabet{\mathscrbf}{OMS}{mdugm}{b}{n}
\newcommand\blfootnote[1]{\begingroup
  \renewcommand\thefootnote{}\footnote{#1}\addtocounter{footnote}{-1}\endgroup
}
\usepackage{array}
\usepackage{array,colortbl,xcolor}


\newcommand\Thickvrule[1]{\multicolumn{1}{!{\vrule width 2pt}c!{\vrule width 2pt}}{#1}}

\newcommand\Thickvrulel[1]{\multicolumn{1}{!{\vrule width 2pt}c|}{#1}}

\newcommand\Thickvruler[1]{\multicolumn{1}{|c!{\vrule width 2pt}}{#1}}

\newlength\Origarrayrulewidth
\newcommand{\Cline}[1]{\noalign{\global\setlength\Origarrayrulewidth{\arrayrulewidth}}\noalign{\global\setlength\arrayrulewidth{2pt}}\cline{#1}\noalign{\global\setlength\arrayrulewidth{\Origarrayrulewidth}}}
\usepackage{color}


\begin{document}
\begin{titlepage}
\begin{center}
\vspace*{-2\baselineskip}
\begin{minipage}[l]{7cm}
\flushleft
\includegraphics[width=2 in]{njit-logo.jpg}
\end{minipage}
\hfill
\begin{minipage}[r]{7cm}
\flushright
\includegraphics[width=1 in]{ANL_LOGO.jpg}\end{minipage}

\vfill

\textsc{\LARGE Content Caching and Distribution in\\ [12pt]
Smart Grid Enabled Wireless Networks}

\vfill
\textsc{\LARGE XUEQING HUANG\12pt]
\LARGE March 17, 2016}\
\begin{array}{l}
\textit{SC Set: }\mathcal{N}_i=\{j\in\mathcal{N}|\alpha_{i,j}= 1\}\\
\textit{UE Set: }\mathcal{K}_i=\{k\in\mathcal{K}|\beta_{i,k}= 1\}.
\end{array}

\label{eq1}
R_i=\sum\limits_{j\in\mathcal{N}_i} B\log_2( 1 + \frac{P_{i,j}\min\limits_{k\in\mathcal{K}_i}\{h_{i,j}^k\}}{N_0 B}),\: i\in\mathcal{M},
\label{on_grid_PW}
P_i^{on}=\left[\sum\limits_{j \in\mathcal{N}_i} {{P_{i,j}}} - (E_i-\delta_i)\right]^+,\: i\in\mathcal{M},
\label{obj}
\begin{array}{l}
\mathop {\min }\limits_{\{\delta_i,{P_{i,j}},{\alpha_{i,j},\beta_{i,k}}\}}  \sum\limits_{i\in\mathcal{M}}{P_i^{on}} \\
\begin{array}{*{20}{l}}
{s.t.}&c_1:&R_i=\left\{\begin{array}{cc}
 DS,&i=1\\
 S, &i\in\mathcal{M'}\\
 \end{array}
 \right.\\
&c_2:&\delta_i\le E_i,i\in\mathcal{M}\\
&c_3:&\sum\limits_{i\in\mathcal{M}}\delta_i (\theta{\bf{1}}_{\{\delta_i>0\}}+{\bf{1}}_{\{\delta_i<0\}})\ge 0\\
&c_4:&\sum\limits_{i\in\mathcal{M}}\alpha_{i,j}=1,j\in\mathcal{N}\\
&c_5:&\beta_{1,k}+\sum\limits_{i\in\mathcal{M'}}\frac{\beta_{i,k}}{D}\ge 1, k\in\mathcal{K}\\
&c_6:&P_{i,j}\ge 0,i\in\mathcal{M},j\in\mathcal{N}\\
&c_7:&{\alpha_{i,j},\beta_{i,k}}\in\{0,1\},i\in\mathcal{M},j\in\mathcal{N},k\in\mathcal{K}\\
\end{array}
\end{array}
\label{2a}
P_{i,j}=\left[\lambda_{i}-\frac{1}{\gamma_{i,j}^{min}}\right]^+,i\in\mathcal{M},j\in\mathcal{N}_{i},
\label{SNR}
\gamma_{i,j}^k={h_{i,j}^k/({N_0 B})},
\label{minSNR}
\gamma_{i,j}^{min}=\min\limits_{k\in\mathcal{K}_i}\{\gamma_{i,j}^k\}.
\label{2a1}
\lambda_{i}=\left\{\begin{array}{l}
{\left({\frac{2^{\frac{DS}{B}}}{\prod\limits_{j\in\mathcal{N}_{i}}{\gamma ^{min}_{i,j}}}}\right)}^{\frac{1}{{N}_{i}}},\: i=1\\
{\left({\frac{2^{\frac{S}{B}}}{\prod\limits_{j\in\mathcal{N}_{i}}{\gamma ^{min}_{i,j}}}}\right)}^{\frac{1}{{N}_{i}}},\: i\in\mathcal{M'}\\
\end{array}
\right.
\label{alpha}
\delta_i=E_i-\sum\limits_{j\in\mathcal{N}_i}P_{i,j}\label{ana1}
\left\{\begin{array}{l}
\sum\limits_{j\in\mathcal{N}_i}P_{i,j}<E_i,i\in\mathcal{M}_{+}\\
P_i^{on}=\left[\sum\limits_{j \in\mathcal{N}_i} {{P_{i,j}}} - (E_i-\delta_i)\right]^+=0,i\in\mathcal{M}^+.\\
\end{array}\right.
\label{simpli}
\sum\limits_{i\in\mathcal{M}}{P_i^{on}}=\sum\limits_{i\in\mathcal{M}^-}{P_i^{on}}.
\label{lem2}
\delta_i=-\min\{\sum\limits_{j\in\mathcal{N}_i}P_{i,j}-E_i,\sum\limits_{i'\in\mathcal{M}^+}\theta\delta_{i'}+\sum\limits_{i'\in\mathcal{M}^-\backslash{i}}\delta_{i'}\}\label{red1}
\sum\limits_{i\in\mathcal{M}}{P_i^{on}}=\sum\limits_{i\in\mathcal{M}^-}[{\sum\limits_{i\in\mathcal{N}_i}P_{i,j}}-(E_i-\delta_i)].
\label{red}
\begin{array}{l}
\sum\limits_{i\in\mathcal{M}}{P_i^{on}}=\sum\limits_{i\in\mathcal{M}^-}({\sum\limits_{j\in\mathcal{N}_i}P_{i,j}}- E_i)+\sum\limits_{i\in\mathcal{M}^-}\delta_i=\\
\sum\limits_{i\in\mathcal{M}^-}({\sum\limits_{j\in\mathcal{N}_i}P_{i,j}}- E_i)-\\
min\{\sum\limits_{i\in\mathcal{M}^-}(\sum\limits_{j\in\mathcal{N}_i}P_{i,j}-E_i),\sum\limits_{i\in\mathcal{M}^+}\theta\delta_{i}\}=\\
\max\{0,\sum\limits_{i\in\mathcal{M}^-}{\sum\limits_{j\in\mathcal{N}_i}P_{i,j}}- \sum\limits_{i\in\mathcal{M}^-}E_i-\sum\limits_{i\in\mathcal{M}^+}\theta\delta_{i}\}=\\
\max\{0,\sum\limits_{i\in\mathcal{M}^-}{\sum\limits_{j\in\mathcal{N}_i}P_{i,j}}+\sum\limits_{i\in\mathcal{M}^+}{\sum\limits_{j\in\mathcal{N}_i}\theta P_{i,j}}- \\
\sum\limits_{i\in\mathcal{M}^-}E_i- \sum\limits_{i\in\mathcal{M}^+}\theta E_i\}.
\end{array}
\label{obj1}
\begin{array}{l}
\mathop {\min }\limits_{\{{\alpha_{i,j},\beta_{i,k}}\}}  \sum\limits_{i\in\mathcal{M}^-}{\sum\limits_{j\in\mathcal{N}_i}P_{i,j}} + \sum\limits_{i\in\mathcal{M}^+}{\sum\limits_{j\in\mathcal{N}_i}\theta P_{i,j}}-\\
\sum\limits_{i\in\mathcal{M}^-}E_i- \sum\limits_{i\in\mathcal{M}^+}\theta E_i\\
\begin{array}{*{20}{l}}
{s.t.}&\mathcal{M}^-=\{i\in\mathcal{M}|\sum\limits_{j\in\mathcal{N}_i}P_{i,j}\ge E_i\}\\
&\mathcal{M}^+=\{i\in\mathcal{M}|\sum\limits_{j\in\mathcal{N}_i}P_{i,j}< E_i\}\\
&\sum\limits_{i\in\mathcal{M}}\alpha_{i,j}=1,j\in\mathcal{N}\\
&\beta_{1,k}+\sum\limits_{j\in\mathcal{M'}}\frac{\beta_{i,k}}{D}\ge 1, k\in\mathcal{K}\\
&{\alpha_{i,j},\beta_{i,k}}\in\{0,1\},i\in\mathcal{M},j\in\mathcal{N},k\in\mathcal{K},\\
\end{array}
\end{array}

\gamma_{i}^k={\sum\limits_{j\in\mathcal{N}}\gamma_{i,j}^k}/{N}.
\label{obj1_2}
\begin{array}{l}
\mathop {\min }\limits_{\{N_i,\beta_{i,k}\}}  \frac{N_1(2^\frac{DS}{N_1B}-1)}{\gamma_1^{min}}+\sum\limits_{i\in\mathcal{M}'}\frac{N_i(2^\frac{S}{N_iB}-1)}{\gamma_i^{min}} \\
\begin{array}{*{20}{l}}
{s.t.}& \sum\limits_{i\in\mathcal{M}}N_i=N\\
&\beta_{1,k}+\sum\limits_{i\in\mathcal{M'}}\frac{\beta_{i,k}}{D}\ge 1, k\in\mathcal{K},\\
\end{array}
\end{array}

\gamma_i^{min}=\min\limits_{k\in\mathcal{K}_i}\{\gamma_{i}^k\}.
D^*=D.D \le D^*\le M.2D \le D^*\le M+D.\label{obj1_3}
\begin{array}{l}
\mathop {\min }\limits_{\{\beta_{i,k}\}} \frac{N}{D^*}(2^{\frac{D^*S}{NB}}-1)(\frac{D}{\gamma_1^{min}}+\sum\limits_{i\in\mathcal{M}^*}\frac{1}{\gamma_i^{min}}) \\
\begin{array}{*{20}{l}}
{s.t.}&\beta_{1,k}+\sum\limits_{i\in\mathcal{M'}}\frac{\beta_{i,k}}{D}\ge 1, k\in\mathcal{K}\\
\end{array}
\end{array}
\label{offload}
P_{i_1}-P'_{i_1}>(P'_{i_2}-P_{i_2})\theta,

P_i^{N_i}=\left\{
\begin{array}{*{20}{c}}
\frac{N_i({2^{\frac{DS}{BN_i}}-1})}{\gamma_i^{min}},\:i=1\\
\frac{N_i({2^{\frac{S}{BN_i}}-1})}{\gamma_i^{min}},\:i\in\mathcal{M}'.\\
\end{array}\right.
\label{alg_eq}
i^*=\arg \max\limits_{i\in\mathcal{M}_{act}} P_i^{N_i}-E_i

}
\LState {Sort SCs in  in the descending order of }
\LState {Index SC with the -th largest SNR to node }
\LState {}
\LState {}
\LState {}
\LState {}
\EndWhile
\LState {}
\LState\Return , , , 
\end{algorithmic}
\end{algorithm}

So far, we have proposed separate algorithms for four sets of variables in (\ref{obj}). To get the optimal solution, the maximal minimum SNR algorithm in Alg. \ref{euclid} is first adopted to obtain the user association scheme, and then the subchannel assignment scheme in Alg. \ref{SC} and Alg. \ref{SC1} are employed to assign the spectrum to each node. Then, the power is allocated according to (\ref{2a}), and subsequently the power flows given in \emph{Lemmas} 1-2 will yield the minimum on-grid power power consumption. 


\section{Simulation Results}
In this section, we assume that the average solar green energy density is  . Different sizes of solar panels equipped in eNB and serving nodes result in diverse average green energy generating rate. For time duration of  , the green energy generating rate for eNB is  , and for SNs,  , . 




The four SNs form a square area with   length. As illustrated in Fig. \ref{SCENARIO_PLOT}, the center of the square area is named the virtual SN center, and users are uniformly distributed within the area of   radius. The rest of the simulation parameters are listed in Table \ref{table_sim}, where  is the multimedia downloading rate required by each user, and  is the size of the total traffic demand.

\begin{figure}[t]
\centering
\includegraphics[width=3in]{cell_plot.eps}\caption{Illustration of the system deployment (Triangle: SN; Cross: Virtual SN center; Dot: UE; Square: eNB).}
\label{SCENARIO_PLOT}
\end{figure}



\begin{table}[ht]
\caption{Simulation Parameters}\label{table_sim} \centering\  \begin{tabular}{|c|c|}
\hline
Carrier Frequency&  \\
\hline
 & \\
\hline
&\\
\hline
&\\\hline
&\\
\hline
Cell Radius&   (eNB)\\
\hline
Path Loss Model& \\
\hline
Shadowing &   Log Normal Fading\\
\hline
 &  \\
\hline
Noise Figure &  (eNB/SN),   (UE)\\
\hline
Antenna Gain &   (eNB),   (SN)\\
\hline 
Fragment size& \\
\hline
\end{tabular}
\label{tab_2}
\end{table}




\begin{figure}
\vspace{-1em}
\centering   
                \begin{subfigure}[!t]{3 in}
                \includegraphics[width=3in,height=2.5in]{theta_0.eps}\caption{}
\end{subfigure}
        ~
        
        \begin{subfigure}[!t]{3 in}
                \includegraphics[width=3in,height=2.5in]{theta_2.eps}\caption{}
\end{subfigure}~
        
         ~\begin{subfigure}[!t]{3 in}
                \includegraphics[width=3in,height=2.5in]{theta_4.eps}\caption{}
\end{subfigure}
        ~ 

        \caption{Total on-grid power consumption versus multimedia download rate (cell edge).}\label{result}
\end{figure}

The on-grid power consumption are averaged over  independent snapshots by Monte-Carlo simulations. Using Case I, where only eNB is active, as the base line, we compare the total on-grid power consumption of our proposed algorithm versus the downloading rate .
 


\subsection{Cell Edge Users}
First, we measure the performance of the virtual SN center located at the cell edge, i.e., the distance between the virtual SN center and eNB is at least  times the radius of the cell. As illustrated in Fig. \ref{result},  the on-grid power consumption will reduce as  increases. More importantly, the reduction of on-grid power consumption in Case I is less obvious than the proposed algorithm. This is because in setting the parameters,  is significantly higher than the energy generating rate of SNs. So, the power flow from SN to eNB will be significantly less than the power flow from eNB to SNs, and thus the latter power flow is more prone to be affected by the energy transfer efficiency.

At the same time, the on-grid power consumption will increase with the data rate, and the proposed scheme outperforms the traditional multicast case. This shows that bringing serving nodes, cached content, and green energy closer to the cell edge users can save energy. Moreover,  means the replication caching scheme. It has the highest effective redundancy factor , and yet brings the least power reduction. The reason is that increased traffic load due to high redundancy trumps the diversity gain in terms of choosing different serving nodes. 

Diverse serving nodes provide the potential to increase the minimum SNR and reduce the power consumption. To leverage this potential, multiple SNs need to be activated. When , the active SNs bring the most amount of traffic load, since  in Table \ref{table_sim} will decrease with . So, as analyzed in Section \ref{assump}, with a given number of subchannels, the increased traffic load can offset the channel diversity gain. 

The on-grid power, however, does not always decrease with . For example,  means no diversity gain, because users have to connect with all SNs, but it always has the minimum traffic load. With  and , the corresponding traffic size in each SN is  and , respectively. By the definition given in Table \ref{table_sim},  will increase with . When  is small, the traffic load size  is not that large as compared to , and so the diversity gain may cancel out the traffic load increment. When  exceeds a certain point, the relatively decreased traffic load in  will trump the diversity gain.


\subsection{Cell Center Users}
As we can see from Fig. \ref{result}, the maximum on-grid power is limited by  , and so the maximum transmission power of eNB and SN are nearly  . For the eNB in Case I, which is located at the cell center, this power limit complies with the current LTE standard. For the virtual SN center located at the cell edge, the maximum transmission power of each SN is between   to  , in which unbearable interference may cause the performance degradation of the other virtual SN center which happens to locate at the edge of the neighboring cell. 

So, in this section, we measure the performance for the virtual SN center located at the cell center, i.e., the distance between the virtual SN center and eNB is less than  times the radius of the cell. As illustrated in Fig. \ref{result2}, the on-grid power consumption for Case I is less, as compared with the cell edge case. This is straight forward because users are closer to the eNB. The improvement for SNs are less obvious because the relative positions of SN and users remain the same. However, the proposed algorithm still outperforms Case I; this proves that bringing content, green energy and serving node closer to users can save the on-grid energy consumption. Moreover, the caching scheme with low effective redundancy factor (small ) can actually save more energy than high redundancy caching scheme when the data rate is high.




\begin{figure}
\vspace{-1em}
\centering   
                \begin{subfigure}[!t]{3 in}
                \includegraphics[width=3in,height=2.5in]{cell_center_theta_0.eps}\caption{}
\end{subfigure}
        ~
        
        \begin{subfigure}[!t]{3 in}
                \includegraphics[width=3in,height=2.5in]{cell_center_theta_2.eps}\caption{}
\end{subfigure}~
        
         ~\begin{subfigure}[!t]{3 in}
                \includegraphics[width=3in,height=2.5in]{cell_center_theta_4.eps}\caption{}
\end{subfigure}
        ~ 

        \caption{Total on-grid power consumption versus multimedia download rate (cell center).}\label{result2}
\end{figure}

\section{Conclusion}
In this paper, we have proposed the content caching and distribution framework for the smart grid enabled wireless multimedia transmission system, where the green energy can be transferred to the serving node that is closer to end users through the grid credit exchange system. To investigate the on-grid energy reduction that is brought by closely located content, downlink node and green energy, the user association problem is designed jointly with radio resource allocation schemes, including power allocation, subchannel allocation, and power flow design. We have analytically derived the optimal power allocation and power flow scheme. Moreover, the user association problem has been successfully decoupled from the subchannel allocation problem. A novel maximal minimum SNR algorithm is proposed along with a two step subchannel assignment algorithm, which takes into account of the green energy generating rate of each node. Simulation results have proven that the proposed schemes outperform the traditional eNB multicasting scheme. That is, bringing content, serving node and green energy closer to the user can save the on-grid power consumption. Moreover, when the downloading rate requirement of each user is high, the caching scheme with low redundancy has been shown to save more energy than the high redundancy caching scheme.
\bibliographystyle{IEEEtran}
\bibliography{mybib}

\end{document}