We now include non-rival resources in the resource allocation problem,
and we compute the capacity of the system under data availability
constraints. A peer can not allocate any of its uplink bandwidth if it
does not have data to transmit first. The non-rival resources are a
set $K$ comprising independent data units. Data are roughly equivalent
in size. The quality of service associated with a peer is then a
function of the number of received data. We denote by $q(u)$ the
service quality for a node $u$. The quality of service is a increasing
function of the number of data units received by the peer.

Each data $k \in K$ is served to nodes on a separate arborescence
$T_k=(V_k,E_k)$, a directed tree rooted at $s$ where $V_k \subseteq V$
and $E_k \subseteq E^\ast$. The children of a client $u \in V_k$ are
denoted by $N_k(u)$, the number of children by $m_k(u)$. The multiple
tree construction takes into account the aforementioned constraint on
upload capacity of node $u$, i.e., $\sum_{k\in K} m_k(u) \leq c(u),
\forall u \in V$.

As stated in the problem DCDA, we are interested in maximizing the
overall quality of service in the overlay. Here, we define this
quantity to be the sum of the qualities of service $q(u)$ experienced
by all clients. Our model can support alternative definitions of the
overall quality of service, as ensuring fairness among the clients or
maximizing the number of clients up to a given quality threshold. We
show below that the $K$-DCDA is NP-complete, even for $K = 1$.

\subsection{NP-Completeness of $K$-DCDA}

A formal formulation of the decision problem related with $k-$DCDA is: \\
{\sc Instance :} A graph $G=(V,E^\ast)$ with $V$ the set of vertices and $E$ the set of edges, a root $s\in V$, a positive integer $K$, a capacity function $c:V \longrightarrow {\mathbb N}$ and a positive integer $\Gamma$.\\
{\sc Question :} Do there exist $K$ arborescences $\left
  (T_k=(V_k,E_k) \right )_{1\leq k\leq K}$ rooted in $s$ such that:
\begin{itemize}
\item[(1)] for any $k$, we have $V_k\subseteq V$ and if $(u \rightarrow
  v)$ is an edge from $T_k$ then $(u \rightarrow v)$ is an edge of $G$,
\item[(2)] for any vertex $u\in V$, the sum of its outdegrees is
  lower or equal to its capacity, i.e., $\sum_{k=1}^K m^+_{T_k}(u)
  \leq c(u)$,
\item[(3)] the total number of vertices belonging to the
  arborescences is greater or equal to $\Gamma$, i.e., $\sum_{k=1}^K
  |V_k| \geq \Gamma$.
\end{itemize}






We now provide a proof of the NP-completeness of $K$-DCDA using a
reduction to the famous 3-SAT problem.

{\noindent\bf 3-SAT\\} {\sc Instance :}
Set $U$ of variables and a collection $C$ of clauses over $U$ such that each clause $c\in C$ has $|c|=3$.\\
{\sc Question :} Is there a truthful assignment for $C$?

\begin{theorem}
  {\bf $K$-DCDA} is NP-complete even for $K=1$.
\end{theorem}

\bpr Given an instance of $K$-DCDA Problem and a family $\left
  (T_k=(V_k,E_k) \right )_{1\leq k\leq K}$ of $K$ arborescence rooted
in $s$, verifying that this family is a valid one is clearly
polynomial in the size of the problem: hence the $K$-DCDA problem
belongs to NP.



Now, given an instance of the 3 SAT problem comprising
$U=\{x_1,\cdots,x_n\}$ a set of variables and
$C=\{C_1,\cdots,C_{|E|}\}$ a set of clauses on $U$ where
$C_j=x_j^1\vee x_j^2\vee x_j^3 $, we define an instance of the
$K$-DCDA problem as follows.  Recall that for any $1\leq j \leq |E| $
and any $1\leq l\leq 3$, we have that there exists $1\leq i\leq n$
such that $x_j^l\in \{x_i, \overline{x_i}\}$. Let $V=\{s\}\cup \{i,
x_i, \overline{x_i} : 1\leq i \leq n\}\cup \{C_1,\cdots,C_{|E|}\}$ and
let $E'=\{\{s,i\} : 1\leq i \leq n\}\cup \{\{i,x_i\},
\{i,\overline{x_i}\} : 1\leq i \leq n\} \cup \{ \{x_j^l,C_j\}: 1\leq j
\leq |E|, 1\leq l\leq 3\}$.  For $1\leq i \leq n$ and for $1\leq j\leq
|E|$, the capacity function is defined as $c(s)=n$, $c(i)=1$,
$c(x_i)=c(\overline{x_i})=|E|$ and $c(C_j)=0$.  Finally, we define
$\Gamma$ as $1+2n+|E|$.  Clearly the instance of the $K$-DCDA problem
can be constructed in polynomial time in the size of the 3-SAT
instance.  We claim that there exists an arborescence $T=(V',F)$
solving our instance of the problem $K$-DCDA if and only if there
exists a truthful assignment for $C$.


For the forward implication, assume that there exists an arborescence
$T=(V',F)$ fulfilling conditions~$(1)$ to~$(3)$ of the problem
$K$-DCDA.  As $K=1+2n+m=|V'|$ and as for any $1\leq k \leq n$, we have
$c(k)=1$ and $c(j)=0$ for $1\leq j \leq |E|$, it follows that
$|\{x_i,\overline{x_i}\}\cap V'|=1$ for any $1\leq i \leq n$.  We
define the assignment function $\varphi$ as follows : $\varphi(x_i)$
is set to True if $x_i\in V'$ and False if $\overline{x_i}\in V'$.
But now, as $|V'|=1+2n+|E|$ and as for any $1\leq i \leq n$ it holds
$|\{x_i,\overline{x_i}\}\cap V'|=1$, we obtain that for any $1\leq
j\leq |E|$, $C_j \in V'$ and thus that there exists a vertex $x'_j$ in
$\{x_i,\overline{x_i} : 1\leq i \leq n\}$ such that $(x'_j,C_j)$ is an
edge of $T$.  But now, by definition of $\varphi$, we obtain that the
literal associated to $x'_j$ has a True value and thus we obtain that
the clause $C_j$ has also a true value, and thus that $\varphi$ is a
truth assignment for $C$.

For the backward implication, assume that we have a truth assignment
$\varphi$ for $C$.  We define $U'$ the set of true litterals for
$\varphi$, that is $U'=\{x_i : x_i \in U, \varphi(x_i)=True\}\cup
\{\overline{x_i} : x_i \in U, \varphi(x_i)=False\}$.  Now let
$V'=\{s\}\cup \{1,\cdots,n\} \cup U' \cup C$, clearly we have
$|V'|=1+2n+|E|$.  As $C$ is True, this means that for any $1\leq j\leq
|E|$, there exists at least one literal $y_j \in C_j$ such that
$\varphi(x_j)=True$.  We denote by $y_j$ one literal from $C_j$ which
is True by $\varphi$.  We define $F=\{(s,i): 1\leq i \leq n\}\cup \{
(i,x_i) : 1\leq i\leq n, x_i \in U'\}\cup \{ (i,\overline{x_i}) :
1\leq i\leq n, \overline{x_i} \in U'\}\cup \{(y_j,C_j) : 1\leq j \leq
|E|\}$.  As by definition of $y_j$, the literal $y_j$ is set to True
and by the definition of $F$, it is obvious that $D=(V',F)$ is an
arborescence rooted in $s$ and that edges from $D$ are also edges from
$G$.  Now we remain with the capacity constraint.  Clearly we have
$m_D(s)=n$, for any $1\leq j \leq |E|$, $m_D(C_j)=0$.  Now, as for any
$1\leq i \leq n$, we have $|\{x_i,\overline{x_i}\}\cap U'|=1$, we
obtain that $m_D(k)=1$.  Moreover, for any $1\leq i \leq n$, we have
both $m_D(x_i)\leq |E|$ and $m_D(\overline{x_i})\leq |E|$, thus $D$ is
an arborescence fullfilling conditions~$(1)$ to $(3)$ and having
$\Gamma$ elements.  \epr

Note that the backward implication above only considers the case $K =
1$ since showing that 1-DCDA is NP-complete also implies that $k-DCDA$
is NP-complete.





\subsection{$1$-DCDA Problem Decomposition}

As the $K$-DCDA problem is NP-complete even for $K=1$, we focus now on
the particular instance of the $1$-DCDA problem where the quality of
service is a binary function. The peers either fulfills their demand
$d(u)= 1, \ \forall u \in V$, or not. We propose a decomposition of
the $1$-DCDA problem into a master problem and several subproblems,
which can be solved efficiently in polynomial time. We introduce first
the concept of \emph{level}. The vertex $s$ corresponds to the only
vertex at level $0$; the vertices adjacent to $s$ are at level $1$,
the vertices adjacent to those at level 1 are at level $2$, and so
forth.  The level of a vertex therefore represents its distance (in
terms of hops) to vertex $s$ in the tree. We denote by $J =
\{1,2,3,\cdots, n-1\}$ the set of possible levels. We also denote by
$V_s$ the set of nodes $V \setminus \{s\}$.

Let $x \in \{0,1\}^{(n-1)^2}$ be a matrix defined as:
\begin{equation*}
x^j_v = \left\{
\begin{array}{ll}
1 & \textrm{ if } v \textrm{ is at level } j,\\
0 & \textrm{ otherwise,}
\end{array}
\right.
\end{equation*}
for all $v \in V_s$ and $j \in J$.  Furthermore, let $y \in
\{0,1\}^{|E|(n-1)}$ be the matrix defined as:
\begin{equation*}
y^j_e = \left\{
\begin{array}{ll}
1 & \textrm{ if } e \textrm{ is selected from level } j-1 \textrm{ to level } j,\\
0 & \textrm{ otherwise,}
\end{array}
\right.
\end{equation*}
for all $e \in E$ and $j \in J$.  Then, the $1$-DCDA problem is equivalent to
the following mixed-integer linear program P1
\[\text{P1} : \, \, \max z(x) = \sum\limits_{j=1}^{n-1} \sum\limits_{v \in V_s} x_v^j \, , \quad \text{s.t.} \]
\vspace{-0.25cm}
\begin{align}
\label{inq:vertex:cover} & \sum\limits_{j=1}^{n-1} x_v^j \leq 1 \, , \, \textrm{for } v \in V_s,\\
\label{inq:degree:s} & \sum\limits_{v \in V_s} x^1_v \leq c(s),\\
\label{inq:degree:t:t+1} & \sum\limits_{v \in V_s} x_v^j - \sum\limits_{v \in V_s} c(v)x_v^{j-1} \leq 0 \, , \, \textrm{for } j \in J\setminus\{1\},\\
\label{inq:edge:assign} & \sum\limits_{j=1}^{n-1} y_e^j \leq 1 \, , \,  \textrm{for } e \in E,\\
\label{inq:edge:left:s} & \sum\limits_{e \in \delta(s)} y_e^1 - c(s) \leq 0,\\
\label{inq:edge:left} & \sum\limits_{e \in \delta(v)} y_e^j - c(v)x_v^{j-1} \leq 0 \, , \, \textrm{for } v \in V_s,\ j \in J\setminus\{1\},\\
\label{inq:edge:right} & \sum\limits_{e \in \delta(v)} y_e^j - x_v^j = 0 \, , \, \textrm{for } v \in V_s,\ j \in J, \\
\label{inq:vertex:integral} & x \in \{0,1\}^{(n-1)^2},\\
\label{inq:edge:integral} & y \in \{0,1\}^{|E|(n-1)}.
\end{align}

The assignment of vertex $v \in V_s$ to at most one level is expressed
by inequalities \eqref{inq:vertex:cover}. Inequalities
\eqref{inq:degree:s} and \eqref{inq:degree:t:t+1} bound from above the
number of vertices at level $j+1$, based on the number of vertices at
level $j$ and the node capacity function $c$.  Inequalities
\eqref{inq:edge:assign} guarantee that edge $e \in E$ is selected at
most once in the induced tree.  Inequalities \eqref{inq:edge:left:s}
and \eqref{inq:edge:left} ensure that vertex $v \in V$ at level $j$ is
adjacent to at most $c(v)$ vertices at level $j+1$, whereas
inequalities \eqref{inq:edge:right} ensure that vertex $v \in V_s$ at
level $j$ is adjacent to exactly one vertex at level $j-1$.

This model can be reformulated without the $y$ variables using
Benders' decomposition~\cite{benders1962pps}.  The main principle of
this decomposition consists of separating the variables of the
problem. A master problem, still NP-complete, is in charge of
determining a solution for one variable, while the sub-problems are
responsible to complete the assignment on the other variables. If this
assignment is possible, the whole problem is solved, otherwise a new
constraint is added to the master problem, which makes its computation
quicker.

Now, let $X = \left\{ x \in \mathbb{R}^{(n-1)^2} : \, x \textrm{
    satisfies } \eqref{inq:vertex:cover}-\eqref{inq:degree:t:t+1}
  \textrm{ and } \eqref{inq:vertex:integral}\right\}$.  
Moreover, let (\ref{inq:edge:left}.$j$) and (\ref{inq:edge:right}.$j$) denote
inequalities \eqref{inq:edge:left} and \eqref{inq:edge:right} for a
specific value $j$ in $J\setminus\{1\}$ and $J$, respectively. Then,
let
\[Y(1) = \{y^1 \in \mathbb{R}^{|E|} : y^1 \textrm{ satisfies }
  \eqref{inq:edge:left:s}, (\ref{inq:edge:right}.1) \textrm{ and } y^1
  \in \{0,1\}^m\},\] \noindent while for $j\in
J\setminus\{1\}$ let
\[Y(j) = \{y^j \in \mathbb{R}^{|E|} : y^j \textrm{ satisfies } (\ref{inq:edge:left}.j), (\ref{inq:edge:right}.j) \textrm{ and } y^j \in \{0,1\}^m\}.\]




Finally, the program P1 can be rewritten as
\begin{equation}
  \max\limits_{x \in X} z(x) + \zeta(s,x^1) + \sum\limits_{j=2}^{n-1} \zeta(x^{j-1},x^j) \, ,
  \label{eqn:master_slave_program}
\end{equation}
where the subproblems have no incidence on the value of the final
solution, therefore they can be abusively written as:
\begin{eqnarray}\label{eq:subpb:s}
  \zeta(s,x^1) & = & \max_{y^1 \in Y(1)} 0 \, , \\
\label{eq:subpb:t}
  \zeta(x^{j-1},x^j) & = & \max_{y^j \in Y(j)} 0 \, , \, \text{for } \, j\in J\setminus\{1\} \, .
\end{eqnarray}


The idea behind the decomposition in (\ref{eqn:master_slave_program})
is that a master problem generates a solution where the nodes are
assigned to levels, and then the individual sub-problems verify if it
is indeed possible to find edges linking the nodes at a given level
with the nodes at the next level while respecting the node capacity
function $c$. Next, we show that these sub-problems can be solved in
polynomial-time.

Consider an undirected graph $G^j = (V^j,E^j)$, a partition
$\{L^j,R^j\}$ of $V^j$ and a function $b : L^j \longrightarrow
\mathbb{N}$.  A {\em semi-perfect $b$-matching of $G^j$} is a subset
$M$ of edges of $G^j$ such that every vertex $v$ in $L^j$ is incident
with at most $b_v$ edges of $M$ and every vertex in $R^j$ is incident
with exactly one edge of $M$. In our case, the number of used links
from nodes in $L^j$ should not be higher than the capacity of this
node, while only one link should be used to reach the nodes in $R^j$.
Let $M$ be a semi-perfect $b$-matching of $G^j$. Its {\em incidence
  vector} $\chi$ is the $\{0,1\}$-vector in $\mathbb{R}^{E^j}$
satisfying
\begin{equation*}
\chi^M_e = \left\{
\begin{array}{ll}
1 & \textrm{ if } e \in M^J,\\
0 & \textrm{ if } e \in E \setminus M^J.
\end{array}
\right.
\end{equation*}
The incidence vectors of semi-perfect $b$-matchings of $G^J$ are solutions
to the following system of linear inequalities
\begin{align}
\label{eq:sp:matching:left} & x(\delta(v)) \leq b_v && \textrm{ for } v \in L^j,\\
\label{eq:sp:matching:right} & x(\delta(v)) = 1 && \textrm{ for } v \in R^j,\\
\label{eq:sp:matching:non-negativity} & x_e \geq 0 && \textrm{ for } e \in E^j.
\end{align}

A polyhedron $P$ is {\em integral} if $P$ is the convex hull of the
integral vectors in $P$. A pointed polyhedron $P$ (\textit{i.e.},
containing at least one extreme point) is integral if and only if each
vertex is integral~\cite{schrijver2003cop}.  In the next lemma, we
show that the polyhedron defined by inequalities
\eqref{eq:sp:matching:left}-\eqref{eq:sp:matching:non-negativity} is
integral.

\begin{lemma} \label{lemma:semi-perfect:matching polytope}
The polyhedron
\begin{equation*}
SPMP(G^j,b) = \{x \in \mathbb{R}^{E^J} : x \textrm{ satisfies } \eqref{eq:sp:matching:left}-\eqref{eq:sp:matching:non-negativity}\}
\end{equation*}
is integral, providing it is not empty and $G^j$ is bipartite.
\end{lemma}
\bopr Assume $G^j$ is bipartite and $SPMP(G^j,b) \not= \emptyset$.
Let $H^j$ be the incidence matrix of $G^j$ which is known for being
totally unimodular~\cite{motzkin}. Matrix $H^j$ can be partitioned
into $H^{L^j}$ and $H^{R^j}$, where $H^{L^j}$ and $H^{R^j}$ are
composed of the rows of $H^j$ indexed by the vertices of $L^j$ and
$R^j$, respectively.  If $x^\angle$ denotes the vector of slack variables of
\eqref{eq:sp:matching:left}, then system
\eqref{eq:sp:matching:left}-\eqref{eq:sp:matching:non-negativity} can
be rewritten as
\begin{equation*}
A'x' = \begin{pmatrix} H^{L^j} & I_{|L^j|}\\ H^{R^j} & 0 \end{pmatrix} \begin{pmatrix} x\\ x^\angle \end{pmatrix} = \begin{pmatrix} b\\ 1_{|R^j|}\end{pmatrix} = b',~ x'\geq 0
\end{equation*}
From $H^j$ being totally unimodular, we easily conclude that so is
matrix $A'$.  Since $b'$ is an integral vector, the polyhedron
$SPMP(G^j,b)$ is therefore integral~\cite{hoffmankruskal}.  \epr




It is straightforward to see that each of the subproblems
\eqref{eq:subpb:s}-\eqref{eq:subpb:t} corresponds to determining
whether a semi-perfect $b$-matching exists on a graph induced by the
vertices between two consecutive levels. In fact, consider any $j \in
J$, and define $L^j = \{v \in V : x_v^{j-1} = 1\}$ and $R^j = \{v \in
V : x_v^j = 1\}$. (If $j=1$, then $L^1$ is reduced to vertex $s$.)
The subgraph $G^j$ of $G$ clearly is bipartite because of inequalities
\eqref{inq:vertex:cover}.

Using Lemma \ref{lemma:semi-perfect:matching polytope} and Farkas'
Lemma~\cite{farkas1902} (or duality in linear programming), each of
the subproblems \eqref{eq:subpb:s}-\eqref{eq:subpb:t} has a feasible
solution if and only if
\begin{equation} \label{eq:extreme:ray}
 u^J(Cx^{j-1} + x^j) \geq 0
  \quad \textrm{ for every extreme ray } u \textrm{ of } C(j),
\end{equation}
where $C(j) = \{(u^{j-1} u^j) \in \mathbb{R}^{n_l+n_r} : (u^{j-1}
u^j)^TH^j \geq 0,\ u^{j-1} \geq 0\}$ and $H^j$ is the incidence matrix
of the subgraph $G^j$.  Therefore, the integer linear programming
formulation
\begin{equation*}
  \max\limits_{x,y} z(x) \textrm{ s.t. } \eqref{inq:vertex:cover}-\eqref{inq:edge:integral}
\end{equation*}
is equivalent to solving
\begin{equation*}
  \max\limits_{x} z(x) \textrm{ s.t. } \eqref{inq:vertex:cover}-\eqref{inq:degree:t:t+1},\eqref{inq:vertex:integral},\eqref{eq:extreme:ray},
\end{equation*}
with the separation problem of inequalities \eqref{eq:extreme:ray}
being solvable in polynomial time (it reduces to solving linear
programs).

