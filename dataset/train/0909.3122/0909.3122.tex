We now include non-rival resources in the resource allocation problem,
and we compute the capacity of the system under data availability
constraints. A peer can not allocate any of its uplink bandwidth if it
does not have data to transmit first. The non-rival resources are a
set  comprising independent data units. Data are roughly equivalent
in size. The quality of service associated with a peer is then a
function of the number of received data. We denote by  the
service quality for a node . The quality of service is a increasing
function of the number of data units received by the peer.

Each data  is served to nodes on a separate arborescence
, a directed tree rooted at  where 
and . The children of a client  are
denoted by , the number of children by . The multiple
tree construction takes into account the aforementioned constraint on
upload capacity of node , i.e., .

As stated in the problem DCDA, we are interested in maximizing the
overall quality of service in the overlay. Here, we define this
quantity to be the sum of the qualities of service  experienced
by all clients. Our model can support alternative definitions of the
overall quality of service, as ensuring fairness among the clients or
maximizing the number of clients up to a given quality threshold. We
show below that the -DCDA is NP-complete, even for .

\subsection{NP-Completeness of -DCDA}

A formal formulation of the decision problem related with DCDA is: \\
{\sc Instance :} A graph  with  the set of vertices and  the set of edges, a root , a positive integer , a capacity function  and a positive integer .\\
{\sc Question :} Do there exist  arborescences  rooted in  such that:
\begin{itemize}
\item[(1)] for any , we have  and if  is an edge from  then  is an edge of ,
\item[(2)] for any vertex , the sum of its outdegrees is
  lower or equal to its capacity, i.e., ,
\item[(3)] the total number of vertices belonging to the
  arborescences is greater or equal to , i.e., .
\end{itemize}






We now provide a proof of the NP-completeness of -DCDA using a
reduction to the famous 3-SAT problem.

{\noindent\bf 3-SAT\\} {\sc Instance :}
Set  of variables and a collection  of clauses over  such that each clause  has .\\
{\sc Question :} Is there a truthful assignment for ?

\begin{theorem}
  {\bf -DCDA} is NP-complete even for .
\end{theorem}

\bpr Given an instance of -DCDA Problem and a family  of  arborescence rooted
in , verifying that this family is a valid one is clearly
polynomial in the size of the problem: hence the -DCDA problem
belongs to NP.



Now, given an instance of the 3 SAT problem comprising
 a set of variables and
 a set of clauses on  where
, we define an instance of the
-DCDA problem as follows.  Recall that for any 
and any , we have that there exists 
such that . Let  and
let .  For  and for , the capacity function is defined as , ,
 and .  Finally, we define
 as .  Clearly the instance of the -DCDA problem
can be constructed in polynomial time in the size of the 3-SAT
instance.  We claim that there exists an arborescence 
solving our instance of the problem -DCDA if and only if there
exists a truthful assignment for .


For the forward implication, assume that there exists an arborescence
 fulfilling conditions~ to~ of the problem
-DCDA.  As  and as for any , we have
 and  for , it follows that
 for any .  We
define the assignment function  as follows : 
is set to True if  and False if .
But now, as  and as for any  it holds
, we obtain that for any ,  and thus that there exists a vertex  in
 such that  is an
edge of .  But now, by definition of , we obtain that the
literal associated to  has a True value and thus we obtain that
the clause  has also a true value, and thus that  is a
truth assignment for .

For the backward implication, assume that we have a truth assignment
 for .  We define  the set of true litterals for
, that is .  Now let
, clearly we have
.  As  is True, this means that for any , there exists at least one literal  such that
.  We denote by  one literal from  which
is True by .  We define .  As by definition of , the literal  is set to True
and by the definition of , it is obvious that  is an
arborescence rooted in  and that edges from  are also edges from
.  Now we remain with the capacity constraint.  Clearly we have
, for any , .  Now, as for any
, we have , we
obtain that .  Moreover, for any , we have
both  and , thus  is
an arborescence fullfilling conditions~ to  and having
 elements.  \epr

Note that the backward implication above only considers the case  since showing that 1-DCDA is NP-complete also implies that 
is NP-complete.





\subsection{-DCDA Problem Decomposition}

As the -DCDA problem is NP-complete even for , we focus now on
the particular instance of the -DCDA problem where the quality of
service is a binary function. The peers either fulfills their demand
, or not. We propose a decomposition of
the -DCDA problem into a master problem and several subproblems,
which can be solved efficiently in polynomial time. We introduce first
the concept of \emph{level}. The vertex  corresponds to the only
vertex at level ; the vertices adjacent to  are at level ,
the vertices adjacent to those at level 1 are at level , and so
forth.  The level of a vertex therefore represents its distance (in
terms of hops) to vertex  in the tree. We denote by  the set of possible levels. We also denote by
 the set of nodes .

Let  be a matrix defined as:

for all  and .  Furthermore, let  be the matrix defined as:

for all  and .  Then, the -DCDA problem is equivalent to
the following mixed-integer linear program P1

\vspace{-0.25cm}


The assignment of vertex  to at most one level is expressed
by inequalities \eqref{inq:vertex:cover}. Inequalities
\eqref{inq:degree:s} and \eqref{inq:degree:t:t+1} bound from above the
number of vertices at level , based on the number of vertices at
level  and the node capacity function .  Inequalities
\eqref{inq:edge:assign} guarantee that edge  is selected at
most once in the induced tree.  Inequalities \eqref{inq:edge:left:s}
and \eqref{inq:edge:left} ensure that vertex  at level  is
adjacent to at most  vertices at level , whereas
inequalities \eqref{inq:edge:right} ensure that vertex  at
level  is adjacent to exactly one vertex at level .

This model can be reformulated without the  variables using
Benders' decomposition~\cite{benders1962pps}.  The main principle of
this decomposition consists of separating the variables of the
problem. A master problem, still NP-complete, is in charge of
determining a solution for one variable, while the sub-problems are
responsible to complete the assignment on the other variables. If this
assignment is possible, the whole problem is solved, otherwise a new
constraint is added to the master problem, which makes its computation
quicker.

Now, let .  
Moreover, let (\ref{inq:edge:left}.) and (\ref{inq:edge:right}.) denote
inequalities \eqref{inq:edge:left} and \eqref{inq:edge:right} for a
specific value  in  and , respectively. Then,
let
 \noindent while for  let





Finally, the program P1 can be rewritten as

where the subproblems have no incidence on the value of the final
solution, therefore they can be abusively written as:



The idea behind the decomposition in (\ref{eqn:master_slave_program})
is that a master problem generates a solution where the nodes are
assigned to levels, and then the individual sub-problems verify if it
is indeed possible to find edges linking the nodes at a given level
with the nodes at the next level while respecting the node capacity
function . Next, we show that these sub-problems can be solved in
polynomial-time.

Consider an undirected graph , a partition
 of  and a function .  A {\em semi-perfect -matching of } is a subset
 of edges of  such that every vertex  in  is incident
with at most  edges of  and every vertex in  is incident
with exactly one edge of . In our case, the number of used links
from nodes in  should not be higher than the capacity of this
node, while only one link should be used to reach the nodes in .
Let  be a semi-perfect -matching of . Its {\em incidence
  vector}  is the -vector in 
satisfying

The incidence vectors of semi-perfect -matchings of  are solutions
to the following system of linear inequalities


A polyhedron  is {\em integral} if  is the convex hull of the
integral vectors in . A pointed polyhedron  (\textit{i.e.},
containing at least one extreme point) is integral if and only if each
vertex is integral~\cite{schrijver2003cop}.  In the next lemma, we
show that the polyhedron defined by inequalities
\eqref{eq:sp:matching:left}-\eqref{eq:sp:matching:non-negativity} is
integral.

\begin{lemma} \label{lemma:semi-perfect:matching polytope}
The polyhedron

is integral, providing it is not empty and  is bipartite.
\end{lemma}
\bopr Assume  is bipartite and .
Let  be the incidence matrix of  which is known for being
totally unimodular~\cite{motzkin}. Matrix  can be partitioned
into  and , where  and  are
composed of the rows of  indexed by the vertices of  and
, respectively.  If  denotes the vector of slack variables of
\eqref{eq:sp:matching:left}, then system
\eqref{eq:sp:matching:left}-\eqref{eq:sp:matching:non-negativity} can
be rewritten as

From  being totally unimodular, we easily conclude that so is
matrix .  Since  is an integral vector, the polyhedron
 is therefore integral~\cite{hoffmankruskal}.  \epr




It is straightforward to see that each of the subproblems
\eqref{eq:subpb:s}-\eqref{eq:subpb:t} corresponds to determining
whether a semi-perfect -matching exists on a graph induced by the
vertices between two consecutive levels. In fact, consider any , and define  and . (If , then  is reduced to vertex .)
The subgraph  of  clearly is bipartite because of inequalities
\eqref{inq:vertex:cover}.

Using Lemma \ref{lemma:semi-perfect:matching polytope} and Farkas'
Lemma~\cite{farkas1902} (or duality in linear programming), each of
the subproblems \eqref{eq:subpb:s}-\eqref{eq:subpb:t} has a feasible
solution if and only if

where  and  is the incidence matrix
of the subgraph .  Therefore, the integer linear programming
formulation

is equivalent to solving

with the separation problem of inequalities \eqref{eq:extreme:ray}
being solvable in polynomial time (it reduces to solving linear
programs).

