\documentclass[a4paper,leqno]{llncs}

\bibliographystyle{splncs}

\pagestyle{plain}

\makeatletter
\let\LNCSproof\proof
\let\LNCSendproof\endproof
\let\proof\@undefined
\let\endproof\@undefined
\makeatother

\newcommand{\keywords}[1]{\par\addvspace\baselineskip
\noindent\keywordname\enspace\ignorespaces#1}

\usepackage{amsmath}
\usepackage{amssymb}
\usepackage[usenames,dvipsnames]{color}
\usepackage{enumitem}
\usepackage{url}
\usepackage{comment}


\newcommand{\mybox}[1]{\ensuremath{\text{\mbox{\ensuremath{#1}}}}}
\newcommand{\spacedwith}[2]{\ensuremath{\phantom{#2}#1\phantom{#2}}}
\newcommand{\spaced}[1]{\spacedwith{#1}{m}}
\newcommand{\dspaced}[1]{\spacedwith{\spaced{#1}}{i}}
\newcommand{\primed}[1]{\ensuremath{#1^{\prime}}}
\newcommand{\dprimed}[1]{\ensuremath{#1^{\prime\prime}}}
\newcommand{\defn}{\spacedwith{\stackrel{\mathbf{def}}{=}}{o}}

\newcommand{\aleq}{\ensuremath{\subseteq}}
\newcommand{\ageq}{\ensuremath{\supseteq}}
\newcommand{\aeq}{\ensuremath{=}}
\newcommand{\aneq}{\ensuremath{\neq}}
\newcommand{\askip}{\ensuremath{\textit{skip}}}
\newcommand{\aabort}{\ensuremath{\textit{abort}}}
\newcommand{\abottom}{\ensuremath{\bot}}
\newcommand{\atopp}{\ensuremath{\top}}
\newcommand{\asemicolon}{\ensuremath{\,;}}
\newcommand{\abacksemicolon}{\ensuremath{\,\breve{;}\,}}
\newcommand{\asemicolontext}{\ensuremath{;}}
\newcommand{\abacksemicolontext}{\ensuremath{\breve{;}}}
\newcommand{\aor}{\ensuremath{\cup}}
\newcommand{\aand}{\ensuremath{\cap}}
\newcommand{\aunion}{\aor}
\newcommand{\aintersect}{\aand}
\newcommand{\alub}{\ensuremath{\bigcup}}
\newcommand{\aglb}{\ensuremath{\bigcap}}
\newcommand{\astar}{\ensuremath{\parallel}}
\newcommand{\kleene}[1]{\ensuremath{{#1}^*}}

\newcommand{\htriple}[3]{\mybox{{#1}\spacedwith{\{#2\}}{\,}{#3}}}
\newcommand{\config}[2]{\mybox{\langle {#1},\,{#2} \rangle}}
\newcommand{\plotkin}[4]{\mybox{\config{#1}{#2}\longrightarrow\config{#3}{#4}}}
\newcommand{\plotkiniter}[5]{\mybox{\config{#1}{#2}\longrightarrow^{#3}\config{#4}{#5}}}
\newcommand{\plotkinstar}[4]{\mybox{\config{#1}{#2}\longrightarrow^*\config{#3}{#4}}}
\newcommand{\milner}[3]{\mybox{{#1}\stackrel{#2}{\longrightarrow}{#3}}}
\newcommand{\kahn}[3]{\mybox{\config{#1}{#2}\longrightarrow{#3}}}
\newcommand{\btriple}[3]{\mybox{{#1}\spacedwith{\ll\mspace{-4mu}{#2}\mspace{-4mu}\gg}{\mspace{-2mu}}{#3}}}
\newcommand{\ftriple}[3]{\mybox{{#1}\spacedwith{[#2]}{\,}{#3}}}
\newcommand{\astriple}[3]{\mybox{{#1}\spacedwith{\##2\#}{\,}{#3}}}
\newcommand{\vtriple}[3]{\mybox{\{{#1}\}\spacedwith{#2}{\,}\{{#3}\}}}
\newcommand{\sembrack}[1]{\ensuremath{[\![#1]\!]}}

\newcommand{\Actions}{\mybox{\mathit{Actions}}}
\newcommand{\Atoms}{\mybox{\mathit{Atoms}}}
\newcommand{\AtomicOperations}{\mybox{\mathit{AtomicOperations}}}
\newcommand{\States}{\mybox{\Sigma}}
\newcommand{\Inconsistent}{\mybox{\mathit{Inconsistent}}}

\newcommand{\s}{\ensuremath{\sigma}}
\newcommand{\sprime}{\primed{\sigma}}
\newcommand{\sdprime}{\dprimed{\s}}
\newcommand{\stprime}{\ensuremath{\s^{\prime\prime\prime}}}
\newcommand{\sone}{\ensuremath{\sigma_1}}
\newcommand{\stwo}{\ensuremath{\sigma_2}}

\newcommand{\trace}{\ensuremath{t}}
\newcommand{\tprime}{\primed{t}}
\newcommand{\tdprime}{\dprimed{\t}}
\newcommand{\tone}{\ensuremath{t_1}}
\newcommand{\ttwo}{\ensuremath{t_2}}

\newcommand{\ic}[1]{\ensuremath{\mathit{ic}(#1)}}
\mathchardef\mhyphen="2D
\newcommand{\icTracesEndingInState}[1]{\ensuremath{\mathit{ic\mhyphen traces\mhyphen ending\mhyphen in\mhyphen state}(#1)}}
\newcommand{\tracesOfView}[1]{\ensuremath{\mathit{traces\mhyphen of\mhyphen view}(#1)}}
\newcommand{\traceSetOfAtomString}[1]{\ensuremath{\mathit{trace\mhyphen set\mhyphen of\mhyphen atom\mhyphen sequence}(#1)}}
\newcommand{\traceSetOfCommand}[1]{\ensuremath{\mathit{trace\mhyphen set\mhyphen of\mhyphen command}(#1)}}

\newcommand{\lfp}[1]{\ensuremath{\mathit{lfp\,}{#1}}}

\newcommand{\Views}{\mybox{\mathit{Views}}}
\newcommand{\vstar}{\ensuremath{*}}
\newcommand{\vunit}{\ensuremath{u}}
\newcommand{\vmodels}{\ensuremath{\models}}
\newcommand{\vlub}{\ensuremath{\bigvee}}
\newcommand{\vglb}{\ensuremath{\bigwedge}}
\newcommand{\vangelic}{\ensuremath{\prec}}
\newcommand{\view}{\ensuremath{v}}
\newcommand{\vprime}{\primed{v}}
\newcommand{\vdprime}{\dprimed{\view}}
\newcommand{\vtprime}{\ensuremath{\view^{\prime\prime\prime}}}
\newcommand{\vone}{\ensuremath{v_1}}
\newcommand{\vtwo}{\ensuremath{v_2}}
\newcommand{\voneprime}{\ensuremath{v_1^\prime}}
\newcommand{\vtwoprime}{\ensuremath{v_2^\prime}}
\newcommand{\erase}[1]{\ensuremath{\lfloor{#1}\rfloor}}
\newcommand{\Axioms}{\mybox{\mathit{Axioms}}}

\newcommand{\sand}{\spaced{\&}}
\newcommand{\siff}{\spaced{\Leftrightarrow}}
\newcommand{\simplies}{\spaced{\Rightarrow}}
\newcommand{\sor}{\spaced{\mid\mid}}
\newcommand{\saleq}{\spaced{\aleq}}
\newcommand{\sageq}{\spaced{\ageq}}
\newcommand{\saeq}{\spaced{=}}

\newcommand{\lemmaAlignSpace}{\mbox{}\
&\rulename{Hskip} \htriple{P}{\askip}{P} \\
&\rulename{Hseq} \htriple{P}{Q}{R}  \sand  \htriple{R}{\primed{Q}}{S}  \simplies  \htriple{P}{Q \asemicolon \primed{Q}}{S} \\
&\rulename{Hchoice} (\forall Q \in X : \htriple{P}{Q}{R})  \simplies  \htriple{P}{\alub X}{R} \\
&\rulename{Hiter} \htriple{P}{Q}{P}  \simplies  \htriple{P}{\kleene{Q}}{P} \\
&\rulename{Hcons} \primed{P} \aleq P \sand \htriple{P}{Q}{R} \sand R \aleq \primed{R} \simplies \htriple{\primed{P}}{Q}{\primed{R}} \\
&\rulename{Hdisj} (\forall P \in X : \htriple{P}{Q}{R}) \simplies \htriple{\alub X}{Q}{R}

&\rulename{Hconj} \htriple{P}{Q}{R}  \sand  \htriple{\primed{P}}{\primed{Q}}{\primed{R}}  \simplies  \htriple{P \aand \primed{P}}{Q \aand \primed{Q}}{R \aand \primed{R}} \\
&\rulename{Hframe} \htriple{P}{Q}{R} \simplies \htriple{F \astar P}{Q}{F \astar R} \\
&\rulename{Hconc} \htriple{P}{Q}{R} \sand \htriple{\primed{P}}{\primed{Q}}{\primed{R}} \simplies \htriple{P \astar \primed{P}}{Q \astar \primed{Q}}{R \astar \primed{R}}

&\rulename{Paction} P \in \Actions \sand s^\prime \aleq s \asemicolon P \simplies \plotkin{P}{s}{\askip}{s'} \\
&\rulename{Pseq1} \plotkin{\askip \asemicolon P}{s}{P}{s} \\
&\rulename{Pseq2} \plotkin{P}{s}{R}{s^\prime} \simplies \plotkin{P \asemicolon P^\prime}{s}{R \asemicolon P^\prime}{s^\prime} \\
&\rulename{Pchoice} P \in X \simplies \plotkin{\alub X}{s}{P}{s} \\
&\rulename{Piter1} \plotkin{\kleene{P}}{s}{\askip}{s} \\
&\rulename{Piter2} \plotkin{\kleene{P}}{s}{P \asemicolon \kleene{P}}{s} \\
&\rulename{Pconc1} \plotkin{\askip \astar P}{s}{P}{s} \\
&\rulename{Pconc2} \plotkin{P \astar \askip}{s}{P}{s} \\
&\rulename{Pconc3} \plotkin{P}{s}{R}{s^\prime} \simplies \plotkin{P \astar P^\prime}{s}{R \astar P^\prime}{s^\prime} \\
&\rulename{Pconc4} \plotkin{P}{s}{R}{s^\prime} \simplies \plotkin{P^\prime \astar P}{s}{P^\prime \astar R}{s^\prime}

&\rulename{Maction} P \in \Actions \simplies \milner{P}{P}{\askip} \\
&\rulename{Mseq1} \milner{\askip \asemicolon P}{\askip}{P} \\
&\rulename{Mseq2} \milner{P}{Q}{R} \simplies \milner{P \asemicolon P^\prime}{Q}{R \asemicolon P^\prime} \\
&\rulename{Mchoice} P \in X \simplies \milner{\alub X}{\askip}{P} \\
&\rulename{Miter1} \milner{\kleene{P}}{\askip}{\askip} \\
&\rulename{Miter2} \milner{\kleene{P}}{\askip}{P \asemicolon \kleene{P}} \\
&\rulename{Mconc1} \milner{\askip \astar P}{\askip}{P} \\
&\rulename{Mconc2} \milner{P \astar \askip}{\askip}{P} \\
&\rulename{Mconc3} \milner{P}{Q}{R} \simplies \milner{P \astar P^\prime}{Q}{R \astar P^\prime} \\
&\rulename{Mconc4} \milner{P}{Q}{R} \simplies \milner{P^\prime \astar P}{Q}{P^\prime \astar R}

&\rulename{Kskip}   \kahn{\askip}{s}{s}   \\
&\rulename{Kseq}   \kahn{P}{s}{s^\prime} \sand \kahn{\primed{P}}{s^\prime}{s^{\prime\prime}}  \simplies  \kahn{P \asemicolon \primed{P}}{s}{s^{\prime\prime}}  \\
&\rulename{Kchoice}  P \in X \sand \kahn{P}{s}{s^\prime} \simplies \kahn{\alub X}{s}{s^\prime} \\
&\rulename{Kiter1}   \kahn{\kleene{P}}{s}{s}    \\
&\rulename{Kiter2}   \kahn{P}{s}{s^\prime} \sand \kahn{\kleene{P}}{s^\prime}{s^{\prime\prime}}  \simplies  \kahn{\kleene{P}}{s}{s^{\prime\prime}} \\
&\rulename{Kconc1}   \kahn{P}{s}{s^\prime} \sand \kahn{\primed{P}}{s^\prime}{s^{\prime\prime}}  \simplies  \kahn{P \astar \primed{P}}{s}{s^{\prime\prime}} \\
&\rulename{Kconc2}   \kahn{P^\prime}{s}{s^\prime} \sand \kahn{P}{s^\prime}{s^{\prime\prime}} \simplies  \kahn{P \astar \primed{P}}{s}{s^{\prime\prime}}

&\rulename{Batom} (\view, a, \vprime) \in \Axioms \simplies \btriple{\view}{a}{\vprime}

&\rulename{Bskip} \btriple{\view}{\askip}{\view} \\
&\rulename{Bseq} \btriple{\view}{P}{\vprime} \sand \btriple{\vprime}{P^\prime}{\vdprime} \simplies \btriple{\view}{P \asemicolon P^\prime}{\vdprime} \\
&\rulename{Bchoice} (\forall P \in X : \btriple{\view}{P}{\vprime}) \simplies \btriple{\view}{\alub X}{\vprime}\\
&\rulename{Biter} \btriple{\view}{P}{\view} \simplies \btriple{\view}{\kleene{P}}{\view}

&\rulename{Bcons} \view \vangelic \vprime \sand \btriple{\vprime}{P}{\vdprime} \sand \vdprime \vangelic \vtprime \simplies \btriple{\view}{P}{\vtprime}

&\rulename{Bdisj} (\forall \view \in V : \btriple{\view}{P}{\vprime}) \simplies \btriple{\vlub V}{P}{\vprime}

&\rulename{Fatom} (\view, a, \vprime) \in \Axioms \simplies \ftriple{\view}{a}{\vprime}

&\rulename{Fskip} \ftriple{\view}{\askip}{\view} \\
&\rulename{Fseq} \ftriple{\view}{P}{\vprime} \sand \ftriple{\vprime}{P^\prime}{\vdprime} \simplies \ftriple{\view}{P \asemicolon P^\prime}{\vdprime} \\
&\rulename{Fchoice} (\forall P \in X : \ftriple{\view}{P}{\vprime}) \simplies \ftriple{\view}{\alub X}{\vprime}\\
&\rulename{Fiter} \ftriple{\view}{P}{\view} \simplies \ftriple{\view}{\kleene{P}}{\view}

&\rulename{Fcons} \view \vangelic \vprime \sand \ftriple{\vprime}{P}{\vdprime} \sand \vdprime \vangelic \vtprime \simplies \ftriple{\view}{P}{\vtprime}

&\rulename{Fdisj} (\forall \view \in V : \ftriple{\view}{P}{\vprime}) \simplies \ftriple{\vlub V}{P}{\vprime}

&\rulename{Fframe} \ftriple{\view}{P}{\vprime} \simplies \ftriple{\view \vstar \vdprime}{P}{\vprime \vstar \vdprime}

&\rulename{Vatom} (\view, a, \vprime) \in \Axioms \simplies \vtriple{\view}{a}{\vprime}

&\rulename{Vskip} \vtriple{\view}{\askip}{\view}

&\rulename{Vseq} \vtriple{\view}{C}{\vprime} \sand \vtriple{\vprime}{C^\prime}{\vdprime} \simplies \vtriple{\view}{C \asemicolon C^\prime}{\vdprime}

&\rulename{Vchoice} (\forall C \in Y : \vtriple{\view}{C}{\vprime}) \simplies \vtriple{\view}{\alub Y}{\vprime}

&\rulename{Viter} \vtriple{\view}{C}{\view} \simplies \vtriple{\view}{\kleene{C}}{\view}

&\rulename{Vcons} \view \vangelic \vprime \sand \vtriple{\vprime}{C}{\vdprime} \sand \vdprime \vangelic \vtprime \simplies \vtriple{\view}{C}{\vtprime}

&\rulename{Vdisj} (\forall \view \in V : \vtriple{\view}{C}{\vprime}) \simplies \vtriple{\vlub V}{C}{\vprime}

&\rulename{Vframe} \vtriple{\view}{C}{\vprime} \simplies \vtriple{\view \vstar \vdprime}{C}{\vprime \vstar \vdprime}

&\rulename{Vconc} \vtriple{\vone}{C_1}{\voneprime} \sand \vtriple{\vtwo}{C_2}{\vtwoprime} \simplies \vtriple{\vone \vstar \vtwo}{C_1 \astar C_2}{\voneprime \vstar \vtwoprime}

&\rulename{PDatom} a \in \AtomicOperations \sand \sprime \in a(\s) \simplies \plotkin{a}{\s}{\askip}{\sprime}

&\rulename{PDseq1} \plotkin{\askip \asemicolon P}{\s}{P}{\s} \\
&\rulename{PDseq2} \plotkin{P}{\s}{R}{\sprime} \simplies \plotkin{P \asemicolon P^\prime}{\s}{R \asemicolon P^\prime}{\sprime} \\
&\rulename{PDchoice} P \in X \simplies \plotkin{\alub X}{\s}{P}{\s} \\
&\rulename{PDiter1} \plotkin{\kleene{P}}{\s}{\askip}{\s} \\
&\rulename{PDiter2} \plotkin{\kleene{P}}{\s}{P \asemicolon \kleene{P}}{\s} \\
&\rulename{PDconc1} \plotkin{\askip \astar P}{\s}{P}{\s} \\
&\rulename{PDconc2} \plotkin{P \astar \askip}{\s}{P}{\s} \\
&\rulename{PDconc3} \plotkin{P}{\s}{R}{\sprime} \simplies \plotkin{P \astar P^\prime}{\s}{R \astar P^\prime}{\sprime} \\
&\rulename{PDconc4} \plotkin{P}{\s}{R}{\sprime} \simplies \plotkin{P^\prime \astar P}{\s}{P^\prime \astar R}{\sprime}

&\rulename{KDatom} a \in \AtomicOperations \sand \sprime \in a(\s) \simplies \kahn{a}{\s}{\sprime}

&\rulename{KDskip}   \kahn{\askip}{\s}{\s}   \\
&\rulename{KDseq}   \kahn{P}{\s}{\sprime} \sand \kahn{\primed{P}}{\sprime}{\sdprime}  \simplies  \kahn{P \asemicolon \primed{P}}{\s}{\sdprime}  \\
&\rulename{KDchoice}  P \in X \sand \kahn{P}{\s}{\sprime} \simplies \kahn{\alub X}{\s}{\sprime} \\
&\rulename{KDiter1}   \kahn{\kleene{P}}{\s}{\s}    \\
&\rulename{KDiter2}   \kahn{P}{\s}{\sprime} \sand \kahn{\kleene{P}}{\sprime}{\sdprime}  \simplies  \kahn{\kleene{P}}{\s}{\sdprime} \\
&\rulename{KDconc1}   \kahn{P}{\s}{\sprime} \sand \kahn{\primed{P}}{\sprime}{\sdprime}  \simplies  \kahn{P \astar \primed{P}}{\s}{\sdprime} \\
&\rulename{KDconc2}   \kahn{P^\prime}{\s}{\sprime} \sand \kahn{P}{\sprime}{\sdprime} \simplies  \kahn{P \astar \primed{P}}{\s}{\sdprime}

&\rulename{PCatom} a \in \AtomicOperations \sand \sprime \in a(\s) \simplies \plotkin{a}{\s}{\askip}{\sprime} \\
&\rulename{PCseq1} \plotkin{\askip \asemicolon C}{\s}{C}{\s} \\
&\rulename{PCseq2} \plotkin{C}{\s}{C^\prime}{\sprime} \simplies \plotkin{C \asemicolon C^{\prime\prime}}{\s}{C^\prime \asemicolon C^{\prime\prime}}{\sprime} \\
&\rulename{PCchoice} C \in Y \simplies \plotkin{\alub Y}{\s}{C}{\s} \\
&\rulename{PCiter1} \plotkin{\kleene{C}}{\s}{\askip}{\s} \\
&\rulename{PCiter2} \plotkin{\kleene{C}}{\s}{C \asemicolon \kleene{C}}{\s} \\
&\rulename{PCconc1} \plotkin{\askip \astar C}{\s}{C}{\s} \\
&\rulename{PCconc2} \plotkin{C \astar \askip}{\s}{C}{\s} \\
&\rulename{PCconc3} \plotkin{C}{\s}{C^\prime}{\sprime} \simplies \plotkin{C \astar C^{\prime\prime}}{\s}{C^\prime \astar C^{\prime\prime}}{\sprime} \\
&\rulename{PCconc4} \plotkin{C}{\s}{C^\prime}{\sprime} \simplies \plotkin{C^{\prime\prime} \astar C}{\s}{C^{\prime\prime} \astar C^\prime}{\sprime}

&\rulename{MCatom} a \in \AtomicOperations \simplies \milner{a}{a}{\askip} \\
&\rulename{MCseq1} \milner{\askip \asemicolon C}{\askip}{C} \\
&\rulename{MCseq2} \milner{C}{C^\prime}{C^{\prime\prime}} \simplies \milner{C \asemicolon C_1}{C^\prime}{C^{\prime\prime} \asemicolon C_1} \\
&\rulename{MCchoice} C \in Y \simplies \milner{\alub Y}{\askip}{C} \\
&\rulename{MCiter1} \milner{\kleene{C}}{\askip}{\askip} \\
&\rulename{MCiter2} \milner{\kleene{C}}{\askip}{C \asemicolon \kleene{C}} \\
&\rulename{MCconc1} \milner{\askip \astar C}{\askip}{C} \\
&\rulename{MCconc2} \milner{C \astar \askip}{\askip}{C} \\
&\rulename{MCconc3} \milner{C}{C^\prime}{C^{\prime\prime}} \simplies \milner{C \astar C_1}{C^\prime}{C^{\prime\prime} \astar C_1} \\
&\rulename{MCconc4} \milner{C}{C^\prime}{C^{\prime\prime}} \simplies \milner{C_1 \astar C}{C^\prime}{C_1 \astar C^{\prime\prime}}

&\rulename{KCatom} a \in \AtomicOperations \sand \sprime \in a(\s) \simplies \kahn{a}{\s}{\sprime} \\
&\rulename{KCskip}   \kahn{\askip}{\s}{\s}   \\
&\rulename{KCseq}   \kahn{C}{\s}{\sprime} \sand \kahn{\primed{C}}{\sprime}{\sdprime}  \simplies  \kahn{C \asemicolon \primed{C}}{\s}{\sdprime}  \\
&\rulename{KCchoice}  C \in Y \sand \kahn{C}{\s}{\sprime} \simplies \kahn{\alub Y}{\s}{\sprime} \\
&\rulename{KCiter1}   \kahn{\kleene{C}}{\s}{\s}    \\
&\rulename{KCiter2}   \kahn{C}{\s}{\sprime} \sand \kahn{\kleene{C}}{\sprime}{\sdprime}  \simplies  \kahn{\kleene{C}}{\s}{\sdprime} \\
&\rulename{KCconc1}   \kahn{C}{\s}{\sprime} \sand \kahn{\primed{C}}{\sprime}{\sdprime}  \simplies  \kahn{C \astar \primed{C}}{\s}{\sdprime} \\
&\rulename{KCconc2}   \kahn{C^\prime}{\s}{\sprime} \sand \kahn{C}{\sprime}{\sdprime} \simplies  \kahn{C \astar \primed{C}}{\s}{\sdprime}

&\rulename{PDfuturechoice} \plotkin{P \asemicolon (P^\prime \aor P^{\prime\prime})}{\s}{P \asemicolon P^\prime}{\s}

&\rulename{Hrec} (\forall Q : \htriple{P}{Q}{R} \simplies \htriple{P}{f(Q)}{R}) \simplies \htriple{P}{\lfp{f}}{R} \\
&\rulename{Brec} (\forall P : \btriple{\view}{P}{\vprime} \simplies \btriple{\view}{f(P)}{\vprime}) \simplies \btriple{\view}{\lfp{f}}{\vprime} \\
&\rulename{Frec} (\forall P : \ftriple{\view}{P}{\vprime} \simplies \ftriple{\view}{f(P)}{\vprime}) \simplies \ftriple{\view}{\lfp{f}}{\vprime} \\
&\rulename{Vrec} (\forall C : \vtriple{\view}{C}{\vprime} \simplies \vtriple{\view}{f(C)}{\vprime}) \simplies \vtriple{\view}{\lfp{f}}{\vprime}

&\rulename{PCrec} \plotkin{f(\lfp{f})}{\s}{C}{\sprime} \simplies \plotkin{\lfp{f}}{\s}{C}{\sprime} \\
&\tag{PCrec} \plotkin{\lfp{f}}{\s}{f(\lfp{f})}{\s} \\
&\rulename{MCrec} \milner{f(\lfp{f})}{C}{C^\prime} \simplies \milner{(\lfp{f})}{C}{C^\prime} \\
&\tag{MCrec} \milner{(\lfp{f})}{\askip}{f(\lfp{f})} \\
&\rulename{KCrec} \kahn{f(\lfp{f})}{\s}{\sprime} \simplies \kahn{\lfp{f}}{\s}{\sprime}


\subsection*{Further remarks}
The Knaster-Tarski characterization of the least fixpoint and the Kleene star laws imply that iteration is a special case of recursion:
\begin{lemma}

\end{lemma}
The monotonicity of the function  follows from the mo\-no\-to\-ni\-ci\-ty of (\aor) and (\asemicolontext). It is possible to derive the rules for iteration by using the ones for recursion.

Users of the framework can also apply the Kleene fixpoint theorem, which provides an alternative characterization of the least fixpoint of some functions. It uses the auxiliary notions of directed sets (of languages) and Scott-continuity:
\BEGINMYDEF
 \\

\ENDMYDEF
The theorem says that if  is Scott-continuous (and therefore monotone), then:
\BEGINMYDEF

\ENDMYDEF
The composition of Scott-continuous functions is again Scott-continuous, and it is simple to prove the Scott-continuity of (\asemicolontext), (\aor), (\astar) and the Kleene star.

\end{document}
