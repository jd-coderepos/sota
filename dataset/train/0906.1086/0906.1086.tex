\documentclass{amsart}
\usepackage{graphicx}
\usepackage{epic,eepic,eepicemu}
\usepackage{subfigure}
\usepackage[normalem]{ulem}\usepackage{epsfig}

\usepackage[usenames,dvipsnames]{pstricks}
\usepackage{pst-grad} \usepackage{pst-plot} 


\vfuzz2pt \hfuzz2pt \newtheorem{thm}{Theorem}[section]
\newtheorem{cor}[thm]{Corollary}
\newtheorem{lem}[thm]{Lemma}
\newtheorem{prop}[thm]{Proposition}
\newtheorem{prob}[thm]{Problem}
\newtheorem{conj}[thm]{Conjecture}
\newtheorem{Clm}{Claim}[thm]
\newtheorem{rem}[thm]{Remark}
\theoremstyle{definition}
\newtheorem{defn}[thm]{Definition}
\newtheorem{Obs}[thm]{Observation}
\theoremstyle{remark}

\newenvironment{prf}{{\bf \noindent Proof } }{\hfill\\}
\newenvironment{PrfClaim}{{\bf Proof }}{{\hfill\tiny{\\}}}
\newcommand{\norm}[1]{\left\Vert#1\right\Vert}
\newcommand{\abs}[1]{\left\vert#1\right\vert}
\newcommand{\set}[1]{\left\{#1\right\}}
\newcommand{\Real}{\mathbb R}
\newcommand{\eps}{\varepsilon}
\newcommand{\To}{\longrightarrow}
\newcommand{\BX}{\mathbf{B}(X)}
\newcommand{\A}{\mathcal{A}}
\newcommand{\prp}{}
\newcommand{\cprp}{}
\newcommand{\Ajoute}[1]{{{\uwave{#1}}} }\newcommand{\Enleve}[1]{{\xout{#1}} }\DeclareGraphicsRule{.bmp}{bmp}{.bb}{}

\title[]{On Fulkerson conjecture}
\author{J.L. Fouquet and J.M. Vanherpe}
\address{L.I.F.O., Facult\'e des Sciences, B.P. 6759 \\
Universit\'e d'Orl\'eans, 45067 Orl\'eans Cedex 2, FR}


\subjclass{035 C} \keywords{Cubic graph;  Perfect Matchings}

\begin{document}
\input{epsf.sty}
\begin{abstract}
If  is a bridgeless cubic graph, Fulkerson conjectured that we
can find   perfect matchings (a {\em Fulkerson covering})  with
the property that every edge of  is contained in exactly two of
them. A consequence of the Fulkerson conjecture would be that every
bridgeless cubic graph has  perfect matchings with empty
intersection (this problem is known as the Fan Raspaud Conjecture).
A {\em FR-triple} is a set of  such perfect matchings. We show
here how to derive a Fulkerson covering from two FR-triples.

Moreover, we give a simple proof that the Fulkerson conjecture holds
true for some classes of well known snarks.
\end{abstract}

\maketitle

\section{Introduction}
The following conjecture is due to Fulkerson, and appears first in
\cite{Ful71}.
\begin{conj}\label{Conjecture:Fulkerson} If  is a bridgeless
cubic graph, then there exist  perfect matchings 
of  with the property that every edge of  is contained in
exactly two of .
\end{conj}

We shall say that  , in the above
conjecture, is a {\em Fulkerson covering}. A consequence of the
Fulkerson conjecture would be that every bridgeless cubic graph has
 perfect matchings with empty intersection (take any  of the
 perfect matchings given by the conjecture). The following
weakening of this (also suggested by Berge) is still open.

\begin{conj}\label{Conjecture:Berge2}There exists a fixed integer 
such that every bridgeless cubic graph has a list of  perfect
matchings with empty intersection.
\end{conj}


For  this conjecture is known as the Fan Raspaud Conjecture.

\begin{conj}\cite{FanRas} \label{Conjecture:FanRaspaud} Every
bridgeless cubic graph contains perfect matching , , 
such that

\end{conj}

Let  be a cubic graph with  perfect matchings 
and  having an empty intersection. Since  satisfies the
Fan Raspaud conjecture, when considering these perfect matchings,
we shall say that  is a {\em
FR-triple}. We define  () as the set of
edges of  which are covered  times by . It will be
convenient to use () for the FR-triple .

\section{FR-triples and Fulkerson covering}


 In
this section, we are concerned with the relationship between
FR-triples and Fulkerson coverings.

\subsection{On FR-triples}

\begin{prop} \label{Proposition:StructureFR}
Let  be a bridgeless cubic graph with  a FR-triple.
Then  and  are disjoint matchings.
\end{prop}
\begin{prf}
Let  be a vertex incident to an edge of . Since  must
be incident to each perfect matching of  and since the
three perfect matchings have an empty intersection, one of the
remaining edges incident to  must be contained in  perfect
matchings while the other is contained in exactly one perfect
matching. The result follows.
\end{prf}


We introduce now  concepts and definitions coming from
\cite{HaoNiuWanZhaZha2009}. Let  be an edge of bridgeless cubic
graph . We shall say that we have {\em splitted} the edge 
when we have applied the operation depicted in Figure
\ref{Figure:SplittingEdge}. The resulting graph is no longer cubic
since we get  vertices with degree  instead of two vertices of
degree . Let  and  be two disjoint matchings of 
(we insist to say that these matchings are not, necessarily, perfect
matchings).  For , let  be the graph obtained by
splitting the edges of  and let  be the
graph homeomorphic to  when the degree  vertices are
deleted. The connected component of  are cubic
graphs and {\em vertexless loop graphs} (graph with one edge and no
vertex). We shall say that   is edge
colourable whenever the cubic components are edge colourable
(any colour can be given to the vertexless loops).

The following Lemma can be obtained from the work of Hao and
al. \cite{HaoNiuWanZhaZha2009} when considering FR-triples.

\begin{figure}
\includegraphics{SplittingEdge.ps}
\caption{ Splitting an edge}
\label{Figure:SplittingEdge}
\end{figure}

\begin{lem} \label{Lemma:FRGBarreT2}
Let  be a bridgeless cubic graph and let  be a
FR-triple. Then  is edge colourable.
\end{lem}
\begin{prf}



Assume that  is a FR-triple.  Let  be an
edge of  then the two edges of  incident with 
must be in the same perfect matching of . Hence, these
two edges are identified in some sens.  If we colour the edges of
 with ,  or  when they are in ,  or
 respectively, we get a natural edge colouring of
.
\end{prf}

\begin{lem} \label{Lemma:GBarreA1FR}
Let  be a bridgeless cubic graph containing  two disjoint
matchings  and  such that  is
edge colourable and  forms an union of disjoint cycles. Then  has a FR-triple   where
 and .
\end{lem}
\begin{prf}
Obviously, forms an union of disjoint even cycles
in . Let  be an even cycle of
 and assume that  when .

Let ,  and   be the three matchings associated
to a  edge-colouring of .  Thanks to the
construction of  for some , the
third edge incident to , say , and the third one incident to
, say  lead to a unique  edge of .
Assume that this edge of  is in , then
 can be extended naturally to a matching of  containing
. Moreover we add  to  and 
to . When applying this process to all edges of  on all
cycles of  we extend the colours of
 into perfect matchings of . Since every edge
of  belongs to at most  matchings in 
we have a FR-triple with . By
construction, we have  and , as claimed.

\end{prf}


\begin{prop} \label{Proposition:EquivalenceStructureFR}
Let  be a bridgeless cubic graph then  has a FR-triple if and
only if  has  two disjoint matchings  and  such
that  forms an union of disjoint cycles, moreover  or  is edge
colourable.
\end{prop}
\begin{prf}
Assume that  has  two disjoint matchings  and  such
that, without loss of generality,  is edge
colourable. From Lemma \ref{Lemma:GBarreA1FR},  has a FR-triple
 where  and .


Conversely, assume that  is a FR-triple. From Lemma
\ref{Lemma:FRGBarreT2}  is -edge
colourable.  Let  and . Then  and  are two disjoint matchings and  is
edge colourable.
\end{prf}

\subsection{On compatible FR-triples}
As pointed out in the introduction, any three perfect matchings in a
Fulkerson covering lead to a FR-triple. Is it possible to get a
Fulkerson covering when we know one or more FR-triples?   In fact,
we can characterize a Fulkerson covering in terms of FR-triples in
the following way.

Let  be a bridgeless cubic graph with  and  two
FR-triples. We shall say that  and  are
{\em compatible} whenever  and  (and
hence  ).


\begin{thm}\label{Theorem:CompatibleFRTriples}
Let  be a bridgeless cubic graph then  can be provided with a
Fulkerson covering if and only if  has  two compatible
FR-triples.
\end{thm}

\begin{prf}
Let  be a Fulkerson covering of
 and let  and .  and  are two
FR-triples and we claim that they are compatible. Since each edge of
  is covered exactly twice by ,  the set of
edges covered only once by  must be covered also only
once by ,  the set of edges not covered by
 must be covered exactly twice by  and
 the set of edges covered exactly twice by  is
not covered by . Which means that ,
 and , that is  and
 are compatible.


Conversely, assume that  and  are two
FR-triples compatible. Then it is an easy task to check that each
edge of  is contained in exactly  perfect matchings of the 
perfect matchings involved in  or .
\end{prf}

\begin{prop}\label{Proposition:EquivalenceFRTripleFulkerson}
Let  be a bridgeless cubic graph then  has two compatible
FR-triples if and only if  has  two disjoint matchings 
and  such that  forms an union of disjoint cycles and  and
 are edge colourable.
\end{prop}
\begin{prf}
Let  and  be  compatible FR-triples.
From Lemma \ref{Lemma:FRGBarreT2} we know that
 and  are edge
colourable. Since  and  by the
compatibility of  and , the result holds
when we set  and .


Conversely, assume that  has  two disjoint matchings  and
 such that  and 
are edge colourable. From Lemma \ref{Lemma:GBarreA1FR},  has
a FR-triple  where  and  as
well as a FR-triple  where  and
. These two FR-triples are obviously compatible.
\end{prf}




\begin{prop}\cite{HaoNiuWanZhaZha2009}\label{Proposition:HaoNiuWanZhanZhan}
Let  be a bridgeless cubic graph then  can be provided with a
Fulkerson covering if and only if  has  two disjoint matchings
 and  such that  forms an union of disjoint cycles and  and
 are edge colourable.
\end{prop}
\begin{prf}
Obvious in view of Theorem \ref{Theorem:CompatibleFRTriples} and
Proposition \ref{Proposition:EquivalenceFRTripleFulkerson}.
\end{prf}

\section{Fulkerson covering for some classical snarks}
A non edge colourable, bridgeless, cyclically edge-connected
cubic graph is called a {\em snark}. 

For an odd , let  be the cubic graph on  vertices , , ,
 such that  is an
induced cycle of length ,   is an induced cycle of length  and for 
the vertex  is adjacent to ,  and . The set
 induces the claw . In Figure
\ref{Figure:J3} we have a representation of , the half edges
(to the left and to the right in the figure) with same labels are
identified.
For  those graphs were introduced by Isaacs in \cite{Isa75} under the name of flower snarks in order to provide an infinite family of snarks.

Proposition \ref{Proposition:HaoNiuWanZhanZhan} is essentially used in
\cite{HaoNiuWanZhaZha2009} in order to show that the so called
flower snarks and Goldberg snarks can be provided with a Fulkerson
covering. We shall see, in this section, that this result can be
directly obtained.
\begin{figure}
\centering \epsfsize=0.45 \hsize \noindent \epsfbox{j3.eps}
\caption{} \label{Figure:J3}
\end{figure}

\begin{thm}  \label{Theorem:FlowerSnark} For any odd ,  can be
provided with a Fulkerson covering.
\end{thm}
\begin{prf}
For  the Fulkerson covering is given in Figure \ref{Figure:J3}.
We obtain a Fulkerson covering of  by inserting a suitable
number of subgraphs isomorphic to the subgraph depicted in Figure
\ref{Figure:JMaillon} when we cut  along the dashed line of
Figure \ref{Figure:J3}. The labels of the edges of the two sets of
three semi-edges (left and right) are identical which insures that
the process can be repeated as long as necessary. These labels lead
to the perfect matchings of the Fulkerson covering.

\begin{figure}
\includegraphics{JMaillon.ps}
\caption{A block for the flower snark} \label{Figure:JMaillon}
\end{figure}
\end{prf}


Let  be the graph depicted in Figure \ref{Figure:Hi}

\begin{figure}
\includegraphics{Hi.ps}
\caption{} \label{Figure:Hi}
\end{figure}


Let  ( odd) be a cubic graph obtained from  copies of
 ( where the name of vertices are indexed
by ) by adding edges  , ,
,  and  (subscripts are
taken modulo ).

If , then  is known as the Goldberg snark (see
\cite{Gol81}). Accordingly, we refer to all graphs  as Goldberg
graphs. The graph  is shown in Figure \ref{Figure:Goldberg5}.
The half edges (to the left and to the right in the figure) with
same labels are identified.


\begin{figure}
\centering 
\includegraphics[scale=0.8]{Goldberg5.ps}
\caption{Goldberg snark } \label{Figure:Goldberg5}
\end{figure}

\begin{thm}  \label{Theorem:GoldbergSnark} For any odd ,  can be
provided with a Fulkerson covering.
\end{thm}
\begin{prf}
We give first a Fulkerson covering of  in Figure
\ref{Figure:Goldberg3Fulkerson}. The reader will complete easily the
matchings along the cycles by remarking that these cycles are
incident to  edges with a common label from  to  and to
exactly one edge of each remaining label. We obtain a Fulkerson
covering of  with odd  by inserting a suitable
number of subgraphs isomorphic to the subgraph depicted in Figure
\ref{Figure:MaillonGoldberg} when we cut  along the dashed
line. The labels of the edges of the two sets of three semi-edges
(left and right) are identical which insures that the process can be
repeated as long as necessary. These labels lead to the perfect
matchings of the Fulkerson covering.
\begin{figure}
\centering 
\subfigure[A Fulkerson covering for ]{
\includegraphics[scale=0.8]{Goldberg3.ps}
\label{Figure:Goldberg3Fulkerson}
}
\subfigure[A block for the Goldberg snark]{
\includegraphics[scale=0.8]{GoldbergMaillon.ps}
\label{Figure:MaillonGoldberg}
}
\caption{Fulkerson covering for the Golberg Snarks}
\end{figure}
\end{prf}


\section{A technical tool}

Let  be a perfect matching, a set  is an {\em
balanced matching} when we can find a perfect matching 
 such that . Assume that   are 
 pairwise disjoint balanced matchings, we shall say that
  is an {\em F-family for } whenever the three following
 conditions are fulfilled:

 \begin{itemize}
   \item [i \label{Item:1}] Every odd cycle of  has exactly one
vertex
   incident with one edge of each matching in .
   \item [ii \label{Item:2}] Every even cycle of  incident with
some matching in    contains  vertices such that
   two of them are incident to one matching in  while the other are
incident to another matching in  or the  vertices are incident to the same matching in 
   .
   \item [ iii \label{Item:3}] The subgraph induced by  vertices so determined in the previous
items has a matching.
 \end{itemize}
It will be convenient to denote the set of edges described in the
third item by .
\begin{thm} \label{Theorem:TechnicalTool}  Let  be a bridgeless cubic graph
together with
a perfect matching  and an F-family for  . Then 
can be provided with a Fulkerson covering.
\end{thm}
\begin{prf} Since  and  are balanced matchings, we can
find  perfect matchings , ,  and 
such that


Let , we will prove that
 is a Fulkerson
covering of .



\begin{Clm}\label{Claim:NperfectMatching}
 is a perfect matching
\end{Clm}
\begin{PrfClaim}
The vertices of  which are not incident with some edge in  are precisely those which are end vertices of
edges in . From the definition of an  F-family, the 
vertices defined on each cycle of 
incident to edges of  form a matching with two edges,
which insures that  is a perfect matching.

\end{PrfClaim}
Let  be the set of cycles
of  and let  and  be two distinct members of
.
\begin{Clm}\label{Claim:OddCycles}
Let  be an odd cycle. Assume that  and 
have ends  and  on . Then   is the only edge of
 not covered by 
\end{Clm}
\begin{PrfClaim}
Since  ( respectively) is a perfect matching, the
edges of  ( respectively) contained in 
saturate every vertex of  with the exception of  (
respectively). The result follows.
\end{PrfClaim}


\begin{Clm}\label{Claim:EvenCyclesFirst}
Let  be an even cycle. Assume that  and
 have ends  and  on  with
 and . Then  and
 are the only edges of  not covered by 
\end{Clm}
\begin{PrfClaim} The perfect matching  must saturate every vertex of
 with the
exception of  and . The same holds with  and
 and . Since  and  are edges of
, these two edges are not covered by  and we
can easily check that the other edges are covered.

\end{PrfClaim}

\begin{Clm}\label{Claim:EvenCyclesSecond}
Let  be an even cycle. Assume that  and
 have ends  and  on  with
 and . Then either 
and  are the only edges of  not covered by
 or  induces a perfect matching
on  such that every edge in that perfect matching is covered
by  and  with the exception of  which
belongs to  and  which belongs to .
\end{Clm}
\begin{PrfClaim}The perfect matching  must saturate every vertex of
 with the
exception of the two consecutive vertices  and . The
same holds with  and  and .


Let us recall here that, since   ( respectively) is a balanced
matching, the paths determined by  and  on 
have odd lengths (the paths determined by  and 
respectively). Two cases may occur.

{\bf case 1:} {\it The two paths obtained on  by deleting
the edges  and   have odd lengths} We can
check that  determines a perfect matching on
 such that every edge in that perfect matching is covered by
 and  with the exception of  which belongs
to  and  which belongs to 



{\bf case 2:} {\it The two paths obtained on  by deleting
 and   have even lengths} We can check that
 covers every edge of  with the exception
of  and .

\end{PrfClaim}


\begin{Clm}\label{Claim:EvenCyclesThird}
Let  be an even cycle. Assume that  have
ends  and  on  with  and .  Then we can choose a perfect
matching  in such a way that  and 
are the only edges of  not covered by .
\end{Clm}
\begin{PrfClaim}
Since  is a perfect matching, the edges of   contained
in  saturate every vertex of  with the exception of
 and . Since   is not incident to
 the perfect matching  can be chosen in two ways
(taking one of the two perfect matchings contained in this cycle).
We can see easily that we can choose  in such a way that
every edge distinct from  and  is covered by
 or .
\end{PrfClaim}

Since  is a partition of , each edge of 
is covered twice by some perfect matchings of .

Let  be an odd cycle, each edge of 
distinct from the two edges of  (Claim \ref{Claim:OddCycles}) is
covered twice by some perfect matchings of . The two
edges of   are covered by exactly one perfect matching belonging
to  and by the perfect matching .
Hence every edge of  is covered twice by .

Let  be an even cycle. Assume first that 
vertices of  are ends of some edges in  while no other
set of  is incident with . From Claim
\ref{Claim:EvenCyclesThird} we can choose  in such a way that
 every edge distinct from the two edges of   is
covered by   or . We can then choose  in such a
way that one of the two edges of  belongs to . Finally, we
can choose  in order to cover the other edge of . Each
edge of  distinct from the two edges of  (Claim
 \ref{Claim:EvenCyclesThird}) is
covered twice by some perfect matchings of . The two
edges of   are covered by exactly one perfect matching belonging
to  and by the perfect matching .
Hence every edge of  is covered twice by .


Assume now that  vertices of  are ends of some edges in
 (say  and ) and  other vertices are ends of
some edges in  (say  and ).



 {\bf case 1:}  and . We can choose 
and  in order to cover every edge of . From Claim
\ref{Claim:EvenCyclesFirst} every edge of  is covered by
 with the exception of  and
. Hence every edge of  is covered twice by
 while  and
 are covered twice by 
Hence every edge of  is covered twice by .

 {\bf case 2:}  and .
Assume that  and  are the only edges of
 not covered by  (Claim
\ref{Claim:EvenCyclesSecond}). Then we can choose  and 
in such a way that every edge of  is covered by . In that case every edge of  is covered twice by
 with the exception of
 and  which are covered twice by .

Assume now that   induces a perfect matching on
 where  and 
while the other edges of this perfect matchings are in  (Claim \ref{Claim:EvenCyclesSecond}). Then we can choose
 and  such that every edge of  not contained
in  is covered twice by  ( induces a perfect matching on ). hence every
edge of  is covered twice by  or by  with the exception of  which is covered
twice by  and  which is covered twice
by .


Finally, assume that  has no vertex as end of some edge in
. Then we can choose easily  and
 such that every edge of  is covered twice by


Hence  is a Fulkerson covering of .
\end{prf}
\begin{rem} \label{Rem:TousDistincts}
Observe that the matchings of the Fulkerson covering described in the above
proof are all distinct.
\end{rem}
\subsection{Dot products which preserve an F-family}
In \cite{Isa75} Isaacs defined the {\em dot product} operation in order to
describe infinites families of non trivial snarks.

Let  be two bridgeless cubic graphs and
,  and
 with
 and
.

The {\em dot product} of   and , denoted by 
is the bridgless cubic graph  defined
as  (see Figure \ref{Fig:DotProduct})~:


\begin{figure}
\begin{center}
\includegraphics[scale=0.8]{DotProductOperation.ps}
\end{center}
\caption{The dot product operation\label{Fig:DotProduct}} 
\end{figure}

It is well known that the dot product of two snarks remains to be a
snark. It must be pointed out that in general the dot product
operation does not permit to extend a Fulkerson covering, in other
words, whenever  and  are snarks together with a Fulkerson
covering, we do not know how to get a Fulkerson covering for .


However, in some cases, the dot product operation can preserve, in
some sense, an family, leading thus to a Fulkerson covering of the
resulting graph.
\begin{prop}\label{Proposition:DotProduct_1}
Let  be a perfect matching of a snark  such that
 contains only two (odd) cycles, namely  and
.  Let  be an edge of  and  be an edge of .

Let   be a perfect matching of a snark  where   is an family for .
Let  be an edge of , with  and   vertices of two distinct odd cycles of
.

Then  is an -family for the perfect matching  of
 with .
\end{prop}
\begin{prf}
Obvious by the definition of the family and the  construction of
the graph resulting of the dot product.
\end{prf}
\begin{prop}\label{Proposition:DotProduct_2}
Let   be a perfect matching of a snark  where   is an family for .
Let  and  be two edges of   not
contained in .

Let  be a perfect matching of a snark  such that
 contains only two (odd) cycles, namely  and
. Let , with  and .


Then  is an -family for the perfect matching  of
 with .
\end{prop}
\begin{prf}
Obvious by the definition of the family and the  construction of
the graph resulting of the dot product.
\end{prf}

We remark that the graphs obtained via Propositions
\ref{Proposition:DotProduct_1} and \ref{Proposition:DotProduct_2}
can be provided with a Fulkerson covering by Theorem
\ref{Theorem:TechnicalTool}. 

The dot product operations described in Propositions \ref{Proposition:DotProduct_1} and \ref{Proposition:DotProduct_2} will be said to {\em preserve} the F-family.

\section{Applications}
\begin{figure}
\includegraphics[scale=1]{FfamilyDansPetersen.ps}
 \caption{An F-family  for the Petersen graph.} \label{Fig:FfamilyDansPetersen}
\end{figure}
\begin{figure}
\subfigure[An F-family  for the flower snark .]{
  \includegraphics[scale=1]{FfamilyDansJ5.ps}\label{subfig:FfamilyDansJ5}}

\subfigure[Two more Claws.]{
  \includegraphics[scale=1]{TwoMoreClaws.ps}\label{subfig:TwoMoreClaws}
}
  \caption{An F-family  for the flower snark .} 
\label{Fig:FfamilyDansJk}
\end{figure}
\subsection{Fulkerson coverings, more examples}
Figures \ref{Fig:FfamilyDansPetersen} and \ref{subfig:FfamilyDansJ5} show
that the Petersen Graph as well as the flower snark  have oddness  and
have an F-family (the dashed edges denote the related perfect matching).




Moreover, as shown in Figure \ref{subfig:TwoMoreClaws} the F-family of  can be extended by induction to all the 's ( odd).

Thus, following the above Propositions we can define a sequence  of cubic graphs as follows ~:
\begin{itemize}
 \item Let  be the Petersen graph or the flower snark  (,  odd).
 \item For ,  where  is either the
Petersen graph or the flower snark  (,  odd) and the dot product operation preserves
the F-family.
\end{itemize}
\begin{figure}
\includegraphics[scale=0.6]{FfamilyDansSzekeresSnark.ps}
\caption{An F-family  for the Szekeres Snark.} \label{Fig:FfamilyDansSzekeres}
\end{figure}
\begin{figure}
 \subfigure[Blanu\v{s}a snark of type ]{
  \includegraphics[scale=0.7]{FfamilyDansBlanusa1.ps}\label{subfig:FfamilyDansBlanusa1}
 }
\subfigure[Blanu\v{s}a snark of type ]{
  \includegraphics[scale=0.7]{FfamilyDansBlanusa2.ps}\label{subfig:FfamilyDansBlanusa2}
 }
\caption{An F-family  for the Generalized Blanu\v{s}a snarks} \label{Figure:GeneralizedBlanusa}
\end{figure}



As a matter of fact this sequence of iterated dot products of the Petersen graph
and/or the flower snark  forms a family of exponentially many snarks
including the Szekeres Snark (see Figure \ref{Fig:FfamilyDansSzekeres}) as well as the two types of generalized Blanu\v{s}a snarks proposed by Watkins in \cite{Wat89} (see Figure \ref{Figure:GeneralizedBlanusa}). 

The family obtained when reducing the possible values of  to   has already been defined by Skupie\'n in \cite{Sku89}, in order to provide a family
of hypohamiltonian snarks in using the so-called {\em Flip-flop construction}
introduced by Chv\'atal in \cite{Chv73}.

As far as we know there is no Fulkerson family for the Golberg snark.

\subsection{Graphs with a -factor of 's.}
Let  be a bridgeless cubic graph having a factor where each
cycle is isomorphic to a chordless . We denote by  the
multigraph obtained from  by shrinking each  of this
factor in a single vertex. The graph  is regular and
we can easily check that it is bridgeless.

\begin{thm} \label{Theorem:2FactorC5}  Let  be a bridgeless cubic graph
having a factor
of chordless . Assume that  
 has chromatic index . Then  can be provided with a Fulkerson covering.
\end{thm}
\begin{prf} Let  be the perfect matching complementary of the factor of
.
Let  be a edge colouring of . Each colour
corresponds to a matching of  (let us denote these matchings by
 and ). Then it is an easy task to see that  is an F-family for  and the result follows from
Theorem \ref{Theorem:TechnicalTool}.
\end{prf}

\begin{thm} \label{Cor:2FactorC5Biparti}  Let  be a bridgeless cubic graph
having a factor of
 chordless . Assume that  
 is bipartite. Then  can be provided with a Fulkerson covering.
\end{thm}
\begin{prf}
It is well known, in that case, the chromatic index of  is
. the result follows from Theorem \ref{Theorem:2FactorC5}.
\end{prf}

Remark that, when considering the Petersen graph  , the graph
associated  is reduced to two vertices and is thus bipartite.

We can construct  cubic graphs with chromatic index   which are cyclically - edge connected ({\em
snarks} in the literature)
and having a -factor of 's. Indeed, let  be cyclically
-edge connected snark of size  and  be a perfect matching
of , . Let  be 
 cyclically -edge connected snarks (each of them having a
-factor of ). For each  () we
consider two edges  and  of the perfect
matching which is the complement of the factor.

We construct then a new cyclically -edge connected snark   by
applying the dot-product operation on  and the edge  (). We remark that the vertices of  vanish in the
operation and the resulting graph  has a  factor of ,
 which is the union of the factors of  of the
. Unfortunately, when considering the graph , derived
from , we cannot insure, in general, that  is edge
colourable in order to apply Theorem \ref{Theorem:2FactorC5} and
obtain hence a Fulkerson covering of .

An interesting case is obtained when, in the above construction of
, each graph  is isomorphic to the Petersen graph. Indeed,
the factor of 's obtained then is such that we can find a
partition of the vertex set of  in sets of   joined
by  edges. These sets lead to pairs of vertices of  joined
by three parallel edges. We can thus see  as a cubic graph
where a perfect matching is taken  times. Let us denote by
 this cubic graph (by the way  is
-connected). It is an easy task to see that, when  is
edge colourable,  is edge colourable and hence,
Theorem \ref{Theorem:2FactorC5} can be applied.

\begin{figure}
\includegraphics[scale=1]{PetersenRizzi.ps}
 \caption{ isomorphic to } \label{Figure:UnslicableP3}
\end{figure}

Let us consider by example the graph  obtained with  copies of
the Petersen graph following the above construction (let us remark
that the graph   involved in our construction  must be isomorphic
also to the Petersen graph). This graph is a snark on  vertices.
Since  is a bridgeless cubic graph, the only case for
which we cannot say whether  has a Fulkerson covering occurs when
 is isomorphic to the Petersen graph and, hence 
is isomorphic to the {\em unslicable} graph 
described by Rizzi \cite{Riz99} (see Figure
\ref{Figure:UnslicableP3}). As a matter of fact we do not know if it is possible to construct a graph  as described above such that  is isomorphic to the graph .

By the way, we do not know example of cyclically -edge connected
snarks (excepted the Petersen graph)  with a -factor of induced
cycles of length . We have proposed in \cite{FouVan09a} the
following problem.

\begin{prob} Is there any -edge connected snark distinct from the Petersen
graph
with a -factor of 's~?
\end{prob}





\section{On proper Fulkerson covering\label{Section:FulkersonProperCovering}}


As noticed in the introduction, when a cubic graph is edge
colourable, we can find a Fulkerson covering by using a edge
colouring and considering each colour twice.
\begin{prop} \label{Proposition:ProperIndex4}
Let  be a bridgeless cubic graph with chromatic index . Assume
that  has a Fulkerson covering  of its edge set. Then the 
perfect matchings are distinct.
\end{prop}
\begin{prf}Assume, without loss of generality that .
Since each edge is contained in exactly  perfect matchings of
, we must have . Hence 
is edge colourable, a contradiction.
\end{prf}

Let us say that a Fulkerson covering is {\em proper} whenever the
 perfect matchings involved in this covering are distinct. An
interesting question is thus to determine which cubic bridgeless
graph have a proper Fulkerson covering.

A -edge colourable graph is said to be {\em bi-hamiltonian} whenever in any
-edge colouring, there are at least two colours whose removing leads to an
hamiltonian -factor.

\begin{prop} \label{Proposition:PropernonHamiltonian}
Let  be a bridgeless -edge colourable cubic graph which is not bi-hamiltonian. Then  has a
proper Fulkerson covering.
\end{prop}
\begin{prf}
Let  be a  edge
colouring of . When  and  are colours in
,   denotes the set of disjoint
even cycles induced by the two colours  and .
\newline Since the graph  is not bi-hamiltonian we may assume that the
-factors  and  are not hamiltonian
cycles.
Let  be a
cycle in , we get a new edge colouring
 by exchanging the two colours  and  along
. We get hence a partition of  into  perfect matching
 and . In the same way, when
considering a cycle  in , we get a new
edge colouring  of  by exchanging  and
 along . Let , and  be
the  perfect matchings so obtained.

Since we have two distinct edge colourings of ,
and , the set of  perfect matchings so
involved
 is a
Fulkerson covering. It remains to show that this set is actually a
proper Fulkerson covering.

The exchange operated in
order to get  involve some edges in  and some
edges in  (those which are on ) while the other edges
keep their colour. In the same way,  the exchange operated in order
to get  involve some edges in  and some edges in
 (those which are on ) while the other edges keep
their colour.

The  perfect matchings of  ( and
 ) are pairwise disjoint as well as those of 
( and ).  We have  since  contains some edges of . We
have  and  since we have exchanged  and 
in order to obtain  and . We have
 since  contains some edges
of  while  contains some edges of . We
have  since  contains some
edges of  and  contains only edges in 
or in . We have  since
 contains some edges of .

Hence
 is a
proper Fulkerson covering.

\end{prf}

The {\em theta graph }(2 vertices joined by 3 edges), ,
 are examples of small bridgeless cubic graph without
proper Fulkerson covering. The infinite family of bridgeless cubic
bi-hamiltonian graphs obtained by doubling the edges of a perfect matching of an
even cycle has no proper Fulkerson covering.
On the other hand, we can provide a bi-hamiltonian graph together with a proper
Fulkerson covering. Consider for example the graph  on  vertices which
have a  factor of 's, namely  and  with the additional
edges edges , , ,  and , it is not difficult to check that
this graph is bi-hamiltonian. Moreover since the following four balanced
matchings , ,  and  form an F-family for the
perfect matching , due to Theorem
\ref{Theorem:TechnicalTool} and Remark \ref{Rem:TousDistincts}, the graph 
has a proper Fulkerson covering.


A challenging problem is thus
to characterize those bridgeless cubic graphs having a proper Fulkerson
covering.

{\bf Acknownledgement.} The authors are gratefull to Professor Skupie\'n for his
helpfull comments on the Flip-flop construction.
\bibliographystyle{amsplain}
\bibliography{BibliographieBergeFulkerson}
\end{document}
