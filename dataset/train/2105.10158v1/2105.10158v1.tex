

\documentclass[11pt,a4paper]{article}
\usepackage[hyperref]{acl2021}
\usepackage{times}
\usepackage{latexsym}
\renewcommand{\UrlFont}{\ttfamily\small}

\usepackage{microtype}
\aclfinalcopy 




\usepackage[utf8]{inputenc}
\usepackage{graphicx}
\usepackage{url}
\usepackage{booktabs}
\usepackage{bbm}
\usepackage{amsmath}
\usepackage{amsthm}
\usepackage{amssymb}
\usepackage{algorithm}
\usepackage{algorithmic}
\usepackage{subfigure}
\usepackage{cases}
\usepackage{multirow}
\usepackage{multicol}

\newcommand\BibTeX{B\textsc{ib}\TeX}
\newcommand{\mymethod}[0]{\textsc{ReRe} }
\newcommand{\casrel}[0]{\textsc{CasRel} }
\DeclareMathOperator*{\argmin}{arg\,min}
\DeclareMathOperator*{\argmax}{arg\,max}
\newcommand{\bci}[0]{\mathbf{c}_i}
\newcommand{\sro}[0]{\langle s,r,o \rangle}
\newcommand{\hyrc}[0]{\mathbf{\hat{y}}_{rc}}
\newcommand{\yrc}[0]{\mathbf{y}_{rc}}
\newcommand{\hy}[0]{\hat{y}}
\newcommand{\by}[0]{\mathbf{y}}
\newcommand{\hby}[0]{\mathbf{\hat{y}}}
\newcommand{\hyee}[0]{\mathbf{\hat{y}}_{ee}}
\newcommand{\yee}[0]{\mathbf{y}_{ee}}
\newcommand{\bw}[1]{\mathbf{W}_{#1}}
\newcommand{\bb}[1]{\mathbf{b}_{#1}}
\newcommand{\ep}[0]{\mathbb{E}}

\title{Revisiting the Negative Data of Distantly Supervised Relation Extraction}

\author{Chenhao Xie, Jiaqing Liang, Jingping Liu, Chengsong Huang, Wenhao Huang, Yanghua Xiao\\
Fudan University, Shanghai, China\\
\{redreamality, l.j.q.light\}@gmail.com \quad \{jpliu17,huangcs19,whhuang17,shawyh\}@fudan.edu.cn
}

\date{}

\begin{document}
\maketitle
\begin{abstract}
Distantly supervision automatically generates plenty of training samples for relation extraction. 
However, it also incurs two major problems: noisy labels and imbalanced training data.
Previous works focus more on reducing wrongly labeled relations (false positives) while few explore the missing relations that are caused by incompleteness of knowledge base (false negatives).
Furthermore, the quantity of negative labels overwhelmingly surpasses the positive ones in previous problem formulations.
In this paper, we first provide a thorough analysis of the above challenges caused by negative data.
Next, we formulate the problem of relation extraction into as a positive unlabeled learning task to alleviate false negative problem. 
Thirdly, we propose a pipeline approach, dubbed \textsc{ReRe}, that performs sentence-level relation detection then subject/object extraction to achieve sample-efficient training.
Experimental results show that the proposed method consistently outperforms existing approaches and remains excellent performance even learned with a large quantity of false positive samples.
\end{abstract}


\section{Introduction}
Relational extraction is a crucial step towards knowledge graph construction.
It aims at identifying relational triples from a given sentence in the form of subject, relation, object, in short, .
For example, given S1 in Figure \ref{fig:re_example}, we hope to extract \textsc{William Shakespeare, Birthplace, stratford-upon-Avon}.




This task is usually modeled as a supervised learning problem and \textit{distant supervision} \cite{mintz2009distant} is utilized to acquire large-scale training data. 
The core idea is to obtain training data is through automatically labeling a sentence with existing relational triples from a knowledge base (KB). 
For example, given a triple  and a sentence, if the sentence contains both  and , distant supervision methods regard  as a valid sample for the sentence.
If no relational triples are applicable, the sentence is labeled as ``NA". 



\begin{figure}[t]
  \centering
  \includegraphics[width=\columnwidth]{resources/re_example.png}
  \caption{Illustration of distant supervision process. S2-S5 are examples for four kinds of label noise. 
  TP, FP, FN and PL mean \textit{true positive, false positive, false negative} and \textit{partially labeled}, respectively.
  ``R-" or ``E-" indicates whether the error occurs at \textit{relation-level} or \textit{entity-level}. \textbf{Bold tokens} are ground-truth subjects/objects. \underline{Underlined tokens} together with the relation in the third column are labeled by distant supervision. ``NA" means no relation.}
  \label{fig:re_example}
\end{figure}



Despite the abundant training data obtained with distant supervision, nonnegligible errors also occur in the labels.
There are two types of errors. 
In the first type, the labeled relation does not conform with the original meaning of sentence, and this type of error is referred to as \textit{false positive} (FP). 
For example, in \textit{S2}, The ``\textit{Shakespeare spent the last few years of his life in Stratford-upon-Avon.}" does not express the relation \textsc{Birthplace}, thus being a FP. 
Second, large amounts of relations in sentences are missing due to the incompleteness of KB, which is referred to as \textit{false negative} (FN). 
For instance, in \textit{S3}, ``\textit{Buffett was born in 1930 in Omaha, Nebraska.}" is wrongly labeled as NA since there is no relation (e.g., \textsc{Birthplace}) between \textsc{Buffett} and \textsc{Omaha, Nebraska} in the KB. 
Many efforts have been devoted to solving the FP problem, including pattern-based methods~\cite{jia2019arnor}, multi-instance learning methods~\cite{lin2016neural,zeng2018large} and reinforcement learning methods~\cite{feng2018reinforcement}. Significant improvements have been made. 

\textit{However, FN problem receives much less attention}~\cite{min2013distant,xu2013filling,roller2015improving}.
To the best of our knowledge, none existing work with deep neural networks to solve this problem.
We argue that this problem is fatal in practice since there are massive FN cases in datasets. 
For example, there exist at least 33\% and 35\% FN's in NYT and SKE datasets, respectively. We will deeply analyze the problem in Section~\ref{sec:fn}

Another huge problem for relation extraction is \textit{overwhelming negative labels}.
As is widely acknowledged, information extraction tasks are highly imbalanced in class labels~\cite{Chowdhury2012ImpactOL,lin2018adaptive,Li2020DiceLF}.
In particular, the negative labels account for most of the labels in relation extraction under almost any problem formulation, which makes relation extraction a hard machine learning problem. 
We systematically analyze this in Section~\ref{sec:neglabel}.

In this paper, we address these challenges caused by negative data. Our main contribution can be summarized as follows. 
\begin{itemize}
    \item We systematically compare the class distributions of different problem modeling and explain why first extract relation then entities, i.e., the third paradigm (P3) in Section~\ref{sec:neglabel},  is superior to the others.
    \item Based on the first point, we adopt P3 and propose a novel two-staged pipeline model dubbed \textsc{ReRe}. 
    It first detects relation at sentence level and then extracts entities for a specific relation.
    We model the false negatives in relation extraction as ``unlabeled positives" and propose a multi-label collective loss function.
    \item Our empirical evaluations show that the proposed method consistently outperforms existing approaches, and achieves excellent performance even learned with a large quantity of false positive samples. 
    We also provide two carefully annotated test sets aiming at reducing the false negatives of previous annotation, namely, NYT21 and SKE21, with 370 and 1150 samples, respectively.
\end{itemize}




\section{Problem Analysis and Pilot Experiments}

We use  to denote a training instance, where  is a sentence consisting of  tokens  labeled by a set of triples  from the training set .
For rigorous definition,  can be viewed as an ordered set  so that set operations can be applied.
We assume , where  is a finite set of all relations in . 
Other model/task-specific notations are defined after each problem formulation.

We now clarify some terms used in the introduction and title without formal definition.
A \textbf{negative sample} refers to a triple  .
\textbf{Negative label} refers to the negative class label (e.g., usually ``0'' for binary classification), used for supervision with respect to task-specific models. Under different task formulation, the negative labels can be different.
\textbf{Negative data} is a general term that includes both negative labels and negative samples.
There are two kinds of \textit{false negatives}.
\textbf{Relation-level false negative} (S3 in Figure~\ref{fig:re_example}) refers to the situation where there exists , but  is actually expressed by , and does not appear in other .
Similarly, \textbf{Entity-level false negative} (S4 and S5 in Figure~\ref{fig:re_example}) means  appears in other .
\textbf{Imbalanced class distribution} means that the quantity of negative labels are much larger than positive ones.

\subsection{Addressing the False Negatives}
\label{sec:fn}
As shown in Table~\ref{tab:stat}, the triples in NYT (SKE) datasets\footnote{Detailed description of datasets is in Sec.~\ref{sec:data}} labeled by Freebase\footnote{\cite{bollacker2008freebase}}
(BaiduBaike\footnote{\url{https://baike.baidu.com/}}) is 88,253 (409,767), 
while the ones labeled by Wikidata\footnote{\cite{Vrandecic2014WikidataAF}} (CN-DBPedia\footnote{~\cite{Xu2017CNDBpediaAN}}) are 58,135 (342,931). 
In other words, there exists massive FN matches if only labeled by one KB due to the incompleteness of KBs. 
Notably, we find that the FN rate is underestimated by previous researches~\cite{min2013distant,xu2013filling}, based on the manual evaluation of which there are 15\%-35\% FN matches.
This discrepancy may be caused by human error.
In specific, a volunteer may accidentally miss some triples.
For example, as pointed out by Wei et al.~\shortcite[in Appendix C]{Wei2020ANC}, the test set of NYT11~\cite{Hoffmann2011KnowledgeBasedWS} missed lots of triples, especially when multiple relations occur in a same sentence, though labeled by human.
That also provides an evidence that FN's are harder to discover than FP's.






















\begin{table}[!htbp]
  \centering
\small
        \begin{tabular}{lrrrr}
        \toprule
                 & \multicolumn{2}{c}{NYT (English)} & \multicolumn{2}{c}{SKE (Chinese)} \\
        \cmidrule{2-5}    \# Sentence & \multicolumn{2}{c}{56,196} & \multicolumn{2}{c}{194,747} \\
        \midrule
                 & \multicolumn{1}{l}{\# Triples} & \multicolumn{1}{l}{\# Rels} & \multicolumn{1}{l}{\# Triples} & \multicolumn{1}{l}{\# Rels} \\
        \cmidrule{2-5}    Original & 88,253   & 23       & 409,767  & 49 \\
        Re-labeled & 58,135   & 57       & 342,931  & 378 \\
        Intersection & 13,848   & 18       & 121,326  & 46 \\
        Union    & 132,540  & 62       & 631,372  & 381 \\
        \midrule
        Original FNR &  0.33     &          &  0.35     &  \\
        Relabel FNR &  0.56     &          &  0.46     &  \\
        \bottomrule
    \end{tabular}\caption{Statistics of the quantity of distantly labeled relational triples by using different KB's. The ``original" refers to freebase for NYT and BaiduBaike for SKE. The ``relabeled" means aligning using Wikidata and CN-DBpedia to re-label NYT and SKE datasets.
    In specific, we consider triples with the same subject and object to be candidate triples and use a relation mapping table to determine whether the triples match.
    The intersection of SKE dataset has two values because the original relation has a one-to-many mapping with relations in CN-DBpedia. FNR stands for false negative rates, calculated by using the \# Triples in Original(Re-labeled) divided by the union.}
\label{tab:stat}\end{table}



\begin{table*}[!htbp]
    \centering
    \small
      \begin{tabular}{lllrrrrrr}
          \toprule
          \multicolumn{1}{r}{\multirow{3}[6]{*}{Paradigm}} & \multicolumn{2}{c}{\multirow{2}[4]{*}{Theoretical}} & \multicolumn{2}{l}{NYT10-HRL} & \multicolumn{2}{l}{NYT11-HRL} & \multicolumn{2}{l}{SKE} \\
          \cmidrule{4-9}             & \multicolumn{2}{r}{} & \multicolumn{2}{l}{=31,= 39.08} & \multicolumn{2}{l}{=11, =39.46} & \multicolumn{2}{l}{=51,= 54.67} \\
          \cmidrule{2-9}             &   &   & \multicolumn{1}{l}{} & \multicolumn{1}{l}{} & \multicolumn{1}{l}{} & \multicolumn{1}{l}{} & \multicolumn{1}{l}{} & \multicolumn{1}{l}{} \\
          \midrule
            then  & --       &  & \multicolumn{1}{l}{--} & 0.01421  & \multicolumn{1}{l}{--} & 0.00280  & \multicolumn{1}{l}{--} & 0.00494 \\
            then  &  &  & 0.0585   & 0.00093  & 0.0574   & 0.00257  & 0.0405   & 0.00067 \\
           then  &  &  & 0.0390   & 0.00842  & 0.0826   & 0.00835  & 0.0344   & 0.00927 \\
          \bottomrule
      \end{tabular}\caption{Comparison of class prior under different relation extraction paradigms.  means the total number of relations and  is the average sentence length.  () refers to the class prior for the first (second) task in the pipeline.  for the first paradigm is omitted because it is often considered a preceding step.  is the summation of 1's in labels, of using which our intention is to represent the information a positive sample conveys.}
    \label{tab:neglabel}\end{table*}

\subsection{Addressing the Overwhelming Negative Labels}
\label{sec:neglabel}

We point out that some of the previous paradigms designed for relation extraction aggravate the imbalance and lead to inefficient supervision.
The mainstream approaches for relation extraction mainly fall into three paradigms depending on what to extract first.

\begin{itemize}
    \item[P1] The first paradigm is a pipeline that begins with named entity recognition (NER) and then classifies each entity pair into different relations, i.e., [ then ]. 
    It is adopted by many traditional approaches~\cite{mintz2009distant,chan2011exploiting,zeng2014relation,Zeng2015DistantSF,gormley2015improved,dos2015classifying,lin2016neural}. 
    \item[P2] The second paradigm first detects all possible subjects in a sentence then identifies objects with respect to each relation, i.e., [ then ]. Specific implementation includes modeling relation extraction as multi-turn question answering~\cite{Li2019EntityRelationEA}, span tagging~\cite{yu2020joint} and cascaded binary tagging \cite{Wei2020ANC}.
\item[P3] The third paradigm first perform sentence-level relation detection (cf. P1, which is at entity pair level.) then extract subjects and entities, i.e., [ then ].
    This paradigm is largely unexplored.
    HRL~\cite{Takanobu2019AHF} is hitherto the only work to apply this paradigm based on our literature review.
\end{itemize}
We provide theoretical analysis of the output space and class prior with statistical support from three datasets (see Section~\ref{sec:data} for description) of the three paradigms in Table~\ref{tab:neglabel}.
The second step of P1 can be compared with the first step of P3.
Both of them find relation from a sentence (P1 with target entity pair given).
Suppose a sentence contains  entities\footnote{Below the same.}, the classifier has to decide relation from  entity pairs, while in reality, relations are often sparse, i.e., . 
In other words, most entity pairs in P1 do not form valid relation, thus resulting in a low class prior.
The situation is even worse when the sentence contains more entities, such as in NYT11-HRL.
In addition, P1 is not sample efficient because the classifier will be trained/queried  for the same sentences.
For P2, we demonstrate with the problem formulation of \textsc{CasRel}~\cite{Wei2020ANC}.
The difference of the first-step class prior between P2 and P3 depends on the result of comparison between \# relations and average sentence length (i.e.,  and ), which varies in different scenarios/domains.
However,  of P2 is extremely low, where a classifier has to decide from a space of .
In contrast, P3 only has to decide from  based on our task formulation (Section~\ref{sec:pdef})

Other task formulations include jointly extracting the relation and entities \cite{yu2010jointly,Li2014IncrementalJE,miwa2014modeling,gupta2016table,katiyar2017going,Ren2017CoTypeJE} and recently in the manner of sequence tagging \cite{Zheng2017JointEO}, sequence-to-sequence learning \cite{Zeng2018ExtractingRF}.
In contrast to the aforementioned three paradigms, most of these methods actually provide an \textit{incomplete decision space} that cannot handle all the situation of relation extraction, for example, the overlapping one~\cite{Wei2020ANC}.









\section{Solution Framework}

\subsection{Framework of \mymethod}
\label{sec:pdef}

Given an instance  from , the goal of training is to maximize the likelihood defined in Eq.~\eqref{eq:likelihood}. It is decomposed into two components by applying the definition of conditional probability, formulated in Eq. \eqref{eq:2stage}.


where we use  as a shorthand for , which means that  occurs in the triple set w.r.t.\ ; Similarly, ,  stands for  and , respectively.
 represents a subset of  with a common relation .
 is an indicator function;  when the condition happens.
We denote by  the model parameters.

Under this decomposition, relational triple extraction task is formulate into two subtasks: \textit{relation classification} and \textit{entity extraction}.
\paragraph*{Relation Classification.} As is discussed, building relation classifier at entity-pair level will introduce excessive negative samples that forms a hard learning problem.
Therefore, we alternatively model the relation classification at sentence level.
Intuitively speaking, we hope that the model could capture \textit{what relation a sentence is expressing.}
We formalize it as a multi-label classification task. 

where  is the probability that  is expressing , the -th relation\footnote{ is parameterized by , omitted in the equation for clarity, below the same.}.
 is the ground truth from the labeled data;  is equivalent to  while  means the opposite.

\paragraph*{Entity Extraction.} 
We then model entity extraction task. 
We observe that given the relation  and context , it naturally forms a machine reading comprehension (MRC) task~\cite{chen2018neural}, where  naturally fits into the paradigm of (\textsc{query, context, answer}).
Particularly, the subjects and objects are continuous spans from , which falls into the category of \textit{span extraction}.
We adopt the boundary detection model with answer pointer~\cite{wang2016machine} as the output layer, which is widely used in MRC tasks.
Formally, for a sentence of  tokens,

where  represents the identifier of each pointer;  refers to the probability of -th token being the start/end of the subject/object.
 is the ground truth from the training data;
if  occurs in  at position from  to , then  and , otherwise ; the same applies for the \textit{objects}.

\subsection{Advantages}
Our task formulation shows several advantages.
By adopting P3 as paradigm, the first and foremost advantage of our solution is that it suffers less from the imbalanced classes (Section~\ref{sec:neglabel}).
Secondly, relation-level false negative is easy to recover.
When modeled as a standard classification problem, many off-the-shelf methods on positive unlabeled learning can be leveraged.
Thirdly, entity-level false negatives do not affect relation classification.
Taking S5 in Figure~\ref{fig:re_example} as an example, even though the \textsc{Birthplace} relation between \textsc{William Swartz} and \textsc{Scranton} is missing, the relation classifier can still capture the signal from the other sample with a same relation, i.e.,  \textsc{Joe Biden, Birthplace, Scranton} .
Fourthly, this kind of modeling is easy to update with new relations without the need of retraining a model from bottom up.
Only relation classifier needs to be redesigned, while entity extractor can be updated in an online manner without modifying the model structure.
Last but not the least, relation classifier can be regarded as a pruning step when applied to practical tasks. Many existing methods treat relation extraction as question answering~\cite{Li2019EntityRelationEA,Zhao2020AskingEA}. 
However, without first identifying the relation, they all need to iterate over all the possible relations and ask diverse questions.
This results in extremely low efficiency where time consumed for predicting one sample may takes up to  times larger than our method.














\section{Our Model}
\label{sec:model}

The relational triple extraction task decomposed in Eq.~\eqref{eq:2stage} inspires us to design a two-staged pipeline, in which we first detect relation at sentence level and then extract subjects/objects for each relation.
The overall architecture of \mymethod is shown in Figure \ref{fig:framework}.


\begin{figure*}[!ht]
  \centering
  \includegraphics[width=1.3\columnwidth]{resources/framework.pdf}
  \caption{The overall architecture of \textsc{ReRe}. In this example, there are two relations, \textsc{Nationality}and \textsc{Creator}, can be found in the Relation Classifier, which will be sent to the Entity Extractor one by one along with the sentence. When The relation \textsc{Nationality} is extracted, the Entity Extractor will find the position of the subject and object of Nationality. The word \textsc{American} and \textsc{Dick Dillin} will be found. The relation \textsc{Creator} will then be handled similarly. The values of grey blocks in  are zero.}
  \label{fig:framework}
\end{figure*}




\subsection{Sentence-level Relation Classifier}
\label{sec:rc}
We first detect relation at sentence level.
The input is a sequence of tokens  and we denote by  the output vector of the model, which aims to estimate  in Eq.~\eqref{eq:pr}.
We use BERT~\cite{Devlin2019BERTPO} for English and RoBERTa~\cite{liu2019roberta} for Chinese, pre-trained language models with multi-layer bidirectional Transformer structure \cite{vaswani2017attention}, to encode the inputs\footnote{For convenience, we refer to the pre-trained Transformer as BERT hereinafter.}.
Specifically, the input sequence , which is fed into BERT for generating a token representation matrix , where  is the hidden dimension defined by pre-trained Transformers.
We take , which is the encoded vector of the first token \texttt{[CLS]}, as the representation of the sentence.
The final output of relation classification module  is defined in Eq.~\eqref{eq:yrc}.


where  and  are trainable model parameters, representing weights and bias, respectively;  denotes the sigmoid activation function.






\subsection{Relation-specific Entity Extractor}
After the relation detected at sentence-level, we extract subjects and objects for each candidate relation.
We aim to estimate , of which each element corresponds to  in Eq.~\eqref{eq:pso}, using a deep neural model.
We take , the one-hot output vector of relation classifier, and generate query tokens  using each of the detected relations (i.e., the ``1''s in ). 
We are aware that many recent works~\cite{Li2019EntityRelationEA,Zhao2020AskingEA} have studied how to generate diverse queries for the given relation, which have the potential of achieving better performance.
Nevertheless, that is beyond the scope of this paper.
To keep things simple, we use the \textit{surface text} of a relation as the query.

Next, the input sequence is constructed as .
Like Section~\ref{sec:rc}, we get the token representation matrix  from BERT.
The -th output pointer of entity extractor is defined by

where  is in accordance to Eq.~\eqref{eq:pso};  and  are the corresponding parameters.

The final subject/object spans are generated by pairing the nearest / with /. 
Next, all subjects are paired to the nearest object. 
If multiple objects occur before the next subject appears, all subsequent objects will be paired with it until next subject occurs.


\subsection{Multi-label Collective Loss function}

In normal cases, the log-likelihood is taken as the learning objective.
However, as is emphasized, there exists many false negative samples in the training data.
Intuitively speaking, the negative labels cannot be simply considered as negative. 
Instead, a small portion of the negative labels should be considered as unlabeled positives and their influence towards the penalty should be eradicated.
Therefore, we adopt cPU~\cite{Xie2020CollectiveLF}, a collective loss function that is designed for \textit{positive unlabeled learning} (PU learning).
To briefly review, cPU considers the learning objective to be the
\textit{correctness} under a surrogate function,

where they redefine the \textit{correctness} function for PU learning as 

where  is the ratio of false negative data (i.e., the unlabeled positive in the original paper).

We extend it to multi-label situation by embodying the original expectation at sample level. 
Due to the fact that class labels are highly imbalanced for our tasks, we introduce a class weight  to downweight the positive penalty.
For relation classifier,

For entity extractor,


In practice, we set , where  is the ratio of false negative and  is the class prior. 
Note that  is not difficult to estimate for both relation classification and entity extraction task in practice.
Besides various of methods in the PU learning  \cite{Plessis2015ClasspriorEF,Bekker2018EstimatingTC} for estimating it, an easy approximation is  when , which happens to be the case for our tasks.



\section{Experiments}
\subsection{Datasets}
\label{sec:data}
Our experiments are conducted on these four datasets\footnote{We do not use WebNLG~\cite{Gardent2017CreatingTC} and ACE04\footnote{\url{https://catalog.ldc.upenn.edu/LDC2005T09}} because these datasets are not automatically labeled by distant supervision. WebNLG is constructed by natural language generation with triples. ACE04 is manually labeled.}.
Some statistics of the datasets are provided in Table~\ref{tab:stat} and Table~\ref{tab:neglabel}.
In relation extraction, some datasets with the same names involve different preprocessing, which leads to unfair comparison. 
We briefly review all the datasets below and specify the operations to perform before applying each dataset.
\begin{itemize}
    \normalsize
    \item \textbf{NYT} \cite{Riedel2010ModelingRA}. 
    NYT is the very first version among all the NYT-related datasets.
    It is based on the articles in New York Times\footnote{\url{https://www.nytimes.com/}}. 
    We use the sentences from it to conduct the pilot experiment in Table~\ref{tab:stat}.
    However, 1) it contains duplicate samples, e.g., 1504 in the training set; 2) It only labels the last word of an entity, which will mislead the evaluation results.
    \item \textbf{NYT10-HRL.} \& \textbf{NYT11-HRL.}
    These two datasets are based on NYT. 
    The difference is that they both contain \textit{complete} entity mentions.
    NYT10~\cite{Riedel2010ModelingRA} is the original one.
    and NYT11~\cite{Hoffmann2011KnowledgeBasedWS} is a small version of NYT10 with 53,395 training samples and a manually labeled test set of 368 samples.
    We refer to them as NYT10-HRL and NYT11-HRL after preprocessed by HRL~\cite{Takanobu2019AHF} where they removed 1) training relation not appearing in the testing and 2) ``NA" sentences.
    These two steps are almost adopted by all the compared methods.
    To compare fairly, we use this version in evaluations.

    \item \textbf{NYT21}. We provide relabel version of the test set of NYT11-HRL. 
    The test set of NYT11-HRL still have false negative problem.
    Most of the samples in the NYT11-HRL has only one relation. 
    We manually added back the missing triples to the test set.
    
    \item \textbf{SKE2019/SKE21}\footnote{\url{http://ai.baidu.com/broad/download?dataset=sked}}.
    SKE2019 is a dataset in Chinese published by Baidu.
    The reason we also adopt this dataset is that it is currently the largest dataset available for relation extraction.
    There are 194,747 sentences in the train set and 21,639 in the validate set.
    We manually labeled 1,150 sentences from the test set with 2,765 annotated triples, which we refer to as SKE21. 
    No preprocessing for this dataset is needed.
    We provide this data for future research\footnote{download url.}.
\end{itemize}

\begin{table*}[ht!]
  \centering
  \small
  \addtolength{\tabcolsep}{-2.5pt}
    \begin{tabular}{p{4.2cm}rrrrrrrrrlll}
    \toprule
             & \multicolumn{3}{c}{NYT10-HRL}  & \multicolumn{3}{c}{NYT11-HRL}  & \multicolumn{3}{c}{NYT21}      & \multicolumn{3}{c}{SKE21} \\
\cmidrule{2-13}             & \multicolumn{1}{c}{Prec.} & \multicolumn{1}{c}{Rec.} & \multicolumn{1}{c}{F1} & \multicolumn{1}{c}{Prec.} & \multicolumn{1}{c}{Rec.} & \multicolumn{1}{c}{F1} & \multicolumn{1}{c}{Prec.} & \multicolumn{1}{c}{Rec.} & \multicolumn{1}{c}{F1} & \multicolumn{1}{c}{Prec.} & \multicolumn{1}{c}{Rec.} & \multicolumn{1}{c}{F1} \\
    \midrule
    KB Match & 38.10       & 32.38    & 34.97    & 47.92    & 31.08    & 37.7     & 47.92    & 29.56    & 36.57    & 69.12    & 28.1     & 39.96 \\
    MultiR \cite{Hoffmann2011KnowledgeBasedWS} & \multicolumn{1}{c}{-} & \multicolumn{1}{c}{-} & \multicolumn{1}{c}{-} & 32.8     & 30.6     & 31.7     & \multicolumn{1}{c}{-} & \multicolumn{1}{c}{-} & \multicolumn{1}{c}{-} & \multicolumn{1}{c}{-} & \multicolumn{1}{c}{-} & \multicolumn{1}{c}{-} \\
    SPTree \cite{Miwa2016EndtoEndRE} & 49.2     & 55.7     & 52.2     & 52.2     & 54.1     & 53.1     & \multicolumn{1}{c}{-} & \multicolumn{1}{c}{-} & \multicolumn{1}{c}{-} & \multicolumn{1}{c}{-} & \multicolumn{1}{c}{-} & \multicolumn{1}{c}{-} \\
    NovelTagging \cite{Zheng2017JointEO} & 59.3     & 38.1     & 46.4     & 46.9     & 48.9     & 47.9     & \multicolumn{1}{c}{-} & \multicolumn{1}{c}{-} & \multicolumn{1}{c}{-} & \multicolumn{1}{c}{-} & \multicolumn{1}{c}{-} & \multicolumn{1}{c}{-} \\
    CoType \cite{Ren2017CoTypeJE} & \multicolumn{1}{c}{-} & \multicolumn{1}{c}{-} & \multicolumn{1}{c}{-} & 48.6     & 38.6     & 43.0       & \multicolumn{1}{c}{-} & \multicolumn{1}{c}{-} & \multicolumn{1}{c}{-} & \multicolumn{1}{c}{-} & \multicolumn{1}{c}{-} & \multicolumn{1}{c}{-} \\
    CopyR \cite{Zeng2018ExtractingRF} & 56.9     & 45.2     & 50.4     & 34.7     & 53.4     & 42.1     & \multicolumn{1}{c}{-} & \multicolumn{1}{c}{-} & \multicolumn{1}{c}{-} & \multicolumn{1}{c}{-} & \multicolumn{1}{c}{-} & \multicolumn{1}{c}{-} \\
    HRL \cite{Takanobu2019AHF} & 71.4     & 58.6     & 64.4     & 53.8     & 53.8     & 53.8     & \multicolumn{1}{c}{-} & \multicolumn{1}{c}{-} & \multicolumn{1}{c}{-} & \multicolumn{1}{c}{-} & \multicolumn{1}{c}{-} & \multicolumn{1}{c}{-} \\
    TPLinker \cite{wang2020tplinker}* & \textbf{81.19} & 65.41    & 72.45    & 56.2     & 55.14    & 55.67    & 59.78    & 55.78    & 57.71    & \multicolumn{1}{c}{-} & \multicolumn{1}{c}{-} & \multicolumn{1}{c}{-} \\
    CasRel \cite{Wei2020ANC}* & 77.7     & 68.8     & 73.0       & 50.1     & 58.4     & 53.9     & 58.64    & 56.62    & 57.61    & \multicolumn{1}{c}{-} & \multicolumn{1}{c}{-} & \multicolumn{1}{c}{-} \\
    \midrule
    \mymethod - LSTM  & 56.71    & 42.00       & 48.26    & \textbf{56.46} & 35.4     & 43.52    & \textbf{62.06} & 37.01    & 46.37    & \multicolumn{1}{c}{-} & \multicolumn{1}{c}{-} & \multicolumn{1}{c}{-} \\
    \mymethod & 75.45    & \textbf{72.50} & \textbf{73.95} & 53.12    & \textbf{59.59} & \textbf{56.23} & 57.69    & \textbf{61.69} & \textbf{59.62} & \multicolumn{1}{c}{\textbf{-}} & \multicolumn{1}{c}{\textbf{-}} & \multicolumn{1}{c}{\textbf{-}} \\
    \midrule
    \midrule
    TPLinker \cite{wang2020tplinker}*(exact) & \underline{80.34} & 65.11    & 71.93    & \underline{55.43} & 55.12    & 55.28    & \underline{58.96} & 55.78    & 57.33    & 83.86    & 84.77    & 84.32 \\
    CasRel \cite{Wei2020ANC}*(exact) & 75.12    & 65.72    & 70.11    & 47.88    & 55.13    & 51.25    & 55.06    & 54.49    & 54.78    & 86.94    & \underline{85.96} & 86.45 \\
\mymethod (exact) & 74.90     & \underline{71.97} & \underline{73.4} & 52.40     & \underline{58.91} & \underline{55.47} & 56.97    & \underline{60.93} & \underline{58.88} & \underline{90.44} & 84.20     & \underline{87.21} \\
    \bottomrule
    \end{tabular}\addtolength{\tabcolsep}{2.5pt}
  \caption{The main evaluation results of different models on NYT10-HRL,  NYT11-HRL, and two hand labeled test set NYT21 and SKE21 on by the compared method on the datasets. 
  The results with only one decimal are quoted from~\cite{Wei2020ANC}. The methods with * are based on our re-implementation.
  Best partial (exact) match results are marked \textbf{bold} (\underline{underlined}). }
  \label{tab:overall}\end{table*}


\subsection{Compared Methods and Metrics}
We evaluate our model by comparing with several models on the same datasets, which are SOTA graphical model MultiR~\cite{Hoffmann2011KnowledgeBasedWS}, joint models  SPTree~\cite{Miwa2016EndtoEndRE} and NovelTagging~\cite{Zheng2017JointEO},
recent strong SOTA models CopyR~\cite{Zeng2018ExtractingRF}, HRL~\cite{Takanobu2019AHF}, CasRel~\cite{Wei2020ANC}, TPLinker~\cite{wang2020tplinker}.
We also provide the result of automatically aligning Wikidata/CN-KBpedia with the corpus, namely \textit{Match}, as a baseline.
To note, we only keep the intersected relations, otherwise it will result in low precision due to the false negative in the original dataset.
We report standard micro Precision (Prec.), Recall (Rec.) and F1 score for all the experiments.
Following the previous works~\cite{Takanobu2019AHF,Wei2020ANC}, we adopt partial match on these data sets for fair comparison. 
We also provide the results of exact match results of the methods we implemented, and only exact match on SKE2019.

\subsection{Overall Comparison}

We show the overall comparison result in Table~\ref{tab:overall}.
First, we observe that \mymethod consistently outperforms all the compared models.
We find an interesting result that by purely aligning the database with the corpus, it already achieves surprisingly good overall result (surpassing MultiR) and relatively high precision (comparable to CoType in NYT11-HRL).
However, the recall is quite low, which accords with our discussion in Section ~\ref{sec:fn} that distant supervision leads to many false negatives.
We also provide an ablation result where BERT is replaced with a bidirectional LSTM encoder~\cite{graves2013speech} with randomly initialized weights.
From the results we discover that even without BERT, our framework achieves competitive results against the previous approaches such as CoType and CopyR.
This further prove the effectiveness of our \mymethod  framework.




\begin{figure*}[!ht]
    \centering
    \begin{subfigure}
        \centering
        \includegraphics[width=0.99\columnwidth]{resources/NYT10-HRL.pdf}
    \end{subfigure}
\begin{subfigure}
        \centering
        \includegraphics[width=0.99\columnwidth]{resources/NYT11-HRL.pdf}
    \end{subfigure}
    \caption{Precision-Recall Curve of \mymethod and \textsc{CasRel} under different false negative rate. 
    Lines are better in the upper-right corner than the opposite.
    Note that the coordinates do not starts from 0.}
    \label{fig:pr}
\end{figure*}




\subsection{How Robust is \mymethod against False Negatives?}

To further study how our model behaves when training data includes different quantity of false negatives, we conduct experiments on synthetic datasets.
We construct five new training data by randomly removing triples with probability of 0.1, 0.3 and 0.5, simulating the situation of different FN rates.
We show the precision-recall curves of our method in comparison with \textsc{CasRel}~\cite{Wei2020ANC}, the best performing competitor, in Figure~\ref{fig:pr}.
1) The overall performance of \mymethod is superior to competitor models even when trained on a dataset with a 0.5 FN rate.
2) We show that the intervals of \mymethod between lines are smaller than \textsc{CasRel}, indicating that the performance decline under different FN rates of \mymethod is smaller.
3) The straight line before curves of our model means that there is no data point at the places where recall is very low.
This means that our model is insensitive with the decision boundary and thus more robust. 







\section{Conclusion}
In this paper, we revisit the negative data in relation extraction task.
We first show that the false negative rate is largely underestimated by previous researches.
We then systematically compare three commonly adopted paradigms and prove that our paradigm suffers less from the overwhelming negative labels.
Based on this advantage, we propose \textsc{ReRe}, a pipelined framework that first detect relations at sentence level and then extract entities for each specific relation and provide a multi-label PU learning loss to recover false negatives.
Empirical results show that \mymethod consistently outperforms the existing state-of-the-arts by a considerable gap, even when learned with large false negative rates.





\bibliographystyle{acl_natbib}
\bibliography{refer}



\end{document}
