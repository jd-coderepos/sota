\documentclass[copyright,creativecommons]{eptcs}
\providecommand{\event}{M. Bujorianu and M. Fisher (Eds.):\\ \hbox{\quad}Workshop on Formal Methods for Aerospace (FMA)}
\providecommand{\volume}{20}
\providecommand{\anno}{2010}
\providecommand{\firstpage}{80}
\providecommand{\eid}{9}
\usepackage{breakurl}             \usepackage{alltt}
\usepackage{graphicx}
\usepackage{url}

\usepackage{amsmath,xcolor,moreverb,url,float,amssymb,listings,subfigure}


\title{Polychronous Interpretation of Synoptic, a Domain Specific Modeling Language for Embedded Flight-Software}
\author{L. Besnard, T. Gautier, J. Ouy, J.-P. Talpin, J.-P. Bodeveix, A. Cortier, \\ M. Pantel, M. Strecker, G. Garcia, A. Rugina, J. Buisson, F. Dagnat
}
\def\titlerunning{Polychronous interpretation of Synoptic}
\def\authorrunning{SPaCIFY Project}


\newcommand{\setz}{\mathbb{Z}}
\newcommand{\ind}[1]{{\protect\raisebox{-0.6ex}{#1}}}
\newcommand{\ZZZ}{ {\setz}/\ind{}}
\newcommand{\signal}{{\sc Signal}}
\newcommand{\polychrony}{Polychrony}
\newcommand{\spacify}{SPaCIFY}
\def\ligne{\protect{\mbox{}\\\mbox{}\indent}}

\newcommand{\VV}{{\mathbb V}}
\newcommand{\TT}{{\mathbb T}}
\newcommand{\XX}{{\mathbb X}}
\newcommand{\B}{{\cal{B}}}
\newcommand{\T}{{\cal T}}
\newcommand{\V}{{\cal V}}
\newcommand{\D}{{\cal D}}
\newcommand{\X}{{\cal X}}
\newcommand{\C}{{\cal C}}
\newcommand{\I}{{\cal I}}
\newcommand{\F}{{\cal F}}
\newcommand{\W}{{\cal W}}
\newcommand{\G}{{\cal G}}
\renewcommand{\S}{{\cal{S}}}
\renewcommand{\P}{{\cal{P}}}
\newcommand{\tags}{\T}
\newcommand{\vars}{\V}
\newcommand{\op}[1]{{\sf #1}}
\newcommand{\true}{1}
\newcommand{\false}{0}
\newcommand{\kw}[1]{{{\,\op{#1}\,}}}
\newcommand{\Not}{\kw{not}}
\newcommand{\Pre}{\kw{\\hat{~}\block\,x\,AxAxx.\triggerx.\resetx.\triggerAAx.\triggerx.\resetxx.\triggerAxy\Event\,x\ra yyzfx\Data\,y\,f\,z\ra xyx\Data\,y\Pre v\ra xxx_0=vn>0yx_n=y_{n-1}x=y\,f\,zxyzf\Skip\If x\Then A\Else BAxBA;BABx.\triggerx.\reset\Init SS:\Do AATS\ra^{\On\,x} TxTS\rra^{\On\,x} TTSA\Par BPx=f(y,z)x,y,zx= y\Pre vxyxvyx= y\When zxyzx=y\Default zxyyzx=y\,f\,zfn^{th}xfn^{th}yzP\Par QPQP/xxPx\hat{~}xxx\Sync yxytS\rra Tc_i[t_i, t_{i+1}[A;BA\ra B\Sq{A}^{rt}=\sq{P}PAP\Sq{A}\sq{P}\Data\,y\Pre v\ra xxvry\prod_{i\leq n}P_iP_1\Par\dots P_n\bigvee_{i\leq n} e_ie_1\Default\dots e_n\Data\,y\,f\,z\ra xx(y,z)f\Event\,y\ra xyx?(y)y\hat{~}(y)y\In(A)\Out(A)AA\Skipx!xx=y\,f\,zyzxf\If x \Then A \Else BABxxEsn0n>0\Skip\Sq{A}^{s,m,g,E}=\sq{P}_{n,h,F}AsmgEPnhF\Use^g_E(x)xg\Def^g_E(x)xEx\in\vars(E)gs=0ts=nttsr0s0xEx=\Use^g_E(x)sn+1e(s\Pre 0)=n+1x!xeA;BAPn_Ag_AE_ABP\Par QBA{g\When x}PB{g\When\Not x}QP\Par Qg_A\Default g_Bx\in XE_AE_BFE\uplus FEFx\in\vars(E)\cup\vars(F)EFx!;x!xx!;\Skip;x!\st A\dt{A}A\Pred_{\st A}(S)=\{T\,|\,(T,x,S)\in R\}\Succ_{\st A}(S)=\{T\,|\,(S,x,T)\in R\}\Pred_{\dt A}(S)\Succ_{\dt A}(S)\st A\dt{A}SA\vec SSSAS\in\S(A)\st A\vec SA\Sq{\Auto x\,A}^{r\!t}Ast0rs'S_iAi=\idx{S_i}\idx{A}S_0iA_iix_{ij}S_iS_j\Sq{S_i}^{s}0\leq i<\idx{A}S_iS_jS_iA_iA_is_is=is'=ss_i\neq 0A_ig_iE_ig_iA_i0S_iS_jx_{ji}E_iF_jS_jA_iP_ih_iF_iA_iS_iQ_ix_{ij}S_iF_iF_i\Def_{h_i}(F_i)S_iS_jh_iijg_{ij}x_{ij}ss_i\neq 0iis'=s'_i\prod_{i< \idx{A}}s'=s'_is'=\bigvee_{i< \idx{A}}s'_i$
\hrulefill
\caption{Recursive interpretation of a mode automaton}\label{fig6}
\end{figure}








\section{Conclusion}
\label{Conclusion}

Synoptic has a formal semantics, defined in terms of the synchronous language \signal. On the one hand, this allows for neat integration of verification environments for ascertaining properties of the system under development. On the other hand, a formal semantics makes it possible to encode the meta-model in a proof assistant. In this sense, Synoptic will profit from the formal correctness proof and subsequent certification of a code generator that is  under way in the GeneAuto project.
Moreover, the formal model of \signal\ is the basis for the Eclipse-based polychronous modeling environment SME~\cite{Polychrony,SME}. SME is used to transform Synoptic diagrams and generate executable C code.


\bibliographystyle{eptcs} 

{\small
\begin{thebibliography}{1}

\bibitem{GeneAuto} A. Toom, T. Naks, M. Pantel, M. Gandriau and I. Wati: 
\newblock \emph{GeneAuto: An Automatic Code Generator for a safe subset of 
SimuLink/StateFlow}.
\newblock {\sl European Congress on Embedded Real Time Software (ERTS'08)},
Soci\'et\'e des Ing\'enieurs de l'Automobile, (2008).

\bibitem{Signal} P. Le Guernic, J.-P. Talpin and J.-C. Le Lann:
\newblock \emph{Polychrony for system design}.
\newblock {\sl Journal for Circuits, Systems and Computers},
Special Issue on Application Specific Hardware Design,
World Scientific, (2003). 

\bibitem{Polychrony} Polychrony and SME. 
\newblock Available at \url{http://www.irisa.fr/espresso/Polychrony}. 

\bibitem{SME} C. Brunette, J.-P. Talpin, A. Gamati\'e and T. Gautier:
\newblock \emph{A metamodel for the design of polychronous systems}.
\newblock {\sl The Journal of Logic and Algebraic Programming}, 78,
Elsevier, (2009). 

\end{thebibliography}
}

\end{document}
