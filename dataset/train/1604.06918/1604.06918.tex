\documentclass[11pt]{article}
\usepackage{times}
\usepackage[pdftex]{graphicx}
\usepackage{amssymb}
\usepackage{amsmath}
\usepackage{amsthm}
\usepackage{mdframed}
\usepackage{algorithm}
\usepackage{mathtools}
\usepackage{comment}
\usepackage{cleveref}
\usepackage{fullpage}
\newcommand*\samethanks[1][\value{footnote}]{\footnotemark[#1]}
\newcommand{\R}{\mathbb{R}}
\newcommand{\C}{\mathbb{C}}
\newcommand{\Z}{\mathbb{Z}}
\newcommand{\N}{\mathbb{N}}
\newcommand{\F}{\mathbb{F}}
\newcommand{\U}{\mathbb{U}}
\newcommand{\E}{\mathbb{E}}
\renewcommand{\cref}{\Cref}
\newcommand{\cA}{\mathcal A}
\newcommand{\cB}{\mathcal B}
\newcommand{\cC}{\mathcal C}
\newcommand{\cD}{\mathcal D}
\newcommand{\cE}{\mathcal E}
\newcommand{\cF}{\mathcal F}
\newcommand{\cG}{\mathcal G}
\newcommand{\cH}{\mathcal H}
\newcommand{\cI}{\mathcal I}
\newcommand{\cJ}{\mathcal J}
\newcommand{\cK}{\mathcal K}
\newcommand{\cL}{\mathcal L}
\newcommand{\cM}{\mathcal M}
\newcommand{\cN}{\mathcal N}
\newcommand{\cO}{\mathcal O}
\newcommand{\cP}{\mathcal P}
\newcommand{\cQ}{\mathcal Q}
\newcommand{\cR}{\mathcal R}
\newcommand{\cS}{\mathcal S}
\newcommand{\cT}{\mathcal T}
\newcommand{\cU}{\mathcal U}
\newcommand{\cV}{\mathcal V}
\newcommand{\cW}{\mathcal W}
\newcommand{\cX}{\mathcal X}
\newcommand{\cY}{\mathcal Y}
\newcommand{\cZ}{\mathcal Z}

\newcommand{\sA}{\mathsf A}
\newcommand{\sB}{\mathsf B}
\newcommand{\sC}{\mathsf C}
\newcommand{\sD}{\mathsf D}
\newcommand{\sE}{\mathsf E}
\newcommand{\sF}{\mathsf F}
\newcommand{\sG}{\mathsf G} 
\newcommand{\sH}{\mathsf H} 
\newcommand{\sI}{\mathsf I} 
\newcommand{\sJ}{\mathsf J} 
\newcommand{\sK}{\mathsf K}
\newcommand{\sL}{\mathsf L}
\newcommand{\sM}{\mathsf M}
\newcommand{\sN}{\mathsf N}
\newcommand{\sO}{\mathsf O}
\newcommand{\sP}{\mathsf P}
\newcommand{\sQ}{\mathsf Q}
\newcommand{\sR}{\mathsf R}
\newcommand{\sS}{\mathsf S}
\newcommand{\sT}{\mathsf T}
\newcommand{\sU}{\mathsf U}
\newcommand{\sV}{\mathsf V}
\newcommand{\sW}{\mathsf W}
\newcommand{\sX}{\mathsf X}
\newcommand{\sY}{\mathsf Y}
\newcommand{\sZ}{\mathsf Z}

\newcommand{\bbA}{\mathbb A}
\newcommand{\bbB}{\mathbb B}
\newcommand{\bbS}{\mathbb S}
\newcommand{\bbR}{\mathbb R}
\newcommand{\bbZ}{\mathbb Z}
\newcommand{\bbI}{\mathbb I}
\newcommand{\bbQ}{\mathbb Q}
\newcommand{\bbP}{\mathbb P}
\newcommand{\bbE}{\mathbb E}
\newcommand{\OPT}{\mathsf{OPT}}
\newcommand{\kk}{k_{\scriptscriptstyle \OPT}}
\newcommand{\kkk}{k'}
\newcommand{\kkkk}{k_{\scriptscriptstyle \OPT}'}

\DeclarePairedDelimiter\floor{\lfloor}{\rfloor}
\DeclarePairedDelimiter\ceil{\lceil}{\rceil}
\newtheorem{theorem}{Theorem}
\newtheorem{lemma}{Lemma}
\newtheorem{claim}{Claim}
\newtheorem{definition}{Definition}
\newtheorem{proposition}{Proposition}
 \newtheorem{cor}{Corollary}
 \newtheorem{obs}{Observation}
 \newtheorem{remark}{Remark}
 \newtheorem{conjecture}{Conjecture}
\begin{document}
\title{Graph Balancing with Two Edge Types }
\author{Deeparnab Chakrabarty\thanks{Microsoft Research, 9 Lavelle Road, Bangalore, India 560001. {\tt deeparnab, kirankumar.shiragur @gmail.com}} \and Kirankumar Shiragur\samethanks}\date{}
\maketitle
\begin{abstract}
	In the graph balancing problem the goal is to orient a weighted undirected graph to minimize the maximum weighted in-degree. This special case of makespan minimization is NP-hard to approximate to a factor better than  even when there are only two types of edge weights. In this note we describe a simple  approximation for the graph balancing problem with two-edge types, settling this very special case of makespan minimization.
\end{abstract}
\section{Introduction}
In the graph balancing problem, we are given an undirected graph  with weights  on edge . The graph could have parallel edges and self loops. The goal is to find an orientation of the edges so as to minimize the maximum weighted in-degree. 

This problem is a special case of makespan minimization, a classic problem in approximation algorithms, where
the input is a collection of jobs  and machines , and processing time  for  and .
The goal is to find an allocation of all jobs to machines so as to minimize . Note that graph balancing is a special case 
when nodes correspond to machines and each edge corresponds to a job which has processing time  on each of the endpoints and  on all other machines. In fact, this is a special case of the so-called restricted assignment machine scheduling problem where each job has an inherent processing time but can only be assigned to a subset of the machines.

In 1990, Lenstra, Shmoys, and Tardos~\cite{LST} described a now classic -approximation for this problem, and also proved it is NP hard to obtain an approximation factor better than . Closing this gap has challenged many researchers in the past three decades. The graph balancing problem was introduced in 2008 by  Ebenlendr, Krcal and Sgall~\cite{EKS} as an interesting special case of makespan minimization, and they gave an -approximation algorithm for this problem. \cite{EKS} also showed that even in the graph balancing problem, it is NP-hard to obtain an approximation factor better than . 
It is perhaps testament to the difficulty of the makespan minimziation problem that we do not know the ``true answer'' even for graph balancing. In 2011, Kolliopoulos and Moysoglou~\cite{KM} looked at an even more special case -- where there are only two types of edge weights! It turns out that even in this special, special case, one can't get better than -approximation; the reduction of Ebenlendr et al~\cite{EKS} uses only two distinct edge weights. Kolliopoulos and Moysoglou~\cite{KM} give a -factor approximation for this case. 

The purpose of this note is to show that a slight modification of techniques of ~\cite{KM}  in fact gives an optimal -approximation for graph balancing with two types of edge weights.


\section{Algorithm}
By scaling, we may assume that the edge weights are either  or  for some real number . 
In fact, Kolliopoulos and Moysoglu~\cite{KM} showed if  for some integer , then there ia a -approximation algorithm. 
Our main observation is with a bit of case analysis, we can generalize to arbitrary .

Fix an instance of the graph balancing problem and let  be its optimum value. The following lemma implies we may assume .
\begin{lemma}\label{lem:1}
	There is a -approximation algorithm if . \end{lemma}
\begin{proof}
	The algorithm in~\cite{LST} returns an orientation with weighted in-degree  if . \end{proof}
Therefore, we may assume  or  for some non-negative integers  and . 
\noindent
Let . Given some integers , we describe a flow networks  very similar to that defined by~\cite{KM} differing only in the capacities. For each arc in the network we define a flow lower bound and upper bound; a flow is feasible if the flow through all arcs lies within their range. The lower bound unless explicitly mentioned is , and the upper bound unless explicitly mentioned is .
Call an edge  small if , and big otherwise. \\

\noindent
{\bf Network Description  for integers .} We have a single source . For each edge  in the graph balancing instance, we have a node  in the network. There is an arc from  to each such  
with flow {\em lower bound} of  if  is a small edge, and a flow lower bound of  otherwise. We have two nodes,  and , for each vertex  in the graph balancing instance.
There is an arc of flow upper bound  from  to .
For every big edge , there is an arc from  to  iff  is adjacent to  with flow upper bound . 
For every small edge , there is an arc from  to  iff  is adjacent to  with flow upper bound . 
Therefore, every  has out-degree  unless it is a loop in which case it has out-degree . 
Finally, we have one sink node  and each node  has an arc to  with flow upper bound  . 
An illustration of the flow network is shown in Figure~\ref{fig:gb}.
\begin{figure}[h!]
	\begin{center}
		\includegraphics[scale=0.3]{gb-new.png}
	\end{center}
	\caption{\small Big circles denote big edges and small ones denote small edges. All the red arcs have weight , blue arcs have weight , and black arcs have weight }
	\label{fig:gb}
\end{figure}

\begin{lemma}\label{lem:21}
If  for some integer , then there is a feasible flow in the network . \end{lemma}
\begin{proof}
	For small edges , we send  unit of flow from  to  where  is the vertex towards which  is oriented. 
	For every big edge , we send  units of flow from  to  to , where  is the vertex towards which  is oriented. 
	Finally  sends all the flow it receives to .
	
	
	Since , in the optimal allocation each vertex has at most one big edge oriented towards it, and so no  receives flow from two separate 's. So the flow on  edge is at most . If a machine  receives  units of flow from , then it must be that in the graph balancing solution  has one big edge oriented towards it and since  is has at most  small edges are oriented towards . Thus, the total flow that  receives is at most . If machine  receives  units of flow from , then in the graph balancing solution there are at most  small edges oriented towards , implying the total flow from  to  is at most .
\end{proof}

\begin{lemma}\label{lem:22}
If  for some integer  and not equal to  for any non-negative integer , there is a feasible flow in the network . \end{lemma}
\begin{proof}
	Since  for any integer , we get  is not an integer for otherwise . Therefore, .
	We will use this in the proof.
	
	For small edges , we send  unit of flow from  to  where  is the vertex towards which  is oriented. 
	For every big edge , we send  units of flow from  to  to , where  is the vertex towards which  is oriented. 
	Finally  sends all the flow it receives to .
	
	Since , in the optimal allocation each vertex has at most one big edge oriented towards it, and so no  receives flow from two separate 's. So the flow on  edge is at most . If a machine  receives  units of flow from , then it must be that in the graph balancing solution  has one big edge oriented towards it.
	Therefore the number of small edges oriented towards  is at most  .
		 Thus, the total flow that  receives is at most . If machine  receives  units of flow from , then in the graph balancing solution there are at most  small edges oriented towards , implying the total flow from  to  is at most .
\end{proof}
\noindent
Since all capacities in  are integral, by integrality of flows,  if there is a feasible fractional flow in , then there must be a feasible integral flow in .

\begin{lemma}\label{lem:3}
	Given an integral feasible flow in , we can obtain a schedule with makespan .
\end{lemma}
\begin{proof}
For any small edge , we know that  receives one unit of flow from , and using the integrality of flows,  sends one unit of flow to exactly one machine . We orient  towards vertex .
    For any large edge , if the flow on any of the out-going arcs, say ,  is {\em strictly} greater than , then orient  towards .  This defines a partial orientation of all the edges of the graph balancing instance. \\
    
   \noindent
    {\bf Case 1:  is odd.} Let . Since a node  corresponding to a big edge  receives a flow of  units from , and there are at most two outgoing arcs from , 
    by integrality of flows and pigeonhole principle, it must be the case that one of the two arcs  carries at least  flow. In particular, all big edges are oriented in the partial orientation implying it is in fact a complete one.
   
   Let us now analyze the total weighted in-degree of any vertex .  Firstly note that if a vertex  only has small edges oriented towards it, then its weighted in-degree is . 
Secondly, any vertex  can have at most one big-edge oriented towards it. To see this, note that if  got two big edges  and  oriented towards it, then the flow on  and  are both . All this flow of  units must flow through the  arc, whose upper bound is  contradicting the feasibility of the flow. So each vertex  has at most one big-edge oriented towards it.  Furthermore, if this is the case, then the flow on the  arc is  implying the total flow  obtains from arcs of the form  for small edges  is at most . Thus the total weighted indegree of  is at most .\\ 
   	
\noindent
	{\bf Case 2:  is even.} Let . In this case, the current orientation may indeed be partial.
	Let  be the collection of nodes  which correspond to unoriented big edges, and  be the out-neighbors of  in the flow network. 
	Let  be one such edge which hasn't been oriented yet. It must be the case then that the flow of  units that  receives from  must flow out 
	on {\em exactly} two arcs  and , each carrying flow exactly  (in particular,  cannot be a loop).
	Furthermore, if  receives  units of flow from some , then since the capacity of  arc is  we have (a) it receives flow from at most one other vertex in ,
	and (b)	it cannot receive  units of flow from any other ; in particular, the vertex  doesn't have any big edge oriented towards it in the partial orientation.
	
	Now in the induced directed graph between  and , the above discussion implies that the out-degree of every vertex in  is exactly , and the in-degree of every vertex in  is at most .
	This implies there is a matching between  and  which completely matches ; one can easily check Hall's condition. If  is matched to , then we orient  towards , thereby extending the partial orientation to a complete one. Note that any node  has at most one edge corresponding to  oriented towards it.
	
	Let us now analyze the weighted in-degree of a vertex . Once again, as in Case 1, if  has only small edges oriented towards it, then its weighted indegree is at most .
	Also as in Case 1, any vertex  can have at most one big edge oriented towards it in the partial orientation. 
Furthermore, by point (b) above, if  receives a big edge in the partial orientation, it doesn't receive any other big edges in the extension.
    On the other hand if  doesn't receive a big edge in the partial orientation, then by the matching property of the extension, it receives at most one big edge in the extension.
    Finally, note that if  does receive exactly one big edge, then the total flow on the  arc must be . So, the total flow on arcs  for small edges  is at most . Therefore, the weighted in-degree of a vertex  which receives at lease one big edge is at most .
\end{proof}


\begin{lemma}\label{lem:41}
	Given , we can obtain a schedule with makespan .
\end{lemma}
\begin{proof}
By Lemma~\ref{lem:21} and Lemma~\ref{lem:3}, there is a feasible flow in the network  and we can obtain a schedule with makespan .
Now, since , we have  implying .
Also, we get , and so we get that the .
\end{proof}

\begin{lemma}\label{lem:42}
	Given  and is not equal to  for any non negative integer , we can obtain a schedule with makespan .
\end{lemma}
\begin{proof}
By Lemma~\ref{lem:22} and Lemma~\ref{lem:3}, there is a feasible flow in the network  and we can obtain a schedule with makespan  (Since )
\end{proof}
The above two lemmas along with Lemma~\ref{lem:1} implies the following theorem.
\begin{theorem}\label{thm:main}
	There is a -approximation for graph balancing with two types of edge weights.
\end{theorem}
\iffalse
\begin{proof}
	One first checks if  or  has a feasible flow; if not, by Lemma~\ref{lem:21} and~\ref{lem:22} we are guaranteed , and by Lemma~\ref{lem:1} we have a -approximation algorithm.
	By binary search, we find the smallest integer  for which  has a feasible flow. Set  if  is feasible, else . Note that  is not feasible, otherwise  would be feasible by sending an extra  flow of each big job along a path to a machine where this job is assigned by Lemma~\ref{lem:3} contradicting the minimality of . Get orientations as described by Lemma~\ref{lem:3} with respect to both  and , and output the orientation with  minimum makespan.
	If  -- by Lemma~\ref{lem:21} . So by Lemma~\ref{lem:41}, we get an orientation with all weighted in-degrees at most .  
	Otherwise  -- by Lemma~\ref{lem:22} . So by Lemma~\ref{lem:42}, we get an orientation with all weighted in-degrees at most .
\end{proof}
\fi

\noindent
{\bf Remark.} The above theorem can also show that the integrality gap of the so-called configuration LP is  too. To see this, note it is easy to prove that if the conf LP is feasible for some guess  or  of , 
then there is a feasible flow in the network  or , respectively.\smallskip

\noindent
{\bf Remark.}  (added 9th June, 2016.) After posting the version 1 of our paper, it was brought to our notice that our result is not new. Huang and Ott~\cite{HO15}, and independently, Page and Solis-Oba~\cite{PS16} also obtain the same result; the former paper in fact gives a 3/2-factor algorithm even in the case when the smaller jobs can go to any number of machines. It is easy to see that our algorithm easily modifies to this case (we never use the fact that small jobs go to at most two machines). The algorithm and analysis in this note is arguably simpler.



\bibliographystyle{alpha}
\bibliography{GBpapers}
\end{document}
