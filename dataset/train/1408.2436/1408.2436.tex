\documentclass[11pt]{patmorin}
\usepackage[utf8]{inputenc}
\usepackage{amsmath,amsfonts,amssymb,amsthm,graphicx,graphics}
\listfiles
\usepackage{graphicx}

\usepackage{hyperref}
\usepackage[dvipsnames]{xcolor}
\definecolor{linkblue}{named}{Blue}
\hypersetup{colorlinks=true, linkcolor=linkblue,  anchorcolor=linkblue,
citecolor=linkblue, filecolor=linkblue, menucolor=linkblue,
urlcolor=linkblue} 








\newcommand{\etal}{\emph{et al.}}

\newtheorem{theorem}{Theorem}[section]
\newtheorem{corollary}[theorem]{Corollary}
\newtheorem{lemma}[theorem]{Lemma}
\newtheorem{proposition}[theorem]{Proposition}
\newtheorem{observation}[theorem]{Observation}
\newtheorem{problem}[theorem]{Problem}
\newtheorem{definition}[theorem]{Definition}
\newtheorem{conjecture}[theorem]{Conjecture}
\newtheorem{question}[theorem]{Question}

\DeclareMathOperator{\rank}{rank}

\newcommand{\R}{\mathbb{R}}




\newcommand{\red}[1]{{\color{red} #1}}
  
\newcommand{\ch}[1]{\ensuremath{\textsc{ch}(#1)}}

\title{\MakeUppercase{Compatible Connectivity-Augmentation \newline of Planar Disconnected Graphs}}

\author{Greg Aloupis,\thanks{Department of Computer Science, Tufts University, 
                             \email{aloupis.greg@gmail.com}}\,\,
       Luis Barba,\thanks{School of Computer Science, Carleton University
                          and Département d'Informatique, 
                          Université Libre de Bruxelles,
                          \email{lbarbafl@ulb.ac.be}}\,\,
       Paz Carmi,\thanks{Department of Computer Science,
                         Ben-Gurion University of the Negev,
                         \email{carmip@cs.bgu.ac.il}}\,\,
       Vida Dujmović,\thanks{School of Computer Science 
                             and Electrical Engineering,
                             University of Ottawa,
                             \email{vida.dujmovic@uottawa.ca}}\,\,
       Fabrizio Frati,\thanks{School of Information Technologies,
                              The University of Sydney,
                              \email{fabrizio.frati@sydney.edu.au}}\,\,
       and Pat Morin\thanks{School of Computer Science, Carleton University,
                            \email{morin@scs.carleton.ca}}}




\begin{document}

\begin{titlepage}

\maketitle
\begin{abstract}
Motivated by applications to graph morphing, we consider the following
\emph{compatible connectivity-augmentation problem}: We are given
a labelled -vertex planar graph, , that has 
connected components, and  isomorphic planar straight-line drawings,
, of . We wish to augment 
by adding  vertices and edges to make it connected in such a way that
these vertices and edges can be added to  as points and
straight-line segments, respectively, to obtain  planar straight-line
drawings isomorphic to the augmentation of .  We show
that adding  edges and vertices to 
is always sufficient and sometimes necessary to achieve this goal.
The upper bound holds for all  and  and is
achievable by an algorithm whose running time is  for
 and whose running time is  for general values of .
The lower bound holds for all  and .
\end{abstract}

\end{titlepage}



\section{Introduction}


Consider the following problem, which will be more carefully formalized
below.  We are given several different labelled planar straight-line drawings
(or simply drawings) of the same disconnected labelled graph, .
We wish to make  connected by adding vertices and edges in
such a way that these vertices and edges can also be added to the 
drawings of  while preserving planarity.  
The objective is to do this while minimizing the number
of edges and vertices added.  As the example in Figure~\ref{fig:bad-example} shows, it is not always possible to just add edges to ; sometimes additional vertices are necessary.

\begin{figure}[hb]
  \centering{
    \begin{tabular}{ccc}
      \includegraphics{img/bad-example-3} &
      \includegraphics{img/bad-example-1} & 
      \includegraphics{img/bad-example-2} \\
       &  &  
    \end{tabular}
  }
  \caption{Two drawings,  and , of the same graph, , where making  connected requires the addition both of edges and vertices. In this case,  is made connected by adding the hollow vertex and two dashed edges.}
  \label{fig:bad-example}
\end{figure}

The motivation for this work comes from the
problem of morphing planar graphs, which has many applications
\cite{erten.kobourov.ea:intersection-free,friedrich.eades:graph,gotsman.surazhsky:guaranteed,surazhsky.gotsman:controllable,surazhsky.gotsman:intrinsic}
including computer animation.  Imagine an animator who wishes to animate a
scene in which a character's expression goes from neutral, to surprised,
to happy (see Figure~\ref{fig:faces}). The animator can draw these
three faces, but does not want to hand-draw the 30--60 frames required
to animate the change of expression.  The strokes used to draw the
character's features can be converted into paths and these can be merged
into components corresponding to the character's eyes, nose, mouth and
so on.  A correspondence between the same elements in different pictures
is also given.\footnote{In many cases, the correspondence is a byproduct
of the creation process. For example, in Figure~\ref{fig:faces}, the
second two faces were obtained by copying and then editing the first one.}
Thus, the input is three isomorphic drawings of the same planar graph.

\begin{figure}
  \centering{ 
  \begin{tabular}{c@{\hspace{.3cm}}c@{\hspace{.3cm}}c@{\hspace{.3cm}}c}
    \includegraphics{img/face-zoom} &
    \includegraphics{img/faces-1} &
    \includegraphics{img/faces-2} &
    \includegraphics{img/faces-3} 
  \end{tabular}}
  \caption{Computer-assisted animation frequently involves morphing between
   a sequence of drawings of the same planar graph.  Zooming in on a section
   of the image reveals that the artist's strokes are approximated by polygonal
   paths}
  \label{fig:faces}
\end{figure}

In this setting, animating the face becomes a problem of
\emph{morphing} (i.e., continuously deforming) one drawing of
a planar graph into another drawing of the same planar graph
while maintaining planarity of the drawing throughout the
deformation. This morphing problem has been studied since 1944,
when Cairns \cite{cairns:deformations} showed such a transformation
always exists.  Since then, a sequence of results has shown that morphs
can be done efficiently, so that the motion can be described concisely
\cite{alamdari.angelini.ea:morphing,angelini.dalozzo.ea:morphing,grunbaum.shephard:geometry,thomassen:deformations}.
The most recent such result \cite{angelini.dalozzo.ea:morphing} shows
that any planar drawing of an -vertex \emph{connected} planar graph
can be morphed into any isomorphic drawing using a sequence of 
\emph{linear morphs}, in which vertices move along linear trajectories
at constant speed.

The morphing algorithms discussed above require that the input graph,
, be connected. In many applications of morphing (for example
in Figure~\ref{fig:faces}) the input graph is not connected. Before these
morphing algorithms can be used,  must be augmented into a
connected graph, , but this augmentation must be compatible
with the drawings of .  At the same time, the complexity
of the morph produced by a morphing algorithm depends on the number of
vertices of .  Therefore, we want to find an augmentation
with the fewest number of vertices.  This motivates the theoretical
question studied in the current paper.





\subsection{Formal Problem Statement and Main Result}

A \emph{drawing} of a graph  is a one-to-one
function .  A drawing is \emph{planar} if (a)~for
every pair of edges  and  in , the open line segment with
endpoints  and  is disjoint from the open line
segment with endpoints  and  and (b)~for every edge
 and every vertex ,  is not contained in the open line
segment with endpoints  and .  Two planar drawings,
 and , of  are \emph{isomorphic} if
there exists a continuous family of planar drawings  of  such that 
and .\footnote{By Cairn's result, this is equivalent
to saying that the two drawings of  have the same rotation schemes,
the same cycle-vertex containment relationship, and the same outer face.}

We call a graph,  a \emph{geometric planar graph} if it is
the image of some planar drawing of a graph .  That is, ,
, and  is a planar
drawing of .  When clear from context, we will sometimes
treat a geometric planar graph interchangeably with the set of points
and line segments defined by its vertices and edges, respectively.

We will avoid repeatedly referencing drawing
functions like .  Instead, we will talk about a graph 
and  isomorphic drawings  of .  This
means that each  is the geometric graph given by the drawing of
 with some function  and that 
are pairwise isomorphic.  When necessary, we may talk about the vertex 
in  where  is actually a vertex of ; this should
be taken to mean the vertex  in .  

We are now ready to state the main problem studied in this paper.
Given  planar isomorphic drawings 
of , a \emph{compatible augmentation}, , of
 is a supergraph of  such that (1) 
is connected, and (2) there exist planar isomorphic drawings,
, of  such that  for every
.  We prove the following result:

\noindent\textbf{Main Result:} {\itshape If  is a graph
with  vertices and  connected components and
, , are isomorphic planar drawings of , then there always exists a compatible augmentation of 
whose size is .

Furthermore, this bound is tight; for every  and , there exists a graph 
with  components and  isomorphic planar drawings for which any compatible
augmentation has size .}

These results show that the (worst-case) cost of an augmentation is very
sensitive to the number, , of drawings, but only up to a point.
For a fixed value of , our bounds range from  (when
) to  (for ).  On the other hand, for the
common case where , our bounds vary from  (when ) up to  (when ).  Neither  nor 
causes the complexity of the augmentation to blow up beyond .

\subsection{Related Work}

To the best of our knowledge, there is little work on
compatible connectivity-augmentation of planar graphs, though
there is work on isomorphic triangulations of polygons.  Refer to
Figure~\ref{fig:compatible-triangs}.  In this setting, the graph
 is a cycle and one has two non-crossing drawings, 
and , of . The goal is to augment  (and the
two drawings  and ) so that  becomes a near-triangulation,
and  and  become (geometric) triangulations of the interiors
of the polygons whose boundaries are  and .  Aronov \etal\
\cite{aronov.seidel.ea:on} showed that this can always
be accomplished with the addition of  vertices and that
 vertices are sometimes necessary.  Kranakis and Urrutia
\cite{kranakis.urrutia:isomorphic} showed that this result can be made
sensitive to the number of reflex vertices of  and , so that the
number of triangles required is  where  and  are the
number of reflex vertices of  and , respectively.

\begin{figure}
  \centering{
    \includegraphics{img/compatible-triangs}
  }
  \caption{Compatible triangulations of two 4-gons  and .}
  \label{fig:compatible-triangs}
\end{figure}

Babikov \etal\ \cite{babikov.souvaine.ea:constructing} showed that the result of Aronov \etal\ can be extended to polygons with holes. This work is the most closely related to ours because it encounters (the special case  of) our problem as a subproblem. In their setting, the graph  is a collection of  cycles, the drawings  and  are such that one cycle, , of  contains all the others in its interior and no other pair of cycles is nested in  or . In the first stage of their algorithm, they build a connected supergraph  of , but their supergraph has size  in the worst case.  The main theorem in the current paper shows that this step of their algorithm could be done with a graph  having only  edges (but completing this graph to a triangulation may still requires  edges in the worst case).

Finally, several papers have dealt with the problem
of increasing the connectivity of a (single) geometric
planar graph while adding few vertices and edges.  Abellenas
\etal~\cite{abellanas.olaverri.ea:augmenting} consider the problem of
adding edges to a planar drawing in order to make it 2-edge connected
and showed that  edge are sometimes necessary
and  edges are always sufficient.  T\'oth \cite{toth:connectivity}
later obtained the tight upper-bound of  for
the same problem.  Rutter and Wolff \cite{rutter.wolff:augmenting} show
that finding the minimum number of edges required to achieve 2-edge
connectivity is NP-hard.


\subsection{Outline}

To guide the reader, we give a rough sketch of our upper bound proof,
which is illustrated in Figure~\ref{figure:example}.  We will assume,
for the sake of simplicity, that every component has at least one vertex
incident to the outer face.

For each component, , of  we select
a distinguished \emph{corner}, , of . (A corner is the
space between two consecutive edges incident to some vertex of the
outer face). The corner  is called the \emph{attachment corner}
for component .  Notice that, since 
are isomorphic,  appears as a corner in each of .
The augmentation that we ultimately create will consist of a path that
connects to each component  at its attachment corner .

\begin{figure}
  \begin{center}
   \begin{tabular}{c}
     \includegraphics{img/example-1} \2ex]
      \\
     \includegraphics{img/example-3} \2ex]
      \\
     \includegraphics{img/example-4} 
   \end{tabular}
  \end{center}
  \caption{The algorithm for making a compatible augmentation of  works by defining corners , taking a spanning path on each augmented drawing  that visits all corners, and using this information to compute a permutation of  that can be drawn efficiently in each .}
  \label{figure:example}
\end{figure}

Next, for each drawing, , we add  edges to make  into a
connected graph, ; these edges are not, in general, edges that
take part in the final augmentation of  (in fact the
edges of  not in  might be different from the edges of
 not in , if ).  We then traverse the boundary
of the outer face of  to obtain a polygonal path, ,
of length  that comes close to every corner of .  The path
 is then used to define an integer distance 
for any two attachment corners  and . This distance includes
information about the number of edges of  between 
and  as well the sizes of the some of the components visited while
walking from  to  along .

Next, we take a leap into  dimensions by using the distance
functions  to produce a -dimensional point set
 that lives in a hypercube of side-length .
This mapping has the property that, by adding a path of length  to , the attachment corners  and 
can be joined in each of  while preserving planarity.

Now, since the point set  is in , has  points, and lives in a
hypercube of side-length , a classic argument about the geometric
Travelling Salesman Problem \cite{few:shortest,moran:on} implies that it has a
spanning path whose length, measured in the  norm, is
.\footnote{The original argument was for the standard
 norm and has an extra  factor.  This factor disappears
in the  norm.  See Appendix~\ref{app:uniform-norm} for details.}
This implies that  can be made connected with a collection
of  paths, whose endpoints are the corners , having
total size , and that each of these paths can be
drawn in a planar fashion in each of .

At this point, all that remains is to show that each of these  paths
be drawn in each of  without crossing
each other.  This part of the proof involves carefully winding these
paths around the components in  using paths close
to the paths  defined above. This part
of the proof resembles the first part of the proof of Babikov \etal\
\cite{babikov.souvaine.ea:constructing}, but is complicated by the fact
that we have to be quite careful that the number of edges in these paths
remains in . 



The remainder of the paper is organized as follows: In
Section~\ref{section:Trivial components} we start by solving
the special case in which the graph  has no edges. This
special case is already non-trivial and introduces some of the
main ideas used in solving the full problem, which is tackled in
Section~\ref{section:General}. Section~\ref{section:Lower bound}
presents a lower bound construction that matches our upper bound.



\section{Upper bounds for trivial components}\label{section:Trivial components}
As a warmup, we consider a (trivial) graph containing  vertices and no edges.
Before constructing a compatible augmentation, we provide a subroutine
that constructs a ``short'' planar spanning path of a given ordered set of points.

\subsection{Spanning paths of point sets}
Let  be a set of  points in the plane with distinct -coordinates. 
Given a point , let  denote the number of points of  that lie to the left of (having smaller -coordinate than) .

Given an arbitrary order  of the points of , we want to construct a path  that connects them in this order and such that:  

This paper uses the  operator in several different ways, depending on the type of its argument.  For a real number, ,  is the absolute value of .  For a walk, ,  denotes the number of edges traversed by . For a (weakly-)simple polygon,  whose vertices---as encountered during a counterclockwise traversal---are , , denotes the number of edges of .

Consider a horizontal line  below  and let  be the closed halfspace supported by  that contains .  We present an algorithm that constructs  iteratively; during the th iteration of the algorithm, the path is extended with  vertices to include .  For each , after the th iteration of the algorithm, we maintain the invariant that  does not intersect , and we also maintain the \emph{escape invariant} which is defined as follows:
For each , there is a closed cone  with apex  such that (1)  lies above  and has the same -coordinate as  ( if ), (2)  contains  and no other point of , (3)  contains the ray originating at  in the direction of the negative -axis, (4)  does not intersect , and (5)  and  are disjoint inside , for every  with .

Initialize  as a path that consists of the single vertex . In order to establish the escape invariant, we define ; also, for each , we define  as an arbitrary translation up of ; further, for each , we let  be a cone with apex on  sufficiently narrow so that these cones do not intersect inside ; see Figure~\ref{fig:Escape Invariant}.

\begin{figure}[tb]
\centering 
\includegraphics{img/EscapeInvariant.pdf}
\caption{The halfplane  and the cones  with apexes at .}
\label{fig:Escape Invariant}
\end{figure}



Now assume that  is a path connecting  with , for some . We extend  by appending a
path that connects  with .

First, we translate  down until its apex  coincides with . Let  be the closure of the set obtained from  by removing , for every ; see Figure~\ref{fig:Dented Halfspace} (right). That is,  is a halfspace with dents made by the removal of  cones.
Observe that, for every pair of apexes  and , with , the boundary of 
contains a path from  to  with  edges.
Because  does not intersect  and by the escape invariant, the boundary of  intersects  only at . Moreover, again by the escape invariant, for each  with ,  lies outside of~ except for  that lies on its boundary. Because both  and  lie on the boundary of , which does not intersect  other than at , we can connect  with  via a path contained in the boundary of~ with length . In this way, we extend  to a planar path that contains .

\begin{figure}[tb]
\centering
\includegraphics[width=.98\textwidth]{img/DentedHalfspace.pdf}
\caption{The boundary of  is in the path from  to .}
\label{fig:Dented Halfspace}
\end{figure}

After connecting  with , for each , either  is disjoint from , or it shares some portion of its boundary with . However, the interior of  does not intersect .
To preserve the escape invariant, for each , we translate  and  downwards by a sufficiently small amount, , and we scale  horizontally down, while keeping its apex at . To conclude, each translated or scaled cone is contained in the previous one,  lies above , for each , and  coincides with . Therefore, by choosing  sufficiently small, we maintain the escape invariant and obtain the following result.

\begin{lemma}\label{lemma:Compatible augmentation for trivial components}
Given an order  of the vertices of , there exists a planar path  that connects every point of  in the given order such that the number of vertices of  between  and  is , for each .
\end{lemma}
\begin{proof}
Recall that in each iteration, the algorithm computes a path connecting  with  that does not cross the portion of the path already constructed. Because this invariant is maintained throughout, the resulting path is planar.

Since the path that connects  with  follows the boundary of  and since this boundary has length  between  and , the path that connects  with  has length . Consequently,  the total length of  is given by .
\end{proof}



\subsection{Compatible drawings of point sets}

Recall that in this section  is a graph with  trivial components.
Let  be  isomorphic drawings of , i.e.,  is a sequence of  points in the plane.
Assume without loss of generality that no two points of  share the same -coordinate.
Given a vertex  of , let  denote the number of points of  having smaller -coordinate than , and let  be a point in the integer grid  in .  Let  and let  be the shortest Hamiltonian path of  when distance is measured using the  norm, so that

It is known that the length of  is  (see
Corollary~\ref{cor:tsp} in Appendix~\ref{app:uniform-norm}).  Note that
the order of the points of  induces an order on the vertices of
 and hence, an order on the vertices of each .

\begin{theorem}\label{theorem:points}
For each , we can construct a path  of length  that connects every point of  so that  is planar. Moreover, for any distinct ,  and  are~isomorphic.
\end{theorem}
\begin{proof}
By relabelling, let  denote the order of the vertices of  induced by the Hamiltonian path, , described above.  The graph , which is an augmentation of , is a path that visits the vertices  in this order. Letting  denote the  distance between  and , the path  includes an additional  vertices between  and .  It follows that the number of vertices
in  is proportional to the length of , which is .

For each , we use Lemma~\ref{lemma:Compatible augmentation for trivial components} to draw  as a planar path, ,
that connects the vertices  in this order in the drawing .
Since , the 
vertices in  between  and  are enough to
draw the  vertices in 
between  and .
Since the vertices of each  are connected in the same order,
 is isomorphic to  for each .
\end{proof}


\section{The general problem}\label{section:General}
In this section, we extend the result presented in
Section~\ref{section:Trivial components} to graphs with
non-trivial components.  We follow the same general scheme used in
Section~\ref{section:Trivial components} for the case of trivial
(isolated vertex) components:  We define  different
orderings of the components of  and use these orderings (and
the sizes of these components) to define an -point set, , in . A
short path that visits all points in  is then translated back into
a short path, , that visits all components of . The path
 is then
added, as a polygonal path, , to each drawing, , of .

Unlike the case in which
components are isolated vertices, there is no natural ordering of the
components of , so we must define one. Also, the drawing
of path  is considerably more complicated.  In Section~\ref{section:Trivial components},  is drawn incrementally, and always passes above components that are not yet included in  and below components that are already included in .  In this section, we redefine ``above'' and ``below''. The number of edges required to go above or below a component depends on its size and structure.

\subsection{Preliminaries}\label{section:Preliminaries} 
Let  be a connected geometric planar graph. Let  be the sequence of vertices of  visited by a counterclockwise
Eulerian tour along the boundary of the outer face of . Note that
 may be equal to  for some .  A vertex 
in this sequence is called a \emph{corner} of .  We consider the boundary of , denoted by , to be the
boundary of the weakly-simple polygon  whose
vertex set is the set of corners of .\footnote{More formally,  is the boundary of the unbounded component of , when we treat  as the union of all its edges (line segments) and vertices (points).}

Let . For each corner  of , let  be the half-line starting at  that bisects the angle between the edges  and  in the outer face of . Let  be the point at distance  from  along . We call  the \emph{-copy} of . Let  be the piecewise-linear cycle defined by the sequence . We call  the \emph{-fattening} of .
An -fattening  is \emph{simple} if   is a simple polygon that contains~.
Note that  is simple, provided that  is sufficiently small. In this paper, we consider only simple -fattenings; see Figure~\ref{fig:Blowing}. Note that the number of edges between two corners of  along the boundary of  is the same as the number of edges between their -copies along~.

\subsection{Connected augmentations}\label{section: connected augmentations}
Let  be a geometric planar graph with  connected components such that each component is adjacent to the outer face.
Two vertices are {\em visible} if the open segment joining them does not intersect .
Let  be a smallest set of edges of the visibility graph of  that need to be added to  to make it connected.
As there are always two components containing mutually visible vertices, we can connect them and repeat recursively.  Thus,  has  edges. (Loosely, we can think of  as a spanning tree of 's components.) Let .  We say that  is a \emph{connected augmentation} of ; see Figure~\ref{fig:Blowing}.

\begin{figure}[tb]
\centering
\includegraphics{img/Blowing.pdf}
\caption{The graph  (left); a connected augmentation, , of  (middle); and the -fattening,  (right).}
\label{fig:Blowing}
\end{figure}

Let  be the components of .
For each , let  be an arbitrary corner of  (note that  is adjacent to the outer face).
We call  the \emph{attachment corner} of .

Let  be the path on the corners of  (hence  is also a walk on the vertices of ) obtained by splitting  at the corner . That is,  is a path that visits every corner of  exactly once except for , which is visited twice.
Given two corners  and  in , let  denote
the unique path in  that connects  with . Let  be
the set of attachment corners of  visited by . For two attachment corners  and , define

which we call the \emph{cost} of going from  to .  The definition of  is designed to capture the fact that, if  and  occur consecutively on the path  we construct, then the portion of  between  and  will have length at least  since it follows , and it may also take a detour around every attachment corner .  If it takes this detour at some , then it does so by walking around  which requires an additional  edges.

\begin{lemma}\label{lemma:Contained in integer grid}
    If  is an attachment corner of , then . Moreover, if , with , is another attachment corner of , then
  .
\end{lemma}
\begin{proof}
Recall that  is a graph with  vertices, so 
is a weakly-simple polygon with at most  distinct vertices, so
.  Furthermore, , so .  Similarly,
.
Therefore, , which proves the first part of the lemma.

To prove the second part of the lemma, first observe that  by definition. To prove the second equality, denote
the relevant attachment corners of  by
. Then , , and
, so

\end{proof}

\subsection{Spanning paths for connected augmentations}\label{section:Spanning paths for connected augmentations}
Let  be an arbitrary order of the attachment corners of  (we can get the incremental indexing by relabeling the components). 
Given a path  that passes through all attachment corners of , we say that  \emph{lies to the right of}  if (1)  is the only vertex of  that belongs to , and (2) if  for some , then  and  appear as consecutive vertices when sorting---in the graph ---the neighbors of  in clockwise order around  (see Figure~\ref{figure:right-of}).


\begin{figure}
\centering
  \includegraphics{img/right-of}
  \caption{The component  is to the right of the path .}
   \label{figure:right-of}
\end{figure}

We now show how to construct a path  that connects the attachment corners of  in the given order, i.e., if , then  is visited before  by . We want to construct  so that each component  of  lies to the right of~ and so that  is a planar geometric graph.
Moreover, we want the subpath of  between  and  to have  vertices. 
We initialize  with the trivial path that contains only , and then extend  iteratively, so that each new corner  is included in .
Recall that  for any given ,  denotes the -fattening of  (see Section~\ref{section:Preliminaries}).
Let  be a small constant to be specified later.
Initially, let  and let . Let  be a constant sufficiently small so that  for any distinct .
Throughout,  remains constant while  and  are redefined at each iteration. However, as an invariant we maintain .

For each , let  be the -copy of  .
Split  at , i.e.,  is a path with both endpoints equal to .
By choosing  sufficiently small, we guarantee that  is simple, i.e.,  is isomorphic to .
We say that two points in the plane are \emph{-visible} if the open segment joining them does not intersect .
Let . For each  such that  is not an interior point of , consider the set of points  that are at distance at most  from .
Let  be the convex hull of , i.e.,  is a ``cone'' with apex at ; see Figure~\ref{fig:Neighborhood}. (We deliberately misuse the word ``cone'' here because the ``cones''  in this section play the same roles as the cones  in Section~\ref{section:Trivial components}.)


While constructing , we also maintain the \emph{escape invariant} which is defined as follows. Assume that  so far connects , for some . Then: (1)~ intersects neither  nor its unbounded face; (2)~for each ,  intersects neither the simple polygon bounded by  nor the cone ; (3)~, for any distinct ; and (4)~ is -visible from .

In particular, Conditions~(2) and (4) of the escape invariant imply that, for
each , every point in  (including ) is
-visible from .  The escape invariant holds when ,
provided that  is sufficiently small.

\begin{figure}[tb]
\centering
\includegraphics{img/Neighborhood.pdf}
\caption{The -fattening of  and the ``cones'' .}
\label{fig:Neighborhood}
\end{figure}

Assume that we have constructed a path  that connects  with , for some , and that the escape invariant holds.  To extend , we create a new path that connects  with  without crossing  while maintaining the escape invariant.  Recall that we consider  to be a path with both endpoints on~.

The first part of the path connecting  with  consists of a path connecting  with . If , or if  and  together with the edge  leaves  to its right, then connect  with  via a straight-line segment; since  is -visible from , this segment does not cross . Otherwise, connect  with  via a straight-line segment and traverse  clockwise before moving to  on .  By the escape invariant, no crossing occur in this drawing.  In this way, we guarantee that  lies to the right of the constructed path; see Figure~\ref{fig: Component to the right} for an illustration. Because  and since , the escape invariant is preserved.

\begin{figure}[tb]
\centering
\includegraphics[width=1\textwidth]{img/ComponentToTheRight.pdf}
\caption{When extending  from  to  we have to take care to
    keep  to the right of .}
\label{fig: Component to the right}
\end{figure}

The path from  to  continues with a path from  to , which follows the unique path in  from  to . However, whenever we reach an endpoint of  for some , we take a \emph{detour} to the other endpoint of  while avoiding its interior so that the points in the interior of  remain -visible from ; see Figure~\ref{fig:Skip Component}. Formally, we walk from the reached endpoint of  to  along the boundary of . Then, we traverse the path  before moving to the other endpoint of  from the endpoint of . Note that  does not intersect the interior of the simple polygon bounded by  nor the interior of . Moreover,  remains inside the simple polygon bounded by .


\begin{figure}[tb]
\centering
\includegraphics[width=.98\textwidth]{img/SkipComponent.pdf}
\caption{The ``detour'' taken to avoid crossing the cone  (left, middle); and the narrowing of the cone  as well as the redefinition of the - and -fattenings of  and , respectively.}
\label{fig:Skip Component}
\end{figure}

Once we go around , we are back on  on the other endpoint of . In this way, we continue going towards~ along  until reaching an endpoint of .
Once we reach an endpoint of , we move directly from this endpoint to .

Because  is isomorphic to , the constructed path between  and  has length at most  plus the length of the boundaries of the components for which the path detoured. Because each component we walked around has its attachment corner on the path , and thus in , the length of the constructed path between  and  is


After reaching , we increase  by a factor of two. Similarly, we decrease the value of  by a factor of two. That is, after reaching ,  while  and hence, we guarantee that .
Also,  is still simple, provided that  is initially chosen to be sufficiently small.
Finally, we reduce  by a factor of two and update  and  accordingly, for each ; see Figure~\ref{fig:Skip Component} (c).

Recall that for each ,  intersected neither the interior of  nor the interior of the polygon bounded by . Moreover,  remained within .
Therefore, after increasing (\emph{resp.} reducing)  (\emph{resp.} ), we preserve the escape invariant for the next iteration of the algorithm.
We iterate until all attachment corners of  are visited by .

\begin{lemma}\label{lemma:Path for connected augmentations}
Given an arbitrary order  of the attachment corners of , there is a path  connecting all attachment corners of  in the given order such that  is planar, every component  of  lies to the right of  when oriented from ~to~, and the subpath of  between  and  has  vertices, for each .
\end{lemma}
\begin{proof}
By construction, the attachment corners are visited by  in the given order; also, the subpath, , of  between  and  has  vertices. For each component ,  is the only vertex of  visited by . Moreover, the construction guarantees that  lies to the right of  when oriented from  to .

To prove that  is planar, recall that in each round we extend  by constructing  a path  that connects  with . We claim that at this point, no edge of  crosses the portion of  constructed so far.
Indeed, because the value of  (\emph{resp.} ) increases (\emph{resp.} decreases) in each round, the edges of  that lie on the boundaries of some  or on  cannot cross  by the escape invariant.
Moreover, this invariant states that for each ,  does not intersect~.
Because each cone  is narrowed in each round, the edges of  that lie on the boundary of this cone cannot cross . Finally, because , the edges of  that lie on  do not cross . Therefore, we conclude that by concatenating  and , we obtain a planar path.
\end{proof}

Figure~\ref{figure:big-example} illustrates the algorithm of Lemma~\ref{lemma:Path for connected augmentations} on a small example.  In this example, the path from  to  passes by , so  detours around  in order to preserve the escape invariant at .  After  attaches to  and , it winds around components  and , respectively, in order to ensure that these components attach to the right of .

\begin{figure}
   \centering{\includegraphics{img/big-example}}
   \caption{An example of the algorithm for generating a spanning path that connects .}
   \label{figure:big-example}
\end{figure}

\subsection{Compatible drawings of planar graphs}
Let  be a planar graph with  vertices and  connected
components.  Let  be  planar isomorphic drawings
of .  For now, we will assume that, in these drawings,
every component of  has at least one vertex incident to
the outer face.  
We show how to construct a compatible augmentation of 
of size .

Let  be the connected components of .  Because  are isomorphic, we can select one attachment corner from each component in the drawing , and this attachment corner also appears in each of . Thus, for each , we choose an attachment corner  of  such that  is incident to the outer face of .

For each , let  be a connected augmentation of , as defined in Section~\ref{section: connected augmentations}. For each  and , let . For each , let  be a point corresponding to the component  such that . Let  denote the resulting set of points. Lemma~\ref{lemma:Contained in integer grid} implies that  is contained in an integer grid of side length .

Let  be the shortest Hamiltonian path of  under the  norm. As before, because  is contained in the -dimensional integer grid of side-length  and , the maximum () length of  is . Note that the order of the points in  induces an order of the components of  and hence an order of the attachment corners of each .

\begin{theorem}\label{theorem:main}
For each , we can construct a path  of length  that connects every component of  such that  is planar. Moreover, for each ,  is isomorphic to .
\end{theorem}
\begin{proof}
By relabelling, let  denote the order of the attachment corners of  induced by .  Letting  denote the  distance between  and , we denote by  a path that passes through (the vertices corresponding to corners)  in this order, and that includes an additional  vertices between  and .  Thus, the number of vertices in  is proportional to the length of , which is .

For each , we use Lemma~\ref{lemma:Path for connected augmentations}
to draw  as a planar path, , that connects  in this order. By construction, 

so the  vertices in 
between  and  are enough to draw the 
 vertices in  between  and
, since .






To conclude, each  visits each component only at its attachment corner, the attachment corners of each  are connected in the same order, and  leaves every component to the right when oriented from  to . Therefore,  is isomorphic to  for each .
\end{proof}

\subsection{Handling interior components}

In the preceding section, we assumed that the drawings  of  were such that every component was incident to the outer face.
To see that the assumption is not necessary, observe that we can first use the preceding algorithm to connect all the components that do appear on the outer face using a polygonal path that is contained on the outer face.  The number of vertices used in this path is , where  is the number of vertices on the outer face.

Next, for each interior face, , that has multiple components
 on its boundary, we can use (a small modification of)
the preceding algorithm to connect  and the outer
boundary of  using a path that is contained in .  This path
has length .  We then repeat this step on each face.
The result is a connected augmentation of  whose total size
is  where  is the total size of all faces.

\subsection{An algorithm}


We remark that Theorem~\ref{theorem:main} yields an efficient algorithm for constructing the augmentation , and even the drawings .  The main steps involved are: 

\begin{enumerate}
  \item Finding connected planar supergraphs 
  of the drawings .  For each planar graph , this
  can easily be done in  time using, for example, a plane
  sweep algorithm that maintains the invariant that all components with
  a vertex to the left of the sweep-line are already joined by edges.
  Thus, this step takes  time.

  \item Constructing the point set  and finding the path
  .  Constructing  takes  time, while a path  of length
   can be obtained from an (approximate) minimum spanning
  tree of .  For constant values of , an approximate MST can
  be computed in  time using the algorithm of Calahan and
  Kosaraju \cite{callahan.kosaraju:faster}.  For larger values of ,
  the actual minimum spanning tree can be computed in  time.

  \item Constructing each of the paths .
  Each of these paths is easily constructed in  time
  once we have determined values of , , , and
   that are sufficiently small.  A more careful examination
  of our algorithm reveals that all that is really needed is
  a value of  such that  is
  simple, for each .  Once we have this value of
  , the values of the remaining variables can taken from
  the set .

  It turns out that a value , where  is the
  minimum non-zero difference between  coordinates or  coordinates
  in , and  is a constant, is sufficiently small.
  Thus, a suitable  can be computed in  time
  by sorting.
\end{enumerate}

This yields the following algorithmic result about connected augmentations:

\begin{theorem}
  An augmentation satisfying the conditions of Theorem~\ref{theorem:main}
  can be computed in  time for any value of .  If  is
  constant, then the augmentation can be computed in 
  time.
\end{theorem}

The latter result is worst-case optimal since, in the next section we will show that there exists inputs where every augmentation has size .


\section{Lower Bounds}\label{section:Lower bound}

Our lower bounds are based on the following lemma. It says that we can find  permutations of  such that for half the indices , and every , there is a permutation in which  and  are at distance .

\begin{lemma}\label{lem:permutations}
Let .  There exists permutations  of  such that for at least half the values of  and for every ,

\end{lemma}

\begin{proof}
  This proof is an application of the probabilistic method.  Select each
  of  independently and uniformly from among
  all  permutations of .  Fix a particular index 
  and a particular index .  For a particular , the
  probability that  is at most 
  since the set
   is a random subset of at most  elements drawn without
  replacement from .

  Therefore, since  are chosen independently, 
  
  In particular, the expected number of such
   is at most  so, by Markov's
  Inequality, the probability that there exists at least one such 
  is at most .  Thus, with probability at least , the index 
  satisfies \eqref{eq:perm} and therefore the expected number of indices 
  that satisfy \eqref{eq:perm} is .  We conclude that there must exist
  some permutations  that satisfy \eqref{eq:perm}
  for at least half the indices .
\end{proof}

Using Lemma~\ref{lem:permutations}, we can prove a lower bound that matches the upper bound obtained in our general construction.

\begin{theorem}\label{thm:lower-bound}
  For every positive integer  and every ,
  there exists a graph  having  vertices,  connected
  components, and  isomorphic drawings  such that
  any compatible augmentation of  has size .
\end{theorem}

\begin{proof}
Since the lemma only claims an asymptotic result, we may assume without
loss of generality that  is even and that  divides .

The graph  consists of  disjoint paths,
, each of length .  Each of the
drawings ,\ldots, has the vertices of  on the
same point and edge set. The point set consists of the vertices of
 nested regular -gons, , each centered at the
origin and having nearly the same size. Refer to Figure~\ref{figure:lower-bound} (left). More precisely,  and the sizes are chosen so that any segment
joining two non-consecutive vertices of  intersects the interior
of~.
The drawings  are obtained from the permutations
 given by Lemma~\ref{lem:permutations}.
In the drawing , the path  is drawn on the vertices
of . If  is even, the drawing uses
all the edges of  except the left-most edge.  If  is odd, the
drawing uses all the edges of  except the right-most edge.

\begin{figure}
  \centering{
    \begin{tabular}{c@{\hspace{1cm}}c}
      \includegraphics{img/lower-bound-1} & \includegraphics{img/lower-bound-2}
    \end{tabular}
  }
  \caption{In the construction in Theorem~\ref{thm:lower-bound}, all drawings use the same set of vertices and line segments and the drawing of a path that joins  to  must travel around all paths drawn between the drawing of  and .}
  \label{figure:lower-bound}
\end{figure}

Now, without loss of generality, consider some edge-minimal compatible
augmentation  of .  For each component
 of , let  be any path in  that has
one endpoint on , one endpoint on some other component
, , and no vertices of  in its
interior.
Now, for each of the  indices  that satisfy
\eqref{eq:perm}, the path  joins a vertex of
 to a vertex of , , and .  This path must
have length  since it has to ``go around'' the
paths between  and ; see
Figure~\ref{figure:lower-bound} (right).

Thus far, we have shown that for at least  values of
, the component  is the endpoint of a
path, , of length at least .
It is tempting to claim the result at this point, since
. Unfortunately, there
is a little more work that needs to be done, since two such paths 
and  may not be disjoint, so summing their lengths double-counts
the contribution of the shared portion.

To finish up we note that, since the augmentation  is minimal,
it is a tree;  contains no cycles, so any cycle in  contains an edge not in  that could be removed.  Now, observe that if we traverse the outer face of (any planar drawing of)  then we obtain a non-simple path, , that traverses each edge of  exactly twice. If we consider the set of maximal subpaths of  with no vertex of  in their interior, we obtain a set of  edge-disjoint paths,  and, for every component  of , there is a vertex of  that is an endpoint of at least one such path.  Therefore, from the preceding discussion, the total length of  is .  But since each edge of  appears at most twice in these subpaths, we conclude that  has  edges.  Since  is a tree, it has  vertices.
\end{proof}


\section*{Acknowledgement}

This work was initiated at the \emph{Second Workshop on Geometry and Graphs},
held at the Bellairs Research Institute, March 9-14, 2014.  We are
grateful to the other workshop participants for providing a stimulating
research environment.




\bibliographystyle{plain}
\bibliography{augmentations}

\appendix
\newpage
\section{Shortest Tour in the Uniform Norm}
\label{app:uniform-norm}

Under the \emph{ metric}, the distance between two points  and  is .  

\begin{lemma}\label{lemma:tsp}
  Let  be a set of  points contained in the -dimensional
  cube , for .  Then there exists a spanning path of 
  whose length under the  metric is at most ,
  where  is a universal constant. In particular,  does not depend
  on  or .
\end{lemma}

\begin{proof}
The following proof is a rehashing of an argument used by Moran
\cite{moran:on}. An argument of Few \cite{few:shortest}, which begins
by stabbing  with a grid of  lines, could
also be used to establish the same asymptotic result.

First, we note that it is sufficient to upper-bound the -length of the minimum spanning-tree of , since this
can be transformed into a path of at most twice its length \cite{rosenkrantz.stearns.ea:analysis}.

For any point , the \emph{uniform ball} of radius  centered at ,
defined as

is a cube of side-length  and has volume .  If  and
, then  contains a cube of side-length , so  has volume at least .  

The preceding implies that the set  contains two points  and ,
such that ; otherwise, one could pack 
disjoint cubes, each of volume greater than  into .

Now, we can construct a spanning tree of  by repeatedly taking the
pair of points  that minimize , adding the edge 
to our spanning tree and then removing  from . Since, at the th
step of this algorithm, the set  contains  points, the total -length of all the edges added to this tree is

for a sufficiently large constant  and any . Thus, the result holds for .
\end{proof}

By uniformly scaling the point set  by a factor of , we obtain the following corollary of Lemma~\ref{lemma:tsp}, which is used in our algorithm:

\begin{corollary}\label{cor:tsp}
  Let  be a set of  points contained in the cube ,
  for .  Then there exists a spanning path of
   whose length under  metric is at
  most , where  is a universal constant. In particular,
   does not depend on , , or .
\end{corollary}












\end{document}  
