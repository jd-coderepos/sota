\documentclass[a4paper,12pt]{article}

\usepackage{amsmath,amsthm,amssymb,multicol,epsf}
\usepackage{color}
\usepackage{graphicx}
\usepackage{amsmath}
\usepackage{amsfonts}
\usepackage{amssymb}
\usepackage{amsthm}
\usepackage{epstopdf}
\usepackage{algorithmic}
\usepackage{setspace}
\usepackage[cm]{fullpage}
\usepackage{multirow}

\newcommand{\alert}[1]{{\color{green}#1}}
\newcommand{\alertg}[1]{{\color{red}#1}}
\newcommand{\alertb}[1]{{\color{blue}#1}}



\newtheorem{defi}{Definition}[section]
\newtheorem{theorem}{Theorem}[section]
\newtheorem{lem}{Lemma}[section]
\newtheorem{corollary}{Corollary}[section]
\newtheorem{remark}{Remark}[section]

\theoremstyle{remark}
\newtheorem{algorithm}{Algorithm}
\newtheorem{problem}{Problem}

\newcommand{\reviewone}[1]{{\color{blue}#1}}
\newcommand{\reviewtwo}[1]{{\color{red}#1}}
\newcommand{\reviewour}[1]{{\color{green}#1}}


\newcommand{\res}{{_{\textstyle|}}_}
\newcommand{\valpha}{{\boldsymbol{\alpha}}}
\newcommand{\vlambda}{{\boldsymbol{\lambda}}}
\newcommand{\vmu}{{\boldsymbol{\mu}}}
\newcommand{\vxi}{{\boldsymbol{\xi}}}
\newcommand{\vbeta}{{\boldsymbol{\beta}}}
\newcommand{\vvarphi}{{\boldsymbol{\varphi}}}
\def\set#1{{\cal #1}}
\newcommand{\R}{\mathbb{R}}
\newcommand{\dabR}{\mathbb{R}}
\newcommand{\dabV}{\mathbb{V}}
\newcommand{\Image}{\textit{Im\,}}
\newcommand{\Ker}{\textit{Ker\,}}
\newcommand{\diag}{\textit{diag\,}}

\newcommand{\mbf}[1]{\mbox{\boldmath}}


\numberwithin{equation}{section}

\title{Subdifferential-based implicit return-mapping operators in Mohr-Coulomb plasticity}
\author{S. Sysala, M. Cermak \\ \\ Institute of Geonics, Czech Academy of Sciences, Ostrava, Czech
Republic\\ V\v SB--Technical University of Ostrava, Ostrava, Czech Republic}

\begin{document}

\maketitle

\begin{abstract}
The paper is devoted to a constitutive solution, limit load analysis and Newton-like methods in elastoplastic problems containing the Mohr-Coulomb yield criterion. Within the constitutive problem, we introduce a self-contained derivation of the implicit return-mapping solution scheme using a recent subdifferential-based treatment. Unlike conventional techniques based on Koiter's rules, the presented scheme a priori detects a position of the unknown stress tensor on the yield surface even if the constitutive solution cannot be found in closed form. This fact eliminates blind guesswork from the scheme, enables to analyze properties of the constitutive operator, and simplifies construction of the consistent tangent operator which is important for the semismooth Newton method applied on the incremental boundary value elastoplastic problem. The incremental problem in Mohr-Coulomb plasticity is combined with the limit load analysis. Beside a conventional direct method of the incremental limit analysis, a recent indirect one is introduced and its advantages are described. The paper contains 2D and 3D numerical experiments on slope stability with publicly available Matlab implementations.

\end{abstract}

\noindent
Keywords: infinitesimal plasticity, Mohr-Coulomb yield surface, implicit return-mapping scheme, consistent tangent operator, semismooth Newton method, incremental limit analysis, slope stability


\section{Introduction}

This paper is a continuation of \cite{SCKKZB15} which was devoted to a solution of elastoplastic constitutive problems using a subdifferential formulation of the plastic flow rule. It leads to simpler and more correct implicit constitutive solution schemes. While a broad class of elastoplastic models containing 1 or 2 singular points (apices) on the yield surface was considered in \cite{SCKKZB15}, the aim of this paper is to approach the subdifferential-based treatment to models that are usually formulated in terms of principal stresses.  For example, the principal stresses are used in models containing the Mohr-Coulomb, the Tresca, the Rankine, the Hoek-Brown or the unified strength yield criteria \cite{NPO08,CDA15,LL15,LR96}. Such criteria have a {\it multisurface} representation leading to a relatively complex structure of singular points. 

Due to technical complexity of implicit solution schemes for these models, we focus only on a particular but representative yield criterion: the Mohr-Coulomb one. This criterion is broadly exploited in soil and rock mechanics and its surface is a hexagonal pyramid aligned with the hydrostatic axis (see, e.g., \cite{NPO08}). We consider the Mohr-Coulomb model introduced in \cite[Section 8]{NPO08} which can optionally contain the {\it nonassociative flow rule} and the {\it nonlinear isotropic hardening}. The nonassociative flow rule enables to catch the dilatant behavior of a material. Further, due to the presence of the nonlinear hardening, one cannot find the implicit constitutive solution in closed form, and thus the problem remains challenging. As in \cite{NPO08}, we let a hardening function in an abstract form. For a particular example of the nonlinear hardening in soil mechanics, we refer, e.g., \cite{BSS03}.

In literature, there are many various concepts of the constitutive solution schemes for models containing yield criteria written in terms of the principal stresses. For their detailed overview and historical development, we refer the recent papers \cite{CDA15} and \cite{K13}, respectively. It is worth mentioning that the solution schemes mainly depend on a formulation of the plastic flow rule, its discretization and other eventual approximations.

In engineering practice, the plastic flow rule is usually formulated using the so-called Koiter rule introduced in \cite{K53} for associative models with multisurface yield criteria. Consequently, this rule was also extended for nonassociative models, see, e.g., \cite{dB87}. It consists of several formulas that depend on a position of the unknown stress tensor  on the yield surface. The formulas have a different number of plastic multipliers. Within the Mohr-Coulomb pyramid, one plastic multiplier is used for smooth portions, two multipliers at edge points, and six multipliers at the apex. For each Koiter's formula, a different solution scheme is introduced. However,  only one of which usually gives the correct stress tensor. Moreover, the handling with different numbers of plastic multipliers is not suitable for analysing the stress-strain operator even if the solution can be found in closed form. If an elastoplastic model contains a {\it convex plastic potential} as the Mohr-Coulomb one then it is possible to replace the Koiter rule with a subdifferential of the potential (see, e.g., \cite{NPO08}). Such a formulation is independent of the unknown stress position, contains just one plastic multiplier, and thus it is more convenient for mathematical analysis of the constitutive operators. In \cite{SCKKZB15}, it was shown that this formulation is also convenient for a solution of some constitutive problems. Further, in some special cases, the constitutive problem can  be also defined using the principle of maximum plastic dissipation \cite{NPO08, HR99} or by the theory of bipotentials \cite{B12} and solved by techniques based on mathematical programming. 

We focus on the (fully) implicit Euler discretization of the flow rule, which is frequently used in elastoplasticity. Beside other Euler-type methods (see, e.g., \cite{NPO08, SH98}), the cutting plane methods are also popular. We refer, e.g., \cite{SHVS14} for the literature survey and recent development of these methods. When the constitutive problems are discretized by the implicit Euler methods, the solution is searched by the elastic predictor -- plastic correction method. Within the plastic correction, the so-called (implicit) return-mapping scheme is constructed. It is worth mentioning that plastic correction problems can be reduced to problems formulated only in terms of the principal stresses \cite{CDA06, CDA15, NPO08}.

In order to simplify the solution schemes for nonsmooth yield criteria, many various approximative techniques have been suggested. These techniques are based on local or global smoothing of yield surfaces or plastic potentials. For literature survey, we refer \cite[Section 1.2]{CDA15} or \cite{ALSH11, B13, BSS03}. However, such an approach is out of the scope of this paper.

The constitutive problem is an essential part of the overall initial boundary value elastoplastic problem. Its time discretization leads to the incremental boundary value problem which is mostly solved by nonsmooth variants of the Newton method \cite{CKSV14, GrVa09, SaWi11, Sy09, Sy14} in each time step. Then, it is useful to construct the so-called consistent tangent operator representing a generalized derivative of the discretized constitutive stress-strain operator. We use the framework based on the eigenprojections of symmetric second order tensors, see, e.g., \cite{CaHo86, NPO08}. A similar approach is also used in the recent book \cite{B13} with slightly different terminology like the spectral directions or the spin of a tensor. Another approach is introduced, e.g., in \cite{C97, CDA06, CDA15} where the consistent tangent operator is determined by the tangent operator representing the relation between the stress and strain rates.

Further, this paper is devoted to the limit load problem which is frequently combined with the Mohr-Coulomb model. It is an additional problem to the elastoplastic one where the load history is not fully prescribed. It is only given a fixed external force that is multiplied by a scalar load parameter whose limit value is unknown. It is well known that the investigated body collapses when this critical value is exceeded. Therefore, this value is an important safety parameter and beyond it no solution exists. Strip-footing collapse or slope stability are traditional applications on this problematic (see, e.g. \cite{CL90, NPO08}).  The simplest computational technique is based on the so-called {\it incremental limit analysis} where the load parameter is enlarged up to its limit value. Then, the boundary-value elastoplastic problem is solved for investigated values of this parameter. Beside the conventional {\it direct method} of the incremental limit analysis, we also introduce the {\it indirect method} and describe its advantages based on recent expertise introduced in \cite{SHHC15, CHKS15, HRS16, HRS16b}.

The rest of the paper is organized as follows. In Section \ref{sec_spectrum}, an auxilliary framework related to the subdifferential of an eigenvalue function and derivatives of eigenprojections is introduced. In Section \ref{sec_model}, the Mohr-Coulomb constitutive initial value problem is formulated using the subdifferential of the plastic potential and discretized by the implicit Euler method. In Section \ref{sec_time_discret}, the existence and uniqueness of a solution to the discretized problem is proven and the improved solution scheme is derived. In Section \ref{sec.Stress-strain_relation}, the stress-strain and the consistent tangent operators are constructed. In Section \ref{sec_realization}, the direct and indirect methods of the incremental limit analysis are introduced. Both methods are combined with the semismooth Newton method. In Section \ref{sec_experiments}, 2D and 3D numerical experiments related to slope stability are introduced. In Section \ref{sec_conclusion}, some concluding remarks are mentioned. The paper also contains Appendix with some useful auxilliary results. In Appendix A, the solution scheme is simplified under the plane strain assumptions. In Appendix B, algebraic representations for second and fourth order tensors within the 3D and plane strain problems are derived.

In this paper, second order tensors, matrices, and vectors are denoted by
bold letters.  Further, the fourth order tensors
are denoted by capital blackboard letters, e.g.,  or . The symbol  means the tensor product \cite{NPO08}. We also use the following
notation:  and  for the space of symmetric, second order tensors. The standard scalar product in  and the biscalar product in  are denoted as  and , respectively.


\section{Subdifferentials and derivatives of eigenvalue functions}
\label{sec_spectrum}

In this section, we introduce an auxilliary framework that will be crucial for an efficient construction of the constitutive and consistent tangent operators in Mohr-Coulomb plasticity. Let

be the spectral decomposition of a tensor . Here, , , , denote the eigenvalues, and the eigenvectors of , respectively. The eigenvalues   can be computed using the Haigh-Westargaard coordinates (see, e.g., \cite[Appendix A]{NPO08}), and they are uniquely determined with respect to the prescribed ordering. Let  denote the corresponding eigenvalue functions, i.e. , . Further, we define the following set of admissible eigenvectors of : 


\subsection{Subdifferential of an eigenvalue function}
\label{subsec_subdif}

Recall the definition of the subdifferential to a convex function  at :

To receive the Mohr-Coulomb yield function or the plastic potential, we specify  as follows:

where the parameters  are sufficiently chosen. Notice that the convexity of the eigenvalue function  can be derived from:

Specific form of  with respect to (\ref{g_ab}) can be found using a framework introduced in \cite[Chapter 2]{Ru06}. We derive another form of  that is convenient for purposes of this paper.

\begin{lem}
Let  be defined by (\ref{g_ab}). Then for any , it holds:

\label{lem_subdif}
\end{lem}

\begin{proof}
Since  and  the standard definition of  is equivalent to:

First, we derive necessary and sufficient conditions on  ensuring

To this end, consider the following spectral decomposition of :

Choose , where  is the unit tensor in . Then from (\ref{ineq_g_T}), (\ref{cond_sub_0}) we have:

Choose  and .  Then from (\ref{ineq_g_T}), (\ref{cond_sub_0}) we derive, respectively:

Let  be arbitrarily chosen and denote , , . Then,

follow from  and (\ref{eig_maxmin}), respectively. Consequently,

Thus the conditions (\ref{cond_sub_0})-(\ref{cond_sub_2}) are necessary and sufficient for (\ref{ineq_g_T}).

Secondly, assume that  belongs to . Then (\ref{cond_sub_0})-(\ref{cond_sub_2}) hold. Since  , the equalities must hold within the derivation of (\ref{ineq_T}) for , i.e., we have:


It is easy to see that the equalities in (\ref{cond_sub_4a}) imply:


We have proven that for any element  the conditions (\ref{cond_sub_0})-(\ref{cond_sub_2}), (\ref{cond_sub_7}) and (\ref{cond_sub_8}) hold. Therefore,

Conversely, one can easily check that any element from the set on the right hand side in (\ref{sub_g_ab2}) belongs to  using (\ref{sub_def}) and (\ref{ineq_T}).
\end{proof}

\begin{remark}
\emph{One can easily specify the eigenvalues ,  and  in (\ref{sub_g_ab}) depending on a number of distinct eigenvalues of  . If  then ,  and . If  then , , and . If  then  and , . }
\label{remark_subdif2}
\end{remark}


\subsection{First and second derivatives of eigenvalue functions}
\label{subsec_eigenprojection}

It is well-known that differentiability of eigenvalue functions depends on multiplicity of the eigenvalues. For example, the function  is differentiable at  with  as follows from  Remark \ref{remark_subdif2}. Following \cite{NPO08, CaHo86}, we derive the first and second Fr\'echet derivatives of the eigenvalue functions using eigenprojections.  The derivative of function  at  is denoted as . Analogous notation, , is also used for tensor-valued function  . Further, it is worth mentioning that some derivatives introduced below cannot be extended on .

First, assume three distinct eigenvalues of , i.e., . Then one can introduce the eigenprojections , , of  as follows:

It holds:



for any  , where the components of the fourth order tensors  and  satisfy
 and , respectively\footnote{In \cite[Appendix A]{NPO08}, instead of  and , their symmetric parts are introduced. For example, instead of , the tensor  with the components  is considered. One can easily check that  for any . A similar identity also holds for .}.
We use the notation ,  . 

Now, assume . In this more general case, one can introduce the derivatives of  and . From (\ref{eigenprojection_case1}), it is readily seen that the function  can be continuously extended for  satisfying  unlike  and . Hence and from (\ref{eigen_prop}), (\ref{der_eigen_case1}), one can write:

To continuously extend the function , we use the equality

and substitute it into (\ref{E_deriv_case1}) for . We obtain
3pt]
&&+\frac{(\eta_1+\eta_2-2\eta_3)\mbf E_3\otimes\mbf E_3+\eta_3[\mbf E_{12}\otimes\mbf E_3+\mbf E_3\otimes\mbf E_{12}]}{(\eta_3-\eta_1)(\eta_3-\eta_2)}.
\label{E3_deriv_case2}

\mathcal D\omega_1(\mbf\eta)=\mbf E_1(\mbf\eta),\quad \mathcal D\omega_{23}(\mbf\eta)=\mbf I-\mbf E_{1}(\mbf\eta)=: \mbf E_{23}(\mbf\eta),
\label{der_eigen_case3}

\mathbb E_1(\mbf\eta)=\mathcal D\mbf E_1(\mbf\eta)&=&\frac{\mathcal D(\mbf\eta^2)-(\eta_2+\eta_3)\mathbb I-[\mbf\eta\otimes\mbf E_{23}+\mbf E_{23}\otimes\mbf\eta]+(\eta_2+\eta_3)\mbf E_{23}\otimes\mbf E_{23}}{(\eta_1-\eta_2)(\eta_1-\eta_2)}+\nonumber\
Notice that if  then  has only two eigenprojections:  and , and . Conversely, if , then .

In the general case , it holds that  and thus

Notice that if  then  has only one eigenprojection: .

\begin{remark}
\emph{The mentioned derivatives can be found in simpler forms when plane strain assumptions are considered, see Appendix A of this paper.}
\end{remark}


\section{The Mohr-Coulomb constitutive problems}
\label{sec_model}

In this section, we introduce the Mohr-Coulomb constitutive initial value problem and its implicit Euler discretization. We use the model proposed in \cite{NPO08} containing the Mohr-Coulomb yield criterion, the  nonassociative plastic flow rule, and the nonlinear isotropic hardening.

\subsection{The initial value constitutive problem}

The initial value constitutive problem reads as:

\medskip\noindent
\textit{Given the history of the strain tensor , , and the initial values

Find  such that 
1mm]
\dot{\mbf{\varepsilon}^p}\in\dot\lambda\partial g(\mbf\sigma),\;\; \dot{\bar{\varepsilon}}^p=-\dot\lambda\frac{\partial f(\mbf\sigma,\kappa)}{\partial \kappa},\
hold for each instant }.

\medskip\noindent
Here,  denote the Cauchy stress tensor, the plastic strain, the hardening variable, and the plastic multiplier, respectively. The dot symbol means the pseudo-time derivative of a quantity. The functions  and  represent the yield function and the plastic potential for the Mohr-Coulomb model, respectively. They are defined as follows:

where  and  are the maximal and minimal eigenvalue functions introduced in Section \ref{sec_spectrum}, and
he material parameters ,  represent the initial cohesion, the friction angle, and the dilatancy angle, respectively. Notice that  are convex functions with respect to the stress variable. Recall that the function  was already introduced in Section \ref{subsec_subdif} for the choice

and thus one can define  using Lemma \ref{lem_subdif}.
Clearly, .

Further, the fourth order tensor  represents linear isotropic elastic law:

where  is the elastic part of the strain tensor and  denotes the bulk, and shear moduli, respectively. 

Finally, we let the function  representing the non-linear isotropic hardening in an abstract form and assume that
it is a nondecreasing, continuous, and piecewise smooth function satisfying . 

It is worth mentioning that the value  need not be always known, see Section \ref{sec_realization}.

\subsection{The discretized constitutive problem}

Let  be a partition of the interval  and denote , , , , , , and . Here, the superscript  is the standard notation for the so-called trial variables (see, e.g., \cite{NPO08}) which are known. If it is clear that the step  is fixed then we will omit the subscript  and write , , , , , , and  to simplify the notation. The -th step of the incremental constitutive problem discretized by the implicit Euler method reads as: 

\medskip\noindent
{\it Given  and . Find ,  , and  satisfying:}
1mm]
\bar{\varepsilon}^p=\bar{\varepsilon}^{p,tr}+\triangle\lambda (2\cos\phi),\
Unlike problem (\ref{CIVP_MC}), the unknown  is not introduced in (\ref{k_step_problem}). It can be simply computed from the formula  and used as the input parameter for the next step.



\section{Solution of the discretized constitutive problem}
\label{sec_time_discret}

The aim of this section is to derive an improved solution scheme to problem (\ref{k_step_problem}).  The solution scheme builds on the standard {\it elastic predictor - plastic corrector method} and its improvement is based on the form of  introduced in Lemma \ref{lem_subdif}. Within the {\it elastic prediction}, we assume . Then, it is readily seen that the triple

is the solution to (\ref{k_step_problem}) under the condition

The {\it plastic correction} happens when . Then the unknown generalized stress  lies on the yield surface and thus the corresponding plastic correction problem reads as:
{\it Given  and . Find ,  , and  satisfying:}
1mm]
\bar{\varepsilon}^p=\bar{\varepsilon}^{p,tr}+\triangle\lambda (2\cos\phi),\
The solution scheme to problem (\ref{correction_problem}) is usually called the {\it implicit return-mapping scheme}. Since its derivation is technically complicated, we divide the rest of this section into several subsections for easier orientation in the text. In Section \ref{subsec_corrections}, problem (\ref{correction_problem}) is reduced and written in terms of principal stresses. In parallel Sections \ref{subsec_smooth}-\ref{subsec_apex}, we introduce solution schemes for returns to the smooth portion, to the ``left" edge, to the ``right" edge, and to the apex of the pyramidal yield surface, respectively.  In Section \ref{subsec_lambda}, we derive a nonlinear equation for the unknown plastic multiplier. This equation is common for all types of the return and has the unique solution. Hence, we derive: existence and uniqueness of problems (\ref{k_step_problem}) and (\ref{correction_problem}), a priori decision criteria for the return types, and other useful results describing a dependence of the unknown stress tensor on the trial stress tensor.
 
\subsection{Plastic correction problem in terms of principal stresses}
\label{subsec_corrections}

First, we reduce problem (\ref{correction_problem}) using the spectral decomposition of  (see Section \ref{sec_spectrum}):

From the definition of  introduced in Section \ref{sec_model}, it is easy to see that the equation (\ref{correction_problem}) can be written only in terms the principal stresses  instead of the whole stress tensor . To re-formulate (\ref{correction_problem}), we use Lemma \ref{lem_subdif} and (\ref{def_ab}): there exists  such that , where
1 mm]
( \nu_1-1-\sin\psi)(\sigma_1-\sigma_2)=0,\quad ( \nu_3+1-\sin\psi)(\sigma_2-\sigma_3)=0.
\end{array}
\right\}
\label{prop_n_i}

\mathbb D_e:\mbf\nu=\sum_{i=1}^3 \left[\frac{2}{3}(3K-2G)\sin\psi+2G\nu_i\right]\mathbf e_i\otimes\mathbf e_i.
\label{Den_spectrum}

\mbf{\sigma}^{tr}=\mbf\sigma+\triangle\lambda\mathbb D_e:\mbf\nu=\sum_{i=1}^3\sigma_i^{tr}\mathbf e_i\otimes\mathbf e_i,\;\;\mbox{where}\quad \sigma_i^{tr}=\sigma_i+\triangle\lambda\left[\frac{2}{3}(3K-2G)\sin\psi+2G\nu_i\right].
\label{flow_eigen}

\left.
\begin{array}{c}
\sigma_i=\sigma_i^{tr}-\triangle\lambda\left[\frac{2}{3}(3K-2G)\sin\psi+2G\nu_i\right],\quad i=1,2,3,\1mm]
(1+\sin\phi)\sigma_1-(1-\sin\phi)\sigma_3-2(c_0+H(\bar\varepsilon^p))\cos\phi=0,
\end{array}
\right\}
\label{correction_problem2}

\sigma_1&=&\sigma_1^{tr}-\triangle\lambda\left[\frac{2}{3}(3K-2G)\sin\psi+2G(1+\sin\psi)\right],\label{flow11}\\
\sigma_2&=&\sigma_2^{tr}-\triangle\lambda\left[\frac{2}{3}(3K-2G)\sin\psi\right],\label{flow12}\\
\sigma_3&=&\sigma_3^{tr}-\triangle\lambda\left[\frac{2}{3}(3K-2G)\sin\psi-2G(1-\sin\psi)\right].\label{flow13}

q^{tr}_s(\gamma)&=&(1+\sin\phi)\sigma_1^{tr}-(1-\sin\phi)\sigma_3^{tr}-2\left[c_0+H\left(\bar{\varepsilon}^{p,tr}+\gamma (2\cos\phi)\right)\right]\cos\phi\nonumber\\
&&-\gamma\left[\frac{4}{3}(3K-2G)\sin\psi\sin\phi+ 4G(1+\sin\psi\sin\phi)\right]. \label{q1}

\gamma^{tr}_{s,l}:=\frac{\sigma_1^{tr}-\sigma_2^{tr}}{2G(1+\sin\psi)}\geq0,\quad\gamma^{tr}_{s,r}:=\frac{\sigma_2^{tr}-\sigma_3^{tr}}{2G(1-\sin\psi)}\geq0.
\label{gamma1}

\frac{1}{2}(\sigma_1+\sigma_2)=\sigma_1&=&\frac{1}{2}(\sigma_1^{tr}+\sigma_2^{tr})-\triangle\lambda\left[\frac{2}{3}(3K-2G)\sin\psi+G(1+\sin\psi)\right],\label{flow212}\\
\sigma_3&=&\sigma_3^{tr}-\triangle\lambda\left[\frac{2}{3}(3K-2G)\sin\psi-2G(1-\sin\psi)\right],\label{flow23}

0=\sigma_1-\sigma_2=\sigma_1^{tr}-\sigma_2^{tr}-\triangle\lambda[2G(\nu_1-\nu_2)]\geq\sigma_1^{tr}-\sigma_2^{tr}-\triangle\lambda[2G(1+\sin\psi)].
\label{est_left}

q^{tr}_l(\gamma)&=&\frac{1}{2}(1+\sin\phi)(\sigma_1^{tr}+\sigma_2^{tr})-(1-\sin\phi)\sigma_3^{tr}-2\left[c_0+H\left(\bar{\varepsilon}^{p,tr}+\gamma (2\cos\phi)\right)\right]\cos\phi-\nonumber\\
&&\gamma\left[\frac{4}{3}(3K-2G)\sin\psi\sin\phi+ G(1+\sin\psi)(1+\sin\phi)+ 2G(1-\sin\psi)(1-\sin\phi)\right].\qquad
\label{q2}

\gamma^{tr}_{l,a}=\frac{\sigma_1^{tr}+\sigma_2^{tr}-2\sigma_3^{tr}}{2G(3-\sin\psi)}=\frac{1+\sin\psi}{3-\sin\psi}\gamma^{tr}_{s,l}+\left(1-\frac{1+\sin\psi}{3-\sin\psi}\right)\gamma^{tr}_{s,r}\geq0
\label{gamma2}

\sigma_1&=&\sigma_1^{tr}-\triangle\lambda\left[\frac{2}{3}(3K-2G)\sin\psi+2G(1+\sin\psi)\right],\label{flow31}\\
\frac{1}{2}(\sigma_2+\sigma_3)=\sigma_3&=&\frac{1}{2}(\sigma_2^{tr}+\sigma_3^{tr})-\triangle\lambda\left[\frac{2}{3}(3K-2G)\sin\psi- G(1-\sin\psi)\right].\label{flow323}

0=\sigma_2-\sigma_3=\sigma_2^{tr}-\sigma_3^{tr}-\triangle\lambda[2G(\nu_2-\nu_3)]\geq\sigma_2^{tr}-\sigma_3^{tr}-\triangle\lambda[2G(1-\sin\psi)].
\label{est_right}

q^{tr}_r(\gamma)&=&(1+\sin\phi)\sigma_1^{tr}-\frac{1}{2}(1-\sin\phi)(\sigma_2^{tr}+\sigma_3^{tr})-2\left[c_0+H\left(\bar{\varepsilon}^{p,tr}+\gamma (2\cos\phi)\right)\right]\cos\phi-\nonumber\\
&&\gamma\left[\frac{4}{3}(3K-2G)\sin\psi\sin\phi+ 2G(1+\sin\psi)(1+\sin\phi)+ G(1-\sin\psi)(1-\sin\phi)\right].\qquad
\label{q3}

\gamma^{tr}_{r,a}=\frac{2\sigma_1^{tr}-\sigma_2^{tr}-\sigma_3^{tr}}{2G(3+\sin\psi)}=\frac{1-\sin\psi}{3+\sin\psi}\gamma^{tr}_{s,r}+\left(1-\frac{1-\sin\psi}{3+\sin\psi}\right)\gamma^{tr}_{s,l}\geq0.
\label{gamma3}

\sigma_1=\frac{1}{3}(\sigma_1+\sigma_2+\sigma_3)=\frac{1}{3}(\sigma_1^{tr}+\sigma_2^{tr}+\sigma_3^{tr})-\triangle\lambda[2K\sin\psi]\quad\label{flow4123}

0&=&2\sigma_1-\sigma_2-\sigma_3\geq2\sigma_1^{tr}-\sigma_2^{tr}-\sigma_3^{tr}-\triangle\lambda[2G(3+\sin\psi)],\label{est_apex1}\\
0&=&\sigma_1+\sigma_2-2\sigma_3\geq\sigma_1^{tr}+\sigma_2^{tr}-2\sigma_3^{tr}-\triangle\lambda[2G(3-\sin\psi)].\label{est_apex2}

q^{tr}_a(\gamma)=\frac{2}{3}(\sigma_1^{tr}+\sigma_2^{tr}+\sigma_3^{tr})\sin\phi-2\left[c_0+H\left(\bar{\varepsilon}^{p,tr}+\gamma (2\cos\phi)\right)\right]\cos\phi-\gamma[4K\sin\psi\sin\phi]. \label{q4}

q^{tr}(\gamma)&=&(1+\sin\phi)\sigma_1^{tr}-(1-\sin\phi)\sigma_3^{tr}-\gamma\left[\frac{4}{3}(3K-2G)\sin\psi\sin\phi+4G(1+\sin\psi\sin\phi)\right]\nonumber\\
&&+G(1+\sin\psi)(1+\sin\phi)(\gamma-\gamma^{tr}_{s,l})^++\frac{1}{3}G(3-\sin\psi)(3-\sin\phi)(\gamma-\gamma^{tr}_{l,a})^+\nonumber\\
&&-2\left[c_0+H\left(\bar{\varepsilon}^{p,tr}+\gamma (2\cos\phi)\right)\right]\cos\phi,\quad\gamma\in(0,+\infty),
\label{q1_tr}

q^{tr}(\gamma)&:=&(1+\sin\phi)\sigma_1^{tr}-(1-\sin\phi)\sigma_3^{tr}-\gamma\left[\frac{4}{3}(3K-2G)\sin\psi\sin\phi+4G(1+\sin\psi\sin\phi)\right]\nonumber\\
&&+G(1-\sin\psi)(1-\sin\phi)(\gamma-\gamma^{tr}_{s,r})^++\frac{1}{3}G(3+\sin\psi)(3+\sin\phi)(\gamma-\gamma^{tr}_{r,a})^+\nonumber\\
&&-2\left[c_0+H\left(\bar{\varepsilon}^{p,tr}+\gamma (2\cos\phi)\right)\right]\cos\phi.
\label{q2_tr}

\mbf{\sigma}(t_k):=\mbf T\left(\mbf{\varepsilon}(t_k);\mbf{\varepsilon^p}(t_{k-1}),\bar{\varepsilon}^p(t_{k-1})\right).
\label{T0}

\mbf{\sigma}:=\mbf T\left(\mbf{\varepsilon};\mbf{\varepsilon^p}(t_{k-1}),\bar{\varepsilon}^p(t_{k-1})\right)=\mbf S\left(\mbf{\varepsilon}^{tr},\bar{\varepsilon}^{p,tr}\right)
\label{T}

M^{tr}_{e}&=&\{\mbf\varepsilon^{tr}\in\mathbb R^{3\times 3}_{sym}\; |\; q^{tr}(0)=q^{tr}_s(0)=f\left(\mbf{\sigma}^{tr},H(\bar{\varepsilon}^{p,tr})\right)<0\},\\
M^{tr}_{s}&=&\{\mbf\varepsilon^{tr}\in\mathbb R^{3\times 3}_{sym}\; |\; q^{tr}_s(0)>0,\; q^{tr}_s(\min\{\gamma^{tr}_{s,l},\gamma^{tr}_{s,r}\})< 0\},\\
M^{tr}_{l}&=&\{\mbf\varepsilon^{tr}\in\mathbb R^{3\times 3}_{sym}\; |\; \gamma^{tr}_{s,l}<\gamma^{tr}_{l,a},\; q^{tr}_l(\gamma^{tr}_{s,l})> 0,\; q^{tr}_l(\gamma^{tr}_{l,a})< 0\},\\
M^{tr}_{r}&=&\{\mbf\varepsilon^{tr}\in\mathbb R^{3\times 3}_{sym}\; |\; \gamma^{tr}_{s,r}<\gamma^{tr}_{r,a},\; q^{tr}_r(\gamma^{tr}_{s,r})> 0,\; q^{tr}_r(\gamma^{tr}_{r,a})< 0\},\\
M^{tr}_{a}&=&\{\mbf\varepsilon^{tr}\in\mathbb R^{3\times 3}_{sym}\; |\; q^{tr}_a(\max\{\gamma^{tr}_{l,a},\gamma^{tr}_{r,a}\})>0\}.

\sigma_i^{tr}=\frac{1}{3}(3K-2G)(\varepsilon^{tr}_1+\varepsilon^{tr}_2+\varepsilon^{tr}_3)+2G\varepsilon^{tr}_i,\quad i=1,2,3.
\label{sigma_i^tr}

\mbf S\left(\mbf{\varepsilon}^{tr},\bar{\varepsilon}^{p,tr}\right)=\mathbb D_e:\mbf\varepsilon^{tr},\quad \mathcal D\mbf S\left(\mbf{\varepsilon}^{tr},\bar{\varepsilon}^{p,tr}\right)=\mathbb D_e.

\mbf S\left(\mbf{\varepsilon}^{tr},\bar{\varepsilon}^{p,tr}\right)=\sum_{i=1}^3\sigma_i\mbf E_i^{tr},\quad \mathcal D\mbf S\left(\mbf{\varepsilon}^{tr},\bar{\varepsilon}^{p,tr}\right)=\sum_{i=1}^3\left[\sigma_i\mathbb E_i^{tr}+\mbf E_i^{tr}\otimes\mathcal D\sigma_i\right].

\mathcal D\sigma_1&\stackrel{(\ref{flow11})}{=}&\frac{1}{3}(3K-2G)\mbf I+2G\mbf E_1^{tr}-\mathcal D(\triangle\lambda)\left[\frac{2}{3}(3K-2G)\sin\psi+2G(1+\sin\psi)\right],\\
\mathcal D\sigma_2&\stackrel{(\ref{flow12})}{=}&\frac{1}{3}(3K-2G)\mbf I+2G\mbf E_2^{tr}-\mathcal D(\triangle\lambda)\left[\frac{2}{3}(3K-2G)\sin\psi\right],\\
\mathcal D\sigma_3&\stackrel{(\ref{flow13})}{=}&\frac{1}{3}(3K-2G)\mbf I+2G\mbf E_3^{tr}-\mathcal D(\triangle\lambda)\left[\frac{2}{3}(3K-2G)\sin\psi-2G(1-\sin\psi)\right],

\mathcal D\mbf S\left(\mbf{\varepsilon}^{tr},\bar{\varepsilon}^{p,tr}\right)&=&\sum_{i=1}^3\left[\sigma_i\mathbb E_i^{tr}+2G\mbf E_i^{tr}\otimes\mbf E_i^{tr}\right]+\frac{1}{3}(3K-2G)\mbf I\otimes\mbf I-\nonumber\\
&&-\left[2G(1+\sin\psi)\mbf E_1^{tr}-2G(1-\sin\psi)\mbf E_3^{tr}+\frac{2}{3}(3K-2G)\sin\psi\mbf I\right]\otimes\mathcal D(\triangle\lambda),\qquad

\mathcal D(\triangle\lambda)\stackrel{(\ref{q1})}{=}\frac{2G(1+\sin\phi)\mbf E_1^{tr}-2G(1-\sin\phi)\mbf E_3^{tr}+\frac{2}{3}(3K-2G)\sin\phi\mbf I}{\frac{4}{3}(3K-2G)\sin\psi\sin\phi+4G(1+\sin\psi\sin\phi)+4H_1\cos^2\phi}.

\mbf S\left(\mbf{\varepsilon}^{tr},\bar{\varepsilon}^{p,tr}\right)=\sigma_1\mbf E_{12}^{tr}+\sigma_3\mbf E_3^{tr},\quad \mathcal D\mbf S\left(\mbf{\varepsilon}^{tr},\bar{\varepsilon}^{p,tr}\right)=(\sigma_3-\sigma_1)\mathbb E_3^{tr}+\mbf E_{12}^{tr}\otimes\mathcal D\sigma_1+\mbf E_{3}^{tr}\otimes\mathcal D\sigma_3.
\label{sigmaE2}

\mathcal D\sigma_1&\stackrel{(\ref{flow212})}{=}&\frac{1}{3}(3K-2G)\mbf I+G\mbf E_{12}^{tr}-\mathcal D(\triangle\lambda)\left[\frac{2}{3}(3K-2G)\sin\psi+G(1+\sin\psi)\right],\\
\mathcal D\sigma_3&\stackrel{(\ref{flow23})}{=}&\frac{1}{3}(3K-2G)\mbf I+2G\mbf E_3^{tr}-\mathcal D(\triangle\lambda)\left[\frac{2}{3}(3K-2G)\sin\psi-2G(1-\sin\psi)\right],

\mathcal D\mbf S\left(\mbf{\varepsilon}^{tr},\bar{\varepsilon}^{p,tr}\right)&=&(\sigma_3-\sigma_1)\mathbb E_3^{tr}+G\mbf E_{12}^{tr}\otimes\mbf E_{12}^{tr}+2G\mbf E_3^{tr}\otimes\mbf E_3^{tr}+\frac{1}{3}(3K-2G)\mbf I\otimes\mbf I-\nonumber\\
&&-\left[G(1+\sin\psi)\mbf E_{12}^{tr}-2G(1-\sin\psi)\mbf E_3^{tr}+\frac{2}{3}(3K-2G)\sin\psi\mbf I\right]\otimes\mathcal D(\triangle\lambda),\qquad

\mathcal D(\triangle\lambda)\stackrel{(\ref{q2})}{=}\frac{G(1+\sin\phi)\mbf E_{12}^{tr}-2G(1-\sin\phi)\mbf E_3^{tr}+\frac{2}{3}(3K-2G)\sin\phi\mbf I}{\frac{4}{3}(3K-2G)\sin\psi\sin\phi+G(1+\sin\psi)(1+\sin\phi)+2G(1-\sin\psi)(1-\sin\phi)+4H_1\cos^2\phi}.

\mbf S\left(\mbf{\varepsilon}^{tr},\bar{\varepsilon}^{p,tr}\right)=\sigma_1\mbf E_{1}^{tr}+\sigma_3\mbf E_{23}^{tr},\quad \mathcal D\mbf S\left(\mbf{\varepsilon}^{tr},\bar{\varepsilon}^{p,tr}\right)=(\sigma_1-\sigma_3)\mathbb E_1^{tr}+\mbf E_{1}^{tr}\otimes\mathcal D\sigma_1+\mbf E_{23}^{tr}\otimes\mathcal D\sigma_3.
\label{sigmaE3}

\mathcal D\sigma_1&\stackrel{(\ref{flow31})}{=}&\frac{1}{3}(3K-2G)\mbf I+2G\mbf E_{1}^{tr}-\mathcal D(\triangle\lambda)\left[\frac{2}{3}(3K-2G)\sin\psi+2G(1+\sin\psi)\right],\\
\mathcal D\sigma_3&\stackrel{(\ref{flow323})}{=}&\frac{1}{3}(3K-2G)\mbf I+G\mbf E_{23}^{tr}-\mathcal D(\triangle\lambda)\left[\frac{2}{3}(3K-2G)\sin\psi-G(1-\sin\psi)\right],

\mathcal D\mbf S\left(\mbf{\varepsilon}^{tr},\bar{\varepsilon}^{p,tr}\right)&=&(\sigma_1-\sigma_3)\mathbb E_1^{tr}+2G\mbf E_{1}^{tr}\otimes\mbf E_{1}^{tr}+G\mbf E_{23}^{tr}\otimes\mbf E_{23}^{tr}+\frac{1}{3}(3K-2G)\mbf I\otimes\mbf I-\nonumber\\
&&-\left[2G(1+\sin\psi)\mbf E_{1}^{tr}-G(1-\sin\psi)\mbf E_{23}^{tr}+\frac{2}{3}(3K-2G)\sin\psi\mbf I\right]\otimes\mathcal D(\triangle\lambda),\qquad

\mathcal D(\triangle\lambda)\stackrel{(\ref{q3})}{=}\frac{2G(1+\sin\phi)\mbf E_{1}^{tr}-G(1-\sin\phi)\mbf E_{23}^{tr}+\frac{2}{3}(3K-2G)\sin\phi\mbf I}{\frac{4}{3}(3K-2G)\sin\psi\sin\phi+2G(1+\sin\psi)(1+\sin\phi)+G(1-\sin\psi)(1-\sin\phi)+4H_1\cos^2\phi}.

\mbf S\left(\mbf{\varepsilon}^{tr},\bar{\varepsilon}^{p,tr}\right)=p\mbf I,\quad p=p^{tr}-(2K\sin\psi)\triangle\lambda,\quad p^{tr}=\frac{1}{3}(\sigma_1^{tr}+\sigma_2^{tr}+\sigma_3^{tr})=K(\varepsilon_1^{tr}+\varepsilon_2^{tr}+\varepsilon_3^{tr}),
\label{sigmaE4}

\mathcal D\mbf S\left(\mbf{\varepsilon}^{tr},\bar{\varepsilon}^{p,tr}\right)\stackrel{(\ref{sigmaE4})}{=}\frac{\partial p}{\partial p^{tr}}K\mbf I\otimes\mbf I=\left(1-2K\sin\psi\frac{\partial\triangle\lambda}{\partial p^{tr}}\right)K\mbf I\otimes\mbf I.

\frac{\partial\triangle\lambda}{\partial p^{tr}}\stackrel{(\ref{q4})}{=}\frac{\sin\phi}{2K\sin\psi\sin\phi+2H_1\cos^2\phi}.
\label{dlambda4}

\mathcal D\mbf S\left(\mbf{\varepsilon}^{tr},\bar{\varepsilon}^{p,tr}\right)=K\left(1-\frac{K\sin\psi\sin\phi}{K\sin\psi\sin\phi+H_1\cos^2\phi}\right)\mbf I\otimes\mbf I.
(\mathcal P_k)_\zeta \qquad\mbox{given }\zeta_{k}\in\mathbb R_+, \mbox{ find } \mbf u_{k}\in\mathbb R^n:\quad \mbf{F}_{k}(\mbf{u}_{k})=\zeta_{k}\mbf{l},0<\zeta_1<\zeta_2<\ldots<\zeta_{k}<\zeta_{k+1}<\ldots<\zeta_{lim}(\mathcal P_k)^\alpha \qquad \mbox{given }(\mbf b,\alpha_{k})\in\mathbb R^n\times\mathbb R_+,\;\mbox{find } (\mbf u_{k}, \zeta_{k})\in\mathbb R^n\times\mathbb R_+:\quad \left\{
\begin{array}{c}
\mbf{F}_{k}(\mbf{u}_{k})=\zeta_{k}\mbf{l},\ 
Clearly, if  is the solution to  then  also solves  for  and .
Unlike to problem , one can expect that problem  has the solution for any . Since the parameter  can be enlarged arbitrary, the indirect method is more stable and does not include any blind guesswork unlike the direct one.
This is the main advantage of the indirect method. For the associative Mohr-Coulomb model, one can expect that  as . This is proven in \cite{CHKS15, HRS16} for  and the generalized Hencky's plasticity. For the nonassociative Mohr-Coulomb model with , we observe that  for some finite  and for , the sequence  is nonincreasing. In such a case, the material exhibits softening behavior and the direct method is too convenient. It is also worth mentioning that the indirect method is similar to the arc-length method introduced, e.g., in \cite{C97, NPO08}.

We solve problems  and  by the semismooth Newton method:

\begin{algorithm}[ALG-]
\hspace{0.2cm}
\begin{spacing}{1.2}
\begin{algorithmic}[1]
  \STATE initialization: 
  \FOR{}
    \STATE find : 
    \STATE compute 
    \STATE {\bf if
    }{} {\bf then stop}
  \ENDFOR
  \STATE set .
\end{algorithmic}
\end{spacing}
\end{algorithm}

\begin{algorithm}[ALG-]
\hspace{0.2cm}
\begin{spacing}{1.2}
\begin{algorithmic}[1]
  \STATE initialization: , 
  \FOR{}
    \STATE find : 
    \STATE compute 
    \STATE compute 
    \STATE set 
    \STATE {\bf if
    }{} {\bf then stop}
  \ENDFOR
  \STATE set , .
\end{algorithmic}
\end{spacing}
\end{algorithm}
If   is semismoothness in  then one can easily show that  is semismooth in . The semismoothness is an essential assumption ensuring local superlinear convergence of these algorithms (see, e.g., \cite{CHKS15}). Further, we initialize ALG- and ALG- using the linear extrapolation of the solutions from two previous steps. In particular, we prescribe

in ALG-  for , and analogously, in ALG-. We observe that this initialization is more convenient than , .

The direct and indirect methods of incremental limit analysis are compared in Section \ref{subsec_comparison}.



\section{Numerical experiments - slope stability}
\label{sec_experiments}

We have implemented the direct and indirect methods of incremental limit analysis in MatLab for 3D slope stability problem and its plane strain reduction. These experimental codes denoted as SS-MC-NP-3D, SS-MC-NH and SS-MC-NH-Acontrol are available in \cite{Mcode}. The codes are vectorized and include the improved return-mapping scheme for the Mohr-Coulomb model in combination with ALG- or ALG-.  One can choose: a) several types of finite elements with appropriate numerical quadratures; b) locally refined meshes with various densities. 

\begin{figure}[htbp]
\center
  \includegraphics[width=0.6\textwidth]{mesh_Q2_level1_2.jpg}
   \caption{\small{Cross section of the body with the coarsest mesh for  elements.}}
   \label{fig.mesh_Q2}
\end{figure}

We consider the benchmark plane strain problem introduced in \cite[Page 351]{NPO08} and its extension for 3D case.
The 2D cross-section of the body with the coarsest mesh considered in \cite[SS-MC-NH]{Mcode} is depicted in Figure \ref{fig.mesh_Q2}. The 3D geometry and the corresponding hexahedral mesh arise from 2D by extruding. The slope height is 10 m and its inclination is . On the bottom, we assume that the body is fixed and, on the lateral sides,  zero normal displacements are prescribed. The body is subjected to self-weight. We set the specific weight kN/m with  being the mass density and  the gravitational acceleration. Such a volume force is multiplied by the load factor . The parameter  is here the settlement at the corner point  on the top of the slope to be in accordance with \cite{NPO08}.
Further, we set kPa, ,  and kPa, where  denotes the cohesion for the perfect plastic model. Hence, kPa and kPa. The remaining parameters of the Mohr-Coulomb model will be introduced below depending on a particular experiment.

We introduce one experiment for the plane strain (2D) problem and one for the 3D problem. The primary aim of these experiments is to numerically illustrate that the formulas derived in Sections \ref{sec_time_discret}, \ref{sec.Stress-strain_relation} and Appendix A work well. This can be confirmed by observing the superlinear convergence of ALG- and ALG- and their stability in vicinity of the limit load. We also prescribe a high precision of these algorithms by the setting  in both experiments. Other aims will be specified below.


\subsection{Comparison of the direct and indirect methods in 2D}
\label{subsec_comparison}

We compare the direct method (code SS-MC-NH) and the indirect method (code SS-MC-NH-Acontrol) of the incremental limit analysis on the slope stability benchmark in 2D. We consider the associative Mohr-Coulomb model containing the nonlinear isotropic hardening defined as in \cite{SCKKZB15}:

Here,  represents the initial slope of  and the material response is perfect plastic for sufficiently large values of . The function  is smooth and its influence on the limit load factor is negligible based on expertise introduced in \cite{SCKKZB15}. We set  to have the associative model. 

Further, we use the  elements (i.e. eight-noded quadrilaterals) with  integration quadrature and the mesh with 37265 nodal points including the midpoints and with 110592 integration points. The mesh has a similar scheme as in  Figure \ref{fig.mesh_Q2} but, of course, it is much more finer. Since the Matlab code is vectorized, we fix 10 inner Newton's iterations for finding the unknown plastic multipliers in each integration point.

Recall that in each step of the direct method, we solve problem  using ALG-. We set the initial load increment .  If  ALG- converges during 50 iterations for step  and if the computed increment of the settlement satisfies m  then we set . Otherwise, the increment is divided by two. Within the indirect method where  problem  is solved using ALG- we set the initial increment  of the settlement to have comparable results with the direct method. If the computed load increment satisfies  then we set . Otherwise, . The loading process is terminated when the computed settlement exceeds 4 meters for both methods. 

\begin{figure}[htbp]
\begin{minipage}[t]{0.47\textwidth}
  \center
  \includegraphics[width=\textwidth]{load_path_level4.jpg}
   \caption{\small{Load path for the direct method.}}
   \label{fig.load_path}
\end{minipage}
\hfill
\begin{minipage}[t]{0.47\textwidth}
  \center
   \includegraphics[width=\textwidth]{load_path_level4_A-control.jpg}
   \caption{\small{Load path for the indirect method.}}
   \label{fig.load_path_A-control}
\end{minipage}
\end{figure}

\begin{figure}[htbp]
\begin{minipage}[t]{0.47\textwidth}
  \center
  \includegraphics[width=\textwidth]{iterations_level4.jpg}
   \caption{\small{Number of iterations for ALG-.}}
   \label{fig.iterations}
\end{minipage}
\hfill
\begin{minipage}[t]{0.47\textwidth}
  \center
   \includegraphics[width=\textwidth]{iterations_level4_A-control.jpg}
   \caption{\small{Number of iterations for ALG-.}}
   \label{fig.iterations_A-control}
\end{minipage}
\end{figure}

\begin{figure}[htbp]
\begin{minipage}[t]{0.47\textwidth}
  \center
  \includegraphics[width=\textwidth]{convergence_level4.jpg}
   \caption{\small{Convergence at selected steps for the direct method.}}
   \label{fig.convergence}
\end{minipage}
\hfill
\begin{minipage}[t]{0.47\textwidth}
  \center
   \includegraphics[width=\textwidth]{convergence_level4_A-control.jpg}
   \caption{\small{Convergence at selected steps for the indirect method.}}
   \label{fig.convergence_A-control}
\end{minipage}
\end{figure}

The comparison of the direct and indirect methods is depicted in Figures \ref{fig.load_path}-\ref{fig.convergence_A-control}. The resulting loading paths practically coincide for both methods and they are in accordance with \cite{NPO08, SCKKZB15, HRS16b}. The computed limit value is equal to 4.057 which is close to the estimate 4.045 known from \cite{CL90}. Both methods need 18 load step and have superlinear convergence in each step. Their convergence is similar up to step 11. However, other comparisons turn out that the indirect method behaves better than the direct one. First, the indirect method has less number of iterations between steps 12 and 18. Secondly, the direct method contained 8 additional load steps without successful convergence while the indirect one convergences in each step. The successful load steps for both methods are depicted  by the circular points in Figures \ref{fig.load_path} and \ref{fig.load_path_A-control}, respectively. We see that the positions of these points are more convenient in Figure \ref{fig.load_path_A-control} than in Figure \ref{fig.load_path} with respect to the curvature of the loading path. Thirdly, we see in Figure \ref{fig.convergence} that the convergence in steps 11 and 16 is superlinear only up to 1e-10. Then, values of the stopping criterion oscillate. This is also observed for a few other steps of the direct method (e.g., steps 14 and 15). For the indirect method, this is not observed at any step. Finally, the computational times of the direct and indirect methods on a current laptop were approximately 9 and 7 minutes, respectively.


\subsection{Associative perfect plastic 3D problem}

Within the 3D slope stability experiment (code  SS-MC-NP-3D), we compare the loading paths for the Q1 and Q2 hexahedral elements with 8 and 20 nodes, respectively. We consider  and  noded integration quadratures for these element types, respectively. Two hexahedral meshes are prepared for this experiment. For the Q1 elements, the meshes contain 5103 and 37597 nodal points, 34560 and 276480 integration points, respectively. For the Q2 elements, the meshes contain 19581 and 147257 nodal points, 116640 and 933120 integration points, respectively. We use the direct method of the incremental limit analysis which is terminated when the computed settlement exceeds 5 meters.

\begin{figure}[htbp]
\center
  \includegraphics[width=0.5\textwidth]{load_path_h_3D.jpg}
   \caption{\small{Comparison of the loading paths for  and  elements.}}
   \label{fig.load_path_3D}
\end{figure}

The corresponding loading paths are depicted in Figure \ref{fig.load_path_3D}. We observe that the estimated limit values of  are close to the expected value of 4.045 for the  elements but not for the  elements. To estimate  using the  elements, it would be necessary to use much finer meshes. Figures \ref{fig.displacement} and \ref{fig.multiplier} illustrate failure at the end of the loading process for the Q2 elements and the finer mesh. 

\begin{figure}[htbp]
\begin{minipage}[t]{0.47\textwidth}
  \center
  \includegraphics[width=\textwidth]{total_displacement_3D.jpg}
   \caption{\small{Total displacement and deformed shape at the end of the loading process.}}
   \label{fig.displacement}
\end{minipage}
\hfill
\begin{minipage}[t]{0.47\textwidth}
  \center
   \includegraphics[width=\textwidth]{plastic_multiplier_3D.jpg}
   \caption{\small{Plastic multipliers at the end of the loading process.}}
   \label{fig.multiplier}
\end{minipage}
\end{figure}




\section{Conclusion}
\label{sec_conclusion}

This paper extended the subdifferential-based constitutive solution technique proposed in \cite{SCKKZB15} to elastoplastic models containing the Mohr-Coulomb yield criterion. It enabled deeper analysis of the constitutive problem discretized by the implicit Euler method and consequently led to several improvements within solution schemes. For example, a priori decision criteria characterizing each type of the return-mapping were derived even if the solution could not be found in closed form. The construction of the consistent tangent operator was also simplified. Moreover, the paper brought self-contained derivation of the constitutive operators which is not too often in Mohr-Coulomb plasticity.

The improved constitutive solution schemes were implemented within slope stability problems in 2D and 3D. To this end, the direct and also the indirect methods of the incremental limit analysis were used and combined with the semismooth Newton method. Local superlinear convergence in each step of both methods was observed. Further, it was illustrated that the indirect method led to more stable control of the loading process or that higher order finite elements reduced strong dependence on mesh. 

\section*{Acknowledgements}
The authors would like to thank to Pavel Mar\v s\'alek for generating the quadrilateral meshes with and without midpoints in 2D and 3D.
This work was supported by The Ministry of Education, Youth and Sports (of the Czech Republic)
from the National Programme of Sustainability (NPU II), project
``IT4Innovations excellence in science - LQ1602". 




\section*{Appendix}
\label{sec_appendix}

\subsection*{A. Simplified constitutive handling for the plane strain problem}

The results of Section \ref{sec_time_discret} and \ref{sec.Stress-strain_relation} are, of course, valid also for the plane strain problem. Nevertheless, in this case, one can simplify the forms of eigenprojections and their derivatives since we work only on the subspace  of  containing trial tensors in the form

To distinguish the derivatives of functions defined in , we use the symbol  instead of .  Define the functions

in . Then , , are the eigenvalues of . These values are not ordered in general. We only know that . Further, define

2mm]
\tilde{\mbf I}, & \tilde\eta_1=\tilde\eta_2
\end{array}
\right.,\quad \tilde{\mbf E}_2(\mbf\eta):=\tilde{\mbf I}-\tilde{\mbf E}_1(\mbf\eta)\tilde{\mathbb E}_1(\mbf\eta):=\left\{
\begin{array}{c l}
\frac{1}{\tilde\eta_1-\tilde\eta_2}[\tilde{\mathbb I}-\tilde{\mbf E}_1\otimes\tilde{\mbf E}_1-\tilde{\mbf E}_2\otimes\tilde{\mbf E}_2], & \tilde\eta_1>\tilde\eta_2\
where  denotes the zeroth fourth order tensor and , , otherwise . Clearly, . If  then

It is worth mentioning that these formulas need not hold in  in general. Similar formulas are also introduced in \cite[Appendix A]{NPO08}.

Now, it is necessary to reorder the eigenvalues of . Denote the ordered eigenvalues as , i.e.,  and . Consequently, we reorder the functions , , , , leading to the functions , , , . To complete the notation, one can easily set 


Finally, one can straightforwardly use the functions , , , ,  and  within Section \ref{sec.Stress-strain_relation} when the plane strain assumptions are considered.


\subsection*{B. Algebraic representation of second and fourth order tensors}

Within our implementation, we use the standard algebraic representation of stress and strain second order tensors specified below but a little bit different representation of fourth order tensors in comparison to \cite[Appendix D]{NPO08}. We assume that a fourth order tensor  represents a linear mapping from  into . Therefore, the components   of  satisfy

The choice  implies that  for any  and

Notice that in \cite[Appendix D]{NPO08}, the stronger assumptions on the components are required: .

We distinguish two cases: the 3D problem and its plane strain reduction.

\subsubsection*{The 3D problem}

Let  denote stress and strain tensors, respectively. Then they are represented by vectors  and  where  and  are the components of , and , respectively. Clearly, . A fourth order tensor  is represented by matrix . Since fourth order tensors are applied on strain tensors within the implementation, we require that

holds for any strain tensors  and . Here,  and  denote the algebraic counterparts of  and , respectively. From (\ref{C_sym}) and (\ref{C_algebra}), one can derive that
1mm]
C_{2211} & C_{2222} & C_{2233} & \frac{1}{2}[C_{2212}+C_{2221}] & \frac{1}{2}[C_{2223}+C_{2232}] & \frac{1}{2}[C_{2213}+C_{2231}]\1mm]
C_{1211} & C_{1222} & C_{1233} & \frac{1}{2}[C_{1212}+C_{1221}] & \frac{1}{2}[C_{1223}+C_{1232}] & \frac{1}{2}[C_{1213}+C_{1231}]\1mm]
C_{1311} & C_{1322} & C_{1333} & \frac{1}{2}[C_{1312}+C_{1321}] & \frac{1}{2}[C_{1323}+C_{1332}] & \frac{1}{2}[C_{1313}+C_{1331}]
\end{array}
\right).[\mbf C]_{45}=\mathbf n\cdot \mbf C\mathbf e\stackrel{(\ref{C_algebra})}{=}\eta:\mathbb C:\varepsilon=\frac{1}{4}[C_{1223}+C_{1232}+C_{2123}+C_{2132}]\stackrel{(\ref{C_sym})}{=}\frac{1}{2}[C_{1223}+C_{1232}].\mbf C=\left(
\begin{array}{c c c c c c}
2\eta_{11} & 0 & 0 & \eta_{12} & 0 & \eta_{13}\1mm]
0 & 0 & 2\eta_{33} & 0 & \eta_{23} & \eta_{13}\1mm]
0 & \eta_{23} & \eta_{23} & \frac{1}{2}\eta_{13} & \frac{1}{2}[\eta_{22}+\eta_{33}] & \frac{1}{2}\eta_{12}\
\end{enumerate}

\subsubsection*{The plane strain problem}

Let  and  denote stress and strain second order tensors, respectively. Then they are represented by the vectors  and  where  and  are components of , and , respectively. Clearly, . Notice that the component  vanishes for the strain tensor but not for the plastic strain tensor.

The fourth order tensor  can be represented by matrix . Similarly as for the 3D problem, one can derive that
1mm]
C_{2211} & C_{2222} & \frac{1}{2}[C_{2212}+C_{2221}] & C_{2233} \1mm]
C_{3311} & C_{3322} & \frac{1}{2}[C_{3312}+C_{3321}] & C_{3333} 
\end{array}
\right).
Finally, it is worth mentioning that for assembling the tangent stiffness matrix, it is sufficient to save only the components  where .


\bibliographystyle{wileyj}
\bibliography{MC_SysCer}




\end{document}
