\documentclass[runningheads,a4paper]{llncs}
\usepackage{amssymb}
\usepackage{amsmath}
\setcounter{tocdepth}{3}
\usepackage{graphicx}
\usepackage{url}
\usepackage{subfigure}
\usepackage[ruled,linesnumbered]{algorithm2e}
\usepackage[english]{babel}
\urldef{\mailsa}\path|{jatin.agarwal, nadeem.moidu}@research.iiit.ac.in|
\urldef{\mailsb}\path|{kkishore, srinathan}@iiit.ac.in|

\newcommand{\keywords}[1]{\par\addvspace\baselineskip
\noindent\keywordname\enspace\ignorespaces#1}
\begin{document}

\mainmatter  

\title{Efficient Range Reporting of Convex Hull}

\titlerunning{Efficient Range Reporting of Convex Hull}

\author{Jatin Agarwal \and Nadeem Moidu \and Kishore Kothapalli \and Kannan Srinathan}

\authorrunning{J Agarwal, N Moidu, K Kothapalli, K Srinathan}


\institute{International Institute of Information Technology, Hyderabad, India\\
\mailsa\\
\mailsb\\}
\maketitle

\begin{abstract}

We consider the problem of reporting convex hull points in an orthogonal range query in two dimensions.
Formally, let  be a set of  points in .
A point lies on the convex hull of a  point set   if it lies on the
boundary of the minimum convex polygon formed by . In this paper, we
are interested in finding the points that lie on the boundary of the
convex hull of the points in  that also fall with in an orthogonal range
. 
We propose a  space data structure that can support reporting points on a convex hull inside
an orthogonal range query, in time . Here  is the size of the output.
This work improves the result of (Brass et al. 2013) \cite{brass} that
builds a data structure that uses  space and has a 
 query time. 
Additionally, we show that our data structure can be modified slightly to
solve other related problems. For instance, for counting the number of
points on the convex hull in an orthogonal query rectangle, we propose an
 space data structure that can be queried upon in  time. 
We also propose a  space data structure that can compute the 
and  of the convex hull inside an orthogonal range query in ) time.

\keywords{ Convex Hull, Plane, Orthogonal range, Reporting, Counting, Area, Prerimeter }
\end{abstract}


\section{Introduction}

\noindent Range searching is one of the most commonly studied topics in
spatial databases and computational geometry.  Informally, range searching can
be stated as follows: Given a point set  we wish to pre-process  into a
data structure such that given any rectangular query region , we can
efficiently report the points in  or count their number in . Range aggregate geometric querying is a variant of range searching where
we calculate a geometrical function  inside . In this work we
efficiently compute  where the function  is to compute the
convex hull for the point set . Range aggregate querying has been
studied recently by Kalavagatttu et al.\cite{Anil}, Das et al. \cite{asdas}
and Brodal et al. \cite{brodal} for maximal points.


However, advances in the creation and archival of digital information have led
to an information explosion and therefore the number of objects inside a query
range can be huge. In such cases, reporting a sample of the result set is
preferred. In this paper, we use the concept of convex hull query for
reporting the boundary points. Reporting of convex points can be useful in
situations where we need a shape with minimum area/perimeter for a set of
points. Therefore, it may have applications in spatial databases \cite{Tao}, computer graphics and computer
vision.

In this work, we propose a data structure to solve the problem of reporting
convex hull points in an orthogonal query rectangle.  We also study the
problem of counting the number of points on the convex hull inside ,
and computing its area/perimeter.
\subsection{Problem Definitions}
\label{def}
In this section, we formally define the problems and related terminology.
Given a set of  points  in
. We use  to represent the  co-ordinate of a point
 and  to represent its  co-ordinate.
The convex hull of any point set  
in the Euclidean plane is the smallest convex polygon that contains .
Henceforth, when we refer to convex hull points,
we essentially mean the points on the boundary of the convex hull.
We are given a set  of  points in  and a query .
We find the convex hull on the point set , denoted by . We also denote points with minimum 
co-ordinate, maximum  co-ordinate, minimum  co-ordinate and maximum  co-ordinate of point set 
by , ,  and  respectively. 

In this work, we study the following problems:
 We wish to pre-process 
into a data structure such that given an orthogonal query region , we can efficiently

{\em Problem 1:} {\bf report} the points on convex hull of .

{\em Problem 2:} {\bf count} the number of points on the convex hull of .

{\em Problem 3:} find the {\bf area} of the convex hull of .

{\em Problem 4:} find the {\bf perimeter} of the convex hull of .
\subsection{Previous Work} \label{prev-work} The convex hull for a static data
set of two-dimensional points can be computed in  time
\cite{opchan} where  is the number of points on the convex boundary.
Gupta et al. discuss non-intersecting queries on aggregated geometric data \cite{gupta}.
Brass et al. gave the first solution for {\em Problem 1} that takes  space,  preprocessing time and  query
time \cite{brass}.  For any given orthogonal range query on a standard
2d-range tree they identify  disjoint canonical convex
hulls. There are  pairs of disjoint convex hulls and they
compute the tangent for each pair of disjoint convex hulls.  Computing
tangents between each pair of local convex hulls takes  time \cite{kirkpatrick}. They
use a method similar to the gift-wrapping algorithm.  Therefore, total query
time is .
\subsection{Our Results} \label{our-results}
The following table summarizes the results presented in this paper.
\begin{center}
    \begin{tabular}{ | l | l | l | l | p{2cm} |}
    \hline
    \textbf{Query Type} & \textbf{Query Time} & \textbf{Preprocessing Time} &  \textbf{Storage Space} & \textbf{Theorems} \\ \hline
    Reporting &  &  &  & Theorem 1 \\ \hline
    Counting &  &  &  & Theorem 2 \\ \hline
    Area &  &  &  & Theorem 3 \\ \hline
    Perimeter &  &  &  & Theorem 4 \\ \hline
    \hline
    \end{tabular}
\end{center}
The rest of the paper is organized as follows. In Section \ref{Basic idea}, we
give the details of the preprocessing and the query algorithm for reporting
convex points inside an orthogonal range query. In Section \ref{other} we
study the problem of counting convex hull points inside  and the
problem of computing the area/perimeter of the convex hull inside . We discuss future work and conclude in Section \ref{conclusion}.
\section{Our Solution}\label{Basic idea}
In this section we explain our solution to the problem of reporting the convex hull points inside .
In Section \ref{preprocessing} we describe how to construct the required data structure of 
size  in time  on a static point set .
In Section \ref{query-al} we explain how to report points on the convex hull inside  for a monotonic chain from  to .
\begin{figure}
\vspace{-0.5cm}
\centering
\includegraphics[scale=1.0]{hull4.pdf}
\caption{() Convex hull inside an orthogonal range query (the shaded region is
  the query). () Convex hull with four monotone chains}
\label{fig1}
\vspace{-0.5cm}
\end{figure}
Points on the convex hull can be broken into four monotone chains, namely
maximal chain from  to , maxY-minX chain from 
to , minimal chain from  to  and
minY-maxX chain from  to  as shown in Figure \ref{fig1}().
Area of such a convex hull gets divided into four quadrants Q1, Q2, Q3 and Q4 as shown in Figure \ref{fig1}.
In this work we present an algorithm to construct the maximal chain from
 to .  A similar approach can be applied for the other
monotone chains.

\subsection{Preprocessing}\label{preprocessing}
Our solution uses a -range tree as described in \cite{berg}.  Each
internal node of the primary tree  represents a horizontal range
 for .  The set of points rooted at an internal node
 are represented by . For each internal node  of  we have a
secondary(binary) tree  such that leaves of tree  store
the points in  in non-decreasing order of their -coordinate.  Each
internal node of the secondary tree  represents a vertical range
 for .  Given a query , we search  and  in the primary tree to get canonical nodes
.
as shown in Figure \ref{fig3}. Here  is the height of the primary
tree. Therefore, the range query  is divided into  canonical subsets .
\begin{figure}
\vspace{-0.7cm}
\centering
\includegraphics[scale=0.30]{srt.pdf}
\caption{(.)Range tree with the canonical nodes highlighted (.) Processing an orthogonal range query}
\label{fig3}
\vspace{-0.7cm}
\end{figure}
Similarly we search for  and  in the secondary tree
 for  to get canonical nodes

Here  is the height of the secondary
tree. Therefore, in every secondary tree, the range query  gets
divided into  canonical subsets , ,, .
For each internal node  of the secondary tree, we compute the convex hull over the set .
We store points of this convex hull in an array 
in the anti-clockwise direction starting from the point with the maximum -coordinate. 
We also find the point  with maximum  co-ordinate from the set  and store it at each internal node .
Therefore, any given query  gets divided into  disjoint canonical subsets and 
for each of these subsets we have stored a convex hull  on the set .

It takes  time to compute convex hull  over set
 and  space to store  at internal node  of
secondary tree . Therefore it takes
 preprocessing time and
 space for secondary tree . For any given query
 on the primary tree  we have  secondary
trees. Therefore it takes  preprocessing time
and  storage space.

\begin{lemma}\label{lem1}
  Given a set  of  points in , we can pre-process 
  into a data structure that takes  storage space and  pre-processing time (as explained above).
\end{lemma}

\subsection{Query Algorithm}\label{query-al}In this algorithm we use two stacks, a hull stack  and a tangent stack . An element of hull stack
 is a pointer to some canonical convex hull . An element of tangent stack  is a tuple
 where  and  are indices of points on two different convex hulls  and
 as shown in Figure \ref{fig4}.
Given an orthogonal range query , the query algorithm
for reporting convex hull points in  is as follows:
\begin{enumerate}
\item 
Express the range of  co-ordinates in  as the
disjoint union of  =  canonical subsets. Let the canonical
nodes be  from left to right in that order, as
shown in Figure \ref{fig3}().

\item\label{intial-step} Consider a node  of the primary tree
  . Make a range query  on the associated secondary
  tree  to find canonical nodes , ,
  . Do a linear search on , ,
  ,  to find point  with maximum 
  co-ordinate. Then traverse the canonical nodes from right to left, starting
  from  back to , as shown in Figure \ref{fig3}().
Initialize  and .

\item\label{repeat-step}
Consider a node  of the primary tree
  . For each internal node  of  there is an associated
  secondary tree . Make a range query  on
  secondary tree  to find canonical nodes , as shown in the Figure \ref{fig3}() where
  .

\item\label{secondtree}
Consider the array . If  the array  is empty then go to step \ref{primarytree},
otherwise set  and then traverse the canonical nodes from bottom to top,
starting from  back to  as shown in Figure \ref{fig3}() as follows:\\

{\em for}  to  call Algorithm Merge(,,) (see Section 2.2.1).\\

\item\label{primarytree}
At this point, we have processed the nodes {}. 
Set  and if , move to the node  and
goto step \ref{repeat-step}, else exit.

\item
Call Algorithm Report(,)(see Section 2.2.2) to report the points on the convex monotone from 
maximum  to maximum  inside .\end{enumerate}
\subsubsection{2.2.1  Merge Algorithm}\label{merge-al}
\paragraph{}
In this section we explain the Algorithm \emph{Merge()} used for merging canonical
convex hulls for any given query.  Note that the algorithm proposed is similar
to the Graham Scan algorithm \cite{graham} where we combine points which are
sorted on  co-ordinate.
Instead of points we have disjoint convex hulls sorted on non-increasing  co-ordinate in the stack .
\begin{algorithm}\label{merging}
\KwData{, , }
\KwResult{Updated Stacks  and  }
\While{}{
;;\;
find points  and  on tangent =(,) joining hulls  and \;
find points  and  on tangent =(,) joining hulls  and \;
\eIf{ }{\tcc{(refer Figure \ref{fig4}())}
 pop(); pop()\;
}{\tcc{(refer Figure \ref{fig4}())}
push();
break\;}
}
push  onto stack \;
\caption{\em Merge()}
\end{algorithm}
In line 1 of the algorithm, a while loop checks whether hull stack  has
sufficient elements to continue.  In line 2 we store hulls at ,
 and  in variables ,  and .  In line
3 we find points  and  incident with tangent =(,) between hulls
 and  as shown in the Figure \ref{fig4}.  Computing such a tangent takes 
points where  and  are sizes of the convex
hulls.  Similarly in line 4 we find points  and
 incident with tangent =(,)
between hulls  and  as shown in the Figure \ref{fig4}.  In line 5 we compute the orientation of
the points ,  and  using function (a primitive opeation in computational Geometry). If the orientation of the points is
clockwise (negative) as shown in Figure \ref{fig4}(b) then we pop out an element from the stack  and
the stack  in line 6, else we push tangent  into stack  and break out of the loop in line 8.
\begin{figure}
\vspace{-0.5cm}
\centering
\includegraphics[scale=0.30]{merging233.pdf}
\caption{()Positive Orientation () Negative Orientation}
\label{fig4}
\vspace{-0.7cm}
\end{figure}
Every time an element gets pushed onto the stack we iterate the above
procedure on the top two elements of stack  and  to discard canonical set of
convex hulls  that do not contribute points to .
For every call to Algorithm \emph{Merge()} there
can be zero or more pop() operations on stack . An element once popped out
of the stack  never gets pushed in again. Therefore, the total number of
push and pop operations is . For each pop and push operation on
the stack , the required tangent can be computed in  time.
Finding the tangent between any two canonical convex hulls takes 
time \cite{kirkpatrick}.  Therefore, the merging algorithm takes no more than
 time where  is size of the stack. We end this section with the following lemma.

\begin{lemma}\label{lem2}
Given the stack , the stack  and the hull  as input to {\bf Algorithm Merge()}, it takes 
 time where  is size of the stack .
\end{lemma}

\subsubsection{2.2.2  Reporting algorithm}\label{putting}
In this section we explain the algorithm \emph{Report()} used for reporting
points on the convex hull.  This reporting algorithm is the last step of the
query algorithm.
\begin{algorithm}\label{reporting}
\KwData{Stack ,Stack }
\KwResult{Convex hull points in  }
);
index \;
Report the point  with maximum  co-ordinate\;
\While{}{
tangent )\;
index )\;
Report points from  to \;
)\;
array ; 
}
Report points from  ch[] till \ in array \;
\caption{Report()}
\end{algorithm}
\vspace{-0.7cm}
We report points from canonical convex hulls stored in
stack  using information about tangents stored in the stack .
In line  of the algorithm we pop out a hull from the stack  and index 
is assigned with the size of array . In line  we report the point  with maximum  co-ordinate.
In line  we check the while loop condition, if the stack is not empty then we enter the while loop.  In line 4
we pop out tangent  from the stack  and in line 5 we assign  with
the first index  of the tangent.  In line 6 we report points from
ch[] to ch[].  In line 7 we assign  with the second index  of
tangent  and in line 8 we pop out a canonical convex hull from the stack .

In the algorithm \emph{Report()} we iterate till the stack  is empty and in each iteration of the while loop
we report a set of points. Let total number points reported by Algorithm \emph{Report()} be .
It takes  time for the algorithm \emph{Report()} because  while loop gets iterated  times.
Therefore, it takes   time for algorithm \emph{Report()}.
We end this section with the following lemma.
\begin{lemma}\label{lem3}
Given the stack  and the stack  as input to {\bf Algorithm Report()}, it takes 
 time where  is size of the stack  and  is the number of points
reported.
\end{lemma}

\subsection{Putting Everything Together}
We will now prove the following theorem:
\begin{theorem}
  Given a set  of  points in , we can pre-process 
  into a data structure of size  in time 
  such that, given an orthogonal range query , we can report the points of the convex hull inside  in time , where  is the number of convex hull
  points reported.
\end{theorem}
 : Lemma \ref{lem1} for preprocessing indicates the storage space
used by our data structure and the preprocessing time to build it.  Now we analyze the
complexity of our query algorithm explained in Section \ref{query-al}.  In
step 1 it takes  time to find  canonical nodes
because the height of primary tree  is .  In step 2 it takes
 time to find canonical subset of nodes ,
, . It takes  time to perform a linear
search on , , ,  to
find point  with maximum  co-ordinate.  In step 3 we consider
each  from  to 1. In  iteration of step 3 we
spend  time finding the canonical subset of nodes , ,  by making an updated query on the associated
secondary tree . Therefore, the total time spent in finding all
canonical sets of nodes for any given query is  
for  then total time is . In step 4 at
most  calls are made to Algorithm \emph{Merge()}. Therefore total calls made to algorithm \emph{Merge()} is .
From Lemma \ref{lem2} we know that time taken for Algorithm \emph{Merge()} is 
Therefore, it takes  time to find updated  that contains convex hulls that contribute
at least one point to .
In step 6 of the query algorithm a call to the algorithm Report(,) is made.
According to Lemma \ref{lem3} it takes  time where  is size of the stack  and  is the number of points
reported. Therefore, it takes  time for step 6 to execute where . Recall that time taken for step  is .
Therefore it takes  to report the points of  in the first quadrant Q1 (see Figure \ref{fig1}).

\section{Related Problems}\label{other}
In this section we study related problems such as counting the number of
convex hull points and finding the area/perimeter of the convex hull inside .  One may argue that it is not necessary to study these problems
separately once we have reported the points of  in query time
 using the algorithm described in Section \ref{query-al}
where  is the total number of points on the convex hull. However,  can
be as large as  for some queries.
If these problems can be solved independent of  then the running time is
unaffected by output size.  With a few clever modifications on the data
structure explained in Section \ref{preprocessing} and the algorithm proposed
in Section \ref{query-al} we can achieve results that are independent of
output size for the above mentioned problems.

\subsection{The Counting Problem}\label{Count}
Most of the preprocessing of the point set  for the counting problem is the
same as the preprocessing described in Section \ref{preprocessing}. In
addition to it, at each internal node  of every secondary tree 
we store three indices ,  and  where
,  and  are
points with maximum  co-ordinate, minimum  coordinate and minimum 
co-ordinate.  To get the stack  which contains all the canonical convex
hulls that contribute to convex hull  we use same Query Algorithm
of Section \ref{query-al}. Below is the algorithm for counting the number of
points on the convex hull of set . To count the points in , 
algorithm counting() can be used as subroutine in step 6 of the query algorithm.
\begin{algorithm}
 \KwData{stack ,stack }
 \KwResult{ of the points of  from maximum  to maximum  }
    array );
    index \;
    \;
  \While{}{
    tangent \;
    index \;
    \;
    \;
    array ); 
   }
     \;
 \caption{Counting()}
\end{algorithm}
\vspace{-0.7cm}
Algorithm \emph{Counting()} is similar in spirit to the
Algorithm {\emph{Report()} of Section 2.2.2. In
Step 2 of this algorithm  is initialized to  because we count all the
points from point  till the point with maximum  co-ordinate in . In Step
6 the variable  gets updated with the number of points that the current
canonical hull contributes to the output. This is computed as the difference
 in each iteration of while loop. This process is repeated until the
stack  is empty.  In step 10  gets updated with the difference
 after exiting from the while loop, where 
is the index of point with the maximum  co-ordinate in the array .

It can be seen that Algorithm \emph{Counting()} runs in  query time to count points of the convex hull 
from maximum  to maximum . Similar counting algorithms can be used to
find the count of points of the convex hull  for the other three
monotone chains (see also Figure \ref{fig1}).
\begin{theorem}
Given a set  of  points in , we can pre-process 
into a data structure of size  in time 
such that, given an orthogonal range query , we can count the points of the convex hull inside  in time .
\end{theorem}

\subsection{The Perimeter Problem}
For computing the perimeter of the convex hull of the points in  on a
two-dimensional point set  and an orthogonal range , we perform a
preprocessing phase that is similar to the one described in Section
\ref{preprocessing}. In addition, at every internal node  of every
secondary tree, we store an auxiliary array  that stores the cumulative perimeter.
 where
 is the distance between points  and 
respectively.  Below Algorithm  {\em Perimeter()} used for finding the perimeter of the convex
hull  from maximum  to maximum .
\vspace{-0.7cm}
\begin{algorithm}
 \KwData{,}
 \KwResult{ of  from maximum  to maximum }
  convex hull );
  index \;
   \;
 \While{}{
   tangent \;
   index \;
   point \;
   \;
   \;
   \; 
   point \;
   \;
 }
  \;
  \caption{Perimeter()}
\end{algorithm}
\vspace{-0.7cm}
Algorithm {\em Perimeter()} is similar in spirit to the reporting
algorithm (Algorithm Report) of Section \ref{Count}.
In Step 2 of this algorithm, the variable  is initialized to
. In Step 6 we store point  of some hull  into point .  In
Step 7 the variable  is updated with difference 
in each iteration of while loop. This process is repeated until the stack 
is empty.
In Step 13 the variable  is updated with difference
 after exiting from while loop where
 is index of point with the maximum  co-ordinate in array
. All other steps are similar to counting algorithm in section \ref{Count}

It can be noticed that the Algorithm Perimeter runs in  time to find perimeter of the convex hull  from
maximum  to maximum . A similar algorithm can be used to find the
perimeter of the convex hull  for other three monotone
chains. (See also Figure \ref{fig1}.)

\begin{theorem}
  Given a set  of  points in , we can pre-process 
  into a data structure of size  in time 
  such that, given an orthogonal range query , we can compute the perimeter of the convex hull inside  in time .
\end{theorem}

\subsection{The Area Problem}
During the preprocessing phase, an auxiliary array  is stored at each
internal node of every secondary tree.  Each element of such an array stores
the cumulative area  where  is the area of the triangle between points
,  and  respectively.   because
it represents the area of point  and  because it is the area
of the line joining points  and . Below is an algorithm for computing the area of the convex hull  .
\begin{algorithm}
 \KwData{Stacks ,,}
 \KwResult{Area of the convex hull of Q1 in Figure \ref{fig1}}
  \;
  point 
  array );
  index \;
   onto \;
  \;
 \While{}{
   tangent )\;
   index )\;
    onto \;
   \;
   )\;
    onto \;
    ch=pop(); 
 }
  \;
   onto \;
  \;
  \;
  point \;
  \While{}{
  point \;
  \;
  \;
  }
\caption{Area()}\end{algorithm}
\vspace{-0.6cm}
The  algorithm {\em Area()} finds the area of quadrant Q1 (Figure
\ref{fig1}). Similarly we can find areas of the other quadrants Q2,Q3 and
Q4. The area of  can be obtained using the following formula.\\
Area(ch() =  where , ,  and 
are the points with minimum , maximum , minimum  and maximum  in .
\begin{theorem}
  Given a set  of  points in , we can pre-process 
  into a data structure of size  in time 
  such that, given an orthogonal range query , we can compute the area of the convex hull inside  in time .
\end{theorem}

\section{Conclusion}\label{conclusion}
In this work, we studied the problem of reporting convex hull points for any
given orthogonal range query, in the \emph{Pointer Machine Model}. We also
solved the problem of counting, area and perimeter in  time.
We restricted the point set to static two-dimensional points. It will be
interesting to see these problems in higher dimensions and dynamic versions of
the problems.  It will be also interesting to study these problems related to convex
hull in the \emph{Multi-shot Model}. It will be interesting to study these problems
in {word-RAM} model and \emph{Cell-probe Model}.

\begin{thebibliography}{8}

\bibitem{opchan} T.\ M.\ Chan.
\emph{Optimal output-sensitive convex hull algorithms in two and three dimensions.}
Discrete and Computational Geometry, April 1996, Volume 16, Issue 4, pp 361-368.

\bibitem{Anil} A.\ K.\ Kalavagattu, J.\ Agarwal, A.\ S.\ Das,  K.\ Kothapalli.
\emph{On Counting Range Maxima Points in Plane}
In Proceedings of 23rd International Workshop on Combinatorial Algorithms (IWOCA 2012)
LNCS Volume 7643, 2012, pp 263-273.

\bibitem{asdas} A.\ S.\ Das, P.\ Gupta, A.\ K.\ Kalavagattu, K.\ Srinathan, K.\ Kothapalli, J.\ Agarwal
\emph{Range Aggregate Maximal Points in the Plane.}
In Proceedings of Workshop on Algorithms and Computation(WALCOM 2012)LNCS Volume 7157, 2012, pp 52-63.

\bibitem{brodal} G.\ S.\ Brodal, K.\ Tsakalidis. 
\emph{Dynamic Planar Range Maxima Queries.} 
In Proceedings of International Colloquium on Automata, Languages and Programming (ICALP 2011),
Springer Verlag LNCS, volume 6755, pp.\ 256--267.

\bibitem{brass} P.\ Brass, C.\ Knauer, C.\ S.\ Shin, M.\ Schmid, I.\ Vigan.
\emph{Range-Aggregate Queries for Geometric Extent Problems}
Australasian Theory Symposium (CATS 2013).

\bibitem{berg} M.\ de.\ Berg, M.\ van\ Kreveld, M.\ Overmars, O.\ Schwarzkopf.
\emph{Computational Geometry: Algorithms and Applications.}
ISBN 3-540-65620-0, Springer Verlag, 2000

\bibitem{kirkpatrick} D.\ Kirkpatrick, J.\ Snoeyink.
\emph{Computing common tangents without a separating line}
Algorithms and Data Structures LNCS Volume 955, 1995, pp 183-193 

\bibitem{graham} R.\ L.\ Graham.
\emph{An Efficient Algorithm for Determining the Convex Hull of a Finite Planar Set}
 Information Processing Letters 1, 132-133, 1972.

\bibitem{gupta} P.\ Gupta, R.\ Janardan \& Smid, M. 
\emph{Efficient non-intersection queries on aggregated geometric data.}
Int.\ J.\ of Computational Geometry and Applications, 19(6), pp. 479-506.

\bibitem{Tao} Y.\ Tao \& D.\ Papadias(2004),\emph{Range aggregate
processing in spatial databases.} IEEE Trans. on
Knowledge and Data Engineering, 16, pp. 1555-1570.
\end{thebibliography}
\end{document}
