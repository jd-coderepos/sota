\begin{figure*}[!t]
\footnotesize
\centering
\pgfplotstabletypeset[
col sep=comma,
    string type,
    columns/BLOCK/.style={column name=\textsc{Block}, column type={|c}},
    columns/QUESTION/.style={column name=\textsc{Question}, column type={|l}},
    columns/OPTIONS/.style={column name=\textsc{Options}, column type={|l}},
    columns/EXCLUSIVE/.style={column name=\textsc{Exclusive}, column type={|c}},
    columns/ORDERED/.style={column name=\textsc{Ordered}, column type={|c}},
    columns/BRANCH/.style={column name=\textsc{Branch}, column type={|c|}},
    every head row/.style={before row=\hline,after row=\hline},
    every last row/.style={after row=\hline},
    every row no 3/.style={before row=\hline},
    every row no 6/.style={before row=\hhline{|=|=|=|=|=|=|}\rowcolor[gray]{0.9}},
    every row no 11/.style={before row=\hhline{|=|=|=|=|=|=|}},
    every row no 18/.style={before row=\hline\rowcolor[gray]{0.9}},
    every even row/.style={before row={\rowcolor[gray]{0.9}}},
]{example.csv}
\caption{An example survey written using the \surveyman{} domain-specific language, adapted from Ipeirotis~\cite{ipeirotis2010demographics}. For clarity, a horizontal line separates each question, and a double horizontal line separates distinct \emph{blocks}, which optionally define a partial order: all questions in block  appear in random order before questions in blocks . When blocks are not specified, all questions may appear in a random order. This relaxed ordering enables \surveyman{}'s error analyses.\label{fig:example}}
\end{figure*}

\subsection{Overview}

The \surveyman{} programming language is a tabular, lightweight
language aimed at end users. In particular, it is designed to both make
it easy for survey authors without programming experience to create
simple surveys, and let more advanced users create sophisticated
surveys. Because of its tabular format, \surveyman{} users can enter
their surveys directly in a spreadsheet application, such as Microsoft
Excel. Unlike text editors or IDEs, spreadsheet applications are tools
that our target audience knows well. \surveyman{} can read in
\texttt{.csv} files, which it then checks for validity
(\ref{sec:static-analysis}), compiles and deploys
(\ref{sec:runtime-system}), and reports results, including errors
(\ref{sec:dynamic-analysis}).

A key distinguishing feature of \surveyman{}'s language is its support
for randomization, including question order, question variants, and
answers. From \surveyman{}'s perspective, more randomization is
better, both because it makes error detection more effective and
because it provides experimental control for possible biases
(\ref{sec:dynamic-analysis}). \surveyman{} is designed so that
survey authors must go out of their way to state when randomization is
\emph{not} to be used. This approach encourages authors to avoid
imposing unnecessary ordering constraints.

\paragraph{Basic Surveys:}
To create a basic survey, a survey author simply lists questions and
possible answers in a sequence of rows. When the survey is deployed,
all questions are presented in a random order, and the order of all
answers is also randomized.

\paragraph{Ordering Constraints:}
Survey authors can assign numbers to questions that they wish
to order; we use the terminology of the survey literature and call these numbers
\emph{blocks}. Multiple questions can have the same number, which
causes them to appear in the same block. Every question in the same block will be
presented in a random order to each respondent. All questions inside
blocks of a lower number will be presented before questions inside
higher-numbered blocks.

\surveyman{}'s block construct has additional features that can give
advanced survey authors finer-grained control over ordering
(\ref{sec:advanced-blocks}).

\paragraph{Logic and Control Flow:}
\surveyman{} also includes both logic and control flow in the form of
branches depending on the survey taker's responses. Each answer can
contain a target branch (a block), which is taken if the survey taker
chooses that answer.



\punt{
The mix of control
flow, blocks, and randomization raises language design and
implementation challenges. The untrammeled use of branches could lead
to counterintuitive results. For instance, a respondent could
theoretically experience different paths through a survey (answering
different questions) simply as a result of randomizaton. By limiting
the expressiveness of branches, \surveyman{} rules out this
possibility.
}


\begin{table}[!t]
\centering
\begin{tabular}{>{\small}l|>{\small}l}
\textbf{Column} & \textbf{Description} \\
\hline
\textsc{Question}  & The text for a question \\
\textsc{Options}   & Answer choices, one per row \\
\hline
\textsc{Block}     & Numbers used to partially order questions \\
\textsc{Exclusive} & Only one choice allowed (default) \\
\textsc{Ordered}   & Present options in order \\
\textsc{Branch}    & For this response, go to this block \\
\textsc{Randomize} & Randomize option orders (default) \\
\textsc{Freetext}  & Allow text entry, optionally constrained \\
\textsc{Correlated} & Used to indicate questions are correlated \\
\end{tabular}
\caption{Columns in \surveyman{}. All except the first two (\textsc{Question} and \textsc{Options}) are optional.\label{tab:columns}}
\end{table}

\subsection{Syntax}

Figure~\ref{fig:example} presents a sample \surveyman{} program, a
survey from a Mechanical Turk demographic survey modified to
illustrate \surveyman{}'s features~\cite{ipeirotis2010demographics}.

Every \surveyman{} program contains a first row of column headers,
which indicate the contents of the following rows. The only mandatory
columns are \textsc{Question} and \textsc{Options}; all other columns
are optional. If a given column is not present, its default value is
used. The columns may appear in any order. 

\paragraph{Blocks.}
Each question can have an optional \textsc{Block} number associated
with it. Blocks establish a partial order. Questions with the same
block number may appear in any order, but must precede all questions
with a higher block number. If no block column is specified, all
questions are placed in the same block, meaning that they can appear
in any order.

\paragraph{Questions and Answers.}
The \textsc{Question} column contains the question text, which may
include HTML. Users may wish to specify multiple variants of the same question
to control for or detect question wording bias. To do this, users place
all of the variants in a particular block and have every question branch
to the same target.

The survey author specifies each question as a series of consecutive
rows. A row with the \textsc{Question} column filled in indicates the
end of the previous question and the start of a new one. All other
rows leave this column empty (as in Figure~\ref{fig:example}).

\textsc{Options} are the possible answers for a question. \surveyman{}
treats a row with an empty question text field as belonging to the
question above it. These may be rendered as radio buttons, checkboxes,
or freetext. If there is only one option listed and the cell is empty,
this question text is presented to the respondent as instructions.

\paragraph{Radio Button or Checkbox Questions.}
By default, all options are \textsc{Exclusive}: the respondent can
only choose one. These correspond to ``radio button'' questions. If
the user specifies a \textsc{Exclusive} column and fills it with
\texttt{false}, the respondent can choose multiple options, which are
displayed as ``checkbox'' questions.

\paragraph{Ordering.}
By default, options are unordered; this corresponds to nominal
data like a respondent's favorite food, where there is no ordering
relationship. Ordered options include so-called Likert scales, where
respondents rate their level of agreement with a statement; when the
options comprise a ranking (e.g., from 1 to 5); and when ordering is
necessary for navigation (e.g., for a long list of countries).

When options are unordered, they are presented to each respondent as one
of  possible permutations of the answers, where  is the
number of options. To order the options, the user must fill in a
\textsc{Ordered} column with the value \textsc{True}. When they are
ordered, they can still be randomized: they are either
presented in the forward or backwards order. The user can only force the
answers to appear in exactly the given order by also including a
\textsc{Randomize} column and filling in the value \textsc{False}.

Unordered options eliminate option order bias, because the
position of each option is randomized. They also make inattentive
respondents who always click on a particular option choice
indistinguishable from random respondents. This property lets
\surveyman{}'s quality control algorithm simultaneously identify both
inattentive and random responses.


\begin{figure*}[!t]
 \centering
 \begin{subfigure}[b]{0.3\textwidth}
                \includegraphics[width=\textwidth]{BranchDiagram.pdf}
                \caption{In this example of survey branches, the respondent is asked the branching question (Q3) as the second question. The execution of this branch is deferred until the respondent answers Q2, the last question in the block.}
                \label{fig:branch-diagram}
        \end{subfigure}\quad\quad
 \begin{subfigure}[b]{0.3\textwidth}
                \includegraphics[width=\textwidth]{BlockDiagram.pdf}
                \caption{Two different instances of a survey with top-level blocks (\texttt{1} and \texttt{2}), nested blocks (\texttt{1.1} and \texttt{1.2}), and a floating block (\texttt{1.B}). The floating block can move anywhere within block \texttt{1}, while the other blocks retain their relative orders (\ref{sec:advanced-blocks}).}
                \label{fig:block-diagram}
        \end{subfigure}\caption{Examples of branches and blocks.\label{fig:blocks-and-branches}}
\end{figure*}

\paragraph{Branches.}
The \textsc{Branch} column provides control flow through the survey
when the survey designer wants respondents who answer a particular
question one way to take one path through the survey, while
respondents who answer differently take a different path. 
For example, in the survey shown in Figure~\ref{fig:example}, only
when respondents answer that they live in the United States will they
be asked what state they live in.

During the survey, branching is deferred until the end of a block, as
Figure~\ref{fig:blocks-and-branches}(\subref{fig:branch-diagram}) depicts. By avoiding premature branching, this
approach ensures that all users answer the same questions, regardless
of randomization. It also avoids the biasing of question order that
would result by forcing branches to appear in a fixed position.

Branching
from a particular question response must go to a higher numbered
block, preventing cycles.



\subsection{Advanced Features}

\surveyman{} several advanced features to give survey authors more control over their surveys and their interaction with \surveyman{}.

\paragraph{Correlated questions.}
\surveyman{} lets authors include a
\textsc{Correlated} column to assist in quality control. \surveyman{}
checks for correlations between questions to see if any redundancy can
be removed, since shorter surveys help reduce survey fatigue. However,
sometimes correlations are desired, whether to confirm a hypothesis or
as a form of quality control. \surveyman{} thus lets authors mark sets
of questions as correlated by filling in the column with arbitary
text: all questions with the same text are assumed to be
correlated. \surveyman{} will then only report if the answers to these
questions are \emph{not} correlated. If not, this information can be
used to help identify inattentive or adversarial respondents.

\subsubsection*{Advanced Blocks Notation}
\label{sec:advanced-blocks}

So far, we have described blocks as if they were limited to
non-negative integers. In fact, \surveyman{}'s block syntax is more
general\footnote{Formally, the syntax of blocks is given by the
  following regular expression:
  \texttt{[1-9][0-9]*(.[a-z1-9][0-9]*)*}.}.

There are three kinds of blocks: \emph{top-level blocks}, indicated by
a single number; \emph{nested blocks}, indicated by numbers separated
by periods; and \emph{floating blocks}, indicated by alphanumerics,
which can move around inside their parent blocks. We describe each in
terms of their interaction with other blocks and with
branches. Figure~\ref{fig:blocks-and-branches}(\subref{fig:block-diagram}) gives an example of each.

\paragraph{Top-level Blocks.}
A survey is composed of blocks. If there are no blocks (i.e., the
survey is completely flat), we say that all questions belong to a
single, top-level block (block \texttt{1}). Only one branch question
is allowed per top-level block. Additionally, branch question targets
must also be top-level blocks, though they cannot be floating blocks
(described below).

\paragraph{Nested Blocks.}
As with outlines, a period in a block indicates hierarchy: we can think of blocks
numbered \texttt{1.1} and \texttt{1.2} as being \emph{contained} within block
\texttt{1}. All questions numbered \texttt{1.1} precede all questions
numbered \texttt{1.2}, and both strictly precede blocks with numbers 2
or higher.

\paragraph{Floating Blocks.}
Survey authors may want certain questions to be allowed to appear anywhere
within a containing block. For example, a survey author might want
questions to appear within block \texttt{1}, but does not care whether
they appear before or after other questions contained in that block
(e.g., questions with block numbers \texttt{1.1} and
\texttt{1.2}). Such questions can be placed in \emph{floating
  blocks}. A question is marked as floating by using a
block name that is an alphanumeric string rather than a number.

In this case, the survey author could give a floating question the
block number \texttt{1.A}. That block is contained within block
\texttt{1}, and the question is free to float anywhere within that
block. The string itself has no effect on ordering; for example,
blocks \texttt{1.A} and \texttt{1.B} can appear in any order inside
block \texttt{1}. All questions in the same floating block float
together inside their containing block.

\punt{
\subsection{Semantics}

}
