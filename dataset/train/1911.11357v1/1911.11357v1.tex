\documentclass[10pt,twocolumn,letterpaper]{article}

\usepackage{cvpr}
\usepackage{times}
\usepackage{epsfig}
\usepackage{graphicx}
\usepackage{amsmath}
\usepackage{amssymb}
\usepackage{booktabs}
\usepackage{array}
\newcolumntype{C}[1]{>{\centering\arraybackslash}p{#1}}
\usepackage{enumitem}
\usepackage{stmaryrd}
\usepackage[toc,page]{appendix}


\usepackage{amsmath,amsfonts,bm}

\DeclareMathOperator{\IS}{IS} \DeclareMathOperator{\FID}{FID} \newcommand{\imagenet}{\textsc{imagenet}}

\newcommand{\figleft}{{\em (Left)}}
\newcommand{\figcenter}{{\em (Center)}}
\newcommand{\figright}{{\em (Right)}}
\newcommand{\figtop}{{\em (Top)}}
\newcommand{\figbottom}{{\em (Bottom)}}
\newcommand{\captiona}{{\em (a)}}
\newcommand{\captionb}{{\em (b)}}
\newcommand{\captionc}{{\em (c)}}
\newcommand{\captiond}{{\em (d)}}

\newcommand{\newterm}[1]{{\bf #1}}


\def\figref#1{figure~\ref{#1}}
\def\Figref#1{Figure~\ref{#1}}
\def\twofigref#1#2{figures \ref{#1} and \ref{#2}}
\def\quadfigref#1#2#3#4{figures \ref{#1}, \ref{#2}, \ref{#3} and \ref{#4}}
\def\secref#1{section~\ref{#1}}
\def\Secref#1{Section~\ref{#1}}
\def\twosecrefs#1#2{sections \ref{#1} and \ref{#2}}
\def\secrefs#1#2#3{sections \ref{#1}, \ref{#2} and \ref{#3}}
\def\eqref#1{equation~\ref{#1}}
\def\Eqref#1{Equation~\ref{#1}}
\def\plaineqref#1{\ref{#1}}
\def\chapref#1{chapter~\ref{#1}}
\def\Chapref#1{Chapter~\ref{#1}}
\def\rangechapref#1#2{chapters\ref{#1}--\ref{#2}}
\def\algref#1{algorithm~\ref{#1}}
\def\Algref#1{Algorithm~\ref{#1}}
\def\twoalgref#1#2{algorithms \ref{#1} and \ref{#2}}
\def\Twoalgref#1#2{Algorithms \ref{#1} and \ref{#2}}
\def\partref#1{part~\ref{#1}}
\def\Partref#1{Part~\ref{#1}}
\def\twopartref#1#2{parts \ref{#1} and \ref{#2}}

\def\ceil#1{\lceil #1 \rceil}
\def\floor#1{\lfloor #1 \rfloor}
\def\1{\bm{1}}
\newcommand{\train}{\mathcal{D}}
\newcommand{\valid}{\mathcal{D_{\mathrm{valid}}}}
\newcommand{\test}{\mathcal{D_{\mathrm{test}}}}

\def\eps{{\epsilon}}


\def\reta{{\textnormal{}}}
\def\ra{{\textnormal{a}}}
\def\rb{{\textnormal{b}}}
\def\rc{{\textnormal{c}}}
\def\rd{{\textnormal{d}}}
\def\re{{\textnormal{e}}}
\def\rf{{\textnormal{f}}}
\def\rg{{\textnormal{g}}}
\def\rh{{\textnormal{h}}}
\def\ri{{\textnormal{i}}}
\def\rj{{\textnormal{j}}}
\def\rk{{\textnormal{k}}}
\def\rl{{\textnormal{l}}}
\def\rn{{\textnormal{n}}}
\def\ro{{\textnormal{o}}}
\def\rp{{\textnormal{p}}}
\def\rq{{\textnormal{q}}}
\def\rr{{\textnormal{r}}}
\def\rs{{\textnormal{s}}}
\def\rt{{\textnormal{t}}}
\def\ru{{\textnormal{u}}}
\def\rv{{\textnormal{v}}}
\def\rw{{\textnormal{w}}}
\def\rx{{\textnormal{x}}}
\def\ry{{\textnormal{y}}}
\def\rz{{\textnormal{z}}}

\def\rvepsilon{{\mathbf{\epsilon}}}
\def\rvtheta{{\mathbf{\theta}}}
\def\rva{{\mathbf{a}}}
\def\rvb{{\mathbf{b}}}
\def\rvc{{\mathbf{c}}}
\def\rvd{{\mathbf{d}}}
\def\rve{{\mathbf{e}}}
\def\rvf{{\mathbf{f}}}
\def\rvg{{\mathbf{g}}}
\def\rvh{{\mathbf{h}}}
\def\rvu{{\mathbf{i}}}
\def\rvj{{\mathbf{j}}}
\def\rvk{{\mathbf{k}}}
\def\rvl{{\mathbf{l}}}
\def\rvm{{\mathbf{m}}}
\def\rvn{{\mathbf{n}}}
\def\rvo{{\mathbf{o}}}
\def\rvp{{\mathbf{p}}}
\def\rvq{{\mathbf{q}}}
\def\rvr{{\mathbf{r}}}
\def\rvs{{\mathbf{s}}}
\def\rvt{{\mathbf{t}}}
\def\rvu{{\mathbf{u}}}
\def\rvv{{\mathbf{v}}}
\def\rvw{{\mathbf{w}}}
\def\rvx{{\mathbf{x}}}
\def\rvy{{\mathbf{y}}}
\def\rvz{{\mathbf{z}}}

\def\erva{{\textnormal{a}}}
\def\ervb{{\textnormal{b}}}
\def\ervc{{\textnormal{c}}}
\def\ervd{{\textnormal{d}}}
\def\erve{{\textnormal{e}}}
\def\ervf{{\textnormal{f}}}
\def\ervg{{\textnormal{g}}}
\def\ervh{{\textnormal{h}}}
\def\ervi{{\textnormal{i}}}
\def\ervj{{\textnormal{j}}}
\def\ervk{{\textnormal{k}}}
\def\ervl{{\textnormal{l}}}
\def\ervm{{\textnormal{m}}}
\def\ervn{{\textnormal{n}}}
\def\ervo{{\textnormal{o}}}
\def\ervp{{\textnormal{p}}}
\def\ervq{{\textnormal{q}}}
\def\ervr{{\textnormal{r}}}
\def\ervs{{\textnormal{s}}}
\def\ervt{{\textnormal{t}}}
\def\ervu{{\textnormal{u}}}
\def\ervv{{\textnormal{v}}}
\def\ervw{{\textnormal{w}}}
\def\ervx{{\textnormal{x}}}
\def\ervy{{\textnormal{y}}}
\def\ervz{{\textnormal{z}}}

\def\rmA{{\mathbf{A}}}
\def\rmB{{\mathbf{B}}}
\def\rmC{{\mathbf{C}}}
\def\rmD{{\mathbf{D}}}
\def\rmE{{\mathbf{E}}}
\def\rmF{{\mathbf{F}}}
\def\rmG{{\mathbf{G}}}
\def\rmH{{\mathbf{H}}}
\def\rmI{{\mathbf{I}}}
\def\rmJ{{\mathbf{J}}}
\def\rmK{{\mathbf{K}}}
\def\rmL{{\mathbf{L}}}
\def\rmM{{\mathbf{M}}}
\def\rmN{{\mathbf{N}}}
\def\rmO{{\mathbf{O}}}
\def\rmP{{\mathbf{P}}}
\def\rmQ{{\mathbf{Q}}}
\def\rmR{{\mathbf{R}}}
\def\rmS{{\mathbf{S}}}
\def\rmT{{\mathbf{T}}}
\def\rmU{{\mathbf{U}}}
\def\rmV{{\mathbf{V}}}
\def\rmW{{\mathbf{W}}}
\def\rmX{{\mathbf{X}}}
\def\rmY{{\mathbf{Y}}}
\def\rmZ{{\mathbf{Z}}}

\def\ermA{{\textnormal{A}}}
\def\ermB{{\textnormal{B}}}
\def\ermC{{\textnormal{C}}}
\def\ermD{{\textnormal{D}}}
\def\ermE{{\textnormal{E}}}
\def\ermF{{\textnormal{F}}}
\def\ermG{{\textnormal{G}}}
\def\ermH{{\textnormal{H}}}
\def\ermI{{\textnormal{I}}}
\def\ermJ{{\textnormal{J}}}
\def\ermK{{\textnormal{K}}}
\def\ermL{{\textnormal{L}}}
\def\ermM{{\textnormal{M}}}
\def\ermN{{\textnormal{N}}}
\def\ermO{{\textnormal{O}}}
\def\ermP{{\textnormal{P}}}
\def\ermQ{{\textnormal{Q}}}
\def\ermR{{\textnormal{R}}}
\def\ermS{{\textnormal{S}}}
\def\ermT{{\textnormal{T}}}
\def\ermU{{\textnormal{U}}}
\def\ermV{{\textnormal{V}}}
\def\ermW{{\textnormal{W}}}
\def\ermX{{\textnormal{X}}}
\def\ermY{{\textnormal{Y}}}
\def\ermZ{{\textnormal{Z}}}

\def\vzero{{\bm{0}}}
\def\vone{{\bm{1}}}
\def\vmu{{\bm{\mu}}}
\def\vtheta{{\bm{\theta}}}
\def\va{{\bm{a}}}
\def\vb{{\bm{b}}}
\def\vc{{\bm{c}}}
\def\vd{{\bm{d}}}
\def\ve{{\bm{e}}}
\def\vf{{\bm{f}}}
\def\vg{{\bm{g}}}
\def\vh{{\bm{h}}}
\def\vi{{\bm{i}}}
\def\vj{{\bm{j}}}
\def\vk{{\bm{k}}}
\def\vl{{\bm{l}}}
\def\vm{{\bm{m}}}
\def\vn{{\bm{n}}}
\def\vo{{\bm{o}}}
\def\vp{{\bm{p}}}
\def\vq{{\bm{q}}}
\def\vr{{\bm{r}}}
\def\vs{{\bm{s}}}
\def\vt{{\bm{t}}}
\def\vu{{\bm{u}}}
\def\vv{{\bm{v}}}
\def\vw{{\bm{w}}}
\def\vx{{\bm{x}}}
\def\vy{{\bm{y}}}
\def\vz{{\bm{z}}}

\def\evalpha{{\alpha}}
\def\evbeta{{\beta}}
\def\evepsilon{{\epsilon}}
\def\evlambda{{\lambda}}
\def\evomega{{\omega}}
\def\evmu{{\mu}}
\def\evpsi{{\psi}}
\def\evsigma{{\sigma}}
\def\evtheta{{\theta}}
\def\eva{{a}}
\def\evb{{b}}
\def\evc{{c}}
\def\evd{{d}}
\def\eve{{e}}
\def\evf{{f}}
\def\evg{{g}}
\def\evh{{h}}
\def\evi{{i}}
\def\evj{{j}}
\def\evk{{k}}
\def\evl{{l}}
\def\evm{{m}}
\def\evn{{n}}
\def\evo{{o}}
\def\evp{{p}}
\def\evq{{q}}
\def\evr{{r}}
\def\evs{{s}}
\def\evt{{t}}
\def\evu{{u}}
\def\evv{{v}}
\def\evw{{w}}
\def\evx{{x}}
\def\evy{{y}}
\def\evz{{z}}

\def\mA{{\bm{A}}}
\def\mB{{\bm{B}}}
\def\mC{{\bm{C}}}
\def\mD{{\bm{D}}}
\def\mE{{\bm{E}}}
\def\mF{{\bm{F}}}
\def\mG{{\bm{G}}}
\def\mH{{\bm{H}}}
\def\mI{{\bm{I}}}
\def\mJ{{\bm{J}}}
\def\mK{{\bm{K}}}
\def\mL{{\bm{L}}}
\def\mM{{\bm{M}}}
\def\mN{{\bm{N}}}
\def\mO{{\bm{O}}}
\def\mP{{\bm{P}}}
\def\mQ{{\bm{Q}}}
\def\mR{{\bm{R}}}
\def\mS{{\bm{S}}}
\def\mT{{\bm{T}}}
\def\mU{{\bm{U}}}
\def\mV{{\bm{V}}}
\def\mW{{\bm{W}}}
\def\mX{{\bm{X}}}
\def\mY{{\bm{Y}}}
\def\mZ{{\bm{Z}}}
\def\mBeta{{\bm{\beta}}}
\def\mPhi{{\bm{\Phi}}}
\def\mLambda{{\bm{\Lambda}}}
\def\mSigma{{\bm{\Sigma}}}

\DeclareMathAlphabet{\mathsfit}{\encodingdefault}{\sfdefault}{m}{sl}
\SetMathAlphabet{\mathsfit}{bold}{\encodingdefault}{\sfdefault}{bx}{n}
\newcommand{\tens}[1]{\bm{\mathsfit{#1}}}
\def\tA{{\tens{A}}}
\def\tB{{\tens{B}}}
\def\tC{{\tens{C}}}
\def\tD{{\tens{D}}}
\def\tE{{\tens{E}}}
\def\tF{{\tens{F}}}
\def\tG{{\tens{G}}}
\def\tH{{\tens{H}}}
\def\tI{{\tens{I}}}
\def\tJ{{\tens{J}}}
\def\tK{{\tens{K}}}
\def\tL{{\tens{L}}}
\def\tM{{\tens{M}}}
\def\tN{{\tens{N}}}
\def\tO{{\tens{O}}}
\def\tP{{\tens{P}}}
\def\tQ{{\tens{Q}}}
\def\tR{{\tens{R}}}
\def\tS{{\tens{S}}}
\def\tT{{\tens{T}}}
\def\tU{{\tens{U}}}
\def\tV{{\tens{V}}}
\def\tW{{\tens{W}}}
\def\tX{{\tens{X}}}
\def\tY{{\tens{Y}}}
\def\tZ{{\tens{Z}}}


\def\gA{{\mathcal{A}}}
\def\gB{{\mathcal{B}}}
\def\gC{{\mathcal{C}}}
\def\gD{{\mathcal{D}}}
\def\gE{{\mathcal{E}}}
\def\gF{{\mathcal{F}}}
\def\gG{{\mathcal{G}}}
\def\gH{{\mathcal{H}}}
\def\gI{{\mathcal{I}}}
\def\gJ{{\mathcal{J}}}
\def\gK{{\mathcal{K}}}
\def\gL{{\mathcal{L}}}
\def\gM{{\mathcal{M}}}
\def\gN{{\mathcal{N}}}
\def\gO{{\mathcal{O}}}
\def\gP{{\mathcal{P}}}
\def\gQ{{\mathcal{Q}}}
\def\gR{{\mathcal{R}}}
\def\gS{{\mathcal{S}}}
\def\gT{{\mathcal{T}}}
\def\gU{{\mathcal{U}}}
\def\gV{{\mathcal{V}}}
\def\gW{{\mathcal{W}}}
\def\gX{{\mathcal{X}}}
\def\gY{{\mathcal{Y}}}
\def\gZ{{\mathcal{Z}}}

\def\sA{{\mathbb{A}}}
\def\sB{{\mathbb{B}}}
\def\sC{{\mathbb{C}}}
\def\sD{{\mathbb{D}}}
\def\sF{{\mathbb{F}}}
\def\sG{{\mathbb{G}}}
\def\sH{{\mathbb{H}}}
\def\sI{{\mathbb{I}}}
\def\sJ{{\mathbb{J}}}
\def\sK{{\mathbb{K}}}
\def\sL{{\mathbb{L}}}
\def\sM{{\mathbb{M}}}
\def\sN{{\mathbb{N}}}
\def\sO{{\mathbb{O}}}
\def\sP{{\mathbb{P}}}
\def\sQ{{\mathbb{Q}}}
\def\sR{{\mathbb{R}}}
\def\sS{{\mathbb{S}}}
\def\sT{{\mathbb{T}}}
\def\sU{{\mathbb{U}}}
\def\sV{{\mathbb{V}}}
\def\sW{{\mathbb{W}}}
\def\sX{{\mathbb{X}}}
\def\sY{{\mathbb{Y}}}
\def\sZ{{\mathbb{Z}}}

\def\emLambda{{\Lambda}}
\def\emA{{A}}
\def\emB{{B}}
\def\emC{{C}}
\def\emD{{D}}
\def\emE{{E}}
\def\emF{{F}}
\def\emG{{G}}
\def\emH{{H}}
\def\emI{{I}}
\def\emJ{{J}}
\def\emK{{K}}
\def\emL{{L}}
\def\emM{{M}}
\def\emN{{N}}
\def\emO{{O}}
\def\emP{{P}}
\def\emQ{{Q}}
\def\emR{{R}}
\def\emS{{S}}
\def\emT{{T}}
\def\emU{{U}}
\def\emV{{V}}
\def\emW{{W}}
\def\emX{{X}}
\def\emY{{Y}}
\def\emZ{{Z}}
\def\emSigma{{\Sigma}}

\newcommand{\etens}[1]{\mathsfit{#1}}
\def\etLambda{{\etens{\Lambda}}}
\def\etA{{\etens{A}}}
\def\etB{{\etens{B}}}
\def\etC{{\etens{C}}}
\def\etD{{\etens{D}}}
\def\etE{{\etens{E}}}
\def\etF{{\etens{F}}}
\def\etG{{\etens{G}}}
\def\etH{{\etens{H}}}
\def\etI{{\etens{I}}}
\def\etJ{{\etens{J}}}
\def\etK{{\etens{K}}}
\def\etL{{\etens{L}}}
\def\etM{{\etens{M}}}
\def\etN{{\etens{N}}}
\def\etO{{\etens{O}}}
\def\etP{{\etens{P}}}
\def\etQ{{\etens{Q}}}
\def\etR{{\etens{R}}}
\def\etS{{\etens{S}}}
\def\etT{{\etens{T}}}
\def\etU{{\etens{U}}}
\def\etV{{\etens{V}}}
\def\etW{{\etens{W}}}
\def\etX{{\etens{X}}}
\def\etY{{\etens{Y}}}
\def\etZ{{\etens{Z}}}

\newcommand{\pdata}{p_{\rm{data}}}
\newcommand{\ptrain}{\hat{p}_{\rm{data}}}
\newcommand{\Ptrain}{\hat{P}_{\rm{data}}}
\newcommand{\pmodel}{p_{\rm{model}}}
\newcommand{\Pmodel}{P_{\rm{model}}}
\newcommand{\ptildemodel}{\tilde{p}_{\rm{model}}}
\newcommand{\pencode}{p_{\rm{encoder}}}
\newcommand{\pdecode}{p_{\rm{decoder}}}
\newcommand{\precons}{p_{\rm{reconstruct}}}

\newcommand{\laplace}{\mathrm{Laplace}} 

\newcommand{\E}{\mathbb{E}}
\newcommand{\Ls}{\mathcal{L}}
\newcommand{\R}{\mathbb{R}}
\newcommand{\emp}{\tilde{p}}
\newcommand{\lr}{\alpha}
\newcommand{\reg}{\lambda}
\newcommand{\rect}{\mathrm{rectifier}}
\newcommand{\softmax}{\mathrm{softmax}}
\newcommand{\sigmoid}{\sigma}
\newcommand{\softplus}{\zeta}
\newcommand{\KL}{D_{\mathrm{KL}}}
\newcommand{\Var}{\mathrm{Var}}
\newcommand{\standarderror}{\mathrm{SE}}
\newcommand{\Cov}{\mathrm{Cov}}
\newcommand{\normlzero}{L^0}
\newcommand{\normlone}{L^1}
\newcommand{\normltwo}{L^2}
\newcommand{\normlp}{L^p}
\newcommand{\normmax}{L^\infty}

\newcommand{\parents}{Pa} 

\DeclareMathOperator*{\argmax}{arg\,max}
\DeclareMathOperator*{\argmin}{arg\,min}

\DeclareMathOperator{\sign}{sign}
\DeclareMathOperator{\Tr}{Tr}
\let\ab\allowbreak
 
\usepackage[pagebackref=true,breaklinks=true,letterpaper=true,colorlinks,bookmarks=false]{hyperref}
\usepackage{xcolor}

\hypersetup{
  colorlinks,
  breaklinks=true,
  anchorcolor={blue!50!black},
  linkcolor={blue!50!black},
  citecolor={blue!50!black},
  urlcolor={blue}
}
\cvprfinalcopy

\newcommand{\lucic}[1]{{\color{red}{Lucic: #1}}}
\newcommand{\et}[1]{{\color{blue}{Eric: #1}}}
\newcommand{\sazadi}[1]{{\color{orange}{Samaneh: #1}}}
\def\httilde{\mbox{\tt\raisebox{-.5ex}{\symbol{126}}}}

\ifcvprfinal\pagestyle{empty}\fi
\begin{document}

\title{Semantic Bottleneck Scene Generation}

\author{\quad Samaneh Azadi\thanks{Email: sazadi@eecs.berkeley.edu},
Michael Tschannen,
Eric Tzeng,
Sylvain Gelly,
Trevor Darrell,
Mario Lucic\
\label{gumbel}
S^{ij}_k = \frac{\exp\{(\log P^{ij}_k + G_k)/\tau\}}{\sum_{i=1}^K \exp\{(\log P^{ij}_i + G_i)/\tau\}}.

 L_{D_{\text{SPD}}} &=& -\mathbb{E}_{y,x} [\min(0, -1 + D_{\text{SPD}}(y, x))] \nonumber\\
 && -\mathbb{E}_{y} [\min(0, -1-D_{\text{SPD}}(y,G_{\text{SPD}}(y)))] \nonumber \\
 L_{G_{\text{SPD}}} &=&  -\mathbb{E}_{y} [D_{\text{SPD}}(y, G_{\text{SPD}}(y)))] \label{eq:SPD}\\
 && + \lambda_1 L_1^{\text{VGG}} + \lambda_2 L_1^{\text{Feat}}, \nonumber

 L_{D_2} &=& -\mathbb{E}_{x} [\min(0, -1 + D_2(x))] \label{D2} \nonumber\\
 && -\mathbb{E}_{z} [\min(0, -1-D_2(G(z)))] \\
 L_G &=&  -\mathbb{E}_{z} [D_2(G(z)))] + L_{G_{\text{SPD}}} +  \lambda L_{G_{\text{SB}}} \nonumber\\
 G(z) &=& G_{\text{SPD}}(G_{\text{SB}}(z)), \nonumber
  where  represents the semantic bottleneck synthesis generator, and  is the improved WGAN loss used to pretrain  described in Section~\ref{SB-network}.
In contrast to the conditional discriminator in SPADE, which enforces consistency between the input semantic map and the output image,  is primarily concerned with the overall quality of the final output.
 The hyper parameter  determines the ratio between the two generators during fine-tuning. The parameters of both generators,  and , as well as the corresponding discriminators,  and , are updated in this end-to-end fine-tuning.

We illustrate our final end-to-end network in Figure~\ref{fig:end2end}.
Jointly fine-tuning the two networks in an end-to-end fashion allows the two networks to reinforce each other, leading to improved performance.
The gradients with respect to RGB images synthesized by SPADE are back-propagated to the segmentation synthesis model, thereby encouraging it to synthesize segmentation layouts that lead to higher quality final images. Hence, SPADE plays the role of a loss function for synthesizing segmentations, but in the RGB space, hence providing a goal that was absent from the initial training.
Similarly, fine-tuning SPADE with synthesized segmentations allows it to adapt to a more diverse set of scene layouts, which improves the quality of generated samples.


\section{Experiments and Results}
We evaluate the performance of the proposed approach on two datasets containing images with complex scenes, where the ground truth segmentation masks are available during training (possibly only for a subset of the images). We also study the role of the two network components, semantic bottleneck and semantic image synthesis, on the final result. We compare the performance of SB-GAN against the state-of-the-art BigGAN model~\cite{biggan} as well as a ProGAN~\cite{PGGAN} baseline that has been trained on the RGB images directly. We evaluate our method using Fr\'echet Inception Distance (FID) as well as a user study.

\begin{figure*}[t!]
\centering
\includegraphics[width=\textwidth]{comp_city5k_tagged.pdf}
\caption{Images synthesized by different methods trained on Cityscapes-5K. Zoom in for more detail. Although both models capture the general scene layout, SB-GAN (1st row) generates more convincing objects such as buildings and cars.}
\label{fig:city5k}
\end{figure*}

\begin{figure*}
\centering
\includegraphics[width=\textwidth]{comp_city25k_tagged.pdf}
\caption{Images synthesized by different methods trained on Cityscapes-25K. Zoom in for more detail. Images synthesized by BigGAN (3rd row) are blurry and sometimes defective in local structures.}
\label{fig:city25k}
\end{figure*}

\vspace{-2mm}
\paragraph{Datasets} We study the performance of our model on the Cityscapes and ADE-indoor datasets as the two domains with complex scene images.

\begin{itemize}[itemsep=0pt,topsep=1pt]
    \item Cityscapes-5K~\cite{cityscapes} contains street scene images in German cities with training and validation set sizes of 3,000 and 500 images, respectively. Ground truth segmentation masks with 33 semantic classes are available for all images in this dataset.
    \item Cityscapes-25K~\cite{cityscapes} contains street scene images in German cities with training and validation set sizes of 23,000 and 500 images, respectively. Cityscapes-5K is a subset of this dataset, providing 3,000 images in the training set here as well as the entire validation set. Fine ground truth annotations are only provided for this subset, with the remaining 20,000 training images containing only coarse annotations.
    We extract the corresponding fine annotations for the rest of training images using the state-of-the-art segmentation model~\cite{DRN, Yu2016} trained on the training annotated samples from Cityscapes-5K. This dataset contains 19 semantic classes. 
    \item ADE-Indoor is a subset of the ADE20K dataset~\cite{ADE20k} containing 4,377 challenging training images from indoor scenes and 433 validation images with 95 semantic categories. 
\end{itemize}

\begin{figure*}
\centering
\includegraphics[width=\textwidth]{comp_ade_tagged.pdf}
\caption{Images synthesized by different methods trained on ADE-Indoor. This dataset is very challenging, causing mode collapse for the BigGAN model (3rd row). In contrast, samples generated by SB-GAN (1st row) are generally of higher quality and much more structured than those of ProGAN (2nd row).}
\label{fig:ade}
\end{figure*}

\vspace{-3mm}
\paragraph{Evaluation} We use the Fr{\'e}chet Inception Distance (FID)~\cite{heusel2017gans} as well as a user study to evaluate the quality of the generated samples. To compute FID, the real data and generated samples are first embedded in a specific layer of a pre-trained Inception network. Then, a multivariate Gaussian is fit to the data, and the distance is computed as
,
where  and  denote the empirical mean and covariance, and subscripts  and  denote the real and generated data respectively. FID is shown to be sensitive to both the addition of spurious modes and to mode dropping~\cite{sajjadi2018assessing,lucic2018}. On the Cityscapes dataset, we ran five trials where we computed FID on 500 random synthetic images and 500 real validation images, and report the average score. On ADE-Indoor, the same process is repeated on batches of 433 images.  

\paragraph{Implementation details} In all our experiments, we set , and . The initial generator and discriminator learning rates for training SPADE both in the pretraining and end-to-end steps are  and , respectively. The learning rate for the semantic bottleneck synthesis sub-network is set to  in the pretraining step and to  in the end-to-end fine-tuning on Cityscapes, and to  for ADE-Indoor. The temperature hyperparameter, , is always set to 1.
For BigGAN, we followed the setup in~\cite{luvcic2019high}\footnote{Configuration as in \url{https://github.com/google/compare_gan/blob/master/example_configs/biggan_imagenet128.gin}}, where we modified the code to allow for non-square images of Cityscapes. We used one class label for all images to have an unconditional BigGAN model. For both datasets we varied the batch size (using values in ), the learning rate, and the location of the self-attention block. We trained the final model for K iterations.

\subsection{Qualitative results}
In Figures~\ref{fig:city5k},~\ref{fig:city25k}, and~\ref{fig:ade}, we provide qualitative comparisons of the competing methods on the three aforementioned datasets. We observe that both Cityscapes-5K and ADE-Indoor are very challenging for the state-of-the-art ProGAN and BigGAN models, likely due to the complexity of the data and small number of training instances. Even at a resolution of  on the ADE-Indoor dataset, BigGAN suffers from mode collapse, as illustrated in the last row of Figure~\ref{fig:ade}. In contrast, SB-GAN significantly improves on the structure of the scene distribution, and provides samples of higher quality. On Cityscapes-25K, the performance improvement of SB-GAN is more modest due to the large number of training images available. It is worth emphasizing that in this case only 3K ground truth segmentations for training SB-GAN are available. Compared to BigGAN, images synthesized by SB-GAN are sharper and contain more structural details (e.g., one can zoom-in on the synthesized cars). More qualitative examples are presented in the Appendix.


\subsection{Quantitative evaluation}
To provide a thorough empirical evaluation of the proposed approach, we generate samples for each dataset and report the FID scores of the resulting images (averaged across 5 sets of generated samples).
We evaluate SB-GAN both before and after end-to-end fine-tuning, and compare our method to two strong baselines, ProGAN~\cite{PGGAN} and BigGAN~\cite{biggan}. The results are detailed in Tables~\ref{table:results} and~\ref{table:results128}.


\begin{table}[h]
\setlength{\tabcolsep}{3pt}
\setlength{\extrarowheight}{5pt}
\renewcommand{\arraystretch}{0.75}
\centering
\begin{tabular}{l C{1.2cm}  C{1.7cm}C{1.8cm}}
\toprule
& \multicolumn{3}{c}{\textsc{Method}} \\ \cmidrule{2-4}
               & ProGAN  & SB-GAN W/O FT & SB-GAN \\ \midrule
\textsc{Cityscapes-5k}  &  92.57   & 83.20       & \textbf{65.49} \\
\textsc{Cityscapes-25k} &  63.87   & 71.13       & \textbf{62.97} \\
\textsc{ADE-Indoor}     & 104.83  & 91.80       &  \textbf{85.27}\\
\bottomrule
\end{tabular}
\vspace{2mm}
\caption{FID of the synthesized samples (lower is better), averaged over 5 random sets of samples. Images were synthesized at resolution of  on Cityscapes and  on ADE-Indoor. }
\label{table:results}
\end{table}

\begin{figure*}[h]
\centering
\includegraphics[width=\textwidth]{ablate_city25k_tagged.pdf}
\caption{The effect of fine-tuning on the baseline setup for the Cityscapes-25K dataset. We observe that both the global structure of the segmentations and the performance of semantic image synthesis improve after fine-tuning, resulting in images of higher quality. \label{fig:ablate_city25k}}
\end{figure*}

 First, in the low-data regime, even without fine-tuning, our Semantic Bottleneck GAN produces higher quality samples and significantly outperforms the baselines on Cityscapes-5K and ADE-Indoor. The advantage of our proposed method is even more striking on smaller datasets. While competing methods are unable to learn a high-quality model of the underlying distribution without having access to a large number of samples, SB-GAN is less sensitive to the number of training data points. Secondly, we observe that by jointly training the semantic bottleneck and image synthesis components, SB-GAN produces state-of-the-art results across all three datasets.

We were not able to successfully train BigGAN at a resolution of  due to instability observed during training and mode collapse. Table~\ref{table:results128}, however, shows the results for a lower-resolution setting, for which we were able to successfully train BigGAN. We report the results before the training collapses. BigGAN is, to a certain extent, able to capture the distribution of Cityscapes-25K, but fails completely on ADE-Indoor. Interestingly, BigGAN fails to capture the distribution of Cityscapes-5K even at  resolution. The standard deviation of the FID scores computed in Tables~\ref{table:results} and~\ref{table:results128} is within  of the mean for Cityscapes and within  of the mean for ADE-Indoor. 


\begin{table}[h!]
\setlength{\tabcolsep}{4pt}
\setlength{\extrarowheight}{5pt}
\renewcommand{\arraystretch}{0.75}
\centering
\begin{tabular}{l C{1.2cm} C{1.5cm}C{1.5cm}}
\toprule
 & \multicolumn{3}{c}{\textsc{Method}} \\ \cmidrule{2-4}
               & ProGAN  & BigGAN & SB-GAN \\ \midrule
\textsc{Cityscapes-25k} &  56.7   &  64.82   & \textbf{54.92} \\
\textsc{ADE-Indoor}      & 85.94 & 156.65  &  \textbf{81.39}\\
\bottomrule
\end{tabular}
\vspace{2mm}
\caption{FID of the synthesized samples (lower is better), averaged over 5 random sets of samples. Images were synthesized at resolution of  on Cityscapes and  on ADE-Indoor.}
\label{table:results128}
\end{table}

\paragraph{Generating by conditioning on real segmentations}
To independently assess the impact of end-to-end training on the conditional image synthesis sub-network, we evaluate the quality of generated samples when conditioning on ground truth validation segmentations from each dataset. 
Comparisons to the baseline network SPADE~\cite{SPADE} are provided in Table~\ref{table:gtlabel-results} and Figure~\ref{fig:ablate_spade}. We observe that the image synthesis component of SB-GAN consistently outperforms SPADE across all three datasets, indicating that fine-tuning on synthetic labels produced by the segmentation generator improves the conditional image generator. Please refer to the Appendix for more qualitative examples.


\begin{table}[h]
\setlength{\tabcolsep}{4pt}
\setlength{\extrarowheight}{5pt}
\renewcommand{\arraystretch}{0.75}
\centering
\begin{tabular}{lcc}
\toprule
  & \multicolumn{2}{c}{\textsc{Method}} \\ \cmidrule{2-3}
& SPADE & SB-GAN \\ \midrule
\textsc{Cityscapes-5k}     & 72.12 & \textbf{60.39} \\
\textsc{Cityscapes-25k}    & 60.83 & \textbf{54.13} \\
\textsc{ADE-Indoor}        & 50.30 & \textbf{48.15} \\
\bottomrule
\end{tabular}
\vspace{3mm}
\caption{FID of the synthesized samples when conditioned on the ground truth labels (lower is better), averaged over 5 random sets of samples. For SB-GAN, we train the entire model end-to-end, extract the trained SPADE sub-network, and synthesize samples conditioned on the ground truth labels.}
\label{table:gtlabel-results}
\end{table}



\begin{figure*}[t!]
\centering
\includegraphics[width=\textwidth]{ablate_ade_tagged.pdf}
\caption{The effect of fine-tuning (FT) on the baseline setup for ADE-Indoor dataset. We observe that both the global structure of the segmentations and the performance of semantic image synthesis have been improved after fine-tuning, resulting in images of higher quality.}
\label{fig:ablate_ade}
\end{figure*}

\begin{figure*}[t!]
\centering
\includegraphics[width=\textwidth]{ablate_spade_tagged.pdf}
\caption{The effect of SB-GAN on improving the performance of the state-of-the-art semantic image synthesis model (SPADE~\cite{SPADE}) on ground truth segmentations of Cityscapes-25K (left) and ADE-Indoor (right) validation sets. For SB-GAN, we train the entire model end-to-end, extract the trained SPADE sub-network, and synthesize samples conditioned on the ground truth labels.}
\label{fig:ablate_spade}
\end{figure*}

\paragraph{Fine-tuning ablation study}
To further dissect the effect of end-to-end training, we perform a study 
on different components of SB-GAN. In particular, we consider three settings: (1) SB-GAN before end-to-end fine-tuning, (2) fine-tuning only the semantic bottleneck synthesis component, (3) fine-tuning only the conditional image synthesis component, and (4) fine-tuning all components jointly. The results on the Cityscapes-5K dataset (resolution ) are reported in Table~\ref{table:ablation}. Finally, the impact of fine-tuning on the quality of samples can be observed in Figures~\ref{fig:ablate_city25k} and ~\ref{fig:ablate_ade}. 

\begin{table}[h]
\setlength{\tabcolsep}{4pt}
\setlength{\extrarowheight}{5pt}
\renewcommand{\arraystretch}{0.75}
\centering
\begin{tabular}{C{1.5cm} C{2cm} C{2cm}C{1.3cm}}
\toprule
\multicolumn{4}{c}{\textsc{Method}} \\ \cmidrule{1-4}
    No FT    & FT SB    & FT SPADE & FT Both \\ \midrule
  70.15 & 66.22   &  63.04  & \textbf{58.67} \\
\bottomrule
\end{tabular}
\vspace{2mm}
\caption{Ablation study of various components of SB-GAN. We report FID scores of SB-GAN before fine-tuning, fine-tuning only the semantic bottleneck synthesis component, fine-tuning only the image synthesis component, and full end-to-end fine-tuning. Experiments are performed on the Cityscapes-5K dataset at a resolution of .}
\label{table:ablation}
\vspace{-2mm}
\end{table}


\subsection{Human evaluation}
We used Amazon Mechanical Turk (AMT) to study and compare the performance of different methods in terms of user assessments. We evaluate the performance of each model on each dataset through 600 pairs of (synthesized images, evaluators) containing 200 unique synthesized images. For each image, evaluators were asked to select a quality score from 1 to 4, indicating terrible and high quality images, respectively. Results are summarized in Table~\ref{table:userstudy} and are consistent with FID-based evaluations, with SB-GAN as the winner in all datasets once again.




\begin{table}[h]
\setlength{\tabcolsep}{5pt}
\setlength{\extrarowheight}{5pt}
\renewcommand{\arraystretch}{0.75}
\centering
\begin{tabular}{lccc}
\toprule
& \multicolumn{3}{c}{\textsc{Method}} \\ \cmidrule{2-4}
& ProGAN & BigGAN & SB-GAN \\ \midrule
\textsc{Cityscapes-5k}     & 2.08 & - & \textbf{2.48} \\
\textsc{Cityscapes-25k}    & 2.53 & 2.27 & \textbf{2.61} \\
\textsc{Ade-Indoor}        & 2.35 & 1.96 & \textbf{2.49} \\
\bottomrule
\end{tabular}
\vspace{2mm}
\caption{Average user evaluation scores when each user has selected a quality score in the range of 1 (terrible quality) to 4 (high quality) for each image.}
\label{table:userstudy}
\vspace{-2mm}
\end{table}

\vspace{-1mm}
\section{Conclusion}
We proposed an end-to-end Semantic Bottleneck GAN model that synthesizes semantic layouts from scratch, and then generates photo-realistic scenes conditioned on the synthesized layouts. Through extensive quantitative and qualitative evaluations, we showed that this novel end-to-end training pipeline significantly outperforms the state-of-the-art models in unconditional synthesis of complex scenes. In addition, Semantic Bottleneck GAN strongly improves the performance of the state-of-the-art semantic image synthesis model in synthesizing photo-realistic images from ground truth segmentations.

We believe that the idea of applying a semantic bottleneck to other generative models should be explored in future work. In addition, novel ways to train GANs with discrete outputs could be explored, especially techniques to deal with the non-differentiable nature of the generated outputs.

\paragraph{Acknowledgments}
This work was supported by Google through Google Cloud Platform research credits. We thank Marvin Ritter for help with issues related to the compare\_gan library \cite{lucic2018}. We are grateful to the members of BAIR for fruitful discussions. SA is supported by the Facebook graduate fellowship. 


{\small
\bibliographystyle{ieee_fullname}
\bibliography{egbib}
}

\newpage
\appendix
\section*{Appendix}
\section{Additional results}
In Figures~\ref{fig:ablate_city},~\ref{fig:ablate_city2},~\ref{fig:ablate_ade}, and~\ref{fig:ablate_ade2}, we show additional synthetic results from our proposed SB-GAN model including both the synthesized segmentations and their corresponding synthesized images from the Cityscapes-25K and ADE-Indoor datasets. As mentioned in the paper, on the Cityscapes-25K dataset, fine ground truth annotations are only provided for the Cityscapes-5k subset. We extract the corresponding fine annotations for the rest of training images using the state-of-the-art segmentation model~\cite{DRN, Yu2016} trained on the training annotated samples from Cityscapes-5K. 

Moreover, Figures~\ref{fig:ablate_city_spade} and~\ref{fig:ablate_ade_spade} present additional examples illustrating the impact of SB-GAN on improving the performance of SPADE~\cite{SPADE}, the state-of-the-art semantic image synthesis model on ground truth segmentations. The third row in these two figures show examples of the synthesized images conditioned on ground truth labels when the SPADE sub-network is extracted from a trained SB-GAN model.



\begin{figure*}[t!]
\centering
\includegraphics[width=\textwidth]{ablate_city25k_sup_tagged.pdf}
\caption{Segmentations and their corresponding images synthesized by SB-GAN trained on the Cityscapes-25K dataset.}
\label{fig:ablate_city}
\end{figure*}

\begin{figure*}[t!]
\centering
\includegraphics[width=\textwidth]{ablate_city25k_sup_tagged2.pdf}
\caption{Segmentations and their corresponding images synthesized by SB-GAN trained on the Cityscapes-25K dataset.}
\label{fig:ablate_city2}
\end{figure*}


\begin{figure*}[t!]
\centering
\includegraphics[width=\textwidth]{ablate_ade_sup_tagged.pdf}
\caption{Segmentations and their corresponding images synthesized by SB-GAN trained on the ADE-Indoor dataset.}
\label{fig:ablate_ade}
\end{figure*}


\begin{figure*}[t!]
\centering
\includegraphics[width=\textwidth]{ablate_ade_sup_tagged2.pdf}
\caption{Segmentations and their corresponding images synthesized by SB-GAN trained on the ADE-Indoor dataset.}
\label{fig:ablate_ade2}
\end{figure*}


\begin{figure*}[!t]
\centering
\includegraphics[width=\textwidth]{ablate_spade_sup_tagged1.pdf}
\caption{The effect of SB-GAN on improving the performance of the state-of-the-art semantic image synthesis model (SPADE) on ground truth segmentations of Cityscapes-25K validation set. For SB-GAN, we train the entire model end-to-end, extract the trained SPADE sub-network, and synthesize samples conditioned on the ground truth labels.}
\label{fig:ablate_city_spade}
\end{figure*}

\begin{figure*}[t!]
\centering
\includegraphics[width=\textwidth]{ablate_spade_sup_tagged2.pdf}
\caption{The effect of SB-GAN on improving the performance of the state-of-the-art semantic image synthesis model (SPADE) on ground truth segmentations of ADE-Indoor validation set. For SB-GAN, we train the entire model end-to-end, extract the trained SPADE sub-network, and synthesize samples conditioned on the ground truth labels.}
\label{fig:ablate_ade_spade}
\end{figure*}
\end{document}
