\documentclass[amsthm]{JSC_LaTex_2007_Mar_12/elsart}

\usepackage{JSC_LaTex_2007_Mar_12/yjsco}
\usepackage{JSC_LaTex_2007_Mar_12/natbib}
\usepackage{myMacLite}


\def\bpf{{\em Proof.}~}
\def\epf{\qed}




\renewcommand{\myfig}[4]{
    \begin{figure}[htb]
    \begin{center}
    \scalebox{#4}{
          \input{figs/#1.pstex_t}}
    \caption{#2}
    \label{fig:#3}
    \end{center}
    \end{figure}
}

\newcommand{\disc}{\textrm{disc}}
\newcommand{\balg}{\begin{algorithm}}
\newcommand{\ealg}{\end{algorithm}}
\newcommand{\difx}[1]{\frac{\partial #1}{\partial x}}
\newcommand{\supp}{\texttt{Support}}
\newcommand{\resl}{\textrm{resl}}


\newcommand{\w}{\text{w}}
\newcommand{\Zero}{\texttt{ZERO}}
\newcommand{\izero}{\intbox \zero}
\newcommand{\EVB}{\mbox{\em EB}}
\newcommand{\SB}{\mbox{\em SB}}
\newcommand{\MRB}{\mbox{\em MRB}}
\newcommand{\SL}{\mbox{\em SL}}
\newcommand{\bfw}{{\bf w}}
\newcommand{\Refine}{\mbox{\rm Refine}}
\newcommand{\ZeroTest}{\mbox{\rm ZeroTest}}
\newcommand{\RootIsol}{\mbox{\rm RootIsoTS}}
\newcommand{\RootIsolMM}{\mbox{\rm RootIsoSQFree}}
\newcommand{\bfalphaab}{\boldsymbol{\alpha}}

\newcommand{\newOld}[2]{#1}

\setlength{\topmargin}{-11mm} \setlength{\textwidth}{139mm}
\setlength{\textheight}{205mm} \pagestyle{empty}
\renewcommand\refname{References and Notes}
\renewcommand{\abstractname}{Abstract}
\renewcommand{\thefootnote}{\fnsymbol{footnote}}

\def\qed{\hfil {\vrule height5pt width2pt depth2pt}}

\def\bref#1{(\ref{#1})}

\def\R{{\mathbb{R}}}
\def\C{{\mathbb{C}}}
\def\Q{{\mathbb{Q}}}
\def\I{{\mathbb{I}}}
\def\B{{\bf B}}


\def\PS{ {\mathcal{P}} }
\def\GB{{\mathcal{G}}}
\def\IS{{\mathcal{I}}}
\def\BS{{\mathbf{BS}}}

\def\RB{\hbox{\rm{RB}}}
\def\Dis{\hbox{\rm{Dis}}}

\def\Re{\hbox{\rm{Re}}}
\def\Im{\hbox{\rm{Im}}}

\def\sharedaffiliation{\end{tabular}
\begin{tabular}{c}}

\begin{document}
\begin{frontmatter}





\title{Root Isolation of Zero-dimensional\\ Polynomial Systems with \\Linear Univariate Representation}
\author{Jin-San Cheng, Xiao-Shan Gao, Leilei Guo}

\address{
KLMM, Institute of Systems Science, AMSS, Chinese Academy of Sciences\\
Email: xgao@mmrc.iss.ac.cn,jcheng@amss.ac.cn}


\footnotetext[1]{Partially supported by a National Key Basic
Research Project of China and by a grant from NSFC.}

\begin{abstract}
In this paper, a linear univariate representation for the roots of a
zero-dimensional polynomial equation system is presented, where the
roots of the equation system are represented as linear combinations
of roots of several univariate polynomial equations. The main
advantage of this representation is that the precision of the roots
can be easily controlled. In fact, based on the linear univariate
representation, we can give the exact precisions needed for
isolating the roots of the univariate equations in order to obtain
the roots of the equation system to a given precision. As a
consequence, a root isolation algorithm for a zero-dimensional
polynomial equation system can be easily derived from its linear
univariate representation.
\end{abstract}
\begin{keyword}
Zero-dimensional polynomial system,
 linear univariate representation, local generic position, root isolation
\end{keyword}

\end{frontmatter}

\section{Introduction}
Solving polynomial equation systems is a basic problem in the field
of computational science and has important engineering applications.
In most cases, we consider zero-dimensional polynomial systems. We
will discuss how to solve this kind of systems in this paper. In
particular, we will consider how to isolate the complex roots for
such a system.

One of the basic methods to solve polynomial equation systems is
based on the concept of separating elements, which can be traced
back to Kronecker \cite{kronecker} and has been studied extensively
in the past twenty years
\cite{abrw,canny,chengjsc,gaochou,gm,gh,gls,kmh,kff,ll,renegar,rur,ynt}.
The idea of the method is to introduce a new variable  which is a linear combination of the variables to be solved
such that  takes different values when evaluated
at different roots of the polynomial equation system
. In such a case, we say that  is a {\bf
separating element} for . If  is a
{separating element} for ,  the roots of
 have the following rational univariate
representation (RUR):

where  and  are rational functions in . As a
consequence, solving multi-variate equation systems is reduced to
solving a univariate equation  and to substituting the roots
of  into rational functions .
Along this line, better complexity bounds and effective software
packages for solving polynomial equations such as the Maple package
RootFinding by Rouillier \cite{rur} and the Magma package Kronecker
by Giusti, Lecerf, and Salvy \cite{gls} are given.

The above approaches still have the following problem: for an
isolation interval  of a real root  of , to
determine the isolation interval of  under a given
precision is not a trivial task.
In this paper, we propose a new representation for the roots of a
polynomial system which will remedy this drawback.


\begin{figure}[ht]
\centering
\begin{minipage}{0.9\textwidth}
\centering
 \includegraphics[scale=0.32]{fig/distribution.eps}
 \caption{The distribution of the roots of . The red diamonds (blue crosses, black circles) are roots of  () and red (blue) boxes are neighborhoods for the red diamonds (blue crosses).}
\label{fig-slope}
\end{minipage}
\end{figure}


In the ISSAC paper \cite{lgp-bi}, based on ideas similar to
separating elements, a local generic position method is introduced
to solve bivariate polynomial systems and experimental results show
that the method is quite efficient for solving equation systems with
multiple roots.
In this paper, we extend the method to solve general
zero-dimensional polynomial systems. A local generic position for a
polynomial equation system  is also a linear combination of
the variables to be solved:  which satisfies two
conditions.
First,   is a separating element of
 for
, and the roots of  have a one-to-one
correspondence with the roots of a univariate equation .
Second, for a root  of 
represented by a root  of , all the roots 
of  corresponding to the roots of
, say , ``lifted" from
 are projected into a fixed square neighborhood of . This
``local" property is illustrated in Figure \ref{fig-slope}.
We prove that if  is a local generic
position for , then the roots of  can be
be represented as special linear combinations of the roots of
univariate equations :
 
where  are certain positive rational numbers and the
 matching  are in certain square
neighborhood of  to be defined in Section 2.
Such a representation is called a {\bf linear univariate
representation} (LUR for short)  of the polynomial system.

The main advantage of the LUR representation is that the precision
of the roots can be easily controlled. For RUR, computing solutions
with a given precision is not a trivial task as we mentioned before.
It is not easy to know with which precision to isolate the roots of
 is enough in order for the roots of the system 
to satisfy a given precision.
For LUR, precision control becomes very easy. We can give an
explicit formula for the precision of the roots of  in
order to obtain the roots of the system with a given precision. So
we can obtain the solutions of the system by refining the roots of
 at most once. Another advantage of LUR is that when we
isolate the roots of , we need only to consider a
fixed neighborhood of each root of .

We propose an algorithm to compute an LUR for a zero-dimensional
polynomial system. The key ingredients of the algorithm are to
estimate the root bounds of  and to estimate the separating
bounds for the roots of  lifted from a root of
. We adopt a computational approach to estimate such
bounds in order to obtain tight bound values. For the root bounds of
, we use Gr\"obner basis computation to obtain the generating
polynomial of the principal ideal  and use this
polynomial to estimate the root bound for the  coordinates of
the roots of .
The separating bounds for  are obtained from the isolating
boxes for the roots of the . These bounds in turn will be
used to compute the isolating boxes for the roots of .
Hence, the algorithm to compute an LUR also gives a set of isolating
boxes for the roots of .

In Section 2, we give the definition of LUR and the main result of
the paper.
In Section 3, we present an algorithm to compute an LUR of a
zero-dimensional polynomial system as well as a set of isolation
boxes of the roots of the equation system.
In Section 4, we provide some illustrative examples.
We conclude the paper in Section 5.


\section{Linear univariate representation}
In this section, we will define LUR and prove its main properties.
Let

be a zero-dimensional polynomial system in ,
where  is the field of rational numbers. Let
 
where  is the ideal generated by .
We use  to denote its complex roots in .

Since we will use rectangles to isolate complex numbers, we adopt
the following norm for a complex number :

The ``distance\footnote{The results in this section are also valid
if we use the usual distance for complex numbers.}" between two
complex numbers  and  is defined to be . It is
easy to check that this is indeed a distance satisfying the
inequality  for any complex
number .
Let  be a complex number and  a positive rational number.
Then the set of points having distance less than  with ,
that is , is an open square with
 as the center.


By an LUR, we mean a set like

where  are univariate polynomials,  and 
are positive rational numbers. The {\bf roots} of \bref{eq-lur} are
defined to be

where . Geometrically, we match a root  of
 with those roots of  inside a squared
neighborhood centered at . See Figure \ref{fig-slope} for
an illustration.
An {\bf LUR for } is a set of form \bref{eq-lur} whose
roots are exactly the roots of .

It is clear that an LUR represents the roots of  as
linear combinations of the roots of some univariate polynomial
equations. The LUR representation has the following advantage: we
can easily derive the precision of the roots of  from
that of the univariate equations as shown by the following lemma.

\begin{lem}
Let \bref{eq-lur} be an LUR for a polynomial system . If
 is a root of  and
 is an approximation of  with
precision , then the approximate root


of  has precision .
\end{lem}
\begin{pf} Since  and the
approximate root  of  has precision
,  the approximate root 
has precision no larger than
.
\end{pf}

For a zero-dimensional polynomial system , let  (), and  () be positive
rational numbers satisfying

where .
Geometrically,  is half of the root separation bound for roots
of  considered as points on a ``fiber" over each root of
,  is twice the root bound for the -th
coordinates of the roots of , and , the inverse of the
slope of certain line, is a key parameter to be used in our method.
If , , we can choose any positive number as .

For  satisfying \bref{eq-dn3}, consider the ideal

 where  is a new variable.
It is clear that  is a zero-dimensional ideal in
. And the elimination ideal
 is principal. Let  be the generator of this
ideal:

The following is the main result of this paper.
\begin{thm}\label{th-1}
If  satisfy conditions \bref{eq-dn1}, \bref{eq-dn3} and
 is defined in \bref{eq-tii}, then the corresponding set
\bref{eq-lur} is an LUR for .
\end{thm}

We will prove two lemmas which will lead to a proof for the theorem.
For a root  of , let

where .
Note that  is an open square whose center is
 and whose edge has length .
With this notation, the {roots} of \bref{eq-lur} can be written as

In Figure \ref{fig-slope},  are interior
parts of the squares.
We have
\begin{lem}\label{lm-ww0}
Under assumptions of Theorem \ref{th-1}, we have
i=1,\ldots,n-1 where   and

\end{lem}
\begin{proof}
From the definition of  in \bref{eq-bi},  is
a root of ,  is a root of , and
each root of  has the form \bref{eq-ww3}.

We first prove that  .
Using \bref{eq-dn2} and \bref{eq-dn3}, we have

As a consequence,  is in .

We now prove that . By \bref{eq-dn1}, we have . Therefore, for any , by \bref{eq-ww1}, we have . Hence 
and the lemma is proved.
\end{proof}



For rational numbers , we call  a
{\bf separating element} of , if ,   implies  (see paper \cite{rur}).

Theorem \ref{th-1} follows from (d) of the following lemma.
\begin{lem}\label{lm-ww1}
Under assumptions of Theorem \ref{th-1}, for , we have
\begin{description}
\item[] (a)  is a separating element of
.

\item[] (b) Each root  of  is in an 
for a root  of . Furthermore, if
,
then all roots of  in  are of
the following form

where .

\item[] (c)  are disjoint for all roots  of
.

\item[] (d)  is an LUR for .
\end{description}
\end{lem}
\begin{proof}
We will prove the lemma by induction on .
For , since ,  statements (a) and (d) are
obviously true. We do not need prove (b).
From \bref{eq-dn1}, we have . As a consequence,  are
disjoint for all roots  of . Statement (c) is
proved.

Suppose that the result is correct for . We will prove
the result for .

We first prove statement (a). Let  and
 be two distinct elements in
.
We consider two cases. If  is different from
, then by the induction hypothesis
 is also
different from . By (c) of the induction hypothesis,
 and  are disjoint.
By Lemma \ref{lm-ww0},   and
. Then, in this case we have
.
In the second case, we have
. Then,
 and . It is clear
that  is
 different from . Thus, (a) is proved.

We now prove statement (b). Use notations in \bref{eq-ww4} and
\bref{eq-ww3}. By Lemma \ref{lm-ww0}, we have
. Then, each root of
 is in an  for a root
 of .
Let  such that
 is another element in  .
We claim that  must be the same as
. Otherwise, by the induction hypothesis
(a),  is different from . By the induction hypothesis
(c),  and   are
disjoint which is impossible since by Lemma \ref{lm-ww0},
 and
.
Thus,  and hence
. This proves
equation \bref{eq-ww2} and hence statement (b).

We now prove statement (c).
Use notations in \bref{eq-ww4} and \bref{eq-ww3}. By Lemma
\ref{lm-ww0}, .
As a consequence, we need only to prove that the squares
 contained in the same
 are disjoint. Let 
be two roots of  in . By
statement (b) just proved, we have

where  is defined in \bref{eq-ww4} and
, 
are roots of .   Then, by \bref{eq-dn1},

So,  and  are
disjoint. Statement (c) is proved.

Finally, we prove statement (d). Let
 and
. By the induction hypothesis, we
have

where . Note
that the inequality is equivalent to that .
By \bref{eq-ww2}, we can recover the  with the following
equation
 
From Lemma \ref{lm-ww0}, we have
 or equivalently
. Then the root

is a root of the LUR . We thus proved that the roots of  are the
same as the roots of the LUR and hence statement (d).
\end{proof}

We have the following corollaries.
\begin{cor}\label{cor-l1}
If \bref{eq-lur} is an LUR for a polynomial system , then the
roots of  are in a one to one correspondence with the roots
of  for .
\end{cor}
\begin{proof}
Let . Then
 is a root of .
By (a) of Lemma \ref{lm-ww1}, this mapping is injective. This
mapping is clearly surjective.
\end{proof}

\begin{cor}
The real roots of  are in  a one to one correspondence with
the real roots of . More precisely, if  is a
real root of , then in the corresponding root

of ,  is a real root of .
\end{cor}
\begin{proof}
For each root  of , let   be
the open square neighborhood of  defined in \bref{eq-nbh}.
We claim that a real root of  cannot be in
 for a complex root  of .
Since  has rational numbers as coefficients, the complex
roots of  appear as pairs which are symmetric with the
real axis and the open square neighborhoods for a pair of complex
roots are disjoint. Then the open square neighborhood of any complex
root has no intersection with the real axis. This proves the claim.
As a consequence, if  is a real root of , then
 is in the open square neighborhood of a real root
 of . Repeating the process, we obtain a
real root

for  where all  are real numbers.
The other side is obvious: a real root of  will correspond to
a real root of .
\end{proof}
From the lemma, we can consider the real roots of an LUR if we only
interest in the real roots of .


\section{Algorithm for computing an LUR and roots isolation}
In this section, we will present an algorithm to compute an LUR for
a zero-dimensional polynomial system. The algorithm will isolate the
roots of the system in  at the same time.


\subsection{Complex isolation intervals and isolation boxes}
We introduce some basic concepts of interval computation. For more
details, we refer to \cite{intervalbook}.

Let  denote the set of intervals of the form ,
where . The {\bf length} of an interval  is defined to be .
Assuming , we define the distance between two intervals
as


A pair of intervals   is called a {\bf complex
interval}, which represents a rectangle in the complex plane. A
complex number  () is said to be in a complex
interval  if  and  .
 The length of a complex interval  is defined to
be .
We define the distance between two complex intervals as {\small
}

A set  of disjoint complex intervals  is called {\bf
isolation intervals} of  if each interval in 
contains only one root of  and each root of  is
contained in one interval in .
Methods to isolate the complex roots of a univariate polynomial
equation are given in \cite{collins-c1,pinkert,sy,wilf}.

Let  denote the set of complex intervals. An element
 in  is called a {\bf
complex box}.
A set  of {\bf isolation boxes} for a zero dimensional
polynomial system  in  is a set of
disjoint complex boxes in   such that each box in
 contains only one root of  and each
root of  is in one of the boxes.
Furthermore,  if each box  
in  satisfies , then  is called an {\bf
-isolation boxes} of .
The aim of this paper is to compute a set of -isolation
boxes for a zero-dimensional polynomial system .

\subsection{Gr\"obner basis and computation of  and }
In this subsection, we will show how to use Gr\"obner basis to
compute  defined in \bref{eq-dn2} and  defined in
\bref{eq-bi} supposing the parameters  are given.

Let  be a zero-dimensional polynomial
system. Then   is a
finite dimensional linear space over . Let  be a
Grbner basis of  with any ordering. Then the set of
remainder monomials

forms a basis of  as a linear space over , where
 are non-negative integers.

Let . Then  gives a multiplication map

defined by  for .
It is clear that  is a linear map. We can construct the matrix
representation for  from   and . The following theorem
is a basic property for  \cite{lazard1}.

\begin{thm}[Stickelberger's Theorem]\label{thm-laz}
Assume that  has a finite positive
number of solutions over . The eigenvalues of  are the
values of  at the roots of  over  with respect to
multiplicities of the roots of .
\end{thm}


Let  be rational numbers satisfying \bref{eq-dn3} and


We can compute  and  such that

In fact, we can construct the matrixes for  and  based
on  and , and  and  are the minimal
polynomials for  and , respectively  (See reference
\cite{cox}).
Note that we can also use the method introduced in reference
\cite{fglm} to compute .

From Theorem \ref{thm-laz} and (a) of Lemma \ref{lm-ww1},  the
-th coordinates of all the roots of  are roots of
, and all the possible values of  on the roots of  are roots of
.

Now we show how to estimate  defined in \bref{eq-dn2}. At
first, compute . Then we have the
following result.

\begin{lem}\label{lm-rb1}
Use the notations introduced before. Then

satisfies the condition \bref{eq-dn2}, where  is the root
bound of a univariate polynomial equation .
\end{lem}
\begin{pf}
The lemma is obvious since for any root ,  is a root of .
\end{pf}


\subsection{Theoretical preparations for the algorithm}

In this subsection, we will outline an algorithm to compute an LUR
for  and to isolate the roots of  under a given
precision . The algorithm is based on an interval version
of Theorem \ref{th-1}.

We define the {\bf isolation boxes} for an LUR defined in
\bref{eq-lur} as: {\small }
where  is a set of isolation boxes for the complex
roots of  and .
In  Theorem \ref{thm-eps1} to be proved below, we will give criteria
under which the
isolation boxes for  are the isolation boxes of an LUR.\\

Let  be a zero-dimensional polynomial
system.
We will compute an LUR for  and a set of
-isolation boxes for the roots of 
inductively.

At first, consider . We compute  as defined in equation
\bref{eq-ti}. Let  be a set of isolation intervals
for the complex roots of . Then, we can set  to be
the minimal distance between any two intervals in .

For  from  to , assuming that we have computed

\quad An LUR  for .

\quad A set of -isolation boxes for .

\quad The parameter .

We will show how to compute  , , ,
, and a set of -isolation boxes of the roots of
. The procedure consists of three steps.



{\bf Step 1.} We will compute  as introduced in
\bref{eq-dn2} and \bref{eq-dn3}. With , we can compute
 as defined in \bref{eq-ti}.

Here  can be computed with the method in Lemma
\ref{lm-rb1}. Note that  is known from the induction
hypotheses. Then we can choose a rational number  such that
condition \bref{eq-dn3} is valid. Finally,  can be
computed with the methods mentioned below equation \bref{eq-ti}.

{\bf Step 2.} We are going to compute the isolation intervals of the
roots of . Let   be a root of
. We are going to find the roots of  ``lifted"
from , that is, roots of the form

To do that, we need only to find a set of isolation intervals for
 with lengths no larger than , since we
already have an -box for .

Let

Then,  is a root of .
By (b) of Lemma \ref{lm-ww1} the roots  of 
correspond to  are

We have

\begin{lem}\label{lm-rr1}
Let  be an isolation interval for the
root  of  such that 
where . Then all  in
\bref{eq-n12} are in the following complex interval

Furthermore, the intervals  are disjoint for all
roots  of .
\end{lem}
\begin{pf}
In Figure \ref{fig-box1}, let square  be  and  square
.
By equation \bref{eq-ww1}, we know  So,  is inside .
Let rectangle  be the complex interval  and
rectangle  the complex interval 
which is obtained by adding  in each direction of the
rectangle .
Then,  contains  and hence
 is inside .

For any , we have  where  is one of the points  to make
 maximal. It is clear that . So, ,
. Since 
is the set of complex numbers satisfying ,
we have .
By (c) of Lemma \ref{lm-ww1},  are disjoint for
all roots of . Then   are disjoint for
all roots of  too.
\end{pf}

\begin{figure}[ht]
\centering
\begin{minipage}{0.7\textwidth}
\centering
 \includegraphics[scale=0.40]{fig/box1.eps}
 \caption{The isolation intervals , ,  for a root  of .
  is represented by .}
\label{fig-box1}
\end{minipage}
\end{figure}

The following lemma shows what is the precision needed to isolate
the roots of  in order for the isolation boxes to be
contained in some . It can be seen as an effective
version of the fact  proved in
Lemma \ref{lm-ww0}.
\begin{lem}\label{lm-rr2}
Use the notations introduced in Lemma \ref{lm-rr1}. Let  be a set of -isolation boxes for
the roots  of . Then, for all


\end{lem}
\begin{pf}
From the proof of Lemma \ref{lm-rr1},  the distance from  to
line  in Figure \ref{fig-box1} is  and the distance from
 to line  is less than . So, the
distance between line  and  is at least
. This fact is also valid for the pairs of lines
, , and . Since the isolation boxes
 are of size smaller than  and their centers are
inside , the boxes  must be inside .
Note that  is rectangle . Since  is
inside both  and rectangle  and the distance from
 to each edge of  is , the distance
between   and  must be smaller than .
\end{pf}

Isolate the roots of  with precision
. By Lemma \ref{lm-rr2}, all the roots are in one
of the intervals  where  is an isolation interval
for a root  of .

Let  be the
isolation intervals for the roots  of 
inside .
Then from \bref{eq-n12}, the isolation intervals of
 are



We have
\begin{lem}With the notations introduced above, if the following
conditions


are valid, then  defined in \bref{eq-inti} are
-isolation intervals of  in equation \bref
{eq-n10}.
\end{lem}
\begin{pf}
It is clear that condition \bref{eq-p1} is used to ensure the
precision: .

We consider \bref{eq-cd} below. Assume that  are any two intervals defined in \bref{eq-inti}
for the -th coordinates of the roots
,
 of , respectively.
Since we have derived the -isolation boxes for the roots
of , we need only to ensure that the intervals of the
-th coordinates of the roots of  lifted from a
fixed root of  are isolation intervals. That is, to show
.

Assume that  and
 are the isolation intervals
of the roots ,  of . Here ,
 correspond to ,
, respectively. So 
correspond to , respectively. Assume that
. Then we have {\small
} and


 and  are disjoint since they are isolation intervals of
. So

It is clear that  if  or .
Then we conclude if inequality \bref{eq-cd} is true, then
. This proves the lemma.
\end{pf}

Geometrically,  is the separation bound for the roots of
 corresponds to those roots of  lifted from
the root of  corresponding to the root  of
.

\begin{rem}\label{rem-2} Note that in \bref{eq-cd}, we obtain
 first and
 later. We will refine the
isolation interval  of  such that inequality
\bref{eq-cd} is true. After the refinement, all other conditions are
still valid. We need to do this kind of refinement only once.
\end{rem}

As a consequence of the above lemma, we have
\begin{cor}\label{cor-b2}
Let  be an -isolation box for the root 
of  and  defined in \bref{eq-inti}. If
\bref{eq-p1}, \bref{eq-cd} are valid, then  are -isolation boxes for the roots
 of , which are lifted from .
\end{cor}


{\bf Step 3.} We will show how to compute  from the
isolation intervals of .

\begin{lem}\label{lem-di}
Let

where  is the minimal distance between any two isolation
intervals of . Then  satisfies conditions
\bref{eq-dn1}.
\end{lem}
\begin{pf}
Let  and  be two different roots of
 defined in \bref{eq-n12}. Then we have

Therefore,   is the parameter defined in
\bref{eq-dn0}, where  is determined as in \bref{eq-cd}. It
is clear that  is not larger than  which is the
minimal distance between any two isolation intervals of
. Then, the first condition in \bref{eq-dn1} is
satisfied.
In order for the second condition in \bref{eq-dn1} to be satisfied,
we also require  So the lemma is
proved.
\end{pf}




We can summarize the result as the following theorem which is an
interval version of Theorem \ref{th-1}.


\begin{thm}\label{thm-eps}
Let \bref{eq-lur} be an LUR such that , , and 
satisfy \bref{eq-di}, \bref{eq-dn2}, and \bref{eq-dn3} respectively,
 the -isolation boxes for the roots of
, and . If
 where ,
 then \bref{eq-ibox} is a set of -isolation boxes  for .
\end{thm}
\begin{pf}
We  first explain the functions of the inequalities in
\bref{eq-eps1}.
The first two inequalities in \bref{eq-eps1} are introduced in
\bref{eq-p1} to ensure the  precision for the isolation
boxes.
The third inequality in \bref{eq-eps1} is introduced in Lemma
\ref{lm-rr1} to ensure  and
 are disjoint.
The fourth inequality is introduced in Lemma \ref{lm-rr2} to ensure
the isolation intervals of the roots of  are inside
their corresponding interval .
The last inequality is introduced in \bref{eq-cd} to ensure the
recovered isolation boxes of  are disjoint.

Now the theorem is a consequence of Corollary \ref{cor-b2}. Here, we
give the explicit expression for the isolation boxes. The expression
for interval  in \bref{eq-inti} is directly given. The
matching condition  is from
condition \bref{eq-rr2}.
\end{pf}

We have the following effective version of Theorems \ref{th-1} and
\ref{thm-eps} by giving an explicit formula for .
\begin{thm} \label{thm-eps1} Using the same notations as Theorem \ref{thm-eps}.
Let  be the given precision to isolate the roots of
.
Let
\
where , , .
If we isolate the roots of  with precision ,
then  \bref{eq-ibox} is a set of -isolation boxes  for
.
\end{thm}
\begin{proof}
By \bref{eq-eps3}, we have
  and
 .
Then the second inequality in \bref{eq-eps1},
, is valid. All
other inequalities in \bref{eq-eps1} are clearly satisfied and the
theorem is proved.
\end{proof}

We can also compute the multiplicities of the roots with the LUR
algorithm.

\begin{cor}
If we compute the last univariate polynomial  in the LUR as
the characteristic polynomial of , then the multiplicities of
the roots of  are the multiplicities of the corresponding
roots of .
\end{cor}
\begin{proof}
By (a) of Lemma \ref{lm-ww1},  is a separating element. By Theorem \ref{thm-laz}, the
characteristic polynomial of  keeps the multiplicities of the
roots of the system. The corollary is proved.
\end{proof}

\subsection{Algorithm}

Now, we can give the full algorithm based on LUR.

\begin{alg} \label{alg-1}
The input is a zero dimensional polynomial system
 in  and a positive
rational number . The output is an LUR for  and a set
of -isolation boxes for the roots of .
\end{alg}

\begin{description}
\item[S1]
Compute a Grbner basis  of  with the graded
reverse lexicographic order and a monomial basis  for linear
space  over .

\item[S2]
Compute  as defined in \bref{eq-ti} with the method given in
Section 3.2; compute a set of -isolation boxes
 for the complex roots of ; set  .

\item[S3]
For , do steps  {\bf S4}-{\bf S9}; output the set of
boxes \bref{eq-ibox}.

\item[S4]
Compute  with the method in Lemma
\ref{lm-rb1}. Select a rational number  such that condition
\bref{eq-dn3} is valid.

\item[S5]
Compute  as defined in \bref{eq-ti} with the method
given in Section 3.2.

\item[S6]
Set  and compute a set of
-isolation boxes  for the
complex roots of 

\item[S7]
Set .

\item[S8]
Compute  with formula \bref{eq-di}.

\item[S9]
Compute  with formula \bref{eq-eps3}; refine the
isolation boxes  of  with precision
; still denote the isolation boxes as .

\end{description}

\begin{rem}
From Lemma \ref{lm-rr1}, the roots of  are in the
rectangle .
So, we need only to isolate the roots of  inside these
rectangles. This property is very useful in practice, see Figure
\ref{fig-slope} for an illustration.
\end{rem}


\section{Examples}
In this section, we will give some examples to illustrate our
method.

We first use the following  example to show how to isolate the roots
of a system with our method. Note that with an LUR, we can also use
floating point numbers to compute the roots of  if the
floating point number can provide the required precision as shown in
the following example.

\begin{exmp} Consider the system
 The coordinate order is .

The Grbner basis  with the graded reverse
lexicographic order  of  is: {\small
} The leading monomials of the
basis are . So the monomial basis of the
quotient algebra  is .

Let , we can compute:
 The minimal polynomial of
 is  Compute its
complex roots with the function ``Analytic'' in Maple package
[RootFinding], we obtain

Computing the roots distance with formula \bref{eq-rtsep}, we obtain
. We can set 

In a similar way, we compute  and its minimal polynomial
. The root bound of  is
. So we have . Since , we
set  Let . We can compute a matrix
 and its minimal polynomial

Computing its complex roots, we have

Computing the minimal distance between any two roots, we have
. From equation \bref{eq-di}, we can obtain



Compute  and its minimal polynomial . Then the root bound of  is . We have
. We can set  Let . Compute
 and its minimal polynomial

Computing its complex roots, we have


We use  to represent the -th element of .  are similarly defined. Since
 and the
absolute values of its  real part and imaginary part are lese than
,  is a root of
. But  and its imaginary part is larger than . Then
 does not correspond to .
 and the
absolute values of its real part and imaginary part are lese than
, so

is a root of . In a similar way, we can find all other
complex roots of . And from Theorem \ref{thm-eps1}, we can
set , where  is the
given precision. So if we refine the roots of  to five
digits, we can obtain the roots of  with three digits.

We also obtain an LUR for  as follows:


The roots of  are: 

Assuming that the final precision for the real roots of the system
is  and isolating the real roots of 
with precision , respectively, we can
obtain the following two real roots of  with the given
precision: {\footnotesize

}
\end{exmp}
In the next example, we will compare our method with RUR \cite{rur}.
\begin{exmp} Consider the following example from paper \cite{rur}.
 The
coordinate order is .

The RUR is as follows.

where


 An LUR of  is as follows:
{\small }
where
{\small

}

The roots of  are: 

\end{exmp}

\section{Conclusion}
We give a new representation, LUR, for the roots of a
zero-dimensional polynomial system  and propose an
algorithm to isolate the roots of  under a given
precision . For the LUR, the roots of the system are
represented as a linear combination of the roots of some univariate
polynomial equations. The main advantage of LUR is that precision
control of the roots of the system is much clearer.

The main drawback of the LUR method is that when the parameters
 becomes very small, the coefficients of  could become
very large, which will slow down the algorithm. To improve the
efficiency of the LUR algorithm is our future work.


\begin{thebibliography}{99}
\bibitem{abrw}
M.~E. Alonso, E.~Becker, M.~F. Roy, and T.~W{\"{o}}rmann.
\newblock Zeros, multiplicities, and idempotents for zerodimensional systems.
\newblock In {\em Algorithms in Algebraic Geometry and Applicatiobns},
  1--15. Birkhauser, 1996.


\bibitem{canny}
J.~F. Canny.
\newblock Some algebraic and geometric computation in pspace.
\newblock In {\em ACM Symp. on Theory of Computing}, 460--469. SIGACT,
  1988.
\bibitem{lgp-bi}
J.~S. Cheng, X.~S. Gao, J. Li,
\newblock Root isolation for bivariate polynomial
systems with local generic position method.
\newblock {\em Proc. ISSAC 2009},
103-109, ACM Press, 2009.
\bibitem{chengjsc}
J.~S. Cheng, X.~S. Gao, and C.~K. Yap.
\newblock Complete numerical isolation of real roots in zero-dimensional
  triangular systems.
\newblock {\em Journal of Symbolic Computation}, 44(7): 768--785, 2009.

\bibitem{collins-c1}
G.~E. Collins and W. Krandick.
\newblock A tangent-secant method for
polynomial complex root calculation.
\newblock {\em Proc. ISSAC 1996},
137-141, ACM Press, 1996.

\bibitem{cox}
D.~A. Cox.
\newblock Solving equations via algebras.
\newblock In {\em Solving Polnomial Equations}, Editors: Alicia Dichenstein  Ioannis Z. Emiris, Springer, 2005.

\bibitem{fglm} J. C. Faug\`{e}re, P. Gianni, d. Lazard, and T. Mora,
\newblock Efficient computation of zero-dimensional Gr\"{o}bner basis by
changing of order.
\newblock {\em Journal of Symbolic Computation}, 16(4): 329-344, 1993.

\bibitem{gaochou}
X.~S. Gao and S.~C. Chou.
\newblock On the theory of resolvents and its applications.
\newblock {\em Sys. Sci. and Math. Sci.}, 12: 17--30, 1999.


\bibitem{gm}
P. Gianni and T. Mora.
 \newblock Algebraic solution of systems of polynomial equations using Groebner bases.
 \newblock {\em AAECC5}, LNCS 356, 247-257, 1989.



\bibitem{gh}
M.~Giusti and J.~Heintz.
\newblock Algorithmes - disons rapides -pour la d\`ecomposition d'une
  vari\`et\'e alg\'ebrique en composantes irr\'educibles et
  \'equidimensionnelles.
\newblock In {\em Proc MEGA' 90}, pages 169--193. Birkh{\"a}user, 1991.

\bibitem{gls} M. Giusti, G. Lecerf, and B. Salvy,
\newblock em A Grbner free alternative for polynomial system solving.
 \newblock {\em Journal of Complexity}, 17: 154-211, 2001.

\bibitem{kmh}
H.~Kobayashi, S.~Moritsugu, and R.~W. Hogan.
\newblock Solving systems of algebraic equations.
\newblock {\em Proc. ISSAC 1988}, 139--149, ACM Press, 1988.

\bibitem{kff}
H. Kobayashi, T. Fujise, and A.~ Furukawa.
\newblock Solving systems of algebraic equations by a general elimination method.
\newblock {\em Journal of Symbolic Computation}, 5(3): 303--320, 1988.

\bibitem{kronecker}
L. Kronecker.
\newblock Grundzge einer arithmetischen theorie der algebraischen grssen.
\newblock {\em J. Reine Angew. Math.} 92: 1-22,1882.

\bibitem{ll}
Y.N. Lakshman and D. Lazard,
\newblock On the complexity of zero-dimensional algebraic systems.
\newblock In ``Effecitve Methods in Algebraic Geometry," Progess in Mathematics, 94: 217-225, Birkhuser,Basel, 1991.


\bibitem{lazard1}
D. Lazard.
\newblock Resolution des Systemes d'Equations Algebriques.
\newblock {\em Theoretical Computer Science}, 15: 77-110, 1981.

\bibitem{intervalbook}
A. Neumaier.
\newblock Interval methods for systems of equations.
\newblock Cambridge University Press, 1990.

\bibitem{pinkert}
J. R. Pinkert.
\newblock An exact method for finding the roots of a complex
polynomial.
\newblock {\em ACM Transactions on Mathematical Software} 2(4): 351-363, 1976.  

\bibitem{renegar}
J. Renegar,
\newblock On the computaional complexity and geometry of the first-order theoery of the reals.
\newblock Part I, {\em Journal of Symbolic Computation}, 13: 255-299, 1992.

\bibitem{rur}
F.~Rouillier.
\newblock Solving zero-dimensional systems through the rational univariate
  representation.
\newblock {\em Applicable Algebra in Engineering, Communication and Computing},
  9(5): 433--461, 1999.
\bibitem{RZ03} F. Rouillier and P. Zimmermann.
\newblock Efficient isolation of
polynomial real roots.
\newblock {\em J. of Comp. and App. Math.}, {162}(1):
33-50, 2003.

\bibitem{sy}
M. Sagraloff and C. K. Yap.
\newblock An efficient exact subdivision algorithm for isolating complex roots
of a polynomial and its complexity analysis.
\newblock Submitted, Oct. 2009.

\bibitem{wilf}
H. S. Wilf.
\newblock A global bisection algorithm for computing the
zeros of polynomials in the complex plane.
\newblock {\em Journal of the ACM}, 25(3): 415-420, 1978. 
\bibitem{ynt}
K.~Yokoyama, M.~Noro, and T.~Takeshima.
\newblock Computing primitive elements of extension fields.
\newblock {\em Journal of Symbolic Computation}, 8(6): 553--580, 1989.
\end{thebibliography}


\end{document} 
