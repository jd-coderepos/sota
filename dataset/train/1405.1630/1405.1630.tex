

\documentclass[letterpaper, 10 pt, conference]{ieeeconf}  



\usepackage{srcltx}
\IEEEoverridecommandlockouts                              

\overrideIEEEmargins                                      \usepackage{bm}
 \usepackage{amssymb,latexsym,amsfonts,amsmath,amscd} 
\usepackage{caption}
\usepackage{subcaption}
\usepackage{cite}    
\usepackage{float}
\usepackage{verbatim}







\usepackage{cite}
\usepackage{acronym}
 
\usepackage{graphicx}
\usepackage{paralist}
\newtheorem{definition}{Definition}
\newtheorem{theorem}{Theorem}
\newtheorem{lemma}{Lemma}
\newtheorem{proposition}{Proposition}
\newtheorem{example}{Example}

\newtheorem{assumption}{Assumption}
\newtheorem{remark}{Remark}
\newtheorem{problem}{Problem}


  \newcommand{\nat}{\mathbb{N}}         \acrodef{sa}[SA]{semiautomaton}
\acrodef{fsa}[FSA]{finite state automaton}
\acrodef{dfa}[DFA]{deterministic finite state automaton}
\acrodef{pta}[PTA]{prefix tree acceptor}
\acrodef{smt}[SMT]{Satisfiability Modulo Theories}
\acrodef{ltl}[LTL]{Linear temporal logic}
\acrodef{ts}[TS]{transition system}


\newcommand{\truev}{\mathsf{true}}
\newcommand{\falsev}{\mathsf{false}}

\newcommand{\rayna}[1]{\textcolor{red}{{\bf Rayna:} #1}}\acrodef{pomdp}[POMDP]{partially observable Markov decision process}

\acrodef{act}[\textsc{ACT}]{abstract counterexample tree}
\acrodef{cegar}[CEGAR]{counterexample guided abstraction refinement}
\providecommand{\abs}[1]{\lvert#1\rvert}

\newcommand{\eqdef}{\overset{\mathrm{def}}{=\joinrel=}}
\providecommand{\card}[1]{\lVert#1\rVert}



\title{Abstractions and sensor design in partial-information, reactive
controller synthesis}



\author{Jie Fu, Rayna Dimitrova and Ufuk Topcu\thanks{This work was supported in part by the AFOSR (FA9550-12-1-0302) and ONR (N00014-13-1-0778).}
\thanks{The authors are with the University of Delaware (USA), Max
  Planck Institute for Software Systems (Germany), and the University
  of Pennsylvania (USA), respectively.}}

\begin{document}


\maketitle
\thispagestyle{empty}
\pagestyle{empty}


\begin{abstract}
  Automated synthesis of reactive control protocols from temporal
  logic specifications has recently attracted considerable attention
  in various applications in, for example, robotic motion planning,
  network management, and hardware design. An implicit and often
  unrealistic assumption in this past work is the availability of
  complete and precise sensing information during the execution of the
  controllers. In this paper, we use an abstraction procedure for
  systems with partial observation and propose a formalism to
  investigate effects of limitations in sensing. The abstraction
  procedure enables the existing synthesis methods with partial
  observation to be applicable and efficient for systems with infinite
  (or finite but large number of) states. This formalism enables us to
  systematically discover sensing modalities necessary in order to
  render the underlying synthesis problems feasible. 
  We use counterexamples, which witness unrealizability potentially
  due to the limitations in sensing and the coarseness in the abstract
  system, and interpolation-based techniques to refine the model and
  the sensing modalities, i.e., to identify new sensors to be
  included, in such synthesis problems. We demonstrate the method on
  examples from robotic motion planning.
\end{abstract}


\section{Introduction}
\label{sec:intro}

Automatically synthesizing reactive controllers with proofs of
correctness for given temporal logic specifications has
emerged as a methodology complementing post-design verification
efforts in building assurance in system operation. Its recent
applications include autonomous robots \cite{ram-hadas,itac}, hardware
design \cite{GR(1)case}, and vehicle management systems
\cite{vms0}. This increasing interest is partly due to both
theoretical advances \cite{pnueli-hardmis, piterman} and software
toolset developments
\cite{pnueli2010jtlv,bloem2010ratsy,wongpiromsarn2011tulip}.

An implicit and often unrealistic assumption in the past work on
reactive synthesis is the availability of complete and precise 
information during the execution of controllers. For example,
while navigating through a workspace, a robot rarely (if ever) has
global awareness about its surrounding dynamic environment and its
sensing of even its own configuration is imprecise. This paper takes
an initial step toward explicitly accounting for the effects of such
incompleteness and imperfectness in sensing (and other means through
which information is revealed to the controller at runtime). 

More specifically, we use an abstraction procedure for games with
partial observation \cite{rayna-report} and propose a formalism to
investigate the effects of limitations in sensing. The abstraction
reduces the size of the control synthesis problem with sensing
limitations by focusing on relevant properties of the control
objective and enables automatic synthesis for systems with potentially
large state spaces using the solutions for partial-information,
turn-based, temporal-logic games \cite{apt2011lectures,
    chatterjee2006algorithms}. Given unrealizable specifications,
where a potential cause for unrealizability is the lack of runtime
information, a simple question we investigate is
what new sensing modalities and with what precision shall be included
in order to render the underlying synthesis problem feasible. We focus on
particular safety type temporal logic specifications for which
counterexamples witness the unrealizability. Using such
counterexamples and interpolation-based techniques
\cite{cimatti2012smt}, the method searches for predicates to be
included in the abstraction. We interpret addition of such
newly discovered predicates as abstraction refinements as well as
adding new sensing modalities or increasing the precision of the
existing sensors.
Besides the partial-information, turn-based games (see
\cite{arnold2003games,Martin2006} in addition to the earlier
references mentioned)  the problem we study in this paper has
similarities with the partially observable Markov decision processes
\cite{kaelbling1998planning,pineau2002high,kurniawati2008sarsop}. The
main deviation in the formalism we employ is the inclusion of a second
player which represents a dynamic, possibly adversarial environment,
particularly well suited for reactive synthesis in a number of
applications, for example, autonomous navigation.

The rest of the paper is organized as follows. We begin with an
overview of the setup, problem, and solution approach. In section
\ref{sec:problemformulation}, we discuss some preliminaries as they
build toward a formal statement of the problem. The solution approach
is detailed in the following two sections in which first an
abstraction procedure and then refinements in abstractions based on
counterexamples are presented. This presentation partly follows the
development in \cite{rayna-report}. Section \ref{subsec:sensorconf}
gives an interpretation of the results in the reconfiguration of
sensing modalities and section \ref{subsec:case} is on a case
study. Throughout the paper, we consider motivating and running
examples loosely from the context of autonomous robotic motion
planning subject to temporal logic specifications.




\section{Overview}
\label{sec:overview}

We begin with a running example and an overview of the problem and our solution approach.

\begin{example}
\label{ex}
Consider a robot in the environment as shown in Fig.~\ref{fig:ex} with
two other dynamic obstacles. The position of this robot is represented
by variables  and  in the coordinate system and the initial
position is at  and . At each time instance, it can
apply the control input  to change its position. The domain of 
is . At each time, with input
 (resp. ) the robot can move in the -direction
precisely with  (resp. ) units, however, in the -direction
there is uncertainty: by  (resp. ), the robot
proceeds some distance ranging from  to  (resp. from  to ) unit.  
There are two uncontrollable moving
obstacles, obj1 and obj2, whose behaviors are not known a priori but are known to
satisfy certain temporal logic formulas. 
Suppose as an
example design question that the available sensor for  has slow
sampling rate, for example, the value of  cannot be observed at
every time instance.
Can it eventually reach and stay in  while
avoiding all the obstacles and not hitting the walls?
\end{example}
\begin{figure}[H]
\vspace{-2ex}
  \centering
  \includegraphics[width=0.3\textwidth]{map}
  \caption{An environment including a robot (represented by the
    red dot) and two dynamic obstacles, obj1, obj2. Regions  and
     are connected by a door.}
  \label{fig:ex}
\end{figure}
\vspace{-2ex}

A reactive controller senses the environment and decides an
action in response based on that
sensor reading (or a finite history of sensor readings).  For control
synthesis in reactive systems with partial observation, two problems
are critical.  One is a synthesis problem: given the \emph{current} sensor
design, is there a controller that realizes the specification?  Another
is a design problem: given an unrealizable specification, would it be
possible to find a controller by introducing new sensing modalities?
If so, what are the necessary modalities to add?


To answer these questions, we consider the counterexample guided
abstraction refinement procedure for two-player games with partial
observation in \cite{DimitrovaF08} . First, we formalize the interaction between a system
and its environment as a \emph{(concrete) game}.  A
safety specification determines the winning conditions for both
players. Then, an initial set of  predicates is selected to construct an
\emph{abstract game} with finite state space. The abstraction is
\emph{sound} in the sense that if the specification is realizable with the
system's partial observation in the abstract game, then it is so in
the concrete game.  However, if there does not exist such a
controller, a \emph{counterexample} that exhibits a violation of the
specification can be found. The procedure checks whether
this counterexample exists in the concrete game. If it does not, i.e., it is 
spurious, then the abstract game is refined until a
controller is obtained, or a genuine counterexample is found.

In the latter case, the task is not realizable by the system with its
current sensor design.  Then, we check whether it is realizable under
the assumption of complete information, using the same abstraction
refinement procedure. If the answer is yes, then the set of predicates
obtained in the abstraction refinement indicates the sensing
modalities that are sufficient, with respect to the given specification.

\section{Problem formulation}
\label{sec:problemformulation}


In this section we provide necessary background for presenting the
results in this paper.  
For a variable  we denote with  its domain. Given a set of
variables , a \emph{state}  is a function  that maps each variable  to a value in
. For , we write  for the projection
of  on .  Let the set of states over  be .  A
\emph{predicate} (atomic formula)  is a statement over a set of variables .  For
a given state ,  has a unique value --- (1) or 
 (0). We write  if  is evaluated to  by the state
 . Otherwise, we write . Given a state , we write , if the valuation of
 at  is . Otherwise, we write .
Given a formula  over a
set of predicates , let  be the set of predicates that occur in . 
 A \emph{substitution} of all variables  in  with
the set of new variables  is denoted .

\subsection{The model}
\label{subsec:model}
A \emph{(first-order) transition system} symbolically represents an
infinite-state transition system \cite{cimatti2012smt}.
\begin{definition}
A \ac{ts}  is a tuple  with components as follows.
\begin{itemize}
\item  is a finite set of variables. 

\item  is a (quantifier-free) first-order logic formula describing the
  transition relation.  relates the variables
   which represent the \emph{current} state, with the
  variables  which represent the state after this
  transition.
 
\item  is a (quantifier-free) first-order formula over 
  which denotes the set of initial states  of
  . \end{itemize}
\end{definition}


The interaction between system and its environment is captured by a
reactive system formalized as a \ac{ts}.  
  \begin{example}
\label{ex2}
    We consider a modified version of Example \ref{ex} in which the
    environment does not contain any obstacle or internal walls.
    The set of variables is  where  is a
    Boolean variable. When , the values of variables  
    are updated. Formally,  the transition relation is



Initially,  holds.
  \end{example}
  A \ac{ts} can be considered in a game formulation in which the
  system is player 1 and the environment is player 2. For this
  purpose, the set of variables  is partitioned into , where  is the set of \emph{input variables},
  controlled by the environment, and  is the set of
  \emph{output variables}, controlled by the system, and  is a
  Boolean \emph{turn variable} indicating whose turn it is to make a transition:
   for the system and  for the environment.  In
  Example~\ref{ex2}, the set of input variables is , the
  set of output variables is , and the turn variable is
  .  We assume the domain of each output variable
  is finite. Without loss of generality \footnote{For a set of output
    variables, each of which has a finite domain, one can always
    construct a single new output variable to replace the set, and the
    domain of this new variable is the Cartesian product of the
    domains of these output variables.}, let  be a
  \emph{singleton}  and , which is a finite
  alphabet.

A  \ac{ts}  defines a game structure. In
this paper, we
assume that the system and its environment do not perform concurrent
actions, and thus the game structure is turn-based.
  \begin{definition} A \emph{game structure} capturing the
    interactions of a system (player 1) and its environment (player 2)
    in a \ac{ts}  is a tuple 
    \begin{itemize}
\item  is the set of states over .  is the set of states at which player 1 makes a
  move ().  consists of the states at
  which player 2 makes a move.
 \item  is the transition relation: 
\begin{itemize}
 \item  if and only if 
  and  evaluates to .
\item  if and only if  and
   evaluates to .
\end{itemize}
 \item  is the set of
   initial states.
\end{itemize} 
\end{definition}
A \emph{run} is a finite (or infinite) sequence of states  (or ) such
that , for each 
where  is the length of . We assume the game is
nonblocking, that is, for all , there exists  such
that . This can be achieved by including 
``idle'' action in the domain of the output variable.



\begin{definition}[Sensor model]
  Assuming the output variable  and the Boolean variable  are
  globally observable, the sensor model is given as a set of formulas
  , where for each input ,
   is a formula over the set of input variables 
  such that the value of the input variable  is observable at state
   if and only if the formula  evaluates to true at
  the state .
\end{definition}

For a state , the set of \emph{observable variables} at  is
.  The \emph{observation} of  is , which is the projection of  onto the set of
variables observable at .  Two states  are \emph{
  observation-equivalent}, denoted  if and only if
. The observation-equivalence can be
extended to sequences of states: let  and , for  and (or ). Two runs () are observation equivalent, denoted ,
if and only if .

This sensor model is able to capture both global and local sensing
modalities: if a variable  is globally observable (globally
unobservable),  (resp. ). Here  and  are symbols for unconditional true
and false, respectively. As an example of a local sensing modality,
consider a sensor model in which an obstacle at  is
observable if it is in close proximity of the robot at , can be
described as .

 


\subsection{Specification language}
We use \ac{ltl} formulas \cite{emerson1990temporal} to specify a set
of desired system properties such as safety, liveness, persistence  and
stability.  


In this paper, we consider safety objectives: the given specification
is in the form , where  is the
\ac{ltl} operator for ``always'' and  is a formula
specifying a set of unsafe states  The objective of the system is to always avoid the
states in  and the goal of the environment is to drive the game
into a state in .  


Let  be the designated initial state of the system.  We
obtain the game ,
corresponding to the reactive system  with the initial
state .  From now on,  and  are
referred to as the \emph{concrete} game and \emph{concrete} reactive
system, respectively. The state set  is the set of \emph{concrete}
states.  
A run  is winning for player 1 if it does not 
contain any state in the set of unsafe states .


A \emph{strategy} for player  is a function  which maps a finite run  into a state
, to be reached, such that , where  is the last
state in ,  and . The set of runs in  with the
initial state  induced by a pair of strategies 
is denoted by .  Given the initial state , a
strategy  is winning for player 1, if and only if for any
strategy  of player 2, any run in  is winning
for player 1. A winning strategy for player 2 is defined dually.

Since the system (player 1) has partial observability, the strategies
it can use are limited to the following class.
\begin{definition}
  An \emph{observation-based} strategy for player 1 is a function
   that satisfies: \begin{inparaenum}[(1)]
\item  is a strategy of player 1; and 
\item for all , if , then given
  , it holds that for the output variable , , and .
\end{inparaenum}
\end{definition}
For a game with partial observation, one can use knowledge-based subset
construction to obtain a game with complete observation. The winning
strategy for player 1 in the latter is an observation-based winning
strategy for player 1 in the former. The reader is referred to
\cite{chatterjee2006algorithm} for the solution of games with partial observation.
\subsection{Problem statement}
\noindent
We now formally state the problem investigated in this paper.
\begin{problem}
  Given a transition system  with the initial state , with a sensor model  and
  a safety specification , determine
  whether there exists an observation-based strategy (i.e. controller)
   such that for any strategy of the environment  and for
  any , . If no such controller exists, then determine a new
  sensor model for which one can find such a controller, if there
  exists one.
\end{problem}







\section{Predicate abstraction}

Since the game  may have a large number of states, the synthesis methods for finite-state games cannot be directly applied or are not efficient. To remedy this problem, we apply an abstraction procedure which combines predicate abstraction and knowledge-based
subset construction and yields an
\emph{abstract finite-state} game with \emph{complete} information 
from the (symbolically represented) concrete game .

\subsection{An abstract game}

Given a \emph{finite} set of predicates, the  abstraction procedure
constructs a finite-state reactive system (game structure).  
Let
 be an indexed set of
predicates over variables .  The \emph{abstraction function}
 maps
a concrete state into a binary vector as
follows. 
where  is the th entry of binary vector .  The  \emph{concretization function}  does the reverse:
 

In the following, we omit the subscript  in the notation
for the abstraction and concretization functions wherever they are
clear from the context.  The following lemma shows that with a proper
choice of predicates, we can ensure that a set of concrete states
grouped by the abstraction function shares the same set of observable
and unobservable variables.
\begin{lemma}
\label{lm1}
Let . Then for any binary vector  and any two states , it holds that .
\end{lemma}
\begin{proof}Since for any , , 
  for any ,  has the same truth value at states  and . Thus, for any , the formula
  , for which , has the same
  value at  and . Hence, if  is observable (or
  unobservable) at , then it must be observable (or unobservable)
  at  and vice versa. 
\end{proof}

Intuitively, by including the predicates in the formulas
defining the sensor model, for each , the set of 
concrete states  share the same sets of observable and
unobservable variables. Hence, we use  to denote set of
observable/visible input variables in  and  for the set of
unobservable/hidden input variables.

A predicate  is observable at a state  if and only if the variables
in  are observable at . According to Lemma~\ref{lm1}, if
there exists  such that  is observable at , then  is
observable for all  and we say that  is observable at
. Slightly abusing the notation , the
\emph{observation of a binary vector}  is
,
which is a set of assignments for observable predicates. Two binary
vectors  are observation-equivalent, denoted , if and
only if
.

The abstraction of the concrete game 
with respect to a finite set of predicates  is a game
with \emph{complete information}
:
\begin{itemize}
\item  is the set of abstract states with sets of player 1's and player 2's abstract states respectively, \\
  and\\
 .
\item  is the initial state.
\item  where 
\begin{itemize}
\item  if and only if the following conditions
  (1), (3) and (4) are satisfied.
\item  if and only if the following
  conditions (2), (3) and (4)
  are satisfied.
\end{itemize}
\begin{enumerate}[(1)]
\item \label{must} for every
     and every
  , there exist  and
   such that ; 
\item \label{may}  there exists , ,
   and  such that ;
\item\label{rel2} for every  , there exist ,  and  such that
  ;
\item \label{rel3} for every , if
  ,  and there exist ,  and  with , then also .
\end{enumerate}
\item  is the set of unsafe states. 
\end{itemize}
In what follows, we refer to a state  as an \emph{abstract
  state}.  By definition, each  in  is a set
of observation-equivalent binary vectors in .




We relate a binary vector  with a formula  that is a conjunction such that 
 where if ,
then , otherwise . 
Further, for any , we define the following formula in disjunctive normal
form .

\addtocounter{example}{-1}
\begin{example}[cont.]We assume  is globally
  observable and  is globally unobservable and require
  that the robot shall never hit the boundary, that is,  where . Let .
 Let . The initial state of
   is , and the corresponding
  initial state in  is 
  where the values for the predicates in  are given in the same
  order in which they are listed in . Given , since , we determine  where  indicating 
  and .
\end{example}
We show that by a choice of predicates, it is ensured that for any
, all concrete states in the set  share the same observable and unobservable
variables.
\begin{lemma}
  If , then for any  and , it holds that
  .
\end{lemma}
\begin{proof}
  By Lemma \ref{lm1}, since for any , for any , , then it suffices to
  prove that for any ,  and
  . By definition,  implies that the set
  of observable (unobservable) predicates is the same in both  and
  . Thus, the set of observable (unobservable) variables that
  determines the observability of predicates has to be the same in both  and . That is,  and .
\end{proof}Let  (resp. ) be the
observable (resp. unobservable) input variables in the abstract  state
. That is,  for any , for .

\subsection{Concretization of strategies}
In the abstract game , there exists a winning strategy
for one of the players. We show that a winning strategy for the system
in  can be concretized into a set of observation-based
winning strategies for the system in .

For , the
  \emph{concretization of a strategy}  in  is a \emph{set} of strategies in , denoted
   and can be obtained as follows. Consider , ,  in the following,
  where
\vspace{-1ex}
  
and ,  for each .
Given , the output  such that there exist  and  such that . In other words,  is a concrete state
reachable from the current state  and can be abstracted into a
binary vector  in the abstract state . Intuitively,
given the run , one can find a run in the abstract system
, and uses the output of  on  to generate an
abstract state. Then  picks a reachable concrete state, which
can also be abstracted into a binary vector contained this abstract
state. A strategy  is \emph{concretizable} if . Otherwise it is \emph{spurious}.


\begin{theorem}\label{thm:soundness}
The concretization  of a player 1's winning strategy
 in  is a non-empty set that consists of observation-based winning
strategies for player 1 in the concrete game .
\end{theorem}
\begin{proof}
Follows from the proof in \cite{rayna-report}.
\end{proof}

In case there is no winning strategy for player 1 in ,
the synthesis algorithm gives us a winning strategy for player 2 in ,
which we refer to as \emph{counterexample}. Then we need to check if
it is spurious, as explained in the next
section.


\section{Abstraction refinement}
We consider an initial set of predicates  which consists
of the predicates occurring in , the predicates
describing the output  of the system, and those
occurring in the sensor model.  With this initial choice of predicates,
if player 1 wins the game , then the abstraction does not need
to be further refined,  according to Theorem \ref{thm:soundness},
the winning strategy of player 1 is concretizable in the concrete
game.  However, if player 2 wins, there exists a deterministic winning strategy  in the game
. The next step is to check if  is
spurious. If it is, then the abstract model is too coarse and needs to
be further refined.

\subsection{Constructing abstract counterexample tree}
We construct a formula from the strategy tree generated from this
counterexample that characterizes the concretizability of this counterexample in the
concrete system , and then we construct a formula from the
 tree. If the formula is satisfiable, then the counterexample is genuine.

 Given the initial state , the \ac{act} for  is
  where
  are nodes and  are edges.  Each node  in  is labeled by
 a state  and we denote the labeling .  A node
  belongs to player  if , for .


In the case of a safety specification,  is a
finite tree in which the following conditions hold. 
\begin{inparaenum}[1)]
\item The root  is labeled by , that is, .
\item If  is a player 1's node and  is not a leaf, then for
  each  such that , add a new child  of 
  and label  with . Let  for which
  .
\item If  is a player 2's node and  is not a leaf, then
  add one child  of , labeled with , where
   is the sequence of nodes' labels (states) on the path from
  the root to the node . Let  where
   is the empty string. \item For a node ,  is a leaf if either  or there is no outgoing transition from .
\item Each node has at most one parent.
\end{inparaenum} 




We illustrate the \ac{act} construction on the small
example. Fig.~\ref{ex:act} shows a fragment of \ac{act} for
Example~\ref{ex2}.  First we define the root , labeled with the
abstract state . At ,
player 1 can select any output in .  Therefore, the children
of  are , one for each input in .  For instance,
the output  labels the edge from  to  and we have
. The only child of  is , labeled with .  Clearly, the actual value of  after executing
 is . Yet the reached state  in which the
predicate  is  is because there exists some  at state , and will make  satisfied after
action . This is caused by the coarseness of the
abstraction.  If player 1 takes action , then it will have
no information about the value of the predicate , as this
predicate is not observable. In Fig.~\ref{ex:act}, each state  is related with a formula  (shown below the figure).
\vspace{-2ex}
\begin{figure}[H]
\centering
\includegraphics[width=0.45\textwidth]{atree.png}
\caption{A fragment of \ac{act} for Example~\ref{ex2}. }
\label{ex:act}
\end{figure}
\vspace{-2ex}
Note that nodes
   and  are labeled with the same state . In the formulas
  , 




For a node ,  is the set of children of 
and  is the set of paths
from the node  to a leaf.  For a path ,
the \emph{trace} of , denoted , is the sequence of labels on the edges in the path.  A
node  is related with a set of traces
. For a leaf,  by
default. For example, .

Note that in the tree structure defined here, for each node , there exists exactly one path from the root to , and hence
there is one trace  that labels that path.











We annotate each node  with a set of variables 
as  where  and  where  are observable
variables in  and  are hidden variables in  when
the trace from  is .  For example, we annotate  with
 as
 is not observable.  With this annotation, the unobservable
variables  at node  can be assigned with different values for
different traces from . It corresponds to the fact that the
concrete states, grouped into an abstract state, share the same values
for observable variables but may have different values for
unobservable ones.


In what follows, we relate a trace with a \emph{trace formula}. By
checking the satisfiability of a \emph{tree formula}, built from trace
formulas with a \ac{smt} solver, we can determine whether the
counterexample is spurious.

\subsection{Analyzing the counterexample}
Given a trace , the trace formula  is
constructed recursively as follows. 

\begin{itemize}
\item If  is a leaf, then  is a singleton. Let , which is satisfiable if there exists
  a concrete state  for  such that
  .
  
\item If  is a player 1's node and not a leaf,
  then for each , for each child
   such that , let  
\vspace{-1ex}
   where  is false if . Then let . Intuitively,  can be satisfied if
  there exist a state  for  and  for  such
  that  and  evaluate to  at  and ,
  respectively; action  enables the transition from  to
  ; and  is satisfied. The disjunction is needed because
  for a node , there can be more than one -successors.
 
\item If  is a player 2's node and not a leaf, there
  exists exactly one child of , say, ,
then for each , let 
\end{itemize}
The \emph{tree formula} is 
\vspace{-1ex}
 
\begin{theorem}
  Let  be a winning strategy for the environment in the game
  , the strategy  is genuine, i.e.,
  , if and only if the tree formula  is
  satisfiable.
\end{theorem}
\begin{proof}
The reader is referred to  \cite{DimitrovaF08}.
\end{proof}
\addtocounter{example}{-1}
\begin{example}[cont.]
 Consider, for instance, the trace
   corresponds to a labeled path
  . Since , we have 
 where . Then given ,
, where . In above equations, for instance
.
\end{example}

\subsection{Refining the abstract transition relations}
\label{subsec:refinetrans}


Given a node  and a trace , if 
is unsatisfiable, then the occurrence of the spurious counterexample
is due to the approximation made in abstracting the transition
relation. To rule out this counterexample, we need to refine the
abstract transition relation. For this purpose, we define a \emph{node
  formula}  as described below.

First, we define the \emph{pre-condition of a formula}: for a
formula  and , the pre-condition of
 with respect to , 
is a formula such that  if and only if
there exists  such that , 
and . Intuitively, at any state  that satisfies
this formula , the system, after
initiating the output , can reach a state  at which
 is satisfied.  Let . Correspondingly,
 is a formula such that  if and only if there exists ,  and
.

Now, we define the \emph{node formula}  as follows.

\begin{itemize}
\item If  is a leaf node, then 
  and .
\item If  belongs to player 1 and is not a leaf, and , then
\vspace{-1ex}

where  is false if . Here, the set
 is a set of
-successors of . \item If  belongs to player 2's and is not a leaf, then 
\vspace{-0.5ex}

where  is the unique child of node .
\end{itemize}

We augment the current set  with all
predicates that occur in the formula , i.e.,

For each node  and each  such that  is unsatisfiable, the procedure generates a set of predicates
, which are then combined with the
current predicate set to generate a new abstract game. We repeat
this procedure iteratively until a set of predicates is found such
that for any  and any ,  is satisfiable.

\subsection{Refining the abstract observation equivalence}
\label{subsec:refineobeq}
If each trace formula for the considered counterexample tree is satisfiable, but the tree formula is not, then we need to check
whether the existence of a counterexample is because of the
coarseness in the abstraction observation-equivalence.

We are in the case when for all ,   is satisfiable.  Let
. Since  is unsatisfiable, there exists a subset  of
 such that  is satisfiable and
a formula  such that  is unsatisfiable. Let the sets of
free variables in  and  be  and 
respectively. Since only observable variables are shared between
different traces,  only consists of \emph{observable}
variables.


A \emph{Craig interpolant} \cite{smullyan1995first} for the pair  is
a formula  such that \begin{inparaenum}
\item  implies ,
\item  is unsatisfiable.
\end{inparaenum}
To illustrate, consider the following example. Let  and . Clearly,
 because  in  needs
to satisfy .  Then the formula  is an
interpolant for the pair of formulas
. For a number of logical
theories commonly used in verification, including linear real
arithmetic, Craig interpolants can be automatically computed
\cite{McMillan2011}.

After computing the interpolant  for , we update the set of
predicates to be
. 
In the end, Algorithm~1 describes
the refinement procedure.
\vspace{-2ex}
\begin{figure}[ht]
\centering
\includegraphics[width=0.5\textwidth]{algorithm1}
\label{alg:abstractrefine}
\end{figure}
\vspace{-4ex}

\section{Sensor reconfiguration}
\label{subsec:sensorconf}
Suppose the task specification is unrealizable given the current
sensor model. Then, a prelude to refining the sensor is identifying
whether the source of unrealizability is limited sensing. To this end,
we first check whether it is realizable under the assumption that the
system has perfect observation over its environment. For this purpose,
we run the procedure \texttt{AbstractAndRefine} with the concrete game
 and a sensor model defined as , which means all the input variables are globally
observable.  If player 1 wins the abstract game, then we can conclude
that the task is not realizable because of the limited sensing
capability.

The procedure \texttt{SensorReconfigure}, shown as Algorithm~2, computes
a set of predicates that we need to observe in order to
satisfy a given specification. The algorithm takes the concrete
system, its current sensor model and an unrealizable specification as
input. Then by making all variables observable, we use the procedure \texttt{AbstractAndRefine} to determine if the task is realizable given complete observation. If \texttt{AbstractAndRefine} terminates with a positive answer, then, the set of predicates obtained by the refinement suffices for realizing the specification. Further, the predicates involving
unobservable variables indicate the set of new sensing modalities to
be added, and provide the requirements on the sensors'
precision and accuracy for both observable and unobservable variables.

\vspace{-2ex}
\begin{figure}[ht]
\centering
\includegraphics[width=0.5\textwidth]{algorithm2}
\label{alg:sensor_reconfig}
\end{figure}
\vspace{-4ex}

\section{Case study}
\label{subsec:case}
We demonstrate the method by revisiting Example \ref{ex}. Assuming the
dynamics of obstacle obj1 with position  is given in form
of logical formula  where 
is a formula that is satisfied when the obstacle hits the wall or the
robot.  For obstacle obj2 , we have . Here we have a liveness condition which specifies that
the robot has to visit and then stay within region . To enforce
such constraint, we introduce a Boolean variable  and set  if , which means if the robot in 
returns to , an error occurs and the system reaches an unsafe
state.
\paragraph*{Case 1}
Due to the limited sampling rate in the sensor for variable , the
system receives the exact value of  intermittently (every other
step). In this case, we introduce a predicate  such that if  
then the exact value of  is observed, otherwise there is no data
sampled. The transition relation  is modified to
capture this type of partial observation. For example, given , the transition is  where  and 
are auxiliary variables used by the robot to keep track of the upper
and lower bounds, respectively, for the value of . Intuitively,
when there is no data received, the robot makes a move such that for
every  within the upper and lower bounds, for all possible changes
in its obstacles, it will not encounter any unsafe state.  Then the
sensor data received in the next step resolves the ambiguity it
had earlier about  and an action is selected accordingly.


The abstraction refinement procedure starts with an initial set of
 predicates. After  iterations, we obtained an abstract game
in which the system has a winning strategy. The abstract game is
computed from  predicates and has  states. The computation
takes  min in a computer with  GB RAM, Intel Xeon
processors. The obtained predicates relating to the variable  falls
into the following categories:
\begin{inparaenum}[(1)]\item
Predicates over the unobservable variable : , , , , , , , , , , , .
\item
Predicates over the observable variable :  , . And \item there is no predicate over the upper bound
variable .
\end{inparaenum} The predicates relating to the obstacles  are the following: , , , , , , , , , . 

With the obtained set   of predicates, we can decide the
requirement on the precision of sensor for this task. For
every  , the constants in  has at most one
decimal place, for example, . Thus, a sensor which can
reliably measure just one decimal place would suffice. Besides, there is no need to keep track of the upper bound  for
 and also the value of   for obj2.

\paragraph*{Case 2}In this case we consider the sensor model with an
extra limitation: the robot cannot observe obj2 if it is in , or
obj1 when it is in . To capture this local sensing modality, we
made  unobservable and introduce another four
auxiliary observable variables  and . When
the robot is in , the values of  equal that of 
and . But when it is in ,  can be any point
in  following the dynamic in robot's assumption of obj1.  Similar
rules applied to  and . For the same task
specification, after 21 iterations, which takes about  min, the
abstraction refinement outputs an abstract system with  states
using  predicates and finds the robot a winning strategy.


\section{Conclusion and future work}
\label{sec:conclusion}

We took a first step toward explicitly accounting for the effects of
sensing limitations in reactive protocol synthesis. The formalism we
put forward is based on partial-information, turn-based,
temporal-logic games. Using witnesses for unrealizability in such
synthesis problems and interpolation methods, we proposed an
abstraction refinement procedure. An interpretation of this procedure
is systematical identification of new sensing modalities and precision
in existing sensors to be included in order to construct feasible
control policies in reactive synthesis problems.  A potential
bottleneck of the proposed formalism is the rapid increase in the
problem size due to, for example knowledge-based subset
construction. A pragmatic future direction is to consider so-called
lazy abstraction methods \cite{henzinger2002lazy} for partial
observation control synthesis, so that different parts of the concrete
game can be abstracted using different sets of predicates. In this
manner, the system is abstracted with different degree of precision
and thus its sensor model can also be configured ``locally" for
different parts of the system.  Furthermore, besides precision, one
would also be interested in refinements in sensing with respect to
accuracy; therefore, extensions to partially observable stochastic
two-player games are also of interest.

\bibliography{mybib}
\bibliographystyle{IEEEtran}
\end{document}
