


\titleformat{\section}[hang]{\normalfont\huge}{\thesection}{1em}{\filcenter\S\;}[]

\titleformat{\subsection}[hang]{\Large\sffamily}{\thesubsection}{1em}{\filcenter}[]


\newcommand\UseKEquationNumbering{
  \renewcommand{\theequation}{K}
}

\newcommand\UseGreekEquationNumbering{
  \renewcommand{\theequation}{\greek{equation}}
  \setcounter{equation}{0}
}

\newtheorem{problem}{Problem}
\renewcommand{\theproblem}{(\Roman{problem})}

\setcounter{page}{493}
\newcommand\mypage[1]{\newpage}






\noindent
{\Large\em Problems concerning the transformation of
symbol sequences according to given rules}\\

\medskip
\noindent
Axel Thue\marginnote{This translation by James F. Power <jpower@cs.nuim.ie>}

\medskip
\noindent
1914\marginnote{\today}


\begin{abstract}
\noindent This document is a translation of Axel Thue's paper 
\marginnote{Based on the version published in his
  \emph{Selected Mathematical Papers} as paper \#28, pp. 493-524.
  Page numbers in this document follow that pagination.}
  \emph{Probleme
  \"uber Ver\"anderungen von Zeichenreihen nach gegebenen Reglen}
(Kra. Videnskabs-Selskabets
Skrifter. I. Mat. Nat.Kl. 1914. No. 10)
\marginnote{Any margin notes (like this) are not part of the original
  paper, but typos noted here are typos in the original paper.}
\end{abstract}


\section{I}\label{sec:one} 

In a previous work I have posed the general
question whether two given concepts depicted as trees, but defined in
different ways, must be equivalent to each other.

In this paper I will deal with a problem concerning the transformation
of symbol sequences using rules.  This problem, that in certain respects is a
special case of one of the most fundamental problems that can be
posed, is also of immediate significance for the general case.
Since this task seems to be extensive and of the utmost difficulty,
I must be satisfied with only treating the question in a piecewise and
fragmentary manner.

In a previous year's work I have already solved a special case
concerning symbol sequences.  On this occasion I will just settle some
simple cases of the aforementioned general problem.  I will not enter
into a discussion here on the wider significance of investigations of
this type.

\section{II}


We are given two series of symbol sequences:

where each symbol in each sequence  and in each sequence  is a
symbol from some group of given symbols.


\vfill

\rule{.3\textwidth}{.5pt}\P \sim QC_1, C_2, \ldots, C_rX \sim C_1 \sim C_2 \sim \ldots \sim C_r \sim Y,X = YC_0 \sim C_1 \sim C_2 \sim \ldots \sim C_r \sim C_{r+1},S \equiv M A_k N,\begin{array}{rcl}
S &\equiv& M A_p L A_q N\\
H &\equiv& M B_p L A_q N\\
K &\equiv& M A_p L B_q N\\
\end{array}P \equiv QMRN \hbox{ and } MN,P \sim Q.X \sim C_1 \sim C_2 \sim \cdots \sim C_r \sim YX = Y
H_0 \sim H_1 \sim H_2 \sim \cdots \sim H_p,\\
K_0 \sim K_1 \sim K_2 \sim \cdots \sim K_q,
X \sim C_1 \sim C_2 \sim \cdots \sim C_m \sim Y\begin{array}{rcl}
C_{t-1} &\equiv& xRyz\\
C_{t} &\equiv& xyz\\
C_{t+1} &\equiv& xyRz\\
\end{array}\cdots \sim C_{t-1} \sim xRyRz \sim C_{t+1} \sim \cdotsR \equiv CU \equiv UDC \sim C(UD) \equiv (CU)D \sim DC = DTT \cdots TXY \equiv YXX \equiv \theta^p, \qquad Y \equiv \theta^qX \equiv Y
\theta \equiv X \equiv Y\\
p = q = 1\\
X \equiv YZ(YZ)Y \equiv XY \equiv YX \equiv Y(YZ)YZ \equiv ZYZ \equiv \theta^\gamma, \quad Y \equiv \theta^\delta, \quad X
\equiv \theta^{\gamma+\delta}\renewcommand\arraystretch{.4}
\begin{array}{@{\extracolsep{-8pt}}*{8}{c}}
\jpLetter{4}{A} & \jpLetter{4}{A} \\
\jpOverBrace{4} & \jpOverBrace{4}\\
\hline
\jpLetterR{1}{x} & \jpLetterR{3}{y} & \jpLetterR{1}{x} \\
\cline{2-5}
\jpBlank{1} & \jpUnderBrace{4}\


or consequently


\bigskip

Let  now denote an arbitrary given symbol sequence.  
\marginnote{This gives us a kind of Euclidean (GCD) algorithm for
  sequences, with each  as the quotient and  as the remainder.}
We can then
define such sequences ,  and  such that:

where  for any value of  is the largest sequence for which one
can find such sequences  and  that

\mypage{500}
where two subsequences  in the sequence can never have a common part.

If one has

then  must be equal to one of the sequences , as can be immediately seen.

If  denotes an arbitrary given symbol sequence and  the largest
sequence for which one can find two sequences  and  such that

or

then first


where the sequence  is either wholly missing, or is composed
of fewer symbols than .

Consequently there exists  a sequence  such that

-5pt]
\jpLetterRL{11}{S} & \jpLetterR{3}{D} \-10pt]
\jpLetterRL{3}{C} & \jpLetter{3}{C} & \jpLetterR{3}{C} & \jpLetterR{2}{\alpha}& \jpLetterR{3}{}  \-10pt]
\jpLetterR{3}{} & \jpLetterR{3}{} & \jpLetterR{3}{}
& \jpOverBrace{2} & \jpOverBrace{1}& \jpLetterR{2}{}\\
\cline{4-14}
\jpLetter{3}{} & \jpUnderBrace{3} & \jpUnderBrace{3}
& \jpUnderBrace{3}& \jpUnderBrace{2}\



\medskip

 is the smallest sequence for which one can find a sequence 
such that


\medskip

If  contains at least one symbol, then we never have that


Otherwise we would get:

where  is composed of fewer symbols than .

\mypage{501}

Each of the sequences

where  is composed of at least one symbol, contains only a
single subsequence  and a single subsequence .

2pt]
\hline
 \jpUnderBrace{2} &  \jpUnderBrace{1} &  \jpUnderBrace{2} &  \jpUnderBrace{3} &  \jpUnderBrace{2} \3pt]
\cline{3-10}
\jpBlank{2} & \jpUnderBrace{3} & \jpUnderBrace{5}\


If we say that  contains an inner subsequence
, one can write

or

or 


However,  here would clearly have to be composed of fewer
symbols than , which is impossible.  In this way it is
also proven that  is not composed of an inner
subsequence .

\medskip

Further, if  is composed of an inner
subsequence , then we have:

2pt]
\hline
 \jpUnderBrace{2} &  \jpUnderBrace{1} &  \jpUnderBrace{3} &  \jpUnderBrace{1} &  \jpUnderBrace{2} \3pt]
\cline{3-10}
\jpBlank{2} & \jpUnderBrace{4} & \jpUnderBrace{3}\


or

or 


That  is composed of fewer symbols than  is however impossible.
In this way it is
also proven that  is not composed of an inner
subsequence .

\mypage{502}

If one has

where the number of symbols in  and  are not less
than\marginnote{So here we're
deliberately selecting some  \emph{shorter} than } the
number in  and , then there is
consequently a whole number  between  and  for which
\marginnote{Thus  and } 


In order to find expressions for the sequence  that belongs to the
sequence , we now write:

where  when , contains fewer symbols
than , while  for  is
greater than .

Furthermore, let  for  be completely
missing,\marginnote{When the process terminates, either  and the
  last factoring is the trivial , or
   and the last factoring is to .}
while
 is the largest sequence that the
two sequences of , and  is the
largest sequence that the two sequences of  can have in common. \marginnote{That is,  is the `overlap'  on each line}
Depending on whether  is greater than or equal to , we can now
treat  or  as  respectively, etc.

\bigskip

If  denotes an arbitrary given sequence, and  denotes the
largest sequence for which one can find sequences  and  such
that

then 

is the shortest sequence for this largest sequence  where
one can find sequences  and  such that

One gets here that



\mypage{503}
\section{V}

Let it be the case that in respect of some null sequence :

where two equivalent sequences  and  for each value of 
are composed of the same number of symbols.  One sees immediately that 
  and  are composed of equally many\marginnote{Same number
  of s, same number of s etc.  Thus   is just a permutation
  of .} of each kind of symbol.


One can write in place of a possible sub-sequence  or  of
some sequence  the other of these equivalent sequences, so that the
sequence  constructed in this way is equivalent to  in respect
of .  We say that  is constructed from  through a
\emph{homogeneous transformation} according to system \eqref{eq:one}.

Two sequences  and , equivalent in respect of , which are
also equivalent in respect of system \eqref{eq:one} are called \emph{parallel}
sequences in respect of  and \eqref{eq:one}.  We indicate this by writing 
\marginnote{Corollary: Parallel sequences always contain the same number of symbols}


If two sequences  and  are parallel to one another in respect of
system \eqref{eq:one}, there thus exist such sequences  where  and  denote  and 
respectively, so that one can get one of the consecutive sequences 
and  from the other by exchanging a possible sub-sequence
 with the corresponding sequence .

When one can not derive any of the equivalences \eqref{eq:one} from the others
through homogeneous transformation we say that the equivalences \eqref{eq:one}
are independent of one another.

\bigskip

Given the sequences

where  denotes the null sequence, if for any symbol  one can
always transform them into one another through homogeneous
transformation by the system \eqref{eq:one}, so that

then we say that \eqref{eq:one} forms a
\emph{complete}\footnote{vollst\"andiges} system of equivalences.

\mypage{504}
Each sub-sequence  of an arbitrary sequence  can thus
through \eqref{eq:one} be moved arbitrarily in the sequence 
without changing the order of the remaining symbols of .

In this case we have the following theorem.

\bigskip

If one can get a sequence  from a sequence , and a sequence
 from a sequence  by removing a sequence , and meanwhile
one can transform  and  into one another by successive
homogeneous transformations according to a complete system of
equivalences, then the sequences  and  have this same property.

We indeed get that e.g.


\bigskip

If a system of equivalences derived from a null sequence  has the
property that  whenever , then we say that
the system is perfect\footnote{vollkommenes} in respect of .

A complete and perfect system of equivalences in respect of a
null sequence  thus has the property that, in respect of the system
it is always the case that

where  denotes an arbitrary sequence, meanwhile, when
 we always have .

\newthought{Theorem.}
If one can get a sequence  from a sequence , and a sequence
 from a sequence  by removing a sequence , and meanwhile
in respect of a complete and perfect system of equivalences in respect
of 

then we also have


Then:

or



\bigskip

If

or

\mypage{505}
so then in respect of a complete and perfect system of equivalences in respect
of a null sequence   we always have 


For


\bigskip


If one has found a complete and perfect system of equivalences in respect
of a null sequence  , then we can immediately see how in this way
our problem \ref{prob:II} is easily solved.

Namely, if  denotes an arbitrary sequence, then one can set up a
series of sequence systems

that for each value of  all sequences of  are parallel, while
 is equal to one of the sequences of .  Further the series 
can be so chosen that no sequence of  contains a sub-sequence ,
while it is possible to obtain for any value of  a sequence of
 from a sequence of  through removal of a sub-sequence
.

Finally, the series  is so chosen that every sequence parallel to a
sequence of the series  is contained in the series .

Having removed then from an arbitrary sequence of a series  a
possible sequence , one can obtain in this way for any value of  one of the sequences in the series .

We say now that  forms an irreducible sequence system belonging
to .

Our problem \ref{prob:II} is now completed through the remark that similar
sequences, and thus equivalent sequences, must have the same irreducible
sequence system.

For a complete and perfect system of equivalences, a null sequence  
must also be parallel to equivalent sequences with equally many symbols
in respect of the aforementioned system.


We can, however, decide for certain whether or not two sequences are
parallel in a calculable number of steps.


\mypage{506}
\section{VI}

Let there be given the two series of symbol sequences 

where  and  for each value of  are - as before -
corresponding sequences.

Two arbitrary sequences  and  are called equivalent in respect
of the  pairs of corresponding sequences  and  when there
exist such sequences , where  and
 denote  and  respectively, that one can obtain
 from  for each value of  through the exchange of a
subsequence  or  for its corresponding sequence.

We represent this, as before, through the equivalence

 and  are called, as before, equivalent sequences, and
we write
.

We have here the equivalences:


\newthought{Theorem.}
For arbitrary values of  and , let  and 
always start with different symbols on the left, and also for any two
of the sequences in , so we can write:

where  and  are such symbols that each  is different from
each of the other symbols  and .  If , ,  and  are
any such sequences that in respect of system \eqref{eq:two}:

\mypage{507}
where

then it is also the case that


We only need to show here that if

for any symbol , then  and  must always be equivalent.
For convenience, we will prove the following more comprehensive 
theorem:

Let  and , where  denotes a single symbol, be two sequences
that are equivalent in respect of system \eqref{eq:two}, i.e.\ we are given such
sequences  that

then one can find such sequences  that

where the number of -sequences  is not greater than the number
of -sequences .

This 
\marginnote{First, three base cases where the derivation is 0, 1 or 2 steps}
theorem is clearly true when


Further, also when


Finally, 
\marginnote{If this happens we must have applied a rule forwards and then
 backwards, as two different rules must change the first letter.}
the theorem must also be true when 

because here .

We wish now 
\marginnote{Now three inductive cases...}
to assume that the theorem is true when .
We will then prove that the theorem is true when .
We can then write:

where each  denotes a single symbol.
If 
\marginnote{Case 1: If  is
  different from each  and  then the equations
in \eqref{eq:two}  won't allow you to change }
 is different from each  and  one has:

or


\mypage{508}

If one of the symbols  e.g.\  equals  then
the theorem is also quite clear:
Then we have

then 
\marginnote{by the inductive hypothesis}
there exist such sequences  and  that

where


We thus need only consider now the case where  denotes and  or a
, while each of the symbols  in \eqref{eq:five} is
different from .

If 
\marginnote{Case 2: start with an }
 were an , e.g.

then we have in \eqref{eq:five}

 in particular being different from , i.e.\

If one has further that  while  is different
from  then we have

which is clearly impossible.

We thus obtain here

or

where


However, since here 

then one can find such sequences  that

\mypage{509}
where 
and thus


Finally, 
\marginnote{Case 3: start with a ; much the same as case 2.}
if   were equal to a , e.g.\ , then we can get from
\eqref{eq:five},

or

where


Since

then there exists such sequences  that

where 
and thus


In this way our theorem is proven.

\bigskip

Let  denote an arbitrary sequence such that for each value of a
symbol  it is always the case that


Further, let  denote an arbitrary sequence equivalent to .
If then 

where  are single symbols, then we would have


For

or


Thus if  contains  symbols, then  arbitrary consecutive
symbols of the sequence  form a sequence equivalent to .

\bigskip

We will now demonstrate some null sequences  for which one can find
a perfect and complete system of equivalences.

\mypage{510}
\section{VII}

\subsection{Example 1.}

Let  be a null sequence defined using the following relations:

where  and  denote different individual
symbols,\marginnote{These differences are important: 
it means \eqref{eq:seven} will
  then fit the format required for the theorem in \S VI}
while  and each  signify an arbitrary positive whole 
number.

Here we thus have

where


Further we also have:

where


We get now e.g.

or


More generally, one obtains for each relevant value of 

or one gets the equivalence:


\mypage{511}
We thus have the following  equivalences in respect of :

\parbox{1.2\textwidth}{
}

We remark here 
\marginnote{and  is different from any }
that  and  begin on the left with .

We add to \eqref{eq:seven} all possible equivalences:

where  is not equal to any of the symbols 
and , so the system formed in this way, which we will call ,
\marginnote{It's \emph{perfect} because it fits the format required
  for the main theorem of
  \S VI, and we can apply this theorem with }
is a perfect system.

We will now prove that  is also a complete system in respect of
, or that in respect of  it is always the case that

when  denotes a single arbitrary symbol.

We however need only prove the case where  is equal to one of the
symbols  or .

We will however first prove that in respect of  or
\eqref{eq:seven}:

\parbox{1.15\textwidth}{}
where  is arbitrary.

\bigskip

The theorem is valid according to \eqref{eq:seven} for , .
But if the theorem is valid for ,  and for , , so it is
also valid according to \eqref{eq:seven} for , .

For\marginnote{Some typos in the following:\\
- added superscript  in line 2\\
- changed  to  in line 3
}


Thus \eqref{eq:eight} is also valid for  or for  .

In this way is \eqref{eq:eight} proven.

\mypage{512}
We now get according to \eqref{eq:seven} and \eqref{eq:eight}
\marginnote{Typo: added superscript  in the second line}


Thus  is also a complete system in respect of .

\bigskip

Our problem \ref{prob:II} is accordingly solved by this means for the given
null sequence .

The above theory keeps its validity if  and
 are sequences provided that they cannot overlap with one another.

\subsection{Example 2.}


where  and  denote single symbols.


or

or

or


In respect of the system\marginnote{As before, this system of
  equivalences fits the format
  required for the main theorem in \S VI, and is thus perfect.}

we get however


\subsection{Example 3.}

Let

where  and  are such sequences that  is the largest
sequence that the two sequences of  have in common.

Further let

where  is not a power sequence, and where  contains more symbols
than .

\mypage{513}

Thus we have here

or


Since , we get

or


If  contains more symbols than , or 

then we get


If  and  here represent single different symbols, or sequences
which cannot have an overlap with each other, then our problem \ref{prob:II} is
solved through these latest equivalences.

\subsection{Example 4.}


where  and  are single symbols.

We get

or

which is sufficient.


\subsection{Example 5.}

Let


Here we have first\marginnote{Thus we can always pull all the 's to
  the left (moving any  to the right) or vice versa.}


\mypage{514}
Second one thus gets:

or


\UseKEquationNumbering
We will now show that the equivalences:

form a complete and perfect system \ref{eq:K} in respect of .

\bigskip

First 
\marginnote{Proving completeness is the easy part.}
it is certainly\marginnote{Typo: changed  to  at the
  end of the first equation}


\smallskip
Second we will show the following:

\marginnote{This is just the definition of a perfect system being
  spelt out}
If  and  are such sequences that, in respect of the System \ref{eq:K},

so that one can thus find a sequences  where

where  is an arbitrary symbol denoting  or , then in respect
of  \ref{eq:K} we would also have


There is then such a sequence  that


\marginnote{The proof will be by induction over\\
(a) the length of a derivation and \\
(b) the number of symbols in .}  
Through the figure

we will indicate here that one can get  from  by exchanging a
subsequence  or  or  or  for its corresponding
sequence.  

\bigskip

The theorem 
\marginnote{For the base
  cases, note that both the equivalences in  \ref{eq:K} must change the
  leftmost symbol.}
is valid now first when

Then clearly


Second the theorem is also valid when

\mypage{515}
Then clearly


Third, the theorem is valid when both  and  denote just a single symbol.
Then clearly


We assume now \marginnote{Inductive hypothesis}
in advance that the theorem is always true when both
 and  are composed of at most  symbols.
Further we assume that the theorem remains true when both  and 
contain  symbols, and where the number of -sequences  is
not greater than .

We then need only to prove that the theorem remains true when both  and 
are composed of  symbols, while in the derivation

the number  of -sequences is equal to .

If it is the case that e.g.

so we thus have


If here e.g.

so we get


In the opposite case\marginnote{i.e. no  is } one gets however


If here either

or

then one gets respectively


In both cases we get

or


\mypage{516}

We need then only\marginnote{Since  is symmetric this covers the
  other ``two'' cases} to consider the case where e.g.\


We get then:

or


We get here the alternatives:


In the first alternative

or

or\marginnote{Typo: changed  to  at the
  end of this equation}


In the second alternative

or

or


In this way the theorem is proved.

\section{VIII}
\UseGreekEquationNumbering

Finally we wish to make a few remarks.

If  denotes an arbitrary null sequence, then there exists three
series of symbol sequences

with the following properties:
\begin{enumerate}
\item  and  are - for each value of  - equivalent to each
  other in respect of , and each of these sequences contains fewer
  symbols than .

\mypage{517}
\item All sequences , each of which denote
  a null sequence, are equivalent to one another in respect of the
  equivalences 


\item For each  the series \eqref{eq:gamma} contains two sequences
   and  such that

where  denotes a symbol sequence.

\item If for two arbitrary  sequences
   and  of the series \eqref{eq:gamma} 
there exist such symbol sequences  and  that

then the equivalence

forms one of the equivalences of \eqref{eq:delta}.
\end{enumerate}

\bigskip

One sees immediately that all the  \eqref{eq:gamma}-sequences contain
equally many symbols, and similarly for  and  for each value
of .

\bigskip

We will now show how one can gradually form the sequences in
\eqref{eq:gamma} and the equivalences in \eqref{eq:delta}.

Let 

denote  series of symbol sequences 

where each  denotes a single symbol sequence, while

for each  is said to denote the series .

\mypage{518}

Further, we signify by

 systems of equivalences

where each  and each  denotes a single symbol sequence, and where
 for each value of  is said to represent the system



\bigskip

The series  only contains the null sequence .

For each value of  we form  from  and
further  from  as follows:

First, if the system

contains two such sequences  and  that

where  and  denote single symbols or sequences, then

is equal to one of the equivalences from .

For each equivalence

from  that are opposite for each value of  in the group,
the series  contains such sequences 
and  that 

\mypage{519}
where   denotes a single symbol or a sequence.  In this way
 is completely defined.

Finally, let  be formed from all of those unique sequences
, that are equivalent to all -sequences in the series
 in
respect of the equivalences of the system .

One sees immediately then that  is contained in
 and that  is contained in .

One can however choose  so large that

and thus also

In this way our claim is proven.

\bigskip

From the system \eqref{eq:delta} we can now choose a system
\eqref{eq:epsilon} of equivalences independent from each other

that one can derive each equivalence in \eqref{eq:delta} from
\eqref{eq:epsilon} while one can thus derive no equivalence in
\eqref{eq:epsilon} from the others.

\eqref{eq:epsilon} can be so chosen that the number  of these
equivalences is minimised.  Further, one can choose \eqref{eq:epsilon}
so that none of these equivalences can be replaced by another with
fewer symbols.

In \eqref{eq:epsilon} it is never the case that

and further we never have simultaneously

or


\newthought{Theorem.}
The system \eqref{eq:epsilon} contains no equivalences of
the form

where e.g.\  starts the left of one of the sequences  of the
\eqref{eq:gamma}-sequences, i.e.

\mypage{520} 

Since the named equivalence must also occur in \eqref{eq:delta}, there
are such sequences  and  in \eqref{eq:gamma} that:

or the equivalence

is contained in \eqref{eq:delta}, or


However \eqref{eq:epsilon} then contains the equivalence

from which one can clearly derive\marginnote{Impossible, since
  by the definition of \eqref{eq:epsilon} you can't derive one of its
  equations   from any of the others.}  


\newthought{Theorem.}
The system \eqref{eq:epsilon} contains no equivalence of the
form:

where  forms the right end of a  \eqref{eq:gamma}-sequence.

Since the named equivalence must also occur in \eqref{eq:delta}, there
are such sequences  and  in \eqref{eq:gamma} that:

or one obtains the equivalence  which is thus contained in
\eqref{eq:delta}.
Finally


However \eqref{eq:delta} then contains the equivalence

which is impossible.

\mypage{521}
\newthought{Theorem.}
If

\end{array}P = Q
PR = RP\\
QR = RQ\\
UR = RU

PR = PR_y \equiv PUQ \equiv R_xQ = R_xP = RP\\
UR = UR_x \equiv UPU = UQU \equiv R_yU = RU.

PU \equiv R_x = R_y \equiv UQ = UP.
C \equiv NM,
NR = RN\\
MR = RM.
\begin{array}{r@{}c@{}l}
R_y \equiv NM &U& \\
      &U& D \equiv R_z
\end{array}\begin{array}{r@{}c@{}l}
R_z = UN&M&\\
      &M&W \typo{\equiv} R_x
\end{array}UN = W
NR = NR_x \equiv NMW = NMUN \equiv R_yN = RN.

MR = MR_z \equiv MUNM = MWM \equiv R_xM = RM.

M \equiv aR_zb\\
N \equiv c R_\mu d

R_x \equiv ab\\
R_y \;\typo{\equiv}\; cd.
\begin{array}{r@{}c@{}l}
M \equiv C&U&\\
      &U&D \equiv N
\end{array}C = D
forms one of the equivalences of \eqref{eq:delta}, or one can obtain
\marginnote{In cases 1-5 below we get , while in cases 6-8  and
   differ by some .}
sequences from  and  through removal of subsequences  of the
systems \eqref{eq:gamma} that are equivalent in respect of
\eqref{eq:delta}.

Letting the symbol  represent equivalence in respect of
\eqref{eq:delta}, we can distinguish the following
cases:\marginnote{These 8 cases cover all possible configurations
  of the overlap between  and }

\begin{enumerate}
\item 

\bigskip

\item 

\mypage{523}
\item 

\bigskip

\item 

\bigskip

\item 

\bigskip

\item 

\mypage{524}


\bigskip

\item 

\bigskip


\item 

\end{enumerate}

\bigskip
\noindent
\textsc{i}. May 1914.

\hfill \textsc{\large{\emph{Axel Thue.}}}

\begin{center}
\begin{tabular}{p{.2\textwidth}}
\\ \hline\hline
\end{tabular}
\end{center}





