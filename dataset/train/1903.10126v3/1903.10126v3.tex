

\documentclass[11pt,a4paper]{article}
\usepackage[hyperref]{naaclhlt2019}
\usepackage{times}
\usepackage{latexsym}
\usepackage{amsfonts}
\usepackage{amsmath}
\usepackage{graphicx}
\usepackage{xspace}
\usepackage{multirow}

\usepackage{url}

\aclfinalcopy \def\aclpaperid{497} 



\newcommand\BibTeX{B{\sc ib}\TeX}

\def\HRERE{{\sc Hrere}\xspace}

\title{Connecting Language and Knowledge with Heterogeneous Representations for Neural Relation Extraction}

\author{Peng Xu \\
  Department of Computing Science \\
  University of Alberta\\
  Edmonton, Canada\\
  {\tt pxu4@ualberta.ca} \\\And
  Denilson Barbosa \\
  Department of Computing Science \\
  University of Alberta\\
  Edmonton, Canada\\
  {\tt denilson@ualberta.ca} \\}

\date{}

\begin{document}
\maketitle
\begin{abstract}
Knowledge Bases (KBs) require constant updating to reflect changes to the world they represent.
For general purpose KBs, this is often done through Relation Extraction (RE), the task of predicting KB relations expressed in text mentioning entities known to the KB. 
One way to improve RE is to use KB Embeddings (KBE) for link prediction.
However, despite clear connections between RE and KBE, little has been done toward properly unifying these models systematically.
We help close the gap with a framework that unifies the learning of RE and KBE models leading to significant improvements over the state-of-the-art in RE.
The code is available at \url{https://github.com/billy-inn/HRERE}.
\end{abstract}

\section{Introduction}


Knowledge Bases (KBs) contain structured information about the world and are used in support of many natural language processing applications such as semantic search and question answering.
Building KBs is a never-ending challenge because, as the world changes, new knowledge needs to be harvested while old knowledge needs to be revised.
This motivates the work on the Relation Extraction (RE) task, whose goal is to assign a KB relation to a \emph{phrase} connecting a pair of entities, which in turn can be used for updating the KB.
The state-of-the-art in RE builds on neural models using distant (a.k.a. weak) supervision~\cite{mintz2009distant} on large-scale corpora for training.

\begin{figure*}[ht]
\begin{center}
 \includegraphics[height=1.7in]{workflow.png}
\end{center}
\caption{Workflow of the proposed framework.} \label{workflow}
\end{figure*}

A task related to RE is that of Knowledge Base Embedding (KBE), which is concerned with representing KB entities and relations in a vector space for predicting missing links in the graph.
Aiming to leverage the similarities between these tasks, \citet{weston2013connecting} were the first to show that \emph{combining} predictions from RE and KBE models was beneficial for RE.
However, the way in which they combine RE and KBE predictions is rather naive (namely, by adding those scores).
To the best of our knowledge, there have been no systematic attempts to further unify RE and KBE, particularly during model \emph{training}. 

We seek to close this gap with \HRERE (Heterogeneous REpresentations for neural Relation Extraction), a novel neural RE framework that learns language and knowledge representations \emph{jointly}.
Figure~\ref{workflow} gives an overview.
\HRERE's backbone is a bi-directional long short term memory (LSTM) network with multiple levels of attention to learn representations of text expressing relations. 
The knowledge representation machinery, borrowed from {\bf ComplEx}~\cite{trouillon2016complex}, nudges the language model to agree with facts in the KB.
Joint learning is guided by three loss functions: one for the language representation, another for the knowledge representation, and a third one to ensure these representations do not diverge.
In effect, this contributes to \HRERE's generalization power by preventing over-fitting by either model.

We build on state-of-the-art methods for learning the separate RE and KBE representations and on learning tools that allow us to scale to a moderately large training corpus. (We use a subset of Freebase with 3M entities as our KB.)
We validate our approach on an established benchmark against state-of-the-art methods for RE, observing not only that our base model significantly outperforms previous methods, but also the fact that jointly learning the heterogeneous representations consistently brings in improvements.
To the best of our knowledge, ours is the first principled framework to combine and jointly learn heterogeneous representations from both language and knowledge for the RE task.

\paragraph*{Contributions.}
This paper describes and evaluates a novel neural framework for jointly learning representations for RE and KBE tasks that uses a cross-entropy loss function to ensure both representations are learned together, resulting in significant improvements over the current state-of-the-art for the RE task. 
\section{Related Work}




Recent neural models have been shown superior to approaches using hand-crafted features for the RE task.
Among the pioneers, \newcite{zeng2015distant} proposed a piecewise convolutional network with multi-instance learning to handle weakly labeled text mentions.
Recurrent neural networks (RNN) are another popular architecture \cite{wu2017adversarial}. 
Similar fast progress has been seen for the KBE task for representing entities and relations in KBs with vectors or matrices.
\citet{bordes2013translating} introduced the influential translation-based embeddings (TransE), while
\citet{yang2014embedding} leveraged latent matrix factorization in their DistMult method.
We build on ComplEx \cite{trouillon2016complex}, which extends DistMult into the complex space and has been shown significantly better on several benchmarks.






\citet{weston2013connecting} were the first to connect RE and KBE models for the RE task. Their simple idea was to  train the two models \emph{independently} and only combine them at inference time.
While they showed that combining the two models is better than using the RE model alone, newer and better models since then have obviated the net gains of such a simple strategy \cite{xu2018investigations}. 
We propose a much tighter integration of RE and KBE models: we not only use them for prediction, but also \emph{train} them together, thus mutually reinforcing one another.


Recently, many methods have been proposed to use information from KBs to facilitate relation extraction.
\newcite{sorokin2017context} considered other relations in the sentential context while predicting the target relation.
\newcite{vashishth2018reside} utilized additional side information from KBs for improved RE.
However, these methods didn't leverage KBE method to unify RE and KBE in a principled way.
\newcite{han2018neural} used a mutual attention between KBs and text to perform better on both RE and KBE, but their method was still based on TransE \cite{bordes2013translating} which can not fully exploit the advantage of the information from KBs.

%
 
\section{Background and Problem}


The goal in the task of Relation Extraction is to predict a KB relation that holds for a pair of entities given a set of sentences mentioning them (or \emph{NA} if no such relation exists).
The input is a KB  with relation set , a set of relations of interest , and an automatically labelled training dataset  obtained via distant supervision.
Given a sentence mentioning entities , the output is a relation  that holds for  or the catch-all relation {\em NA} if no such  exists.

\paragraph*{Knowledge Base and Distant Supervision.}

As customary, we denote a KB  with relation scheme  as a set of \emph{triples} , where  is the set of entities of interest.
Distant supervision exploits the KB to automatically annotate sentences in a corpus containing mentions of entities with the relations they participate in.
Formally, a labeled dataset for relation extraction consists of fact triples  and a multi-set of extracted sentences for each triple , such that
each sentence  mentions both the head entity  and the tail entity .

\paragraph*{Problem Statement.}

Given an entity pair  and a set of sentences  mentioning them, the RE task is to estimate the probability of each relation in .
Formally, for each relation , we want to predict .

In practice, the input set of sentences  can have arbitrary size. 
For the sake of computational efficiency, we normalize the set size to a fixed number  by splitting large sets and oversampling small ones. 
We also restrict the length of each sentence in the set by a constant  by truncating long sentences and padding short ones.  
\section{Methodology}


We now go over the details of our framework outlined in Figure~\ref{workflow} for unifying the learning of the language and the knowledge representations used for relation extraction. 
In a nutshell, we use LSTM with attention mechanisms for language representation and we follow the approach of~\newcite{trouillon2016complex} for KB embedding.




\subsection{Language Representation}



\textbf{Input Representation.}
For each word token, we use pretrained word embeddings and randomly initialized position embeddings \cite{zeng2014relation} to project it into -dimensional space, where  is the size of word embedding and  is the size of position embedding.

\noindent \textbf{Sentence Encoder.}
For each sentence , we apply a non-linear transformation to the vector representation of  to derive a feature vector  given a set of parameters .
In this paper, we adopt bidirectional LSTM with  hidden units as  \cite{zhou2016attention}.


\noindent \textbf{Multi-level Attention Mechanisms.}
We employ attention mechanisms at both word-level and sentence-level to allow the model to softly select the most informative words and sentences during training~\cite{zhou2016attention,lin2016neural}.
With the learned language representation , the conditional probability  is computed through a \emph{softmax} layer, where  is the parameters of the model to learn language representation.




\subsection{Knowledge Representation}
\label{subsection:k}



Following the score function  and training procedure of \citet{trouillon2016complex}, we can get the knowledge representations .
With the knowledge representations and the scoring function, we can obtain the conditional probability  for each relation :


\noindent where  corresponds to the knowledge representations .
Since , we use a randomized complex vector as .




\subsection{Connecting the Pieces}

As stated, this paper seeks an elegant way of connecting language and knowledge representations for the RE task.
In order to achieve that, we use separate loss functions (recall Figure~\ref{workflow}) to guide the language and knowledge representation learning and a third loss function that ties the predictions of these models thus nudging the parameters towards agreement.






The cross-entropy losses based on the language and knowledge representations are defined as:
 

\noindent where  denotes the size of the training set.
Finally, we use a cross-entropy loss to measure the dissimilarity between two distributions, thus connecting them, and formulate model learning as minimizing :



\noindent where .




\subsection{Model Learning}

Based on Eq.~\ref{eq:lloss},~\ref{eq:kloss},~\ref{eq:closs}, we form the joint optimization problem for model parameters as

\noindent where . The knowledge representations are first trained on the whole KB independently and then used as the initialization for the \emph{joint} learning.
We adopt the stochastic gradient descent with mini-batches and Adam~\cite{kingma2014adam} to update , employing different learning rates  and  on  and  respectively








\subsection{Relation Inference}
In order to get the conditional probability , we use the weighed average to combine the two distribution  and :


\noindent where  is the combining weight of the weighted average. Then, the predicted relation  is

 
\section{Experiments}




\paragraph*{Datasets.}

We evaluate our model on the widely used {\bf NYT} dataset~\cite{riede2010modeling} by aligning Freebase relations mentioned in the New York Times Corpus.
Articles from years 2005-2006 are used for training while articles from 2007 are used for testing.
As our KB, we used a Freebase subset with the 3M entities with highest degree (i.e., participating in most relations).
Moreover, to prevent the knowledge representation from memorizing the true relations for entity pairs in the test set, we removed all entity pairs present in the NYT.








\noindent {\bf Evaluation Protocol:}
Following previous work \cite{mintz2009distant}, we evaluate our model using held-out evaluation which approximately measures the precision without time-consuming manual evaluation. 
We report both Precision/Recall curves and Precision@N (P@N) in our experiments, ignoring the probability predicted for the {\em NA} relation.
Moreover, to evaluate each sentence in the test set as in previous methods, we append  copies of each sentence into  for each testing sample.

\noindent {\bf Word Embeddings:}
In this paper, we used the freely available 300-dimensional pre-trained word embeddings distributed by \newcite{pennington2014glove} to help the model generalize to words not appearing in the training set.

\begin{table}[t]
\small
\begin{center}
\begin{tabular}{| l|l|l |}
\hline
learning rate on  &  &  \\
learning rate on  &  &  \\
size of word position embedding &  &  \\
state size for LSTM layers &  &  \\
input dropout keep probability &  &  \\
output dropout keep probability &  &  \\
L2 regularization parameter &  & \\
combining weight parameter&  &  \\
\hline
\end{tabular}
\end{center}
\caption{Hyperparameter setting} \label{param}
\end{table}

\noindent{\bf Hyperparameter Settings:}
For hyperparameter tuning, we randonly sampled  of the training set as a development set.
All the hyperparameters were obtained by evaluating the model on the development set.
With the well-tuned hyperparameter setting, we run each model five times on the whole training set and report the average P@N.
For Precision/Recall curves, we just select the results from the first run of each model.
For training, we set the iteration number over all the training data as .
Values of the hyperparameters used in the experiments can be found in Table~\ref{param}.





\begin{figure}[ht]
\begin{center}
\vskip-1.25em
 \includegraphics[height=2.4in]{comparison.png}
\end{center}
\caption{The Precision/Recall curves of previous state-of-the-art methods and our proposed framework.}\label{baseline}
\end{figure}

\begin{table}[h]
\begin{center}
\begin{tabular}{l | l | l | l}
\hline
P@N(\%) &  &  &  \\ \hline
Weston &  &  &  \\
\HRERE-base &  &  &  \\
\HRERE-naive & &  &  \\
\HRERE-full &  &  &  \\

\hline
\end{tabular}
\end{center}
\caption{P@N of {\bf Weston}  and variants of our proposed framework.}\label{prec}
\end{table}

\begin{table*}[ht]
\small
\begin{center}
\begin{tabular}{| l | p{8cm} | l | l | l |}
\hline
Relation & Textual Mention & base & naive & full \\ \hline
{\em contains} & Much of the {\bf middle east} tension stems from the sense that shiite power is growing, led by {\bf Iran}. & 
 &  &  \\ \hline
{\em place\_of\_birth} & Sometimes I rattle off the names of movie stars from {\bf Omaha}: Fred Astaire, {\bf Henry Fonda}, Nick Nolte \dots & 
 &  &  \\ \hline
{\em country} & Spokesmen for {\bf Germany} and Italy in Washington said yesterday that they would reserve comment until the report is formally released at a news conference in {\bf Berlin} today. &
 &  &  \\ \hline
\end{tabular}
\end{center}
\caption{Some examples in NYT corpus and the predicted probabilities of the true relations.}\label{examples}
\end{table*}

\paragraph*{Methods Evaluated.}
\label{analysis}
We study three variants of our framework:
(1) {\bf \HRERE-base}: basic neural model with local loss  only;
(2) {\bf \HRERE-naive}: neural model with both local loss  and global loss  but without the dissimilarities ;
(3) {\bf \HRERE-full}: neural model with both local and global loss along with their dissimilarities.
We compare against two previous state-of-the-art neural models, {\bf CNN+ATT} and {\bf PCNN+ATT} \cite{lin2016neural}.
We also implement a baseline {\bf Weston} based on the strategy following \newcite{weston2013connecting}, namely to combine the scores computed with the methods stated in this paper directly without joint learning.




\paragraph*{Analysis.}
Figure~\ref{baseline} shows the Precision/Recall curves for all the above methods.
As one can see, {\bf \HRERE-base} significantly outperforms previous state-of-the-art neural models and {\bf Weston} over the entire range of recall.
However, {\bf \HRERE-base} performs worst compared to all other variants, while {\bf \HRERE-full} always performs best as shown in Figure~\ref{baseline} and Table~\ref{prec}.
This suggests that introducing knowledge representation consistently results in improvements, which validates our motivating hypothesis.
{\bf \HRERE-naive} simply optimizes both local and global loss at the same time without attempting to connect them.
We can see that {\bf \HRERE-full} is not only consistently superior but also more stable than {\bf \HRERE-naive} when the recall is less than 0.1.
One possible reason for the instability is that the results may be dominated by one of the representations and biased toward it.
This suggests that (1) jointly learning the heterogeneous representations bring mutual benefits which are out of reach of previous methods that learn each independently; (2) connecting heterogeneous representations can increase the robustness of the framework.










\paragraph*{Case Study.}





 Table~\ref{examples} shows two examples in the testing data.
 For each example, we show the relation, the sentence along with entity mentions and the corresponding probabilities predicted by {\bf \HRERE-base} and {\bf \HRERE-full}. 
 The entity pairs in the sentence are highlighted with bold formatting.

 From the table, we have the following observations:
 (1) The predicted probabilities of three variants of our model in the table match the observations and corroborate our analysis.
(2) From the text of the two sentences, we can easily infer that {\em middle east contains Iran} and {\em Henry Fonda was born in Omaha}.
 However, {\bf \HRERE-base} fails to detect these relations, suggesting that it is hard for models based on language representations alone to detect implicit relations, which is reasonable to expect.
 With the help of KBE, the model can effectively identify implicit relations present in the text.
(3) It may happen that the relation cannot be inferred by the text as shown in the last example.
It's a common wrong labeled case caused by distant supervision.
It is a case of an incorrectly labeled instance, a typical occurrence in distant supervision.
However, the fact is obviously true in the KBs.
As a result, {\bf \HRERE-full} gives the underlying relation according to the KBs.
This observation may point to one direction of de-noising weakly labeled textual mentions generated by distant supervision.




 
\section{Conclusion}


This paper describes an elegant neural framework for jointly learning heterogeneous representations from text and from facts in an existing knowledge base.
Contrary to previous work that learn the two disparate representations independently and use simple schemes to integrate predictions from each model, we introduce a novel framework using an elegant loss function that allows the proper connection between the the heterogeneous representations to be learned seamlessly during training.
Experimental results demonstrate that the proposed framework outperforms previous strategies to combine heterogeneous representations and the state-of-the-art for the RE task.
A closer inspection of our results show that our framework enables both independent models to enhance each other.
 
\section*{Acknowledgments}

This work was supported in part by grants from the Natural Sciences and Engineering Research Council of Canada and a gift from Diffbot Inc.

\bibliographystyle{acl_natbib}
\bibliography{ref}

\end{document}
