\documentclass{article}
\usepackage{fullpage}
\usepackage{graphicx}
\usepackage{amsmath}
\usepackage{amsfonts}
\usepackage{bbm}
\usepackage{amssymb}
\usepackage{amsthm}
\usepackage{subfigure}
\usepackage{cite}
\newtheorem{thm}{Theorem}
\newtheorem{deff}{Definition}

\begin{document}
\title{Numerical stability of explicit Runge-Kutta finite-difference schemes for the nonlinear Schr{\"o}dinger equation}
\author{R. M. Caplan\footnote{Corresponding author.  Present address:  Predictive Science Inc.  9990 Mesa Rim Rd, Suite 170, San Diego, CA 92121. email: caplanr@predsci.com, phone: 858-225-2314}~~and R. Carretero-Gonz{\'a}lez\
\label{nlse}
i\frac{\partial \Psi}{\partial t} + a\nabla^2\Psi - V({\bf r})\Psi + s|\Psi|^2 \Psi = 0,

\label{scheme}
\frac{\partial \vec \Psi}{\partial t} = {\cal A}\, \vec \Psi,

\label{hhatdef}
\vec p = {\bf k}\,\vec \lambda,

\label{req}
R(p) = 1 + p + \frac{p^2}{2} + \frac{p^3}{6} + \frac{p^4}{24},

\label{rbound}
\lVert \vec R(\vec p)\rVert_\infty < 1,
{2}
\label{req2}
\left|R(\lambda)\right|^2 = 1 & + \frac{1}{576}\,{\bf k}^8 |\lambda|^8 -
 \frac{1}{72}\,{\bf k}^6 |\lambda|^6 + \left(\frac{{\bf k}^6|\lambda|^6}{6} - {\bf k}^4| \lambda|^4 + 24 \right)\,\frac{{\bf k}}{12}\,\mbox{Re}(\lambda)    \\
&+ \left({\bf k}^4|\lambda|^4 + 24\right)\,\frac{{\bf k}^2}{12}\,(\mbox{Re}(\lambda))^2 + \left({\bf k}^2|\lambda|^2 + 4\right)\,\frac{{\bf k}^3}{3}\,(\mbox{Re}(\lambda))^3 + \frac{2{\bf k}^4}{3}\,(\mbox{Re}(\lambda))^4.
\notag

\label{req3}
\left|R(\vec \lambda)\right|^2 = 1 + \frac {1}{576}\,{\bf k}^8|\vec \lambda|^8-\frac{1}{72}\,{\bf k}^6|\vec \lambda|^6,

\label{rk4stbeasy}
{\bf k} < \frac{\sqrt{8}}{\lVert\vec \lambda\rVert_\infty}.

\frac{\partial \vec \Psi}{\partial t} = {\cal A} \vec \Psi = \frac{i\,a}{h^2}\,A \vec \Psi,

\label{rk4stbeasy2}
{\bf k} < \frac{\sqrt{8}}{\lVert\vec \lambda_{A}\rVert_\infty}\frac{h^2}{a}.
 \lambda_j = c_0 + c_{N-1}\,\omega_j + c_{N-2}\,\omega_j^2 + ... + c_1\,\omega_j^{N-1}, \qquad j = 0,..., {N-1},

\omega_j = \exp \left(\frac{2\pi\, i\, j}{N}\right).

\label{bb}
\left. \frac{\partial \Psi}{\partial t}\right|_b = \frac{i\,a}{h^2} B_b \Psi_b,

\label{db}
\nabla^2\Psi_b = \frac{1}{h^2} D_b \Psi_b,

\Psi_b = C,

\left. \frac{\partial \Psi}{\partial t}\right|_b = 0,

\nabla^2\Psi_b = -\frac{1}{a} (s|\Psi_b|^2 - V_b) \Psi_b,

|\Psi_b|^2 = C,

\label{msdfixedstb}
\Psi_{t,b} \approx i\,\mbox{Im}\left[\frac{\Psi_{t,b-1}}{\Psi_{b-1}}\right]\,\Psi_b.

\left. \frac{\partial \Psi}{\partial t}\right|_b \approx i\,\Omega_{b-1} \Psi_b,

\label{msdlap}
\nabla^2\Psi_b \approx \left[ \mbox{Im}\left(i\, \frac{\nabla^2\Psi_{b-1}}{\Psi_{b-1}}\right) + \frac{1}{a}\,\left(N_{b-1} - N_b\right)\right]\Psi_b,

N_b = s\,|\Psi_b|^2 - V_b, \qquad N_{b-1} = s\,|\Psi_{b-1}|^2 - V_{b-1}.

\nabla^2\Psi_b = 0,

\left. \frac{\partial \Psi}{\partial t}\right|_b = i\,(s|\Psi_b|^2 - V_b)\,\Psi_b,

\nabla^2\Psi_i = \left. \frac{\partial^2 \Psi}{\partial x^2} \right|_i \approx \frac{\Psi_{i+1} -2\Psi_i +\Psi_{i-1}}{h^2},

A = \left[\begin {array}{cccccc} -2& 1& 0& 0& 1\\\noalign{\medskip}
                                  1&-2& 1& 0& 0\\\noalign{\medskip}
                                  0& \ddots&\ddots & \ddots & 0\\\noalign{\medskip}
                                  0& 0& 1 &-2& 1\\\noalign{\medskip}
                                  1& 0& 0& 1&-2\end {array} \right],

\lambda_j = -2 + \exp\left[\frac{2\pi i j }{N}\right] +  \exp\left[\frac{2\pi i j (N-1)}{N}\right], \qquad j\in\{0,...N-1\}.

|\lambda|_{\max} = \left|-2 + \exp\left[\pi i\right] + \exp\left[\pi i\right]^{N-1}\right|,

|\lambda|_{\max} = 4.

|\lambda|_{\max} = \left|-2 - (-1)^{1/N} + (-1)^N\,(-1)^{-1/N}\right|,

|\lambda|_{\max} = \left|-2 - 2\,\cos\left(\frac{\pi}{N}\right)\right|.

|\lambda|_{\max} < 4.

\label{1dcircbound}
{\bf k} < \frac{\sqrt{8}}{4}\frac{h^2}{a}.

\label{1dcircboundL}
{\bf k} < \frac{\sqrt{8}}{\max\{\lVert\vec L\rVert_{\infty},\lVert \vec L-4\rVert_{\infty}\}}\,\frac{h^2}{a},

\label{Leq}
L_i = \frac{h^2}{a}(s\,|\Psi_i|^2 - V_i),

A = \left[ \begin {array}{cccccc}  B_0&0     &0  &0  &0         \\\noalign{\medskip}
                                   1  &L_1-2 &1  &0  &0         \\\noalign{\medskip}
                                   0  &\ddots   &\ddots &\ddots          &0\\\noalign{\medskip}
                                   0  &0      &1  &L_{N-2}-2 &1\\\noalign{\medskip}
                                   0  &0       &0  &0         &B_{N-1}\end {array} \right]. 

A^\prime = \left[\begin {array}{cccccc}
L_1-2&1    &0  &0        &0\\\noalign{\medskip}
1    &L_2-2&1  &0        &0\\\noalign{\medskip}
0    &\ddots    &\ddots &\ddots       &0\\\noalign{\medskip}
0    &0    &1 &L_{N-3}-2&1\\\noalign{\medskip}
0    &0    &0  &1        &L_{N-2}-2
\end {array} \right].

\label{kbg}
{\bf k} < \frac{\sqrt{8}}{\max\{\lVert\vec B\rVert_{\infty},\lVert \forall L_i, L_i - \vec G\rVert_{\infty}\}}\,\frac{h^2}{a},

\label{G1DCD}
\vec G = \left\{4,3,1,0 \right\}.

\label{1d4cd}
\nabla^2\Psi_i = \left. \frac{\partial^2 \Psi}{\partial x^2} \right|_i \approx \frac{-\Psi_{i+2} +16\Psi_{i+1} -30\Psi_i +16\Psi_{i-1} - \Psi_{i-2}}{12\,h^2}.

A = \left[\begin {array}{cccccccc}
-15/6 &   4/3 & -1/12 &     0 &     0 &     -1/12 &   4/3\\\noalign{\medskip}
  4/3 & -15/6 &   4/3 & -1/12 &     0 &         0 & -1/12\\\noalign{\medskip}
-1/12 &   4/3 & -15/6 &   4/3 & -1/12 &         0 &     0\\\noalign{\medskip}
    0 &   \ddots &   \ddots &   \ddots &  \ddots &  \ddots &         0\\\noalign{\medskip}
    0 &         0 & -1/12 &   4/3 & -15/6 &   4/3 & -1/12\\\noalign{\medskip}
-1/12 &     0 &        0 & -1/12 &   4/3 & -15/6 &   4/3\\\noalign{\medskip}
  4/3 & -1/12 &        0 &     0 & -1/12 &   4/3 & -15/6\end {array} \right],
{2}
\lambda_j = &-\frac{15}{6} + \frac{4}{3}\,\exp\left[\frac{2\pi i j }{N}\right] - \frac{1}{12}\,\exp\left[\frac{4\pi i j }{N}\right]  \notag \\
&- \frac{1}{12}\,\exp\left[\frac{2(N-2)\pi i j }{N}\right] + \frac{4}{3}\,\exp\left[\frac{2(N-1)\pi i j }{N}\right]. \notag

\lambda_{N/2} = -\frac{15}{6} - \frac{4}{3} - \frac{1}{12} - \frac{1}{12}(-1)^{N-2} + \frac{4}{3}(-1)^{N-1} = -\frac{16}{3}.

\lambda_{(N+1)/2} = -\frac{15}{6} - \frac{4}{3}\,\left((-1)^{1/N} + (-1)^{-1/N}\right) - \frac{1}{12}\left((-1)^{2/N} + (-1)^{-2/N}\right),

\lambda_{(N+1)/2} = -\frac{15}{6} -\frac{4}{3}\,\left(2\,\cos\left(\frac{\pi}{N}\right)\right) - \frac{1}{12}\,\left(2\,\cos\left(\frac{2\pi}{N}\right)\right).

\label{klinhoc1d}
{\bf k} < \left(\frac{3}{4}\right) \frac{\sqrt{8}}{4} \frac{h^2}{a},

{\bf k} < \frac{6\sqrt{2}}{\max\{\lVert3\vec L-16\rVert_{\infty},\lVert 3\vec L+1\rVert_{\infty}\}} \frac{h^2}{a}.
{3}
 &1) \qquad &D_i &= \frac{1}{h^2}\left(\Psi_{i+1} - 2\Psi_i + \Psi_{i-1}\right), \label{2shoc1dstb} \\
 &2) \qquad &\nabla^2\Psi_i &\approx \frac{7}{6}D_i - \frac{1}{12}\left(D_{i+1} + D_{i-1}\right). \label{2shoc1d2stb}

A = \left[\begin {array}{cccccccc}
B_0         & 0         & 0        & 0      & 0            & 0             & 0 \\\noalign{\medskip}
\frac{14-D_0}{12} & L_1-\frac{29}{12} & \frac{4}{3}     & -\frac{1}{12}  & 0            & 0             & 0 \\\noalign{\medskip}
-\frac{1}{12}       & \frac{4}{3}       & L_2-\frac{15}{6} & \frac{4}{3}    & -\frac{1}{12}        & 0             & 0 \\\noalign{\medskip}
0           & \ddots    & \ddots   & \ddots & \ddots       & \ddots        & 0 \\\noalign{\medskip}
0           & 0         &-\frac{1}{12}    & \frac{4}{3}    & L_{N-3}-\frac{15}{6} & \frac{4}{3}           & -\frac{1}{12} \\\noalign{\medskip}
0           & 0         & 0        & -\frac{1}{12}  & \frac{4}{3}          & L_{N-2}-\frac{29}{12} & \frac{14-D_{N-1}}{12} \\\noalign{\medskip}
0           & 0         & 0        & 0      & 0            & 0             & B_{N-1}
\end {array} \right],

A^{\prime} = \left[\begin {array}{ccccccc}  
L_1-\frac{29}{12} & \frac{4}{3}      & -\frac{1}{12}    & 0      & 0            & 0            & 0 \\\noalign{\medskip}
\frac{4}{3}      & L_2-\frac{15}{6} & \frac{4}{3}       & -\frac{1}{12}  & 0            & 0            & 0 \\\noalign{\medskip}
-\frac{1}{12}     & \frac{4}{3}      & L_3-\frac{15}{6} & \frac{4}{3}    & -\frac{1}{12}        & 0            & 0 \\\noalign{\medskip}
0         & \ddots   & \ddots   & \ddots & \ddots       & \ddots       & 0 \\\noalign{\medskip}
0         & 0        & -\frac{1}{12}    & \frac{4}{3}     & L_{N-4}-\frac{15}{6} & \frac{4}{3}          & -\frac{1}{12} \\\noalign{\medskip}
0         & 0        & 0        & -\frac{1}{12}  & \frac{4}{3}           & L_{N-3}-\frac{15}{6} & \frac{4}{3} \\\noalign{\medskip}
0         & 0        & 0        & 0      & -\frac{1}{12}        & \frac{4}{3}          & L_{N-2}-\frac{29}{12}
\end {array} \right],

\label{G1D2SHOC}
\vec G = \frac{1}{12} \times \left \{64,63,46,12,-3,-4\right\}.

\begin{tabular}{c} 

\begin{tabular}{|c|c|c|} \hline
  &   &   \\ \hline
 &  &  \\ \hline
  &   &  \\ \hline
\end{tabular}

\end{tabular}

\label{G2DCD}
\vec G = \{0,1,2,6,7,8\}.

\label{kbound2dcdlin}
{\bf k} < \frac{\sqrt{8}}{8}\,\frac{h^2}{a}.

\label{2dnonhoc}
\begin{tabular}{c} 

\begin{tabular}{|c|c|c|c|c|} \hline
  &     &    &     &   \\ \hline
  &     &  &     &   \\ \hline
 &  &   &  &  \\ \hline
  &     &  &     &   \\ \hline
  &     &    &     &   \\ \hline
\end{tabular}

\end{tabular}
{3}
&\begin{tabular}{ll} 
 &  
\begin{tabular}{|c|c|c|} \hline
  &   &   \\ \hline
 &  &  \\ \hline
  &   &  \\ \hline
\end{tabular}

\end{tabular}
\\
\;&\; \notag
\\
&\begin{tabular}{ll} 
 & 
\begin{tabular}{|c|c|c|} \hline
  &    &   \\ \hline
 &  &  \\ \hline
  &    &   \\ \hline
\end{tabular}
  
\begin{tabular}{|c|c|c|} \hline
 &    &  \\ \hline
  &  &   \\ \hline
 &    &  \\ \hline
\end{tabular}

\end{tabular}

\label{G2D2SHOC}
\vec G = \dfrac{1}{12} \times \left\{128,127,126,110,109,92,24,9,8,-6,-7,-8\right \}.

\label{kbound2d2shoclin}
{\bf k} < \frac{3\sqrt{8}}{32}\,\frac{h^2}{a},

\label{3dcds}
\begin{tabular}{ll}
 &  \\
\; \; \\
\; & 
111-6111 
 \end{tabular}

\label{G3DCD}
\vec G = \{12,11,10,9,3,2,1,0\}.

\label{kbound3dcdlin}
{\bf k} < \frac{\sqrt{8}}{12}\,\frac{h^2}{a}.
{2}
\label{3dnonhoc}
\nabla^2 \Psi &\approx \frac{1}{12\,h^2}\left[\Psi_{i+2,j,k} + \Psi_{i-2,j,k} + \Psi_{i,j+2,k} + \Psi_{i,j-2,k} + \Psi_{i,j,k+2} + \Psi_{i,j,k-2}\right. \\
\; &\left. - 16\,(\Psi_{i+1,j,k} + \Psi_{i-1,j,k} + \Psi_{i,j+1,k} + \Psi_{i,j-1,k} + \Psi_{i,j,k+1} + \Psi_{i,j,k-1}) + 90\,\Psi_{i,j,k}\right]. \notag
{5}
&1) \; D_{i,j,k} = \label{3d2shocs}
\\
&\begin{tabular}{l} 

\end{tabular} \notag
\\
\;&\; \notag
\\
&2) \; \nabla^2\Psi_{i,j,k} \approx \label{3d2shocs2} 
\\
&\begin{tabular}{l}

\end{tabular} \notag
\\
&\begin{tabular}{l}

\end{tabular} \notag

\label{G3D2SHOC}
\vec G = \frac{1}{12} \times \left\{192,191,190,189,174,173,172,156,155,138,36,21,20,6,5,4,-9,-10,-11,-12 \right\}.

\label{kbound3d2shoclin}
{\bf k} < \frac{\sqrt{8}}{16}\,\frac{h^2}{a},

\label{soliton}
\Psi(x,t) = \sqrt{\frac{2\,\Omega}{s}}\,\mbox{sech}\left[\sqrt{\frac{\Omega}{a}}\,x\right]\,\mbox{exp}\left(i\,\Omega\,t\right),

\label{exmp2Dvort}
\Psi(r,\theta,t) = f(r)\,\mbox{exp}[i\,(m\,\theta + \Omega\,t)],

\Psi(x,y,t=0) = f(r_1)f(r_2)\,\mbox{exp}[i\,m\,(\theta_1 + \theta_2)],

\label{3dexpsol}
\Psi(x,y,z,t=0) = \mbox{exp}\left(-\frac{x^2 + y^2 + z^2}{2\,a}\right) \mbox{exp}\left(-i\,\frac{x}{2}\right),

\label{3dexpsolv}
V(x,y,z) = \frac{x^2 + y^2 +z^2}{a},
2.0ex]
 \includegraphics[width=2in]{linear_T0_MOD2_3DVOL.ps}
 \includegraphics[width=2in]{linear_n11070_MOD2_3DVOL_t=101999.ps}
 \includegraphics[width=2in]{linear_CDsTBn10854_MOD2_3DVOL_t=999979.ps}
 \caption{(Color online) Examples of integrating the LSE and NLSE before and after the numerical stability bound for the examples described in Sec.~\ref{s:num}.  Left to right columns:  Initial condition, solution with , and solution with .  Top to bottom:  one-, two-, and three-dimensional test cases using the CD, 2SHOC, and CD schemes respectively. For the one-dimensional test, the predicted stability bounds are  and the solution is shown with  (middle) and  (right) at . 
For the two-dimensional test, the predicted stability bounds are  and .  The solution is shown with  (middle) and  (right) at  and 
 respectively. 
For the three-dimensional test, the predicted stability bounds are  and .  The solution is shown with  (middle) and  (right) at .\label{f:results}}
\end{figure} 


To test the stability bounds, each solution is integrated to an ending time of  and it is observed if the solution remains stable.  We increase the time-step  until the solution becomes unstable within the  simulation time, at which point the largest time-step that was stable is denoted .  This is then compared to the computed linear [Eqs.~(\ref{kdcdlin}) and (\ref{kd2shoclin}) denoted ] and fully linearized [Eq.~(\ref{kdfull}) denoted ] stability bounds formulated in Secs.~\ref{s:1dstb}--\ref{s:3dstb}.  The time-step is incremented to yield the numerical stability limit to within four significant figures.  All of the simulations are performed using the NLSEmagic software package \cite{NLSEMAGIC}.
  
\begin{table}[p]
\begin{center}
\begin{tabular}{|l|r|r|r|r|r|} \hline
Example: &  &  &  & \%-diff  & \%-diff  \\
\hline
1D CD    & 0.02828  & 0.02828  & 0.02832  &  0.14 &  0.14 \\
1D 2SHOC & 0.02121  & 0.02121  & 0.02124  &  0.14 &  0.14 \\
2D CD    & 0.01414  & 0.01407  & 0.01402  & -0.85 & -0.36 \\
2D 2SHOC & 0.01061  & 0.01057  & 0.01054  & -0.66 & -0.28 \\
3D CD    & 0.009428 & 0.008650 & 0.009213 & -2.28 &  6.51 \\
3D 2SHOC & 0.007071 & 0.006624 & 0.006992 & -1.12 &  5.56 \\
\hline
\end{tabular}
\caption{Numerical test results of finding the numerical stability bound () for the example problems described in Sec.~\ref{s:num} compared to the predicted linear () and linearized () bounds.\label{t:results}}
\end{center}
\end{table}


Before displaying the results, we point out that there are some sources of error to consider.  First, the predicted stability bounds are linearized and therefore will not be the same as the corresponding true nonlinear stability bounds.  Second, in our analysis, we chose to use every possible combination of  which may lead to predictions of the bounds which are stricter than the true bound.  Finally, it is sometimes difficult to determine the true stability bound numerically, as some unstable time-steps may only exhibit their instability after a very long simulation time.  For our test, we choose a moderately long simulation time, but the exact bound may be slightly higher than the given result. 

The results are shown in Table~\ref{t:results}, while Fig.~\ref{f:results} shows the solutions before and after the recorded numerical stability bounds for three chosen examples.
We see that overall, the numerical results match the predicted stability 
values quite well (especially in one and two dimensions) with a maximum 
percent difference of  when  in the three-dimensional
example, but with a typical percent difference less that  when 
.  
It is noted that in some cases the predicted bounds are stricter than the numerical result, while in other cases, they are too lenient, noting that the examples with  were all too strict, while those with  were all too lenient.  However, due to the small number of tests, no conclusions about the effect of the sign and presence of the parameters and external potential of the LSE and NLSE on the stability bound predictions can be drawn from these observations.  
In terms of choosing a stable time-step for LSE and NLSE simulations, the results given are well within a tolerable range, and in practice one would use a time-step some percentage (say --) lower than the predicted bound to ensure stability over long integration times.

\section{Conclusion and summary of results}
\label{s:sum}
In this paper we have formulated linearized stability bounds for using second- and fourth-order spatial finite-differencing with fourth-order Runge-Kutta time-stepping for the multi-dimensional nonlinear Schr{\"o}dinger equation (NLSE) with Dirichlet, modulus-squared Dirichlet, Laplacian-zero, and periodic boundary conditions.  

A summary of the stability results for easy reference is given presently.  For the nonlinear Schr{\"o}dinger equation defined as 

where  and  are parameters of the system and  is an external potential, the numerical stability bounds on the time-step when using the fourth-order Runge-Kutta time-stepping scheme is as follows: 

In the linear case where  and with no external potential (), utilizing periodic, Dirichlet, or Laplacian-zero boundary conditions, the stability bound on the time-step  when using the second-order central difference (CD) scheme in a -dimensional setting is 

while that of using a fourth-order central difference scheme (with interior points computed in the two-step high-order compact (2SHOC) methodology of Ref.~\cite{ME_2SHOC}) is

The linearized stability bounds for the general NLSE are

where  are the boundary points as defined by Table~\ref{t:bc2} (or in the periodic case is ignored), the elements of  is defined as

where the index  spans the entire grid, and  is a set of values defined in Table~\ref{t:sumresults}, determined by the dimension and method being used.
\begin{table}[htbp] 
\caption{Values of  in Eq.~(\ref{kdfull}).}
\begin{center}
\begin{tabular}{|l|c|c|c|} \hline
  & Dirichlet () & Laplacian-zero ()        & MSD () \\ \hline
 &        &  &  \\ \hline
\end{tabular}
\end{center}
\label{t:bc2}
\end{table}
\begin{table}[htbp] 
\caption{Values of  in Eq.~(\ref{kdfull}).}
\begin{center}
\begin{tabular}{|l|c|c|} \hline
Scheme  & CD          & 2SHOC  
\\ \hline
\;     & \;                  &\;              
\\
1D     &        &     
\\
       & \;                  &
\\
2D     &    &   
\\
\;     &  \;                 & 
\\
3D     &   & 
\\
\;     & \;                        & 
\\
\;     & \;                        &   
\\ \hline
\end{tabular}
\end{center}
\label{t:sumresults}
\end{table}

We have found through numerical testing (those of Sec.~\ref{s:num}, as well as others not reported here) that to ensure stability in all dimensions for typical problems, the bounds must be lowered by about -- (most likely due to nonlinear effects).  Also, we note that the reduced linear results are often similar to the full linearized bounds and can therefore be used as a good quick estimate of the stability bound.

\section*{Acknowledgments}
This research was supported by NSF-DMS-0806762 and the Computational
Science Research Center (CSRC) at SDSU.
We gratefully acknowledge insightful discussions with Peter Blomgren.

\def\myitemsep{5pt}
\bibliographystyle{elsart-num-sort}
\bibliography{RK4STB5}  
\end{document}
