\documentclass[10pt, twocolumn, final]{IEEEtran}
\usepackage{algorithm, algorithmic}
\usepackage{epsfig, graphics, color}
\usepackage{amsmath, amssymb, amsfonts, amsbsy, mathrsfs, bm}
\usepackage{cases}
\usepackage{enumerate, fancyvrb}
\usepackage{theorem}
\usepackage[compress]{cite}
\DeclareMathAlphabet{\mathsfsl}{OT1}{cmss}{m}{sl}

\ifCLASSOPTIONcompsoc
 \usepackage[tight, normalsize, sf, SF]{subfigure}
\else
 \usepackage[tight, footnotesize]{subfigure}
\fi

\def\red#1{{\color{red} #1}}
\def\blue#1{{{\color{blue} #1}}}
\def\old#1{{\color{green} [DELETED: #1]}}

\title{Cooperative and Distributed Localization for Wireless Sensor Networks in Multipath Environments}
\author{Mei Leng, Wee Peng Tay, and Tony~Q.S.~Quek
\thanks{M. Leng and W.P. Tay are with the Nanyang Technological University, Singapore. (e-mail:lengmei,wptay@ntu.edu.sg).}
\thanks{T.Q.S. Quek is with the Institute for Infocomm Research, Singapore. (e-mail:qsquek@i2r.a-star.edu.sg).}}

\newtheorem{theorem}{\bf Theorem}
\newtheorem{lemma}{\bf Lemma}
\newtheorem{collary}{\bf Collary}
\setlength{\unitlength}{1 cm}
\renewcommand{\baselinestretch}{1}

\newcommand{\bA}{\mathbf{A}}
\newcommand{\bD}{\mathbf{D}}
\newcommand{\bH}{\mathbf{H}}
\newcommand{\bK}{\mathbf{K}}
\newcommand{\bL}{\mathbf{L}}
\newcommand{\bP}{\mathbf{P}}
\newcommand{\bQ}{\mathbf{Q}}
\newcommand{\bI}{\mathbf{I}}
\newcommand{\bU}{\mathbf{U}}
\newcommand{\bW}{\mathbf{W}}
\newcommand{\bZ}{\mathbf{Z}}

\newcommand{\be}{\mathbf{e}}
\newcommand{\bs}{\mathbf{s}}

\newcommand{\bSigma}{\mathbf{\Sigma}}
\newcommand{\N}[2]{{\mathcal{N}\left(#1\ ;\ #2\right)}}
\newcommand{\tc}[1]{^{(#1)}}
\newcommand{\norm}[1]{{\left\lVert #1 \right\rVert}}
\newcommand{\Fnorm}[1]{{\left\lVert #1 \right\rVert}_{\mathrm{F}}}
\newcommand{\EE}[1]{{\mathbb{E}\left[ #1 \right]}}

\begin{document}
\maketitle

\begin{abstract}
We consider the problem of sensor localization in a wireless network in a multipath environment, where time and angle of arrival information are available at each sensor. We propose a distributed algorithm based on belief propagation, which allows sensors to cooperatively self-localize with respect to one single anchor in a multihop network. The algorithm has low overhead and is scalable. Simulations show that although the network is loopy, the proposed algorithm converges, and achieves good localization accuracy.
\end{abstract}

\begin{IEEEkeywords}
Distributed localization, Wireless Sensor Network, belief propagation, non-line-of-sight.
\end{IEEEkeywords}

\section{Introduction}
A wireless sensor network (WSN) consists of many devices (or nodes) capable of onboard sensing, computing and communications. WSNs are used in industrial and commercial applications, such as environmental monitoring and pollution detection, control of industrial machines and home appliances, event detection, and object tracking \cite{Akyildiz2007,Bulusu2005,Tay2009}. In most applications, the data collected by the sensor nodes can only be meaningfully interpreted if it is correlated with the location of the corresponding sensors. The Global Positioning System (GPS) is widely used for localization in outdoor environments\cite{Bulusu2000}. However, GPS is a costly option and is not suitable for power-limited sensor nodes in WSNs. Furthermore, GPS signals do not penetrate well to indoor environments. Therefore, alternatives to GPS localization have been widely studied\cite{Bulusu2000, Wymeersch2009}. In a WSN, nodes whose positions are known are called ``\emph{anchors}''. By making use of pairwise range or angle measurements between anchors and/or other sensor nodes whose positions are unknown, sensors with no access to GPS can perform self-localization. Typical techniques include the use of time-of-arrival (TOA), time-difference-of-arrival (TDOA), received-signal-strength (RSS) and/or angle-of-arrival (AOA) information in triangulating the location of a node. This is usually studied in line-of-sight (LOS) environments\cite{Sun2005, Mao2007}.

However, LOS signals do not always exist in urban or cluttered environments, where signals usually experience multiple refections and diffractions. Such signals are referred to as nonline-of-sight (NLOS) signals and are commonly encountered in both indoor (e.g., residential buildings, offices and shopping malls) and outdoor (e.g., metropolitan and urban) environments. NLOS errors mitigation techniques have been extensively investigated \cite{Guvenc2009, Al-Jazzar2007, Cong2005, Al-Jazzar2009, Xie2009, Miao2007, Seow2008}, but most of the algorithms in the literature focus on locating one single sensor with several anchors. Since they require each sensor to have direct signal paths to anchors in the network, such algorithms cannot be applied in network-wide localization. On the other hand, several algorithms for network-wide localization have been proposed in the literature \cite{Biswas2006, Tseng2007, Srirangarajan2008, Wymeersch2009, Ihler2005}. Unfortunately, only LOS signals are considered in these algorithms.

Distributed localization algorithms for multipath environments were proposed in \cite{Ananthasubramaniam2008,Ekambaram2010}, where NLOS error is modeled as a positive bias in range and angle measurements, and its statistical characteristics are inferred by numerical methods, such as bootstrap sampling in \cite{Ananthasubramaniam2008}, and particle filters in \cite{Ekambaram2010}. One of the major disadvantages is that these Bayesian inference techniques require a large number of observations and are computationally expensive. Generally, the statistical model for NLOS errors depends on various factors, such as signal bandwidth, propagation medium and environment temperature. Different models including the uniform, exponential and Rayleigh distributions, have been proposed in the literature \cite{Pedersen2000,Zekavat2011}.

Instead of modeling NLOS errors in multipath environments as random biases, geometric analysis can be applied in pairwise localization \cite{Xie2009,Miao2007,Seow2008}, where measurements of different paths are modeled using closed-form expressions, which significantly simplifies the system model. In this paper, we derive a distributed localization algorithm based on range and direction measurements at each node, and where nodes exchange information to cooperatively perform self-localization relative to a fixed reference node. We show through simulation that by exchanging limited information, all the nodes in the network can perform localization to a good accuracy. We compare the performance to that achieved without cooperation, and show that cooperation amongst neighboring nodes significantly improves the localization accuracy.

The rest of this paper is organized as follows. In Section \ref{Section:System}, we define the system model. We describe our algorithm in Section \ref{Section:DistributedLocalization}, and provide simulation results in Section \ref{Section:SimulationResults}. In Section \ref{Section:Conclusions}, we summarize and conclude.

\section{System model}\label{Section:System}
Consider a network of  sensors, . The position of  is , where  and  are its - and -coordinates respectively. Without loss of generality, we assume that the position of  is known and that . The objective of each  is to perform self-localization relative to .

In the following, we consider two nodes  and . Similar to \cite{Seow2008}, we describe a model to relate the range and direction measurements at each node to their positions. Suppose that there are  LOS or NLOS paths between  and . An example of a single-bounce scattering path is shown in Figure \ref{Figure-Scatter}, where the signal from  to  is reflected at a nearby scatter, and the communication link between two nodes is assumed to be symmetric. Let  be the distance measured by  using the time-of-arrival information of the signal along the  path from , and  be the corresponding angle-of-arrival information. Nodes  and  exchange measurements with each other, so that both nodes have the measurements .

\begin{figure}[!t]
\centering
\includegraphics[width=0.35\textwidth]{Figure-Scatter}
\caption{An example for single-bounce scattering path between  and .} \label{Figure-Scatter}
\end{figure}

Consider the  path between  and . Given the position of  and , the position of  cannot be determined with certainty even in the absence of measurement and communication noise. As shown in Figure \ref{Figure-Scatter}, the estimated position for  can be any point along the line . If there are multiple paths between  and  from non-parallel scatters, the position of  can be found as the intersection point of two such lines. Suppose that there is no measurement noise, then a straightforward geometric consideration shows that

where  and  are the positions of  and  respectively. A vector perpendicular to , is

Since both  and  are on the line , we have , from which we obtain

which can be further simplified as

where

We have made use of the fact that  for symmetric communication links between  and  in \eqref{relation}. We have not factored in measurement and communication noise up to this point. Let the corresponding noisy measurements be . Modeling the total effect of noise as a Gaussian random error , we have

We assume that measurements for the  paths are such that  are linearly independently, otherwise some of the paths are duplicates of each other. We also assume that the noise terms  are i.i.d. Gaussian random variables with zero mean and variance . Stacking the measurements from all  paths into a vector , and letting , and , we can estimate the position of  from  using

where  is the Moore-Penrose pseudo-inverse of the matrix . If there are more than one signal paths between nodes  and , we have . If there is only one signal path, . Let . The posterior distribution of the node locations is given by


Similar analysis in \cite{Seow2008} proposes a least square estimator to localize a single node with respect to a reference node. However, to localize every node in a network, it is necessary to consider the interaction between  and all the other nodes . The MAP estimator for  is given by (\ref{joint}) on top of next page.
\begin{figure*}[!t]
\setcounter{equation}{6}

\hrulefill
\end{figure*}
The joint posterior distribution in (\ref{joint}) depends on interactions amongst all the variables and is difficult to compute by brute-force integration.

In order to provide a computationally efficient algorithm to calculate the marginal distributions, we observe that the state at any sensor depends directly only on its neighboring sensors whose number is usually far less than the total number of variables. To explore such conditional independence structure, we make use of belief propagation in the following and propose an efficient algorithm which computes a set of marginal distributions from the joint posterior distribution without performing a full integration as in (\ref{joint}).

\emph{Remark 1}: When there exists a LOS path between  and , it is easy to see that  and , which is a special case of (\ref{relation}).

\emph{Remark 2}: When there exists paths with multiple bounces between  and , a two-step proximity detection scheme suggested in \cite{Seow2008} can be applied to detect and discard such paths, leaving measurements from either LOS and/or single-bounce paths for processing.

\section{Distributed localization based on belief propagation}\label{Section:DistributedLocalization}
\begin{figure}[!t]
\centering
\includegraphics[width=0.35\textwidth]{NetworkFactorGraph}
\caption{An example for the factor graph of a network with  nodes, where dashed lines indicate that a one-bounce scattering path exists between the corresponding two nodes, and the arrows indicate the direction of message flows.} \label{Figure-FG}
\end{figure}

We use Belief Propagation (BP) on a factor graph\cite{Kschischang2001} in this paper. An example for a network with  nodes is shown in Figure \ref{Figure-FG}. Each random variable  is represented by a variable node (circle). The interaction between two sensors  and  is represented by a factor node (square) connected to both variable nodes  and . We split the interaction between  and  into two factors , and . Since messages only flow in one direction along the edges in the factor graph, they can be broadcast by the sensor nodes.

Without loss of generality, suppose the sensor  is the anchor with a known position . For each variable , , the marginal posterior distribution  is found by iterative belief propagation, where two kinds of messages are involved,
\begin{itemize}
\item{: belief of its own state at the variable node  after the  iteration,

where  is the index set of 's neighboring sensors.}
\item{: message from the factor node  to the variable node  in the  iteration, which represents 's belief of 's state, resulting from interactions between  and ,
}
\end{itemize}
Therefore, setting the initial belief  to be the corresponding prior distribution , beliefs (\ref{belief}) and messages (\ref{message}) are iteratively updated at each sensor, and the estimation for  is found by maximizing the converged belief  with respect to . When prior distributions  are Gaussian, closed-form expressions for beliefs and messages can be obtained as follows.

\subsection{Derivation for closed-form beliefs and messages}\label{Subsection:Derivation}
To derive closed-form expressions for (\ref{belief}) and (\ref{message}), we first consider the message from the anchor  to a neighboring sensor . Since , its belief is a constant and can be represented as , where  is the Krocnecker delta function. Therefore, the message from  to  will be constant over all iterations, and is given by

where  and  with .

Other the other hand, for all nodes  other than , we set the initial belief of  as a Gaussian distribution with zero mean and large variance, i.e., for , , where , and we define

where  is a large positive value of at least an order of magnitude larger than the dimension of the environment in which the sensor nodes are located. Let  be the largest eigenvalue of . We assume that .

As the BP algorithm proceeds, the belief  for  after  iterations is updated as  and its message to  in the  iteration is

where

The message  is a Gaussian distribution, so only the mean  and covariance matrix  need to be passed to node .

The belief of  after the  iteration then follows from (\ref{belief}). Since all the messages in the product in \eqref{belief} are Gaussian distributions, we have  with

and

At the end of the  iteration, sensor  estimates its position by maximizing the belief  with respect to . Since  is a Gaussian distribution, the estimator for  at the  iteration is given by . This iterative procedure is formally given in Algorithm \ref{algorithm}. Notice that factor nodes are virtual, and are introduced only to facilitate the derivation of the algorithm. In practice, the update is performed at each individual sensor directly.

\begin{algorithm}[!t]
\caption{Distributed Localization in Multi-path Environments} \label{algorithm}
\begin{algorithmic}[1]
\STATE{\textbf{Initialization}:}
\STATE{Set the position at the anchor  as .}
\STATE{Set  and
}
\STATE{\textbf{Iteration until convergence}:}
\FOR{\textrm{the  iteration}}
\STATE{\textbf{sensors}  with } \textbf{in parallel}
\STATE{broadcast the current belief  to neighboring sensors;}
\STATE{receive  from neighboring sensors , where ;}
\STATE{update its belief as  with

and

where  and  are given in (\ref{inv}) and (\ref{invmu}) respectively. }
\STATE{estimate its position as .}
\STATE{\textbf{end parallel}}
\ENDFOR
\end{algorithmic}
\end{algorithm}

\section{Simulation Results}\label{Section:SimulationResults}
Numerical simulations are conducted to validate the effectiveness of our proposed algorithm. We consider a network with  nodes randomly distributed in a  square area. The factor graph is that in Figure \ref{Figure-FG}, where  represents the anchor with a fixed location at , while the other nodes are the remaining sensors' locations. We set , , , and . Any two nodes that can communicate with each other through single-bounce scattering paths are connected with their connection indicated by a dashed line. The ranging measurement errors are i.i.d. Gaussian random variables with zero mean and standard variance . The measurement error for AOA is assumed to be uniformly distributed in . Each point in the figures is an average of  independent simulation runs.

\begin{figure}
  \centering
  \subfigure[absolute errors on x-coordinates.]{
    \label{Figure-CDF-x-Ortho}
    \includegraphics[width=0.5\textwidth]{Figure-CDF-x-Ortho}}
  \subfigure[absolute errors on y-coordinates.]{
    \label{Figure-CDF-y-Ortho}
    \includegraphics[width=0.5\textwidth]{Figure-CDF-y-Ortho}}
  \caption{Cumulative distribution function of absolute localization errors where scatters are orthogonal.}\label{Figure-CDF-Ortho}
\end{figure}

\begin{table*}[!t]
\caption[short title]{Mean error of estimated location at each sensor} \label{Table-ME}
\centering
\begin{tabular}{|c||c|c|c|c|c|c|}

\hline Localization scheme & Mean Error &  &  &  &  \\
\hline

Cooperative &  &   &    &    &   \\
\cline{2-6}
Localization &  &  &   &    &  \\
\hline

Pairwise &  &   &    &    &   \\
\cline{2-6}
Localization &  &  &   &    &  \\
\hline

\end{tabular} \3ex]
\end{table*}

First, we consider scenarios where scatters are orthogonal. We compare the performance of cooperative and pairwise localization. The cumulative distribution functions for absolute localization error of each sensor are shown in Figure \ref{Figure-CDF-Ortho}. Corresponding to the factor graph in Figure \ref{Figure-FG}, sensors  and  are directly connected to the anchor. Sensors  and  do not have any paths to the anchor. Instead each has a NLOS path to  or  respectively, and a NLOS path between themselves. In pairwise localization,  localizes using only measurements from , and  localizes with respect to . It can be seen from Figure \ref{Figure-CDF-x-Ortho} that  and  are localized with larger errors than  and , and this is because errors are accumulated over hops. In cooperative localization,  and  exchange information and incorporate measurements from the NLOS path between themselves. As shown in Figure \ref{Figure-CDF-x-Ortho}, the cooperative localization achieves better performances with more than  of the localization errors less than  and all errors smaller than . Moreover, results for estimation on -coordinates are shown in Figure \ref{Figure-CDF-y-Ortho}. It can be seen that both schemes have similar performances for sensors directly connected to the anchor (e.g.,  and ), and cooperation localization for  and  achieves better performances. The mean absolute errors for both schemes are further shown in Table \ref{Table-ME}.

\begin{figure}
  \centering
  \subfigure[absolute errors on x-coordinates.]{
    \label{Figure-CDF-x-Bitho}
    \includegraphics[width=0.5\textwidth]{Figure-CDF-x-Bitho}}
  \subfigure[absolute errors on y-coordinates.]{
    \label{Figure-CDF-y-Bitho}
    \includegraphics[width=0.5\textwidth]{Figure-CDF-y-Bitho}}
  \caption{Cumulative distribution function of absolute localization errors when scatters are horizontal or at .} \label{Figure-CDF-Bitho}
\end{figure}

\begin{figure}[!t]
\centering
\includegraphics[width=0.45\textwidth]{Figure-Convergence-Bitho}
\caption{Convergence of the mean absolute error when scatters are horizontal or at , corresponding to that in Figure \ref{Figure-CDF-Bitho}.} \label{Figure-Convergence}
\end{figure}

Second, we consider scenarios where scatters are horizontal or at angle  to the horizontal. As can be seen from Figure \ref{Figure-CDF-Bitho}, cooperation among neighboring sensors improves performance on both - and - coordinates. We also note that compared with Figure \ref{Figure-CDF-y-Ortho}, estimation errors for -coordinates deteriorate due to correlation between measurements on the vertical direction. On the other hand, we investigate the convergence rate for our proposed algorithm in this biorthogonal scenarios, and the result is shown in Figure \ref{Figure-Convergence}. Simulations are also conducted when scatters are at ,  and  to the horizontal. Similar results as in Figure \ref{Figure-Convergence} are obtained and hence omitted here. These numerical results suggest that even for scenarios with non-orthogonal scatters, the mean of belief at each sensor converges in the proposed algorithm.

\section{Conclusions}\label{Section:Conclusions}
In this paper, we propose a distributed algorithm based on belief propagation for network-wide localization in multipath environments. The proposed algorithm requires communications only between neighboring sensors, with each sensor processing only information local to itself. The proposed algorithm has low overhead and can achieve robust and scalable localization. By utilizing both TOA and AOA information of the single-bounce scattering paths, we require only one anchor in the whole network, and sensors that do not have LOS/NLOS paths to the anchor can be localized by cooperation with its neighboring sensors. Simulation results show that our proposed algorithm achieve accuracy less than  in a  square area.

\bibliographystyle{IEEEtran}
\bibliography{IEEEabrv,Distributed_Localization}

\end{document} 