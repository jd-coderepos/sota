\begin{figure*}[ht]
\begin{center}
\small
    \begin{tabular}{p{16cm}}
\textit{P6418} \begin{otherlanguage*}{russian} Термин Computer science (Компьютерная наука) появился \underline{\textcolor{green}{в 1959 году}} в научном журнале Communications of the ACM, в котором \underline{\textcolor{red}{Луи Фейн}} (Louis Fein) ратовал за создание Graduate School in Computer Sciences (Высшей школы в области информатики) \ldots Усилия Луи Фейна, численного аналитика Джорджа Форсайта и других увенчались успехом: университеты пошли на создание программ, связанных с информатикой, начиная \underline{\textcolor{blue}{с Университета Пердью}} в 1962. \end{otherlanguage*}   \\\hline
\textit{P6418} The term "computer science" appears in a \underline{\textcolor{green}{1959}} article in Communications of the ACM, 
in which \underline{\textcolor{red}{Louis Fein}} argues for the creation of a Graduate School in Computer Science \ldots   Louis Fein's efforts, and those of others such as numerical analyst George Forsythe, were rewarded: universities went on to create such departments, starting with \underline{\textcolor{blue}{Purdue}} in 1962.
 \\\hline 
\textit{\textcolor{green}{Q11870}} \begin{otherlanguage*}{russian}Когда впервые был применен термин Computer science ( Компьютерная наука )? \end{otherlanguage*}\\
\textit{\textcolor{green}{Q11870}} When did the term "computer science" appear? \\\hline
\textit{\textcolor{red}{Q28900}} \begin{otherlanguage*}{russian}Кто впервые использовал этот термин?\end{otherlanguage*} \\
\textit{\textcolor{red}{Q28900}} Who was the first to use this term? \\\hline
\textit{\textcolor{blue}{Q30330}} \begin{otherlanguage*}{russian}Начиная с \underline{каого} учебного заведения стали применяться учебные программы, связанные с информатикой?\end{otherlanguage*}\\
\textit{\textcolor{blue}{Q30330}} Starting with \underline{wich} university were computer science programs created?\\\hline
\end{tabular}
\end{center}
\caption{\label{fig:examples} A sample SberQuAD entry (both the original and the translation): answers are underlined and colored. 
The word \textbf{which} in \textit{Q30330} is misspelled on purpose to reflect the fact that the original has a misspelling.}
\end{figure*}
