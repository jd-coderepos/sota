\documentclass[12pt]{article}


\usepackage{amsmath,amsfonts,amssymb,amsthm,amscd}
\usepackage{epsfig}


\newtheorem{theorem}{Theorem}[section]             
\newtheorem{proposition}[theorem]{Proposition}            
\newtheorem{lemma}[theorem]{Lemma}
\newtheorem{corollary}[theorem]{Corollary}
\newtheorem*{conjecture}{Conjecture}
\newtheorem*{problem}{Problem}
\newtheorem*{question}{Question}
\newtheorem{obs}[theorem]{Observation}
\newtheorem{claim}[theorem]{Claim}

\newtheorem*{claim_}{Claim}                               
\newtheorem*{lemma_}{Lemma}                               

\usepackage{hyperref}

\def\pfoftmain{\noindent{\em Proof of Theorem~\ref{t:main} (even case)}}
\def\pf{\noindent{\em Proof.}}
\newcommand{\keywords}[1]{\par\addvspace\baselineskip
\noindent\keywordname\enspace\ignorespaces#1}

\def\inst#1{}
\renewcommand{\thefootnote}{\fnsymbol{footnote}}

\renewcommand{\qedsymbol}{\rule{1ex}{.8em}}

\long\def\onefigure#1#2{\begin{figure*}[tbh]
\begin{center}
\includegraphics{#1}
\end{center}
\caption{#2}
\end{figure*}
} 
\newcommand{\myfig}[2]  {\onefigure{{g-#1.eps}}{\label{f:#1} #2} }

\begin{document}





\title{Universal Sets for Straight-Line Embeddings of Bicolored Graphs\thanks{
Research was supported by the project 1M0545 of the Ministry of Education of the Czech Republic 
and by the grant SVV-2010-261313 (Discrete Methods and Algorithms).
Viola M\'{e}sz\'{a}ros was also partially supported by OTKA grant K76099
and by European project IST-FET AEOLUS.
Josef Cibulka and Rudolf Stola\v{r} were also 
supported by the Czech Science Foundation under the contract no.\ 201/09/H057.
Josef Cibulka, Jan Kyn\v{c}l and Pavel Valtr were also supported by the Grant Agency 
of the Charles University, GAUK 52410.
\newline 
Part of the research was conducted during the Special Semester on Discrete
and Computational Geometry at \'Ecole Polytechnique F\'ed\'erale de Lausanne, organized and
supported by the CIB (Centre Interfacultaire Bernoulli) and the SNSF (Swiss
National Science Foundation).
} \thanks{A preliminary version appeared in 
proceedings of Graph Drawing 2008~\cite{gdversion}.}} 
\author{Josef Cibulka\inst{1}, Jan Kyn\v{c}l\inst{2}, Viola M\'{e}sz\'{a}ros\inst{2,3}, \\
Rudolf Stola\v{r}\inst{1} and Pavel Valtr\inst{2}
} 
\date{}

\maketitle

\begin{center}
{\footnotesize
\inst{1} 
Department of Applied Mathematics, \\
Charles University, Faculty of Mathematics and Physics, \\
Malostransk\'e n\'am.~25, 118~ 00 Prague, Czech Republic; \\ 
\texttt{cibulka@kam.mff.cuni.cz, ruda@kam.mff.cuni.cz} 
\\\ \\
\inst{2}
Department of Applied Mathematics and Institute for Theoretical Computer Science, \\
Charles University, Faculty of Mathematics and Physics, \\
Malostransk\'e n\'am.~25, 118~ 00 Prague, Czech Republic; \\
\texttt{kyncl@kam.mff.cuni.cz}
\\\ \\
\inst{3}
Bolyai Institute, University of Szeged, \\
Aradi v\'ertan\'uk tere 1, 6720 Szeged, Hungary; \\
\texttt{viola@math.u-szeged.hu}
}
\end{center}  


\begin{abstract}
A set  of  points is \emph{-color universal} for a graph  on  vertices if for every proper 
-coloring of  and for every -coloring of  with the same sizes of color classes as  has, 
 is straight-line embeddable on .

We show that the so-called double chain is -color universal for paths if each of the two
chains contains at least one fifth of all the points, but not if one of the chains is more than 
approximately  times longer than the other.

A -coloring of  is \emph{equitable} if the sizes of the color classes differ by at most .
A bipartite graph is \emph{equitable} if it admits an equitable proper coloring.
We study the case when  is the double-chain with chain sizes differing 
by at most  and  is an equitable bipartite graph. We prove that this  is not -color universal 
if  is not a forest of caterpillars and that it is -color universal for equitable caterpillars 
with at most one half non-leaf vertices. 
We also show that if this  is equitably -colored, then equitably properly -colored 
forests of stars can be embedded on it.

\end{abstract}



\maketitle


\section{Introduction}
\subsection{Previous Results}
It is frequently asked in geometric graph theory whether
a given graph  can be drawn without edge crossings on a given planar point set  
under some additional constraints on the drawing. In this paper, we always assume 
.

One possibility is to prescribe a fixed position for each vertex of .
If the edges are allowed to be arbitrary curves, then we can obtain a planar drawing 
of an arbitrary graph  by moving vertices from an arbitrary planar drawing of  
to the given points. Pach and Wenger showed~\cite{pachwenger} that every planar  with 
prescribed vertex positions can be drawn so that each edge is a piecewise-linear curve with  
bends and that this bound is tight even if  is a path.

In another setting, we are only given the graph  and the set of points , but we
are allowed to choose the mapping between  and .
Kaufmann and Wiese~\cite{kaufmannwiese} showed that two bends per edge are then always enough 
and for some graphs necessary. An \emph{outerplanar graph} is a graph admitting a planar drawing 
where one face contains all vertices. By a result of Gritzmann et al.~\cite{gritzmann91}, 
outerplanar graphs are exactly the graphs with a straight-line planar drawing on 
an arbitrary set of points in general position.

In this paper we are dealing with a combination of these two versions. The vertices and
points are colored with two colors and each vertex has to be placed on a point of the 
same color. An obvious necessary condition to find such a drawing is that each color class
has the same number of vertices as points. 
We then say that the -coloring of  is \emph{compatible} with the -coloring of .

A \emph{caterpillar} is a tree in which the non-leaf vertices induce a path. 
The coloring of  is \emph{proper} if it doesn't create any monochromatic edge.
It is known that drawing some bicolored planar graphs on some bicolored point sets requires 
at least  bends per edge~\cite{giacomograph},
but caterpillars can be drawn with two bends per edge~\cite{giacomograph} and one bend per
edge is enough for paths~\cite{giacomopath} and properly colored caterpillars~\cite{giacomograph}.

We restrict our attention to the proper -colorings of a bipartite graph . Then 
the question of embeddability of  on a bicolored point set is very similar to finding a non-crossing
copy of  in a complete geometric graph from which we removed edges of the two complete subgraphs
on points of the two color classes. 
The only difference is that in the latter case, we can swap colors on some connected components of .
A related question was posed by Micha Perles on DIMACS Workshop on Geometric Graph Theory in 2002.
He asked what is the maximum number  such that if we remove arbitrary  edges  
from a complete geometric graph on an arbitrary set of  points in general position, we
can still find a non-crossing Hamiltonian path.
\v{C}ern\'y et al.~\cite{cerny07} showed that 
and also that it is safe to remove the edges of an arbitrary complete graph
on  vertices and that this bound is asymptotically optimal. 
Aichholzer et al.~\cite{aichholzer10} summarize history and results of this type also for
graphs different from the path.

It is thus impossible to find a non-crossing Hamiltonian alternating (that is, properly colored) path (\emph{NHAP} for short) 
on some bicolored point sets. Kaneko et al.~\cite{kanekokanosuzuki} proved that the smallest such point set 
has  points if we allow only even number  of points and  for arbitrary . 

Several sufficient conditions are known under which a bicolored point set admits an NHAP. An NHAP exists whenever
the two color classes are separable by a line~\cite{abellanas99} or if one of them is composed of 
the points of the convex hull of ~\cite{abellanas99}.

The result on sets with color classes separable by a line readily implies that any -colored 
set  with each color class of size  admits a non-crossing alternating path (NAP) on at least  
points of . It is an open problem if this lower bound can be improved to , where 
is unbounded (see also Chapter 9.7 of the book~\cite{brassmoserpach}). 
On the other hand, there are such -colored sets admitting no NAP of length more than 
\cite{abellanas03,kynclpt}.
This upper bound is proved for certain colorings of points in convex position. The above
general lower bound  can be slightly improved for sets in convex position~\cite{kynclpt, hajnalmeszaros}, 
the best bound is currently  by Hajnal and M\'esz\'aros~\cite{hajnalmeszaros}.


The main result of this paper is that some point sets contain an NHAP for any equitable 
-coloring of their points. We call such point sets \emph{-color universal} for a path.

See the survey of Kaneko and Kano~\cite{kanekokano} for more results on embedding graphs on bicolored point sets.
One of the results mentioned in the survey is the possibility to embed some graphs with a fixed -coloring on
an arbitrary compatibly -colored point set. Let  be a forest of stars with centers colored black and 
leaves white and let  be a -colored point set. If we map the centers of stars arbitrarily and then 
we map the leaves so that the sum of lengths of edges is minimized, then no two edges cross.

Previously, a different notion of universality was considered in the context of embedding colored graphs 
on colored point sets. A -colored set  of  points is \emph{universal} for a class  
of graphs if every (not necessarily proper) coloring of vertices of  on  vertices 
admits an embedding on . Brandes et al.~\cite{brandes10+} find, for example, universal -colored 
point sets for the class of caterpillars for every .



\subsection{Our Results}
A {\em convex\/} or a {\em concave chain\/} is a finite set of points in the plane
lying on the graph of a strictly convex or a strictly concave function, respectively.
A {\em double-chain}  consists of a convex chain  and a concave chain
 such that each point of  lies strictly below every line determined
by  and similarly, each point of  lies strictly above every line determined
by  (see Fig.~\ref{f:double-chain}).
Double-chains were first considered in \cite{garcia}. 

The size of a chain  is the number  of its points.
Note that we allow different sizes of the chains  and . If the sizes , 
of the chains differ by at most , we say that the double-chain is \emph{balanced}.


We consider only -colorings and we use {\em black\/} and {\em white\/} as the colors.
A -coloring of a set  of  points in the plane is \emph{compatible} with a -coloring of a graph  
on  vertices if the number of black points of  is the same as the number of black vertices in . 
This implies the equality of numbers of white points and white vertices as well. 

A graph  with -colored vertices is \emph{embeddable} on a -colored point set  if the vertices 
of  can be mapped to the points of  so that the colors match and no two edges cross if they are drawn as 
straight-line segments.

A set  of points is \emph{-color universal} for a bipartite graph  if for every proper -coloring
of  and for every compatible -coloring of ,  is embeddable on .

If the properly colored path on  vertices, , can be embedded on a -colored set  of  points, 
then we say that  has an NHAP (non-crossing Hamiltonian alternating path).

A -coloring of a set  of  points is \emph{equitable} if it is compatible with some 
proper -coloring of , that is, if the sizes of the two color classes differ by at most .

\myfig{double-chain}{an equitable -coloring of a double-chain }

Section~\ref{s:mainthm} contains the proof of the following theorem.

\begin{theorem}\label{t:main}
Let  be a double-chain satisfying  for .
Then  is -color universal for , where . Moreover, an NHAP on an equitably 
colored  can be found in linear time.
\end{theorem}

Note that this doesn't always hold if we don't require the coloring of the path to be proper. For example, 
if we color first two vertices of  black and the other two white, then it cannot be embedded on the 
double-chain with both chains of size  if the two black points are the top-left one and the bottom-right one.

In Section~\ref{s:nopath}, we show that double-chains with highly unbalanced sizes of chains
do not admit an NHAP for some equitable -colorings.

\begin{theorem} \label{t:nopath}
Let  be a double-chain, and let  be periodic with the following period of length 16:
2 black, 4 white, 6 black and 4 white points.
If , then  has no NHAP.
\end{theorem}

An \emph{equitable coloring} of a graph is a coloring where the sizes of any two color classes differ by at most .
A bipartite graph is \emph{equitable} if it admits a proper equitable -coloring.

Section~\ref{sec:emb} mainly studies other graphs for which the balanced double-chain is -color universal.

\begin{theorem}
\label{thm:eqonbal}
The balanced double-chain is -color universal for equitable caterpillars with at least as many leaves 
as non-leaf vertices. 
\par If a forest of stars is -colored equitably and properly, then it can be embedded on every 
compatibly -colored balanced double-chain.
\par If the balanced double chain is -color universal for an equitable bipartite graph , then
 is a forest of caterpillars.
\end{theorem}

We also present examples of equitable bipartite planar graphs for which no set of points is -color universal.






\section{Proof of Theorem~\ref{t:main}}\label{s:mainthm}

The main idea of our proof is to cover the chains  by a special type of
pairwise non-crossing paths, so called
hedgehogs, and then to connect these hedgehogs into an NHAP by adding some
edges between  and .

\subsection{Notation Used in the Proof}

For , let  be the number of black points of  and
let  denote the number of white points of
. 

Since the coloring is equitable, we may assume that 
 and .
Then black is {\em the major color of}  and
{\em the minor color of} , and white is {\em the major color
of}  and {\em the minor color of} . Points in the major
color, i.e., black points on  and white points on , are
called {\em major points}. Points in the minor color are called
{\em minor points}.

Points on each  are linearly ordered according to the -coordinate.
An {\em interval\/} of  is a sequence of consecutive points of .
An {\em inner point\/} of an interval  is any point of  which is
neither the leftmost nor the rightmost point of .

\myfig{hedgehog}{a hedgehog in }

A {\em body}  is a non-empty interval of a chain  
such that all inner points of  are major.
If the leftmost point of  is minor, then we call it a {\em head\/}
of . Otherwise  has no head. If the rightmost point of 
is minor, then we call it a {\em tail\/} of . Otherwise  has no tail.
If a body consists of just one minor point, this point is both the head
and the tail.

Bodies are of the following four types. A {\em -body\/} is a body with
no head and no tail. A {\em -body\/} is a body with both head
and tail. The bodies of remaining two types have exactly one endpoint major
and the other one minor. We will call the body a {\em -body\/} or
a {\em -body\/} if the minor endpoint is a head or a tail, respectively.

Let  be a body on . A {\em hedgehog (built on the body
)\/} is a non-crossing alternating path 
with vertices in  satisfying
the following three conditions: (1)  contains all points of , (2) 
contains no major points outside of , (3) the endpoints of  are
the first and the last point of .
A hedgehog built on an -body is an {\em -hedgehog\/}
(). If a hedgehog  is built on a body , then  is
{\em the body of \/} and the points of  that do not lie in  are
{\em spines}. Note that each spine is a minor point. All possible types of 
hedgehogs can be seen on Fig.~\ref{f:htypes} (for better lucidity, we will draw
hedgehogs with bodies on a horizontal line and spines indicated only by a
``peak'' from now on).

\myfig{htypes}{types of hedgehogs (sketch)}

On each , maximal intervals containing only major points are called {\em runs}.
Clearly, runs form a partition of major points.
For , let  denote the number of runs in .

\subsection{Proof in the Even Case}

Throughout this subsection,  denotes a double-chain with
 even. Since the coloring is equitable, we have .
Set 

First we give a lemma characterizing collections of bodies
on a chain  that are bodies of some pairwise non-crossing
hedgehogs covering the whole chain .

\begin{lemma}\label{l:hedgehogs}
Let . Let all major points of  be covered by a set
 of pairwise disjoint bodies. Then the bodies of  are the bodies
of some pairwise non-crossing hedgehogs covering the whole  if and only if
, where  is the number
of -bodies in .
\end{lemma}

\begin{proof}
An -hedgehog containing  major points contains
 minor points. It follows that the equality
 is necessary for the existence of a covering
of  by disjoint hedgehogs built on the bodies of .

Suppose now that . Let  be the set of minor points
on  that lie in no body of , and let  be the set of
the mid-points of
straight-line segments connecting pairs of consecutive major points
lying in the same body. It is easily checked that .
Clearly  is a convex or a concave chain. Now it is easy to prove
that there is a non-crossing perfect matching formed by  straight-line
segments between  and  (for the proof, take any segment connecting
a point of  with a neighboring point of , remove the two points,
and continue by induction); see Fig.~\ref{f:spines}.

\myfig{spines}{a non-crossing matching of minor points and midpoints (in }

If  is connected to a point  in the matching,
then  will be a spine with edges going from it to those two major
points that determined . Obviously, these spines and edges define
non-crossing hedgehogs with bodies in  and with all the required properties.
\end{proof}

\bigskip

The following three lemmas and their proofs show how to construct an NHAP
in some special cases.

\begin{lemma}\label{l:deltalarge}
If  then  has an NHAP.
\end{lemma}

\begin{proof}
Let .
Since , the runs in  may be partitioned
into  -bodies. By Lemma~\ref{l:hedgehogs}, these -bodies
may be extended to pairwise non-crossing hedgehogs covering .
This gives us  hedgehogs on the double-chain.
They may be connected into an NHAP by  edges between the chains
in the way shown in Fig.~\ref{f:lemma2}.
\end{proof}

\myfig{lemma2}{-hedgehogs connected to an NHAP}

\begin{lemma}\label{l:equalruns}
If  then  has an NHAP.
\end{lemma}

\begin{proof}
Set .
If  then we may apply Lemma~\ref{l:deltalarge}.
Thus, let .

Suppose first that . We cover each run on each  by a single body
whose type is as follows. On  we take  -bodies
followed by  -bodies. On  we take (from left to right)
 -bodies,  -bodies, and one -body.
By Lemma~\ref{l:hedgehogs}, the  bodies on each  can be extended
to hedgehogs covering . Altogether we obtain  hedgehogs. They can be connected
to an NHAP by  edges between  and  (see Fig.~\ref{f:lemma3}).

\myfig{lemma3}{an NHAP in the case }

Suppose now that . We add one auxiliary major point on each 
as follows. On , the auxiliary point extends the leftmost run on the left.
On , the auxiliary point extends the rightmost run on the right.
This does not change the number of runs and increases  to .
Thus, we may proceed as above. The NHAP obtained has the two auxiliary points
on its ends. We may remove the auxiliary points from the path, obtaining
an NHAP for .
\end{proof}
\bigskip

A {\em singleton}  is an inner point of  () such that
its two neighbors on  are colored differently from .

\begin{lemma}\label{l:nosingletons}
Suppose that  has no singletons and  can be covered by 
pairwise disjoint hedgehogs.
Then  has an NHAP.
\end{lemma}

\begin{proof}
For simplicity of notation, set .
We denote the  hedgehogs on  by 
in the left-to-right order in which the bodies of these hedgehogs appear
on . For technical reasons, we enlarge
the leftmost run of  from the left by an auxiliary major point .

Our goal is to find  hedgehogs 
on  such that they may be connected with the hedgehogs
 into an NHAP. For each , the body 
of the hedgehog  will be denoted by . For each ,
 covers the -th run of  (in the left-to-right order).
We now finish the definition of the bodies  by specifying
for each  if it has a head and/or a tail.
The body  is without head. For ,  has a head if and
only if  has a tail. The last body  is without tail and 
, has a tail if and only if  has a head.

\myfig{lemma4}{an NHAP in the case of no singletons on }

It follows from Lemma~\ref{l:hedgehogs} that we
may add or remove some minor points on 
so that  can then be extended to pairwise
non-crossing hedgehogs  covering the ``new'' .
More precisely, there is a double-chain  such that
 can be extended to pairwise
non-crossing hedgehogs  covering , where
either  or  is obtained from 
by adding some minor (white) points on the left of  (say)
or  is obtained from  by removal of some minor (white)
points lying in none of the bodies .
Then the concatenation 
shown in Fig.~\ref{f:lemma4} gives an NHAP on .
This NHAP starts with the point . Removal
of  from it gives an NHAP 
for the double-chain . The endpoints of 
have different colors. Thus,  covers the same number
of black and white points. Black points on  are the 
black points of . Thus,  covers exactly 
points. It follows that  and thus
.
The path  is an NHAP on the double-chain .
\end{proof}

\bigskip

The following lemma will be used to find a covering needed in 
Lemma~\ref{l:nosingletons}.

\begin{lemma}\label{l:paths}
Suppose that ,  and  for some  and for some integer . Then  can be covered by  pairwise
disjoint hedgehogs.
\end{lemma}

\begin{proof}
The idea of the proof is to start with the set  of 
bodies, each of them being a single point, and then gradually decrease the 
number of bodies in  by joining some of the bodies together.
We see that , where  is the number of
-bodies in . If we join two neighboring -bodies to one
-body and withdraw a single-point -body from  (to let the
minor point become a spine) at the same time, the difference between the number
of -bodies and the number of -bodies remains the same and
 decreases by two. We can reduce  by one
while preserving the difference 
by joining a -body with a neighboring single-point
-body into a - or a -body. Similarly we can join a - or
a -body with a neighboring (from the proper side) single-point -body
into a new -body to decrease  by one as well. When we are
joining two -bodies, we choose the single-point -body to remove in such
a way to keep as many single-point -bodies adjacent to -bodies as
possible. This guarantees that we can use up to  of them for heads and
tails.

We start with joining neighboring -bodies and we do this as long as
 and . Note that by the assumption 
, we will have enough single-point -bodies to do that.
When we end, one of the following conditions holds: ,
 or . In the first case we are done. If
, we just add one head or one tail (we can do this since
, which implies
). If , then each run is covered by just one -body.
We need to add  heads and tails. We have enough
single-point -bodies to do that since 
.
On the other hand, , so the number of heads and
tails needed is at most . Therefore, all the single-point -bodies are
adjacent to -bodies and we can use them to form heads and tails.

In all cases we get a set  of  bodies. Now we can apply
Lemma~\ref{l:hedgehogs} to obtain  pairwise disjoint hedgehogs covering
.
\end{proof}

\bigskip

By a {\em contraction\/} we mean removing a singleton with both its
neighbors and putting a point of the color of its neighbors in its place
instead. It is easy to verify that if there is an NHAP in the new
double-chain obtained by this contraction, it can be expanded to an NHAP
in the original double-chain.

Now we can prove our main theorem in the even case.

\bigskip

\begin{pfoftmain}
Without loss of generality we may assume that . In the case
, we get an NHAP by Lemma~\ref{l:equalruns}. In the case
, we get an NHAP by Lemma~\ref{l:deltalarge}. Therefore,
the only case left is , .

If there is a singleton on , we make a contraction of it. By this we
decrease  by one and both  and  remain unchanged. If
now  or , we again get an NHAP, otherwise
we keep making contractions until one of the previous cases appears or there
are no more singletons to contract.

If there is no more singleton to contract on  and still  and
, we try to cover  by  pairwise disjoint paths.
Before the contractions,  did hold and by the
contractions we could just decrease , therefore it still holds.

All the maximal intervals on the chain  (with possible exception of the
first and the last one) have now length at least two, which implies that
.
Hence , so we can create 
pairwise disjoint hedgehogs covering  using Lemma~\ref{l:paths}.
Then we apply Lemma~\ref{l:nosingletons} and expand the NHAP obtained
by Lemma~\ref{l:nosingletons} to an NHAP on the original double-chain.

There is a straightforward linear-time algorithm for finding an NHAP
on  based on the above proof.
\qed
\end{pfoftmain}




\subsection{Proof in the Odd Case}\label{s:odd}
In this subsection we prove Theorem \ref{t:main} for the case when 
is odd. We set  and proceed similarly as in the even case.
On several places in the proof we will add one auxiliary point
 to get the even case (its color will be chosen to equalize
the numbers of black and white points). We will be able to apply one of the
Lemmas \ref{l:deltalarge}--\ref{l:nosingletons} to obtain an NHAP.
The point  will be at some end of the NHAP and by removing
 we obtain an NHAP for .

Without loss of generality we may assume that . In the case
, we add an auxiliary major point , which is placed either as
the left neighbor of the leftmost major point on
 or as the right neighbor of the rightmost major point on . Then
we get an NHAP by Lemma \ref{l:equalruns} and the removal of  gives
us an NHAP for . 

In the case , we add an auxiliary point  to the same place
and we get an NHAP by Lemma \ref{l:deltalarge}. Again, the removal of
 gives us an NHAP for . 

Now, the only case left is , .
If there are any singletons on , we make the contractions exactly the same
way as in the proof of the even case. If Lemma \ref{l:deltalarge} or
\ref{l:equalruns} needs to be applied, we again add an auxiliary point
 and proceed as above.

If there is no more singleton to contract on  and still  and
, we have  as in the proof of
the even case and we can use Lemma
\ref{l:paths} to get  pairwise disjoint hedgehogs covering . Now we need to consider two cases: (1) If , then we find
an NHAP for  in the same way as in the proof of Lemma
\ref{l:nosingletons}, except we do not add the auxiliary point .
(2) If , we add an auxiliary point  as the
right neighbor of the rightmost major point on . The number  didn't
change so Lemma \ref{l:nosingletons} gives us an NHAP. Again, the removal
of  gives us an NHAP for .

There is a straightforward linear-time algorithm for finding an NHAP
on  based on the above proof.
\qed


\section{Unbalanced Double-Chains with no NHAP}\label{s:nopath}
In this section we prove Theorem~\ref{t:nopath}.
Let  be a double-chain whose points are colored by an equitable 2-coloring, and let  be periodic with the following period: 2 black, 4 white, 6 black and 4 white points. 
Let . We want to show that  has no NHAP.

Suppose on the contrary that  has an NHAP.
Let  denote the maximal subpaths of the NHAP containing only points of . Since between every two consecutive paths ,  in the NHAP there is at least one point of , we have . In the following we think of  as of a cyclic sequence of points on the circle. Note that we get more intervals in this way. Theorem~\ref{t:nopath} now directly follows from the following theorem.


\begin{theorem}\label{t:circle_covering}
Let  be a set of points on a circle periodically -colored with the following period of length :
 black,  white,  black and  white points.
Suppose that all points of  are covered by a set of  non-crossing alternating and pairwise disjoint paths . Then .
\end{theorem}

\begin{proof}
Each maximal interval spanned by a path  on the circle is called a {\em base}. Let  denote the number of bases of . A path with one base only is called a {\em leaf}. We consider the following special types of edges in the paths. {\em Long edges\/} connect points that belong to different bases. {\em Short edges\/} connect consecutive points on . Note that short edges cannot be adjacent to each other. A maximal subpath of a path  spanning two subintervals of two different bases and consisting of long edges only is called a {\em zig-zag}. A path is {\em separated\/} if all of its edges can be crossed by a line. Note that each zig-zag is a separated path. A maximal separated subpath of  that contains an endpoint of  and spans one interval only is a {\em rainbow\/}. We find all the zig-zags and rainbows in each , . Note that two zig-zags, or a zig-zag and a rainbow, are either disjoint or share an endpoint.
A {\em branch\/} is a maximal subpath of  that spans two intervals and is induced by a union of zig-zags. 
For each path  that is not a leaf construct the following graph . The vertices of  are the bases of . We add an edge between two vertices for each branch that connects the corresponding bases.
If  has a cycle (including the case of a ``-cycle''), then one of the corresponding branches consists of a single edge that lies on the convex hull of . We delete such an edge from  and don't call it a branch anymore. By deleting a corresponding edge from each cycle of  we obtain a graph , which is a spanning tree of . The {\em branch graph}  is a union of all graphs .
 
Let  denote the set of leaves and  the set of branches. Let  =  .

\begin{obs}\label{o:branches}
The branch graph  is a forest with components . Therefore,  \qed
\end{obs}

The branches and rainbows in  do not necessarily cover all the points of . Each point that is not covered is adjacent to a deleted long edge and to a short edge that connects this point to a branch or a rainbow. It follows that between two consecutive branches (and between a rainbow and the nearest branch) there are at most two uncovered points, that are endpoints of a common deleted edge. By an easy case analysis it can be shown that this upper bound can be achieved only if one of the nearest branches consists of a single zig-zag.

In the rest of the paper, a {\em run\/} will be a maximal monochromatic interval of any color.
In the following we will count the runs that are spanned by the paths . The {\em weight\/} of a path , , is the number of runs spanned by . If  spans a whole run, it adds one unit to . If  partially spans a run, it adds half a unit to .

\begin{obs}\label{o:mbranch}
The weight of a zig-zag or a rainbow is at most . A branch consists of at most two zig-zags, hence it weights at most three units. \qed
\end{obs}

 
\begin{lemma} \label{l:path}
A path  that is not a leaf weights at most  units where  is the number of branches in .
\end{lemma}

\begin{proof}
According to the above discussion, for each pair of uncovered points that are adjacent on  we can join one of them to the adjacent branch consisting of a single zig-zag. To each such branch we join at most two uncovered points, hence its weight increases by at most one unit to at most  units. The number of the remaining uncovered points is at most . Therefore, .
\end{proof}


\begin{lemma}\label{l:leaf}
A leaf weights at most  units.
\end{lemma}



\begin{proof}
Let  be a leaf spanning at least two points. Consider the interval spanned by . Cut this interval out of  and glue its endpoints together to form a circle. Take a line  that crosses the first and the last edge of . Note that the line  doesn't separate any of the runs. Exactly one of the arcs determined by  contains the gluing point . 

Each of the ending edges of  belongs to a rainbow, all of whose edges cross . It follows that if  has only one rainbow, then this rainbow covers the whole leaf  and .
Otherwise  has exactly two rainbows,  and . We show that  and  cover all edges of  that cross the line . Suppose there is an edge  in  that crosses  and does not belong to any of the rainbows , . Then one of these rainbows, say , is separated from  by . Then the edge of  that is the second nearest to  also has the same property as the edge . This would imply that  spans two whole runs, a contradiction. It follows that all the edges of  that are not covered by the rainbows are consecutive and connect adjacent points on the circle. There are at most three such edges; at most one connecting the points adjacent to , the rest of them being short on . But this upper bound of three cannot be achieved since it would force both rainbows to span two whole runs. Therefore, there are at most two edges and hence at most one point in  uncovered by the rainbows. The lemma follows.
\end{proof}


\begin{lemma}\label{l:leaves}
.
\end{lemma}

\begin{proof}
The number of runs in  is at least . By Lemma~\ref{l:leaf}, if all the paths  are leaves, then at least  of them are needed to cover  and the lemma follows. 

If not all the paths are leaves, we order the paths so that all the leaves come at the end of the ordering. 
The path  spans  bases. Shrink these bases to points. These points divide the circle into  arcs each of which contains at least one leaf. If  is not a leaf then continue. The path  spans  intervals on one of the previous arcs. Shrink them to points. These points divide the arc into  subarcs. At least  of them contain leaves. This increased the number of leaves by at least . The case of , , is similar to . The lemma follows by induction.
\end{proof}





\begin{corollary} 

\end{corollary}

\begin{proof}
Combining Lemma~\ref{l:leaves} and Observation~\ref{o:branches} we get the following:

\end{proof}
\bigskip

Now we are in position to finish the proof of Theorem~\ref{t:circle_covering}.
If the whole  is covered by the paths , then . Therefore,


\end{proof}






\section{Embedding equitable bipartite graphs}
\label{sec:emb}
\subsection{Embedding on balanced double-chains}
We already know that the balanced double-chain is -color universal for the path . 
In this subsection, we further study the class of graphs for which the balanced double-chain 
is -color universal. The three lemmas of this subsection prove the three claims of 
Theorem~\ref{thm:eqonbal}.

\begin{lemma}
\label{lem:forcat}
If the balanced double-chain is -color universal for an equitable bipartite graph  
then  is a forest of caterpillars.
\end{lemma}
\begin{proof}
Let  be the -star with subdivided edges (see Fig.~\ref{f:k13}).
A connected graph is a caterpillar if and only if it contains no cycle and no  as a subgraph.

We will color all points of one chain white and points of the other chain black so that the resulting 
coloring is compatible with the -coloring of . We assume for contradiction that  can be 
embedded on it and that it contains either a cycle or .

As the double-chain has monochromatic chains, all the edges connect the two chains.
Because the embedding has no edge crossings, we can consider the leftmost edge of the cycle 
and let  and  be its endvertices. Then the two edges of the cycle incident to exactly one of 
 and  cross.


We now assume that  can be embedded on the double-chain and let the color of its root vertex 
be white. Let  be 
the white point where the root of  is mapped and let , ,  be 
(from left to right) the three black points where the middle vertices are mapped. Then  
is connected by an edge to some white leaf vertex of , but this edge is crossed either by the 
segment  or by . See Fig.~\ref{f:k13}.
\myfig{k13}{a)  and b) impossibility of its embedding on the double chain with monochromatic chains}
\end{proof}

The \emph{central path} in a caterpillar is the set of non-leaf vertices. 

\begin{lemma}
\label{lem:shortcat}
If an equitable bipartite graph  on  vertices is a caterpillar with at most  
vertices on the central path, then the balanced double-chain is -color universal for .
\end{lemma}
\begin{proof}
Let  be the number of black points on the chain  and  the number of white points on .
Since the coloring is equitable, we may assume that  and .
Then black is {\em the major color} of  and {\em the minor color} of , and white is 
{\em the major color} of  and {\em the minor color} of . Points in the major
color are called {\em major points}. Points in the minor color are called {\em minor points}.


Observe that the number of minor points on each chain is at most the number of leaves of  of that color. 
Let  be the graph with  black and  white vertices obtained from  by removing some 
 black and  white leaves. 

In the first phase, we embed  on the set of major points of 
the two chains. We take the vertices of the central path of  starting from one of its ends. 
A vertex  of the central path is placed on the leftmost unused major point on the chain 
where the color of  is major. The leaves of  in  are then successively placed on the leftmost unused 
major points of the other chain.

In the second phase, we map all the leaves removed in the first phase on minor points.
In the same greedy way as in the proof of Lemma~\ref{l:hedgehogs}, we keep connecting the closest pair
of an unused white point of  and a black point of the central path that still misses at least 
one leaf. The same is done on .

This guarantees that no crossing appears and that every vertex is mapped to some point. See 
Fig.~\ref{f:caterpillar}.
\myfig{caterpillar}{embedding a caterpillar on a balanced double-chain; the bold edges form the central path}
\end{proof}


\begin{lemma}
If a forest of stars  is -colored equitably and properly, then  can be embedded on every 
compatibly -colored balanced double-chain.
\end{lemma}
\begin{proof}
We take some fixed proper equitable -coloring of . 

We show that by adding edges to , we are able to create a properly -colored
caterpillar on the set of all vertices of  and with at most  non-leaf vertices.
By Lemma~\ref{lem:shortcat} this caterpillar can be embedded on every compatibly -colored balanced 
double-chain and thus  can be embedded.

The cases when  and when  has no edges are trivial.

For every , let  () be the number of stars on  vertices and with black (white) central vertex.
In case of -vertex components of , we cannot distinguish the 
central vertex and we let  be their number. We also let  be the total number of -vertex 
components of  as it is not necessary for the proof to count black and white ones separately.

We assume without loss of generality, that at least half of the stars on at least three vertices have black central vertex.
We start connecting the central vertices of stars on at least three vertices into an alternating path starting 
with a black vertex. At some point we run out of stars with white central vertex. We then use the stars on 
two vertices as stars with 
white central vertex. If we run out of stars with black central vertex, we use every second star on two vertices as 
a star with black central vertex. Otherwise we run out of stars on two vertices. Then we start connecting 
each of the remaining stars on at least three vertices by an edge between one of its white leaves and the last 
black vertex on the path.

The resulting graph is composed of a connected graph  and all -vertex components of . The graph  is 
a properly colored caterpillar and the created path is its central path . 
Since  has some edges,  is not empty.

If  contains only one vertex , we pick one of its leaves, , and connect every -vertex component of 
either to  or to , depending on its color. The central path then has  vertices, which is at most 
.

If  has at least two vertices, we connect every -vertex component of  to a vertex of 
the other color on the central path. 

See Fig.~\ref{f:starstocat}.

\myfig{starstocat}{connecting stars to form a caterpillar}

It remains to show that the central path is not too long. The total number of vertices is


If , then every vertex of the central path has at least one leaf
and thus the caterpillar has at most  non-leaf vertices. 

Otherwise, the central path starts and ends with a black vertex and the black vertices of the central path are 
exactly the black centers of stars on at least three vertices. The central path thus has  
vertices.

Because the -coloring of  is equitable, the number of black vertices of  is at least 
 and thus


At most  vertices of  are white, which leads to


Putting the two inequalities together gives us


The number of vertices of the central path is at most , because


\end{proof}




\subsection{Open problems}

It seems plausible that the balanced double-chain is -color universal for all equitable forests
of caterpillars.

\begin{conjecture}
The balanced double-chain is -color universal for an equitable bipartite graph  if and only if
 is a forest of caterpillars.
\end{conjecture}

Some graphs for which the balanced double-chain is not -color universal have a different -color 
universal set. For example, the balanced double-chain is not -color universal for  by 
Lemma~\ref{lem:forcat}, but it is easy to verify that the double-chain with one chain composed of only 
one vertex is.

There even exist graphs with a -color universal set of points, but no double chain is -color universal for 
them. Consider the properly colored  with black central vertex. It is not embeddable on double-chains 
colored as in Fig.~\ref{f:k14-doublechain}. But a modification of the double chain in Fig.~\ref{f:glued-double-chain} 
is -color universal for .

\myfig{k14-doublechain}{colorings of double-chains not admitting }

\myfig{glued-double-chain}{a -color universal point set for }

Some equitable bipartite planar graphs have no -color universal set of points.

\begin{claim}
If  is a bipartite planar quadrangulation on at least five vertices, then  has no -color universal 
set of points.
\end{claim}
\begin{proof}
Because the bipartite graph  has no -cycle, each of its faces has at least four vertices. 
Then, by Euler's formula, every planar drawing of  is a quadrangulation.

Take a set  of points and let  be the set of points of  on its convex hull. In a straight-line 
planar drawing of a graph on a set  of points, the points of  lie all on the outer face of the drawing. 
Thus  can only be drawn on  if . In a proper coloring of  on at least five vertices,
one color class contains at least three vertices. If we color three points of  by this color and the rest
arbitrarily,  cannot be embedded, because no face in a drawing of  can contain three vertices of one color. 
\end{proof}

The results of this paper solve only a few particular cases of the following general question.
\begin{question}
Which planar bipartite graphs have a -color universal set of points?
\end{question}


\section*{Acknowledgment} We thank Jakub \v{C}ern\'y for his active participation at
the earlier stages of our discussions.


\begin{thebibliography}{ABC}

\bibitem{abellanas99}
M. Abellanas, J. Garc{\'\i}a, G. Hernandez, M. Noy, P. Ramos, 
Bipartite embeddings of trees in the plane, 
Discrete Appl. Math. 93 (1999), 141--148.

\bibitem{abellanas03}
M. Abellanas, J. Garc{\'\i}a, F. Hurtado, J. Tejel, Caminos Alternantes (in Spanish),
Proc. X Encuentros de Geometr{\'\i}a Computacional, Sevilla, June 2003, pp.\ 7--12. English version
available on Ferran Hurtado's web page \url{http://www-ma2.upc.es/~hurtado/mypapers.html}.

\bibitem{aichholzer10}
O. Aichholzer, S. Cabello, R. Fabila-Monroy, D. Flores-Pe\~{n}aloza, T. Hackl, C. Huemer, F. Hurtado, D.R. Wood,
Edge-removal and non-crossing configurations in geometric graphs,
Discrete Math. Theor. Comput. Sci. 12 (2010), no. 1, 75--86.

\bibitem{brandes10+}
U. Brandes, C. Erten, A. Estrella-Balderrama, J. Fowler, F. Frati, M. Geyer, C. Gutwenger, Seok-Hee Hong, M. Kaufmann,
S. Kobourov, G. Liotta, P. Mutzel, A. Symvonis,
Colored Simultaneous Geometric Embeddings and Universal Pointsets,
Algorithmica, to appear, available online at \url{http://dx.doi.org/10.1007/s00453-010-9433-x}.


\bibitem{brassmoserpach}
P. Brass, W. Moser, J. Pach, Research Problems in Discrete Geometry,
Springer, New York, 2005.

\bibitem{cerny07}
J. \v{C}ern\'y, Z. Dvo\v{r}\'ak, V. Jel\'inek, J. K\'ara,
Noncrossing Hamiltonian paths in geometric graphs,
Discrete Appl. Math. 155 (2007), no. 9, 1096--1105.

\bibitem{gdversion}
J. Cibulka, J. Kyn\v{c}l, V. M\'{e}sz\'{a}ros, R. Stola\v{r}, P. Valtr,  
Hamiltonian alternating paths on bicolored double-chains,
in: I. G. Tollis, M. Patrignani (Eds.), Graph Drawing 2008, Lecture Notes in Computer Science 5417,
Springer, New York, 2009, pp.\ 181--192.

\bibitem{garcia}
A. Garc{\'\i}a, M. Noy, J. Tejel,
Lower bounds on the number of crossing-free subgraphs of ,
Comput. Geom. 16 (2000), 211--221.

\bibitem{giacomograph}
E. Di Giacomo, G. Liotta, F. Trotta,
On embedding a graph on two sets of points,
Int. J. of Foundations of Comp. Science 17(2006), no. 5, 1071--1094.

\bibitem{giacomopath}
E. Di Giacomo, G. Liotta, F. Trotta,
How to embed a path onto two sets of points,
in: P. Healy, N. Nikolov (Eds.), Graph Drawing 2005, Lecture Notes in Computer Science 3843,
Springer, New York, 2006, pp.\ 111--116.

\bibitem{gritzmann91}
P. Gritzmann, B. Mohar, J. Pach, R. Pollack,
Embedding a planar triangulation with vertices at specified points,
in: Am. Math. Monthly 98 (1991), 165--166 (Solution to problem E3341)

\bibitem{hajnalmeszaros}
P. Hajnal, V. M\'esz\'aros, 
Note on noncrossing path in colored convex sets, accepted to Discrete Math. Theor. Comput. Sci..

\bibitem{kanekokano}
A. Kaneko, M. Kano,
Discrete geometry on red and blue points in the plane --- a survey,
in: B. Aronov et al. (Eds.), Discrete and computational geometry,
The Goodman-Pollack Festschrift, Springer,
Algorithms Comb. 25 (2003), 551--570.

\bibitem{kanekokanosuzuki}
A. Kaneko, M. Kano, K. Suzuki, Path coverings of two sets of points in the plane,
in: J. Pach (Ed.), Towards a Theory of Geometric Graphs, Contemporary Mathematics 342
(2004), 99--111.

\bibitem{kaufmannwiese}
M. Kaufmann, R. Wiese,
Embedding vertices at points: Few bends suffice for planar graphs,
Journal of Graph Algorithms and Applications 6 (2002), no. 1, 115--129.

\bibitem{kynclpt}
J. Kyn\v cl, J. Pach, G. T\'oth, Long alternating paths in bicolored point sets, 
Discrete Mathematics 308 (2008), no. 19, 4315--4321.

\bibitem{pachwenger}
J. Pach, R. Wenger, 
Embedding planar graphs at fixed vertex locations, 
Graphs and Combinatorics 17 (2001), no. 4, 717--728.


\end{thebibliography}

\end{document}
