Forming and maintaining healthy eating habits is important to individuals' long-term physical well-being. However, despite the availability of numerous dietary guidelines, few people are able to do so as demonstrated by the prevalence of chronic diseases such as obesity and type-2 diabetes. Arguably, part of the reason for such failure is that these guidelines are one-size-fits-all suggestions, making them difficult to be adopted habitually by individuals. In contrast to dietary guidelines, suggestions tailored to specific individuals from recommender systems may be more effective at facilitating incremental behavior change. More specifically, by learning about users' dietary behavior through data from mobile food consumption tracking apps such as MyFitnessPal (MFP), the systems can nudge the users towards ``similar but healthier'' alternatives by recommending food substitutes personalized to the users' current dietary needs and preferences. In this work, we explored a data-driven approach to extracting food substitutes from personal food consumption data as the first step into the healthier food recommendation. Thanks to the rise of self-monitoring practices enabled by mobile and wearable technology, we turned to a wealth of public food consumption data created by MFP users. 


\begin{figure}[t]
\centering
\includegraphics[width=\columnwidth]{MFP_diary}
\caption{Screenshot of a food diary on MFP.}
\label{fig:mfp_screenshot}
\end{figure}


MyFitnessPal (MFP) is a popular mobile app and website for fitness and health with 80 million registered users\footnote{\url{https://en.wikipedia.org/wiki/MyFitnessPal}}. One of its core features is an online food diary which helps users track their food consumption to achieve specific health goals, such as losing, gaining, or maintaining weight. Each food diary page consists of a sequence of meals where each meal contains a collection of food entries and nutrition information. As we can see from the sample diary page shown in Figure~\ref{fig:mfp_screenshot}, the user has logged 9 food entries across 2 meals (named ``One'' and ``Four''). When logging a new entry into a diary, users can either enter a new food entry and nutritional values or search for existing food entries shared by other users in the food database. Users can control the  diary sharing setting such that their diaries can be viewed by anyone (``Public''), their friends (``Friends Only''), or only the users themselves (``Private'').

Our main assumption of the substitution relationship between foods is inspired by the distributional hypothesis in linguistics: words that occur in the same contexts tend to convey similar meanings. By applying the same notion to food consumption, we hypothesize that foods consumed in similar contexts are more likely to be a substitute of each other. More specifically, our method is based on the vector space models of semantics~\cite{Turney2010} commonly used in related natural language processing (NLP) tasks, such as word similarity and analogy recovery.

Past studies have investigated the applications of recommender systems in food and cooking domains. One of the most common tasks explored by researchers is cooking recipe recommendation~\cite{Freyne:2010:IFP:1719970.1720021, Harvey2013, Ge2015} where popular recommendation algorithms, such as collaborative filtering~\cite{Freyne:2010:IFP:1719970.1720021} and matrix factorization~\cite{Harvey2013, Ge2015} were employed to predict ratings of cooking recipes. Others have focused on extracting substitutable ingredients from recipes using network analysis~\cite{Teng2012} and statistical approaches~\cite{ Boscarino2014}. While many past studies relied on recipe ratings data collected through recipe-sharing websites such as AllRecipes.com, to the best of our knowledge, our work is the first to study the food substitute extraction problem using a real-world self-reported food consumption data. In addition, we identified the substitution relationship directly from the food consumption data instead of relying on external knowledge sources\cite{Teng2012, Boscarino2014}. Lastly, we evaluated the effectiveness of the method on the human-labeled dataset of food substitutes constructed through an online crowdsourcing service.

The rest of the paper is organized as followed. First, we describe our method in Section~\ref{sec:method}. In Section~\ref{sec:data}, we describe the procedures to collect and process data. Then, we present the experimental evaluation in Section~\ref{sec:experiments} and discuss the results in Section~\ref{sec:results}. Lastly, we conclude the paper in Section~\ref{sec:conclusion}.









%
