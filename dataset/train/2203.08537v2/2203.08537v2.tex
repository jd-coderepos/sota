\documentclass[10pt,twocolumn,letterpaper]{article}



\usepackage[pagenumbers]{cvpr} \usepackage{graphicx}
\usepackage{amsmath, mathtools}
\usepackage{amssymb}
\usepackage{booktabs}
\usepackage{bm}
\usepackage{bbm}
\usepackage{xcolor}
\usepackage[linesnumbered,ruled,vlined]{algorithm2e}
\usepackage{rotating}
\usepackage{pifont}
\usepackage{cuted}
\usepackage{enumitem}
\usepackage{lipsum}
\usepackage[accsupp]{axessibility}

\usepackage{tabularx}
\newcolumntype{b}{X}
\newcolumntype{s}{>{\hsize=.5\hsize}X}

\definecolor{fgreen}{rgb}{0.13, 0.55, 0.13}

\newcommand{\red}[1]{\textcolor{red}{#1}}
\DeclareMathOperator*{\argmax}{argmax}
\newcommand{\floor}[1]{\lfloor #1 \rfloor}
\newcommand{\lturn}[1]{\begin{turn}{90} #1 \end{turn}}

\usepackage[pagebackref,breaklinks,colorlinks]{hyperref}
\usepackage[capitalize]{cleveref}
\crefname{section}{Sec.}{Secs.}
\Crefname{section}{Section}{Sections}
\Crefname{table}{Table}{Tables}
\crefname{table}{Tab.}{Tabs.}

\def\cvprPaperID{3680}
\def\confName{CVPR}
\def\confYear{2022}

\begin{document}

\title{Scribble-Supervised LiDAR Semantic Segmentation}

\author{
Ozan Unal \qquad Dengxin Dai \qquad Luc Van Gool \\
ETH Zurich, MPI for Informatics, KU Leuven \\
{\tt\small \{ozan.unal, dai, vangool\}@vision.ee.ethz.ch}
}
\maketitle

\begin{abstract}
Densely annotating LiDAR point clouds remains too expensive and time-consuming to keep up with the ever growing volume of data. While current literature focuses on fully-supervised performance, developing efficient methods that take advantage of realistic weak supervision have yet to be explored. In this paper, we propose using scribbles to annotate LiDAR point clouds and release ScribbleKITTI, the first scribble-annotated dataset for LiDAR semantic segmentation. Furthermore, we present a pipeline to reduce the performance gap that arises when using such weak annotations. Our pipeline comprises of three stand-alone contributions that can be combined with any LiDAR semantic segmentation model to achieve up to  of the fully-supervised performance while using only  labeled points. Our scribble annotations and code are available at github.com/ouenal/scribblekitti.
\end{abstract}

\section{Introduction}

With the increase of LiDAR's popularity on autonomous vehicles, data acquisition has significantly ramped up. However, it is very hard to keep pace with the volume of data, as the dense data annotation process is very expensive and time-consuming for large scale datasets, especially in 3D where the navigation of the annotation tool is not trivial. Even with powerful annotation tools~\cite{iccv2019semantickitti} that allow labeling of superimposed LiDAR frames, a single 100m by 100m tile can take up to 4.5 hours for an experienced annotator~\cite{iccv2019semantickitti}.

\begin{figure}
    \centering
    \includegraphics[width=\columnwidth]{figures/teaser.pdf}
    \caption{Example of scribble-annotated LiDAR point cloud scenes of a single frame (top) and  superimposed frames (bottom). Compared are  the proposed ScribbleKITTI (left) with  the fully labeled counterpart from SemanticKITTI~\cite{iccv2019semantickitti} (right).}
    \label{fig:teaser}
\end{figure}

In stark contrast to the 2D cases~\cite{cvpr2016scribblesup, iccv2015boxsup, arxiv2015weaklyandsemi, cvpr2018imagelevel}, current efforts in 3D semantic segmentation mainly focus on designing networks for densely annotated data (e.g.~\cite{cvpr2021cylindrical, eccv2020spvnas, wacv2021improving}), as opposed to developing efficient methods for creating more labels or learning from cheap/weak supervision. It is clear that only by doing the latter, the scaling of 3D semantic segmentation can keep up with the growth of applications and data volume.
In this paper, we present a method for this very purpose, by firstly introducing a new annotation strategy and later developing a pipeline to directly exploit such annotations.

Using scribbles as annotations has proven to be a popular and effective method for 2D semantic segmentation~\cite{cvpr2016scribblesup, miccai2020scribble2label, dlmia2018medicalscribble}. The weak annotation method allows annotators to simply mark object centers, avoiding the time consuming task of determining class boundaries. 

We adopt this idea for LiDAR point clouds to supervise 3D semantic segmentation. As opposed to 2D images, 3D point clouds preserve the metric space and therefore \textit{things} and \textit{stuff} follow highly geometric structures. To accompany this, we propose using the more geometric \textit{line}-scribble to annotate LiDAR point clouds. Compared to free-formed scribbles, annotators only need to determine the start and end points of a line annotation. This allows faster labeling of classes that span large distances (e.g. roads, buildings, fences), while also providing as sufficient information for smaller object classes (e.g. cars, trucks), as short lines and free-formed scribbles become less distinguishable.

We provide scribble-annotations for the \textit{train}-split of SemanticKITTI~\cite{iccv2019semantickitti} for 19 classes. The resulting scribble-annotated data, which we call ScribbleKITTI, contains 189 million labeled points corresponding to 8.06\% of the total point count. Fig.~\ref{fig:teaser} shows an example from ScribbleKITTI.

Furthermore, in this paper we develop a novel learning method for 3D semantic segmentation that directly exploits scribble annotated LiDAR data. Learning from scribble annotations provides a unique challenge as no supervision/regularization is available from unlabeled points, which form the majority of the training data. A performance gap between scribble-supervised and fully supervised training could be very large if no special methods are designed for the former. To tackle this issue, we introduce three stand-alone contributions that can be combined with any 3D LiDAR segmentation model: a teacher-student consistency loss on unlabeled points, a self-training scheme designed for outdoor LiDAR scenes, and a novel descriptor that improves pseudo-label quality.

Specifically, we first introduce a weak form of supervision from unlabeled points via a consistency loss. Secondly, we strengthen this supervision by fixing the confident predictions of our model on the unlabeled points and employing self-training with pseudo-labels. The standard self-training strategy is however very prone to confirmation bias due to the long-tailed distribution of classes inherent in autonomous driving scenes and the large variation of point density across different ranges inherent in LiDAR data. To combat these, we develop a class-range-balanced pseudo-labeling strategy to uniformly sample target labels across all classes and ranges. Finally, to improve the quality of our pseudo-labels, we augment the input point cloud by using a novel descriptor that provides each point with the semantic prior about its local surrounding at multiple resolutions.

\noindent In summary, our contributions are as follows:
\setlist{nolistsep}
\begin{itemize}[noitemsep]
    \item We present ScribbleKITTI, the first scribble-annotated LiDAR semantic segmentation dataset.
    \item We propose class-range-balanced self-training to combat the inherent bias towards dominant classes and close ranged dense regions in pseudo-labels.
    \item We further improve the pseudo-labeling quality by augmenting the input point cloud with a pyramid local semantic-context descriptor.
    \item Putting these two contributions along with the mean teacher framework, our scribble-based pipeline achieves up to  relative performance of fully supervised training while using only  labeled points. 
 \end{itemize}
Our contributions remain orthogonal to the development of better neural network architectures and can be combined with any 3D LiDAR segmentation model.

\section{Related Work}

\noindent \textbf{LiDAR Semantic Segmentation}:
As point clouds are irregular geometric data structures, current literature for 3D semantic segmentation mainly focuses on identifying and understanding various representation strategies amongst:
operating directly on point coordinates~\cite{cvpr2017pointnet, arxiv2017pointnet++, cvpr2020randla,iccv2019kpconv,wacv2021improving, eccv2020kprnet}, projecting the LiDAR scene onto images and employ 2D architectures~\cite{iros2019rangenet++,icra2018squeezeseg, icra2019squeezesegv2, eccv2020squeezesegv3, isvs2020salsanext, ral2020mininet}, utilizing sparse 3D voxel grids~\cite{eccv2020spvnas,cvpr2019minkowski,cvpr2021cylindrical,arxiv2020amvnet, arxiv2020sparse}, or utilizing multiple representations~\cite{arxiv2021rpvnet, eccv2020fusionnet,wacv2021multi}.
All of these models are developed under the fully-supervised framework, which requires densely annotated LiDAR point clouds that are time-consuming and tedious to acquire.
In this work, our focus is different and our contributions are complementary. Our developed pipeline can be used with any such network in order to reduce the performance gap between fully-supervised and scribble-supervised training.



\noindent \textbf{2D Scribble-supervised Semantic Segmentation}: 
To alleviate the strenuous task of dense data annotation, two training methods can be used: weakly-supervised~\cite{iccv2015constrained, cvpr2016scribblesup, miccai2020scribble2label, eccv2016seed, eccv2020weakly, nipsw2019scribble}, where only a subset of points are labeled on every frame, and semi-supervised~\cite{ipmi2019semi, dlmia2018deepsemi, iccv2021dars, nc2021semi}, where only a subset of frames are labeled within the dataset. Scribbles have been adopted as a user-friendly form of weak supervision~\cite{cvpr2016scribblesup}. The common approach when dealing with such weak annotations is to either employ online labeling through a consistency check using mean teacher~\cite{nips2017meanteacher,pr2022weakly,media2021weakly,iccv2021click, isbd2019wssmt}, or to employ a self-training scheme where data is iteratively processed by generating offline target pseudo-labels and retraining~\cite{cvpr2016scribblesup, dlmia2018medicalscribble, miccai2020scribble2label, tvcg2020scribble3d}. However, the naive approach of self-training on all predictions can introduce confirmation bias~\cite{ijcnn2020confirmationbias}. To combat this, threshold-based filtering can help reduce possible errors by only sampling confident predictions~\cite{aaai2020curriculum,cvpr2020selftraining, arxiv2020rethinking}. 
When facing long tailed distributions, CB-ST~\cite{eccv2018classbalanced} uses class-balanced sampling to avoid the domination of head classes in the pseudo labels. DARS~\cite{iccv2021dars} extends CB-ST by re-distributing biased pseudo labels after thresholding.
We extend the previously available methods to also include balancing against range to avoid undersampling points from distant, sparser regions of the LiDAR point cloud.


\noindent \textbf{Incomplete Supervision in 3D Semantic Segmentation}: 
In contrast to 2D, incomplete supervision for point clouds have remained underexplored.
When tackling semi-supervised segmentation on LiDAR point clouds, Semi-sup~\cite{iccv2021guided} implements a pseudo-label guided point contrastive loss to extend supervision to unlabeled frames. Li~\etal~\cite{nc2021semi} and SSPC~\cite{arxiv2021sspc} employ self-training to achieve the same goal. Xu~\etal~\cite{cvpr2020towards10x} compares semi-supervised training to weakly-supervised on point clouds and argues that under a fixed labelling budget, weak supervision performs better for semantic segmentation. PSD~\cite{iccv2021selfdistillation} uses consistency check across perturbed branches to utilize unlabeled points in weakly supervised learning. However, the weak labels from existing methods~\cite{cvpr2020towards10x,iccv2021selfdistillation} are generated through offline uniform sampling from dense annotations which cannot be easily adopted during the dense labeling itself.
In this work, we tackle a form of weakly-supervised segmentation based on line-scribbles. Instead of using simulated weak labels, we provide a human annotated dataset to realistically validate our method. Compared to uniform sampled labels, scribbles vitally do not provide any information on class boundaries and appear only in scribble-clusters, i.e. are much less spatially distributed within a scene.

\nocite{tang2018regularized}
\nocite{liu2021unbiased}

\section{The ScribbleKITTI Dataset}

While LiDAR point cloud semantic segmentation has gained popularity over the past years, the number of large-scale datasets still remains low due to the complexity and time consumption of the data annotation process. Inspired by 2D scribble annotations~\cite{cvpr2016scribblesup} that are efficient and easy to generate, we propose using line-scribbles to annotate LiDAR point clouds for semantic segmentation and release ScribbleKITTI, the first scribble-annotated LiDAR point cloud dataset.

We annotate the \textit{train}-split of SemanticKITTI~\cite{iccv2019semantickitti} based on KITTI~\cite{ijrr2013kitti} which consists of 10 sequences, 19130 scans, 2349 million points. ScribbleKITTI contains 189 million labeled points corresponding to only 8.06\% of the total point count. We choose SemanticKITTI for its current wide use and established benchmark. We retain the same 19 classes to encourage easy transitioning towards research into scribble-supervised LiDAR semantic segmentation. The class-wise label distribution is visualized in Fig.~\ref{fig:count}.

When annotating, we use line-scribbles rather than free-forming scribbles. LiDAR point clouds preserve the metric space and therefore \textit{things} (e.g. car, truck) and \textit{stuff} (e.g. terrain, road) mostly follow highly geometric structures. While both drawings are valid approaches, we found that line scribbles allow faster labeling of such geometric classes that span large distances (e.g. roads, sidewalks, buildings, fences), as annotators only need to provide two clicks (start and end) to annotate an entire segment. We illustrate this by showcasing an example annotated tile in Fig.~\ref{fig:annotation_process}.

\begin{figure}[t]
    \centering
    \includegraphics[width=\columnwidth]{figures/count.pdf}
    \caption{Number of points labeled in ScribbleKITTI () visualized against SemanticKITTI () in log-scale.}
    \label{fig:count}
\end{figure}

\noindent \textbf{Data Annotation:} We use the help of student annotators. Following Behley~\etal~\cite{iccv2019semantickitti}, we initially screen the annotators until they are comfortable navigating within the 3D space to ensure good results. We subdivide a sequence of superimposed point clouds into 100m by 100m tiles and label on a per-tile basis. We generate scribble annotations through line drawings using an adapted point labeling tool\footnote{https://github.com/jbehley/point\_labeler, MIT License}~\cite{iccv2019semantickitti}. We overlap neighboring tiles to allow labeling consistency across the entire sequence. Finally, we do a comparison to SemanticKITTI to stay consistent with their class definitions.
We provide further information in the supplementary materials on the labeling process.


\begin{figure}
    \centering
    \includegraphics[width=\linewidth]{figures/annotation-compressed.pdf}
    \caption{Line-annotation process illustrated on a 100m by 100m tile. Classes that span large distances such as building (yellow) and road (pink) can be annotated with only two clicks. As the tile is annotated using 2D lines projected onto the 3D surface, scribbles may become indistinguishable once the viewing angle changes (e.g. bottom right).}
    \label{fig:annotation_process}
\end{figure}

An annotator needs on average 10-25 minutes per tile depending on the contents (e.g. highway vs. city) as opposed to the reported 1.5-4.5 hours for full annotations~\cite{iccv2019semantickitti}. This corresponds to roughly a 90\% time saving, which can account to over a thousands hours for large scale datasets~\cite{iccv2019semantickitti}.

\section{Scribble-Supervised LiDAR Segmentation}

\begin{figure*}
    \centering
    \includegraphics[width=\textwidth]{figures/pipeline.pdf}
    \caption{Illustration of the proposed pipeline for scribble-supervised LiDAR semantic segmentation comprising of three steps: training, pseudo-labeling, distillation. During training, we preform pyramid local semantic-context (PLS) augmentation before training the mean teacher model on the available scribble-annotations. During pseudo-labeling, we generate target labels in a class-range-balanced (CRB) manner. Finally during distillation, we retrain the mean teacher on the generated pseudo-labels.  and  denote the losses applied to the supervised- and unsupervised set of points respectively. Gray arrows propagate label information.}
    \label{fig:pipeline}
\end{figure*}

The naive approach of tackling scribble-supervised semantic segmentation is to treat the problem similarly to any fully supervised task and employ a loss  (typically cross-entropy) on the available labeled points. 

We define a LiDAR point cloud  as the set of points  with  denoting the 3D coordinates and  the reflectance intensity. We further define  as the set of labeled points. The objective function over  frames can therefore be formulated as:

with  denoting the predicted class distribution for the point  of frame  given the network parameters , and  denoting the ground truth label.

In this baseline approach the unlabeled points which contain vital boundary information are not used. Furthermore due to the sheer lack of labeled data points, performance degradation is unavoidable, as confidence on long tailed object classes suffer due to the reduced supervision.

In the following sections, we address these issues by introducing three stand-alone methods that utilize unlabeled points and expand the annotated dataset: partial consistency loss with mean teacher (Sec.~\ref{sec:mt}), class-range-balanced self-training (Sec.~\ref{sec:crb}), and pyramid local semantic-context (Sec.~\ref{sec:pls}). Our overall pipeline can be seen in Fig.~\ref{fig:pipeline}.

\subsection{Partial Consistency Loss with Mean Teacher} \label{sec:mt}

Firstly, we introduce further weak supervision to the unlabeled set of points via a consistency loss applied using mean teacher. The mean teacher framework is formed of two models, namely the student, parametrized by , and the teacher, parametrized by ~\cite{nips2017meanteacher}. Unlike the student network, which is traditionally trained using gradient descent, the teacher weights are computed as the exponential moving average (EMA) of successive student weights, resulting in the update function:

for time step , with  denoting the smoothing coefficient which determines the update speed. Stochastic averaging of weights has been shown to yield more accurate models than using the final training weights directly~\cite{siam1992averaging, nips2017meanteacher}, allowing the teacher predictions to be used as a form of weak supervision for the student under varying small perturbations.

We further define  as the set of unlabeled points, i.e. . We introduce a consistency loss between the student and teacher networks, but unlike Tan~\etal~\cite{isbd2019wssmt}, we restrict the consistency loss to only unlabeled points . This allows a sharper supervision on labeled points in  by eliminating the teacher injected uncertainties, while retaining the unlabeled supervision that takes advantage of the more accurate teacher predictions. This restriction is more in alignment with the applications of the mean teacher framework in semi-supervised tasks~\cite{nips2017meanteacher, arvix2020structured, cvpr2021temporalaction}.

We extend our objective function (Eq.~\ref{eq:partial_h}) to include supervision on unlabeled points as:

with  denoting the predicted class distribution for the point  given the network parameters ,  denoting the ground truth label. To reduce the Shannon mutual information, i.e. to increase the training signal from the consistency loss, we apply a heavier augmentation the student input in the form of global rotation, translation, random flip and white Gaussian noise~\cite{arxiv2020fixmatch, arxiv2019mixmatch, cvpr2021pixmatch}.


While mean teacher introduces supervision on unlabeled points, the information gain is limited by the teachers performance. Even if the teacher predicts the correct label for a point, due to the soft pseudo-labeling, the confidences on other classes will continue to guide the student's output.

\subsection{Class-range-balanced Self-training (CRB-ST)} \label{sec:crb}

To combat this uncertainty injection and more directly utilize the confident predictions of unlabeled points, we expand the annotated dataset and employ self-training. 
Our goal by introducing self-training alongside mean teacher, is to keep the soft pseudo-label guidance of the mean teacher for uncertain predictions while hardening the pseudo-labels of certain predictions. Using the teacher's most confident predictions, we generate target labels for a subset of unlabeled points. We define this set of pseudo-labeled points as  and later retrain our network on .


Formally, we extend our objective function (Eq.~\ref{eq:partial_consistency}) to also learn target labels as hidden variables:

where  is the pseudo-label vector,  denoting a one-shot vector,  denoting the number of classes and  denoting the negative log-confidence threshold. The generated pseudo-label set is given by .
To exploit the increased performance generated from stochastic weight averaging, we sample labels from the teacher's output ().


We initialize the optimization of Eq.~\ref{eq:st} by setting the latent variable  for all points, i.e. by only selecting the scribble-annotation (). The self-training protocol from pseudo-labels can then be summarised in two steps:
\begin{enumerate}
    \item \textbf{Training}: We fix  and optimize the objective function with respect to .
    \item \textbf{Pseudo-labeling}:  We fix  (and effectively ) and optimize the objective function with respect to . We update  given .
\end{enumerate}
The two steps can be repeated to take advantage of the improved representation capability of the model through pseudo-labeling.

While self-training with pseudo-labels has been proven to be an effective strategy in scribble-supervised semantic segmentation~\cite{cvpr2016scribblesup, tvcg2020scribble3d}, the class distribution in autonomous driving scenes are inherently long tailed, which may result in the gradual dominance of large and easy-to-learn classes on generated pseudo-labels. CB-ST~\cite{eccv2018classbalanced} proposes to sample labels while retaining the overall class distribution by setting thresholds in a class-wise manner. While this is sufficient in the 2D setting, we observe that 3D LiDAR data presents an additional unique challenge.

Due to the nature of the LiDAR sensor, the local point density varies based on the beam radius, as sparsity increases with distance. This results in sampling of pseudo-labels mainly from denser regions, which tend to show a higher estimation confidence. To reduce this bias in the pseudo-label generation, we propose a revised self-training scheme that not only balances based on the overall class-wise distribution, but also on range. We call our method class-range-balanced (CRB) pseudo-labeling and provide a visual sample in Fig.~\ref{fig:pseudo_label} comparing it to CB-ST.

We initially coarsely divide the transverse plane into  annuli of width  centered around the ego-vehicle. In Fig.~\ref{fig:pseudo_label}.b we illustrate the first three in red dashed lines. Each annulus contains points that fall between a range of distances, from which we pseudo-label the globally highest confident predictions on a per-class basis. This ensures that we obtain reliable labels while distributing them proportionally across varying ranges and across all classes.

\begin{figure}[t]
    \centering
    \includegraphics[width=\columnwidth]{figures/pseudo_label.pdf}
    \caption{Visual comparison of (50\%) (a) class-balanced pseudo-labeling~\cite{eccv2018classbalanced} and (b) proposed CRB. As seen right, generated pseudo-labels lack distant sparse region representation when balancing solely on class. Red lines For the quantitative analysis, see Tab.~\ref{tab:pseudo_label}.
    \label{fig:pseudo_label}}
\end{figure}

We redefine the self-training objective function (Eq.\ref{eq:st}) to include CRB as:

with  denoting the negative log-threshold for a class-annulus pairing. To solve the nonlinear integer optimization task, we employ the following solver:


When determining , we take the maximum output probability of each point, i.e. the networks confidence for the predicted label, and store the confidence values of all points in all frames for each class-annulus pairing in a global vector. Each vector is then sorted in descending order. We define a hyperparameter  which determines the percentage of pseudo-labels to be sampled, and find a threshold confidence for each vector by taking the value at index  times the vectors length.  is set as the negative logarithm of the threshold confidence. The process is summarized in Algorithm~\ref{alg:cbr}.

\begin{algorithm}[t]
\SetKwInput{KwInput}{Input}
\SetKwInput{KwOutput}{Return}
\DontPrintSemicolon

\KwInput{Dataset containing  point clouds, trained neural network , annulus count , portion  of selected pseudo-labels}
\For{} {
    value, class = max(, \ axis=0) \\
    B = max() / R \\
    range =  // B \\
    \For{} {
        mask = (class == c) \\
        \For{} {
            mask = (range == r) \\
            M = value[mask \& mask] \\
            M = [M, M]
        }
    }
}
\For{} {
    \For{} {
        M = sort(M, \ order=descending) \\
        thresh =  length(M) \\
         = -(M[:thresh])
    }
}
\KwOutput{}
\caption{Determination of  in CRB}
\label{alg:cbr}
\end{algorithm}

\subsection{Pyramid Local Semantic-context (PLS)} \label{sec:pls}

With self-training, the performance of the final network (Sec.~\ref{sec:crb}, \textbf{training}) is highly reliant on the pseudo-label quality. To ensure higher quality pseudo-labels, we further introduce a novel descriptor to enrich the features of the initial points by utilizing available scribbles.

We make the following two observations for the distribution of semantic classes in 3D space: (1) There exists a spatial smoothness constraint, i.e. a point in space is likely have the same class label as at least one of its neighbors since objects have nonzero dimensions; (2) There exists a semantic pattern constraint, i.e. a set of complex high-level rules governing inter-class spatial relations. For example, in outdoor autonomous driving scenes, vehicles lie on ground classes such as roads and parking areas, pedestrians often appear on sidewalks, buildings and vegetation outline roads.

\begin{figure}[t]
    \centering
    \includegraphics[width=\columnwidth]{figures/pls.pdf}
    \caption{Illustration of pyramid local semantic-context (PLS) augmentation based on scribble ground-truth (not to scale). As seen, the semantic-context can provide highly descriptive information about the local neighborhood of a point at scaling resolutions.}
    \label{fig:pls}
\end{figure}

We therefore argue that a local semantic prior can be used as a rich point descriptor to encapsulate the two stated cues. We propose using local \textit{semantic-context} at scaling resolutions to reduce the ambiguity when propagating information between the labeled-unlabeled point sets and to improve pseudo-labeling quality. 
We identify that the distribution of class labels over global coordinates is a robust, compact semantic descriptor, especially for unlabeled points.

We initially discretize the space into coarse voxels. This step is crucial as to avoid over-descriptive features that cause the network to overfit to the scribble annotations, reducing its capability to generalize well and understand meaningful geometric relations. We use multiple sizes of bins in cylindrical coordinates in order to follow the inherent point distribution of the LiDAR sensor at different resolutions. For each bin  we compute a coarse histogram: \vspace{-5px}

as illustrated in Fig.~\ref{fig:pls}. The \textit{pyramid local semantic-context} (PLS) of all points  is then defined as the concatenation of the normalized histograms:

for  resolutions. We append PLS to the input features and redefine the input LiDAR point cloud as the augmented set of points . When optimizing Eq.~\ref{eq:crb}, during the \textbf{training} step (Sec.~\ref{sec:crb}) we substitute  with  such that we generate better quality pseudo-labels during \textbf{pseudo-labeling}. 

At the end of the self-training pipeline, we require one extra \textbf{distillation} stage because PLS augmentation cannot be used during test-time as the scribble-information is not available. During distillation, we again set the input point cloud to . The resulting three stages of the overall pipeline is illustrated in Fig.~\ref{fig:pipeline}.

\section{Experiments}

We carry out our experiments using Cylinder3D~\cite{cvpr2021cylindrical} but forego the applied test-time-augmentation (TTA) and test the performance on the fully annotated SemanticKITTI~\cite{iccv2019semantickitti} \textit{valid}-set unless stated otherwise. Alongside the mean-Intersection-over-Union (mIoU), we also provide the relative performance of scribble-supervised (SS) training to the fully supervised upper-bound (FS) in percentages (SS/FS).

\noindent \textbf{Implementation Details:} For MT we set . For CRB, we define  annuli. For PLS, we divide  into ,  and  voxels. We only apply one iteration of self-training  () as we don't observe a significant increase in performance in consecutive steps.

\subsection{Results}

\begin{table*}[th]
    \tabcolsep=0.11cm
    \resizebox{\textwidth}{!}{
    \begin{tabular}{|l|c|c|cc|ccccccccccccccccccc|}
        \hline
        Model
        & Supervision
        & Ours
        & mIoU
        & SS/FF
        &\lturn{car}
        &\lturn{bicycle}
        &\lturn{motorcycle}
        &\lturn{truck} 
        &\lturn{other vehicle } 
        &\lturn{person}
        &\lturn{bicyclist} 
        &\lturn{motorcyclist} 
        &\lturn{road}
        &\lturn{parking}  
        &\lturn{sidewalk} 
        &\lturn{other ground} 
        &\lturn{building} 
        &\lturn{fence}
        &\lturn{vegetation} 
        &\lturn{trunk}
        &\lturn{terrain} 
        &\lturn{pole}
        &\lturn{traffic sign} \\
        [0.5ex] 
        \hline
        & fully & & 64.3 & - & 96.3 & 49.8 & 69.4 & 84.3 & 50.6 & 71.9 & 88.0 & 0.0 & 94.4 & 39.4 & 80.9 & 0.1 & 90.5 & 58.9 & 88.1 & 68.1 & 75.5 & 63.2 & 50.2 \\
        Cylinder3D~\cite{cvpr2021cylindrical} & scribble & & 57.0 & 88.6 & 88.5 & 39.9 & 58.0 & 58.4 & 48.1 & 68.6 & 77.0 & 0.5 & 84.4 & 30.4 & 72.2 & 2.5 & 89.4 & 48.4 & 81.9 & 64.6 & 59.8 & 61.2 & 48.7 \\
        & scribble & \ding{51} & 61.3 & 95.3 & 91.0 & 41.1 & 58.1 & 85.5 & 57.1 & 71.7 & 80.9 & 0.0 & 87.2 & 35.1 & 74.6 & 3.3 & 88.8 & 51.5 & 86.3 & 68.0 & 70.7 & 63.4 & 49.5 \\
        \hline
        & fully & & 61.1 & - & 95.7 & 20.4 & 63.9 & 70.3 & 45.5 & 65.0 & 78.5 & 0.0 & 93.5 & 49.6 & 81.0 & 0.2 & 91.1 & 63.8 & 87.2 & 68.5 & 72.3 & 64.4 & 49.1 \\
        MinkowskiNet~\cite{cvpr2019minkowski} & scribble & & 55.0 & 90.0 & 88.1 & 13.2 & 55.1 & 72.3 & 36.9 & 61.3 & 77.1 & 0.0 & 83.4 & 32.7 & 71.0 & 0.3 & 90.0 & 50.0 & 84.1 & 66.6 & 65.8 & 61.6 & 35.2 \\
        & scribble & \ding{51} & 58.5 & 95.7 & 91.1 & 23.8 & 59.0 & 66.3 & 58.6 & 65.2 & 75.2 & 0.0 & 83.8 & 36.1 & 72.4 & 0.7 & 90.2 & 51.8 & 86.7 & 68.5 & 72.5 & 62.5 & 46.6 \\
        \hline
        & fully &  & 63.8 & - & 97.1 & 35.2 & 64.6 & 72.7 & 64.3 & 69.7 & 82.5 & 0.2 & 93.5 & 50.8 & 81.0 & 0.3 & 91.1 & 63.5 & 89.2 & 66.1 & 77.2 & 64.1 & 49.4  \\
        SPVCNN~\cite{eccv2020spvnas}  & scribble & & 56.9 & 89.2 & 88.6 & 25.7 & 55.9 & 67.4 & 48.8 & 65.0 & 78.2 & 0.0 & 82.6 & 30.4 & 70.1 & 0.3 & 90.5 & 49.6 & 84.4 & 67.6 & 66.1 & 61.6 & 48.7 \\
        & scribble & \ding{51} & 60.8 & 95.3 & 91.1 & 35.3 & 57.2 & 71.1 & 63.8 & 70.0 & 81.3 & 0.0 & 84.6 & 37.9 & 72.9 & 0.0 & 90.0 & 54.0 & 87.4 & 71.1 & 73.0 & 64.0 & 50.5 \\
        \hline
    \end{tabular}
    }
    \caption{3D semantic segmentation results evaluated on the SemanticKITTI \textit{valid}-set. Alongside the per-class metrics we show the relative performance of the scribble supervised approach against the fully supervised (SS/FS).
    \label{tab:model_independent}}
\vspace{-10px} \end{table*}

We present the 3D semantic segmentation results from the SemanticKITTI -set in Tab.~\ref{tab:model_independent} for three state-of-the-art networks (Cylinder3D~\cite{cvpr2021cylindrical}, MinkowskiNet~\cite{cvpr2019minkowski}, SPVCNN~\cite{eccv2020spvnas}) to demonstrate the model independence of our approach. For the training schedule and architecture details, please refer to the respective publications.  In Fig.~\ref{fig:results} we present visual results using Cylinder3D.

Due to the lack of available supervision, the three presented models trained on scribble-annotations show a relative performances (SS/FS) of ,  and  compared to their respective fully supervised upper-bound. While the reduction in the number of supervised points reduce the class-wise performance across the board, this effect is further amplified for long tailed classes such as bicycle, truck and other-vehicle.

By applying our proposed pipeline for scribble-supervised LiDAR semantic segmentation, we are able to reduce the gap between the two training strategies significantly, reaching , ,  relative performance for all three models. As observed, the major performance gains originate from the same long tailed classes that initially show a deficit against their respective baselines.

\subsection{Ablation Studies}

\begin{table}[t]
    \centering
    \tabcolsep=0.11cm
    \resizebox{.95\columnwidth}{!}{
    \begin{tabular}{|l|cc|c|c|}
        \hline
        & \multicolumn{2}{c|}{Labeled}
        & Unlabeled
        & Valid \\
        Model
        & Volume
        & Type
        & Used
        & mIoU \\
        \hline
        Cylinder3D~\cite{cvpr2021cylindrical} & 10\% frames & fully & & 46.8 \\
        Cylinder3D~\cite{cvpr2021cylindrical} & 8\% points & scribbles &  & 57.0 \\
        \hline
        Sup-only~\cite{iccv2021guided} & 10\% frames & fully & & 43.9 \\
        Sup-only~\cite{iccv2021guided} & 8\% points & scribbles & & 55.0 \\
        \hline
        Semi-sup~\cite{iccv2021guided} & 10\% frames & fully & \ding{51} & 49.9 \\
        Sup-only+Ours & 8\% points & scribbles & \ding{51} & 58.5 \\
        \hline
    \end{tabular}
    }
    \caption{Compared are different annotation strategies for incomplete supervision. Sup-only refers to the baseline sparse U-Net model employed by Semi-sup~\cite{iccv2021guided}.  frames fully labeled correspond to  annotated points.
    \label{tab:comparison}}
    \vspace{-5px}
\end{table}

\begin{table}[t]
    \centering
    \tabcolsep=0.11cm
    \resizebox{.85\columnwidth}{!}{
    \begin{tabular}{|cccc|c|cc|} 
        \hline
        \multicolumn{4}{|c|}{Method} & Train & \multicolumn{2}{c|}{Valid} \\  
        Baseline  & MT        & CRB-ST    & PLS       & mIoU & mIoU & SS/FS \\
        \hline
        \ding{51} &           &           &           & 77.6 & 57.0 & 88.6  \\
        \ding{51} & \ding{51} &           &           & 78.0 & 59.3 & 92.2  \\
        \ding{51} & \ding{51} & \ding{51} &           &  -   & 60.6 & 94.2  \\
        \ding{51} & \ding{51} &           & \ding{51} & 86.0 &  -   &  -    \\
        \ding{51} & \ding{51} & \ding{51} & \ding{51} &  -   & 61.3 & 95.3  \\
        \hline
    \end{tabular}
    }
    \caption{Ablation study on proposed methods. PLS results are given after the first iteration, while CRB-ST results are given after the last iteration. Performances are reported on the SemanticKITTI \textit{train}- and \textit{valid}-sets respectively, along with the relative performance against fully supervised (SS/FS).
    \label{tab:ablation}}
\vspace{-10px} \end{table}

\noindent \textbf{Scribbles as Annotations}: 
We compare our proposed labeling strategy of weakly labeling all frames to fully labeling partial frames under a fixed labeling budget in Tab.~\ref{tab:comparison} and present the results for both Cylinder3D~\cite{cvpr2021cylindrical} and Sup-only, the baseline U-Net model employed in Semi-sup~\cite{iccv2021guided}. As seen, both models perform significantly better using scribble annotations compared to having full annotations on  of the \textit{train}-set by up to  and  mIoU.

Furthermore in Tab.~\ref{tab:comparison}, we also compare the current state-of-the-art on semi-supervised LiDAR semantic segmentation with our proposed scribble-supervised approach. Semi-sup~\cite{iccv2021guided} which further makes use of the  unlabeled frames still shows a  lower mIoU performance than a its baseline Semi-sup trained on scribble-annotations. Moreover, the same baseline model trained with our proposed pipeline further increases the gap to . 

\noindent \textbf{Effects of Network Components}: We perform ablation studies to investigate the effects of the different components of our proposed pipeline for scribble-supervised LiDAR semantic segmentation. We report the performance on the SemanticKITTI -set for intermediate steps, as this metric provides an indication of the pseudo-labeling quality, and on the -set to assess the performance benefits of each individual component. 

As seen in Tab.~\ref{tab:ablation}, by adding a weak form of supervision to the unlabeled point set via MT, we observe a  increase in mIoU, which alone reduces the relative performance drop of scribble-supervised training below . However the fully labeled training performance does not increase significantly. Applying CRB-ST at this point yields an mIoU of . Using PLS, we can further increase the training mIoU by , which has the benefit of boosting pseudo-labeling accuracy from  to  and improving mIoU performance in the subsequent step of the self-training protocol. Self-training with CRB pseudo-labeling now yields a further  increase in mIoU.

\begin{figure}[t]
    \centering
    \includegraphics[width=\columnwidth]{figures/results.pdf}
    \caption{Example results from the SemanticKITTI -set comparing (a) the ground truth frame; to Cylinder3D~\cite{cvpr2021cylindrical} trained (b) scribble-supervised, and  (c) scribble-supervised using our proposed pipeline.}
    \label{fig:results}
\vspace{-10px} \end{figure}

\noindent \textbf{Pseudo-label Filtering for Self-training}: We perform further ablation studies on the pseudo-labeling strategy used in the proposed self-training (ST) protocol and report the results in Tab.~\ref{tab:pseudo_label}. We replace our proposed CRB pseudo-labeling module with naive sampling (where all predictions are taken as pseudo-labels), threshold-based sampling~\cite{aaai2020curriculum,cvpr2020selftraining, arxiv2020rethinking}, class-balanced sampling (CB)~\cite{eccv2018classbalanced} and DARS~\cite{iccv2021dars}. For all given strategies we use the same input predictions generated from the PLS augmented MT.

Due to the long-tailed nature of outdoor LiDAR scenes for semantic segmentation, CB and DARS show great improvements over naive and threshold based sampling strategies with improvements of up to . Here we observe that in 3D semantic segmentation, the confidence overlapping is not as prevalent as in 2D. Applying DARS on CB generated pseudo-labels results in a reduction of only  data points on the entire \textit{train}-set with  (at most  for head classes). Therefore both CB and DARS perform similarly at  on the \textit{valid}-set.
After applying further balancing on range with our proposed CRB, we observe an improvement of  over CB, reaching a relative performance of  to fully-supervised.

\begin{table}[t]
    \centering
    \tabcolsep=0.11cm
    \resizebox{.9\columnwidth}{!}{
    \begin{tabular}{|l|cc|cc|} 
        \hline
        & \multicolumn{2}{c|}{Labeling} & \multicolumn{2}{c|}{Valid} \\   
        Pseudo-labeling Method &  & Acc & mIoU & SS/FS \\  
        \hline
        Naive & - & 86.3 & 59.4 & 92.4 \\
        Threshold-based~\cite{aaai2020curriculum,cvpr2020selftraining, arxiv2020rethinking} & 50\% & 99.0 & 59.1 & 91.9 \\
        Class-balanced~\cite{eccv2018classbalanced} & 50\% & 99.4 & 60.8 & 94.6 \\
        DARS~\cite{iccv2021dars} & 50\% & 99.3 & 60.8 & 94.6 \\
        CRB (Ours) & 50\% & 99.0 & 61.3 & 95.3 \\
        \hline
    \end{tabular}
    }
    \caption{Ablation study on the pseudo-labeling strategies comparing naive (all predictions), threshold-based, class-balanced labeling and DARS with our proposed CRB.  determines the percentage of labeled points. Performances are reported on the SemanticKITTI \textit{valid}-set. All methods use the same initial labels.
    \label{tab:pseudo_label}}
\end{table}

\begin{table}[t]
    \centering
    \tabcolsep=0.11cm
    \resizebox{.8\columnwidth}{!}{
    \begin{tabular}{|l|cc|cc|} 
        \hline
        & \multicolumn{2}{c|}{Scribble} & \multicolumn{2}{c|}{CRB-PL (50\%)} \\
        Consistency-loss & mIoU & SS/FS & mIoU & SS/FS \\
        \hline
        All points~\cite{cvpr2020towards10x} & 59.1 & 91.9 & 60.4 & 93.9 \\
        Partial (Ours) & 59.3 & 92.2 & 61.3 & 95.3 \\
        \hline
    \end{tabular}
    }
    \caption{Compared is the application of the consistency-loss on all points~\cite{cvpr2020towards10x} to our proposed partial application on only unlabeled points. We conduct experiments using the mean teacher pipeline with scribble annotations () and CRB pseudo-labels ().} 
    \label{tab:mt}
    \vspace{-10px}
\end{table}

\noindent \textbf{Consistency-loss within Mean Teacher}: We perform further ablation studies on the consistency loss within the mean teacher framework and compare our partial application on unlabeled points to the application on all points~\cite{cvpr2020towards10x}.

As seen in Tab.~\ref{tab:mt}, the difference between the two losses is negligible when training with scribble annotations. Scribbles only account for roughly  of the total point count, therefore the loss is mainly dominated by the unsupervised points in either setting. However, when training on generated pseudo-labels, we observe that the teacher network can inject uncertainties to labeled points, weakening the introduced supervision from the pseudo-labels and causing a decrease in mIoU of .

\section{Conclusion}

We have presented a weakly-supervised pipeline for LiDAR semantic segmentation based on scribble annotations. Our pipeline comprises of three stand-alone contributions that can be combined with any LiDAR semantic segmentation model to reduce the gap between fully-supervised and scribble-supervised training.

\noindent \textbf{Limitations}: We only annotate the \textit{train}-split of SemanticKITTI~\cite{iccv2019semantickitti}. We haven't applied our method to different datasets and LiDAR sensors due to annotation cost.

\noindent \textbf{Acknowledgements}: Special thanks to Zeynep Demirkol and Tim Br\"odermann for their efforts during annotation.


\clearpage
{\small
\bibliographystyle{ieee_fullname}
\bibliography{references}
}
\clearpage

\section{Supplementary Material}

\subsection{The ScribbleKITTI Dataset}

The goal of generating scribble-annotations is to to be fast and efficient while retaining as much information as possible to allow relatively high performance when compared to fully-supervised training. To this end, we formulate a set of guidelines for our annotators that also allows us to remain consistent across the dataset.

\noindent \textbf{Process: } We modify the point labeler~\cite{iccv2019semantickitti} to include line annotations. An example of the labeler GUI can be seen in Fig.~\ref{fig:labeling}. As seen, the annotator draws lines on the LiDAR scene by determining its start and end points. The tool also allows multi-segment lines (when providing more than two points) to allow easier labeling of curved surfaces. As LiDAR point clouds are inherently sparse, we add a thickness to the drawn line. All points, who's projections fall onto the thickened line, are labeled. At  height we set the line thickness to  pixels. We adjust the thickness proportionally to the zoom settings to remain consistent throughout the labeling.

\noindent \textbf{Guidelines: } During labeling, each object in a scene (e.g. vehicle, person, sign, trunk) is marked with a single line. To ease the process and eliminate any spillage to the ground points, the annotators can use a threshold based filter for the z-axis (which was already implemented in the point labeler~\cite{iccv2019semantickitti}) to hide ground points. An example can be see in Fig.~\ref{fig:labeling} bottom-right. However, unlike the dense annotated case, annotators do not need to later remove the filter in order to determine difficult border points between objects and ground classes. 

For classes that cover large distances, e.g ground classes (e.g. road, sidewalk, parking) and structure fa\c cades (e.g. building, fence), we try to annotate each segment using the least amount of scribbles. For example, given a north-south facing road segment that later turns right, the annotator draws two line-scribbles: 1) a north-south facing scribble that extends from the tile edge to junction, and 2) a west-east facing scribble that extends from the junction to the corresponding tile edge. If object interfere with the line-scribble (e.g. a car is in the middle of the road) the annotator can chose to scribble on either side of the object. For vegetation, each patch of greenery is annotated once. When periodically placed trees or bushes have similar heights, the threshold based filter can be used to isolate them, allowing a single annotation line to cover multiple individual trees. This also holds for sparse vegetation clusters in empty space (see main text Fig.~3 - bottom right). As 2D lines are projected onto the 3D surface to generate annotations, such scribbles may become indistinguishable once the viewing angle changes.

\begin{figure}[t]
    \centering
    \includegraphics[width=\columnwidth]{figures/supplementary/labeling.PNG}
    \caption{Screenshot of the labeling GUI and illustration of the process. As seen, the labeling tool~\cite{iccv2019semantickitti} has been modified to be able to generate line annotations. The annotator needs to only select the starting and ending positions of the line.}
    \label{fig:labeling}
\end{figure}

\subsection{Ablation Studies}

\noindent \textbf{Semi-supervised dataset:} Our line-scribbles label roughly  of the total point count and take  of the time to acquire compared to their fully labeled counterpart (based on the reported times of SemanticKITTI~\cite{iccv2019semantickitti}). Under a fixed labeling budget, we show that scribble-annotating all frames enables better representation capabilities compared to fully labeling partial frames (see main text Sec.~5.2). For these experiments, when simulating the semi-labeled setting, we follow the data generation process of Semi-sup~\cite{iccv2021guided} with  labeling.

\noindent \textbf{Labeling Percentage for CRB-ST:} We further investigate the effect of the labeling percentage  for CRB-ST. In Tab.~\ref{tab:beta} we compare results for three  values at , , . As seen, the mIoU performance does depend on the percentage of predictions selected as pseudo-labels.  outperforms  and  by  and  respectively, achieving a better balance between the introduction of more supervision through pseudo-labeling, and the reduction of errors propagating from pseudo-labeling to distillation.

\begin{table}[t]
        \centering
        \tabcolsep=0.11cm
        \resizebox{0.36\columnwidth}{!}{
        \begin{tabular}{|c|cc|} 
            \hline
             & mIoU & SS/FS \\
            \hline
            30\% & 60.9 & 94.7 \\
            50\% & 61.3 & 95.3 \\
            70\% & 60.8 & 94.6 \\
            \hline
        \end{tabular}
        }
        \caption{Investigating the effect of  for CRB-ST.}
        \label{tab:beta}
\end{table}


\end{document}
