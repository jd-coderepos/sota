
\documentclass[10pt]{article}

\usepackage{amsmath}
\usepackage{amssymb}

\usepackage[sc]{mathpazo}
\linespread{1.05}         \usepackage[T1]{fontenc}


\usepackage{graphicx}

\usepackage{cite}

\usepackage{color} 



\topmargin 0.0cm
\oddsidemargin 0.5cm
\evensidemargin 0.5cm
\textwidth 16cm 
\textheight 21cm

\usepackage[labelfont=bf,labelsep=period,justification=raggedright]{caption}
\usepackage{url}

\graphicspath{{./figures/}{./ieee_sensors/figures/}}
\DeclareGraphicsExtensions{.pdf,.jpeg,.png,.jpg}
\bibliographystyle{plos2009}
\usepackage{amsmath,amssymb}
\newcommand{\bm}[1]{\boldsymbol{#1}}    

\def\x{{\bf x}}
\def\X{{\bf X}}
\def\I{{\bf I}}
\def\s{{\bf s}}
\def\S{{\bf S}}
\def\y{{\bf y}}
\def\Y{{\bf Y}}
\def\z{{\bf z}}
\def\Z{{\bf Z}}
\def\A{{\bf A}}
\def\B{{\bf B}}
\def\D{{\bf D}}
\def\u{{\bf u}}
\def\U{{\bf U}}
\def\v{{\bf v}}
\def\V{{\bf V}}
\def\Npdf{{\mathcal N}}		\def\Gpdf{{\mathcal G}}		\def\Wpdf{{\mathcal W}}		\def\L{{\mathcal L}}		\def\F{{\mathcal F}}
\def\vecA{{\bf z}}
\def\vecY{{\bf m}}
\def\R{{\mathcal R}}
\def\E{\bm {\mathcal{E}}}

 \makeatletter
\renewcommand{\@biblabel}[1]{\quad#1.}
\makeatother

\usepackage{caption}
\usepackage{subcaption}
\usepackage[font=footnotesize,caption=false]{subfig}
\usepackage{cite}

\date{}

\pagestyle{myheadings}




\begin{document}

\begin{center}
{\Large
\textbf{The Smartphone Brain Scanner: \\ A Mobile Real-time Neuroimaging System}
}
\0.05in]
arks@dtu.dk, csta@dtu.dk, jaeg@dtu.dk, mkai@dtu.dk, lkai@dtu.dk
\\label{eq:ForwardProblem}
    {\bf Y} = {\bf AS} + {\E}.
\label{eq:Likelihood}
p \left( {{\bf Y} \left| {\bf S} \right. } \right) &=& { \prod_{t=1}^{N_t} {\Npdf} \left( {{\bf y}_t \left| {{\bf As}_t ,\beta^{-1}{\bf I}_{N_c } } \right.} \right)} \\
p \left( {\bf S} \right) &=& { \prod_{t=1}^{N_t} {\Npdf} \left( {{\bf s}_t \left| {{\bf 0} ,\alpha^{-1}{\bf L}^T {\bf L} } \right.} \right)}
\label{eq:MNsolution}
p \left( {{\bf S} \left| {\bf Y} \right. } \right) &=& { \prod_{t=1}^{N_t} {\Npdf} \left( {{\bf s}_t \left| {{\bm \mu}_t ,{\bf \Sigma }_{s} } \right.} \right)} \nonumber \\
{\bf \Sigma }_{s} &=& \alpha^{-1} {\bf I}_{N_d} - \alpha^{-1} {\bf A}^T {\bf \Sigma}_{y} {\bf A} \alpha^{-1}  \nonumber \\
{\bf \Sigma }_{y}^{-1}  &=&  {\alpha ^{ - 1} {\bf A}{\bf L}^{T}{\bf L}{\bf A}^{T}  + \beta ^{ - 1} {\bf I}_{N_c } }   \\
 {\bar{{\bf s}}_t}  &=& \alpha ^{ - 1} {\bf A}^{T} {\bf \Sigma }_{y} {\bf y}_t .

Here,  denotes a spatial coherence matrix, which in the current form take advantage of graph Laplacian using a fixed smoothness parameter ().  \section{Methods: Experimental Designs}
In this section we briefly describe the design of experiments that demonstrate and validate the potential of the SBS2 framework, the specific hardware, and the mobile approach in general.

	\subsection{Timing and Data Quality}

First, we analyze the data and timing quality. Many neuroscience paradigms rely heavily on accurate synchronization between EEG signal and stimuli, user response, or data from other sensors (e.g. P300, steady state visual evoked potentials). However, we can also envision applications in which the present 'low-cost' mobile setup will be used to collect data from many subjects over extended periods, where the precise synchronization is less important.

		\subsubsection{Emotiv EEG sampling}
The measurements are all based on the Emotiv EEG neuroheadset. The nominal sampling frequency of this neuroheadset is 128Hz (downsampled from internal 2048Hz). For validation purposes we test the actual sampling rate obtained from 3 randomly picked Emotiv devices (10  10 min measurements for each).
\subsubsection{Data Quality}
The Emotiv hardware adds a modulo  counter () to every packet transmitted from the device. This allows for data quality control (dropped packets) with accuracy of modulo . It is possible to obtain long recordings (over one hour) using this neuroheadset and SBS2. The battery in the Emotiv hardware is rated at  of continuous operation; in recording-only setup mobile device such as Galaxy Note (offline mode, screen off, only decrypting and recording) lasts for around . Provided that good visibility between the Emotiv EEG neuroheadset transmitter (located in the back part of the headset) and USB receiver is maintained, we were able  to achieve zero packet loss in the full rundown recording.
In order to acquire an EEG signal of good quality, the impedance between the electrodes and the scalp should be kept under . The Emotiv headset embeds the channel quality information in the signal directly (2Hz per channel, multiplexed into the signal). The values are unscaled, and come from applying a square wave of  to the DRL feedback circuit and extracting the amplitude of the inherent square wave using phase-locked detection on each channel. In principle the obtained values can be calibrated using a known impedance. For regular usage however, the hardware manufacturer assures that the green color of the indicator (channel quality value greater than 407) corresponds to sufficiently low impedance of the electrode. From our experience with the system this appears correct.
		\subsubsection{Timing}
\begin{figure}[!t]
\centering
\includegraphics[width=1\columnwidth]{timing_arch}
\caption{The timing measurement setup. 10Hz sinusoid is generated with a sound card, amplified, and fed into an oscilloscope and the EEG hardware.}
\label{figure_timing_arch}
\end{figure}
In order to measure the total delay in the system, we use the setup as depicted in Fig.~\ref{figure_timing_arch}. A sinusoidal audio tone of  and trailing and following periods of silence is generated and amplified so it can be detected by the EEG hardware and split into oscilloscope and EEG hardware. The software on the device performs peak detection on the signal and visualizes the peaks by changing the screen color from black to white. This change is detected by a photocell, connected to the second channel of the oscilloscope. We can then calculate , indicating the total delay of the system from the physical signal reaching the EEG hardware to it being visualized on the screen (without any additional processing), see Figure \ref{figure_timing_all}. We also look at the jitter  as the difference between  and  values of . The observed delta depends on the EEG sampling rate (here ), the processing power of the device, and screen refresh rate ( for all tested devices).

	\subsection{Imagined Finger Tapping}
One of the best known and examined experiments from the BCI literature is a task in which a subject is instructed to select between two or more different imagined movements \cite{Muller-Gerking1999:DesigningOptimalSpatialFilters,Babiloni2000:LinearClassLowResEEGImagHand,Dornhege2004,Blankertz2006:BerlinBCI}. Such experiments are rooted in a central aim of many BCI systems, namely of being able to assist patients with severe motor disabilities to communicate by 'thought'.
In this contribution we replicate a classical experiment with imagined finger tapping (left vs. right) inspired by \cite{Blankertz2006:BerlinBCI}. The setup consisted of a set of three different images with instructions, \textit{Relax}, \textit{Left}, and \textit{Right}. In order to minimize the effect of eye movements, the subject was instructed to focus on the center of the screen, where the instructions also appeared (3.5 inch display size, 800 x 480 pixels resolution, at a distance of 0.5 m). The instructions \textit{Left} and \textit{Right} appeared in random order. A total of 200 trials were conducted for a single subject.

\iffalse{
To illustrate the potential for performing such study in a completely mobile context all stimulus delivery and data recording were carried out using the SBS2 framework, while post-processing and decoding were conducted off-line using standard analysis tools. In particular, we applied a common spatial pattern (CSP) approach \cite{Muller-Gerking1999:DesigningOptimalSpatialFilters} to extract spatial filters that maximize the variance for one class while minimizing the variance of the other class and vice versa. A quadratic Bayesian classifier for decoding was applied on features transformed as in \cite{Muller-Gerking1999:DesigningOptimalSpatialFilters}.
}\fi

















 \section{Results and Discussion}
In this section we present and discuss the results of the experiments, validating the performance of the software, the used platforms, and EEG hardware.
\subsection{Timing and Data Quality}
\subsubsection{Emotiv EEG sampling}
From Fig.~\ref{figure_sampling} we can see that the Emotiv EPOC hardware a) has an actual sampling rate close to  and b) keeps this sampling rate in a fairly consistent manner.  Depending on the analysis performed on the data, one can assume , , or measure the actual sampling rate for every Emotiv EPOC hardware device individually.
\begin{figure}[!t]
\centering
\includegraphics[width=1\columnwidth]{sampling_rate_2_2}
\caption{Measured sampling frequency, including measurement resolution for 3 random Emotiv EPOC devices,  recordings for each.  All measured rates, including uncertainty are between  and , corresponding to  and  of nominal . The measurements were performed with  resolution ( accuracy) on  EEG packets. All tests were performed in a normal temperature on a single day.}
\label{figure_sampling}
\end{figure}
\subsubsection{Timing}
The results of the timing measurements (20 per device) are depicted in Fig.~\ref{figure_timing_all}.
\begin{figure}[!t]
\centering
\includegraphics[width=0.8\columnwidth]{timing_all}
\caption{System response timings. Galaxy Note running Android 4.0.1, 60Hz AMOLED screen, , ; Nexus 7 running Android 4.1.1, 60Hz IPS LCD screen, , ; MacbookPro, LCD screen (60Hz), , .}
\label{figure_timing_all}
\end{figure}
We can see in the results that for all devices there is a significant delay between the signal reaching the EEG hardware and being fully processed in the software (). This delay however, although significant is fairly stable ( jitter) and thus can be corrected for.

In the second set of measurements, we test the stability of the timing of the packets as they appear in the system. To measure this, we collect the packets from the Emotiv EPOC device and change the screen color every 4 packets (limited by screen refresh rate, ). This change is then measured by a photocell and fed into the oscilloscope and the distance between the 4-packet packages is calculated. Fig.~\ref{figure_distance_4} shows these measurements.
\begin{figure}[!t]
\centering
\includegraphics[width=1\columnwidth]{distance_sample}
\caption{Distances between 4-sample frames. Red line indicates expected distance of . The bars indicate the observed distance. We can see that the Emotiv system compensates every  samples to keep the average (black line) at the correct level.}
\label{figure_distance_4}
\end{figure}
In summary, the stability and quality of the acquired signal is excellent. Most of the variations, including imperfect sampling rate or timing jitters are constant and can be largely accounted for in the data analysis if necessary.


		\subsubsection{3D Source Reconstruction On-Device Performance}
 
Source imaging was obtained using the Bayesian inverse solver for the linear model in Eq.\ (\ref{eq:ForwardProblem}). The forward matrix  and cortical source mesh grid was based on a coarse resolution (5124 vertices) of the SPM8 template brain \cite{litvak2011eeg}, further reduced to 1028 using Matlab's function reducepatch. We tested the performance of 3D reconstruction and hyper-parameters calculation on 1s of signal: MacBookPro8,2 (Intel Core i7 Sandy Bridge 2.2GHz): 2ms/2s, Nexus 7: 8ms/1s, Galaxy Note: 8ms/11s, and Acer Iconia: 14ms/13s.

	\subsection{Imagined Finger Tapping -- Online Source Reconstruction}
In order to demonstrate the applicability of discriminating a simple task as the left and right imagined finger tapping on the cortical source level in an online framework, the EEG data were acquired with the Emotiv EPOC neuroheadset and compared with EEG recordings acquired with a Biosemi Active-II device 64 channels. The 64-channels were subsampled to represent the same channel locations as the Emotiv device.



Imagined fingertapping is known to lead to a suppression of the alpha (8-13 Hz) activity over the premotor/motor regions, with the contra lateral areas normally being more desynchronized \cite{pfurtscheller1999event}. Thus, imagined right finger tapping, should lead to the alpha activity being suppressed both in the left and right pre-motor region with the Left as the dominant one.
This is confirmed in Fig.~\ref{figure_Emotiv_PowerInPrecentralAALregions} and Fig.~\ref{figure_Biosemi14ch_PowerInPrecentralAALregions}, which demonstrate the SBS2 framework ability to online reconstruct meaningful current sources within the brain. Fig.~\ref{figure_Emotiv_PowerInPrecentralAALregions} shows how alpha power (8-13 Hz) is suppressed over time in the two regions of interest; Precentral Left and Right AAL. The responses are calculated as the averaged response over 87 Right cued trials. Note, that even though this result is an average over runs, the source localization was carried out in online mode with model parameters ( and ) and current sources () estimated online. By collecting these source estimates over time we have just presented the averaged response at the end of the experiment. Similarly, Fig.~\ref{figure_Biosemi14ch_PowerInPrecentralAALregions} demonstrates the averaged power response across 79 Right cued trials.
Interestingly, the suppression of the alpha power in the Left and Right Precentral AAL regions to right imagined finger tapping trials, looks quite similar for both devices (Emotiv EPOC and Biosemi), with the contralateral frontal regions (Left) being mostly suppressed.
\iffalse{
Similarly, a left imagined fingertapping cue leads to more suppressed alpha activity of right frontal regions during task execution as demonstrated in Fig.~\ref{figure_Emotiv_3Dsourcerecon_Trial190Left_6timepoints}. Additionally, the figure shows how the alpha power in the end of the trial is recovering to its initial state.
}\fi

\iffalse{
We here explore the SBS2 framework ability to adaptive estimating the model parameters  and , which determines the overall regularization of the source solution, . Figure~\ref{figure_3Dbrain_hyperparameters} demonstrate online estimation of the model parameters  and , the signal-to-noise ratio (SNR), and the regularization .
\begin{figure}[!t]
\centering
\includegraphics[width=0.8\columnwidth]{figures/Emotiv_MN8_13Hz_OnlineTrack_Regularization_Zoom.png}
\caption{Adaptive estimation of regularization term - \comment{CS}{Note we actually update the hyperparameters more often here than on on the phone/tablet - should we leave the figure or take it out?}}
\label{figure_3Dbrain_hyperparameters}
\end{figure}
}\fi







\begin{figure}
        \centering
        \begin{subfigure}[b]{0.8\columnwidth}
                \centering
        		\includegraphics[width=\textwidth]{figures/Emotiv_MN8_13Hz_MeanPowerRightTrials_SelectedRegions.png}
				\caption{Emotiv EPOC.}
				\label{figure_Emotiv_PowerInPrecentralAALregions}
        \end{subfigure}

        ~ \begin{subfigure}[b]{0.8\columnwidth}
                \centering
               \includegraphics[width=\textwidth]{figures/Biosemi14ch_MN8_13Hz_MeanPowerRightTrials_SelectedRegions.png}
\caption{Biosemi 14 channels. }
\label{figure_Biosemi14ch_PowerInPrecentralAALregions}
        \end{subfigure}
     \caption{Mean (solid lines) and standard deviation (dashed lines) of reconstructed current source power in the left (L) and right (R) Precentral AAL regions calculated across Right cued imagined finger tapping conditions. Online estimation of the  and  parameters. Minimum Norm Solution.}
\end{figure}

\iffalse{
\begin{figure}[!t]
\centering
\includegraphics[width=0.8\columnwidth]{figures/Emotiv_3Dsourcerecon_Trial190Left_6timepoints.png}
\caption{Emotiv EPOC. Reconstructed alpha (8-13 Hz) source power obtained using the MN method at different time points after a left imagined finger tapping cue (0 ms).}
\label{figure_Emotiv_3Dsourcerecon_Trial190Left_6timepoints}
\end{figure}
}\fi


\iffalse{
\begin{figure}[!t]
\centering
\subfloat[Emotiv EPOC]{\label{figure_Emotiv_PowerInPrecentralAALregions}\includegraphics[width=0.4\columnwidth]{figures/Emotiv_MN8_13Hz_MeanPowerRightTrials_SelectedRegions.png}}
\subfloat[Biosemi - 14 channels]{\label{figure_Biosemi14ch_PowerInPrecentralAALregions}\includegraphics[width=0.4\columnwidth]{figures/Biosemi_MN8_13Hz_MeanPowerRightTrials_SelectedRegions.png}}
\caption{Mean and standard derivation of reconstructed current source power in the left (L) and right (R) Precentral AAL regions calculated across Right cued imagined fingertapping conditions. Online estimation of the  and  parameters. Minimum Norm Solution.}
\label{figure_PowerInPrecentralAALregions}
\end{figure}
}\fi


\iffalse{
\begin{figure}[!t]
\centering
\subfloat[2][1]{[Emotiv EPOC]{\label{figure_Emotiv_PowerInPrecentralAALregions}\includegraphics[width=0.4\columnwidth]{figures/Emotiv_MN8_13Hz_MeanPowerRightTrials_SelectedRegions.png}}}
\subfloat[2][2]{[Biosemi - 14 channels]{\label{figure_Emotiv_PowerInPrecentralAALregions}\includegraphics[width=0.4\columnwidth]{figures/Biosemi_MN8_13Hz_MeanPowerRightTrials_SelectedRegions.png}}}
\caption{Mean and standard derivation of reconstructed current source power in the left (L) and right (R) Precentral AAL regions calculated across Right cued imagined fingertapping conditions. Online estimation of the  and  parameters. Minimum Norm Solution.}
\label{figure_PowerInPrecentralAALregions}
\end{figure}
}\fi


\iffalse{
	\subsection{Imagined Finger Tapping}

In order to validate the applicability of the Emotiv EPOC decoding imagined left and right finger tapping the EEG data were bandpass filtered (8-32 Hz) and we used the data in the interval 0.75-2.00s after stimuli onset as input to the common spatial pattern (CSP) algorithm \cite{Muller-Gerking1999:DesigningOptimalSpatialFilters}.
One important parameter in the CSP algorithm to be controlled is the number of spatial filters. To determine the number of spatial filters we applied cross-validation and examined the performance (accuracy of classifier) as function of training size (number of trials used for training), see Fig.~\ref{figure_imagfingertap_classification}.
The classifier was trained on a balanced set of trials (i.e. equal number of left and right trials), which was carried out 200 times for each training set size.

Figure~\ref{figure_imagfingertap_classification} demonstrate that in our configuration for this subject the optimal number of spatial filters seems to be around  and , corresponding to total number of filters being between 4 to 6, respectively. For  two spatial filters are used to maximize the variance for class 1 while minimizing the variance for class 2 and additional two spatial filters are used to minimize the variance for class 1 while maximizing the variance for class 2. It is interesting that only a relative low number of spatial filters is required to obtain a reasonable performance close to 60\%  correctly classified. Another interesting point for the results presented in this figure is that the performance would have increase further if more training examples had been acquired.
\begin{figure}[!t]
\centering
\includegraphics[width=0.8\columnwidth]{ImaginedFingertap_CSPquadBayes.png}
\caption{Mean accuracy left and right imagined finger tapping classification for a single subject. Mean accuracy is based on 200 splits in training and test data.
Classification is based on CSP and a quadradic Bayes classifier focusing on bandpass filtered data 8-32Hz in interval 0.75-2.00 s after onset.}
\label{figure_imagfingertap_classification}
\end{figure}

}\fi


































 \section{Conclusions}

We have presented the design, implementation, and evaluation of the first fully mobile 3D EEG imaging system: The \emph{Smartphone Brain Scanner}. The open source software allows realtime EEG data acquisition and source imaging on standard off-the-shelf Android mobile smartphones and tablets with a good spatial resolution and frame rates in excess of 40 fps. In particular, we have implemented a real-time solver for the ill-posed inverse problem with online Bayesian optimization of hyper-parameters (noise level and regularization).

The evaluation showed that the combined system provides for a stable imaging pipeline with a delay of 80-120ms. We showed results of a cue imagined finger tapping experiment and compared the smartphone brain scanner's average power in the alpha band in a relevant motor area, and we found that these aggregate signals compare favorably with those obtained with standard laboratory equipment. Both show the expected de-synchronization on initiation of imagined motor actions.

We suggest that the mobility and simplified application development may enable completely new research directions for imaging neuroscience and thus offset the expected reduced signal quality of a mobile off-the-shelf, low-density neuroheadset relative to more conventional and controlled, high-density laboratory equipment.
 \section*{Acknowledgment}
This work is supported in part by the Danish Lundbeck Foundation through CIMBI, Center for Integrated Molecular Brain Imaging.  

\bibliography{bibliography}

\end{document}
