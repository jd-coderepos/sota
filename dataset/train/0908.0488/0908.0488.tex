\documentclass{article}
\pdfoutput=1 
\usepackage{graphicx,amsthm,amssymb,amsmath,url}
\usepackage[thinlines]{easybmat}
\usepackage[letterpaper,hmargin=1.3in,vmargin=1.2in]{geometry}


\theoremstyle{plain} \newtheorem{thm}{Theorem}[section]
\newtheorem{lem}{Lemma}[section]
\newtheorem{obs}{Observation}[section]
\newtheorem{cor}{Corollary}[section]
\newtheorem{defn}{Definition}
\newtheorem{prop}{Proposition}

\newcommand{\p}{\mathbf{p}}
\newcommand{\ot}{\tilde{\omega}}
\newcommand{\R}{\mathbb{R}}
\newcommand{\etal}{\emph{et~al.}}



\title{Small grid embeddings of 3-polytopes}

\author{  Ares {Rib\'o Mor}\thanks{Gesellschaft zur F{\"o}rderung angewandter Informatik e.V., Berlin, Germany,  \texttt{ribo@gfai.de}.
Partially supported by the
Deutsche Forschungsgemeinschaft within the European Research Training Network \emph{Combinatorics,
Geometry and Computation} (No.~GRK~588/2).}
	\and
	G\"unter Rote\thanks{Institut f\"ur Informatik, Freie Universit\"at Berlin, Germany, \texttt{rote}\texttt{@inf.fu-berlin.de}.}
	\and
	Andr\'e Schulz\thanks{Institut f\"ur Mathematsche Logik und Grundlagenforschung, Universit\"at M\"unster, \protect\url{andre.schulz@uni-muenster.de}. Partially supported by
the German Research Foundation (DFG) under grant SCHU 2458/1-1.}
}




\begin{document}

\maketitle


\begin{abstract}
We introduce an algorithm that embeds a given 3-connected planar graph as a convex 3-polytope with integer coordinates. The size of the  coordinates is bounded by . If the graph contains a triangle we can bound the integer coordinates by . If the graph contains a quadrilateral we can bound the integer coordinates by . The crucial part of the algorithm is to find a convex plane embedding whose edges can be weighted such that the sum of the weighted edges, seen as vectors, cancel at every point. It is well known that this can be guaranteed for the interior vertices by applying a technique of Tutte. We show how to extend Tutte's ideas to construct a plane embedding where the weighted vector sums cancel also on the vertices of the boundary face. 
\end{abstract}

\section{Introduction}

\paragraph{Problem Setting.}
The graph of a polytope is an abstraction from its geometric realization.
For a 3-polytope,
 the graph determines the complete combinatorial structure.
The graphs of 3-polytopes are characterized by Steinitz' seminal
 theorem~\cite{s-emw-22}, which asserts that they are exactly the planar 3-connected graphs.



A natural question is to ask for a geometric realization of a 3-polytope when its combinatorial structure is given. One might be interested in a realization that fulfills additionally certain optimality criteria. For example a good resolution is desirable to obtain aesthetic drawings~\cite{cgt-cdgtt-96,s-dpgvr-09}. 
We address a different problem and ask for an embedding whose vertices can be placed on a small integer grid. The vertex coordinates of such an embedding can be stored efficiently.

\paragraph{Related Work.}
Suppose we are given the combinatorial structure of a 3-polytope by
a graph 
with  vertices.
The original proof of Steinitz' theorem transforms the
3-connected planar graph  into the graph of the tetrahedron by a
sequence of elementary operations.
The transformation preserves the realizability as a 3-polytope. Since all operations can be carried out in the rationals, the proof gives a method to construct a realization of a 3-polytope with integer coordinates. However, it is not easy to keep track of the size and the denominators of the coordinates, which makes it difficult to apply this approach for our problem. An alternative proof of Steinitz' theorem goes back to the Koebe-Andreev-Thurston Circle Packing Theorem (see for example Schramm~\cite{s-eupsc-91}). This
approach relies on non-linear methods, which make the (grid) size of the embedding intractable. 
A third proof of Steinitz' theorem relies on the ``liftability''  of planar barycentric embeddings. Since this \emph{barycentric approach} is based on linear methods, its construction favors computational aspects of the embedding.  
This led to a series of embedding algorithms: Hopcroft and Kahn~\cite{hk-prga-92}, Onn and Sturmfels~\cite{os-qst-94}, Eades and Garvan~\cite{eg-dspgt-95}, Richter-Gebert~\cite{rg-rsp-96}, 
Chrobak, Goodrich, and Tamassia~\cite{cgt-cdgtt-96}.
Our work also follows this paradigm.

As a first quantitative analysis of Steinitz' theorem, Onn and Sturmfels~\cite{os-qst-94} showed that integer coordinates smaller than  suffice to realize a 3-polytope. 
Richter-Gebert improved this bound to . A more careful analysis of Richter-Gebert's approach shows that the size of the integer coordinates can be bounded by ~\cite{r-rcpps-06}.


Integer realizations with at most exponentially large 
coordinates in terms of  
were previously known for polytopes whose graph contains a
triangle (Richter-Gebert  \cite{rg-rsp-96}). We describe this method in Section~\ref{sec:case1} (p.~\pageref{sec:case1}) as  Case~1
of our embedding algorithm.
In Richter-Gebert's approach (and already in
Onn and Sturmfels~\cite{os-qst-94}),
 graphs without triangles are embedded by first embedding the polar polytope, whose graph in this case has to contain a triangle. Based on the polar, an embedding of the original polytope is constructed. However, this
operation yields coordinates with a quadratic term in the exponent.

For \emph{triangulated} 3-polytopes,
Das and Goodrich~\cite{dg-copdc-97} showed that they can be embedded with coordinates of size , using an incremental method which can be carried out in  arithmetic operations.
Triangulated 3-polytopes are easier to realize on the grid than general polytopes, since each
vertex can be perturbed within some small neighborhood while
maintaining the combinatorial structure of the polytope. An explicit
bound on the coordinates has not been worked out by the authors.  For
stacked polytopes a better upper bound exists \cite{z-gcsg-07}, but it
is still exponential.

\paragraph{Lower Bounds.}
Little is known about the lower bound of a grid embedding of a 3-polytope. 
An integral convex embedding of an -gon in the plane needs
an area of  \cite{az-mnecd-95,a-lbvsc-61,t-ep-91,vk-mnecd-82}.
Therefore, realizing a 3-polytope with an -gonal face requires
at least one dimension of size .

\paragraph{Two Dimensions.}
In the plane, planar 3-connected graphs  can be embedded on a very small grid.
For a crossing-free straight-line embedding  an  grid
is sufficient \cite{fpp-hdpgg-90,s-epgg-90}.
This is also true if the embedding has to be convex
\cite{bfm-cdpg-06}. A
strictly convex drawing can be realized on an  grid \cite{br-scdpg-06}.
\paragraph{Higher Dimensions.}
Already in dimension 4, there are polytopes that cannot be realized
with rational coordinates, and a 4-polytope that can be realized
on the grid might require coordinates that are doubly exponential in
the number of its vertices. Moreover, it is NP-hard to even decide
if a lattice is a face lattice of a
4-polytope~\cite{rg-rsp-96,gz-rspu-95}.

\paragraph{Results.}
In this article we develop an algorithm that realizes  as a 3-polytope with
integer coordinates not greater than . 
This implies that for any -polytope a combinatorially  equivalent polytope can be stored
with  bits per vertex. 
For the case that  contains a triangle we show that  admits an integer realization with no coordinate larger than , if  contains a quadrilateral  face, the size of the coordinates can be bounded by . The most difficult part of the algorithm is to locate the boundary face of the plane embedding such that a lifting into  exists. This problem can be reduced to a non-linear system which is most complex when  contains neither a triangle nor a quadrilateral face.



Partial results containing
the essential ideas for graphs with quadrilateral faces (Case~2
of Section~\ref{sec:case2}) were presented by the second author
at the workshop \emph{The Future of Discrete Mathematics}
at {\v Sti\v r\'\i n} Castle, Czech Republic, in May 1997.
The results of this paper were presented in a different form
at the 23rd Annual Symposium on Computational Geometry
 in Gyeongju, Korea, in June 2007~\cite{rrs-epsg-07}. Since then, we were able to simplify the computation of the explicit bounds with help of Lemma~\ref{lem:zsize}. The simplification yields slightly different bounds. 
By improving the bound of Lemma~9 in~\cite{rrs-epsg-07} by a polynomial factor (now  Lemma~\ref{lem:foresttree}) 
we obtain better bounds in the end.
However, our analysis could be further improved with help of the more complicated construction of~\cite{rrs-epsg-07}. 
Since the improvement would only result in a constant factor we decided to present the simpler and more elegant analysis. 

 A follow-up work~\cite{s-dpgvr-09} extends the techniques of this article
and studies more general barycentric embeddings.
With help of these modifications, a grid embedding with -coordinates
smaller than  can be constructed. The small -coordinates are
realized at the expense of the size of the  and -coordinates,
which are bounded by .


\paragraph{Remark:}
Most recently, Buchin and Schulz~\cite{BS10} improved the upper bound for the maximum number of spanning trees contained in a planar graph. 
This has a direct consequence for our results, since we obtain the bound for the necessary grid size in terms of this quantity. In particular, the new bounds of~\cite{BS10} yield that our algorithm requires a grid of size  (general case),  ( contains a quadrilateral face), and  ( contains a triangular face).





\section{Lifting Planar Graphs}
\label{sec:lifting}

Let  be a -connected planar graph with vertex set  embedded in the plane with straight edges and no crossings.
The coordinates of a vertex  in the (plane) embedding are called , the whole embedding is denoted as . 
Let  be a height assignment for the vertices in . We write  for .
If the vertices  of every face of  lie on a common plane, we call the height assignment  a \textit{lifting} of .


\begin{defn}[Equilibrium, Stress]
  An assignment  of scalars \textup(denoted as
  \textup) to the edges of  is
  called a \emph{stress}.
\begin{enumerate}
\item A vertex  is in equilibrium in , if

\item The embedding  is in equilibrium if all vertices
  are in equilibrium.
\item If  is in equilibrium for the stress , then
   is called an equilibrium stress for .
\end{enumerate}
\end{defn}
It is well known that equilibrium stresses and liftings are related.
Maxwell observed
in the 19th century
 that there is a correspondence between
embeddings with equilibrium stress and projections of 3-dimensional
polytopes \cite{m-rfdf-64}. 
There are different versions of Maxwell's
theorem. For the scope of this article the following formulation is the most
suitable. 
\begin{thm}[Maxwell,
  Whiteley] Let  be a planar 3-connected graph with embedding  and 
designated face . There exists a correspondence between
\begin{itemize}
\item[\rm A)] equilibrium stresses  on ,
\item[\rm B)] liftings of  in , where face  lies in the
  -plane.
\end{itemize}
\end{thm}
The proof that A induces B (which is the important direction for our purpose) is  due to Walter Whiteley \cite{w-mspp-82}. The Maxwell-Cremona correspondence finds interesting applications in different areas (see for example
Hopcroft and  Kahn~\cite{hk-prga-92}, and Connelly, Demaine and Rote~\cite{cdr-spcpc-03}).

To describe a lifting, we have to specify for each face  of the
graph the plane  on which it lies.
 We define  by the two
parameters  and . The plane  is characterized
by the function that assigns to every point  in the plane a third coordinate by

Here,  denotes the dot product.
The correspondence between liftings and stresses comes from the
observation that the ``slope difference'' 
between two adjacent faces  and  is perpendicular to the
edge  that separates them:

for some scalar .
Here,  denotes the vector 
rotated by 90 degrees.
It is not hard to show that these numbers  form an
equilibrium stress.

The other direction, the computation of the lifting of  induced by  is straightforward,
see Crapo and Whiteley~\cite{cw-psspp-93}.
We follow the presentation of  Connelly, Demaine and Rote \cite{cdr-spcpc-03} for the computation of the lifting. 

The parameters  and  can be computed by the following iterative method:
We pick  as the face that lies in the -plane, and set  and . Then we lift the remaining faces one by one. This is achieved by selecting a face  that is incident to an already lifted face . 
Let  be the common edge of  and . Assume that in  the face   lies left of the  directed edge , and  lies right of it. The parameters of  can be computed by
 
The formula~\eqref{equ:liftingit} comes directly from~\eqref{eq:perpendicular}, and
\eqref{equ:liftingit2} comes from the fact that the two planes must intersect above  and .

The sign of the stresses allows us to say something about the curvature of the lifted graph. 
According to~\eqref{eq:perpendicular} and \eqref{equ:liftingit}, the sign of  that separates  and  tells us if the lifted face  lies below or above .
As a consequence we obtain the following:
\begin{prop}\label{obs:signstress}
Let  be a straight-line embedding of a planar 3-connected graph
  with equilibrium stress. 
If the stresses on the boundary edges are
negative and all other stresses positive then the
lifting induced by
such equilibrium stress results in a convex 3-polytope. 
\end{prop}

\begin{lem}\label{lem:scaling1}
If  has integer coordinates only and the equilibrium stress is integral on all interior edges, then the -coordinates
of the lifted embedding are also integers.
\end{lem}
\begin{proof}
We select an interior face as face .
The gradient  and
the scalar  are clearly integral. 
For all other interior faces  the parameters ,  
of the planes  can be computed with
help of equations~\eqref{equ:liftingit} and~\eqref{equ:liftingit2}. 
By an inductive argument 
these parameters are integral as well. 
Computing the -coordinate of some point 
by~\eqref{eq:plane}
boils down to the multiplication and addition of integers.
\end{proof}

\section{The Grid Embedding Algorithm}
\label{sec:embedding}
\subsection{The Plane Embedding}

The embedding of  as a 3-polytope uses the following high level approach.
First we embed  in the plane, such that it is liftable (see Section~\ref{sec:lifting}), then we lift the embedding to , finally we scale to obtain integer coordinates as described in Section~\ref{sec:scaling}. The analysis of the algorithm in Section~\ref{sec:analysis} gives the new upper bound. The most challenging part is to construct a liftable 2d-embedding. 










An embedding is called \emph{barycentric} if every vertex that is not
on the outer face is in the barycenter of its neighbors. Tutte showed
that for planar 3-connected graphs the barycentric embedding for a
fixed convex outer face is unique~\cite{t-crg-60,t-hdg-63}. Moreover,
if embedded with straight lines, no two edges cross, and all faces are
realized as convex polygons.  In the barycentric embedding all
vertices that are not on the outer face are in equilibrium according
to the stress . Our embedding algorithm uses this
special stress only, although we state the lemmas as general as possible. Since our techniques might find applications in
other settings we develop our main tools for arbitrary
stresses. (Note that Tutte's approach works with
arbitrary stresses that are positive on interior
edges, see for example Gortler \etal~\cite{ggt-dofma-06}.)




We describe now how to compute the barycentric embedding of . 
Let  be a face of  that we picked as the outer face, and let 
 be the number of vertices in . For simplicity we want  as small as 
possible. 
Euler's formula implies that every planar and -connected graph has a face  with  edges. 
We assume that the vertices in  are labeled such that the first  vertices
belong to  in cyclic order. 
Let  be the index set of the boundary vertices and let  denote the index set
of the interior  vertices.
The edges of  are called \textit{boundary edges},
all other edges \textit{interior edges}. 
The stresses on the exterior edges will be defined later, but since they don't matter for the barycentric embedding we set them to zero for now.


We denote with  the \emph{Laplacian} matrix of  (short \emph{Laplacian}), which is defined as follows
For the special ``weights''  the Laplacian equals the
negative adjacency matrix of .with vertex degrees on the diagonal.
We subdivide  into block
matrices indexed by the  sets  and , and obtain , , , and .
The matrix  is called the \textit{reduced Laplacian matrix}  of . For convenience we write  instead of 
\paragraph{Example.}
  \begin{figure}
    \centering
    \includegraphics{figs/smallexample}
    \caption{A small example graph.}
    \label{fig:smallexample}
  \end{figure}
Consider the graph of Figure~\ref{fig:smallexample},
with  and . We have

\iffalse MAPLE CODE
with(linalg):
LBB := matrix([
[   2,0,0],
[   0,2,0],
[   0,0,2]]);
LBI := matrix([
[ -1,-1,0,0],
[  0,-1,0,-1],
[  0, 0,-1,-1]]);
LII := matrix([
[   3,-1,-1,0],
[  -1, 4,0,-1],
[  -1, 0,3,-1],
[  0,-1,-1,4]]);
Ls := evalm( LBB - LBI &* inverse(LII) &* transpose(LBI));
       [ 96             -39 ]
       [ --     -3/5    --- ]
       [ 95             95  ]
       [                    ]
       [-3/5    6/5     -3/5]
       [                    ]
       [-39              96 ]
       [---     -3/5     -- ]
       [95               95 ]
\fi
In the example the presence of the boundary edges , ,
and  is not reflected in the Laplacian, because the stress is set to zero on the boundary.

The location of the boundary vertices is given by the vectors 
 and  .
Since every vertex should lie at the (weighted) barycenter of its neighbors, the coordinates of the interior vertices    and
   have to satisfy the  equilibrium condition
\eqref{equ:equilibrium} for the stress . In particular, 
the equations  and   have to hold. 
Thus, we can express the interior coordinates as

For non-zero weights  the matrix  is irreducible (that is, the underlying graph is connected, see Lemma~\ref{lem:technical}) and diagonally dominant. As a consequence  is invertible and \eqref{equ:tuttex} has a unique solution \cite[page 363]{hj-ma-90}.

The barycentric embedding assures that the interior vertices are in equilibrium. However, to make  liftable we have to guarantee the equilibrium also for the vertices on . 
We define the vectors

as the non-resolving ``forces'', which arise at the boundary vertices and
cannot be canceled by the interior stresses:

Our goal is to define the yet unassigned stresses on the boundary
edges such that they cancel the forces in .  However, this is not
always possible, depending on the shape of the outer face.
In order to pick a good embedding of , we have to know how
changing the coordinates of the outer face changes the forces in .
The following lemma helps to express this dependence.

\index{substitution lemma} \index{stress!equilibrium}

\begin{lem}[Substitution Lemma] \label{lemma:1}
There are  weights , for ,
 independent of the location of the boundary vertices, such that 
 
The weights  are the off-diagonal entries of  .
If  is integral,
each  is a  multiple of .
\end{lem}
\begin{proof}
Let  denote the vector , 
where  is the -component of the vector .
We rephrase \eqref{equ:nonresolve} as   
.
With help of \eqref{equ:tuttex} we eliminate  and 
obtain

(The matrix  is the Schur complement of  in .)
For the -coordinates, we obtain a similar formula with the same matrix . 
We define  as the off-diagonal entries  of
. Since , the matrix  is
symmetric and therefore  holds. 

To show that the expression
 has the form stated in the lemma we have
to check that all row sums in  equal 0. Let 
denote the vector  where all entries are , equivalently  denotes the vector that contains only zeros as entries. Since each of the last  rows of  sums up to  we have ; and hence 
. Plugging this
  expression into
   gives us 
, which equals
. 
The matrix  can be written as a rational expression whose
denominator is the determinant of , and thus the
weights  are multiples of .
\end{proof}

In linear algebra terms, the lemma can be rephrased as saying that the
Schur complement of a submatrix of a weighted Laplacian, if it exists,
has again the form of a weighted Laplacian.

The proof assumes that  is invertible. This is the case
whenever the graph  has no connected component that is a subset
of~. If such components exist, they can simply be omitted, since
they are completely disconnected from  and hence have no effect on
the forces in . Hence, the lemma holds for arbitrary graphs, without any connectivity assumptions.

\paragraph{Example.}
For the example of Figure~\ref{fig:smallexample},
we obtain the following substitution stresses.








We emphasize that the  values are independent
of the location of  and : they only depend on the
combinatorial structure of . In other words, the stresses
 contain all the necessary information about the
combinatorial structure of . Thus, we have a compact
description (of size ) of the structure of  that 
is responsible
for the forces in . We call the stresses  \textit{substitution
  stresses} \index{substitution stresses}
to emphasize that they are used
as a substitution for the combinatorial structure of~.

For the later analysis of the grid size it is necessary to bound the
size of the substitution stresses. We first state a technical lemma.
\begin{lem}\label{lem:technical}
  Removing all vertices and edges of a face  from a 3-connected
  planar graph  leaves a connected graph.
\end{lem}
\begin{proof}
  After realizing  as a polytope, the
  claim becomes a special case of the well-known statement that a graph of a
  polytope in any dimension remains connected if the vertices of some
  face are removed. This statement can be proved by defining a linear
  objective function that realizes the minimum entirely on the removed
  vertices. Every remaining vertex is connected to the maximum vertex
  by a monotone increasing path.  The objective function
  can be perturbed such that there is a unique maximum.
\end{proof}
\begin{lem} \label{lem:2}
\begin{enumerate}
\item Let  be a stress that is positive on every interior edge. Any induced substitution stress  is positive.
\item Let  be the stress that is  on every interior edge. Any induced substitution stress  is smaller than .
\end{enumerate}
\end{lem}
\begin{proof}
The substitution stresses are independent of the location of . 
Therefore, we can choose the positions for 
the boundary vertices freely.
We place vertex  at position  and all other boundary vertices at .
All vertices lie on the segment between  and , which is the convex hull of the boundary vertices. 

We now show that all interior vertices lie in the interior of this
segment.
If an interior vertex lies at  then all its neighbors have to lie at  as well. Otherwise the vertex cannot be in equilibrium. But since due to Lemma~\ref{lem:technical} all interior vertices are connected by interior edges this would imply that all interior vertices must lie at . This is a contradiction, since  is also the neighbor of an interior vertex. By the same arguments one can show that no interior vertex can lie at . 
Therefore, all interior vertices have a positive -coordinate strictly smaller than .

In our special embedding the force  () can be expressed as .
By \eqref{equ:nonresolve} we 
have . 
Due to the results of the previous paragraph, this sum consists of at most   summands, which are positive, and in the case  smaller than 1. Both statements of the lemma follow.
\end{proof}

We are now ready to introduce the embedding algorithm. 
As a first step we construct a 2d
embedding in equilibrium with respect to a stress  with  on the interior edges.
In order to get equilibrium on the boundary vertices as well,
we have to choose their locations and the stresses on the boundary edges appropriately.
This leads to a
non-linear system in the  unknowns , and  and the 
unknown boundary stresses .
Let  be the Laplacian of the graph that consists of the outer face  only, with unknown stresses
 
for the boundary edges. The  equations of the system are given by 

Since these equations are dependent, the system is
under-constrained. To solve it, we fix as many boundary coordinates as
necessary to obtain a unique solution.  We also have to ensure that the
solution defines a \emph{convex} face. We continue with a case
distinction on .

\subsubsection*{Case 1:  contains a triangular face}\label{sec:case1}
The triangular case is easy: we can position the boundary  vertices at any
convenient position (see for example~\cite{hk-prga-92}). We choose:

\begin{lem}
If  contains a triangle and we place the boundary
vertices as
stated in \eqref{triangle} then the boundary forces can be resolved.
\end{lem}
\begin{proof}
We embed  as barycentric embedding and calculate the substitution
stresses. 
After setting 
all points are in equilibrium. 
\end{proof}

\subsubsection*{Case 2:  contains a quadrilateral but no triangular face}\label{sec:case2}
If  is a quadrilateral we have to fix some 
coordinates of the boundary vertices such that
it is possible to cancel the forces in . 
We used computer algebra software to experiment with various
possibilities to constrain the coordinates and solve the non-linear
system~\eqref{equ:system}.
A unique solution can be obtained by setting 

with

The solution of the equation system~\eqref{equ:system} provides also the stresses on the boundary edges. These stresses are not necessary for our further computations; we mention them here for completeness only.

We assume that .
(Otherwise we cyclically relabel the vertices on .) 
Since  on the interior edges the substitution stresses are positive by Lemma~\ref{lem:2}. Under this assumption we can deduce that . Hence,   forms a \emph{convex} face.


Note that the substitution stresses  between adjacent
vertices (on the boundary) are irrelevant. The 
forces resulting by the boundary stresses  
can be directly canceled by the
corresponding stresses . 
This can also be observed by looking at the solution  
of the corresponding equation system:
Boundary stresses do not appear in the solution for . (Furthermore 
the sum  for boundary edges 
does not depend on any other boundary stress either.)

\begin{lem}
If  contains a quadrilateral and we place the boundary vertices as
stated in \eqref{equ:4goncoordinates} and \eqref{equ:sol4}, 
then  forms a convex quadrilateral 
and the boundary stresses
 \eqref{equ:sol4-2}
 cancel the
forces~.
\qed
\end{lem}


\noindent
\subsubsection*{Case 3:  contains no triangular and no quadrilateral
  face}
The case if the smallest face of  is a  pentagon is more complicated. 
We have  substitution stresses , but the adjacent ones
do not count (by the same reasons given in the previous case).
So we are left with five ``diagonal'' substitution stresses
, and . 

Like in the previous cases we determine a unique solution of the equation system
by fixing some of the coordinates of the outer face. 
However, we have to make more effort to guarantee the
convexity of . We first observe:
\begin{lem}\label{lem:lemma5gon}
We can relabel the boundary points for any stress  such that

\end{lem}
\begin{proof}
Without loss of generality we assume that the largest stress on an
interior edge is . If
 we are done. Otherwise we relabel the vertices by
exchanging  and
.
\end{proof}
For the rest of this section we label the vertices such that Lemma~\ref{lem:lemma5gon} holds.
The way we embed   depends on the substitution stresses .\\label{equ:case1}
\ot_{35}\ot_{14}+\ot_{14}\ot_{25}+\ot_{25}\ot_{24}+\ot_{13}\ot_{35}>\ot_{35}\ot_{25}.

 \mathbf{p}_1= \begin{pmatrix}0 \\  0\end{pmatrix},
 \mathbf{p}_2= \begin{pmatrix}1 \\  0\end{pmatrix},
 \mathbf{p}_3= \begin{pmatrix}1 \\ 1 \end{pmatrix},
 \mathbf{p}_4= \begin{pmatrix}0 \\ 1 \end{pmatrix},
 \mathbf{p}_5= \begin{pmatrix}x_5\\y_5 \end{pmatrix}.
x_5=
 \frac{(\ot_{13}-\ot_{25}-\ot_{24})(\ot_{35}+\ot_{13}-\ot_{24})}       {\ot_{35}\ot_{14}+\ot_{14}\ot_{25}+\ot_{25}\ot_{24}+\ot_{13}\ot_{35}-\ot_{35}\ot_{25}},y_5=\frac{\ot_{35}+\ot_{13}-\ot_{24}}{\ot_{35}+\ot_{25}} .
\omega_{12}  &= 
 \frac{\ot_{13}(\ot_{25}^2+\ot_{24}\ot_{35}+2\ot_{24}\ot_{25}-\ot_{13}\ot_{25})
       \!+\!\ot_{14}(\ot_{25}^2+\ot_{25}\ot_{35}+\ot_{24}\ot_{25}+\ot_{35}\ot_{24})}
       {\ot_{35}\ot_{25}-\ot_{14}\ot_{25}-\ot_{25}\ot_{24}-\ot_{13}\ot_{35}-\ot_{35}\ot_{14}}- \ot_{12},\\         
\omega_{34} &=  
 \frac{ \ot_{14}(\ot_{35}^2+\ot_{35}\ot_{13}+\ot_{25}\ot_{35}+\ot_{13}\ot_{25})
       \!+\!\ot_{24}(\ot_{35}^2+\ot_{13}\ot_{25}+2\ot_{13}\ot_{35}-\ot_{35}\ot_{24})}
       {\ot_{35}\ot_{25}-\ot_{14}\ot_{25}-\ot_{25}\ot_{24}-\ot_{13}\ot_{35}-\ot_{35}\ot_{14}} - \ot_{34},  \\
\omega_{23} & = 
\frac{\ot_{13}\ot_{25}+\ot_{25}\ot_{35}+\ot_{24}\ot_{25}}
         {-\ot_{25}-\ot_{35}}     -\ot_{23}, \\
\omega_{45}  & = 
\frac{\ot_{24}\ot_{25}+\ot_{25}\ot_{14}+\ot_{14}\ot_{35}+\ot_{24}\ot_{35}}
         {\ot_{13}-\ot_{24}-\ot_{25}}     -\ot_{45}, 
\


We have to check that  forms a convex polygon. 
Clearly, , since
the 's  are greater than zero and . Moreover 
, because . The numerator of
 is negative and due to \eqref{equ:case1} the denominator
of  is positive. Therefore,  and  forms
a convex polygon.\\label{equ:case2}
 \ot_{35}\ot_{14}+\ot_{14}\ot_{25}+\ot_{25}\ot_{24}+\ot_{13}\ot_{35}\leq\ot_{35}\ot_{25}.
\mathbf{p}_1\!=\!
\begin{pmatrix} 0 \\ -1 \end{pmatrix},\mathbf{p}_2\!=\!\begin{pmatrix} 1 \\ y_2\end{pmatrix},
\mathbf{p}_3\!=\!\begin{pmatrix} 1 \\ y_3 \end{pmatrix} ,\mathbf{p}_4\!=\!\begin{pmatrix} 0 \\
  1\end{pmatrix},\mathbf{p}_5\!=\!\begin{pmatrix} -1 \\ 0\end{pmatrix}.
 y_2& =  -2\cdot \frac{\ot_{24}\ot_{13}+\ot_{24}\ot_{35}+
 \ot_{25}\ot_{13}+2\ot_{25}\ot_{35}-\ot_{13}^2-2\ot_{13}\ot_{35}
 -\ot_{35}\ot_{14}}
 {\ot_{24}\ot_{35}+\ot_{25}\ot_{13}+2\ot_{25}\ot_{35}}, 
 \\
 y_3&=\phantom{-} 2\cdot\frac{\ot_{24}\ot_{13}+\ot_{24}\ot_{35}+
 \ot_{25}\ot_{13}+2\ot_{25}\ot_{35}-\ot_{24}^2-2\ot_{24}\ot_{25}
 -\ot_{14}\ot_{25}}
 {\ot_{24}\ot_{35}+\ot_{25}\ot_{13}+2\ot_{25}\ot_{35}}.

\omega_{12} & =  -\ot_{24}-2\ot_{25}-\ot_{12}, \1.1ex] 
\omega_{34} & =
  -\ot_{14}-2\ot_{15}-\ot_{34},  \1.1ex]
\omega_{15} &=
  \ot_{13}-2\ot_{25}-\ot_{24}-\ot_{15}. 

-\ot_{13}^2-\ot_{35}\ot_{14}+\ot_{13}(\ot_{24}-2\ot_{35})&<0 
 \text{ and} \\
-\ot_{24}^2-\ot_{14}\ot_{25}+\ot_{24}(\ot_{13}-2\ot_{25})&<0.
\label{equ:case2convex}
\ot_{13}^2+\ot_{24}^2+2\ot_{13}\ot_{35}+2\ot_{24}\ot_{25}+ \ot_{25}\ot_{14}+\ot_{35}\ot_{14}<\\
  2\ot_{24}\ot_{35}+2\ot_{25}\ot_{13}+4\ot_{25}\ot_{35}+2\ot_{24}\ot_{13}. 
x_i=\det{\bar{L}^{(i)}} / \det{\bar{L}},
 S_x^B & = 
 {(\ot_{35}\ot_{14}+\ot_{14}\ot_{25}+\ot_{25}\ot_{24}+\ot_{13}\ot_{35}-\ot_{35}\ot_{25}})
 (\det{\bar{L}})^2,
 \\
 S_y^B & =  (\ot_{35}+\ot_{25})  \det{\bar{L}}. 
 S^B_y=(\ot_{24}\ot_{35}+\ot_{25}\ot_{13}+2\ot_{25}\ot_{35}) (\det{\bar{L}})^2.  \mathcal{F}_B(G) <  \mathcal{T}(G).\mathcal{T}(G) <  \prod_i \mathrm{deg}(v_i).
 \mathcal{T}(G)&
=\det\bar L
\le 
\prod_{i=1}^{n-1} l_{ii}=\prod_{i=1}^{n-1} \deg(v_i)

\mbox{if  is of type 3:} \quad  \mathcal{F}_B(G) <  \mathcal{T}(G) & \leq   5.\bar{3}^n, \\
\mbox{if  is of type 4:} \quad  \mathcal{F}_B(G) <  \mathcal{T}(G) & \leq  3.529988^n,   \\
\mbox{if  is of type 5A/5B:} \quad  \mathcal{F}_B(G) <  \mathcal{T}(G) & \leq  2.847263^n.
\Delta x = (\ot_{25}(\ot_{13}+\ot_{14})+\ot_{35}(\ot_{14}+\ot_{25})-(\ot_{13}-\ot_{24})^2)(\det\bar L)^3.0\leq z_i < 2n\Delta x \Delta y z_1=\langle \mathbf{a_k},\p_1 \rangle + d_k = d_k.
-d_k & \leq  \sum_{(i,j)\in\mathcal{C}}|\langle \p_i,\p_j^\bot
\rangle|
<  2n  \max\{\,\lvert\langle \p_i,\p_j^\bot\rangle\rvert \colon 1\leq i,j \leq n\}.
\langle
\p_i,\p_j^\bot\rangle = x_iy_j-x_jy_i.\setlength{\arraycolsep}{0pt}
\begin{array}{rcl}
    0  \leq{}& x_i,y_i&{} <   5.\bar{3}^n, \\
    0 \leq{}& z_i&{} <   2 n \cdot 28.\bar{4}^n. \\
  \end{array}
   0  \leq  x_i & <  2 \cdot 3{.}530^n, \\
   0  \leq  y_i & <  2 n \cdot 12{.}46	1^n, \\
   0  \leq  z_i & <  8 n^2 \cdot 43{.}987^n. 
  
   0  \leq  x_i & <   16 n^2 \cdot 23{.}083^n, \\
   0  \leq  y_i & <  2n  \cdot 8.107^n, \\
   0  \leq  z_i & <  16 n^4 \cdot 187.128^n. 
 \mathbf{p}_1=\binom{ 0 }{ 0 },\mathbf{p}_2=\binom{ 1 }{ 0},
 \mathbf{p}_3=\binom{ 1 }{ 1 } ,\mathbf{p}_4=\binom{ 0 }{
   1},\mathbf{p}_5=\binom{ -1/3 }{ 1/2}. 
  \bar{S}_x =  1\,264\,158\,727\,403\,904, \quad
  \bar{S}_y =  26\,069\,428\,512. 
|z_{3}|= 11\,083\,163\,098\,782\,678\,334\,820\,352 \approx 2^{83.19}, 
which is smaller than the bound  from Corollary~\ref{cor:3d}.

\begin{figure}[htb]
 \center 
\includegraphics{figs/dodeca-alt} 
    \caption{Two views of the dodecahedron embedded with our algorithm, with scaled -axis.
      The right picture includes also the equilibrium-stressed plane embedding.}
    \label{fig:dodecaresult}
\end{figure}

The computed embedding allows a smaller integer realization. Due to
the fact that the greatest common divisor of the -coordinates is
 and the greatest common divisor
of the -coordinates is , scaling down
by these factors yields a smaller integer embedding. We obtain an
integral plane embedding on the grid .  The
corresponding -coordinates range between 0 and .  This
reduction is due to the fact that all substitution stresses  are
equal. Thus one might have replaced them by  in the subsequent calculations.


A much smaller grid embedding of the dodecahedron was constructed by hand by
Francisco Santos.
It is centrally symmetric and fits inside a 
box, see Figure~\ref{fig:dodecahedron}(b).
It is hard to believe that a smaller realization would be possible.
Another, more symmetric, realization of the dodecahedron is the
pyritohedron (one of the possible crystal shapes of the mineral
pyrite), as pointed out to us by G\'abor G\'evay.  It fits in a
 box, see Figure~\ref{fig:dodecahedron}(c).  It has 8 vertices of the form
, plus 12 vertices, which are the 4 vertices of the form
 and their cyclic rotations of the coordinates.  The
normals of the 12 faces are the vectors  and their
cyclic rotations.

\paragraph{
Acknowledgements.}
We thank a referee for a very thorough reading of the manuscript.

\bibliographystyle{abbrv}
\bibliography{3-polytopes}





\end{document}
