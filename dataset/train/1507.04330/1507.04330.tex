In this section we describe a template for maintaining a maximal independent set (MIS). Initially, we are given a graph $G=(V,E)$ along with an MIS that satisfies certain properties, and after a topology change occurs in the graph, applying the template results in an MIS that satisfies the same properties. That is, the template describes what we do after a single topology change, and if one considers a long-lived process of topology changes, then this would correspond to having initially an empty graph and maintaining an MIS as it evolves. We emphasize that the template describes a process that is not in any particular model of computation, and later in Section~\ref{sec:dist_impl} we show how to implement it efficiently in our dynamic distributed setting. This also means that there are only four topology changes we need to consider: edge-insertion, edge-deletion, node-insertion and node-deletion. For example, the notions of abrupt and graceful node deletions are defined with respect to the dynamic distributed setting because they affect communication, and therefore the implementation of the template will have to address this distinction, but the template itself is only concerned with a single type of node deletion, not in any particular computation model.


Throughout, we assume a uniformly random permutation $\pi$ on the nodes $v \in V$. We define two \emph{states} in which each node can be: $\M$  for an MIS node, and $\NM$ for a non-MIS node. We abuse notations and also denote by $\M$ and $\NM$ the sets of all MIS and non-MIS nodes, respectively. Our goal is to maintain the following \emph{\MIS invariant}: A node $v$ is in $\M$ if and only if all of its neighbors $u \in N(v)$ which are ordered before it according to $\pi$, i.e., for which $\pi(u) < \pi(v)$, are not in $\M$. It is easy to verify that whenever the \MIS invariant is satisfied, it holds that the set $\M$ is a maximal independent set in $G$.
Furthermore, it is easy to verify that this invariant simulates the greedy sequential algorithm, as defined in the introduction.

When any of the four topology changes occurs, there is at most a single node for which the MIS invariant no longer holds. We denote this node by $v^* = v^*(\Gold,\Gnew,\pi)$, where $\Gold$ and $\Gnew$ are the graphs before and after the topology change. For an edge insertion or deletion, $v^*$ is the endpoint with the larger order according to $\pi$. For a node insertion or deletion, $v^*$ is the  node. \footnote{For a node deletion, we slightly abuse the definition of $v^*$ in order to facilitate the presentation, and consider it to be the deleted node. This means that here we consider an intermediate stage of having $v^*$ still belong to the graph w.r.t. the \MIS invariant of all the other nodes, but for $v^*$ the \MIS invariant no longer holds. This is in order to unify the four cases, otherwise we would have to consider all of the neighbors of a deleted node as nodes for which the \MIS invariant no longer holds after the topology change.}
In case the topology change is an edge change, we will need also to take into consideration its other endpoint. We denote it by $v^{**} =v^{**}(\Gold,\Gnew,\pi)$, and notice that by our notation, it must be the case that $\pi(v^{**}) < \pi(v^{*})$. In order to unify our proofs for all of the four possible topology changes, we talk about a node $v^{**}$ also for node changes. In this case we define $v^{**}$ to be $v^*$ itself, and we have that $\pi(v^{**}) = \pi(v^{*})$. Therefore, for any topology change, it holds that $\pi(v^{**}) \leq \pi(v^{*})$.

To describe our template, consider the case where a new edge is inserted and it connects two nodes $\pi(v^{**}) < \pi(v^{*})$, where both nodes are in $\M$. As a result, $v^*$ must now be deleted from the MIS and hence we need to change its state. Notice that as a result of the change in the state of $v^*$, additional nodes may need their state to be changed, causing multiple state changes in the graph.
An important observation is that it is possible that during this process of propagating local corrections of the \MIS invariant, we change the state of a node more than once. As a simple example, consider the case in which $v^*$ has two neighbors, $u_1$ and $u_2$, for which $\pi(v^*) < \pi(u_1),\pi(u_2)$, and that $u_1$ and $u_2$ are connected by a path $(u_1,w_1,w_2,u_2)$, with $\pi(u_1) < \pi(w_1) < \pi(w_2) < \pi(u_2)$. Now, when we change the state of $v^*$ to $\NM$, both $u_1$ and $u_2$ need to be changed to $\M$, for the \MIS invariant to hold. This implies that $w_1$ needs to be changed to $\NM$ and $w_2$ needs to be changed to $\M$. In this case, since $\pi(w_2) < \pi(u_2)$, the node $u_2$ needs to be changed back to state $\NM$.


The above observation leads us to define a set of \emph{influenced} nodes, denoted by $S=S(\Gold,\Gnew,\pi)$, containing $v^*$ in the scenario where we need to change its state, and all other nodes whose state we must  subsequently change as a result of the state change of $v^*$.
To formally define the set $S$ we introduce some notations. The notations rely on the graph structure of $\Gnew$ unless the change is a node deletion in which case the rely on $\Gold$.
For each node $u$, we define $I_{\pi}(u) = \{ v \in N(u) ~|~ \pi(v) <  \pi(u)\}$, the set of neighbors of $u$ that are ordered before it according to $\pi$.
These are the nodes that can potentially \emph{influence} the state of $u$ according to the \MIS invariant. The definition of $S$ is recursive, according to the ordering induced by $\pi$.
If immediately after the topology change, in the new graph $G$ with the order $\pi$ it holds that the \MIS invariant still holds for $v^*$, then we define $S=\emptyset$. (This is motivated by the fact that no node is influenced by this change.) Otherwise, we denote $S_0 = \{v^*\}$, and inductively define
\begin{equation} \label{eq:defS}
S_{i} = \{u ~|~ u \in \M \mbox{, and } S_{i-1}\cap I_{\pi}(u)\neq\emptyset\} \cup \{u ~|~ u \in \NM \mbox{, and every }  v \in I_{\pi}(u) \cap \M \mbox{ is in }\cup_{j=0}^{i-1}{S_j}) \}.
\end{equation}

The set $S$ is then defined as $S=\bigcup_i{S_i}$. Notice that a node $u$ can be in more than one set $S_i$, as is the case for $u_2$ in the example above, which is in both $S_1$ and $S_4$. The impact of a node $u$ being in more than one $S_i$ is that in order to maintain the \MIS invariant, we need to make sure that we update the state of $u$ after we update that of $w$, for any $w$ such that $w \in I_{\pi}(u)$. Instead of updating the state of $u$ twice, we can simply wait and update it only after the state of every such $w$ is updated. For this, we denote by $i_u = \max\{i~|~ u \in S_i\}$ the maximal index $i$ for which $u$ is in $S_i$.

\begin{algorithm}
Initially, $G=(V,E)$ satisfies the \MIS invariant.\\
On topology change at node $v^*$ do:\\
1. Update state of $v^*$ if required for \MIS to hold\\
2. For $i \leftarrow 1$, until $S_{i} = \emptyset$, do:\\
3. \quad For every $u \in S_{i}$ such that $i = i_u$:\\
4. \quad\quad Update state of $u$\\
5. \quad $i \leftarrow i+1$
\caption{A Template for Dynamic Correlation Clustering.}
\label{alg:template}
\end{algorithm}


We formally describe our template in Algorithm~\ref{alg:template}.
By construction, the updated states after executing Algorithm~\ref{alg:template} satisfy the \MIS invariant.
In addition, the crucial property that is satisfied by the above template is that in expectation, the size of the set $S$ is $1$. The remainder of this section is devoted to proving the following, which is our main technical result.

\begin{thm}
\label{thm:ES-const}
For every two graphs $\Gold$ and $\Gnew$ that differ only by a single edge or a single node, it holds that  
$\E_\pi \left[|S(\Gold, \Gnew, \pi)|\right] \leq 1$. 
\end{thm}


\paragraph{Outline of the proof:} In order to prove that $\E[|S|] \leq 1$, 
instead of analyzing the set $S$ directly, we analyze the set $S' = S'(\Gold,\Gnew,\pi,v^*)$,  which is defined via recursion similarly to $S$ with three modifications: (1) It is always the case that $S_0' = \{v^*\}$ (2) The graph according to which $S'$ is defined is $\Gold$ in the case of a node deletion or an edge insertion, and $\Gnew$ otherwise. (3) The permutation according to $S'$ is defined as $\pi'$, that is identical to $\pi$ other than its value for $v^*$ that is forced to be the minimal among all other $\pi$ values. Notice that $S'$ does not depend on $\pi(v^*)$ and in particular, having knowledge about its elements does not give any information as to whether $\pi(v^*) < \pi(v^{**})$ or vice versa. 






In Lemma~\ref{lem:SandS'}, we prove that if $\pi(v^*) \neq \min\left\{ \pi(u)\mid u\in S'\right\}$ then $S=\emptyset$, and otherwise $S=S'$ (in fact, it would be enough that $S\subseteq S'$). Then, in Lemma~\ref{lem:one-over-P}, we prove that for any set $P \subseteq V$, given the event that $P=S'$, the probability, over the random choice of $\pi$, that $\pi(v^*) =\min\left\{ \pi(u)\mid u\in P\right\}$ is $1/|P|$. This leads to the required result of Theorem~\ref{thm:ES-const}.
Lemma~\ref{lem:one-over-P} would be trivial if there was no correlation between $\pi$ and $S'$. However, the trap we must avoid here is that $S'$ is defined according to $\pi$, and therefore when analyzing its size we cannot treat $\pi$ as a \emph{uniformly} random permutation. To see why, suppose we know that inside $S'\setminus\{v^*\}$ we have nodes with large order in $\pi$. Then the probability that the order of $v^*$ in $\pi$ is smaller than all nodes in $S'\setminus\{v^*\}$, is much larger than $1/|S'|$, and can in fact be as large as $1-o(1)$. In other words, $S'$ gives some information over $\pi$. Nevertheless, we show that this information is either about the order between nodes outside of $S'$, or about the order between nodes within $S'\setminus\{v^*\}$. Both types of restrictions on $\pi$ do not affect the probability that $v^*$ is the minimal of $S'$.



We now formally prove our result as outlined above. Throughout we use the notation $u \in \M$ or $u \in \NM$. This applies only to nodes $u$ for which we are guaranteed that their states remain the same despite the topology change.
\begin{lem}
\label{lem:SandS'}
If $\pi(v^*)\neq\min\left\{ \pi(u)\mid u\in S'\right\} $ then $S=\emptyset$. Otherwise, $S\subseteq S'$.
\end{lem}
\begin{proof}
First, assume that $\pi(v^*)\neq\min\left\{ \pi(u)\mid u\in S'\right\}$. We show that the \MIS invariant still holds after the topology change, and so $S=\emptyset$.
Consider the node $w$, for which $\pi(w)=\min\left\{ \pi(u)\mid u\in S'\right\}$. Notice that $w\not\in S$, because $\pi(w) < \pi(v^*)$. We claim that $w \in \M$. Assume, towards a contradiction, that $w \in \NM$. This implies that $w$ has a neighbor $u \in \M$ such that $\pi(u) < \pi(w)$. For this node $u$ we must have $u \notin S'$ due to the minimality of $\pi(w)$. It follows, according to the construction of $S'$ that $w$ cannot be an element of $S'$, leading to a contradiction.

We have that $w \in \M$ and due to the minimality of $\pi(w)$, it must be that $w \in S'_1$, which implies that $w$ is a neighbor of $v^*$. But then, when considering $S$, $v^*$ has a neighbor other than $v^{**}$ which is ordered before it according to $\pi$ which is in $\M$. In the case of an edge insertion or deletion, this means that $v^{*}$ remains in $\NM$ despite the topology change meaning that $S=\emptyset$. In the case of a node deletion, $v^*$ was not in $\M$ prior to the change hence $S=\emptyset$. In the case of a node insertion, $v^*$ does not enter $\M$ hence again, $S=\emptyset$.

Next, assume that $\pi(v^*)=\min\left\{ \pi(u)\mid u\in S'\right\}$. We show that either $S=\emptyset$ or $S=S'$.
If there is no need to change the state of $v^*$ as a result of the topological change then $S_0=\emptyset$, and so $S=\emptyset$ and the claim holds. It remains to analyze the case where $S_0=S'_0=\{v^*\}$. 
If $u \in S'_1$ then $\pi(v^*) < \pi(u)$ hence according to its definition $u \in S_1$. If $u \notin S'_1$ then $u$ must have a neighbor $w \in \M$ with $\pi(w)<\pi(u)$ meaning that $u \notin S_1$. We have that $S_1=S'_1$ and similarly $S_i=S'_i$ for all $i >1$. We conclude that $S'=S$ as required.
\end{proof}

The following lemma shows that the probability of having $S=S'$ is $1/|S'|$, which immediately lead to Theorem~\ref{thm:ES-const} as the only other alternative is $S=\emptyset$. \begin{lem}
\label{lem:one-over-P}
For any set of nodes $\S\subseteq V$, it holds that $$\Pr\left[\pi(v^*)=\min\left\{ \pi(u)\mid u\in \S\right\} \mid S'=\S \mbox{ and } \pi(v^{**})\leq\pi(v^{*}) \right]=\frac{1}{|\S|}.$$
\end{lem}


To prove this lemma we focus on $S'$. Notice that the events we considered in the previous lemma depend only on the ordering 
implied by $\pi$ and hold for any configuration of states for the nodes that satisfy the \MIS invariant. Roughly speaking, the lemma 
will follow from the fact the the event $S'=\S$ does not give any information about the order implied by $\pi$ between nodes in $\S$ 
and nodes in $\barS$. To this end, for every permutation $\tau$ on $V$, we define $S'(\tau) = S'(\Gold,\Gnew,\tau,v^*)$ as the set corresponding to $S'$ under the ordering induced by $\tau$. 
We denote by $\Pi_{\S}$ the set of all permutations $\tau$ for which it holds that $S'(\tau)=\S$. We first need to establish the 
following about permutations in $\Pi_{\S}$: If $\pi$ and $\sigma$ are two permutations on $V$ such that $\pi|_{\S}=\sigma|_{\S}$ 
and $\pi|_{\barS}=\sigma|_{\barS}$, then $\sigma\in\Pi_{\S}$ if and only if $\pi\in\Pi_{\S}$.



\begin{claim}
\label{claim:u-in-barS}
Let $P \subseteq V$ be a set of nodes, and let $\pi$ and $\sigma$ be two permutations such that $\pi|_{\S}=\sigma|_{\S}$ and $\pi|_{\barS}=\sigma|_{\barS}$. Assume $\pi \in \Pi_\S$. We have that $\barS \subseteq V \setminus S'(\sigma)$ and every $u \in \barS$ has the same state according to $\pi$ and $\sigma$.
\end{claim}

\begin{proof}
Let $u \in \barS$. We prove that $u \in V \setminus S'(\sigma)$ and that its state under $\sigma$ is the same as it is under $\pi$ by induction on the order of nodes in $\barS$ according to $\pi$ (which is equal to their order according to $\sigma$).



For the base case, assume that $u$ has the minimal order in $\barS$.
We claim that $u$ cannot have a neighbor in $\S$.
Assume, towards a contradiction, that $u$ has a neighbor $w\in\S$. Since $w\in\S$ then it is possible that after the a $w$ will be in $\M$. Since two nodes in $\M$ cannot be neighbors and $u \not\in\S$, then $u$ must be in $\NM$ according to $\pi$. In this case there is a node $z\in I_{\pi}(u)\cap\barS$ that is in $\M$ according to $\pi$. But this cannot occur due to the minimality of $\pi(u)$ in $\barS$. Therefore, $u$ has no neighbors in $\S$ as required.

We have that all of the neighbors of $u$ are in $\barS$ and that $u$ is the minimal among its neighbors according to $\pi$. Since $\pi|_{\barS}=\sigma|_{\barS}$ we have that $u$ has the minimal order among its neighbors according to $\sigma$. This translates into $u$ having a state of $M$ under $\sigma$ and in particular, $u$ is not an element of $S'(\sigma)$, thus proving our base case.


For the induction step, consider a node $u\in\barS$, and assume the claim holds for every $w \in \barS \cap I_{\pi}(u)$. We consider two cases, depending on whether $u$ has a neighbor in $\S$ or not.

Case 1:  $u$ does not have any neighbor in $\S$.
If $u \in \NM$, then there is a node $z\in I_{\pi}(u)\cap \barS$ that is in $\M$ according to $\pi$. By the induction hypothesis, $z \in V \setminus S'(\sigma)$ and $z \in \M$ also according to $\sigma$. Since $\pi|_{\barS}=\sigma|_{\barS}$, we have that $u$ is in $\NM$ according to $\sigma$ too.
Otherwise, if $u \in \M$, then every $w\in I_{\pi}(u)$ (which is also $\barS$) is in $\NM$ according to $\pi$.
Any node $w \in I_{\sigma}(u)$ is also in $w \in I_{\pi}(u)$, since it is not in $\S$ and $\pi|_{\barS}=\sigma|_{\barS}$. The induction hypothesis on $w$ gives that it is also in $V \setminus S'(\sigma)$ (otherwise it would be in $S'_{\pi}=\S$ in contradiction to the assumption of case 1), and its state according to $\sigma$ is $\NM$. Hence, $u$ must be in $V \setminus S'(\sigma)$ as well, and in state $\M$ according to $\sigma$.


Case 2: Assume that $u$ has a neighbor $w\in\S$. Since $w\in\S$ then it is possible that after the algorithm $w$ will be in $\M$. Since two nodes in $\M$ cannot be neighbors and $u \not\in\S$, then $u$ must be in $\NM$ according to $\pi$. In this case there is a node $z\in I_{\pi}(u)\backslash\S$ that is in $\M$ according to $\pi$. By the induction hypothesis, $z \in V \setminus S'(\sigma)$ and $z \in \M$ also according to $\sigma$. Since $\pi|_{\barS}=\sigma|_{\barS}$, we have that $u$ is in $\NM$ and in $V \setminus S'(\sigma)$ according to $\sigma$ too.
\end{proof}


\begin{claim}
\label{claim:sigma-pi}
Let $P \subseteq V$ be a set of nodes, and let $\pi$ and $\sigma$ be two permutations such that $\pi|_{\S}=\sigma|_{\S}$ and $\pi|_{\barS}=\sigma|_{\barS}$. Assume $\pi \in \Pi_\S$. We have that $\S \subseteq S'(\sigma)$.
\end{claim}
\begin{proof}
We prove that every node $u \in \S$ is also in $S'(\sigma)$ by induction on the order of nodes in $\S$ according to $\pi$ (which is equal to their order according to $\sigma$), with the modification forcing $v^*$ to be the first among the nodes of $P$.
The base case is for $v^*$, which is clearly in both sets $S'(\pi)$ and $S'(\sigma)$.
Consider a node $u \in \S$ and assume that the claim holds for every node in $\S$ which is ordered before $u$ according to $\pi$.
Since $u \in \S$ and $u \neq v^*$ there must be some $w \in I_\pi(u) \cap \S$ since $\pi|_{\S}=\sigma|_{\S}$ and $u \in \S$ we have according to our induction hypothesis that $w \in S'(\sigma)$, meaning that $I_\sigma(u) \cap S'(\sigma)$ is non-empty.

Consider now an arbitrary $w \in I_\sigma(u)$. If $w \in \S$ then since $\pi|_{\S}=\sigma|_{\S}$ and $u \in \S$ we have according to our induction hypothesis that $w \in S'(\sigma)$. If $w \notin \S$ then it must be the case that $w \in \NM$ according to $\pi$, otherwise $u$ cannot be in $\S$. We thus have according to Claim~\ref{claim:u-in-barS} that (1) $w \in V \setminus S'(\sigma)$ and (2) $w \in \NM$ according to $\sigma$. It follows that all neighbors of $u$ in $I_\sigma(u)$ are either in $S'(\sigma)$ or in $\NM$ according to $\sigma$, hence since $I_\sigma(u) \cap S'(\sigma) \neq \emptyset$ it must be the case that $u \in S'(\sigma)$.
\end{proof}





Claims~\ref{claim:u-in-barS} and~\ref{claim:sigma-pi} combined imply that if $\pi|_{\S}=\sigma|_{\S}$ and $\pi|_{\barS}=\sigma|_{\barS}$ then $\sigma \in \Pi_{\S}$ if and only if $\pi \in \Pi_{\S}$. We are now ready for the proof of Lemma~\ref{lem:one-over-P}.
\begin{proof}
\emph{(of Lemma~\ref{lem:one-over-P})~}
Given two permutations $\sigma^{+}$ and $\sigma^{-}$ on $\S\backslash\{v^*\}$ and $\barS$, respectively, we define $\rho_{\sigma^{+},\sigma^{-}}$ as 
$\rho_{\sigma^{+},\sigma^{-}} = \Pr\left[\forall u\in\S, \pi(v^*) \leq \pi(u) \mid\pi|_{\S\backslash\{v^*\}}=\sigma^{+} \mbox{ and } \pi|_{\barS}=\sigma^{-}\right].$



First, we observe that for two pairs of permutations $\sigma_{1}^{+},\sigma_{1}^{-}$ and $\sigma_{2}^{+},\sigma_{2}^{-}$ as above, it holds that $\rho_{\sigma_{1}^{+},\sigma_{1}^{-}}=\rho_{\sigma_{2}^{+},\sigma_{2}^{-}}$. This is because given the condition for $\sigma_{1}^{+},\sigma_{1}^{-}$, applying the permutation $(\sigma_{1}^{+})^{-1}\sigma_{2}^{+}$ to nodes in $\S\backslash\{v^*\}$ and applying the permutation $(\sigma_{1}^{-})^{-1}\sigma_{2}^{-}$ to nodes in $\barS$ has no affect on whether the event $\forall u\in\S, \pi(v^*)\leq\pi(u)$ holds.
Next, since $\Pr\left[\forall u\in\S, \pi(v^*)\leq\pi(u)\right]=\frac{1}{|\S|}$, we have that for any pair of permutations $\sigma^{+},\sigma^{-}$ on $\S\backslash\{v^*\}$ and $\barS$, respectively:
\vspace{-0.1in}
\begin{eqnarray*}
\frac{1}{|\S|} & = & \Pr\left[\forall u\in\S, \pi(v^*)\leq\pi(u)\right]=\sum_{\tau^{+},\tau^{-}}\rho_{\tau^{+},\tau^{-}}\Pr\left[\pi|_{\S\backslash\{v^*\}}=\tau^{+}\mbox{ and }\pi|_{\barS}=\tau^{-}\right]\\
 & = & \sum_{\tau^{+},\tau^{-}}\rho_{\sigma^{+},\sigma^{-}}\Pr\left[\pi|_{\S\backslash\{v^*\}}=\tau^{+}\mbox{ and }\pi|_{\barS}=\tau^{-}\right]=\rho_{\sigma^{+},\sigma^{-}}.
\end{eqnarray*}

Finally, Claims~\ref{claim:u-in-barS} and~\ref{claim:sigma-pi} imply that for every set $\S\subseteq V$ there is a set of $t=t_{\S}$ pairs of permutations $\{(\sigma_{1}^{+}, \sigma_{1}^{-}),\dots,(\sigma_{t}^{+}, \sigma_{t}^{-}) \}$ on $\S\backslash\{v^*\}$ and $\barS$, respectively, such that $\Pi_{\S}=\{\pi\mid \exists i, \pi|_{\S\backslash\{v^*\}}=\sigma_{i}^{+}\mbox{ and }\pi|_{\barS}=\sigma_{i}^{-}\}$. We conclude that for a given set $\S\subseteq V$:
\vspace{-0.2in}
\begin{eqnarray*}
\Pr_{\pi\in\Pi_{\S}}\left[\forall u\in\S, \pi(v^*)\leq\pi(u)\right] & = & \sum_{i=1}^{t}\rho_{\sigma_{i}^{+},\sigma_{i}^{-}}\Pr\left[\pi|_{\S\backslash\{v^*\}}=\sigma_{i}^{+}\mbox{ and }\pi|_{\barS}=\sigma_{i}^{-}\mid\pi\in\Pi_{\S}\right]
\end{eqnarray*}
\vspace{-0.25in}
\begin{eqnarray} \label{eq:vs_min}
& = & \frac{1}{|\S|}\sum_{i=1}^{t}\Pr\left[\pi|_{\S\backslash\{v^*\}}=\sigma_{i}^{+}\mbox{ and }\pi|_{\barS}=\sigma_{i}^{-}\mid\pi\in\Pi_{\S}\right]=\frac{1}{|\S|}.
\end{eqnarray}
To complete the proof, we argue that knowing that $\pi(v^{**}) \leq \pi(v^*)$ can only decrease the probability that $\pi(v^*)\leq\pi(u)$ for all $u \in \S$. Formally,
\begin{eqnarray*}
 &  & \Pr_{\pi\in\Pi_{\S}}\left[\forall u\in\S,\pi(v^{*})\leq\pi(u)~|~\pi(v^{**})\leq\pi(v^{*})\right]\\
 & = & \Pr_{\pi\in\Pi_{\S}}\left[\forall u\in\S,\pi(v^{*})\leq\pi(u)\mbox{ and }\pi(v^{**})\leq\pi(u)~|~\pi(v^{**})\leq\pi(v^{*})\right]\\
& = & \frac{\Pr_{\pi\in\Pi_{\S}}\left[\forall u\in\S,\pi(v^{*})\leq\pi(u)\mbox{ and }\pi(v^{**})\leq\pi(u) \mbox{ and }    \pi(v^{**})\leq\pi(v^{*})     \right]}{\Pr_{\pi\in\Pi_{\S}}\left[\pi(v^{**})\leq\pi(v^{*})\right]}\\
 & \leq & \frac{\Pr_{\pi\in\Pi_{\S}}\left[\forall u\in\S,\pi(v^{*})\leq\pi(u)\mbox{ and }\pi(v^{**})\leq\pi(u)\right]}{\Pr_{\pi\in\Pi_{\S}}\left[\pi(v^{**})\leq\pi(v^{*})\right]}
\end{eqnarray*}
To bound the above expression, we separate our discussion into three possible cases. In the first $v^{**} \neq v^*$ and $v^{**} \in P$. The value of the expression is clearly 0 in this case. In the second we have $v^* = v^{**}$ and according to Equation~\eqref{eq:vs_min} we have that the quantity is bounded by $1/|P|$. The last case is the one where $v^{**} \notin P$. Here, because $v^* \in P$ and $v^{**} \notin P$ we have that the events of $\pi(v^*)$ being the minimal in $\{\pi(u)\}_{u\in\S}$ and $\pi(v^{**})$ being the smaller than each of the elements of $\{\pi(u)\}_{u\in\S}$ are independent for uniform $\pi \in \Pi_{\S}$. This is due to the first event being dependent of the inner order inside $\S$ and the second being independent of the same inside order.
Hence,
\begin{eqnarray*} 
 &  & \frac{\Pr_{\pi\in\Pi_{\S}}\left[\forall u\in\S,\pi(v^{*})\leq\pi(u)\mbox{ and }\pi(v^{**})\leq\pi(u)\right]}{\Pr_{\pi\in\Pi_{\S}}\left[\pi(v^{**})\leq\pi(v^{*})\right]} \\
  & = & \frac{\Pr_{\pi\in\Pi_{\S}}\left[\forall u\in\S,\pi(v^{**})\leq\pi(u)\right]}{\Pr_{\pi\in\Pi_{\S}}\left[\pi(v^{**})\leq\pi(v^{*})\right]} \cdot \Pr_{\pi\in\Pi_{\S}}\left[\forall u\in\S,\pi(v^{*})\leq\pi(u)\right] \\
 & \leq & \Pr_{\pi\in\Pi_{\S}}\left[\forall u\in\S,\pi(v^{*})\leq\pi(u)\right]  \leq 1/|P|
\end{eqnarray*}
\vspace{-0.2in}

\end{proof}
\vspace{-0.1in}

Lemma~\ref{lem:SandS'} and Lemma~\ref{lem:one-over-P} immediately lead to Theorem~\ref{thm:ES-const}. Also, as an immediate corollary of Theorem~\ref{thm:ES-const} we get

\begin{corollary}
A direct distributed implementation of Algorithm~\ref{alg:template} has, in expectation, both a single adjustment and round, in both the synchronous and asynchronous models.
\end{corollary}
