\let\normalvec\vec
\documentclass[oribibl,envcountsame,envcountsect]{llncs}
\let\vec\normalvec

\usepackage{amsmath,amssymb}
\usepackage[english]{babel}
\usepackage[T1]{fontenc}
\usepackage[ruled,noend,linesnumbered]{algorithm2e}
\usepackage{xspace}
\usepackage{graphicx}
\usepackage{subfigure}


\newenvironment{alg}{
  \begin{algorithm}[htbp]
    \SetKwInOut{Input}{input}
    \SetKwInOut{Output}{output}
  }{
  \end{algorithm}
}

\newcommand{\mc}[1]{\ensuremath{\mathcal{#1}}}
\newcommand{\bo}[1]{\mathcal{O}(#1)}
\newcommand{\conv}{\mathsf{conv}}
\newcommand{\real}{\mathbb{R}}
\newcommand{\nat}{\mathbb{N}}
\newcommand{\algo}[1]{\ensuremath{\mathcal{A}_{#1}}}

\newcommand{\Soberon}{Sober\'{o}n}
\newcommand{\GarciaColin}{Garc\'{i}a-Col\'{i}n}
\newcommand{\Caratheodory}{Carath\'{e}odory}

\title{Algorithms for Tolerant Tverberg Partitions}
\author{Wolfgang Mulzer\thanks{Supported in part by DFG Grants MU 3501/1 and MU 3501/2.} \and Yannik Stein\thanks{
    Supported by the Deutsche Forschungsgemeinschaft within the research
    training group `Methods for Discrete Structures' (GRK 1408).
    }
}
\institute{ Institut f\"ur Informatik, Freie Universit\"at Berlin\\
            Takustr. 9, 14195 Berlin, Germany\\
            \email{\{mulzer, yannikstein\}@inf.fu-berlin.de}
          }
\pagestyle{plain}

\begin{document}
\maketitle

\begin{abstract}
  Let  be a -dimensional -point set. A partition  of 
  is called a \emph{Tverberg} partition if the convex hulls of all sets
  in  intersect in at least one point. We say that  is
  \emph{-tolerant} if it remains a Tverberg partition after deleting
  any  points from . Sober\'{o}n and Strausz proved that there is always a
  -tolerant Tverberg partition with  sets.
  However, no nontrivial algorithms for computing or approximating
  such partitions have been presented so far.

  For , we show that the Sober\'{o}n-Strausz bound can be improved,
  and we show how the corresponding partitions can be found in polynomial time.
  For , we give the first polynomial-time approximation algorithm
  by presenting a reduction to the regular Tverberg problem (with no
  tolerance).
  Finally, we show that it is coNP-complete to determine whether a given
  Tverberg partition is -tolerant.

  \keywords{Tverberg partition; tolerant Tverberg partition; high-dimensional
    approximation; coNP-completeness.}
\end{abstract}

\section{Introduction}
Let  be a point set of size . A point  has
\emph{(Tukey) depth}   with respect to  if every closed half-space containing
 also contains at least  points from . A point of depth  is called a \emph{centerpoint} for . The well-known centerpoint
theorem~\cite{Rado1946} states that every point set has a centerpoint. Centerpoints
are of great interest since they constitute a natural generalization of the median to
higher-dimensions and are invariant under scaling or translations
and robust against outliers.

Chan~\cite{Chan2004} described a randomized algorithm that
finds a -dimensional centerpoint in expected time
. In fact, Chan solves the seemingly harder problem of finding a
point with maximum depth, and he conjectures that his result is optimal.
Since this is infeasible in higher
dimensions, approximation algorithms are of interest. Already in 1993, Clarkson et
al.~\cite{Clarkson1996} developed a Monte-Carlo algorithm that finds a point
with depth  in time , where  is the error-probability.
Teng~\cite{Teng1992} proved that it is coNP-complte to test whether
a given point is a centerpoint. Thus, we do not know how to verify
efficiently the output of the algorithm by Clarkson et al. For a subset
of centerpoints, \emph{Tverberg partitions}~\cite{Tverberg1966} provide
polynomial-time checkable proofs for the depth: a \emph{Tverberg -partition} for a
point set  is a partition
 of  into  sets such
that .  Each half-space that
intersects  must contain at least one
point from each , so each point in  has
depth at least .
Tverberg's theorem states that depth  can always be
achieved.
Thus, there is always a centerpoint with a corresponding Tverberg partition.
Miller and Sheehy~\cite{Miller2010} developed
a deterministic algorithm that computes a
point of depth  in time  together
with a corresponding Tverberg partition. This was recently improved by Mulzer
and Werner~\cite{Mulzer2013}. Through recursion on the dimension, they can find
a point of depth  and a corresponding Tverberg partition
in time .

\begin{figure}[htbp]
  \newcommand{\imgwidth}{0.42\textwidth}
  \begin{center}
    \subfigure[-tolerant Tverberg -partition]{
        \label{intro:fig:tolex:untol}\includegraphics[width=\imgwidth]{img/ut_vs_tol1.eps}
    }
    \hspace{1cm}
    \subfigure[-tolerant Tverberg -partition]{\label{intro:fig:tolex:tol}\includegraphics[width=\imgwidth]{img/ut_vs_tol2.eps}}
  \end{center}
  \caption{
    \subref{intro:fig:tolex:untol} A regular Tverberg partition
    for 13 points. One set of the partition consists of
    a single point. The removal of any point would separate the
    convex hulls.
    \subref{intro:fig:tolex:tol} A -tolerant Tverberg partition
    for the same point set. The Tverberg partition is not
    -tolerant since the removal of both white points would
    separate the convex hulls.
  }
  \label{intro:fig:tolex}
\end{figure}

Let  be a Tverberg -partition for . If any nonempty subset
 is removed from , we no longer know if . In the worst-case, the maximum number of
sets in  whose convex hulls still have a nonempty intersection is
. Thus, after removing only  points, the convex hulls of sets in 
could be pairwise disjoint and do not longer serve as a depth-certificate for
points in the intersection of the convex hulls. In order to give stronger
guarantees if
points are removed, we study Tverberg
partitions that remain Tverberg partitions even after removing  arbitrary
points from . We call a Tverberg partition \emph{-tolerant}
if  is nonempty for any subset  of size at most . To distinguish tolerant Tverberg partitions from
Tverberg partitions with no tolerance, we  call the latter \emph{regular}
Tverberg partitions. Furthermore, we say that a set  \emph{separates} the convex hulls of the
sets in  if . See
Figure~\ref{intro:fig:tolex} for two examples.
In 1972, Larman~\cite{Larman1972} proved that
every set of size  admits a -tolerant Tverberg
-partition.  This was motivated by a problem proposed by
McMullen: find the largest number  such that any -point set
can be made convex by applying a permissible projective
transformation. Here, permissible means
that no point in the set is mapped to a point at infinity.
Larman also showed that there are sets of size 
that do not have a -tolerant Tverberg
-partition if . This lower bound was later improved by
Ram\'irez Alfons\'in~\cite{Alfonsin2001} to
 for . \GarciaColin{}~\cite{GarciaColin2007} generalized
Larman's upper bound, showing that sets of size  always have a
-tolerant Tverberg -partition, and asked for a general bound to guarantee the
existence of -tolerant Tverberg -partitions. Later, Montejano and
Oliveros~\cite{Montejano2011} conjectured that every set of size  admits a
-tolerant Tverberg -partition. This
was proved by \Soberon{} and Strausz~\cite{Soberon2012} who adapted Sarkaria's
proof of Tverberg's theorem~\cite{Sarkaria1992} to the tolerant setting:

\begin{theorem}[\Soberon{}-Strausz-Theorem~\cite{Soberon2012}]\label{thm:sobstrau}
  Let  be a set of size . Then, there exists
  a -tolerant Tverberg -partition for .

  Equivalently, for all -point sets there
  exists a -tolerant Tverberg -partition.
\end{theorem}

\Soberon{} and Strausz~\cite{Soberon2012} also
conjectured this bound to be tight. A lower bound was recently
proven by \Soberon{}~\cite{Soberon2014}: at least
 points are necessary to guarantee the
existence of a -tolerant Tverberg -partition.

So far,
no exact or approximation algorithms for tolerant Tverberg partitions appear in
the literature.

\paragraph{Our contribution.}
\newenvironment{prfTODO}{\begin{proof}[TODO]}{\qed\end{proof}}

In Section~\ref{sec:ld}, we consider the problem of computing
tolerant Tverberg partitions in low dimensions. We present an
algorithm for the one-dimensional case and use a dimension-reduction
argument to extend the algorithm to multidimensional input:
\begin{theorem}\label{thm:dr_tverberg}
  Given a set  of size ,
  a -tolerant Tverberg -partition
  for  can be computed in time .
\end{theorem}
For , the bound on the number of points is tight and improves
the \Soberon{}-Strausz bound from Theorem~\ref{thm:sobstrau} by
. For , the new bound improves the bound
by \Soberon{} and Strausz for large enough  and .

For higher dimensions, we describe in Section~\ref{sec:redregtver} an
approximation-preserving reduction to the regular Tverberg problem based on a
lemma by \GarciaColin{}. Thus, we
can apply existing and possible future algorithms for the regular Tverberg
problem in the tolerant setting:

\begin{proposition} \label{apr:prop:cb}
  Let  and let  be an
  algorithm that computes a regular Tverberg -partition
  for any point set of size  in time .
  Then, a -tolerant Tverberg
  -partition for  can be computed in time
  .
\end{proposition}

Finally, we show in Section~\ref{sec:complexity} that it is
coNP-complete to determine whether a given Tverberg partition has
tolerance  if the dimension is part of the input:
\begin{theorem}\label{thm:complexity}
  \textsc{TestingTolerantTverberg} is coNP-complete.
\end{theorem}
This holds even if we restrict the input to Tverberg partitions of size 2.

\section{Low Dimensions} \label{sec:ld}
We start with an algorithm for the one-dimensional case that yields a
tight bound. This can be bootstrapped to higher dimensions with a lifting approach
similar to the algorithm by Mulzer and Werner~\cite{Mulzer2013}. In two dimensions, we also get
an improved bound if the size of the desired partition and the
tolerance is large enough.

\subsection{One Dimension}
Let  with , and let
 be a -tolerant Tverberg -partition of . By
definition, there is no
subset  whose removal separates the convex hulls of the
sets in . Bounding the size of the sets in \mc{T} gives us more insight
into the structure.

\begin{lemma}\label{lem:1d_sizes}
  Let  with  and let  be
  a -tolerant Tverberg -partition of . Then,
  \begin{enumerate}
    \renewcommand{\labelenumi}{(\roman{enumi})}
    \item for , we have ; and
    \item for , , we have .
  \end{enumerate}
\end{lemma}
\begin{proof}
  \begin{enumerate}
    \renewcommand{\labelenumi}{(\roman{enumi})}
    \item Suppose . After removing  from , the
          intersection of the convex hulls of the sets in  becomes empty,
          and  would not be -tolerant.
    \item Suppose there are   with . By , we have .
    Let  and assume w.l.o.g. that 
    (see Figure~\ref{fig:1d_sizes}).
    Then , and removing the set  separates the convex hulls of  and .
    This again contradicts   being -tolerant.
  \end{enumerate}
\end{proof}

\begin{figure}[htbp]
  \begin{center}
    \includegraphics{img/1d_lowerbound.eps}
  \end{center}
  \caption{The convex hulls of two sets of size  can be separated by
    removing  points.}
  \label{fig:1d_sizes}
\end{figure}

Lemma~\ref{lem:1d_sizes} immediately implies a lower bound on the size of any point set
that admits a -tolerant Tverberg -partition.

\begin{corollary}
  \label{ld:cor:lb}
  Let  with . Then  has no -tolerant
  Tverberg -partition.
\end{corollary}

Now what happens for ? Note that for  and , we
have , the bound by \Soberon{} and Strausz.
Thus, proving that a -tolerant Tverberg -partition exists for any
one-dimensional point set of size  would disprove the conjecture by
\Soberon{} and Strausz.

Let  be of size . By Lemma~\ref{lem:1d_sizes}, in any
-tolerant Tverberg partition of , one set has to be of size  and all
other sets have to be of size . Let  be
a Tverberg -partition of  such that  contains every th
point of  and each other set  () has one point in each interval
defined by the points of ; see Fig.~\ref{ld:fig:ttp} for  and .
Note that  and  for .
We will show that  is -tolerant.
Intuitively, 
maximizes the interleaving of the sets, making the convex hulls more
robust to removals.

\begin{figure}[htbp]
  \begin{center}
    \includegraphics[width=0.7\textwidth]{img/1d_ttp.eps}
  \end{center}
  \caption{A -tolerant Tverberg -partition for  ()
    points.}
  \label{ld:fig:ttp}
\end{figure}

\begin{lemma}
  \label{ld:lem:1d}
  Let  with , and let
   be an -partition of .
  Suppose that , and write , sorted
  from left to right. Suppose that each interval ,
  , , contains one point
  from each , for . Then 
  is a -tolerant Tverberg -partition for .
\end{lemma}
\begin{proof}
  Suppose there exist , , and a subset  of size  such that removing   from  separates the convex hulls of
   and .  Let  be a point that separates 
  and .  Define  and
  , where  if all points in 
  are greater than  and  if all points in  are less than .
  Let  and , and define ,  similarly.
  Since removing  separates the convex hulls of  and  at , 
  must contain either  or . Figure~\ref{ld:fig:lem1dproof} shows the situation. By
  construction, we know that  and . We
  thus have . However, since
  , this is a contradiction.

  Thus, even after removing  points, the convex hulls of the sets in 
  intersect pairwise. Helly's theorem~\cite{Matouvsek2002} now
  guarantees that the convex hulls of all sets in  have a common
  intersection point. Hence,  is -tolerant.
\end{proof}

\begin{figure}[htbp]
  \begin{center}
    \includegraphics[width=0.6\textwidth]{img/1d_proof.eps}
  \end{center}
  \caption{The convex hulls of two elements in \mc{T} are separated after the
    removal of . Crosses mark removed points (i.e., points in ).
    Points not used in the proof of Lemma~\ref{ld:lem:1d} are left out for
    clarification.}
  \label{ld:fig:lem1dproof}
\end{figure}


Lemma~\ref{ld:lem:1d} immediately gives a way to compute a -tolerant Tverberg
-partition in  time for  by sorting
. However, it is not necessary to know the order of all of .
Algorithm~\ref{ld:alg:1d} exploits this fact to improve the running time. It
repeatedly partitions the point set until it has selected all points whose
ranks are multiples of . These points form the set .
Initially, the set 
contains only the input  (line \ref{ld:alg:1d:qinit}). In lines
\ref{ld:alg:1d:splitstart}--\ref{ld:alg:1d:splitend}, we select from each set in
 an element whose rank is a multiple of  (line \ref{ld:alg:1d:select}) and
we split the set at this element. Here,  is a procedure that
returns the element with rank  of . After termination of both loops
in lines \ref{ld:alg:1d:outerstart}--\ref{ld:alg:1d:outerend}, all remaining sets
in  correspond to points in  between two consecutive points in . In
lines \ref{ld:alg:1d:fstart}--\ref{ld:alg:1d:fend}, the points in the sets in
 are distributed equally among the elements  () of the returned
partition.

\begin{alg}
  \Input{, size of partition }
  \SetKwFunction{Select}{select}
  \;\label{ld:alg:1d:rstart}
  \While{}{
    \;\label{ld:alg:1d:se}\label{ld:alg:1d:rend}
  }
  ;\label{ld:alg:1d:qinit}
  \;
  \While{}{
    \label{ld:alg:1d:outerstart}
    \ForEach{}{\label{ld:alg:1d:splitstart}
        remove  from \;
        \;\label{ld:alg:1d:select}
        \;\label{ld:alg:1d:qadd}
        \;\label{ld:alg:1d:splitend}
    }
	\;\label{ld:alg:1d:r}
    \label{ld:alg:1d:outerend}
  }
    \ForEach{}{\label{ld:alg:1d:fstart}
  \ForEach{}{
      remove any point from  and add it to \; \label{ld:alg:1d:fend}
      }
    }
  \Return{}\;
  \caption{1d-tolerant-Tverberg}
  \label{ld:alg:1d}
\end{alg}

\begin{theorem}
  Let  be a set of size . On input ,
  Algorithm~\ref{ld:alg:1d} returns a -tolerant Tverberg partition for  in
  time .
  \label{ld:thm:bound}
\end{theorem}
\begin{proof}
  After each complete run of the inner
  for-loop (lines \ref{ld:alg:1d:splitstart}--\ref{ld:alg:1d:splitend}),
  each element  has size strictly less than
  :
  initially,  contains only  and  is strictly greater than .
  Hence, both new sets added to  in line~\ref{ld:alg:1d:qadd}
  are of size strictly less than .
  Since  is halved after each run (line~\ref{ld:alg:1d:r}), the invariant is maintained.

  We will now check that Lemma~\ref{ld:lem:1d} applies. We only split the sets
  in  at elements whose rank is a multiple of , so the ranks do not change
  modulo . By the invariant, after the termination of the outer while-loop
  in lines \ref{ld:alg:1d:outerstart}--\ref{ld:alg:1d:outerend}, each set in 
  has size strictly less than . Thus,  contains the
  points in  whose rank is a multiple of  and
  each set  in  contains all points between two consecutive
  points in . Since these are distributed equally among
  , Lemma~\ref{ld:lem:1d} now shows the correctness of the
  algorithm.

  Let us consider the running time. Computing the initial  in
  lines~\ref{ld:alg:1d:rstart}--\ref{ld:alg:1d:rend} requires
   time. The split-element in
  line~\ref{ld:alg:1d:select} can be found in time
  ~\cite{Cormen2009}. Thus, since the sets are disjoint,
  one iteration of the outer while-loop requires  time, for a total of
  . By the same argument, both for-loops
  in lines \ref{ld:alg:1d:fstart}--\ref{ld:alg:1d:fend} require linear time in
  the size of . This results in a total time complexity of ,
  as claimed.
\end{proof}

\subsection{Higher Dimensions}
We use a lifting argument~\cite{Mulzer2013} to extend Algorithm~\ref{ld:alg:1d} to
higher-dimensional input.
Given a point set  of size , let  be a hyperplane that
splits  evenly (if  is odd,  contains exactly one point of ). We
then partition  into  pairs , where
 and . We obtain a -dimensional point set
with  elements by mapping each pair to the intersection of
the connecting line segment with .

Let  be the mapped point for  and
 a
-tolerant Tverberg -partition of .
We obtain a Tverberg -partition  with tolerance  for
 by replacing each  in  by its corresponding pair
. Thus, we can repeatedly project the set 
until Algorithm~\ref{ld:alg:1d} is applicable. Then, we lift the one-dimensional
solution back to higher dimensions.

Algorithm~\ref{ld:alg:dred} follows this approach.
For , Algorithm~\ref{ld:alg:1d} is applied
(lines~\ref{ld:alg:dred:basestart}--\ref{ld:alg:dred:baseend}).
Otherwise, we take an appropriate hyperplane orthogonal to the -axis and
compute the lower-dimensional point set
(lines~\ref{ld:alg:dred:h}--\ref{ld:alg:dred:pairend}). This is always possible
since we can assume w.l.o.g.
that all points have distinct  coordinates by a simple rotation argument.
Finally, the result for  dimensions is lifted back to  dimensions
(lines~\ref{ld:alg:dred:liftstart}--\ref{ld:alg:dred:liftend}).
\begin{alg}
  \Input{point set , tolerance parameter ,
  size of partition }
  \Output{-tolerant Tverberg partition for P of size }
  \SetKwFunction{solveOned}{1d-tolerant-Tverberg}
  \SetKwFunction{rec}{DimReduct-Tolerant-Tverberg}
  \If{}{\label{ld:alg:dred:basestart}
    \Return{\solveOned}
    \label{ld:alg:dred:baseend}
  }
   hyperplane that halves  according to the -coordinate\;
  \label{ld:alg:dred:h}
  \ForEach{}{
    \label{ld:alg:dred:pairstart}
     remove any point from  that belongs to \;
     remove any point from  that belongs to \;
     first  coordinates of \;
    \label{ld:alg:dred:pairend}
  }
  \;
  \;
  \label{ld:alg:dred:reccall}
  \ForEach{}{
    \label{ld:alg:dred:liftstart}
    \;
    \label{ld:alg:dred:liftend}
  }
  \label{ld:alg:dred:return}
  \Return{}\;
  \caption{DimReduct-Tolerant-Tverberg}
  \label{ld:alg:dred}
\end{alg}
Using Theorem~\ref{ld:thm:bound}, it is easy to show that
Algorithm~\ref{ld:alg:dred} achieves the bounds claimed in Theorem~\ref{thm:dr_tverberg}:
\begin{theorem}[Theorem~\ref{thm:dr_tverberg} restated]
  Given a set  of size ,
  a -tolerant Tverberg -partition
  for  can be computed in time .
\end{theorem}
\begin{proof}
  Since the size of  halves in each recursion step,  points
  suffice to ensure that Algorithm~\ref{ld:alg:1d} can be applied to 
  points in the base case. Each projection and lifting step can be performed in linear
  time, using a median computation. Since the size of the point set decreases
  geometrically, the total time for projection and lifting is thus .
  Since Algorithm~\ref{ld:alg:1d} has running time , the result follows.
\end{proof}

For , the bound from Proposition~\ref{thm:dr_tverberg}
is worse than the \Soberon{}-Strausz bound. However, in two dimensions, we have


This holds for instance if  or .
Thus, Algorithm~\ref{ld:alg:dred} gives a strict improvement over
the \Soberon{}-Strausz bound for large enough  and .

\section{Reduction to the Regular Tverberg Problem}\label{sec:redregtver}

We now show how to use any algorithm that computes approximate
regular Tverberg partitions in order to find tolerant Tverberg partitions.
For this, we must increase the tolerance of a Tverberg partition.
In the following, we show that one can merge elements of several
Tverberg partitions for disjoint subsets of 
to obtain a Tverberg partition with higher tolerance for the whole set .
The following lemma is also implicit in the Ph.D. thesis of
\GarciaColin{}~\cite{GarciaColin2007}.

\begin{lemma}
  \label{apr:lem:cb}
  Let  be Tverberg -partitions
  for disjoint point sets , ,.
  Let  be the th element of  and 
  the tolerance of .
  Then,
    
  is a Tverberg -partition of  with tolerance
  .
\end{lemma}
\begin{proof}
  Take  with . As
  , there is an  with
  .
  Since  is -tolerant, we have
  . Because each
   is contained in the corresponding set  of ,
  the convex hulls of the elements in  still intersect after the removal
  of .
\end{proof}

From a mathematical perspective, the main motivation for
introducing tolerance to Tverberg partitions is the possibility
to achieve better bounds than by just combining regular Tverberg
partitions. This provides deeper insight in the intersection pattern
of convex sets. Nevertheless, Lemma~\ref{apr:lem:cb} is interesting
from an algorithmic viewpoint as it enables us to benefit from
existing approximation algorithms for regular Tverberg partitions
by implying a simple algorithm: compute regular Tverberg
partitions for disjoint subsets of  and then merge them using
Lemma~\ref{apr:lem:cb}. This proves Proposition~\ref{apr:prop:cb}:

\begin{proposition}[Proposition~\ref{apr:prop:cb} restated]
  Let  and let  be an
  algorithm that computes a regular Tverberg -partition
  for any point set of size  in time .
  Then, a -tolerant Tverberg
  -partition for  can be computed in time
  .
\end{proposition}
\begin{proof}
  We split  into  disjoint sets and use
   to obtain for each subset a regular Tverberg partition. Applying
  Lemma~\ref{apr:lem:cb}, we obtain a -tolerant Tverberg -partition. Since the merging step in
  Lemma~\ref{apr:lem:cb} takes linear time in , the total running time is
  , as
  claimed.
\end{proof}

Table~\ref{apr:tab:conc} shows specific values for Proposition~\ref{apr:prop:cb}
applied to Miller \& Sheehy's and Mulzer \& Werner's algorithm.

\begin{table}
  \begin{center}
    \newcommand{\tableheader}[1]{\multicolumn{1}{|c|}{\textbf{#1}}}
    \renewcommand{\arraystretch}{1.5}
    \begin{tabular}{|l|c|c|}
      \hline
      \tableheader{Algorithm} &
      \tableheader{Tolerance} &
      \tableheader{Running time}
      \tabularnewline \hline \hline
      Proposition~\ref{apr:prop:cb} with Miller-Sheehy &
       &
      
      \tabularnewline \hline
      Proposition~\ref{apr:prop:cb} with Mulzer-Werner &
        &
      
      \tabularnewline \hline
    \end{tabular}
  \end{center}
  \caption{Proposition~\ref{apr:prop:cb} applied to existing approximation
    algorithms for the regular Tverberg problem.}
  \label{apr:tab:conc}
\end{table}


\begin{remark}
Lemma~\ref{apr:lem:cb} gives a quick proof of a slightly weaker version of the
\Soberon{}-Strausz bound: partition  into  disjoint sets of size at least
. By Tverberg's theorem, for each subset there exists a Tverberg
partition with no tolerance of size
. Using Lemma~\ref{apr:lem:cb},
we obtain a -tolerant Tverberg partition of size
 of , which is at most one less than the \Soberon{}-Strausz bound.
This weaker bound was also stated by \GarciaColin{}~\cite{GarciaColin2007}.
Again, as already mentioned after Lemma~\ref{apr:lem:cb}, this is
interesting mostly from an algorithmic perspective since it implies
that computing slightly worse tolerant Tverberg partitions than
guaranteed by the \Soberon{}-Strausz bound is polynomial-time
equivalent to computing regular Tverberg partitions.
\end{remark}

\section{Hardness of Tolerance Testing}\label{sec:complexity}
Teng~\cite{Teng1992} proved that deciding whether a point is a
centerpoint (\textsc{TestingCenter}) is coNP-complete. We show the same for
deciding whether a Tverberg partition is -tolerant
(\textsc{TestingTolerantTverberg}) by a reduction from \textsc{TestingCenter}.
The problems are formally defined as follows:

\begin{problem}[\textsc{TestingCenter}]
  \begin{enumerate}
    \item[]
      \begin{description}
    \item[Given] a point set , and a centerpoint candidate
      , where  is part of the input.
    \item[Decide] whether  is a centerpoint of .
  \end{description}
  \end{enumerate}
\end{problem}

\begin{problem}[\textsc{TestingTolerantTverberg}]
  \begin{enumerate}
    \item[]
      \begin{description}
    \item[Given] a point set , a
      partition  of , and a conjectured tolerance , where
       is part of the input.
    \item[Decide] whether  is a -tolerant Tverberg partition of .
  \end{description}
  \end{enumerate}
\end{problem}

Note that the size of the partition  in the definition of
\textsc{TestingTolerantTverberg} can be constant.

The following lemma is folklore. We include the proof for completeness.
It is used in the reduction to connect the tolerance of a
Tverberg partition with the depth of points in the intersection of the convex
hulls.

\begin{lemma}\label{com:lem:depth}
  Let  and let . Then  has depth  w.r.t.
   if and only if for all subsets  with , we have .
\end{lemma}
\begin{proof}
  We prove both directions by showing the contrapositive.\\
  \indent``''
        Suppose there is some  with . Then, there is a
        half-space  that contains  but no points from .
        Thus,
         and ,
        and hence  has depth at most  w.r.t. .

    ``''
        Assume  has depth  w.r.t. . Let  be a half-space that
        contains  and  points from . Set . Then,
         and .
\end{proof}

We are now ready to prove Theorem~\ref{thm:complexity}:
\begin{theorem}[Theorem~\ref{thm:complexity} restated]
  \textsc{TestingTolerantTverberg} is coNP-complete.
\end{theorem}
\begin{proof}
  We first check that \textsc{TestingTolerantTverberg} is indeed contained in
  coNP. Let  be a Tverberg partition of  that is claimed to have tolerance . A witness to
   not being a -tolerant Tverberg partition is a subset
   of size at most  such that
  . Checking if
   is a witness reduces to testing the feasibility  of the linear
  program defined by the following constraints for each element
   in \mc{T}:
  
  where  denotes the th point in .
  The linear program is feasible if and only if , i.e., if  is not a witness.
  Since the number of constraints and variables is polynomial in the input size,
  feasibility checking of the above linear program can be carried out in
  polynomial time.

  Let  be an input to
  \textsc{TestingCenter}. We embed the vector space  in 
  by identifying it with the hyperplane .  Define  and let  be the line that is orthogonal to  and
  passes through . Furthermore, let  and  be sets of 
  arbitrary points in  and , respectively.  Set . We claim that  is a Tverberg 2-partition for  with tolerance  if and only if  is a centerpoint of ; see
  Figure~\ref{com:fig:red}.

  ``'' Assume  is a -tolerant Tverberg
    2-partition. By construction of , we have .
    Thus,  lies in the intersection
    of both convex hulls even if an arbitrary subset of size at most  is removed.
    Lemma~\ref{com:lem:depth} implies that  has depth  w.r.t. , so  is a centerpoint for .

  ``'' Assume  is a centerpoint for . By
      definition,  has depth at least
       w.r.t. . Lemma~\ref{com:lem:depth}
      then implies that  is contained in the
      convex hull of  even if any  points from  are removed. Since
       contains  points on both sides of a line through ,  is also
      contained in  if any  points from  are removed.
      Thus,
       is a -tolerant Tverberg 2-partition for .
\end{proof}
\begin{figure}[htbp]
  \begin{center}
    \includegraphics[width=0.5\textwidth]{img/complexity.eps}
  \end{center}
  \caption{Reduction of \textsc{TestingCenter} to
    \textsc{TestingTolerantTverberg}}
  \label{com:fig:red}
\end{figure}


\section{Conclusion}
We have shown that for each set  of size , a
-tolerant Tverberg partition of size  can be found in time
. The bound on the size of  is tight, and it improves
the \Soberon{}-Strausz bound in one dimension.
Combining this with a lifting method, we could also get improved bounds in
two dimensions and an efficient algorithm for tolerant Tverberg partitions in
any fixed dimension.
However, the running time is exponential in the dimension.

This motivated us to look for a way of reusing the existing technology for
the regular Tverberg problem.
We have presented a reduction to the regular Tverberg problem
that enables us to
reuse the approximation algorithms by Miller \& Sheehy and Mulzer \& Werner.

Finally, we proved that testing whether a given Tverberg partition is of
some tolerance  is coNP-complete.
Unfortunately, this does not imply anything about the complexity of
finding tolerant Tverberg partitions. It is not even clear
whether computing tolerant Tverberg partitions is harder than computing
regular Tverberg partitions. However, we could show that computing tolerant
Tverberg partitions with smaller tolerance than guaranteed by the
\Soberon{}-Strausz bound is polynomial-time equivalent to computing regular
Tverberg partitions.

It remains open whether the bound by \Soberon{} and Strausz is tight for .
We believe that our results in one and two dimensions indicate that
the bound can be improved also in general dimension.
Another open problem is finding a \emph{pruning strategy} for tolerant
Tverberg partitions. By this, we mean an
algorithm that efficiently reduces the sizes of the sets in a -tolerant Tverberg
partition without deteriorating the tolerance.
Such a pruning strategy could be used to improve the quality of our algorithms.
In Miller \& Sheehy's and Mulzer \& Werner's algorithms,
\Caratheodory{}'s theorem was used for this task. Unfortunately, this result does not
preserve the tolerance of the pruned partitions. The generalized tolerant
\Caratheodory{} theorem~\cite{Montejano2011} also does not seem to help. It remains an
interesting problem to develop criteria for superfluous points
in tolerant Tverberg partitions.
\\

\noindent\textbf{Acknowledgments.}
We would like to thank the anonymous reviewers for their helpful
and detailed comments that helped to improve the quality of the
paper. In particular, we would like to thank an anonymous referee
for pointing out that the algorithm in
Proposition~\ref{apr:prop:cb} could be greatly simplified.


\bibliographystyle{abbrv}
\bibliography{library}
\end{document}
