\documentclass{sig-alternate}
\let\proof\undefined
\let\endproof\undefined
\usepackage{amsthm}
\usepackage{amssymb,amsmath}
\usepackage{graphics}
\usepackage{amsfonts}
\usepackage{mathrsfs}
\usepackage{amscd}
\usepackage{comment}
\usepackage{color}
\usepackage{url}
\usepackage{bbm}
\usepackage[plainpages=false,pdfpagelabels,colorlinks=true,citecolor=blue,hypertexnames=false]{hyperref}

\newcommand{\blue}{\color{blue}}
\newcommand{\red}{\color{red}}
\newcommand{\green}{\color{green}}

\newtheorem{theorem}{Theorem}
\newtheorem{prop}[theorem]{Proposition}
\newtheorem{corollary}[theorem]{Corollary}
\newtheorem{lemma}[theorem]{Lemma}
\newtheorem{algorithm}[theorem]{Algorithm}
\newtheorem{problem}[theorem]{Problem}
\newtheorem{defi}[theorem]{Definition}
\newtheorem{example}[theorem]{Example}
\newtheorem{fact}[theorem]{Fact}
\def\qed{\quad\rule{1ex}{1ex}}

\def\rmod{\mathrel{\mathrm{rmod}}}
\let\set\mathbbm
\def\K{\set K}
\def\O{\mathrm{O}}
\def\sqfp#1{#1^\ast}
\def\lc{\operatorname{lc}}
\def\deg{\operatorname{deg}}
\def\ff#1#2{#1^{\underline{#2}}}
\def\rf#1#2{#1^{\overline{#2}}}
\def\iverson#1{[\![#1]\!]}
\def\clap#1{\hbox to0pt{\hss#1\hss}}
\def\lin{\operatorname{L}}

\begin{document}

\title{Order-Degree Curves for Hypergeometric Creative~Telescoping}

\numberofauthors{2}

\author{\alignauthor Shaoshi Chen\titlenote{Supported by the National Science Foundation (NFS) grant CCF-1017217.}\\smallskipamount]
      \email{\strut schen21@ncsu.edu}
 \alignauthor \strut Manuel Kauers\titlenote{Supported by the Austrian Science Fund (FWF) grant Y464-N18.}\\smallskipamount]
      \email{\strut mkauers@risc.jku.at}
}

\maketitle
\begin{abstract}
    Creative telescoping applied to a bivariate proper hypergeometric term
    produces linear recurrence operators with polynomial coefficients, called
    telescopers. We provide bounds for the degrees of the polynomials appearing
    in these operators. Our bounds are expressed as curves in the {-plane}
    which assign to every order~ a bound on the degree~ of the telescopers. 
    These curves are hyperbolas, which reflect the phenomenon that higher order 
    telescopers tend to have lower degree, and vice versa.
\end{abstract}

\kern-\medskipamount

\category{I.1.2}{Computing Methodologies}{Symbolic and Algebraic Manipulation}[Algorithms]

\kern-\medskipamount

\terms{Algorithms}

\kern-\medskipamount

\keywords{Symbolic Summation, Creative Telescoping, Degree Bounds}

\kern-\medskipamount

\section{Introduction}

We consider the problem of finding linear recurrence equations with polynomial
coefficients satisfied by a given definite single sum over a proper
hypergeometric term in two variables. This is one of the classical problems
in symbolic summation. Zeilberger~\cite{zeilberger90} showed that such a
recurrence always exists, and proposed the algorithm now named after him for
computing one~\cite{zeilberger90a,zeilberger91}. Also explicit bounds are known
for the order of the recurrence satisfied by a given
sum~\cite{wilf92a,mohammed05,bostan10}. Little is known however about the
degrees of the polynomials appearing in the recurrence. These are investigated
in the present paper.

Ideally, we would like to be able to determine for a given sum all the pairs
 such that the sum satisfies a linear recurrence of order~ with
polynomial coefficients of degree at most~. This is a hard question which we
do not expect to have a simple answer. The results given below can be viewed as
answers to simplified variants of the problem. One simplification is that we
restrict the attention to the recurrences found by creative
telescoping~\cite{zeilberger91}, called ``\emph{telescopers}'' in the symbolic summation
community (see Section~\ref{sec:hg} below for a definition). The second
simplification is that instead of trying to characterize all the pairs ,
we confine ourselves with sufficient conditions.

Our main results are thus formulas which provide bounds on the degree~
of the polynomial coefficients in a telescoper, depending on its order~. The
formulas describe curves in the -plane with the property that for every
integer point  above the curve, there is a telescoper of order~ with
polynomial coefficients of degree at most~. As the curves are hyperbolas,
they reflect the phenomenon that higher order recurrence equations may have
lower degree coefficients. This feature can be used to derive a complexity
estimate according to which, at least in theory, computing the minimum order
recurrence is more expensive than computing a recurrence with slightly higher
order (but drastically smaller polynomial coefficients). This phenomenon is
analogous to the situation in the differential case, which was first analyzed by
Bostan et al.~\cite{bostan07} for algebraic functions, and recently for
integrals of hyperexponential terms by the authors~\cite{chen11}.

Our analysis for non-rational proper hypergeometric input
(Section~\ref{sec:trans}) follows closely our analysis for the differential
case~\cite{chen11}. It turns out that the summation case considered here is
slightly easier than the differential case in that it requires fewer cases to
distinguish and in that the resulting degree estimation formula is much simpler
than its differential analogue. For rational input (Section~\ref{sec:rat}), we
derive a degree estimation formula following Le's algorithm for computing
telescopers of rational functions~\cite{abramov02,le03}.

\section{Proper Hypergeometric Terms and Creative Telescoping}\label{sec:hg}

Let  be a field of characteristic zero and let  be the field of rational functions
in  and~. We will be considering extension fields  of  on which two
isomorphisms~ and  are defined which commute with each other,
leave every element of  fixed, and act on  and  via ,
, , .
A \emph{hypergeometric term} is an element~ of such an extension field~ with
 and . Proper hypergeometric terms are hypergeometric
terms which can be written in the form

where , ,  is fixed,
, , , , , , ,  are nonnegative integers,
, , ,  and the expressions ,
, and  refer to elements of  on which  and 
act as suggested by the notation, e.g.,

Throughout the paper the symbols , , , , , , , \dots will be used
with the meaning they have in~\eqref{eq:hgdef}.
For fitting long formulas into the narrow columns of this layout, we also use the abbreviations


We follow the paradigm of creative telescoping. For a given hypergeometric
term~ as above, we want to determine polynomials
 (free of~, not all zero), and a rational
function  (possibly involving~, possibly zero), such that

In this case, the operator  is called a \emph{telescoper} for~, and the rational
function  is called a \emph{certificate} for~ (and~). The
number~ is called the \emph{order} of~, and 
is called its \emph{degree}. If  represents an actual sequence , then
a recurrence for the definite sum  can be obtained from
such a pair  as explained in the literature on symbolic
summation~\cite{petkovsek97}. We shall not embark on the technical subtleties of
this correspondence here but restrict ourselves to analyzing of the set of all pairs
 for which there exists a telescoper of order  and degree~.

The following notation will be used.
\begin{itemize}
\item For  and , let
  
  with the conventions  and .
\item For ,  and  denote the degree of  with respect
  to  or~, respectively.  without any subscript denotes the total degree of~.
\item For , let .
\end{itemize}
With this notation, we have


\section{The Non-Rational Case}\label{sec:trans}

We consider in this section the case where  cannot be split into 
for  and another hypergeometric term  with
. Informally, this means that we exclude terms  where 
and every -term involving  can be cancelled against another one to
some rational function. Those terms are treated separately in
Section~\ref{sec:rat} below.  If  cannot be split as indicated, then also
 cannot be split in this way, for any rational function . In
particular, we can then not have  and therefore we always have
. This implies that whenever we have a pair  with
, we can be sure that  is not the zero operator, and we need
not worry about this requirement any further.

The analysis in the present case is similar to that carried out by
Apagodu and Zeilberger~\cite{mohammed05}, who used it for deriving a bound on the order
 of~, and similar to our analysis~\cite{chen11} of the differential case.
The main idea is to follow step by step the execution of Zeilberger's
algorithm when applied to~. This eventually leads to a linear system of
equations with coefficients in~ which must have a solution whenever it is
underdetermined. The condition of having more variables than equations in this
linear system is the source of the estimate for choices  that lead
to a solution.

\subsection{Zeilberger's Algorithm}

Recall the main steps of Zeilberger's algorithm: for some choice of~, it makes
an ansatz  with undetermined coefficients
, and then calls Gosper's algorithm on~.
Gosper's algorithm~\cite{gosper78} proceeds by writing

for some polynomials  such that  for all . It
turns out that the undetermined coefficients  appear
linearly in~ and not at all in  or~. Next, the algorithm searches for
a polynomial solution  of the \emph{Gosper equation}

by making an ansatz  for some suitably chosen
degree~, substituting the ansatz into the equation, and comparing powers of 
on both sides. This leads to a linear system in the variables
 with coefficients in~. Any solution
of this system gives rise to a telescoper~ with the corresponding certificate .
If no solution exists, the procedure is repeated with a greater value
of~.

For a hypergeometric term~ and an operator , we have {\allowdisplaybreaks
}We can write

where

Depending on the actual values of the coefficients appearing in~, this decomposition may or may not
satisfy the requirement  for all . But even if
it does not, it only means that we may overlook some solutions, but every
solution we find still gives rise to a correct telescoper and
certificate. Since we are interested only in bounding the size of the
telescopers of~, it is sufficient to study under which circumstances the
Gosper equation

with the above choice of  has a solution.

\subsection{Counting Variables and Equations}

Apagodu and Zeilberger~\cite{mohammed05} proceed from here by analyzing the linear system over
 resulting from the Gosper equation for a suitable choice of the degree
of~. They derive a bound on~ but give no information on the
degree~. General bounds for the degrees of solutions of linear systems with
polynomial coefficients could be applied, but they turn out to overshoot quite
much. In particular, it seems difficult to capture the phenomenon that increasing 
may allow for decreasing  using such general bounds.

We proceed differently. Instead of a coefficient comparison with respect to powers
of  leading to a linear system over~, we consider a coefficient comparison
with respect to powers of  and  leading to a linear system over~. This
requires us to make a choice not only for the degree of  in~ but also for the
degree of  in  as well as for the degrees of the  () in~.
For expressing the number of variables and equations in this system, it is helpful
to adopt the following definition.

\begin{defi}\label{def:greek}
  For a proper hypergeometric term  as in~\eqref{eq:hgdef}, let {\allowdisplaybreaks
  -2pt]
      \lambda &=\sum_{m=1}^M(u_m+v_m),\\
          \mu &=\sum_{m=1}^M (a_m+b_m-u_m-v_m),\}
\end{defi}

Note that these parameters are integers which only depend on  but not on  or~.
Except for , they are all nonnegative. Note also that we have 
and .

\begin{lemma}\label{lem:1}
  Let  (). Then
  
  Furthermore, .
\end{lemma}
\begin{proof}
  It suffices to observe that
  
  for all  and all . For the degree with respect to~,
  observe also that  for all~.
\end{proof}

We have some freedom in choosing the~. The choice influences the number of variables
in the ansatz

as well as the number of equations. We prefer to have many variables and few equations.
For a fixed target degree~, the maximum possible number of variables is 
by choosing . But this choice also leads to many equations.
A better balance between number of variables and number of equations is obtained by lowering
some of the  with indices close to zero (if  is negative) or with indices
close to~ (if  is positive). Specifically, we choose

See~\cite[Ex.~11, Ex.~15.5 and the remarks after Thm.~14]{chen11} for a detailed motivation of
the corresponding choice in the differential case. The support of the ansatz for  looks
as in the following diagram, where every term  is represented by a bullet at
position :
\begin{center}
  \begin{picture}(80,102)(0,-10)
    \multiput(0,0)(0,10){9}{\circle*{3}}
    \multiput(10,0)(0,10){9}{\circle*{3}}
    \multiput(20,0)(0,10){9}{\circle*{3}}
    \multiput(30,0)(0,10){9}{\circle*{3}}
    \multiput(40,0)(0,10){9}{\circle*{3}}
    \multiput(50,0)(0,10){9}{\circle*{3}}
    \multiput(60,0)(0,10){7}{\circle*{3}}
    \multiput(70,0)(0,10){5}{\circle*{3}}
    \multiput(80,0)(0,10){3}{\circle*{3}}
    \put(0,-5){\line(0,-1){3}}\put(0,-15){\hbox to0pt{\hss \hss}}
    \put(80,-5){\line(0,-1){3}}\put(80,-15){\hbox to0pt{\hss \hss}}
    \put(-5,0){\line(-1,0){3}}\put(-12,-2){\hbox to0pt{\hss }}
    \put(-5,80){\line(-1,0){3}}\put(-12,78){\hbox to0pt{\hss }}
    \put(83,73){}
    \put(83,51){}
    \put(83,33){}
    \put(56,85){}
  \end{picture}
  \medskip
\end{center}
With this choice for the degrees~, the number of variables in the ansatz for  is

provided that .
The number of resulting equations is as follows.

\begin{lemma}\label{lem:2}
  If the  are chosen as above, then  contains at most
  
  terms .
\end{lemma}
\begin{proof}
  If , we have
  
  for all . Likewise, when , we have
  
  for all . Together with Lemma~\ref{lem:1}, it follows that
  
  regardless of the sign of~. We also have 
  from Lemma~\ref{lem:1}. For the number of terms  in  we have
  
  Plugging the estimates for  and  into the right hand side gives
  the expression claimed in the Lemma.
\end{proof}

The support of  has a trapezoidal shape which is determined by the total degree and 
the degree with respect to~:

\begin{center}
  \begin{picture}(30,80)(0,-10)
    \multiput(0,0)(0,10){8}{\circle*{3}}
    \multiput(10,0)(0,10){7}{\circle*{3}}
    \multiput(20,0)(0,10){6}{\circle*{3}}
    \multiput(30,0)(0,10){5}{\circle*{3}}

    \put(-4,-5){\hbox{}}
    \put(-5,35){\hbox to 0pt{\hss}}
  \end{picture}
  \medskip
\end{center}

The next step is to choose the degrees for  in  and~. This is done in such a way that
 only contains terms which are already expected to occur in~, so that no
additional equations will appear.

\begin{lemma}\label{lem:3}
  Let the  be chosen as before and suppose that  is such that
   and . Then
   contains at most
  
  terms .
\end{lemma}
\begin{proof}
  As for Lemma~\ref{lem:2}, using also .
\end{proof}

Lemma~\ref{lem:3} suggests the ansatz

with  and , which provides us with

variables. We are now ready to formulate the main result of this section.
Note that the inequality for~ is a considerably simpler formula
than the corresponding result in the differential case (Thm.~14 in~\cite{chen11}).

\begin{theorem}\label{thm:trans}
  Let  be a proper hypergeometric term which cannot be written  for
  some  and a hypergeometric term  with .
  Let  be as in Definition~\ref{def:greek}, let  and
  
  Then there exists a telescoper  for  of order~ and degree~.
\end{theorem}
\begin{proof}
  A sufficient condition for the existence of a telescoper of order  and degree~ is that
  for some particular ansatz, the equation
  
  has a nontrivial solution. A sufficient condition for the existence of a solution is that the
  linear system resulting from coefficient comparison has more variables than equations.
  For all  in question, we have . Therefore,
  with the ansatz described above, we have
  
  variables  in~,
  
  variables  in~, and
  
  equations. Solving the inequality
  
  under the assumption  for  gives the claimed degree estimate.
\end{proof}

\subsection{Examples and Consequences}

\begin{example}\label{ex:trans}
\begin{enumerate}
\item For
  
  we have , , , . Theorem~\ref{thm:trans} predicts
  a telescoper of order~ and degree~ whenever  and
  
  The left figure below shows the curve defined by the right hand side (black) together with
  the region of all points  for which we found telescopers of  with order~
  and degree~ by direct calculation (gray).
  In this example, the estimate overshoots by very little only.
\item The corresponding picture for
  
  is shown below on the right. Here, , , ,  and
  Theorem~\ref{thm:trans} predicts a telescoper of order~ and degree~ whenever 
  and
  
  In this example, the estimate is less tight.



\end{enumerate}
\end{example}

\centerline{\includegraphics{curve-a}\hfil\includegraphics{curve-b}}

\medskip

The points  in the portion of the gray region which is below the black curve
represent telescopers where the corresponding linear system resulting from the ansatz considered
in our proof is overdetermined but, for some strange reason, nevertheless nontrivially solvable. 
The small portions of white space which lie above the curves are not in contradiction with our 
theorem because they do not contain any points with integer coordinates. (The theorem says that
every point  above the curve belongs to the gray region.)

Theorem~\ref{thm:trans} supplements the bound given in \cite{mohammed05} on the order
of telescopers for a hypergeometric term by an estimate for the degree that these
operators may have. In addition, it provides lower degree bounds for higher orders
and admits a bound on the least possible degree for a telescoper.

\begin{corollary}
  With the notation of Theorem~\ref{thm:trans},  admits a telescoper of
  order  and degree
  
  as well as a telescoper of order
  
  and degree .
\end{corollary}
\begin{proof}
  Immediate by checking that the two choices for  and  satisfy the conditions
  stated in Theorem~\ref{thm:trans}.
\end{proof}

An accurate prediction for the degrees of the telescopers can also be used for improving the efficiency
of creative telescoping algorithms. Although most implementations today compute the telescoper
with minimum order, it may be less costly to compute a telescoper of slightly higher order. If
we know in advance the degrees  of the telescopers for every order~, we can select before
the computation the order  which minimizes the computational cost. Of course, the cost depends
on the algorithm which is used. It is not necessary (and not advisable) to follow the steps in the
derivation of Theorem~\ref{thm:trans} and do a coefficient comparison over~. Instead, one should
follow the common practice~\cite{koutschan10b} of comparing coefficients only with respect to powers of 
and solve a linear system over~. For nonminimal choices of , this system will have a
nullspace of dimension greater than one, of which we need not compute a complete basis,
but only a single vector with components of low degree. There are algorithms known for computing
such a vector using  field operations when the system has at most  variables and equations and the
solution has degree at most  \cite{beckermann94,storjohann05,bostan07}. In the situation at hand, we have
 variables and a solution of degree .
Therefore, in order to compute a telescoper
and its certificate most efficiently, we should minimize the cost function

According to the following theorem, for asymptotically large input it is significantly better to
choose  slightly larger than the minimal possible value.

\begin{theorem}\label{thm:trans:compl}
  Let  and  be as in Theorem~\ref{thm:trans}, ,
  and suppose that  is a constant such that degree  solutions of a linear system
  with  variables and at most  equations over  can be computed with  operations
  in~. Then:
  \begin{enumerate}
  \item A telescoper of order  along with a corresponding certificate can be computed using
  
  operations in .
  \item\label{it:2} If  is some constant and  is chosen such that
  ,
  then a telescoper of order  and a corresponding certificate can be computed using
  
  operations in~.
  \end{enumerate}
  In particular, a telescoper for  and a corresponding certificate can be computed
  in polynomial time.
\end{theorem}
\begin{proof}
  According to Theorem~\ref{thm:trans}, for every  there exists a telescoper of
  order  and degree  for any
  
  By assumption, such a telescoper can be computed using no more than
  
  operations in~. The claim now follows from the asymptotic expansions of
   and  for , respectively.
\end{proof}

The leading coefficient in part~\ref{it:2} is minimized for . This suggests that
when  and  are large and approximately equal, the order of the cheapest telescoper
is about 20\% larger than the minimal expected order.

\section{The Rational Case}\label{sec:rat}

We now turn to the case where  can be written as  for some hypergeometric term 
with . By the following transformation, we may assume without loss of generality .

\begin{lemma}
  Let  be a hypergeometric term and suppose that  for some  and a hypergeometric
  term  with . Let  be such that .
  Let  be a telescoper for  of order  and degree~.
  Then there exists a telescoper for  of order  and degree at most
  .
\end{lemma}
\begin{proof}
  Write  and let  be a certificate
  for~ and~, so . For , let
  
  and . Then
  
  Because of
  
  the operator  is free of~. Thus, after clearing denominators,  is
  a telescoper for  with coefficients of degree at most .
\end{proof}

{}From now on, we assume that  is at the same time a proper hypergeometric term and a rational
function, or equivalently, that  is a rational function whose denominator factors into integer-linear
factors. Le~\cite{le03} gives a precise description of the structure of telescopers in this case, and he
proposes an algorithm different from Zeilberger's for computing them. Our degree estimate is derived
following the steps of his algorithm, so we start by briefly summarizing the main steps of Le's approach.

\subsection{Le's Algorithm}\label{sec:le}

Given a rational proper hypergeometric term~, Le's algorithm computes a telescoper  for  as follows.

\begin{enumerate}
\item Compute  and polynomials  with  for all 
  such that
  
  Then an operator  is a telescoper for  if and only if  is a telescoper for .
  Abramov~\cite{abramov95a} and Paule~\cite{paule95} explain how to compute such a decomposition.
\item Compute a polynomial , operators  in , and
  rational functions  of the form  () such that
  
  Such data always exists according to Lemma~5 in \cite{le03} in combination with the assumption 
  (). It can be further assumed that the  are chosen such that , ,
   for all~, and
  
  for all  with .
\item For , compute an operator  such that  is a right divisor
  of . It follows from Le's Lemma~4 that the operators  with this property are precisely
  the telescopers of the rational functions~.
\item Compute a common left multiple  of the operators .
  Then  is a telescoper for~.
\end{enumerate}

The main part of the computational work happens in the last two steps. It
therefore appears sensible to assume in the following degree analysis that we
already know the data  computed in step~2, and
to express the degree bounds in terms of their degrees and coefficients rather
than in terms of the degrees of numerator and denominator of~, say.

\subsection{Counting Variables and Equations}

Also in the present case, the degree estimate is obtained by balancing the number of variables and
equations of a certain linear system over~. The linear system we consider originates from a
particular way of executing steps 3 and~4 of the algorithm outlined above.

\begin{theorem}\label{thm:curve:rat}
  Let  and let 
  be operators of degree~ ().
  Let  for some ,
   with  and , , suppose
  
  for all  with . Let .
  Then for every  and every
  
  there exists a telescoper  for  of order~ and degree~.
\end{theorem}
\begin{proof}
  According to Le's algorithm, it suffices to find some  and operators
   with the property that  for all~.

  Denote by  the order of~.  Writing , we
  make an ansatz  with
  
  so that  has degree~ and  ().
  It thus remains to construct operators  with .
  Since  has order  and degree , we consider
  ansatzes for the  of order  and degree , respectively,
  because  has order  and degree~.
  Then we have altogether
  
  variables in  and the~, and comparing coefficients with respect to  and 
  in all the required identities  leads to a linear system with
  
  equations. This system will have a nontrivial solution whenever the number of variables
  exceeds the number of equations. Under the assumption , the inequality
  
  is equivalent to
  
  This completes the proof.
\end{proof}

\subsection{Examples and Consequences}

\begin{example}\label{ex:rat}
  \begin{enumerate}
  \item The rational function
    
    can be written in the form  where
    ,
    the  are such that , , and
    the  are such that .
    Therefore, Theorem~\ref{thm:curve:rat} predicts a telescoper of order~ and degree~
    whenever  and
    
    This curve together with the region of all points  for which 
    a telescoper of order~ and degree~ exists is shown in the left figure below.
  \item The corresponding picture for the rational function
    
    is shown in the figure below on the right. This input can be written in the form
    
    with ,
    ,
    the  such that , , and
    the  such that , .
    According to Theorem~\ref{thm:curve:rat}, we therefore expect a telescoper for 
    of order~ and degree~ whenever  and
    
    In this example, the estimate is not as tight as in the previous one.
  \end{enumerate}
\end{example}



\centerline{\includegraphics{curve-rat-a}\hfil\includegraphics{curve-rat-b}}

\smallskip

Again, it is an easy matter to specialize the general degree bound to a degree estimate
for a low order telescoper, or to an order estimate for a low degree telescoper.

\begin{corollary}
  With the notation of Theorem~\ref{thm:curve:rat},  admits a telescoper of
  order  and degree 
  as well as a telescoper of order  and degree .
\end{corollary}
\begin{proof}
  Clear by checking that the proposed choices for  and~ are consistent with
  the bounds in Theorem~\ref{thm:curve:rat}.
\end{proof}

Also like in the non-rational case, the bounds for the degrees of the telescopers can be used for
deriving bounds on the computational cost for computing them. In the present situation,
let us assume for simplicity that the cost of steps 1 and~2 of Le's algorithm is negligible,
or equivalently, that the input~ is of the form  with .
We shall analyze the algorithm which carries out steps 3 and~4 of Section~\ref{sec:le} in one
stroke by making an ansatz over  for an operator~, computing
the right reminders of  with respect to  and equating their coefficients to
zero. We assume, as before, that the resulting linear system is solved using an
algorithm whose runtime is linear in the output degree and cubic in the matrix size. Then
the algorithm requires  operations in~.

\begin{theorem}\label{thm:rat:compl}
  Let , , and  be as in Theorem~\ref{thm:curve:rat}
  and consider .
  Suppose that  is a constant such that degree  solutions of a linear system
  with  variables and at most  equations over  can be computed with  operations
  in~. Assume  and  are fixed.
  Then:
  \begin{enumerate}
  \item A telescoper of order  can be computed using
  
  operations in .
  \item\label{it:rat:2} If  is some constant and  is chosen such that
  
  then a telescoper of order~ can be computed using
  
  operations in~.
  \end{enumerate}
  In particular, a telescoper for  can be computed in polynomial time.
\end{theorem}
\begin{proof}
  According to the first estimate stated in Theorem~\ref{thm:curve:rat}, for every 
  there exists a telescoper of order~ and degree~ for any
  
  By assumption, such a telescoper can be computed using no more than 
  operations in~. The claim now follows from the asymptotic expansions of 
  and  for , respectively.
\end{proof}

When , the leading coefficient in part~\ref{it:rat:2} is minimized
for . This suggests that when  is large and all the ,
and  are approximately equal, the order of the cheapest operator
exceeds the minimal expected order by around~50\%.

It must not be concluded from a literal comparison of the exponents in
Theorems~\ref{thm:trans} and~\ref{thm:curve:rat} that Le's algorithm is faster
than Zeilberger's, because  in Theorem~\ref{thm:trans} and  in
Theorem~\ref{thm:curve:rat} measure the size of the input differently. Nevertheless,
it is plausible to expect that Le's algorithm is faster, because it finds
the telescopers without also computing a (potentially big) corresponding
certificate. Our main point here is not a comparison of the two approaches, but
rather the observation that both of them admit a degree analysis which fits to the
general paradigm that increasing the order can cause a degree drop which is
significant enough to leave a trace in the computational complexity.

It can also be argued that the situations considered in
Theorems~\ref{thm:trans:compl} and~\ref{thm:rat:compl} are chosen somewhat
arbitrarily (~and  growing while ~remains fixed; resp.\
~growing while all the  and  remain fixed). Indeed, it would
be wrong to take these theorems as an advice which telescopers are most easily
computed for a particular input at hand. Instead, in order to speed up an actual
implementation, one should let the program calculate the optimal choice for~
from the degree estimates given Theorems~\ref{thm:trans} and~\ref{thm:curve:rat}
with the particular parameters of the input.

Unfortunately, we are not able to illustrate the speedup obtained in this way by
an actual runtime comparison for a concrete example, because for examples which
can be handled on currently available hardware, the computational cost turns out
to be minimized for the least order operator. But already for examples which are
only slightly beyond the capacity of current machines, the degree predictions in
Theorems~\ref{thm:trans} and~\ref{thm:curve:rat} indicate that computing the
telescoper of order one more than minimal will start to give an advantage. We
therefore expect that the results presented in this paper will contribute to the
improvement of creative telescoping implementations in the very near future.


\begin{thebibliography}{10}

\bibitem{abramov95a}
Sergei~A. Abramov.
\newblock Indefinite sums of rational functions.
\newblock In {\em Proceedings of ISSAC'95}, pages 303--308, 1995.

\bibitem{abramov02}
Sergei~A. Abramov and Ha~Q. Le.
\newblock A criterion for the applicability of {Z}eilberger's algorithm to
  rational functions.
\newblock {\em Discrete Mathematics}, 259(1--3):1--17, 2002.

\bibitem{beckermann94}
Bernhard Beckermann and George Labahn.
\newblock A uniform approach for the fast computation of matrix-type {P}ad{\'e}
  approximants.
\newblock {\em SIAM Journal on Matrix Analysis and Applications},
  15(3):804--823, 1994.

\bibitem{bostan10}
Alin Bostan, Shaoshi Chen, Fr\'{e}d\'{e}ric Chyzak, and Ziming Li.
\newblock Complexity of creative telescoping for bivariate rational functions.
\newblock In {\em Proceedings of ISSAC'10}, pages 203--210, 2010.

\bibitem{bostan07}
Alin Bostan, Fr{\'e}d{\'e}ric Chyzak, Bruno Salvy, Gr{\'e}goire Lecerf, and
  {\'E}ric Schost.
\newblock Differential equations for algebraic functions.
\newblock In {\em Proceedings of ISSAC'07}, pages 25--32, 2007.

\bibitem{chen11}
Shaoshi Chen and Manuel Kauers.
\newblock Trading order for degree in creative telescoping.
\newblock Technical Report 1108.4508, ArXiv, 2011.

\bibitem{gosper78}
William Gosper.
\newblock Decision procedure for indefinite hypergeometric summation.
\newblock {\em Proceedings of the National Academy of Sciences of the United
  States of America}, 75:40--42, 1978.

\bibitem{koutschan10b}
Christoph Koutschan.
\newblock A fast approach to creative telescoping.
\newblock {\em Mathematics in Computer Science}, 4(2--3):259--266, 2010.

\bibitem{le03}
Ha~Q. Le.
\newblock A direct algorithm to construct the minimal {Z}-pairs for rational
  functions.
\newblock {\em Advances in Applied Mathematics}, 30(1--2):137--159, 2003.

\bibitem{mohammed05}
Mohamud Mohammed and Doron Zeilberger.
\newblock Sharp upper bounds for the orders of the recurrences output by the
  {Z}eilberger and -{Z}eilberger algorithms.
\newblock {\em Journal of Symbolic Computation}, 39(2):201--207, 2005.

\bibitem{paule95}
Peter Paule.
\newblock Greatest factorial factorization and symbolic summation.
\newblock {\em Journal of Symbolic Computation}, 20:235--268, 1995.

\bibitem{petkovsek97}
Marko Petkov{\v s}ek, Herbert Wilf, and Doron Zeilberger.
\newblock {\em }.
\newblock AK Peters, Ltd., 1997.

\bibitem{storjohann05}
Arne Storjohann and Gilles Villard.
\newblock Computing the rank and a small nullspace basis of a polynomial
  matrix.
\newblock In {\em Proceedings of ISSAC'05}, pages 309--316, 2005.

\bibitem{wilf92a}
Herbert~S. Wilf and Doron Zeilberger.
\newblock An algorithmic proof theory for hypergeometric (ordinary and )
  multisum/integral identities.
\newblock {\em Inventiones Mathematicae}, 108:575--633, 1992.

\bibitem{zeilberger90a}
Doron Zeilberger.
\newblock A fast algorithm for proving terminating hypergeometric identities.
\newblock {\em Discrete Mathematics}, 80:207--211, 1990.

\bibitem{zeilberger90}
Doron Zeilberger.
\newblock A holonomic systems approach to special function identities.
\newblock {\em Journal of Computational and Applied Mathematics}, 32:321--368,
  1990.

\bibitem{zeilberger91}
Doron Zeilberger.
\newblock The method of creative telescoping.
\newblock {\em Journal of Symbolic Computation}, 11:195--204, 1991.

\end{thebibliography}

\end{document}
