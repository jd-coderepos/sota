\documentclass{article}


\usepackage{graphicx}
\usepackage{subfigure}
\usepackage{wrapfig}



\usepackage{url}  
\usepackage[utf8]{inputenc}  
\usepackage{subfigure}  
\usepackage{latexsym}  
\usepackage{amsmath} 
\usepackage{amssymb} 
\usepackage{algorithm}
\usepackage{algorithmic}
\urlstyle{tt}  



\newtheorem{definition}{Definition}  
\newtheorem{proposition}{Proposition}  
\newtheorem{prop}{Property}  
\newtheorem{lemma}{Lemma}  
\newtheorem{cor}{Corollary}  
\newtheorem{corollary}{Corollary}  
\newtheorem{example}{Example}  
\newtheorem{invariant}{Invariant}  
\newtheorem{property}{Property}  
 


\newtheorem{theorem}{Theorem}  
 
\newenvironment{proof}{\noindent {\it Proof.}}{\vskip1ex}  
\newenvironment{preuve}{\noindent {\it Proof.}}{\vskip1ex}  
\newenvironment{preuveA1}{\noindent {\it Proof of conservation of   
Property .}}{\vskip1ex}  
\newenvironment{preuveA2}{\noindent {\it Proof of conservation of   
Property .}}{\vskip1ex}  
\newenvironment{preuveB}{\noindent {\it Proof of the invariants.}}  
{\vskip1ex}  
\newenvironment{OJO}{\noindent {\bf OJO:}}{\vskip1ex}  
  
 
\newcommand{\Nat}{{\mathbb N}}
\newcommand{\Real}{{\mathbb R}}
\def\lastname{Habib}
\graphicspath{{figures/}}

\begin{document}


\title{\vspace*{-2cm}Faster and Simpler Minimal Conflicting\\ Set
  Identification\thanks{This work is partly supported by the french MAPPI project
    (ANR-2010-COSI-004).}}

\author{Aida Ouangraoua\thanks{INRIA, Centre de recherche INRIA
    Haute-Borne, Bât. A, Park Plaza 40 avenue Halley, 59650 Villeneuve
    d'Ascq, France.  {\tt aida.ouangraoua@inria.fr}} \and Mathieu
  Raffinot\thanks{CNRS/LIAFA, Universit\'e Paris Diderot - Paris 7,
    France, {\tt raffinot@liafa.jussieu.fr}} 
}

\maketitle


\begin{abstract}
Let  be a finite set of  elements and  a family of  subsets of . A subset  of  verifies the \emph{Consecutive Ones
Property (C1P)} if there exists a permutation  of  such
that each  in  is an interval of . A \emph{Minimal
Conflicting Set (MCS)}  is a subset of
  that does not verify the C1P, but such that any of its proper 
subsets does. In this paper, we present a new simpler and faster algorithm 
to decide if a given element  belongs to at least one MCS. 
Our algorithm runs in , largely improving
the current  fastest algorithm of
[Blin {\em et al}, CSR 2011]. The new algorithm is based on an alternative
approach considering minimal forbidden induced subgraphs of interval
graphs instead of Tucker matrices.
\end{abstract}  


\section{Introduction}

Let  be a finite set of  elements
and  a family of  subsets of
. Those sets can be seen as a  0-1 matrix
, such that the set  represents the 
columns of the matrix, and the set  the rows of the matrix:
each  represents the set of columns where row  has
an entry 1.









A subset  of  verifies the consecutive ones
property (C1P) if there exists a permutation  of  such
that each  in  is an interval of . Testing the
consecutive ones property is the core of many algorithms that have
applications in a wide range of domains, from VLSI circuit conception
through planar embeddings \cite{nr-pgd-04} to computational biology
for the reconstruction of ancestral genomes
\cite{BBCC2004,Stephane2010,CHSY2009,Chauve08,SW2009}. 
We focus on this last field in this paper.



On real biological matrices, the C1P is rarely verified, and only some
subsets of rows might verify the desired property.However, the combinatorics of such sets is difficult to
handle, and a strategy to deal with them has been proposed
in \cite{BBCC2004,Chauve08,SW2009}. It consists in identifying
the rows belonging to minimal conflicting subsets of rows that 
do not verify the C1P, but such that any of their
row subset does. 

\begin{definition}
A set  is a {\em Minimal
  Conflicting Set} (MCS) if  does not verify the C1P, but such 
that , the set  verifies the C1P.
\end{definition}






\begin{wrapfigure}[15]{r}{5cm}
  \centering
\includegraphics[width=4cm]{MCS-mat4.pdf}
\caption{A matrix not verifying the C1P and such
  that each set of 3 rows is a MCS.}
 \label{MCS-exp}
\end{wrapfigure}

\noindent
However, it is not difficult to build examples of  matrices such that the
number of MCS is polynomial or even exponential in the number of
rows.

 Figure \ref{MCS-exp} shows such an example in which each sub set of
 rows is a MCS. Thus, such a construction with  rows
gives   MCS. Note that, on this example, a single
row is included in  MCS.



 Figure \ref{MCS-exp2}-(a) shows another example where the number of MCS
 is exponential in the number of rows. Let  be the number of nodes
 of {\em external} rows, which are  and  on the
 figure. The total number of rows is , the number of columns ,
 and the number of MCS is  since any induced chordless cycle in
 the row intersection graph of the matrix  (Figure \ref{MCS-exp2}-(b)) 
constitutes a MCS.

\begin{figure}[htb]
  \centering
\includegraphics[width=9cm]{exp5.pdf}
\caption{(a) A matrix not verifying the C1P and such that the number 
of MCS is exponential in the  number of rows. (b) A row intersection
graph of the matrix whose vertices correspond to the rows of the matrix,
and such that there exists an edge between two rows  and  
if .}
 \label{MCS-exp2}
\end{figure}


From a computational point of view, the first question that arises is the
following: is a given row  included in at least one MCS ? 
This question has been raised in \cite{BBCC2004}, recalled in 
\cite{CHSY2009,Chauve08} and recently solved in polynomial time
 in \cite{Blin2011}. This currently fastest 
algorithm is based on the identification of minimal Tucker forbidden 
submatrices \cite{T1972,D2009}.

In this paper we present a new simpler  time
algorithm for deciding if a given row belongs to at least one MCS and
if true exhibit one. Our algorithm is based on an
alternative approach considering minimal forbidden induced subgraphs
of interval graphs \cite{LB62} instead of Tucker matrices. Moreover,
our central paradigm consists in reducing the recognition of complex
forbidden induced subgraphs to the detection of induced cycles in
ad-hoc graphs, while in \cite{Blin2011} only induced paths are
considered. Our approach is faster and simpler, but a limit
shared by both approaches 
resides in  avoiding to report the number of MCS to which a given row belongs.


\section{MCS and Forbidden induced subgraphs}



The \emph{row-column intersection graph} of a 0-1 matrix 
 is a vertex-colored 
bipartite graph  whose set of vertices is  ;  
the vertices corresponding to rows (resp. columns) are black (resp. white) ;
there exists an edge between two rows  and  
if , and there exists an edge between a row 
 and a column  if . 

It should be noted that a column vertex (white) is only connected to row vertices (black).

The \emph{neighborhood}  of a row  is the set of rows intersecting ,
 and  
. The \emph{span}  of a column  
is the set of rows containing , .

\begin{figure}[h]
  \centering
\includegraphics[width=11cm]{forbid-all.pdf}
\caption{Forbidden induced subgraphs for the row-column intersection graph of  to verify C1P.}
 \label{forbid}
\end{figure}

\begin{theorem}[\cite{LB62}, Theorem 4]
\label{thm-fordid}
A 0-1 matrix  verifies the C1P if and only 
if its row-column intersection 
graph does not contain a forbidden induced subgraph of the form I, II, III, IV, 
or V (Figure \ref{forbid}).
\end{theorem}

\begin{property}
From Theorem \ref{thm-fordid}, a set  is a MCS
if the row-column intersection graph  contains a 
subgraph of the 
form I, II, III, IV, or V; and for any , 
 does not contain a subgraph of the form I, II, III, 
IV, or V. 
\label{feat-mcs}
\end{property}

Given a MCS , a forbidden induced subgraph 
contained in  
 is said to be \emph{responsible} for the MCS 
. 
If this forbidden induced subgraph is of the form I (resp. II; III; IV; V), 
we simply say that  is a MCS of the form I (resp. II; III; IV; V).

\begin{definition}
A row of a MCS   that intersects all other rows of 
is called a \emph{kernel} of . In a forbidden  induced subgraph  
responsible for , any kernel of  constitutes a  black 
vertex that is connected to all other  black vertices.
\label{def-kernel}
\end{definition}

\begin{property}
Note that an induced subgraph of the form II, III, IV, or V necessarily contains
at least one kernel, while an induced subgraph of the form I contains no kernel.
\label{mcs-kernel}
\end{property}

We denote by , the subgraph of  induced by the set of rows 
, thus containing only black vertices.

\noindent
{\bf Graph sizes.}  has  vertices and at most  
edges, while   has  vertices and at most  
edges.

\section{A global algorithm}

Our algorithm to decide if a row  of a 0-1 matrix 

belongs to at least one MCS, is based on a sequence of algorithms for finding 
a forbidden subgraph of  responsible for a MCS 
containing . It looks for forbidden subgraph of the form I, III, II, IV, V, 
in the following order: 1. MCS of type I, 2. MCS of size  (types IV or V),
3. MCS of type II, 4. MCS of type III, 5. MCS of type IV and size larger or 
equal to , and MCS of type V and size larger or equal to . See Figure  
\ref{Algo}  for an overview. The steps 1 to 4 are based on straightforward 
brute-force algorithms, while the two last steps relies to a reduction to the 
detection of induced chordless cycles in ad-hoc graphs.

In the following, we simply write 
 as  and  as .

\begin{figure}[htb]
  \centering
\includegraphics[width=10cm]{algo.pdf}
\caption{The different steps of the algorithm: in each case, when row  has 
a specific location in the forbidden induced subgraph that is looked for, this 
location is indicated in bold character. Other rows and columns of the 
forbidden induced subgraph are indicated in grey color characters.}
 \label{Algo}
\end{figure}

\subsection{Step 1: Forbidden induced subgraph I}

We first test if  belongs to a MCS  of the form I.
If it is true, then  belongs to an induced chordless cycle of 
of length at least  containing only black vertices. 
Such a cycle exists in  if and only if is also a chordless
cycle in  since  is the subgraph of  induced 
by the set of rows .
Thus it suffices to search for an induced chordless cycle 
in .

\begin{algorithm}[htpb]   
\rule{11.7cm}{0.01cm}
\caption{Check\_I (, ) -- }
\rule{11.7cm}{0.01cm}
\\               
{\bf Input:} a row , the subgraph .\\
{\bf Output:} returns a MCS  given by a forbidden induced subgraph of 
the form I containing  if such a MCS exists, otherwise returns  "NO''.
\rule{11.7cm}{0.01cm}               
\begin{algorithmic}[1] 
\FOR{ any  of  containing }
\STATE Consider the graph  obtained from  after removing the two 
internal vertices of the  and their neighborhood from the graph, and 
consider the extremities  and  of the 
\IF{ there exists a -path in }
\STATE find a chordless path  in this graph linking  and .
\STATE return the set of vertices of the  plus the set of vertices of 
\ENDIF
\ENDFOR
\STATE return "NO''
\end{algorithmic}
\rule{11.7cm}{0.01cm}
\end{algorithm}



\begin{proposition}
  Algorithm Check\_I is correct and runs in worst case  time.
\end{proposition}
\begin{proof}
The correctness of Algorithm \mbox{Check\_I} comes from the fact that,  
 is contained in a MCS of the form I if and only if  belongs to 
an induced chordless cycle of  of length at least  whose set
of vertices  constitutes the MCS (Figure \ref{Algo}.I). 
A  of  is an induced chordless path of  containing
 vertices.
In this case, Algorithm \mbox{Check\_I} returns such a set of vertices
since an induced chordless cycle of  of length at least 
containing  is a  containing  whose extremities are
linked by a chordless path in the subgraph of  that does not contain
the neighborhood of the internal vertices of the . This set  cannot contain a smaller subset of rows that is a MCS, as no
subset of  can be a MCS of the form I, or a MCS of any other form
because of Property \ref{mcs-kernel}.

Algorithm \mbox{Check\_I} might be implemented in . 
The test performed on a give  containing  (lines 2-5 of the algorithm)
can be achieved in   as follows: 
removing the neighborhood of its internal vertices
might be done in  time, and finding a chordless path
between the two extremities might be performed using Dijkstra's
algorithm in  time.
Enumerating all  containing  might be done in time  using
a BFS from  stopping at depth .
Eventually, the whole algorithm is in   time.
\end{proof}


\noindent
{\bf Precomputation.} In the following steps, we assume that the following 
precomputations have been achieved:
\begin{itemize}
\item For any triplet of rows  that are pairwise intersecting,
\emph{i.e} each couple is an edge in , 
 and  are precomputed ; 
\item Two rows   and  are  \emph{overlapping} if  
 and  and 
.
The overlapping relation between any couple of rows is precomputed ; 
\item For any quadruplet of rows  such that , and 
 overlap ,   is precomputed.
\end{itemize}

All those precomputations can simply be performed in  time
using straightforward algorithms, that is, scanning the  columns of
the input matrix for each triplet or quadruplet of rows.  


\subsection{Step 2: Forbidden induced subgraph responsible for a MCS of size }

We test here if  belongs to a MCS of size . A MCS of size  is 
necessarily caused by a forbidden induced subgraph of the form IV or V.
As a consequence, the following property is immediate.

\begin{property}
A MCS of size  is always composed of  rows that are pairwise 
overlapping.
\label{mcs-3}
\end{property}


\begin{algorithm}[htpb]                    
\rule{11.7cm}{0.01cm}
\caption{Check\_IV\_V\_3 (, ) -- }
\rule{11.7cm}{0.01cm}
\\               
{\bf Input:} a row , the row-column intersection graph .\\
{\bf Output:} returns a MCS  of size  given by a forbidden 
induced subgraph of the form IV or V containing  if such a MCS exists, 
otherwise returns  "NO''.
\rule{11.7cm}{0.01cm}              
\begin{algorithmic}[1] 
\FOR{ any couple  of black vertices that both overlap , and overlap each other}
\IF{  and 
and }
\STATE return 
\ENDIF
\IF{  and 
and }
\STATE return 
\ENDIF
\ENDFOR
\STATE return "NO''
\end{algorithmic}
\rule{11.7cm}{0.01cm}
\end{algorithm}

\begin{proposition}
Algorithm \mbox{Check\_IV\_V\_3} is correct and runs in  time.
\end{proposition}
\begin{proof}
The correctness of Algorithm \mbox{Check\_IV\_V\_3} comes from the fact that, 
 is contained in a MCS of size  if and only if this MCS is caused 
by a forbidden induced subgraph of the form IV or V (Property \ref{mcs-3}). 
Thus,  should belong to a triplet of rows  that are pairwise
overlapping, and satisfy the conditions given in:
\begin{itemize}
\item either, line 2 of the algorithm to produce a forbidden induced subgraph of the form IV (left-end graph in Figure \ref{Algo}.IV\_V\_3),
\item or, line 5 of the algorithm to produce a forbidden induced subgraph of the form V (right-end graph in Figure \ref{Algo}.IV\_V\_3).
\end{itemize}
In both cases, Algorithm \mbox{Check\_IV\_V\_3} returns the set  
as a MCS if such a set of rows exists. This set cannot contain a smaller subset
of rows that is a MCS as  is the minimum size of any MCS.

Algorithm Check\_IV\_V\_3 runs in  time since, given ,
there might be  couples  on which the tests performed
(lines 2-8 of the algorithm) might be
achieved in , thanks to the precomputations that have been done.
\end{proof}

\subsection{Step 3: Forbidden induced subgraph II}

We test here if  belongs to a MCS of the form II, with
the assumption that  is not contained in any MCS of size .
Note that such a MCS is of size .

\begin{algorithm}[htpb]                    
\rule{11.7cm}{0.01cm}
\caption{Check\_II\_4 (, )--}
\rule{11.7cm}{0.01cm}
\\               
{\bf Input:} a row , the row-column intersection graph .\\
{\bf Assumption:}  is not contained in a MCS of size .\\
{\bf Output:} returns a MCS  given by a forbidden induced subgraph of 
the form II containing  if such a MCS exists, otherwise returns  "NO''.
\rule{11.7cm}{0.01cm}
\begin{algorithmic}[1]
\FOR{ any triplet  of black vertices such that  overlap }
\IF{ there are no edges , , and  in }
\STATE return 
\ENDIF
\ENDFOR 
\FOR{ any triplet  of black vertices of such that  overlaps , and  overlap }
\IF{ there are no edges , , or  in }
\STATE return 
\ENDIF
\ENDFOR 
\STATE return "NO''
\end{algorithmic}
\rule{11.7cm}{0.01cm}
\end{algorithm}



\begin{proposition}
Algorithm \mbox{Check\_II\_4} is correct and  runs in  time.
\end{proposition}
\begin{proof}
The correctness of Algorithm \mbox{Check\_II\_4} comes from the fact that,  if  belongs to a MCS of the form II,
then  should belong to a quadruplet of rows  such that one
these rows is a kernel, and the three other rows do not intersect each other. 
Thus, the row  is:
\begin{itemize}
\item either, a kernel of the MCS, tested in lines 1-5 of the algorithm (left-end graph 
in Figure \ref{Algo}.II\_4),
\item or, not a kernel of the MCS tested in lines 6-10 of the algorithm (right-end 
graph in Figure \ref{Algo}.II\_4).
\end{itemize}
In both cases, Algorithm \mbox{Check\_II\_4} returns the set 
 as a MCS if such a set of rows exists. 
This set cannot contain a smaller subset of rows that is a MCS as this 
subset would be a subset of  rows that cannot satisfy Property \ref{mcs-3}.

Algorithm Check\_IV\_V\_4 runs in  time since all the tests
performed on a given triplet  in lines 2-4 and 7-9 of 
algorithm can be achieved in , and given  there might be 
 such triplets.
\end{proof}

\subsection{Step 4: Forbidden induced subgraph III}

We test here if  belongs to a MCS of the form III, 
with the assumption that   is not contained in a MCS of size .
Note that such a MCS is of size .

\begin{algorithm}[htpb]                    
\rule{11.7cm}{0.01cm}
\caption{Check\_III\_4 (, )--}
\rule{11.7cm}{0.01cm}
\\
{\bf Input:} a row , the row-column intersection graph .\\
{\bf Assumption:}   is not contained in a MCS of size .\\
{\bf Output:} returns a MCS  given by a forbidden induced subgraph of 
the form III containing  if such a MCS exists, otherwise returns  "NO''.
\rule{11.7cm}{0.01cm}
\begin{algorithmic}[1]
\FOR{ any triplet  of black vertices such that  overlap , and }
\IF{ there are no edge  in , and , and }
\STATE return 
\ENDIF
\ENDFOR 

\FOR{ any triplet  of black vertices of such that   overlaps , and  overlap , and  , and  is not a MCS}
\IF{ there are no edge  in , and , and  }
\STATE return 
\ENDIF
\ENDFOR 
\FOR{ any triplet  of black vertices of such that  overlaps , and  overlap , and  }
\IF{ there are no edge  in , and , and }
\STATE return 
\ENDIF
\ENDFOR 
\STATE return "NO''
\end{algorithmic}
\rule{11.7cm}{0.01cm}
\end{algorithm}

\begin{proposition}
Algorithm \mbox{Check\_III\_4} is correct and runs in  time.
\end{proposition}
\begin{proof}
The correctness of Algorithm \mbox{Check\_III\_4} comes from the fact that, 
 belongs to a MCS of the form 
III if and only if  should belong to a quadruplet of rows 
 included in an induced subgraph of the form III
such that two of these rows are kernels of the subgraph, and one of these 
kernels contains a column of the induced subgraph that is not shared 
with any of the other rows. Let us call this kernel kernel\_1, and the 
other kernel kernel\_2. For example in the left-end graph 
in Figure \ref{Algo}.III\_4, kernel\_1=, and kernel\_2=.

Thus, the row  is:
\begin{itemize}
\item either, kernel\_1, tested in lines 1-5 of the algorithm (left-end graph 
in Figure \ref{Algo}.III\_4),
\item or, not a kernel, tested in  lines 6-10 of the algorithm (middle 
graph in Figure \ref{Algo}.III\_4).
\item or, kernel\_2, tested in  lines 11-15 of the algorithm (right-end 
graph in Figure \ref{Algo}.III\_4).
\end{itemize}
In the first, and third cases, the set   cannot be a MCS
because such a set cannot satisfy Property \ref{mcs-3}
In all cases, Algorithm \mbox{Check\_III\_4} returns the set 
 as a MCS if 
such a set of rows exists, and  is not a MCS 
(in the second case). 
Since we made the assumption that  is not contained in a MCS of size ,
there cannot exists a smaller subset of   containing 
that is a MCS.

Algorithm Check\_III\_4 runs in  time using a similar proof
as the complexity proof for Check\_IV\_V\_4: all the tests performed by 
the algorithm (lines 2-4, 7-9, and 12-14 of the algoritms) on a given triplet 
 are achieved in  thanks to the precomputations, 
and given  there might be  such triplets.
\end{proof}

\subsection{Step 5: Forbidden induced subgraph IV}

We test here if  belongs to a MCS of the form IV, with the assumption that  
 is contained, neither in a MCS of size , nor in a MCS of type I.
Depending on whether the size of the MCS is  or larger than , 
we describe two algorithms.

\subsubsection{MCS of size }

We first test if  belongs to a MCS of the form IV of size .
We look for a triplet of rows  such that 
the set  is a MCS of the form IV (Figure \ref{Algo}.IV\_4).
In an induced subgraph of the form IV containing  rows  , 
two rows are kernels, and in that case,  is either a kernel of the MCS, 
or not. If  is a kernel, then it is either a kernel --called kernel\_1-- 
containing a column of the induced subgraph that is not shared with any 
of the other rows , or not --called kernel\_2--. For example, in the 
left-end graph in Figure 4.IV\_4, the two kernel are the two central black
vertices of the graph: the top one is a kernel\_1, and the bootom one a
 kernel\_2.
Algorithm \mbox{Check\_IV\_4} looks for each of these configurations: 
\begin{itemize}
\item   is a kernel\_1, tested in lines 1-5 of the algorithm;  
\item  is not a kernel,tested in lines 6-10 of the algorithm; 
\item  is a kernel\_2, tested in lines 11-15 of the algorithm.
\end{itemize}
The proof of the correctness of Algorithm \mbox{Check\_IV\_4} is similar to the
proof for Algorithm \mbox{Check\_III\_4}.

\begin{algorithm}[htpb]                      
\rule{11.7cm}{0.01cm}

\caption{Check\_IV\_4 ( , ) -- } 
\rule{11.7cm}{0.01cm}
\\
{\bf Input:} a row , the row-column intersection graph .\\
{\bf Assumption:}   is not contained in a MCS of size .\\
{\bf Output:} returns a MCS  of size  given by a forbidden 
induced subgraph of the form IV containing  if such a MCS exists, otherwise returns  "NO''.
\rule{11.7cm}{0.01cm}

\begin{algorithmic}[1]
\FOR{ any triplet  of black vertices such that  
are connected to , and }
\IF{ there are no edge  in , and , and , and , and
}
\STATE return 
\ENDIF
\ENDFOR 

\FOR{ any triplet  of black vertices of such that  
is connected to , and  are connected to , and  , and  is not a MCS}
\IF{ there are no edge  in , and , and , and , and
}
\STATE return 
\ENDIF
\ENDFOR 
\FOR{ any triplet  of black vertices of such that  
is connected to , and  are connected to , and  }
\IF{ there are no edge  in , and , and , and , and
}
\STATE return 
\ENDIF
\ENDFOR 
\STATE return "NO''

\end{algorithmic}
\rule{11.7cm}{0.01cm}
\end{algorithm}


\begin{proposition}
Algorithm  is correct and runs in  time.
\end{proposition}
\begin{proof}
The proof for Algorithm \mbox{Check\_IV\_4} is similar to the
proof for Algorithm \mbox{Check\_III\_4}.
\end{proof}


\subsubsection{MCS of size larger than }

We test here if  belongs to a MCS of the form IV of size larger than .
A MCS of the form IV of size larger than  contains one and only
one kernel. Depending on whether  is the kernel or not, we
distinguish two cases here.\\

{\bf Case 1: If row  is the kernel of the MCS}\\

Algorithm \mbox{Check\_IV} recovers a MCS  of the form IV of size 
larger than  containing  as a kernel, with the assumption that   
is not contained in a MCS of size  (Figure \ref{Algo}.IV).
The principle of the algorithm relies in first choosing the column 
, of the
forbidden induced subgraph of type IV responsible for , that is 
contained in , and in no other row of the MCS  
(see Figure \ref{Algo}.IV).
Next, it considers the subgraph  of  induced by the set of black 
vertices (rows) that are neighbors of , but do not contain the column .
We denote this subgraph by  .
Then, it looks for a set of rows , constituting a chordless path in , 
such that  is a MCS of the form IV.


\begin{algorithm}[htpb]                      
\rule{11.7cm}{0.01cm}

\caption{Check\_IV (, ) -- } 
\rule{11.7cm}{0.01cm}
\\
{\bf Input:} a row , the row-column intersection graph .\\
{\bf Assumption:}   is not contained in a MCS of size .\\
{\bf Output:} returns a MCS  of size larger that  given by a 
forbidden induced subgraph of the form IV whose kernel is  if such a 
MCS exists, otherwise returns  "NO''.
\rule{11.7cm}{0.01cm}

\begin{algorithmic}[1]
\FOR{ any column }
\STATE 
\FOR{any connected component  of }
\STATE pick a a couple  of black vertices in  that satisfies 
1)  and  are not connected, and 2)   overlap .
\STATE find a chordless path  in  linking  and 
\STATE pick the smallest subpath  of  linking two vertices  and 
, such that the couple  also satisfies 1) and 2)
\STATE return 
\ENDFOR
\ENDFOR
\STATE return ``NO''
\end{algorithmic}
\rule{11.7cm}{0.01cm}
\end{algorithm}


\begin{proposition}
Algorithm  is correct and runs in  time.
\end{proposition}
\begin{proof}
 Note that, if the MCS exists, then all the rows belonging to the MCS, 
except , belong to a same 
connected component of . Thus, in each connected component of , the
algorithm looks for a chordless path  linking two vertices  
satisfying 1)  and  are not connected, and 2)   overlap 
, and 3)  does not contain any smaller subpath satisfying conditions 
1) and 2). 
These conditions are necessary and sufficient for the set  
to form the rows of a induced subgraph of the form . The set  
cannot contain a subset that is a MCS as such a smaller MCS should be:
\begin{itemize}
 \item either a MCS of size  including , which impossible by assumption, 
 \item or a MCS of type II or III necessarily including  as kernel,
 \item or a MCS of  type IV and size larger than   having  as kernel.
\end{itemize}
The two last cases are also impossible, since  would not have satisfy
condition 3) in these cases.

Next, there might be  columns  and up to 
couples  of black vertices to test before 
finding a valid couple  satisfying
the conditions in line 4 of the algorithm. 
Up to this point, the complexity is in . Assume now
that such a couple exist. Then finding a chordless path between 
and  might be done by searching for a shortest path between 
and  in the connected component  using Dijkstra's algorithm, 
which thus requires at worst  time. The path 
is of length at most , and thus identifying  and  is 
bounded by testing each pair on this path in , which requires at 
worst 
time. Thus, in total, the algorithm is  worst case time.
\end{proof}

{\bf  Case 2: If row  is not the kernel of the  MCS}\\

Algorithm \mbox{Check\_IV} recovers a MCS  of the form IV 
of size larger 
than  containing , but not as a kernel, with the assumptions that  is 
not contained in a MCS of size , and  does not belong to an induced 
chordless cycle of  (Figure 4.IV). The principle of the algorithm
consists in first choosing the kernel  of   among the black 
vertices (rows) neighbors of , and the column , of the
induced subgraph of type IV responsible for , that is contained 
in , but in no other row of the MCS.
(see Figure \ref{Algo}.IV).
Next, the algorithm calls Algorithm \mbox{Check\_IV} to look for the MCS 
 with , , , and  given as parameters.

\begin{algorithm}[htpb]                      
\rule{11.7cm}{0.01cm}
\caption{Check\_IV (, ) -- } 
\rule{11.7cm}{0.01cm}
\\
{\bf Input:} a row , the row-column intersection graph .\\
{\bf Assumption:}   is not contained in a MCS of size .\\
 does not belong to an induced chordless cycle of .\\
{\bf Output:} returns a MCS  of size larger that  given by a 
forbidden induced subgraph of the form IV containing  whose kernel is 
not  if such a MCS exists, otherwise returns  "NO''.
\rule{11.7cm}{0.01cm}
\begin{algorithmic}[1]                
\FOR{ any black vertex }
\FOR{ any column }
\STATE return Check\_IV(, , , )
\ENDFOR
\ENDFOR
\STATE return ``NO''
\end{algorithmic}
\rule{11.7cm}{0.01cm}
\end{algorithm}

Algorithm \mbox{Check\_IV} is called in Algorithm \mbox{Check\_IV}. It 
recovers a  MCS  of the form IV of size larger 
than  containing , given the row , the  kernel  of 
the MCS , and the column , of the
induced subgraph of type IV responsible for , that is contained 
in , but in no other row of the MCS (Figure \ref{Algo}.IV).

 \begin{algorithm}[htpb]                      
 \rule{11.7cm}{0.01cm}
 \caption{Check\_IV (, , , )-- }
 \rule{11.7cm}{0.01cm}
 \\
 {\bf Input:} two rows  and , and a column  such that
   .\\
 {\bf Assumption:}   is not contained in a MCS of size .\\
  does not belong to an induced chordless cycle of .\\
 {\bf Output:} returns a MCS  of size larger that  given by a 
 forbidden induced subgraph of the form IV containing  and , whose 
 kernel is  if such a MCS exists, otherwise returns  "NO''.
 \rule{11.7cm}{0.01cm}
 \begin{algorithmic}[1]                
\STATE 
 \STATE let  be the connected component of  to which  belongs.


 \STATE let  be the set of vertices .
 \STATE let  be the set of edges .

 \STATE let  be the graph such that  and .

 \STATE  =  Check\_I (, )
 \IF{ "NO''}
 \STATE return 
 \ENDIF
 \STATE return "NO''
 \end{algorithmic}
 \rule{11.7cm}{0.01cm}
 \end{algorithm}



\begin{proposition}
Algorithm  is correct, and runs in  time.
\end{proposition}
\begin{proof}


The correctness and the complexity of  follows
directly from the the correctness and the complexity of 
Algorithm  that is called in Algorithm .

The correctness of  comes from the fact that, 
 does not belong to any chordless cycle in the graph  
computed at line 2 of the algorithm by assumption.  
Then at line 6 of the algorithm, any
chordless cycle in the graph  containing vertex  necessarily
contains at least one edge  belonging to the set
. The number of edges belonging to the
set  in such a chordless cycle  cannot be greater than  
as any couple of such edges in the chordless cycle would induce a chord. 
Indeed, if  contains more
than one edge belonging to , any two such edges would have to
extremities in , one from each of the two edges, that are not
connected in the graph . These extremities would thus be linked 
by an edge in , creating a chord for the cycle  in the graph .

Therefore, the set of vertices of the chordless cycle  induces 
  a chordless path in  such that each vertex of  is connected to 
 vertex  by definition of the graph , and the extremities
  and  of  satisfy 1)  and  are not connected in , 
 and 2)   overlap , and 3)  does not contain any smaller 
 subpath satisfying conditions 1) and 2). These conditions are necessary 
 and sufficient for the set  
 to form the rows of an induced subgraph of the form , and this set  
 cannot contain a smaller MCS since such a  MCS would be:
\begin{itemize}
 \item either a MCS of size  including , 
 \item or a MCS of type II or III necessarily including  as kernel,
 \item or a MCS of  type IV and size larger than   having  as kernel.
\end{itemize}
The 3 cases are impossible,  since they would induce a chord from the
set  in the chordless cycle induced by  in the graph .

Algorithm  calls Algorithm . 
Both algorithms have the same time complexity in  time.
It follows immediately that  Algorithm 
runs  in  time.
\end{proof}
 
\subsection{Step 6: Forbidden induced subgraph V}
 
We test here if  belongs to a MCS of the form V, with the assumption that  
 is contained neither in a MCS of size , nor in a MCS of type I.
Depending on whether the size of the MCS is ,  or larger than , 
we describe three algorithms.

\subsubsection{MCS of size  or }

We first test if  belongs to a MCS of the form V of size  or .
For a MCS of size 4, we look  for a triplet of rows 
 such that the set  is a MCS of the
form V. In such a case, we look for an induced
subgraph responsible for the MCS, containing  as four black 
vertices pairwise connectedr, and we can pick three different couples of
 such that each couple shares a column (white vertex) that 
is not shared with the two other of the MCS (see Figure \ref{Algo}.V\_4). 

\begin{algorithm}[htpb]                      
\rule{11.7cm}{0.01cm}
\caption{Check\_V\_4 ( , ) -- } 
\rule{11.7cm}{0.01cm}
\\
{\bf Input:} a row , the row-column intersection graph .\\
{\bf Assumption:}   is not contained in a MCS of size .\\
{\bf Output:} returns a MCS  of size  given by a forbidden 
induced subgraph of the form V containing  if such a MCS exists, otherwise returns  "NO''.
\rule{11.7cm}{0.01cm}
\begin{algorithmic}[1]
\FOR{ any triplet  of black vertices such that  
are connected to  , and are pairwise connected}
\IF{   is not a MCS, and , and }
\IF{}
\STATE return 
\ENDIF
\IF{}
\STATE return 
\ENDIF
\ENDIF
\ENDFOR 
\STATE return "NO''
\end{algorithmic}
\rule{11.7cm}{0.01cm}
\end{algorithm}

\begin{proposition}
Algorithm  is correct and runs in  time.
\end{proposition}
\begin{proof}
Algorithm \mbox{Check\_V\_4} looks for an induced subgraph with 
 black vertices , that are  pairwise connected 
to each other. 
These  black vertices should be such that there exist three different 
couples of vertices among them, such that two couples are disjoint and the 
third one (called couple\_kernel) overlaps the two first, and the  rows 
of each of these couples share a column that is not shared with the two 
other rows of the set. 
In this case, if  is not a MCS, then the subgraph induced
by  and the  columns (white vertices) connected to 
the   couples
of rows is of the form V, and is responsible for a MCS .
Algorithm \mbox{Check\_V\_4} looks for two cases, depending on whether
 belong to couple\_kernel (lines 3-5), or not (lines 6-8). 

Next, all the tests performed by 
Algorithm \mbox{Check\_V\_4} (lines 2-9 of the algoritm) on a given triplet 
 are achieved in  thanks to the precomputations, 
and given  there might be  such triplets.
Thus, Algorithm \mbox{Check\_V\_4} runs in  time.
\end{proof}


Next, for a MCS of size 5, we look  for a quadruplet of rows 
 such that the set  is a MCS of the
form V (Figure \ref{Algo}.V\_5). Algorithm Check\_V\_5 looks for an
induced subgraph of the form V, consisting of  rows (black vertices) 

that are pairwise connected, except for a on missing edge,
say   in ,
and three columns (white vertices) satisfying the configuration
of Figure \ref{Algo}.V\_5. 

\begin{algorithm}[htpb]                      
\rule{11.7cm}{0.01cm}
\caption{Check\_V\_5 ( , ) -- } 
\rule{11.7cm}{0.01cm}
\\
{\bf Input:} a row , the row-column intersection graph .\\
{\bf Assumption:}   is not contained in a MCS of size  or .\\
{\bf Output:} returns a MCS  of size  given by a forbidden 
induced subgraph of the form V containing  if such a MCS exists, otherwise returns  "NO''.
\rule{11.7cm}{0.01cm}
\begin{algorithmic}[1]                
\FOR{any quadruplet  of black vertices such that 
  are pairwise connected, except for one edge 
   in  missing}
\IF{ is C1P}
\FOR{any pair  in }

\IF{, and , and }
\STATE  return 
\ENDIF
\ENDFOR
\ENDIF
\ENDFOR
\STATE  return "NO''
\end{algorithmic}
\rule{11.7cm}{0.01cm}
\end{algorithm}

\begin{proposition}
Algorithm  is correct and runs in  time.
\end{proposition}
\begin{proof}
Algorithm \mbox{Check\_V\_5} looks for an induced subgraph with 
 black vertices , that are  pairwise connected, 
except for one missing edge   
in .
The  black vertices that belong to the set with , should correspond to
a set of rows that is C1P. Moreover, there should exist two particular 
rows (black vertices) of the set, with three columns (white vertices)
that satisfy the conditions on line 4 of the algorithm in order to fit 
the configuration depicted in Figure \ref{Algo}.V\_5.

Next, all the tests performed by 
Algorithm \mbox{Check\_V\_5} (lines 2-8 of the algoritm) on a given quatruplet 
 are achieved in  thanks to the precomputations, 
and given  there might be  such triplets.
Thus, Algorithm \mbox{Check\_V\_5} runs in  time.
\end{proof}


\subsubsection{MCS of size larger than }

A MCS of the form V of size larger than  contains exactly two kernels.
Depending on whether  is a kernel or not, we distinguish two cases.\\


{\bf Case 1: If row  is a kernel of the MCS}

Algorithm \mbox{Check\_V} recovers a MCS  of the form V of size 
larger than  containing  as a kernel, with the assumption that   
is not contained in a MCS of size , or  (Figure \ref{Algo}.V).
The principle of the algorithm is similar to Algorithm \mbox{Check\_IV}. 
It relies in first choosing the second kernel  of the MCS, and the column 
, of the induced subgraphof type V responsible for , that is 
contained in both  and , but in no other row of the MCS  
(see Figure \ref{Algo}.V).
Next, it considers the subgraph  of  induced by the set of black 
vertices (rows) that are neighbors of  and , but do not contain 
. We denote this subgraph by . Then, it looks for 
a set of rows , constituting a chordless path in ,
 such that  is a MCS of the form V.


\begin{algorithm}[htpb]                      
\rule{11.7cm}{0.01cm}
\caption{Check\_V (, ) -- } 
\rule{11.7cm}{0.01cm}
\\
{\bf Input:} a row , the row-column intersection graph .\\
{\bf Assumption:}   is not contained in a MCS of size , or .\\
{\bf Output:} returns a MCS  of size larger that  given by a 
forbidden induced subgraph of the form V such that  is one of its kernel,
if such a MCS exists, otherwise returns  "NO''.
\rule{11.7cm}{0.01cm}
\begin{algorithmic}[1]
\FOR{ any black vertex }
\FOR{ any column }
\STATE 
\FOR{any connected component  of }
\STATE pick a a couple  of black vertices in  that satisfies 
1)  and  are not connected, and 2) , 
and 3) .
\STATE find a chordless path  in  linking  and 
\STATE pick the smallest subpath  of  linking two vertices  and 
, such that the couple  also satisfies 1) and 2) and 3)
\STATE return 
\ENDFOR
\ENDFOR
\ENDFOR
\STATE return ``NO''
\end{algorithmic}
\rule{11.7cm}{0.01cm}
\end{algorithm}

\begin{proposition}
Algorithm  is correct and runs in  time.
\end{proposition}
\begin{proof}
The proofs are similar to the proofs for the correctness and the
complexity of  Algorithm  as the two algorithms
are based on the same principle. However, here the complexity is 
multiplied by a factor  due to considering all black vertices 
.
\end{proof}

{\bf Case 2: If row  is not a kernel of the MCS}

Algorithm Check\_V recovers a MCS {\cal S} of the form V of size 
larger than  containing r, but not as a kernel, with the assumptions 
that  is not contained in a MCS of size  or , and r does not 
belong to an induced chordless cycle of  (Figure \ref{Algo}.V). 

The principle of the algorithm is similar to the principle 
of Algorithm Check\_IV.
It consists in first choosing the two kernels  of {\cal S} among the 
black vertices (rows) neighbors of , and the column , of the 
induced subgraph responsible for {\cal S}, that is contained in both
 and , but in no other row of the MCS. Next, the algorithm calls 
Algorithm Check\_V to look for the MCS  {\cal S} with , , , 
and  given as parameters.

\begin{algorithm}[htpb]                      
\rule{11.7cm}{0.01cm}
\caption{Check\_V (, ) -- } 
\rule{11.7cm}{0.01cm}
\\
{\bf Input:} a row , the row-column intersection graph .\\
{\bf Assumption:}   is not contained in a MCS of size , .\\
 does not belong to an induced chordless cycle of .\\
{\bf Output:} returns a MCS  of size larger that  given by a 
forbidden induced subgraph of the form V containing , but not as a kernel,
if such a MCS exists, otherwise returns  "NO''.
\rule{11.7cm}{0.01cm}
\begin{algorithmic}[1]                
\FOR{ any couple of connected black vertices }
\FOR{ any column }
\STATE return Check\_V (, , , )
\ENDFOR
\ENDFOR
\STATE return ``NO''
\end{algorithmic}
\rule{11.7cm}{0.01cm}
\end{algorithm}

Algorithm Check\_V is called in Algorithm Check\_V. It recovers a MCS
 of the form V of size larger than 5 containing , given the 
row ,  the kernels  and  of the MCS, and the column , of the 
induced subgraph responsible for , that is contained in  and 
, but in no other row of the MCS.

\begin{algorithm} [htpb]                   
\rule{11.7cm}{0.01cm}
\caption{Check\_V (, , , )-- }
\rule{11.7cm}{0.01cm}
\\
{\bf Input:} three rows ,  and , and a column  such that
  .\\
{\bf Assumption:}   is not contained in a MCS of size ,  or .\\
 is not contained in a MCS of type .\\
 does not belong to an induced chordless cycle of .\\
{\bf Output:} returns a MCS  of size larger that  given by a 
forbidden induced subgraph of the form V containing , , and , and 
whose kernels are  and , if such a MCS exists, otherwise returns  "NO''.
\rule{11.7cm}{0.01cm}
\begin{algorithmic}[1] 
\STATE 
\STATE let  be the connected component of  to which  belongs.
\STATE let  be the set of vertices .
\STATE let  be the set of vertices .
\STATE let  be the set of edges .
\STATE let  be the set of vertices  , and  be the set of edges .
\STATE let  be the set of vertices  , and  be the set of edges .
\STATE let  be the graph such that  and  
\STATE  =  Check\_I (, )
\IF{  "NO''}
\STATE return 
\ENDIF
\STATE return "NO''
\end{algorithmic}
\rule{11.7cm}{0.01cm}
\end{algorithm}

\begin{proposition}
Algorithm  is correct and runs in  time.
\end{proposition}
\begin{proof}
In order to prove the correctness and the complexity of Algorithm 
, we need to prove the correctness and give the
complexity of Algorithm  that is called in 
.

The correctness of Check\_V comes from the fact that  does not
belong to any chordless cycle in the graph  computed at line 2 of the
algorithm by assumption. Let  be a chordless cycle in the graph  
containing vertex , computed at line 9 of the algorithm. 
Since  does not belong to an induced chordless cycle of the  by 
assumption, then  necessarily contains at least one edge belonging to 
the set .

\noindent
We first give two trivial but useful properties for the remaining of the proof:
\begin{itemize}
\item[(i)] For any two edges of , there always exists two extremities 
 and  of these edges, one in each edge, that are not disjoint in the graph 
 , 
i.e  
\item[(ii)] , and  .
\end{itemize}

\noindent
We also prove the following useful property: 
\begin{itemize}
\item[(iii)]  and . Let  there exists  such that  and
    Then, either  in
  which case , or  which implies
  that . The proof is similar for  
\end{itemize}

We now prove that the cycle  necessarily contains at most one edge of
the set . Indeed, if  contains two edges of
, let  be two disjoint extremities of these edges
(Property (i)). We can distinguish  cases according to the belonging
of  and  to the sets , ,  and , and we show 
in the following that, in all these cases, a chord is induced in the
chordless cycle  in the graph : contradiction.

\begin{enumerate}
\item If  (resp. ), then from Property (ii), 
 (resp.  ), and thus 
(resp. ).
\item If  (resp.  ), then   (resp. ).
\item If  (or the symmetric), then  .
\item If  (or the symmetric), then  from Property (iii),
 or , and thus  or  from cases 3. and 6.
\item If  (or the symmetric), then  from Property (iii), 
 or , and thus  or  from cases 3. and 6.
\item If  (resp.  ) (or the symmetric), then from Property (ii),   (resp.  ), and thus  (resp. ).
\item If  (or the symmetric), then from Property (iii), 
 or , and thus  or  from cases 1 and 5.
\end{enumerate}


\noindent
In consequence, there exits at most one edge, and then exactly one edge 
of the set  in the cycle  in the graph .
Next, let  be the only edge of  belonging to 
.
We show that  . Indeed, if 
(resp. ), then the set  (resp. ) 
satisfies the conditions
to be a MCS of type IV with  (resp. ) as kernel, which is impossible 
by assumption.


So, we have . Finally, removing the edge  
 from the cycle yields a chordless path  in  
containing  such that each vertex of  
is connected to vertices   and , and the extremities  and  of 
 satisfy 1)  and  are not connected, and 2) 
, and 3) .
and 4)  does not contain any smaller subpath satisfying conditions 
1) and 2) and 3). 
These conditions are necessary and sufficient for the 
set  to form the rows of an induced subgraph of the form V, 
and this set cannot contain a smaller MCS since such a MCS would be:
\begin{itemize}
 \item either a MCS of size  including  or , 
 \item or a MCS of type II or III necessarily including  or  as kernel,
 \item or a MCS of  type IV and size larger than   having  or  as 
kernel.
 \item or a MCS of  type IV and size larger than   having  and  as 
kernels.
\end{itemize}
The 3 cases are impossible,  since they would induce a chord from the
set  in the chordless cycle induced by  in 
the graph .

The correctness of Algorithm Check\_V follows
immediately from the correctness of Algorithm Check\_V.

Algorithm  calls Algorithm . 
Both algorithms have the same time complexity in  time.
It follows immediately that  Algorithm 
runs  in  time.
\end{proof}




\paragraph{Acknowledgment} We would like to thanks Nicolas Trotignon for 
his valuable comments on induced subgraphs and also Juraj Stacho for
his participation to some meeting on the subject.

\bibliographystyle{plain}
\bibliography{consecutives}
\end{document}
