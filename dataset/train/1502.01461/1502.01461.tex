\documentclass[11pt]{article}
\usepackage[utf8]{inputenc}
\usepackage[T2A]{fontenc}
\usepackage[english]{babel}
\usepackage{amsmath,amssymb,amsthm}
\selectlanguage{english}
\usepackage[x11names, rgb]{xcolor}
\usepackage{fullpage}
\usepackage{verbatim}
\usepackage{tikz}
\usepackage{bbold,xspace}
\usepackage[textwidth=23mm,colorinlistoftodos]{todonotes}
\usetikzlibrary{calc,arrows,shapes,backgrounds,patterns,fit,decorations,decorations.pathmorphing}







\newtheorem{theorem}{Theorem}
\newtheorem{lemma}{Lemma}
\newtheorem{remark}{Remark}
\newtheorem{definition}{Definition}
\newtheorem{corollary}{Corollary}

\newcommand{\tw}{{\mathbf{tw}}}
\newcommand{\adj}{\textrm{adj}}
\newcommand{\inc}{\textrm{inc}}
\newcommand{\card}{\textrm{card}}
\newcommand{\df}{\textrm{\rm def}}
\newcommand{\prefix}{\textrm{prefix}}
\newcommand{\suffix}{\textrm{suffix}}
\newcommand{\overlap}{\textrm{overlap}}

\newcommand{\cO}{\mathcal{O}}
\newcommand{\cOs}{\mathcal{O}^*}

\DeclareMathOperator{\operatorClassP}{P}
\newcommand{\classP}{\ensuremath{\operatorClassP}}
\DeclareMathOperator{\operatorClassNP}{NP}
\newcommand{\classNP}{\ensuremath{\operatorClassNP}}
\DeclareMathOperator{\operatorClassCoNP}{coNP}
\newcommand{\classCoNP}{\ensuremath{\operatorClassCoNP}}
\DeclareMathOperator{\operatorClassFPT}{FPT}
\newcommand{\classFPT}{\ensuremath{\operatorClassFPT}}
\DeclareMathOperator{\operatorClassW}{W}
\newcommand{\classW}[1]{\ensuremath{\operatorClassW[#1]}}
\DeclareMathOperator{\operatorClassParaNP}{Para-NP}
\newcommand{\classParaNP}{\ensuremath{\operatorClassParaNP}}
\DeclareMathOperator{\operatorClassXP}{XP}
\newcommand{\classXP}{\ensuremath{\operatorClassXP}}
\newcommand{\cost}{W}
\newcommand{\costfunction}{\omega}
\newcommand{\defproblemu}[3]{
  \vspace{1mm}
\noindent\fbox{
  \begin{minipage}{0.95\textwidth}
  #1 \\
  {\bf{Input:}} #2  \\
  {\bf{Question:}} #3
  \end{minipage}
  }
  \vspace{1mm}
}
\newcommand{\defparproblem}[4]{
  \vspace{1mm}
\noindent\fbox{
  \begin{minipage}{0.96\textwidth}
  \begin{tabular*}{\textwidth}{@{\extracolsep{\fill}}lr} #1  & {\bf{Parameter:}} #3 \\ \end{tabular*}
  {\bf{Input:}} #2  \\
  {\bf{Question:}} #4
  \end{minipage}
  }
  \vspace{1mm}
}




\newcommand{\problemdef}[4]{
  \vspace{5mm} 
  \setlength{\fboxrule}{2pt}
\noindent\fbox{
  \begin{minipage}{\textwidth}
  \begin{tabular*}{\textwidth}{@{\extracolsep{\fill}}lr} #1 
   \\ \end{tabular*}
  {\bf{Instance:}} #2  \\
   {\bf{Parameter:}} #3 \\
  {\bf{Question:}} #4
  \end{minipage}
  }
    \vspace{4mm}
} 

\pagestyle{plain}

\newcounter{firstbib}

\begin{document}


\title{Parameterized Complexity of Superstring Problems\thanks{The research leading to these results has received funding from the 
Government of the Russian Federation (grant 14.Z50.31.0030).}}

\author{
Ivan Bliznets\thanks{St.~Petersburg Department of Steklov Institute of Mathematics of the Russian Academy of Sciences} \addtocounter{footnote}{-1}
\and
Fedor V. Fomin \footnotemark {} \thanks{Department of Informatics, University of Bergen, Norway}  \addtocounter{footnote}{-2} 
\and
Petr A. Golovach \footnotemark {}  \footnotemark \addtocounter{footnote}{-2}
\and
Nikolay Karpov \footnotemark \addtocounter{footnote}{-1}
\and
Alexander S. Kulikov \footnotemark 
\and
Saket Saurabh \footnotemark {} \thanks{Institute of Mathematical Sciences, Chennai, India}
}


\maketitle
\sloppy
\begin{abstract}
In the  \textsc{Shortest Superstring} problem we are given a set of  strings   and integer  and the question is to decide whether there is a superstring  of length at most   containing all strings of  as substrings. We obtain several parameterized algorithms and complexity results for this problem. 

In particular, we give an algorithm which in time  finds a superstring of length at most  containing at least  strings of . We  complement this by the lower bound showing that such a  parameterization does not admit a polynomial kernel up to some complexity assumption. We also obtain several results about ``below guaranteed values" parameterization of the problem. We show that parameterization by compression admits a polynomial   kernel while parameterization ``below matching" is hard.  
\end{abstract}


\section{Introduction}
We consider the \textsc{Shortest Superstring} problem defined as follows:\\

\defproblemu{\textsc{Shortest Superstring}}{A set of  strings  over an alphabet  and
a non-negative integer~.}
{Is there a string  of length at most  containing all strings from  as substrings?}\\
This is a well-known NP-complete problem \cite{GareyJ79} with a range of practical  applications from 
DNA assembly \cite{EvansW11} till  data compression \cite{GallantMS80}. Due to this fact approximation algorithms
for it are widely studied. The currently best known approximation guarantee  is
due to Mucha~\cite{mucha}. At the same time the best known exact algorithms run in roughly 
 steps
and are known for more than 50 years already. More precisely, using known algorithms for the \textsc{Traveling Salesman} problem, \textsc{Shortest Superstring} can be solved either in time 

and the same space
by dynamic programming over subsets \cite{bellman,held} or in time  and only
polynomial space
by inclusion-exclusion~\cite{karp1982dynamic,kohn1977}
(here,  hides factors that are polynomial in the input length, i.e., ).
Such algorithms can only be used in practice to solve instances of very moderate size. Stronger upper bounds are known for a special case when input strings have bounded 
length~\cite{golovnev2013solving,golovnev2014solving}. There are
  heuristic methods for solving 
\textsc{Traveling Salesman}, and hence also \textsc{Shortest Superstring}, 
they are efficient in practice, however  have no efficient provable guarantee on the running time 
(see, e.g., \cite{concordeTSP}).

In this paper, we study the \textsc{Shortest Superstring} problem from the parameterized complexity point of view.
This field studies the complexity of computational problems with respect not only to input size,
but also to some additional parameters and tries to identify parameters of input instances that make
the problem tractable. 
Interestingly, prior to our work, except observations following from the known reductions to \textsc{Traveling Salesman}, not much about the parameterized complexity of  \textsc{Shortest Superstring} was known. We refer to the  survey of  Bulteau  et al. \cite{BulteauHKN14} for a nice overview of known results on parameterized algorithms and complexity of strings problems. Thus our work can be seen as the first non-trivial step towards the study of this interesting and important   problem from the perspective of parameterized complexity. 


\paragraph{Our results}
In this paper we study two types of parameterization for \textsc{Shortest Superstring} and present two kind of results. 
The first set of results concerns ``natural" parameterization of the problem. 
We consider the following  generalization
of \textsc{Shortest Superstring}:

\medskip
\defproblemu{\textsc{Partial Superstring}}{A collection (multiset) of strings  over an alphabet ,  and   non-negative integers .}
{Is there a string  of length at most  such that  is a superstring of a collection of at least  strings ?}\\
If  , then this is  \textsc{Shortest Superstring}.
Notice that  can contain copies of the same string and a string of  can be a substring of another string of the collection. For \textsc{Shortest Superstring}, such cases could be easily avoided, but for \textsc{Partial Superstring} it is natural to assume that we have such possibilities.

Here we show that \textsc{Partial Superstring} is fixed parameter tractable (FPT) 
when parameterized by  or~. We complement this result by showing that
it is unlikely that the problem admits a polynomial kernel with respect to these parameters.

\medskip

The second set of results concerns ``below guaranteed value" parameterization. 
 Note that
an obvious (non-optimal) superstring  of  is a string of length 
  formed by 
concatenating  all  strings from . 
For a superstring  of  the value  is called by \emph{compression
of  with respect to~}. Then finding a shortest superstring is equivalent to
  finding
an order of  such that the consecutive strings have the largest possible total overlap.
We first show that it is FPT with respect to  to check whether one can achieve a compression at least  by construction a kernel of size . 
We complement this result by a hardness result about ``stronger" parameterization.
Let us partition  input strings into  pairs such that the sum of the  resulting overlaps is maximized.
Such a partition can be found in polynomial time by constructing a maximum weight matching in an auxiliary graph.
Then this total overlap provides a lower bound on the maximum compression (or, equivalently, an upper bound
on the length of a shortest superstring). We show that already deciding whether at least one additional
symbol can be saved beyond the maximum weight matching value is already NP-complete. 

\section{Basic definitions and preliminaries}\label{sec:defs}
{\bf Strings.} Let  be a string. By  is denoted the \emph{length} of . By , where , is denoted the -th symbol of , and  for . We assume that  is the empty string if .
We denote  and   the \emph{-th prefix} and \emph{-th suffix} of  respectively for ;
 is the empty string.
Let  be strings. We write  to denote that  is a \emph{substring} of . If , then  is a \emph{superstring} of .
We write  and  to denote proper sub and superstrings.
For a collection of strings , a string  is a superstring of  if  is a superstring of each string in .
The \emph{compression measure} of a superstring  of a collection of strings  is .
 If , then ; otherwise, if  and , then , where . We denote by  the \emph{concatenation} of  and . 
For strings , we define the \emph{concatenation with overlap}  as follows. If , then . If  and , then , where  and .

We need the following folklore property  of superstrings.

\begin{lemma}\label{lem:overlap}
Let  be a superstring of a collection   of  strings. 
Let  be a set of inclusion maximal pairwise distinct strings of  such that each string of  is a substring of a string from .
Let also   for  and assume that .
Then  is a superstring of  of length at most . 
\end{lemma}




\medskip
\noindent
{\bf Graphs.}
We consider finite directed and undirected graphs without loops or multiple
edges. The vertex set of a (directed) graph  is denoted by , 
the edge set of an undirected graph and the arc set of a directed graph  is denoted by .
To distinguish edges and arcs, the edge with two end-vertices   is denoted by , and we write  for the corresponding arc.
For an arc ,  is the \emph{head} of  and  is the tail. 
Let 
 be a directed graph.
For a vertex , we say that  is an \emph{in-neighbor} of  if . The set of all in-neighbors of  is denoted by . The \emph{in-degree} .
Respectively,  is an  \emph{out-neighbor} of  if , the set of all out-neighbors of  is denoted by , and the \emph{out-degree} .
For a directed graph , a (directed) \emph{trail} of \emph{length}  is a sequence  of vertices and arcs  of  such that , ,
the arcs   are pairwise distinct, and 
for , . 
We omit the word ``directed'' if it does not create a confusion. Slightly abusing notations we often write a trail as a sequence of its vertices  or arcs .
If  are pairwise distinct, then  is a (directed) path. Recall that a path of length  is a \emph{Hamiltonian} path.
For an undirected graph , 
a set  is a \emph{vertex cover} of  if for any edge  of ,  or . A set of edges  with pairwise distinct end-vertices is a \emph{matching}.

We consider the following auxiliary problem:\\
\defproblemu{\textsc{Long Trail}}{A directed graph  and a non-negative integer .}
{Is there a trail of length at least  in ?}

\begin{lemma}\label{lem:trail}
\textsc{Long Trail} is \classNP-complete. In particular, the problem is \classNP-complete if .
\end{lemma}

\begin{proof}
We reduce the \textsc{Hamiltonian Path} problem for directed graphs that is well known to be \classNP-complete (see, e.g., \cite{GareyJ79}). Let  be a directed graph with  vertices. We construct the graph  as follows.
\begin{itemize}
\item For each , construct two vertices  and an arc .
\item For each , construct an arc .
\item Construct two vertices  and for each , construct arcs . 
\end{itemize}
We claim that  has a trail of length at least  if and only if  has a Hamiltonian path.

Suppose that  has a Hamiltonian path . Then the trail  in  has length .

Assume that  has a trail  of 
length at least . Without loss of generality we can assume that  is the first vertex of  and  is the last. To see it, suppose that  is the first vertex of . Notice that 
 is not in , because . If  for , then  we can consider the extended trail . If   for , then let  be the next vertex in  after . We consider the path  obtained from  by the replacement of  and  by  and  respectively. Clearly,  has the same length as . By the symmetric arguments, we obtain that we can assume that  is the last vertex of . We have that any vertex of  occurs exactly once in , because
 and  for .    
Moreover,  for each vertex ,  in , because  is the unique in-neighbor of  and  is the unique out-neighbor of  respectively for . Hence,  can be written as  for . It remains to observe that  is a Hamiltonian path in .
\end{proof}



\smallskip
\noindent
{\bf Parameterized Complexity.}
Parameterized complexity is a two dimensional framework
for studying the computational complexity of a problem. One dimension is the input size
and another one is a parameter. We refer to the books of Downey and Fellows~\cite{DowneyF13},
Flum and Grohe~\cite{FlumG06}, and   Niedermeier~\cite{Niedermeierbook06} for  detailed introductions  to parameterized complexity. 

Formally, a parameterized problem , where  is a finite alphabet, i.e., an instance of  is a pair  for  and , where  is an input and  is a parameter.
It is said that a problem is \emph{fixed parameter tractable} (or \classFPT), if it can be solved in time  for some function . 
A \emph{kernelization} for a parameterized problem is a polynomial algorithm that maps each instance   to an instance  such that 
\begin{itemize}
\item[i)]  is a yes-instance if and only if  is a yes-instance of the problem, and
\item[ii)] the size of  and  are bounded by  for a computable function . 
\end{itemize}
The output  is called a \emph{kernel}. The function  is said to be a \emph{size} of a kernel. Respectively, a kernel is \emph{polynomial} if  is polynomial. 
While a parameterized problem is \classFPT{} if and only if it has a kernel, it is widely believed that not all \classFPT{} problems have polynomial kernels.

 In particular, Bodlaender, Jansen and Kratsch~\cite{BodlaenderJK14} introduced techniques that allow to show that a parameterized problem has no polynomial kernel unless  .


Let  be a finite alphabet. An equivalence relation  on the set of strings  is called a \emph{polynomial equivalence relation} if the following two conditions hold:
\begin{itemize}
\item[i)] there is an algorithm that given two strings  decides whether  and  belong to
the same equivalence class in time polynomial in ,
\item[ii)] for any finite set , the equivalence relation  partitions the elements of  into a
number of classes that is polynomially bounded in the size of the largest element of .
\end{itemize}


Let  be a language, let  be a polynomial
equivalence relation on , and let    
be a parameterized problem.  An \emph{OR-cross-composition of  into } (with respect to ) is an algorithm that, given  instances  
of  belonging to the same equivalence class of , takes time polynomial in
 and outputs an instance  such that:
\begin{itemize}
\item[i)] the parameter value  is polynomially bounded in ,
\item[ii)] the instance  is a yes-instance for  if and only if at least one instance  is a yes-instance for  and .
\end{itemize}
It is said that  \emph{OR-cross-composes into}  if a cross-composition
algorithm exists for a suitable relation .

In particular, Bodlaender, Jansen and Kratsch~\cite{BodlaenderJK14} proved the following theorem.
\begin{theorem}[\cite{BodlaenderJK14}]\label{thm:BJK}
If an \classNP-hard language  OR-cross-composes into the parameterized problem ,
then  does not admit a polynomial kernelization unless
.
\end{theorem}


We use randomized algorithms for our problems. Recall that a \emph{Monte Carlo algorithm} is a randomized algorithm whose running time is deterministic, but whose output may be incorrect with a certain (typically small) probability. A Monte-Carlo algorithm is \emph{true-biased} (\emph{false-biased} respectively) if it always returns a correct answer when it returns a yes-answer (a no-answer respectively).
 
\section{\classFPT-algorithms for Partial Superstring}
In this section we show that \textsc{Partial Superstring} is \classFPT, when parameterized by  or . For technical reasons, we consider the following variant of the problem with weights:\\
\defproblemu{\textsc{Partial Weighted Superstring}}{A collection of strings  over an alphabet  with a weight function , and   non-negative integers  and .}
{Is there a string  of length at most  such that  is a superstring of a collection of  strings  with ?}\\
Clearly, if  and , then we have the  \textsc{Partial Superstring} problem.

\begin{theorem}\label{thm:FPT-weight}
\textsc{Partial Weighted Superstring} can be solved  
in time  by  a true-biased Monte-Carlo algorithm
and in time  by a deterministic algorithm for a collection of  strings of length at most .
\end{theorem}

\begin{proof}
First, we describe the randomized algorithm and then explain how it can be derandomized. The algorithm uses the color coding technique proposed by Alon, Yuster and Zwick~\cite{AlonYZ95}. 

If , then the problem is trivial, as the concatenation of any  strings of  has length at most  and we can greedily choose  strings of maximum weight. Assume that 

We color the strings of  by  colors  uniformly at random independently from each other.
Now we are looking for a string  that is a superstring of  strings of maximum total weight that have pairwise distinct colors. 


To do it, we apply the dynamic programming across subsets. 
For simplicity, we explain only how to solve the existence problem, but our algorithm can be modified to find a colorful superstring as well.
For , a string  and a positive integer ,
the algorithm computes the maximum weight  of a string  of length at most  such that
\begin{itemize}
\item[i)]  is a superstring of a collection of  strings  of pairwise distinct colors from , 
\item[ii)]  is inclusion maximal string of  and . 
\end{itemize}
If such a string  does not exist, then .

We compute the table of values of  consecutively for . To simplify computations, we assume that  for .
If , then for each string , we set  if  is colored by the unique color of  and . In all other cases .
Assume that  and the values of  are already computed if . 
Let

and

where  is the color of ; we assume that  if there is no substring  of  of color , and 
  if every string  is a sub or superstring of . 
We set .

We show that  is the maximum weight  of  strings of  colored by distinct colors that have a superstring of length at most ; if this value equals , then there is no string of length at most  that is a superstring of  string of  of distinct colors.

To prove this, it is sufficient to show that the values  computed by the algorithms are the maximum weights of strings of length at most  that satisfy (i) and (ii). The proof is by induction on the size of . It is straightforward to verify that it holds if . Assume that  and the  claim holds for sets of lesser size. 
Denote by   the maximum weight of a string  of length at most  that satisfies (i) and (ii).  
By the description of the algorithm, . We show that  . 

Let  be a collection of  strings of pairwise distinct colors from  that have  as a superstring. Denote by  a set of inclusion maximal distinct strings of  that contains  such that every string of  is a substring of a string of . Assume that  and  for .
Clearly, . 

Suppose that there is  such that . Let  be a color of . Then  is a superstring of  and the total weight of these string is . By induction,  and we have that .

Suppose now that  does not contain substrings of . Then . Let  and .  Observe that .
Notice that  is a superstring of . 
Because  has no substrings of , every string in  is a substring of any superstring of   and, therefore,  is a superstring of  of length at most .  The weight of  is . By induction, 
. Hence .

To evaluate the running time of the dynamic programming algorithm, observe that we can check whether  is a substring of  or find  in time  using, e.g., the algorithm of Knuth, Morris, and Pratt~\cite{KnuthMP77}, 
and we can construct the table of the overlaps and their sizes in time . 
Hence, for each , the values  can be computed in time , as 
. 
Therefore, the running time is . 

We proved that an optimal colorful solution can be found in time  . Using the standard color coding arguments (see~\cite{AlonYZ95}), we obtain that it is sufficient to consider  random colorings of  to claim that with probability , where  is a constant that does not depend on the input size and the parameter, we get a coloring for which  string of  that have a superstring of length at most  and the total weight at least  are colored by distinct colors if such a string exists.  It implies that 
\textsc{Partial Weighted Superstring} can be solved  
in time  by  our randomized algorithm.

To derandomize the algorithm, we apply the  technique proposed by Alon, Yuster and Zwick~\cite{AlonYZ95}
using the -perfect hash functions constructed by Naor, Schulman and Srinivasan~\cite{NaorSS95}. 
The random colorings are replaced by the family of at most  hash functions  that have the following property:
there is a hash function  that colors 
 string of  that have a superstring of length at most  and the total weight at least  by distinct colors if such a string exists.
It implies that \textsc{Partial Weighted Superstring} can be solved in time  deterministically.
\end{proof}

Because \textsc{Partial Superstring} is a special case of \textsc{Partial Weighted Superstring}, Theorem~\ref{thm:FPT-weight} implies that this problem is \classFPT\ when parameterized by . We show that the same holds if we parameterize the problem by .

\begin{corollary}\label{cor:FPT-ell}
\textsc{Partial Superstring} is \classFPT\ when parameterized by .
\end{corollary}

\begin{proof}
Consider an instance  of \textsc{Partial Superstring}.  Recall that  can contain several copies of the same string. We construct a set of weighted strings  by replacing a string  that occurs  times in  by the single copy of  of weight . Let .
Observe that there is a string  of length at most  such that  is a superstring of a collection of at least  strings of  if and only if there a string  of length at most  such that  is a superstring of a set of  strings of  of total weight at least .  A string of length at most  has at most  distinct substrings. We consider the instances 
 of \textsc{Partial Weighted Superstring} for . For each of these instances, we solve the problem using Theorem~\ref{thm:FPT-weight}.
It remains to observe that  there is a string  of length at most  such that  is a superstring of a set of  strings of  of total weight at least  if and only if one of the instances 
 is a yes-instance of \textsc{Partial Weighted Superstring}.
\end{proof}

We complement the above algorithmic results by showing that we hardly can expect that \textsc{Partial Superstring} has a polynomial kernel when parameterized by  or . 
\begin{theorem}\label{thm:no-kernel}
\textsc{Partial Superstring} does not admit a  polynomial kernel when parameterized by  or  for strings of length at most  over the alphabet  unless .
\end{theorem}

 \begin{proof}[Theorem~\ref{thm:no-kernel}]
We show that \textsc{Long Trail} OR-cross-composes into \textsc{Partial Superstring}. Recall that \textsc{Long Trail} was shown to be \classNP-complete in Lemma~\ref{lem:trail}.

We assume that two instances  and  of  \textsc{Long Trail} are equivalent if  and .
Consider equivalent instances  of   \textsc{Long Trail} for . Let  for . 
Let . Denote by  the string of length  that encodes a positive integer  in binary for .  
Let  for , i.e., . Notice that if , then the first symbol of  is '0'. 
For each arc  of , we construct a string . 
Clearly, .
We define
 and let , . 
We claim that there is  such that  has a trail of length  if and only if there is a string  of length at most  that is a superstring of  strings of .

Suppose that  there is  such that  has a  trail  .  Consider . Because the length of each  is  and , we obtain that . Hence,  is a string of length at most  that is a superstring of  strings.

Assume now that there is a string  of length at most  that is a superstring of  strings of . Because no string of  is a substring of another one, we can assume that 
 for some  by Lemma~\ref{lem:overlap}.  
We use the following properties of the overlap of two strings . Recall that if  of , then ,  and 
the first symbol of  is ''. It implies that  and   if and only if  for some 
and ,  for some . Since  and ,  for .
Hence,   is a  trail in some .

It remains to observe that  and  to complete the proof. 
\end{proof}


\section{Shortest Superstring below guaranteed values}\label{sec:below}
In this section we discuss \textsc{Shortest Superstring} parameterized by the difference between upper bounds for the length of a shortest superstring and the length of a solution superstring.  
For a collection of strings , the length of the shortest superstring is trivially upper bounded by . We show that \textsc{Shortest Superstring} admits a polynomial kernel when parameterized by the compression measure of a solution.

\begin{theorem}\label{thm:kernel}
\textsc{Shortest Superstring} admits a kernel of size  when parameterized by .
\end{theorem}

\begin{proof}
Let  be an instance of \textsc{Shortest Superstring}, . First, we apply the following reduction rules for the instance.

\medskip
\noindent
{\bf Rule~1.} If there are distinct elements  and  of  such that , then delete  and set . If , then return a yes-answer and stop.


\medskip
\noindent
{\bf Rule~2.} If there is  such that for any , , then delete  and set . 
If  and , then return a yes-answer and stop. If , then return a no-answer and stop.


\medskip
\noindent
{\bf Rule~3.} If there are distinct elements   and  of  such that , then return a yes-answer and stop. 

\medskip
It is straightforward to verify that these rules are \emph{safe}, i.e., by the application of a rule we either solve the problem or obtain an equivalent instance.
We exhaustively apply Rules~1--3. To simplify notations, we assume that  is the obtained set of strings and  and  are the obtained values of the parameters.
Notice that all strings in  are distinct and no string is a substring of another.
Our next aim is to bound the lengths of considered strings.

\medskip
\noindent
{\bf Rule~4.} If there is  with , then set  and . If , then return a no-answer and stop.

\medskip
To see that the rule is safe, recall that  is not a sub or superstring of any other string of , and  and  for any  distinct from  after the applications of Rule~3. As before, we apply Rule~4 exhaustively. 

Now we construct an auxiliary graph  with the vertex set  such that two distinct  are adjacent in  if and only if  or . 
We greedily select a maximal matching  in  and apply the following rule.

\medskip
\noindent
{\bf Rule~5.} If , then return a yes-answer and stop. 

\medskip
To show that the rule is safe, it is sufficient to observe that if ,  for  and , then the string  obtained by the consecutive concatenations with overlaps of    and then all the other strings of  in arbitrary order, then the compression measure of  is at least . 

Assume from now that we do not stop here, i.e., . Let  be the set of end-vertices of the edges of  and . Let . Clearly, . Observe that  is a vertex cover of  and  is an independent set of . 

For each ordered pair  of distinct , find an ordering  of the elements of  sorted by the decrease of  for . We construct the set  that contains the first  elements of the sequence.

For each , find an ordering  of the elements of  sorted by the decrease of  for . We construct the set  that contains the first  elements of the sequence.

For each , find an ordering  of the elements of  sorted by the decrease of  for . We construct the set  that contains the first  elements of the sequence.

Let 

\medskip
\noindent
{\bf Claim~.} {\it There is a superstring  of  with the compression measure at least  if and only if there is a superstring  of of  with the compression measure at least . }

\begin{proof}[Proof of Claim~]
If  is a superstring of  with the compression measure at least , then the string  obtained from  by the concatenation of  and the strings of  (in any order) is a superstring of  with the same compression measure as .

Suppose that  is a shortest superstring of  and the compression measure  at least . By Lemma~\ref{lem:overlap}, , where .
Let  
we assume that  are empty strings. 

We show that . Suppose that . If , then , because  and any two strings of  have the empty overlap.
By the same arguments, if   , then . Because , we have that .

Suppose that the shortest superstring  is chosen in such a way that  is minimum. We prove that  in this case.
To obtain a contradiction, assume that there is . We consider three cases.

\medskip
\noindent
{\bf Case 1.}  and . Recall that  in this case. Since ,  for  and .
In particular, it means that . As  and , there is  such that , i.e., . By the definition of , .  Consider 
 assuming that  (the other case is similar). Because 
, . Moreover, since  is a shortest superstring of , 
 and, therefore, . But then for  the set  constructed for  in the same way as the set  for , we obtain that 
; a contradiction.


\medskip
\noindent
{\bf Case 2.}  and . Then .
 Since ,  for  and . 
As  and , there is  such that , i.e., 
. By the definition of ,
.  
As in Case~1, consider  obtained by the exchange of  and  in the sequence of strings that is used for the concatenations with overlaps.   
In the same way, we obtain a contradiction with the choice of , because for  the set  constructed for  in the same way as the set  for , we obtain that 
.

\medskip
\noindent
{\bf Case 3.}  and . To obtain contradiction in this case, we use the same arguments as in Case~2 using symmetry. Notice that we should consider  instead of .

\medskip
Now  let , where  is the sequence of string of  obtained from  by the deletion of the strings of . Because we have that , the overlap of each deleted string with its neighbors is empty and, therefore,  has the same compression measure as .
\end{proof}

To finish the construction of the kernel, we define  and apply the following rule that is safe by Claim .

\medskip
\noindent
{\bf Rule~6.} If , then return a no-answer and stop. Otherwise, return the instance  and stop. 

\medskip
Since , . Because each string of  has length at most , the kernel has size . 

It is easy to see that  Rules~1-3 can be applied in polynomial time.  Then graph   and  can be constructed in polynomial time and, trivially, Rule~5 demands  time. The sets , , ,  and  can be constructed in polynomial time. Hence,  and  can be constructed in polynomial time. Because Rule~6 can be applied in time , we conclude that the kernel is constructed in polynomial time.
\end{proof}

Now we consider another upper bound for the length of the shortest superstring. Let  be a collection of strings. We construct 
 an auxiliary weighted graph  with the vertex set  by assigning the weight 
for any  two distinct . Let  be the size of a maximum weighted matching in . Clearly,  can be constructed in polynomial time and the computation of  is well known to be polynomial~\cite{Edmonds65}.
If  and  for , then the string  obtained by the consecutive concatenations with overlaps of    and then (possibly) the remaining string of  has the compression measure  at least . Hence,  is the upper bound for the length of the shortest superstring of . We show that it is \classNP-hard to find a superstring that is shorter than this bound.

\begin{theorem}\label{thm:ss_NP_mat}
\textsc{Shortest Superstring} is \classNP-complete for  even if restricted to the alphabet .
\end{theorem} 

\begin{proof}
We reduce \textsc{Long Trail} that was shown to be \classNP-complete in Lemma~\ref{lem:trail} for . Let  be an instance of the problem, . We assume that . Let  and .
Let also  and . Denote by  and  the strings of length  such that the first symbol of  is '0' and all the other symbols are '1'-s and  is a strings of '1'-s.
For a positive integer , denote by  the string of length  that encodes  in binary and by  the string of length  that encodes . 
Notice that  and , because . Hence, the first symbols of  and  are '0' if . Observe also that the last symbol of each  is '0'. 
For each , we consider the arc  of  and construct two strings:
\begin{itemize}
\item , 
\item .
\end{itemize}
We define .

We need the following properties of the strings of .
\begin{itemize}
 \item[i)] For ,  and .
 \item[ii)] For distinct ,  if the head of  coincides with the tail of  and   otherwise.
 \item[iii)] For distinct , .
\end{itemize}
These properties immediately follow from the definition of  and the facts that , the strings  start with '0', the last symbol of  is \lq{}0',  and , . It is sufficient to notice that if the overlap of two strings is not empty, then the -th prefix and suffix of the overlap is always  and  respectively.

Now we consider the weighted graph  and observe that  is a maximum weight matching in  and  by (i)--(iii).

We claim that  has a  trail of length at least  if and only if  has a superstring of length at most .

Suppose that the sequence of arcs  composes a trail in . Let .
Consider 
Since  for  by (i),  for  by (ii) and
 by (i), the compression measure of  is  and 
. Hence,  is a superstring of  of length at most .

Assume that  is a shortest superstring of  and . By Lemma~\ref{lem:overlap}, we can assume that  is obtained from a sequence  of the strings of  by the concatenations with overlaps.

We show that for every , either  or  are consecutive in . To obtain a contradiction, assume first that for some ,  occurs in  before  but these strings are not consecutive.  Let  be the predecessor of ,  be a predecessor of  and  be a successor of  in ; if  is the first element of  or  is the last element, we assume that  or  is the empty string respectively. 
Then  by (iii) and  by (ii) and (iii). Consider the sequence 
 obtained from  by the placement of  between  and . Because  by (1), the string  obtained from  by the concatenations with overlaps has length at  most ; a contradiction.  Suppose now that      
for some ,  occurs in  before  but these strings are not consecutive. Let   be the successor of ,  be a predecessor of  and  be a successor of  in ; if  is the  last element of , we assume that  is the empty string. 
We have that   by (iii) and  by (ii) and (iii). Consider the sequence 
 obtained from  by the placement of  between  and . Because  by (i), the string  obtained from  by the concatenations with overlaps has length at  most ; a contradiction. 

We decompose  into inclusion maximal subsequences  such that the overlap between any two consecutive strings in each subsequence is not empty.
Because  either  or  are consecutive in  for  and   and  by (i), each pair  is in the same subsequence. In particular, it means that the number of elements in each subsequence is even. Let  be the size of  and let  be the string obtained by the concatenation with overlaps from  for . Because ,  and the compression measure of  is at least , 
there is  such that the compression measure  of  is at least .

Suppose that  are in  for some . Then they are consecutive. If  has a predecessor  in , then , and if 
 has a successor  in , then  by (iii). Hence  and  in this case, but then by (i), ; a contradiction. It follows that , where distinct  and . 
Since for , the overlap between  and  is not empty,  and the head of the arc  is the tail of .
Hence,  is a  trail in . By (i) and (ii), we have that . Therefore, , i.e.,  has a  trail of length at least . 
\end{proof}





\bibliographystyle{splncs03}
\bibliography{superstring}



\end{document}
