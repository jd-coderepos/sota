\documentclass[11pt]{article}
\def\fileversion{v1.00}
\def\filedate{2000/10/06}
\NeedsTeXFormat{LaTeX2e}[1998/06/01]


\RequirePackage{amsmath}
\RequirePackage{amssymb}
\RequirePackage{amsthm}
\RequirePackage{epsfig}
\RequirePackage{mathrsfs}
\usepackage{amsfonts}
\usepackage{array}
\usepackage{enumerate}
\usepackage{graphicx}
\usepackage{longtable}
\usepackage{amsmath}
\usepackage{latexsym}
\usepackage{fancyhdr}
\usepackage[hidelinks]{hyperref}
\renewcommand{\footrulewidth}{2 pt}
\fancyheadoffset{2 pc}
\fancyfootoffset{2 pc}
\hypersetup{
 colorlinks = true,
 urlcolor = blue,
 linkcolor = blue,
 citecolor = red
}
\pagestyle{fancy}
\lhead{}
\chead{\textcolor{red}{\footnotesize{\textit{International Journal of Mathematics Trends and Technology - Volume 7 Number 2 - March 2014}}}}
\rhead{}
\lfoot{ISSN: 2231-5373}
\cfoot{\url{http://www.ijmttjournal.org}}
\rfoot{Page \thepage}
\fancypagestyle{plain}{\lhead{}
\chead{\textcolor{red}{\footnotesize{\textit{International Journal of Mathematics Trends and Technology - Volume 7 Number 2- March 2014}}}}
\rhead{}
\lfoot{ISSN: 2231-5373}
\cfoot{\url{http://www.ijmttjournal.org}}
\rfoot{Page \thepage}
}
\setlength\oddsidemargin  {14 pt}
\setlength\evensidemargin {16 pt}
\setlength\marginparwidth {60 pt}
\setlength\footskip {20 pt}
\setlength\hoffset{7mm}
\setlength\voffset{5mm}
\setlength\lineskip{1pt}
\setlength\normallineskip{1pt}
\setlength\overfullrule{0pt}
\setlength\textwidth{35pc}
\setlength\textheight{46\baselineskip}
\setlength\headsep {9 pt}
\renewcommand\baselinestretch{1.02}
\setlength\parindent{1em}
\setlength\parskip{0 pt}
\setcounter{topnumber}{3}
\renewcommand\topfraction{.99}
\setcounter{bottomnumber}{1}
\renewcommand\bottomfraction{.3}
\setcounter{totalnumber}{3}
\renewcommand\textfraction{.07}
\renewcommand\floatpagefraction{.93}
\renewcommand{\arraystretch}{1.05}
\newtheorem{theorem}{Theorem}[section]
\newtheorem{lemma}[theorem]{Lemma}
\newtheorem{corollary}[theorem]{Corollary}
\newtheorem{proposition}[theorem]{Proposition}
\newtheorem{conjecture}[theorem]{Conjecture}
\newtheorem{claim}[theorem]{Claim}
\newtheorem*{Theorem}{Theorem}
\newtheorem*{Corollary}{Corollary}
\newtheorem*{Proposition}{Proposition}
\newtheorem{definition}[theorem]{Definition}
\newtheorem{example}[theorem]{Example}
\newtheorem{remark}[theorem]{Remark}
\pagenumbering{arabic}
\clubpenalty=4000
\widowpenalty=10000
\hbadness = 10000
\tolerance = 10000
\pretolerance = 10000
\pagestyle{fancy}
\thispagestyle{empty}
\setcounter{page}{156}






\title{\textbf{ Cooperating distributed context-free hexagonal array grammar systems with permitting contexts }}
\author {\footnote{email-nksujathakumari@gmail.com} ,\ Dersanambika K.S\footnote{email-dersanapdf@yahoo.com}\\
\date{}
\small{ Department of Mathematics}\\\small{S.N College,
Punalur, Kerala 691305, India}\\\small{ Department of
Mathematics}\\\small{Fatima Mata National College, Kollam 691 001,
India}\\}
\begin{document}
\maketitle
\begin{abstract}{In this paper we associate permitting symbols with rules of Grammars in the components of cooperating
 distributed context-free hexagonal array grammar systems as a control mechanism and investigating the generative power
  of the resulting systems in the terminal mode. This feature of associating permitting symbols with rules when extended
   to patterns in the form of connected arrays also requires checking of symbols, but this is simpler than usual pattern
    matching. The benefit of allowing permitting symbols is that it enables us to reduce the number of components required
     in a cooperating distributed hexagonal array grammar system for generating a set of picture arrays.}
\end{abstract}

\noindent\textbf{Subject Classification: 68RXX}

\noindent\textbf{Keywords: Hexagonal arrays, Cooperating hexagonal
array grammar systems, Generative power}
\section{Introduction}
In the Context of image analysis and image processing a variety of
generative models for digitalized picture arrays in the two
dimensional plane have been proposed \cite{10}. Out of the
different techniques adopted for various models, grammar based
techniques utilize the rich theory of formal grammars and
languages and develop array grammars generating two dimensional
languages whose elements are picture arrays. There are two
distinct types of array grammars, isometric array grammars and
non-isometric array grammars. Since application of rewriting rule
can increase or decrease the length of the rewritten part, the
dimension of rewritten sub array can change in the case of
non-isometric grammars but application of such a rule is shape
preserving in the case of isometric grammars due to the fact that
the left and right sides of an array rewriting rule is
geometrically identical.

In order to handle more context with rewriting systems, a system
with several components is composed and defined a cooperation
protocol for these components to generate a common sentential
form. Such devices are known as cooperating distributed
(CD)grammar systems \cite{3}. Components are represented by
grammars or other rewriting devices, and the protocol for mutual
cooperation modifying the common sentential form  according to
their own rules. A variety of string grammar system models
\cite{3} have been introduced and studied in the literature.
Rudolf Freund extended the concept of grammar system to arrays
\cite{2} by introducing array grammar system and further J.
Dassow, R. Freund and Gh. Pun elaborated the power of
cooperation in array grammar system  (cooperating array grammar
system) for various non-context-free sets of arrays which can be
generated in a simple way by cooperating array grammar systems and
simple picture description \cite{1}. They also proved that the
cooperation increases the generative capacity even in the case of
systems with regular array grammar components.

Different kinds of control mechanism that are added to component
grammars for regulated rewriting rules have been considered in
string grammar systems and such control devices are known to
increase the generative power of the grammar in many cases
\cite{1}. Random context grammar is viewed as one of the prototype
mechanism in which components grammars that permit or forbid the
application of a rule based on  the presence or absence of a set
of symbols.

Hexagonal arrays and hexagonal patterns are found in the
literature on picture processing and image analysis. The class of
Hexagonal kolam array language (HKAL) was introduced by Siromoneys
\cite{9}. The class of Hexagonal array language was introduced by
Subramanian. The class of local and recognizable picture languages
were introduced  by Dersanambika et.al. \cite{6}. Recently we
extended cooperative distributed grammar system to Hexagonal
arrays and different capabilities of the system are studied
\cite{8}.

In this paper we associate permitting symbols with rules of the
grammar in the components of cooperating distributed context-free
hexagonal array grammar systems as a control mechanism and
investigating the generative power of the resulting systems in the
terminal mode. This feature of associating permitting symbols with
rules when extended to patterns in the form of connected arrays
also requires checking of symbols, but this is simpler than usual
pattern matching. The benefit of allowing permitting symbols is
that it enables us to reduce the number of components required in
a cooperating distributed hexagonal array grammar system for
generating a set of picture arrays.
\section{Preliminaries and definitions}
Let  be a finite non-empty set of symbols. The set of all
hexagonal arrays made up of elements of  is denoted by
. The size of the hexagonal array is defined by the
parameters (left upper), (left lower), (right upper),
(right lower), (upper), (lower) as shown in Figure 1.
For  the length of the left upper side of  is
denoted by . Similarly we define
,,, and .


\begin{figure}
    \begin{center}
        \includegraphics[scale=0.4]{pict2}
    \end{center}
    \centerline{Figure 1}
\end{figure}

\begin{definition}An isometric hexagonal array grammar is a construct  where  and  are disjoint
 alphabets of non terminals and terminals respectively,  is the start symbol,  is a special symbol called
 blank symbol and  is a finite set of rewriting rules of the form  where  and 
  are finite subpatterns of a hexagonal pattern over  satisfying the following conditions:
\begin{enumerate}
\item The shape of  and  are identical.
\item  contains at least one element of  The elements of  appearing in  are not rewritten.
\item A non  symbol in  is not replaced by a blank symbol in 
\item The application of the production  preserves connectivity of the hexagonal array.
\end{enumerate}

For a hexagonal array grammar  we can define
 for  if there is a
rule  such that  is a
subpattern of  and  is obtained by replacing  in 
by  The reflexive closure of  is denoted by
 The hexagonal array language generated
by  is defined by


\end{definition}
\begin{definition}A hexagonal array grammar is said to be context free if in the rule 
\begin{enumerate}
\item non  symbol in  are not replaced by  in 
\item  contain exactly one non-terminal and some occurrences of blank symbol.
\end{enumerate}.The family of languages generated by a context
free hexagonal array grammar is denoted by .
\end{definition}
\vspace{5cm }
\begin{definition}A context free hexagonal array grammar is said to be regular if rules are of the form




The family of languages generated by a regular hexagonal array grammar is denoted by 
\end{definition}
\begin{definition}A cooperating hexagonal array grammar system (of type  and degree ), is a construct  where  and  are non-terminal and terminal alphabets respectively,
 and  are finite sets of regular
respectively context free rules over 
\end{definition}
\begin{definition}Let  be a cooperating hexagonal array grammar system. Let  Then we write  if and only if there are words  such that
\begin{enumerate}
\item  \item  that is,
 
\end{enumerate}

Moreover, we write\\
\indent\indent if and only if  for some \\
\indent\indent  if and only if  for some \\
\indent\indent  if and only if  for some \\
\indent\indent  if and only if  and there is no  with 

By  we denote family of hexagonal array language
generated by cooperating hexagonal array grammar system consisting
of at most  components of type  in the mode

\end{definition}
\begin{definition}
A random context grammar is a quadruple  where  is
the alphabet of non-terminals,  is the alphabet of terminals
such that  is the start symbol,and
 is a finite set of productions of the form  where  is a context free
production, and ,and .
For  and a production , the relation  holds provided
that  and . A
permitting (forbidding)grammar is a random context grammar
 where for each production\\ , it holds that 
\end{definition}



\section {Cooperating distributed context-free hexagonal array grammar system with permitting symbols}

The set of all symbols in the labeled cells of the array  is denoted by alph A permitting CF hexagonal
 array rule is an array grammar  is of the form  where 
 is a context-free hexagonal array rewriting rule and , where is the set of non -terminals of the grammar.
 If , then we avoid mentioning it in the rule.For any two arrays  and a  permitting CF hexagonal array rule
 , the array  is derived from  by replacing  in  by  provided that
 . A permitting cooperating distributed context-free hexagonal array grammar systems
 (pCDCFHAGS) is  where  is a finite  set of non terminals, is
 the start symbol, is a finite set of terminals,  and each , for  is a finite set
 of permitting CF hexagonal array rewriting rules.

For any two hexagonal arrays , we denote 
an array rewriting step performed by applying a permitting cooperating CF Hexagonal array rule in ,
 and by   the transitive closure of 
 Also we say that the array  derives an array  in the terminal mode or  mode and write , if and there is no array  such that  The array language generated by  in the  mode is defined as
 for 

Note that  is any sequence of symbols belonging to  where repeated symbols are allowed.
 Also  denote the family of array languages generated in the  mode by permitting cooperating CF hexagonal array grammar systems with at most  components
\begin{example}Consider the context-free hexagonal array grammars with rules
 where


\centerline{Figure 2}
 \par  generates hexagonal arrays over 
is in the shape of left arrow head  but size of left upper arm and
left lower arm are not necessarily equal.





\end{example}
\begin{example}
The pCDCFHAGS
 where  consists  of the following rules.

generates (in the  mode) the set  of all arrays over
 in the shape of a left arrow head with left upper arm and
left lower arm are equal in size (Figure 3).

\centerline{Figure 3} The derivations starts with rule 1 followed
by rule 2 which can be applied as the permitting symbol  is
present in the array. This grows left upper arm (LU) by one cell.
Then rule 3 can be applied due to the presence of permitting
symbol  and this grows left lower arm (LL) by one cell. An
application of rule 4 followed by 5, again noting that the
permitting symbols of the respective rules are present changes
 to  and  to . the repeated application of the
process growing both the left upper arm and left lower arm equal
in size. Rules 6 and 7 are applied changing  to  and  to
 so that the derivation can be terminated by the application of
rules 8 and 9 thus yielding a hexagonal array in the shape of a
left arrow head with size of  and  are equal.
\end{example}



\begin{remark}A  where the set of permitting symbols in all the components is empty, is simply a cooperating distributed  hexagonal array system ()\cite{8}. The family of array languages generated in the  mode by a  with at most  components is denoted by  If the rules in all the components are only in the form of rules of a regular array grammar, then this family is denoted by 
\end{remark}
We now show that the set  of all  arrays over  in the form of hollow hexagonal frame.
Figure 3 can be generated (in the  mode) by a  with
only two components.

\begin{lemma} 
\end{lemma}
\begin{proof}The set  is generated (in the  mode) by the
.
 The rules in the component are given by

 The rules in the component  are given by


\centerline{Figure 4} Using - mode of derivation, starting with
the symbol  an application of rule (1) in the first component
followed by rule(2) which can be applied as the permitting symbol
 is present in the array, grows in the  arm by one place.
Since  is the permuting symbol for rule (3), rule (3) can then
applied which results the growth of  arm by one place. Now the
situations are ready for applying rules (4) and (5) and at this
stage  becomes  and  becomes . The process can be
repeated and this in turn results the growth of  and  arms
equal in length. Instead of rule (4) rule (6) is applied followed
by (7),(8),(9),(10),(11) allows upper and lower arms to grow in
equal length, with permitting symbols in all these rules directing
the sequence of applications in the right order. If rule (12) is
used instead of rule (10) and this is followed by rule (13) Right
upper () and Right lower () arms grows equal in size and
correct application of rule (18) and (19) will result in the
symbol  in the  arm and  in the lower right arm  where
 is at the position of right end point of  and  arm so
that further application of productions in  is not possible
at any non-terminals. At this stage applying productions in
 and this in turn terminate the derivation yielding a
hollow hexagon with its parallel arms are equal in size.
\end{proof}

\begin{lemma}
\begin{enumerate}
\item  \item 
\end{enumerate}
\end{lemma}
Proof follows from the result in \cite{8}.
\\\\\\
\begin{theorem}
\begin{enumerate}
\item  \item
 \item  for any 
\end{enumerate}
\begin{proof}
The equality follows from the results from \cite{8}. We know that,
a  with empty set of permitting symbols associated with
the rules is same as a cooperating distributed  hexagonal
array grammar system and so  if
. Examples (1) and (2) illustrated the fact that the set
 of all hexagonal arrays over  in the shape of left
arrow head with left upper arm and left lower arm with equal size
is generated by the  with only one component and
working in the -mode and hence the inclusion is proper which
proves (1). Similar arguments for inclusion in statement(2) are
hold. From the proof of the lemma(1) it is very clear that under
the strict application of the derivation rules with the respective
permitive symbols generate a hollow hexagon with parallel arms
equal in size. Such a generation is not possible in  since   and incorrect application of
terminating rule will leads to non-completion of the hollow
hexagon with parallel arms equal in size. Thus 

Consider the array languages generated by  in example
(1) and in the proof of lemma (1). It can be seen that generating
the arrow head of patterns of the language, we should require two
growing heads at the same time. But in the  with any
number of components the array rules contains only one growing
head. So the same language cannot be generated by
 and this proves (3).
\end{proof}
\end{theorem}
To show the power of the cooperating hexagonal array grammar system with the array rules controlled by permitting symbols, consider the following example of a language of set of all hexagons with parallel arms are equal in length over a one letter alphabet.
\begin{example}Consider the \\
.\\
 The rules in the component  are\\


 

The rules in the component  are given by



\centerline{Figure 5} Starting with  repeated application of
the first five rules of the component  in this order generate
 having equal arms, with  arm having non-terminal  and
 arm having non-terminal  Once two rules  and  of component 
are used, the generations of these two arms end with terminal 
and then starts the generations of upper and lower arms of the
hexagon using rules (6) to (11). Then the right application of
rule (12) to (17) then (18), (19),  (45) subjected to the
permitting symbols will result to a hexagonal picture and finally
by the application of rules in  we get the required hexagon
over the one letter alphabet  as in Figure 5.
\end{example}
 \par In the Siromoney matrix grammar (9) rectangular arrays are generated in
two phases; one in horizontal and the other in vertical. Further
it was extended by associating a finite set of rules in the second
phase of generation with each table having either right linear
non-terminal rules of the form  or right-linear
terminal rules of the form  and such array
languages are denoted by  and  and we have a well
known result 
(refer 11). Correspondingly it can be established for hexagonal
arrays. Here we compare  with these classes.
\begin{theorem}
\end{theorem}
\begin{proof}
Consider the 
\\
\\
where the components are\\




\centerline{Figure 6} Except the rules  in  generates the  and  arm
 with a middle marker  and the symbols  above and below  in both arms and which are equal in number.
 The first two rules of  changes the symbols in the left most cell in to .Then the remaining rules of 
 and the last two rules of  (,) and the rules of the component 
 generate the arms such that each cell in the middle arm in the horizontal direction is made up of  and all other
 cells above and below are made up of  except the leftmost arms.The generation of the cells finally terminates,
 yielding the rightmost cells are rewritten by . Thus a hexagonal array in the shape of a left arrowhead
 (describing as Figure(6) is generated).If we treat  as blank, such arrays cannot be in  and
 which in turn proves that .
 \end{proof}

\noindent{\bf Conclusion.} In this paper, the picture array
generating power of cooperating  hexagonal array grammar
systems endowed with permitting symbols are studied. It is seen
that the control mechanism which we here used namely the
'permitting symbols' is shape preserving in picture generation and
also it reduce size complexity.

 \begin{thebibliography}{99}

\bibitem{1} H. Bordihm, M. Holzer, in: C. Matin-Vide, F. Otto, H. Fernau (Eds.), \textit{Random Context in Regulated Rewriting Versus Cooperating Distributed Grammar Systems}, in:Lecture Notes in Computer Science, Vol. 5159, Springer-Verlag, 2008,pp 125--136.

\bibitem{2}J. Dassow, R. Freund, Gh. P\u aun, \textit{Cooperating Array Grammer Systems}, J. Pattern Recognit, Artif. Intell. 9,1995, pp 1029--1053.

\bibitem{3}J. Dassow, Gh. P\u aun, G. Rozenberg, \textit{Grammar systems, in: G. Rozenberg, A. Salomaa (Eds)}, Vol. 2, Springer-Verlag, 1997, pp 155--213.

\bibitem{4}R. Freund, \textit{Array Grammars}, Tech. Report 15/00, XI Tarrogena Seminar in Formal Syntax and Semantics, 2000.

\bibitem{5}K. G. Subramanian et. al., \textit{On the Power of Permitting Features in Cooperating Context-Free Array Grammar
System}, Discrete Applied Mathematics, Vol. 161(15),2013, pp
2328-2335.

\bibitem{6}K. S. Dersanambika, K. Krithivasan, Martin-vide, K. G. Subramanian, \textit{Local and Recognizable Hexagonal Picture Languages}, International Journal of Pattern Recognition and Artificial Intelligence, 19(7), 2012, pp 553--571.

\bibitem{7}K. S. Dersanambika, K. Krithivasan, H. K. Agarwal, J. Guptha \textit{Hexagonal Contextual Array p-Systems, Formal Models, Languages and Application}, Series in Machine Perception Artificial Intelligence 66, 2006, pp 79--96.

\bibitem{8}Jismy Joseph, K. S. Dersanambika, K. Sujathakumari, \textit{Cooperating Hexagonal Array Grammar System}, Communicated.

\bibitem{9}G. Siromoney R. Siromoney, \textit{Hexagonal Arrays and Rectangular Blocks}, Computer Graphics and Image Processing, 1976, pp 353--381.

\bibitem{10}A. Rozenfield, \textit{Formal Models for Picture Recognition}, Academic Press, New York, 1979.

\bibitem{11} R. Siromoney, K. G. Subramanian, K. Rangarajan,
\textit{Parallel  Sequential rectangular arrays with
tables}, Int. J. Comput. Math.6A, 1977,pp 143-158.

\end{thebibliography}
\end{document}
