\documentclass[11pt]{article}
\author{Gabriel Nivasch\thanks{\texttt{gabriel.nivasch@cs.tau.ac.il}.
Work was supported by ISF Grant 155/05 and by the Hermann
Minkowski--MINERVA Center for Geometry at Tel Aviv University.}
\quad Micha Sharir\thanks{\texttt{michas@post.tau.ac.il}. Work was
partially supported by NSF grant CCF-05-14079, by a grant from the
U.S.-Israel Binational Science Foundation, by ISF Grant 155/05, and
by the Hermann Minkowski--MINERVA Center for Geometry
at Tel Aviv University.}\\
\footnotesize School of Computer Science\-1mm]
\footnotesize Tel Aviv 69978, Israel}
\date{July 27, 2008}

\title{Eppstein's bound on intersecting triangles revisited}

 \addtolength{\oddsidemargin}{-1.5cm}
 \addtolength{\textwidth}{3cm}
 \addtolength{\topmargin}{-1cm}
 \addtolength{\textheight}{2.5cm}

\usepackage{amsmath,amsfonts,amssymb,amsthm}
\usepackage{hyperref}
\usepackage{graphics}

\newtheorem{lemma}{Lemma}

\begin{document}

\maketitle

\begin{abstract}
Let  be a set of  points in the plane, and let  be a set of
 triangles with vertices in . Then there exists a point in the
plane contained in  triangles of .
Eppstein (1993) gave a proof of this claim, but there is a problem
with his proof. Here we provide a correct proof by slightly
modifying Eppstein's argument.

\emph{Keywords:} Triangle; Simplex; Selection Lemma; -Set
\end{abstract}

\section{Introduction}

Let  be a set of  points in the plane in general position (no
three points on a line), and let  be a set of  triangles with vertices in . Aronov et al.~\cite{ACEGSW}
showed that there always exists a point in the plane contained in
the interior of

triangles of . Eppstein \cite{eppstein} subsequently claimed to
have improved this bound to

There is a problem in Eppstein's proof, however.\footnote{The very
last sentence in the proof of Theorem 4 (Section 4) in
\cite{eppstein} reads: ``So , and , from which it follows that .'' This is patently false, since what actually follows is
that , and the entire argument falls through.}
In this note we provide a correct proof of (\ref{eq_bound_log2}), by
slightly modifying Eppstein's argument.

\subsection{The Second Selection Lemma and -sets}

The above result is the special case  of the following lemma
(called the \emph{Second Selection Lemma} in \cite{matou}), whose
proof was put together by B\'ar\'any et al.~\cite{BFL}, Alon et
al.~\cite{ABFK}, and \v Zivaljevi\'c and Vre\'cica \cite{ZV}:

\begin{lemma}\label{lemma_2nd_sel}
If  is an -point set in  and  is a family of
 -simplices spanned by , then there
exists a point  contained in at least

simplices of , for some constants  and  that depend
only on .
\end{lemma}

(Note that , so the smaller the constant ,
the stronger the bound.) Thus, for  the constant  in
(\ref{eq_2nd_SL}) can be taken arbitrarily close to . The general
proof of Lemma \ref{lemma_2nd_sel} gives very large bounds for
; roughly .

The main motivation for the Second Selection Lemma is deriving upper
bounds for the maximum number of \emph{-sets} of an -point set
in ; see \cite[ch.~11]{matou} for the definition and
details.

\section{The proof}

We assume that , since otherwise the
bound (\ref{eq_bound_log2}) is trivial. The proof, like the proof of
the previous bound (\ref{eq_bound_log5}), relies on the following
two one-dimensional \emph{selection lemmas} \cite{ACEGSW}:

\begin{lemma}[Unweighted Selection Lemma]\label{lemma_unw}
Let  be a set of  points on the real line, and let  be a
set of  distinct intervals with endpoints in . Then there
exists a point  lying in the interior of 
intervals of .
\end{lemma}

\begin{lemma}[Weighted Selection Lemma]\label{lemma_weighted}
Let  be a set of  points on the real line, and let  be a
\emph{multiset} of  intervals with endpoints in . Then there
exists a multiset  of  intervals, having as
endpoints a subset  of  points, such that all
the intervals of  contain a common point  in their interior,
and such that

\end{lemma}

The proof of the desired bound (\ref{eq_bound_log2}) proceeds as
follows:

Assume without loss of generality that no two points of  have the
same -coordinate. For each triangle in  define its \emph{base}
to be the edge with the longest -projection. For each pair of
points , let  be the set of triangles in  that
have  as base, and let . (Thus, .)

Discard all sets  for which . We discarded
at most  triangles, so we are left with a
subset  of at least  triangles, such that either  or  for each base .\footnote{This critical
discarding step is missing in \cite{eppstein}, and that is why the
proof there does not work.}

Partition the bases into a logarithmic number of subsets  for , so that each  contains
all the bases  for which

Let  denote the set of triangles
with bases in , and  denote their number. There
must exist an index  for which

since otherwise the total number of triangles in  would be less
than . From now on we fix this , and work only with the
bases in  and the triangles in .

For each pair of triangles ,  having the same base , project the segment  into the -axis, obtaining segment
. We thus obtain a multiset  of horizontal segments, with

(Each of the  triangles in  is paired with all other
triangles sharing the same base, and each such pair is counted
twice.)

We now apply the Weighted Selection Lemma
(Lemma~\ref{lemma_weighted}) to , obtaining a multiset  of
segments delimited by  distinct endpoints, all segments
containing some point  in their interior, with



\begin{figure}
\centerline{\includegraphics{fig_triangle_pair.pdf}}
\caption{\label{fig_triangle_pair} Pairing two triangles with a
common base.}
\end{figure}

Let  be the vertical line passing through . For each
horizontal segment , each of its (possibly multiple)
instances in  originates from a pair of triangles , ,
where points  and  lie to the left of , and points 
and  lie to the right of . Let  be the intersection of
 with , and let  be the intersection of  with
. Then,  is a vertical segment along , contained in
the union of the triangles ,  (see Figure
\ref{fig_triangle_pair}). Let  be the set of all these segments
 for all .

Note that the vertical segments in  are all distinct, since
each such segment  uniquely determines the originating points
, , ,  (assuming  was chosen in general position).

Let  be the number of endpoints of the segments in . We
have , since each endpoint (such as ) is uniquely
determined by one of  ``inner'' vertices (such as ) and one
of at most  ``outer'' vertices (such as ).

Next, apply the Unweighted Selection Lemma (Lemma~\ref{lemma_unw})
to , obtaining a point  that is contained in

segments in . Thus,  is contained in at least these many
\emph{unions of pairs of triangles} of . But by
(\ref{eq_bound_m_ab}), each triangle in  participates in at
most  pairs. Therefore,  is contained in

triangles of .

\section{Discussion}

Eppstein \cite{eppstein} also showed that there always exists a
point in  contained in  triangles of .
This latter bound is stronger than (\ref{eq_bound_log2}) for small
, namely for .

On the other hand, as Eppstein also showed \cite{eppstein}, for
every -point set  in general position and every , , there exists a set  of 
triangles with vertices in , such that no point in the plane is
contained in more than  triangles of . Thus, with the
current lack of any better lower bound, the bound
(\ref{eq_bound_log2}) appears to be far from tight. Even achieving a
lower bound of , without any logarithmic factors,
is a major challenge still unresolved.

It is known, however, that if  is a set of  points in  in general position (no four points on a plane), and  is a
set of  triangles spanned by , then there exists a \emph{line}
(in fact, a line spanned by two points of ) that intersects the
interior of  triangles of ; see \cite{DE} and
\cite{smo_phd} for two different proofs of this.

\begin{thebibliography}{9}

\bibitem{ABFK}
N. Alon, I. B\'ar\'any, Z. F\"uredi, and D. Kleitman, Point
selections and weak -nets for convex hulls, \emph{Combin.,
Probab. Comput.}, 1:189--200, 1992.

\bibitem{ACEGSW}
B. Aronov, B. Chazelle, H. Edelsbrunner, L. J. Guibas, M. Sharir,
and R. Wenger, Points and triangles in the plane and halving planes
in space, \emph{Discrete Comput. Geom.}, 6:435--442, 1991.

\bibitem{BFL}
I. B\'ar\'any, Z. F\"uredi, and L. Lov\'asz, On the number of
halving planes, \emph{Combinatorica}, 10:175--183, 1990.

\bibitem{DE}
T. K. Dey and H. Edelsbrunner, Counting triangle crossings and
halving planes, \emph{Discrete Comput. Geom.}, 12:281--289, 1994.

\bibitem{eppstein} D. Eppstein, Improved bounds for intersecting
triangles and halving planes, \emph{J. Combin. Theory Ser. A},
62:176--182, 1993.

\bibitem{matou}
J. Matou\v sek, \emph{Lectures on Discrete Geometry},
Springer-Verlag, New York, 2002.

\bibitem{smo_phd}
S. Smorodinsky, \emph{Combinatorial problems in computational
geometry}, Ph.D.~Thesis, Tel~Aviv University, June 2003.
\url{http://www.cs.bgu.ac.il/~shakhar/my_papers/phd.ps.gz}.

\bibitem{ZV}
R. T. \v Zivaljevi\'c and S. T. Vre\'cica, The colored Tverberg's
problem and complexes of injective functions, \emph{J. Combin.
Theory Ser. A}, 61:309--318, 1992.

\end{thebibliography}

\end{document}
