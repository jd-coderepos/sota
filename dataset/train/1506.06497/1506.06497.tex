\documentclass[12pt]{report}


\usepackage{amssymb}
\usepackage{amsmath}
\usepackage{amsthm}
\usepackage{stmaryrd}
\usepackage{graphicx}
\usepackage{tikz}
\usepackage{geometry}
\usepackage{verbatim}
\usepackage{caption}
\usepackage{subcaption}

\usetikzlibrary{shapes}


\tikzstyle{noir}=[circle,draw,fill=black!40]
\tikzstyle{debut}=[circle,draw,very thick,fill=white]
\tikzstyle{blanc}=[circle,draw, fill=white]
\title{Towards an algebraic theory of rational word functions}
\author{Nathan Lhote}

\newtheorem{thm}{Theorem}[section]
\newtheorem{prp}{Proposition}[section]
\newtheorem{lem}{Lemma}[section]
\newtheorem{cor}{Corollary}[section]

\theoremstyle{definition}
\newtheorem{dfn}{Definition}[section]
\newtheorem{xmp}{Example}[section]

\theoremstyle{remark}
\newtheorem{rmk}{Remark}[section]

\newenvironment{prf}
{\textit{Proof:}}
{\hfill \\}


\newcommand{\HRule}{\rule{\linewidth}{0.5mm}}

\begin{document}

\begin{titlepage}

\begin{center}

\includegraphics[width=0.4\textwidth]{./u-bordeaux.jpg}~\hfill ~\includegraphics[width=0.4\textwidth]{./labri.jpg}~\0.5cm]

\HRule \0.4cm] }

\HRule \\forall u,v\in \Sigma^\ast, \forall \sigma\in \Sigma, u\sim v\Rightarrow u\sigma\sim v\sigma u\approx_L v \Leftrightarrow \left( \forall x,y\quad xuy\in L \Leftrightarrow xvy\in L \right)u\equiv_A v \Leftrightarrow \left( \forall p,q\in Q \quad u\in L_{p,q} \Leftrightarrow v\in L_{p,q} \right)u\sim_A v \Leftrightarrow \left( \forall p\in Q \quad u\in L_{q_0,p} \Leftrightarrow v\in L_{q_0,p} \right)\varphi \mathrel{::=}\exists X\ \varphi\mid \exists x\ \varphi\mid \varphi\wedge \varphi \mid \neg \varphi\mid x\in X \mid \sigma(x) \mid x<y\mid (\varphi)\mathsf{succ}(x,y)=(x<y)\wedge (\neg\exists z\ (x<z)\wedge(z<y))\mathsf{min}(x)=\neg \exists y\ (y<x)\mathsf{max}(x)=\neg \exists y\ (x<y)\begin{array}{rl}\varphi_{even}=\exists X\ \exists Y\ \forall x\ \forall y\ & x\in  X\leftrightarrow \neg(x\in Y)\\
\wedge&\mathsf{min}(x)\rightarrow x\in X\\
\wedge& \mathsf{max}(y)\rightarrow y\in Y\\
\wedge& \mathsf{succ}(x,y)\rightarrow (x\in X\wedge y\in Y)\vee (x\in Y\wedge y\in X)
\end{array}\varphi_{ends}=\exists x\ \exists y\ \mathsf{min}(x)\wedge a(x) \wedge\mathsf{max}(y)\wedge a(y)\omega(l,u,r)=\omega(l,\sigma_1,r_{n-1})\cdots\omega(l_{n-1},\sigma_n,r)\llbracket B \rrbracket(u)=\lambda(r_n)\omega(l_0,u,r_0)\rho(l_n)\mathcal T=\left(k,S,\left(\varphi_{j,\sigma,v}^<,\varphi_{j,\sigma,v}^>\right)_{\begin{smallmatrix} 1\leq j\leq k\\ \sigma \in \Sigma, v\in S \end{smallmatrix}},\left(\varphi_v^i\right)_{v\in S},\left(\varphi_v^t\right)_{v\in S}\right)\widehat f(vm_1)=ix_1\bigwedge \left\{ f_p(w)\mid  w\in \mathrm{dom} (f_p) \right\}\widehat f(vm_2)=ix_2\bigwedge \left\{ f_p(w)\mid  w\in \mathrm{dom} (f_p) \right\}\widehat f(vm_1)^{-1}f(vm_1w)=s^{-1}x_1^{-1}i^{-1}ix_1f_p(w)=s^{-1}f_p(w)\widehat f(vm_2)^{-1}f(vm_2w)=s^{-1}x_2^{-1}i^{-1}ix_2f_p(w)=s^{-1}f_p(w)v_{i_1,j}x_{i_1,i_2}=x_{j+1}v_{i_2,j+1}\begin{array}{rll} 
v_{i_0,j}&x_{i_0,i_1}\cdots x_{i_{t-1},i_t}=x_{j+1}\cdots x_{j+t}v_{i_t,j}\\
v_{i_0,j'}&x_{i_0,i_1}\cdots x_{i_{t-1},i_t}=x_{j'+1}\cdots x_{j'+t}v_{i_t,j'}
\end{array}\begin{array}{rcl} 
v_{i,j}Y&=&(X_j)^sv_{i,j}\\
v_{i,j+1}Y&=&(X_{j+1})^sv_{i,j+1}\\
v_{i,j}Y'&=&(X_j)^sx_{j}v_{i,j+1}\\
\end{array}\begin{array}{rclr} 
v_{i,j}Y&=&(X_j)^sv_{i,j}&\quad (\alpha)\\
wv_{i,j}Y&=&(X_{j+1})^swv_{i,j}&\quad(\beta)\\
v_{i,j}Y'&=&(X_j)^sx_{j}wv_{i,j}&\quad(\gamma)\\
\end{array}X_j=(x_jw)^k X_{j+1}=(wx_j)^k\begin{array}{rcl} 
w_lv_{l,j+1}Y_l&=&(X_j)^sw_lv_{l,j+1}\\
v_{l,j+1}Y_l&=&(X_{j+1})^sv_{l,j+1}\\
v_{l,j+1}Y'_l&=&(X_{j+1})^{s}x_{j+1}\cdots x_{j-1} w_lv_{l,j+1}\\
\end{array}\begin{array}{rcl}
(x_{j+1}\cdots x_{j-1} w_l)^k&=&((wx_j)^k)^{k-1}\\
x_{j+1}\cdots x_{j-1} w_l&=&(wx_j)^{k-1}\\
wx_jx_{j+1}\cdots x_{j-1} w_l&=&(wx_j)^k\\
wX_jw_l&=&X_{j+1}
\end{array}\begin{array}{rcl} 
v_{i,j}Z_l&=&(X_j)^sv_{l,j}\\
wv_{i,j}Z_l&=&(X_{j+1})^sv_{i,j+1}\\
\end{array}\begin{array}{c}
\forall w\in \Sigma^\ast,\  wu\in \mathrm{dom}(f) \Leftrightarrow wv\in \mathrm{dom}(f)\\
\text{and}\\
\sup_{w\in \Sigma^\ast}\left\{\lVert f(wu),f(wv)\rVert \right\}<\infty
\end{array}\widehat f_r(u)=\bigwedge \left\{f(uv)\mid  vr_0=r \right\}\begin{array}{rcl}
\widehat f_{\sigma r}(uw)^{-1}\widehat f_{r}(uw\sigma) &=& \widehat f_{\sigma r}(vw)^{-1}\widehat f_{r}(vw\sigma)\\
\widehat f_{r_0}(uw)^{-1} f(uw) &=& \widehat f_{r_0}(vw)^{-1} f(vw)
\end{array} S=\left\{ (q',wu)\mid  \exists (q,w)\in P\ \mathrm{and}\ q\xrightarrow{\sigma\mid u}q'\right\}P'=\left\{ (q',w')\mid   (q',vw')\in S\right\}\begin{array}{l}
\widehat f_r(m_1w\sigma)=u_1vv't_r(P_3,r) \\
\widehat f_r(m_2w\sigma)=u_2vv't_r(P_3,r)\\
\widehat f_{\sigma r}(m_1w)=u_1vt_{\sigma r}(P_2,\sigma r)\\
\widehat f_{\sigma r}(m_2w)=u_2vt_{\sigma r}(P_2,\sigma r)\\
\end{array}\begin{array}{l}
f(m_1w)=u_1vt_f(P_2,r_0) \\
f(m_2w)=u_2vt_f(P_2,r_0)\\
\widehat f_{r_0}(m_1w)=u_1vt_{r_0}(P_2,r_0)\\
\widehat f_{r_0}(m_2w)=u_2vt_{r_0}(P_2,r_0)\\
\end{array}{L'} \sqsubseteq {L^R}\begin{array}{rcl}
\lVert f(wu),f(wv)\rVert &=&
\lVert \lambda(r_n)\omega(l_0,w,r)\omega(l_n,u,r_0)\rho(l),\lambda(r_n)\omega(l_0,w,r)\omega(l_n,v,r_0)\rho(l')\rVert\\
&=&\lVert \omega(l_n,u,r_0)\rho(l),\omega(l_n,v,r_0)\rho(l')\rVert\\
&\leq& k(|u|+|v|+1)
\end{array}{L^0} \sqsubseteq {L^R}\begin{array}{cccc}\bar f:& \Sigma^\ast &\rightarrow & (\Sigma\uplus\left\{\bot \right\})^\ast \\
&u & \mapsto & \left\{\begin{array}{l l} f(u) & \mathrm{if} \ u\in \mathrm{dom}(f) \\ \bot & \mathrm{otherwise}\end{array}  \right.
\end{array} f: \begin{array}{rcl}
a(b+c)^+ \ni w &\mapsto & a\\
aa^+b(b+c)^\ast \ni w &\mapsto & b\\
aa^+c(b+c)^\ast \ni w &\mapsto & c
\end{array}f: \begin{array}{rcl}
a(b+c)^+ \ni w &\mapsto & a\\
aa^+b(b+c)^\ast \ni w &\mapsto & b\\
aa^+c(b+c)^\ast \ni w &\mapsto & c
\end{array}f: a^n\mapsto\left\{ \begin{array}{c} a\ \mathrm{if}\ n=1\\ b\ \mathrm{if}\ n>1   \end{array} \right. \begin{array}{rcl}
ab &\mapsto& a\\
aab & \mapsto& a\\
aba &\mapsto& ab\\
ba &\mapsto& c \\
baa &\mapsto& c\\

\end{array}(0,\varnothing)\xrightarrow{(ab)^{k}}(2,\varnothing)(0,\varnothing)\xrightarrow{(ac)^{k}}(4,\varnothing)
If and only if  is odd. In both cases the obtained automaton is periodic.
\end{xmp}

\chapter*{Conclusion}
Using techniques inspired by \cite{choffrut03,reutenauers91}, we have been able to extend decidability results for varieties of rational languages to varieties of rational functions.
In the deterministic case the notion of minimal DFT gives a natural way to decide if a subsequential functions belongs to a given variety.
In the non-deterministic case we need to consider not one minimal machine, but a finite set of canonical bimachines and if one of these machines satisfies the desired algebraic property, then the function it defines belongs to the corresponding variety.

Some of the questions that remain are for instance the complexity of the decision procedures described in this report.
The question of the equivalence between \textbf V-NFT and unambiguous \textbf V-NFT for a general variety also remains unanswered. 

We foresee several interesting directions in which the techniques of this report can be taken.
The first one would be to try to extend the results to the more general class of regular functions, defined by two-way transducers \cite{engelfrieth01} and streaming string transducers \cite{alurc10}, or some intermediate class.
A second possible direction is the generalization to relations, with probably some condition of finite value, for instance by bounding the ambiguity of the transducers.
Another extension would be to consider infinite words and see for instance if some minimization is possible for DFTs on infinite words.
There is also the domain of weighted automata for which common techniques are shared with transducers \cite{filiotgr14}.



\bibliographystyle{alpha}
\bibliography{biblio}


\end{document}
