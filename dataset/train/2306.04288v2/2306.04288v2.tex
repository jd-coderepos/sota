












\documentclass{bmcart}

\usepackage[utf8]{inputenc} 









\startlocaldefs
\endlocaldefs


\begin{document}

\begin{frontmatter}

\begin{fmbox}
\dochead{Research}



\title{A sample article title}



\author[
   addressref={aff1},                   corref={aff1},                       noteref={n1},                        email={jane.e.doe@cambridge.co.uk}   ]{\inits{JE}\fnm{Jane E} \snm{Doe}}
\author[
   addressref={aff1,aff2},
   email={john.RS.Smith@cambridge.co.uk}
]{\inits{JRS}\fnm{John RS} \snm{Smith}}



\address[id=aff1]{\orgname{Department of Zoology, Cambridge}, \street{Waterloo Road},                     \city{London},                              \cny{UK}                                    }
\address[id=aff2]{\orgname{Marine Ecology Department, Institute of Marine Sciences Kiel},
  \street{D\"{u}sternbrooker Weg 20},
  \postcode{24105}
  \city{Kiel},
  \cny{Germany}
}



\begin{artnotes}
\note[id=n1]{Equal contributor} \end{artnotes}

\end{fmbox}



\begin{abstractbox}

\begin{abstract} \parttitle{First part title} Text for this section.

\parttitle{Second part title} Text for this section.
\end{abstract}



\begin{keyword}
\kwd{sample}
\kwd{article}
\kwd{author}
\end{keyword}



\end{abstractbox}


\end{frontmatter}





\section*{Content}
Text and results for this section, as per the individual journal's instructions for authors. 

\section*{Section title}
Text for this section \ldots
\subsection*{Sub-heading for section}
Text for this sub-heading \ldots
\subsubsection*{Sub-sub heading for section}
Text for this sub-sub-heading \ldots
\paragraph*{Sub-sub-sub heading for section}
Text for this sub-sub-sub-heading \ldots
In this section we examine the growth rate of the mean of ,  and . In
addition, we examine a common modeling assumption and note the
importance of considering the tails of the extinction time  in
studies of escape dynamics.
We will first consider the expected resistant population at  for
some , (and temporarily assume )

If we assume that sensitive cells follow a deterministic decay
 and approximate their extinction time as
, then we can heuristically
estimate the expected value as

Thus we observe that this expected value is finite for all  (also see \cite{koon,khar,zvai,xjon,marg}).




\begin{backmatter}

\section*{Competing interests}
  The authors declare that they have no competing interests.

\section*{Author's contributions}
    Text for this section \ldots

\section*{Acknowledgements}
  Text for this section \ldots


\bibliographystyle{bmc-mathphys} \bibliography{bmc_article}      







\section*{Figures}
  \begin{figure}[h!]
  \caption{\csentence{Sample figure title.}
      A short description of the figure content
      should go here.}
      \end{figure}

\begin{figure}[h!]
  \caption{\csentence{Sample figure title.}
      Figure legend text.}
      \end{figure}



\section*{Tables}
\begin{table}[h!]
\caption{Sample table title. This is where the description of the table should go.}
      \begin{tabular}{cccc}
        \hline
           & B1  &B2   & B3\\ \hline
        A1 & 0.1 & 0.2 & 0.3\\
        A2 & ... & ..  & .\\
        A3 & ..  & .   & .\\ \hline
      \end{tabular}
\end{table}



\section*{Additional Files}
  \subsection*{Additional file 1 --- Sample additional file title}
    Additional file descriptions text (including details of how to
    view the file, if it is in a non-standard format or the file extension).  This might
    refer to a multi-page table or a figure.

  \subsection*{Additional file 2 --- Sample additional file title}
    Additional file descriptions text.


\end{backmatter}
\end{document}
