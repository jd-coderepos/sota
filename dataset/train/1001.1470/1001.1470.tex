In this section we consider another application that has received significant attention in the recent past: the max-min fair allocation problem
\cite{dani:05,julia:focs09,asadpour:stoc07,asadpour-feige-saberi,bansal:stoc06}. We provide a new algorithm for max-min fair allocation based on bipartite dependent rounding \cite{gkps:dep-round} and its generalization to weighted graphs. Bipartite dependent rounding has found many applications in combinatorial optimization \cite{srin:level-sets,gkps:dep-round,kmps:unified-sched}, and can be seen as a special case of {\bf RandMove} on bipartite graphs. We also consider an ``equitable allocations'' version of
such problems, in Theorem~\ref{thm:santa2}: this theorem follows from \cite{shmoys-tardos:gap} as pointed out by the referee. 

In the max-min fair allocation problem,
there are $m$ goods that need to be distributed indivisibly among $k$ persons. Each person $i$ has a non-negative integer valuation $u_{i,j}$ for good $j$. The valuation functions are linear, i.e., $u_{i,C}=\sum_{j \in C} u_{i,j}$ for any set of $C$ goods. The goal is to allocate each good to a person such that the ``least happy person is as happy as possible'': i.e., $\min_{i} u_{i,C}$ is maximized. Our main contribution  in this regard
is to near-optimally pin-point the integrality gap of a {\em configuration LP} previously proposed and analyzed in \cite{bansal:stoc06,asadpour:stoc07}.


\noindent
\paragraph{\textbf{The Configuration LP for Max-Min Fair Allocation}}

The configuration LP formulation for the max-min fair allocation problem was first considered in
\cite{bansal:stoc06}. A {\it configuration} is a subset of items,
and the LP has a
variable for each valid configuration. Using binary search, first the optimum solution value $T$ is guessed and then
we define valid configurations based on the approximation factor $\lambda$ sought; we will set
\begin{equation}
\label{eqn:lambda}
\lambda = 26 \sqrt{k \ln k}.
\end{equation}
We call a configuration $C$  {\it valid for person $i$} if either of the following two conditions hold:
\begin{itemize}
\item  $u_{i,C} \geq T$ and each item  in $C$ has value less than $\frac{T}{\lambda} $. These are called {\em small} items.
\item $C$ contains only one item $j$ and $u_{i,j} \geq \frac{T}{\lambda}$. We call such an item $j$ to be a {\em big} item for person $i$.
\end{itemize}

We define a variable $x_{i,C}$ for assigning a valid configuration $C$ to person $i$.
Let $C(i,T)$ denote the set of all valid configurations corresponding to person $i$ with respect to $T$. The configuration LP relaxation
 of the problem is as follows:

\begin{eqnarray}
\label{eqn:lp}
\forall j: \sum_{C \ni j}\sum_{i} x_{i,C} \leq 1 \\
\forall i: \sum_{C \in C(i,T)} x_{i,C}=1 \nonumber\\
\forall i, C: x_{i,C} \geq 0 \nonumber
\end{eqnarray}

The above LP formulation may have an exponential number of variables, However, if the LP is feasible, then a fractional allocation where each person receives either a big item or at least a utility of $T(1-\epsilon)$ can be computed in
polynomial time for any constant $\epsilon > 0$ \cite{bansal:stoc06}.
In the subsequent discussion and analysis, we ignore the multiplicative
$(1 - \epsilon)^{-1}$ factor; it is hidden in the $\Theta$ notation of
the ultimate approximation ratio.

The worst-case integrality gap of the above configuration LP is
lower-bounded by $\Omega(\frac{1}{\sqrt{k}})$ \cite{bansal:stoc06}.
In \cite{asadpour:stoc07}, Asadpour and Saberi
gave a rounding procedure for the configuration LP that achieved an approximation factor of $O\left(\frac{1}{\sqrt{k}(\ln{k})^3}\right)$. Here we further lower
the gap and prove the following theorem;  our proof is also
significantly simpler than that of \cite{asadpour:stoc07}.


\begin{theorem}
\label{thm:santa1}
Given any feasible solution to the configuration LP, it can be rounded to a feasible integer solution such that every person gets  at least $\Theta\left(\frac{1}{\sqrt{k\ln{k}}}\right)$ fraction of the optimal utility with probability at least $1-\Theta(\frac{1}{k})$, in polynomial time.
\end{theorem}


Note that the work of Chakrabarty, Chuzhoy and Khanna \cite{Chuzhoy09}
yields an improved approximation factor of $m^{\epsilon}$
for any positive constant
$\epsilon$, but it does not use the configuration LP
(also note that $m \geq k$).


In the context of fair allocation, an additional important criterion can be
an \emph{equitable partitioning} of goods: we may impose an upper
bound on the number of items a person might receive. For example,
we may want each person to receive at most $\lceil \frac{m}{k} \rceil$ goods. Theorem \ref{thm:matching} then directly leads to the following.
\begin{theorem}
\label{thm:santa2}
Suppose, in max-min allocation, we are given upper bounds $c_{i}$ on the
number of items that each person $i$ can receive, in addition to the
 utility values $u_{i,j}$. Let $T$ be the optimal max-min allocation value
that satisfies $c_i$ for all $i$. Then, we can efficiently construct an allocation
in which for each person $i$ the bound $c_i$ holds and she receives a total
utility of at least $T - \max_{j} u_{i,j}$.
\end{theorem}

This generalizes the result of \cite{dani:05}, which yields the
``$T - \max_{j} u_{i,j}$'' value when no bounds such as the $c_i$ are given.
To our knowledge, the results of
\cite{julia:focs09,asadpour:stoc07,asadpour-feige-saberi,bansal:stoc06}
do not carry over to the setting of such ``fairness bounds'' $c_i$.

\subsection{Algorithm for Max-Min Fair Allocation}
\label{subsec:algo}

We now describe the algorithm and proof for Theorem \ref{thm:santa1}.

\subsubsection{Algorithm}

We define a weighted bipartite graph $G$ with the vertex set $A \bigcup B$ corresponding to the persons and the items respectively. There is an edge between a vertex corresponding to person $i \in A$ and item $j \in B$, if a configuration $C$ containing $j$ is fractionally assigned to $i$. Define
$$w_{i,j}=\sum_{C \ni j} x_{i,C},$$
i.e., $w_{i,j}$ is the
fraction of item $j$ that is allocated to person $i$ by the fractional solution of the LP. An edge $(i,j)$ is called a {\em matching edge}, if the item $j$ is big for person $i$. Otherwise it is called a {\em flow edge}.





Let $M$ and $F$ represent the set of matching and flow edges respectively. For each vertex $v \in A \bigcup B$, let $m_{v}$ denote the total fractional weight of the matching edges incident to it. That is if $v$ is a person then $$m_v=\sum_{j \text{ is a big item for } v} w_{v,j}=\sum_{j \in C, C \text{ contains the big item $j$ for } v} x(v,C).$$ And if $v$ is a job then $$m_v=\sum_{i: \text{ $v$ is a big item for } i} w_{i,v}=\sum_{i,  C \text{ contains $v$ as a big item for } i} x(i,C).$$  Also define $f_{v}=1-m_v$. The main steps of the algorithm are as follows.

\bigskip


\begin{enumerate}
\item[1.] Guess the value of the optimal solution $T$ by doing a binary search. Solve LP (\ref{eqn:lp}).
  Obtain the set $M$ and $m_v, f_v$ for each vertex $v$ in $G$ constructed from the LP solution.

  \bigskip

\item[2] {\bf Allocating Big Items}: Select a random matching from edges in $M$ using { \em bipartite dependent rounding}
(see Section \ref{subsec:match}) such that for every $v \in A \bigcup B$, the probability that $v$ is matched by the matching is $m_v=1-f_v$.

\bigskip

\item[3] {\bf Allocating Small Items}: Let $\epsilon_1=\sqrt{\frac{\ln{k}}{k}}$.
\begin{enumerate}
\item Discard any item $j$ with $m_{j} \geq (1-\epsilon_1)$, and also discard all the persons and the items matched by the matching.
\item (Scaling) In the remaining graph containing only flow edges for unmatched persons and items, set for each
person $i$, $w'_{i,j}=\frac{w_{i,j}}{f_{i}}, ~ \forall j$.
\item Further discard any item  $j$ with $\sum_{i} w'_{i,j} \geq \psi(k)$, where $\psi(k)$ is defined below.
\item Scale down the weights on all the remaining edges by a factor of $\psi(k)$ and run the algorithm of \cite{dani:05} to assign the small items.
\end{enumerate}

matched
\end{enumerate}

 \noindent{\bf Choice of $\psi(k)$.}



Let us consider the functions $\Phi(k) = 100 \ln\ln\ln k / \ln\ln k$ (note that
$\Phi$ is asymptotically zero.) and
\begin{equation}
\label{eqn:psi-bound}
\psi = \psi(k) = \frac{3 \ln k}{\ln \ln k} \cdot (1 + \Phi(k));
\end{equation}
For large enough $k$, say $k \geq 10$, the following holds.
\begin{equation}
\label{eqn:Phi-bound}
(1 + \psi) \ln(1 + \psi) - \psi \geq 3 \ln k
\end{equation}
 This is easily verified by plugging the fact $\ln(1 + \psi) \geq \ln\ln k - \ln\ln\ln k$ into (\ref{eqn:Phi-bound}).

\bigskip


We now analyze each step. The main proof idea is in showing that there remains enough left-over utility in the flow graph for each person
not matched by the matching. This is obtained through proving a negative correlation property among the random variables
defined on a collection of vertices. Previously, the negative correlation property due to bipartite dependent
rounding was known for variables defined on edges incident on any particular vertex. We adapt the proof according to our need.

\subsubsection{Allocating Big Items}
\label{subsec:match}
Consider the edges in $M$ in the person-item bipartite graph. Remove all the edges $(i,j)$ that have already been rounded to $0$ or $1$. Additionally, if an edge is rounded to $1$, remove both its endpoints $i$ and $j$.  We initialize for each $(i,j) \in M$, $y_{i,j}=w_{i,j}$, and modify the $y_{i,j}$ values probabilistically in rounds using bipartite dependent rounding.


\paragraph{{\bf Bipartite Dependent Rounding}\cite{gkps:dep-round}}
\label{subsec:bipartite}
 We give a brief sketch of bipartite dependent rounding introduced in \cite{gkps:dep-round} for the sake of completeness.

  The bipartite dependent rounding selects an even cycle $\mathcal{C}$ or a maximal path $\mathcal{P}$ in $G$, and partitions the edges in $\mathcal{C}$ or $\mathcal{P}$ into two matchings $\mathcal{M}_{1}$ and $\mathcal{M}_{2}$. Then, two positive scalars $\alpha$ and $\beta$ are chosen as follows:
 $$\alpha =\min\{\eta > 0: ((\exists(i,j) \in \mathcal{M}_{1}: y_{i,j}+\eta=1) \bigcup(\exists(i,j) \in \mathcal{M}_{2}: y_{i,j}-\eta=0))\};$$
 $$\beta =\min\{\eta > 0: ((\exists(i,j) \in \mathcal{M}_{1}: y_{i,j}-\eta=0)\bigcup(\exists(i,j) \in \mathcal{M}_{2}: y_{i,j}+\eta=1))\};$$

Now with probability $\frac{\beta}{\alpha+\beta}$, set
\begin{eqnarray*}
&& y'_{i,j}=y_{i,j}+\alpha \text{ for all } (i,j) \in \mathcal{M}_{1} \\
\text{ and } && y'_{i,j}=y_{i,j}-\alpha \text{ for all } (i,j) \in \mathcal{M}_{2};
\end{eqnarray*}
with complementary probability of $\frac{\alpha}{\alpha+\beta}$, set
\begin{eqnarray*}
&& y'_{i,j}=y_{i,j}-\beta \text{ for all } (i,j) \in \mathcal{M}_{1} \\
\text{ and }&& y'_{i,j}=y_{i,j}+\beta \text{ for all } (i,j) \in \mathcal{M}_{2};
\end{eqnarray*}

The above rounding scheme satisfies the following two properties, which are easy to verify:
\begin{equation}
\label{dep:prop1}
\forall \,i,j,\, \expect{y'_{i,j}}=y_{i,j}
\end{equation}
\begin{equation}
\label{dep:prop2}
\exists\,i,j,\, y'_{i,j}\in \{0,1\}
\end{equation}


Thus, if $Y_{i,j}$ denotes the final rounded values then Property (\ref{dep:prop1}) guarantees for every edge $(i,j)$,  $\expect{Y_{i,j}}=w_{i,j}$. This gives the following corollary.



\begin{corollary}
\label{cor:match1}
The probability that a vertex $v \in A \bigcup B$ is matched in the matching generated by the algorithm is $m_v$.
\end{corollary}

\begin{proof}
Let there be $l \geq 0$ edges $e_1,e_2,..e_l \in M$ that are incident on $v$. Then,
\begin{eqnarray*}
\prob{v~ \text{is matched}}&=&\prob{\exists\, e_i, i \in [1,l]~~ s.t~~ v~\text{is matched with }~e_i}\\
&=&\sum_{i=1}^{l} \prob{ v \text{ is matched with } e_{i}}=\sum_{i=1}^{l} w_{i}=m_{v}
\end{eqnarray*}

Here the second equality follows by replacing the union bound by sum since the events are mutually exclusive.
\end{proof}

\paragraph*{Negative Correlation over Multiple Vertices} 
Now we show additional properties of this rounding to be used crucially for the analysis of the next step.
Recall the notion of negative correlation from Definition~\ref{defn:neg-correl}.
We show a useful negative-correlation property for dependent rounding on bipartite graphs over multiple vertices. The proof is syntactically similar to Lemma~2.2 of \cite{gkps:dep-round}. However, \cite{gkps:dep-round} only shows negative correlation property for random variables defined on edges incident to a single vertex; here a stronger negative correlation property is proven for random variables defined on multiple vertices. We state the theorem here, and prove it in the appendix. 



\begin{theorem}
\label{theorem:neg1}
Define an indicator random variable $z_{j}$ for each item $j \in B$ with $m_{j} < 1$, such that $z_{j}=1$
 if item $j$ is matched by the matching. Then, the indicator random variables $\{z_{j}\}$ are negatively correlated.
 \end{theorem}



 As a corollary of Theorem \ref{theorem:neg1}, we get the following:
 \begin{corollary}
\label{cor:neg2}
Define an indicator random variable $z_{i}$ for each person $i \in A$, such that $z_{i}=1$
 if person $i$ is matched by the matching. Then, the indicator random variables $\{z_{i}\}$ are negatively correlated.
 \end{corollary}

\begin{proof}
Do the same analysis as in Theorem \ref{theorem:neg1} with items replaced by persons.
\end{proof}


\subsubsection{Allocating small items}
\label{subsec:alloc}

We start by proving in Lemma \ref{lemma:small} that after the matching phase, we have with high probability that
each unmatched person has available items with utility at least $\sqrt{\frac{\ln{k}}{k}}\frac{T}{5}$ in the flow graph. Additionally we prove in Lemma  \ref{lemma:item} that given any \emph{particular} item $j$, we have with probability at least $1 - O(1/k)$ that $j$ is claimed at most
$\psi(k)$ times. Note that this probability is not large enough to afford a union bound over all the $m$ possible values of $j$, since $m$ is not
bounded as a function of $k$; Lemma~\ref{lemma:stepc} shows how to get around this issue. Both of these probabilistic results use
Theorem~\ref{thm:chbound-negcorrel}.



\begin{lemma}
\label{lemma:small}
After Step 2 of  allocation of big items by bipartite dependent rounding, we have the probability for all unmatched person to have a total utility of at least
$\sqrt{\frac{\ln{k}}{k}}\frac{T}{5}$ from the unmatched items is at least $1-\frac{1}{k}$.
\end{lemma}


\begin{proof}
Consider a person $v$ who is unsatisfied by the matching. Define $w'_{v,j}=\frac{w_{v,j}}{f_{v}}$. Then according to LP (\ref{eqn:lp}) solution
\begin{equation}
\sum_{j}w'_{v,j} u_{v,j} =T
\end{equation}


In step (a) of {\bf Allocation of Small Items}, all items $j$ with $m_{j}$
 at least $(1-\epsilon_1)$ are discarded; recall that $\epsilon_1=\sqrt{\frac{\ln{k}}{k}}$. Since the total sum of $m_j$ can be at most $k$ (the number of persons), there can be at most $\frac{k}{1-\epsilon_1}$ items with $m_j$ at least $1-\epsilon_1$. Therefore, for the remaining items, we have $f_j \geq \epsilon_1$. Each person is connected only to small items in the flow graph. After removing the items with $m_j$ at least $1-\epsilon_1$, the remaining utility in the flow graph for person $v$ is  at least
 \begin{equation}
 \label{eq:1}
 \sum_{j: f_j \geq \epsilon_1}w'_{v,j} u_{v,j} =
 \left(T-\sum_{j: f_{j}\leq \epsilon_1} u_{v,j} f_{j}\right)\geq \left( T-\frac{\epsilon_1 k}{1-\epsilon_1}\frac{T}{\lambda}\right).
\end{equation}
 Now consider random variables $Y_{v,j}$ for each of these unmatched items:
\begin{equation}
\label{eqn:def}
Y_{v,j} = \begin{cases}  \frac{w'_{v,j} u_{v,j}}{T/\lambda} &: \text{if item $j$ is not matched} \\
0 &: \text{ otherwise} \end{cases}
\end{equation}

Since $u_{v,j} \leq T/\lambda$ and $w_{v,j} \leq f_{v}$, the
$Y_{v,j}$ are random variables bounded in $[0,1]$. Person $v$ is unmatched by the matching
with probability $1-m_v=f_v$. Each such person $v$  gets
a fractional utility of $w'_{v,j} u_{v,j}$ from the small (with respect to the person) item $j$ in the flow graph, if item $j$ is
not matched by the matching. The latter happens with probability $f_{j}$.

Define $G_{v}=\sum_{j}Y_{v,j}$. Then $\frac{T}{\lambda} G_v$ is the total fractional utility after step (b). It
follows from (\ref{eq:1}) that
\begin{eqnarray*}
\expect{G_{v}}&=&\sum_{j}\frac{w'_{v,j}u_{v,j}f_j}{T/\lambda}\geq \epsilon_1 \lambda \left(1-\frac{\epsilon_1 k}{(1-\epsilon_1)\lambda}\right)
\end{eqnarray*}

Thus, since $\lambda = 26 \sqrt{k \ln k}$, we have for sufficiently large $k$ that
\[ \expect{G_{v}} \geq \epsilon_1\lambda \left(1-\frac{\epsilon_1 k}{(1-\epsilon_1)\lambda}\right) \geq 24\ln{k}. \]

That the $Y_{v,j}$'s are negatively correlated follows from Theorem \ref{theorem:neg1}. Therefore, applying Theorem~\ref{thm:chbound-negcorrel}(i)
with $\delta = 1/2$,
\[ \prob{G_v \leq \frac{1}{2} \expect{G_v}} \leq e^{-24\ln{k}/12} = \frac{1}{k^2}; \]
i.e.,
 \begin{equation*}
\prob{\frac{T}{\lambda}G_v \leq \frac{1}{2} \frac{T}{\lambda}\expect{G_v}} \leq \frac{1}{k^2}
\end{equation*}

Hence,
 \begin{equation*}
\prob{\exists v: ~\frac{T}{\lambda}G_v \leq \frac{1}{2} \frac{T}{\lambda}\expect{G_v}} \leq \frac{1}{k}.
\end{equation*}

Therefore the net fractional utility that remains for each person  in the flow graph after scaling is at least $ \frac{1}{2} \frac{T}{\lambda}\expect{G_v}= \frac{1}{2} \frac{T}{26\sqrt{k\ln{k}}}12\ln{k} \geq \frac{T}{5}\sqrt{\frac{\ln{k}}{k}}$, with probability at least $1-\frac{1}{k}$.
\end{proof}

\begin{lemma}
\label{lemma:item}
Fix any item $j$ that is unmatched after Step 2. After the matching and the scaling (step (b)), $j$ has a total fractional incident edge-weight
from the unmatched persons to be at most $\psi(k)$,  with probability at least $1-\frac{1}{k^3}$.
\end{lemma}

\begin{proof}
Note that for any person $v$ for which $j$ is small for $v$, $w_{v,j} \leq f_{v}$; hence,
$w'_{v,j}=\frac{w_{v,j}}{f_{v}}\leq 1$. Define a random variable $Z_{v,j}$ for each person $v$ as:
\begin{equation}
\label{eqn:itemdef}
Z_{v,j} = \begin{cases} w'_{v,j} &: \text{if person $v$ is not matched} \\
0 &: \text{ otherwise} \end{cases}
\end{equation}

Let $X_{j}=\sum_{v} Z_{v,j}$. Then $X_{j}$ is the total weight of all the edges incident on item $j$ in the flow graph
after scaling and removal of all matched persons. We have $\expect{X_{j}}= \sum_{v}w'_{v,j} f_{v}=\sum_{v} w_{v,j} \leq 1$.
The fact that the variables $Z_{v,j}$ are negatively correlated follows from Corollary \ref{cor:neg2}. Thus, applying
Theorem~\ref{thm:chbound-negcorrel}(ii) with $\mu = 1$ and $\delta = \psi(k)$ along with (\ref{eqn:Phi-bound}), we obtain
\[
\prob{X_{j} \geq \psi(k)} \leq \frac{1}{k^3}.
\]
This completes the proof.
\end{proof}

Recall the third step, step (c), of {\bf Allocating Small Items}. Any job in the remaining flow graph with total weight of
incident edges more than $\psi(k)$ is discarded in this step. We now calculate the utility that remains for
each person in the flow graph after step (c).

\begin{lemma}
\label{lemma:stepc}
After removing all the items that have total degree more than $\psi(k)$ in the flow graph, that is after step (c)
of {\bf Allocating Small Items}, the probability that all unmatched persons have remaining utility in the flow graph at least $\sqrt{\frac{\ln{k}}{k}}\frac{T}{2*(3+o(1))}$ is at least $1-\frac{2}{k}$.
\end{lemma}
\begin{proof}
Fix a person $v$ and consider the utility that $v$ obtains from the fractional assignments in the flow graph before step (c). It is at least $\sqrt{\frac{\ln{k}}{k}}\frac{T}{5}$ from Lemma \ref{lemma:small}.
Define a random variable for each item that $v$ claims with nonzero value in the flow graph at step (b):

\begin{equation}
\label{eqn:newdef}
Z'_{v,j} = \begin{cases} u_{v,j} &: \text{if item $j$ has total weighted degree at least $\psi(k)$ } \\
0 &: \text{ otherwise} \end{cases}
\end{equation}

We have $\prob{Z'_{v,j}=u_{v,j}} \leq \frac{1}{k^3}$ from Lemma \ref{lemma:item}.
Therefore, the expected utility for $v$ from all the items in the flow graph that have total incident weight
more than  $\psi(k)$  is at most $\frac{T}{k^3}$. By Markov's
inequality, the probability that the utility for $v$ from the discarded items is more than $\frac{T}{k}$, is at most $\frac{1}{k^2}$. Applying
the union bound, the probability of the utility from the discarded items being more than $\frac{T}{k}$ for some person,
is at most $\frac{1}{k}$. The initial utility before step (c) was at least $\sqrt{\frac{\ln{k}}{k}}\frac{T}{5}$ with probability $1-\frac{1}{k}$.
Thus after step (c), the remaining utility is at least $\sqrt{\frac{\ln{k}}{k}}\frac{T}{5}-\frac{T}{k}$ with probability at least $1-\frac{2}{k}$.
\end{proof}

The next and the final step (d) of allocations is to run \cite{dani:05} on a {\em scaled-down} flow graph. 
The weight on the remaining edges is scaled down by a factor of $\psi(k)$ and hence
for every item node that has not been matched after step (c), the total edge-weight incident on it is at most $one$. Hence after scaling down the utility of any person $v$ in the flow graph is
$\sum_{j} u_{v,j} W_{v,j} \geq \frac{\ln{\ln{k}}}{\ln{k}}\sqrt{\frac{\ln{k}}{k}}\frac{T}{18(1+o(1))}=
\frac{\ln{\ln{k}}}{\sqrt{k\ln{k}}}\frac{T}{18(1+o(1))}$, where $W_{v,j}$ denote the scaled down weight on the edge $(v,j)$. Also, note that the maximum utility of any item in the flow graph is at most $\frac{T}{\lambda}=\frac{T}{26\sqrt{k\ln{k}}}$. Hence, by running the algorithm of \cite{dani:05}, which is a simpler version of Theorem~\ref{thm:matching}, we get the following lemma.







\begin{lemma}
\label{lemma:unsat}
For all persons unmatched by the matching, the total utility received is at least $\Omega(\frac{\ln{\ln{k}}}{\sqrt{k\ln{k}}}T)$ after step (d) with probability at least $1-\frac{2}{k}$.
\end{lemma}
 \begin{proof}
 Let $W_{v,j}$ denote the fractional weight on the scaled down flow graph. Then for every item $j$ in the flow graph, $\sum_{v} W_{v,j} \leq 1$. And for every person $v$ considering the items in the flow graph, $\sum_{j} W_{v,j} u_{v,j} \geq \frac{\ln{\ln{k}}}{\sqrt{k\ln{k}}}\frac{T}{18(1+o(1))}$ with probability at least $1-\frac{2}{k}$. We can now employ the rounding algorithm of \cite{dani:05} which is a simplification of Theorem \ref{thm:matching} without any capacity constraint. We get an integer solution where each person receives a utility of at least $\frac{\ln{\ln{k}}}{\sqrt{k\ln{k}}}\frac{T}{18(1+o(1))}-\frac{T}{\lambda}$, and every item is assigned to at most one person. Since $\lambda = 26\sqrt{k\ln{k}}$, we get the desired result.
\end{proof}
~\\
{\it Theorem~\ref{thm:santa1}
Given any feasible solution to the configuration LP, it can be rounded to a feasible integer solution such that every person gets  at least $\Theta\left(\frac{1}{\sqrt{k\ln{k}}}\right)$ fraction of the optimal utility with probability at least $1-\Theta(\frac{1}{k})$, in polynomial time.
}
\begin{proof}
Any person that is matched by step $2$ of the algorithm {\bf Allocating Big Items} receives a utility of $\frac{T}{\lambda}$. From Lemma~\ref{lemma:unsat}, each person unmatched by the matching receives a utility of $\Omega(\frac{\ln{\ln{k}}}{\sqrt{k\ln{k}}}T)$ with probability at least $1-\frac{2}{k}$. Noting that $\lambda=26\sqrt{k\ln{k}}$, we therefore, get the theorem.
\end{proof}

Thus, our approximation ratio is $\Theta(\frac{1}{\sqrt{k\ln{k}}})$.
This provides an upper bound of $O(\sqrt{k\ln{k}})$ on the integrality gap of the configuration LP for max-min fair allocation, nearly
matching the  lower bound of $\Omega(\sqrt{k})$ due to \cite{bansal:stoc06}. 


\section{Designing Overlay Multicast Networks For Streaming}
\label{section:spaa}
The work of \cite{DBLP:conf/spaa/AndreevMMS03} studies approximation algorithms
for designing a multicast overlay network. We first describe the problem and state the results in \cite{DBLP:conf/spaa/AndreevMMS03} (Lemma \ref{lem:spaa1} and Lemma \ref{lem:spaa2}). Next, we show our main improvement in Lemma \ref{lem:color}.

\subsection{Background} The background text here is largely borrowed from \cite{DBLP:conf/spaa/AndreevMMS03}. An overlay network can be represented as a tripartite digraph $N=(V,E)$. The nodes $V$ are partitioned into sets of entry points called sources ($S$), reflectors ($R$), and edge-servers or sinks ($D$). There are multiple commodities or streams, that must be routed from sources, via reflectors, to the sinks that are designated to serve that stream to end-users. Without loss of generality, we can assume that each source holds a single stream. 
There is a cost associated with usage of every link and reflector. 
 There are capacity constraints, especially on the reflectors,
that dictate the maximum total bandwidth (in bits/sec) that the reflector is allowed to send. 
To ensure reliability, multiple copies of each stream may be sent to the designated edge-servers.
\begin{table*}
\begin{eqnarray}
\text{min} && \sum_{i \in R} r_{i}z_{i} +\sum_{i \in R}\sum_{k \in S}c_{k,i,k}y_{i,k}+\sum_{i \in R}\sum_{k \in S}\sum_{j \in D}c_{i,j,k}x_{i,j,k} \nonumber\\
s.t. &&\\
&& y_{i,k} \leq z_{i}~~\forall i \in R,~~\forall k \in S \label{eqn:cons1} \\
&& x_{i,j,k} \leq y_{i,k}~~\forall i \in R, ~~\forall j \in D, ~~\forall k \in S \label{eqn:cons2} \\
&& \sum_{k \in S} \sum_{j \in D} x_{i,j,k} \leq F_{i}z_{i}~~\forall i \in R
\label{eqn:fanout} \\
&& \sum_{i \in R}x_{i,j,k}w_{i,j,k} \geq W_{j,k}~~\forall j \in D, \forall k \in S \label{eqn:weight} \\
&& x_{i,j,k} \in \{0,1\}, y_{i,k} \in \{0,1\}, z_{i} \in \{0,1\}
\label{eqn:integrality}
\end{eqnarray}
\caption{Integer Program for Overlay Multicast Network Design}
\label{table:spaa}
\end{table*}

All these requirements can be captured by an integer program. Let us use indicator variable $z_{i}$ for building reflector $i$, $y_{i,k}$ for delivery of $k$-th stream to the $i$-th reflector and $x_{i,j,k}$  for delivering $k$-th stream to the $j$-th sink through the $i$-th reflector. $F_{i}$ denotes the fanout constraint for each reflector $i \in R$. Let $p_{x,y}$ denote the failure probability on any edge (source-reflector or reflector-sink). We transform the probabilities into weights: $w_{i,j,k}=-\log{(p_{k,i}+p_{i,j}-p_{k,i}p_{i,j})}$. Therefore, $w_{i,j,k}$ is the negative log of the probability of a commodity $k$  failing to reach sink $j$ via reflector $i$. On the other hand, if $\phi_{j,k}$ is the minimum required success probability for commodity $k$ to reach sink $j$, we instead use $W_{j,k}=-\log{(1-\phi_{j,k})}$. Thus $W_{j,k}$ denotes the negative log of maximum allowed failure. $r_i$ is the cost for opening the reflector $i$ and $c_{x,y,k}$ is the cost for using the link $(x,y)$ to send commodity $k$. Thus we have the IP (see Table \ref{table:spaa}).

Constraints (\ref{eqn:cons1}) and (\ref{eqn:cons2}) are natural consistency
requirements; constraint (\ref{eqn:fanout}) encodes the fanout restriction.
Constraint (\ref{eqn:weight}), the \emph{weight} constraint, ensures
quality and reliability. Constraint (\ref{eqn:integrality}) is the standard
integrality-constraint that will be relaxed to construct the LP relaxation.

There is an important  stability requirement that is referred as \emph{color constraint} in \cite{DBLP:conf/spaa/AndreevMMS03}. Reflectors are grouped into $m$ color classes, $R=R_{1} \cup R_{2} \cup \ldots \cup R_{m}$. We want each group of reflectors to deliver not more than one copy of a stream into a sink. This constraint translates to
\begin{equation}
\label{eqn:color}
\sum_{i \in R_{l}} x_{i,j,k} \leq 1~\forall j \in D, ~\forall k \in S, ~\forall l \in [m]
\end{equation}

Each group of reflectors can be thought to belong to the same ISP. Thus we want to make sure that a client is served only with one -- the best -- stream possible from a certain ISP. This diversifies the stream distribution over different ISPs and provides stability. If an ISP goes down, still most of the sinks will be served. We refer the LP-relaxation of  integer program
(Table \ref{table:spaa}) with the color constraint (\ref{eqn:color})
as \textbf{LP-Color}.

All of the above is from \cite{DBLP:conf/spaa/AndreevMMS03}.
The work of \cite{DBLP:conf/spaa/AndreevMMS03} uses a two-step rounding procedure and obtains the following guarantee.

First stage rounding: Rounds $z_{i}$ and $y_{i,k}$ for all $i$ and $k$ to decide which reflector should be open and which streams should be sent to a reflector. The results here can be summarized in the following
lemma:

\begin{lemma}
\textbf{(\cite{DBLP:conf/spaa/AndreevMMS03})}
\label{lem:spaa1}
The first-stage rounding algorithm incurs a cost at most a factor of $64\log{|D|}$ higher than the optimum cost, and with high probability violates the weight constraints by at most a factor of $\frac{1}{4}$ and the fanout constraints by at most a factor of $2$. Color constraints are all satisfied.
\end{lemma}




Second stage rounding: Rounds $x_{i,j,k}$'s using the open reflectors and streams that are sent to different reflectors in the first stage. The results in this stage can be summarized as follows:

\begin{lemma}
\textbf{(\cite{DBLP:conf/spaa/AndreevMMS03})}
\label{lem:spaa2}
The second-stage rounding incurs a cost at most a factor of $14$ higher than the optimum cost and violates each of fanout, color and weight constraint by at most a factor of $7$.
\end{lemma}

\subsection{Main Contribution}

Our main contribution is an improvement of the second-stage rounding
through the use of repeated \textbf{RandMove} and by judicious choices of
constraints to drop. Let us call the linear program that remains just at the end of first stage \textbf{LP-Color2}:

\begin{eqnarray*}
&&\min{\sum_{i \in R}\sum_{k \in S}\sum_{j \in D} c_{i,j,k}x_{i,j,k}}\\
&& \text{s.t.}\\
&& \sum_{k \in S}\sum_{j \in D}x_{i,j,k} \leq F_{i}~\forall i \in R ~(\text{Fanout})\\
&& \sum_{i \in R}x_{i,j,k}w_{i,j,k} \geq W_{j,k}~\forall j \in D, \forall k \in S ~(\text{Weight})\\
&& \sum_{i \in R_{l}} x_{i,j,k} \leq 1~\forall j \in D, ~\forall k \in S, ~\forall l \in [m]~(\text{Color})\\
&& x_{i,j,k} \in \{0,1\} ~\forall i \in R,\forall j \in D, \forall k \in S\\
\end{eqnarray*}

We show:

\begin{lemma}
\label{lem:color}
\textbf{LP-Color2} can be efficiently rounded such that the cost and weight constraints are satisfied exactly, fanout constraints are violated at most by additive $1$, and the color constraints are violated at most by additive $3$.
\end{lemma}

The proof is very similar to Theorem~\ref{thm:matching}. Note that, here instead of having capacity constraints, we have fanout constraints. Weight constraints correspond to load constraints in Theorem~\ref{thm:matching}, but now they provide lower bounds. Moreover, the color constraints can be thought of as additional capacity constraints imposed on a set of reflectors. This constitutes the main change from Theorem~\ref{thm:matching}, and we need new conditions to drop color constraints $(D2'')$. The color constraints being all disjoint help us in the rounding.

\begin{proof}
Let $x^{*}_{i,j,k} \in [0,1]$ denote the fraction of stream generated from source $k \in S$ reaching destination
$j \in D$ routed through reflector $i \in R$ after the first stage of rounding. Initialize $X=x^*$. The algorithm
consists of several iterations. the random value at the end of iteration $h$ is denoted by $X^{h}$. Each iteration $h$ conducts a randomized update using {\bf RandMove} on the polytope of a linear system constructed from a subset of constraints of {\bf LP-Color2}. Therefore by induction on $h$, we will have for all $(i,j,h)$ that $\expect{X_{i,j}^{h}}=x^{*}_{i,j}$. Thus the cost constraint is maintained exactly on expectation. The entire procedure can be derandomized by the method of conditional probabilities, yielding the required bounds on the cost.

Let $R$ and $SD$ denote the set of reflectors and (source, destination) pairs respectively.
Suppose we are at the beginning of some iteration $(h+1)$ of the overall algorithm and currently looking
at the values $X_{i,j,k}^{h}$. We will maintain two invariants:
\begin{description}
\item[(I1'')] Once a variable $x_{i,j,k}$ gets assigned to $0$ or $1$, it is never changed;
\item[(I2'')] Once a constraint is dropped in some iteration, it is never reinstated.
\end{description}
Iteration $(h+1)$ of rounding consists of three main steps:

\begin{enumerate}
\item Since we aim to maintain ({\bf I1''}), let us remove all $X_{i,j,k}^{h} \in \{0,1\}$; i.e.,
we project $X^{h}$ to those coordinates $(i,j,k)$ for which $X_{i,j,k}^{h} \in (0,1)$, to obtain the
current vector $Y$ of floating (yet to be rounded) variables; let $\mathcal{S}\equiv(A_{h}Y=u_{h})$ denote
the current linear system that represents {\bf LP-Color2}. In particular, the fanout constraint
for a reflector in $\mathcal{S}$ is its residual fanout $F'_{i}$; i.e., $F_{i}$ minus the
number of streams that are routed through it.
\item Let $v$ denote the number of floating variables, i.e., $Y\in (0,1)^{v}$. We now drop the following constraint:
    \begin{description}
    \item[(D1'')] Drop fanout constraint for degree $1$ reflector denoted $R_1$, i.e, reflectors with only one floating variable associated with it. For any degree $2$ reflectors denoted $R_2$, if
        it has a tight fanout of $1$ drop its fanout constraint.
    \item[(D2'')] Drop color constraint for a group of reflectors $R_{l}$, if they have at most four floating variables associated with them.
    \end{description}
    \end{enumerate}
Let $\mathcal{P}$ denote the polytope defined by this reduced system of constraints.
A key claim is that $Y$ is not a vertex of $\mathcal{P}$ and thus we can apply
{\bf RandMove} and make progress either by rounding a new variable or by dropping
a new constraint. We count the number of variables $v$ and the number of tight constraints
 $t$ separately. We have $$t=\sum_{i \in R \setminus R_1} 1+\sum_{k \in S}\sum_{j \in D}(l_{k,j}+1),$$ where
 $l_{j,k}$ is the number of tight color constraints for the stream generated at source $k$
 and to be delivered to the  destination $j$.  We further have $v \geq \sum_{i \in R} (F_i +1)$, and that
$v \geq \sum_{k \in S, \j \in D, l_{k,j} > 0} 4l_{k,j} + \sum_{k \in S, \j \in D, l_{k,j}= 0} 2$. Thus by averaging, $$v \geq \frac{\sum_{i \in R} (F_i +1)}{2}+  \sum_{k \in S, \j \in D, l_{k,j} > 0} 2l_{k,j}+\sum_{k \in S, \j \in D, l_{k,j}= 0} 1.$$ A moment's reflection shows that the
 system can become underdetermined only if there is no color constraint associated with a stream $(j,k)$, each reflector $i$ has two floating variables associated with it with total contribution $1$ towards fanout and each stream $(j,k)$ is routed fractionally through two reflectors. But in this situation all the fanout constraints are dropped violating fanout at most by an additive one and making the system underdetermined once again.
The color constraints are dropped only when there are less than four
floating variables associated with that group  of reflectors; hence, the color constraints can get violated at most by an additive $3$. The fanout constraint is dropped only for singleton reflectors or degree-2 reflectors with fanout equaling $1$. Hence the fanout is violated only by an additive
excess of $1$. The weight constraint is never dropped, and
is hence maintained exactly.
\end{proof}
