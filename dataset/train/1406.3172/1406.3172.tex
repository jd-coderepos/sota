\documentclass[journal,onecolumn]{IEEEtran}
\usepackage{graphics}
\usepackage{graphicx}
\usepackage{url}
\usepackage{setspace}

\title{Optimal Gaussian Filter for Effective Noise Filtering}
\author{Sunil Kopparapu and M Satish
\thanks{Sunil Kumar Kopparapu and M Satish are with the TCS Innovation Labs - Mumbai, Yantra Park, Thane (West), Maharastra, INDIA.
Email: SunilKumar.Kopparapu@TCS.Com}}

\newcommand{\f}{{\cal F}}
\newcommand{\fs}{f_{s}}
\newcommand{\fmax}{f_{max}}
\newcommand{\noise}{n}
\newcommand{\Noise}{N}
\newcommand{\Signal}{X}
\newcommand{\signal}{x}
\newcommand{\x}{\signal}
\newcommand{\g}{f}
\renewcommand{\S}{{\cal S}}
\newcommand{\BW}{{\cal B}}
\newcommand{\slength}{{\cal N}}
\newcommand{\flength}{{\cal M}}
\newcommand{\w}{\omega}
\renewcommand{\a}{\alpha}
\newcommand{\dt}{m}
\newtheorem{mynote}{Note}

\begin{document}
\maketitle

\doublespace
\begin{abstract}

In this paper we show that the knowledge of noise statistics 
contaminating a signal can be effectively used to choose an optimal
 Gaussian filter to 
eliminate noise. Very specifically, we show that the additive white 
Gaussian noise (AWGN) contaminating a signal can be filtered best by 
using a Gaussian filter of specific characteristics. The design of the 
Gaussian filter bears relationship with the noise 
statistics and also some basic information about the signal. We first 
derive a relationship between the properties of the Gaussian filter, 
noise statistics and the signal and later show through experiments that 
this relationship can be used effectively to identify the optimal 
Gaussian filter that can effectively filter noise.
\end{abstract}

\begin{IEEEkeywords}
Filtering, Gaussian Smoothing, Noise removal
\end{IEEEkeywords}


\section{Introduction}

Signal smoothing or noise filtering or denoising 
has been an area of active research and continues to hold 
the attention of researchers in various fields, for example,
\cite{Crisan_Kouritzin_Xiong_2008,Oktem_Egiazarian_Lukin_Ponomarenko_Tsymbal_2007,Buades_Silva_Santos_2010,Huang_Wang_Long_2009,Yang_Wei_2010}.
Noise is inherent in signals 
\cite{Bruni_Piccoli_Vitulano_2008,Narayana:2009:ENM:1946497.1946503}
and a necessary first step is noise removal before any other processing can
take place. A successful pre-processing step to remove noise improves the
performance of the {\em actual processing} on the signal \cite{lajish_2010}.
There are essentially two ways of taking
care of noise in the signal, namely, (a) pre-processing of the signal to
enable noise removal or (b) use of a set of robust algorithms that can
compensate for the inherent noise. In signal processing 
literature pre-processing of the signal is
the preferred approach.

\subsection{Problem}
\label{sec:problem}
Let  be a band 
limited () 
digitized signal which is sampled at a sampling frequency of  and
Let 
 be the noise sequence.
Further assume that  is 
Gaussian distributed with mean  and variance 
. Let 
  
represent the signal  
contaminated by AWGN . 
Now the problem can be stated as, given 
 estimate  such that the error in the 
estimate is minimum, namely 
  
Typically the process of 
estimating  given the noise contaminated 
 is called noise filtering or denoising. 
We will restrict our 
discussion, in this paper to the usage of a Gaussian smoothing filter for noise 
removal.
We 
describe Gaussian 
filtering in Section \ref{sec:gaussian_filtering} which is characterized 
by  which determines the amount of smoothing. We build theory 
in Section \ref{sec:our_approach} 
which allows identification of an optimal . We show 
experimentally how the identification of the actual Gaussian filter can 
be found in Section \ref{sec:experimental_results} and conclude in 
Section \ref{sec:conclusions}.


\section{Gaussian Smoothing}
\label{sec:gaussian_filtering}


A Gaussian filter is parametrized by its means  and variance 
 and represented by
 
\begin{mynote}
Given  and  one can construct a Gaussian filter 
(\ref{eq:gaussian_filter}) with  running between .  
\end{mynote}
\begin{mynote}
It 
is well known that spanning  between  and  
covers  \% of the total area under the Gaussian. 
\end{mynote}
So we can approximate 
 from  to  
as
 from  to  
for the purpose of 
discussion and subsequent experimentation. Let the discrete version of 
 from  to  
be 
represented by 
  from 
 to 
, where  represents the ceil
of .
Let 
 
smoothed with  
result in , 
namely, 
  
 for . Let the error 
in the estimate be 

We hypothesize that one can achieve an optimal estimate 
 for some  
such that  is minimized. 
We further hypothesize that
 is based on the 
variance of the noise affecting the signal and some properties of the signal. 
Specifically,  is 
dependent directly or indirectly on  and .  

\section{Our Approach}
\label{sec:our_approach}

In the frequency domain we can write (\ref{eq:tdomain}) as
 
 and the Gaussian filter as

  The estimate of the signal 
 due to filtering by Gaussian filter can be
written as
  The error in the filtered output is given by
 
 As seen in (\ref{eq:error_components}) the error in the estimate ()
due to 
filtering has two components namely, one due to distortion of signal 
()
and 
the other due to the reminiscent noise  ()
in the signal after filtering. 
Let  denote the power in the signal , then
input and output signal to noise () ratios are given by 
 
\begin{mynote}
For a certain ,  the Gaussian filter is able to filter the signal
such that . Namely,
simultaneously 
remove the
noise and 
not distort the signal. 
\end{mynote}
\begin{mynote}
If we increase  then the cutoff frequency and the 
bandwidth of Gaussian filter will decrease as seen in 
(\ref{eq:gaussian_frequency}) and subsequently this will lead to more 
noise removal but on same account the signal distortion will also increase. 
\end{mynote}
In the limiting case when 
,  we have an all pass filter and hence 
. 
Let for some  
, such that  
if we increase  further then . 
One can hypothesize that for  in the range 
, . 
We further hypothesize that there exists a 
 (in the range ) for which  
peaks to achieve . We show through curve fitting and later 
experimentally that we can determine the optimal  
such that  is maximized.

\subsection{Determining }

With an aim to identify  the optimal choice of 
Gaussian filter to remove noise we constructed three different signals 
() with different bandwidths (). We constructed the noisy 
signal () by appending  with  with 
varying . For each of this noisy signal we used 
different  Gaussian to filter noise and for each of this 
 is computed.
The band limited   is constructed by first generating 
a random sequence of length  having a normal distribution with mean
zero and variance one. This random signal is smoothened using a filter of length . The impulse response of the smoothing filter is given by

Note that if we take a  point DFT of this smoothed signal, then most of the energy is
limited to  Hz or  points. We cut off the high frequency region of the signal, namely,
we set the points from   to 
to zero. The inverse DFT of this low-pass filtered signal is the test signal with maximum frequency 
 Hz. Note that different values of  produce a filtered signal with different  and 
hence bandwidths ().
In this manner we constructed three different signals, each  of length 
 with . We denote these three signals as
 and  having  of 
,
,
 Hz
respectively. 
An additive white Gaussian 
noise with  and  denoted by  is generated.
In all we had   as our test bed. Namely,
 , 
 , 
 , 
 , 
 , 
 , 
 , 
 , 
 .

 These signals  are denoised using a 
Gaussian filter (\ref{eq:gaussian_filter}) with different . 
We varied  from  to  in steps 
of  ( data points). For all 
these filtered output signal, namely, , the  is 
calculated. Fig. \ref{fig:snr_plot} shows the  of the filtered 
 for different values of . The x-axis shows 
the different values of  and the bell shaped curve is the 
; also  ( dB) is shown as a horizontal line. 
 \begin{figure} 
 \centerline{\includegraphics[width=0.5\textwidth]{snr_105_35}} 
 \caption{The  of filtered  for different values 
of .}
 \label{fig:snr_plot}
 \end{figure}
We had   for varying  for each of the  
noisy signals. We now try to fit a curve so as to relate the  in terms of 
,  and 
.  
We did this in two steps using \cite{web:curve_fit}.

\begin{enumerate}
 \item[Step 1] For a fixed ,  we fit a 3-D curve to relate , 
 
and  for   separately
using the reciprocal full quadratic function\footnote{Experimented with
several functions before converging onto the reciprocal full quadratic function}, namely, 

with minimize
the sum of squared absolute error criteria. 
For each  we obtained a set of coefficients 
 and , so in all we had  coefficients, namely, 
,
,
,
,
 and


\item [Step 2] We then fit a quadratic curve 
for each coefficient set, namely,   and  separately.
Using  we found that  in (\ref{eq:cf1}) is related to  as 
. 
Similarly coefficients
 can be written in terms of . Namely, 
 \end{enumerate}


 Now we have (\ref{eq:cf1}), we get  by 
differentiating (\ref{eq:cf1}) with
respect to
 and setting 
namely, 

where , ,  are given in (\ref{eq:cf2}). We get
 by substituting the value of  in
(\ref{eq:cf1}), namely,


\section{Experimental Results}
\label{sec:experimental_results}

We conducted a number of experiments to verify the correctness of
(\ref{eq:sopt}) and (\ref{eq:ssnr}) in identifying  and
 respectively, these results are shown 
in Table \ref{tab:table1} and Table \ref{tab:table2}.
\begin{table}
\begin{center}
\begin{tabular}{|c|c|c|c|c|c|c|}
\hline \hline
 &  &  &   &  & 
 &  \\
&  &  & (\ref{eq:sopt}) &  & (\ref{eq:ssnr})&  \\
\hline \hline
 &  &  &  &  &  &  \\ \hline
 &  &  &  &  &  &  \\ \hline
 &  &  &  &  &  &  \\ \hline \hline
 &  &  &  &  &  &  \\ \hline
 &  &  &  &  &  &  \\ \hline
 &  &  &  &  &  &  \\ \hline \hline
 &  &  &  &  &  &  \\ \hline
 &  &  &  &  &  &  \\ \hline
 &  &  &  &  &  &  \\ \hline
\end{tabular}
\caption{Comparison of actual ,  with derived
 using (\ref{eq:sopt}),  using (\ref{eq:ssnr}).}
\label{tab:table1}
\end{center}
\end{table}
Table \ref{tab:table1} tries to access the goodness of the curve fit, namely,
the choice of the curve and the construction of (\ref{eq:sopt}) 
and (\ref{eq:ssnr}) from the data. 
As can be seen, the column four ( calculated from
(\ref{eq:sopt})) and column five (actual
 computed from the data) are very close to each other. This is
to be expected when the choice of the curve to fit the data is good. 
However to verify the validity of our approach to identify the 
we conducted another set of experiments. We generated several 
test
signals with different  and
 with different , such that these test 
signals were not part of the signals used to
construct (\ref{eq:sopt}) using curve fitting.  As can be seen in Table
\ref{tab:table2}, the estimation of  using (\ref{eq:sopt}) is very
close to the actual  for all signals in Table \ref{tab:table2}. 
As expected, a similar match is seen for  obtained using 
(\ref{eq:ssnr}) and actual  .
\begin{small}
\begin{table}
\begin{center}
\begin{tabular}{|c|c|c|c|c|c|c|}
\hline \hline
 &  &  &
  &  &
 &  \\
 &  &  &  (\ref{eq:sopt}) &  & (\ref{eq:ssnr})&  \\
\hline \hline
 &  &  &  &  &  &  \\ \hline
 &  &  &  &  &  &  \\ \hline
 &  &  &  &  &  &  \\ \hline \hline 
 &  &  &  &  &  &  \\ \hline
 &  &  &  &  &  &  \\ \hline \hline
 &  &  & &  &  &  \\ \hline
 &  &  &  &  &  &  \\ \hline
 &  &  &  &  &  &  \\ \hline
\end{tabular}
\caption{Comparison of actual ,  with derived
 using (\ref{eq:sopt}),  using (\ref{eq:ssnr}) for test
signals.}
\label{tab:table2}
\end{center}
\end{table}
\end{small}

\section{Conclusions}
\label{sec:conclusions}

Noise removal is a mandatory pre-processing step in many signal processing
applications. In this paper, we have show that it is possible to identify the
optimal Gaussian filter that best filters noise, under the assumption that the
noise is AWGN. The major contribution of this paper is identification of a
method to obtain the optimal Gaussian filter that best filters a
signal contaminated with AWGN. We have shown experimentally that the identified
method works well for signals whose bandwidth and the input signal to noise ratio 
is know. 
We are in the process of verifying the validity of our approach for 
practical signals like speech.



\bibliographystyle{IEEEtran}
\bibliography{estimating_opt_sigma}

\end{document}
