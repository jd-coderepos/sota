\documentclass{article}



\usepackage[final,nonatbib]{neurips_2021}









\usepackage[utf8]{inputenc} \usepackage[T1]{fontenc}    \usepackage{hyperref}       \usepackage{url}            \usepackage{booktabs}       \usepackage{amsfonts}       \usepackage{amsmath}       \usepackage{nicefrac}       \usepackage{microtype}      \usepackage{xcolor}        \usepackage{verbatim}        

\definecolor{olivegreen}{rgb}{0, 0.6, 0}
	\definecolor{emerald}{rgb}{0.31, 0.78, 0.47}
\definecolor{black}{HTML}{000000}
\definecolor{white}{HTML}{ffffff}
\definecolor{color1}{HTML}{ACE5EE}
\definecolor{color2}{HTML}{0093AF}
\definecolor{color3}{HTML}{CC0000}\definecolor{color4}{HTML}{0087BD}
\definecolor{color5}{HTML}{333399}
\definecolor{color6}{HTML}{20B2AA}


\usepackage{xspace}         \usepackage{graphicx}\usepackage{floatrow}
\newfloatcommand{capbtabbox}{table}[][\FBwidth]


\usepackage{pgfplots}
\usepackage{pgfplotstable}
\usepackage{amsmath}

\usepackage{array, makecell} \newcolumntype{x}[1]{>{\centering\arraybackslash\hspace{0pt}}p{#1}}

\usepackage{subcaption}

\usepackage[backend=bibtex, sorting=none]{biblatex}
\AtBeginBibliography{\small}
\bibliography{refs}

\usepackage{multirow}


\usepackage{pifont}\newcommand{\cmark}{\color{olivegreen}\ding{51}}\newcommand{\xmark}{\color{red}\ding{55}}



\newcommand{\aname}{Qimera\xspace}
\newcommand{\JL}[1]{{\color{magenta}[\textbf{\sc JLee}: \textit{#1}]}}
\newcommand{\KH}[1]{{\color{purple}[\textbf{\sc KH}: \textit{#1}]}}
\newcommand{\DK}[1]{{\color{red}[\textbf{\sc DK}: \textit{#1}]}}


\newcommand{\rev}[1]{{\color{olivegreen}#1}}
\renewcommand{\rev}[1]{#1}

\renewcommand{\JL}[1]{}
\renewcommand{\KH}[1]{}
\renewcommand{\DK}[1]{}


\newcommand{\loss}{\mathcal{L}}
\newcommand{\round}[1]{\ensuremath{\lfloor#1\rceil}}

\def\blfootnote{\xdef\@thefnmark{}\@footnotetext}












\title{Qimera: Data-free Quantization with Synthetic Boundary Supporting Samples}



\author{Kanghyun Choi\textsuperscript{1} \And Deokki Hong\textsuperscript{2} \And Noseong Park\textsuperscript{1,2} \And Youngsok Kim\textsuperscript{1,2} \And Jinho Lee\textsuperscript{1,2}\thanks{Corresponding author} \AND \\
  \textsuperscript{1} Department of Computer Science, Yonsei University\\
  \textsuperscript{2} Department of Artificial Intelligence, Yonsei University\\ 
  \texttt{\{kanghyun.choi, dk.hong, noseong, youngsok, leejinho\}@yonsei.ac.kr }


  
}

\begin{document}

\maketitle


\begin{abstract}
Model quantization is known as a promising method to compress deep neural networks, especially for inferences on lightweight mobile or edge devices. 
However, model quantization usually requires access to the original training data to maintain the accuracy of the full-precision models, which is often infeasible in real-world scenarios for security and privacy issues.
A popular approach to perform quantization without access to the original data is to use synthetically generated samples, based on batch-normalization statistics or adversarial learning.
However, the drawback of such approaches is that they primarily rely on random noise input to the generator to attain diversity of the synthetic samples. 
We find that this is often insufficient to capture the distribution of the original data, especially around the decision boundaries.
To this end, we propose Qimera, a method that uses superposed latent embeddings to generate synthetic boundary supporting samples.
For the superposed embeddings to better reflect the original distribution, we also propose using an additional disentanglement mapping layer and extracting information from the full-precision model.
The experimental results show that Qimera achieves state-of-the-art performances for various settings on data-free quantization. 
Code is available at \url{https://github.com/iamkanghyunchoi/qimera}.

\end{abstract}
 \begin{tikzpicture}[overlay, remember picture]
 \path (current page.north) ++(+3,-1) node[below left] {Accepted to NeurIPS 2021. Author's copy};
 \end{tikzpicture}

\section{Introduction}
Among many neural network compression methodologies, quantization is considered a promising direction because it can be easily supported by accelerator hardwares~\cite{tpu} than pruning~\cite{deepcompression} and is more lightweight than knowledge distillation~\cite{hinton2015distilling}.
However, quantization generally requires some form of adjustment (e.g., fine-tuning) using the original training data~\cite{dorefa, zhuang2018towards, robustnessdefensive, jacob2018quantization, shen2020q} to restore the accuracy drop due to the quantization errors.
Unfortunately, access to the original training data is not always possible, especially for deployment in the field, for many reasons such as privacy and security.
For example, the data could be medical images of patients, photos of confidential products, or pictures of military assets.


Therefore, data-free quantization is a natural direction to achieve a highly accurate quantized model without accessing any training data.
Among many excellent prior studies~\cite{aciq, dfq, choi2020data, outlier}, generative methods~\cite{gdfq, zaq, zeroq} have recently been drawing much attention due to their superior performance.
Generative methods successfully generate synthetic samples that resemble the distribution of the original dataset and achieve high accuracy using information from the pretrained full-precision network, such as batch-normalization statistics~\cite{zeroq, gdfq} or intermediate features~\cite{zaq}.



However, a significant gap still exists between data-free quantized models and quantized models fine-tuned with original data.
What is missing from the current generative data-free quantization schemes?
We hypothesize that the synthetic samples of conventional methods lack \emph{boundary supporting samples}~\cite{heo2019knowledge}, which lie on or near the decision boundary of the full-precision model and directly affect the model performance.
The generator designs are often based on conditional generative adversarial networks (CGANs)~\cite{cgan, acgan} that take class embeddings representing class-specific latent features. 
Based on these embeddings as the centroid of each class distribution, generators rely on the input of random Gaussian noise vectors to gain diverse samples. 
However, one can easily deduce that random noises have difficulty reflecting the complex class boundaries.
In addition, the weights and embeddings of the generators are trained with cross-entropy (CE) loss, further ensuring that these samples are well-separated from each other.



\JL{flat --> align , or cite stylegan here}

In this work, we propose \aname, a method for data-free quantization employing superposed latent embeddings to create boundary supporting samples.
First, we conduct a motivational experiment to confirm our hypothesis that samples near the boundary can improve the quantized model performance. 
Then, we propose a novel method based on inputting superposed latent embeddings into the generator to produce synthetic boundary supporting samples. 
In addition, we provide two auxiliary schemes for flattening the latent embedding space so that superposed embeddings could contain adequate features. 


\aname achieves significant performance improvement over the existing techniques. 
The experimental results indicate that \aname sets new state-of-the-art performance for various datasets and model settings for the data-free quantization problem. 
Our contributions are summarized as the following:
\begin{itemize}
    \item We identify that boundary supporting samples form an important missing piece of the current state-of-the-art data-free compression. 
    \item We propose using superposed latent embeddings, which enables a generator to synthesize boundary supporting samples of the full-precision model.
    \item We propose disentanglement mapping and extracted embedding initialization that help train a better set of embeddings for the generator.
    \item We conduct an extensive set of experiments, showing that the proposed scheme outperforms the existing methods.
\end{itemize}


\JL{maybe itemized contributions}





\section{Related Work}
\subsection{Data-free Compression}
Early work on data-free compression has been led by knowledge distillation~\cite{hinton2015distilling}, 
which usually involves pseudo-data created from teacher network statistics~\cite{lopes2017data, nayak2019zero}.
Lopes et al.~\cite{lopes2017data} suggested generating pseudo-data from metadata collected from the teacher in the form of activation records. 
Nayak et al.~\cite{nayak2019zero} proposed a similar scheme but with a zero-shot approach by modeling the output space of the teacher model as a Dirichlet distribution, which is taken from model weights.
More recent studies have employed generator architectures similar to GAN~\cite{gan} to generate synthetic samples replacing the original data~\cite{chen2019data, micaelli2019zero, fang2019data, yin2020dreaming}. \KH{deepdream is not using generator}
In the absence of the original data for training, DAFL~\cite{chen2019data} used the teacher model to replace the discriminator by encouraging the outputs to be close to a one-hot distribution and by maximizing the activation counts.
KegNet~\cite{kegnet} adopted a similar idea and used a low-rank decomposition to aid the compression.
Adversarial belief matching~\cite{micaelli2019zero} and data-free adversarial distillation~\cite{fang2019data} methods suggested adversarially training the generator, such that the generated samples become harder to train.
One other variant is to modify samples directly using logit maximization as in DeepInversion~\cite{yin2020dreaming}.
While this approach can generate images that appear natural to a human, it has the drawback of skyrocketing computational costs because each image must be modified using backpropagation.



Data-free quantization is similar to data-free knowledge distillation but is a more complex problem because quantization errors must be recovered.
The quantized model has the same architecture as the full-precision model; thus, the early methods of post-training quantization were focused on how to convert the full-precision weights into quantized weights by limiting the range of activations~\cite{aciq, outlier}, correcting biases~\cite{aciq, dfq}, and equalizing the weights~\cite{dfq, outlier}.
ZeroQ~\cite{zeroq} pioneered using synthetic data for data-free quantization employing statistics stored in batch-normalization layers.
GDFQ~\cite{gdfq} added a generator close to the ACGAN~\cite{acgan} to generate better synthetic samples, 
\rev{and AutoReCon~\cite{autorecon} suggests a better performing generator found by neural architecture search.
In addition, DSG~\cite{zhang2021diversifying} further suggests relaxing the batch-normalization statistics alignment to generate more diverse samples.}
ZAQ~\cite{zaq} adopted adversarial training of the generator on the quantization problem and introduced intermediate feature matching between the full-precision and quantized model.
However, none of these considered aiming to synthesize boundary supporting samples of the full-precision model. 
Adversarial training of generators have a similar purpose, but it is fundamentally different from generating boundary supporting samples. 
In addition, adversarial training can be sensitive to hyperparameters and risks generating samples outside of the original distributions~\cite{outofmanifold}.
To the extent of our knowledge, this work is the first to propose using superposed latent embeddings to generate boundary supporting samples explicitly for data-free quantizations.






























































\subsection{Boundary Supporting Samples}
In the context of knowledge distillation, boundary supporting samples~\cite{heo2019knowledge} are defined as samples that lie near the decision boundary of the teacher models.
As these samples contain classification information about the teacher models, they can help the student model correctly mimic the teacher’s behavior.
Heo et al.~\cite{heo2019knowledge} applied an adversarial attack~\cite{advatk} to generate boundary supporting samples and successfully demonstrated that they improve knowledge distillation performance.
In AMKD~\cite{dong2020adversarial}, triplet loss was used to aid the student in drawing a better boundary.
Later, DeepDig~\cite{karimi2019characterizing} devised a refined method for generating boundary supporting samples and analyzed their characteristics by defining new metrics.
Although boundary supporting samples have been successful for many problems, such as domain adaptation~\cite{saito2020universal} and open-set recognition~\cite{xu2020open}, they have not yet been considered for data-free compression.















\begin{table}[]    
\begin{tabular}{l|c|c}
    \toprule
       Data & Accuracy & Data-free\\
       \midrule
       Fine-tuned with Original data & 68.28\%  & \xmark\\
       Synthetic samples & 63.39\% & \cmark\\
       Synthetic + unconfusing real samples  &  63.91\% (+0.52)& \xmark\\
       Synthetic + confusing real samples & 65.75\% (+2.36) & \xmark\\  
       \midrule
       \aname (This work) & 65.10\% (+1.71) & \cmark\\
    \bottomrule
    \end{tabular}
    \caption{Motivational Experiment\vspace{-3mm}}
    \label{tab:moti}
\end{table}






\section{Motivational Experiment}
To explain the discrepancy between the accuracy of the model fine-tuned with the original training data and the data-free quantized models, we hypothesized that the synthetic samples from generative data-free quantization methods lack samples near the decision boundary.
To validate this hypothesis, we designed an experiment using the CIFAR-100~\cite{cifar} dataset with the ResNet-20 network~\cite{resnet}. 


First, we forwarded images in the dataset into the pre-trained full-precision ResNet-20. 
Among these, we selected 1500 samples (3\% of the dataset, 15 samples per class) from samples where the highest confidence value was lower than 0.25, forming a group of ‘confusing’ samples.
Then, we combined the confusing samples with synthetic samples generated from a previous study~\cite{gdfq} to fine-tune the quantized model with 4-bit weights and 4-bit activations. 
We also selected as a control group an equal number of random samples from the images classified as unconfusing and fine-tuned the quantized model using the same method.


The results are presented in \tablename~\ref{tab:moti}.
The quantized model with synthetic + unconfusing real samples exhibited only 0.52\%p increase in the accuracy from the baseline.
In contrast, adding confusing samples provided 2.36\%p improvement, filling almost half the gap towards the quantized model fine-tuned with the original data. 
These results indirectly validate that the synthetic samples suffer from a lack of confusing samples (i.e., boundary supporting samples). 
We aim to address the issue in this paper.
As we indicate in Section~\ref{sec:exp}, \aname achieves a 1.71\%p \KH{CHECK!} performance gain for the same model in a data-free  setting, close to that of the addition of confusing synthetic samples.


\JL{somewhere: simple yet effective}




\section{Generating Boundary Supporting Samples with Superposed Latent Embeddings}
\label{sec:method}

\subsection{Baseline Generative Data-free Quantization}
\JL{maybe we will remove zaq from here}
Recent generative data-free quantization schemes~\cite{gdfq,zaq} employ a GAN-like generator to create synthetic samples. 
In the absence of the original training samples, the generator  attempts to generate synthetic samples so that the quantized model  can mimic the behavior of the full-precision model .
For example, in GDFQ~\cite{gdfq}, the loss function  is  

where the first term  guides the generator to output clearly classifiable samples, and the second term  aligns the batch-normalization statistics of the synthetic samples with those of the batch-normalization layers in the full-precision model.
In another previous work ZAQ~\cite{zaq},  

where the first term  separates the prediction outputs of  and , and the second term  separates the feature maps produced by  and .
These two losses let the generator be adversarially trained and allow it to determine samples where  cannot mimic  adequately.
Lastly, the third term  maximizes the activation map values of  so that the created samples do not drift too far away from the original dataset. 

The quantized model  is usually jointly trained with , such that 

for GDFQ, and 

for ZAQ, respectively, where  from Eq.~\ref{eq:gdfq_Q} is the usual KD loss with Kullback-Leibler divergence.

\begin{figure}
\centering
\subcaptionbox{Synthetic Images generated \\with Gaussian noise.\label{fig:distr-small}}
	    {\includegraphics[height=0.27\textwidth]{figs/diagram1.pdf}}
\subcaptionbox{Synthetic boundary supporting samples with superposed latent embeddings (proposed).\label{fig:distr-qimera}}
	    {\includegraphics[height=0.27\textwidth]{figs/diagram2.pdf}}
\caption{Diagram of synthetic samples in the feature space of the full-precision network. The black curves represent the decision boundary of the full-precision model, which are considered ideal for the quantized model (orange dotted curves) to mimic. When synthetic images are generated with per-class embeddings and noises as in (a), their features do not support the decision boundary, whereas the proposed approach in (b) generates the boundary supporting samples, helping the quantized model to set the near-ideal decision boundary. }
	\label{fig:distr}
\end{figure}


While these two methods exhibit great performance, they both model the distribution of per-class embeddings in the latent input space as a Gaussian distribution, 
and generate diverse samples using random Gaussian noise inputs. 
However, based on the Gaussian distribution, it is difficult to correctly reflect the boundary between the two different classes, especially when all samples have one-hot labels. 
\figurename~\ref{fig:distr} visualizes such problems in a simplified two-dimensional space. 
With samples generated from gaussian noise (\figurename~\ref{fig:distr-small}), the two per-class distributions are far away, and a void exists between the two classes (see \figurename~\ref{fig:pca_gdfq} for plots from experimental data).  
This can cause the decision boundaries to be formed at nonideal regions.
One alternative is to use noise with higher variance, but the samples would overlap on a large region, resulting in too many samples with incorrect labels.



Furthermore, in the above approaches, the loss terms of Eqs.~\ref{eq:gdfq} and \ref{eq:zaq} such as  and  encourage the class distributions to be separated from each other. 
While this is beneficial for generating clean, well-classified samples, it also prevents generating samples near the decision boundary, which is necessary for training a good model~\cite{heo2019knowledge, dong2020adversarial}. 
Therefore, in this paper, we focus on methods to generate synthetic samples near the boundary from the full-precision model  (\figurename~\ref{fig:distr-qimera}).



\subsection{Superposed Latent Embeddings}
The overview of \aname is presented in \figurename~\ref{fig:qimera}.
Often, generators use learned embeddings to create samples of multiple classes~\cite{acgan, stylegan}.
Given an input one-hot label  representing one of  classes and a random noise vector , a generator  uses an embedding layer  to create a synthetic sample :

To create boundary supporting samples, we superpose the class embeddings so that the generated samples 
have features lying near the decision boundaries of .
With two embeddings superposed, new synthetic sample  becomes 

To avoid too many confusing samples from complicating the feature space, we also apply soft labels in the same manner as a regularizer:

Generalizing to  embeddings, Eqs. \ref{eq:twox} and \ref{eq:twoy} become

where  has  embeddings from  as the column vectors 
(i.e., ), 
and  is a superposer function. Applying this to existing methods is straightforward and incurs only a small amount of computational overhead.
Similar to knowledge distillation with boundary supporting samples~\cite{heo2019knowledge,dong2020adversarial}, the superposed embeddings are supposed to help transfer the decision boundary of the full precision (teacher) model to the quantized (student) model. 




\begin{figure}
\centering
	\includegraphics[width=0.95\columnwidth]{figs/Qimera}
\caption{An overview of the proposed method.
	Proposed components are denoted as colored shapes.} 
	\label{fig:qimera}
\end{figure}


Although the superposed embedding scheme alone produces a substantial amount of performance gains, the generator embedding space is often not flat enough~\cite{oddity}; therefore, linearly interpolating them can result in unnatural samples~\cite{laine}.
To mitigate this, the embedding space used in \aname should possess two characteristics. 
First, the embedding space should be as flat as possible so that the samples generated from Eq. \ref{eq:manyx} reflect the intermediate points in the feature space.
Second, the individual embeddings should still be sufficiently distinct from each other, correctly representing the distance between each class distribution.
In the remaining two subsections, we describe our strategies for enforcing the embeddings to contain the above characteristics.


\subsection{Disentanglement Mapping}
To perform superposing in a flatter manifold, we added a learnable mapping function  before the embeddings are superposed, where  is the embedding dimension and  is the dimension of the target space.
Thus,  from Eq.~\ref{eq:manyx} becomes , where .
Although we do not add any specific loss that guides the output space of  to be flat, training to match the output of the full-precision model (i.e., ) using the superposed  encourages  to map the input to a flatter space.
In practice, we modeled  as a single-layer perceptron, which we call the disentanglement mapping layer.
The experimental results from Section~\ref{sec:exp} demonstrate that the disentanglement mapping provides a considerable amount of performance gain. 



\subsection{Extracted Embedding Initialization}
For the embeddings to be flat, we want the distributions of the embeddings fused with noise to be similar to the feature space.
For this purpose, we utilize the fully connected layer of the full precision model .
Given , the output features of the full-precision model before the last fully connected layer, we minimize 
    ,
where  is the number of classes,  is the input of the generator, and  is some distance metric.
We do not have knowledge of the distribution of ; thus, we modeled it as a Gaussian distribution, such that .
Therefore, solving it against Eq.~\ref{eq:onex} simply yields .
In practice, we use the corresponding column from the weight of the last fully connected layer of the full-precision model because its weights represent the centroids of the activations. 
If the weights of the last fully connected layer , we set .
Our experiments reveal that extracting the weights from the full-precision model and freezing them already works well \rev{(see Appendix B.2)}. 
However, using them as initializations and jointly training them empirically works better. 
We believe this is because fully connected layers have bias parameters in addition to weight parameters.
Because we do not extract these biases into the embeddings, a slight tuning is needed by training them.  
This outcome aligns with the findings from the class prototype scheme used in self-supervised learning approaches~\cite{saito2019semi, saito2020universal}.



\section{Experimental Results}
\label{sec:exp}

\subsection{Experiment Implementation}
Our method is evaluated on CIFAR-10, CIFAR-100~\cite{cifar} and ImageNet (ILSVRC2012~\cite{imagenet}) datasets, which are well-known datasets for evaluating the performance of a model on the image classification task. 
CIFAR10/100 datasets consist of 50k training sets and 10k evaluation sets with 10 classes and 100 classes, respectively, and is used for small-scale experiments in our evaluation.
ImageNet dataset is used for large-scale experiments. 
It has 1.2 million training sets and 50k evaluation sets. 
To keep the data-free environment, only the evaluation sets were used for test purposes in all experiments. 


For the experiments, we chose ResNet-20~\cite{resnet} for CIFAR-10/100, and ResNet-18, ResNet-50, and MobileNetV2~\cite{mobilenetv2} for ImageNet. 
We implemented all our techniques using PyTorch~\cite{pytorch} and ran the experiments using RTX3090 GPUs.
All the model implementations and pre-trained weights before quantization are from pytorchcv library~\cite{pytorchcv}. 
For quantization, we quantized all the layers and activations using -bit linear quantization, described by \cite{jacob2018quantization}, as below:

\KH{Change notation to functional form}
where  is the full-precision value,  is the quantized value,  is calculated as . 
 and  are per-channel minimum and maximum value of . \KH{function notation}
 

To generate synthetic samples, we built a generator using the architecture of ACGAN~\cite{acgan} and added a disentanglement mapping layer after class embeddings followed by a superposing layer. 
Among all batches, the superposing layer chooses between superposed embeddings and regular embeddings in  ratio. 
The dimension of latent embedding and random noise was set to be the same with the channel of the last fully connected layer of the target network. 
For CIFAR, the intermediate embedding dimension after the disentanglement mapping layer was set as 64. 
The generator was trained using Adam~\cite{adam} with a learning rate of 0.001. 
For ImageNet, the intermediate embedding dimension was set to be 100. 
To maintain label information among all layers of the generator, we apply conditional batch normalization~\cite{CBN} rather than regular batch normalization layer, following SN-GAN~\cite{sngan}. 
The optimizer and learning rate were the same as that of CIFAR. 


\JL{need a space before citation brackets! usually a unbreakable space (\textasciitilde) is preferred.}
To fine-tune the quantized model , we used SGD with Nesterov~\cite{nesterov} as an optimizer for  with a learning rate of 0.0001 while momentum and weight decay terms as 0.9 and 0.0001 respectively. 
The generator  and the quantized model  were jointly trained with 200 iterations for 400 epochs while decaying the learning rate by 0.1 per every 100 epochs. The batch size was 64 and 16 for CIFAR and ImageNet respectively. 
While \aname can be applied almost all generator-based approaches, we chose to adopt baseline loss functions for training from GDFQ~\cite{gdfq}, because it was stable and showed better results for large scale experiments.
Thus, loss functions  and  are equal to Eq.~\ref{eq:gdfq} and Eq.~\ref{eq:gdfq_Q} with  and , following the baseline.
\KH{Isn't it too similar with GDFQ?}\JL{I added `following the baseline'}
\JL{hyparm p and K?}

\newcommand{\pcawidth}{0.3\textwidth}
\subsection{Visualizations of \aname-generated Samples on Feature Space} 
\begin{figure*}[t]
    \centering
    \subcaptionbox{Original data\label{fig:pca_original}}{
        \centering
        \includegraphics[width=\pcawidth,trim={5mm 5mm 5mm 5mm},clip]{figs/pca/pca_original.png} }
    \subcaptionbox{GDFQ\label{fig:pca_gdfq}}{
        \centering
        \includegraphics[width=\pcawidth,trim={5mm 5mm 5mm 5mm},clip]{figs/pca/pca_gdfq.png} }    
    \subcaptionbox{\aname\label{fig:eyeriss}}{
        \centering
        \includegraphics[width=\pcawidth,trim={5mm 5mm 5mm 5mm},clip]{figs/pca/pca_qimera.png} }
    \caption{PCA plots of the features before the last layer of the full-precision model. In the plots of Qimera-generated samples (c), the black dots represent synthetic boundary supporting samples generated with the superposed latent embeddings, which fill the space between the clusters. GDFQ-generated samples (b) form clusters that are smaller than that of the original data (a) and lack samples in the mid-ground. Best viewed in color.}
    \label{fig:hardware}
\end{figure*}

To ensure that the superposed latent embedding creates boundary supporting samples, we conducted an experiment to compare generated synthetic samples on feature space visually. 
The experimental results are based on the generators trained with ResNet-20 for CIFAR-10 dataset.
The features were extracted from the intermediate activation before the last fully connected layer of the full-precision model. 
For the \aname-based generator, we set K=2 during the sample generations for clarity. 

After extracting the features, we projected the features into a two-dimensional space using Principal Component Analysis (PCA)~\cite{pca}.
The results are presented in \figurename~\ref{fig:hardware}. 
Compared with the original training set data (\figurename~\ref{fig:pca_original}), the samples from the GDFQ-based generator (\figurename~\ref{fig:pca_gdfq}) show a lack of boundary supporting samples and the class distributions are confined to small regions around the centroid of each class. 
A generator trained with \aname, however, exhibits different characteristics (\figurename~\ref{fig:eyeriss}). 
Samples that are generated from superposed latent embeddings are displayed as black dots. 
The black dots are mostly located on the sparse regions between the class clusters. This experimental result shows that our method, \aname, successfully generates samples near the decision boundaries. In other words, superposed latent embeddings are not only superposed on embedding space but also in the feature space, serving as synthetic boundary supporting samples.

\subsection{Quantization Results Comparison}



Table \ref{tab:exp} displays the classification accuracy on various datasets, target models, and bit-width settings. 
Note that wa means -bit quantization for weights and -bit quantization for activations. 
As baselines, we selected ZeroQ~\cite{zeroq}, ZAQ~\cite{zaq}, and GDFQ~\cite{gdfq} as the important previous works on generative data-free quantization.
In addition, we implemented Mixup~\cite{mixup} and Cutmix~\cite{cutmix} on top of GDFQ, 
which are data augmentation schemes that mix input images.
To implement these schemes, we created synthetic images from GDFQ, and applied the augmentations to build training images and labels.
All baseline results are from official code of the authors, where a small amount of modifications have been made on the GDFQ baseline for applying Mixup and Cutmix.  
We report top-1 accuracy for each experiment.
The numbers inside the parentheses of \aname results are improvements over the highest baseline.

\JL{recheck below later for updated numbers}
The results demonstrate that \aname outperforms the baselines at almost all settings. 
Its performance improvement is especially large for low-bitwidth (4w4a) cases. For 5w5a setting, the gain is still significant, and its performance reaches near that of the full-precision model,
which represents the upper bound.  
In addition, \aname is not limited to small datasets having a low spatial dimension. 
The ImageNet results prove that \aname performs beyond other baselines with considerable gaps, on a large-scale dataset with many classes.  
Interestingly, the result of GDFQ+Mixup and GDFQ+Cutmix did not produce much noticeable improvement except for GDFQ+Mixup in 4w4a ResNet -18 and -50. 
This implies that a mixture of generated synthetic samples of different classes in the sample space is not sufficient to represent boundary supporting samples.
In summary, \aname achieves superior accuracy on various environments regardless of dataset or model scale.  


\begin{table}[]
    \centering
    \footnotesize
    
    \caption{Comparison on data-free quantization schemes \JL{ZAQ and then GDFQ}\KH{Also MIXUP and then CUTMIX}\KH{CHECK THE RESULT!!!}\JL{mixup first, and then cutmix}\JL{lets not forget error bars}\vspace{-2mm}}
    \label{tab:exp}
    
\resizebox{\textwidth}{!}
    {
    \begin{tabular}{x{1.5cm}|c|c|lll|cc|c}
    \toprule
    Dataset & \makecell{Model\\%p improvement)} \\
      \midrule
      \multirow{2}{*}{Cifar-10} &  ResNet-20 & 4w4a  &  79.30 & \textbf{92.13} & 90.25 & 89.58 & 88.69 (-3.44)  & 91.26 0.49 (-0.87) \\
                                & (93.89) & 5w5a & 91.34& 93.36 & 93.38 & 92.75 & 92.79 (-0.59) &  \textbf{93.46 0.03 (+0.08)} \\
      \midrule
      \multirow{2}{*}{Cifar-100} & ResNet-20  & 4w4a & 47.45 & 60.42 &  63.39 & 62.74 & 62.99 (-0.40) & \textbf{65.10 0.33 (+1.71)} \\
                                 &(70.33) & 5w5a & 65.61 & 68.70 & 66.12 & 67.51 & 67.78 (-0.92)  & \textbf{69.02 0.22 (+0.32)} \\
      

      \midrule
       \multirow{6}{*}{ImageNet} &  ResNet-18   & 4w4a  &  22.58 & 52.64 &60.60 & 58.90 & 61.72 (+1.12) & \textbf{63.84 0.30 (+3.24)} \\
                             &  (71.47) & 5w5a & 59.26 & 64.54 & 68.40 & 68.05 & 68.67 (+0.27) &  \textbf{69.29 0.16 (+0.89)} \\
                                 & ResNet-50 & 4w4a & {\color{white}0}8.38 & 53.02 & 52.12 & 51.80 & 59.25 (+6.24) &  \textbf{66.25 0.90 (+13.23)} \\
                                  &  (77.73) & 5w5a & 48.12 & 73.38 & 71.89 & 70.99 & 71.57 (-1.81) & \textbf{75.32 0.09 (+1.94)} \\
    
                            & MobileNetV2 &  4w4a & 10.96 & {\color{white}0}0.10 &59.43 & 57.23 & 59.99 (+0.56)&  \textbf{61.62 0.39 (+2.19)} \\
                                  & (73.03)  & 5w5a & 59.88 & 62.35 & 68.11 & 67.61 & 68.83 (+0.72) & \textbf{70.45 0.07 (+2.34)} \\

               \bottomrule
          \multicolumn{9}{r}{\small Highest among the baselines\ \  Did not converge} \\
    \end{tabular}}
     \vspace{-2mm}
\end{table}

 
\rev{
\subsection{\aname on Top of Various Algorithms}

\begin{table}[]
\rev{
    \centering
    \footnotesize
    
    \caption{Performance of \aname implemented on top of GDFQ, ZAQ and AutoReCon}
    \label{tab:ontopof}
    
\resizebox{\textwidth}{!}
    {
    \begin{tabular}{ccc|cc|cc|cc}
    \toprule
    Dataset & \makecell{Model\
    \loss(G) = \loss_{o}^{P,Q}(G) + \beta\loss_{f}^{P,Q}(G) + \gamma\loss_a^P(G) +\rho\loss_{CE}^P(G),
    \label{eq:zaqimera}

    \loss_(Q) = -\loss_{o}^{P,Q}(G) - \beta\loss_{f}^{P,Q}(G)+\rho\loss_{CE}^Q(G),
    \label{eq:zaq_Q}
FP32 Acc.)} & Bits & ZeroQ & ZAQ & GDFQ & Qimera & \makecell{Extracted Init \\+ Freeze} \\
      \midrule
      \multirow{2}{*}{Cifar-10} &  ResNet-20 & 4w4a  &  79.30 & 92.13 & 90.25 & 91.26 &90.37  \\
                                & (93.89) & 5w5a & 91.34& 93.36 & 93.38 & 93.46&93.25  \\
      \midrule
      \multirow{2}{*}{Cifar-100} & ResNet-20  & 4w4a & 47.45 & 60.42 &  63.39 & 65.10 & 63.83\\
                                 &(70.33) & 5w5a & 65.61 & 68.70 & 66.12 & 69.02& 68.76 \\
      

      \midrule
       \multirow{6}{*}{ImageNet} &  ResNet-18   & 4w4a  &  22.58 & 52.64 &60.60 & 63.84&63.67 \\
                             &  (71.47) & 5w5a & 59.26 & 64.54 & 68.40 & 69.29&69.23  \\
                                 & ResNet-50 & 4w4a & {\color{white}0}8.38 & 53.02 & 52.12 & 66.25&63.20  \\
                                  &  (77.73) & 5w5a & 48.12 & 73.38 & 71.89 & 75.32 & 74.84   \\
    
                            & MobileNetV2 &  4w4a & 10.96 & {\color{white}0}0.10 &59.43 & 61.62&60.46  \\
                                  & (73.03)  & 5w5a & 59.88 & 62.35 & 68.11 & 70.45 & 68.82   \\
               \bottomrule
          \multicolumn{8}{r}{\small Highest among the baselines\ \  Did not converge} \\
    \end{tabular}}
\end{table}




To show that the weight from the last fully connected layer of the full-precision model is a good candidate for the initial embeddings, we have performed an experiment where the embeddings are frozen right after initialization.
The results are presented in \tablename~\ref{tab:freeze}. 
\aname with frozen embeddings are not better than the primary \aname method with trained embeddings.
However, compared to the two baselines (ZAQ and GDFQ), they provides a comparable accuracy on Cifar-10 and better accuracies on Cifar-100 and ImageNet. 
Furthermore, the accuracy on Cifar-10 dataset is close to the upper bound for all techniques under comparison, and thus the differences are minimal.

\rev{
\subsection{Sensitivity Study on Number of DM layers}
\begin{table}[h]
\rev{
    \centering


    \caption{Sensitivity Study on Number of DM Layers}
    \label{tab:dmlayer}
    


    \begin{tabular}{ccc}
    \toprule
    \multirow{2}{*}{Num. DM Layers} & \multicolumn{2}{c}{Accuracy} \\
    \cmidrule(lr){2-3}
    & Cifar-10 & Cifar-100 \\
      \midrule
    (w/o DM) &  90.81 & 64.89 \\
     1 & 91.26 & \textbf{65.10} \\
     2 & 91.18 & 64.90 \\
     4 & 91.49 & 64.96 \\
     8 & \textbf{91.63} & 64.11 \\
      \bottomrule
      
    \end{tabular} 
    }
\end{table}
To have a deeper look into the DM layers, we have conducted a sensitivity study on the number of DM layers in \tablename~\ref{tab:dmlayer}.
In the table, all results are from 4w4a setting with =0.4, =2 for Cifar-10 and =10 for Cifar-100. As displayed, we found that there are sometimes small improvements from using more DM layers above one, but a severe drop in performance has been observed for using too many layers (Cifar-100, 8 layers).


}
\subsection{More Sensitivity Study on Hyperparameters}
\begin{table}[h]
\centering
\caption{Further Sensitivity analysis
     \vspace{-2mm}}
    \label{tab:more_sense}    
{
    \begin{tabular}{crccccccc}
    \toprule
    \multirow{2}{*}{Dataset} &\multirow{2}{*}{}& \multicolumn{6}{c}{} \\
    \cmidrule(lr){3-8}
     &  & 0.10 & 0.25 & 0.40 & 0.55 & 0.70 & 0.85 \\ \midrule
\multirow{4}{*}{\makecell{Cifar-100\ResNet-50)}}&100&58.74	&60.64&	61.43&	61.47&	63.87	&65.73\\
&250&61.30&	61.28	&62.16	&64.03&	64.50&	65.23\\
&500&58.96&	60.11&	58.69&	63.05&	\textbf{66.25}&	66.19\\
&1000&58.65	&59.62&	61.20&	58.86&	65.12&	64.24\\

                                     
    \bottomrule

    \end{tabular}
    }

\end{table}


In addition to our choice of hyperparameters presented in the main body, we have performed a further extensive sensitivity study on those parameters, which is displayed in \tablename~\ref{tab:more_sense}.
All experiments are against 4w4a configuration, equal to the \tablename~3 (Section~5.4) in the main body.
Regardless of the choice in  and , the results are all better than the two baselines ZAQ and GDFQ.
Furthermore, while they all provide a meaningfully good performance, the results show a clear trend: lower  for Cifar-10/100 and higher  for ImageNet as sweet spots. 
This result supports the use of \aname in that these parameters are easily tunable, not something that must be exhaustively seAutoReConhed for optimal values.

\rev{
\subsection{Comparison with DSG}

\begin{table}[h]
\rev{
    \centering    
    \caption{Comparison with DSG}
    \label{tab:dsg}
    
{
    \begin{tabular}{ccccc}
    \toprule
    Dataset & Model & Bits & DSG~\cite{zhang2021diversifying} & Qimera (\%p improvement)\\ 
      \midrule

       \multirow{5}{*}{ImageNet} &   ResNet-18 & 4w4a &	34.53	& 63.84 (+29.31) \\
                                 &   ResNet-50 & 6w6a &	76.07 &	77.18 (+1.11) \\
                                 &   InceptionV3 & 4w4a &	34.89	& 73.31 (+38.42) \\
                                 &   SqueezeNext & 6w6a &	60.50 &	65.97 (+5.47) \\
                                 &   ShuffleNet & 6w6a &	44.88   &	56.16 (+11.28) \\
      
               \bottomrule
          \end{tabular}}
          }
\end{table}

\aname is conceptually similar to DSG~\cite{zhang2021diversifying} which tries to diversify the sample generation by relaxing the batch-norm stat alignments.
However, \aname is different from DSG because we explicitly try to generate boundary supporting samples, instead of relying on diversification.
This would led to better performance as demonstrated in the motivational experiment of Section 3.

\tablename~\ref{tab:dsg} shows the comparison of \aname with DSG.
We use the reported numbers for DSG, and perform a new set of experiments for \aname to match the settings.
We use the lowest-bit settings for each network evaluated in DSG.
As displayed in the table, \aname outperforms DSG in all settings, especially for 4w4a cases.
}





\begin{comment}

\subsection{Applying \aname to ZAQ and AutoReCon}


In this paper, we have applied GDFQ~\cite{gdfq} as a baseline.
However, our design does not particularly depend on a certain method, and can be adopted by many schemes.
Although ZAQ is not a good candidate because it does not utilize conditional generators, we have conducted an experiment of applying \aname to ZAQ~\cite{zaq} and the results are displayed in \tablename~\ref{tab:zaqimera}.
Because original ZAQ does not use per-class sample generation, we have extended the generator with an embedding layer, and added a cross-entropy loss into the loss function as the following:




where  was set to 0.1, and ,  are calculated based on Eqs. 8 and 9 in the main body. \KH{refering eqs on main paper?} 

For 4w4a cases, \aname adds 1.15\%p and 5.14\%p accuracy enhancement from ZAQ for Cifar-10 and Cifar-100. respectively, similar to what \aname gained for GDFQ.
Unfortunately, it did not converge with ImageNet, and thus we did not consider this implementation as the primary version of \aname, even though this version of \aname achieves the best performance in 4w4a settings with Cifar-10/100.



\subsection{Comparison with AutoReCon}

\begin{table}[h]
    \centering    
    \caption{Comparison with AutoReCon}
    \label{tab:autorecon}
    
{
    \begin{tabular}{ccccc}
    \toprule
    Dataset & \makecell{Model\FP32 Acc.)} & Bits & AutoReCon~\cite{autorecon} & Qimera (\%p improvement)\\ 
      \midrule

       \multirow{4}{*}{ImageNet} &   ResNet-50 & 4w4a &64.37 & 66.25 (1.88) \\
            & (77.73) &5w5a &       74.13     &   75.32 (1.19)\\
            \cmidrule(lr){2-5}
            
            & MobileNetV2 &  4w4a & 60.13 & 61.62 (1.49)  \\
              & (73.03) &5w5a & 68.40 & 70.45 (2.05)\\
               \bottomrule
          \end{tabular}}
\end{table}


Close to the Neurips 2021 deadline, a new paper named AutoReCon~\cite{autorecon} has appeared on arXiv that tackles the same data-free quantization problem.
AutoReCon improves the accuracy of the data-free quantization by applying neural AutoReConhitecture seAutoReConh to construct a better generator AutoReConhitecture.
In \tablename~\ref{tab:autorecon}, we provide a comparison based on the reported values from their paper. 
In all results that share the same setting, \aname outperforms AutoReCon by a significant margin.
Moreover, we believe schemes used in AutoReCon is orthogonal to ours because the generator discovered by AutoReCon can be used with \aname to provide extra performance gain.

\end{comment}




\section{Class-Pairwise Visualization}
To look closely onto the visualization of the samples from Section 5.3 (\figurename~3) in the main body, we have plotted them in a pair-wise manner.
Even though 10 classes in total gives 45 possible pairs, we chose nine symbolically adjacent pairs in the figure.
Although being symbolically adjacent does not have much meaning, we believe having nine pairs is enough for our purpose rather than showing all 45 possible pairs.
The colors match that of the \figurename~3, where the \textbf{{\color{emerald}lightgreen}} dots represent the synthetic boundary supporting samples. 
Also, we have plotted the path between the centroids of the two clusters in \textbf{black}, by varying  (the ratio of superposition) from 0 to 1 by 0.01 without any noise.
Each 10th percentile is denoted as larger black dots.
The results show that the samples and the path lie relatively in the middle of the two clusters.
Please note that we have performed PCA plot for each pair to best show the distribution, so the position and orientation of the clusters do not exactly match those from \figurename~3.








\begin{comment}
\subsection{Feature Space Distance}
\tablename~\ref{tab:distance} shows the distance, measured in a piecewise linear manner, with 100 points between the curves. 
The average column shows the average of all 45 pairs, not just the example 9 pairs we display as the examples.
It is clearly shown that the distance between the clusters in the feature space decreases with disentanglement mapping by @\%, although both types of curves are moderately flat.
\end{comment}


\section{More Generated Images}
Lastly, \figurename~\ref{fig:moresynth} shows more samples generated from \aname. 
\figurename~\ref{fig:mix_cifar} displays the synthetic boundary supporting samples generated from Cifar-10 dataset, with  and .
Each row and column represents a class from Cifar-10.
For example, the image at row 0 (airplane) and column 2 (bird) represents a sample generated from superposed embeddings of airplane and bird.
Although still not very human-recognizable, we find that each sample in \figurename~\ref{fig:mix_cifar} has some features adopted from each of the source classes in \figurename~4d.

\figurename~\ref{fig:synth_imagenet} shows the sample images created from ImageNet.
Because there are too many classes within ImageNet (1000), we chose 10 classes from them, which are 
\{0: `tench, Tinca tinca',
100: `black swan, Cygnus atratus',
200: `Tibetan terrier, chrysanthemum dog',
300: `tiger beetle',
400: `academic gown, academic robe, judge's robe",
500: `cliff dwelling',
600: `hook, claw',
700: `paper towel',
800: `slot, one-armed bandit',
900: `water tower'\}, 
and the original samples from those classes are shown in \figurename~\ref{fig:original_imagenet}.
As in Cifar-10, the generated samples are far from human-recognizable, but each row is clearly distinguishable from the others.
In addition, \figurename~\ref{fig:mix_imagenet} contains the synthetic boundary supporting samples from ImageNet, following the same rules from \figurename~\ref{fig:mix_cifar}.
Again, we see that each position in the sample matrix adopts features from the rows of the corresponding class pair in \figurename~\ref{fig:synth_imagenet}.















\begin{figure}
    \centering
    \subcaptionbox{Class 0-1.}{
        \centering
        \includegraphics[width=\pcawidth,trim={25mm 25mm 5mm 5mm},clip]{figs/featurespace_path/feature_path_rearrange_0_1.png} } 
    \subcaptionbox{Class 1-2.}{
        \centering
        \includegraphics[width=\pcawidth,trim={25mm 25mm 5mm 5mm},clip]{figs/featurespace_path/feature_path_rearrange_1_2.png} } 
    \subcaptionbox{Class 2-3.}{
        \centering
        \includegraphics[width=\pcawidth,trim={25mm 25mm 5mm 5mm},clip]{figs/featurespace_path/feature_path_rearrange_2_3.png} }
    \centering
    \subcaptionbox{Class 3-4.}{
        \centering
        \includegraphics[width=\pcawidth,trim={25mm 25mm 5mm 5mm},clip]{figs/featurespace_path/feature_path_rearrange_3_4.png} } 
    \subcaptionbox{Class 4-5.}{
        \centering
        \includegraphics[width=\pcawidth,trim={25mm 25mm 5mm 5mm},clip]{figs/featurespace_path/feature_path_rearrange_4_5.png} } 
    \subcaptionbox{Class 5-6.}{
        \centering
        \includegraphics[width=\pcawidth,trim={25mm 25mm 5mm 5mm},clip]{figs/featurespace_path/feature_path_rearrange_5_6.png} }
    \subcaptionbox{Class 6-7.}{
        \centering
        \includegraphics[width=\pcawidth,trim={25mm 25mm 5mm 5mm},clip]{figs/featurespace_path/feature_path_rearrange_6_7.png} }
    \subcaptionbox{Class 7-8.}{
        \centering
        \includegraphics[width=\pcawidth,trim={25mm 25mm 5mm 5mm},clip]{figs/featurespace_path/feature_path_rearrange_7_8.png} }
    \subcaptionbox{Class 8-9.}{
        \centering
        \includegraphics[width=\pcawidth,trim={25mm 25mm 5mm 5mm},clip]{figs/featurespace_path/feature_path_rearrange_8_9.png} }
    \caption{Visualization of the generated samples in the feature space. The lightgreen cloud represents the synthetic boundary supporting samples. The black dots represents the path between the two embeddings without any noise, where every 10th percentile is denoted as larger dots. The colors match that of the \figurename~3 of the main body, but the PCA dimension has been adjusted to best show each chosen class pair.}
\end{figure}

\begin{figure}
    \centering
    
    \subcaptionbox{Synthetic boundary supporting samples from Cifar-10.\label{fig:mix_cifar}\vspace{2mm}}{
        \centering\tikz[baseline=0.6ex, white]\draw (0,0) rectangle (0.3\textwidth,0);
        \includegraphics[width=0.27\textwidth]{figs/qimera_mix.png} \tikz[baseline=0.6ex, white]\draw (0,0) rectangle (0.3\textwidth,0);
    }  
    
    \subcaptionbox{Original ImageNet samples from the selected 10 classes.\label{fig:original_imagenet}\vspace{2mm}}{
        \centering
        \includegraphics[width=0.45\textwidth]{figs/imagenet_generated_image/imagenet_original.png} 
    }     
    \subcaptionbox{Synthetic samples from ImageNet.\label{fig:synth_imagenet}\vspace{2mm}}{
        \centering
        \includegraphics[width=0.45\textwidth]{figs/imagenet_generated_image/imagenet_fc.png} 
    } 
    
    \subcaptionbox{Synthetic boundary supporting samples from ImageNet.\label{fig:mix_imagenet}}{
        \centering
        \includegraphics[width=0.45\textwidth]{figs/imagenet_generated_image/imagenet_fc_mix.png} 
    } 
    \caption{Additional synthetic samples.}
    \label{fig:moresynth}
\end{figure}



\end{document}