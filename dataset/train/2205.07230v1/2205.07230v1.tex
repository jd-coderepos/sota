

\documentclass[10pt,twocolumn,letterpaper]{article}

\usepackage{cvpr}              

\usepackage{graphicx}
\usepackage{amsmath}
\usepackage{amssymb}
\usepackage{booktabs}
\usepackage{multirow}
\usepackage{bm}
\usepackage{pifont}

\usepackage[pagebackref,breaklinks,colorlinks]{hyperref}

\newcommand{\cmark}{\ding{51}}\newcommand{\xmark}{\ding{55}}


\usepackage[capitalize]{cleveref}
\crefname{section}{Sec.}{Secs.}
\Crefname{section}{Section}{Sections}
\Crefname{table}{Table}{Tables}
\crefname{table}{Tab.}{Tabs.}


\def\cvprPaperID{2082} \def\confName{CVPR}
\def\confYear{2022}

\begin{document}
	
\title{Video Frame Interpolation with Transformer}
	
	\author{
		Liying Lu\textsuperscript{1} \quad Ruizheng Wu\textsuperscript{2} \quad Huaijia Lin\textsuperscript{2} \quad  Jiangbo Lu\textsuperscript{2} \quad Jiaya Jia\textsuperscript{1}\\
		 The Chinese University of Hong Kong \qquad
		 SmartMore\\
		{\tt\small \{lylu,rzwu,linhj,leojia\}@cse.cuhk.edu.hk, jiangbo@smartmore.com} 
	}
	
	
	\maketitle
	
\begin{abstract}
		Video frame interpolation~(VFI), which aims to synthesize intermediate frames of a video, has made remarkable progress with development of deep convolutional networks over past years. Existing methods built upon convolutional networks generally face challenges of handling large motion due to the locality of convolution operations. To overcome this limitation, we introduce a novel framework, which takes advantage of Transformer to model long-range pixel correlation among video frames. Further, our network is equipped with a novel cross-scale window-based attention mechanism, where cross-scale windows interact with each other. This design effectively enlarges the receptive field and aggregates multi-scale information. Extensive quantitative and qualitative experiments demonstrate that our method achieves new state-of-the-art results on various benchmarks. The source code is available at \href{https://github.com/dvlab-research/VFIformer}{https://github.com/dvlab-research/VFIformer}.
\end{abstract}
	


	
	
	
	
	\section{Introduction}
	\label{sec:intro}
	
	
	
	
	Video frame interpolation~(VFI) is a fundamental video processing task in which intermediate frames are synthesized between given consecutive ones to increase the frame rate. It is effective in alleviating motion blur and judder, and has become a compelling strategy for numerous applications, such as novel view synthesis~\cite{flynn2016deepstereo,kalantari2016learning}, video compression~\cite{wu2018video}, video restoration~\cite{haris2020space,xiang2020zooming,kim2020fisr}, and slow motion generation~\cite{niklaus2017video,liu2017video,jiang2018super,niklaus2018context,peleg2019net,bao2019depth}.
	Many popular algorithms adopt optical flow warping~\cite{liu2017video,jiang2018super,bao2019depth,niklaus2018context,softsplat,bao2019memc,liu2019deep,xu2019quadratic,park2020bmbc,asymmetric,xvfi} to tackle this challenging task. Though achieving remarkable performance, these methods built upon convolutional networks generally face challenges of capturing long-range spatial interactions due to the intrinsic locality of convolution operations, thus limited in handling large motion, which is one of the main challenges of VFI. 


	Recently, natural language processing~(NLP)~\cite{vaswani2017attention,devlin2018bert,brown2020language} and computer vision~\cite{dosovitskiy2020image,carion2020end,liu2021swin} tasks achieve notable progress using Transformers, which is a highly adaptive architecture with strong modeling capability. In this work, we are inspired to explore the application of Transformers in the context of video frame interpolation and introduce a novel network, VFIformer. With the attention mechanism as the core operation, VFIformer is able to model pixel correspondence between different frames. Besides, its strong capability of capturing long-range dependency is helpful for handling large motion~(see Fig.~\ref{fig:teaser}). 
	
	
	\begin{figure}[t]
		\begin{center}
\includegraphics[width=1.0\linewidth]{figures/teaser.pdf}
		\end{center}
\caption{Visual comparison on a challenging sample from Middlebury dataset~\cite{middlebury}. Our method produces the result more appealing than the three leading VFI methods, \ie, AdaCoF~\cite{lee2020adacof}, RIFE~\cite{rife} and ABME~\cite{asymmetric}.}
		\label{fig:teaser}
	\end{figure}
	
	
	Since the vanilla Transformer needs high memory and computational cost, the proposed VFIformer is designed in a UNet~\cite{unet} architecture where features are processed in different scales to reduce the computational complexity and enlarge the receptive field. Besides, to overcome the quadratic complexity, inspired by recent work~\cite{liu2021swin,wang2021uformer,liang2021swinir,chu2021twins}, our VFIformer is built upon window-based attention where feature maps are divided into non-overlapping sub-windows. Self-attention is only performed within each sub-window. Despite computationally efficient, such an approach prohibits information interaction between different windows and leads to limited receptive field. We address this problem by proposing cross-scale window-based attention, where the attention is computed among feature windows of different scales. 
	
	Our design enjoys two merits. (1) Compared with windows at the original scale, the corresponding windows at coarser scales cover more content. As a result, the interaction among these windows effectively enlarges the receptive field. (2) Features at coarser scales naturally contain smaller displacement and thus provide informative motion priors for the original scale to facilitate synthesis. 
	
	
	Our contributions are summarized as follows:
	\begin{itemize}
		\item We propose a novel framework integrated with the Transformer for the VFI task, which takes advantage of the Transformer to model long-range pixel correlations among the video frames.
		\item A cross-scale window-based attention mechanism is introduced to enlarge the receptive field of current window-based attention to adapt to the challenges of large motions in the VFI task.
		\item Our model achieves state-of-the-art performance for the VFI task on multiple public benchmarks.
	\end{itemize}
	
	
	


\section{Related Work}
	
	\subsection{Video Frame Interpolation}
	
	Video frame interpolation, aiming to synthesize intermediate frames between existing ones of a video, is a long-standing problem. Existing methods can be classified into three categories. They are phase-based, kernel-based, and motion-based ones. 
	
	\textit{Phase-based} methods represent motion in the phase shift of individual pixels. They interpolate phase and amplitude across the levels of a multi-scale pyramid through optimization~\cite{meyer2015phase} or neural networks~\cite{meyer2018phasenet}. A common drawback of these approaches is that they are only applicable to limited-range motion.
	
	\textit{Kernel-based} methods jointly perform motion estimation and motion compensation in a single step. Niklaus~\etal~\cite{niklaus2017videoaa} estimate a spatially-adaptive convolution kernel for each pixel using a convolutional network. The intermediate frame is then generated by convolving the input frames with the predicted kernels. Further development in this field includes using adaptive separable convolutions~\cite{niklaus2017video} to reduce network parameters, adopting deformable convolution or its alternatives to estimate both kernel weights and offset vectors~\cite{lee2020adacof,shi2020video,cheng2021multiple}, integrating optical flow and interpolation kernels together~\cite{bao2019memc,bao2019depth} to improve the performance, and developing loss functions that combine adaptive convolution and trilinear interpolation~\cite{peleg2019net}. 
	
	These methods tend to yield blurry results when handling fast-moving objects since they hallucinate pixel values directly. Besides, to handle large motion, the estimated kernels are designed to be large, leading to a large number of parameters to learn.
	
	For \textit{motion-based} methods, optical flow is estimated to warp the input frames. Liu~\etal~\cite{liu2017video} introduce a deep network that produces 3D optical flow vectors across space and time, and warps input frames by trilinear sampling. Jiang~\etal~\cite{jiang2018super} linearly combine optical flow between the input images to approximate the intermediate flow. 
	
	Recent work has explored a few strategies for improving the performance of such methods. These efforts include utilizing additional contextual information to interpolate high-quality results~\cite{niklaus2018context}, developing unsupervised techniques by cycle consistency~\cite{reda2019unsupervised}, detecting the occlusion by exploring the depth information~\cite{bao2019depth}, forward-warping input frames using softmax splatting~\cite{softsplat}, using quadratic interpolation to overcome the limitation of linear models~\cite{xu2019quadratic,liu2020enhanced}, leveraging the distillation loss to supervise the intermediate flows~\cite{rife}, and constructing efficient architectures for large resolution images~\cite{sim2021xvfi,choi2021motion}. We note that methods built upon convolutional networks generally face challenges of modeling long-term dependencies thus limiting large motion handling. 


	\begin{figure*}[t]
		\begin{center}
			\includegraphics[width=1.0\linewidth]{figures/overview.pdf}
		\end{center}
\caption{Overview of our proposed framework. At first, a convolutional network is used to directly estimate the intermediate optical flows  and . An encoder  is used to extract multi-scale features  and  from the input frames, where . The input frames and extracted features are then backward warped by the estimated flow, producing , , , and . At last, to generate the final results, the input frames and the warped features are fed into the proposed VFIformer, in which cross-scale attention is employed to enlarge the receptive field.}
		\label{fig:overview}
	\end{figure*}
	
	
	\subsection{Transformer}
	
	Transformer was first proposed by Vaswani~\etal~\cite{vaswani2017attention} for machine translation. It consists of stacked self-attention layers for modeling dense relation among input tokens and has shown great flexibility. After breakthrough with the advent of Transformer in NLP, research of Transformer in computer vision becomes popular. Carion~\etal~\cite{carion2020end} propose an end-to-end detection Transformer~(DETR) for direct set prediction. Dosovitskiy~\etal~\cite{dosovitskiy2020image} propose ViT, which is a pure Transformer for image classification and achieves decent results. Liu~\etal~\cite{liu2021swin} present a general-purpose backbone, called Swin Transformer, which achieves linear computational complexity by computing self-attention within non-overlapping windows. A shifted window scheme is also proposed for cross-window connection. 
	
	Apart from high-level vision tasks, several attempts have also been made to integrate the Transformer into low-level vision tasks~\cite{ttsr,cao2021video,chen2021pre,liang2021swinir}. Chen~\etal~\cite{chen2021pre} develop a pre-trained model for image processing using the Transformer architecture. Liang~\etal~\cite{liang2021swinir} propose SwinIR for image restoration based on the Swin Transformer. Cao~\etal~\cite{cao2021video} adapt Transformer for video super-resolution, and an optical flow-based feed-forward layer is integrated for feature alignment. In this work, we introduce the Transformer into the VFI task, aiming to leverage its power of capturing long-range correlation.
	
	


	
	
	\section{Our Method}
	
	
	Given two input frames  and , video frame interpolation is to synthesize an intermediate frame . Our framework is illustrated in Fig.~\ref{fig:overview}. At first, we utilize a convolutional network~(called flow estimator in the following) and an encoder  to obtain the preliminary elements, including optical flows  and , corresponding warped features  and , and warped images  and . 
	
	With these preliminary results as input, VFIformer (Sec.~\ref{sec:vfiformer}) is utilized to capture long-range pixel interaction, generating the mask and residual for final synthesis. To enlarge the receptive field of the window-based attention in VFIformer, we design a cross-scale window-based attention (Sec.~\ref{sec:csma}) mechanism to make trade-off between efficiency and performance.
	
	
	
	\subsection{VFIformer}
	\label{sec:vfiformer}
	
	\begin{figure*}[t]
		\begin{center}
			\includegraphics[width=0.8\linewidth]{figures/vfiformer.png}
		\end{center}
\caption{Structure of the proposed VFIformer. (a) VFIformer is designed in a UNet architecture. Its encoder consists of several Transformer blocks~(TFBs). (b) Structure of the Transformer blocks~(TFB). (c) Structure of the Transformer layers~(TFL), where Cross-Scale Window-based Attention (CSWA) is a key component.}
		\label{fig:fe_transformer}
		\vspace{-0.1in}
	\end{figure*}
	
	
	Since the locality of convolution constrains its receptive field, previous VFI methods built upon convolutional networks generally face challenges of capturing long-range spatial interactions. In contrast, our work builds upon the recent advance that integrates Transformers into vision models to learn long-range dependencies. We propose the VFIformer, which is able to aggregate information over large receptive fields and is effective in handling large  displacement.
	
	As shown in Fig.~\ref{fig:fe_transformer}a, our VFIformer is designed in a UNet architecture, where features are processed in different scales to reduce computational complexity and enlarge the receptive field.
	The encoder of VFIformer consists of several Transformer blocks~(TFB), and the decoder consists of standard convolutions and transposed convolutions.
	For the -th TFB, its output feature  is produced as
	
	where  is the feature from the last TFB,  and  are the features from  warped by the intermediate optical flow  and .  denotes the operations of the -th TFB. The first TFB takes as input the concatenation of frames , , , and  without features. 
	
	At last, the decoder of VFIformer produces a soft mask  and an image residual  to synthesize the final intermediate frame  as
	 
	where  denotes the Hadamard product. The soft mask is used to blend the two warped frames , . The residual is used to compensate flow errors and occlusion.
	
	
	We then look into the detailed structures of TFB. As illustrated in Fig.~\ref{fig:fe_transformer}(b), each TFB consists of several Transformer layers TFL (see Fig.~\ref{fig:fe_transformer}(c)) and convolutional layers. 
	The features in the -th TFL are processed as
	
	where  is the feature generated by the -th TFL. LN and MLP denote the LayerNorm and Multi-Layer Perceptrons. CSWA denotes our proposed cross-scale window-based attention, which is explained in the following.
	
	
	\subsection{Cross-Scale Window-based Attention~(CSWA)}
	\label{sec:csma}
	
	\paragraph{Window-based Attention~(WA)}
	We first revisit the window-based attention method.
	Though the key component self-attention of Transformer, has shown great flexibility and strong modeling capability, a known fact is that its power comes at the price of high computational complexity. Inspired by \cite{liu2021swin,liang2021swinir}, we employ window-based attention~(WA) to reduce the computational cost, where feature maps are divided into sub-windows. Self-attention is only performed within each sub-window. Specifically, for a feature map , we divide it into  sub-windows of size .
	Taking one of the windows  as an example, its \textit{query}, \textit{key} and \textit{value} matrices ,  and  are computed as
	
	where , , and  are projection matrices shared across different windows. Afterwards, the self-attention is computed as
	
	where  is the learnable positional encoding, and  is the  dimension.
	
	\begin{figure}[t]
		\begin{center}
\includegraphics[width=1.0\linewidth]{figures/msattn.pdf}
		\end{center}
		\vspace{-0.1in}
		\caption{Proposed cross-scale window-based attention~(CSWA). The fine-scale feature  is divided into non-overlapping windows~(dark-colored block), and the coarse-scale feature  is divided into overlapping windows~(light-colored block). The feature windows with the same color~(\eg,  and ) interact with each other.}
		\label{fig:msattn}
		\vspace{-0.1in}
	\end{figure}
	
	\paragraph{Cross-Scale Window-based Attention~(CSWA)}
	Although window-based attention is computationally efficient, the drawback is that the receptive field is still limited, resulting in limited information interaction between different windows. We address this by introducing the cross-scale window-based attention~(CSWA), which enlarges the receptive field in an effective way. 
	
	The details are shown in Fig.~\ref{fig:msattn}, for the input feature map , we first down-sample it by scale 2 to get . Then  is divided into  \textit{non-overlapping} sub-windows, following the same procedure in WA as mentioned above. As for , we first pad it with padding size  in the mode of reflection, and then divide it into \textit{overlapping} sub-windows of size . 
	
	Taking one of the windows  in , we denote  as its corresponding window in . Following Eq.~\eqref{eq:qkv}, we calculate the query  only for the window  from the original feature . As for the \textit{key} and \textit{value}, we calculate them for both the window  and window  to interact features in different scales. The procedure is written as
	
	where , , ,  and  are projection matrices. Afterwards, the attention is computed within the set of  and  in a similar way as Eq.~\eqref{eq:attn}, producing  and . The final result is generated as
	
	where  denotes concatenation in the channel dimension.
	
	
	As shown in Fig.~\ref{fig:msattn}, windows with the same color of  and  interact with each other, introducing multi-scale information and therefore generating more representative features. On the other hand, windows of  cover larger context than those of . For example, window  in  actually covers 4 times as much context as the window  in .
	
	In this way, the receptive field of self-attention is enlarged effectively. We adopt the widely used effective receptive field~(ERF)~\cite{erf} to visualize the ERFs of WA and CSWA. Fig.~\ref{fig:receptive} shows the ERF of a TFB equipped with {\bf left}: WA, {\bf right}: CSWA, it is obvious that the ERF of CSWA is much larger than that of WA.


	\begin{figure}[t]
		\begin{center}
\includegraphics[width=0.7\linewidth]{figures/receptive.pdf}
		\end{center}
		\vspace{-0.1in}
		\caption{The effective receptive field of WA and CSWA.}
		\label{fig:receptive}
		\vspace{-0.1in}
	\end{figure}
	
	


\subsection{Loss Functions}
	
	\noindent\textbf{Reconstruction loss.} We adopt  loss as the reconstruction loss as
	
	where  and  denote the ground-truth intermediate frame and the generated one.
	
	\vspace{0.05in}
	\noindent\textbf{Census loss.} Census loss~\cite{meister2018unflow,zhong2019unsupervised}  is robust to illumination changes, which is defined as the soft Hamming distance between census-transformed~\cite{zabih1994non} image patches of  and .

	\vspace{0.05in}	
	\noindent\textbf{Distillation loss.} Following \cite{rife}, we use distillation loss to explicitly supervise the estimated flows as
	
	where  and  are flows generated by a pretrained flow estimation network~\cite{hui2018liteflownet}.  and  are flow estimated by our flow estimator. 


\vspace{0.05in}	
	\noindent\textbf{Full objective.} Our full objective is defined as
	
	where ,  and  are loss weights for ,  and , respectively.
	
\section{Experiments}
	
	\subsection{Datasets}
	
	Our model is trained on the Vimeo90K training set and evaluated on various datasets.

	\vspace{0.05in}	
	\noindent\textbf{Vimeo90K~\cite{vimeo90k}.} The Vimeo90K training set contains  triplets, where each triplet consists of three consecutive video frames with resolution . The Vimeo90K training set contains  triplets whose resolution is also .

	\vspace{0.05in}	
	\noindent\textbf{UCF101~\cite{ucf101}.} It contains videos with a large variety of human actions. There are 379 triplets with a resolution of .

	\vspace{0.05in}	
	\noindent\textbf{Middlebury.} The Middlebury dataset has two subsets, in which the OTHER dataset provides the ground-truth intermediate frames. The image resolution in this dataset is around . Following previous methods, we report the average interpolation error~(IE) on the OTHER dataset. A lower IE indicates better performance.
	
	
\vspace{0.05in}	
	\noindent\textbf{SNU-FILM~\cite{snufilm}.} It contains  triplets of resolutions up to . There are four different settings according to the motion types: Easy, Medium, Hard and Extreme.
	
	
	\begin{table*}[t]
		\setlength{\belowcaptionskip}{0pt}
		\scalebox{0.95}{
			\begin{tabular}{l|ccccccc}
				\hline 
				\multicolumn{1}{c|}{\multirow{2}{*}{Method}} & \multicolumn{1}{c}{\multirow{2}{*}{Vimeo90K}} & \multicolumn{1}{c}{\multirow{2}{*}{UCF101}} & \multicolumn{1}{c}{\multirow{2}{*}{Middlebury}} & \multicolumn{4}{c}{SNU-FILM} \\
				\cline{5-8} & & & & Easy & Medium & Hard & Extreme \\
				
				
				\hline 
				ToFlow~\cite{vimeo90k} & 33.73/0.9682 & 34.58/0.9667 & 2.15 & 39.08/0.9890 & 34.39/0.9740 & 28.44/0.9180 & 23.39/0.8310 \\ 
				SepConv~\cite{niklaus2017video} & 33.79/0.9702 & 34.78/0.9669 & 2.27 & 39.41/0.9900 & 34.97/0.9762 & 29.36/0.9253 & 24.31/0.8448 \\ 
				CyclicGen~\cite{liu2019deep} & 32.09/0.9490 & 35.11/0.9684 & - & 37.72/0.9840 & 32.47/0.9554 & 26.95/0.8871 & 22.70/0.8083 \\ 
				DAIN~\cite{bao2019depth} & 34.71/0.9756 & 34.99/0.9683 & 2.04 & 39.73/0.9902 & 35.46/0.9780 & 30.17/0.9335 & 25.09/0.8584 \\ 
				CAIN~\cite{snufilm} & 34.65/0.9730 & 34.91/0.9690 & 2.28 & 39.89/0.9900 & 35.61/0.9776 & 29.90/0.9292 & 24.78/0.8507 \\ 
				AdaCoF~\cite{lee2020adacof} & 34.47/0.9730 & 34.90/0.9680 & 2.24 & 39.80/0.9900 & 35.05/0.9754 & 29.46/0.9244 & 24.31/0.8439 \\ 
				BMBC~\cite{park2020bmbc} & 35.01/0.9764 & 35.15/0.9689 & 2.04 & 39.90/0.9902 & 35.31/0.9774 & 29.33/0.9270 & 23.92/0.8432 \\ 
				RIFE-Large~\cite{rife} & 36.10/0.9801 & 35.29/0.9693 & {\color{blue} 1.94} & {\color{blue} 40.02/0.9906} & {\color{blue} 35.92/0.9791} & 30.49/0.9364 & 25.24/0.8621 \\ 
				ABME~\cite{asymmetric} & {\color{blue} 36.18/0.9805} & {\color{blue} 35.38/0.9698} & 2.01 & 39.59/0.9901 & 35.77/0.9789 & {\color{blue} 30.58/0.9364} & {\color{blue} 25.42/0.8639} \\ 
				\textbf{Ours} & {\color{red} 36.50/0.9816} & {\color{red} 35.43/0.9700} & {\color{red} 1.82} & {\color{red} 40.13/0.9907} & {\color{red} 36.09/0.9799} & {\color{red} 30.67/0.9378} & {\color{red} 25.43/0.8643}  \\
				\hline 
		\end{tabular}}
		\caption{Quantitative comparison among different VFI methods on 4 testing datasets. We report the average interpolation error IE (the lower the better) on the Middlebury dataset and report PSNR/SSIM~(the higher the better) on other datasets. The best and second-best results are colored in {\color{red} red} and {\color{blue} blue}.}
		\label{table:quantitative}
\end{table*} 
	
	
	
	
	\begin{table*}[t]
		\setlength{\belowcaptionskip}{0pt}
		\centering
		\scalebox{1}{
			\begin{tabular}{l|cc|ccccc}
				\hline 
				
				
				\multicolumn{1}{c|}{\multirow{2}{*}{ }} & \multicolumn{1}{c}{\multirow{2}{*}{TFLs}} & \multicolumn{1}{c}{\multirow{2}{*}{CSWA}} & \multicolumn{1}{c}{\multirow{2}{*}{Vimeo90K}} & \multicolumn{4}{c}{SNU-FILM} \\
				\cline{5-8} & & & & Easy & Medium & Hard & Extreme \\
				
				
				\hline 
				Model 1 & \xmark & \xmark & 36.27/0.9809 & 40.01/0.9906 & 35.89/0.9793 & 30.58/0.9369 & 25.33/0.8629 \\ 
				Model 2 & \cmark & \xmark & 36.49/0.9815 & 40.06/0.9907 & 36.03/0.9798 & 30.61/0.9375 & 25.40/0.8643  \\ 
				Model 3 & \cmark & \cmark & 36.50/0.9816 & 40.13/0.9907 & 36.09/0.9799 & 30.67/0.9378 & 25.43/0.8643  \\ 
				\hline 
		\end{tabular}}
		\caption{Ablation study on the proposed modules. }
		\label{table:aba_modules}
\end{table*}
	
	
	
	\subsection{Implementation Details}
	
	\noindent\textbf{Network Achitecture.} In the VFIformer, the window size is set to , the channel numbers of linear layers and convolution layers are 180. Each TFB contains 6 TFLs except that the first TFB only contains 2 TFLs. 
	Encoder  contains 4 blocks and each extracts one level of features from  and . Each encoder block consists of 2 convolutions with strides 2 and 1, respectively, and the channel numbers of features are 24, 48, 96, and 192 from shallow to deep layers. 
	The architecture of the flow estimator is included in the supplementary file.
	
	
	
	
		\vspace{0.05in}
	\noindent\textbf{Training Details.} We train our model with the AdamW optimizer. The learning rate is set to . We first train the flow estimator for 0.32M iterations with batch size 48. Then the whole model is trained in an end-to-end manner for 0.47M iterations with batch size 24.
	The weight coefficients , , and  are 1, 1 and 0.01, respectively. We randomly crop  patches from the training samples and augment them by random flip and time reversal. 
	
	
	
	
	
	
	
	


	\subsection{Comparisons with State-of-the-Art Methods}
	
	\begin{figure*}[t]
		\begin{center}
\includegraphics[width=1.0\linewidth]{figures/qualitative.pdf}
		\end{center}
		\vspace{-0.15in}
		\caption{Visual comparison among different VFI methods on the Vimeo90K testing set.}
		\label{fig:qualitative}
		\vspace{-0.1in}
	\end{figure*}
	
	
	\begin{figure*}[t]
		\begin{center}
\includegraphics[width=1.0\linewidth]{figures/multi_frame.pdf}
		\end{center}
		\vspace{-0.1in}
		\caption{ interpolation results of our method on the Vimeo90K testing set.}
		\label{fig:multi_frame}
	\end{figure*}
	
	
	
	
	\begin{figure}[t]
		\begin{center}
\includegraphics[width=1.0\linewidth]{figures/aba_cswa.pdf}
		\end{center}
		\vspace{-0.05in}
		\caption{Ablation study on the cross-scale window-based attention~(CSWA), the samples in the first and second columns are overlaid inputs and ground-truths. Those in the third and fourth are the results of the models without and with CSWA, respectively.}
		\label{fig:aba_cswa}
		\vspace{-0.1in}
	\end{figure}
	
	We compare our model with nine recent, competitive methods, including ToFlow~\cite{vimeo90k}, SepConv~\cite{niklaus2017video}, CyclicGen~\cite{liu2019deep}, DAIN~\cite{bao2019depth}, CAIN~\cite{snufilm}, AdaCoF~\cite{lee2020adacof}, BMBC~\cite{park2020bmbc}, RIFE-Large~\cite{rife} and ABME~\cite{asymmetric}. Table~\ref{table:quantitative} shows the quantitative comparison, where the best and second best results are colored in red and blue. It is observed that our model outperforms recent state-of-the-art methods on all four testing sets.
	It is noteworthy that our method outperforms the second-best method on Vimeo90K testing set by \textbf{0.32} dB.
	
	The visual comparison between our method and other VFI methods is shown in Fig.~\ref{fig:qualitative}. Our proposed method generates more reasonable results with fewer unpleasing artifacts in general. For example, our method successfully interpolates the intermediate frame of the stick with large motion in the first and fourth example of Fig.~\ref{fig:qualitative}.
	
	Meanwhile, to thoroughly investigate the performance of our proposed method, we also conduct multi-frame generation. We recursively apply our model to generate multiple intermediate frames. Specifically, given two input frames  and , we first generate . Then we interpolate between  and  to generate . We show the  interpolation results on Vimeo90K testing set in Fig.~\ref{fig:multi_frame}. Our model yields multiple intermediate frames with smooth motion.
	
	
	
	


	
	
	
	
	\begin{table}
		\setlength{\belowcaptionskip}{0pt}
		\centering
		\scalebox{1}{
			\begin{tabular}{c|c}
				\hline 
				Window Size & ~~~~~~~PSNR/SSIM~~~~~~~ \\ 
				\hline
				4 & ~~~~~~~36.24/0.9806~~~~~~~  \\
				8 & ~~~~~~~36.29/0.9807~~~~~~~  \\
				12 & ~~~~~~~36.31/0.9808~~~~~~~  \\
				\hline 
		\end{tabular}}
		\caption{Ablation study on the window sizes of self-attention. }
		\label{table:aba_ws}
		\vspace{-0.10in}
	\end{table}
	
	
	
	
	
	\subsection{Ablation Study}
	
	
	In this section, we conduct several ablation studies to investigate our proposed method. We verify the effectiveness of the Transformer layers and the proposed cross-scale window-based attention. We also analyze the influence of different window sizes while computing attention.
	
		\vspace{0.05in}
	\noindent\textbf{Effect of the Transformer layers~(TFLs).} TFLs are the key components of our VFIformer, which play the role of capturing long-range dependency. We investigate the effect of TFLs by replacing them with convolutional layers of a similar number of parameters. The ablation results are shown in Table~\ref{table:aba_modules}, where Model 2 is the model with TFLs~(using standard window-based attention) and Model 1 is the model without TFLs.
	Model 2 outperforms Model 1 by 0.22 dB on the Vimeo90K testing set, and also obtains better performance on the SNU-FILM dataset under 4 settings.
	
	
		\vspace{0.05in}
	\noindent\textbf{Effect of Cross-scale Window-based Attention. } Cross-scale window-based attention~(CSWA) is proposed to enlarge the receptive field and aggregate multi-scale information. 
	To further verify its effectiveness, we train two models with and without CSWA respectively. As shown in Table~\ref{table:aba_modules}, Model 2 adopts standard window-based attention~(WA) while Model 3 adopts CSWA.
	
	Compared with Model 2, Model 3 improves it by 0.07, 0.06, and 0.06 dB in the Easy, Medium, and Hard settings of SNU-FILM, respectively, in terms of PSNR. 
	We show the visual comparison of these two models on SNU-FILM~(Hard) in Fig.~\ref{fig:aba_cswa}. It is observed that compared with Model 2, Model 3 generates sharper results with more fine details while dealing with cases with large motion.


	
		\vspace{0.05in}
	\noindent\textbf{Influence of the Attention Window Size. } We further investigate the influence of attention window size, which determines the size of the receptive field. The larger the window is, the larger the range of information can be captured, along with higher computational cost. To enable quick exploration, we train all the models for only 0.36M iterations in this experiment. Table~\ref{table:aba_ws} shows the ablation result. It is observed that the model with window size 12 achieves the best performance. To balance the performance and the computational cost, we choose window size 8 in our experiments.
	
	
	
	
\section{Limitations}
	
	Though our proposed method has achieved decent results, there are several limitations. First, while our model is built upon the window-based attention, the computational cost is still heavier than CNN-based methods due to the complex calculations of self-attention. We will explore more efficient approaches in the future by computing self-attention in the horizontal and vertical stripes in parallel~\cite{dong2021cswin}. 
	
	Second, unlike existing methods~\cite{jiang2018super,asymmetric} that are able to interpolate frames at arbitrary time, our model only synthesizes the intermediate frame. In future work, we will investigate variables that represent the interpolation time and put them into the network to control the generated content.
	
	
	\section{Conclusion}
	
	In this work, we have proposed a novel framework integrated with the Transformer for the video frame interpolation task. The proposed VFIformer endows our framework with a strong capability of modeling long-range dependencies and handling large motions. Further, a novel cross-scale window-based attention mechanism is designed to aggregate multi-scale information and enlarge the receptive field. Extensive experiments show that our proposed method achieves superior performance over existing state-of-the-art methods on multiple popular benchmarks. 
	
	
	
	
	
{\small
		\bibliographystyle{ieee_fullname}
		\bibliography{egbib}
	}


\clearpage
	\newpage
	
	\appendix
	
	\section{Appendix}
	
	\subsection{More Implementation Details}
	\label{sec:imp}
	
	
	\begin{figure}[h]
		\begin{center}
\includegraphics[width=1.0\linewidth]{figures/flow_estimator.pdf}
		\end{center}
		\vspace{-0.1in}
		\caption{(a). The pipeline of our flow estimator, where the Bilateral Local Refinement Block~(BLRB) is used to refine the flows. (b). Details of the Bilateral Local Refinement Block.}
		\label{fig:flow_estimator}
\end{figure}
	
	
	\noindent\textbf{Flow Estimator Achitecture.} The pipeline of our flow estimator is shown in Fig.~\ref{fig:flow_estimator}a, a flow prediction network~\cite{rife} is first used to predict the coarse flows , . Then Bilateral Local Refinement Blocks~(BLRBs) are used to refine the flows in a coarse-to-fine manner, whose details are shown in Fig.~\ref{fig:flow_estimator}b.
	
	For the -th BLRB, given the flows  and  produced by the last BLRB and feature maps ,  of two input frames extracted by the encoder , we first rescale the flows to the current scale~(note that we do not change their notations after rescaling for brevity) and use them to backward warp the features, obtaining  and . Then the warped features are fed into convolutional layers to produce the intermediate feature . The specific process is given by
	
	where  denotes convolutional layers,  denotes concatenation in the channel dimension,  is the generated mask for blending  and , and  is used to ensure the mask in the range of [0, 1].
	
	Afterwards, we compute the local correlation volumes to model the relationships among pixels in  and , . Specifically, for each pixel  in , we map it to its estimated correspondence in ~(here we take refining  for example): . Then a local window around  is defined as
	 
	where  is the radius and is set to  in our experiments. Then we calculate the cosine similarity between  and pixels in , producing a correlation vector.
	The correlation map is then processed by two convolutional layers. Meanwhile, the flow  is also processed with other two convolutional layers. At last, the correlation feature, the flow feature,  and  are concatenated and fed into four convolutional layers to produce the flow residual , the refined flow is obtained as
	 
	Note that all the operations are similar for . 
	
	


	
	\subsection{More Ablation Studies}
	
	\begin{table}[h]
		\setlength{\belowcaptionskip}{0pt}
		\centering
		\scalebox{1}{
			\begin{tabular}{c|c}
				\hline 
				TFL number & ~~~~~~~PSNR/SSIM~~~~~~~ \\ 
				\hline
				2 & ~~~~~~~36.35/0.9810~~~~~~~  \\
				4 & ~~~~~~~36.43/0.9814~~~~~~~  \\
				6 & ~~~~~~~36.49/0.9815~~~~~~~  \\
				8 & ~~~~~~~36.52/0.9817~~~~~~~  \\
				\hline 
		\end{tabular}}
		\caption{Ablation study on the TFL number. }
		\label{table:aba_tflnb}
\end{table}
	
	
	\noindent\textbf{Influence of the TFL number.} We also investigate the influence of the TFL number, the ablation results on the Vimeo90K~\cite{vimeo90k} testing set are shown in Table~\ref{table:aba_tflnb}, we set the numbers of the TFLs in each TFB to 2, 4, 6, and 8, respectively. It is observed that the more TFLs, the higher PSNR/SSIM. To balance the performance and the computational cost, we set the number to 6 in our experiments.
	
	\noindent\textbf{Effect of CSWA.} To have a thorough investigation on the Vimeo90K testing set, we split it into 4 motion levels according to the average flow values estimated by PWC-Net~\cite{pwcnet}. The PSNR gain of CSWA becomes larger as motions become larger as shown in Table~\ref{table:motion}, which further validates the effectiveness of CSWA. 
	
	\begin{table}[h]
		\setlength{\belowcaptionskip}{0pt}
		\centering
		\scalebox{1}{
			\begin{tabular}{|cc|cc|c|}
				\hline
				flow range & avg. flow &  WA & CSWA & gain \\
				\hline
 & &  36.51 & 36.52 & 0.01  \\
				 &  & 36.52 & 36.55 & 0.03 \\
				 &  & 36.91 & 36.94 &  0.03 \\
				 &  & 34.00 & 34.04 & 0.04 \\
				\hline
		\end{tabular}}
		\captionof{table}{PSNR of WA and CSWA under 4 motion levels.}
		\label{table:motion}
\end{table}
	
	\subsection{More Quantitative Comparisons}
	
	\begin{table}[h]
		\setlength{\belowcaptionskip}{0pt}
		\centering
		\footnotesize
		\scalebox{0.9}{
			\begin{tabular}{l|cccc}
				\hline 
				Methods & Runtime~(ms) & Param.~(M) & Vimeo90K \\
				\hline 
				SoftSplat-~\cite{softsplat} & - & -  & 36.10/0.9700 \\
				CDFI~\cite{cdfi} & 60 & 5.0 & 35.17/0.9640 \\
				XVFI~\cite{xvfi} & 96 & 5.5 & 35.07/0.9760 \\
				BMBC~\cite{park2020bmbc} & 478 & 11.0 & 35.01/0.9764 \\
				RIFE-Large~\cite{rife} & 13 & 21.7 & 36.10/0.9801 \\
				ABME~\cite{asymmetric} & 158 & 18.1 & 36.18/0.9805 \\
				Ours & 365 & 24.1 & 36.50/0.9816 \\
				Ours-Small & 213 & 17.0 & 36.38/0.9811 \\
				\hline 
		\end{tabular}}
\caption{More quantitative comparisons. The running time is tested on images with  resolution on an NVIDIA TITAN V GPU.}
		\label{table:runtime_comparison}
\end{table}
	
	We compare the running time and network parameters with some recent SOTA methods in Table~\ref{table:runtime_comparison}.
	As mentioned in the \textit{Limitation} section of the main paper, the computational cost of our model is still heavier than CNN-based methods. To increase the practicability of our model, we further train a light-weight version, which is denoted as `Ours-Small'. It adopts a simpler flow estimator architecture~\cite{rife} and its window size and channel number are  and 136. As shown in Table~\ref{table:runtime_comparison}, the light-weight version has moderate running time and parameters yet still yields the SOTA result.
	
	
	


	\subsection{More Visual Results}
	
	More visual results are shown in Fig.~\ref{fig:comparison_supp_0} and Fig.~\ref{fig:comparison_supp_1}.
	We compare our proposed method and other recent state-of-the-art methods, including AdaCoF~\cite{lee2020adacof}, RIFE-Large~\cite{rife} and ABME~\cite{asymmetric}. It can be observed that our method restores more appealing results with sharper structures. 


	
	
	\begin{figure*}[t]
		\begin{center}
\includegraphics[width=1.0\linewidth]{figures/comparison_supp_0.pdf}
		\end{center}
		\vspace{-0.15in}
		\caption{Visual comparison among different VFI methods on the Vimeo90K~\cite{vimeo90k} testing set.}
		\label{fig:comparison_supp_0}
\end{figure*}
	
	
	\begin{figure*}[t]
		\begin{center}
\includegraphics[width=1.0\linewidth]{figures/comparison_supp_1.pdf}
		\end{center}
		\vspace{-0.15in}
		\caption{Visual comparison among different VFI methods on the Vimeo90K~\cite{vimeo90k} testing set.}
		\label{fig:comparison_supp_1}
\end{figure*}
	





	
\end{document}
