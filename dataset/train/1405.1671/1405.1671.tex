




We present the 
proof of  Theorem~\ref{thm:bmmb:r} using a more formal description of the abstract MAC layer.
We can model our systems using \emph{Timed I/O Automata} \cite{TIOA2010}.

Consider some positive integer .
The 'th power  of a graph  is a graph with the same vertex set , and two nodes 
are adjacent, , when their distance in  is at most .
That is,
.
(The th power  graph  does not includes self-loops.)
For a given node ,
let  denote the set of nodes that are within  hops of 
in , including  itself.
A graph  is a subgraph of  (denoted by ), if  and .








\subsubsection{Guarantees for the Abstract MAC Layer}
\label{sec:model:prop}

Here we provide a set of properties that constrain the
behavior of the abstract MAC layer automaton.
Technically, these properties are expressed for admissible timed
executions of the timed I/O automaton modeling the complete system.

\paragraph{Well-Formedness Properties.}
We assume some constraints on the behavior of the user automata.
Let  be an admissible execution of the system consisting of
user and abstract MAC layer automata.
We say  is {\em user-well-formed} if the following hold, for
every :
\begin{enumerate}
\item
Every two  events have an intervening  or  event.
\item
Every  is preceded by a  (for the same )
with no intervening , , or  events.
\end{enumerate}

The rest of this subsection gives constraints on the behavior of the
abstract MAC layer automaton, in system executions that are
user-well-formed.
Thus, from now on in the section, we assume that  is a
user-well-formed system execution.

\paragraph{Constraints on Message Behavior.}
We assume that there exists a ``cause'' function that maps every 
event in  to a preceding  event, where ,
and that maps each  and  to a preceding .

We use the term \emph{message instance} to refer to the set
consisting of a  event and all the other events that are
related to it by the \emph{cause} function.


We now define two safety conditions and one liveness condition
regarding the relationships captured by the cause function:

\begin{enumerate}
\item {\bf Receive correctness:}
Suppose that  event  causes  event 
in .
Then , and no other  event or  event
caused by  precedes .

\item  {\bf Acknowledgment correctness:}
Suppose that  event  causes  event  in .
Then for every  such that , a  event caused
by  precedes .
Also, no other  event or  caused by 
precedes .

\item  {\bf Termination:}
Every  causes either an  or an .
\end{enumerate}

\paragraph{Time Bounds.}
We now impose upper bounds on the time from a  event to
its corresponding  and  events.\footnote{
  We express these here as constants rather than functions, because we
  will not worry about adaptive bounds in this paper.
  }

Let  and  be positive real numbers.
We use these to bound delays for a specific message to be delivered
and an acknowledgment received (the ``acknowledgment delay''), and for
{\em some} message from among many to be received (the ``progress
delay'').
We think of  as smaller than , because the time to
receive {\em some} message among many is typically less than the time
to receive a specific message.
The statement of the acknowledgment bound is simple:

\begin{enumerate}
\setcounter{enumi}{3}
\item
{\bf Acknowledgment bound:}
Suppose that a  event  causes an  event
 in .
Then the time between  and  is at most .
\end{enumerate}

The statement of the progress bound is a bit more involved, and
requires some auxiliary definitions.
Let  be a closed execution fragment within the given execution
,\footnote{Formally, that means that there
  exist fragments  and 
  such that , and moreover, the first
  state of  is the last state of . Notice,
  this allows  to begin and/or end in the middle of a trajectory.}
and let  be a process.
Then define:
\begin{itemize}
\item
 is the set of message instances in  such that
 is wholly contained between the  and terminating
event ( or ) of the instance, and , where 
is the originator of the  of the message instance.
\item
 is the set of message instances in  for
which the terminating event does not precede the beginning of , and
, where  is the originator of the  of the
message instance.
\end{itemize}

\begin{lemma}
For every  and ,
.
\end{lemma}

\begin{enumerate}
\setcounter{enumi}{4}
\item
{\bf Progress bound:}
For every closed fragment  within , and for every
process , it is not the case that all three of the following
conditions hold:
  \begin{enumerate}
  \item The total time of  is strictly greater than .
  \item .
  \item No  event from a message instance in 
    occurs by the end of .
  \end{enumerate}
In other words,  should receive {\em some} message within time
 provided that at least one message is being sent by a -neighbor.
\end{enumerate}

Note that our definitions allow a  for a particular 
to occur after an  for that .
We impose a (small) bound  on the amount of time
after an  when such a  may occur.






\subsubsection{The Multi-Message Broadcast Problem}
\label{sec:mmb:prelim}

A user automaton is considered to be an {\em MMB protocol} provided
that its external interface includes an  input
and  output for each user process 
and message .

We say an execution of an MMB protocol is {\em MMB-well-formed} if and only if it
contains at most one 
event for each ;
that is, each broadcast message is unique.
We say an MMB protocol {\em solves the MMB problem}
if and only if for every MMB-well-formed (admissible) execution 
of the MMB protocol composed with a MAC layer,
the following hold:
\begin{enumerate}
\item[(a)]
For every  event in  and every process ,
 contains a  event.
\item[(b)]
For every  and every process ,
 contains at most one  event and it
comes after an  event for some .
\end{enumerate}




We describe a simple MMB protocol that achieves efficient runtime.

\begin{quote}\small
{\bf The Basic Multi-Message Broadcast (BMMB) Protocol }\\
Every process  maintains a FIFO queue named  and a set
named .
Both are initially empty.

If process  is not currently sending a message (i.e., not waiting
for an  from the MAC layer) and  is not empty, the
process immediately (without any time-passage) s the message at
the head of  on the MAC layer.

When process  receives an  event, it immediately
performs a local  output and adds  to the back of its
, and to its  set.

When  receives a  message  from the MAC layer it checks
its  set.
If , procsss  discards the message.
Otherwise,  immediately performs a  event,
and adds  to the back of its  and to its  set.
\end{quote}


\begin{theorem}
The BMMB protocol solves the MMB problem.
\label{thm:bmmb}
\end{theorem}

We give two definitions that we will use in our complexity proof.
In the following, let  be some MMB-well-formed
execution of the BMMB protocol composed with a MAC layer.
We begin with two definitions that we will use in our complexity proof.

\paragraph{ events.}
We define a  event with respect to ,
for some arbitrary message  and process ,
to be one in which process  first learns about message .
Specifically,  is the first  event
in case message  arrives at
process ,
otherwise,  is the first  event.

\paragraph{ events}
Let  be a message for which an  event
occurs in .  We define  to describe the final
 event in  for any process .\footnote{
  By the definition of BMMB, if an  occurs, then 
  eventually s , so  eventually occurs. Furthermore,
  by the definition of BMMB, there can be at most one  event for
  every process . Therefore,  is well-defined.}



\subsubsection{Proof Preliminaries}
\label{subsec:Proof Prelim}

For the rest of Section \ref{sec:standard}, we consider the special case of the general MMB problem
in which all messages arrive from the environment at time , that is, all {\em arrive} events occur at time 0.
We fix a particular MMB-well-formed execution  of the BMMB protocol composed with a MAC layer.
We assume that an  event occurs in  for
some message  at some node , at time .

For each node  and each time , we introduce two sets of messages,
which we call  (for "received messages") and  (for "completed messages").
We define:
\cmpdell{\footnote{So , , , and  are implicit parameters.}}
\begin{itemize}
\item
 is the set of messages  for which the   event occurs by time .
\item
 is the set of messages
 for which the  event occurs by time
.
\end{itemize}
That is,  is the set of messages that have been received by
process  by time , and  is the set of messages that
process  has finished (completed) processing by time .\footnote{
Note that both  and  may include .}



The following two lemmas express some very basic properties of the  and  sets.


\begin{lemma}

For every  and  such that :
  \begin{enumerate}
  \item  and .
  \label{item:lemma: t<t' => C(t) <= C(t')}

  \item .
  \label{item:lemma: C_i(t) subseteq R_i(t)}

  \item If  and  are neighbors in , then .
  \label{item:lemma: i and j neigh t'<=t C_i(t') subseteq R_j(t)}

  \end{enumerate}
\label{lemma:bmmb: C(t') subset C(t'')}
\label{lemma: CG1}
\label{lemma: CG4}
\end{lemma}
\begin{proof}
Straightforward.
\end{proof}



\begin{lemma}
\label{lemma: CG6}
Fix  and , and let  be the final state at time .
Then, in state ,  contains exactly the messages in .
\cmpdell{
\begin{enumerate}
\item
If in state ,  is in , then .
\item
In state ,  contains exactly the messages in .
\end{enumerate}
}
\end{lemma}

\begin{proof}
By the operation of the algorithm,  contains exactly the
messages that have had a  and no  at node .
These
are exactly the elements of .
\end{proof}




\begin{lemma}
\label{lemma:  CG2}
\label{lemma: CG3}
Fix  and ,
and let  be the final state at time .
\begin{enumerate}
\item
If in state ,  is in position  of , then .
\item
If in state ,  has length at least , then
.
\item
If , then .
\end{enumerate}
\end{lemma}

\begin{proof}
\begin{enumerate}
\item
For ,
the abstract MAC layer properties say that, within
time ,  is acknowledged at .
Therefore, , which yields the result for .
The statement for general  follows from repeated application of the
statement for .
\item
The statement is trivial for , so consider .
Part 1, applied to the first  messages in  in state ,
implies that all of these messages are in .
Lemma~\ref{lemma: CG6},
implies that none of these messages
are in .
Therefore, .
\item
Suppose that  .
By Lemma~\ref{lemma: CG6}, every element of  is on
 in state , so the length of  in state  is
at least .
We consider two cases. If , then

as needed for Part 3, where the first inequality follows, since

by
Part \ref{item:lemma: t<t' => C(t) <= C(t')}
of Lemma \ref{lemma:bmmb: C(t') subset C(t'')}
(with  and );
the second inequality follows by Part 2 of the lemma;
and the last inequality holds by the assumption of Part 3 (of the lemma).
If , then
, by Part 2 of the lemma.
Part 3 follows.
\end{enumerate}
\end{proof}


The following corollary is a special case of Part 2 of Lemma \ref{lemma:  CG2}.



\begin{corollary}
\label{lemma:bmmb: ack and prog progress}
\label{coro:bmmb: ack and prog progress}
Fix ,  and .
If  and , then .
\end{corollary}
\begin{proof}
The case where  follows
by Part \ref{item:lemma: t<t' => C(t) <= C(t')}
of Lemma \ref{lemma:bmmb: C(t') subset C(t'')}.
Suppose that .
Since , it follows that .
Therefore,
by
Lemma \ref{lemma: CG6},
at the final state  at time ,  has length at least at least one.
Thus, by Part 2 of Lemma \ref{lemma:  CG2}, .
The Corollary follows.
\end{proof}




Now we have two key lemmas that describe situations when a process 
is guaranteed to receive a new message.
The first deals with  receiving its first message.\footnote{
  Actually, this lemma is formally a corollary to the following one,
  but it might be nicer to see this proof as a ``warm-up''.
  }


\begin{lemma}
\label{lemma: CG5}
Let  and  be neighboring nodes in , and suppose .
If , then .
\end{lemma}

\begin{proof}
Assume for contradiction that .
Choose  to be some time strictly after ,
when ;
this is possible because the next discrete event after time  must occur some positive amount of time after .

We obtain a contradiction to the progress bound.
Let  be the closed execution fragment of  that begins
with the final state  at time  and ends with the final state
 at time .
We show that  provides a contradiction to the progress
bound.  We verify that the three conditions in the definition of the
progress bound are all satisfied for .  Condition (a), that
the total time of  is strictly greater than ,
is immediate.

Condition (b) says that .
Since , Lemma~\ref{lemma: CG4}, Part 3, implies
that .
Since , Lemma~\ref{lemma: CG6},
implies
that, in state ,  is nonempty.
Let  be the message at the head of  in state .
Since  is the final state at time  and the protocol has 
delay for performing s, it must be that the  event for
process 's instance for  occurs before the start of .
Since ,  is not received by process  by the
end of .
This implies that the  event, which terminates 's
instance for , must occur after the end of .
It follows that 's instance for  is in ,
so that , which shows Condition
(b).

Condition (c) says that no  event from a message instance in
 occurs by the end of .
We know that no  occurs by the end of , because .
So no  event from a message instance in 
occurs by the end of , which shows Condition (c).

Thus,  satisfies the combination of three conditions that
are prohibited by the progress bound assumption, yielding the needed
contradiction.
\end{proof}




The next lemma deals with the fast positive progress scenario in which  process  receives some ``new'' message in  time.
\begin{lemma}
Let  and  be neighboring nodes in , and suppose . Suppose that:
\begin{enumerate}
\item  for every neighbor  of  in .
\item .
\end{enumerate}

Then,  .
\label{lemma: z-neighborhood prog bound}
\end{lemma}
This says that, if every message that  has already received is already completed at all of 's
neighbors in  and some neighbor  of  in  has received some message that  hasn't yet received, then  will
receive a new message within  time.


\begin{proof}
Assume for contradiction that
 for every neighbor  of  in ,
that , and
that .
Then it must be that .
Choose  to be some time strictly after ,
when ; this is possible because the next
discrete event after time  must occur some positive
amount of time after .



We obtain a contradiction to the progress bound.
Let  be the closed execution fragment of  that begins
with the final state  at time  and ends with the final state
 at time .  We verify that the three conditions in the
definition of the progress bound are all satisfied for .
\begin{itemize}
\item
Condition (a):  The total time of  is strictly greater
than .  \\
This is immediate.

\item
Condition (b):  . \\
Since  and 
(by Part
\ref{item:lemma: i and j neigh t'<=t C_i(t') subseteq R_j(t)}
of Lemma \ref{lemma:bmmb: C(t') subset C(t'')} with , ),
we have
.


Then Lemma~\ref{lemma: CG6},
implies that, in state ,
 is nonempty.
Let  be the message at the head of  in state .
Since  is the final state at time  and the protocol has 
delay for performing s, it must be that the  event for
process 's instance for  occurs before the start of .

Also, we know that , because  and
.

Since , we also know that .
Therefore,  is not received by process  by the end of
.
This implies that the  event, which terminates 's
instance for , must occur after the end of .
It follows that 's instance for  is in ,
so that .

\item
Condition (c):  No  event from a message instance in
 occurs by the end of .
We claim that, if a message  has an instance in
, then .
To see this, let  be a neighbor of  in  originating an instance of
 in .
If , then by hypothesis 1 (of the lemma), also .
That means that the  event of node 's instance of 
occurs before the start of , which implies that the instance
is not in .



With this claim, we can complete the proof for Condition (c).
Suppose for contradiction that a  event from some message
instance in  occurs by the end of .
Using the first claim above, .
But by the second claim above, .
But we have assumed that , which yields a
contradiction.
\end{itemize}

Thus,  satisfies the combination of three conditions that
are prohibited by the progress bound assumption, yielding the needed
contradiction.
\end{proof}





The next lemma deals with the slow positive progress scenario in which  process  is guaranteed to receive some ``new'' message in  time (for some positive integer ).


\begin{lemma}
Fix some time . Suppose that:
\begin{enumerate}

\item
.

\item
 for every node .

\item
There exists some  such that .
\label{item: condition: there exists some j'' s t Rj''(t) geq ell}

\end{enumerate}
Then .
\label{lemma: progress in z cdot Fack}
\end{lemma}

\noindent This says that, if
(1)  completes at least  messages by time ;
(2) every message that  has completed by time  is also completed at all of 's
neighbors in  by time ; and
(3) there exists at least one neighbor of  in  that receives at least  messages by that time,
then  receives
at least  messages by time ~.


\begin{proof}
We prove this lemma by induction on .
For the base, , the statement trivially follows, since , which implies together with condition
\ref{item: condition: there exists some j'' s t Rj''(t) geq ell} that .
For the inductive step, we assume .
We assume the lemma statement for  and prove it for .
If , then
by Part \ref{item:lemma: t<t' => C(t) <= C(t')}
of Lemma \ref{lemma:bmmb: C(t') subset C(t'')},
 and we are done.


If , then by  assumption 1
and Part
\ref{item:lemma: t<t' => C(t) <= C(t')}
of Lemma \ref{lemma:bmmb: C(t') subset C(t'')},
,
which implies
by Part \ref{item:lemma: C_i(t) subseteq R_i(t)}
of Lemma \ref{lemma:bmmb: C(t') subset C(t'')}
that , as needed.


It remains to consider the case where  and  .
Let  be a closest neighbor of  such that .
Note that,  must be at distance at least 1 from  (by the assumptions for this case) and  must be at distance at most  from  (by assumption 3 of the lemma).
Moreover, by the first two assumptions of the lemma, it follow that .
Combining this inequality with   and
Corollary \ref{coro:bmmb: ack and prog progress},
we get that



Let  be the next-closer node on a shortest path in  from  to .
We now apply the inductive hypothesis
for  and time .
Note that, .
To do this, we show that the three assumptions of the lemma indeed hold for  and .
The first assumption of the lemma holds as

where the first inequality holds since

(by Part \ref{item:lemma: t<t' => C(t) <= C(t')}
of Lemma \ref{lemma:bmmb: C(t') subset C(t'')},
with );
and the second inequality holds by assumption 1 for  and .
The second assumption of the lemma holds as,

where the first inequality holds since, in this case,  and  (by assumption 2 of the lemma for  and );
and the second inequality holds by Part \ref{item:lemma: t<t' => C(t) <= C(t')}
of Lemma \ref{lemma:bmmb: C(t') subset C(t'')} with .
We next argue that the third assumption holds as well.
Specifically, we claim that , in particular, satisfies this assumption.
That is,
 and .
The first statement holds, since  and .
The second statement holds as,

where the first inequality holds,
since 
(by Part \ref{item:lemma: i and j neigh t'<=t C_i(t') subseteq R_j(t)}
of Lemma \ref{lemma:bmmb: C(t') subset C(t'')})
and the second inequality holds
by combining together Inequality (\ref{ineq:lemma:|C_j''(t + Ffac)| geq ell}) with .


Having shown the three assumptions, we can now invoke the inductive hypothesis for  and .
We have .
In addition, , since  and .
Combining these two inequalities together with
Part \ref{item:lemma: t<t' => C(t) <= C(t')}
of Lemma \ref{lemma:bmmb: C(t') subset C(t'')},
we get that
, as needed.
The lemma follows.
\end{proof}






\subsubsection{The Key Lemma}
\label{subsec: The Key Lemma}

We continue to assume all the context we established earlier in Subsection \ref{subsec:Proof Prelim}.
The lemma below summarizes some helpful complexity bounds.

\begin{lemma}[``Complexity Bounds'']
\label{lemma:bmmb: t d ell values}
\label{lemma: t-inequalities}
The following hold for the  values:
  \begin{enumerate}
  \item For ,  (monotonically increasing in terms of ).
  \label{item:lemma: t dl monotonically increasing}



  \item For , , .
  \label{item:lemma com-bounds r fack}

  \item For , , .
  \label{item:lemma com-bounds fack t d+r ell-1 + Fack leq t d-1 ell}


  \item For , , .
  \label{item:lemma com-bounds fprog}

  \item For , .
  \label{item:t- t0 + ell fack leq t 0 ell}


  \end{enumerate}
\end{lemma}


\begin{proof}
By simple algebraic calculations.
\end{proof}


To prove the key lemma, we show a double induction
for  as an ``outer'' induction and for distance  as an ``inner'' induction.
To warm up, let us begin with two special cases.
The first (Lemma \ref{lemma: ell=1} below) will be used in the base case for  for the outer induction of the inductive proof
in the main lemma.
The second (see Lemma \ref{lemma: d=0}) will be used in the base case for  for the inner induction of the inductive proof
in the main lemma.


\begin{lemma}
\label{lemma: ell=1}
Let  be a node at distance  from .
Then:
\begin{enumerate}
\item
.
\item
.
\end{enumerate}
\end{lemma}

\begin{proof}
\begin{enumerate}
\item
For Part 1, we use induction on .
For the base case, consider .
Then  and .
Since , we see that ,
as needed.

For the inductive step, assume Part 1 for  and prove it
for .
Let  be the predecessor of  on a shortest path in  from  to
; then .
By inductive hypothesis, we know that
.
Then Lemma~\ref{lemma: CG5} implies that .
Since , this implies that
, as needed.

\item
Part 2 follows from Part 1 using Lemma~\ref{lemma: CG3}, Part 3,
applied with .
\end{enumerate}
\end{proof}


\begin{lemma}
\label{lemma: d=0}
Let .
Then:
\begin{enumerate}
\item
.
\item
Either  or .
\end{enumerate}
\end{lemma}

\begin{proof}
Since , and , we have , which yields Part 1.

For Part 2, if , then clearly , which suffices.
So suppose that .
Then , so  is on 
in the final state  at time .

If in state , the position of  on  is , then Lemma~\ref{lemma: CG2}, Part 1, implies that
.
By Part \ref{item:t- t0 + ell fack leq t 0 ell} of Lemma~\ref{lemma: t-inequalities}, ,
which implies together with
Part \ref{item:lemma: t<t' => C(t) <= C(t')}
of Lemma \ref{lemma:bmmb: C(t') subset C(t'')}
that ,
which is sufficient to establish the claim.

On the other hand, if in state , the position of  on
 is strictly greater than , then we apply
Lemma~\ref{lemma: CG2}, Part 2, to conclude that
.
That implies that , which
again suffices.
\end{proof}

And now, for the main lemma.







\begin{lemma}\label{lemma:bmmblemma-full}
  Let  be a node at distance  from  in .
  Let  be any positive integer. Then:
\begin{enumerate}
  \item  Either  or .

  \item  Either  or .

  \end{enumerate}



\label{lem:bmmb:time}
\end{lemma}

\begin{proof}
We prove both parts together by induction on .
For the base, , both statements follow
immediately from Lemma \ref{lemma: ell=1}.
For the inductive step, let .
We assume the lemma
statement for  (and for all ) and prove it for .
To prove the lemma statement for , we use a second, ``inner''induction, on the distance
 from  and the destination .
For the base, , both statements
follow from Lemma \ref{lemma: d=0}.


\noindent {\bf Inductive Step: .}
For the inductive step, we assume .
Assume both parts of the lemma for
(1)  (as ``outer'' induction hypothesis) for all distances;
and (2) for  for distance  (as ``inner'' induction hypothesis).
We prove both parts of the lemma for  and distance .


By our ``outer'' inductive hypothesis, all processors of the network satisfy the two parts of the lemma for  and all values of .
In particular, by combining the inductive hypothesis for   and all values of  with
Part \ref{item:lemma: t dl monotonically increasing} of Lemma \ref{lemma:bmmb: t d ell values} and
Part \ref{item:lemma: t<t' => C(t) <= C(t')} of Lemma \ref{lemma:bmmb: C(t') subset C(t'')},
it follows that for every node ,
either
\begin{itemize}

\item [(S1)] 

\end{itemize}
We use distance  for  because  is at distance at most  from  in 
( is at distance  from  in ; and  is either  itself or it at distance at most  from  in , since ).
Let .
Recall that,
by Part \ref{item:lemma com-bounds r fack} of Lemma \ref{lemma:bmmb: t d ell values},


\begin{enumerate}

\item We now prove Part 1 of the lemma (for  and ).
Suppose that   or .
Then, either
 or ,
since ,
by Inequality (\ref{ineq:lemma: t d+r ell-1 leq t d ell}) and Part \ref{item:lemma: C_i(t) subseteq R_i(t)} of Lemma \ref{lemma:bmmb: C(t') subset C(t'')}
(with  and ).
This implies that  satisfies Part 1 of the lemma statement for . This implies Part 1.
Now, suppose the contrary, that  and .
Since,  does satisfy (S1) for , it follows that ,  since .
Thus, in the remaining case, we have

We next prove that  (which implies Part 1 of the lemma).
We consider two cases regarding the set of messages that  completes by time 
and the sets of messages that are completed (by that time) by all neighbors of  in .



\noindent {\bf Case 1:} There exists some neighbor  of  in  such that .


Choose a closest node  to  in  with this property, and let 
be the next-closer node on some shortest path from  to  in .
That is,
,  and .
Recall that, in this case, there exists such a neighbor  with this property, hence .


To apply Lemma \ref{lemma: progress in z cdot Fack}, with  and , we first need to show that the three hypothesis of the lemma hold.
First, by the second conjunct of (\ref{ineq:lemma: m not in Cj(t*) and |cj(t*)|= ell-1}),
we have , which implies the first hypothesis of Lemma
\ref{lemma: progress in z cdot Fack}.
Second,
we need to show that , for every .
This follows from the fact that  and the fact that  is a closest node to  in  with the property that .
This implies that , and in particular , for every , as needed.
Third, we need to show that  for some neighbor .
We show that  (that is, , in particular, does satisfy this property).


The fact that  is a closest node with this property and node  is closer than  to  in , implies that
 and that .
By the inductive hypothesis for , we obtain that either  or ;
either way, by Inequality
(\ref{ineq:lemma: m not in Cj(t*) and |cj(t*)|= ell-1}),
there is some message ,
which implies by
Part \ref{item:lemma: i and j neigh t'<=t C_i(t') subseteq R_j(t)}
of Lemma \ref{lemma:bmmb: C(t') subset C(t'')}
(with ,  and ),
that .
This, in turn implies that
.
Then Lemma \ref{lemma: progress in z cdot Fack}
(using  and ),
yields that .
Since ,
by Part \ref{item:lemma com-bounds r fack}
of Lemma \ref{lemma:bmmb: t d ell values},
we have that .
Thus,  as needed for Part 1 of the lemma.




\noindent {\bf Case 2:} , for all neighbors .

Since 
\footnote{
Note that this is the first place that we use this assumption.
}, it holds, in particular, that , for all neighbors  of  in .
Let's focus on time ~.
By Part \ref{item:lemma com-bounds fack t d+r ell-1 + Fack leq t d-1 ell}
and Part \ref{item:lemma com-bounds fprog}
of Lemma \ref{lemma:bmmb: t d ell values},
we have

Now, if , then , by
Part \ref{item:lemma: t dl monotonically increasing}
of Lemma \ref{lem:bmmb:time} and we are done.
So suppose that .
Recall that

where the second inequality holds by combining
Inequality (\ref{ineq:lemma: t^* leq t d-1 ell + fprog = t d ell})
together with Part \ref{item:lemma: C_i(t) subseteq R_i(t)}
of Lemma \ref{lemma:bmmb: C(t') subset C(t'')}
(with  and ).
This implies that .
Hence, , which implies together with Inequality
(\ref{eq-subset:lemma: Cj(t*) with t d-1 ell})  that



Now we will apply Lemma \ref{lemma: z-neighborhood prog bound}, with ~.
To do this, we need to show the two hypotheses of that lemma:
First, we
need to show that  for every neighbor  of  in .
Consider some neighbor  of  in .
We have

where the first equality holds by the case analysis assumption and Equality (\ref{eq:lemma: Rj t d-1 ell = Cj(t*)});
and the second inequality holds by combining the first inequality of
(\ref{ineq:lemma: t^* leq t d-1 ell + fprog = t d ell})
with
Part \ref{item:lemma: t<t' => C(t) <= C(t')}
of Lemma \ref{lemma:bmmb: C(t') subset C(t'')}.
This implies, in particular, that
, for every neighbor  of  in  (since ), as needed for the first hypothesis of Lemma \ref{lemma: z-neighborhood prog bound}.


To show the second hypothesis of Lemma \ref{lemma: z-neighborhood prog bound},
we need to show that
, for some neighbor  of  in .
So, fix a neighbor  of  in  at distance  from .
By the inductive hypothesis for ,
we obtain that either  or ;
either way,
by Inequality
(\ref{ineq:lemma: m not in Cj(t*) and |cj(t*)|= ell-1}),
there is some message .




Then, Lemma \ref{lemma: z-neighborhood prog bound}
yields that .
This implies that , since 
(by Part \ref{item:lemma com-bounds fprog}
of Lemma \ref{lemma:bmmb: t d ell values}).
Part 1 of the lemma follows.




\item Now, we prove Part 2 of the lemma.
Before proceeding recall that

(where the left inequality holds
by Part
\ref{item:lemma com-bounds fack t d+r ell-1 + Fack leq t d-1 ell}
of Lemma
\ref{lemma:bmmb: t d ell values}), which implies together with
Part \ref{item:lemma: t<t' => C(t) <= C(t')}
of Lemma \ref{lemma:bmmb: C(t') subset C(t'')}, that

Now, suppose that .
Then,

(since ) and we are done.
Next, assume that .
Then,  by the inductive hypothesis for , .
This implies, in particular, that

(since ).
By Part 1, either  or ;
either way, we obtain that , since, ,  and .
Then Corollary \ref{coro:bmmb: ack and prog progress} implies that
, so
 as needed.

\end{enumerate}
\end{proof}



\subsubsection{The Main Theorem}

Let  be the set of messages that arrive at the nodes in a given execution .


\begin{theorem}
\label{thm:bmmb:time-full}
If  then  for every node ,
where .
\end{theorem}
The conclusion of this theorem says all the messages of  are received at all
nodes by time .
\begin{proof}
Follows directly from Lemma \ref{lemma:bmmblemma-full}.
\end{proof}













