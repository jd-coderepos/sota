




We present the 
proof of  Theorem~\ref{thm:bmmb:r} using a more formal description of the abstract MAC layer.
We can model our systems using \emph{Timed I/O Automata} \cite{TIOA2010}.

Consider some positive integer $r$.
The $r$'th power $G^r(V,E^r)$ of a graph $G$ is a graph with the same vertex set $V$, and two nodes $v,u\in V$
are adjacent, $(u,v)\in E^r$, when their distance in $G$ is at most $r$.
That is,
$E^r=\{ (u,v) \mid u\not= v \mbox{ and } d_G(u,v) \leq r\}$.
(The $r$th power  graph $G^r$ does not includes self-loops.)
For a given node $j\in V$,
let $N^r_G(j)=\{ j' \mid d_G(j,j')\leq z \}$ denote the set of nodes that are within $r$ hops of $j$
in $G$, including $j$ itself.
A graph $G'=(V',E')$ is a subgraph of $G=(V,E)$ (denoted by $G\subseteq G'$), if $V=V'$ and $E' \subseteq E$.








\subsubsection{Guarantees for the Abstract MAC Layer}
\label{sec:model:prop}

Here we provide a set of properties that constrain the
behavior of the abstract MAC layer automaton.
Technically, these properties are expressed for admissible timed
executions of the timed I/O automaton modeling the complete system.

\paragraph{Well-Formedness Properties.}
We assume some constraints on the behavior of the user automata.
Let $\alpha$ be an admissible execution of the system consisting of
user and abstract MAC layer automata.
We say $\alpha$ is {\em user-well-formed} if the following hold, for
every $i$:
\begin{enumerate}
\item
Every two $bcast_i$ events have an intervening $ack_i$ or $abort_i$ event.
\item
Every $abort(m)_i$ is preceded by a $bcast(m)_i$ (for the same $m$)
with no intervening $bcast_i$, $ack_i$, or $abort_i$ events.
\end{enumerate}

The rest of this subsection gives constraints on the behavior of the
abstract MAC layer automaton, in system executions that are
user-well-formed.
Thus, from now on in the section, we assume that $\alpha$ is a
user-well-formed system execution.

\paragraph{Constraints on Message Behavior.}
We assume that there exists a ``cause'' function that maps every $rcv(m)_j$
event in $\alpha$ to a preceding $bcast(m)_i$ event, where $i \ne j$,
and that maps each $ack(m)_i$ and $abort(m)_i$ to a preceding $bcast(m)_i$.

We use the term \emph{message instance} to refer to the set
consisting of a $bcast$ event and all the other events that are
related to it by the \emph{cause} function.


We now define two safety conditions and one liveness condition
regarding the relationships captured by the cause function:

\begin{enumerate}
\item {\bf Receive correctness:}
Suppose that $bcast(m)_i$ event $\pi$ causes $rcv(m)_j$ event $\pi'$
in $\alpha$.
Then $(i,j) \in E'$, and no other $rcv(m)_j$ event or $ack(m)_i$ event
caused by $\pi$ precedes $\pi'$.

\item  {\bf Acknowledgment correctness:}
Suppose that $bcast(m)_i$ event $\pi$ causes $ack(m)_i$ event $\pi'$ in $\alpha$.
Then for every $j$ such that $(i,j) \in E$, a $rcv(m)_j$ event caused
by $\pi$ precedes $\pi'$.
Also, no other $ack(m)_i$ event or $abort(m)_i$ caused by $\pi$
precedes $\pi'$.

\item  {\bf Termination:}
Every $bcast(m)_i$ causes either an $ack(m)_i$ or an $abort(m)_i$.
\end{enumerate}

\paragraph{Time Bounds.}
We now impose upper bounds on the time from a $bcast(m)_i$ event to
its corresponding $ack(m)_i$ and $rcv(m)_j$ events.\footnote{
  We express these here as constants rather than functions, because we
  will not worry about adaptive bounds in this paper.
  }

Let $\fack$ and $\fprog$ be positive real numbers.
We use these to bound delays for a specific message to be delivered
and an acknowledgment received (the ``acknowledgment delay''), and for
{\em some} message from among many to be received (the ``progress
delay'').
We think of $\fprog$ as smaller than $\fack$, because the time to
receive {\em some} message among many is typically less than the time
to receive a specific message.
The statement of the acknowledgment bound is simple:

\begin{enumerate}
\setcounter{enumi}{3}
\item
{\bf Acknowledgment bound:}
Suppose that a $bcast(m)_i$ event $\pi$ causes an $ack(m)_i$ event
$\pi'$ in $\alpha$.
Then the time between $\pi$ and $\pi'$ is at most $\fack$.
\end{enumerate}

The statement of the progress bound is a bit more involved, and
requires some auxiliary definitions.
Let $\alpha'$ be a closed execution fragment within the given execution
$\alpha$,\footnote{Formally, that means that there
  exist fragments $\alpha''$ and $\alpha'''$
  such that $\alpha = \alpha'' \alpha' \alpha'''$, and moreover, the first
  state of $\alpha'$ is the last state of $\alpha''$. Notice,
  this allows $\alpha'$ to begin and/or end in the middle of a trajectory.}
and let $j$ be a process.
Then define:
\begin{itemize}
\item
$connect(\alpha',j)$ is the set of message instances in $\alpha$ such that
$\alpha'$ is wholly contained between the $bcast$ and terminating
event ($ack$ or $abort$) of the instance, and $(i,j) \in E$, where $i$
is the originator of the $bcast$ of the message instance.
\item
$contend(\alpha',j)$ is the set of message instances in $\alpha$ for
which the terminating event does not precede the beginning of $\alpha'$, and
$(i,j) \in E'$, where $i$ is the originator of the $bcast$ of the
message instance.
\end{itemize}

\begin{lemma}
For every $\alpha'$ and $j$,
$connect(\alpha', j) \subseteq contend(\alpha',j)$.
\end{lemma}

\begin{enumerate}
\setcounter{enumi}{4}
\item
{\bf Progress bound:}
For every closed fragment $\alpha'$ within $\alpha$, and for every
process $j$, it is not the case that all three of the following
conditions hold:
  \begin{enumerate}
  \item The total time of $\alpha'$ is strictly greater than $\fprog$.
  \item $connect(\alpha', j) \neq \emptyset$.
  \item No $rcv_j$ event from a message instance in $contend(\alpha', j)$
    occurs by the end of $\alpha'$.
  \end{enumerate}
In other words, $j$ should receive {\em some} message within time
$\fprog$ provided that at least one message is being sent by a $G$-neighbor.
\end{enumerate}

Note that our definitions allow a $rcv$ for a particular $bcast$
to occur after an $abort$ for that $bcast$.
We impose a (small) bound $\epsilon_{abort}$ on the amount of time
after an $abort$ when such a $rcv$ may occur.






\subsubsection{The Multi-Message Broadcast Problem}
\label{sec:mmb:prelim}

A user automaton is considered to be an {\em MMB protocol} provided
that its external interface includes an $arrive(m)_i$ input
and $deliver(m)_i$ output for each user process $i$
and message $m\in {\cal M}$.

We say an execution of an MMB protocol is {\em MMB-well-formed} if and only if it
contains at most one $arrive(m)_i$
event for each $m\in {\cal M}$;
that is, each broadcast message is unique.
We say an MMB protocol {\em solves the MMB problem}
if and only if for every MMB-well-formed (admissible) execution $\alpha$
of the MMB protocol composed with a MAC layer,
the following hold:
\begin{enumerate}
\item[(a)]
For every $arrive(m)_i$ event in $\alpha$ and every process $j$,
$\alpha$ contains a $deliver(m)_j$ event.
\item[(b)]
For every $m\in {\cal M}$ and every process $j$,
$\alpha$ contains at most one $deliver(m)_j$ event and it
comes after an $arrive(m)_i$ event for some $i$.
\end{enumerate}




We describe a simple MMB protocol that achieves efficient runtime.

\begin{quote}\small
{\bf The Basic Multi-Message Broadcast (BMMB) Protocol }\\
Every process $i$ maintains a FIFO queue named $bcastq$ and a set
named $rcvd$.
Both are initially empty.

If process $i$ is not currently sending a message (i.e., not waiting
for an $ack$ from the MAC layer) and $bcastq$ is not empty, the
process immediately (without any time-passage) $bcast$s the message at
the head of $bcastq$ on the MAC layer.

When process $i$ receives an $arrive(m)_i$ event, it immediately
performs a local $deliver(m)_i$ output and adds $m$ to the back of its
$bcastq$, and to its $rcvd$ set.

When $i$ receives a  message $m$ from the MAC layer it checks
its $rcvd$ set.
If $m \in rcvd$, procsss $i$ discards the message.
Otherwise, $i$ immediately performs a $deliver(m)_i$ event,
and adds $m$ to the back of its $bcastq$ and to its $rcvd$ set.
\end{quote}


\begin{theorem}
The BMMB protocol solves the MMB problem.
\label{thm:bmmb}
\end{theorem}

We give two definitions that we will use in our complexity proof.
In the following, let $\alpha$ be some MMB-well-formed
execution of the BMMB protocol composed with a MAC layer.
We begin with two definitions that we will use in our complexity proof.

\paragraph{$get$ events.}
We define a $get(m)_i$ event with respect to $\alpha$,
for some arbitrary message $m$ and process $i$,
to be one in which process $i$ first learns about message $m$.
Specifically, $get(m)_i$ is the first $arrive(m)_i$ event
in case message $m$ arrives at
process $i$,
otherwise, $get(m)_i$ is the first $rcv(m)_i$ event.

\paragraph{$clear$ events}
Let $m\in\mathcal{M}$ be a message for which an $arrive(m)_i$ event
occurs in $\alpha$.  We define $clear(m)$ to describe the final
$ack(m)_j$ event in $\alpha$ for any process $j$.\footnote{
  By the definition of BMMB, if an $arrive(m)_i$ occurs, then $i$
  eventually $bcast$s $m$, so $ack(m)_i$ eventually occurs. Furthermore,
  by the definition of BMMB, there can be at most one $ack(m)_j$ event for
  every process $j$. Therefore, $clear(m)$ is well-defined.}



\subsubsection{Proof Preliminaries}
\label{subsec:Proof Prelim}

For the rest of Section \ref{sec:standard}, we consider the special case of the general MMB problem
in which all messages arrive from the environment at time $t_0 = 0$, that is, all {\em arrive} events occur at time 0.
We fix a particular MMB-well-formed execution $\alpha$ of the BMMB protocol composed with a MAC layer.
We assume that an $arrive(m)_{i_0}$ event occurs in $\alpha$ for
some message $m \in \mathcal{M}$ at some node $i_0$, at time $t_0=0$.

For each node $i\in V$ and each time $t$, we introduce two sets of messages,
which we call $\cR$ (for "received messages") and $\cC$ (for "completed messages").
We define:
\cmpdell{\footnote{So $\alpha$, $m$, $i_0$, and $\mathcal{M}'$ are implicit parameters.}}
\begin{itemize}
\item
$\cR_i(t) \subseteq \mathcal{M}\cmpdell{'}$ is the set of messages $m' \in
\mathcal{M}\cmpdell{'}$ for which the  $get(m')_i$ event occurs by time $t$.
\item
$\cC_i(t)\subseteq\mathcal{M}\cmpdell{'}$ is the set of messages
$m'\in\mathcal{M}\cmpdell{'}$ for which the $ack(m')_i$ event occurs by time
$t$.
\end{itemize}
That is, $\cR_i(t)$ is the set of messages that have been received by
process $i$ by time $t$, and $\cC_i(t)$ is the set of messages that
process $i$ has finished (completed) processing by time $t$.\footnote{
Note that both $\cR_i(t)$ and $\cC_i(t)$ may include $m$.}



The following two lemmas express some very basic properties of the $\cR$ and $\cC$ sets.


\begin{lemma}

For every $i,i',t$ and $t'$ such that $t\leq t'$:
  \begin{enumerate}
  \item $\cR_i(t)\subseteq \cR_i(t')$ and $\cC_i(t)\subseteq \cC_i(t')$.
  \label{item:lemma: t<t' => C(t) <= C(t')}

  \item $\cC_{i}(t) \subseteq \cR_{i}(t')$.
  \label{item:lemma: C_i(t) subseteq R_i(t)}

  \item If $i$ and $i'$ are neighbors in $G$, then $\cC_{i'}(t) \subseteq \cR_{i}(t')$.
  \label{item:lemma: i and j neigh t'<=t C_i(t') subseteq R_j(t)}

  \end{enumerate}
\label{lemma:bmmb: C(t') subset C(t'')}
\label{lemma: CG1}
\label{lemma: CG4}
\end{lemma}
\begin{proof}
Straightforward.
\end{proof}



\begin{lemma}
\label{lemma: CG6}
Fix $i$ and $t \geq 0$, and let $s$ be the final state at time $t$.
Then, in state $s$, $\bcastq_i$ contains exactly the messages in $\cR_i(t) - \cC_i(t)$.
\cmpdell{
\begin{enumerate}
\item
If in state $s$, $m'$ is in $\bcastq_i$, then $m' \in \mathcal{M}'$.
\item
In state $s$, $\bcastq_i$ contains exactly the messages in $\cR_i(t) - \cC_i(t)$.
\end{enumerate}
}
\end{lemma}

\begin{proof}
By the operation of the algorithm, $\bcastq_i$ contains exactly the
messages that have had a $get$ and no $ack$ at node $i$.
These
are exactly the elements of $\cR_i(t) - \cC_i(t)$.
\end{proof}




\begin{lemma}
\label{lemma:  CG2}
\label{lemma: CG3}
Fix $i$ and $t \geq 0$,
and let $s$ be the final state at time $t$.
\begin{enumerate}
\item
If in state $s$, $m'$ is in position $k\geq 1$ of $bcastq_i$, then $m'
\in \cC_i(t + k \fack)$.
\item
If in state $s$, $bcastq_i$ has length at least $k\geq 0$, then
$|\cC_i(t + k \fack)| \geq |\cC_i(t)| + k$.
\item
If $|\cR_i(t)| \geq k \geq 0$, then $|\cC_i(t + k \fack)| \geq k$.
\end{enumerate}
\end{lemma}

\begin{proof}
\begin{enumerate}
\item
For $k=1$,
the abstract MAC layer properties say that, within
time $\fack$, $m'$ is acknowledged at $i$.
Therefore, $m' \in \cC_i(t + \fack)$, which yields the result for $k=1$.
The statement for general $k$ follows from repeated application of the
statement for $k=1$.
\item
The statement is trivial for $k=0$, so consider $k \geq 1$.
Part 1, applied to the first $k$ messages in $bcastq_i$ in state $s$,
implies that all of these messages are in $\cC_i(t + \fack)$.
Lemma~\ref{lemma: CG6},
implies that none of these messages
are in $\cC_i(t)$.
Therefore, $|\cC_i(t + \fack)| \geq |\cC_i(t)| + k$.
\item
Suppose that  $|\cR_i(t)| = |\cC_i(t)| + |\cR_i(t) - \cC_i(t)| \geq k$.
By Lemma~\ref{lemma: CG6}, every element of $\cR_i(t) - \cC_i(t)$ is on
$bcastq_i$ in state $s$, so the length of $bcastq_i$ in state $s$ is
at least $|\cR_i(t) - \cC_i(t)|$.
We consider two cases. If $|\cR_i(t) - \cC_i(t)|\leq k$, then
$$
|\cC_i(t+k \cdot \fack)| \geq |\cC_i(t+ |\cR_i(t) - \cC_i(t)| \cdot \fack)| \geq |\cC_i(t)|+|\cR_i(t) - \cC_i(t)| \geq k,
$$
as needed for Part 3, where the first inequality follows, since
$\cC_i(t+ |\cR_i(t) - \cC_i(t)|\cdot \fack) \subseteq \cC_i(t+k \cdot \fack) $
by
Part \ref{item:lemma: t<t' => C(t) <= C(t')}
of Lemma \ref{lemma:bmmb: C(t') subset C(t'')}
(with $t= t+ |\cR_i(t) - \cC_i(t)|\cdot \fack$ and $t'=t+k \cdot \fack$);
the second inequality follows by Part 2 of the lemma;
and the last inequality holds by the assumption of Part 3 (of the lemma).
If $|\cR_i(t) - \cC_i(t)| > k$, then
$|\cC_i(t+k \cdot \fack)| \geq |\cC_i(t)|+k$, by Part 2 of the lemma.
Part 3 follows.
\end{enumerate}
\end{proof}


The following corollary is a special case of Part 2 of Lemma \ref{lemma:  CG2}.



\begin{corollary}
\label{lemma:bmmb: ack and prog progress}
\label{coro:bmmb: ack and prog progress}
Fix $i$, $t\geq 0$ and $\ell>0$.
If $|\cC_i(t)| \geq \ell-1$ and $|\cR_i(t)| \geq \ell$, then $|\cC_i(t+\fack)| \geq \ell$.
\end{corollary}
\begin{proof}
The case where $|\cC_i(t)|\geq \ell$ follows
by Part \ref{item:lemma: t<t' => C(t) <= C(t')}
of Lemma \ref{lemma:bmmb: C(t') subset C(t'')}.
Suppose that $|\cC_i(t)| = \ell-1$.
Since $|\cR_i(t)| \geq \ell$, it follows that $\cR_i(t)-\cC_i(t)\not=\emptyset$.
Therefore,
by
Lemma \ref{lemma: CG6},
at the final state $s$ at time $t$, $bcastq_i$ has length at least at least one.
Thus, by Part 2 of Lemma \ref{lemma:  CG2}, $|\cR_i(t+\fack)| \geq |\cR_i(t)|+1$.
The Corollary follows.
\end{proof}




Now we have two key lemmas that describe situations when a process $i$
is guaranteed to receive a new message.
The first deals with $i$ receiving its first message.\footnote{
  Actually, this lemma is formally a corollary to the following one,
  but it might be nicer to see this proof as a ``warm-up''.
  }


\begin{lemma}
\label{lemma: CG5}
Let $i$ and $j$ be neighboring nodes in $G$, and suppose $t \geq 0$.
If $\cR_j(t) \neq \emptyset$, then $\cR_i(t + \fprog) \neq \emptyset$.
\end{lemma}

\begin{proof}
Assume for contradiction that $\cR_i(t + \fprog) = \emptyset$.
Choose $t' > t + \fprog$ to be some time strictly after $t + \fprog$,
when $\cR_i(t') = \emptyset$;
this is possible because the next discrete event after time $t +
\fprog$ must occur some positive amount of time after $t + \fprog$.

We obtain a contradiction to the progress bound.
Let $\alpha'$ be the closed execution fragment of $\alpha$ that begins
with the final state $s$ at time $t$ and ends with the final state
$s'$ at time $t'$.
We show that $\alpha'$ provides a contradiction to the progress
bound.  We verify that the three conditions in the definition of the
progress bound are all satisfied for $\alpha'$.  Condition (a), that
the total time of $\alpha'$ is strictly greater than $\fprog$,
is immediate.

Condition (b) says that $connect(\alpha',i) \neq \emptyset$.
Since $\cR_i(t) = \emptyset$, Lemma~\ref{lemma: CG4}, Part 3, implies
that $\cC_j(t) = \emptyset$.
Since $\cR_j(t) \neq \emptyset$, Lemma~\ref{lemma: CG6},
implies
that, in state $s$, $bcastq_j$ is nonempty.
Let $m'$ be the message at the head of $bcastq_j$ in state $s$.
Since $s$ is the final state at time $t$ and the protocol has $0$
delay for performing $bcast$s, it must be that the $bcast$ event for
process $j$'s instance for $m'$ occurs before the start of $\alpha'$.
Since $m' \notin \cR_i(t')$, $m'$ is not received by process $i$ by the
end of $\alpha'$.
This implies that the $ack_j(m')$ event, which terminates $j$'s
instance for $m'$, must occur after the end of $\alpha'$.
It follows that $j$'s instance for $m'$ is in $connect(\alpha', i)$,
so that $connect(\alpha',i) \neq \emptyset$, which shows Condition
(b).

Condition (c) says that no $rcv_i$ event from a message instance in
$contend(\alpha',i)$ occurs by the end of $\alpha'$.
We know that no $rcv_i$ occurs by the end of $\alpha'$, because $\cR_i(t') = \emptyset$.
So no $rcv_i$ event from a message instance in $contend(\alpha',i)$
occurs by the end of $\alpha'$, which shows Condition (c).

Thus, $\alpha'$ satisfies the combination of three conditions that
are prohibited by the progress bound assumption, yielding the needed
contradiction.
\end{proof}




The next lemma deals with the fast positive progress scenario in which  process $j$ receives some ``new'' message in $\fprog$ time.
\begin{lemma}
Let $i$ and $j$ be neighboring nodes in $G$, and suppose $t \geq 0$. Suppose that:
\begin{enumerate}
\item $\cR_i(t) \subseteq \cC_{i'}(t)$ for every neighbor $i'$ of $i$ in $G'$.
\item $\cR_j(t) - \cR_i(t) \not= \emptyset$.
\end{enumerate}

Then,  $|\cR_i(t + \fprog) | > |\cR_i(t)|$.
\label{lemma: z-neighborhood prog bound}
\end{lemma}
This says that, if every message that $i$ has already received is already completed at all of $i$'s
neighbors in $G'$ and some neighbor $j$ of $i$ in $G$ has received some message that $i$ hasn't yet received, then $i$ will
receive a new message within $\fprog$ time.


\begin{proof}
Assume for contradiction that
$\cR_i(t) \subseteq \cC_{i'}(t)$ for every neighbor $i'$ of $i$ in $G'$,
that $\cR_j(t) - \cR_i(t) \neq \emptyset$, and
that $|\cR_i(t+\fprog)| = |\cR_i(t)|$.
Then it must be that $\cR_i(t+\fprog) = \cR_i(t)$.
Choose $t' > t + \fprog$ to be some time strictly after $t + \fprog$,
when $\cR_i(t') = \cR_i(t)$; this is possible because the next
discrete event after time $t + \fprog$ must occur some positive
amount of time after $t + \fprog$.



We obtain a contradiction to the progress bound.
Let $\alpha'$ be the closed execution fragment of $\alpha$ that begins
with the final state $s$ at time $t$ and ends with the final state
$s'$ at time $t'$.  We verify that the three conditions in the
definition of the progress bound are all satisfied for $\alpha'$.
\begin{itemize}
\item
Condition (a):  The total time of $\alpha'$ is strictly greater
than $\fprog$.  \\
This is immediate.

\item
Condition (b):  $connect(\alpha',i) \neq \emptyset$. \\
Since $\cR_j(t) - \cR_i(t) \neq \emptyset$ and $\cC_j(t) \subseteq \cR_i(t)$
(by Part
\ref{item:lemma: i and j neigh t'<=t C_i(t') subseteq R_j(t)}
of Lemma \ref{lemma:bmmb: C(t') subset C(t'')} with $i=i$, $i'=j$),
we have
$\cR_j(t) - \cC_j(t) \neq \emptyset$.


Then Lemma~\ref{lemma: CG6},
implies that, in state $s$,
$\bcastq_j$ is nonempty.
Let $m'$ be the message at the head of $\bcastq_j$ in state $s$.
Since $s$ is the final state at time $t$ and the protocol has $0$
delay for performing $bcast$s, it must be that the $bcast$ event for
process $j$'s instance for $m'$ occurs before the start of $\alpha'$.

Also, we know that $m' \notin \cR_i(t)$, because $m' \notin \cC_j(t)$ and
$\cR_i(t) \subseteq \cC_j(t)$.

Since $\cR_i(t') = \cR_i(t)$, we also know that $m' \notin \cR_i(t')$.
Therefore, $m'$ is not received by process $i$ by the end of
$\alpha'$.
This implies that the $ack_j(m')$ event, which terminates $j$'s
instance for $m'$, must occur after the end of $\alpha'$.
It follows that $j$'s instance for $m'$ is in $connect(\alpha', i)$,
so that $connect(\alpha',i) \neq \emptyset$.

\item
Condition (c):  No $rcv_i$ event from a message instance in
$contend(\alpha',i)$ occurs by the end of $\alpha'$.
We claim that, if a message $m''$ has an instance in
$contend(\alpha',i)$, then $m'' \notin \cR_i(t)$.
To see this, let $i'$ be a neighbor of $i$ in $G'$ originating an instance of
$m''$ in $contend(\alpha',i)$.
If $m'' \in \cR_i(t)$, then by hypothesis 1 (of the lemma), also $m'' \in \cC_{i'}(t)$.
That means that the $ack$ event of node $i'$'s instance of $m''$
occurs before the start of $\alpha'$, which implies that the instance
is not in $contend(\alpha,i)$.



With this claim, we can complete the proof for Condition (c).
Suppose for contradiction that a $rcv_i(m'')$ event from some message
instance in $contend(\alpha,i)$ occurs by the end of $\alpha'$.
Using the first claim above, $m'' \in \cR_i(t')$.
But by the second claim above, $m'' \notin \cR_i(t)$.
But we have assumed that $\cR_i(t') = \cR_i(t)$, which yields a
contradiction.
\end{itemize}

Thus, $\alpha'$ satisfies the combination of three conditions that
are prohibited by the progress bound assumption, yielding the needed
contradiction.
\end{proof}





The next lemma deals with the slow positive progress scenario in which  process $j$ is guaranteed to receive some ``new'' message in $z \cdot \fack$ time (for some positive integer $z$).


\begin{lemma}
Fix some time $t\geq 0$. Suppose that:
\begin{enumerate}

\item
$|\cC_{j}(t)|\geq \ell-1$.

\item
$\cC_{j}(t) \subseteq \cC_{j'}(t)$ for every node $j' \in N_G^z(j)$.

\item
There exists some $j'' \in N_G^z(j)$ such that $|\cR_{j''}(t) |\geq \ell$.
\label{item: condition: there exists some j'' s t Rj''(t) geq ell}

\end{enumerate}
Then $|\cR_j(t + z \fack)| \geq \ell$.
\label{lemma: progress in z cdot Fack}
\end{lemma}

\noindent This says that, if
(1) $j$ completes at least $\ell-1$ messages by time $t$;
(2) every message that $j$ has completed by time $t$ is also completed at all of $j$'s
neighbors in $G^z$ by time $t$; and
(3) there exists at least one neighbor of $j$ in $G^z$ that receives at least $\ell$ messages by that time,
then $j$ receives
at least $\ell$ messages by time $t+ z\cdot \fack$~.


\begin{proof}
We prove this lemma by induction on $z$.
For the base, $z= 0$, the statement trivially follows, since $N^0(j)=\{j\}$, which implies together with condition
\ref{item: condition: there exists some j'' s t Rj''(t) geq ell} that $|\cR_{j}(t) |\geq \ell$.
For the inductive step, we assume $z \geq 1$.
We assume the lemma statement for $z'<z$ and prove it for $z$.
If $|\cR_j(t)| \geq \ell$, then
by Part \ref{item:lemma: t<t' => C(t) <= C(t')}
of Lemma \ref{lemma:bmmb: C(t') subset C(t'')},
$|\cR_j(t+\fack)| \geq \ell$ and we are done.


If $\cC_{j}(t+\fack) \not= \cC_{j}(t)$, then by  assumption 1
and Part
\ref{item:lemma: t<t' => C(t) <= C(t')}
of Lemma \ref{lemma:bmmb: C(t') subset C(t'')},
$|\cC_{j}(t+\fack)|\geq \ell$,
which implies
by Part \ref{item:lemma: C_i(t) subseteq R_i(t)}
of Lemma \ref{lemma:bmmb: C(t') subset C(t'')}
that $|\cR_{j}(t+z\cdot \fack)|\geq \ell$, as needed.


It remains to consider the case where $|\cR_j(t)| = \ell-1$ and  $\cC_{j}(t+\fack) = \cC_{j}(t)$.
Let $j''$ be a closest neighbor of $j$ such that $|\cR_{j''}(t)| \geq \ell$.
Note that, $j''$ must be at distance at least 1 from $j$ (by the assumptions for this case) and $j''$ must be at distance at most $z$ from $j$ (by assumption 3 of the lemma).
Moreover, by the first two assumptions of the lemma, it follow that $|\cC_{j''}(t)| \geq \ell-1$.
Combining this inequality with  $|\cR_{j''}(t)| \geq \ell$ and
Corollary \ref{coro:bmmb: ack and prog progress},
we get that
\begin{equation}
|\cC_{j''}(t+\fack)| \geq \ell.
\label{ineq:lemma:|C_j''(t + Ffac)| geq ell}
\end{equation}


Let $j^*$ be the next-closer node on a shortest path in $G$ from $j''$ to $j$.
We now apply the inductive hypothesis
for $z' = z-1$ and time $t'=t+\fack$.
Note that, $j^*\in N^{z'}_G(j)$.
To do this, we show that the three assumptions of the lemma indeed hold for $z'$ and $t'$.
The first assumption of the lemma holds as
$$|\cC_{j}(t')|\geq |\cC_{j}(t)|\geq \ell-1,$$
where the first inequality holds since
$\cC_{j}(t) \subseteq \cC_{j}(t')$
(by Part \ref{item:lemma: t<t' => C(t) <= C(t')}
of Lemma \ref{lemma:bmmb: C(t') subset C(t'')},
with $t\leq t'$);
and the second inequality holds by assumption 1 for $z$ and $t$.
The second assumption of the lemma holds as,
$$
\cC_{j}(t') \subseteq \cC_{j'}(t) \subseteq \cC_{j'}(t'), \mbox{ for every node } j' \in N_G^{z'},
$$
where the first inequality holds since, in this case, $\cC_{j}(t') = \cC_{j}(t)$ and $\cC_{j}(t) \subseteq \cC_{j'}(t)$ (by assumption 2 of the lemma for $z>z'$ and $t$);
and the second inequality holds by Part \ref{item:lemma: t<t' => C(t) <= C(t')}
of Lemma \ref{lemma:bmmb: C(t') subset C(t'')} with $t\leq t'$.
We next argue that the third assumption holds as well.
Specifically, we claim that $j^*$, in particular, satisfies this assumption.
That is,
$j^*\in N_G^{z'}(j)$ and $|\cR_{j^*}(t')| \geq \ell$.
The first statement holds, since $z'=z-1$ and $d_G(j,j^*) < d_G(j,j'')\leq z$.
The second statement holds as,
$$
|\cR_{j^*}(t')| \geq |\cC_{j''}(t')|\geq \ell,
$$
where the first inequality holds,
since $\cC_{j''}(t') \subseteq \cR_{j^*}(t')$
(by Part \ref{item:lemma: i and j neigh t'<=t C_i(t') subseteq R_j(t)}
of Lemma \ref{lemma:bmmb: C(t') subset C(t'')})
and the second inequality holds
by combining together Inequality (\ref{ineq:lemma:|C_j''(t + Ffac)| geq ell}) with $t'=t+\fack$.


Having shown the three assumptions, we can now invoke the inductive hypothesis for $z'$ and $t'$.
We have $|\cR_{j}(t'+z'\cdot \fack)| \geq \ell$.
In addition, $t'+z'\cdot \fack \leq t+z\cdot \fack$, since $t'=t+\fack$ and $z'\leq z-1$.
Combining these two inequalities together with
Part \ref{item:lemma: t<t' => C(t) <= C(t')}
of Lemma \ref{lemma:bmmb: C(t') subset C(t'')},
we get that
$|\cR_{j}(t+z\cdot \fack)| \geq \ell$, as needed.
The lemma follows.
\end{proof}






\subsubsection{The Key Lemma}
\label{subsec: The Key Lemma}

We continue to assume all the context we established earlier in Subsection \ref{subsec:Proof Prelim}.
The lemma below summarizes some helpful complexity bounds.

\begin{lemma}[``Complexity Bounds'']
\label{lemma:bmmb: t d ell values}
\label{lemma: t-inequalities}
The following hold for the $t_{d,\ell}$ values:
  \begin{enumerate}
  \item For $d'\leq d''$, $t_{d',\ell} \leq t_{d'',\ell} $ (monotonically increasing in terms of $d$).
  \label{item:lemma: t dl monotonically increasing}



  \item For $\ell\geq 2$, $d\geq 1$, $t_{d+r,\ell-1} + r\cdot \fack \leq t_{d,\ell}$.
  \label{item:lemma com-bounds r fack}

  \item For $\ell\geq 2$, $d\geq 1$, $t_{d+r,\ell-1} + \fack \leq t_{d-1,\ell}$.
  \label{item:lemma com-bounds fack t d+r ell-1 + Fack leq t d-1 ell}


  \item For $\ell\geq 1$, $d\geq 1$, $t_{d-1,\ell} + \fprog = t_{d,\ell}$.
  \label{item:lemma com-bounds fprog}

  \item For $\ell > 1$, $\ell\cdot \fack \leq t_{0,\ell} $.
  \label{item:t- t0 + ell fack leq t 0 ell}


  \end{enumerate}
\end{lemma}


\begin{proof}
By simple algebraic calculations.
\end{proof}


To prove the key lemma, we show a double induction
for $\ell$ as an ``outer'' induction and for distance $d$ as an ``inner'' induction.
To warm up, let us begin with two special cases.
The first (Lemma \ref{lemma: ell=1} below) will be used in the base case for $\ell=1$ for the outer induction of the inductive proof
in the main lemma.
The second (see Lemma \ref{lemma: d=0}) will be used in the base case for $d=0$ for the inner induction of the inductive proof
in the main lemma.


\begin{lemma}
\label{lemma: ell=1}
Let $j$ be a node at distance $d=d_G(i_0,j)$ from $i_0$.
Then:
\begin{enumerate}
\item
$\cR_j(t_{d,1}) \neq \emptyset$.
\item
$\cC_j(t_{d,1} + \fack) \neq \emptyset$.
\end{enumerate}
\end{lemma}

\begin{proof}
\begin{enumerate}
\item
For Part 1, we use induction on $d$.
For the base case, consider $d=0$.
Then $j = i_0$ and $t_{d,1} = t_{0,1} = 0$.
Since $m \in \cR_{i_0}(0)$, we see that $\cR_j(t_{d,1}) \neq \emptyset$,
as needed.

For the inductive step, assume Part 1 for $d-1$ and prove it
for $d$.
Let $j'$ be the predecessor of $j$ on a shortest path in $G$ from $i_0$ to
$j$; then $d_G(i_0,j') = d-1$.
By inductive hypothesis, we know that
$\cR_{j'}(t_{d-1,1}) \neq \emptyset$.
Then Lemma~\ref{lemma: CG5} implies that $\cR_j(t_{d-1,1} + \fprog) \neq
\emptyset$.
Since $t_{d,1} = t_{d-1,1} + \fprog$, this implies that
$\cR_j(t_{d,1}) \neq \emptyset$, as needed.

\item
Part 2 follows from Part 1 using Lemma~\ref{lemma: CG3}, Part 3,
applied with $k=1$.
\end{enumerate}
\end{proof}


\begin{lemma}
\label{lemma: d=0}
Let $\ell \geq 1$.
Then:
\begin{enumerate}
\item
$m \in \cR_{i_0}(t_{0,\ell})$.
\item
Either $m \in \cC_{i_0}(t_{0,\ell} + \fack)$ or $|\cC_{i_0}(t_{0,\ell} + \fack)| \geq \ell$.
\end{enumerate}
\end{lemma}

\begin{proof}
Since $m \in \cR_{i_0}(0)$, and $0 \leq t_{0,\ell}$, we have $m \in
\cR_j(t_{0,\ell})$, which yields Part 1.

For Part 2, if $m \in \cC_{i_0}(0)$, then clearly $m \in \cC_{i_0}(t_{0,\ell} + \fack)$, which suffices.
So suppose that $m \notin \cC_{i_0}(0)$.
Then $m \in \cR_{i_0}(0) - \cC_{i_0}(0)$, so $m$ is on $bcastq_{i_0}$
in the final state $s_0$ at time $t=0$.

If in state $s_0$, the position of $m$ on $bcastq_{i_0}$ is $\leq
\ell$, then Lemma~\ref{lemma: CG2}, Part 1, implies that
$m \in \cC_{i_0}(\ell \fack)$.
By Part \ref{item:t- t0 + ell fack leq t 0 ell} of Lemma~\ref{lemma: t-inequalities}, $ \ell \fack \leq t_{0,\ell} + \fack$,
which implies together with
Part \ref{item:lemma: t<t' => C(t) <= C(t')}
of Lemma \ref{lemma:bmmb: C(t') subset C(t'')}
that $m \in \cC_{i_0}(t_{0,\ell} + \fack)$,
which is sufficient to establish the claim.

On the other hand, if in state $s_0$, the position of $m$ on
$bcastq_{i_0}$ is strictly greater than $\ell$, then we apply
Lemma~\ref{lemma: CG2}, Part 2, to conclude that
$|\cC_{i_0}( \ell \fack)| \geq \ell$.
That implies that $|\cC_{i_0}(t_{0,\ell} + \fack)| \geq \ell$, which
again suffices.
\end{proof}

And now, for the main lemma.







\begin{lemma}\label{lemma:bmmblemma-full}
  Let $j$ be a node at distance $d=d_G(i_0,j)$ from $i_0$ in $G$.
  Let $\ell$ be any positive integer. Then:
\begin{enumerate}
  \item  Either $m\in \cR_j(t_{d,\ell})$ or $|\cR_j(t_{d,\ell})|\geq \ell$.

  \item  Either $m\in \cC_j(t_{d,\ell}+\fack)$ or $|\cC_j(t_{d,\ell}+\fack)|\geq \ell$.

  \end{enumerate}



\label{lem:bmmb:time}
\end{lemma}

\begin{proof}
We prove both parts together by induction on $\ell$.
For the base, $\ell= 1$, both statements follow
immediately from Lemma \ref{lemma: ell=1}.
For the inductive step, let $\ell\geq 2$.
We assume the lemma
statement for $\ell-1$ (and for all $d$) and prove it for $\ell$.
To prove the lemma statement for $\ell$, we use a second, ``inner''induction, on the distance
$d$ from $i_0$ and the destination $j$.
For the base, $d = 0$, both statements
follow from Lemma \ref{lemma: d=0}.


\noindent {\bf Inductive Step: $d \geq 1$.}
For the inductive step, we assume $d \geq 1$.
Assume both parts of the lemma for
(1) $\ell -1$ (as ``outer'' induction hypothesis) for all distances;
and (2) for $\ell$ for distance $d-1$ (as ``inner'' induction hypothesis).
We prove both parts of the lemma for $\ell$ and distance $d$.


By our ``outer'' inductive hypothesis, all processors of the network satisfy the two parts of the lemma for $\ell-1$ and all values of $d$.
In particular, by combining the inductive hypothesis for $\ell-1$  and all values of $d$ with
Part \ref{item:lemma: t dl monotonically increasing} of Lemma \ref{lemma:bmmb: t d ell values} and
Part \ref{item:lemma: t<t' => C(t) <= C(t')} of Lemma \ref{lemma:bmmb: C(t') subset C(t'')},
it follows that for every node $j' \in N^r_G(j)$,
either
\begin{itemize}

\item [(S1)] $m\in \cC_{j'}(t_{d+r,\ell-1}+\fack) \mbox{ or } |\cC_{j'}(t_{d+r,\ell-1}+\fack)|\geq \ell-1.$

\end{itemize}
We use distance $d+r$ for $j'$ because $j'$ is at distance at most $d+r$ from $i_0$ in $G$
($j$ is at distance $d$ from $i_0$ in $G$; and $j'$ is either $j$ itself or it at distance at most $r$ from $j$ in $G$, since $j' \in N^r_G(j)$).
Let $t^*= t_{d+r,\ell-1}+\fack$.
Recall that,
by Part \ref{item:lemma com-bounds r fack} of Lemma \ref{lemma:bmmb: t d ell values},
\begin{equation}
t^* + (r-1) \fack \leq t_{d,\ell}~.
\label{ineq:lemma: t d+r ell-1 leq t d ell}
\end{equation}

\begin{enumerate}

\item We now prove Part 1 of the lemma (for $\ell$ and $d$).
Suppose that $m\in \cC_j(t^*)$  or $|\cC_{j}(t^*)| \geq \ell$.
Then, either
$m\in \cR_{j}(t_{d,\ell})$ or $|\cR_{j}(t_{d,\ell})| \geq \ell $,
since $\cC_{j}(t^*) \subseteq \cR_{j}(t_{d,\ell})$,
by Inequality (\ref{ineq:lemma: t d+r ell-1 leq t d ell}) and Part \ref{item:lemma: C_i(t) subseteq R_i(t)} of Lemma \ref{lemma:bmmb: C(t') subset C(t'')}
(with $t=t^*$ and $t'= t_{d,\ell}$).
This implies that $j$ satisfies Part 1 of the lemma statement for $\ell$. This implies Part 1.
Now, suppose the contrary, that $m\not\in \cC_j(t^*)$ and $|\cC_{j}(t^*)| < \ell$.
Since, $j$ does satisfy (S1) for $\ell-1$, it follows that $|\cC_{j}(t^*)| \geq \ell-1$,  since $m\not\in \cC_j(t^*)$.
Thus, in the remaining case, we have
\begin{equation}
m\not\in \cC_j(t^*) \mbox{ and } |\cC_{j}(t^*)| = \ell-1.
\label{ineq:lemma: m not in Cj(t*) and |cj(t*)|= ell-1}
\end{equation}
We next prove that $|\cR_{j}(t_{d,\ell})| \geq \ell$ (which implies Part 1 of the lemma).
We consider two cases regarding the set of messages that $j$ completes by time $t^*$
and the sets of messages that are completed (by that time) by all neighbors of $j$ in $G^r$.



\noindent {\bf Case 1:} There exists some neighbor $j'$ of $j$ in $G^r$ such that $\cC_{j'}(t^*) \not =  \cC_j(t^*)$.


Choose a closest node $j''$ to $j$ in $G$ with this property, and let $j^*$
be the next-closer node on some shortest path from $j''$ to $j$ in $G$.
That is,
$j'' \in \arg\min\{d_G(j,i') \mid \cC_{i'}(t^*)\not= \cC_{j}(t^*) \}$, $(j'',j^*)\in E$ and $d_G(j,j^*)=d_G(j,j'')-1$.
Recall that, in this case, there exists such a neighbor $j''\in N^r_G(j)$ with this property, hence $0\leq d_G(j,j^*)<d_G(j,j'')\leq r$.


To apply Lemma \ref{lemma: progress in z cdot Fack}, with $z = d_G(j,j^*)\leq r-1$ and $t = t^*$, we first need to show that the three hypothesis of the lemma hold.
First, by the second conjunct of (\ref{ineq:lemma: m not in Cj(t*) and |cj(t*)|= ell-1}),
we have $|\cC_{j}(t^*)| = \ell-1$, which implies the first hypothesis of Lemma
\ref{lemma: progress in z cdot Fack}.
Second,
we need to show that $\cC_{j}(t^*) \subseteq \cC_{j'}(t^*)$, for every $j' \in N_G^z(j)$.
This follows from the fact that $d_G(j,j'')=z+1$ and the fact that $j''$ is a closest node to $j$ in $G$ with the property that $\cC_{j''}(t^*) \neq \cC_{j}(t^*)$.
This implies that $\cC_{j}(t^*) = \cC_{j'}(t^*)$, and in particular $\cC_{j}(t^*) \subseteq \cC_{j'}(t^*)$, for every $j' \in N_G^z(j)$, as needed.
Third, we need to show that $|\cR_{j'}(t^*)|\geq \ell$ for some neighbor $j' \in N^z_G(j)$.
We show that $|\cR_{j^*}(t^*)|\geq \ell$ (that is, $j^*$, in particular, does satisfy this property).


The fact that $j''$ is a closest node with this property and node $j^*$ is closer than $j''$ to $j$ in $G$, implies that
$\cC_{j^*}(t^*) = \cC_j(t^*)$ and that $\cC_{j''}(t^*) \neq \cC_{j^*}(t^*)$.
By the inductive hypothesis for $\ell-1$, we obtain that either $m\in \cC_{j''}(t^*)$ or $|\cC_{j''}(t^*)| \geq \ell-1$;
either way, by Inequality
(\ref{ineq:lemma: m not in Cj(t*) and |cj(t*)|= ell-1}),
there is some message $m'\in \cC_{j''}(t^*) \setminus \cC_{j^*}(t^*)$,
which implies by
Part \ref{item:lemma: i and j neigh t'<=t C_i(t') subseteq R_j(t)}
of Lemma \ref{lemma:bmmb: C(t') subset C(t'')}
(with $t=t'=t^*$, $i'=j''$ and $i=j^*$),
that $m'\in \cR_{j''}(t^*) \setminus \cC_{j''}(t^*)$.
This, in turn implies that
$|\cR_{j^*}(t^*)| \geq \ell$.
Then Lemma \ref{lemma: progress in z cdot Fack}
(using $z = d_G(j,j^*)\leq r-1$ and $t = t^*$),
yields that $|\cR_j(t^* + (r-1) \fack)| \geq \ell$.
Since $t^*= t_{d+r,\ell-1}+ \fack$,
by Part \ref{item:lemma com-bounds r fack}
of Lemma \ref{lemma:bmmb: t d ell values},
we have that $t_{d,\ell} \geq t^* + (r-1) \fack$.
Thus, $|\cR_j(t_{d,\ell})| \geq \ell$ as needed for Part 1 of the lemma.




\noindent {\bf Case 2:} $\cC_{j'}(t^*) = \cC_j(t^*)$, for all neighbors $j'\in N^r_G(j)$.

Since $G' \subseteq G^r$
\footnote{
Note that this is the first place that we use this assumption.
}, it holds, in particular, that $\cC_{j'}(t^*) = \cC_{j}(t^*)$, for all neighbors $j'$ of $j$ in $G'$.
Let's focus on time $t_{d-1,\ell}$~.
By Part \ref{item:lemma com-bounds fack t d+r ell-1 + Fack leq t d-1 ell}
and Part \ref{item:lemma com-bounds fprog}
of Lemma \ref{lemma:bmmb: t d ell values},
we have
\begin{equation}
t^* \leq t_{d-1,\ell} + \fprog = t_{d,\ell} ~.
\label{ineq:lemma: t^* leq t d-1 ell + fprog = t d ell}
\end{equation}
Now, if $|\cR_{j}(t_{d-1,\ell})| \geq \ell$, then $|\cR_{j}(t_{d,\ell})| \geq \ell$, by
Part \ref{item:lemma: t dl monotonically increasing}
of Lemma \ref{lem:bmmb:time} and we are done.
So suppose that $|\cR_{j}(t_{d-1,\ell})| \leq \ell-1$.
Recall that
\begin{equation}
|\cC_{j}(t^*)|=\ell-1 \mbox{ and } \cC_{j}(t^*) \subseteq \cR_{j}(t_{d-1,\ell}),
\label{eq-subset:lemma: Cj(t*) with t d-1 ell}
\end{equation}
where the second inequality holds by combining
Inequality (\ref{ineq:lemma: t^* leq t d-1 ell + fprog = t d ell})
together with Part \ref{item:lemma: C_i(t) subseteq R_i(t)}
of Lemma \ref{lemma:bmmb: C(t') subset C(t'')}
(with $t=t^*$ and $t'=t_{d-1,\ell}$).
This implies that $|\cR_{j}(t_{d-1,\ell})| \geq \ell-1$.
Hence, $|\cR_{j}(t_{d-1,\ell})| = \ell-1$, which implies together with Inequality
(\ref{eq-subset:lemma: Cj(t*) with t d-1 ell})  that
\begin{equation}
\cR_{j}(t_{d-1,\ell})=\cC_{j}(t^*).
\label{eq:lemma: Rj t d-1 ell = Cj(t*)}
\end{equation}


Now we will apply Lemma \ref{lemma: z-neighborhood prog bound}, with $t=t_{d-1,\ell}$~.
To do this, we need to show the two hypotheses of that lemma:
First, we
need to show that $\cR_j(t_{d-1,\ell}) \subseteq  \cC_{j'}(t_{d-1,\ell})$ for every neighbor $j'$ of $j$ in $G'$.
Consider some neighbor $j'$ of $j$ in $G^r$.
We have
$$\cR_{j}(t_{d-1,\ell}) = \cC_{j'}(t^*) \subseteq \cC_{j'}(t_{d-1,\ell}), \mbox{ for every neighbor } j' \mbox{ of } j \mbox{ in } G^r,$$
where the first equality holds by the case analysis assumption and Equality (\ref{eq:lemma: Rj t d-1 ell = Cj(t*)});
and the second inequality holds by combining the first inequality of
(\ref{ineq:lemma: t^* leq t d-1 ell + fprog = t d ell})
with
Part \ref{item:lemma: t<t' => C(t) <= C(t')}
of Lemma \ref{lemma:bmmb: C(t') subset C(t'')}.
This implies, in particular, that
$\cR_{j}(t_{d-1,\ell}) \subseteq \cC_{j'}(t_{d-1,\ell})$, for every neighbor $j'$ of $j$ in $G'$ (since $G' \subseteq G^r$), as needed for the first hypothesis of Lemma \ref{lemma: z-neighborhood prog bound}.


To show the second hypothesis of Lemma \ref{lemma: z-neighborhood prog bound},
we need to show that
$\cR_{j'}(t_{d-1,\ell}) - \cR_{j}(t_{d-1,\ell})\not= \emptyset$, for some neighbor $j'$ of $j$ in $G$.
So, fix a neighbor $j^*$ of $j$ in $G$ at distance $d-1$ from $i_0$.
By the inductive hypothesis for $d$,
we obtain that either $m\in \cR_{j^*}(t_{d-1,\ell})$ or $|\cR_{j^*}(t_{d-1,\ell})| \geq \ell$;
either way,
by Inequality
(\ref{ineq:lemma: m not in Cj(t*) and |cj(t*)|= ell-1}),
there is some message $m'\in \cR_{j^*}(t_{d-1,\ell}) \setminus \cC_{j}(t^*)$.




Then, Lemma \ref{lemma: z-neighborhood prog bound}
yields that $|\cR_{j}(t_{d-1,\ell}+ \fprog)| > |\cR_{j}(t_{d-1,\ell})| = \ell-1$.
This implies that $|\cR_{j}(t_{d,\ell})| \geq \ell$, since $t_{d,\ell} = t_{d-1,\ell}+\fprog$
(by Part \ref{item:lemma com-bounds fprog}
of Lemma \ref{lemma:bmmb: t d ell values}).
Part 1 of the lemma follows.




\item Now, we prove Part 2 of the lemma.
Before proceeding recall that
$t^*=t_{d+r,\ell-1}+\fack \leq t_{d,\ell} < t_{d,\ell}+\fack$
(where the left inequality holds
by Part
\ref{item:lemma com-bounds fack t d+r ell-1 + Fack leq t d-1 ell}
of Lemma
\ref{lemma:bmmb: t d ell values}), which implies together with
Part \ref{item:lemma: t<t' => C(t) <= C(t')}
of Lemma \ref{lemma:bmmb: C(t') subset C(t'')}, that
$$
\cC_j(t^*) \subseteq \cC_j(t_{d,\ell}) \subseteq \cC_j(t_{d,\ell}+\fack).
$$
Now, suppose that $m\in \cC_j(t^*)$.
Then,
$m\in \cC_{j}(t_{d,\ell}+\fack)$
(since $\cC_{j}(t^*) \subseteq \cC_{j}(t_{d,\ell}+\fack)$) and we are done.
Next, assume that $m\not\in \cC_j(t^*)$.
Then,  by the inductive hypothesis for $\ell-1$, $|\cC_j(t^*)| \geq \ell-1$.
This implies, in particular, that
$|\cC_j(t_{d,\ell})| \geq \ell-1$
(since $\cC_j(t^*)\subseteq \cC_j(t_{d,\ell})$).
By Part 1, either $m\in \cR_j(t_{d,\ell})$ or $|\cR_j(t_{d,\ell})|\geq \ell$;
either way, we obtain that $|\cR_j(t_{d,\ell})|\geq \ell$, since, $\cC_j(t^*)\subseteq \cR_j(t_{d,\ell})$, $|\cC_j(t^*)|\geq \ell-1$ and $m\not\in \cC_j(t^*)$.
Then Corollary \ref{coro:bmmb: ack and prog progress} implies that
$|\cC_j(t_{d,\ell}+\fack)|>|\cC_j(t_{d,\ell})|$, so
$|\cC_j(t_{d,\ell}+\fack)|\geq \ell$ as needed.

\end{enumerate}
\end{proof}



\subsubsection{The Main Theorem}

Let $\mathcal{K}\subseteq \cM$ be the set of messages that arrive at the nodes in a given execution $\alpha$.


\begin{theorem}
\label{thm:bmmb:time-full}
If $|\mathcal{K}| \leq k$ then $\cR_j(t_1)=\mathcal{K}$ for every node $j$,
where $t_1 = (D + (r+1)k - 2) \fprog + r(k-1) \fack$.
\end{theorem}
The conclusion of this theorem says all the messages of $\mathcal{K}$ are received at all
nodes by time $t_1$.
\begin{proof}
Follows directly from Lemma \ref{lemma:bmmblemma-full}.
\end{proof}













