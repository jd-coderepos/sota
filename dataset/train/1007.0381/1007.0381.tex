\documentclass{article}
\usepackage{amssymb, amsmath, latexsym}
\usepackage[dvips]{epsfig}
\newtheorem{theorem}{Theorem}
\newtheorem{lemma}{Lemma}
\newtheorem{proposition}{Proposition}
\newtheorem{corollary}{Corollary}
\newtheorem{definition}{Definition}
\newtheorem{notation}{Notation}
\newtheorem{remark}{Remark}
\linespread{1.6}
\begin{document}
\title{\bf Parallel Chip Firing Game associated with -{\em cube} orientations}  
\author{Ren\'e Ndoundam, Maurice Tchuente, Claude Tadonki \\
 {\small Department of Computer Science, Faculty of Science, 
 University of Yaound\'e I,  } \\
{\small P.o. Box. 812 Yaound\'e, Cameroon} \\
 {\small Department of Mathematics and Computer Science, Faculty of Science, University of Ngaound\'er\'e } \\
{\small University of Paris-Sud 11 ,  CNRS , IEF (AXIS Group) } \\
        {\small UMR 8622 - B\^at 220 - Centre Scientifique d'Orsay - F91405 Orsay Cedex , France}    \\
{\small E.mail : ndoundam@yahoo.com , Maurice.Tchuente@ens-lyon.fr , claude.tadonki@u-psud.fr } \\
                 } 
\maketitle
\begin{abstract} We study the cycles generated by the chip firing game
 associated with -{\em cube} orientations. We show the existence of the cycles generated by
{\em parallel evolutions} of even lengths from  to  on  (), and of odd lengths different
 from  and ranging from  to  on  ().
\end{abstract}
{\bf Keywords :}
Graph, chip firing game, parallel evolution, cycle, transient, -{\em cube} orientation.

\pagestyle{myheadings}
\thispagestyle{plain}
\markboth{R. Ndoundam, M. Tchuente and C. Tadonki}{Parallel Chip firing game
  on hypercube}

\section{Introduction}
Consider a digraph , where  is the set of vertices
and  is the set of arcs. The {\em out-degree} ( resp. {\em in-degree} ) of a
vertex , hereafter denoted by  (resp.  ), is the number of vertices  such
that  (resp. ). A vertex
with {\em out-degree} zero is called a {\em sink}. All these notions apply to an
undirected graph  by considering an edge  as two opposite
arcs  and .

In the {\em parallel chip firing game} played on , a state is a mapping
 which can be viewed as a distribution of chips onto the
vertices of . A vertex is said to be {\em active} in a state  if
, otherwise it is said to be {\em passive}. In a move of the
game, a state  is transformed into a new state as follows : every vertex
tries to send one chip to every {\em out-neighbor}.

\hspace{2mm} If it is not possible, i.e. if  is {\em passive}, then it
resigns ;

\hspace{2mm} Otherwise, vertex  is {\em active} and sends the chips.\\
It is easily seen that the number of chips remains constant. Therefore, the
evolution is ultimately {\em periodic}. More precisely, if ,
denotes the state of the system at time , then there exists an integer 
called {\em transient length} and another integer  called {\em period}
or  {\em cycle length } such that \\
\\
The sequence  is called the {\em transient} and every
sequence of  consecutive states , such that
, is called a {\em cycle} of the evolution.

Following Spencer's introductory paper \cite{SPE:86} which was devoted to the
chip firing game on chains, many authors have been interested in this
problem. The most interesting questions concern the relationships between the
structure of the graph on one hand, and the transients and periods generated
by the chip firing game on the other hand. Concerning the {\em period}, Bitar and
Goles have shown that if  is a tree, then only periods one and two occur
\cite{BIT:92}. Later, Prisner has studied a generalization of the game by
considering multigraphs, i.e. digraphs with multiplicities on the arcs. He
has then shown that there is a sharp contrast in the behavior for eulerian
digraphs (i.e. digraphs where the {\em in-degre} of each vertex equals it {\em
  out-degree}). More precisely, he has proved that in every strongly connected
euleurian multigraph, any divisor of every dicycle length occurs as a period
\cite{PRI:94}. He has also shown that there is no polynomial  such that
the periods generated by the chip firing game on digraphs of order  are
bounded by . Readers interested by other results on periods and
transients of the chip firing game may refer to \cite{TAR:88,AND:89,BIT:89,BJO:91,ERI:91,GOL:93}.
 Readers interested by combinatorial games may refer to \cite{Gol:02, GM:02, Gol:04, Sjo:05, Fra:09}.

There is a particular case where the chip firing game is related to graph
orientations. Indeed, let us consider an undirected graph , and let
us
assume that initially, the edges of  can be oriented in such a way that the
number of chips of every vertex equals the {\em in-degree} of that vertex. If
this property is true in the initial configuration, then it remains true throughout the
game. One step of the game then consists in reversing the orientations of all
edges going into sinks. Goles and Prisner \cite{GOL:00} have studied {\em gardens
of Eden}, i.e. states that can appear only at time . They have also
studied the relationships between graph orientations and evolutions induced by
states with  chips. Moreover, Kiwi, Ndoundam, Tchuente and Goles
\cite{KIW:94} have exhibited cycles of exponential length 
generated by the chip firing game associated with the orientations of cascades
of rings. Other results on this particular case may be found in \cite{ERI:94}.

In this paper, we study the dynamics generated by the chip firing game
associated with - orientations. More precisely, using a
recurrent approach, we show that for , there exists cycles of even lengths from
  to  on  (), and of odd lengths different from  and ranging from  to
  on  ().

The remainder of this paper is organized as follows. In the next section, we
present some basic notations and definitions related to -. Section 3 is devoted to the recurrent
 construction of {\em left cyclic partitions} and possible period lengths whereas section 4 presents some concluding remarks.

\section{Basic notations and definitions}
An {\em n-dimensional hypercube} (or {\em -cube}) is an undirected graph
, where  is the set of vertices and two nodes
 and  are neighbors if and
only if they differ in only one bit in their binary representations, i.e.
there is an integer  such that  and  for . One can define recursively the {\em -cube} as follows :

 The -{\em cube} is reduced to one vertex ;

  is obtained by taking two copies of  and connecting
all pairs of equivalent vertices.\\
Fig. 1 illustrates this constructions for .
\begin{figure}[htbp]
\centering
\epsfxsize=8.6cm
\epsfbox{fig1.eps}
\caption{\label{figgraph} -{\em cubes} for .}
\end{figure}

Hereafter, given a set  and a boolean value , we denote . With this notation, we can write .

{\bf Definition 1.} A {\em block-sequential evolution} of the chip firing game
associated with graph orientations and played on an -{\em cube} is
obtained as follows. Consider a sequence of non empty subsets  of . At time , every vertex  of  is
considered. If  is a sink then the orientation of all its {\em in-going}
arcs are reversed, otherwise no action is undertaken. Hereafter, we say that a vertex {\em fires} at time 
if it belongs to  and is a {\em sink} at time .\vspace{2mm} \\

The {\em parallel evolution} is therefore a particular case of the general
scheme described above. Another classical evolution scheme is the {\em
  sequential evolution} where  is reduced to one vertex (i.e. )
and there is a permutation  of  such that  is periodic of period . Both parallel and sequential evolutions are particular cases of the
so-called {\em serial-parallel} evolutions where the sequence  is periodic of period , with the constraint that
   is a partition of .\vspace{2mm}\\

{\bf Definition 2.} A partition  of the
vertices of an -{\em cube} is called a {\em left cyclic} partition if the
two following statements hold.

 For all  from  to , every vertex of  has a
neighbor in , where index operations are performed modulo .

 For all  from  to , there is no edge between two
vertices of .\vspace{2mm}\\
{\bf Comment 2.} Canonical decompositions defined in \cite{GOL:00} for
acyclic digraphs are obtained from {\em left cyclic partitions} by orienting
the edges such that all arcs from the set  go to sets
 such that . On the other hand, {\em left cyclic partitions} are more
restrictive than the partitions introduced in \cite{PRI:94} since we do not
allow edges joining two vertices of the same subset. Indeed, in the chip
firing game associated with graph orientations, two neighbors cannot fire
simultaneously, whereas this situation is possible for the general chip firing
game. \\

We present an important property of left cyclic partitions on an -{\em cube}.

\begin{theorem}
If a partition  of the vertices of an -
  cube  is a left cyclic partition then there is a
  cyclic evolution of the chip firing game associated with graph orientations
  and played on , such that for every ,  is the set of
  vertices which are fired at time .
\end{theorem}
{\bf Proof.} Let  be a {\em left cyclic partition}. Consider
an orientation where every edge  such that  and , is oriented from  to . It is easily seen that in the parallel chip
firing game starting with such a configuration, the subsets of vertices
which fire at successive steps correspond to a periodic sequence of period
.\\


\section{Recurrent construction of left cyclic partitions}
In this section, we first present the construction of {\em left cyclic
  partitions} of even lengths.
\begin{lemma}   \label{lem:allpair}
An -cube admits left cyclic partitions of all even lengths from
 to .
\end{lemma}
{\bf Proof.} Let  be an -{\em cube} an let  be an even
integer between  and . It is well known that, since  is even,
there is a cycle 
of length  in . Now, for every vertex , let  denote the
set of all neighbors of  in . This notation is naturally extended to a
set of vertices. A {\em left cyclic partition} of order  is
obtained  as follows.

\ {\bf For}   {\bf do}

\ \ \ 

\ {\bf endfor}

\ 

\ {\bf while}  {\bf do}

\ \ \ {\bf For }  {\bf to }   {\bf do}

\ \ \ \ \hspace{2mm} 

\ \ \ \ \hspace{2mm} 

\ \ \ \ {\bf endfor}

\ {\bf endwhile}

It is obvious that  is a partition of  and that every
vertex in  has at least one neighbor in . So we just need to show
that two vertices of the same subset  cannot be neighbors. Let  and
 be two vertices of .

 There is a path from a to  of length  such that  {\bf  mod },

 There is a path from b to  of length  such that  {\bf  mod },\\
Since  is  even, it follows that  {\bf mod }. Hence, if
 and  were neighbors, there would exist a cyclic path of odd length
 joining  and  in , which is not possible
since  is a bipartite graph. This shows that two vertices of the same
subset cannot be neighbors.\\
.\\
The following figure displays the partition of order  in  obtained by
the previous procedure starting with the cycle .
\begin{figure}[htbp]
\centering
\epsfxsize=6cm
\epsfbox{fig2.eps}
\caption{\label{tracegraph} {\em Left cyclic partition} of orders  generated in .}
\end{figure}\\
Let us now turn to the construction of {\em left cyclic partitions} of odd
lengths.

\begin{lemma}  \label{lem:23}
If  is a {\em left cyclic partition} of , then every vertex
of  has at least two neighbors in  for .
\end{lemma}
{\bf Proof.} Because of symmetry considerations, we can assume that . So
let  be a vertex of . From the definition of {\em left cyclic
  partitions},

 
has a neighbor  in , where  is the {\sc xor}
operator and  is a

\ \ vector of the canonical basis.

 similarly,   has a neighbor
 in . \\
Now consider the vertex . 

 It is a neighbor of , hence it does not belong to .

 It is a neighbor of , hence it does not belong to .\\
It then follows that  belongs to , hence  admits two
neighbors  and  which are both in .\\
\\

\begin{lemma} \label{lem:3rec}
If ,  admits a {\em left cyclic partition} of order , then 
 admits a {\em left cyclic partition} of order .
\end{lemma}
{\bf Proof.} Let  be a {\em left cyclic partition} of order  of 
. Let  be a vertex of . We can assume without loss of generality
that . Since , 
 is the unique neighbor of  in . Consequently, from lemma \ref{lem:23},  admits a neighbor in . This shows that the
subgraph  which is isomorphic to , contains a {\em left
  cyclic partition} of order .\\



\begin{proposition}  \label{prop:pas3}
-cubes do not  admit left cyclic partitions of order 3. 
\end{proposition}
{\bf Proof.} An -{\em cube} with  has less than  vertices and
 cannot admit a {\em left cyclic partition} of order . On the other hand,
 from lemma \ref{lem:23}, if  is a {\em left cyclic partition} of
 an -{\em cube}, , then every  contains at least two elements
 (i.e. ). Consequently, the -{\em cube}  which is of
 cardinality  cannot admit a {\em left cyclic partition} of order . By application of lemma
 \ref{lem:3rec}, we deduce that no -{\em cube},  admits a {\em left
 cyclic partition} of order .\\
 
\\
Proposition \ref{prop:pas3} gives the lower bound for {\em left cyclic partitions} of odd lengths. 
Let now study the upper bound.
\begin{proposition}   \label{prop:ibound}
If  is a left cyclic partition of odd order  of , then 
. 
\end{proposition}
{\bf Proof.} We just have to show that in such a case,   
 for . Indeed, starting from a vertex , 
we construct a chain ,, , 
, , , ,  such 
that  for . It is clear that , otherwise 
we would have displayed a closed path of odd length in  which is not
possible.\\
 
\\
Now that we have established lower and upper bounds for {\em left cyclic
  partitions} of odd lengths, let us show that 
all intermediate lengths are admissible.
\begin{lemma}   \label{lem:rec}
If  admits a {\em left cyclic partition} of order , then  
admits left cyclic partition of order .
\end{lemma}
{\bf Proof.} If  is a left cyclic partition of order 
 in , then it is easily checked that  is
a {\em left cyclic 
partition} of order  in \\
 .

\begin{lemma}   \label{lem:suiv}
If  admits a {\em left cyclic partition} of odd order ,  then   
admits a {\em left cyclic partition} of order . Moreover, if , then 
 admits a left cyclic partition of order .
\end{lemma}
{\bf Proof.} Let  be a {\em left cyclic partition} of
odd order .

 Case \\
The following sequence is a {\em left cyclic partition} of order  in
.\\
, , , , , , 
,
 , , , ,..., , ,
 , , , ,\\ 

 Case \\
A {\em left cyclic partition} of order  in
 is obtained from the {\em left cyclic partition} exhibited in the
case  by replacing the subsequence 
, , , , , , ,  
by 
 , , , , , .\\


\begin{lemma}   \label{lem:57}
 admits left 
cyclic partitions of orders  and .
\end{lemma}
{\bf Proof.}\\
  A {\em left cyclic partition} of order  in  is the following
:\\
, , 
, ,\\ 
.\\
 A {\em left cyclic partition} of order  in  is the following
:\\
, , 
, , 
, \\, 
.\\
Fig. 3 displays the partitions. 
\begin{figure}[htbp]
\centering
\epsfxsize=8.6cm
\epsfbox{fig3.eps} 
\caption{\label{partgraph} {\em Left cyclic partitions} of orders  and  in .}
\end{figure}\\


\begin{lemma}  \label{lem:ipmax}
, , admits a {\em left 
cyclic} partition of order .
\end{lemma}
{\bf Proof.} Consider the sequence ,
defined by , where  is the
-position binary representation of the integer , and symbol  denotes integer
division. It can be easilly checked that this sequence corresponds to a {\em
  hamiltonian cycle} in . Now, let us denote  (i.e.  is obtained from  by changing the first and last
bits) and . It is also easy to check that  is a {\em hamiltonian cycle} of . Let us now consider the following
sets\\
\\
At this step, it is important to recall that two vertices referenced by 
and  are neighbors in the hypercube if and only if there is an integer 
such that . Observe that . Hence,  and  are
not neighbors in the hypercube  . On the other hand, ,
 ,  and . Hence, \\
\\
Moreover, ,  and
. Hence 
\\
Properties \ref{eqn:endc} and \ref{eqn:firstc} together with the fact that
 and  are both {\em hamiltonian cycles} of , imply that the
partition exhibited in \ref{eqn:bigleft} is a {\em left cyclic partition}. \\    
 \\
\begin{proposition}   \label{prop:ip}
, , admits left cyclic partitions of all odd orders 
from  to . 
\end{proposition}
{\bf Proof.} We proceed by induction on .
For  the result follows from lemma \ref{lem:57}.\\
Assuming that the result holds for , let us consider an -{\em cube} together with an odd integer .

 Case 1 : . The result follows from the 
induction hypothesis by application of lemma \ref{lem:rec}.

 Case 2 : . There is an odd integer , 
, such that  or  
. The result follows from the induction hypothesis by application of
lemma \ref{lem:suiv}.

 Case 3 : . The result follows from lemma \ref{lem:ipmax}.\\

\\

We are now ready to state the main theorem.

\begin{theorem}  \label{th:main}
There exists cycles generated by the parallel chip firing game associated with n-cube
orientations, , are of even lengths from  to , and of odd
lengths different from 3 and ranging from  to .
\end{theorem}
{\bf Proof.} We just need to show that this property holds for {\em left
  cyclic partitions} of vertices of {\em n-cubes}. The existence of {\em left cyclic partitions} of all even lengths
 from  to  follows from lemma \ref{lem:allpair}. Let us turn to odd periods . \\
 Case 1 : . Consider and orientation which contains a {\em
  hamiltonian cycle}. Clearly, such a configuration is a {\em fixed point} for
  the chip firing game associated with graph orientations. \\
 Case 2 : . The non existence of period  follows from proposition \ref{prop:pas3}. \\
 Case 3 : . The existence of this period
follows from proposition \ref{prop:ip}. \\


\section{Conclusion}
 We show in the particular case of {\em parallel evolutions} on -{\em cube}, 
 the existence of cycles of even lengths from  to , and of odd lengths different
 from  and ranging from  to . In case of {\em parallel evolutions} on 
 -{\em cube}, the existence of cycles of lengths greater than  remains an open question. \\
\\
{\bf Aknowledgement.} This work was supported by the French Agency
 {\em Aire d\'eveloppement} through the project {\em Calcul Parall\`ele} and
 by the Project NTIC of the University of Ngaound\'er\'e.

\bibliographystyle{fplain}

\begin{thebibliography}{AND 89}
\bibitem[AND 89]{AND:89} R.J. Anderson, L. Lowasz, P. W. Shor, J. Spencer, E. Tardos,
 S. Winograd,
 {\it Disks, balls and walls : analysis of a combinatorical game},
 American Mathematical Monthly, vol.96, pp. 481-493,
 1989.
\bibitem[BIT 89]{BIT:89} J. Bitar,
 {\it Juegos Combinatoricales en redos automatas},
 T\'esis de ingenieros Matematico, Fac de Cs. Fisicas y Matematicas, U. de
 Chile, Santiago, Chile
 1989.
\bibitem[BIT 92]{BIT:92} J. Bitar and E. Goles,
 {\it Parallel chip firing games on graphs},
 Theoretical Computer Science, 92, pp. 291-300,
 1992.
\bibitem[BJO 91]{BJO:91} A. Bjorner, L. Lovasz, P. W. Shor,
 {\it Chip firing game on graphs},
 European J. Combin. 12, pp. 283-291,
 1991 .
\bibitem[ERI 91]{ERI:91} K. Erikson,
 {\it No polynomial bound for the chip firing game on directed graphs},
Proc. Amer. Math. Soc. 112, pp. 1203-1205,
 1991.
\bibitem[ERI 94]{ERI:94} H. Erikson and K. Erikson,
 {\it Chip firng game and coxeter elements},
 Proceedings of FPSAC 94,
 1994.
\bibitem [Fra 09]{Fra:09} Aviezri S. Fraenkel,
 {\it Combinatorial Games: Selected Bibliography with a Succint Gourmet Introduction},
 The Electronic Journal of Combinatorics (2009).
\bibitem [GOL 93]{GOL:93} E. Goles, M. A. Kiwi,
 {\it Games on line graphs and sand piles},
 Theoretical Computer Science, 115, pp. 321-349,
 1993.
\bibitem[GOL 00]{GOL:00} E. Goles, Erich Prisner
 {\it Source Reversal and Chip Firing Game},
 Theoretical Computer Science, Volume 233, pages 287-295
 2000.
\bibitem[Gol 02]{Gol:02} E. Goles, M. Morvan, H. Duong Phan, {\it Sandpiles and order structure of integer partitions},
 Discrete Applied Mathematics, Volume 117, 2002, pages 51-54
\bibitem[GM 02]{GM:02} E. Goles, M. Morvan, H. Duong Phan, {\it The structure of a linear chip firing game and related models},
 Theoretical Computer Science, Volume 270, 2002, pages 827-841.
\bibitem[Gol 04]{Gol:04} E. Goles, M. Latapy, C. Magnien, M. Morvan, H. Duong Phan, {\it Sandpile models and lattices : a
 comprehensive survey}. Theoretical Computer Science, Volume 322, 2004, pages 383-407
\bibitem[KIW 94]{KIW:94} M. A. Kiwi, R. Ndoundam, M. Tchuente and E. Goles
 {\it No polynomial bound for the period of the parallel chip firing game on graphs},
 Theoretical Computer Science, 136, pp. 527-532,
 1994.
\bibitem[PRI 94]{PRI:94} E. Prisner,
 {\it Parallel chip firing on digraphs},
Complex Systems, No. 8, pp. 367-383,
 1994.
\bibitem[NDO 94]{NDO:94} R. Ndoundam and M. Tchuente,
 {\it Comportement dynamique d'un r\'eseau d'automates associ\'e aux
 orientations d'un graphe},
Acte du CARI'94, Ouagadougou, pp. 495-505,
 Octobre 1994.
\bibitem[NDO 95]{NDO:95} R. Ndoundam,
 {\it Analyse et Synth\`ese de certains reseaux d'automates},
Th\`ese de Doctorat de  cycle, Universit\'e de Yaound\'e I,
 1995.
\bibitem[Sjo 05]{Sjo:05} J. Sj\"{o}strand, {\it The cover pebbling theorem},
 The Electronic Journal of Combinatorics {\bf 12} (2005). 
\bibitem[SPE 86]{SPE:86} J. Spencer,
 {\it Balancing vectors in the max norm},
Combinatorica, 6, pp. 55-66,
 1986.
\bibitem[TAR 88]{TAR:88} G.Tardos,
 {\it Polynomial bound for the chip firing game on graphs},
SIAM Journal of Discrete Mathematics, 1, 3, pp. 397-398,
 1988.
\end{thebibliography}
\end{document}
