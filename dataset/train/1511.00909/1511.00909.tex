\documentclass{llncs}

\begin{document}

\title{Data Language Specification via Terminal Attribution}

\author{Alexander Sakharov, Timothy Sakharov}

\institute{}

\maketitle

\begin{abstract}
Unstructured data have to be parsed in order to become usable. The complexity of grammar notations and the difficulty of grammar debugging limit the use of parsers for data preprocessing. We introduce a notation in which grammars are defined by simply dividing terminals into predefined classes and then splitting elements of some classes into multiple layered sub-groups. These LL(1) grammars are designed for data languages. They simplify the task of developing data parsers.
\end{abstract}

\section{Introduction}
Most data are unstructured. Parsing unstructured data opens up opportunities for further processing, querying, and extracting knowledge from the data, as well as loading data into databases or publishing web pages. Many software professionals are involved in the development of data parsers without even realizing that the code they wrote is actually a hard-coded parser. Hard-coded parsing is a typical step in big data preprocessing. These hard-coded data parsers require software updates with every change in data format. The emergence of standardized data notations such as XML and JSON is essentially a response to the technical difficulties associated with parsing data, but the reality is that only a small percentage of data conform to these standards.

Context-free grammars (CFG) \cite{Aho06} are an excellent mechanism for specifying the syntax of programming languages, but they are rarely used in data preprocessing. Few software developers are familiar with CFGs. In most cases, it is still easier to hard-code an ad hoc data parser than to write and debug a CFG grammar. Understanding how to create a grammar for predictive top-down parsing or a grammar without conflicts for bottom-up parsing requires some deep understanding of the theory of parsing. Basically, CFGs are not for everyone but only for gurus in the domain of formal languages and compilers. Over the course of time, several alternative grammar notations were developed. With the exception of regular expressions, none of these alternatives really simplified the task of creating and debugging grammars. 

The output of parsing both programming and data languages is the same: it is an abstract syntax tree (AST) \cite{Aho06} which contains syntactic information extracted from the source. Nonetheless, there are some fundamental differences between parsing programming languages and parsing data. Data languages mostly consist of aggregation constructs and references. The former represent structures with named fields or sets including maps. i.e. key-value pairs. A grammar defining a programming language should be very constraining. It should disallow inappropriate strings of symbols because language semantics should be applicable to any well-formed program according to the grammar defining the language. 

The situation is drastically different with the parsing of data. The goal of parsing data is not to verify the conformance, but to build a rich parse tree, subsequently extract and tag pieces of information, and possibly derive relationships among these pieces based on the structure of the parse tree. Almost always, some portions of data have an incorrect format or somehow diverge from any given standard. Therefore, grammars for defining the syntax of data should be inclusive in order to avoid undesirable exceptions when processing these data. In contrast to programming languages, data formats are plentiful and evolve all the time. It is important, especially for big data, to be able to easily modify data grammars without the danger of compromising their properties. It is also important to be able to parse data using an incomplete grammar because the exact syntax of big data may not be known in advance.

An adequate notation for defining data languages should be much simpler than CFGs. It should be on par with regular expressions in terms of comprehensibility. Unfortunately, regular expressions themselves are not a good choice for defining data languages because of their limited expressiveness and because they do not help build informative parse trees. The use of such notation should not require sophisticated tools for parser generation, and parsing should be feasible in linear time. A notation that satisfies the above criteria was introduced in \cite{Sakharov15}. It is called tier grammars. This paper provides a more detailed description of tier grammar properties. 

Tier grammars have no nonterminals, no grammar productions, and no formulas. A language is defined by simply dividing terminals into predefined classes. Each class has its role. Some classes are split into multiple tiers, i.e. layered sub-groups. Note that the choice of terminal classes in this notation is not motivated by theoretical considerations but rather is driven by the intent to cover more constructs used in practice while having a clear meaning of every terminal class. Tier grammars define a subset of LL(1) languages, which makes predictive parsing possible \cite{Aho06}. Tier languages are unambiguous. They are devised to be very inclusive. We give a simple characterization of strings belonging to these languages. This notation is rich enough for specifying data formats of various kinds of documents including machine-generated documents such as log files. Our notation facilitates the definition of constructs representing data aggregates and references.  

\section{Definition of Tier Languages}

Following the tradition for programming languages, it is assumed that lexical analysis using regular expressions is done before parsing. The output of lexical analysis is a sequence of tokens whose names are terminals for parsing. As usual, the longest lexeme is selected in case of conflicts \cite{Aho06}. If the syntax is known for portions of the input, then regular expressions are also used to select these fragments before parsing them. 

Suppose the set of terminals T is a union of disjoint sets , , , , , , .  is the set of base terminals. Terminals from  and  define bracketed constructs. Terminals from  are opening brackets, and terminals from  are closing ones. Terminals from  are called markers. These terminals are split into groups by their priority. Their role is to serve as delimiters that combine items left and right to them in groups. 

Terminals from  are called postfixes. They are postfix operators. Terminals from  are called prefixes. They are unary prefix operators. Terminals from  are connectives that serve as binary operators in expressions or separators like in the comma-separated values format. Prefixes, postfixes, and connectives are also split into disjoint groups by their priority. They share the range of priorities but only one kind of terminal is allowed for a given priority. Let  be the highest priority for markers and  be the highest priority for postfixes, prefixes, and connectives. We use  to denote the number of distinct markers, postfixes, prefixes, or connectives of priority .

Tier language  for family of terminal classes , , , , , ,  is defined recursively by the following rules. Understanding these rules does not require any knowledge of CFGs, but tier languages can still be expressed via CFGs. We give CFG productions along with the rules in order to demonstrate how the rules map to them.  will denote the start nonterminal of the corresponding CFG. Symbol  will denote the empty string.  will denote respective terminals of priority . Note that only one of  may be non-empty for any .

1. If , then .

\noindent


2. If , then  .

\noindent


\noindent


\noindent


3. Let either  (connective),  (prefix), or  (postfix). If , ,  are defined by rules 1, 2, or this rule for terminals of higher priority,  is defined by rules 1, 2, or this rule for terminals of the same or higher priority, then , , or .

\noindent


postfix:

\noindent
 (for )

\noindent 


\noindent


prefix:

\noindent
 (for )

\noindent


connective:

\noindent
 (for )

\noindent


\noindent
 (for )

\noindent
 

4. If , , then , ,  provided that this string follows the beginning of the input string, or a terminal from  or a marker of lower priority, and precedes the end of the input string, or a terminal from  or a marker of lower priority.

\noindent
 (for )

\noindent


\noindent
 (for )

\noindent


\noindent


Now we only need to add one more production to complete the definition of the corresponding CFG: 
.
The above context-free productions have to be slightly modified when some terminal sets are empty. In case the sets of connectives, postfixes, and prefixes are all empty:
.
In case the set of markers is empty:
.

These terminal classes  are suitable for various representations of data aggregates and references: prefixes, postfixes, and connectives for named fields in structures; brackets and markers for structures, sets, and maps; connectives for key/value pairs and for separating set elements; prefixes and brackets for references. Rule applications define parse trees for tier languages. Applications of Rule 1 constitute the terminal nodes of these parse trees. Every application of all other rules corresponds to a nonterminal node of the parse tree.


\section{Predictive Parsing}

Let  and  denote strings of terminals and nonterminals. A CFG is LL(1) if and only if the following holds for every two productions ,  \cite{Aho06}:

\noindent
1.  and  never derive strings beginning with the same terminal

\noindent
2. At most one of  and  can derive the empty string

\noindent
3. If  can derive the empty string, then  does not derive any string beginning with a terminal from 

We use symbol \T_{4i}T_{5i}T_{6i}T_{7i}E_iFIRST(E_i) = \{ T_1, T_2, T_{6i},...,T_{6k} \}FIRSTFOLLOWG_i \rightarrow \epsilon | s_{i1} | ... | s_{i\overline i} FOLLOW(G_i) = \{ T_{71},...,T_{7i-1}, T_{41},...,T_{4q}, T_1, T_2, T_3,T_{61},...,T_{6k}, T_{51},...,T_{5i-1},\

2.  (or )

\noindent
   \} R_i \rightarrow \epsilon | m_{i1} Q_{i+1} R_i | ... | m_{i\overline i} Q_{i+1} R_iR_q \rightarrow \epsilon | m_{q1} D R_q | ... | m_{q\overline q} D R_qFOLLOW(R_i) = \{ T_3, T_{41},...,T_{4i-1}, \

4. 

\noindent


\noindent
 \}FIRSTFOLLOWABE_iE_i \rightarrow E_{i+1} G_iG_iE_i \rightarrow p_{ij} E_iE_i \rightarrow E_{i+1} L_iL_iQ_iQ_i \rightarrow Q_{i+1} R_iR_iC, F, H, G_i ,L_i, R_iP_1P_2P_1P_2P_1P_2S \Rightarrow^* NNEventHandlerT_1, T_2, T_3(ab)^*R := ( b_1 | ... | b_{\underline b} | ( r_1 | ... | r_{\underline r} ) S ( e_1 | ... | e_{\underline e} ) )i=k,...,1R := ( R ( \epsilon | s_{i1} | ... | s_{i\underline i} ) )R := ( ( p_{i1} | ... | p_{i\underline i} )^* R )R := ( R ( ( c_{i1} | ... | c_{i\underline i} ) R )^* )R := R^*i=q,...,1R := ( R ( ( m_{i1} | ... | m_{i\underline i} ) R )^* )B \rightarrow F S HBCA \rightarrow t_{11}ST_2T_3LASTFIRSTFIRST(E_i) = \{ T_1, T_2, T_{6i},...,T_{6k}\}LAST(E_i) = \{ T_1, T_3, T_{5i},...,T_{5k} \}iE_{i+1}E_iiE_iiE_{i+1}tktbtS \Rightarrow^* ... E_k ... \Rightarrow ... C G_k ... \Rightarrow ... C ... \Rightarrow^* ... b ...S \Rightarrow^* ... E_k ... \Rightarrow ... C G_k ... \Rightarrow ... C t ... \Rightarrow^* ... b t ...tbpttS \Rightarrow^* ... E_k ... \Rightarrow ... C ... \Rightarrow^* ... b ...S \Rightarrow^* ... E_k ... \Rightarrow^* ... t C ... \Rightarrow^* ... t b ...tb_1b_2tb_2S \Rightarrow^* ... E_k ... \Rightarrow ... C L_k ... \Rightarrow ... C ... \Rightarrow^* ... b_1 ...S \Rightarrow^* ... E_k ... \Rightarrow ... C L_k ... \Rightarrow^* ... b_1 L_k ...S \Rightarrow^* ... E_k ... \Rightarrow ... C L_k ... \Rightarrow ... C t C L_k ... \Rightarrow^* ... b_1 t b_2 ...S \Rightarrow^* ... E_k ... \Rightarrow ... C L_k ... \Rightarrow ... C t C L_k ... \Rightarrow^* ... b_1 t b_2 L_k ...mS \Rightarrow^* ... Q_q ... \Rightarrow^* ... D ... \Rightarrow^* ... E_1 ... E_1 ...S \Rightarrow^* ... Q_q ... \Rightarrow ... D R_q ... \Rightarrow^* E_1 ... E_1 R_qE_1E_1E_1E_1E_1S \Rightarrow^* ... Q_q ... \Rightarrow ... D R_q ... \Rightarrow^* ... E_1 ... E_1 m E_1 ... E_1S \Rightarrow^* ... Q_q ... \Rightarrow ... D R_q ... \Rightarrow^* ... E_1 ... E_1 m E_1 ... E_1 R_q ...p_1p_2bS \Rightarrow^* ... C ... \Rightarrow ... A ... \Rightarrow ... b ...uS \Rightarrow^* uS \Rightarrow^* ... C ... \Rightarrow^* ... B ... \Rightarrow^* ... p_1 S p_2 ...T'_2T'_3T'_4T'_5T'_6T'_7T_2T_3T_4T_5T_6T_7s \in \Lambda(\{T_1T_2T_3T_4T_5T_6T_7\})T_2 \setminus T'_2T_3 \setminus T'_3ss \in \Lambda(\{T_1 \cup T'_2 \cup T'_3 \cup T'_4 \cup T'_5 \cup T'_6 \cup T'_7, T_2 \setminus T'_2, T_3 \setminus T'_3, T_4 \setminus T'_4, T_5 \setminus T'_5, T_6 \setminus T'_6, T_7 \setminus T'_7\})E_i \rightarrow p_i E_i ... E_iE_ip_i$. Also, multiple tier grammars can be combined so that every source grammar applies only to a relevant portion of a document. The advantage of combining multiple tier grammars vs CFGs is that the simplicity of the notation is not compromised. These extended and combined grammars remain LL(1). For more information about extending and combining tier grammars, see \cite{Sakharov15}. Tier grammars can be further generalized, and a description of such generalization will be published separately. 


\bibliographystyle{splncs}
{\small
\bibliography{TierGrammarsArxiv}}


\end{document}
