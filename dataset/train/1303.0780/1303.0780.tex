\documentclass[11pt]{article}

\usepackage{a4wide}
\addtolength{\topmargin}{-12mm}

\usepackage{xspace}

\usepackage[all]{xy}
\usepackage{amssymb}
\usepackage{amsmath}
\usepackage{graphics}
\usepackage{wrapfig}
\sloppy




\hyphenation{bi-si-mu-la-tion}



\newtheorem{theorem}{Theorem}[section]
\newtheorem{conjecture}[theorem]{Conjecture}
\newtheorem{corollary}[theorem]{Corollary}
\newtheorem{proposition}[theorem]{Proposition}
\newtheorem{lemma}[theorem]{Lemma}
\newtheorem{definition}[theorem]{Definition}



\newcommand{\push}{\textsc{push}\xspace}
\newcommand{\pop}{\textsc{pop}\xspace}
\newcommand{\pda}{\text{PDA}\xspace}
\newcommand{\dpda}{\text{DPDA}\xspace}



\newcommand{\todo}[1]
{\marginpar{\baselineskip0ex\rule{2,5cm}{0.5pt}\5mm]}

\author{Petr Jan\v{c}ar \3mm] 
{\normalsize Department of Computer Science}\\
{\normalsize Aalborg University}\\ 
{\normalsize Denmark}\\
\texttt{srba@cs.aau.dk}}

\date{ }
\begin{document}

\maketitle

\vspace{-6mm}
\begin{abstract}
\noindent
Broadbent and G\"oller (FSTTCS 2012)
proved the undecidability of bisimulation equivalence for
processes generated by
-free second-order pushdown automata.
We add a few remarks concerning the
used proof technique, called Defender's forcing, 
and the related undecidability proof for  
first-order pushdown automata with -transitions 
(Jan\v{c}ar and Srba, JACM 2008).
\end{abstract}\vspace{3mm}

\noindent
Language equivalence of pushdown automata (\pda)
is a well-known problem in computer science community.
There are standard textbook proofs showing
the undecidability even for \emph{-free}
\pda, i.e. for \pda that have no -transitions;
such \pda are sometimes called \emph{real-time} \pda.
The decidability question of language equivalence 
for \emph{deterministic} \pda (\dpda)
was a famous
long-standing open problem. It was positively answered by 
Oyamaguchi~\cite{DBLP:journals/jacm/Oyamaguchi87}
for -free \dpda and later by 
S\'enizergues~\cite{Senizergues:TCS2001} for the whole class 
of \dpda.


Besides their role of language acceptors, 
\pda can be also viewed as generators of (infinite) labelled transition
systems; in this context it is natural to study another fundamental
equivalence, namely \emph{bisimulation equivalence}, also called
\emph{bisimilarity}. This equivalence
is finer than language equivalence, but the 
two equivalences in principle coincide on deterministic systems.

S\'enizergues~\cite{Senizergues:SIAM:05} 
showed an involved proof of the
decidability of bisimilarity for (nondeterministic)
-free \pda but also for \pda in which 
-transitions are deterministic, popping and 
do not collide with ordinary input -transitions.
It turned out that 
a small relaxation, allowing for nondeterministic
popping -transitions, already leads to 
undecidability~\cite{DBLP:journals/jacm/JancarS08}.
In~\cite{DBLP:journals/jacm/JancarS08}, the authors of this note also
explicitly describe
a general proof technique called \emph{Defender's forcing}. 
It is a simple, yet powerful, idea related to the bisimulation game
played between Attacker and Defender; it was used, sometimes
implicitly, also in context of other hardness results
for bisimilarity on various classes of infinite state systems.


The classical PDA, to which we have been referring so far, are
the \emph{first-order} \pda in the hierarchy of 
\emph{higher-order} \pda that were introduced 
in connection with higher-order recursion
schemes already in 1970s.
The decidability question for equivalence of deterministic
th-order \pda,
where ,
seems to be open so far. A step towards a solution was made by
Stirling~\cite{DBLP:conf/concur/Stirling06} who 
showed
the decidability for a subclass of 
-free
second-order \dpda.




Recently 
Broadbent and G{\"o}ller~\cite{DBLP:conf/fsttcs/BroadbentG12} 
noted that the results in~\cite{DBLP:journals/jacm/JancarS08}, or
anywhere else in the literature, do not answer the decidability question for
bisimilarity of \emph{-free} second-order \pda.
They used the above mentioned technique of Defender's
forcing to show that this problem is also undecidable.
This result helps to further 
clarify the (un)decidability border, now in another
direction:
a mild use of second-order operations 
(on a stack of stacks)
is sufficient 
to establish undecidability without using
-transitions (that are necessary in the first-order
undecidability proof~\cite{DBLP:journals/jacm/JancarS08}).


The authors of~\cite{DBLP:conf/fsttcs/BroadbentG12} concentrate on
giving a complete self-contained technical construction yielding the
undecidability proof, however, they do not discuss in detail its relation to the
constructions in~\cite{DBLP:journals/jacm/JancarS08}. 
Here, in Section~\ref{sec:undecepsilon},
we try to concisely  present the idea of the relevant first-order proof 
from~\cite{DBLP:journals/jacm/JancarS08}, 
and then, in Section~\ref{sec:undecsecondorder}, we highlight the
idea in~\cite{DBLP:conf/fsttcs/BroadbentG12} that makes it 
possible to replace
the use of -transitions 
in the undecidability proof
with second-order operations.


We hope that this note may help to popularize 
the Defender's forcing technique,
and that it might be found useful by other researchers
tackling further open problems in the area. 


\section{Definitions}\label{sec:definitions}

A \emph{labelled transition system} (LTS) is a 
(possibly infinite) directed multigraph with action-labelled
edges. By a triple , called a \emph{transition}, 
or an \emph{-transition},
we denote
that there is 
an edge from node  to node  labelled with \,;
we also refer to the nodes as
to the \emph{states}.
A \emph{symmetric} binary relation  on the set of states
is a \emph{bisimulation} if for any  and any 
transition  there is a transition  
(with the same label )
such that . Two states  and  are
\emph{bisimilar}, written , if there is a bisimulation
containing .

Bisimilarity is often presented
in terms of a two-player game
between Attacker (he) and Defender (she).
In the current game position, that is a pair of states  
in an LTS,
Attacker chooses a transition  (for )
and Defender then chooses a transition
; 
the pair  becomes the new current position.
If one player gets stuck then the other player wins;
an infinite play is a win of Defender. It is easy to verify
that  are
bisimilar iff Defender has a winning strategy when starting from
the position .

An \emph{-free second-order pushdown system}
is a tuple  consisting of
four finite nonempty sets:
 contains the
\emph{control states}, 
 the \emph{stack symbols},
 the
\emph{actions} (corresponding to classical input letters),
and  the \emph{rules} of the following three types:

where  , 
,
, and .
The LTS generated by 
has the set  as the set of states; 
a state is written in the form 
where  is a control state and 
 is a nonempty sequence of stack symbols (for ).
By  we denote the empty sequence; hence
 when .
The transitions in the generated LTS are induced by the rules from  as
follows:
\begin{itemize}
\item
the rule  implies 
 if
,
\\
and   if \,;
\item the rule
 implies 
\,;
\item the rule
 implies 
\,.
\end{itemize}
We remark that the definitions of second-order pushdown systems in the
literature vary in details that are insignificant for us.
If we restrict the rules to the type  then we get 
\emph{-free first-order pushdown systems}.
In this paper we do not introduce -rules 
(of the types~(\ref{eq:rules}) with );
their restricted use in our paper is handled by a remark at the
respective place. 


\section{Undecidability of bisimilarity for \pda with
-transitions}\label{sec:undecepsilon}

In this section, we briefly explain a result
from~\cite{DBLP:journals/jacm/JancarS08}, namely the undecidability of
bisimilarity for (normal, i.e. first-order) pushdown systems with 
popping -rules (of the type ). 
The text closely follows the beginning of Section 5.1
from~\cite{DBLP:journals/jacm/JancarS08}, though it is 
a bit modified, concentrating on illustrating the ideas.


The undecidability result is 
achieved by a reduction from the following variant of
Post's Correspondence Problem (PCP). 
As usual, by a \emph{word}  \emph{over an alphabet} we
mean a finite sequence of letters;  denotes the length of .

\begin{definition}
A \emph{PCP-instance} INST
is a nonempty sequence
 of pairs of nonempty words
over the alphabet  where
 for all .
An \emph{infinite initial solution} of INST, a \emph{solution} of INST for short,
is an infinite sequence of
indices  from the set 
such that  and the infinite words
 and

are equal. A finite sequence  is a
\emph{partial solution} of INST if  and

is a prefix of .
\\
The problem \emph{inf-PCP} asks if there is 
a solution
for a given INST.
\end{definition}

The next proposition can be shown by standard arguments, related to
simulations of \emph{nonterminating} Turing machine computations;
the respective reduction easily guarantees our 
technical condition  
(see also \cite{DBLP:journals/jacm/JancarS08}). 


\begin{proposition}
\label{prop:infandrecPCP}
Problem inf-PCP is undecidable; more precisely, 
inf-PCP is -complete.
\end{proposition}

We now consider
a fixed instance INST of inf-PCP,
i.e. 
as above.
Let us imagine the following game, played between Attacker (he)
and  Defender (she); this game is more abstract, it will be only later
implemented as the bisimulation game.

Starting with the one-element sequence
, where , Attacker repeatedly asks Defender to prolong 
the current sequence 
 by one
 (of her choice),
to get . 
(We use prolongations to the left, to ease the later implementation by
a pushdown system.)
Attacker can thus ask Defender indefinitely, 
in which case the play is
a win for Defender, or he can
eventually decide to
switch to checking whether the current sequence
represents a partial solution, i.e., 
whether  is a prefix of
; the negative case is a win for
Attacker, the positive case is a win for Defender. 
In another formulation, 
the checking phase finds out
whether
  is equal to 
 a suffix
of 
 , where 
  denotes the reverse of .
It is obvious that 


With an eye to the later implementation of the game by pushdown rules,
we formulate an intermediate version of the game as follows.
(In fact, this intermediate game replaces the arguments 
given in~\cite{DBLP:journals/jacm/JancarS08} to justify
the rules of Fig.~\ref{fig:rulesystem}.)

\begin{itemize}
\item
(\emph{Generating phase}) 
\\
The game starts with a pair  where 
and  are auxiliary symbols that we can call ``control
states''.
Attacker repeatedly asks Defender to prolong 
both sequences in the current pair 
 
by some ,
thus creating the next current pair 
. 
\item
(\emph{Switching phase}) 
\\
For any current pair 

Attacker can decide to switch (to the verification): 
the control state in the left-hand sequence changes to ;
in the right-hand side sequence 
the control state changes to  but before that
Defender can erase a chosen prefix 
 and replace 
 with a suffix  of ; we thus get

\item
(\emph{Verification phase})
\\
Here the play is completely determined, verifying (step by step)
that 
  is equal
to
 .
The control state  signals that  is interpreted as 
, and  signals that  is interpreted as 
.
If a mismatch is encountered then Attacker wins, otherwise Defender wins.
\end{itemize}
Property~(\ref{eq:soliffwin}) obviously holds  
for the above (intermediate) game as well.
We now show that 
this game is implemented as the bisimulation game 
in the LTS generated by the pushdown system 
in Fig.~\ref{fig:rulesystem}, 
starting in the position .
We use the symbol  instead of ; 
the ``bottom-of-the-stack'' symbol
 is used for technical reasons.



\begin{figure}[ht]


\begin{tabular}{lllll}
(G1) rules: \hspace{1mm} &  \\  
\vspace{2mm}
&\fbox{} \hspace{5mm} &   
\\
& &
 \\
&& \fbox{} & where 
\end{tabular}




\begin{tabular}{llll}
(S1) rules: \hspace{1mm} & \\
&\fbox{} \hspace{3mm} &
 \hspace{-2mm} &
for all suffixes  of \\
\end{tabular}




\medskip

\begin{tabular}{llll}
(V1) rules: \hspace{1mm} & 
 
 \hspace{4mm} &  
\\
&  &   & \hspace*{2cm} \\
& &   \\
\end{tabular} 


\begin{quote}
\emph{Notation.} 
A rule  replaces the family 
 
for all stack symbols .
By  we denote the first symbol of ;
 is the rest of , and thus 
. 
By  (head-action) we mean 
if , and  if .
Subscripts  range over ;
thus the rule  stands for the  rules
, , , ,
the rule
, , stands for 
rules like , ,
etc. (Rules with  in (S1) are explained in the text.) 
\end{quote}

\caption{Rules from~\cite{DBLP:journals/jacm/JancarS08}, 
showing undecidability in the first-order 
case}\label{fig:rulesystem}
\end{figure}







Any position  in the bisimulation game
is trivially winning for Defender. 
To avoid this ``equality-win'', when
starting from the position , 
Attacker obviously must not use the framed rule  
(for any ), nor  which would
allow Defender to choose the framed rule to install equality. 
The frames just highlight
the use of Defender's forcing;
the rules are constructed so that Attacker must ensure
that neither him nor Defender ever uses a framed rule.

In the first round of the game, Attacker is thus forced
to use either  ( for ``generating'')
or  ( for ``switching'').
In the first case Defender uses
 for some (freely chosen) 
;
the current position becomes .
Attacker is now forced to use 
or , since using an 
-transition for  allows Defender to install equality.
After Defender's response,
the current position is 
 where  has been
chosen by Defender. 
We can thus see that the rules (G1) implement the generating phase. 
As long as Attacker chooses , the play goes
through longer and longer pairs

Since any infinite play is a win of Defender,
Attacker needs to enter the switching phase eventually, by using 
 from (S1).
The rules , 
 constitute the only place where
-transitions enter the stage.
These rules stand for the 
following family of rules
given in (S1-) in~\cite{DBLP:journals/jacm/JancarS08} 
(where  ranges over ):

We note that the last rule 

could be made 
-popping by remembering  
in the control state but we prefer the given form for simplicity.
The -rules generate 
-transitions
in the respective \emph{fine-grained} LTS. 
Nevertheless 
we refer to the -free LTS where 
 iff 
in the fine-grained LTS.

It is thus clear that Attacker is indeed forced to start 
the switching phase by choosing the rule 
and performing the transition 
.
Defender then chooses  and ,
and the corresponding transition

(where  is a suffix of 
). 
The next current pair thus becomes

Rules~(\ref{eq:familyepsilonrules}) also allow us to choose
;
but once we understand the verification phase, 
we can easily check that this is of no help for Defender.
The verification phase is implemented by the rules (V1).
Defender can no longer threaten with installing equality but this is
not needed anymore; this phase is completely determined,
giving no real choice to any of the players.  
It is obvious that Defender  wins iff

Since the described
bisimulation game closely mimicks our previous (intermediate) game,
it is easy to check that it also has Property~(\ref{eq:soliffwin}).








\section{Second-Order Pushdown Systems}\label{sec:undecsecondorder}

The ``first-order'' proof in Section~\ref{sec:undecepsilon}
(captured by the rules in Fig.~\ref{fig:rulesystem})
trivially shows the undecidability of bisimilarity for second-order
pushdown systems when 
-transitions are allowed.
When we explore the decidability question for
\emph{-free}
second-order pushdown systems then
it is natural to ask whether
we can implement the switching phase
(captured by (S1))
without using -rules, when
we have second-order \push and \pop at
our disposal.
So in terms of our intermediate game, we want to implement 
the switching from~(\ref{eq:switchstart}) 
to~(\ref{eq:switchend}). 
Without -transitions  we cannot implement erasing 
a prefix of  
(in the right-hand side string) in one move.
A natural idea is to shorten the right-hand side string 
step-by-step while Defender should 
decide when to finish. But it is not clear how 
to implement this in the ``first-order'' 
bisimulation game since Defender 
loses the possibility
of threatening with equality during such a step-by-step process. 
(S\'enizergues's decidability 
result~\cite{Senizergues:SIAM:05} shows that 
such an implementation is
indeed impossible in the first-order case.)

The idea (i.e., the crucial point in the undecidability proof 
in~\cite{DBLP:conf/fsttcs/BroadbentG12}) can be explained as follows.
When Attacker wants to switch at the position

then the stacks are doubled (using \push),
and the next position becomes

Now the top stacks are being synchronously shortened, the play 
going through positions

for decreasing .
During this process
Defender can threaten with equality, so it is
possible to implement that it is Defender who decides when the process should
stop, forcing 
 on the left-hand side (with entering )
and choosing a suffix  of  on the 
right-hand side; the reached position is then

The bottom stack on the right-hand side is now superfluous;
it only served for the previous threatening
with equality.
The verification phase is the same as previously
(with no choice for any player).
Implementing the described switching via the second-order rules 
is now a routine,
once we understand the Defender's forcing technique.
We just replace the rules (S1) in Fig.~\ref{fig:rulesystem}
with (S1-) in Fig.~\ref{fig:modifrules}
(where  ranges over ).

\begin{figure}[ht]


\begin{tabular}{llll}
\vspace{1mm}
(S1-) rules:  &  
 & & \\ 
&  &  & 
, 
\\
\vspace{1mm}
& \fbox{, }
\\ 
&  &&  \\
\vspace{1mm}
& && \fbox{} \\  
&  &&  \\
& && \fbox{} \\ 
\vspace{1mm}
&  \\   
& \\
& \fbox{} 
& &  \hspace{3mm}
 (for each suffix  of )\\ \\
\end{tabular}
\caption{A replacement of (S1) to show undecidability
for -free second-order \pda}\label{fig:modifrules}
\end{figure}
\noindent
Now if Attacker chooses to switch (by action ) then
a position corresponding to (\ref{eq:doubled}) is reached
(where  is replaced with , and  is added).
By Defender's forcing, Attacker must now use
the rule  and Defender decides whether
to enter the control state  (meaning that she wishes to erase
a further symbol  
from the top stacks, by the rules 
and ) or the control state  (meaning
that she wishes to enter the verification phase,
by the rules  
and ). In the next
round Attacker must follow these choices otherwise he will lose
(by the framed rules). 
The rule  forces 
Defender to choose the second option (entering ) 
before the top stacks are emptied (otherwise she loses).
Finally, once a position 
\begin{center}

\end{center}
is reached,
the last application of Defender's forcing
results in an analogue of~(\ref{eq:check}):
\begin{center}
\,.
\end{center}


\subsection{Normedness}

Bisimilarity problems like those we discuss here
are
often simpler when restricted to \emph{normed systems};
in our case, a \emph{state}  in the LTS generated by  
a pushdown system is  \emph{normed} if from each
state that is reachable from  we can reach 
a state where the stack is empty.
But restricting to the normed case does not affect the undecidability
here.
The states ,  in the system 
defined by
Fig.~\ref{fig:rulesystem} are normed, if 
we view the states ,  as having the empty stack;
otherwise we can add
the rules
,
.
(In~\cite{DBLP:journals/jacm/JancarS08}, there are used
the rules ,
 in the context 
of prefix-rewrite system definition.)

The authors of~\cite{DBLP:conf/fsttcs/BroadbentG12} 
are also interested in normedness for higher-order \pda as a natural extension of normedness for
first-order \pda.
We can note that after replacing (S1) in 
Fig.~\ref{fig:rulesystem}
with (S1-)
in Fig.~\ref{fig:modifrules}, the system is
not normed anymore 
(exemplified by the state
).
In~\cite{DBLP:conf/fsttcs/BroadbentG12} a triple
copy of the stack 
is used to handle the specific normedness definition there.
Another possibility is to start from

and add a new control state  and
the rules  and
 for all control states 
where , 
including the rule .

\subsection*{Additional comments}

As already mentioned, the undecidability result 
for -free 
second-order pushdown systems 
in~\cite{DBLP:conf/fsttcs/BroadbentG12}
clarifies the (un)decidability border in another direction 
than the undecidability result 
for first-order pushdown systems with -transitions
in~\cite{DBLP:journals/jacm/JancarS08}.
The border can be surely explored further. 
For example it seems that we cannot avoid 
using several control states in the above undecidability proofs
(though we can surely decrease their number 
by extending the stack alphabet).
Hence (normed) second-order simple grammars, 
studied in~\cite{DBLP:conf/concur/Stirling06}, 
are a possible target for exploring.







\bibliographystyle{abbrv}
\bibliography{concur}


\end{document}
