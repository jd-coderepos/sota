
This paper is about employing context-sensitive rewriting to model outermost rewriting.
We do so by marking redexes, and forbid rewriting below them.
As we have seen, contracting a redex may create another redex higher up in the term tree.
Hence it may be necessary to update some labels during a rewrite step.
In Section~\ref{sec:cxtext} we defined a transformation where this updating 
was accounted for by extending rules with contexts.
Here we give an alternative transformation from TRSs to context-sensitive TRS{s}.
We call this tranformation `dynamic labeling'.
Instead of extending rules with contexts, 
we now employ rewriting to propagate the changed information upward in the term tree,
and set the labels in the surrounding context right, step by step.
Again the \cdepth{} (Definition~\ref{def:cmodel}) serves as a bound:
here on the number of successive ancestor nodes that have to be relabeled.
Each original rewrite step will give rise to a corresponding step and a bounded number ( the \cdepth{}) 
of auxiliary steps in the transformed system. 
Thus, although the derivational complexity (the length of rewrite sequences) 
is changed, this is only by a constant factor.
We prove that dynamic labeling is sound for arbitrary TRSs.
Moreover, for left-linear TRSs, the method is complete in a weakened sense, 
see Theorem~\ref{thm:dynlab:complete}.
In Section~\ref{sec:evaluation}, we compare the performance of this method 
to the one of dynamic context extension described in Section~\ref{sec:cxtext}.

We begin with an analysis for evaluating which value changes can occur by rewriting and need to be propagated upward.
As we will see, this restricts the number of auxiliary `relabel symbols', and, in particular,
the number of `relabeling rules'.

\begin{definition}\label{def:value-change-pairs}
  Let  be a TRS over , 
  and let  be a \clabeling\ for .
  For \,, we define the set  inductively by:
  
  Then we define the set  of \emph{value-change pairs} by:
  
  with \, the least number
  such that .
\end{definition}

The `dynamic labeling'  of a TRS~\, is partitioned into two sets of rules.
The first set is denoted by 
and consists of a semantic labeling of the original rules,
where, additionally, a right-hand side is prefixed by a symbol  
whenever application of the rule causes a change of interpretation from  to .
The second set, , 
is a set of rules for relabeling the context of the rule application.
A symbol , with ,
indicates that the value of its subterm has changed from  to ,
and the rules in  
take care of propagating this change of value upward in the term.

\begin{definition}[Dynamic labeling]\label{def:dynlab}
  Let  be a TRS over , 
  and let  be a \clabeling\ for .
  The TRS  over the signature 
  
is defined by 
  . 
  Here the set  of \emph{labeled rules} contains,   
  for each rule  and assignment ,
  one of the rules:
  
Secondly, the set  of 
  \emph{relabeling rules} contains,
  for each -ary , , 
  and  
such that 
   
  with ,
  one of the rules:
  
  where 
  ,
  ,
  ,
  , 
  and .

  The \emph{dynamic labeling of  (with respect to the \clabeling~)}
  is the context-sensitive TRS~,
  where the replacement map  is defined by 
   for all ,
   if ,
  and  otherwise,
  for all .
Whenever  is clear from the context, we leave  implicit,
  and overload the notation  to denote 
  .
\end{definition}


\begin{example}
  We revisit the TRS~ from Example~\ref{ex:fffx:cmodel} 
  for which we worked out the static and dynamic context extensions 
  in Examples~\ref{ex:fffx:static:cxtext} and~\ref{ex:fffx:dyn:cxtext}.
  We repeat its definition and the \clabeling{} from Example~\ref{ex:fffx:clabeling}:  is the TRS over  consisting of the rules:
   
  A \cmodel{} for  is formed by the -algebra 
  
  with interpretation:
  
  for all .
  Furthermore,  denotes the maximal labeling for , 
  and .
Then 
  forms a sound and complete \clabeling\ of .
Also note that  forms a core algebra;
  for each value  there is a ground term  such that 
  .
We first compute the set  of value-change pairs. 
  For the initial set , note that the rule 
   
  changes the interpretation from  to , regardless of the value assigned to .
The other rule creates three value-change pairs; one for each of the values  assigned to . 
  If the interpretation of  is  there is no change.
  Hence we get:
  
All symbols  with  
  will disappear in one relabeling step, whence .
The dynamic labeling of  then is
  
  where  consists of the rules:
  
  and where  is formed by:
  
  In total the dynamic labeling of  has 13 rules. 
Had we not restricted the construction of the set of the relabeling rules to 
  the `reachable' symbols  
  (by the requirement  in Definition~\ref{def:dynlab}),
  we would have come up with 18 instead of 5 relabeling rules.
\end{example}

\begin{example}
  We reconsider the term rewrite system~ from Example~\ref{ex:dupl_rhs}:
  
  and the algebra  
  with  defined, for all , as follows:
  
  Moreover, we employ minimal labeling again; see Example~\ref{ex:dupl_rhs}. 

  The set of change-value pairs is:
  
  The set  of labeled rules is constructed thus: 
  There are four rules with left-hand side ,
  one for each value assigned to~. 
  In case  there is no change of interpretation, 
  for we have that  for all 
  and so no  symbol is inserted.
  But if, for instance,  for some substitution ,
  then some labels in the context~ of a rewrite step 
  
  have to be updated, since the value of  has changed from  to ,
  whence the insertion of  to the right-hand side .

  The set  
  of relabeling rules is formed by:
  
  Some remarks for clarification: 
  First, note that all  symbols disappear upon one relabeling step.
  Secondly, observe the overlap in, for example, the rules with left-hand side 
  .
  If the value assigned to  is , 
  then a redex is created; this is witnessed by the marked symbol  on the right.
  For other values for , this is not the case.
  Also note that there is no rule for
  . 
  This is because when the left argument of  is interpreted as ,
  then  forms a redex, and so  should be marked. 
  Definition~\ref{def:dynlab} does not allow  symbols 
  to commute with redex symbols.
  Intuitively, a  symbol is a witness of a rewrite step 
  which we do not want to occur inside other redexes, as we want to model outermost terminination. 
  However, more technically, sometimes illegal (i.e., non-outermost)  steps are allowed.
  This is illustrated in Example~\ref{ex:dyncomplete}.
  The point is that by preventing  symbols to commute with redex symbols,
  for local completeness (Theorem~\ref{thm:dynlab:complete})
  it is as if illegal steps never happened.
\end{example}

\begin{remark}
\newcommand{\ssrelabelto}{\ssrelabel}\newcommand{\srelabelto}{\super{\ssrelabelto}}\newcommand{\relabelto}[1]{\funap{\srelabelto{#1}}}\newcommand{\fgfxnr}{8}\newcommand{\trsfgfx}{\iatrs{\fgfxnr}}\newcommand{\algfgfx}{\iaalg{\fgfxnr}}We elaborate on the role of the element~ in .
  Whenever the application of a rule 
   
  changes the interpretation,
  i.e., ,
  then a symbol  is inserted.
  A term of the form  can be thought of as a witness 
  of a rewrite step  
  causing a change of interpretation from  to .
  This change of the value then needs to be propagated upward to update the labels accordingly,
  using the relabeling rules from .
  At first sight, the value  in  seems redundant for relabeling:
  why would we store the previous value? 
  However, the label  is important in order to restrict the number of applicable rules,
  and to have a bound on the number of relabeling steps. 
  To see this, consider the system~ with single rewrite rule:

  and the algebra  with
   for all ,
  ,  for all ,
  and .
  We employ minimal labeling, that is, 
  only \,,
  and all the other symbols are unlabeled.
  
  The dynamic labeling  gives rise to two labelings of the original rule:
  
  And, among the fourteen rules in  for updating labels,
  we find the following two:
  

  \noindent
  Now consider the term ,
and the rewrite sequence:
  
  After an application of~\eqref{rule:fgfx:dynlab:1}, relabeling takes two steps,
  resulting in a correctly labeled term.

  In the alternative, let us say `forgetful' version of dynamic labeling,
  where the `from' value  in symbols  is omitted,
  the rules~\eqref{rule:fgfx:dynlab:1}--\eqref{rule:fgfx:dynlab:4} look like this:
  
Due to the overlap in rules~\eqref{rule:fgfx:dynlab:forgetful:3} and~\eqref{rule:fgfx:dynlab:forgetful:4},
  the resulting \csTRS\ has a rewrite sequence 
  from  where the symbol  goes up all the way to the top:
  
\end{remark}





From the following lemma it follows that every  symbol
can be rewritten at most  times (before it vanishes).
By rewriting a ` symbol' we refer to a notion of
residuals that extends the usual definition of orthogonal projection~\cite{terese:03}
with a concept suggested by the definition of \,:
Whenever we have a rule of the form:

then we call  in the right-hand side
a residual of  in the left-hand side.

\newcommand{\reachrel}{\leadsto}
\newcommand{\srelabelw}{w}
\newcommand{\relabelw}{\funap{\srelabelw}}
\begin{lemma}\label{lem:relabel:bound}
  Let  be a TRS over , 
  and let  be a \clabeling\ for .
  We define the relation  by:
  
  Then  is well-founded and every  path has length .
\end{lemma}
\begin{proof}
  By definition of value-change pairs we have that for every pair 
  there exists a rule  and assignment 
  such that .
  
  Assume, to arrive at a contradiction, there exists a sequence 
  
  with .
  For  we construct thin contexts  
  and assignments  
  such that 
  
  and .
  We begin with  and .
  Then we have  and .
  For  there exist
  , and 
  such that
  
  and .
  We pick fresh variables  and 
  with  and ,
  and define ,
  and  is  extended by
  mapping variables  to the corresponding  and  to .
  It follows that 
  and .
  But then
  
  which contradicts that  is the \cdepth{} of .
\end{proof}



\begin{corollary}
  Every  symbol disappears at latest 
  after having applied  many 
  relabeling rules (to this symbol).
\end{corollary}
\begin{proof}
  For every rule in  of the form:
  
  we have that .
\end{proof}

For the dynamic context extension, the `intended' terms in 
are those terms that can be obtained by correctly labeling terms in .
For the purpose of adapting this definition to dynamic labeling,
we enrich the (unlabeled) signature  to :

and extend the \clabeling{} to  by:

for all  such that  for some .
Then labeled symbols are identified by .


We obtain the following lemma: 
\begin{lemma}\label{lem:relabel}
  Let  be a TRS over , 
  and let  be a \clabeling\ for .
Whenever we have a ground term  of the form:
  
  with , ,
  and where the displayed  symbol is at a -replacing position,
  then one of the following steps applies:
  
  where .
\end{lemma}
\begin{proof}
  Let . 
  Note that .
  Then:

  where 
  and .

  By Definition~\ref{def:dynlab} the dynamic labeling  contains a rule of the form:
  
  with  or .
  Consequently we have a step of the form:
  
  with  or .

  Now the claim follows since .
\end{proof}

\begin{lemma}\label{lem:dynlab:sound}
  Let  be a TRS over , 
  and let  be a sound \clabeling\ for .
  Let  be ground terms
  such that . 
  Then, for some \,:
  
\end{lemma}
\begin{proof}
  Assume  for some position .
  Then there exists a rule , 
  a context  with 
  and a ground substitution  such that 
   
  and .
\newcommand{\labacxt}[1]{\overline{\acontext}_{#1}}\newcommand{\labasubst}{\overline{\asubst}}Let  
  and , then by Lemma~\ref{lem:zzz} we obtain:
  
  \newcommand{\refeqsimplt}{\ensuremath{\ref{eq:simpl:t}}}and, likewise:
  

  By definition of dynamic labeling, one of the following rules is in : 

  (Note that 
  , 
  and  by Lemma~\ref{lem:xyz}.)
  
  In case , 
  no relabeling is needed and we take :
  
  
  If , 
  we get:
   
By Lemma~\ref{lem:relabel} the  symbol can `walk' upward until it disappears,
  and at the latest it vanishes when it meets . 
  Hence we have:
  
  for some  by Lemma~\ref{lem:relabel:bound}.
\end{proof}

\begin{theorem}\label{thm:dynlab:sound}
  Let  be a TRS over , 
  and  a sound \clabeling\ for .
  Then \ is outermost ground terminating if  is terminating.
\end{theorem}
\proof
  Assume that \, admits an infinite outermost rewrite sequence of ground terms:
  
  Then from Lemma~\ref{lem:dynlab:sound} it follows that 
   admits an infinite rewrite sequence: 


We note that Theorem~\ref{thm:dynlab:sound} can be strengthened
by weakening the termination of 
to local termination of  on 
the set of correctly labeled ground terms without  symbols.
Let us denote this set by :


Theorem~\ref{thm:dynlab:complete} below states
that dynamic labeling is complete with respect to 
local termination on .
More precisely, outermost ground termination of  implies termination of 
 on .
The following example helps to understand the proof of that theorem;
it illustrates that even when starting from terms in
, not every rewrite step in 
 corresponds to an outermost step in .

\begin{example}\label{ex:dyncomplete}
  Let  consist of the following rules:
  
  Moreover, let  with
  , ,
  and  for all .
  Labeling symbols with the value of their arguments, 
  we obtain for :
  
  and for :
  
  where .
  Then we obtain the following rewrite sequence in :
  
  The second step in this rewrite sequence
  does not correspond to an outermost step.
  Nevertheless, Theorem~\ref{thm:dynlab:complete}
  states that such `illegal' steps do not harm completeness of the transformation.
  The reason is that if the relabeling rules create a redex above
  some  symbol, then this  symbol
  is prevented from further propagating its information upward
  (until it becomes -replacing again).
  The crucial point is that above  symbols
  the labels are unchanged, thus as if the step would not have taken place.
  Moreover, it is essential that  
  prohibits  to propagate over symbols from .
  For instance, in the above example 
  does not contain a rule of the form:
  
  This rule would cause non-termination: 
  
\end{example}

\begin{theorem}\label{thm:dynlab:complete}
  Let  be a left-linear TRS over , 
  and  a complete, maximal, and core \clabeling\ for .
  Then  is terminating on the set of terms 
  if  is outermost ground terminating.
\end{theorem}
\begin{proof}
  \newcommand{\sto}{\hookrightarrow}\newcommand{\nto}{\stackrel{\makebox(0,0){{\scriptsize}}}{\longrightarrow}}Define ,
  and .
  Note that the relation  is restricted to terms which contain no  symbols.
  Hence always the maximal number of relabeling rules is applied.
  It is clear that each  rewrite step corresponds to an outermost rewrite step in the original TRS .
  Therefore it suffices to show that any infinite rewrite sequence
  
  gives rise to an infinite rewrite sequence .
  We prove the claim by a kind of standardization of reductions.
We first classify the rules from :
  
For , we analyse the steps 
  and construct 
  in such a way that 
  where we use  to denote standard term rewriting 
  ignoring the replacement map , and using rules from \eqref{dyn2} and \eqref{dyn4} only.
  Observe that then the maximal prefix  of  
  not containing  symbols is also a prefix of 
  (since everything changed by \eqref{dyn2} and \eqref{dyn4} is `hidden' inside a  symbol).
  We begin with , and .
  For , we consider the step .

  If  is a step with respect to a rule from:
  \begin{enumerate}[]

  \item
\eqref{dyn2} or \eqref{dyn4}, then
  we append  to the rewrite sequence 
  
  yielding 
  .
  Note that this leaves the -rewrite sequence  untouched.

  \item
\eqref{dyn1},
  then the pattern of  lies entirely in  which is also prefix of .
  Then we append  to  (using left-linearity of ) 
  yielding .
  We have 
  by orthogonal projection of the steps 
  over  (all steps in  are below the prefix ).
  
  \item
\eqref{dyn3}, then a  symbol `disappears'.
  We can trace this symbol back to a sequence of steps
  ,
  that is, it must have been created in  by a \eqref{dyn2} step,
  followed by a number of \eqref{dyn4} steps.
  We combine  and  to a  step,
  yielding .
  Then 
  as the remaining steps from
   are not harmed by the permutation (performing  first).

  \end{enumerate}

  \noindent
  It remains to be shown that the constructed sequence  is infinite.
  This follows from the fact that an infinite number of steps in
  
  must be of type \eqref{dyn1} or \eqref{dyn3}.
  This is a direct consequence of the fact that
   is terminating
  (with every step the prefix in which rewriting is allowed gets smaller).
\end{proof}

The following example demonstrates why the completeness result for dynamic labeling (Theorem~\ref{thm:dynlab:complete}) 
is restricted to the set  of correctly labeled terms 
which do not contain  symbols.
The point is that, although the original TRS is outermost terminating
the transformed system may in general be non-terminating 
due to the existence of `non-reachable' terms.
\begin{example}\label{ex:dynlab:noncomplete}
  We consider the following term rewriting system :
  
  We explain why this TRS is outermost ground terminating.
  Without the rule , 
  the system would even be terminating.
  Now note that the rule  can only be applied once 
  to each occurrence of 
  since  implies that ,
  and then the rule  has priority by the strategy of outermost rewriting.

  We define a maximal, complete, core \clabeling{} 
   for 
  (isomorphic to the result of the construction given in the next section) where the algebra  
  with the interpretation function defined by:
  
  for all  with ,
  and with .


  The dynamic labeling  of  with respect to this \clabeling{} then includes the rules:
  
  Now the context-sensitive TRS  
  admits the following infinite rewrite sequence:
  
Observe that this anomaly is caused by the subterm ,
  which is not reachable from any term in .
\end{example}

\begin{remark}
Theorem~\ref{thm:dynlab:complete} 
  states completeness of dynamic labeling with respect to 
  local termination on the set of terms .
We briefly indicate how the theorem can be generalized 
  to termination on 
  by altering the definition of .
  Note that
  .
  In particular,
  the set  includes terms that are not correctly labeled.
  The necessary modification of the definition of dynamic labeling concerns the elimination of collapsing rules .
  This can be achieved by wrapping the right-hand side into 
  even when the interpretations of the left and right-hand side are equal.
  Additionaly, we let the symbols  disappear after one relabeling step.
  By an application of Theorem~\ref{thm:adapt:ohlebusch} it then follows that 
  termination on  
  coincides with termination on .
\end{remark}
