
This paper is about employing context-sensitive rewriting to model outermost rewriting.
We do so by marking redexes, and forbid rewriting below them.
As we have seen, contracting a redex may create another redex higher up in the term tree.
Hence it may be necessary to update some labels during a rewrite step.
In Section~\ref{sec:cxtext} we defined a transformation where this updating 
was accounted for by extending rules with contexts.
Here we give an alternative transformation from TRSs to context-sensitive TRS{s}.
We call this tranformation `dynamic labeling'.
Instead of extending rules with contexts, 
we now employ rewriting to propagate the changed information upward in the term tree,
and set the labels in the surrounding context right, step by step.
Again the \cdepth{} (Definition~\ref{def:cmodel}) serves as a bound:
here on the number of successive ancestor nodes that have to be relabeled.
Each original rewrite step will give rise to a corresponding step and a bounded number ($\leq$ the \cdepth{}) 
of auxiliary steps in the transformed system. 
Thus, although the derivational complexity (the length of rewrite sequences) 
is changed, this is only by a constant factor.
We prove that dynamic labeling is sound for arbitrary TRSs.
Moreover, for left-linear TRSs, the method is complete in a weakened sense, 
see Theorem~\ref{thm:dynlab:complete}.
In Section~\ref{sec:evaluation}, we compare the performance of this method 
to the one of dynamic context extension described in Section~\ref{sec:cxtext}.

We begin with an analysis for evaluating which value changes can occur by rewriting and need to be propagated upward.
As we will see, this restricts the number of auxiliary `relabel symbols', and, in particular,
the number of `relabeling rules'.

\begin{definition}\label{def:value-change-pairs}
  Let $\atrs$ be a TRS over $\asig$, 
  and let $\triple{\aalg}{\semlab}{\asigred}$ be a \clabeling\ for $\atrs$.
  For $i = 0,1,2,\ldots$\,, we define the set $\ireachables{i}\subseteq\aalg\times\aalg$ inductively by:
  \begin{align*}
    \ireachables{0}
    =
    \{\,
      \pair{\interpreta{l}{\alpha}}{\interpreta{r}{\alpha}}
      \where
      {\ell \to r} \in \atrs,\,
      & \mbox{}
      \alpha \funin \vars{\ell} \to \aalg,\,
      \interpreta{l}{\alpha} \neq \interpreta{r}{\alpha}
    \,\}
    \\
    \ireachables{i+1} 
    =
    \ireachables{i}
    \join
    \{\,
      \pair{\funap{\interpret{\tf}}{\vec{a},b,\vec{c}}}{\funap{\interpret{\tf}}{\vec{a},b',\vec{c}}}
      \where \mbox{}
      & 
      \tf\in\asig,\,
\vec{a} \cdot b \cdot \vec{c}\,\in\aalg^{\arity{\tf}},\,
      \pair{b}{b'} \in \ireachables{i}, 
      \\
      &
      \svoodoolabel{\tf}{\funap{\slabelf{\tf}}{\vec{a},b,\vec{c}}} \in { \labelsig{\topsig{\asig}} \setminus \asigred }      
    \,\}
  \end{align*}
  Then we define the set $\reachables{\semlab}{\atrs}$ of \emph{value-change pairs} by:
  \[
    \reachables{\semlab}{\atrs} = \ireachables{i} \setminus \{\pair{a}{a}\}_{a\in\aalg}
  \]
  with $i$\, the least number
  such that $\ireachables{i+1} = \ireachables{i}$.
\end{definition}

The `dynamic labeling' $\dynlab{\aalg}{\semlab}{\atrs}$ of a TRS~$\atrs$\, is partitioned into two sets of rules.
The first set is denoted by $\dynlaborg{\aalg}{\semlab}{\atrs}$
and consists of a semantic labeling of the original rules,
where, additionally, a right-hand side is prefixed by a symbol $\srelabel{a}{a'}$ 
whenever application of the rule causes a change of interpretation from $a$ to $a'$.
The second set, $\dynlabprp{\aalg}{\semlab}{\atrs}$, 
is a set of rules for relabeling the context of the rule application.
A symbol $\srelabel{a}{a'}$, with $\pair{a}{a'}\in\reachables{\semlab}{\atrs}$,
indicates that the value of its subterm has changed from $a$ to $a'$,
and the rules in $\dynlabprp{\aalg}{\semlab}{\atrs}$ 
take care of propagating this change of value upward in the term.

\begin{definition}[Dynamic labeling]\label{def:dynlab}
  Let $\atrs$ be a TRS over $\asig$, 
  and let $\triple{\aalg}{\semlab}{\asigred}$ be a \clabeling\ for $\atrs$.
  The TRS $\dynlab{\aalg}{\semlab}{\atrs}$ over the signature 
  $\labelsig{\asig_{\topsymb}} \join \{ \srelabel{a}{a'} \where { \pair{a}{a'} \in \reachables{\semlab}{\atrs} } \}$
is defined by 
  $\dynlab{\aalg}{\semlab}{\atrs} = \dynlaborg{\aalg}{\semlab}{\atrs} \join \dynlabprp{\aalg}{\semlab}{\atrs}$. 
  Here the set $\dynlaborg{\aalg}{\semlab}{\atrs}$ of \emph{labeled rules} contains,   
  for each rule ${\ell\to r}\in\atrs$ and assignment $\alpha\funin{\vars{\ell}\to\aalg}$,
  one of the rules:
  \begin{align*}
    \dolabel{\ell}{\alpha} \to 
    \begin{cases}
      \dolabel{r}{\alpha}
      & \text{if $\interpreta{\ell}{\alpha} \eq \interpreta{r}{\alpha}$}
      \\
      \relabel{\interpreta{\ell}{\alpha}}{\interpreta{r}{\alpha}}{\dolabel{r}{\alpha}}
      & \text{otherwise}
    \end{cases}  
  \end{align*}
Secondly, the set $\dynlabprp{\aalg}{\semlab}{\atrs}$ of 
  \emph{relabeling rules} contains,
  for each $n$-ary $\tf\in\asig$, $\pair{b}{b'}\in\reachables{\semlab}{\atrs}$, 
  and $\tuple{\vec{a},b,\vec{c}} \in \aalg^n$ 
such that 
  $\svoodoolabel{\tf}{\alab} \in { \labelsig{\topsig{\asig}} \setminus \asigred }$ 
  with $\alab = \funap{\slabelf{\tf}}{\vec{a},b,\vec{c}}$,
  one of the rules:
  \begin{align*}
    \funap{\svoodoolabel{\tf}{\alab}}{\vec{x},\relabel{b}{b'}{y},\vec{z}}
    \to
    \begin{cases}
      \funap{\svoodoolabel{\tf}{\alab'}}{\vec{x},y,\vec{z}}
      &\text{if $d \eq d'$} \\
      \relabel{d}{d'}{\funap{\svoodoolabel{\tf}{\alab'}}{\vec{x},y,\vec{z}}}
      &\text{otherwise}
    \end{cases}
  \end{align*}
  where 
  $\alab' = \funap{\slabelf{\tf}}{\vec{a},b',\vec{c}}$,
  $d = \funap{\interpret{\tf}}{\vec{a},b,\vec{c}}$,
  $d' = \funap{\interpret{\tf}}{\vec{a},b',\vec{c}}$,
  $\veclength{\vec{x}} = \veclength{\vec{a}}$, 
  and $\veclength{\vec{z}} = \veclength{\vec{c}}$.

  The \emph{dynamic labeling of $\atrs$ (with respect to the \clabeling~$\triple{\aalg}{\semlab}{\asigred}$)}
  is the context-sensitive TRS~$\pair{\dynlab{\aalg}{\semlab}{\atrs}}{\samumap}$,
  where the replacement map $\samumap$ is defined by 
  $\amumap{\srelabel{a}{a'}} = \setemp$ for all $\pair{a}{a'}\in\reachables{\semlab}{\atrs}$,
  $\amumap{\tf} = \setemp$ if $\tf \in \asigred$,
  and $\amumap{\tf} = \{1,\ldots,\arity{\tf}\}$ otherwise,
  for all $\tf\in\labelsig{\asig_{\topsymb}}$.
Whenever $\asigred$ is clear from the context, we leave $\samumap$ implicit,
  and overload the notation $\dynlab{\aalg}{\semlab}{\atrs}$ to denote 
  $\pair{\dynlab{\aalg}{\semlab}{\atrs}}{\samumap}$.
\end{definition}


\begin{example}
  We revisit the TRS~$\trsfffx$ from Example~\ref{ex:fffx:cmodel} 
  for which we worked out the static and dynamic context extensions 
  in Examples~\ref{ex:fffx:static:cxtext} and~\ref{ex:fffx:dyn:cxtext}.
  We repeat its definition and the \clabeling{} from Example~\ref{ex:fffx:clabeling}: $\trsfffx$ is the TRS over $\asig = \{\ta,\tf,\tg\}$ consisting of the rules:
  \begin{align}
    \f{\g{x}} &\to \f{\f{\g{x}}} 
    & \f{\f{\f{x}}} &\to x
    \tag{$\trsfffx$}
  \end{align} 
  A \cmodel{} for $\trsfffx$ is formed by the $\asig$-algebra 
  $\algfffx = \{\bot,f,\mit{ff},g\}$
  with interpretation:
  \begin{align}
    \interpret{\fun{c}} = \bot 
    && 
    \funap{\interpret{\tf}}{\bot} = \funap{\interpret{\tf}}{g} = f
    && 
    \funap{\interpret{\tf}}{f} = \funap{\interpret{\tf}}{\mit{ff}} = \mit{ff}
    &&
    \funap{\interpret{\tg}}{x} = g
    \tag{$\algfffx$}
  \end{align}
  for all $x\in\algfffx$.
  Furthermore, $\pair{\algfffx}{\semlab}$ denotes the maximal labeling for $\trsfffx$, 
  and $\sigredex{\asig} = \{ \svoodoolabel{\tf}{g}, \svoodoolabel{\tf}{\mit{ff}} \}$.
Then $\triple{\algfffx}{\semlab}{\sigredex{\asig}}$
  forms a sound and complete \clabeling\ of $\trsfffx$.
Also note that $\algfffx$ forms a core algebra;
  for each value $e\in\algfffx$ there is a ground term $t$ such that 
  $\interpret{t} = e$.
We first compute the set $\reachables{\semlab}{\trsfffx}$ of value-change pairs. 
  For the initial set $\ireachables{0}$, note that the rule 
  $\f{\g{x}} \to \f{\f{\g{x}}}$ 
  changes the interpretation from $f$ to $\mit{ff}$, regardless of the value assigned to $x$.
The other rule creates three value-change pairs; one for each of the values $g,\bot,f$ assigned to $x$. 
  If the interpretation of $x$ is $\mit{ff}$ there is no change.
  Hence we get:
  \[
    \ireachables{0} = \{ \pair{f}{\mit{ff}} , \pair{\mit{ff}}{\bot} , \pair{\mit{ff}}{f} , \pair{\mit{ff}}{g} \}
  \]
All symbols $\srelabel{e}{e'}$ with $\pair{e}{e'} \in \ireachables{0}$ 
  will disappear in one relabeling step, whence $\reachables{\semlab}{\trsfffx} = \ireachables{0}$.
The dynamic labeling of $\trsfffx$ then is
  $\dynlab{\algfffx}{\semlab}{\trsfffx} = \dynlaborg{\algfffx}{\semlab}{\trsfffx} \join \dynlabprp{\algfffx}{\semlab}{\trsfffx}$
  where $\dynlaborg{\algfffx}{\semlab}{\trsfffx}$ consists of the rules:
  \begin{align*}
    \funap{\svoodoolabel{\tf}{g}}{\funap{\svoodoolabel{\tg}{e}}{x}} 
    & \to \relabel{f}{\mit{ff}}{\funap{\svoodoolabel{\tf}{f}}{\funap{\svoodoolabel{\tf}{g}}{\funap{\svoodoolabel{\tg}{e}}{x}}}}
    & \text{for all $e \in \algfffx$} \\
    \funap{\svoodoolabel{\tf}{\mit{ff}}}{\funap{\svoodoolabel{\tf}{\mit{ff}}}{\funap{\svoodoolabel{\tf}{e'}}{x}}} 
    & \to \relabel{e}{e'}{x}
    & \text{for all $\pair{e}{e'} \in \{ \pair{\mit{ff}}{\bot} , \pair{\mit{ff}}{f} , \pair{\mit{ff}}{g} \}$}
    \\
    \funap{\svoodoolabel{\tf}{\mit{ff}}}{\funap{\svoodoolabel{\tf}{\mit{ff}}}{\funap{\svoodoolabel{\tf}{\mit{ff}}}{x}}} &\to x
  \end{align*}
  and where $\dynlabprp{\algfffx}{\semlab}{\trsfffx}$ is formed by:
  \begin{align*}
    \funap{\svoodoolabel{\tg}{e}}{\relabel{e}{e'}{x}} &\to \funap{\svoodoolabel{\tg}{e'}}{x}
    &\text{for all $\pair{e}{e'}\in\reachables{\semlab}{\trsfffx}$ } \\
    \funap{\svoodoolabel{\tf}{f}}{\relabel{f}{\mit{ff}}{x}} &\to \funap{\svoodoolabel{\tf}{\mit{ff}}}{x}
  \end{align*}
  In total the dynamic labeling of $\trsfffx$ has 13 rules. 
Had we not restricted the construction of the set of the relabeling rules to 
  the `reachable' symbols $\srelabel{e}{e'}$ 
  (by the requirement $\pair{e}{e'}\in\reachables{\semlab}{\trsfffx}$ in Definition~\ref{def:dynlab}),
  we would have come up with 18 instead of 5 relabeling rules.
\end{example}

\begin{example}
  We reconsider the term rewrite system~$\trsduplrhs$ from Example~\ref{ex:dupl_rhs}:
  \begin{align*}
    \fb{\h{x}}{\fun{c}}
    & \to \fb{\funap{\fun{i}}{x}}{\funap{\fun{s}}{x}}
    &
    \funap{\fun{i}}{x}
    & \to \h{x}
    \\
    \fb{\funap{\fun{i}}{x}}{y}
    & \to x
    &
    \h{x}
    & \to \fb{\h{x}}{\fun{c}}
  \end{align*}
  and the algebra $\algduplrhs = \pair{\{\bot,c,h,i\}}{\sinterpret}$ 
  with $\sinterpret$ defined, for all $x,y\in\algduplrhs$, as follows:
  \begin{align*}
    \interpret{\fun{c}}=c
  &&
  \funap{\interpret{\sh}}{x} = h
  &&
  \funap{\interpret{\fun{i}}}{x} = i
  &&
  \bfunap{\interpret{\tf}}{x}{y} = \funap{\interpret{\fun{s}}}{x} = \bot
  \end{align*}
  Moreover, we employ minimal labeling again; see Example~\ref{ex:dupl_rhs}. 

  The set of change-value pairs is:
  \begin{align*}
    \reachables{\semlab}{\trsduplrhs}
    = \{ \pair{\bot}{c} , \pair{\bot}{h} , \pair{\bot}{i} , \pair{i}{h} , \pair{h}{\bot} \}
  \end{align*}
  The set $\dynlaborg{\algduplrhs}{\semlab}{\trsduplrhs}$ of labeled rules is constructed thus: 
  \begin{align*}
    \bfunap{\rmark{\tf}}{\funap{\rmark{\sh}}{x}}{\fun{c}}
    & \to \bfunap{\rmark{\tf}}{\funap{\rmark{\fun{i}}}{x}}{\funap{\fun{s}}{x}} 
    \\
    \bfunap{\rmark{\tf}}{\funap{\rmark{\fun{i}}}{x}}{y}
    & \to x
    \\
    \bfunap{\rmark{\tf}}{\funap{\rmark{\fun{i}}}{x}}{y}
    & \to \relabel{e}{e'}{x}
    & \text{$\pair{e}{e'} \in \{ \pair{\bot}{c} , \pair{\bot}{h} , \pair{\bot}{i} \}$}
    \\
    \funap{\rmark{\fun{i}}}{x}
    & \to \relabel{i}{h}{\funap{\rmark{\sh}}{x}}
    \\
    \funap{\rmark{\sh}}{x}
    & \to \relabel{h}{\bot}{\bfunap{\rmark{\tf}}{\funap{\rmark{\sh}}{x}}{\fun{c}}}
  \end{align*}There are four rules with left-hand side $\ell = \bfunap{\rmark{\tf}}{\funap{\rmark{\fun{i}}}{x}}{y}$,
  one for each value assigned to~$x$. 
  In case $\funap{\alpha}{x} = \bot$ there is no change of interpretation, 
  for we have that $\interpreta{\ell}{\alpha} = \bot$ for all $\alpha\funin\{x,y\}\to\algduplrhs$
  and so no $\ssrelabel$ symbol is inserted.
  But if, for instance, $\interpret{\funap{\asubst}{x}} = c$ for some substitution $\asubst$,
  then some labels in the context~$\acxt$ of a rewrite step 
  $\cxtfill{\acxt}{\subst{\asubst}{\ell}} \to \cxtfill{\acxt}{\funap{\asubst}{x}}$
  have to be updated, since the value of $\cxthole$ has changed from $\bot$ to $c$,
  whence the insertion of $\srelabel{\bot}{c}$ to the right-hand side $x$.

  The set $\dynlabprp{\algduplrhs}{\semlab}{\trsduplrhs}$ 
  of relabeling rules is formed by:
  \begin{align*}
    \fb{\relabel{e}{e'}{x}}{y}
    & \to \fb{x}{y} 
    & \text{$\pair{e}{e'} \in \{ \pair{\bot}{c} , \pair{\bot}{h} , \pair{h}{\bot} \}$}
    \\
    \fb{\relabel{e}{e'}{x}}{y}
    & \to \bfunap{\rmark{\tf}}{x}{y}
    & \text{$\pair{e}{e'}\in\{\pair{\bot}{h} , \pair{\bot}{i}\}$}
    \\
    \fb{x}{\relabel{\bot}{c}{y}}
    & \to \bfunap{\rmark{\tf}}{x}{y} 
    \\
    \fb{x}{\relabel{\bot}{c}{y}}
    & \to \fb{x}{y} 
    & \text{$\pair{e}{e'} \in \reachables{\semlab}{\trsduplrhs}$}
    \\
    \funap{\fun{s}}{\relabel{e}{e'}{x}}
    & \to \funap{\fun{s}}{x} 
    & \text{$\pair{e}{e'}\in\reachables{\semlab}{\trsduplrhs}$}
    \\
    \funap{\topsymb}{\relabel{e}{e'}{x}}
    & \to \funap{\topsymb}{x} 
    & \text{$\pair{e}{e'}\in\reachables{\semlab}{\trsduplrhs}$}
  \end{align*}
  Some remarks for clarification: 
  First, note that all $\ssrelabel$ symbols disappear upon one relabeling step.
  Secondly, observe the overlap in, for example, the rules with left-hand side 
  $\fb{\relabel{\bot}{h}{x}}{y}$.
  If the value assigned to $y$ is $c$, 
  then a redex is created; this is witnessed by the marked symbol $\rmark{\tf}$ on the right.
  For other values for $y$, this is not the case.
  Also note that there is no rule for
  $t = \fb{\relabel{i}{h}{x}}{y}$. 
  This is because when the left argument of $\tf$ is interpreted as $i$,
  then $t$ forms a redex, and so $\tf$ should be marked. 
  Definition~\ref{def:dynlab} does not allow $\ssrelabel$ symbols 
  to commute with redex symbols.
  Intuitively, a $\ssrelabel$ symbol is a witness of a rewrite step 
  which we do not want to occur inside other redexes, as we want to model outermost terminination. 
  However, more technically, sometimes illegal (i.e., non-outermost) $\ssrelabel$ steps are allowed.
  This is illustrated in Example~\ref{ex:dyncomplete}.
  The point is that by preventing $\ssrelabel$ symbols to commute with redex symbols,
  for local completeness (Theorem~\ref{thm:dynlab:complete})
  it is as if illegal steps never happened.
\end{example}

\begin{remark}
\newcommand{\ssrelabelto}{\ssrelabel}\newcommand{\srelabelto}{\super{\ssrelabelto}}\newcommand{\relabelto}[1]{\funap{\srelabelto{#1}}}\newcommand{\fgfxnr}{8}\newcommand{\trsfgfx}{\iatrs{\fgfxnr}}\newcommand{\algfgfx}{\iaalg{\fgfxnr}}We elaborate on the role of the element~$a$ in $\srelabel{a}{a'}$.
  Whenever the application of a rule 
  $\contextfill{\acontext}{\subst{\asubst}{\ell}} \to \contextfill{\acontext}{\subst{\asubst}{r}}$ 
  changes the interpretation,
  i.e., $\interpret{\subst{\asubst}{\ell}} \neq \interpret{\subst{\asubst}{r}}$,
  then a symbol $\srelabel{\interpret{\subst{\asubst}{\ell}}}{\interpret{\subst{\asubst}{r}}}$ is inserted.
  A term of the form $\relabel{a}{a'}{t'}$ can be thought of as a witness 
  of a rewrite step $t\to t'$ 
  causing a change of interpretation from $a = \interpret{t}$ to $a' = \interpret{t'}$.
  This change of the value then needs to be propagated upward to update the labels accordingly,
  using the relabeling rules from $\dynlabprp{\aalg}{\semlab}{\atrs}$.
  At first sight, the value $a$ in $\relabel{a}{a'}{t}$ seems redundant for relabeling:
  why would we store the previous value? 
  However, the label $a$ is important in order to restrict the number of applicable rules,
  and to have a bound on the number of relabeling steps. 
  To see this, consider the system~$\trsfgfx$ with single rewrite rule:
\begin{align}
    \f{\g{\f{x}}} \to \fun{d}
    \tag{$\trsfgfx$}
    \label{ex:fgfx}
  \end{align}
  and the algebra $\algfgfx = \{ \bot, f, \mit{gf} \}$ with
  $\funap{\interpret{\tf}}{x} = f$ for all $x \in \algfgfx$,
  $\funap{\interpret{\tg}}{\mit{f}} = \mit{gf}$, $\funap{\interpret{\tg}}{x} = \bot$ for all $x \ne f$,
  and $\interpret{\fun{d}} = \bot$.
  We employ minimal labeling, that is, 
  only $\funap{\slabelf{\tf}}{\mit{gf}} = \star$\,,
  and all the other symbols are unlabeled.
  
  The dynamic labeling $\dynlab{\algfgfx}{\semlab}{\trsfgfx}$ gives rise to two labelings of the original rule:
  \begin{align}
    \funap{\rmark{\tf}}{\g{\f{x}}} &\to \relabel{f}{\bot}{\fun{d}}
    \label{rule:fgfx:dynlab:1}
    \\
    \funap{\rmark{\tf}}{\g{\funap{\rmark{\tf}}{x}}} &\to \relabel{f}{\bot}{\fun{d}}
    \label{rule:fgfx:dynlab:2}
  \end{align}
  And, among the fourteen rules in $\dynlab{\algfgfx}{\semlab}{\trsfgfx}$ for updating labels,
  we find the following two:
  \begin{align}
    \g{\relabel{f}{\bot}{x}} &\to \relabel{\mit{gf}}{\bot}{\g{x}}
    \label{rule:fgfx:dynlab:3}
    \\ 
    \g{\relabel{\mit{gf}}{\bot}{x}} &\to \g{x}
    \label{rule:fgfx:dynlab:4}
  \end{align}

  \noindent
  Now consider the term $t = \funap{\topsymb}{\g{\cdots(\g{\g{\g{\funap{\svoodoolabel{\tf}{\star}}{\g{\f{\fun{d}}}}}}})}}$,
and the rewrite sequence:
  \begin{align*}
    t
    \to_{\ref{rule:fgfx:dynlab:1},\samumap} {}
    &
    \funap{\topsymb}{\g{\cdots(\g{\g{\g{\relabel{f}{\bot}{\fun{d}}}}})}}
    \\
    {} \to_{\ref{rule:fgfx:dynlab:3},\samumap} {}
    &
    \funap{\topsymb}{\g{\cdots(\g{\g{\relabel{\mit{gf}}{\bot}{\g{\fun{d}}}}})}}
    \\
    {} \to_{\ref{rule:fgfx:dynlab:4},\samumap} {}
    &
    \funap{\topsymb}{\g{\cdots(\g{\g{\g{\fun{d}}}})}}
  \end{align*}
  After an application of~\eqref{rule:fgfx:dynlab:1}, relabeling takes two steps,
  resulting in a correctly labeled term.

  In the alternative, let us say `forgetful' version of dynamic labeling,
  where the `from' value $a$ in symbols $\srelabel{a}{b}$ is omitted,
  the rules~\eqref{rule:fgfx:dynlab:1}--\eqref{rule:fgfx:dynlab:4} look like this:
  \begin{align}
    \funap{\rmark{\tf}}{\g{\f{x}}} &\to \relabelto{\bot}{\fun{d}}
    \label{rule:fgfx:dynlab:forgetful:1}
    \tag{$\ref{rule:fgfx:dynlab:1}'$}
    \\
    \funap{\rmark{\tf}}{\g{\funap{\rmark{\tf}}{x}}} &\to \relabelto{\bot}{\fun{d}}
    \label{rule:fgfx:dynlab:forgetful:2}
    \tag{$\ref{rule:fgfx:dynlab:2}'$}
    \\ 
    \g{\relabelto{\bot}{x}} &\to \relabelto{\bot}{\g{x}}
    \label{rule:fgfx:dynlab:forgetful:3}
    \tag{$\ref{rule:fgfx:dynlab:3}'$}
    \\
    \g{\relabelto{\bot}{x}} &\to \g{x}
    \label{rule:fgfx:dynlab:forgetful:4}
    \tag{$\ref{rule:fgfx:dynlab:4}'$}
  \end{align}
Due to the overlap in rules~\eqref{rule:fgfx:dynlab:forgetful:3} and~\eqref{rule:fgfx:dynlab:forgetful:4},
  the resulting \csTRS\ has a rewrite sequence 
  from $t$ where the symbol $\srelabelto{\bot}$ goes up all the way to the top:
  \begin{align*}
    t \to_{\text{\ref{rule:fgfx:dynlab:forgetful:1}},\samumap} {}
    &
    \funap{\topsymb}{\g{\cdots(\g{\g{\g{\relabelto{\bot}{\fun{d}}}}})}}
    \\
    {} \to_{\text{\ref{rule:fgfx:dynlab:forgetful:3}},\samumap} {}
    &
    \funap{\topsymb}{\g{\cdots(\g{\g{\relabelto{\bot}{\g{\fun{d}}}}})}}
    \\
    {} \to_{\text{\ref{rule:fgfx:dynlab:forgetful:3}},\samumap} {}
    &
    \funap{\topsymb}{\g{\cdots(\g{\relabelto{\bot}{\g{\g{\fun{d}}}}})}}
    \\
    {} \to_{\text{\ref{rule:fgfx:dynlab:forgetful:3}},\samumap} {}
    &
    \ldots
  \end{align*}
\end{remark}





From the following lemma it follows that every $\ssrelabel$ symbol
can be rewritten at most $\cdpth{\aalg}{\atrs}$ times (before it vanishes).
By rewriting a `$\ssrelabel$ symbol' we refer to a notion of
residuals that extends the usual definition of orthogonal projection~\cite{terese:03}
with a concept suggested by the definition of $\dynlabprp{\aalg}{\semlab}{\atrs}$\,:
Whenever we have a rule of the form:
\begin{align*}
  \funap{\svoodoolabel{\tf}{\alab}}{\vec{x},\relabel{a}{a'}{\dolabelg{t}},\vec{z}}
  \to \relabel{b}{b'}{\funap{\svoodoolabel{\tf}{\blab}}{\vec{x},\dolabelg{t},\vec{z}}}
\end{align*}
then we call $\srelabel{b}{b'}$ in the right-hand side
a residual of $\srelabel{a}{a'}$ in the left-hand side.

\newcommand{\reachrel}{\leadsto}
\newcommand{\srelabelw}{w}
\newcommand{\relabelw}{\funap{\srelabelw}}
\begin{lemma}\label{lem:relabel:bound}
  Let $\atrs$ be a TRS over $\asig$, 
  and let $\triple{\aalg}{\semlab}{\asigred}$ be a \clabeling\ for $\atrs$.
  We define the relation ${\reachrel} \subseteq {\reachables{\semlab}{\atrs} \times \reachables{\semlab}{\atrs}}$ by:
  \begin{align*}
    \pair{b}{b'} \reachrel \pair{\funap{\interpret{\tf}}{\vec{a},b,\vec{c}}}{\funap{\interpret{\tf}}{\vec{a},b',\vec{c}}}
    && \text{ for every } &\ \tf\in\asig,\, 
       \pair{b}{b'} \in \reachables{\semlab}{\atrs},\,
       \vec{a} \cdot b \cdot \vec{c}\,\in\aalg^{\arity{\tf}},\\
    &&   &\ \svoodoolabel{\tf}{\funap{\slabelf{\tf}}{\vec{a},b,\vec{c}}} \in { \labelsig{\topsig{\asig}} \setminus \asigred }
  \end{align*}
  Then $\reachrel$ is well-founded and every $\reachrel$ path has length $\le \cdpth{\aalg}{\atrs}$.
\end{lemma}
\begin{proof}
  By definition of value-change pairs we have that for every pair $\pair{b}{b'} \in \reachables{\semlab}{\atrs}$
  there exists a rule $\arule \in \atrs$ and assignment $\alpha \funin \vars{\alhs} \to \aalg$
  such that $\pair{\interpreta{\alhs}{\alpha}}{\interpreta{\arhs}{\alpha}} \reachrel^* \pair{b}{b'}$.
  
  Assume, to arrive at a contradiction, there exists a sequence 
  \begin{align*}
   \pair{\interpreta{\alhs}{\alpha}}{\interpreta{\arhs}{\alpha}} 
   = \pair{b_0}{b'_0} \reachrel \pair{b_1}{b'_1} \reachrel \ldots \reachrel \pair{b_m}{b'_m}
  \end{align*}
  with $m > \cdpth{\aalg}{\atrs}$.
  For $i = 0,1,\ldots,m$ we construct thin contexts $\bcxt_i$ 
  and assignments $\alpha_i : \vars{\bcxt_i} \to \aalg$ 
  such that 
  $b_i = \interpreta{\cxtfill{\bcxt_i}{\alhs}}{\alpha_i}$
  and $b'_i = \interpreta{\cxtfill{\bcxt_i}{\arhs}}{\alpha_i}$.
  We begin with $\bcxt_0 = \cxthole$ and $\alpha_0 = \alpha$.
  Then we have $b_0 = \interpreta{\alhs}{\alpha}$ and $b'_0 = \interpreta{\arhs}{\alpha}$.
  For $i = 1,\ldots,m$ there exist
  $\tf_i \in \asig$, and $\vec{a_i} \cdot b_{i-1} \cdot \vec{c_i} \in \aalg^{\arity{\tf_i}}$
  such that
  $b_i = \funap{\interpret{\tf_i}}{\vec{a_i},b_{i-1},\vec{c_i}}$
  and $b'_i = \funap{\interpret{\tf_i}}{\vec{a_i},b'_{i-1},\vec{c_i}}$.
  We pick fresh variables $\vec{x_i}$ and $\vec{z_i}$
  with $\lstlength{\vec{x_i}} = \lstlength{\vec{a_i}}$ and $\lstlength{\vec{z_i}} = \lstlength{\vec{c_i}}$,
  and define $\bcxt_i = \funap{\tf_i}{\vec{x_i},\bcxt_{i-1},\vec{z_i}}$,
  and $\alpha_i$ is $\alpha_{i-1}$ extended by
  mapping variables $\vec{x_i}$ to the corresponding $\vec{a_i}$ and $\vec{z_i}$ to $\vec{c_i}$.
  It follows that $b_i = \interpreta{\cxtfill{\bcxt_i}{\alhs}}{\alpha_i}$
  and $b'_i = \interpreta{\cxtfill{\bcxt_i}{\arhs}}{\alpha_i}$.
  But then
  $\interpreta{\contextfill{D}{\alhs}}{\alpha} = b_m \ne b_m' = \interpreta{\contextfill{D}{\arhs}}{\alpha}$
  which contradicts that $\cdpth{\aalg}{\arule}$ is the \cdepth{} of $\arule$.
\end{proof}



\begin{corollary}
  Every $\ssrelabel$ symbol disappears at latest 
  after having applied $\cdpth{\aalg}{\atrs}$ many 
  relabeling rules (to this symbol).
\end{corollary}
\begin{proof}
  For every rule in $\dynlabprp{\aalg}{\semlab}{\atrs}$ of the form:
  \begin{align*}
    \funap{\svoodoolabel{\tf}{\alab}}{\vec{x},\relabel{a}{a'}{\dolabelg{t}},\vec{z}}
    \to \relabel{b}{b'}{\funap{\svoodoolabel{\tf}{\blab}}{\vec{x},\dolabelg{t},\vec{z}}}
  \end{align*}
  we have that $\pair{a}{a'} \reachrel \pair{b}{b'}$.
\end{proof}

For the dynamic context extension, the `intended' terms in $\ter{\labelsig{\asig}}{\setemp}$
are those terms that can be obtained by correctly labeling terms in $\ter{\asig}{\setemp}$.
For the purpose of adapting this definition to dynamic labeling,
we enrich the (unlabeled) signature $\asig$ to $\asig_{+}$:
\begin{align*}
  \asig_{+} &= \asig \join \{ \srelabelone{a} \where \pair{a}{a'}\in\reachables{\semlab}{\atrs} \} 
\end{align*}
and extend the \clabeling{} to $\asig_{+}$ by:
\begin{align*}
  \labelf{\srelabelone{b}}{b'} &= b' & \interpret{\srelabelone{b}} = b
\end{align*}
for all $b,b'\in\aalg$ such that $\pair{b}{c}\in \reachables{\semlab}{\atrs}$ for some $c\in\aalg$.
Then labeled symbols are identified by $\svoodoolabel{(\srelabelone{a})}{a'} = \srelabel{a}{a'}$.


We obtain the following lemma: 
\begin{lemma}\label{lem:relabel}
  Let $\atrs$ be a TRS over $\asig$, 
  and let $\triple{\aalg}{\semlab}{\asigred}$ be a \clabeling\ for $\atrs$.
Whenever we have a ground term $s$ of the form:
  \[
    s = \dolabelg{\contextfill{\acxt}{\f{s_1,\ldots,\relabelone{a}{t},\ldots,s_n}}}
  \]
  with $\pair{a}{a'}\in\reachables{\semlab}{\atrs}$, $a' = \interpret{t}$,
  and where the displayed $\ssrelabel$ symbol is at a $\samumap$-replacing position,
  then one of the following steps applies:
  \begin{align}
    s
    & \to_{\dynlabprp{\aalg}{\semlab}{\atrs},\samumap} 
    \dolabelg{\contextfill{\acxt}{\relabelone{b}{\f{s_1,\ldots,t,\ldots,s_n}}}}
    \\
    s
    & \to_{\dynlabprp{\aalg}{\semlab}{\atrs},\samumap} 
    \dolabelg{\contextfill{\acxt}{\f{s_1,\ldots,t,\ldots,s_n}}}
  \end{align}
  where $b = \interpreta{\f{s_1,\ldots,\cxthole,\ldots,s_n}}{\cxthole\mapsto a}$.
\end{lemma}
\begin{proof}
  Let $b' = \interpret{\f{s_1,\ldots,t,\ldots,s_n}}$. 
  Note that $b' = \interpreta{\f{s_1,\ldots,\cxthole,\ldots,s_n}}{\cxthole\mapsto a'}$.
  Then:
\begin{align*}
    \dolabelg{\f{s_1,\ldots,\relabelone{a}{t},\ldots,s_{n}}} 
    & = \funap{\svoodoolabel{\tf}{\alab}}{\dolabelg{s_1},\ldots,\relabel{a}{a'}{\dolabelg{t}},\ldots,\dolabelg{s_{n}}} 
    \\
    \dolabelg{\relabelone{b}{\f{s_1,\ldots,t,\ldots,s_n}}} 
    & = \relabel{b}{b'}{\funap{\svoodoolabel{\tf}{\blab}}{\dolabelg{s_1},\ldots,\dolabelg{t},\ldots,\dolabelg{s_n}}}
  \end{align*}
  where $\alab = \labelf{\tf}{\interpret{s_1},\ldots,a,\ldots,\interpret{s_{n}}}$
  and $\blab = \labelf{\tf}{\interpret{s_1},\ldots,a',\ldots,\interpret{s_{n}}}$.

  By Definition~\ref{def:dynlab} the dynamic labeling $\dynlab{\aalg}{\semlab}{\atrs}$ contains a rule of the form:
  \begin{align*}
  \funap{\svoodoolabel{\tf}{\alab}}{\vec{x},\relabel{a}{a'}{\dolabelg{t}},\vec{z}}
  & \to 
  \contextfill{\acxt}{\funap{\svoodoolabel{\tf}{\blab}}{\vec{x},\dolabelg{t},\vec{z}}}
\end{align*}
  with $\acxt = \cxthole$ or $\acxt = \relabel{b}{b'}{\cxthole}$.
  Consequently we have a step of the form:
  \begin{align*}
    \dolabelg{\f{s_1,\ldots,\relabelone{a}{t},\ldots,s_n}} &\to_{\dynlab{\aalg}{\semlab}{\atrs},\samumap} \dolabelg{\contextfill{\bcxt}{\f{s_1,\ldots,t,\ldots,s_n}}}
\end{align*}
  with $\bcxt = \cxthole$ or $\bcxt = \relabelone{b}{\cxthole}$.

  Now the claim follows since $s = \contextfill{\dolabel{\acxt}{\cxthole \mapsto b}}{\dolabelg{\f{s_1,\ldots,\relabelone{a}{t},\ldots,s_n}}}$.
\end{proof}

\begin{lemma}\label{lem:dynlab:sound}
  Let $\atrs$ be a TRS over $\asig$, 
  and let $\triple{\aalg}{\semlab}{\asigred}$ be a sound \clabeling\ for $\atrs$.
  Let $s,t \in \term{\asig}{\setemp}$ be ground terms
  such that $s \outred_{\atrs} t$. 
  Then, for some $m \leq \cdpth{\aalg}{\atrs}$\,:
  \[
    \dolabelg{\funap{\topsymb}{s}} 
    \relcomp
      {\to_{\dynlaborg{\aalg}{\semlab}{\atrs},\samumap}}
      {\to^{m}_{\dynlabprp{\aalg}{\semlab}{\atrs},\samumap}}
    \dolabelg{\funap{\topsymb}{t}}
\]
\end{lemma}
\begin{proof}
  Assume $s \outred_{\atrs,p} t$ for some position $p \in \pos{s}$.
  Then there exists a rule $\arule\in\atrs$, 
  a context $\acontext$ with $\symbat{\acontext}{p} = \contexthole$
  and a ground substitution $\asubst\funin{\avars\to\term{\asig}{\setemp}}$ such that 
  $s = \contextfill{\acontext}{\subst{\asubst}{\alhs}}$ 
  and $t = \contextfill{\acontext}{\subst{\asubst}{\arhs}}$.
\newcommand{\labacxt}[1]{\overline{\acontext}_{#1}}\newcommand{\labasubst}{\overline{\asubst}}Let $\labacxt{a} = \dolabel{\funap{\topsymb}{\acontext}}{\cxthole \mapsto a}$ 
  and $\labasubst = \dolabelg{\asubst}$, then by Lemma~\ref{lem:zzz} we obtain:
  \begin{align}
    \dolabelg{\funap{\topsymb}{s}}
    = 
    \dolabelg{\funap{\topsymb}{\contextfill{\acontext}{\subst{\asubst}{\alhs}}}}
    =
    \contextfill{\labacxt{\interpret{\subst{\asubst}{\ell}}}}{\dolabelg{\subst{\asubst}{\alhs}}}
    =
    \contextfill{\labacxt{\interpret{\subst{\asubst}{\ell}}}}{\subst{\,\labasubst}{\dolabel{\alhs}{\interpret{\asubst}}}}
    \label{eq:simpl:s}
  \end{align}
  \newcommand{\refeqsimplt}{\ensuremath{\ref{eq:simpl:t}}}and, likewise:
  \begin{align}
  \dolabelg{\funap{\topsymb}{t}} = \contextfill{\labacxt{\interpret{\subst{\asubst}{\arhs}}}}{\subst{\,\labasubst}{\dolabel{\arhs}{\interpret{\asubst}}}}
  \label{eq:simpl:t}
\end{align}

  By definition of dynamic labeling, one of the following rules is in $\dynlaborg{\aalg}{\semlab}{\atrs}$: 
\begin{align}
    \dolabel{\alhs}{\interpret{\asubst}} & \to \dolabel{\arhs}{\interpret{\asubst}} 
    & \text{if $\interpret{\subst{\asubst}{\alhs}} \eq \interpret{\subst{\asubst}{\arhs}}$}
    \label{dynlabrule:1}
    \\
    \dolabel{\alhs}{\interpret{\asubst}} 
    & \to \relabel{\interpret{\subst{\asubst}{\alhs}}}{\interpret{\subst{\asubst}{\arhs}}}{\dolabel{\arhs}{\interpret{\asubst}}}
    & \text{if $\interpret{\subst{\asubst}{\alhs}} \neq \interpret{\subst{\asubst}{\arhs}}$}
    \label{dynlabrule:2}
  \end{align}
  (Note that 
  $\interpreta{\alhs}{\interpret{\asubst}} = \interpret{\subst{\asubst}{\alhs}}$, 
  and $\interpreta{\arhs}{\interpret{\asubst}} = \interpret{\subst{\asubst}{\arhs}}$ by Lemma~\ref{lem:xyz}.)
  
  In case $\interpret{\subst{\asubst}{\alhs}} \eq \interpret{\subst{\asubst}{\arhs}}$, 
  no relabeling is needed and we take $m = 0$:
  \begin{align*}
\dolabelg{\funap{\topsymb}{s}}
    & \stackrel{\eqref{eq:simpl:s}}{=} 
    \contextfill{\labacxt{\interpret{\subst{\asubst}{\alhs}}}}{\subst{\,\labasubst}{\dolabel{\alhs}{\interpret{\asubst}}}}
    \stackrel{\eqref{dynlabrule:1}}{\to_{\dynlaborg{\aalg}{\semlab}{\atrs},\samumap}}
    \contextfill{\labacxt{\interpret{\subst{\asubst}{\alhs}}}}{\subst{\,\labasubst}{\dolabel{\arhs}{\interpret{\asubst}}}}
    \stackrel{\refeqsimplt}{=} 
    \dolabelg{\funap{\topsymb}{t}}
\end{align*}
  
  If $\interpret{\subst{\asubst}{\alhs}} \neq \interpret{\subst{\asubst}{\arhs}}$, 
  we get:
  \begin{align*}
    \dolabelg{\funap{\topsymb}{s}}
    \stackrel{\eqref{eq:simpl:s}}{=} &\ 
    \contextfill{\labacxt{\interpret{\subst{\asubst}{\alhs}}}}{\subst{\,\labasubst}{\dolabel{\alhs}{\interpret{\asubst}}}}
    \\
    \stackrel{\eqref{dynlabrule:2}}{\to_{\dynlaborg{\aalg}{\semlab}{\atrs},\samumap}} &\ 
    \contextfill{\labacxt{\interpret{\subst{\asubst}{\alhs}}}}{\subst{\,\labasubst}{\relabel{\interpret{\subst{\asubst}{\alhs}}}{\interpret{\subst{\asubst}{\arhs}}}{\dolabel{\arhs}{\interpret{\asubst}}}}}
    = \dolabelg{\funap{\topsymb}{\contextfill{\acxt}{\relabelone{\interpret{\subst{\asubst}{\alhs}}}{\subst{\asubst}{\arhs}}}}}
  \end{align*} 
By Lemma~\ref{lem:relabel} the $\ssrelabel$ symbol can `walk' upward until it disappears,
  and at the latest it vanishes when it meets $\topsymb$. 
  Hence we have:
  \begin{align*}
    \dolabelg{\funap{\topsymb}{s}}
    \to_{\dynlaborg{\aalg}{\semlab}{\atrs},\samumap}
    \dolabelg{\funap{\topsymb}{\contextfill{\acxt}{\relabelone{\interpret{\subst{\asubst}{\alhs}}}{\subst{\asubst}{\arhs}}}}}
    \to_{\dynlabprp{\aalg}{\semlab}{\atrs},\samumap}^m
    \dolabelg{\funap{\topsymb}{\contextfill{\acxt}{\subst{\asubst}{\arhs}}}}
  \end{align*}
  for some $m \le \cdpth{\aalg}{\atrs}$ by Lemma~\ref{lem:relabel:bound}.
\end{proof}

\begin{theorem}\label{thm:dynlab:sound}
  Let $\atrs$ be a TRS over $\asig$, 
  and $\triple{\aalg}{\semlab}{\asigred}$ a sound \clabeling\ for $\atrs$.
  Then $\atrs$\ is outermost ground terminating if $\dynlab{\aalg}{\semlab}{\atrs}$ is terminating.
\end{theorem}
\proof
  Assume that $\atrs$\, admits an infinite outermost rewrite sequence of ground terms:
  \[
    t_1 \outred_{\atrs} t_2 \outred_{\atrs} t_3 \outred_{\atrs} \ldots
  \]
  Then from Lemma~\ref{lem:dynlab:sound} it follows that 
  $\dynlab{\aalg}{\semlab}{\atrs}$ admits an infinite rewrite sequence: $$
    \dolabelg{\funap{\topsymb}{t_1}} 
    \to_{\dynlab{\aalg}{\semlab}{\atrs},\samumap}^{+}
    \dolabelg{\funap{\topsymb}{t_2}} 
    \to_{\dynlab{\aalg}{\semlab}{\atrs},\samumap}^{+}
    \dolabelg{\funap{\topsymb}{t_3}} 
    \to_{\dynlab{\aalg}{\semlab}{\atrs},\samumap}^{+}
    \ldots\eqno{\qEd}
  $$


We note that Theorem~\ref{thm:dynlab:sound} can be strengthened
by weakening the termination of $\dynlab{\aalg}{\semlab}{\atrs}$
to local termination of $\dynlab{\aalg}{\semlab}{\atrs}$ on 
the set of correctly labeled ground terms without $\ssrelabel$ symbols.
Let us denote this set by $\dolabelg{\ter{\asig}{\setemp}}$:
\[
  \dolabelg{\ter{\asig}{\setemp}}
  =
  \{\dolabelg{t} \where t \in \ter{\asig}{\setemp} \}
\]

Theorem~\ref{thm:dynlab:complete} below states
that dynamic labeling is complete with respect to 
local termination on $\dolabelg{\ter{\asig}{\setemp}}$.
More precisely, outermost ground termination of $\atrs$ implies termination of 
$\dynlab{\aalg}{\semlab}{\atrs}$ on $\dolabelg{\ter{\asig}{\setemp}}$.
The following example helps to understand the proof of that theorem;
it illustrates that even when starting from terms in
$\dolabelg{\ter{\asig}{\setemp}}$, not every rewrite step in 
$\dynlab{\aalg}{\semlab}{\atrs}$ corresponds to an outermost step in $\atrs$.

\begin{example}\label{ex:dyncomplete}
  Let $\atrs$ consist of the following rules:
  \begin{align*}
    \fb{\tb}{x} &\to \ta &
    \fb{x}{\tb} &\to \ta \\
    \fb{\tb}{\tb} &\to \fb{\ta}{\ta} &
    \ta &\to \tb
  \end{align*}
  Moreover, let $\aalg = \{\bot,b\}$ with
  $\interpret{\ta} = \bot$, $\interpret{\tb} = b$,
  and $\bfunap{\interpret{\tf}}{x}{y} = \bot$ for all $x,y \in \aalg$.
  Labeling symbols with the value of their arguments, 
  we obtain for $\dynlaborg{\aalg}{\semlab}{\atrs}$:
  \begin{align*}
    \bfunap{\svoodoolabel{\tf}{b,\bot}}{\tb}{x} &\to \ta &
    \bfunap{\svoodoolabel{\tf}{\bot,b}}{x}{\tb} &\to \ta \\
    \bfunap{\svoodoolabel{\tf}{b,b}}{\tb}{x} &\to \ta &
    \bfunap{\svoodoolabel{\tf}{b,b}}{x}{\tb} &\to \ta \\
    \bfunap{\svoodoolabel{\tf}{b,b}}{\tb}{\tb} &\to \bfunap{\svoodoolabel{\tf}{\bot,\bot}}{\ta}{\ta} &
    \ta &\to \relabel{\bot}{b}{\tb}
  \end{align*}
  and for $\dynlabprp{\aalg}{\semlab}{\atrs}$:
  \begin{align*}
    \bfunap{\svoodoolabel{\tf}{\bot,\bot}}{\relabel{\bot}{b}{x}}{y} &\to \bfunap{\svoodoolabel{\tf}{b,\bot}}{x}{y} 
    & 
    \bfunap{\svoodoolabel{\tf}{\bot,\bot}}{x}{\relabel{\bot}{b}{y}} &\to \bfunap{\svoodoolabel{\tf}{\bot,b}}{x}{y}
  \end{align*}
  where $\sigredex{\asig} = \{\ta, \svoodoolabel{\tf}{b,\bot}, \svoodoolabel{\tf}{\bot,b}, \svoodoolabel{\tf}{b,b}\}$.
  Then we obtain the following rewrite sequence in $\dynlab{\aalg}{\semlab}{\atrs}$:
  \begin{align*}
    \dolabelg{\fb{\ta}{\ta}} =&\ \bfunap{\svoodoolabel{\tf}{\bot,\bot}}{\ta}{\ta}\\
    \to_{\dynlaborg{\aalg}{\semlab}{\atrs},\samumap} &\ 
      \bfunap{\svoodoolabel{\tf}{\bot,\bot}}{\relabel{\bot}{b}{\tb}}{\ta}\\
    \to_{\dynlaborg{\aalg}{\semlab}{\atrs},\samumap}&\ 
      \bfunap{\svoodoolabel{\tf}{\bot,\bot}}{\relabel{\bot}{b}{\tb}}{\relabel{\bot}{b}{\tb}}\\
    \to_{\dynlabprp{\aalg}{\semlab}{\atrs},\samumap}&\ 
      \bfunap{\svoodoolabel{\tf}{b,\bot}}{\tb}{\relabel{\bot}{b}{\tb}}
  \end{align*}
  The second step in this rewrite sequence
  does not correspond to an outermost step.
  Nevertheless, Theorem~\ref{thm:dynlab:complete}
  states that such `illegal' steps do not harm completeness of the transformation.
  The reason is that if the relabeling rules create a redex above
  some $\ssrelabel$ symbol, then this $\ssrelabel$ symbol
  is prevented from further propagating its information upward
  (until it becomes $\samumap$-replacing again).
  The crucial point is that above $\ssrelabel$ symbols
  the labels are unchanged, thus as if the step would not have taken place.
  Moreover, it is essential that $\dynlab{\aalg}{\semlab}{\atrs}$ 
  prohibits $\ssrelabel$ to propagate over symbols from $\sigredex{\asig}$.
  For instance, in the above example $\dynlab{\aalg}{\semlab}{\atrs}$
  does not contain a rule of the form:
  \begin{align}
    \bfunap{\svoodoolabel{\tf}{b,\bot}}{x}{\relabel{\bot}{b}{y}} &\to \bfunap{\svoodoolabel{\tf}{b,b}}{x}{y}
    \label{illrelabel}
  \end{align}
  This rule would cause non-termination: 
  \begin{align*}
    \bfunap{\svoodoolabel{\tf}{\bot,\bot}}{\ta}{\ta} 
    & \mured^2 
    \bfunap{\svoodoolabel{\tf}{\bot,\bot}}{\relabel{\bot}{b}{\tb}}{\relabel{\bot}{b}{\tb}} \\
    & \mured
    \bfunap{\svoodoolabel{\tf}{b,\bot}}{\tb}{\relabel{\bot}{b}{\tb}}
    \\
    & \stackrel{\text{\eqref{illrelabel}}}{\mured}
    \bfunap{\svoodoolabel{\tf}{b,b}}{\tb}{\tb}
    \\
    & \mured
    \bfunap{\svoodoolabel{\tf}{\bot,\bot}}{\ta}{\ta}
  \end{align*}
\end{example}

\begin{theorem}\label{thm:dynlab:complete}
  Let $\atrs$ be a left-linear TRS over $\asig$, 
  and $\triple{\aalg}{\semlab}{\asigred}$ a complete, maximal, and core \clabeling\ for $\atrs$.
  Then $\dynlab{\aalg}{\semlab}{\atrs}$ is terminating on the set of terms $\dolabelg{\ter{\asig}{\setemp}}$
  if $\atrs$ is outermost ground terminating.
\end{theorem}
\begin{proof}
  \newcommand{\sto}{\hookrightarrow}\newcommand{\nto}{\stackrel{\makebox(0,0){{\scriptsize$\neg\samumap$}}}{\longrightarrow}}Define $T = \dolabelg{\ter{\asig}{\setemp}}$,
  and ${\sto} = (\relcomp{ \to_{\dynlaborg{\aalg}{\semlab}{\atrs},\samumap} }{ \to^*_{\dynlabprp{\aalg}{\semlab}{\atrs},\samumap} }) \cap (T \times T)$.
  Note that the relation $\sto$ is restricted to terms which contain no $\ssrelabel$ symbols.
  Hence always the maximal number of relabeling rules is applied.
  It is clear that each ${\sto}$ rewrite step corresponds to an outermost rewrite step in the original TRS $\atrs$.
  Therefore it suffices to show that any infinite rewrite sequence
  $t = t_0 \muredr{\dynlab{\aalg}{\semlab}{\atrs}} t_1 \muredr{\dynlab{\aalg}{\semlab}{\atrs}} \ldots$
  gives rise to an infinite rewrite sequence $t = s_0 \sto s_1 \sto\ldots$.
  We prove the claim by a kind of standardization of reductions.
We first classify the rules from $\dynlab{\aalg}{\semlab}{\atrs}$:
  \begin{align*}
  \dolabel{\ell}{\alpha} &\to \dolabel{r}{\alpha}
  \tag{\ensuremath{c1}}\label{dyn1}\\
  \dolabel{\ell}{\alpha} &\to \relabel{\interpreta{\ell}{\alpha}}{\interpreta{r}{\alpha}}{\dolabel{r}{\alpha}}
        \tag{\ensuremath{c2}}\label{dyn2}\\
  \funap{\svoodoolabel{\tf}{\funap{\slabelf{\tf}}{\vec{a},b,\vec{c}}}}{\vec{x},\relabel{b}{b'}{y},\vec{z}}
  &\to
  \funap{\svoodoolabel{\tf}{\funap{\slabelf{\tf}}{\vec{a},b',\vec{c}}}}{\vec{x},y,\vec{z}}
  \tag{\ensuremath{c3}}\label{dyn3}\\
  \funap{\svoodoolabel{\tf}{\funap{\slabelf{\tf}}{\vec{a},b,\vec{c}}}}{\vec{x},\relabel{b}{b'}{y},\vec{z}}
  &\to
  \relabel{d}{d'}{\funap{\svoodoolabel{\tf}{\funap{\slabelf{\tf}}{\vec{a},b',\vec{c}}}}{\vec{x},y,\vec{z}}}
        \tag{\ensuremath{c4}}\label{dyn4}
  \end{align*}
For $i = 0,1,\ldots$, we analyse the steps $t_i \to_{\dynlab{\aalg}{\semlab}{\atrs},\samumap} t_{i+1}$
  and construct $s_0 \sto s_1 \sto \ldots \sto s_j$
  in such a way that $s_j \nto_{\ref{dyn2},\ref{dyn4}}^* t_{i+1}$
  where we use $\nto_{\ref{dyn2},\ref{dyn4}}$ to denote standard term rewriting 
  ignoring the replacement map $\samumap$, and using rules from \eqref{dyn2} and \eqref{dyn4} only.
  Observe that then the maximal prefix $C_{i+1}$ of $t_{i+1}$ 
  not containing $\ssrelabel$ symbols is also a prefix of $s_j$
  (since everything changed by \eqref{dyn2} and \eqref{dyn4} is `hidden' inside a $\ssrelabel$ symbol).
  We begin with $t = s_0$, and $i = j = 0$.
  For $i = 0,1,\ldots$, we consider the step $\tau_i : t_i \to_{\dynlab{\aalg}{\semlab}{\atrs},\samumap} t_{i+1}$.

  If $\tau_i$ is a step with respect to a rule from:
  \begin{enumerate}[$-$]

  \item
\eqref{dyn2} or \eqref{dyn4}, then
  we append $\tau_i$ to the rewrite sequence 
  $s_j \nto_{\ref{dyn2},\ref{dyn4}}^* t_i$
  yielding 
  $s_j \nto_{\ref{dyn2},\ref{dyn4}}^* t_{i+1}$.
  Note that this leaves the $\sto$-rewrite sequence $s_0 \sto^{\ast} s_j$ untouched.

  \item
\eqref{dyn1},
  then the pattern of $\tau_i$ lies entirely in $C_i$ which is also prefix of $s_j$.
  Then we append $\tau_i$ to $s_0 \sto^{\ast} s_j$ (using left-linearity of $\atrs$) 
  yielding $s_0 \sto^{\ast} s_j \mstackrel{\tau_i}{\sto} s_{j+1}$.
  We have $s_{j+1} \nto_{\ref{dyn2},\ref{dyn4}}^* t_{i+1}$
  by orthogonal projection of the steps $s_j \nto_{\ref{dyn2},\ref{dyn4}}^* t_i$
  over $s_j \mstackrel{\tau_i}{\sto} s_{j+1}$ (all steps in $s_j \nto_{\ref{dyn2},\ref{dyn4}}^* t_i$ are below the prefix $C_i$).
  
  \item
\eqref{dyn3}, then a $\ssrelabel$ symbol `disappears'.
  We can trace this symbol back to a sequence of steps
  $\sigma_i \funin s_j \relcomp{ \nto_{\ref{dyn2}} }{ \nto_{\ref{dyn4}}^+ } s_j'$,
  that is, it must have been created in $s_j$ by a \eqref{dyn2} step,
  followed by a number of \eqref{dyn4} steps.
  We combine $\sigma_i$ and $\tau_i$ to a $\sto$ step,
  yielding $s_0 \sto^* s_j \mstackrel{\relcomp{\sigma_i}{\tau_i}}{\sto} s_{j+1}$.
  Then $s_{j+1} \nto_{\ref{dyn2},\ref{dyn4}}^* t_{i+1}$
  as the remaining steps from
  $s_j \nto_{\ref{dyn2},\ref{dyn4}}^* t_i$ are not harmed by the permutation (performing $\sigma_i$ first).

  \end{enumerate}

  \noindent
  It remains to be shown that the constructed sequence $s_0 \sto s_1 \sto s_2 \sto \ldots$ is infinite.
  This follows from the fact that an infinite number of steps in
  $t_0 \to_{\dynlab{\aalg}{\semlab}{\atrs},\samumap} t_1 \to_{\dynlab{\aalg}{\semlab}{\atrs},\samumap} \ldots$
  must be of type \eqref{dyn1} or \eqref{dyn3}.
  This is a direct consequence of the fact that
  $\to_{\ref{dyn2},\ref{dyn4}}$ is terminating
  (with every step the prefix in which rewriting is allowed gets smaller).
\end{proof}

The following example demonstrates why the completeness result for dynamic labeling (Theorem~\ref{thm:dynlab:complete}) 
is restricted to the set $\dolabelg{\ter{\asig}{\setemp}}$ of correctly labeled terms 
which do not contain $\ssrelabel$ symbols.
The point is that, although the original TRS is outermost terminating
the transformed system may in general be non-terminating 
due to the existence of `non-reachable' terms.
\begin{example}\label{ex:dynlab:noncomplete}
  We consider the following term rewriting system $\atrs$:
  \begin{align*}
    \ta &\to \tb
    &
    \fb{\tb}{y} &\to \tb
    &
    \fb{\fun{c}}{y} &\to \h{\fb{y}{y}} 
    \\
    \h{\fb{x}{\tb}} &\to \tb
    &  
    \h{\fb{x}{\fun{c}}} &\to \tb
  \end{align*}
  We explain why this TRS is outermost ground terminating.
  Without the rule $\rho : \fb{\fun{c}}{y} \to \h{\fb{y}{y}}$, 
  the system would even be terminating.
  Now note that the rule $\rho$ can only be applied once 
  to each occurrence of $\fb{\fun{c}}{\cxthole}$
  since $\h{\fb{t}{t}} \to^* \h{\fb{\fun{c}}{t'}}$ implies that $t = \fun{c}$,
  and then the rule $\h{\fb{x}{\fun{c}}} \to \tb$ has priority by the strategy of outermost rewriting.

  We define a maximal, complete, core \clabeling{} 
  $\triple{\aalg}{\asemlab}{\asigred}$ for $\atrs$
  (isomorphic to the result of the construction given in the next section) where the algebra $\aalg = \{ \mit{bc}, \mit{fbc}, \bot \}$ 
  with the interpretation function defined by:
  \begin{align*}
  \bfunap{\interpret{\tf}}{x}{\mit{bc}} &= \mit{fbc}
  &  
  \interpret{\tb} & = \interpret{\fun{c}} = \mit{bc}
  \\
  \bfunap{\interpret{\tf}}{x}{y} &= \bot
  &
  \interpret{\ta} & = \funap{\interpret{\sh}}{x} = \bot 
  \end{align*}
  for all $x,y \in \aalg$ with $y \ne \mit{bc}$,
  and with $\asigred = \{ \ta , \svoodoolabel{\tf}{\mit{bc}} , \svoodoolabel{\sh}{\mit{fbc}} \}$.


  The dynamic labeling $\dynlab{\aalg}{\semlab}{\atrs}$ of $\atrs$ with respect to this \clabeling{} then includes the rules:
  \begin{align*}
    \ta &\to \relabel{\bot}{\mit{bc}}{\tb} \\
    \bfunap{\svoodoolabel{\tf}{\mit{bc},\bot}}{\fun{c}}{y} &\to \funap{\svoodoolabel{\sh}{\bot}}{\bfunap{\svoodoolabel{\tf}{\bot,\bot}}{y}{y}} \\
    \bfunap{\svoodoolabel{\tf}{\bot,\bot}}{\relabel{\bot}{\mit{bc}}{x}}{y} &\to \bfunap{\svoodoolabel{\tf}{\mit{bc},\bot}}{x}{y}
  \end{align*}
  Now the context-sensitive TRS $\dynlab{\aalg}{\semlab}{\atrs}$ 
  admits the following infinite rewrite sequence:
  \begin{align*}
    \bfunap{\svoodoolabel{\tf}{\mit{bc},\bot}}{\fun{c}}{\relabel{\bot}{\mit{bc}}{\fun{c}}} 
    &\to_{\dynlab{\aalg}{\semlab}{\atrs},\samumap}
    \funap{\svoodoolabel{\sh}{\bot}}{\bfunap{\svoodoolabel{\tf}{\bot,\bot}}{\relabel{\bot}{\mit{bc}}{\fun{c}}}{\relabel{\bot}{\mit{bc}}{\fun{c}}}}\\
    &\to_{\dynlab{\aalg}{\semlab}{\atrs},\samumap}
    \funap{\svoodoolabel{\sh}{\bot}}{\bfunap{\svoodoolabel{\tf}{\mit{bc},\bot}}{\fun{c}}{\relabel{\bot}{\mit{bc}}{\fun{c}}}}\\
    &\to_{\dynlab{\aalg}{\semlab}{\atrs},\samumap}
    \ldots
  \end{align*}
Observe that this anomaly is caused by the subterm $\relabel{\bot}{\mit{bc}}{c}$,
  which is not reachable from any term in $\dolabelg{\ter{\asig}{\setemp}}$.
\end{example}

\begin{remark}
Theorem~\ref{thm:dynlab:complete} 
  states completeness of dynamic labeling with respect to 
  local termination on the set of terms $\dolabelg{\ter{\asig}{\setemp}}$.
We briefly indicate how the theorem can be generalized 
  to termination on $\ter{\labelsig{\asig}}{\setemp}$
  by altering the definition of $\dynlab{\aalg}{\semlab}{\atrs}$.
  Note that
  $\dolabelg{\ter{\asig}{\setemp}} \subsetneq \ter{\labelsig{\asig}}{\setemp}$.
  In particular,
  the set $\ter{\labelsig{\asig}}{\setemp}$ includes terms that are not correctly labeled.
  The necessary modification of the definition of dynamic labeling concerns the elimination of collapsing rules $\alhs \to x$.
  This can be achieved by wrapping the right-hand side into $\relabel{a}{a}{\cxthole}$
  even when the interpretations of the left and right-hand side are equal.
  Additionaly, we let the symbols $\srelabel{a}{a}$ disappear after one relabeling step.
  By an application of Theorem~\ref{thm:adapt:ohlebusch} it then follows that 
  termination on $\dolabelg{\ter{\asig}{\setemp}}$ 
  coincides with termination on $\ter{\labelsig{\asig}}{\setemp}$.
\end{remark}
