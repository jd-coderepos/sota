\begin{figure}[ht]
    \centering
    \includegraphics[width=0.49\columnwidth]{figures/court-detection/bad0.png}\linebreak[0]
    \includegraphics[width=0.49\columnwidth]{figures/court-detection/improved0.png}\linebreak[0]
    \includegraphics[width=0.49\columnwidth]{figures/court-detection/bad1.png}\linebreak[0]
    \includegraphics[width=0.49\columnwidth]{figures/court-detection/improved1.png}\linebreak[0]
    \caption{{\bf Our court detection model is more robust} than the baseline Farin \etal method~\cite{FKWE04}. \emph{Left}: Courts detected with \cite{FKWE04}. \emph{Right}: Courts detected with our model. These two images are from additional annotated matches outside of the TrackNet dataset.
    \label{fig:court-improvements}}
\end{figure}

\section{Automated 3D trajectory reconstruction}
\label{sec:annotation_pipeline}
In this section, we present each part of our system in detail. For each part of the system, we review the existing state-of-the-art in the area, and discuss modification specific to our work where appropriate. As previously described, our system currently only analyzes singles rallies recorded with a fixed camera from an approximate ``broadcast view" (see~\Cref{fig:test-matches} for examples of this view).

\subsection{Court detection}
\label{sec:court-detection}

We base our approach for detecting the court on the model-based algorithm of Farin \etal~\cite{FKWE04,chen_physics-based_2009,LLHG06,CTLY12}. Unfortuntately, we found that the original algorithm fails on many videos in our dataset where the viewing angle is off-center. We propose a graph-based approach to overcome this issue and significantly boost the performance of this algorithm.

We briefly illustrate the original Farin's algorithm~\cite{FKWE04} before introducing our modification. Given an image of the court, candidate lines are first detected using pixel color thresholding followed by application of the Hough transform (see~\Cref{fig:court-detect}). The detected lines are then split into a set  of horizontal lines (those with slopes between -25 and 25 degrees) and  of vertical lines (those with slopes between 60 and 120 degrees). Finally, a combinatorial search is conducted to match a known court layout to the candidate lines, with each search iteration picking two horizontal and two vertical lines for a reference rectangle in the known layout. This results in a  algorithm. Unfortunately, this algorithm fails in about 24\% of videos in our dataset. We found the main culprit to be the hard angle constraints set when partitioning results in line misclassification, which ultimately break the algorithm. This happens most frequently when the angle of the camera is not ideal or when the reference rectangle of the court is not visible (see~\Cref{fig:court-improvements} for examples).

To avoid this issue, we propose a new partitioning algorithm that is free of hard-coded angle constraints. To do this, we frame the problem as a maximum weight bipartite subgraph identification. We represent each line as a node in a complete graph, and for every pair of lines  and , we connect the nodes with an edge of weight  ( is set to be a small constant, e.g. ), where  is the angle between lines  and  closest to . Then we greedily try to partition the graph into two sets of vertices  and  such that the weight between the two sets is maximized. This weight function encourages a partition where the lines in the two parts are roughly orthogonal to each other.

Our implementation is developed on top of the reference code provided by~\cite{court_detect}. To assess the accuracy of the court detection, we measure two metrics over a manually annotated dataset: (i) the average pixel error over all detections, and (ii) the percentage of successful detections. We define a court detection as successful if the IoU of the detected court and ground truth is . On our dataset, our proposed approach increases the success rate of court detection from 73.9\% to 85.5\%, and decreased the average detection time by a factor of 40 while achieving an higher average IoU of 0.97 (vs. 0.96 from the original method).

\begin{figure*}[ht]
\includegraphics[height=0.13\textwidth]{figures/court-detection/1.png}\linebreak[0]
\includegraphics[height=0.13\textwidth]{figures/court-detection/2.png}\linebreak[0]
\includegraphics[height=0.13\textwidth]{figures/court-detection/3.png}\linebreak[0]
\includegraphics[height=0.13\textwidth]{figures/court-detection/5.png}\linebreak[0]
\includegraphics[height=0.13\textwidth]{figures/court-detection/6.png}\linebreak[0]
\caption{{\bf Stages of our court detection} includes applying pixel color thresholding (second) and a Hough transform to obtain line candidates (third), and then partitioning these lines (fourth) in order to efficiently search for the correct court layout (last). \label{fig:court-detect}}
\end{figure*}









\subsection{Pose estimation}
\label{sec:pose-estimation}
We perform pose estimation using a top-down HRNet model~\cite{hrnet} to compute per-frame poses through the {\tt mmpose} framework~\cite{mmpose}. To track poses, instead of using methods developed for unstructured environments~\cite{girdhar_detect-and-track_2018} or recurrent network-based methods~\cite{sadeghian_tracking_2017}, we simply leverage the detected court as a strong cue. We filter all detected poses that do not have feet in the court, and identify near and far players based on their distance to the camera. This strategy is effective due to the fact that no one other than the players can step onto the court, and that players do not switch side during a point. To accommodate jumping motions, which would misplace the player to a deeper position than they actually are, we make two modifications to increase robustness. Firstly, we relax the court boundary slightly. Secondly, if a side of the court has no pose within it, we find the pose closest to the last in-court pose recorded on that side. For all of our videos, this simple approach identifies the two players on every frame.



\subsection{Shuttle detection}
\label{sec:shuttle-tracking}
\begin{figure}[ht]
    \centering
    \includegraphics[width=\columnwidth]{figures/shuttle-arch.pdf}
    \caption{{\bf Architecture of our shuttle detection model} is based on a modified U-net, where we added residual connections and use the weighted dice and binary cross-entropy loss.  represents addition and  represents concatenation.}
    \label{fig:our-tracknet}
\end{figure}

We formulate the shuttle detection problem as semantic segmentation and use a U-net style architecture (\Cref{fig:our-tracknet}) inspired by TrackNetV2~\cite{sun_tracknetv2_2020}. Similar to TrackNetV2, to encourage the network to learn the temporal context, we use a 3-in-3-out architecture that predicts the shuttle masks for three consecutive frames simultaneously (each resized to 288-by-512). However, our model has significantly smaller footprint than the original TrackNetV2 (2.9M parameters in our model versus 11.3M parameters), making it faster to train and perform inference, and higher accuracy ( accuracy from  in the original). This improvement is credited to two main changes we introduced. First, we added residual connections to each convolutional layers (see~\Cref{fig:our-tracknet}). Second, instead of a binary focal loss, we use a weighted combination of the dice loss and the binary cross-entropy to mitigate the input imbalance problem of tiny shuttlecock, inspired by \cite{combo_loss}. Given per-pixel prediction  and ground truth label , the loss  we use is

where  is the blending coefficient, and  is a small constant for numerical stability. Throughout our experiments, we use  and . To generate the final shuttlecock location, we threshold  at  to produce a binary mask per frame, and then use the centroid of the largest connected component in the mask. If no pixels are above  or the area of the component is not large enough, we report the shuttle is undetected.

To improve training, we trained the first few epochs of our network with distillation learning~\cite{HVD15} using parameters from TrackNetV2. In total we used 10 distillation epochs and 40 training epochs with an Adadelta optimizer. Standard augmentation such as random rotations, shears, and zooms were applied.



\subsection{Shot segmentation}
\label{sec:shot_segmentation}
\begin{figure}[ht]
    \centering
    \includegraphics[trim={0cm, 0.75cm, 0cm, 0.75cm}, clip, page=2,width=0.7\columnwidth]{figures/hitnet.pdf}
    \caption{{\bf Architecture of our hit detection model} is based on a simple GRU-based recurrent network that consumes court, pose, and 2D shuttlecock information to make hit predictions.}
    \label{fig:hitnet}
\end{figure}

Identifying the shots of a rally is critical for reconstructing 3D trajectories (\S\ref{sec:3d-recon}) and other downstream applications. A shot happens when a player hits the shuttle with her racket, and ends right before the opposing player hits the shuttle or if the shuttle hits the court. Therefore, in order the segment the video into successive shots, it is equivalent to identify the \emph{hit} events.

We formulate the hit detection as a multi-class classification problem. Since we are focusing on singles matches only, this is a three way classification with labels \emph{no hit}, \emph{near player hit}, and \emph{far player hit} predicted at each frame.

\paragraph{HitNet: Hit detection architecture} In all racket sports, hits at certain parts of the court (e.g. the side lines) occurs more often than the others. Moreover, due to the need to efficiently translate power to the ball, athletes have very consistent poses when hitting, and have to perfect their positioning with respect to the ball. The higher the level of the athlete, the higher this consistency. As a result, we hypothesize that the court layout, the poses, and the shuttlecock location play an important role in identifying hits.

We propose a recurrent model that leverages the previously computed court layout (\S\ref{sec:court-detection}), poses (\S\ref{sec:pose-estimation}), and shuttlecock positions (\S\ref{sec:shuttle-tracking}) and their temporal tracks to predict hits. For each frame of the video, we create a feature vector comprising of the pixel coordinates of the court corners, the two players' poses, and the location of the shuttle, and normalize them jointly in the  and  directions. The features are embedded into 32 dimensions using a fully-connected layer, before feeding into the recurrent unit. The recurrent unit comprises of two GRU layers that takes 12 frames at a time, and predict whether a hit occurred between frame 7 to 12. We then feed the last token embedding of this recurrent unit into a small fully connected layer before passing it to the softmax layer to generate confidence scores. The architecture is shown in~\Cref{fig:hitnet}. The network is compact in size (around 16K parameters) and can reach  accuracy. The network can performance inference over several thousand frames a second. To show that each feature (court, pose, shuttle) contributes significantly to the output accuracy, we perform ablation studies in \Cref{tab:hit-detection}. 

The training is done on our dataset with standard data augmentation. Artificial noise is also added in the pixel coordinates 5\% of the time to simulate noise in the pipeline. Cross-entropy is used as the loss function with the Adam optimizer. Learning rate is set at a constant . For the input normalization, we scale all -pixel coordinates and -pixel coordinates of the feature (court, pose, and shuttlecock) to the interval , and set undetected shuttle and pose coordinates to . All features were normalized together to ensure that the spatial relationship is preserved. Finally, due to the class-imbalance between hit and no-hit events, we rebalance the dataset to ensure an equal number of each type of event were used in the dataset.






\paragraph{Constrained optimization of the network output}
Next, we incorporate badminton domain knowledge to further optimize the HitNet output. The optimization is based on imposing several constraints. In particular,
\begin{itemize}[noitemsep]
\item[I.] We know the approximate number of hits for a given rally. Based on typical shuttle speeds and the court size, the average time between hits is around 1 second, implying that a rally lasting  seconds has approximately  hits.
\item[II.] No two hits can be too close in time. Empirically, we found half a second to be a good threshold. In the TrackNetV2 dataset, no two hits are within 0.5s of each other.
\item[III.] Hits must be alternating between opposing players, hence no two adjacent hits should be classified to the same player.
\end{itemize}
Our optimization aims to find a set of hits roughly maximizing the sum of confidence scores subject to these constraints.

More formally, given  frames and a set of per-frame confidence scores , our goal is to associate each frame with a label  indicating no hit, or a hit by one of the players. Since the three scenarios are mutually exclusive, it suffices to label all the frames on which hits occur (the rest will be labeled as no hits). 

Let  denote the frame indices where a hit has occurred, with  total hits. Suppose the total duration of the video is  seconds (implying  hits on average), and the video is at  fps. We seek to maximize the following objective:

where  is a parameter that encourages the algorithm to use fewer than  hits if possible. Without , the algorithm will always use exactly  hits as none of the confidence scores are negative. In practice, we set  to be the mean of  across all the frames. The first two inequalities enforces constraint I and II, and the third enforces the shuttle to be hit by alternating players. The global optimum of the objective above can be found by standard dynamic programming running in  time.

In \Cref{tab:hit-detection} we compare our model against a naive baseline that simply detects a hit when the second derivative of either the shuttle -coordinate or -coordinate exceeds a predefined threshold (tuned for maximum accuracy). The locations of large second derivative indicate ``discontinuities" in the velocity that occur when a shuttle is struck. To measure the effectiveness of our constrained optimization, we further compare against a naive post-processing which simply classifies using the largest confidence score while ensuring that no two hits are within 0.5 seconds of each other. If two frames are classified as hits and are within 0.5 seconds of each other, the earlier one is classified as a hit and the later one is classified as a no hit.

To measure the accuracy, recall, and precision of the models, we only look at frames where hits occurred. If we were to include all frames, then a trivial detector outputting no hit for all frames would get close to 100\% accuracy. To be precise, suppose the ground truth has hit-player pairs  indicating that player  hit the shutte on frame , and the model predicts . The metrics we use are:

As \Cref{tab:hit-detection} shows, our model offers a substantial () accuracy improvement of the hit detection over the naive model.




\begin{table}[t]
\caption{{\bf Comparison of HitNet over baselines and ablation study} shows that HitNet benefits from all input features with the optimization-based postprocessing. Derivative-based method attempts to detect hits by thresholding on trajectory derivatives, and RF is a random forest-based classifier. HitNet is our model. \label{tab:hit-detection}}
\centering
\begin{tabular}{l|llll}
\toprule
           & recall & acc. & prec. & f1 \\ 
\hline
Derivative-based     & 65.7\%       & 53.8\%       & 74.7\%        & 0.699 \\ 
\hline
RF                   & 65.1\%       & 57.4\%       & 83.0\%        & 0.730 \\
\hline
HitNet (Shuttle)      & 70.2\%      & 68.8\%       & {\bf 97.4\%}  & 0.815 \\
HitNet (Shuttle+Pose) & 73.9\%      & 75.6\%       & 97.1\%  & 0.850 \\
HitNet (All)          & 78.1\%      & 76.4\%       & 97.2\%  & 0.866 \\
\hline
HitNet+optimization  & {\bf 94.3\%} & {\bf 89.7\%} & 94.9\%        & {\bf 0.946} \\
\bottomrule
\end{tabular}
\end{table}


\subsection{3D Reconstruction}
\label{sec:3d-recon}

With the shots segmented (\S\ref{sec:shot_segmentation}), we can now independently reconstruct trajectories for each shot. We pose this as a constrained nonlinear optimization problem.

\paragraph{Physics-based trajectory estimation} The ability to reconstruct shot-by-shot offers great advantage because without the discontinuous forces applied to the shuttle, the 3D trajectory, , can simply be approximated\footnote{The rotation and spin of the shuttle also affects the motion, which can be accounted for with more complicated models. However, we find the simple drag model to be sufficient for our purposes.} by a particle under drag~\cite{cohen_physics_2015}:

with initial position , velocity , and the drag coefficients .  is a constant representing the gravitational acceleration. Given , , and , we can integrate this differential equation to get . Note  can change from shot to shot, as the shuttlecock feathers slowly break over the course of a rally. 

\paragraph{Estimating the initial conditions}
The problem for the above equation is of course that the initial conditions are unknown. However, note that given 3D trajectory estimates and camera parameters, we can project  to image space to obtain 2D trajectory estimates  using the Direct Linear Transform~\cite{abdel2015direct}. This requires  known 3D coordinates, which we have via the  boundary court corners detected in \S\ref{sec:court-detection} plus the  tip points on the net poles\footnote{The 2D frame position of the pole is found by orthogonally projecting the midpoints of the sidelines up towards the closest white line that is approximately parallel to the back line of the court}. Given the camera parameters, we can measure how good a given 3D trajectory is by measuring the reprojection error:

where  is the tracked 2D coordinates of the shuttlecock we introduced in \S\ref{sec:shuttle-tracking}. This problem can then be solved with a non-linear regression optimizer until we find a good set of initial conditions.

\paragraph{3D trajectory reconstruction algorithm}

The vanilla version of our reconstruction algorithm is therefore built on solving \Cref{eq:drag}, and refining the initial conditions by reprojecting the solution back to image space using \Cref{eq:reprojection_loss}. 

The reconstruction can be greatly improved by incorporating additional priors. We can provide priors on the start and end positions of the shot through the players' poses. We can also penalize the unlikely event that the shot goes out by extending the trajectory of the shuttle until it hits the ground.\footnote{In professional play, the shuttle rarely goes out by more than a few inches.} The final loss we use is

where  and  are the 3D position estimates of the hitting and receiving players. These are estimated using their feet position at the time of hit and receive, respectively, with a vertical height estimation of \SI{2}{m}.  is the distance out of the court if we extend the trajectory of the shuttle until it hits the ground. If the shuttle lands inside the court, . We found the use of  is quite important, as it helpfully rules out the cases where the shuttlecock shoots out the back or side of the court in a way that cannot be penalized by the reprojection loss.  is used to adjust the closeness of the reprojection to the initial and final coordinate guesses. In practice, we use , where  is the camera projection matrix. This choice of  allows us to compensate for the fact that  is measured in image coordinates, while the other three terms in \Cref{eq:final_3d_loss} are measured in 3D world coordinates. This optimization is solved with some domain-specific constraints:
\begin{itemize}
    \item The initial and final 3D shuttle coordinates should have height less than 3 metres, i.e., .
    \item The initial velocity of the shuttle is less than 426 kph, or roughly 120 m/s, i.e., .
    \item The initial coordinates of the shuttle is on the same side as the player that hit it. Since the court is 13.4 metres long, this constraint implies that ,  if the closer player hit the shuttle, and  if the farther player hit the shuttle.
    \item The initial velocity is towards the opposing player, i.e. .
\end{itemize}

