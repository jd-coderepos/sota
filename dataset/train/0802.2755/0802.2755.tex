\documentclass[11pt]{article}

\setlength{\headheight}{0mm}
\setlength{\oddsidemargin}{-0mm}
\setlength{\topmargin}{-2mm}        
\setlength{\textwidth}{160mm}
\setlength{\textheight}{230mm}       

\usepackage{theorem}
\usepackage{graphicx}
\usepackage{bm}
\usepackage{amsmath}
\usepackage{amssymb}
\usepackage{algorithm}
\usepackage{algorithmic}
\usepackage{tabularx}

\newcounter{ichi}
\newcounter{ni}
\newcounter{san}
\newcounter{yon}

\theoremstyle{plain}
\newtheorem{theorem}{Theorem}[section]
\newtheorem{lemma}[theorem]{Lemma}\newtheorem{corollary}[theorem]{Corollary}\newtheorem{definition}[theorem]{Definition}\newtheorem{proposition}[theorem]{Proposition}\newtheorem{claim}[theorem]{Claim}\newtheorem{fact}[theorem]{Fact}\newtheorem{example}{Example}
\newsavebox{\ProofSym}
\savebox{\ProofSym}{\begin{picture}(7,7)
    \put(0,0){\framebox(6,6){}}
    \put(0,2){\framebox(4,4){}}
  \end{picture}}
\newcommand{\eop}{\hfill \usebox{\ProofSym}} 
\newenvironment{proof}{\noindent {\it Proof.}}{\eop\par\vspace{0.3cm}}
\newenvironment{proof2}[1]{\noindent {\it Proof of #1.}}{\eop\par\vspace{0.3cm}}
\newenvironment{proof3}{\noindent {\it Sketch of Proof.}}{\eop\par\vspace{0.3cm}}

\title{Covering Directed Graphs by In-trees}
\author{Naoyuki {\sc Kamiyama}\thanks{
Department of Architecture and Architectural Engineering,
Kyoto University, 
Kyotodaigaku-Katsura, Nishikyo-ku,
Kyoto, 615-8540, Japan.
E-mail {\ttfamily is.kamiyama@archi.kyoto-u.ac.jp} 
Supported by JSPS Research Fellowships for Young Scientists.}
\and 
Naoki {\sc Katoh}\thanks{Supported by the project {\it New Horizons in
Computing}, Grant-in-Aid for Scientific Research 
on Priority Areas, MEXT Japan.
Department of Architecture and Architectural Engineering,
Kyoto University, 
Kyotodaigaku-Katsura, Nishikyo-ku,
Kyoto, 615-8540, Japan.
E-mail {\ttfamily naoki@archi.kyoto-u.ac.jp} 
}
}
\date{\today}

\begin{document}

\maketitle

\begin{abstract}
\noindent
Given a directed graph  with a set of  specified vertices
 and a function  where  denotes the set of non-negative integers, 
we consider the problem which asks whether there exist  in-trees denoted by 
 for every  such that
 are rooted at , each 
spans vertices from which  is reachable and the union of all arc sets of
 for  and  covers . In this paper, we prove that such set of 
in-trees covering  can be found by using an algorithm for the weighted matroid intersection problem
in time bounded by a polynomial in  and the size of . 
Furthermore, for the case where  is acyclic, 
we present another characterization of the existence of in-trees covering ,  
and then we prove that in-trees covering  can be 
computed more efficiently than the general case by finding maximum matchings in a series of bipartite graphs.
\end{abstract}

\section{Introduction}


The problem for covering a graph by subgraphs with specified properties (for example, trees or paths) is
very important from practical and theoretical viewpoints and have
been extensively studied. For example, Nagamochi and Okada~\cite{NO07}
studied the problem for covering a set of vertices of a given undirected
tree by subtrees, and Arkin et al.~\cite{AHL06} studied the problem  
for covering a set of vertices or edges of a given undirected graph by
subtrees or paths. These results were motivated by vehicle routing problems. 
Moreover, Even et al.~\cite{EGKRS04} studied the covering problem
motivated by nurse station location problems. 

This paper studies the problem for covering a directed graph by rooted trees which is 
motivated by the following evacuation planning problem. 
Given a directed graph which models a city, 
vertices model intersections and buildings,
and arcs model roads connecting these intersections and buildings.
People exist not only at vertices but also along arcs. 
Suppose we have to give several evacuation instructions for
evacuating all people to some safety place. In order to avoid disorderly
confusion, it is desirable that one evacuation instruction gives a single evacuation path
for each person and these paths do not cross each other. 
Thus, we want each evacuation instruction to become an
in-tree rooted at some safety place. Moreover, the number of instructions
for each safety place is bounded in proportion to a size of each safety place. 

The above evacuation planning problem is formulated as the following
covering problem defined on a directed graph. 
We are given a directed graph  
which consists of a vertex set , an arc set , a set of  specified vertices
 and a function  where  denotes the set of non-negative integers. 
In the above evacuation planning problem,  corresponds to a set of safety
places, and  represents the upper bound of
the number of evacuation instructions for . 
For each , we define  as the
set of vertices in  from which  is reachable in , and we define an in-tree
rooted at  which spans  as a {\it -in-tree}.
We define a set  of  subgraphs of  
as a {\it -canonical set of in-trees}
if  contains exactly  -in-trees for every . 
If every two distinct in-trees of a -canonical set  
of in-trees are arc-disjoint, we call  a {\it 
-canonical set of arc-disjoint in-trees}. 
Furthermore, if the union of arc sets of all
in-trees of a -canonical set  of in-trees 
is equal to , we say that  {\it covers} . 

Four in-trees illustrated in Figure~\ref{fig:covering2}
compose a -canonical set  of in-trees which covers the arc set of a
directed graph   illustrated in Figure~\ref{fig:covering1}(a) where 
, ,  and . However,  
is not a -canonical set of arc-disjoint in-trees. 
\begin{figure}[h]
\begin{minipage}{0.5\hsize}
\begin{center}
\includegraphics[width=5cm]{covering1.eps}
\par(a)
\end{center}
\end{minipage}
\begin{minipage}{0.5\hsize}
\begin{center}
\includegraphics[width=5cm]{covering2.eps}
\par(b)
\end{center}
\end{minipage}
\caption{\small (a) Directed graph  . (b) Transformed graph .}
\label{fig:covering1}
\end{figure}
\begin{figure}[h]
\begin{minipage}{0.245\hsize}
\begin{center}
\includegraphics[width=3.5cm]{covering3.eps}
\par(a)
\end{center}
\end{minipage}
\begin{minipage}{0.245\hsize}
\begin{center}
\includegraphics[width=3.5cm]{covering4.eps}
\par(b)
\end{center}
\end{minipage}
\begin{minipage}{0.245\hsize}
\begin{center}
\includegraphics[width=3.5cm]{covering5.eps}
\par(c)
\end{center}
\end{minipage}
\begin{minipage}{0.245\hsize}
\begin{center}
\includegraphics[width=3.5cm]{covering6.eps}
\par(d)
\end{center}
\end{minipage}
\caption{\small (a) -in-tree. (b) -in-tree. (c) -in-tree. (d) -in-tree.}
\label{fig:covering2}
\end{figure}

We will study the problem for 
{\it covering directed graphs by in-trees} (in short ), 
and we will present characterizations for a directed graph  
for which there exists a feasible solution of , and 
a polynomial time algorithm for . 
\begin{center}
\begin{tabularx}{150mm}{rX}
\hline
{\bf Problem} &  
\\
\hline 
{\bf Input} & a directed graph ; \\
{\bf Output} &
a -canonical set of in-trees which covers the arc set of , if one exists.\\
\hline
\end{tabularx}
\end{center}
A special class of the problem  in which  consists of a single vertex 
was considered by Vidyasankar~\cite{V78}. 
He showed the necessary and sufficient condition 
in terms of linear inequalities 
that there exists a feasible solution of this problem
(a weaker version was shown by Frank~\cite{F79}).
However, to the best of our knowledge, 
an algorithm for  was not 
presented.

We will summerize our results as follows. 
\begin{enumerate}
\item 
We first show that  can be viewed as some type of 
the connectivity augmentation problem. After this, 
we will prove that 
this connectivity augmentation problem can be solved by using 
an algorithm for the weighted matroid intersection problem in time bounded by a polynomial in 
 and the size of  
(this generalizes the result by Frank~\cite{F06}). 
\item
For the case where  is acyclic, we show another characterization for  
that there exists a feasible solution of . Moreover, we prove that 
in this case  can be solved more efficiently than the general case by 
finding maximum matchings in a series of bipartite graphs instead of using 
an algorithm for the weighted matroid intersection problem. \\
\end{enumerate}

\subsection{Outline}

The rest of this paper is organized as follows. 
Section~\ref{Preliminaries} gives necessary definitions and fundamental results. 
In Section~\ref{Algorithm}, we give an algorithm for the problem 
 by using an algorithm for the weighted matroid intersection 
problem. In Section~\ref{Acyclic Case}, we consider the acyclic case. 

\section{Preliminaries}
\label{Preliminaries}

Let  be a connected directed graph which may have multiple arcs. 
Let . 
Since we can always cover by
 -in-trees the arc set of the subgraph
of  induced by , 
we consider the problem by using at most
 -in-trees. That is,
without loss of generality, we assume that .
For , let  (resp.~) be a set of tails (resp.~heads) of arcs in . 
For , we write  and  instead of  and , 
respectively. 
For ,
we define .
For , we write  instead of . 
For two distinct vertices , we denote by  the 
local arc connectivity from  to  in , i.e.,
. 
We call a subgraph  of  {\it forest} if 
 has no cycle when we ignore the direction of arcs in . 
If a forest  is connected, we call  {\it tree}.  
If every arc of an arc set  is parallel to some arc in , 
we say that  is {\it parallel} to . 
We denote a directed graph obtained by adding an arc set  to  by , 
i.e., . 
For , let . 
For , we denote by  a set of vertices in
 which are reachable from  in . 
For ,
let .

For an arc set  which is parallel to , we clearly have
for every  

From (\ref{eq3:directed graphs}), we have
for every 


We define  as a directed graph  obtained from  by
adding a new vertex  and connecting  to  with
 parallel arcs for every  (see Figure~\ref{fig:covering1}).
We denote by  the arc set of .  
From the definition of , 

We say that  is {\it -proper} when 
 holds for every . 

\subsection{Rooted arc-connectivity augmentation by reinforcing arcs}

Given a directed graph  , 
we call an arc set  with  which is parallel to  
a {\it -rooted connector} 
if  holds for every .
Notice that since a -rooted connector  is parallel to , 
 does not contain an arc which is parallel to an arc entering into  in . 
Then, the problem {\it rooted arc-connectivity augmentation by reinforcing arcs} 
(in short ) is formally defined as follows.
\begin{center}
\begin{tabularx}{150mm}{rX}
\hline
{\bf Problem} &  \\
\hline 
{\bf Input} &  of a directed graph ;\\
{\bf Output} &
a -rooted connector  whose size is minimum among all 
-rooted connectors. \\
\hline
\end{tabularx}
\end{center}

Notice that the problem  
is not equivalent to the local arc-connectivity augmentation problem with 
minimum number of reinforcing arcs from  to .
For example, we consider  illustrated in Figure~\ref{example_raa-ra}(a) 
of a directed graph   
where ,  and . The broken lines in Figure~\ref{example_raa-ra}(b)
represent a minimum -rooted connector. For the problem that asks to increase the - local 
arc-connectivity for every  and  to  by adding minimum 
parallel arcs to  (this problem is called the problem 
{\it increasing arc-connectivity by reinforcing arcs} in \cite{J97}, 
in short ), 
an optimal solution is a set of broken lines in Figure~\ref{example_raa-ra}(c). 
While it is known~\cite{J97} that  
is -hard, 
it is known~\cite{F06} that 
 in which  consists of a single element can be solved 
in time bounded by a polynomial in  and the size of 
by using an algorithm for the weighted matroid intersection.    

\begin{figure}[h]
\begin{minipage}{0.33\hsize}
\begin{center}
\includegraphics[width=3cm]{raa-ra1.eps}
\par(a)
\end{center}
\end{minipage}
\begin{minipage}{0.32\hsize}
\begin{center}
\includegraphics[width=3cm]{raa-ra2.eps}
\par(b)
\end{center}
\end{minipage}
\begin{minipage}{0.33\hsize}
\begin{center}
\includegraphics[width=3cm]{raa-ra3.eps}
\par(c)
\end{center}
\end{minipage}
\caption{\small (a) Input.
(b) Optimal solution for . (c) Optimal solution for .}
\label{example_raa-ra}
\end{figure}

\subsection{Matroids on arc sets of directed graphs}

In this subsection, 
we define two matroids  and  
on  for a directed graph , 
which will be used in the subsequent discussion.
We denote by  a matroid on  whose collection of independent sets is 
. Introductory treatment of a matroid is given in \cite{O92}. 

For  and , 
we define  
where  belongs to  
if and only if both of 
a tail and a head of every arc in  are contained in  and 
a directed graph  is a forest. 
 is clearly a matroid (i.e.~graphic matroid). 
Moreover, we denote the union of  
for  and 
by  
in which  belongs to 
if and only if  can be partitioned into  
such that each  belongs to . 
 is also a matroid
(see Chapter~12.3 in \cite{O92}. This matroid is also called {\it matroid sum}).
When  can be partitioned into 
such that a directed graph  is a tree for 
every  and , 
we call  a {\it base of }. 

Next we define another matroid. We define  
where  
belongs to  if and only if  satisfies 

Since  is a direct sum of uniform matroids,  
 is also a matroid (see Exercise~7 of pp.16 and Example~1.2.7 in \cite{O92}). 
We call  a {\it base of } 
when (\ref{eq1:matroids}) holds with equality. 

For two matroids  and , 
we call an arc set 
{\it -intersection} when . 
If a -intersection  is a base of both  and , 
we call  {\it complete}. 

When we are given a weight function  where  
denotes the set of non-negative reals, 
we define the weight of  (denoted by ) 
by the sum of weights of all arcs . 
The {\it weighted matroid intersection problem} (in short ) is then defined as follows~\cite{F81}. 
\begin{center}
\begin{tabularx}{150mm}{rX}
\hline
{\bf Problem} &  
\\
\hline 
{\bf Input} &
 of a directed graph   
and a weight function ;
\\
{\bf Output} &
a complete -intersection  whose weigh is minimum among all complete 
-intersections, if one exists.
\\
\hline
\end{tabularx}
\end{center}

\begin{lemma} \label{lemma:matroid}
We can solve  in 
 time where . 
\end{lemma}
\begin{proof}
To prove the lemma, we use the following theorem concerning a matroid. 

\begin{theorem}[\cite{K74}] \label{theorem:matroid-knuth}
Given a matroid  
which is a union of  () matroids , 
we can test if a given set belongs to 
 in  time where  is the time required 
to test if a given set belongs to . 
\end{theorem}

\begin{theorem}[\cite{F81}] \label{theorem:matroid-frank}
Given two matroids  and  with a weight function 
 and a non-negative integer , 
we can find  with  whose weight is minimum among all 
 with  in  time if one 
exists where  is the time required to test if 
a given set belongs to both  and . 
\end{theorem}

We consider the time required to test if a given set 
belongs to both  and . 
Since it is not difficult to see that 
we can test is a given set belongs to each 
in  time, 
we can test if a given set belongs to  in
 time from Theorem~\ref{theorem:matroid-knuth}. 
For , the time complexity is clearly  time. 
The size of every complete -intersection is equal to 
 from (\ref{eq1:matroids}). 
From this discussion, the total time required for solving  is  from 
Theorem~\ref{theorem:matroid-frank}. 
\end{proof}

\subsection{Results from \cite{KKT08}}

In this section, we introduce results concerning packing of in-trees given by 
Kamiyama et al.~\cite{KKT08} which plays a crucial role in this paper.  

\begin{theorem}[\cite{KKT08}] \label{theorem:KKT08}
Given a directed graph  , the following three statements are equivalent
\begin{enumerate}
\item For every ,  holds.
\item There exists a -canonical set of arc-disjoint in-trees. 
\item There exists a complete -intersection. 
\end{enumerate}
\end{theorem}

Although the following theorem is not explicitly proved in \cite{KKT08},
we can easily obtain it from the proof of Theorem~\ref{theorem:KKT08} in \cite{KKT08}. 

\begin{theorem}[\cite{KKT08}] \label{theorem2:KKT08}
Given a directed graph   which satisfies the condition of 
Theorem~\ref{theorem:KKT08}, we can find a -canonical set of arc-disjoint
in-trees in  time where .
\end{theorem}

From Theorem~\ref{theorem:KKT08}, we obtain the following corollary. 

\begin{corollary} \label{corollary:KKT08}
Given a directed graph   and an arc set  with  
which is parallel to 
, the following three statements are equivalent
\begin{enumerate}
\item  is a -rooted connector.
\item There exists a -canonical set of arc-disjoint in-trees. 
\item There exists a complete -intersection. 
\end{enumerate}
\end{corollary}
\begin{proof}
The equivalence of the statements~2 and 3 follows from Theorem~\ref{theorem:KKT08}.\\
{\bf 12}
Since  is parallel to , we clearly have 

Since  is a -rooted connector,
we have for every 

From this inequality and Theorem~\ref{theorem:KKT08}, 
this part follows. \\
{\bf 21}
Since there exists a -canonical set of  
arc-disjoint in-trees,
we have for every 

This proves that  is a -rooted connector.
\end{proof}

\section{An Algorithm for Covering by In-trees}
\label{Algorithm}

Given a directed graph  , we present in this section an algorithm for . 
The time complexity of the proposed algorithm is bounded by a polynomial in  and the size of . 
We first prove that  can be reduced to . 
After this, we show that  can be solved by using 
an algorithm for the weighted matroid intersection problem. 

\subsection{Reduction from  to }
\label{Reduction from CDGI to RAA-RA}

If  is not -proper, i.e., 
 for some ,  
there exists no feasible solution of  
since there can not be a -canonical set of in-trees 
that covers  
from the definition of a -canonical set of in-trees.  
Thus, we assume in the subsequent discussion that  is -proper. 

\begin{proposition} \label{proposition1:raa-ra}
Given an -proper directed graph , 
the size of a -rooted connector is at least 
\sum.
\end{proposition}
\begin{proof}
Let  be a -rooted connector. 
For every ,  holds 
from the definition of a -rooted connector.
Thus, the number of arcs of  is at least . 
Since the number of arcs of  is equal to 
 from (\ref{eq1:directed graphs}), the proposition holds. 
\end{proof}
For an -proper directed graph , we define  by 

From Proposition~\ref{proposition1:raa-ra}, the size of a -rooted 
connector is at least . 

\begin{lemma} \label{lemma2:raa-ra}
Given an -proper
directed graph  , there exists a feasible solution of 
if and only if there exists a -rooted connector whose size is equal to . 
\end{lemma}
\begin{proof}
{\bf Only if-part}
Suppose there exists a feasible solution of , i.e., 
there exists a -canonical set  of in-trees which covers . 
For each , we denote  -in-trees of  
by .
For each , let .
Since  covers , 
each  is contained in at least one in-tree of . 
Thus,  holds for every . 
We define an arc set  by 
. 
We will prove that  is a -rooted 
connector whose size is equal to .  

We first prove . For this, 
we show that for every  

Let us first consider . 
For ,  contains 
since  spans  and  is reachable from . 
Hence, 
since  is an in-tree and  is not a root of  from , 
 contains exactly one arc , i.e., 
 is contained in  for exactly one arc . 
Thus, 
\sum.
From this equation and since  follows from , 
(\ref{claim1:lemma2:raa-ra}) holds. 
In the case of , for , 
 is contained in  for exactly one arc
 as in the case of . Thus, 
\sum. 
From this equation and , 

This completes the proof of (\ref{claim1:lemma2:raa-ra}).
Since  contains  copies of , 


What remains is to prove that  is a -rooted connector. 
From Corollary~\ref{corollary:KKT08}, it is sufficient to  
prove that there exists a -canonical set of arc-disjoint in-trees.
For this, we will construct from  a set  of arc-disjoint in-trees 
which consists of  for ,
and we prove that  is a -canonical set of in-trees.
Each  is constructed from  as follows. 
When  is contained in more than one in-tree of , 
in order to construct  from , 
we need to replace  of  by an arc in  which is parallel to 
for every  except one in-tree.
For  which is lexicographically smallest in , 
we allow  to use , while for , 
we replace  of  by an arc in  which is parallel to  
so that for distinct , 
the resulting  and  contain distinct arcs which are parallel to , 
respectively (see Figure~\ref{example_replace}).

\begin{figure}[h]
\begin{center}
\includegraphics[width=1.5cm]{example_replace.eps}
\end{center}
\caption{\small Illustration of the replacing operation. Let  be an arc in , and 
let  be arcs in . Assume that . In this case, , 
 and  contain . Then,  contains ,  contains , 
and  contains .}
\label{example_replace}
\end{figure}
 
We will do this operation for every .
Let  be the set of in-trees obtained by performing the above operation for every . 
Here we show that  is a -canonical set of arc-disjoint in-trees. 
Since  and  are arc-disjoint for  
from the way of constructing , 
it is sufficient to prove that  is a -in-tree. 
Since  is constructed by replacing arcs of  by 
the corresponding parallel arc in  and 
 is an in-tree rooted at ,  
 is also an in-tree rooted at . Since  spans  and from (\ref{eq4:directed graphs}), 
 spans . Hence,  
is a -in-tree. This completes the proof. \\
{\bf If-part}
Let  be a -rooted connector with . 
From Corollary~\ref{corollary:KKT08}, there exists a -canonical set  of arc-disjoint 
in-trees. For each , we denote  -in-trees of  
by .
We will prove that we can construct from  
a -canonical set of in-trees covering . 
We first construct from  a set  of in-trees which consists of  for 
and  by the following procedure {\sf Replace}.
\begin{center}
\fbox{
\begin{minipage}{150mm}
{\bf Procedure {\sf Replace}}
For each  and , 
set  to be a directed graph obtained from 
by replacing every arc  which is contained in 
by an arc in  which is parallel to . 
\end{minipage}
}
\end{center}

From now on, we prove that  is a -canonical set of 
in-trees which covers . 
It is not difficult to prove that  is a -canonical set of in-trees 
from the definition of the procedure {\sf Replace} in the same manner as the last part 
of the proof of the ``only if-part''. 
Thus, it is sufficient to prove that  covers . 
For this, we first show that 
 covers . 
From ,  and (\ref{eq1:rooted}), 

Recall that each  is contained in  in-trees of 
from the definition of a -canonical set of in-trees. 
Thus, since in-trees of  are arc-disjoint, 
it holds for each  that  
the number of arcs in 
which are contained in in-trees of  is equal to 

Hence, the number of arcs in  contained in in-trees of 
 is equal to

Since any arc of  is in  and the number of arcs in 
is equal to that of 
from (\ref{eq4:lemma1:raa-ra}) and (\ref{eq6:lemma1:raa-ra}),
 contains all arcs in . 
Thus,  covers  from the definition of the procedure {\sf Replace}. 
\end{proof}

As seen in the proof of the ``if-part'' of Lemma~\ref{lemma2:raa-ra}, 
if we can find a -rooted connector  with , 
we can compute a -canonical set of in-trees which covers  by using the procedure 
{\sf Replace} from a -canonical set of arc-disjoint in-trees. 
Furthermore, we can construct a -canonical set of arc-disjoint in-trees by using 
the algorithm of Theorem~\ref{theorem2:KKT08}. 
Since the optimal value of  is at least  from Proposition~\ref{proposition1:raa-ra}, 
we can test if there exists a -rooted connector whose size is equal to 
by solving . 
Assuming that we can solve , 
our algorithm for finding a -canonical set of in-trees which covers 
called Algorithm {\sf CR} can be illustrated as Algorithm~\ref{Algorithm:Algorithm1} below. 

\begin{algorithm}[h]
\begin{algorithmic}[1]
\INPUT a directed graph 
\OUTPUT a -canonical set of in-trees covering , if one exists
\IF{ is not -proper}
\STATE Halt (there exists no -canonical set of in-trees covering )
\ENDIF
\STATE Find an optimal solution  of 
\IF{}
\STATE Halt (there exists no -canonical set of in-trees covering )
\ELSE
\STATE Construct a -canonical set  of arc-disjoint in-trees
\STATE Construct a set  of in-trees from  by using the procedure {\sf Replace}
\RETURN  
\ENDIF
\end{algorithmic}
\caption{Algorithm {\sf CR}}
\label{Algorithm:Algorithm1}
\end{algorithm}

\begin{lemma} \label{lemma1:computation}
Given a directed graph , Algorithm  correctly 
finds a -canonical set of in-trees which covers  in  time if one exists where 
 is the time required to solve  and 
. 
\end{lemma}
\begin{proof}
The correctness of the algorithm follows from Lemma~\ref{lemma2:raa-ra}. Thus, 
we consider the time complexity. 
In Step~1, we have to compute  for every . 
This can be done in  time by applying depth-first search from every . 
After this, the time required to test whether  for all  
is . Thus, the time required for Step~1 is . 
Since the number of arcs of  is at most  for a -rooted connector  with
 from (\ref{eq1:rooted}), the time required for Step~8 is 
from Theorem~\ref{theorem2:KKT08}. 
Moreover, since the number of arcs of  is at most , the time required for Step~9
is  from the definition of Procedure {\sf Replace}.
Hence, since the time required for Step~4 is , the lemma follows. 
\end{proof}

\subsection{Reduction from  to }

From the algorithm {\sf CR} in Section~\ref{Reduction from CDGI to RAA-RA},
in order 
to present an algorithm for , what remains is to 
show how we solve . 
In this section, we will prove that we can test whether there exists 
a -rooted connector whose size is equal to  
(i.e., Steps~4 and 5 in the algorithm {\sf CR}) 
by reducing it to the problem . 
Our proof is based on the algorithm of \cite{F06} for 
 in which  consists of a single vertex.
We extend the idea of \cite{F06} to the case of 
 by using Theorem~\ref{theorem:KKT08}.  
We define a directed graph  obtained from  by adding  parallel arcs to every 
. 
Then, we will compute a -rooted connector whose size is equal to  
by using an algorithm for 
as described below. 
Since the number of arcs in a -rooted connector whose size is equal to  
which are parallel to one arc in  
is at most , it is enough to add  parallel arcs to 
each arc of  in  in order to find a -rooted connector whose size 
is equal to . 

We denote by  and  the arc sets of  and , respectively. 
If  is a complete 
-intersection, 
since  is a base of  and 
from (\ref{eq1:matroids}) and (\ref{eq3:directed graphs}),  

We define a weight function  by  

The following lemma shows the relation between  and . 
\begin{lemma} \label{lemma2:wmi}
Given an -proper directed graph  , 
there exists a -rooted connector whose size is equal to 
if and only if 
there exists a complete -intersection whose weight is equal to .
\end{lemma}
To prove Lemma~\ref{lemma2:wmi}, we need to show the following two lemmas.

\begin{lemma} \label{lemma1:wmi}
Given a directed graph   and an arc set  which is parallel to ,
\begin{enumerate}
\item if there is a complete -intersection ,  is also a complete -intersection, and
\item if there is a complete -intersection  such that ,  
is also a complete -intersection.  
\end{enumerate}
\end{lemma}
\begin{proof}
{\bf 1}
We first prove that  is a base of . 
Since  is a base of ,  can be partitioned into 
such that a directed graph  is a tree for every 
 and .
Thus, since each  is a tree from (\ref{eq4:directed graphs}),
 is a base of . 

Next we prove that  is a base of . 
Since  is a base of ,
 is equal to 

Furthermore, since  follows from , 
 is equal to  for every . 
Thus, for each ,  is equal to 

This proves that  is a base of .  
\\
{\bf 2}
This part can be proved in the same manner as in the proof of the part~1.  
\hfill\usebox{\ProofSym}
\end{proof}

\begin{lemma} \label{proposition1:wmi}
Given  of an -proper directed graph 
 and a weight function  
defined by (\ref{eq1:algorithm for raa-ra}),
if there exists a complete -intersection , . 
Moreover,  if and only if . 
\end{lemma}
\begin{proof}
From (\ref{eq1:algorithm for raa-ra}), we have 
.  
Furthermore, 
 
Thus,  follows from (\ref{eq1:rooted}). 
From the above equation,  if and only if . 
This proves the rest of the lemma. 
\hfill\usebox{\ProofSym}
\end{proof}

\begin{proof2}{Lemma~\ref{lemma2:wmi}}
{\bf Only if-part}
Assume that there exists a -rooted connector whose size is equal to .
Since  has  parallel arcs to every , there exists 
a -rooted connector  with . 
Let us fix a -rooted connector  with . 
From (i) of Lemma~\ref{lemma1:wmi}, in order to prove the ``only if-part'', it is sufficient to prove that there exists 
a complete -intersection  with . 
Since there exists a complete -intersection  from Corollary~\ref{corollary:KKT08}, 
we will prove that . 
Since the arc set of  is equal to  and 
 is a -intersection,  holds. 
Thus, since  follows from (\ref{eq1:algorithm for raa-ra}), 
 holds. 
Hence,  follows from Lemma~\ref{proposition1:wmi}. This completes the proof. 
\\
{\bf If-part}
Assume that there exists a complete -intersection 
 with . 
Let  be , and we will 
prove that  is a -rooted connector with 
. 
We first prove  is a -rooted connector by using (ii) of 
Lemma~\ref{lemma1:wmi} and Corollary~\ref{corollary:KKT08}. 
We set  and  in Lemma~\ref{lemma1:wmi} to be 
 and , respectively.
Notice that  follows from 
and  is parallel to .
From , we have . 
Thus,  is a complete -intersection
since  is a complete -intersection and from (ii) 
of Lemma~\ref{lemma1:wmi}.  
Hence, from Corollary~\ref{corollary:KKT08}, 
 is a -rooted connector. 

What remains is to prove that . 
From Lemma~\ref{proposition1:wmi} and ,
 holds. 
Thus, from  and (\ref{eq1:proposition1:wmi}), 
 
This equation and (\ref{eq1:rooted}) complete the proof. 
\end{proof2}

As seen in the proof of the ``if-part'' of Lemma~\ref{lemma2:wmi}, 
if we can find a complete -intersection  
with , we can find 
a -rooted connector  with  
by setting . 
Furthermore, we can obtain a complete -intersection 
whose weight is equal to  if one exists by using the algorithm for 
since the optimal value of  is at least  
from Lemma~\ref{proposition1:wmi}. The formal description of the algorithm 
called Algorithm {\sf RW}
for finding a -rooted connector whose size is equal to 
is illustrated in Algorithm~\ref{Algorithm:Algorithm2}.

\begin{algorithm}[h]
\begin{algorithmic}[1]
\INPUT  of an -proper directed graph  
\OUTPUT a -rooted connector whose size is equal to , if one exits
\STATE Find an optimal solution  for  
with a weight function  defined by (\ref{eq1:algorithm for raa-ra})\\
\IF{there exists no solution of  or }
\STATE Halt (There exists no -rooted connector whose size is equal to )
\ENDIF 
\RETURN 
\end{algorithmic}
\caption{Algorithm {\sf RW}}
\label{Algorithm:Algorithm2}
\end{algorithm}

\begin{lemma} \label{lemma2:computation}
Given  of an -proper directed graph , Algorithm  correctly 
finds a -rooted connector whose size is equal to 
in  time if one exists where 
 is the time required to solve  and 
.
\end{lemma}
\begin{proof}
The correctness of the algorithm follows from Lemma~\ref{lemma2:wmi}. We consider the time 
complexity. In Step~1, we can construct  in  time 
since  has  arcs 
parallel to each arc in  and from (\ref{eq1:rooted}). 
Hence, since the time required for Step~2 is equal to ,
the lemma holds.  
\end{proof}

\subsection{Algorithm for }

We are ready to explain the formal description of our algorithm 
called Algorithm {\sf Covering} for .
Algorithm {\sf Covering} is the same as Algorithm {\sf CR} such that 
Steps~4, 5 and 6 are replaced by Algorithm {\sf RW}.

\begin{theorem}
Given a directed graph  , Algorithm {\sf Covering} correctly finds 
a -canonical set of in-trees which covers  in  time 
if one exits where .
\end{theorem}
\begin{proof}
The correctness of the algorithm follows from Lemmas~\ref{lemma1:computation} and 
\ref{lemma2:computation}. We then consider the time complexity of this algorithm.
From Lemmas~\ref{lemma1:computation} and \ref{lemma2:computation}, 
what remains is to analyze the time required to solve . 
If  is -proper,  
\sum\sum.  
Thus, since  has  parallel arcs of every ,  
\sum.
Hence we have . 
Thus, from Lemma~\ref{lemma:matroid}, we can solve  in 
 time. 
From this discussion and Lemmas~\ref{lemma1:computation} 
and \ref{lemma2:computation}, we obtain the theorem.  
\end{proof}

\section{Acyclic Case}
\label{Acyclic Case}

In this section, we show that in the case where  is acyclic, 
a -canonical set of in-trees covering  
can be computed more efficiently than the general case. 
For this, we prove the following theorem. 

\begin{theorem} \label{main_theorem_acyclic}
Given an acyclic directed graph  , 
there exists a -canonical set of in-trees which covers 
 if and only if 

\end{theorem}
\begin{proof}
For each , we define an undirected bipartite graph
 which is necessary to prove the theorem. 
Let  and 
. 
 and  are connected by an edge in  if and only if 
is reachable from  (see Figure~\ref{fig:acyclic}). 

\begin{figure}[h]
\begin{minipage}{0.5\hsize}
\begin{center}
\includegraphics[width=4cm]{acyclic1.eps}
\par(a)
\end{center}
\end{minipage}
\begin{minipage}{0.5\hsize}
\begin{center}
\includegraphics[width=3cm]{acyclic2.eps}
\par(b)
\end{center}
\end{minipage}
\caption{\small (a) Input acyclic directed graph . (b) Bipartite graph 
 for  in (a).}
\label{fig:acyclic}
\end{figure}

It is well-known that (\ref{eq1:main_theorem_acyclic}) is equivalent to the necessary and
 sufficient condition that for any , there exists a matching
 in  which saturates vertices in  (e.g., Theorem~16.7 in Chapter~16 of \cite{S03}). 
Thus it is
 sufficient to prove that there exists a -canonical set of in-trees which
 covers  if and only if for any , there exists a matching
 in  which saturates vertices in . \\
{\bf If-part}
Since  has no cycle, we can label vertices in  as follows, based on 
topological ordering 
(i) A label of each vertex is an integer between  and .
(ii) For any , a label of  is smaller than that of .  
For , we denote by  a subgraph of 
 induced by  with a set of 
specified vertices  and a restriction of  on . 
Let  be the set of all vertices whose label is at most
. We prove by induction on . For , 
it is clear that there exists a -canonical 
set of in-trees covering the arc set of . 
Assume that in the case of , there exists a -canonical set  
of in-trees covering the arc set of . 
For  and , 
let  be an in-tree of  
which is rooted at  and spans vertices in  from which  is reachable. 

Let  be a vertex whose label is equal to .\\ 
{\bf Case1}
We first consider the case of . 
In this case, from , 
we will construct a set  of in-trees which consists of  
for  () such that each  
is obtained from . We first consider  for . 
For , 
from  holds, 
 is also a -in-tree. Thus, we set . 
Next we consider  for . 
For , since  holds,
we need to add an arc in  to . 
Here we use a matching  in  which saturates vertices in . 
For each edge , we set  be an in-tree obtained by adding 
an arc  to . If there exists   
which is not contained in any edge in , 
we arbitrarily choose an arc  such that  is a neighbour of  in 
and we set  to be 
an in-tree obtained by adding  to . 
From the way of construction,  is clearly a -canonical set of in-trees.
Since  saturates vertices in ,  with  contain all arcs in 
. Thus, since  covers the arc set of  from the induction hypothesis, 
 covers the arc set of . \\
{\bf Case2}
Next we consider the case of . In this case, since  holds, 
letting , 
we need to add new in-trees  for every  
to  which is constructed as above. 
This completes the proof of the ``if-part''.\\
{\bf Only if-part}
Assume that there exists a -canonical set  of in-trees covering
. For , we denote  -in-trees
of  by .
Let us fix , and for  and  we define a set  in which
an edge  is contained in  if and only if  is contained in . If  is contained in , 
is reachable from . Thus, 
 is a subset of . 
Since  covers ,
each  is contained in at least one in-tree in
. That is,  saturates . 
Since  is an in-tree, 
each  is contained in exactly one edge in . 
Thus, it is not difficult to see
that a matching in  which saturates vertices in  can be
obtained from . This completes the proof.  
\end{proof}
From Theorem~\ref{main_theorem_acyclic}, 
instead of the algorithm presented in Section~\ref{Algorithm}, 
we can more efficiently find a -canonical set
of in-trees covering  by finding a maximum matching in
a bipartite graph  times.  
In regard to algorithms for finding a maximum matching in a bipartite
graph, see e.g.~\cite{HK73}. 

\begin{corollary} 
Given an acyclic directed graph  , 
we can find a -canonical set of in-trees which covers 
in  time if one exists 
where  represents the time 
required to find maximum matching in a bipartite graph with  vertices and 
 arcs and . 
\end{corollary}
\begin{proof}
From the proof of Theorem~\ref{main_theorem_acyclic}, for each ,  and  hold. Then,
  follows. 
Thus, the corollary follows from 
 and . 
\end{proof}
{\bf Acknowledgement}
We thank Prof.~Tibor Jord\'{a}n who informed us of the paper \cite{F06} and 
we are grateful to Shin-ichi Tanigawa for helpful comments. 

\begin{thebibliography}{14}

\bibitem{AHL06}
E.~M. Arkin, R.~Hassin, and A.~Levin.
\newblock Approximations for minimum and min-max vehicle routing problems.
\newblock {\em J. Algorithms}, 59(1):1--18, 2006.

\bibitem{EGKRS04}
G.~Even, N.~Garg, J.~K{\"o}nemann, R.~Ravi, and A.~Sinha.
\newblock Min-max tree covers of graphs.
\newblock {\em Oper. Res. Lett.}, 32(4):309--315, 2004.

\bibitem{F06}
A.~Frank.
\newblock Rooted {}-connections in digraphs.
\newblock {\em Discrete Applied Mathematics}.
\newblock (to appear).

\bibitem{F79}
A.~Frank.
\newblock Covering branchings.
\newblock {\em Acta Scientiarum Mathematicarum [Szeged]}, 41:77--81, 1979.

\bibitem{F81}
A.~Frank.
\newblock A weighted matroid intersection algorithm.
\newblock {\em J. Algorithms}, 2(4):328--336, 1981.

\bibitem{HK73}
J.~E. Hopcroft and R.~M. Karp.
\newblock An {} algorithm for maximum matchings in bipartite graphs.
\newblock {\em SIAM J. Comput.}, 2(4):225--231, 1973.

\bibitem{J97}
T.~Jordan.
\newblock Two {}-complete augmentation problems.
\newblock Technical Report~8, {D}epartment of {M}athematics and {C}omputer
  {S}cience, {O}dense {U}niversity, 1997.

\bibitem{KKT08}
N.~Kamiyama, N.~Katoh, and A.~Takizawa.
\newblock Arc-disjoint in-trees in directed graphs.
\newblock In {\em Proc. the nineteenth Annual ACM-SIAM Symposium on Discrete
  Algorithms (SODA2008)}, pages 518--526, 2008.

\bibitem{K74}
D.E. Knuth.
\newblock Matroid partitioning.
\newblock Technical Report {STAN-CS}-73-342, {C}omputer {S}cience {D}epartment,
  {S}tanford {U}niversity, 1974.

\bibitem{NO07}
H.~Nagamochi and K.~Okada.
\newblock Approximating the minmax rooted-tree cover in a tree.
\newblock {\em Inf. Process. Lett.}, 104(5):173--178, 2007.

\bibitem{O92}
J.~G. Oxley.
\newblock {\em Matroid theory}.
\newblock Oxford University Press, 1992.

\bibitem{S03}
A.~Schrijver.
\newblock {\em Combinatorial Optimization: Polyhedra and Efficiency (Algorithms
  and Combinatorics)}.
\newblock Springer-Verlag, 2003.

\bibitem{V78}
K.~Vidyasankar.
\newblock Covering the edge set of a directed graph with trees.
\newblock {\em Discrete Mathematics}, 24:79--85, 1978.

\end{thebibliography}

\end{document}