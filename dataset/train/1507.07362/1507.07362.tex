\FactBranchDepthBound*
\begin{proof}
  Suppose, towards a contradiction, that .
  There must exit two nodes 
  with .
  If  then we get a certificate 
  of strictly smaller rank than  by setting the input value
  of  to  and propagating onwards.
  If  then we may replace
  the subtree rooted in  by the subtree rooted in ,
  retaining the flow conditions since 
  by \cref{fact:dropped-values-1},
  and thus get a certificate
  of strictly smaller rank than .
  Both cases contradict the minimality of .
  This concludes the proof that .

  Now suppose, towards a contradiction, that .
  There exists necessarily two nodes 
  with .
  If  then we get a certificate 
  of strictly smaller rank than  by setting the input value
  of  to  and propagating onwards.
  Similarly,
  if  and 
  then we get a certificate 
  of strictly smaller rank than  by setting the output value
  of  to  and propagating onwards.
  If  and  then we may replace
  the subtree rooted in  by the subtree rooted in ,
  and thus get a certificate
  of strictly smaller rank than .
  The remaining case is when .
  In that case,
  we collapse the nodes ,
  preserve the input value of ,
  and relabel all nodes lexicographically larger than 
  with the largest input/output values allowed by the flow conditions.
  In the resulting flow tree ,
  which has a smaller rank than ,
  the node  originating from  has a strictly larger input value than .
  So  is a certificate.
  All cases contradict the minimality of .
  This concludes the proof that .
 \qed
\end{proof}

\FactTraversalInequation*
\begin{proof}
  Suppose, towards a contradiction,
  that  for some node 
  with .
  Recall that  by assumption.
  If the subtree rooted in  has at most  leaves,
  then we may decrease the input and output values of its nodes,
  retaining the flow conditions,
  so that the output value of  is preserved and its new input value
  is at most .
  Notice that this does not modify the main branch since .
  Thus,
  we get a certificate of strictly smaller rank than .
  Otherwise,
  the subtree rooted in  has at least  leaves.
  Hence,
  it is not elementary
  and we may reduce it into a strictly smaller,
  elementary subtree with at most  leaves.
  The latter induces a complete flow tree
  with the same input and output values for .
  Again,
  this does not modify the main branch since .
  Thus,
  we get a certificate of strictly smaller rank than .
  \qed
\end{proof}

\FactDescentOutsideMB*
\begin{proof}
  Assume that  or .
  Observe that the children of  are not on the main branch,
  i.e.,
  none of them is a prefix of .
  If  is the last child of ,
  then  by minimality of .
  Otherwise,
   and  is the last child of .
  It holds that  and 
  by minimality of .
  We derive from \cref{fact:traversal-inequation} that
  .

  Now assume, in addition, that .
  Suppose, towards a contradiction, that .
  If  then we get a certificate 
  of strictly smaller rank than  by setting the input value
  of  to  and propagating onwards.
  If  then we may replace,
  retaining the flow conditions since ,
  the subtree rooted in  by the subtree rooted in ,
  and thus get a certificate
  of strictly smaller rank than .
  Both cases contradict the minimality of .
  It follows that ,
  which concludes the proof of the fact.
  \qed
\end{proof}

\FactBranchValuesBoundOne*
\begin{proof}
If  then 
  by \cref{fact:dropped-values-2},
  so the equality  holds.
  Otherwise, .
  If we had  then we could
  decrease the output value of  by one,
  retaining the flow conditions by \cref{fact:dropped-values-1},
  and get a certificate
  of strictly smaller rank than ,
  contradicting the minimality of .
  Therefore we get that
  
  and in particular the first inequality of the claim.

  Let us now prove that ,
  where .
  Suppose, towards a contradiction, that .
  Observe that  because of
  \cref{eq:fact:branch-values-bound-1}.
  Let us consider the subtrees on the right of the branch from  to .
  The main idea of the proof is to replace these subtrees by smaller ones
  using \cref{thm:derivewitness}.
  Formally,
  let .
  The set  collects the right-children of the main branch from  to the parent of ,
\gsnote{Introduce earlier in this section the ``left'' and ``right'' notions that we use in informal explanations?}
  excluding those that are on the branch themselves.
  Let  denote the elements of ,
  in lexicographic order,
  and let  for .
  Note that  since  is in Chomsky normal form by assumption.
  Observe also that 
\gsnote{This uses displacements  which are not defined (only ).}
  due to the flow conditions.
  It follows from  that .

  If the total size of the subtrees rooted in  is
  at most ,
  then the flow conditions entail that
  the input and output values of their nodes are all strictly positive,
  since  and  by assumption.
  The same holds for the output values of the nodes  with
  .
  So we may decrease all these values by one,
  retaining the flow conditions.
  Indeed,
  \cref{fact:dropped-values-1} guarantees that the first flow condition
  still holds for the parent of .
  We obtain, in this way,
  a certificate of strictly smaller rank than .

  Otherwise,
  the total size of the subtrees rooted in  is
  at least .
  Observe that  by \cref{fact:branch-depth-bound}.
  According to \cref{thm:derivewitness},
  there exists ,
  where each  is a complete parse tree for  with
  yield ,
  such that 
  and .
  Let us replace the subtrees rooted in 
  by , respectively.
  Since ,
  this induces a complete flow tree
  ,
  with , and
  satisfying
  .
  The new output value of  might be smaller,
  but \cref{fact:dropped-values-1} guarantees that the first flow condition
  still holds for the parent of .
  The input values of  and  were not changed,
  so we have .
  Therefore,
   is a certificate
  of strictly smaller rank than .

  In both cases,
  we obtain a contradiction with the minimality of .
  The observation that 
  concludes the proof of the fact.
  \qed
\end{proof}

\FactBranchValuesBoundThree*
\begin{proof}
  We first show that .
  Recall that  by definition of certificates.
  If we had  then we could
  decrease the input value of  by one,
  retaining the flow conditions by \cref{fact:dropped-values-2},
  and get a certificate
  of strictly smaller rank than ,
  contradicting the minimality of .
  Therefore, .

  Observe that  is an internal node since 
  cannot be in .
  Let us bound the input value of its first child.
  According to \cref{fact:traversal-inequation},
  it holds that 
  for each child  of .
  Let  denote the number of children of .
  The flow conditions together with
  the minimality of  guarantee that
  ,
  , and
   for every .
  We derive that .
  It follows from \cref{fact:branch-values-bound-1} that
  ,
  where .

  Let us now prove that ,
  where .
  The proof is similar to the proof of \cref{fact:branch-values-bound-1}.
  Suppose, towards a contradiction, that .
  Observe that .
  Let us consider the subtrees on the left of the branch from  to .
  The main idea of the proof is to replace these subtrees by smaller ones
  using \cref{thm:derivewitness}.
  Formally,
  let .
  The set  collects the left-children of the main branch from  to the parent of ,
  excluding those that are on the branch themselves.
  Let  denote the elements of ,
  in lexicographic order,
  and let  for .
  Note that  since  is in Chomsky normal form by assumption.
  Observe also that .
\gsnote{This uses displacements  which are not defined (only ).}
  It follows that
   and that
   if .

  If the total size of the subtrees rooted in  is
  at most ,
  then the flow conditions entail that
  the input and input values of their nodes are all strictly positive,
  since  and  by assumption.
  The same holds for the input values of the nodes  with
  .
  So we may decrease all these values by one,
  retaining the flow conditions.
  Indeed,
  the first flow condition still holds for 
  since .
  We obtain, in this way,
  a certificate of strictly smaller rank than .

  Otherwise,
  the total size of the subtrees rooted in  is
  at least .
  Observe that  by \cref{fact:branch-depth-bound}.
  According to \cref{thm:derivewitness},
  there exists ,
  where each  is a complete parse tree for  with
  yield ,
  such that 
  and .
  Moreover  only if .
  Let us replace the subtrees rooted in 
  by , respectively.
  Since ,
  this induces a complete flow tree
  ,
  with , and
  satisfying
  .
  The first flow condition still holds for  since
  .
  The output values of  and  were not changed.
  It follows that  or .
  Indeed,
  if  then
  , hence, ,
  which entails that .
  Therefore,
   is a certificate
  of strictly smaller rank than .

  In both cases,
  we obtain a contradiction with the minimality of .
  The observation that 
  concludes the proof of the fact.
  \qed
\end{proof}

\FactBranchValuesBoundTwo*
\begin{proof}
  Let us write  where each ,
  and let  for .
  We first show that for every ,
  
  Indeed,
  the flow conditions together with
  the minimality of  guarantee,
  for every ancestor ,
  that  and
  that  if .
  In the latter case,
  
  by \cref{fact:traversal-inequation},
  hence,
  .
  We have thus shown that
  
  for every ancestor 
  and every 
  such that .


  We derive from \cref{eq:branch-values-bound-2} that
  
  for all .
  Recall that  by \cref{fact:branch-depth-bound}.
  It follows from \cref{fact:branch-values-bound-1}
  that for every node  such that ,
  we have
  
  According to \cref{fact:dropped-values-1},
  it holds that  for every proper ancestor ,
  which concludes the proof of the fact.
  \qed
\end{proof}

\FactBranchValuesBoundFour*
\begin{proof}
  Let us write  where each ,
  and let  for .
  We claim that
  
  for every .
  Indeed,
  the flow conditions together with
  the minimality of  guarantee,
  for every ancestor ,
  that  and
  that  if .
  In the latter case,
  
  by \cref{fact:traversal-inequation},
  hence,
  .
  We have thus shown that
  
  for every ancestor 
  and every 
  such that .
  This concludes the proof of the claim.

  Observe that  where .
  We derive from the claim that, firstly,
  
  for all ,
  and secondly,
  
  for all .
  Recall that  and  by \cref{fact:branch-depth-bound}.
  It follows from \cref{fact:branch-values-bound-3}
  that,
  for every ancestor ,
  we have
  
  which concludes the proof of the fact.
  \qed
\end{proof}
