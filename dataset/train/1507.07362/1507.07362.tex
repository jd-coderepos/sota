\FactBranchDepthBound*
\begin{proof}
  Suppose, towards a contradiction, that $\len{s} > \card{V}$.
  There must exit two nodes $p \pprefix q \pprefix s$
  with $\lsym{p} = \lsym{q}$.
  If $\lin{p} < \lin{q}$ then we get a certificate $(\mathcal{T}', p, q)$
  of strictly smaller rank than $(\mathcal{T}, s, t)$ by setting the input value
  of $t$ to $-\infty$ and propagating onwards.
  If $\lin{p} \geq \lin{q}$ then we may replace
  the subtree rooted in $p$ by the subtree rooted in $q$,
  retaining the flow conditions since $\lout{p} = \lout{q} = -\infty$
  by \cref{fact:dropped-values-1},
  and thus get a certificate
  of strictly smaller rank than $(\mathcal{T}, s, t)$.
  Both cases contradict the minimality of $(\mathcal{T}, s, t)$.
  This concludes the proof that $\len{s} \leq \card{V}$.

  Now suppose, towards a contradiction, that $\len{t} > \len{s} + \card{V} + 1$.
  There exists necessarily two nodes $s \pprefix p \pprefix q \pprefix t$
  with $\lsym{p} = \lsym{q}$.
  If $\lin{p} < \lin{q}$ then we get a certificate $(\mathcal{T}', p, q)$
  of strictly smaller rank than $(\mathcal{T}, s, t)$ by setting the input value
  of $t$ to $-\infty$ and propagating onwards.
  Similarly,
  if $\lin{p} = \lin{q}$ and $\lout{q} < \lout{p}$
  then we get a certificate $(\mathcal{T}', p, q)$
  of strictly smaller rank than $(\mathcal{T}, s, t)$ by setting the output value
  of $s$ to $-\infty$ and propagating onwards.
  If $\lin{p} = \lin{q}$ and $\lout{q} \geq \lout{p}$ then we may replace
  the subtree rooted in $p$ by the subtree rooted in $q$,
  and thus get a certificate
  of strictly smaller rank than $(\mathcal{T}, s, t)$.
  The remaining case is when $\lin{p} > \lin{q}$.
  In that case,
  we collapse the nodes $p \pprefix q$,
  preserve the input value of $p$,
  and relabel all nodes lexicographically larger than $p$
  with the largest input/output values allowed by the flow conditions.
  In the resulting flow tree $\mathcal{T}'$,
  which has a smaller rank than $\mathcal{T}$,
  the node $t'$ originating from $t$ has a strictly larger input value than $s$.
  So $(\mathcal{T}', s, t')$ is a certificate.
  All cases contradict the minimality of $(\mathcal{T}, s, t)$.
  This concludes the proof that $\len{t} \leq \len{s} + \card{V} + 1$.
 \qed
\end{proof}

\FactTraversalInequation*
\begin{proof}
  Suppose, towards a contradiction,
  that $\lin{p} > \lout{p} + 2^{\card{V}}$ for some node $p \in T$
  with $p \not\prefix t$.
  Recall that $A = \{-1,0,1\}$ by assumption.
  If the subtree rooted in $p$ has at most $2^{\card{V}}$ leaves,
  then we may decrease the input and output values of its nodes,
  retaining the flow conditions,
  so that the output value of $p$ is preserved and its new input value
  is at most $\lout{p} + 2^{\card{V}}$.
  Notice that this does not modify the main branch since $p \not\prefix t$.
  Thus,
  we get a certificate of strictly smaller rank than $(\mathcal{T}, s, t)$.
  Otherwise,
  the subtree rooted in $p$ has at least $2^{\card{V}} + 1$ leaves.
  Hence,
  it is not elementary
  and we may reduce it into a strictly smaller,
  elementary subtree with at most $2^{\card{V}}$ leaves.
  The latter induces a complete flow tree
  with the same input and output values for $p$.
  Again,
  this does not modify the main branch since $p \not\prefix t$.
  Thus,
  we get a certificate of strictly smaller rank than $(\mathcal{T}, s, t)$.
  \qed
\end{proof}

\FactDescentOutsideMB*
\begin{proof}
  Assume that $p = t$ or $p \not\prefix t$.
  Observe that the children of $p$ are not on the main branch,
  i.e.,
  none of them is a prefix of $t$.
  If $q$ is the last child of $p$,
  then $\lout{q} = \lout{p}$ by minimality of $(\mathcal{T}, s, t)$.
  Otherwise,
  $q = p0$ and $p1$ is the last child of $p$.
  It holds that $\lout{p0} = \lin{p1}$ and $\lout{p1} = \lout{p}$
  by minimality of $(\mathcal{T}, s, t)$.
  We derive from \cref{fact:traversal-inequation} that
  $\lout{q} \leq \lout{p} + 2^{\card{V}}$.

  Now assume, in addition, that $\lsym{p} = \lsym{q}$.
  Suppose, towards a contradiction, that $\lout{q} \geq \lout{p}$.
  If $\lin{p} < \lin{q}$ then we get a certificate $(\mathcal{T}', p, q)$
  of strictly smaller rank than $(\mathcal{T}, s, t)$ by setting the input value
  of $t$ to $-\infty$ and propagating onwards.
  If $\lin{p} \geq \lin{q}$ then we may replace,
  retaining the flow conditions since $\lout{q} \geq \lout{p}$,
  the subtree rooted in $p$ by the subtree rooted in $q$,
  and thus get a certificate
  of strictly smaller rank than $(\mathcal{T}, s, t)$.
  Both cases contradict the minimality of $(\mathcal{T}, s, t)$.
  It follows that $\lout{q} < \lout{p}$,
  which concludes the proof of the fact.
  \qed
\end{proof}

\FactBranchValuesBoundOne*
\begin{proof}
If $\lin{s} < \lin{t}$ then $\lout{s} = \lout{t} = -\infty$
  by \cref{fact:dropped-values-2},
  so the equality $\lout{s} = \lout{t} + 1$ holds.
  Otherwise, $\lout{t} < \lout{s}$.
  If we had $\lout{t} + 1 < \lout{s}$ then we could
  decrease the output value of $s$ by one,
  retaining the flow conditions by \cref{fact:dropped-values-1},
  and get a certificate
  of strictly smaller rank than $(\mathcal{T}, s, t)$,
  contradicting the minimality of $(\mathcal{T}, s, t)$.
  Therefore we get that
  \begin{equation}
  \label{eq:fact:branch-values-bound-1}
  \lout{s} = \lout{t} + 1
  \end{equation}
  and in particular the first inequality of the claim.

  Let us now prove that $\lout{s} \leq K$,
  where $K \eqdef 3 (\card{V} + 1) 4^{\card{V}+1}$.
  Suppose, towards a contradiction, that $\lout{s} > K$.
  Observe that $\lout{t} \geq K$ because of
  \cref{eq:fact:branch-values-bound-1}.
  Let us consider the subtrees on the right of the branch from $s$ to $t$.
  The main idea of the proof is to replace these subtrees by smaller ones
  using \cref{thm:derivewitness}.
  Formally,
  let $U = \{p1 \in T \mid s \prefix p \pprefix t \wedge p1 \not\prefix t\}$.
  The set $U$ collects the right-children of the main branch from $s$ to the parent of $t$,
\gsnote{Introduce earlier in this section the ``left'' and ``right'' notions that we use in informal explanations?}
  excluding those that are on the branch themselves.
  Let $u_1, \ldots, u_k$ denote the elements of $U$,
  in lexicographic order,
  and let $S_i = \lsym{u_i}$ for $i = 1, \ldots, k$.
  Note that $S_i \in V$ since $G$ is in Chomsky normal form by assumption.
  Observe also that $\lout{s} \leq \lout{t} + \displ{\#_1 \cdots \#_k}$
\gsnote{This uses displacements $\displ[^G]{w}$ which are not defined (only $\displ[^G]{}$).}
  due to the flow conditions.
  It follows from $\lout{s} = \lout{t} + 1$ that $\displ{\#_1 \cdots \#_k} > 0$.

  If the total size of the subtrees rooted in $u_1, \ldots, u_k$ is
  at most $K$,
  then the flow conditions entail that
  the input and output values of their nodes are all strictly positive,
  since $\lout{s} > K$ and $A = \{-1,0,1\}$ by assumption.
  The same holds for the output values of the nodes $p$ with
  $s \prefix p \prefix t$.
  So we may decrease all these values by one,
  retaining the flow conditions.
  Indeed,
  \cref{fact:dropped-values-1} guarantees that the first flow condition
  still holds for the parent of $s$.
  We obtain, in this way,
  a certificate of strictly smaller rank than $(\mathcal{T}, s, t)$.

  Otherwise,
  the total size of the subtrees rooted in $u_1, \ldots, u_k$ is
  at least $K + 1$.
  Observe that $k \leq \card{V} + 1$ by \cref{fact:branch-depth-bound}.
  According to \cref{thm:derivewitness},
  there exists $T_1, \ldots, T_k$,
  where each $T_i$ is a complete parse tree for $G[S_i]$ with
  yield $z_i$,
  such that $\card{T_1} + \cdots + \card{T_k} \leq 3 k 4^{\card{V}+1} \leq K$
  and $\sum z_1\ldots z_k>0$.
  Let us replace the subtrees rooted in $u_1, \ldots, u_k$
  by $T_1, \ldots, T_k$, respectively.
  Since $\lout{t} \geq K$,
  this induces a complete flow tree
  $\mathcal{T}' = (T', \lsymoperator['], \linoperator['], \loutoperator['])$,
  with $\lout[']{t} = \lout{t}$, and
  satisfying
  $\lout[']{s} = \lout[']{t} + \sum z_1\ldots z_k > \lout[']{t}$.
  The new output value of $s$ might be smaller,
  but \cref{fact:dropped-values-1} guarantees that the first flow condition
  still holds for the parent of $s$.
  The input values of $s$ and $t$ were not changed,
  so we have $\lin[']{s} = \lin{s} \leq \lin{t} = \lin[']{t}$.
  Therefore,
  $(\mathcal{T}', s, t)$ is a certificate
  of strictly smaller rank than $(\mathcal{T}, s, t)$.

  In both cases,
  we obtain a contradiction with the minimality of $(\mathcal{T}, s, t)$.
  The observation that $K \leq 6 \card{V} \cdot 4^{\card{V}+1}$
  concludes the proof of the fact.
  \qed
\end{proof}

\FactBranchValuesBoundThree*
\begin{proof}
  We first show that $\lin{s} \leq \lin{t} \leq \lin{s} + 1$.
  Recall that $\lin{s} \leq \lin{t}$ by definition of certificates.
  If we had $\lin{s} + 1 < \lin{t}$ then we could
  decrease the input value of $t$ by one,
  retaining the flow conditions by \cref{fact:dropped-values-2},
  and get a certificate
  of strictly smaller rank than $(\mathcal{T}, s, t)$,
  contradicting the minimality of $(\mathcal{T}, s, t)$.
  Therefore, $\lin{t} \leq \lin{s} + 1$.

  Observe that $t$ is an internal node since $\lsym{s} = \lsym{t}$
  cannot be in $A \cup \{\varepsilon\}$.
  Let us bound the input value of its first child.
  According to \cref{fact:traversal-inequation},
  it holds that $\lin{tj} \leq \lout{tj} + 2^{\card{V}}$
  for each child $tj$ of $t$.
  Let $k \in \{1, 2\}$ denote the number of children of $t$.
  The flow conditions together with
  the minimality of $(\mathcal{T}, s, t)$ guarantee that
  $\lin{t0} = \lin{t}$,
  $\lout{t} = \lout{t(k-1)}$, and
  $\lin{t(j+1)} = \lout{tj}$ for every $j = 0, \ldots, k-1$.
  We derive that $\lin{t0} \leq \lout{t} + 2 \cdot 2^{\card{V}}$.
  It follows from \cref{fact:branch-values-bound-1} that
  $\lin{t0} \leq K + 2^{\card{V} + 1}$,
  where $K \eqdef 6 \card{V} \cdot 4^{\card{V}+1}$.

  Let us now prove that $\lin{t} \leq H$,
  where $H \eqdef K + 2^{\card{V} + 1}$.
  The proof is similar to the proof of \cref{fact:branch-values-bound-1}.
  Suppose, towards a contradiction, that $\lin{t} > H$.
  Observe that $\lin{s} \geq H$.
  Let us consider the subtrees on the left of the branch from $s$ to $t$.
  The main idea of the proof is to replace these subtrees by smaller ones
  using \cref{thm:derivewitness}.
  Formally,
  let $U = \{p0 \in T \mid s \prefix p \pprefix t \wedge p0 \not\prefix t\}$.
  The set $U$ collects the left-children of the main branch from $s$ to the parent of $t$,
  excluding those that are on the branch themselves.
  Let $u_1, \ldots, u_k$ denote the elements of $U$,
  in lexicographic order,
  and let $S_i = \lsym{u_i}$ for $i = 1, \ldots, k$.
  Note that $S_i \in V$ since $G$ is in Chomsky normal form by assumption.
  Observe also that $\lin{t} \leq \lin{s} + \displ{\#_1 \cdots \#_k}$.
\gsnote{This uses displacements $\displ[^G]{w}$ which are not defined (only $\displ[^G]{}$).}
  It follows that
  $\displ{\#_1 \cdots \#_k} \geq 0$ and that
  $\displ{\#_1 \cdots \#_k} > 0$ if $\lin{s} < \lin{t}$.

  If the total size of the subtrees rooted in $u_1, \ldots, u_k$ is
  at most $K$,
  then the flow conditions entail that
  the input and input values of their nodes are all strictly positive,
  since $\lin{t} > H \geq K$ and $A = \{-1,0,1\}$ by assumption.
  The same holds for the input values of the nodes $p$ with
  $s \prefix p \prefix t$.
  So we may decrease all these values by one,
  retaining the flow conditions.
  Indeed,
  the first flow condition still holds for $t$
  since $\lin{t0} \leq H$.
  We obtain, in this way,
  a certificate of strictly smaller rank than $(\mathcal{T}, s, t)$.

  Otherwise,
  the total size of the subtrees rooted in $u_1, \ldots, u_k$ is
  at least $K + 1$.
  Observe that $k \leq \card{V} + 1$ by \cref{fact:branch-depth-bound}.
  According to \cref{thm:derivewitness},
  there exists $T_1, \ldots, T_k$,
  where each $T_i$ is a complete parse tree for $G[S_i]$ with
  yield $z_i$,
  such that $\card{T_1} + \cdots + \card{T_k} \leq 3 k 4^{\card{V}+1} \leq K$
  and $\sum z_1\ldots z_k \geq 0$.
  Moreover $\sum z_1\ldots z_k = 0$ only if $\lin{s} = \lin{t}$.
  Let us replace the subtrees rooted in $u_1, \ldots, u_k$
  by $T_1, \ldots, T_k$, respectively.
  Since $\lin{s} \geq H \geq K$,
  this induces a complete flow tree
  $\mathcal{T}' = (T', \lsymoperator['], \linoperator['], \loutoperator['])$,
  with $\lin[']{s} = \lin{s}$, and
  satisfying
  $\lin[']{t} = \lin[']{s} + \sum z_1\ldots z_k \geq \lin[']{s}$.
  The first flow condition still holds for $t$ since
  $\lin[']{t0} = \lin{t0} \leq H \leq \lin{s} = \lin[']{s} \leq \lin[']{t}$.
  The output values of $s$ and $t$ were not changed.
  It follows that $\lin[']{s} < \lin[']{t}$ or $\lout[']{t} < \lout[']{s}$.
  Indeed,
  if $\lin[']{s} = \lin[']{t}$ then
  $\sum z_1\ldots z_k = 0$, hence, $\lin{s} = \lin{t}$,
  which entails that $\lout[']{t} = \lout{t} < \lout{s} = \lout[']{s}$.
  Therefore,
  $(\mathcal{T}', s, t)$ is a certificate
  of strictly smaller rank than $(\mathcal{T}, s, t)$.

  In both cases,
  we obtain a contradiction with the minimality of $(\mathcal{T}, s, t)$.
  The observation that $H \leq 7 \card{V} \cdot 4^{\card{V}+1}$
  concludes the proof of the fact.
  \qed
\end{proof}

\FactBranchValuesBoundTwo*
\begin{proof}
  Let us write $t = s j_1 \cdots j_k$ where each $j_i \in \{0, 1\}$,
  and let $p_i = s j_1 \cdots j_i$ for $i = 0, \ldots, k$.
  We first show that for every $0 < i \leq k$,
  \begin{equation}
  \label{eq:branch-values-bound-2}
  \lout{p_i} \leq \lout{p_{i-1}} + 2^{\card{V}}.
  \end{equation}
  Indeed,
  the flow conditions together with
  the minimality of $(\mathcal{T}, s, t)$ guarantee,
  for every ancestor $s \prefix p \pprefix t$,
  that $\lout{p} = \lout{p1}$ and
  that $\lout{p0} = \lin{p1}$ if $p0 \prefix t$.
  In the latter case,
  $\lin{p1} \leq \lout{p1} + 2^{\card{V}}$
  by \cref{fact:traversal-inequation},
  hence,
  $\lout{p0} \leq \lout{p} + 2^{\card{V}}$.
  We have thus shown that
  $\lout{pj} \leq \lout{p} + 2^{\card{V}}$
  for every ancestor $s \prefix p \pprefix t$
  and every $j \in \{0, 1\}$
  such that $pj \prefix t$.


  We derive from \cref{eq:branch-values-bound-2} that
  $\lout{p_i} \leq \lout{s} + i 2^{\card{V}}$
  for all $0 < i < k$.
  Recall that $k \leq \card{V} + 1$ by \cref{fact:branch-depth-bound}.
  It follows from \cref{fact:branch-values-bound-1}
  that for every node $p$ such that $s \prefix p \prefix t$,
  we have
  \begin{equation*}
  \lout{p} \leq \lout{s} + \card{V}\cdot2^{\card{V}}
  \leq 6 \card{V} \cdot 4^{\card{V}+1} + \card{V}\cdot 2^{\card{V}}
\leq 4^{2(\card{V}+1)}
  \end{equation*}
  According to \cref{fact:dropped-values-1},
  it holds that $\lout{p} = -\infty$ for every proper ancestor $p \pprefix s$,
  which concludes the proof of the fact.
  \qed
\end{proof}

\FactBranchValuesBoundFour*
\begin{proof}
  Let us write $t = j_1 \cdots j_k$ where each $j_i \in \{0, 1\}$,
  and let $p_i = j_1 \cdots j_i$ for $i = 1, \ldots, k$.
  We claim that
  $\lin{p_{i-1}} \leq \lin{p_i} + 2^{\card{V}}$
  for every $1 < i \leq k$.
  Indeed,
  the flow conditions together with
  the minimality of $(\mathcal{T}, s, t)$ guarantee,
  for every ancestor $\varepsilon \pprefix p \pprefix t$,
  that $\lin{p} = \lin{p0}$ and
  that $\lout{p0} = \lin{p1}$ if $p1 \prefix t$.
  In the latter case,
  $\lin{p0} \leq \lout{p0} + 2^{\card{V}}$
  by \cref{fact:traversal-inequation},
  hence,
  $\lin{p} \leq \lin{p1} + 2^{\card{V}}$.
  We have thus shown that
  $\lin{p} \leq \lin{pj} + 2^{\card{V}}$
  for every ancestor $\varepsilon \pprefix p \pprefix t$
  and every $j \in \{0, 1\}$
  such that $pj \prefix t$.
  This concludes the proof of the claim.

  Observe that $s = j_1 \cdots j_h$ where $h = \len{s}$.
  We derive from the claim that, firstly,
  $\lin{p_i} \leq \lin{s} + (h-i) 2^{\card{V}}$
  for all $0 < i < h$,
  and secondly,
  $\lin{p_i} \leq \lin{t} + (k-i) 2^{\card{V}}$
  for all $h < i < k$.
  Recall that $h \leq \card{V}$ and $(k-h) \leq \card{V} + 1$ by \cref{fact:branch-depth-bound}.
  It follows from \cref{fact:branch-values-bound-3}
  that,
  for every ancestor $\varepsilon \pprefix p \prefix t$,
  we have
  \begin{equation*}
  \lin{p}
  \leq \max\{\lin{s}, \lin{t}\} + \card{V} 2^{\card{V}}
  \leq 7 \card{V} \cdot 4^{\card{V}+1} + \card{V} 2^{\card{V}}
\leq 4^{2(\card{V}+1)}
  \end{equation*}
  which concludes the proof of the fact.
  \qed
\end{proof}
