\documentclass[runningheads,a4paper]{llncs}

\usepackage{amsmath}
\usepackage[xcdraw]{xcolor}
\usepackage{amssymb}
\usepackage{graphicx}
\usepackage{epstopdf}
\usepackage[hyphens]{url}
\usepackage{cite}
\usepackage{geometry}
\geometry{
  a4paper,         textwidth=13.4cm,  textheight=21.3cm, heightrounded,   hratio=1:1,      vratio=2:3,      }


\usepackage{booktabs}


\usepackage[flushleft]{threeparttable}


\usepackage{arydshln}
\usepackage{multirow}

\usepackage{pifont}
\newcommand{\cmark}{\ding{51}}\newcommand{\xmark}{\ding{55}}

\usepackage{listings, color}

\definecolor{gray}{rgb}{0.4,0.4,0.4}
\definecolor{darkblue}{rgb}{0.0,0.0,0.6}
\definecolor{cyan}{rgb}{0.0,0.6,0.6}

\lstloadlanguages{XML}

\lstdefinestyle{listXML}{language=XML, extendedchars=true,  belowcaptionskip=5pt, xleftmargin=1.8em, xrightmargin=0.5em, numbers=left, numberstyle=\small\ttfamily\bf, frame=single, breaklines=true, breakatwhitespace=true, breakindent=0pt, emph={}, emphstyle=\color{red}, basicstyle=\small\ttfamily, columns=fullflexible, showstringspaces=false, commentstyle=\color{gray}\upshape,
	morestring=[b]",
	morecomment=[s]{<?}{?>},
	morecomment=[s][\color{orange}]{<!--}{-->},
	keywordstyle=\color{cyan},
	stringstyle=\color{black},
	tagstyle=\color{darkblue},
	morekeywords={xmlns,version,type}
}

\lstdefinestyle{customc}{
	belowcaptionskip=1\baselineskip,
	belowcaptionskip=5pt, xleftmargin=1.8em, xrightmargin=0.5em, numbers=left, numberstyle=\small\ttfamily\bf,
	breaklines=true, breakatwhitespace=true, breakindent=0pt, emph={}, emphstyle=\color{red}, basicstyle=\small\ttfamily, columns=fullflexible, showstringspaces=false, commentstyle=\color{gray}\upshape,
	breaklines=true,
	extendedchars=true,
	frame=single,
	morecomment=[s][\color{orange}]{/*}{*/},
	xleftmargin=\parindent,
	language=Java,
	showstringspaces=false,
	basicstyle=\footnotesize\ttfamily,
	keywordstyle=\bfseries\color{green!40!black},
	commentstyle=\itshape\color{purple!40!black},
	stringstyle=\color{orange},
	keywordstyle=\color{cyan},
	stringstyle=\color{black},
	tagstyle=\color{darkblue},
	morekeywords={xmlns,version,type}
}


\begin{document}

\title{When Theory and Reality Collide:  Demystifying the Effectiveness of Ambient Sensing for NFC-based Proximity Detection by Applying Relay Attack Data}

\author{Iakovos Gurulian \and Carlton Shepherd \and
	Konstantinos Markantonakis \and Raja Naeem Akram \and  Keith Mayes}

 \institute{Information Security Group, Smart Card Centre, Royal Holloway, University of London. United Kingdom,\\
}




\maketitle


\begin{abstract}
Over the past decade, smartphones have become the point of convergence for many applications and services. There is a growing trend in which traditional smart-card based services like banking, transport and access control are being provisioned through smartphones. Smartphones with Near Field Communication (NFC) capability can emulate a contactless smart card; popular examples of such services include Google Pay and Apple Pay. Similar to contactless smart cards, NFC-based smartphone transactions are susceptible to relay attacks. For contactless smart cards, distance-bounding protocols are proposed to counter such attacks; for NFC-based smartphone transactions, ambient sensors have been proposed as potential countermeasures. In this study, we have empirically evaluated the suitability of ambient sensors as a proximity detection mechanism for contactless transactions. To provide a comprehensive analysis, we also collected relay attack data to ascertain whether ambient sensors are able to thwart such attacks effectively. We initially evaluated 17 sensors before selecting 7 sensors for in-depth analysis based on their effectiveness as potential proximity detection mechanisms within the constraints of a contactless transaction scenario.  Each sensor was used to record 1000 legitimate and relay (illegitimate) contactless transactions at four different physical locations. The analysis of these transactions provides an empirical foundation on which to determine whether ambient sensors provide a strong proximity detection mechanism for security-sensitive applications like banking, transport and high-security access control.  

\end{abstract}



\section{Introduction}\label{Introduction}
Near Field Communication (NFC) \cite{Coskun2013} has opened smartphone platforms to many different application domains, particularly smart cards. By enabling a smartphone to emulate a contactless smart card, users may use a mobile device as a potential replacement for smart cards in applications such as banking, transport and access control.   Leading technology firms, particularly Google and Apple, have already launched smartphone-based payment services.  To this end, the Android platform has proposed Host-based Card Emulation (HCE) \cite{umar2015performance} that is poised to open up card emulation via NFC to any application on an Android smartphone.

Mobile (contactless) payments are being adopted by tech-savvy, young age groups \cite{UKCardsPayment2015a}.  Deloitte predicted that 5\% of the 600-650 million NFC-enabled mobile phones would be used at least once a month in 2015\footnote{Actual figures for 2015 were not available at the time of writing this paper. However, early forecasts for the global market in mobile payments for 2016 and beyond are available at \url{http://www.nfcworld.com/technology/forecast/}} to make a contactless payment \cite{Deloitte2015}.  We can reasonably assume, therefore, that mobile payments will be a significant payment medium in the future, potentially overtaking contactless smart cards. According to Statista, in 2015, 12.7\% of smartphone users in the USA were actively using proximity mobile payment and  the value of such transactions is projected to grow to 114 billion US\<Time_i\approx<<Time_iTime_iTime_i<Time_iTime_iT_iT_iTime_iTime_ittt^\dagTITT\textbf{T}DTI\textbf{T}_{i} = (TI_{i}, TT_{i}, DTI_{i})TI_{i}TT_{i}iTITT\lvert\textbf{T}\rvert \times \lvert \textbf{T}\rvertt(TI_{i}, TT_{j}), i=j(TI_{i}, TT_{j}), i \neq jDTI_{i}TT_{i}(TI_{i}, TT_{i})(DTI_{i}, TT_{i})TI_{i}DTI_{i}TT_{i}TI_{i}TT_{i}(TI_{i}, TT_{i})(DTTI_{i}, T_{i})NA_{i,j}j^{th}i^{th}A\sigma_{A_{i}}A_{i}cov\mu_{A_{i}}A_{i}xyzAB|A_{i,j} - B_{i,j}|A_{i,j}BB_{i,j}ABA_{i,3}B_{i,3}TI_{i}TT_{i}TI_{i}TT_{j}i \neq jTI_{i}TT_{j}TI_{i}TT_{j}TI_{i}, TT_{i}DTI_{i}, TT_{i}[-1,1]EER_{MAE}EER_{corr}_{MAE}_{corr}EER_{MAE}EER_{corr}_{MAE}_{corr}8000 \times 8000xyzms^{-2}^{\circ}xyzms^{-2}xyzrads^{-1}luxms^{-2}xyz\mu ThPaxyz$ axes.


\subsection{Sound}
For the sound sensor's measurement, we use the device's microphone to record the noise in the vicinity of the mobile handsets and retrieve the maximum amplitude that was sampled, every time it becomes available by the Android operating system.

\subsection{WiFi}
This sensor uses traditional WiFi to detect the networks in the vicinity of the mobile device.  The MAC addresses and ESSIDs of the nearby networks are returned.
 
\end{document}
