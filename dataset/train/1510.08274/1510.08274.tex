\documentclass{llncs}
\usepackage{graphicx}
\usepackage{hyperref}
\usepackage{xcolor}
\usepackage{xspace}
\usepackage{paralist}
\usepackage{amsmath,amssymb}
\usepackage{subfig}
\usepackage{wrapfig}

\newtheorem{observation}{Observation}

\newcommand{\rephrase}[3]{\noindent\textbf{#1 #2}.~\emph{#3}}

\newcommand{\NP}{\xspace}
\newcommand{\PP}{\xspace}
\renewcommand{\NP}{}
\newcommand{\NPH}{\mbox{NP-hard}\xspace}
\newcommand{\NPC}{\mbox{NP-complete}\xspace}
\newcommand{\npc}{\mbox{NP-complete}}
\newcommand{\NPCN}{\mbox{NP-completeness}\xspace}
\newcommand{\npcn}{\mbox{NP-completeness}}
\newcommand{\NPHN}{\mbox{NP-hardness}\xspace}
\newcommand{\nphn}{\mbox{NP-hardness}}


\definecolor{blue}{rgb}{0.274,0.392,0.666}
\definecolor{red}{rgb}{0.627,0.117,0.156}
\definecolor{green}{rgb}{0,0.588,0.509}

\newcommand{\red}[1]{{\color{red}{#1\xspace}}}
\newcommand{\blue}[1]{{\color{blue}{#1\xspace}}}
\newcommand{\green}[1]{{\color{green}{#1\xspace}}}

\newcommand{\lp}{{\sc Level Planarity}\xspace}
\newcommand{\bookinstance}[1]{\xspace}
\newcommand{\Vr}[1]{\xspace}
\newcommand{\Vb}[1]{\xspace}

\newcommand{\clpshort}{{\sc CL-Planarity}\xspace}

\newcommand{\Gint}{\xspace}
\newcommand{\Gr}[1]{\xspace}
\newcommand{\Gb}[1]{\xspace}
\newcommand{\Gg}[1]{\xspace}
\newcommand{\clinstance}[1]{\xspace}
\newcommand{\clp}{{\sc Cyclic Level Planarity}\xspace}
\newcommand{\tlp}{{\sc {\em }-Level Planarity}\xspace}

\newcommand{\tlinstance}[1]{\xspace}
\newcommand{\stinstance}[1]{\xspace}
\newcommand{\betp}{{\sc Betweenness}\xspace}


\newcommand{\sefeinstance}[1]{\xspace}
\newcommand{\sefethreeinstance}[1]{\xspace}
\newcommand{\sefe}{SEFE\xspace}

\let\doendproof\endproof
\renewcommand\endproof{~\hfill\qed\doendproof}


\newcommand{\try}{Beyond Level Planarity}
\title{\try\thanks{Research was partially supported by DFG grant Ka812/17-1, by MIUR project AMANDA, prot. 2012C4E3KT\_001, and by DFG grant WA 654/21-1.}
}

\newcommand{\tuba}{} 
\newcommand{\rome}{} 
\newcommand{\kit}{}

\author{Patrizio {Angelini\tuba}, Giordano {Da Lozzo\rome}, Giuseppe {Di Battista\rome}, \\
    Fabrizio {Frati\rome}, Maurizio {Patrignani\rome}, Ignaz {Rutter\kit}
	\institute{
    \tuba T\"ubingen University, Germany --
    \email{angelini@informatik.uni-tuebingen.de}\\
	\rome~Roma Tre University, Italy --
    \email{\{dalozzo,gdb,frati,patrigna\}@dia.uniroma3.it}\\
    \kit~Karlsruhe Institute of Technology, Germany -- 
    \email{rutter@kit.edu}
}}


\begin{document}
\maketitle
\pagestyle{plain}

\begin{abstract}
In this paper we settle the computational complexity of two open problems related to the extension of the notion of level planarity to surfaces different from the plane. Namely, we show that the problems of testing the existence of a level embedding of a level graph on the surface of the rolling cylinder or on the surface of the torus, respectively known by the name of {\sc Cyclic Level Planarity} and {\sc Torus Level Planarity}, are polynomial-time solvable. 
 
Moreover, we show a complexity dichotomy for testing the {\sc Simultaneous Level Planarity} of a set of level graphs, with respect to both the number of level graphs and the number of levels.
\end{abstract}



\section{Introduction and Overview} \label{se:intro}


The study of level drawings of level graphs has spanned a long time; the seminal paper by Sugiyama {\em et al.}~on this subject~\cite{stt-mvuhs-81} dates back to 1981, well before graph drawing was recognized as a distinguished research area. This is motivated by the fact that level graphs naturally model hierarchically organized data sets and level drawings are a very intuitive way to represent such graphs.

Formally, a \emph{level graph}  is a directed graph  together with a function , with . The set  is the -th \emph{level} of .  
A level graph  is \emph{proper} if for each , either , or  and . Let  be  horizontal straight lines on the plane ordered in this way with respect to the -axis. A \emph{level drawing} of  maps each vertex  to a point on  and each edge  to a curve monotonically increasing in the -direction from  to . Note that a level graph  containing an edge  with  does not admit any level drawing. A level graph is \emph{level planar} if it admits a {\em level embedding}, i.e., a level drawing with no crossing; see Fig.~\ref{fig:drawings}(a). The {\sc Level Planarity} problem asks to test whether a given level graph is level planar.
 
\begin{figure}[tb!]
    \centering
      \subfloat[]{\includegraphics[height=.25\textwidth,page=1]{level.pdf}}\hfill
      \subfloat[]{\includegraphics[height=.25\textwidth,page=2]{level.pdf}}\hfill
      \subfloat[]{\includegraphics[height=.25\textwidth,page=3]{level.pdf}}\hfill
      \subfloat[]{\includegraphics[height=.25\textwidth,page=4]{level.pdf}}
    \caption{Level embeddings (a) on the plane, (b) on the standing cylinder, (c) on the rolling cylinder, and (d) on the torus.}
\label{fig:drawings}
\end{figure}

Problem {\sc Level Planarity} has been studied for decades~\cite{dn-hpt-88,hp-rlpdlt-95,jlm-lptlt-98,rsbhkmsc-sfplg-01,Fulek2013}, starting from a characterization of the single-source level planar graphs~\cite{dn-hpt-88} and culminating in a linear-time algorithm for general level graphs~\cite{jlm-lptlt-98}.
A characterization of level planarity in terms of ``minimal'' forbidden subgraphs is still missing~\cite{efk-clptmp-09,hkl-clpg-04}.
The problem has also been studied to take into account a clustering of the vertices ({\sc Clustered Level Planarity}~\cite{tibp-addfr-15,fb-clp-04}) or consecutivity constraints for the vertex orderings on the levels ({\sc -level Planarity}~\cite{tibp-addfr-15,wsp-gktlg-12}).

Differently from the standard notion of planarity, the concept of level planarity does not immediately extend to representations of level graphs on surfaces different from the plane\footnote{We consider connected orientable surfaces; the {\em genus} of a surface is the maximum number of cuttings along non-intersecting closed curves that do not disconnect it.}. When considering the surface  of a sphere, level drawings are usually defined as follows: The vertices have to be placed on the  circles given by the intersection of  with  parallel planes, and each edge is a curve on  that is monotone in the direction orthogonal to these planes. The notion of level planarity in this setting goes by the name of {\sc Radial Level Planarity} and is known to be decidable in linear time~\cite{bbf-rlptelt-05}. This setting is equivalent to the one in which the level graph is embedded on the ``standing cylinder'': Here, the vertices have to be placed on the circles defined by the intersection of the cylinder surface  with planes parallel to the cylinder bases, and the edges are curves on  monotone with respect to the cylinder axis; see~\cite{abbg-cpue-12,bbf-rlptelt-05,b-updsrc-14} and Fig.~\ref{fig:drawings}(b).  


Problem {\sc Level Planarity} has been also studied on the surface  of a ``rolling cylinder''; see~\cite{abbg-cpue-12,bb-ltpte-08,bbk-clpte-07,b-updsrc-14} and Fig.~\ref{fig:drawings}(c). 
In this setting,  straight lines  parallel to the cylinder axis lie on , where  are seen in this clockwise order from a point  on one of the cylinder bases, the vertices of level  have to be placed on , for , and each edge  is a curve  lying on  and flowing monotonically in clockwise direction from  to  as seen from .
Within this setting, the problem takes the name of {\sc Cyclic Level Planarity}~\cite{bbk-clpte-07}.
Note that a level graph  may now admit a level embedding even if it contains edges  with .
Contrary to the other mentioned settings, the complexity of testing {\sc Cyclic Level Planarity} is still unknown, and a polynomial (in fact, linear) time algorithm has been presented only for {\em strongly connected graphs}~\cite{bb-ltpte-08}, which are level graphs such that for each pair of vertices there exists a directed cycle through them. 

In this paper we settle the computational complexity of {\sc Cyclic Level Planarity} by showing a polynomial-time algorithm to test whether a level graph admits a cyclic level embedding (Theorem~\ref{co:cyclicLevel-polynomial}). 
In order to obtain this result, we study a version of level planarity in which the surface~ where the level graphs have to be embedded has genus ; we call {\sc Torus Level Planarity} the corresponding decision problem, whose study was suggested in~\cite{bbk-clpte-07}. It is not difficult to note (Lemmata~\ref{le:radial-torus} and~\ref{le:cyclic-torus}) that the torus surface combines the representational power of the surfaces of the standing and of the rolling cylinder -- that is, if a graph admits a level embedding on one of the latter surfaces, then it also admits a level embedding on the torus surface. Furthermore, both {\sc Radial Level Planarity} and {\sc Cyclic Level Planarity} (and hence {\sc Level Planarity}) reduce in linear time to {\sc Torus Level Planarity}.

The main result of the paper (Theorem~\ref{th:torus}) is a quadratic-time algorithm for proper instances of {\sc Torus Level Planarity} and a quartic-time algorithm for general instances. Our solution is based on a linear-time reduction (Observation~\ref{prop:characterization} and Lemmata~\ref{le:torus-characterization}-\ref{le:reduction}) that, starting from any proper instance of {\sc Torus Level Planarity}, produces an equivalent instance of the {\sc Simultaneous PQ-Ordering} problem~\cite{br-spacep-13} that can be solved in quadratic time (Theorem~\ref{th:2-fixed}).

Motivated by the growing interest in simultaneous embeddings of multiple planar graphs, which allow to display several relationships on the same set of entities in a unified representation, we define a new notion of level planarity in which multiple level graphs are considered and the goal is to find a simultaneous level embedding of them. 
The problem {\sc Simultaneous Embedding} (see the seminal paper~\cite{bcdeeiklm-spge-07} and a recent survey~\cite{bkr-sepg-13}) takes as input  planar graphs  and asks whether they admit planar drawings mapping each vertex to the same point of the plane. We introduce the problem {\sc Simultaneous Level Planarity}, which asks whether  level graphs  admit level embeddings mapping each vertex to the same point along the corresponding level.
As an instance of {\sc Simultaneous Level Planarity} for two graphs on two levels is equivalent to one of {\sc Cyclic Level Planarity} on two levels (Theorem~\ref{th:sim-poly}), we can solve {\sc Simultaneous Level Planarity} in polynomial time in this case. This positive result cannot be extended (unless P=NP), as the problem becomes \NPC even for two graphs on three levels and for three graphs on two levels (Theorem~\ref{thm:sim-level-npc-two-levels}). Altogether, this establishes a tight border of tractability for {\sc Simultaneous Level Planarity}.


\section{Preliminaries} \label{se:preliminaries}

A {\em tree}  is a connected acyclic graph. The degree- vertices of  are the {\em leaves} of , denoted by , while the remaining vertices are the {\em internal} vertices. 

A digraph  without directed cycles is a {\em directed acyclic graph} ({\em DAG}). 
An edge  directed from  to  is an \emph{arc}; vertex  is a {\em parent} of  and  is a {\em child} of .
A vertex is a {\em source} ({\em sink}) if it has no parents~(children).

\smallskip
\noindent
{\bf Embeddings on levels.} 
An {\em embedding} of a graph on a surface  is a mapping  of each vertex  to a distinct point on  and of each edge  to a simple Jordan curve on  connecting  and , such that no two curves cross except at a common endpoint. 
Let  and  denote the unit interval and the boundary of the unit disk, respectively.
We define the surface  of the standing cylinder,  of the rolling cylinder, and  of the torus
as ,
as , and 
as , respectively.
The {\em -th level} of surfaces , , and  with  levels is defined as
 ,
  , and
 , respectively; see Fig.~\ref{fig:levels}.
An edge  on , on , or on  is {\em monotone} if it intersects the levels , where , exactly once and does not intersect any of the other levels.



 \begin{figure}[tb!]
\centering
\subfloat[]{\includegraphics[height=0.31\textwidth,page=3]{levels.pdf}\label{fig:levels-S}}\hfil
\subfloat[]{\includegraphics[height=0.31\textwidth,page=2]{levels.pdf}\label{fig:levels-R}}\hfil
\subfloat[]{\includegraphics[height=0.31\textwidth,page=1]{levels.pdf}\label{fig:levels-T}}
\caption{Levels on (a) , on (b) , and on (c) , respectively.
} 
\label{fig:levels}
\end{figure}

Problems {\sc Radial}, {\sc Cyclic}, and {\sc Torus Level Planarity} take as input a level graph  and ask to find an embedding  of  on , on , and on , respectively, in which each vertex  lies on  and each edge  is monotone. Embedding  is called a {\em radial}, a {\em cyclic}, and a {\em torus level embedding} of , respectively.
A level graph admitting a radial, cyclic, or torus level embedding is called {\em radial}, {\em cyclic}, or {\em torus level planar}, respectively.

Lemmata~\ref{le:radial-torus} and~\ref{le:cyclic-torus} show that the torus surface combines the power of representation of the standing and of the rolling cylinder. To strengthen this fact, we present a level graph in Fig.~\ref{fig:torus-level} that is neither radial nor cyclic level planar, yet it is torus level planar; note that the underlying (non-level) graph is also planar.

\begin{lemma}\label{le:radial-torus}
Every radial level planar graph is also torus level planar.
Further, {\sc Radial Level Planarity} reduces in linear time to {\sc Torus Level Planarity}.
\end{lemma}

\begin{proof}
The first part of the statement can be easily proved by observing that any level embedding on  is also a level embedding on .
We prove the second part of the statement. 
		Given an instance  of {\sc Radial Level Planarity} we construct an instance  of {\sc Torus Level Planarity}, where , for each , , and . 
Suppose that  admits a radial level embedding  on . Consider the corresponding torus level embedding  of  on , which exists by the first part of the statement. Since  does not contain any edge  such that  (as otherwise,  would not be radial level planar), the strip of  between  and  does not contain any edge. Hence, cycle  can be added to  to obtain a torus level embedding of  on .
		Suppose that  admits a torus level embedding  on .
		A radial level embedding of  on  can be obtained by removing the drawing of cycle  in .
		\end{proof}

\begin{lemma}\label{le:cyclic-torus}
Every cyclic level planar graph is also torus level planar.
Further, {\sc Cyclic Level Planarity} reduces in linear time to {\sc Torus Level Planarity}.
\end{lemma}

\begin{proof}
The first part of the statement can be proved by observing that any level embedding on  is also a level embedding on .
We prove the second part. Given an instance  of {\sc Cyclic Level Planarity}, we construct an instance   of {\sc Torus Level Planarity}, where  for each , and .
Suppose that  admits a cyclic level embedding  on . Add a drawing of cycle  to  along the boundary of one of the two bases of , thus obtaining a cyclic level embedding  of  on . From the first part of the statement there exists a torus level embedding of  on .
Suppose that  admits a torus level embedding  on .
A cyclic level embedding of  on  can be obtained by removing the drawing of cycle  in~.
\end{proof}


\begin{figure}[tb!]
\centering
\subfloat[]{\includegraphics{torus-simpler.pdf}\label{fig:torus-level}}\hfil
\subfloat[]{\includegraphics{radial.pdf}\label{fig:2-levels}}
\caption{(a) A level graph that is neither cyclic nor radial level planar, yet it is torus level planar. (b) A radial level embedding  of a level graph  on two levels. Colors are used for edges incident to vertices of degree larger than one to illustrate that the edge ordering on  in  is v-consecutive.} 
\label{fig:something}
\end{figure}

\smallskip
\noindent
{\bf Orderings and PQ-trees.} Let  be a finite set. 
We call {\em linear ordering} any permutation of .
When considering the first and the last elements of the permutation as consecutive, we talk about {\em circular orderings}. 
Let  be a circular ordering on  and let  be the circular ordering on  obtained by restricting  to the elements of . Then  is a {\em suborder} of  and  is an {\em extension} of . 
Let  and  be finite sets, let  be a circular ordering on , let   be an injective map, and let  be the image of  under ;
then  denotes the circular ordering .
We also say that a circular ordering  on  is a {\em suborder} of a circular ordering  on  (and  is an {\em extension} of ) if  is a suborder of .

An \emph{unrooted PQ-tree}  is a tree whose leaves are the elements of a ground set . PQ-tree  can be used to represent all and only the circular orderings  on  satisfying a given set of \emph{consecutivity constraints} on , each of which specifies that a subset of the elements of  has to appear consecutively in all the represented circular orderings on . 
The internal nodes of  are either {\em P-nodes} or {\em Q-nodes}. The orderings in  are all and only the circular orderings on the leaves of  obtained by arbitrarily ordering the neighbours of each P-node and by arbitrarily selecting for each Q-node either a given circular ordering on its neighbours or its reverse ordering.
Note that possibly , if  is the empty tree, or  represents all possible circular orderings on , if  is a star centered at a P-node. In the latter case,  is the {\em universal} PQ-tree on .

We illustrate three linear-time operations on PQ-trees (see~\cite{bl-tcop-76,hrl-tsp-13,hm-ppp-04}). Let  and  be PQ-trees on  and let :
The {\em reduction of  by } builds a new PQ-tree on  representing the circular orderings in  in which the elements of  are consecutive.
The {\em projection of  to }, denoted by , builds a new PQ-tree on  representing the circular orderings on  that are suborders of circular orderings in .
The {\em intersection of  and }, denoted by , builds a new PQ-tree on  representing the circular orderings in .



\smallskip
\noindent
{\bf Simultaneous PQ-Ordering.} Let  be a DAG with vertex set , where  is a PQ-tree, such that each arc  consists of a source , of a target , and of an injective map  from the leaves of  to the leaves of . 
Given an arc  and circular orderings  and , we say that arc  is {\em satisfied} by  if  extends .
The {\sc Simultaneous PQ-Ordering} problem asks to find circular orderings  on , respectively, such that each arc  is satisfied by .

Let  be an arc in . 
An internal node  of  is {\em fixed by} an internal node  of  (and  {\em fixes}  \emph{in} ) if there exist leaves  and  such that 
(i) removing  from  makes , , and  pairwise disconnected in , and
(ii) removing  from  makes , , and  pairwise disconnected in .
Note that by (i) the three paths connecting  with , , and  in  share no node other than , while by (ii) those connecting  with , , and  in  share no node other than .
Since any ordering  determines a circular ordering around  of the paths connecting it with , , and  in , any ordering  extending  determines the same circular ordering around  of the paths connecting it with , , and  in ; this is why we say that  is fixed by .

Theorem~\ref{th:2-fixed} below will be a key ingredient in the algorithms of the next section. However, in order to exploit it, we need to consider {\em normalized} instances of {\sc Simultaneous PQ-Ordering}, namely instances  such that, for each arc  and for each internal node , tree  contains exactly one node  that is fixed by . This property can be guaranteed by an operation, called {\em normalization}~\cite{br-spacep-13}, defined as follows. Consider each arc  and replace  with  in , that is, replace tree  with its intersection with the projection of its parent  to the set of leaves of  obtained by applying mapping  to the leaves  of . 

Consider a normalized instance .
Let  be a P-node of a PQ-tree  with parents  and let  be the unique node in , with , fixed by . The {\em fixedness}  of  is defined as , where  is the number of children of  fixing . A P-node  is \emph{-fixed} if . Also, instance  is \emph{-fixed} if all the P-nodes of any PQ-tree  are -fixed. 

\begin{theorem}[Bl{\"{a}}sius and Rutter~\cite{br-spacep-13}, Theorems~3.2 and~3.3]\label{th:2-fixed} -fixed instances of {\sc Simultaneous PQ-Ordering} can be tested in quadratic time.
\end{theorem}
 


\section{Torus Level Planarity} \label{se:cyclic}

In this section we provide a polynomial-time testing and embedding algorithm for {\sc Torus Level Planarity} that is based on the following simple observation.

\begin{observation}\label{prop:characterization}
A proper level graph  is torus level planar if and only if 
there exist circular orderings  on 
such that, for each  with , there exists a radial level embedding of the level graph  in which the circular orderings on  along  and on  along  are  and , respectively.
\end{observation}


In view of Observation~\ref{prop:characterization} we focus on a level graph  on two levels  and . Denote by  and by  the subsets of  and of  that are incident to edges in , respectively.
Let  be a radial level embedding of .
Consider a closed curve  separating levels  and  and intersecting all the edges in  exactly once.
The {\em edge ordering on  in } is the circular ordering in which the edges in  intersect  according to a clockwise orientation of  on the surface~ of the standing cylinder; see Fig.~\ref{fig:2-levels}.
Further, let  be a circular ordering on the edge set . Ordering  is {\em vertex-consecutive} ({\em v-consecutive}) if all the edges incident to each vertex in  are consecutive in . 

Let  be a v-consecutive ordering on . 
We define orderings  on  and  on  {\em induced by} , as follows.
Consider the edges in  one by one as they appear in . Append the end-vertex in  of the currently considered edge to a list . 
Since  is v-consecutive, the occurrences of the same vertex appear consecutively in , regarding such a list as circular.
Hence,  can be turned into a circular ordering  on  by removing repetitions of the same vertex. Circular ordering  can be constructed analogously. We have the following.


\begin{lemma}\label{le:torus-characterization}
Let  be a circular ordering on  and  be a pair of circular orderings on  and . 
There exists a radial level embedding of  whose edge ordering is  and such that the circular orderings on  and  along  and  are  and , respectively,
if and only if  is v-consecutive, and  and  extend the orderings  and  on  and  induced by , respectively.
\end{lemma}

\begin{proof}
The necessity is trivial. 
For the sufficiency, assume that  is v-consecutive and that  and  extend the orderings  and  on  and  induced by , respectively.
We construct a radial level embedding  of  with the desired properties, as follows. Let  be a radial level embedding consisting of  non-crossing curves, each connecting a distinct point on  and a distinct point on . We associate each curve with a distinct edge in , so that the edge ordering of  is . Note that, since  is v-consecutive, all the occurrences of the same vertex of  and of  appear consecutively along  and , respectively. Hence, we can transform  into a radial level embedding  of
, by continuously deforming the curves in  incident to occurrences of the same vertex in  (in ) so that their end-points on  (on ) coincide. Since the circular orderings on  and on  along  and  are  and , respectively, we can construct  by inserting the isolated vertices in  and  at suitable points along  and , so that the circular orderings on  and on  along  and  are  and , respectively.
\end{proof}


We construct an instance  of {\sc Simultaneous PQ-Ordering} starting from a level graph  on two levels as follows; refer to Fig.~\ref{fig:simpq-instance}, where  corresponds to the subinstance  in the dashed box.
We define the {\em level trees}  and  as the universal PQ-trees on  and , respectively. 
Also, we define the {\em layer tree}  as the PQ-tree on  representing exactly the edge orderings for which a radial level embedding of  exists, which are the v-consecutive orderings on , by Lemma~\ref{le:torus-characterization}. 
The PQ-tree  can be constructed in  time~\cite{bl-tcop-76,hm-ppp-04}.
We define the {\em consistency trees}  and  as the universal PQ-trees on  and , respectively.
Instance  contains , , , , and , together with the arcs ,
,
,
 and , where  denotes the identity map and   () assigns to each vertex in  (in ) an incident edge in . We have the following.

\begin{figure}[tb]
\centering
\includegraphics[width=\textwidth,page=2]{IG.pdf}
\caption{Instance  of {\sc Simultaneous PQ-Ordering} for a level graph . Instance  corresponding to the level graph  induced by levels  and  of  is enclosed in a dashed box.}
\label{fig:simpq-instance}
\end{figure}

\begin{lemma}\label{le:equivalence}
Level graph  admits a radial level embedding in which the circular ordering on  along  is  and the circular ordering on  along  is  if and only if instance  of {\sc Simultaneous PQ-Ordering} admits a solution in which the circular ordering on  is  and the one on  is .
\end{lemma}

\begin{proof}
We prove the necessity. Let  be a radial level embedding of . We construct an ordering on the leaves of each tree in  as follows. Let , , , and  be the circular orderings on  along , on  along , on  along , and on  along  in , respectively. Let  be the edge ordering on  in . Note that  since  is v-consecutive by Lemma~\ref{le:torus-characterization}. The remaining trees are universal, hence , , , and . 

We prove that all arcs of  are satisfied. Arc  is satisfied if and only if  extends . This is the case since  is the identity map, since , and since  and  are the circular orderings on  and  along . Analogously, arc  is satisfied. Arc  is satisfied if and only if  extends . This is due to the fact that  assigns to each vertex in  an incident edge in  and to the fact that, by Lemma~\ref{le:torus-characterization}, ordering  is v-consecutive and  is induced by . Analogously, arc  is satisfied.  

We prove the sufficiency. Suppose that  is a positive instance of {\sc Simultaneous PQ-Ordering}, that is, there exist orderings , , , , and  of the leaves of the trees , , , , and , respectively, satisfying all arcs of . 
Since  is the identity map and since arcs  and  are satisfied, we have that  and  are restrictions of  and  to  and , respectively. Also, since  and  are satisfied, we have that  extends both  and . By the construction of , ordering  is v-consecutive. By Lemma~\ref{le:torus-characterization}, a radial level embedding  of  exists in which the circular ordering on  along  is , for .
\end{proof}



We now show how to construct an instance  of {\sc Simultaneous PQ-Ordering} from a proper level graph  on  levels; refer to Fig.~\ref{fig:simpq-instance}. For each , let  be the instance of {\sc
  Simultaneous PQ-Ordering} constructed as described above starting from the level graph on two levels  (in the construction  takes the role of ,  takes the role of , and  ). Any two instances  and  share exactly the level tree , whereas non-adjacent instances are disjoint. We define  and obtain  by normalizing .  We now present two lemmata about properties of instance . 

\begin{lemma} \label{le:construction}
 is -fixed, has  size, and can be built in  time. 
\end{lemma} 

\begin{proof}
  Every PQ-tree  in  is either a source with exactly two children or a sink with exactly two parents, and the normalization of  to obtain  does not alter this property. Thus every P-node in a PQ-tree  in  is at most -fixed.
  In fact, recall that for a P-node  of a PQ-tree  with parents , we have that , where  is the number
of children of  fixing , and   is the unique node in , with , fixed by . Hence, if  is a source PQ-tree, it holds  and ; whereas, if  is a sink PQ-tree, it holds , , and  for each parent  of . Therefore  is -fixed.

Since every internal node of a PQ-tree in  has degree greater than , to prove the bound on  it suffices to show that the total number of leaves of all PQ-trees in  is in . Since  and , the number of leaves of all level and consistency trees is at most . Also, since , the number of leaves of all layer trees is at most . Thus .

We have already observed that each layer tree  can be constructed in  time; level and consistency trees are stars, hence they can be constructed in linear time in the number of their leaves. Finally, the normalization of each arc  can be performed in  time~\cite{br-spacep-13}. Hence, the  time bound follows.  
\end{proof}

\begin{lemma} \label{le:reduction}
Level graph  admits a torus level embedding if and only if  is a positive instance of {\sc Simultaneous PQ-Ordering}.
\end{lemma}

\begin{proof}
Suppose that  admits a torus level embedding . For , let  be the circular ordering on  along . By Observation~\ref{prop:characterization}, embedding  determines a radial level embedding  of . By Lemma~\ref{le:equivalence}, for , there exists a solution for the instance  of {\sc Simultaneous PQ-Ordering} in which the circular ordering on  () is  (resp.\ ). Since the circular ordering on  is  both in  and  and since each arc of  is satisfied as it belongs to exactly one instance , which is a positive instance of {\sc Simultaneous PQ-Ordering}, it follows that the circular orderings deriving from instances  define a solution for . 

Suppose that  admits a solution. Let  be the circular orderings on the leaves of the level trees  in this solution. By Lemma~\ref{le:equivalence}, for each  with , there exists a radial level embedding of level graph
 in which the circular orderings on  along  and  along  are  and , respectively. By Observation~\ref{prop:characterization},  is torus level planar. 
\end{proof}

We thus get the main result of this paper.

\begin{theorem}\label{th:torus}
{\sc Torus Level Planarity} can be tested in quadratic (quartic) time for proper (non-proper) instances.
\end{theorem}

\begin{proof}
Consider any instance  of {\sc Torus Level Planarity}. Assume first that  is proper. By Lemmata~\ref{le:construction} and~\ref{le:reduction}, a -fixed instance  of {\sc Simultaneous PQ-Ordering} equivalent to  can be constructed in linear time with . By Theorem~\ref{th:2-fixed} instance  can be tested in quadratic time.

If  is not proper, then subdivide every edge  that spans  levels with  vertices, assigned to levels . This increases the size of the graph at most quadratically, and the time bound follows.
\end{proof}

Theorem~\ref{th:torus} and Lemma~\ref{le:cyclic-torus} imply the following result.

\begin{theorem}\label{co:cyclicLevel-polynomial}
{\sc Cyclic Level Planarity} can be solved in quadratic (quartic) time for proper (non-proper) instances.
\end{theorem}



Our techniques allow us to solve a more general problem, that we call {\sc Torus -Level Planarity}, in which a level graph  is given together with a set of PQ-trees  such that , where each tree  encodes consecutivity constraints on the ordering on  along . The goal is then to test the existence of a level embedding of  on  in which the circular ordering on  along  belongs to . 
This problem has been studied in the plane~\cite{tibp-addfr-15,wsp-gktlg-12} under the name of {\sc -Level Planarity}; it is NP-hard in general and polynomial-time solvable for proper instances. While the former result implies the NP-hardness of {\sc Torus -Level Planarity}, the techniques of this paper show that {\sc Torus -Level Planarity} can be solved in polynomial time for proper instances. Namely, in the construction of instance  of {\sc Simultaneous PQ-Ordering}, it suffices to replace level tree  with PQ-tree . Analogous considerations allow us to extend this result to {\sc Radial -Level Planarity} and {\sc Cyclic -Level Planarity}.

\section{Simultaneous Level Planarity} \label{se:simultaneous}

In this section we prove that  {\sc Simultaneous Level Planarity} is \NPC for two graphs on three levels and for three graphs on two levels, while it is polynomial-time solvable for two graphs on two levels. 

Both NP-hardness proofs rely on a reduction from the \NPC problem {\sc Betweenness}~\cite{o-top-79}, which asks for a ground set  and a set  of ordered triplets of , with  and , whether a linear order  of  exists such that, for any , it holds true that  or that . Both proofs exploit the following gadgets. 

The {\em ordering gadget} is a pair \sefeinstance{} of level graphs on levels  and , where  contains  vertices , with  and , and  contains  vertices , with  and . For  and , \Gr{} contains edge  and \Gb{} contains edge . See \Gr{} and \Gb{} in Fig.~\ref{fig:simultaneous-level}(a). Consider any simultaneous level embedding  of \sefeinstance{} and assume, w.l.o.g. up to a renaming, that  appear in this left-to-right order along .

\begin{lemma} \label{le:sl-ordering}
For every , vertices  appear in this left-to-right order along  in~; also, for every , vertices  appear in this left-to-right order along~ in~. 
\end{lemma} 

\begin{proof}
	Suppose, for a contradiction, that the statement does not hold. Then let  be the smallest index such that either: 
	
	\begin{itemize}
		\item[(A)] for every , vertices  appear in this left-to-right order along ; for every , vertices  appear in this left-to-right order along ; and vertices  do not appear in this left-to-right order along ; or 
		\item[(B)] for every , vertices  appear in this left-to-right order along ; for every , vertices  appear in this left-to-right order along ; and vertices  do not appear in this left-to-right order along .
	\end{itemize}
	
	Suppose that we are in Case (A), as the discussion for Case (B) is analogous. Then  appear in this left-to-right order along , while  do not appear in this left-to-right order along . Hence, there exist indices  and  such that  is to the left of  along , while  is to the right of  along . Hence, edges  and   cross, thus contradicting the assumption that  is a simultaneous level embedding, as they both belong to \Gb{}. 
\end{proof}


The {\em triplet gadget} is a path  on two levels, where , , and  belong to a level  and  and  belong to a level . See  in Fig.~\ref{fig:simultaneous-level}(a), with  and . We have the following.

\begin{lemma} \label{le:sl-triplet}
In every level embedding of , vertex  is between  and  along~.
\end{lemma} 

\begin{proof}
	Suppose, for a contradiction, that  is to the left of  and  along ; the case in which it is to their right is analogous. Also assume that  is below , as the other case is symmetric. If  is to the left of  along , then edges  and  cross, otherwise edges  and  cross. In both cases we have a contradiction.
\end{proof}

We are now ready to prove the claimed {\cal NP}-completeness results.

\begin{figure}[tb]
\centering
\subfloat[]{\includegraphics[scale=1]{3Graphs.pdf}} 
\subfloat[]{\includegraphics[scale=1]{3Levels.pdf}} 
\caption{Instances (a)  and (b) \sefeinstance{} corresponding to an instance of {\sc Betweenness} with .} 
\label{fig:simultaneous-level}
\end{figure}

\begin{theorem} \label{thm:sim-level-npc-two-levels} {\sc Simultaneous Level Planarity} is \NPC even for three graphs on two levels and for two graphs on three levels.
\end{theorem}
\begin{proof}
Both problems clearly are in \NP. We prove the \NP-hardness only for three graphs on two levels (see Fig.~\ref{fig:simultaneous-level}(a)), as the other proof is analogous (see Fig.~\ref{fig:simultaneous-level}(b)). We construct an instance  of {\sc Simultaneous Level Planarity} from an instance  of {\sc Betweenness} as follows: Pair \sefeinstance{} contains an ordering gadget on levels  and , where the vertices  of \Gr{} are (in bijection with) the elements of . Graph \Gg{} contains  triplet gadgets , for . Vertices  are all distinct and are on . Clearly, the construction can be carried out in linear time.  We prove the equivalence of the two instances. 


 Suppose that a simultaneous level embedding  of \sefethreeinstance{} exists. We claim that the left-to-right order of  along  satisfies the betweenness constraints in . Suppose, for a contradiction, that there exists a triplet  with  not between  and  along . By Lemma~\ref{le:sl-ordering},  is not between  and . By Lemma~\ref{le:sl-triplet}, we have that  is not planar in  , a contradiction. 

 Suppose that  is a positive instance of {\sc Betweenness}, and assume, w.l.o.g. up to a renaming, that  is a solution for .
Construct a straight-line simultaneous level planar drawing of \sefethreeinstance{} with:
\begin{inparaenum}[(i)]
\item vertices    in this
  left-to-right order along , 
\item  vertices   in this
  left-to-right order along ,
\item  vertices  and  to the left
  of vertices  and , for , and
\item vertex  to the left of vertex  if and only if .
\end{inparaenum}

Properties (i) and (ii) guarantee that, for any two edges  and , vertex  is to the left of  along  if and only if  is to the left of  along , which implies the planarity of \Gr{} in . The planarity of \Gb{} in  is proved analogously. Properties (i) and (iii) imply that no two paths  and  cross each other, while Property (iv) guarantees that each path  is planar. Hence, the drawing of \Gg{} in  is planar.
\end{proof}

The graphs in Theorem~\ref{thm:sim-level-npc-two-levels} can be made connected, by adding vertices and edges, at the expense of using one additional level. Also, the theorem holds true even if the simultaneous embedding is {\em geometric} or {\em with fixed edges} (see~\cite{bkr-sepg-13,bcdeeiklm-spge-07} for definitions).

In contrast to the NP-hardness results, a reduction to a proper instance of {\sc Cyclic Level Planarity} allows us to decide in polynomial time instances composed of two graphs on two levels. Namely, the edges of a graph are directed from  to , while those of the other graph are directed from  to .

\begin{theorem} \label{th:sim-poly}
{\sc Simultaneous Level Planarity} is quadratic-time solvable for two graphs on two levels. 
\end{theorem}

\begin{proof}
Let  be an instance of the {\sc Simultaneous Level Planarity} problem, where each of \Gr{} and \Gb{} is a level graph on two levels  and . We define a proper instance  of {\sc Cyclic Level Planarity} as follows. The vertex set  is the same as the one of \Gr{} and \Gb{}, as well as the function ; further,  contains an edge  for every  and an edge  for every . We prove that  is simultaneous level planar if and only if  is cyclic level planar.

 Consider a simultaneous level embedding of \sefeinstance{}, map it to the surface  of the rolling cylinder, and wrap the edges of \Gb{} around the part delimited by  and  not containing the edges of \Gr{}, hence obtaining a cyclic level embedding of . 

 Consider a cyclic level embedding of  on , reroute the edges of \Gb{} to lie in the part of  delimited by  and  and containing the edges of \Gr{}, and map this drawing to the plane; this results in a simultaneous level embedding of \Gr{} and \Gb{}. 

The statement of the theorem then follows from Corollary~\ref{co:cyclicLevel-polynomial} and from the fact that the described reduction can be performed in linear time.
\end{proof}









\section{Conclusions and Open Problems} \label{se:conclusions}

In this paper we have settled the computational complexity of two of the main open problems in the research topic of level planarity by showing that the {\sc Cyclic Level Planarity} and the {\sc Torus Level Planarity} problems are polynomial-time solvable.
Our algorithms run in quartic time in the graph size; it is hence an interesting challenge to design new techniques to improve this time bound. 
We also introduced a notion of simultaneous level planarity for level graphs and we established a complexity dichotomy for this problem.



\begin{wrapfigure}[5]{r}{0.53\textwidth}
  \vspace{-20pt}
  \centering
\includegraphics{double-torus.pdf}
\label{fig:double-torus}
\end{wrapfigure}

An intriguing research direction is the one of extending the concept of level planarity to surfaces with genus larger than one. 
However, there seems to be more than one meaningful way to arrange  levels on a high-genus surface.
A reasonable choice would be the one shown in the figure, in which the levels are arranged in different sequences between two distinguished levels  and  (and edges only connect vertices on two levels in the same sequence). 
{\sc Radial Level Planarity} and {\sc Torus Level Planarity} can be regarded as special cases of this setting (with only one and two paths of levels between  and , respectively). 




\bibliographystyle{splncs03}
\bibliography{bibliography}

\end{document}
