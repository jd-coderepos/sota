\documentclass[letter,11pt]{article}
\usepackage{amsmath,amssymb,amsthm,color,graphicx,fullpage}
\usepackage{euscript}
\usepackage{times}





\newcommand{\seclab}[1]{\label{sec:#1}}
\newcommand{\theolab}[1]{\label{theo:#1}}
\newcommand{\chaplab}[1]{\label{chap:#1}}
\newcommand{\eqnlab}[1]{\label{eqn:#1}}
\newcommand{\corlab}[1]{\label{cor:#1}}
\newcommand{\figlab}[1]{\label{fig:#1}}
\newcommand{\tablab}[1]{\label{tab:#1}}


\newtheorem{theorem}{Theorem}[section]
\newtheorem{corollary}[theorem]{Corollary}
\newtheorem{lemma}[theorem]{Lemma}
\newtheorem{claim}[theorem]{Claim}
\graphicspath{{Figs/}}

\def\NN{\EuScript{N}}
\def \reals{{\mathbb R}}
\def \sphere{{\mathbb S}}
\def\Re{{\mathbb R}}
\def\dirtour{{\mathcal D}}
\def\xitour {{\mathcal V}}
\def\W{{\mathcal W}}
\def\bd{{\partial}}
\def\eps{{\varepsilon}}
\def\eps{{\varepsilon}}
\def\poly{\diamond}
\def\php{\varphi^\poly_j}
\def\C{\EuScript{C}}
\def\S{\mathcal{S}}
\def\E{\mathcal{E}}
\def\K{\EuScript{K}}
\def\T{\EuScript{T}}
\def\Triag{\mathbb{T}}
\def\PT{\mathbb{PT}}
\def\KK{\mathbb{K}}
\def\Chain{\mathcal{C}}
\def\CC{\mathbb{C}}
\def\TT{{\cal T}}
\newcommand{\ignore}[1]{}

\def\inprod#1#2{\langle #1, #2\rangle}

\def\bisect{b}
\def\Bisect{\EuScript{B}}
\def\Graph{\mathbb{H}}
\def\CH{\mathbb{CH}}
\def\Nbrs{N}
\def\tour{\K}

\def\indset#1#2#3{#1_{#2}^{(#3)}}

\def\distfn{\varphi}
\def\tree{\EuScript{T}}
\def\canonical{\C}
\def\extrpt{\psi}
\def\extrset{\Psi}
\def\extrpair{\xi}
\def\nn{\nu}

\def\conv{\mathcal{CH}}
\def\DDG{\mathop{\mathrm{DDG}}}
\def\G{{\sf G}}
\def\SDG{\mathop{\mathrm{SDG}}}
\def\PSDG{\mathop{\mathrm{PSDG}}}
\def\DT{\mathop{\mathrm{DT}}}
\def\VD{\mathop{\mathrm{VD}}}
\def\Vor{\mathop{\mathrm{Vor}}}
\def\intr{\mathop{\mathrm{int}}}

\def\marrow{{\marginpar[\hfill]{}}}

\def\leo#1{{\sc Leo says: }{\marrow\sf #1}}
\def\jie#1{{\sc Jie says: }{\marrow\sf #1}}
\def\pankaj#1{{\sc Pankaj says: }{\marrow\sf #1}}
\def\vladlen#1{{\sc Vladlen says: }{\marrow\sf #1}}
\def\micha#1{{\sc Micha says: }{\marrow\sf #1}}
\def\haim#1{{\sc Haim says: }{\marrow\sf #1}}
\def\natan#1{{\sc Natan says: }{\marrow\sf #1}}

\begin{document}

\begin{titlepage}

\title{Kinetic Stable Delaunay Graphs\thanks{A preliminary version of this paper appeared in {\it Proc. 26th Annual Symposium on Computational Geometry}, 2010, pp. 127--136.}}

\author{Pankaj K. Agarwal\thanks{Department of Computer Science, Duke University, Durham, NC
    27708-0129, USA, {\tt pankaj@cs.duke.edu}.}
\and
Jie Gao\thanks{Department of Computer Science, Stony Brook University, Stony
Brook, NY 11794, USA, {\tt jgao@cs.sunysb.edu}. }
\and
Leonidas Guibas\thanks{Department of Computer Science, Stanford University, Stanford,
CA 94305, USA, {\tt guibas@cs.stanford.edu}.}
\and
Haim Kaplan\thanks{School of Computer Science, Tel Aviv University, Tel~Aviv 69978, Israel.
{\tt haimk@tau.ac.il}.}
\and
Vladlen Koltun\thanks{Department of Computer Science, Stanford University, Stanford,
CA 94305-9025, USA, {\tt vladlen@cs.stanford.edu}.}
\and
Natan Rubin\thanks{School of Computer Science, Tel Aviv University, Tel~Aviv 69978, Israel.
{\tt rubinnat@tau.ac.il}.}
\and
Micha Sharir\thanks{School of Computer Science, Tel Aviv University, Tel~Aviv 69978, Israel;
and Courant Institute of Mathematical Sciences, New York University,
New York, NY~~10012,~USA.  {\tt michas@tau.ac.il}.}
}

\maketitle


\begin{abstract}
We consider the problem of maintaining the Euclidean Delaunay
triangulation  of a set  of  moving points in the plane, along algebraic tranjectories of constant description complexity.
Since the best known upper bound on the number of topological changes 
in the full Delaunay triangulation is only nearly cubic, we seek 
to maintain a suitable portion of the diagram that is less volatile 
yet retains many useful properties of the full triangulation.  
We introduce the notion of a {\em stable Delaunay graph}, which is 
a dynamic subgraph of the Delaunay triangulation. The stable Delaunay graph (a) is easy to
define, (b) experiences only a nearly quadratic number of discrete 
changes, (c) is robust under small changes of the norm, and (d)
possesses certain useful properties for further applications.

The stable Delaunay graph ( in short) is defined in terms of
a parameter , and consists of Delaunay edges  for
which the (equal) angles at which  and  see the corresponding
Voronoi edge  are at least .
We show that (i)  always contains at least roughly one third of the
Delaunay edges at any fixed time; (ii) it contains the
-skeleton of , for ; (iii) it is
stable, in the sense that its edges survive for long periods of time,
as long as the orientations of the segments connecting (nearby) points
of  do not change by much; and (iv) stable Delaunay edges remain
stable (with an appropriate redefinition of stability) if we
replace the Euclidean norm by any sufficiently close norm.

In particular, if we approximate the Euclidean norm by a polygonal 
norm (with a regular -gon as its unit ball, with
), we can define and keep track of a Euclidean 
by maintaining the full Delaunay triangulation of  under the
polygonal norm (which is trivial to do, and which is known to involve only a
nearly quadratic number of discrete changes).

We describe two kinetic data structures for maintaining  when
the points of  move along pseudo-algebraic trajectories of constant description complexity. The first
uses the polygonal norm approximation noted above, and the second is
slightly more involved, but significantly reduces the dependence of 
its performance on .  Both structures use  storage 
and process  events during the motion, each in  time.
(Here the  notation hides multiplicative factors which 
are polynomial in  and polylogarithmic in .)
\end{abstract}

\end{titlepage}

\section{Introduction}


\paragraph{Delaunay triangulations and Voronoi diagrams.} 
Let  be a (finite) set of points in . 
Let  and  denote the Voronoi diagram and Delaunay
triangulation of , respectively. For a point , let
 denote the Voronoi cell of . 
The Delaunay triangulation  consists of all 
triangles whose circumcircles do not contain points of  in their
interior. Its edges form the {\em Delaunay graph}, which is the 
straight-edge dual graph of the Voronoi diagram of . That is,
 is an edge of the Delaunay graph if and only if
 and  share an edge, which we denote by .
This is equivalent to the existence of a circle passing through 
and  that does not contain any point of  in its interior---any
circle centered at a point on  and passing through  and 
is such a circle. 
Delaunay triangulations and Voronoi diagrams are fundamental to much 
of computational geometry and its applications. 
See \cite{AK,Ed2} for a survey and a
textbook on these structures.

In many applications of Delaunay/Voronoi methods (e.g., mesh generation and kinetic collision detection) the points are moving continuously, so
these diagrams need to be efficiently updated as motion occurs.
Even though the motion of the nodes is continuous, the combinatorial and topological structure of the Voronoi and
Delaunay diagrams change only at
discrete times when certain critical events occur. Their evolution
under motion can be studied within the framework of {\em kinetic data
structures} (KDS in short) of Basch {\em et al.}~\cite{bgh-dsmd-99,285869,g-kdssar-98},
a general methodology for designing efficient algorithms for maintaining
such combinatorial attributes of mobile data. 


For the purpose of kinetic maintenance, Delaunay triangulations are 
nice structures, because, as mentioned above, they admit local 
certifications associated with individual triangles.  This makes 
it simple to maintain  under point motion: an update is 
necessary only when one of these empty circumcircle conditions 
fails---this corresponds to cocircularities of certain subsets of
four points.\footnote{We assume that the motion of the points is sufficiently generic, so that no more than four points can become cocircular at any given time.} Whenever such an event happens, 
a single edge flip easily restores Delaunayhood. Estimating the 
number of such events, however, has been elusive---the problem 
of bounding the number of combinatorial changes in  for 
points moving along semi-algebraic trajectories of constant description complexity has been in the 
computational geometry lore for many years; see \cite{TOPP}.

Let  be the number of moving points in . We
assume that each point moves along an algebraic trajectory of
fixed degree or, more generally, along pseudo-algebraic trajectory of constant description complexity (see Section~\ref{sec:Prelim} for a more formal
definition).
Guibas et al.~\cite{gmr-vdmpp-92} showed a roughly cubic upper bound of
 on the number of discrete (also known as \textit{topological}) changes in , where  is the maximum length
of an -Davenport-Schinzel sequence~\cite{SA95}, and  is a constant
depending on the motions of the points. A substantial gap exists between this upper bound
and the best known quadratic lower bound~\cite{SA95}. 


It is thus desirable to find approaches for maintaining a substantial 
portion of  that {\em provably} experiences only a nearly 
quadratic number of discrete changes, that is reasonably easy to define and 
maintain, and that retains useful properties for further applications.

\paragraph{Polygonal distance functions.} 
If the ``unit ball" of our
underlying norm is {\em polygonal} then things improve considerably.
In more detail, let  be a convex polygon with a constant
number, , of edges.  It induces a {\em convex distance function}
 
 is a metric if  is centrally symmetric with respect to the origin.

We can define the -Voronoi diagram
of a set  of points in the plane in the usual way, as the
partitioning of the plane into Voronoi cells, so that the cell
 of a point  is
.
Assuming that the points of  are in general position with respect
to , these cells are nonempty, have pairwise disjoint interiors,
and cover the plane. 

As in the Euclidean case, the -Voronoi diagram of  has its 
dual representation, which we refer to as the {\em -Delaunay
triangulation} . A triple of points in  define a
triangle in  if and only if they lie on the boundary of some 
homothetic copy of  that does not contain any point of  in its
interior. Assuming that  is in general position, these -Delaunay
triangles form a triangulation of a certain simply-connected polygonal 
region that is contained in the convex hull of .
Unlike the Euclidean case, it does not always coincide with the convex hull (see Figures~\ref{Fig:ConesCertif} and~\ref{Fig:AlmostTriangulation} for examples).
See Chew and Drysdale~\cite{CD} and Leven and Sharir~\cite{LS} for analysis of Voronoi and Delaunay diagrams of this kind.

For kinetic maintenance, polygonal Delaunay triangulations are
``better'' than Euclidean Delaunay triangulations because, as shown by
Chew~\cite{Chew}, when the points of  move (in the algebraic
sense assumed above), the number of topological changes in the 
-Delaunay triangulation is only nearly quadratic in .
One of
the major observations in this paper is that the \textit{stable portions} of the Euclidean Delaunay triangulation and the -Delaunay triangulation are closely related.

\paragraph{Stable Delaunay edges.} 
We introduce the notion of \textit{-stable Delaunay edges}, 
for a fixed parameter , defined as follows.  Let  be 
a Delaunay edge under the Euclidean norm, and let  and  
be the two Delaunay triangles incident to .  Then  is 
called {\em -stable} if its opposite angles in these triangles
satisfy . (The case where
 lies on the convex hull of  is treated as if one of  
lies at infinity, so that the corresponding angle  or 
 is equal to .) An equivalent and more useful definition, in terms of the dual Voronoi diagram, is that 
 is -stable if the equal angles at which  and  
see their common Voronoi edge  are at least .
See Figure \ref{Fig:LongDelaunay}.

\begin{figure}[htbp]
\begin{center}
\begin{picture}(0,0)\includegraphics{LongDelaunay.pstex}\end{picture}\setlength{\unitlength}{2368sp}\begingroup\makeatletter\ifx\SetFigFont\undefined \gdef\SetFigFont#1#2#3#4#5{\reset@font\fontsize{#1}{#2pt}\fontfamily{#3}\fontseries{#4}\fontshape{#5}\selectfont}\fi\endgroup \begin{picture}(3793,3316)(722,-3160)
\put(2656,-739){\makebox(0,0)[lb]{\smash{{\SetFigFont{12}{14.4}{\rmdefault}{\mddefault}{\updefault}{\color[rgb]{0,0,0}}}}}}
\put(3143,-1071){\makebox(0,0)[lb]{\smash{{\SetFigFont{12}{14.4}{\rmdefault}{\mddefault}{\updefault}{\color[rgb]{0,0,0}}}}}}
\put(2176,-1561){\makebox(0,0)[rb]{\smash{{\SetFigFont{12}{14.4}{\rmdefault}{\mddefault}{\updefault}{\color[rgb]{0,0,0}}}}}}
\put(737,-1474){\makebox(0,0)[lb]{\smash{{\SetFigFont{12}{14.4}{\rmdefault}{\mddefault}{\updefault}{\color[rgb]{0,0,0}}}}}}
\put(3785,-987){\makebox(0,0)[lb]{\smash{{\SetFigFont{12}{14.4}{\rmdefault}{\mddefault}{\updefault}{\color[rgb]{0,0,0}}}}}}
\put(2138,-75){\makebox(0,0)[lb]{\smash{{\SetFigFont{12}{14.4}{\rmdefault}{\mddefault}{\updefault}{\color[rgb]{0,0,0}}}}}}
\put(2088,-3064){\makebox(0,0)[lb]{\smash{{\SetFigFont{12}{14.4}{\rmdefault}{\mddefault}{\updefault}{\color[rgb]{0,0,0}}}}}}
\put(2449,-2137){\makebox(0,0)[lb]{\smash{{\SetFigFont{12}{14.4}{\rmdefault}{\mddefault}{\updefault}{\color[rgb]{0,0,0}}}}}}
\put(3151,-1601){\makebox(0,0)[lb]{\smash{{\SetFigFont{12}{14.4}{\rmdefault}{\mddefault}{\updefault}{\color[rgb]{0,0,0}}}}}}
\put(1037,-887){\makebox(0,0)[lb]{\smash{{\SetFigFont{12}{14.4}{\rmdefault}{\mddefault}{\updefault}{\color[rgb]{0,0,0}}}}}}
\put(4439,-851){\makebox(0,0)[lb]{\smash{{\SetFigFont{12}{14.4}{\rmdefault}{\mddefault}{\updefault}{\color[rgb]{0,0,0}}}}}}
\put(1962,-1105){\makebox(0,0)[lb]{\smash{{\SetFigFont{12}{14.4}{\rmdefault}{\mddefault}{\updefault}{\color[rgb]{0,0,0}}}}}}
\end{picture} \caption{\small \sf The points  and  see their common Voronoi edge  at (equal) angles . This is equivalent to the angle condition  for the two adjacent Delaunay triangles.}
\label{Fig:LongDelaunay}
\end{center}
\end{figure}

A justification for calling such edges stable lies in the following
observation: If a Delaunay edge  is -stable then it 
remains in  during any continuous motion of  for which
every angle , for , changes
by at most . This is clear because at the time  is
-stable we have
 for \textit{any} pair of points
,  lying on opposite sides of the line  supporting , so,
if each of these angles change by at most  we still have
, which is easily seen to imply
that  remains an edge of . (This argument also covers the cases when a point  crosses  from side to side: Since
each point, on either side of , sees  at an angle of , it follows that no point can cross
 itself -- the angle has to increase from  to . Any other crossing of  by a point  causes  to
decrease to , and even if it increases to  on the other side of ,  is still an edge of , as is easily checked.)
Hence, as long as the ``small angle change'' condition 
holds, stable Delaunay edges remain a ``long time'' in the
triangulation.

Informally speaking, the non-stable edges  of  are those for  and 
are almost cocircular with their two common Delaunay neighbors
, , and hence are more likely to get flipped ``soon". 

\paragraph{Overview of our results.}
Let  be a fixed parameter.
In this paper we show how to maintain a subgraph of the full Delaunay 
triangulation , which we call a {\em -stable Delaunay graph} ( in short), so that (i) every edge of  is -stable, 
and (ii) every -stable edge of  belongs to , where  is some (small) absolute constant.
Note that  is not uniquely defined, even when  is fixed.

In Section \ref{sec:Prelim}, we introduce several useful definitions and show that the number of discrete changes in the s
that we consider
is nearly quadratic. 
What this analysis also implies is that if the true bound for
kinetic changes in a Delaunay triangulation is really close to cubic, then
the overhelming majority of these changes involve edges which never become stable and just flicker in and out of the diagram by cocircularity with their two Delaunay neighbors.

In Sections \ref{Sec:polygProp} and \ref{Sec:ReduceS} we show that  can be
maintained by a kinetic data structure that uses only near-linear
storage (in the terminology of \cite{bgh-dsmd-99}, it is {\em compact}), 
encounters only a nearly quadratic number of critical events 
(it is {\em efficient}), and processes each event in polylogarithmic 
time (it is {\em responsive}). For the second data structure, described in Section \ref{Sec:ReduceS}, can be slightly modified to ensure that each point appears at any time 
in only polylogarithmically many places in the structure (it then becomes 
{\em local}). 

The scheme described in Section \ref{Sec:polygProp} is based on a useful and interesting ``equivalence" connection
between the (Euclidean)  and a suitably defined ``stable" version of the Delaunay triangulation of  under the ``polygonal" norm whose unit ball  is a regular
-gon, for . As noted above, Voronoi and Delaunay structures under polygonal norms are particularly
favorable for kinetic maintenance because of Chew's
result~\cite{Chew}, showing that the number of topological changes in
 is ; here
the  notation hides a factor that depends
sub-polynomially on both  and . In other words, the scheme simply maintains the ``polygonal" diagram  in its entirety, and selects from it those edges that are also stable edges of the Euclidean diagram .

The major disadvantage of the solution in Section \ref{Sec:polygProp} is
the rather high (proportional to ) dependence on 
 of the bound on the number of
topological changes. We do not know whether the upper bound  on the number of topological changes in
 is nearly tight (in its dependence on ). 
To remedy this, we present in Section \ref{Sec:ReduceS} an 
alternative scheme for maintaining stable
(Euclidean) Delaunay edges. The scheme is reminiscent of the kinetic
schemes used in \cite{KineticNeighbors} for maintaining closest pairs
and nearest neighbors. It extracts  pairs of points of  
that are candidates for being stable Delaunay edges. Each point
 then runs  \textit{kinetic and dynamic tournaments} involving
the other points in its candidate pairs. Roughly, these tournaments
correspond to shooting  rays from  in fixed directions and finding along each ray
the nearest point equally distant from  and from some other
candidate point . We show that  is a stable Delaunay edge if and
only if  wins many (at least some constant number of) consecutive tournaments of  (or  wins many consecutive tournaments of ). A careful analysis shows that
the number of events that this scheme processes (and the overall 
processing time) is only .

Section \ref{Sec:SDGProperties} establishes several useful properties of stable Delaunay graphs. In particular, we show that at
any given time the stable subgraph contains at least  Delaunay
edges, i.e., at least about one third of the maximum possible number of edges. In addition, we 
show that at any moment the  contains the closest pair, the so-called 
\textit{-skeleton} of , for  (see \cite{Crusts,Skeletons}), and the \textit{crust} of a sufficiently densely sampled point set along a smooth curve (see \cite{Amenta,Crusts}). 
We also extend the connection in Section \ref{Sec:polygProp} to arbitrary distance functions  whose unit ball  is sufficiently close (in the Hausdorff sense) to the Euclidean one (i.e., the unit disk).


\section{Preliminaries}\label{sec:Prelim}
\seclab{sdg}\seclab{ddg}
\paragraph{Stable edges in Voronoi diagrams.}
Let  be a set of 
 equally spaced directions in . For 
concreteness take ,  (so our directions  go clockwise as  increases).\footnote{The index arithmetic is modulo , i.e., .} 
For a point  and a unit vector
 let  denote the ray  that
emanates from  in direction .  For a pair of points  
let  denote the perpendicular bisector of  and .
If  intersects , then the expression

is the distance
between  and the intersection point of  with
.
If  does not
intersect  we define .
 The point  minimizes , among all points
 for which  intersects , if and only if the
intersection between  and  lies on the Voronoi edge . We call  the {\em neighbor of  in direction }, 
and denote it by ; see Figure \ref{Fig:StableVoronoi}.

The {\em (angular) extent} of a Voronoi edge  of two points 
 is
the angle at which it is seen from either  or  (these two
angles are equal).  For a given angle , 
  is  called {\em -long} (resp., {\em
-short}) if the extent of  is at least
 (resp., smaller than) . We also say that
 is {\em -long} (resp., {\em
-short}) if  is {\em -long} (resp., {\em
-short}). As noted in the introduction, these notions can also be defined (equivalently) in terms of the angles in the Delaunay triangulation: A Delaunay edge , which is not a hull edge, is -long if and only if ,
where  and  are the two Delaunay triangles incident to . See Figure \ref{Fig:LongDelaunay}; hull edges are handled similarly, as discussed in the introduction.


Given parameters , we seek to construct (and
maintain under motion) an \emph{-stable Delaunay
graph} (or \emph{stable Delaunay graph}, for brevity, which we further abbreviate as ) of , which
is any subgraph  of  with the following properties:
\begin{itemize}
\item[(S1)]
  Every -long edge of  is an edge of
  .
\item[(S2)]
  Every edge of  is an -long edge of .
\end{itemize}
An -stable Delaunay graph need not be
unique. In what follows,  will always be some fixed (and reasonably small) multiple of .

\paragraph{Kinetic tournaments.} 
Kinetic tournaments were first studied by
Basch \textit{et al.}~\cite{bgh-dsmd-99}, for kinetically maintaining the lowest point in a set  of  points moving on some vertical line, say the -axis, so that their trajectories are algebraic of bounded degree, as above. 
These tournaments are a key ingredient in the data structures that we will develop for maintaining stable Delaunay graphs. Such a tournament is 
represented and maintained using the following variant of a heap.
Let  be a minimum-height balanced binary tree, with the points stored
at its leaves (in an arbitrary order). For an internal node ,
let  denote the set of points stored in the subtree rooted at . At any
specific time , each internal node  stores the lowest point
among the points in  at time , which is called the {\em winner\/} at .
The winner at the root is the desired overall lowest point of .

To maintain  we associate a certificate with each internal node , which
asserts which of the two winners, at the left child and at the
right child of , is the winner at . This certificate remains
valid as long as (i) the winners at the children of  do not change,
and (ii) the order along the -axis between these two
``sub-winners'' does not change. The actual certificate caters
only to the second
condition; the first will be taken care of recursively.
Each certificate has an associated failure time, which is the next time
when these two winners switch their order along the -axis.
We store all certificates in another heap, using the failure times
as keys.\footnote{Any ``standard'' heap that supports
  {\bf insert}, {\bf delete}, and {\bf deletemin} in 
  time is good for our purpose.}
This heap of certificates is called the {\em event queue}.


Processing an event is simple. When the two sub-winners  at some node  change their order, we compute the new failure time of the certificate at  (the first future time when  and  meet again), update the event queue accordingly, and propagate the new winner, say , up the tree, revising the certificates at the ancestors of , if needed.

If we assume that the  trajectories of each pair of points intersect at most 
times then
the overall number of changes of winners, and
therefore also the overall number of events, is at most
. Here
, and  is the maximum length of a Davenport-Schinzel sequence
of order  on  symbols; see \cite{SA95}.

This is larger by a logarithmic factor than the maximum possible
number of times the lowest point along the -axis can indeed change,
since this latter number is bounded by the complexity of the lower
envelope of the trajectories of the points in  (which, as noted above, records the changes in the winner at the root of ).

Agarwal {\em et al.}~\cite{KineticNeighbors} show how to make
such a tournament also {\em dynamic\/}, supporting insertions and deletions of points. They replace the balanced binary tree  by
 a {\em weight-balanced  tree} \cite{NR73}
(and see also \cite{Mehlhorn}).  This allows us to insert a new point
anywhere we wish in , and to delete any point from ,
in  time. Each such insertion or deletion may
change  certificates, along the corresponding search path,
and therefore updating the event queue takes  time, including the time for the
structural updates of (rotations in) ; here  denotes the
actual number of points in , at the step where we perform
the insertion or deletion. The analysis of \cite{KineticNeighbors} is summarized in Theorem \ref{thm:kinetic-tour}.



\begin{theorem}[\textbf{Agarwal \textit{et al.}}~\cite{KineticNeighbors}] \label{thm:kinetic-tour}
A sequence of  insertions and deletions into a kinetic tournament,
whose maximum size at any time is  (assuming ), when
implemented as a weight-balanced tree in the manner described above,
generates at most  events, with a total processing cost
of . Here  is the maximum number of times a pair of points intersect, and .
Processing an update or a tournament event takes
 worst-case time. A dynamic kinetic tournament on 
elements can be constructed in  time.
\end{theorem}

\noindent {\it Remarks:} (1) Theorem \ref{thm:kinetic-tour} subsumes the static case too, by inserting all the elements ``at the beginning of time", and then tracing the kinetic changes. \\
\noindent (ii) Note that the amortized cost of an update or of processing a tournament event is only  (as opposed to the  worst-case cost).

\paragraph{Maintenance of an SDG.}
Let  be a set of points 
moving in . Let  denote the position
of  at time . We call the motion of 
\emph{algebraic} if each  is a
polynomial function of , and the \emph{degree} of motion of  is the maximum
degree of these polynomials.
Throughout this paper we assume that the motion of  is
algebraic and that its degree is bounded by a constant.
In this subsection we present a simple technique for maintaining a
-stable Delaunay graph. Unfortunately this
algorithm requires quadratic space.
It is based on the following easy observation (see Figure \ref{Fig:StableVoronoi}),
where  is an integer, and the unit vectors (directions)  are as defined earlier.

\begin{lemma} \label{lem:alpha}
Let .
(i) If the extent of  is larger than  then there are two consecutive directions 
, , such that  is the neighbor of  in directions  and . \\
(ii) If there are two consecutive directions , such that  is the neighbor of  in both directions  and , then
the extent of  is at least .
\end{lemma}

\begin{figure}[htbp]
\begin{center}
\begin{picture}(0,0)\includegraphics{StableVoronoi.pstex}\end{picture}\setlength{\unitlength}{1973sp}\begingroup\makeatletter\ifx\SetFigFont\undefined \gdef\SetFigFont#1#2#3#4#5{\reset@font\fontsize{#1}{#2pt}\fontfamily{#3}\fontseries{#4}\fontshape{#5}\selectfont}\fi\endgroup \begin{picture}(3827,3701)(4564,-5765)
\put(5750,-2789){\makebox(0,0)[lb]{\smash{{\SetFigFont{12}{14.4}{\rmdefault}{\mddefault}{\updefault}{\color[rgb]{0,0,0}}}}}}
\put(5791,-3988){\makebox(0,0)[lb]{\smash{{\SetFigFont{12}{14.4}{\rmdefault}{\mddefault}{\updefault}{\color[rgb]{0,0,0}}}}}}
\put(7317,-3047){\makebox(0,0)[lb]{\smash{{\SetFigFont{11}{13.2}{\rmdefault}{\mddefault}{\updefault}{\color[rgb]{0,0,0}}}}}}
\put(7734,-3445){\makebox(0,0)[lb]{\smash{{\SetFigFont{12}{14.4}{\rmdefault}{\mddefault}{\updefault}{\color[rgb]{0,0,0}}}}}}
\put(8171,-4152){\makebox(0,0)[lb]{\smash{{\SetFigFont{12}{14.4}{\rmdefault}{\mddefault}{\updefault}{\color[rgb]{0,0,0}}}}}}
\put(4727,-4329){\makebox(0,0)[lb]{\smash{{\SetFigFont{11}{13.2}{\rmdefault}{\mddefault}{\updefault}{\color[rgb]{0,0,0}}}}}}
\put(4623,-3462){\makebox(0,0)[lb]{\smash{{\SetFigFont{11}{13.2}{\rmdefault}{\mddefault}{\updefault}{\color[rgb]{0,0,0}}}}}}
\end{picture} \caption{\small \sf  is the neighbor of  in the directions  and , so the Voronoi edge  is -long.}
\label{Fig:StableVoronoi}
\end{center}
\end{figure}

The algorithm maintains Delaunay edges  such that there are two consecutive directions
 and  along which  is the neighbor of .
For each point  and direction  we get a set of at most  piecewise
continuous functions of time, , one for each point , as defined in (\ref{Eq:DirectDist}). (Recall that  when 
does not intersect .) By assumption on the motion of ,
for each  and , the domain in which  is 
defined consists of a constant number of intervals.

For each point , and ray , 
consider each function   as the
trajectory of a point moving
along the ray and corresponding to . The algorithm maintains
these points in a dynamic and kinetic tournament  
(see Theorem \ref{thm:kinetic-tour}) that keeps track of the 
minimum of  over time.
For each pair of points  and  such that 
 wins in two consecutive tournaments,  and , of ,
 it keeps the edge  in
the stable Delaunay graph. It is trivial to update this graph as a by-product of the updates of the various tournaments.
The analysis of this
data structure is straightforward using Theorem \ref{thm:kinetic-tour},
and yields the following result.
\begin{theorem} \label{thm:ddj}
Let  be a set of  moving points in  under algebraic
motion of bounded degree, let  be an integer, and let .
A -stable Delaunay graph
of  can be maintained using
 storage and processing
  events, for a total cost
of  time.
The processing of each event takes
 worst-case time.
Here  is a constant that depends on the degree of motion of .
\end{theorem}

Later on, in Section \ref{Sec:ReduceS}, we will revise this approach and reduce the storage to nearly linear, by letting only 
a small number of points to participate in each tournament. The filtering procedure for the points makes the improved solution 
somewhat more involved.
\section{An SDG Based on Polygonal Voronoi Diagrams}
\label{sec:ViaPolygonal}
\label{Sec:polygProp}


Let  be a regular -gon
 for some even , circumscribed by the unit disk, and let  (this is the angle at which the center of  sees an edge). 
Let  and  denote the -Voronoi diagram and
the dual -Delaunay triangulation of , respectively.
In this section we show that the set of
edges of  with sufficiently many \textit{breakpoints} (see below for details) form a
-stable (Euclidean) Delaunay graph for appropriate multiples
 of .
Thus, by kinetically maintaining  (in its entirety),
we shall get ``for free'' a KDS for keeping track of a stable portion 
of the Euclidean DT.

\subsection{Properties of }
\label{Sec:PolygonalBackground}
We first review the properties of the (stationary)
 and . Then we consider the
kinetic version of these diagrams, as the points of  move, and
review Chew's proof~\cite{Chew} that the number of topological
changes in these diagrams, over time, is only nearly quadratic
in . Finally, we present a straightforward kinetic data structure 
for maintaining  under motion that uses linear storage, 
and that processes a nearly quadratic number of events, 
each in  time. 
Although later on we will take  to be a regular -gon, the analysis in this subsection is more general, and we only assume here that  is an arbitrary convex -gon, lying in general position with respect to .


\paragraph{Stationary -diagrams.}
The {\em bisector}  between
two points  and , with respect to , is the 
locus of all
placements of the center of any homothetic copy  of  that
touches  and .
 can be classified according to the pair of its edges,  and ,
that touch  and , respectively. If we slide  so that its
center moves along  (and its size expands or shrinks to
keep it touching  and ), and the contact edges,  and ,
remain fixed, the center traces a straight segment.
The bisector is a concatenation of  such segments. They
meet at {\em breakpoints}, which are placements of the center of a
copy  that touches  and  and one of the contact points
is a vertex of ; see Figure \ref{Fig:CornerContact}. We call such a 
placement a {\em corner contact} at the appropriate point. 
Note that a corner contact where some vertex  of (a copy  of)  touches 
has the property that the center of  lies on the fixed ray
emanating from  and parallel to the directed segment from 
to the center of .

\begin{figure}[htbp]
\begin{center}
\begin{picture}(0,0)\includegraphics{CornerPlacement.pstex}\end{picture}\setlength{\unitlength}{2171sp}\begingroup\makeatletter\ifx\SetFigFont\undefined \gdef\SetFigFont#1#2#3#4#5{\reset@font\fontsize{#1}{#2pt}\fontfamily{#3}\fontseries{#4}\fontshape{#5}\selectfont}\fi\endgroup \begin{picture}(2990,2991)(1367,-3668)
\put(1567,-3516){\makebox(0,0)[lb]{\smash{{\SetFigFont{10}{12.0}{\rmdefault}{\mddefault}{\updefault}{\color[rgb]{0,0,0}}}}}}
\put(4194,-1610){\makebox(0,0)[lb]{\smash{{\SetFigFont{10}{12.0}{\rmdefault}{\mddefault}{\updefault}{\color[rgb]{0,0,0}}}}}}
\put(1928,-1602){\makebox(0,0)[lb]{\smash{{\SetFigFont{10}{12.0}{\rmdefault}{\mddefault}{\updefault}{\color[rgb]{0,0,0}}}}}}
\end{picture} \caption{\small \sf Each breakpoint on  corresponds to a corner contact of  at one of the points , so that  also touches the other point.}\label{Fig:CornerContact}
\end{center}
\end{figure}



A useful property of bisectors and Delaunay edges, in the special case where  is a regular 
-gon, which will be used in the next subsection, is that the breakpoints along a bisector 
 alternate between corner contacts at  and corner contacts at . 
Indeed, assuming general position, each point  determines a unique
placement of  where it touches  at  and also touches , as
is easily checked. A symmetric property holds when we interchange 
and . Hence, as we slide the center of  along the bisector
, the points of contact of  with  and  vary
continuously and monotonically along . Consider two consecutive
corner contacts, , , of  at  along , and
suppose to the contrary that the portion of  between
them is a straight segment, meaning that, within this portion, 
 touches each of ,  at a fixed edge. Since the center of
 moves along the angle bisector of the lines supporting these
edges (a property that is easily seen to hold for regular -gons), it is easy to see that the distance between the two contact
points of , at the beginning and the end of this sliding, and
the distance between the two contact points of  (measured, say, 
on the boundary of the standard placement of ) are equal. However,
this distance for  is the length of a full edge of , because
the motion starts and ends with  touching a vertex, and therefore
the same holds for , which is impossible (unless  also starts
and ends at a vertex, which contradicts our general position
assumption).

Another well known property of -bisectors and Voronoi edges, for arbitrary convex polygons in general position with respect to , is that
two bisectors , , can intersect at
most once (again, assuming general position), so every -Voronoi edge  is connected. Equivalently, this
asserts that there exists at most one homothetic placement of  at
which it touches , , and . Indeed, since homothetic
placements of  behave like pseudo-disks (see, e.g., \cite{KLPS}),
the boundaries of two distinct homothetic placements of  intersect
in at most two points, or, in degenerate position, in at most two
connected segments. Clearly, in the former case the boundaries 
cannot both contain , , and , and this also holds in the
latter case because of our general position assumption.


Consider next an edge  of . Its dual Voronoi edge
 is a portion of the bisector , and consists of those
center placements along  for which the corresponding copy 
has an {\em empty interior} (i.e., its interior is disjoint from ).
Following the notation of Chew~\cite{Chew}, we call  a
{\em corner edge} if  contains a breakpoint
(i.e., a placement with a corner contact); otherwise it is a
{\em non-corner edge}, and is therefore a straight segment.

\paragraph{Kinetic -diagrams.}
Consider next what happens to  and 
as the points of  move continuously with time.
In this case  changes
continuously, but undergoes topological
changes at certain critical times, called \emph{events}. There are 
two kinds of events:

\smallskip
\noindent (i) \textsc{Flip Event.}
A Voronoi edge  shrinks to a point, disappears, and is
``flipped'' into a newly emerging Voronoi edge .

\smallskip
\noindent (ii) \textsc{Corner Event.}
An endpoint of some Voronoi edge  becomes a breakpoint (a
corner placement). Immediately after this time  either 
gains a new straight segment, or loses such a segment, that it had before the event.

\smallskip
Some comments are in order: 

\smallskip
\noindent(a) A flip event
occurs when the four points  become ``cocircular'':
there is an empty homothetic copy  of  that touches all four points.

\smallskip
\noindent(b) Only non-corner edges can participate in a flip event, as
both the vanishing edge  and the newly emerging edge
 do not have breakpoints near the event.

\smallskip
\noindent(c) A flip event entails a discrete change in the
Delaunay triangulation, whereas a corner event does not.
Still, for algorithmic purposes, we will keep track of both kinds of events.

\smallskip
We first bound the number of corner events.

\begin{lemma} \label{corners}
Let  be a set of  points in  under algebraic
motion of bounded degree, and let  be a convex -gon. 
The number of corner events in  is , 
where  is a constant that depends on the degree of motion
of .
\end{lemma}

\begin{proof}
Fix a point  and a vertex  of , and consider all the corner
events in which  touches . As noted above, at any such event the
center  of  lies on a ray  emanating from  at a fixed
direction. (Since  is moving,  is a moving ray, but its orientation remains fixed.) For each other point , let 
denote the distance, at time , from  along  to the center
of a copy of  that touches  (at ) and .
The value
 represents the intersection of
 with  at time , where  is the Voronoi cell of  in . The point  that attains the
minimum defines the Voronoi edge  (or vertex if the 
minimum is attained by more than one point ) of  that  intersects.

In other words, we have a collection of  partially defined
functions , and the breakpoints of their lower envelope
represent the corner events that involve the contact
of  with . By our assumption on the motion of , each 
function  is piecewise algebraic, 
with  pieces. Each piece encodes a continuous contact of  
with a specific edge of , and has constant description complexity. Hence (see, e.g., \cite[Corollary 1.6]{SA95}) the complexity of
the envelope is at most , for an appropriate constant
. Repeating the analysis for each point  and each vertex  of , the lemma
follows.
\end{proof}

Consider next flip events. As noted, each flip event involves a 
placement of an empty homothetic copy  of  that touches 
simultaneously four points  of , in this 
counterclockwise order along , so that the 
Voronoi edge ,
which is a non-corner edge before the event, shrinks to a point
and is replaced by the non-corner edge . Let 
denote the edge of  that touches , for . 

We fix the quadruple of edges , bound the number
of flip events involving a quadruple contact with these edges,
and sum the bound over all  choices of four edges of .
For a fixed quadruple of edges , we replace  by
the convex hull  of these edges, and note that any flip event
involving these edges is also a flip event
for . We therefore restrict our attention to , which is
a convex -gon, for some .

We note that if  is a Delaunay edge
representing a contact of some homothetic copy  of  where  and 
touch two {\em adjacent} edges of , then  must be a corner
edge---shrinking  towards the vertex common to the two edges,
while it continues to touch  and , will keep it empty, and
eventually reach a placement where either  or  touches a corner
of .
The same (and actually simpler) argument applies to the case when  and  touch the same edge\footnote{In general position this does not occur, but it can happen at discrete time instances during the motion,} of .

\begin{figure}[htbp]
\begin{center}
\input{ChewProof.pstex_t}\hspace{3cm}\begin{picture}(0,0)\includegraphics{ChewProof1.pstex}\end{picture}\setlength{\unitlength}{3947sp}\begingroup\makeatletter\ifx\SetFigFont\undefined \gdef\SetFigFont#1#2#3#4#5{\reset@font\fontsize{#1}{#2pt}\fontfamily{#3}\fontseries{#4}\fontshape{#5}\selectfont}\fi\endgroup \begin{picture}(1912,1782)(436,-1762)
\put(935,-931){\makebox(0,0)[lb]{\smash{{\SetFigFont{12}{14.4}{\rmdefault}{\mddefault}{\updefault}{\color[rgb]{0,0,0}}}}}}
\put(1036,-151){\makebox(0,0)[lb]{\smash{{\SetFigFont{12}{14.4}{\rmdefault}{\mddefault}{\updefault}{\color[rgb]{0,0,0}}}}}}
\put(1185,-1689){\makebox(0,0)[lb]{\smash{{\SetFigFont{12}{14.4}{\rmdefault}{\mddefault}{\updefault}{\color[rgb]{0,0,0}}}}}}
\put(451,-803){\makebox(0,0)[lb]{\smash{{\SetFigFont{12}{14.4}{\rmdefault}{\mddefault}{\updefault}{\color[rgb]{0,0,0}}}}}}
\put(1132,-632){\makebox(0,0)[lb]{\smash{{\SetFigFont{12}{14.4}{\rmdefault}{\mddefault}{\updefault}{\color[rgb]{0,0,0}}}}}}
\put(2333,-916){\makebox(0,0)[lb]{\smash{{\SetFigFont{12}{14.4}{\rmdefault}{\mddefault}{\updefault}{\color[rgb]{0,0,0}}}}}}
\put(510,-235){\makebox(0,0)[lb]{\smash{{\SetFigFont{12}{14.4}{\rmdefault}{\mddefault}{\updefault}{\color[rgb]{0,0,0}}}}}}
\put(1561,-952){\makebox(0,0)[lb]{\smash{{\SetFigFont{12}{14.4}{\rmdefault}{\mddefault}{\updefault}{\color[rgb]{0,0,0}}}}}}
\end{picture} \caption{\small \sf Left: The edge  in the diagram  before disappearing. The endpoint  (resp., ) of  corresponds to the homothetic copy of  whose edges  (resp., ) are incident to the respective vertices  (resp., ). Right: The tree of non-corner edges.}\label{Fig:ChewProof}
\end{center}
\end{figure}


Consider the situation just before the critical event takes place,
as depicted in Figure~\ref{Fig:ChewProof} (left).
The Voronoi edge  (to simplify the notation, we write this edge as , and similarly for the other edges and vertices in this analysis) is delimited by two Voronoi vertices,
one, , being the center of a copy of  which
touches  at the respective edges ,
and the other, , being the center of a copy of
 which touches  at the respective edges
. Consider the two other Voronoi edges  and
 adjacent to , and
the two Voronoi edges  and
 adjacent to .
Among them, consider only those which are non-corner edges;
assume for simplicity that they all are.
For specificity, consider the edge . As we move
the center of  along that edge away from ,
 loses the contact with ; it shrinks on the side of
 which contains  (and , already away from ),
and expands on the other side. Since this is a non-corner edge,
its other endpoint is a placement where the (artificial) edge
 of  between  and  touches another point
. Now, however, since  is adjacent to both edges
, , the new Voronoi edges  and
 are both corner edges.


Repeating this analysis to each of the other three Voronoi edges
adjacent to , we get a tree of non-corner Voronoi edges,
consisting of at most five edges, so that all the other Voronoi edges
adjacent to its edges are corner edges. As long as no discrete change occurs at any of the surrounding corner edges, the tree can undergo only  discrete changes, because all its edges are defined by a total of  points of . When a corner edge undergoes a discrete change, this can affect only  adjacent non-corner trees of the above kind. Hence, the number of changes in non-corner edges is proportional to the number
of changes in corner edges, which, by Lemma \ref{corners} (applied to ) is . Multiplying by the  choices of quadruples of edges of , we thus obtain:

\begin{theorem} \label{Thm:PolygonalVoronoi}
Let  be a set of  moving points in  under algebraic
motion of bounded degree, and let  be a convex -gon.
The number of topological changes in  with respect to 
 is , where  is a
constant that depends on the degree of motion of .
\end{theorem}

\paragraph{Kinetic maintenance of  and 
.}
As already mentioned, it is a fairly trivial task to maintain 
 and  kinetically, as the points of  
move. All we need to do is to assert the correctness of the present 
triangulation by a collection of local certificates, one for each edge 
of the diagram, where the certificate of an edge asserts that the two 
homothetic placements  of  that circumscribe the two 
respective adjacent -Delaunay triangles 
, are such that  does not 
contain  and  does not contain . The failure time of this certificate is the first time (if one exists) at which , and  become -cocircular---they all lie on the boundary of a common homothetic copy of . Such an event corresponds to a flip event in . If  is an edge of the periphery of , so that  exists but  does not, then  is a limiting wedge bounded by rays supporting two {\it consecutive} edges of (a copy of) , one passing through  and one through  (see Figure \ref{Fig:ConesCertif}).
The failure time of the corresponding certificate is the first time (if any) at which  also lies on the boundary of that wedge.


We maintain the breakpoints using ``sub-certificates", each of which asserts that , say, touches each of  at respective specific edges (and similarly for ). The failure time of this sub-certificate is the first failure time when one of  or  touches  at a vertex. In this case we have a corner event---two of the adjacent Voronoi edges terminate at a corner placement. Note that the failure time of each sub-certificate can be computed in  time. Moreover, for a fixed collection of valid sub-certificates, the failure time of an initial certificate (asserting non-cocircularity) can also be computed in  time (provided that it fails before the failures of the corresponding sub-certificates), because we know the four edges of  involved in the contacts.

\begin{figure}[htbp]
\begin{center}
\begin{picture}(0,0)\includegraphics{CertificateCones.pstex}\end{picture}\setlength{\unitlength}{2763sp}\begingroup\makeatletter\ifx\SetFigFont\undefined \gdef\SetFigFont#1#2#3#4#5{\reset@font\fontsize{#1}{#2pt}\fontfamily{#3}\fontseries{#4}\fontshape{#5}\selectfont}\fi\endgroup \begin{picture}(3820,2242)(1496,-3408)
\put(3174,-2817){\makebox(0,0)[lb]{\smash{{\SetFigFont{14}{16.8}{\rmdefault}{\mddefault}{\updefault}{\color[rgb]{0,0,0}}}}}}
\put(4268,-2922){\makebox(0,0)[lb]{\smash{{\SetFigFont{14}{16.8}{\rmdefault}{\mddefault}{\updefault}{\color[rgb]{0,0,0}}}}}}
\put(2250,-2408){\makebox(0,0)[lb]{\smash{{\SetFigFont{14}{16.8}{\rmdefault}{\mddefault}{\updefault}{\color[rgb]{0,0,0}}}}}}
\put(3318,-2159){\makebox(0,0)[lb]{\smash{{\SetFigFont{14}{16.8}{\rmdefault}{\mddefault}{\updefault}{\color[rgb]{0,0,0}}}}}}
\end{picture} \caption{\small \sf If  does not exist then  is a limiting wedge bounded by rays supporting two consecutive edges of (a copy of) .}
\label{Fig:ConesCertif}
\end{center}
\end{figure}

We therefore maintain an event queue that stores and updates all the 
active failure times (there are only  of them at any given time---the bound is 
independent of , because they correspond to actual  edges. When a sub-certificate fails we do not change 
, but only update the corresponding Voronoi edge, by 
adding or removing a segment and a breakpoint, and by replacing the 
sub-certificate by a new one; we also update the cocircularity certificate 
associated with the edge, because one of the contact edges has changed.
When a cocircularity certificate fails we update  and 
construct  new sub-certificates and certificates. Altogether, each update of the diagram takes  time. We thus have

\begin{theorem}\label{Thm:MaintainPolygDT}
Let  be a set of  moving points in  under algebraic
motion of bounded degree, and let  be a convex -gon.
 and  can be maintained using
 storage and  update time, so that
 events are processed, where  is  a
constant that depends on the degree of motion of .
\end{theorem}

\subsection{Stable Delaunay edges in }
We now restrict  to be a regular -gon.
Let  be
the vertices of  arranged in a clockwise direction, with  the leftmost.   
We call a homothetic copy of  whose vertex  touches a point , a
{\em -placement of  at }. Let
 be the direction of the vector that connects  with the 
center of , for each  (as in Section \ref{sec:Prelim}). See Figure \ref{Fig:Placement} (left).

We follow the machinery in the proof of Lemma~\ref{corners}. That is, for any pair  let  denote the distance from  to the point
; we put  if
 does not intersect . If
 then the point  is
the center of the -placement  of  at 
that also touches , and it is easy to see that there is a unique such point.
The value  is equal to the circumradius
of . See Figure \ref{Fig:Placement} (middle).

\begin{figure}[htbp]
\begin{center}
\input{Regular.pstex_t}\hspace{2cm}\input{PolygonalDist.pstex_t}\hspace{2cm}\begin{picture}(0,0)\includegraphics{UndefinedPolygDist.pstex}\end{picture}\setlength{\unitlength}{1973sp}\begingroup\makeatletter\ifx\SetFigFont\undefined \gdef\SetFigFont#1#2#3#4#5{\reset@font\fontsize{#1}{#2pt}\fontfamily{#3}\fontseries{#4}\fontshape{#5}\selectfont}\fi\endgroup \begin{picture}(4059,4059)(750,-4712)
\put(2401,-3211){\makebox(0,0)[lb]{\smash{{\SetFigFont{10}{12.0}{\rmdefault}{\mddefault}{\updefault}{\color[rgb]{0,0,0}}}}}}
\put(1501,-3961){\makebox(0,0)[lb]{\smash{{\SetFigFont{10}{12.0}{\rmdefault}{\mddefault}{\updefault}{\color[rgb]{0,0,0}}}}}}
\put(2251,-4411){\makebox(0,0)[lb]{\smash{{\SetFigFont{10}{12.0}{\rmdefault}{\mddefault}{\updefault}{\color[rgb]{0,0,0}}}}}}
\end{picture} \caption{\small \sf Left: \sf  is the direction of the vector connecting vertex  to the center of . Middle:
The function  is equal to the radius of
the circle that circumscribes the -placement of  at
 that also touches .
Right: The case when  while
  . In this case  must lie in one of
the shaded wedges.
}\label{Fig:Placement}
\end{center}
\end{figure}


The \textit{neighbor}  of  in direction  is defined to be the point  that minimizes . Clearly, for any
 we have  if and only if there is an empty
-placement  of  at  so that  touches one of
its edges. 




\smallskip
\noindent{\bf Remark:}
Note that, in the Euclidean case, we have  if
and only if the angle between  and  is at most
.  In contrast,  if and only if
the angle between  and  is at most
. Moreover, we have . Therefore,  always
implies , but not vice versa; see Figure
\ref{Fig:Placement} (right). Note also that in both the Euclidean and the polygonal cases, the respective quantities  and  may be undefined.


\begin{lemma}\label{Thm:LongEucPoly}
Let  be a pair of points such that  for 
 consecutive indices, say .
Then for each of these indices, except possibly for the first and the last one, we also have .
\end{lemma}

\begin{proof}
Let  (resp., ) be the point at which the ray  (resp., ) hits the edge  in . (By assumption, both points exist.)
Let  and  be the disks centered at  and , respectively, and touching  and . By definition, neither of these disks contains a point of  in its interior. The angle between the tangents to  and  at  or at  (these angles are equal) is ; see Figure \ref{Fig:ProvePolyg} (left).

\begin{figure}[htbp]
\begin{center}
\input{ProvePolyg1.pstex_t}\hspace{2cm}\begin{picture}(0,0)\includegraphics{ProvePolyg2.pstex}\end{picture}\setlength{\unitlength}{5131sp}\begingroup\makeatletter\ifx\SetFigFont\undefined \gdef\SetFigFont#1#2#3#4#5{\reset@font\fontsize{#1}{#2pt}\fontfamily{#3}\fontseries{#4}\fontshape{#5}\selectfont}\fi\endgroup \begin{picture}(1785,1918)(533,-1682)
\put(1706,-1271){\makebox(0,0)[lb]{\smash{{\SetFigFont{12}{14.4}{\rmdefault}{\mddefault}{\updefault}{\color[rgb]{0,0,0}}}}}}
\put(548,-804){\makebox(0,0)[lb]{\smash{{\SetFigFont{12}{14.4}{\rmdefault}{\mddefault}{\updefault}{\color[rgb]{0,0,0}}}}}}
\put(1515,-1622){\makebox(0,0)[lb]{\smash{{\SetFigFont{12}{14.4}{\rmdefault}{\mddefault}{\updefault}{\color[rgb]{0,0,0}}}}}}
\put(1390,-441){\makebox(0,0)[lb]{\smash{{\SetFigFont{12}{14.4}{\rmdefault}{\mddefault}{\updefault}{\color[rgb]{0,0,0}}}}}}
\put(1081,-607){\makebox(0,0)[lb]{\smash{{\SetFigFont{12}{14.4}{\rmdefault}{\mddefault}{\updefault}{\color[rgb]{0,0,0}}}}}}
\put(1231,-499){\makebox(0,0)[lb]{\smash{{\SetFigFont{12}{14.4}{\rmdefault}{\mddefault}{\updefault}{\color[rgb]{0,0,0}}}}}}
\put(1501, 89){\makebox(0,0)[lb]{\smash{{\SetFigFont{12}{14.4}{\rmdefault}{\mddefault}{\updefault}{\color[rgb]{0,0,0}}}}}}
\put(1906,-338){\makebox(0,0)[lb]{\smash{{\SetFigFont{12}{14.4}{\rmdefault}{\mddefault}{\updefault}{\color[rgb]{1,0,0}}}}}}
\put(2303,-893){\makebox(0,0)[lb]{\smash{{\SetFigFont{12}{14.4}{\rmdefault}{\mddefault}{\updefault}{\color[rgb]{0,0,0}}}}}}
\end{picture} \caption{\small \sf Left: The angle between the tangents to  and  at  (or at ) is equal to
  . Right: The line  crosses  in a chord  which is fully
  contained in .}\label{Fig:ProvePolyg}
\end{center}
\end{figure}

Fix an arbitrary index , so  intersects  
and forms an angle of at least  with each of .
Let  be the -placement of 
at  that touches . To see that such a placement exists, we note that, by the preceding remark, it suffices to show that the angle between
 and  is at most ; that is, to rule out the case where  lies in one of the shaded wedges in Figure \ref{Fig:Placement} (right). This case is indeed impossible, because then one of  would form an angle greater than  with 
, contradicting the assumption that both of these rays intersect the (Euclidean) .

We claim that . 
Establishing this property for every  will complete 
the proof of the lemma. 
Let  be the edge of  passing through . See Figure \ref{Fig:ProvePolyg} (right). Let  be the disk 
whose center lies on  and which passes through  and , 
and let  be the circumscribing disk of . 
Since  and is interior to , and since  and 
 are centered on the same ray  and pass through , it 
follows that . 
The line  containing  crosses  in a chord  that
is fully contained in . The angle between the tangent to  at 
 and the chord  is equal to the angle at which  sees . 
This is smaller than the angle at which  sees , which in turn 
is equal to .

Arguing as in the analysis of  and , the tangent to  at  forms an angle of at least  with each of the tangents to  at , and hence  forms an angle of at least  with each of these tangents; see Figure \ref{Fig:ProvePolyg2} (left).
The line  marks two chords  within the respective disks . We claim that  is fully contained in their union . Indeed, the angle  is equal to the angle between  and the tangent to  at , so .
On the other hand, the angle at which  sees  is , which is smaller. This, and the symmetic argument involving , are easily seen to imply the claim.

\begin{figure}[htbp]
\begin{center}
\input{ProvePolyg3.pstex_t} \hspace{2cm} \begin{picture}(0,0)\includegraphics{ProvePolyg4.pstex}\end{picture}\setlength{\unitlength}{3947sp}\begingroup\makeatletter\ifx\SetFigFont\undefined \gdef\SetFigFont#1#2#3#4#5{\reset@font\fontsize{#1}{#2pt}\fontfamily{#3}\fontseries{#4}\fontshape{#5}\selectfont}\fi\endgroup \begin{picture}(3090,1857)(3421,-1686)
\put(4475,-43){\makebox(0,0)[lb]{\smash{{\SetFigFont{12}{14.4}{\rmdefault}{\mddefault}{\updefault}{\color[rgb]{1,0,0}}}}}}
\put(5528,-30){\makebox(0,0)[lb]{\smash{{\SetFigFont{12}{14.4}{\rmdefault}{\mddefault}{\updefault}{\color[rgb]{1,0,0}}}}}}
\put(5925,-1568){\makebox(0,0)[lb]{\smash{{\SetFigFont{12}{14.4}{\rmdefault}{\mddefault}{\updefault}{\color[rgb]{0,0,0}}}}}}
\put(4740,-1613){\makebox(0,0)[lb]{\smash{{\SetFigFont{12}{14.4}{\rmdefault}{\mddefault}{\updefault}{\color[rgb]{0,0,0}}}}}}
\put(5971,-15){\makebox(0,0)[lb]{\smash{{\SetFigFont{12}{14.4}{\rmdefault}{\mddefault}{\updefault}{\color[rgb]{0,0,0}}}}}}
\put(3765,  0){\makebox(0,0)[lb]{\smash{{\SetFigFont{12}{14.4}{\rmdefault}{\mddefault}{\updefault}{\color[rgb]{0,0,0}}}}}}
\put(6496,-143){\makebox(0,0)[lb]{\smash{{\SetFigFont{12}{14.4}{\rmdefault}{\mddefault}{\updefault}{\color[rgb]{0,0,0}}}}}}
\put(3436,-1358){\makebox(0,0)[lb]{\smash{{\SetFigFont{12}{14.4}{\rmdefault}{\mddefault}{\updefault}{\color[rgb]{0,0,0}}}}}}
\put(5076,-222){\makebox(0,0)[lb]{\smash{{\SetFigFont{12}{14.4}{\rmdefault}{\mddefault}{\updefault}{\color[rgb]{1,0,0}}}}}}
\put(3955,-780){\makebox(0,0)[lb]{\smash{{\SetFigFont{12}{14.4}{\rmdefault}{\mddefault}{\updefault}{\color[rgb]{1,0,0}}}}}}
\put(5436,-571){\makebox(0,0)[lb]{\smash{{\SetFigFont{12}{14.4}{\rmdefault}{\mddefault}{\updefault}{\color[rgb]{0,0,0}}}}}}
\end{picture} \caption{\small \sf Left: The line  forms an angle of at least  with each of the tangents to  at . Right: The edge  of  is fully contained in .}\label{Fig:ProvePolyg2}
\end{center}
\end{figure}

Now consider the circumscribing disk  of . Denote the endpoints of  as  and , where  lies in  and  lies in .
Since the ray  hits  before hitting , and the ray  hits these circles in the reverse order, it follows that the second intersection of  and  (other than ) must lie on a ray from  which lies between the rays  and thus crosses . See Figure \ref{Fig:ProvePolyg2} (right).
Symmetrically, the second intersection point of  and  also lies on a ray which crosses .

It follows that the arc of  delimited by these intersections and containing  is fully contained in .
Hence all the vertices of  (which lie on this arc) lie in . This, combined with the argument in the preceding paragraphs, is easily seen to imply that , so its interior does not contain points of , which in turn implies that . 
As noted, this completes the proof of the lemma.
\end{proof}

Since -Voronoi edges are connected, Lemma~\ref{Thm:LongEucPoly} implies that  is ``long", in the sense that it contains at least  breakpoints that represent corner placements at , interleaved (as promised in Section \ref{Sec:PolygonalBackground}) with at least  corner placements at .
This property is easily seen to hold also under the weaker assumptions that: (i) for the first and the last indices , the point  either is equal to  or is undefined, and (ii) for the rest of the indices  we have  and  (i.e., the -placement of  at  that touches  exists).
In this relaxed setting, it is now possible that any of the two points  lies at infinity, in which case the corresponding disk  or  degenerates into a halfplane. This stronger version of 
Lemma~\ref{Thm:LongEucPoly} is used in the proof of the converse 
Lemma~\ref{Thm:LongPolygEuc}, asserting that every edge  in  with sufficiently many breakpoints has a stable counterpart  in .

\begin{lemma}\label{Thm:LongPolygEuc}
Let  be a pair of points such that  for at least three consecutive indices .
Then for each of these indices, except possibly for the first and the last one, we have .
\end{lemma}
\begin{proof} 
Again, assume with no
  loss of generality that  for , with
  .  Suppose to the contrary that, for some , we have .  Since 
  by assumption, we have , so there exists  for which
  .  Assume with no loss of generality
  that  lies to the left of the line from  to . In this case . Indeed, we have (i)  by assumption,
  so , and (ii) . Moreover,
  because  lies to the left of the line from  to , the orientation of  lies counterclockwise to that of ,
  implying that .
  See Figure \ref{Fig:Converse}. Since  hits  before hitting , any ray emanating from  counterlockwise to  must do the same, so we have , as claimed. Similarly, we get that either
   or
   (where the latter
  can occur only for ).  Now applying (the extended version of)
  Lemma~\ref{Thm:LongEucPoly} to the point set  and to
  the index set , we get that
  . But this
  contradicts the fact that .
\end{proof}

\begin{figure}[htbp]
\begin{center}
\begin{picture}(0,0)\includegraphics{Converse.pstex}\end{picture}\setlength{\unitlength}{2565sp}\begingroup\makeatletter\ifx\SetFigFont\undefined \gdef\SetFigFont#1#2#3#4#5{\reset@font\fontsize{#1}{#2pt}\fontfamily{#3}\fontseries{#4}\fontshape{#5}\selectfont}\fi\endgroup \begin{picture}(2867,2675)(1545,-2538)
\put(3133,-2309){\makebox(0,0)[rb]{\smash{{\SetFigFont{12}{14.4}{\rmdefault}{\mddefault}{\updefault}{\color[rgb]{0,0,0}}}}}}
\put(1694,-918){\makebox(0,0)[lb]{\smash{{\SetFigFont{12}{14.4}{\rmdefault}{\mddefault}{\updefault}{\color[rgb]{0,0,0}}}}}}
\put(2478,-94){\makebox(0,0)[lb]{\smash{{\SetFigFont{12}{14.4}{\rmdefault}{\mddefault}{\updefault}{\color[rgb]{0,0,0}}}}}}
\put(3357,-2442){\makebox(0,0)[lb]{\smash{{\SetFigFont{12}{14.4}{\rmdefault}{\mddefault}{\updefault}{\color[rgb]{0,0,0}}}}}}
\put(3329,-1913){\makebox(0,0)[lb]{\smash{{\SetFigFont{12}{14.4}{\rmdefault}{\mddefault}{\updefault}{\color[rgb]{0,0,0}}}}}}
\put(1560,-2086){\makebox(0,0)[rb]{\smash{{\SetFigFont{12}{14.4}{\rmdefault}{\mddefault}{\updefault}{\color[rgb]{0,0,0}}}}}}
\put(1881,-1504){\makebox(0,0)[rb]{\smash{{\SetFigFont{12}{14.4}{\rmdefault}{\mddefault}{\updefault}{\color[rgb]{0,0,0}}}}}}
\end{picture} \caption{\small \sf Proof of Lemma~\ref{Thm:LongPolygEuc}. If  because some , lying to the left of the line from  to , satisfies . Since , we have .}
\label{Fig:Converse}
\end{center}
\end{figure}

\paragraph{Maintaining an SDG using .}
Lemmas~\ref{Thm:LongEucPoly} 
and~\ref{Thm:LongPolygEuc} together imply that an  can be maintained
using the fairly straightforward kinetic algorithm for maintaining 
the whole , provided by Theorem~\ref{Thm:MaintainPolygDT}. 
We use 
 to maintain the graph  on , whose edges are all 
the pairs  such that  and  define an edge 
 in  that contains at least seven 
breakpoints. As shown in Theorem \ref{Thm:MaintainPolygDT}, this can 
be done with  storage,  update time, and 
 updates (for an appropriate ).  We claim that  is a 
- in the Euclidean norm. 

Indeed, if two points  define a -long edge  in  then
this edge stabs at least six rays  emanating from , and at least six rays  emanating from .
Thus, according to Lemma~\ref{Thm:LongEucPoly},  contains the edge  with at least four breakpoints corresponding to corner placements of  at  that touch , and at least four breakpoints corresponding to corner placements of  at  that touch . Therefore,  contains at least  breakpoints, so .

For the second part, if  define an edge  in  with at least  breakpoints then, by the interleaving property of breakpoints, we may assume, without loss of generality, that at least four of these breakpoints correspond to -empty corner placements of  at  that touch .
Thus, Lemma~\ref{Thm:LongPolygEuc} implies that  contains the edge , and that this edge is hit by at least two consecutive rays .
But then, as observed in Lemma \ref{lem:alpha}, the edge  is -long in .
We thus obtain the main result of this section.
 
\begin{theorem}\label{Thm:MaintainSDGPolyg}
Let  be a set of  moving points in  under algebraic
motion of bounded degree, 
and let  be a parameter.  A
-stable Delaunay graph of  can be maintained by a 
KDS of linear size that processes
  events, where  is a
constant that depends on the degree of motion of , and that 
updates the SDG  at each event in  time.
\end{theorem}


\section{An Improved Data Structure}
\label{Sec:ReduceS}

The data structure of Theorem~\ref{Thm:MaintainSDGPolyg} requires
 storage but the best bound we have on the number of
events it may encounter is , which is 
much larger than the number of events encountered by the data structure
of Theorem~\ref{thm:ddj} (which, in terms of the dependence on , is only ).  In this
section we present an alternative data structure that requires
 space and  overall processing
time.  The structure processes each event in  time and is also {\it local}, in the sense that each point is stored at only  places in the structure. 


\paragraph{Notation.}
We use the directions  and the associated quantities  and   defined in
Section \ref{sec:Prelim}. We assume that , the number of canonical directions, is even, and write, as in Section \ref{sec:Prelim}, .
We denote by  the cone (or wedge) with apex at the origin that is bounded by
 and . Note that  and  are antipodal.
As before, for a vector , we denote by  the ray emanating from  in direction . Similarly, for a cone  we denote
by  the translation of  that places its apex at .
Let  be an angle. For a direction 
and for two points , we say that the edge
 is \emph{-long around  the ray } 
if  is the Voronoi neighbor of  in all directions in the range 
, i.e., for all , the ray 
 intersects .
The  {\em -cone around  } is the cone whose apex is 
and each of its bounding rays makes an angle of  with .

\begin{figure}[htbp]
\begin{center}
\input{Jextremal1.pstex_t}\hspace{2cm}\begin{picture}(0,0)\includegraphics{StronglyJextremal1.pstex}\end{picture}\setlength{\unitlength}{3158sp}\begingroup\makeatletter\ifx\SetFigFont\undefined \gdef\SetFigFont#1#2#3#4#5{\reset@font\fontsize{#1}{#2pt}\fontfamily{#3}\fontseries{#4}\fontshape{#5}\selectfont}\fi\endgroup \begin{picture}(3843,2635)(520,-2768)
\put(3351,-340){\makebox(0,0)[lb]{\smash{{\SetFigFont{12}{14.4}{\rmdefault}{\mddefault}{\updefault}{\color[rgb]{0,0,0}}}}}}
\put(4298,-1422){\makebox(0,0)[lb]{\smash{{\SetFigFont{12}{14.4}{\rmdefault}{\mddefault}{\updefault}{\color[rgb]{0,0,0}}}}}}
\put(3857,-2681){\makebox(0,0)[lb]{\smash{{\SetFigFont{12}{14.4}{\rmdefault}{\mddefault}{\updefault}{\color[rgb]{0,0,0}}}}}}
\put(1617,-2205){\makebox(0,0)[lb]{\smash{{\SetFigFont{12}{14.4}{\rmdefault}{\mddefault}{\updefault}{\color[rgb]{0,0,0}}}}}}
\put(543,-1872){\makebox(0,0)[lb]{\smash{{\SetFigFont{12}{14.4}{\rmdefault}{\mddefault}{\updefault}{\color[rgb]{0,0,0}}}}}}
\put(3179,-1651){\makebox(0,0)[lb]{\smash{{\SetFigFont{12}{14.4}{\rmdefault}{\mddefault}{\updefault}{\color[rgb]{0,0,0}}}}}}
\end{picture} \caption{\small \sf Left:  is -extremal for . Right:  is \textit{strongly} -extremal for .}
\label{Fig:Jextremal}
\end{center}
\end{figure}

\paragraph{Definition (-extremal points).}
(i) Let , let  be the index such that ,
and let  be a direction
 such that
   for all .
We say that  is \emph{-extremal} for  if 
  .
That is,  is the nearest point to  in this cone, in the -direction.
Clearly, a point  has at most  -extremal points,
one for every admissible cone , for any fixed . See Figure \ref{Fig:Jextremal} (left).

(ii) For , let  denote the extended cone that
is the union of the seven consecutive cones . 
Let , let  be the index such that ,
and let  be a direction
such that 
  for all  (such 's exist if  is smaller than some appropriate constant).
We say that the point  is \textit{strongly -extremal} for 
if . 

(iii) We say that a pair  is {\it (strongly)} -{\it extremal}, for some , if  is (strongly) -extremal for
 and  is (strongly) -extremal for .

\begin{figure}[htbp]
\begin{center}
\begin{picture}(0,0)\includegraphics{necessaryCond.pstex}\end{picture}\setlength{\unitlength}{1776sp}\begingroup\makeatletter\ifx\SetFigFont\undefined \gdef\SetFigFont#1#2#3#4#5{\reset@font\fontsize{#1}{#2pt}\fontfamily{#3}\fontseries{#4}\fontshape{#5}\selectfont}\fi\endgroup \begin{picture}(5748,5770)(3868,-7214)
\put(5251,-3189){\makebox(0,0)[rb]{\smash{{\SetFigFont{10}{12.0}{\rmdefault}{\mddefault}{\updefault}{\color[rgb]{0,0,0}}}}}}
\put(9601,-5567){\makebox(0,0)[rb]{\smash{{\SetFigFont{10}{12.0}{\rmdefault}{\mddefault}{\updefault}{\color[rgb]{0,0,0}}}}}}
\put(9586,-3094){\makebox(0,0)[rb]{\smash{{\SetFigFont{10}{12.0}{\rmdefault}{\mddefault}{\updefault}{\color[rgb]{0,0,0}}}}}}
\put(5037,-3939){\makebox(0,0)[lb]{\smash{{\SetFigFont{10}{12.0}{\rmdefault}{\mddefault}{\updefault}{\color[rgb]{0,0,0}}}}}}
\put(7176,-4046){\makebox(0,0)[lb]{\smash{{\SetFigFont{10}{12.0}{\rmdefault}{\mddefault}{\updefault}{\color[rgb]{0,0,0}}}}}}
\put(5140,-6821){\makebox(0,0)[lb]{\smash{{\SetFigFont{10}{12.0}{\rmdefault}{\mddefault}{\updefault}{\color[rgb]{0,0,0}}}}}}
\put(7548,-1880){\makebox(0,0)[lb]{\smash{{\SetFigFont{10}{12.0}{\rmdefault}{\mddefault}{\updefault}{\color[rgb]{0,0,0}}}}}}
\put(4762,-4621){\makebox(0,0)[rb]{\smash{{\SetFigFont{10}{12.0}{\rmdefault}{\mddefault}{\updefault}{\color[rgb]{0,0,0}}}}}}
\put(8684,-5409){\makebox(0,0)[b]{\smash{{\SetFigFont{10}{12.0}{\rmdefault}{\mddefault}{\updefault}{\color[rgb]{0,0,0}}}}}}
\put(6840,-3634){\makebox(0,0)[lb]{\smash{{\SetFigFont{10}{12.0}{\rmdefault}{\mddefault}{\updefault}{\color[rgb]{0,0,0}}}}}}
\put(6555,-2164){\makebox(0,0)[lb]{\smash{{\SetFigFont{10}{12.0}{\rmdefault}{\mddefault}{\updefault}{\color[rgb]{0,0,0}}}}}}
\put(6030,-4429){\makebox(0,0)[lb]{\smash{{\SetFigFont{10}{12.0}{\rmdefault}{\mddefault}{\updefault}{\color[rgb]{0,0,0}}}}}}
\put(6075,-4819){\makebox(0,0)[lb]{\smash{{\SetFigFont{10}{12.0}{\rmdefault}{\mddefault}{\updefault}{\color[rgb]{0,0,0}}}}}}
\put(8737,-6232){\makebox(0,0)[lb]{\smash{{\SetFigFont{10}{12.0}{\rmdefault}{\mddefault}{\updefault}{\color[rgb]{0,0,0}}}}}}
\put(9299,-4371){\makebox(0,0)[rb]{\smash{{\SetFigFont{10}{12.0}{\rmdefault}{\mddefault}{\updefault}{\color[rgb]{0,0,0}}}}}}
\put(3883,-5529){\makebox(0,0)[lb]{\smash{{\SetFigFont{10}{12.0}{\rmdefault}{\mddefault}{\updefault}{\color[rgb]{0,0,0}}}}}}
\put(4986,-1738){\makebox(0,0)[lb]{\smash{{\SetFigFont{10}{12.0}{\rmdefault}{\mddefault}{\updefault}{\color[rgb]{0,0,0}}}}}}
\end{picture} \caption{\small \sf Illustration of the setup in Lemma~\ref{Lemma:EmptyCone}: the edge  is -long around , and the ``tip"  of the cone  is empty.}
\label{Fig:EmptyCone}
\end{center}
\end{figure}

\begin{lemma}\label{Lemma:EmptyCone}
Let , and let  be a direction such that the
edge  appears in  and is -long around the ray 
. Let  be the -cone around the ray from  through 
.  Then  for all
.
\end{lemma}
\begin{proof}
Refer to Figure \ref{Fig:EmptyCone}. Without loss of generality, we assume that  is the -direction
and that  lies above and to the right of . (In this case the slope of the bisector  is negative. Note that  has to
lie to the right of , for otherwise  would not cross .) 
Let 
(resp., ) be the direction that makes a counterclockwise (resp.,
clockwise) angle of  with . Let  (resp., ) be
the intersection  of  with  (resp., with ); by assumption, both points exist.
Let  be the vertical line passing through . Let 
(resp., ) be the intersection point of  with the ray
emanating from  (resp., ) in the direction opposite to
 (resp., ); see Figure~\ref{Fig:EmptyCone}.

Note that , and  that
, i.e.,  is the circumcenter
of .   Therefore . That is,  is the intersection of the upper ray of  with .
Similarly,  is the intersection of the lower ray of  with .
Moreover, if there exists a
 point  properly inside the triangle 
then
,
contradicting the fact that  is on . So the interior of  (including the relative interiors of edges ) is disjoint from .
 Similarly, by a symmetric argument, no points of  lie inside 
 or on the relative interiors of its edges .
Hence, the portion of  to the right of  is openly disjoint from , and therefore  is a rightmost point of  (extreme in the  direction) inside .\end{proof}

\begin{corollary}\label{Corol:ExtremalPair}
Let . 
\noindent (i) If the edge  is -long in 
then there are  for which 
is a -extremal pair.
\noindent(ii) If the edge  is -long in 
then there are  for which 
is a strongly -extremal pair.
\end{corollary}
\begin{proof}
To prove part (i), choose , such that  is
-long around each of   and  . By Lemma \ref{Lemma:EmptyCone},  is -extremal in the -cone  around the ray from  through . Let  be the index such that . Since the opening angle of  is , it follows that , so  is -extremal with respect to , and, symmetrically,  is -extremal with respect to . To prove part (ii) choose , such that  is -long around each of 
and  and apply
Lemma~\ref{Lemma:EmptyCone} as in the proof of part (i).
\end{proof}

\paragraph{The stable Delaunay graph.}
We kinetically maintain a
-stable Delaunay graph, whose precise definition is given below,
using a data-structure which is based on a collection of 2-dimensional orthogonal range trees similar to the ones used in \cite{KineticNeighbors}.

Fix , and choose a ``sheared" coordinate frame in
 which the rays  and  form the - and -axes,
 respectively. That is, in this coordinate frame,  if and
 only if  lies in the upper-right quadrant anchored at . 
 
We define a 2-dimensional range tree  consisting of a \textit{primary} balanced binary search tree with the
points of  stored at its leaves ordered by their -coordinates, and of secondary trees, introduced below.
Each internal node  of the primary tree of  is associated with the
\emph{canonical subset}  of all points that are stored at the
leaves of the subtree rooted at . A point  is said to be
\emph{red} (resp., {\it blue) in } if it is stored at the subtree rooted at
the left (resp., right) child of  in . For each primary node
 we maintain a secondary balanced binary search tree , whose
leaves store the points of  ordered by their -coordinates.
We refer to a node  in a secondary
tree    as a {\em secondary node  of }.

Each node  of a secondary tree  is associated with a canonical
subset  of points stored at the leaves of the
subtree of   rooted at . We also associate with  the sets
 and  of points  residing in the
\emph{left} (resp., \emph{right}) \emph{subtree} of  and are red
(resp., blue) in .  It is easy to verify that the
 sum of the sizes of the sets  and 
over all secondary nodes of 
 is .


For each secondary
node  and each  we maintain the points 
 provided that both  are not empty. 
See Figure \ref{Fig:JLextremalNode}. It is straightforward to show that if  is a -extremal pair, so that ,
then there
is a secondary node  for which 
 and .
 
\begin{figure}[htbp]
\begin{center}
\begin{picture}(0,0)\includegraphics{JLextremal.pstex}\end{picture}\setlength{\unitlength}{2368sp}\begingroup\makeatletter\ifx\SetFigFont\undefined \gdef\SetFigFont#1#2#3#4#5{\reset@font\fontsize{#1}{#2pt}\fontfamily{#3}\fontseries{#4}\fontshape{#5}\selectfont}\fi\endgroup \begin{picture}(4061,3307)(726,-3138)
\put(3599,-28){\makebox(0,0)[lb]{\smash{{\SetFigFont{10}{12.0}{\rmdefault}{\mddefault}{\updefault}{\color[rgb]{0,0,0}}}}}}
\put(791,-2844){\makebox(0,0)[lb]{\smash{{\SetFigFont{10}{12.0}{\rmdefault}{\mddefault}{\updefault}{\color[rgb]{0,0,0}}}}}}
\put(3390,-380){\makebox(0,0)[lb]{\smash{{\SetFigFont{10}{12.0}{\rmdefault}{\mddefault}{\updefault}{\color[rgb]{0,0,0}}}}}}
\put(1350,-2500){\makebox(0,0)[lb]{\smash{{\SetFigFont{10}{12.0}{\rmdefault}{\mddefault}{\updefault}{\color[rgb]{0,0,0}}}}}}
\put(873,-2150){\makebox(0,0)[lb]{\smash{{\SetFigFont{10}{12.0}{\rmdefault}{\mddefault}{\updefault}{\color[rgb]{0,0,0}}}}}}
\put(2746,-733){\makebox(0,0)[lb]{\smash{{\SetFigFont{10}{12.0}{\rmdefault}{\mddefault}{\updefault}{\color[rgb]{0,0,0}}}}}}
\put(4035,-743){\makebox(0,0)[lb]{\smash{{\SetFigFont{10}{12.0}{\rmdefault}{\mddefault}{\updefault}{\color[rgb]{0,0,0}}}}}}
\put(2611,-17){\makebox(0,0)[lb]{\smash{{\SetFigFont{10}{12.0}{\rmdefault}{\mddefault}{\updefault}{\color[rgb]{0,0,0}}}}}}
\put(2705,-2045){\makebox(0,0)[lb]{\smash{{\SetFigFont{10}{12.0}{\rmdefault}{\mddefault}{\updefault}{\color[rgb]{0,0,0}}}}}}
\put(3121,-1716){\makebox(0,0)[lb]{\smash{{\SetFigFont{10}{12.0}{\rmdefault}{\mddefault}{\updefault}{\color[rgb]{0,0,0}}}}}}
\end{picture} \caption{\small \sf The points  for a secondary node  of .}
\label{Fig:JLextremalNode}
\end{center}
\end{figure}

For each  we construct a set  containing all points  for which  is a -extremal pair, for some pair of indices . Specifically, for
each , and each secondary node  such that
 for some , we include in
 all the points  such that  for some . Similarly,
for each , and each secondary node
 such that  for some  we include in  all the points  such that
 for some .
It is easy to verify that, for each -extremal pair , for some ,
 is placed in  by the preceding process. The converse, however, does not always hold, so in general  is a superset of
the pairs that we want. 

For each , each
point  belongs to  sets  and , so the size of
 is bounded by . Indeed,  may be coupled with up to  neighbors at each of the  nodes containing it.

For each point  and  we 
maintain all points in  in a kinetic and dynamic
tournament  whose winner  minimizes
the directional distance , as given in (\ref{Eq:DirectDist}). That is, 
the winner in  is  in the Voronoi diagram of
. 

We are now ready to define the stable Delaunay graph  that we maintain.
For each pair of points  we add the edge  to  if 
the following hold.
\begin{itemize}
\item[(G1)] There is an index  such that  wins the 8
consecutive tournaments .
\item[(G2)] The point  is strongly -extremal and strongly
-extremal for .
\end{itemize}

The -stability of  is implied by a combination of Theorems \ref{Thm:Completeness} and \ref{Thm:Soundness}.
\begin{theorem}\label{Thm:Completeness}
For every -long edge , the graph 
contains the edge .
\end{theorem}
\begin{proof}
By Corollary~\ref{Corol:ExtremalPair} (i), 
there are  and  such that  is a -extremal pair.
By the preceding discussion this implies that  is in .
 Now since  is  -long 
  there is an  such that 
 in , and therefore also
in the Voronoi diagram of .
So it follows that  indeed wins the tournaments
 .

By the proof of Corollary~\ref{Corol:ExtremalPair} (ii),  is
strongly  -extremal and
strongly -extremal for .
\end{proof}

\begin{theorem}\label{Thm:Soundness}
For every edge , the edge  belongs to  and is -long there.
\end{theorem}
\begin{proof}


Since  we know that  is in  and wins the tournaments
, for some  and
that the point  is strongly -extremal and -extremal for . 
We prove that the rays  and  stab ,
from which the theorem follows.

Assume then that
 one of the rays  does not stab
 ; suppose it is the ray . (This includes the case when  is not present at
 all in .)  By definition, this means that
 . We use Lemma~\ref{Lemma:qExtremal}, given shortly below, to show that
  cannot win in at least one of the tournaments among
  and thereby get a
 contradiction.


\begin{figure}[htbp]
\begin{center}
\begin{picture}(0,0)\includegraphics{ProvingStable2.pstex}\end{picture}\setlength{\unitlength}{1776sp}\begingroup\makeatletter\ifx\SetFigFont\undefined \gdef\SetFigFont#1#2#3#4#5{\reset@font\fontsize{#1}{#2pt}\fontfamily{#3}\fontseries{#4}\fontshape{#5}\selectfont}\fi\endgroup \begin{picture}(6372,6248)(3991,-7268)
\put(7111,-4726){\makebox(0,0)[rb]{\smash{{\SetFigFont{11}{13.2}{\rmdefault}{\mddefault}{\updefault}{\color[rgb]{0,0,0}}}}}}
\put(4471,-4531){\makebox(0,0)[lb]{\smash{{\SetFigFont{11}{13.2}{\rmdefault}{\mddefault}{\updefault}{\color[rgb]{0,0,0}}}}}}
\put(4006,-3256){\makebox(0,0)[lb]{\smash{{\SetFigFont{11}{13.2}{\rmdefault}{\mddefault}{\updefault}{\color[rgb]{0,0,0}}}}}}
\put(5161,-2476){\makebox(0,0)[lb]{\smash{{\SetFigFont{11}{13.2}{\rmdefault}{\mddefault}{\updefault}{\color[rgb]{0,0,0}}}}}}
\put(5356,-1756){\makebox(0,0)[lb]{\smash{{\SetFigFont{11}{13.2}{\rmdefault}{\mddefault}{\updefault}{\color[rgb]{0,0,0}}}}}}
\put(7141,-3105){\makebox(0,0)[lb]{\smash{{\SetFigFont{11}{13.2}{\rmdefault}{\mddefault}{\updefault}{\color[rgb]{0,0,0}}}}}}
\put(9376,-1486){\makebox(0,0)[lb]{\smash{{\SetFigFont{11}{13.2}{\rmdefault}{\mddefault}{\updefault}{\color[rgb]{0,0,0}}}}}}
\put(8866,-6303){\makebox(0,0)[lb]{\smash{{\SetFigFont{11}{13.2}{\rmdefault}{\mddefault}{\updefault}{\color[rgb]{0,0,0}}}}}}
\put(7658,-3911){\makebox(0,0)[lb]{\smash{{\SetFigFont{11}{13.2}{\rmdefault}{\mddefault}{\updefault}{\color[rgb]{0,0,1}}}}}}
\put(10201,-1966){\makebox(0,0)[lb]{\smash{{\SetFigFont{11}{13.2}{\rmdefault}{\mddefault}{\updefault}{\color[rgb]{0,0,1}}}}}}
\put(9088,-2323){\makebox(0,0)[lb]{\smash{{\SetFigFont{11}{13.2}{\rmdefault}{\mddefault}{\updefault}{\color[rgb]{1,0,0}}}}}}
\put(6601,-2311){\makebox(0,0)[lb]{\smash{{\SetFigFont{11}{13.2}{\rmdefault}{\mddefault}{\updefault}{\color[rgb]{0,0,0}}}}}}
\put(7921,-4951){\makebox(0,0)[lb]{\smash{{\SetFigFont{11}{13.2}{\rmdefault}{\mddefault}{\updefault}{\color[rgb]{1,0,0}}}}}}
\put(10022,-4319){\makebox(0,0)[lb]{\smash{{\SetFigFont{11}{13.2}{\rmdefault}{\mddefault}{\updefault}{\color[rgb]{0,0,0}}}}}}
\put(8896,-7141){\makebox(0,0)[lb]{\smash{{\SetFigFont{11}{13.2}{\rmdefault}{\mddefault}{\updefault}{\color[rgb]{1,0,0}}}}}}
\put(7816,-6211){\rotatebox{333.0}{\makebox(0,0)[rb]{\smash{{\SetFigFont{11}{13.2}{\rmdefault}{\mddefault}{\updefault}{\color[rgb]{0,0,1}}}}}}}
\put(7576,-4561){\makebox(0,0)[lb]{\smash{{\SetFigFont{11}{13.2}{\rmdefault}{\mddefault}{\updefault}{\color[rgb]{0,0,0}}}}}}
\put(8158,-1989){\makebox(0,0)[lb]{\smash{{\SetFigFont{11}{13.2}{\rmdefault}{\mddefault}{\updefault}{\color[rgb]{0,0,1}}}}}}
\end{picture} \end{center}
\caption{\sf \small Proof of Theorem~\ref{Thm:Soundness}: the case when 
is to the right of the line from  to . The line 
orthogonal to  through 
intersects the circle  at a point  outside
, which implies that   
is to the right of the line from  to .
 Assuming , the point  is inside the cone bounded by  and
. Hence,  hits  before
.} \label{Fig:Clockwise}
\end{figure}


According to Lemma~\ref{Lemma:qExtremal},
there exists a point  such that 
and  
 is -extremal for . 
Let   and let
 be the circle which is centered at , and passes through  and ; see Figure \ref{Fig:Clockwise}.




We consider the case where  is to the right of the  line
from  to ; 
the other case is treated symmetrically.  In this case the
intersection of  and  is to the left of
the directed line from  to . Let  be the index for
which .  If  then there is a secondary
node  in the tree  for which  and , and
since  is -extremal for ,  is equal to
. If  then, symmetrically, we have a node 
such that  and  and  is equal to
. We assume that  in the sequel; the other case is treated in a fully symmetric manner.

Let  be the ray from  through , for an appropriate direction , and let  be the direction which lies counterclockwise to  and forms with it an angle of at least  and at most .
Put
, implying that  and
. 
In particular,  belongs
to . 
If  is inside  (and in particular if ) then  cannot win 
 the tournament  which is the contradiction we are after.
So we may assume that  is outside .



Let  be the line through  orthogonal to .  
Clearly,  intersects  at two points,  and another point
 (lying counterclockwise to  along , by the choice of ). Since , and  equal to twice the angle between  and ,  is at least ,
 so  is outside .  By assumption,  lies in the halfplane bounded by  and containing . Since we assume that 
is not in  it must be to the right of the  line
from  to .  It follows that  intersects  at
some point  to the right of the  line from  to ; see
Figure~\ref{Fig:Clockwise}.


We claim that  is inside the cone with apex  bounded by the rays
 and . 
Indeed, suppose to the contrary that the claim is false.
It follows that in the diagram  the edge
 is -long around . Indeed, denote the intersection point of  and  as  (see Figure \ref{Fig:Clockwise}). Then . Since the angle between  and  is between  and , the claim follows. Now, according to
Lemma~\ref{Lemma:EmptyCone}, ,
which contradicts the choice of . It follows that  is in
the cone bounded by  and  and thus
 hits  before , and therefore also before . Hence,  cannot win , and we get the final contradiction which completes the proof of the theorem.
\end{proof}

\begin{figure}[htbp]
\begin{center}
\begin{picture}(0,0)\includegraphics{fix2.pstex}\end{picture}\setlength{\unitlength}{1776sp}\begingroup\makeatletter\ifx\SetFigFont\undefined \gdef\SetFigFont#1#2#3#4#5{\reset@font\fontsize{#1}{#2pt}\fontfamily{#3}\fontseries{#4}\fontshape{#5}\selectfont}\fi\endgroup \begin{picture}(6946,6204)(3086,-7648)
\put(7548,-1880){\makebox(0,0)[lb]{\smash{{\SetFigFont{10}{12.0}{\rmdefault}{\mddefault}{\updefault}{\color[rgb]{0,0,0}}}}}}
\put(4762,-4621){\makebox(0,0)[rb]{\smash{{\SetFigFont{10}{12.0}{\rmdefault}{\mddefault}{\updefault}{\color[rgb]{0,0,0}}}}}}
\put(10017,-4231){\makebox(0,0)[rb]{\smash{{\SetFigFont{10}{12.0}{\rmdefault}{\mddefault}{\updefault}{\color[rgb]{0,0,0}}}}}}
\put(5040,-1909){\makebox(0,0)[lb]{\smash{{\SetFigFont{10}{12.0}{\rmdefault}{\mddefault}{\updefault}{\color[rgb]{0,0,0}}}}}}
\put(4876,-7021){\makebox(0,0)[lb]{\smash{{\SetFigFont{10}{12.0}{\rmdefault}{\mddefault}{\updefault}{\color[rgb]{0,0,0}}}}}}
\put(9466,-2836){\makebox(0,0)[lb]{\smash{{\SetFigFont{10}{12.0}{\rmdefault}{\mddefault}{\updefault}{\color[rgb]{0,0,0}}}}}}
\put(5521,-5836){\makebox(0,0)[lb]{\smash{{\SetFigFont{10}{12.0}{\rmdefault}{\mddefault}{\updefault}{\color[rgb]{0,0,0}}}}}}
\put(5386,-3376){\makebox(0,0)[lb]{\smash{{\SetFigFont{10}{12.0}{\rmdefault}{\mddefault}{\updefault}{\color[rgb]{0,0,0}}}}}}
\put(4306,-3151){\makebox(0,0)[lb]{\smash{{\SetFigFont{10}{12.0}{\rmdefault}{\mddefault}{\updefault}{\color[rgb]{0,0,0}}}}}}
\put(4366,-6016){\makebox(0,0)[lb]{\smash{{\SetFigFont{10}{12.0}{\rmdefault}{\mddefault}{\updefault}{\color[rgb]{0,0,0}}}}}}
\put(7231,-4816){\makebox(0,0)[lb]{\smash{{\SetFigFont{10}{12.0}{\rmdefault}{\mddefault}{\updefault}{\color[rgb]{0,0,0}}}}}}
\put(9031,-5866){\makebox(0,0)[lb]{\smash{{\SetFigFont{10}{12.0}{\rmdefault}{\mddefault}{\updefault}{\color[rgb]{0,0,0}}}}}}
\put(6421,-2731){\makebox(0,0)[lb]{\smash{{\SetFigFont{10}{12.0}{\rmdefault}{\mddefault}{\updefault}{\color[rgb]{0,0,0}}}}}}
\put(6616,-3031){\makebox(0,0)[lb]{\smash{{\SetFigFont{10}{12.0}{\rmdefault}{\mddefault}{\updefault}{\color[rgb]{0,0,0}}}}}}
\put(7471,-5401){\makebox(0,0)[lb]{\smash{{\SetFigFont{10}{12.0}{\rmdefault}{\mddefault}{\updefault}{\color[rgb]{0,0,0}}}}}}
\put(7351,-5791){\makebox(0,0)[lb]{\smash{{\SetFigFont{10}{12.0}{\rmdefault}{\mddefault}{\updefault}{\color[rgb]{0,0,0}}}}}}
\end{picture} \caption{\sf \small The proof of Lemma \ref{Lemma:qExtremal}: The point  is strongly -extremal for  and
-extremal for .}\label{Fig:Extremal1}
\end{center}
\end{figure}

\noindent {\bf Remark:} We have not made any serious attempt to reduce the constants  appearing in the definitions of various -s that we maintain. We suspect, though, that they can be significantly reduced.

To complete the proof of Theorem \ref{Thm:Soundness}, we provide the missing lemma.


\begin{lemma}\label{Lemma:qExtremal}
Let  be a pair of points and  an index,
such that the point  is strongly -extremal for  but
.
Then there exists a point  such that  and
  is -extremal
for  .
\end{lemma}

\begin{proof}

Let  be the index for which  and let
 be the 
line through , orthogonal to .
Assume without loss of generality that  is vertical and
the ray  extends to the right of .


\begin{figure}[htb]
\begin{center}
\input{fix3.pstex_t}\hspace{1cm}\begin{picture}(0,0)\includegraphics{fix4.pstex}\end{picture}\setlength{\unitlength}{1342sp}\begingroup\makeatletter\ifx\SetFigFont\undefined \gdef\SetFigFont#1#2#3#4#5{\reset@font\fontsize{#1}{#2pt}\fontfamily{#3}\fontseries{#4}\fontshape{#5}\selectfont}\fi\endgroup \begin{picture}(10607,8798)(3086,-8353)
\put(4762,-4621){\makebox(0,0)[rb]{\smash{{\SetFigFont{11}{13.2}{\rmdefault}{\mddefault}{\updefault}{\color[rgb]{0,0,0}}}}}}
\put(10017,-4231){\makebox(0,0)[rb]{\smash{{\SetFigFont{10}{12.0}{\rmdefault}{\mddefault}{\updefault}{\color[rgb]{0,0,0}}}}}}
\put(7548,-1880){\makebox(0,0)[lb]{\smash{{\SetFigFont{10}{12.0}{\rmdefault}{\mddefault}{\updefault}{\color[rgb]{0,0,0}}}}}}
\put(4726,-1411){\makebox(0,0)[lb]{\smash{{\SetFigFont{11}{13.2}{\rmdefault}{\mddefault}{\updefault}{\color[rgb]{0,0,0}}}}}}
\put(7231,-4816){\makebox(0,0)[lb]{\smash{{\SetFigFont{10}{12.0}{\rmdefault}{\mddefault}{\updefault}{\color[rgb]{0,0,0}}}}}}
\put(7426,-6811){\makebox(0,0)[lb]{\smash{{\SetFigFont{10}{12.0}{\rmdefault}{\mddefault}{\updefault}{\color[rgb]{0,0,0}}}}}}
\put(4951,-6961){\makebox(0,0)[lb]{\smash{{\SetFigFont{10}{12.0}{\rmdefault}{\mddefault}{\updefault}{\color[rgb]{0,0,0}}}}}}
\put(5251,-6361){\makebox(0,0)[lb]{\smash{{\SetFigFont{10}{12.0}{\rmdefault}{\mddefault}{\updefault}{\color[rgb]{0,0,0}}}}}}
\put(4201,-5611){\makebox(0,0)[lb]{\smash{{\SetFigFont{11}{13.2}{\rmdefault}{\mddefault}{\updefault}{\color[rgb]{0,0,0}}}}}}
\put(8176,-8161){\makebox(0,0)[lb]{\smash{{\SetFigFont{9}{10.8}{\rmdefault}{\mddefault}{\updefault}{\color[rgb]{0,0,0}}}}}}
\put(5386,-3376){\makebox(0,0)[lb]{\smash{{\SetFigFont{9}{10.8}{\rmdefault}{\mddefault}{\updefault}{\color[rgb]{0,0,0}}}}}}
\put(4306,-3151){\makebox(0,0)[lb]{\smash{{\SetFigFont{11}{13.2}{\rmdefault}{\mddefault}{\updefault}{\color[rgb]{0,0,0}}}}}}
\put(6616,-3031){\makebox(0,0)[lb]{\smash{{\SetFigFont{10}{12.0}{\rmdefault}{\mddefault}{\updefault}{\color[rgb]{0,0,0}}}}}}
\put(6421,-2731){\makebox(0,0)[lb]{\smash{{\SetFigFont{10}{12.0}{\rmdefault}{\mddefault}{\updefault}{\color[rgb]{0,0,0}}}}}}
\end{picture} \caption{\sf \small Left: The circular arc  is the locus of all points
which are to the right of  and see it at angle . Right: To minimize  we increase the radius of
 until one of its intersection points with 
coincides with .}\label{Fig:Extremal2}
\end{center}
\end{figure}

Let  be the point at which  intersects the bisector
, and let  be the disk centered at 
whose boundary contains both  and . Since
, the interior of  must contain some other point ; see Figure~\ref{Fig:Extremal1}. 

Let  be the cone 
emanating from  such that each of its bounding rays makes an angle of
 with 
the ray from  through ;
 in particular  contains . 
Let  (resp., ) denote the upper (resp., lower) endpoint of the
intersection of   and  . Since  is strongly -extremal for
, the interior of the triangle  does not contain any
points of . Hence,  must be outside 
the triangle . So either  is above
  (and inside ) or below  (and inside ).


Assume, without loss of generality, that  is below , as shown in
Figure~\ref{Fig:Extremal1}. (The case where  is above
  is fully symmetric.) 
 Let  and  denote the intersection points
 and , respectively. Let 
be the point at which the ray 
from  through 
intersects . Then
the intersection of the triangle  and  is empty.
Among all the points of  in  we choose  so that its -coordinate is
the smallest. For this choice of  we also
have that  is empty (since it is contained in  and lies to the left of ).
In other words,  is empty.

Let  (resp., ) denote the angle 
(resp., ). It remains to show that
 and .
This will imply that the cone  that contains  is fully contained in the cone bounded by the rays from  through  and , so  is extreme in the -direction within , which is what the lemma asserts. Since  is inside , it is clear that
. 
The angle
 however may be smaller than , but, as we next show,
.
Indeed, fix an angle  and
let  denote the circular arc which is the locus of
all points  that are to the right of  and 
the angle  is 
. The endpoints of  are  and
, and its center  is on the (horizontal) bisector of
; see Figure~\ref{Fig:Extremal2} (left). 

Notice that  intersects  at two points, one of which is , which
are symmetric with respect to the line 
through  and .  As  decreases
 moves to the right, and the intersection of 
with  rotates clockwise around .
Consider the smallest  such that  intersects  on or below  . It follows that this intersection is at .
 See Figure~\ref{Fig:Extremal2} (right).


This shows that
for  fixed   and , the position of 
in  below the line  which minimizes  is at
. 
To complete the analysis, we look for the position of  that minimizes  when
 is at . 
Note that, as  moves along , the points  and  do not change.
As shown in Figure~\ref{Fig:Extremal3} (left),
  decreases when 
  tends counterclockwise to .
When  is at ,
  is tangent to .
A simple calculation, 
 illustrated in Figure~\ref{Fig:Extremal3} (right), shows that
 . By the
 inequality , for  sufficiently small, it follows that , implying, as noted above, that the point  is -extremal 
 for . This completes the proof of the lemma.
\end{proof}

\begin{figure}[htbp]
\begin{center}
\input{fix5.pstex_t}\hspace{2cm}\begin{picture}(0,0)\includegraphics{fix6.pstex}\end{picture}\setlength{\unitlength}{1697sp}\begingroup\makeatletter\ifx\SetFigFont\undefined \gdef\SetFigFont#1#2#3#4#5{\reset@font\fontsize{#1}{#2pt}\fontfamily{#3}\fontseries{#4}\fontshape{#5}\selectfont}\fi\endgroup \begin{picture}(6621,6006)(3374,-7450)
\put(4882,-4798){\makebox(0,0)[rb]{\smash{{\SetFigFont{10}{12.0}{\rmdefault}{\mddefault}{\updefault}{\color[rgb]{0,0,0}}}}}}
\put(6151,-2011){\makebox(0,0)[lb]{\smash{{\SetFigFont{10}{12.0}{\rmdefault}{\mddefault}{\updefault}{\color[rgb]{0,0,0}}}}}}
\put(9901,-4261){\makebox(0,0)[rb]{\smash{{\SetFigFont{10}{12.0}{\rmdefault}{\mddefault}{\updefault}{\color[rgb]{0,0,0}}}}}}
\put(5040,-1909){\makebox(0,0)[lb]{\smash{{\SetFigFont{10}{12.0}{\rmdefault}{\mddefault}{\updefault}{\color[rgb]{0,0,0}}}}}}
\put(7231,-4816){\makebox(0,0)[lb]{\smash{{\SetFigFont{10}{12.0}{\rmdefault}{\mddefault}{\updefault}{\color[rgb]{0,0,0}}}}}}
\put(9466,-2836){\makebox(0,0)[lb]{\smash{{\SetFigFont{10}{12.0}{\rmdefault}{\mddefault}{\updefault}{\color[rgb]{0,0,0}}}}}}
\put(6151,-2986){\makebox(0,0)[lb]{\smash{{\SetFigFont{10}{12.0}{\rmdefault}{\mddefault}{\updefault}{\color[rgb]{0,0,0}}}}}}
\put(5851,-2761){\makebox(0,0)[lb]{\smash{{\SetFigFont{10}{12.0}{\rmdefault}{\mddefault}{\updefault}{\color[rgb]{0,0,0}}}}}}
\put(6601,-7261){\makebox(0,0)[lb]{\smash{{\SetFigFont{10}{12.0}{\rmdefault}{\mddefault}{\updefault}{\color[rgb]{0,0,0}}}}}}
\put(5251,-7336){\makebox(0,0)[lb]{\smash{{\SetFigFont{10}{12.0}{\rmdefault}{\mddefault}{\updefault}{\color[rgb]{0,0,0}}}}}}
\put(5251,-5611){\makebox(0,0)[lb]{\smash{{\SetFigFont{10}{12.0}{\rmdefault}{\mddefault}{\updefault}{\color[rgb]{0,0,0}}}}}}
\put(5476,-2386){\makebox(0,0)[lb]{\smash{{\SetFigFont{10}{12.0}{\rmdefault}{\mddefault}{\updefault}{\color[rgb]{0,0,0}}}}}}
\put(5626,-3661){\makebox(0,0)[lb]{\smash{{\SetFigFont{10}{12.0}{\rmdefault}{\mddefault}{\updefault}{\color[rgb]{0,0,0}}}}}}
\put(3751,-3211){\makebox(0,0)[lb]{\smash{{\SetFigFont{9}{10.8}{\rmdefault}{\mddefault}{\updefault}{\color[rgb]{0,0,0}}}}}}
\put(3389,-3856){\makebox(0,0)[lb]{\smash{{\SetFigFont{9}{10.8}{\rmdefault}{\mddefault}{\updefault}{\color[rgb]{0,0,0}}}}}}
\put(6060,-5596){\makebox(0,0)[lb]{\smash{{\SetFigFont{10}{12.0}{\rmdefault}{\mddefault}{\updefault}{\color[rgb]{0,0,0}}}}}}
\put(4649,-3553){\makebox(0,0)[lb]{\smash{{\SetFigFont{10}{12.0}{\rmdefault}{\mddefault}{\updefault}{\color[rgb]{0,0,0}}}}}}
\put(4595,-3898){\makebox(0,0)[lb]{\smash{{\SetFigFont{10}{12.0}{\rmdefault}{\mddefault}{\updefault}{\color[rgb]{0,0,0}}}}}}
\put(4911,-2803){\makebox(0,0)[lb]{\smash{{\SetFigFont{10}{12.0}{\rmdefault}{\mddefault}{\updefault}{\color[rgb]{0,0,0}}}}}}
\end{picture} \caption{\sf \small Left:  is minimized as  tends counterclockwise to .
Right: Proving that  when  and
. The triangles  and  are isosceles and similar, and . Thus . }\label{Fig:Extremal3}
\end{center}
\end{figure}

In Section \ref{Subsec:ReducedMaintenNaive} we describe a naive algorithm for kinetic maintenance of , which encounters a total of  events in the tournaments . In Section \ref{Subsec:ReducedMaintenImprove} we consider a slightly more economical definition of the tournaments , yielding a solution which processes only  events in  overall time.
\subsection{Naive maintenance of }\label{Subsec:ReducedMaintenNaive}
As the points of  move, we need to update the  , which, as we recall, contains those edges  such that  wins  consecutive tournaments  of , and  is strongly -extremal and -extremal for . We thus need to detect and process instances at which one of these conditions changes. There are several events at which such a change can occur:\\
\indent(a) A change in the sets of neighbors , for .\\
\indent(b) A change in the status of being strongly -extremal for some pair .\\
\indent(c) A change in the winner of some tournament  (at which two existing members of  attain the same minimum distance in the direction ).

Note that each of the events (a)--(b) can arise only during a swap of two points in one of the  directions  or in one of the directions orthogonal to these vectors. 

For each  we maintain two lists. The first list, , stores the
points of  ordered by their projections on a line in the -direction, and the second list, , stores the points ordered by their projections on a line orthogonal to the -direction. We note that, as long as the order in each of the  lists  remains unchanged, the discrete structure of the range trees , and the auxiliary items , does not change either. More precisely, the structure of  changes only when two consecutive elements in 
or in  swap their order in the respective list; whereas the auxiliary items , stored at secondary nodes of , may also change when two consecutive points swap their order in the list . 
There are  discrete events where consecutive points in  or 
swap. We call these events {\em -swaps} and \textit{-swaps}, respectively. Each such event happens
when the line trough a pair of points becomes orthogonal or parallel to . We
can maintain each list in linear space for a total of  space
for all lists. Processing a swap takes  time to replace a
constant number of elements in the event queue (and more time to update the various structures, as discussed next).


\smallskip

\noindent{\bf The range trees .}
As just noted, the structure of  changes either at a -swap or at a
-swap. As described in \cite[Section 4]{KineticNeighbors}, we
can update  when such a swap occurs, including the various auxiliary data that it stores, in  time. 
(The factor  is due to the fact that we maintain
 extreme points  and  in each
secondary node  of , whereas in \cite{KineticNeighbors} only
two points are maintained.)

In a similar manner, an -swap of two points  may affect one of the items 
and  stored at any secondary node  of any , for , such that both  belong to  or to . Each  has only  such nodes, and the data structure of \cite{KineticNeighbors} allows us to
update , when an -swap occurs in  time. Summing up over all , we get
that the total update time of the range trees after an -swap is
.  As follows from the analysis in~\cite[Section 4]{KineticNeighbors}, the
trees , for , require a total of  storage (because of the  items
 stored at each secondary node of each of the  trees).

\smallskip

\noindent{\bf The tournaments .}
The kinetic tournament , for  and  contains the points in the set .  Since 
varies both kinetically and dynamically and therefore the tournaments
 need to be maintained as kinetic and dynamic tournaments, in the manner reviewed in Section \ref{sec:Prelim}. 

For , we define  to be
the set of pairs of points , such that there exists a
secondary node  in , and indices , for which
 and .
For a fixed , a point  belongs to  pairs  in
, for a total of  pairs over all sets .  It follows that the total size of all the
sets  is .  Any secondary node of any tree
, for , contributes at most  pairs to
the respective set .


The set  consists of all the points  such that there exists a set
 that contains the pair  or the pair .  So the
total size of the sets , over all points , is . A set  changes only when one of the sets  changes,
which can happen only as the result of a swap.

Specifically, when  changes for some  and , from a point  to a point , we
make the following updates.  
(i) If  for all
 then for every  we delete the pair
 from . (ii) We add the
pair  to .  We make analogous updates
when one of the values  changes.  When a node  is created, deleted,
or involved in a rotation, we update the pairs (,
) in  for every  and . In such a case
we say that node  {\em is changed}.

A change of  or  in an existing
node  generates  changes in  and thereby 
changes to the sets .  Thus, it may generate  updates to the
tournaments . A change of a secondary node may
generate  changes to the sets  and thereby  updates
to the tournaments .

A point  or  changes during either
a , , or -swap.  Each -swap, for any , causes 
points  or  to change (over the entire collection of trees), and
therefore each swap causes  updates to the tournamnets
.  The number of nodes which
change in  by a  or -swap is . Each
such change causes  updates to the tournaments .
Therefore the total number of updates to tournaments due to changes of
nodes is also  per swap.

The number of swaps is , so overall we get 
updates to the tournaments.  The size of each individual tournament is
.
By Theorem \ref{thm:kinetic-tour} these updates generate

tournament
events, which are processed in
 
time.
Processing each individual tournament event takes  time.
 
Since the size of each tournament is  and there are  tournaments, the
total size of all tournaments is .


\smallskip

\noindent{\bf Testing whether  is strongly -extremal for the
  winner of .}
For each , and for each pair  we maintain
those indices  (if there are any) for which
 is  strongly -extremal for .
Recall that each point  belongs to  pairs in the sets
.

We use the trees  for  to find, for a query 
, the  point , for each . The query time is  
Using this information we easily determine,
 for a pair ,
for which values of 
 is  strongly -extremal for .




As explained above, every swap changes
 pairs of the sets .
When a new pair is added to a set   we query the trees
, , to find for which values of ,  is
strongly -extremal for  (and vice versa).
This takes a total of  time for each
swap.

Furthermore, a point  can cease (or start) being  strongly
-extremal for  only during a swap which involves either  or . 
So when we process a swap 
between  and some other point we
recompute, for all pairs  and  in the current 
sets  and for every
,
 whether  is  strongly -extremal for , and whether  remains  strongly -extremal for .
This adds an overhead of
 time at each
swap. 

The following theorem summarizes the results obtained so far in this section.

\begin{theorem}
The   can be maintained using a data structure which
requires 
 space
 and encounters two types of
events: swaps and tournament events.\\ 
There are  swaps, 
each processed in  time. 
There are
  tournament 
events which are processed in overall
 
time.
Processing each individual tournament event takes   time.
\end{theorem}


\subsection{An even faster data structure}\label{Subsec:ReducedMaintenImprove}

We next reduce the overall time and space required to maintain 
roughly by factors of  and , respectively (bringing the dependence on  of both bounds down to roughly ).  We achieve that by
restricting each tournament  to contain a carefully chosen
subset  of size  (recall that the size of the entire set  is ).
The definition of  is based on
the following lemma. Its simple proof is given in Figure \ref{Fig:Compatible}.

\begin{lemma}\label{Lemma:Compatible}
Let  and let  be the index for which 
. Let  be an index, and  a direction such that
the rays  and  intersect
 at the same point. 
Then  lies in one of the two consecutive cones , where .
\end{lemma}

\begin{figure}[htbp]
\begin{center}
\begin{picture}(0,0)\includegraphics{Compatible.pstex}\end{picture}\setlength{\unitlength}{3947sp}\begingroup\makeatletter\ifx\SetFigFont\undefined \gdef\SetFigFont#1#2#3#4#5{\reset@font\fontsize{#1}{#2pt}\fontfamily{#3}\fontseries{#4}\fontshape{#5}\selectfont}\fi\endgroup \begin{picture}(1861,1793)(4976,-3542)
\put(5436,-3478){\makebox(0,0)[lb]{\smash{{\SetFigFont{11}{13.2}{\rmdefault}{\mddefault}{\updefault}{\color[rgb]{0,0,0}}}}}}
\put(6305,-3087){\makebox(0,0)[lb]{\smash{{\SetFigFont{11}{13.2}{\rmdefault}{\mddefault}{\updefault}{\color[rgb]{0,0,0}}}}}}
\put(6384,-2048){\makebox(0,0)[lb]{\smash{{\SetFigFont{11}{13.2}{\rmdefault}{\mddefault}{\updefault}{\color[rgb]{0,0,0}}}}}}
\put(5395,-3155){\makebox(0,0)[lb]{\smash{{\SetFigFont{11}{13.2}{\rmdefault}{\mddefault}{\updefault}{\color[rgb]{0,0,0}}}}}}
\put(6012,-2277){\makebox(0,0)[lb]{\smash{{\SetFigFont{11}{13.2}{\rmdefault}{\mddefault}{\updefault}{\color[rgb]{0,0,0}}}}}}
\put(4991,-2884){\makebox(0,0)[lb]{\smash{{\SetFigFont{11}{13.2}{\rmdefault}{\mddefault}{\updefault}{\color[rgb]{0,0,0}}}}}}
\put(5465,-2158){\makebox(0,0)[lb]{\smash{{\SetFigFont{11}{13.2}{\rmdefault}{\mddefault}{\updefault}{\color[rgb]{0,0,0}}}}}}
\put(5720,-3025){\makebox(0,0)[lb]{\smash{{\SetFigFont{11}{13.2}{\rmdefault}{\mddefault}{\updefault}{\color[rgb]{0,0,0}}}}}}
\put(5644,-1925){\makebox(0,0)[lb]{\smash{{\SetFigFont{11}{13.2}{\rmdefault}{\mddefault}{\updefault}{\color[rgb]{0,0,0}}}}}}
\put(6755,-1928){\makebox(0,0)[lb]{\smash{{\SetFigFont{12}{14.4}{\rmdefault}{\mddefault}{\updefault}{\color[rgb]{0,0,0}}}}}}
\put(5060,-2427){\makebox(0,0)[lb]{\smash{{\SetFigFont{11}{13.2}{\rmdefault}{\mddefault}{\updefault}{\color[rgb]{0,0,0}}}}}}
\put(6129,-1920){\makebox(0,0)[lb]{\smash{{\SetFigFont{12}{14.4}{\rmdefault}{\mddefault}{\updefault}{\color[rgb]{0,0,0}}}}}}
\end{picture} \caption{\sf \small Proof of Lemma \ref{Lemma:Compatible}: We assume that , and that the rays  and  hit  at the \textit{same point} . Then the angle , for some . The orientation of  is . Hence, the orientation of  is . Thus, the direction  lies in the union of the two consecutive cones , for .}\label{Fig:Compatible}
\end{center}
\end{figure}

It follows that in Corollary~\ref{Corol:ExtremalPair},
we can require that the indices , for which
 is a (strongly) -extremal pair, satisfy 
.  
Indeed, we may require that the vectors  hit  at the respective points  and  for which the angle  is at most , which, in turn, happens only if  bounds one of the cones .


For all  and  we define a set  which
consists of all pairs  of points of  such that there exists a
secondary node  in , and indices  and , such that  and
 or  
and .  We define the set  to consist of
all points  such that  .
For a point  the set of points that participate in the 
{\em reduced} tournament  is 
.
(Note that this rule distributes a point  to only seven nearby tournaments. Nevertheless, when the edge  is sufficiently long,  will belong to several consecutive neighborhoods , and therefore will appear in more tournaments, in particular in at least eight consecutive tournaments at which it should win, according to the definition of our .)


We claim that, with this redefinition of the tournaments
, Theorems \ref{Thm:Completeness} and \ref{Thm:Soundness} still hold.
To verify that  Theorem \ref{Thm:Completeness} holds one has to follow its (short) proof and
notice that, by Lemma 
\ref{Lemma:Compatible}, the point   belongs to the eight reduced tournaments 
which it is supposed to win.

We next indicate the changes required in the proof of Theorem
\ref{Thm:Soundness}. We use the same notation as in the original
proof of Theorem \ref{Thm:Soundness}, and recall that it assumed by contradiction that, say,
 even though  wins the tournaments
,
and the point  is strongly - and -extremal for
. 
We use
Lemma~\ref{Lemma:qExtremal} to establish the existence of some point
 such that  and  is
-extremal for .  Let  be the index for which , and let  be the secondary node in  for which 
and .  Note that .  We next choose an
 index  such that the point  either satisfies 
that
 if  is to the
right of the line from  to ,  or that
 if  is to the
left of the line from  to .  To re-establish Theorem \ref{Thm:Soundness} it suffices to show that
 participates in the reduced tournament  (resp., ) if  is to the
right (resp., left) of the line from  to .


It follows from the way we defined  in the original proof and from Lemma \ref{Lemma:Compatible} 
that  
(if  is to the
right of the line from  to )
or  (if  is to the
left of the line from  to ). 
So   and therefore  does participate in the reduced tournament  or .
Indeed, the direction  used in that proof lies in one of the cones . The direction  then forms an angle between  and  with , which lies counterclockwise from  if  lies to the right of the line from  to , or clockwise from  in the other case. This is easily seen to imply the two corresponding constraints on ; see Figure \ref{Fig:Clockwise}.



We change our algorithm accordingly to maintain only the reduced tournaments. 

Now every secondary node  of any range tree  contributes only seven pairs to each set
, for , so the size of each such set is . Since there
are  sets , their total size is .  Each pair in each  contributes an item to a constant
number of tournaments, so the total size of the tournaments is 
.  Each individual tournament  is now
of size , because  belongs to  pairs in each
set  for , , and  inherits only those points  that come from pairs , for  and .

When 
 changes from  to  for some  and , at most a constant number of pairs
 for 
are deleted from , and a constant number of pairs
 for 
are added to .
Similar changes take place in  for those three indices  satisfying
. 
When  
changes from  to  for
some  and ,
at most a constant number of pairs  are deleted
from  for the indices  satisfying , and a constant number of pairs
 are added for the same values of
.  Similarly,
at most a constant number of pairs  are deleted
from  for the indices  satisfying , and a constant number of pairs
 are added for the same values of
.

  A change
of a secondary node  in the tree  causes
  pairs in the sets 
to change.


Any -swap changes 
 nodes in 
   and thereby causes  pairs in the sets 
to change. 
Any -swap changes  extremal points ,
 at secondary nodes  of the trees , and thereby  causes 
  pairs in the sets 
to change. Since each pair in  contributes an item to a constant
number of tournaments it follows that 
 points are inserted to and deleted from
the tournaments  at each swap.


According to Theorem \ref{thm:kinetic-tour}
the size of each tournament is  -- the number
of elements that it contains. So the total size of all tournaments
is . In total we get that there are 
 updates to tournaments during swaps.
These updates generate 
 
tournament events
that are processed in overall
 
time.
Each  individual tournament event 
is processed in  time and each swap
can be processed in  
time.


\smallskip

In addition, for each pair 
we record whether  is strongly
-extremal for .
We maintain this information using the 
trees , for , as described above, which allow
for any  and  to test, in 
time, if  is strongly -extremal for .  At each swap event
we spend  extra time to compute for
 pairs  which are added to the sets 
whether  is strongly
-extremal for .


Consider a pair . The point  may stop being
strongly
-extremal for  only during a swap which involves  
or . So, as before, at each swap we find the  pairs
containing one of the points involved in the swap, and recompute, in 
total time,
for each such pair , whether
the strong extremal relation holds. 
 We thus obtain the following summary result.

\begin{theorem}\label{Thm:ReducedS}
Let  be a set of  moving points in  under algebraic
motion of bounded degree, 
and let  be a sufficiently small parameter. A -SDG of 
can be maintained using a data structure that requires 
 space and encounters two types of
events: swap events 
and tournament events.  There are  swap events, 
each processed in  time.
There are

 tournament 
events, which are handled in a total of 

processing time. The worst-case processing time of a
tournament event is . The data structure is also {\it local}, in the sense that each point
is stored, at any given time, at only  places in the structure.
\end{theorem}
Concerning locality, we note that a point participates in  projection tournaments at each of  tree nodes. If it wins in at least one of the projection tournaments at a node, it is fed to  
directional tournaments. So it appears in  places.

\noindent {\bf Remarks:} (1) Comparing this algorithm with the space-inefficient one of Section~\ref{sec:Prelim}, we note that they both use the 
same kind of tournaments, but here much fewer pairs of points 
( instead of ) participate in the 
tournaments. The price we have to pay is that the test for an edge 
to be stable is more involved. Moreover, keeping track of the subset of 
pairs that participate in the tournaments requires additional work,
which is facilitated by the range trees .

\medskip\noindent
(2) To be fair, we note that our  notation hides polylogarithmic factors in . Hence, comparing the analysis in this section with Theorem \ref{Thm:MaintainSDGPolyg}, we gain when  is smaller than some threshold, which is exponential in .
\section{Properties of  SDG}\label{Sec:SDGProperties}
We conclude the paper by establishing some of the properties of stable Delaunay graphs. 

\paragraph{Near cocircularities do not show up in an SDG.}
Consider a critical event during the kinetic maintenance of the full
Delaunay triangulation, in which four points  become cocircular,
in this order, along their circumcircle, with this circle being empty.
Just before the critical event, the Delaunay triangulation involved
two triangles, say, , . The Voronoi edge  shrinks
to a point (namely, to the circumcenter of  at the critical event),
and, after the critical cocircularity, is replaced by the Voronoi edge
, which expands from the circumcenter as time progresses.

Our algorithm will detect the possibility of such an event before the criticality occurs,
when  becomes -short (or even before this happens). It will then remove this edge from the stable subgraph,
so the actual cocircularity will not be recorded. The new edge 
will then be detected by the algorithm only when it becomes sufficiently long
(if this happens at all), and will then enter the stable Delaunay graph. In short,
critical cocircularities do not arise {\em at all} in our scheme.

As noted in the introduction, a Delaunay edge  (interior to the hull) is just about to become -short or -long when the sum of the opposite angles in its two adjacent Delaunay triangles is  (see Figure \ref{Fig:LongDelaunay}). This shows that changes in the stable Delaunay graph occur when the
``cocircularity defect'' of a nearly cocircular quadruple (i.e., the difference between  and the sum of opposite angles in the quadrilateral spanned by the quadruple) is between
 and , where  is the constant used in our definitions in Section \ref{sec:ViaPolygonal} or Section \ref{Sec:ReduceS}.
Note that a degenerate case of cocircularity is a collinearity on the convex
hull.
Such collinearities also do not show up in the stable
Delaunay graph.\footnote{Even if they did show up, no real damage would be done, because the number of such collinearities is only ; see, e.g., \cite{SA95}.} A hull collinearity between three nodes  is
detected before it happens, when (or before) the corresponding Voronoi edge
becomes -short, in which case the angle , where  is the middle point of the (near-)collinearity becomes  (see Figure \ref{collinearity}). 
Therefore a hull edge is removed from the  if the Delaunay triangle is
almost collinear. The edge (or any new edge about to replace it) re-appears in the  when its corresponding
Voronoi edge is long enough, as before. 

\begin{figure}
\begin{center}
\begin{picture}(0,0)\includegraphics{collinearity.pstex}\end{picture}\setlength{\unitlength}{2368sp}\begingroup\makeatletter\ifx\SetFigFont\undefined \gdef\SetFigFont#1#2#3#4#5{\reset@font\fontsize{#1}{#2pt}\fontfamily{#3}\fontseries{#4}\fontshape{#5}\selectfont}\fi\endgroup \begin{picture}(2725,2624)(1274,-2548)
\put(1289,-1468){\makebox(0,0)[rb]{\smash{{\SetFigFont{12}{14.4}{\rmdefault}{\mddefault}{\updefault}{\color[rgb]{0,0,0}}}}}}
\put(1786,-1601){\makebox(0,0)[lb]{\smash{{\SetFigFont{12}{14.4}{\rmdefault}{\mddefault}{\updefault}{\color[rgb]{0,0,0}}}}}}
\put(1908,-773){\makebox(0,0)[lb]{\smash{{\SetFigFont{12}{14.4}{\rmdefault}{\mddefault}{\updefault}{\color[rgb]{0,0,0}}}}}}
\put(1887,-1886){\makebox(0,0)[lb]{\smash{{\SetFigFont{12}{14.4}{\rmdefault}{\mddefault}{\updefault}{\color[rgb]{0,0,0}}}}}}
\put(1373,-439){\makebox(0,0)[lb]{\smash{{\SetFigFont{12}{14.4}{\rmdefault}{\mddefault}{\updefault}{\color[rgb]{0,0,0}}}}}}
\put(1385,-2183){\makebox(0,0)[lb]{\smash{{\SetFigFont{12}{14.4}{\rmdefault}{\mddefault}{\updefault}{\color[rgb]{0,0,0}}}}}}
\put(2724,-1316){\makebox(0,0)[lb]{\smash{{\SetFigFont{12}{14.4}{\rmdefault}{\mddefault}{\updefault}{\color[rgb]{0,0,0}}}}}}
\put(3792,-505){\makebox(0,0)[lb]{\smash{{\SetFigFont{12}{14.4}{\rmdefault}{\mddefault}{\updefault}{\color[rgb]{0,0,0}}}}}}
\put(3766,-2175){\makebox(0,0)[lb]{\smash{{\SetFigFont{12}{14.4}{\rmdefault}{\mddefault}{\updefault}{\color[rgb]{0,0,0}}}}}}
\put(2846,-986){\makebox(0,0)[lb]{\smash{{\SetFigFont{12}{14.4}{\rmdefault}{\mddefault}{\updefault}{\color[rgb]{0,0,0}}}}}}
\end{picture} \caption{\small \sf The near collinearity that corresponds to a Voronoi edge
becoming -short.} \label{collinearity}
\end{center}
\end{figure}

\paragraph{SDGs are not too sparse.}
Consider the Voronoi cell  of a point , and suppose that  has only one -long edge . Since the angle at which  sees  is at most , the sum of the angles at which  sees the other edges is at least , so  has at least  -short edges. Let  denote the number of points  with this property. Then the sum of their degrees in  is at least . Similarly, if  points do not have any -long Voronoi edge, then the sum of their degrees is at least . Any other point at least two -long Voronoi edges and its degree is at least 3 if it is an interior point, or at least 2 otherwise. So the number of -long
edges is at least (recall that each -long edge is counted twice)

Let  denote the number of hull vertices. Since the sum of the degrees is , we get


implying that

Plugging this inequality in (\ref{Eq:long-edge}), we conclude that the number of -long edges is at least

As  decreases, the number of edges in the SDG is always at 
least a quantity that gets closer to .
This is nearly tight, since there exist -point sets for which the number of stable edges is only roughly , see Figure \ref{Fig:ShiftedGrid}.

\begin{figure}
\begin{center}
\begin{picture}(0,0)\includegraphics{Grid.pstex}\end{picture}\setlength{\unitlength}{1776sp}\begingroup\makeatletter\ifx\SetFigFont\undefined \gdef\SetFigFont#1#2#3#4#5{\reset@font\fontsize{#1}{#2pt}\fontfamily{#3}\fontseries{#4}\fontshape{#5}\selectfont}\fi\endgroup \begin{picture}(3295,3225)(7679,-4218)
\end{picture} \caption{\small \sf If the points of  lie on a sufficiently spaced shifted grid then the number of -long edges in  (the vertical ones) is close to .} \label{Fig:ShiftedGrid}
\end{center}
\end{figure}

\paragraph{Closest pairs, crusts, -skeleta, and the SDG.}
Let , and let  be a set of  points in the plane. 
The \textit{-skeleton} of  is a graph on  that 
consists of all the edges  such that the union of the two disks of 
radius , touching  and , does not contain any 
point of . See, e.g., \cite{Crusts,Skeletons} for 
properties of the -skeleton, and for its applications in surface reconstruction. 
We show that the edges of the -skeleton are -stable 
in , provided .
In Figure \ref{Fig:Skeleton} we sketch a straightforward proof of the fact that the edges of the -skeleton are -stable in , provided that .

\begin{figure}[htbp]
\begin{center}
\begin{picture}(0,0)\includegraphics{Skeleton.pstex}\end{picture}\setlength{\unitlength}{4539sp}\begingroup\makeatletter\ifx\SetFigFont\undefined \gdef\SetFigFont#1#2#3#4#5{\reset@font\fontsize{#1}{#2pt}\fontfamily{#3}\fontseries{#4}\fontshape{#5}\selectfont}\fi\endgroup \begin{picture}(2463,1594)(1734,-1724)
\put(2355,-826){\rotatebox{41.0}{\makebox(0,0)[lb]{\smash{{\SetFigFont{12}{14.4}{\rmdefault}{\mddefault}{\updefault}{\color[rgb]{0,0,0}}}}}}}
\put(2813,-348){\makebox(0,0)[lb]{\smash{{\SetFigFont{12}{14.4}{\rmdefault}{\mddefault}{\updefault}{\color[rgb]{0,0,0}}}}}}
\put(2819,-1548){\makebox(0,0)[lb]{\smash{{\SetFigFont{12}{14.4}{\rmdefault}{\mddefault}{\updefault}{\color[rgb]{0,0,0}}}}}}
\put(2746,-747){\makebox(0,0)[lb]{\smash{{\SetFigFont{12}{14.4}{\rmdefault}{\mddefault}{\updefault}{\color[rgb]{0,0,0}}}}}}
\put(2919,-753){\makebox(0,0)[lb]{\smash{{\SetFigFont{12}{14.4}{\rmdefault}{\mddefault}{\updefault}{\color[rgb]{0,0,0}}}}}}
\put(2202,-969){\makebox(0,0)[lb]{\smash{{\SetFigFont{12}{14.4}{\rmdefault}{\mddefault}{\updefault}{\color[rgb]{0,0,0}}}}}}
\put(3371,-1015){\makebox(0,0)[lb]{\smash{{\SetFigFont{12}{14.4}{\rmdefault}{\mddefault}{\updefault}{\color[rgb]{0,0,0}}}}}}
\end{picture} \caption {\small \sf An edge  of the -skeleton of  (for ).  and  are centers of the two -empty disks of radius  touching  and . Clearly, each of  sees the Voronoi edge  at an angle at least  (so it is -stable). We have  or . That is, for  every edge of the -skeleton is -stable.}\label{Fig:Skeleton}
\end{center}
\end{figure}

 A similar argument shows that the stable Delaunay graph contains the 
closest pair in  as well as the crust of a set of points sampled
sufficiently densely along a 1-dimensional curve (see \cite{Amenta,Crusts} for the definition of crusts and their applications in surface 
reconstruction). 
We only sketch the argument for closest pairs: If  is a closest pair then , and the two adjacent Delaunay triangles  are such that their angles of  are at most  each, so  is -long, ensuring that  belongs to any stable subgraph for  sufficiently small; see \cite{KineticNeighbors} for more details. 
We omit the proof for crusts, which is fairly straightforward.

\begin{figure}
\begin{center}
\begin{picture}(0,0)\includegraphics{norng.pstex}\end{picture}\setlength{\unitlength}{2368sp}\begingroup\makeatletter\ifx\SetFigFont\undefined \gdef\SetFigFont#1#2#3#4#5{\reset@font\fontsize{#1}{#2pt}\fontfamily{#3}\fontseries{#4}\fontshape{#5}\selectfont}\fi\endgroup \begin{picture}(3275,2415)(2435,-3809)
\put(4041,-3722){\makebox(0,0)[lb]{\smash{{\SetFigFont{12}{14.4}{\rmdefault}{\mddefault}{\updefault}{\color[rgb]{0,0,0}}}}}}
\put(3628,-1720){\makebox(0,0)[lb]{\smash{{\SetFigFont{12}{14.4}{\rmdefault}{\mddefault}{\updefault}{\color[rgb]{0,0,0}}}}}}
\put(3988,-1707){\makebox(0,0)[lb]{\smash{{\SetFigFont{12}{14.4}{\rmdefault}{\mddefault}{\updefault}{\color[rgb]{0,0,0}}}}}}
\put(4391,-1739){\makebox(0,0)[lb]{\smash{{\SetFigFont{12}{14.4}{\rmdefault}{\mddefault}{\updefault}{\color[rgb]{0,0,0}}}}}}
\end{picture} \caption{\small \sf  is an edge of the relative neighborhood graph but not of
.}
\label{norng1}
\end{center}
\end{figure}

In contrast, stable Delaunay graphs need not contain all the
edges of several other important subgraphs of the Delaunay
triangulation, including the Euclidean minimum spanning tree, the
Gabriel graph, the relative neighborhood graph, and the
all-nearest-neighbors graph. An illustration for the relative neighborhood graph is given in Figure \ref{norng1}. As a matter of fact, the stable
Delaunay graph need not even be connected, as is illustrated in
Figure~\ref{norng2}.

\begin{figure}
\begin{center}
\begin{picture}(0,0)\includegraphics{wheel.pstex}\end{picture}\setlength{\unitlength}{2368sp}\begingroup\makeatletter\ifx\SetFigFont\undefined \gdef\SetFigFont#1#2#3#4#5{\reset@font\fontsize{#1}{#2pt}\fontfamily{#3}\fontseries{#4}\fontshape{#5}\selectfont}\fi\endgroup \begin{picture}(3914,3776)(2007,-4073)
\put(3985,-2315){\makebox(0,0)[lb]{\smash{{\SetFigFont{12}{14.4}{\rmdefault}{\mddefault}{\updefault}{\color[rgb]{0,0,0}}}}}}
\end{picture} \caption{\small \sf A wheel-like configuration that disconnects  in the
stable Delaunay graph. The Voronoi diagram is drawn with dashed
lines, the stable Delaunay edges are drawn as solid, and the remaining
Delaunay edges as dotted edges. The points of the ``wheel" need not be cocircular.}
\label{norng2}
\end{center}
\end{figure}

\paragraph{Completing SDG into a triangulation.}
As argued above, the Delaunay edges that are missing in the stable
subgraph correspond to nearly cocircular quadruples of points, or
to nearly collinear triples of points near the boundary of the convex
hull. Arguably, these missing edges carry little information, because
they may ``flicker" in and out of the Delaunay triangulation even when the points
move just slightly (so that all angles determined by the triples of points change only slightly). Nevertheless, in many applications it is desirable
(or essential) to complete the stable subgraph into {\em some} triangulation,
preferrably one that is also stable in the combinatorial sense---it undergoes
only nearly quadratically many topological changes.

By the analysis in Section \ref{Sec:polygProp} we can achieve part of this goal by maintaining the full Delaunay triangulation  under the polygonal norm induced by the regular -gon . This diagram experiences only a nearly quadratic number of topological changes, is easy to maintain, and contains all the stable Euclidean Delaunay edges, for an appropriate choice of . Moreover, the union of its triangles is simply connected --- it has no holes. Unfortunately, in general it is not a triangulation of the entire convex hull of , as illustrated in Figure \ref{Fig:AlmostTriangulation}.

\begin{figure}
\begin{center}
\begin{picture}(0,0)\includegraphics{AlmostTriangulation.pstex}\end{picture}\setlength{\unitlength}{3158sp}\begingroup\makeatletter\ifx\SetFigFont\undefined \gdef\SetFigFont#1#2#3#4#5{\reset@font\fontsize{#1}{#2pt}\fontfamily{#3}\fontseries{#4}\fontshape{#5}\selectfont}\fi\endgroup \begin{picture}(3581,2577)(417,-2203)
\put(973,-791){\makebox(0,0)[lb]{\smash{{\SetFigFont{14}{16.8}{\rmdefault}{\mddefault}{\updefault}{\color[rgb]{0,0,0}}}}}}
\put(3899,-754){\makebox(0,0)[lb]{\smash{{\SetFigFont{14}{16.8}{\rmdefault}{\mddefault}{\updefault}{\color[rgb]{0,0,0}}}}}}
\put(2126,-754){\makebox(0,0)[lb]{\smash{{\SetFigFont{14}{16.8}{\rmdefault}{\mddefault}{\updefault}{\color[rgb]{0,0,0}}}}}}
\put(3013,-1502){\makebox(0,0)[lb]{\smash{{\SetFigFont{14}{16.8}{\rmdefault}{\mddefault}{\updefault}{\color[rgb]{0,0,0}}}}}}
\put(2937, -7){\makebox(0,0)[lb]{\smash{{\SetFigFont{14}{16.8}{\rmdefault}{\mddefault}{\updefault}{\color[rgb]{0,0,0}}}}}}
\end{picture} \caption{\small \sf The triangulation  of an 8-point set .
The points , which do not lie on the convex hull of , still lie on the boundary of the union of the triangles of 
because, for each of these points we can place an arbitrary large homothetic interior-empty copy of  which touches that point.}
\label{Fig:AlmostTriangulation}
\end{center}
\end{figure}

For the time being, we leave it as an open problem to come up with a 
simple and ``stable" scheme for filling the gaps between the triangles 
of  and the edges of the convex hull. 
It might be possible to extend the kinetic triangulation scheme 
developed in \cite{KRS}, so as to kinetically maintain a triangulation of the
``fringes" between  and the convex hull of , which is simple to define, easy to maintain, and undergoes only nearly quadratically many topological changes.

Of course, if we only want to maintain a triangulation of  that experiences only a nearly quadratically many topological changes, then we can use the scheme in \cite{KRS}, or the earlier, somewhat more involved scheme in \cite{AWY}. However, if we want to keep the triangulation ``as Delaunay as possible", we should include in it the stable portion of , and then the efficient completion of it, as mentioned above, becomes an issue, not yet resolved.

\paragraph{Nearly Euclidean norms and some of their properties.}
One way of interpreting the results of Section 3 is that the stability of Delaunay edges is preserved, in an appropriately defined sense, if we replace the Euclidean norm by the polygonal norm induced by the regular -gon  (for ). That is, stable edges in one Delaunay triangulation are also edges of the other triangulation, and are stable there too. Here we note that there is nothing special about : The same property holds if we replace the Euclidean norm by any sufficiently close norm (or convex distance function \cite{CD}).

Specifically, let  be a closed convex set in the plane that is contained in the
unit disk  and contains the disk  that
is concentric with  and scaled by the factor .
This
is equivalent to requiring that the Hausdorff distance  
between  and  be at most . 
We define the center of  to coincide with the common center of 
 and .




 induces a convex distance function , defined by . Consider the Voronoi diagram 
of  induced by , and the corresponding Delaunay triangulation . We omit here the detailed analysis of the structure of these diagrams, which is similar to that for the norm induced by , as presented in Section \ref{Sec:polygProp}. See also \cite{Chew,CD} for more details. Call an edge  of  -stable if the following property holds: Let  and  be the endpoints of , and let  be the two homothetic copies of  that are centered at , respectively, and touch  and . Then we require that the angle between the 
supporting lines at 
(for simplicity, assume that  is smooth, and so has a unique supporting line at  (and at ); otherwise, the condition should hold for any pair of supporting lines at  or at ) 
to  and  is at least , and that the same holds at .
In this case we refer to the edge  of  as -stable.

Note that -stability was (implicitly) defined in a different manner in Section \ref{Sec:polygProp}, based on the number of breakpoints of the corresponding Voronoi edges. Nevertheless, it is easy to verify that the two definitions are essentially identical.
\begin{figure}[hbt]
\begin{center}
\begin{picture}(0,0)\includegraphics{norm.pstex}\end{picture}\setlength{\unitlength}{2171sp}\begingroup\makeatletter\ifx\SetFigFont\undefined \gdef\SetFigFont#1#2#3#4#5{\reset@font\fontsize{#1}{#2pt}\fontfamily{#3}\fontseries{#4}\fontshape{#5}\selectfont}\fi\endgroup \begin{picture}(4599,4551)(439,-4158)
\put(2926,-1036){\makebox(0,0)[lb]{\smash{{\SetFigFont{12}{14.4}{\rmdefault}{\mddefault}{\updefault}{\color[rgb]{0,0,0}}}}}}
\put(2341,-2851){\makebox(0,0)[lb]{\smash{{\SetFigFont{12}{14.4}{\rmdefault}{\mddefault}{\updefault}{\color[rgb]{0,0,0}}}}}}
\put(3939,-3632){\makebox(0,0)[lb]{\smash{{\SetFigFont{12}{14.4}{\rmdefault}{\mddefault}{\updefault}{\color[rgb]{0,0,0}}}}}}
\put(2832,-1943){\makebox(0,0)[lb]{\smash{{\SetFigFont{11}{13.2}{\rmdefault}{\mddefault}{\updefault}{\color[rgb]{0,0,0}}}}}}
\put(886,-3278){\makebox(0,0)[lb]{\smash{{\SetFigFont{12}{14.4}{\rmdefault}{\mddefault}{\updefault}{\color[rgb]{0,0,0}}}}}}
\put(3090, 60){\makebox(0,0)[lb]{\smash{{\SetFigFont{12}{14.4}{\rmdefault}{\mddefault}{\updefault}{\color[rgb]{0,0,0}}}}}}
\put(4562,-1060){\makebox(0,0)[lb]{\smash{{\SetFigFont{12}{14.4}{\rmdefault}{\mddefault}{\updefault}{\color[rgb]{0,0,0}}}}}}
\put(3696,-710){\makebox(0,0)[lb]{\smash{{\SetFigFont{12}{14.4}{\rmdefault}{\mddefault}{\updefault}{\color[rgb]{0,0,0}}}}}}
\put(2886,-669){\makebox(0,0)[lb]{\smash{{\SetFigFont{12}{14.4}{\rmdefault}{\mddefault}{\updefault}{\color[rgb]{0,0,0}}}}}}
\put(2442,-1108){\makebox(0,0)[lb]{\smash{{\SetFigFont{11}{13.2}{\rmdefault}{\mddefault}{\updefault}{\color[rgb]{0,0,0}}}}}}
\end{picture} \caption{\small \sf An Illustration for Claim \ref{Q1}. 
 \label{fig:norm1}}
\end{center}
\end{figure}


A useful property of such a set  is the following: 
\begin{claim} \label{Q1}
Let  be a point on  and let
 be a supporting line to  at .
Let  be the point on 
closest to  ( and  lie on the same radius from the center
).
Let  be the arc of  , containing , and bounded by
the intersection points of  with  .
 Then the angle between  and  the tangent, , to  at any point
along , 
 is at most . 
\end{claim} 


\begin{proof}
Denote this angle by .
Clearly  is maximized when  is tangent to 
at an intersection of  and .
See Figure \ref{fig:norm1}.
It is easy to verify that
the distance from 
 to  is . But this distance has to be at least
, or else  would have contained a point inside 
, contrary to assumption. Hence we have 
, and thus , as claimed.
\end{proof}

We need a few more properties:

\begin{figure}[hbt]
\begin{center}
\begin{picture}(0,0)\includegraphics{norm2.pstex}\end{picture}\setlength{\unitlength}{1776sp}\begingroup\makeatletter\ifx\SetFigFont\undefined \gdef\SetFigFont#1#2#3#4#5{\reset@font\fontsize{#1}{#2pt}\fontfamily{#3}\fontseries{#4}\fontshape{#5}\selectfont}\fi\endgroup \begin{picture}(4845,3849)(439,-4642)
\put(3301,-2386){\makebox(0,0)[lb]{\smash{{\SetFigFont{10}{12.0}{\rmdefault}{\mddefault}{\updefault}{\color[rgb]{0,0,0}}}}}}
\put(4501,-1036){\makebox(0,0)[lb]{\smash{{\SetFigFont{10}{12.0}{\rmdefault}{\mddefault}{\updefault}{\color[rgb]{0,0,0}}}}}}
\put(2551,-1036){\makebox(0,0)[lb]{\smash{{\SetFigFont{10}{12.0}{\rmdefault}{\mddefault}{\updefault}{\color[rgb]{0,0,0}}}}}}
\put(3076,-4036){\makebox(0,0)[lb]{\smash{{\SetFigFont{10}{12.0}{\rmdefault}{\mddefault}{\updefault}{\color[rgb]{0,0,0}}}}}}
\put(901,-2686){\makebox(0,0)[lb]{\smash{{\SetFigFont{10}{12.0}{\rmdefault}{\mddefault}{\updefault}{\color[rgb]{0,0,0}}}}}}
\put(3376,-2911){\makebox(0,0)[lb]{\smash{{\SetFigFont{10}{12.0}{\rmdefault}{\mddefault}{\updefault}{\color[rgb]{0,0,0}}}}}}
\end{picture} \caption{\small \sf An Illustration for Claim \ref{Q2}. 
 \label{fig:norm2}}
\end{center}
\end{figure}

\begin{claim} \label{Q2}
Let  and  be two homothetic copies of  and let  be a
point such that (i)  lies on  and on , and
(ii)  and the respective centers ,  of , 
are collinear. Then  and  are tangent to each other at ;
more precisely, they have a common supporting line at , and, assuming  to be smooth,  is the only point of intersection of  (otherwise,  is a single connected arc containing .).
\end{claim}
\begin{proof}
Map each of ,  back to the standard placement of , by
translation and scaling, and note that both transformations map 
to the same point  on . Let  be a supporting line
of  at , and let ,  be the forward images of
 under the mappings of  to  and to , respectively.
Clearly,  and  coincide, and are a common supporting
line of  and  at .
See Figure \ref{fig:norm2}. The other asserted property follows immediately if  is smooth, and can easily be shown to hold in the non-smooth case too; we omit the routine argument.
\end{proof}

\begin{claim} \label{Q3}
Let  and  be two points on , and let  and 
be supporting lines of  at  and , respectively. Then the
difference between the angles that  and  form with
 is at most .
\end{claim}

\begin{proof}
Denote the two angles in the claim by  and ,
respectively.
Let  (resp., ) be the point on  nearest to (and
co-radial with)  (resp., ). Let ,  denote the
respective tangents to  at  and at .  Clearly, the 
respective angles ,  between the chord  
of  and ,  are equal. By Claim~\ref{Q1}, we 
have  and
, and the claim follows.
\end{proof}

\paragraph{The connection between Euclidean stability and -stability.} 
Let  be a -long Voronoi edge of the Euclidean diagram, for , 
and let  denote its endpoints. 
Let  and  denote the disks centered respectively
at , whose boundaries pass through  and , and let  be a
disk whose boundary passes through  and , so that 
 and the angles between the tangents to  and
to  and  at  (or at ) are at least  each, where
. (Recall that the angle between the tangents to  and  us at least .)

\begin{figure}[hbt]
\begin{center}
\begin{picture}(0,0)\includegraphics{norm3.pstex}\end{picture}\setlength{\unitlength}{2960sp}\begingroup\makeatletter\ifx\SetFigFont\undefined \gdef\SetFigFont#1#2#3#4#5{\reset@font\fontsize{#1}{#2pt}\fontfamily{#3}\fontseries{#4}\fontshape{#5}\selectfont}\fi\endgroup \begin{picture}(3909,2711)(3720,-3044)
\put(5321,-540){\makebox(0,0)[lb]{\smash{{\SetFigFont{12}{14.4}{\rmdefault}{\mddefault}{\updefault}{\color[rgb]{0,0,0}}}}}}
\put(3854,-1823){\makebox(0,0)[lb]{\smash{{\SetFigFont{12}{14.4}{\rmdefault}{\mddefault}{\updefault}{\color[rgb]{0,0,0}}}}}}
\put(5543,-1616){\makebox(0,0)[lb]{\smash{{\SetFigFont{12}{14.4}{\rmdefault}{\mddefault}{\updefault}{\color[rgb]{0,0,0}}}}}}
\put(5330,-2957){\makebox(0,0)[lb]{\smash{{\SetFigFont{12}{14.4}{\rmdefault}{\mddefault}{\updefault}{\color[rgb]{0,0,0}}}}}}
\put(5795,-2471){\makebox(0,0)[lb]{\smash{{\SetFigFont{12}{14.4}{\rmdefault}{\mddefault}{\updefault}{\color[rgb]{0,0,0}}}}}}
\put(5958,-1136){\makebox(0,0)[lb]{\smash{{\SetFigFont{12}{14.4}{\rmdefault}{\mddefault}{\updefault}{\color[rgb]{0,0,0}}}}}}
\put(6995,-1819){\makebox(0,0)[lb]{\smash{{\SetFigFont{12}{14.4}{\rmdefault}{\mddefault}{\updefault}{\color[rgb]{0,0,0}}}}}}
\end{picture} \caption{\small \sf The homothetic copy .
 \label{fig:norm3}}
\end{center}
\end{figure}

Let  and  denote  the center and radius of , respectively.
Note that  lies on  ``somewhere in the middle'', because of
the angle condition assumed above.
Let  denote the homothetic copy of  centered at  and
scaled by , so  is fully contained in  and thus
also in , implying that  is {\em empty}---it
does not contain any point of  in its interior. (This scaling makes the
unit circle  bounding  coincide with .) See Figure \ref{fig:norm3}.



Expand  about its center  until the first time it 
touches either  or . Suppose, without loss of generality, 
that it touches . Denote this placement of  as .
Let  denote a supporting line of  at . We claim that the angle between  and the tangent  to  at  is at most . Indeed, let  denote the tangents from  to . By Claim \ref{Q1}, the angles that they form with the tangent  to  at  are at most  each. As  is expanded to , these tangents rotate towards each other, one clockwise and one counterclockwise so when they coincide (at ) the resulting supporting line  lies inside the double wedge between them. Since  also lies inside this double wedge, and forms an angle of at most  with each of them, it follows that  must form an angle of at most  with , as claimed.

Since the angle between the tangent  to  at  and the tangent

to  at  is at least  it follows that the angle between
 and  is at least . 
A similar argument shows that the angle between  and 
the tangent  to  at  is at least  . 
 
\begin{figure}[hbt]
\begin{center}
\begin{picture}(0,0)\includegraphics{norm4.pstex}\end{picture}\setlength{\unitlength}{2763sp}\begingroup\makeatletter\ifx\SetFigFont\undefined \gdef\SetFigFont#1#2#3#4#5{\reset@font\fontsize{#1}{#2pt}\fontfamily{#3}\fontseries{#4}\fontshape{#5}\selectfont}\fi\endgroup \begin{picture}(4869,3279)(1114,-3823)
\put(4159,-3700){\makebox(0,0)[lb]{\smash{{\SetFigFont{12}{14.4}{\rmdefault}{\mddefault}{\updefault}{\color[rgb]{0,0,0}}}}}}
\put(1419,-1608){\makebox(0,0)[lb]{\smash{{\SetFigFont{12}{14.4}{\rmdefault}{\mddefault}{\updefault}{\color[rgb]{0,0,0}}}}}}
\put(4594,-1220){\makebox(0,0)[lb]{\smash{{\SetFigFont{12}{14.4}{\rmdefault}{\mddefault}{\updefault}{\color[rgb]{0,0,0}}}}}}
\put(5401,-3211){\makebox(0,0)[lb]{\smash{{\SetFigFont{12}{14.4}{\rmdefault}{\mddefault}{\updefault}{\color[rgb]{1,0,0}}}}}}
\put(5813,-2094){\makebox(0,0)[lb]{\smash{{\SetFigFont{12}{14.4}{\rmdefault}{\mddefault}{\updefault}{\color[rgb]{0,0,0}}}}}}
\put(4253,-2129){\makebox(0,0)[lb]{\smash{{\SetFigFont{12}{14.4}{\rmdefault}{\mddefault}{\updefault}{\color[rgb]{0,0,0}}}}}}
\put(3600,-1739){\makebox(0,0)[lb]{\smash{{\SetFigFont{12}{14.4}{\rmdefault}{\mddefault}{\updefault}{\color[rgb]{0,0,0}}}}}}
\put(3446,-3064){\makebox(0,0)[lb]{\smash{{\SetFigFont{12}{14.4}{\rmdefault}{\mddefault}{\updefault}{\color[rgb]{0,0,0}}}}}}
\put(5574,-3626){\makebox(0,0)[lb]{\smash{{\SetFigFont{12}{14.4}{\rmdefault}{\mddefault}{\updefault}{\color[rgb]{0,0,0}}}}}}
\put(2665,-3435){\makebox(0,0)[lb]{\smash{{\SetFigFont{12}{14.4}{\rmdefault}{\mddefault}{\updefault}{\color[rgb]{0,0,0}}}}}}
\end{picture} \caption{\small \sf The homothetic copy .
 \label{fig:norm4}}
\end{center}
\end{figure}


Now expand  by moving its center along the line passing
through  and , away from , and scale it appropriately so 
that its boundary continues to pass through , until it touches 
 too. Denote the center of the new placement as , and
the placement itself as . Let  be the 
corresponding homothetic copy of  centered at  and bounding
. See Figure \ref{fig:norm4}.


\begin{figure}[hbt]
\begin{center}
\begin{picture}(0,0)\includegraphics{norm5.pstex}\end{picture}\setlength{\unitlength}{2763sp}\begingroup\makeatletter\ifx\SetFigFont\undefined \gdef\SetFigFont#1#2#3#4#5{\reset@font\fontsize{#1}{#2pt}\fontfamily{#3}\fontseries{#4}\fontshape{#5}\selectfont}\fi\endgroup \begin{picture}(4749,3836)(1234,-3898)
\put(3676,-2086){\makebox(0,0)[lb]{\smash{{\SetFigFont{12}{14.4}{\rmdefault}{\mddefault}{\updefault}{\color[rgb]{0,0,0}}}}}}
\put(3664,-293){\makebox(0,0)[lb]{\smash{{\SetFigFont{12}{14.4}{\rmdefault}{\mddefault}{\updefault}{\color[rgb]{0,0,0}}}}}}
\put(3501,-720){\makebox(0,0)[lb]{\smash{{\SetFigFont{12}{14.4}{\rmdefault}{\mddefault}{\updefault}{\color[rgb]{0,0,0}}}}}}
\put(4852,-581){\makebox(0,0)[lb]{\smash{{\SetFigFont{12}{14.4}{\rmdefault}{\mddefault}{\updefault}{\color[rgb]{0,0,0}}}}}}
\put(1890,-913){\makebox(0,0)[lb]{\smash{{\SetFigFont{12}{14.4}{\rmdefault}{\mddefault}{\updefault}{\color[rgb]{0,0,0}}}}}}
\put(5026,-3811){\makebox(0,0)[lb]{\smash{{\SetFigFont{12}{14.4}{\rmdefault}{\mddefault}{\updefault}{\color[rgb]{0,0,0}}}}}}
\put(3512,-2641){\makebox(0,0)[lb]{\smash{{\SetFigFont{12}{14.4}{\rmdefault}{\mddefault}{\updefault}{\color[rgb]{0,0,0}}}}}}
\put(4285,-2403){\makebox(0,0)[lb]{\smash{{\SetFigFont{12}{14.4}{\rmdefault}{\mddefault}{\updefault}{\color[rgb]{0,0,0}}}}}}
\put(3670,-1580){\makebox(0,0)[lb]{\smash{{\SetFigFont{12}{14.4}{\rmdefault}{\mddefault}{\updefault}{\color[rgb]{0,0,0}}}}}}
\end{picture} \caption{\small \sf The homothetic copy .
 \label{fig:norm5}}
\end{center}
\end{figure}


We argue that   is empty.
 By Claim~\ref{Q2},  is also
a supporting line of  at .
Refer to Figure \ref{fig:norm6}.
We denote by  and  the intersections of the supporting line
 with  and , respectively.
 We denote by  the intersection of  and  that lies on the same side of  as .
The angle  is at most  since by Claim \ref{Q1}
the angle between  and the tangent to  at  is at most 
.
On the other hand the
 angle  is at least  since the angle
between  and  at  is at least 
. So it follows that
the segment  is fully contained in .
Since the ray  meets  (at ) before meeting , and the ray  meets  (at ) before meeting , it follows that  and  intersect at a point on a ray between  and .

\begin{figure}[hbt]
\begin{center}
\begin{picture}(0,0)\includegraphics{norm6.pstex}\end{picture}\setlength{\unitlength}{2763sp}\begingroup\makeatletter\ifx\SetFigFont\undefined \gdef\SetFigFont#1#2#3#4#5{\reset@font\fontsize{#1}{#2pt}\fontfamily{#3}\fontseries{#4}\fontshape{#5}\selectfont}\fi\endgroup \begin{picture}(5153,3642)(1487,-3756)
\put(3421,-309){\makebox(0,0)[lb]{\smash{{\SetFigFont{12}{14.4}{\rmdefault}{\mddefault}{\updefault}{\color[rgb]{0,0,0}}}}}}
\put(3495,-2510){\makebox(0,0)[lb]{\smash{{\SetFigFont{12}{14.4}{\rmdefault}{\mddefault}{\updefault}{\color[rgb]{0,0,0}}}}}}
\put(1506,-1415){\makebox(0,0)[lb]{\smash{{\SetFigFont{12}{14.4}{\rmdefault}{\mddefault}{\updefault}{\color[rgb]{0,0,0}}}}}}
\put(5557,-1105){\makebox(0,0)[lb]{\smash{{\SetFigFont{12}{14.4}{\rmdefault}{\mddefault}{\updefault}{\color[rgb]{0,0,0}}}}}}
\put(4704,-1514){\makebox(0,0)[lb]{\smash{{\SetFigFont{12}{14.4}{\rmdefault}{\mddefault}{\updefault}{\color[rgb]{0,0,0}}}}}}
\put(4457,-452){\makebox(0,0)[lb]{\smash{{\SetFigFont{12}{14.4}{\rmdefault}{\mddefault}{\updefault}{\color[rgb]{0,0,0}}}}}}
\put(3483,-733){\makebox(0,0)[lb]{\smash{{\SetFigFont{12}{14.4}{\rmdefault}{\mddefault}{\updefault}{\color[rgb]{0,0,0}}}}}}
\put(4517,-3395){\makebox(0,0)[lb]{\smash{{\SetFigFont{12}{14.4}{\rmdefault}{\mddefault}{\updefault}{\color[rgb]{0,0,0}}}}}}
\put(3852,-3340){\makebox(0,0)[lb]{\smash{{\SetFigFont{12}{14.4}{\rmdefault}{\mddefault}{\updefault}{\color[rgb]{0,0,0}}}}}}
\put(5795,-3430){\makebox(0,0)[lb]{\smash{{\SetFigFont{12}{14.4}{\rmdefault}{\mddefault}{\updefault}{\color[rgb]{0,0,0}}}}}}
\end{picture} \caption{\small \sf The segment  is fully contained in . The circles
 intersect at a point on a ray emanating from  between  and .
 \label{fig:norm6}}
\end{center}
\end{figure}


Let  denote a 
supporting line of  at . By Claim~\ref{Q3}, the angles
between  and the lines ,  differ by at most
.
Since each of the angles between  and 
the two tangents
 
and  is at least , it follows that
each of the angles between  and the two 
tangents  and   to  and ,
respectively, at , is at least .

Refer now to Figure \ref{fig:norm7}.
We denote by  the intersection of  and  distinct from ,
and we denote by  the intersections between  and
 , respectively. An argument analogous to the one given
before shows that  while 
. It follows that the segment 
 is fully contained in  and we have an intersection between 
 and  on a ray emanating from  between the ray from  to
 and the ray from  to .

\begin{figure}[hbt]
\begin{center}
\begin{picture}(0,0)\includegraphics{norm7.pstex}\end{picture}\setlength{\unitlength}{2763sp}\begingroup\makeatletter\ifx\SetFigFont\undefined \gdef\SetFigFont#1#2#3#4#5{\reset@font\fontsize{#1}{#2pt}\fontfamily{#3}\fontseries{#4}\fontshape{#5}\selectfont}\fi\endgroup \begin{picture}(5124,3577)(1211,-3680)
\put(3489,-298){\makebox(0,0)[lb]{\smash{{\SetFigFont{12}{14.4}{\rmdefault}{\mddefault}{\updefault}{\color[rgb]{0,0,0}}}}}}
\put(3483,-720){\makebox(0,0)[lb]{\smash{{\SetFigFont{12}{14.4}{\rmdefault}{\mddefault}{\updefault}{\color[rgb]{0,0,0}}}}}}
\put(5077,-1022){\makebox(0,0)[lb]{\smash{{\SetFigFont{12}{14.4}{\rmdefault}{\mddefault}{\updefault}{\color[rgb]{0,0,0}}}}}}
\put(5600,-1110){\makebox(0,0)[lb]{\smash{{\SetFigFont{12}{14.4}{\rmdefault}{\mddefault}{\updefault}{\color[rgb]{0,0,0}}}}}}
\put(5808,-3480){\makebox(0,0)[lb]{\smash{{\SetFigFont{12}{14.4}{\rmdefault}{\mddefault}{\updefault}{\color[rgb]{0,0,0}}}}}}
\put(3814,-3314){\makebox(0,0)[lb]{\smash{{\SetFigFont{12}{14.4}{\rmdefault}{\mddefault}{\updefault}{\color[rgb]{0,0,0}}}}}}
\put(3519,-2609){\makebox(0,0)[lb]{\smash{{\SetFigFont{12}{14.4}{\rmdefault}{\mddefault}{\updefault}{\color[rgb]{0,0,0}}}}}}
\put(6320,-1379){\makebox(0,0)[lb]{\smash{{\SetFigFont{12}{14.4}{\rmdefault}{\mddefault}{\updefault}{\color[rgb]{0,0,0}}}}}}
\put(2709,-1727){\makebox(0,0)[lb]{\smash{{\SetFigFont{12}{14.4}{\rmdefault}{\mddefault}{\updefault}{\color[rgb]{0,0,0}}}}}}
\put(1630,-1333){\makebox(0,0)[lb]{\smash{{\SetFigFont{12}{14.4}{\rmdefault}{\mddefault}{\updefault}{\color[rgb]{0,0,0}}}}}}
\put(5806,-2086){\makebox(0,0)[lb]{\smash{{\SetFigFont{12}{14.4}{\rmdefault}{\mddefault}{\updefault}{\color[rgb]{0,0,0}}}}}}
\end{picture} \caption{\small \sf The segment  is fully contained in . The circles
 intersect at a point on a ray emanating from  between  and .
 \label{fig:norm7}}
\end{center}
\end{figure}

Our argument about the position of the intersections between
 and  implies that the entire section of 
 between  and  is contained . Therefore
the portion of  to the right of the line through  and  (in the configuration depicted in the figures) is fully contained in .
A symmetric argument shows that the portion of 
 to the left of the line 
through  and  is fully contained in . Since  is empty we conclude that
 is empty.

 The emptiness of  implies that  and 
 are neighbors in the -Voronoi diagram, and that  lies on 
their common -Voronoi edge .


We thus obtain the following theorem.
\begin{theorem} 
Let , , and  be as above. Then (i) every -stable edge of the Euclidean Delaunay triangulation is an -stable edge of . (ii) Conversely, every -stable edge of  is also an -stable edge in the Euclidean norm.
\end{theorem}
Note that parts (i) and (ii) are generalizations of Lemmas \ref{Thm:LongEucPoly} and \ref{Thm:LongPolygEuc}, respectively (with weaker constants).
\begin{proof}
Part (i) follows directly from the preceding analysis. Indeed, let  be a -stable Delaunay edge, for , whose Voronoi counterpart has endpoints  and . Let  be the homothetic placement of , with center , that touches  and . We have shown that  has empty interior if the ray  lies between  and  and spans an angle of at least  with each of them. Assuming , such rays  form a cone of size , which, in turn, gives the first part of the theorem. 

Part (ii) follows from part (i) by repeating, almost verbatim, the proof of Lemma \ref{Thm:LongPolygEuc}.
\end{proof}

There are many interesting open problems that arise here. One of the main problems is to extend the class of sets  for which a near quadratic bound on the number of topological changes in , under algebraic motion of bounded degree of the points of , can be established.




\begin{thebibliography}{10}










\bibitem{KineticNeighbors}
P. K. Agarwal, H. Kaplan and M. Sharir, Kinetic and dynamic data
structures for closest pair and all nearest neighbors, \emph{ACM
Trans. Algorithms} 5 (1) (2008), Art.~4.

\bibitem{AWY}
P. K. Agarwal, Y. Wang and H. Yu,
A 2D kinetic triangulation with near-quadratic topological changes,
\textit{Discrete Comput. Geom.} 36 (2006), 573--592.



\bibitem{Amenta}
N. Amenta and M. Bern, 
Surface reconstruction by Voronoi filtering,
{\em Discrete Comput. Geom.}, 22 (1999), 481--504.

\bibitem{Crusts}
N. Amenta, M. W. Bern and D. Eppstein, The crust and beta-skeleton: combinatorial curve reconstruction, 
{\it Graphic. Models and Image Processing} 60 (2) (1998), 125--135. 





\bibitem{AK}
F. Aurenhammer and R. Klein,
Voronoi diagrams,
in {\it Handbook of Computational Geometry},
J.-R. Sack and J. Urrutia, Eds.,
Elsevier, Amsterdam, 2000,
pages 201--290.

\bibitem{bgh-dsmd-99}
J.~Basch, L.~J. Guibas and J.~Hershberger,
Data structures for mobile data,
{\em J. Algorithms} 31 (1) (1999), 1--28.

\bibitem{Chew}
L. P. Chew,
Near-quadratic bounds for the  Voronoi diagram of moving points,
{\em Comput. Geom. Theory Appl.}  7 (1997), 73--80.

\bibitem{CD}
L. P. Chew and R. L. Drysdale,
Voronoi diagrams based on convex distance functions,
{\em Proc. First Annu. ACM Sympos. Comput. Geom.}, 1985, pp.~235--244.



\bibitem{d-slsv-34}
B.~Delaunay,
Sur la sph{\`e}re vide. {A} la memoire de {Georges} {Voronoi},
{\em Izv. Akad. Nauk SSSR, Otdelenie Matematicheskih i Estestvennyh
Nauk} 7 (1934), 793--800.

\bibitem{TOPP}
E.~D.~Demaine, J.~S.~B.~Mitchell, and J.~O'Rourke,\\
The Open Problems Project,
\texttt{http://www.cs.smith.edu/\~{ }orourke/TOPP/}.

\bibitem{Ed2}
H. Edelsbrunner,
{\em Geometry and Topology for Mesh Generation},
Cambridge University Press, Cambride, 2001.

\bibitem{285869}
L.~J. Guibas,
Modeling motion,
In J.~E. Goodman and J.~O'Rourke, editors, {\em Handbook of Discrete
and Computational Geometry}. CRC Press, Inc., Boca Raton, FL, USA, second
edition, 2004, pages 1117--1134.

\bibitem{g-kdssar-98}
L.~J. Guibas,
Kinetic data structures --- a state of the art report,
In P.~K. Agarwal, L.~E. Kavraki and M.~Mason, editors, {\em Proc.
Workshop Algorithmic Found. Robot.}, pages 191--209. A. K. Peters, Wellesley,
MA, 1998.

\bibitem{gmr-vdmpp-92}
L.~J. Guibas, J.~S.~B. Mitchell and T.~Roos,
Voronoi diagrams of moving points in the plane,
{\em Proc. 17th Internat. Workshop Graph-Theoret. Concepts Comput.
Sci.}, volume 570 of {\em Lecture Notes Comput. Sci.}, pages 113--125.
Springer-Verlag, 1992.

\bibitem{IKLM}
C. Icking, R. Klein, N.-M. L\^{e} and L. Ma,
Convex distance functions in 3-space are different, {\it Fundam. Inform.} 22 (4) (1995), 331--352.

\bibitem{KRS}
H. Kaplan, N. Rubin and M. Sharir, 
A kinetic triangulation scheme for moving points in the plane, {\it Comput. Geom. Theory Appl.} 44 (2011), 191--205.

\bibitem{KLPS}
K. Kedem, R. Livne, J. Pach, and M. Sharir, On the union of Jordan regions and collision-free translational motion amidst polygonal obstacles, {\em Discrete Comput. Geom.} 1 (1986), 59--70.

\bibitem{Skeletons}
D. Kirkpatrick and J. D. Radke, A framework for computational morphology, \textit{Computational Geometry} (G. Toussaint, ed.), North-Holland (1985), 217--248.

\bibitem{LS}
D. Leven and M. Sharir,
Planning a purely translational motion for a convex object in
two--dimensional space using generalized Voronoi diagrams,
{\it Discrete Comput. Geom.} 2 (1987), 9--31.

\bibitem{Mehlhorn} 
K. Mehlhorn, \textit{Data Structures and Algorithms 1: Sorting and Searching}, Springer Verlag, Berlin 1984.

\bibitem{NR73} 
J. Nievergelt and E. M. Reingold, Binary search trees of bounded balance, {\it SIAM J. Comput.} 2 (1973),
33--43.

\bibitem{SA95}
M.~Sharir and P.~K. Agarwal,
{\em Davenport-Schinzel Sequences and Their Geometric Applications},
Cambridge University Press, New York, 1995.

\end{thebibliography}

\paragraph{Acknowledgements.}
{\small Pankaj Agarwal was supported by NSF under grants CNS-05-40347, CCF-06 -35000, CCF-09-40671
               and DEB-04-25465, by ARO grants
               W911NF-07-1-0376 and W911NF-08-1-0452, by an
               NIH grant 1P50-GM-08183-01, by a DOE grant
               OEG-P200A070505, and by Grant 2006/194 from the
                U.S.--Israel Binational Science Foundation.
    Leo Guibas was supported by NSF grants CCR-0204486,
    ITR-0086013, ITR-0205671, ARO grant DAAD19-03-1-0331, as well as by
    the Bio-X consortium at Stanford.
Haim Kaplan was partially supported by Grant 2006/204 from the U.S.--Israel
Binational Science Foundation, project number 2006204, and by Grants 975/06 and 822/10 from
the 
Israel Science Fund.
 Micha Sharir was supported by NSF Grants CCF-05-14079 and CCF-08-30272, 
    by Grant 338/09 from the Israel Science Fund,
    and by the Hermann Minkowski--MINERVA Center for Geometry at Tel
    Aviv University. Natan Rubin was supported by Grants 975/06 and 338/09 from the Israel Science Fund.}


\end{document}