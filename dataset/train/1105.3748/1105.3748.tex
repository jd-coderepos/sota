\documentclass[11pt]{article}


\usepackage{fullpage}
\usepackage{latexsym}
\usepackage{amsfonts,amssymb}
\usepackage{amstext}
\usepackage{graphicx}
\usepackage{color}
\usepackage{url,hyperref}
\usepackage{amsmath,xspace}
\usepackage{shadow,shadethm}









 \newtheorem{fact}{Fact}[section]
 \newtheorem{lemma}{Lemma}[section]
 \newtheorem{theorem}[lemma]{Theorem}
 \newtheorem{assumption}[lemma]{Assumption}
 \newtheorem{definition}[lemma]{Definition}
 \newtheorem{corollary}[lemma]{Corollary}
 \newtheorem{proposition}[lemma]{Proposition}
 \newtheorem{claim}[lemma]{Claim}
 \newtheorem{remark}[lemma]{Remark}
 \newtheorem{prob}{Problem}
 \newtheorem{conjecture}{Conjecture}
 \newtheorem{observation}{Observation}

\newenvironment{proof}{\vspace{-0.15in}\noindent{\bf Proof:}}{\hspace*{\fill}\par}
\newcommand{\proofname}{\textbf{Proof}}
\newenvironment{proofof}[1]{\smallskip\noindent{\bf Proof of #1:}}{\hspace*{\fill}\par}


\newcommand{\opt}{\textrm{\sc OPT}\xspace}
\newcommand{\etal}{et al.\ }
\newcommand{\eps}{\epsilon}
\newcommand{\Algorithm}[1]{{\texttt{\bf{#1}}}} 

\newcommand{\lp}{\texttt{LP}}
\newcommand{\len}{\mathfrak{\rho}}

\newcommand{\floor}[1]{\lfloor #1 \rfloor}
\newcommand{\ceil}[1]{\lceil #1 \rceil}

\newcommand{\R}{{\mathbb R}}
\newcommand{\tsty}{}

\newcommand{\dx}{{\sf dx}}
\newcommand{\dy}{{\sf dy}}
\newcommand{\alg}{\ensuremath{{\sf Alg}}\xspace}
\newcommand{\opthdf}{\ensuremath{{\sf Opt}_{HDF}}\xspace}
\newcommand{\bcp}{{\sf BCP}\xspace}
\newcommand{\alw}{{\sf ALW}\xspace}
\newcommand{\const}{\frac{2}{\epsilon}}
\newcommand{\constun}{\frac{4}{\epsilon}}
\newcommand{\dphidt}{{\frac{d \Phi_i(t)}{dt}}}
\newcommand{\dphidtside}{d \Phi_i(t)/dt}
\newcommand{\dphdt}{{\frac{d \Phi(t)}{dt}}}
\newcommand{\dadt}{G_A(t)}
\newcommand{\dodt}{G_{OPT}(t)}

\newcommand{\Xomit}[1]{ }



\newcounter{myLISTctr}
\newcommand{\initOneLiners}{\setlength{\itemsep}{0pt}
    \setlength{\parsep }{0pt}
    \setlength{\topsep }{0pt}
}
\newenvironment{OneLiners}[1][\ensuremath{\bullet}]
    {\begin{list}
        {#1}
        {\initOneLiners}}
    {\end{list}}






\begin{document}

\title{Scalably Scheduling Power-Heterogeneous Processors\footnote{A preliminary version of this paper appeared in ICALP 2010}}

\author{
 Anupam Gupta\thanks{ Computer Science Department, Carnegie Mellon
    University, Pittsburgh, PA 15213, USA. 
Supported in part by
     NSF award CCF-0729022 and an Alfred P.~Sloan
     Fellowship.}
 \and Ravishankar Krishnaswamy
\and Kirk Pruhs\thanks{Computer Science Department,
University of Pittsburgh, Pittsburgh, PA 15260, USA. 
Supported in part by
 NSF grants CNS-0325353, IIS-0534531, and CCF-0830558, and
 an IBM Faculty Award.}
 }

\date{}
\maketitle


\begin{abstract}
  We show that a natural online algorithm for scheduling jobs on a
  heterogeneous multiprocessor, with arbitrary power functions, is
  scalable for the objective function of weighted flow plus energy.
\end{abstract}

\section{Introduction}
\label{sec:introduction}



Many prominent computer architects believe that architectures consisting
of heterogeneous processors/cores, such as the STI Cell processor, will
be the dominant architectural design in the
future~\cite{Bower2008,Kumar2004,Kumar2006,Merritt2008,Tomer2006}.  The
main advantage of a heterogeneous architecture, relative to an
architecture of identical processors, is that it allows for the
inclusion of processors whose design is specialized for particular types
of jobs, and for jobs to be assigned to a processor best suited for
that job.  Most notably, it is envisioned that these heterogeneous
architectures will consist of a small number of high-power
high-performance processors for critical jobs, and a larger number of
lower-power lower-performance processors for less critical jobs.
Naturally, the lower-power processors would be more energy efficient in
terms of the computation performed per unit of energy expended, and
would generate less heat per unit of computation.  For a given area and
power budget, heterogeneous designs can give significantly better
performance for standard workloads~\cite{Bower2008,Merritt2008};
Emulations in \cite{Kumar2006} suggest a figure of 40\% better
performance, and emulations in \cite{Tomer2006} suggest a figure of
67\% better performance.
Moreover, even processors that were designed to be homogeneous, are
increasingly likely to be heterogeneous at run time~\cite{Bower2008}:
the dominant underlying cause is the increasing variability in the
fabrication process as the feature size is scaled down (although run
time faults will also play a role).  Since manufacturing yields would be
unacceptably low if every processor/core was required to be perfect, and
since there would be significant performance loss from derating the
entire chip to the functioning of the least functional processor (which
is what would be required in order to attain processor homogeneity),
some processor heterogeneity seems inevitable in chips with many processors/cores.

The position paper~\cite{Bower2008} identifies three fundamental
challenges in scheduling heterogeneous multiprocessors: (1)~the OS must
discover the status of each processor, (2)~the OS must discover the
resource demand of each job, and (3)~given this information about
processors and jobs, the OS must match jobs to processors as well as
possible. In this paper, we address this third fundamental challenge.
In particular, we assume that different jobs are of differing
importance, and we study how to assign these jobs to processors of
varying power and varying energy efficiency, so as to achieve the best
possible trade-off between energy and performance.

Formally, we assume that a collection of jobs arrive in an online
fashion over time. When a job  arrives in the system, the system is
able to discover a \emph{size} , as well as a
\emph{importance/weight} , for that job. The
importance  specifies an upper bound on the amount of energy that
the system is allowed to invest in running  to reduce 's flow by
one unit of time (assuming that this energy investment in  doesn't
decrease the flow of other jobs)---hence jobs with high weight are more
important, since higher investments of energy are permissible to justify
a fixed reduction in flow. Furthermore, we assume that the system knows the
allowable speeds for each processor, and the system also knows the
power used when each processor is run at its
set of allowable speeds.  We make no real restrictions on the allowable speeds,
or on the power used for these speeds.\footnote{So the processors may or
  may not be speed scalable, the speeds may be continuous or discrete or
  a mixture, the static power may or may not be negligible, the dynamic
  power may or may not satisfy the cube root rule, etc.}  The online
scheduler has three component policies:
\begin{description}
\item[Job Selection:] Determines which job to run on each processor at any time.
\item[Speed Scaling:]Determines the speed of each processor at each time.
\item[Assignment:] When a new job arrives, it determines the processor
to which this new job is assigned.
\end{description}


The objective we consider is that of \emph{weighted flow plus energy}.
The rationale for this objective function is that the optimal schedule
under this objective gives the best possible weighted flow for the
energy invested, and increasing the energy investment will not lead to a
corresponding reduction in weighted flow (intuitively, it is not
possible to speed up a collection of jobs with an investment of energy
proportional to these jobs' importance).

We consider the following natural online algorithm that essentially adopts the
job selection and speed scaling algorithms from the uniprocessor
algorithm in \cite{BCP}, and then greedily assigns the jobs based on
these policies.
\begin{shadebox}
\begin{description}
\item[Job Selection:] Highest Density First (HDF)
\item[Speed Scaling:] The speed is set so that the power is the fractional
  weight of the unfinished jobs.
\item[Assignment:] A new job is assigned to the processor
  that results in the least increase in the projected future weighted
  flow, assuming the adopted speed scaling and job selection policies, and
  ignoring the possibility of jobs arriving in the future.
\end{description}
\end{shadebox}
Our main result is then:
\begin{theorem}
  \label{thm:main1}
  This online algorithm is scalable for scheduling jobs on a
  heterogeneous multiprocessor with arbitrary power functions to
  minimize the objective function of weighted flow plus energy.
\end{theorem}
In this context, \emph{scalable} means that if the adversary can run
processor  at speed  and power , the online algorithm is
allowed to run the processor at speed  and power ,
and then for all inputs, the online cost is bounded by 
times the optimal cost.  Intuitively,
a scalable algorithm can handle almost the same load as
optimal;  for further elaboration, see~\cite{PST,Pruhs07}.  Theorem
\ref{thm:main1} extends theorems showing similar results for weighted
flow plus energy on a uniprocessor~\cite{BCP,Lachlan2009}, and for
weighted flow on a multiprocessor without power
considerations~\cite{Chadha2009}.  As scheduling on identical processors
with the objective of total flow, and scheduling on a uniprocessor with
the objective of weighted flow, are special cases of our problem, constant
competitiveness is not possible without some sort of resource
augmentation~\cite{LR,BC09}.


Our analysis is an amortized local-competitiveness argument. As is
usually the case with such arguments, the main technical hurdle is to
discover the ``right'' potential function. The most natural straw-man potential function to try is the sum over all
processors of the single processor potential function used
in~\cite{BCP}.  While one can prove constant competitiveness with this
potential in some special cases (e.g. where for each processor the
allowable speeds are the non-negative reals, and the power satisfies the
cube-root rule), one can not prove constant competitiveness for general
power functions with this potential function.  The reason for this is
that the uniprocessor potential function from~\cite{BCP} is not
sufficiently accurate.  Specifically, one can construct
configurations where the adversary has finished all jobs, and where the
potential is much higher than the remaining online cost. This did not
mess up the analysis in \cite{BCP} because to finish all these jobs by
this time the adversary would have had to run very fast in the past,
wasting a lot of energy, which could then be used to pay for this
unnecessarily high potential.  But since we consider multiple
processors, the adversary may have no jobs left on a particular
processor simply because it assigned these jobs to a different
processor, and there may not be a corresponding unnecessarily high
adversarial cost that can be used to pay for this unnecessarily high
potential. 

Thus, the main technical contribution in this paper is a seemingly
more accurate potential function expressing the additional cost required
to finish one collection of jobs compared to another collection of
jobs. Our potential function is arguably more transparent than the one used in~\cite{BCP},
and we expect that
this potential function will find future application in
the analysis of other power management algorithms.


In section~\ref{dsec:unweighted-flow}, we show that a similar online algorithm is
-competitive with -speedup for
\emph{unweighted} flow plus energy.
We also remark that when the power functions  are
restricted to be of the form , our algorithms give a
-competitive algorithm (with no resource augmentation needed) for the problem of minimizing weighted flow plus energy, and an
-competitive algorithm for minimizing the unweighted flow plus energy,
where .





\subsection{Related Results}
\label{sec:related-results}

Let us first consider previous work for the case of a single processor,
with unbounded speed, and a polynomially bounded power function .  \cite{PUW} gave an efficient offline algorithm to find the
schedule that minimizes average flow subject to a constraint on the
amount of energy used, in the case that jobs have unit work. \cite{AF}
introduced the objective of flow plus energy and gave a constant
competitive algorithm for this objective in the case of unit
work jobs. \cite{BPS} gave a constant competitive algorithm for the
objective of weighted flow plus energy.  The competitive ratio was
improved by \cite{LLTW08} for the unweighted case using a potential
function specifically tailored to integer flow. \cite{BCLL08} extended
the results of \cite{BPS} to the bounded speed model,
and~\cite{STACS2009} gave a \emph{nonclairvoyant} algorithm that is
-competitive.

Still for a single processor, dropping the assumptions of unbounded
speed and polynomially-bounded power functions, \cite{BCP} gave a
-competitive algorithm for the objective of unweighted flow plus
energy, and a -competitive algorithm for fractional weighted flow
plus energy, both in the uniprocessor case for a large class of power
functions.  The former analysis was subsequently improved
by~\cite{Lachlan2009} to show -competitiveness, along with a matching
lower bound.

Now for multiple processors:
\cite{Lam08} considered the setting of multiple \emph{homogeneous}
processors, where the allowable speeds range between zero and some upper
bound, and the power function is polynomial in this range.  They gave an
algorithm that uses a variant of round-robin for the assignment
policy, and job selection and speed scaling policies from
\cite{BPS}, and showed that this algorithm is scalable for the objective 
of (unweighted) flow plus energy.
Subsequently, \cite{GNS09} showed that a randomized machine selection
algorithm is scalable for weighted flow plus energy (and even more
general objective functions) in the setting of polynomial power 
functions. Both these algorithms provide non-migratory schedules and 
compare their costs with optimal solutions which could even be migratory. 
In comparison, as mentioned above, for the case of polynomial power 
functions, our techniques can give a deterministic constant-competitive 
online algorithm for non-migratory weighted flow time plus energy.
(Details appear in the final version.)

In non-power-aware settings, the paper most relevant to this work is
that of~\cite{Chadha2009}, which gives a scalable online algorithm for
minimizing weighted flow on unrelated processors. Their setting is even
more demanding, since they allow the processing requirement of the job
to be processor dependent (which captures a type of heterogeneity that
is orthogonal to the performance energy-efficiency heterogeneity that we
consider in this paper). Our algorithm is based on the same general
intuition as theirs: they assign each new job to the processor that
would result in the least increment in future weighted flow (assuming
HDF is used for job selection), and show that this online algorithm is
scalable using an amortized local competitiveness argument.  However, it
is unclear how to directly extend their potential function to our
power-aware setting; we  had success only in the case that each processor
had allowable speed-power combinations lying in .





\subsection{Preliminaries}
\label{sec:preliminaries}

\subsubsection{Scheduling Basics.}
We consider only non-migratory schedules, which means that no job can
ever run on one processor, and later run on some other processor. In
general, migration is undesirable as the overhead can be significant. We
assume that preemption is allowed, that is, that jobs may be suspended,
and restarted later from the point of suspension.  It is clear that if
preemption is not allowed, bounded competitiveness is not
obtainable.  The speed is the rate at which work is completed; a job 
with size  run at a constant speed  completes in 
seconds. A job is completed when all of its work has been  processed.
The flow of a job is the completion time of the job minus the release time of the job.
The weighted flow of a job is the weight of the job times the flow of the job.
For a , let  be the remaining unprocessed
work on job  at time . The  fractional weight of job  at this
time is .
The fractional weighted flow of a job is the integral over times
between the job's release time
and its completion time of its fractional weight at that time.
The density of a job is its weight divided by its size.
The job selection policy Highest Density First (HDF) always runs the job of highest density.
The inverse density of a job is its size divided by its weight.






\subsubsection{Power Functions.}
The power function for processor  is denoted by , and
specifies the power used when processor is run at speed . We
essentially allow any reasonable power function.  However, we do require
the following minimal conditions on each power function, which we adopt
from~\cite{BCP}.  We assume that the allowable speeds are a countable
collection of disjoint subintervals of . We assume that all
the intervals, except possibly the rightmost interval, are closed on
both ends. The rightmost interval may be open on the right if the power
 approaches infinity as the speed  approaches the rightmost
endpoint of that interval.  We assume that  is non-negative, and
 is continuous and differentiable on all but countably many points.
We assume that either there is a maximum allowable speed , or that
the limit inferior of  as  approaches infinity is not zero
(if this condition doesn't hold then, then the optimal speed scaling
policy is to run at infinite speed). Using transformations specified
in~\cite{BCP}, we may assume without loss of generality that the power
functions satisfy the following properties:
 is continuous and differentiable,
,  is strictly increasing,
 is strictly convex, and   is unbounded.
We use  to denote ; i.e.,  gives us the speed
that we can run processor  at, if we specify a limit of .


\subsubsection{Local Competitiveness and Potential Functions.}
Finally, let us quickly review amortized local competitiveness analysis
on a single processor.
Consider an objective . Let  be the increase in the objective
in the schedule for algorithm A at time .  So when  is fractional weighted flow
plus energy,  is , where  is the power
for A at time  and  is the fractional weight of the unfinished jobs for A at
time .
Let
OPT be the offline adversary that optimizes .
A is locally -competitive if for all times , if
.
To prove A is -competitive using an amortized local competitiveness argument,
it suffices to give a potential function
 such that the following conditions hold (see for example~\cite{Pruhs07}).

\begin{OneLiners}
\item[{\bf Boundary condition:}]  is zero before any job is
  released and  is non-negative after all jobs are finished.
\item[{\bf Completion condition:}]
 does not increase due to completions by either A or OPT.
\item[{\bf Arrival condition:}]
 does not increase more than  due to job arrivals.

\item[{\bf Running condition:}]
  At any time  when no job arrives or is completed,
  
\end{OneLiners}
The sufficiency of these conditions for proving -competitiveness
follows from integrating them over time.

\section{Weighted Flow}
\label{sec:weighted-flow-time}

Our goal in this section is to prove Theorem \ref{thm:main1}.
We first show that the online algorithm is -speed
-competitive for the objective of
\emph{fractional} weighted flow plus energy. Theorem \ref{thm:main1}
then follows since HDF is
-speed
-competitive for fixed processor speeds~\cite{BLMP1}
for the objective of (integer) weighted flow.

Let \opt be some
optimal schedule minimizing fractional weighted flow.
Let 
denote the total fractional weight of jobs in processor
's queue that have an inverse density of at least .
Let  be
the total fractional weight of unfinished jobs in the queue.
Let  be
the total fractional weight of unfinished jobs in all queues.
Let
,
, and
 be similarly defined for \opt.
When the time instant being considered is clear, we drop the superscript of  from all variables.



We assume that once \opt
has assigned a job to some processor, it runs the \bcp
algorithm~\cite{BCP} for job selection and speed scaling---i.e., it sets
the speed of the  processor to , and hence
the  processor uses power ,
and uses HDF for job selection.
We can make such an assumption because the results of~\cite{BCP} show that the
fractional weighted flow plus energy of the schedule output by this algorithm is within a factor of two of
optimal.  Therefore, the only real difference between \opt and the online algorithm is the
assignment policy.





\subsection{The Assignment Policy}

To better understand the online algorithm's assignment policy,
define the ``shadow potential'' for processor  at time  to be

The shadow potential captures (up to a constant factor) the total fractional weighted flow to serve the current set of jobs if no jobs arrive in the
future.
Based on this, the online algorithm's assignment policy can alternatively be described as follows:


\medskip \noindent {\bf Assignment Policy.} When a new job with size  and weight 
arrives at time , the assignment policy assigns it
to a processor which would cause the smallest increase in the
shadow potential; i.e. a processor minimizing






\subsection{Amortized Local Competitiveness Analysis}




We apply a local competitiveness argument as described in subsection \ref{sec:preliminaries}.
Because the online algorithm is using the BCP algorithm on each processor, the power
for the online algorithm is
. Thus . Similarly, since \opt is using BCP on
each processor .

\subsubsection{Defining the potential function}
For processor , define the potential

Here .
The global potential is then defined to be .
Firstly, we observe that the function  is
increasing and subadditive.
Then, the following lemma
will be useful subsequently, the proof of which will appear in the full version of the paper.



\begin{lemma}
  \label{lem:concave-arrival}
  Let  be any increasing subadditive function with , and .
  Then,
  
\end{lemma}

That the boundary and completion conditions are satisfied are obvious.
In Lemma \ref{lem:arrival-wtd} we prove that the arrival condition holds,
and in Lemma \ref{lem:running-wtd} we prove that the running condition holds.




\begin{lemma}
  \label{lem:arrival-wtd}
  The arrival condition holds with .
\end{lemma}
\begin{proof}
Consider a new job  with processing time , weight  and
inverse density , which the algorithm assigns to
processor~ while the optimal solution assigns it to processor .
Observe that  is
the increase in \opt's fractional weighted flow due to this new job .
Thus our goal is to prove that the increase in the potential due to
job 's arrival is at most this amount.
The change in the potential  is:

Now, since  is an increasing function we have that

and hence the change of potential can be bounded by

Since we assigned the job to processor~, we know that

Therefore, the change in potential is at most

Applying Lemma~\ref{lem:concave-arrival}, we get:

\end{proof}


\begin{lemma}
  \label{lem:running-wtd}
The running condition holds with constant .
\end{lemma}


\begin{proof}
Let us consider an infinitesimally small interval  during
which no jobs arrive and analyze the change in the potential .
Since , we can do this on a per-processor
basis. Fix a single processor , and time .
Let , and
.
Let  and  denote the inverse densities of the jobs being
executed on processor  by the algorithm and optimal solution
respectively (which are the densest jobs in their respective queues,
since both run HDF). Define  and . Since we assumed that \opt uses the \bcp algorithm on each
processor,  \opt runs processor  at speed .
Since the online algorithm is also using \bcp, but has
-speed augmentation, the online algorithms
runs the processor at speed
. Hence the
fractional weight of the job the online algorithm works on decreases at
a rate of .  Therefore, the quantity
 drops by  for .
Likewise,  drops by  for  due to
the optimal algorithm working on its densest job. We consider
several different cases based on the values of , and  and establish bounds on ;
Recall the definition of 
from equation~(\ref{eq:pot-wf}):



\medskip \noindent {\bf Case (1): }:
The only
possible increase in potential function occurs due to the decrease in
, which happens for values of .
But for 's in this range,  and .
Thus the inner integral is empty, resulting in no increase in potential.
The running condition then holds since .

\medskip \noindent {\bf Case (2): }:
To quantify the change in potential due to the
online algorithm working, observe that for any , the
inner integral of  decreases by

Here, we have used the fact that  is infinitisemally small to get the above equality. Hence, the total drop in  due to the online algorithm's
processing is

Here, the first inequality holds because  is a non-decreasing function, and for all , we have  and  and hence .


Now to quantify the increase in the potential due to the optimal algorithm
working: observe that for , the inner integral of
 increases by at most

Again notice that we have used that fact that here  is an infinitesimal period of time that in the limit is zero. Hence the total increase in  due to the optimal algorithm's processing
is at most

Again here, the first inequality holds because  is a non-decreasing function, and for all , we have  and  and hence .

Putting the two together, the overall increase in  can be
bounded by

It is now easy to verify that by plugging this bound on  into the
running condition that one gets a valid inequality.

\medskip \noindent {\bf Case (3): }: In this case,
let us just consider the increase due to \opt working. The inner
integral in the potential function starts off from zero (since ) and potentially (in the worst case) could increase to
 (since  drops by  and  cannot
increase). However, since  is a monotone non-decreasing
function, this is at most

Therefore, the total increase in the potential  can be
bounded by

It is now easy to verify that by plugging this bound on 
into the
running condition,
and using the fact that ,
one gets a valid inequality.
\end{proof}


\section{Algorithm for Unweighted Flow}
\label{dsec:unweighted-flow}

In this section, we give an immediate assignment based scheduling policy
and show that it is -competitive against a non-migratory
adversary for the objective of \emph{unweighted} flow plus energy,
assuming the online algorithm has resource augmentation of
 in speed. Note that this result has a better
competitiveness than the result for weighted flow from
Section~\ref{sec:weighted-flow-time}, but holds only for the unweighted case.

We begin by giving intuition behind our algorithm, which is again
similar to that for the weighted case. Let \opt be some optimal
schedule. 
However, for the rest of the section, we assume that on a single machine, the optimal scheduling algorithm for minimizing sum of flow times plus energy on a single machine
is that of Andrew et al.\cite{Lachlan2009} which sets the power at any time to be  when there are  unfinished jobs, and processes jobs according to SRPT.
Since we know that this \alw algorithm~\cite{Lachlan2009} is
-competitive against the optimal schedule on a single processor, we
will imagine that, once \opt has assigned a job to some processor, it
uses the \alw algorithm on each processor. Likewise, once our assignment
policy assigns a job to some processor, our algorithm also runs the \alw algorithm on each processor.
Therefore, just like the weighted case, the crux of our algorithm is in designing a good assignment policy,
and arguing that it is -competitive even though our algorithm and
\opt may schedule a new job on different processors with completely
different power functions.

\subsection{Algorithm}
\label{dsec:algorithm-uf}

Our algorithm works as follows: Each processor maintains a queue of jobs
that have currently been assigned to it.  At some time instant , for
any processor , let  denote the number of jobs in processor
's queue that have a remaining processing time of at least . Let
 denote the total number of unfinished jobs in the queue. Also,
let us define the \emph{shadow potential} for processor  at this time
 as

Note that the shadow potential
 is the total future cost of the online
algorithm (up to a constant factor) assuming no jobs arrive after this
time instant, and the online algorithm runs the \alw algorithm on all
processors (i.e., the job selection is SRPT, and the processor is run at
a speed of ). Now our algorithm is the following:
\begin{shadebox}
  \noindent When a new job arrives, the assignment policy assigns it to a
  processor which would cause the smallest increase in the ``shadow
  potential''; i.e., a processor minimizing
  
  The job selection on each processor is SRPT (Shortest Remaining
  Processing Time), and we set the power of processor  at time  to .  Once the job is
  assigned to a processor, it is never migrated.
\end{shadebox}



\subsection{The Amortized Local-Competitive Analysis}
\label{dsec:analysis-uf}

We again employ a potential function based analysis, similar to the one in Section~\ref{sec:weighted-flow-time}.


\subsubsection{The Potential Function.}

We now describe our potential function .  For time  and
processor , recall the definitions  and  given above;
analogously define  as the number of unfinished jobs assigned
to processor  by the optimal solution at time , and  to be the
number of these jobs with remaining processing time at least .
Henceforth, we will drop the superscript  from these terms whenever the time instant  is clear from the context.

Now, we define the
global potential function to be , where
 is the potential for processor  defined as:

 Recall that , and .
Notice that if the optimal solution has no jobs remaining on processor  at time
, we get  is (within a constant off) simply .


\subsubsection{Proving the Arrival Condition.}

We now show that the increase in the potential  is bounded (up to
a constant factor) by the increase in the future optimal cost when a new
job arrives. Suppose a new job of size  arrives at time , and suppose
 the online algorithm assigns it
to processor~ while the optimal solution assigns it to processor
. Then  increases since  goes up by  for all
,  could decrease due to  dropping by
 for all , and  (for ) does
not change.

Let us first assume that  for all  and for ; we will show below how to remove this
assumption. Under this assumption, the total change in potential 
is

But since  is increasing this is less than

By the greedy choice of processor  (instead of ), this is less
than

Now, since  is subadditive this is less than , which in
turn is (within a factor of ) precisely the \emph{increase in
  the future cost} incurred by the optimal solution, since we had
assumed that \opt also runs the \alw algorithm on its processors.

Now suppose  for some . There
is no increase in the inner sum of  for such values of , and
hence we can trivially upper bound this zero increase by . And to
discharge the assumption for processor , note that if  for some , there is no decrease in the inner sum of
 for this value of , but in this case we can simply use
 for such values of .
Therefore, we get the same bound of  on the increase in all
cases, thus proving the following lemma.
\begin{lemma}
  \label{dlem:arrival} The arrival condition holds for the unweighted
  case with .
\end{lemma}
\subsubsection{Proving the Running Condition.}
In this section, our goal is to analyze the change in  in an
infinitesimally small time interval  and compare 
to  and . We do this on a per-processor basis; let
us focus on processor  at time . Recall (after dropping the  superscripts) the definitions of
 and  from above, and define
. Finally, let  and 
denote the remaining sizes of the jobs being worked on by the algorithm
and optimal solution respectively at time ; recall that both of them
use SRPT for job selection.  Define  and  to be the (unaugmented) speeds of processor 
according to the online algorithm and the optimal algorithm
respectively---though, since we assume resource augmentation, our
processor runs at speed .  Hence
 drops by  for 
and  drops by  for  for the
optimal algorithm. Let  and
 denote these two intervals. Let us consider
some cases:



\medskip \noindent {\bf Case (1): }: The increase in
potential function may occur due to  dropping by  in . However, since , it follows that
 for all ;
the equality follows from \opt using SRPT.  Consequently, even with
 dropping by ,  and there
is no increase in potential, or equivalently .
Hence, in this case, .

\medskip \noindent {\bf Case (2a): , and }: For , the inner summation of  drops by , since
 decreases by~. Moreover,  and
, because both \alg and \opt run
SRPT, and we're considering .  For , the
inner summation of  increases by .  However, we have
 and  because , and  because \opt runs SRPT.  Therefore the
increase is at most 
for all . Combining these two, we get

where we repeatedly use that  is a non-decreasing function.
This implies that .

\medskip \noindent {\bf Case (2b): , and }: In this case, for , the inner summation of 
increases by . Also, we have  and
, because  and both algorithms
run SRPT. Therefore the overall increase in potential function can be
bounded by . Moreover, for , the inner summation of 
drops by . Also,  and ,
because  and there was a job of remaining size  among
the optimal solution's active jobs. Thus the potential function drops by
at least , since  is a non-decreasing function.
Combining these terms,

In the above, we have used the fact that , and consequently, . Therefore, in this case too we get .

\medskip \noindent {\bf Case (2c): , and }: Since , and  is an increasing
function,  and thus . For  in the
interval , the term  drops by  \emph{and} the term
 drops by , and therefore there is no change in
. For , the inner
summation for  drops by . Also,  and
, and the decrease in potential function is . But the analysis in Case~(2a) implies that  in this case as well.

Summing over all , we
get
\begin{lemma}
  \label{dlem:running}
  The running condition holds for the unweighted case with constant .
  At any time  when there are no job arrivals, we have
\end{lemma}
Combining Lemmas~\ref{dlem:arrival} and~\ref{dlem:running} with the
standard potential function argument indicated in
Section~\ref{sec:preliminaries}, we get the following theorem.
\begin{theorem}
  \label{dthm:unwtd}
  There is a -speed -competitive
  immediate-assignment algorithm to minimize the total flow plus energy on
  heterogeneous processors with arbitrary power functions.
\end{theorem}


\bigskip
\noindent
{\bf Acknowledgments:} We thank Sangyeun Cho and Bruce Childers for helpful discussions
about heterogeneous multicore processors, and also Srivatsan Narayanan for several useful discussions.


\bibliographystyle{plain}
\bibliography{icalp}


\appendix

\section{Estimating the Future Cost of BCP}
\label{sec:cost-estimate}
Suppose we have a set of  jobs, with weight  and processing time
 for job  () such that . Let  denote the inverse density of a job .
We now explain how we get the estimate of the future cost of this configuration
when we run the BCP algorithm, i.e. HDF at a speed of , where  is the total fractional weight
of unfinished jobs at time .

Firstly, observe that by virtue of our algorithm running HDF, it schedules job  followed by job , etc.
Also, as long as the algorithm is running job , it runs the processor at speed
, where  and  is
the fractional weight of job  remaining.
Secondly, since our algorithm always uses power equal to the fractional weight remaining, the rate of increase of the objective function at any time  is simply .
Therefore, the following equations immediately follow:

That is, the total cost incurred while job  is being scheduled is

Similarly, while any job  is being scheduled, we can use the same arguments as above to show that the total fractional flow incurred is
 
 Summing over , the total fractional flow incurred by our algorithm
 is
 
Rearranging the terms, it is not hard to see (given ) that this is equal to

where  is the total weight of jobs with inverse density at least .

\section{Subadditivity of x/Q(x)}
\label{sec:subadditive}
Let  be any concave function such that  and  for all , and let . Then the following facts are true about .
\begin{enumerate}
\item[(a)]  is non-decreasing. That is,  for all .
\item[(b)]  is subadditive. That is,  for all 
\end{enumerate}

To see why the first is true, consider  and  for some . Then,
showing (a) is equivalent to showing

But this reduces to showing  which is true because  is a concave function. To prove the second property, we first observe that the function  is convex. This is because the second derivative of  is

which is always non-negative for all , since  is non-negative and  is non-positive for all . Therefore, since  is convex, it holds for any , , and ,  that

Plugging in  and , we get

which implies

But since  is non-decreasing, we have  and hence



\section{Missing Proofs}
\label{sec:missing-proofs}

\begin{proofof}{Lemma~\ref{lem:concave-arrival}}
We first show that
 and then argue that  because  is subadditive.
To this end, we consider several cases and prove the lemma.
Suppose  is such
that : in this case we can discard the
 operators on all the limits to get

Here, the final inequality follows because  is a subadditive function.
On the other hand, suppose it is the case that , then both limits  and  are zero, and therefore we only need to bound , which can be done as follows:

Finally, if  for some ,
we first observe that  simplifies to . Therefore, we are interested in bounding

Here again, in the second to last inequality, we used the fact that  is subadditive and therefore , for all values of ; hence we get .

\medskip \noindent To complete the proof, we need to show that . To see this, consider the following sequence of steps:

For any , since  is subadditive, we have

Integrating both sides from  to  we get

which is, by variable renaming, equivalent to

But since  is non-decreasing, we have  and therefore

which is what we want.
\end{proofof}



\end{document}
