








\documentclass[twocolumn]{autart}
\usepackage{bbm}
\usepackage{}    

\usepackage{graphicx}          \usepackage{subfigure}
\usepackage{cite}
\usepackage{amsmath}
\usepackage{amssymb}
\usepackage{mathrsfs}
\usepackage{algorithm}
\usepackage{algorithmicx}
\usepackage{arydshln} \usepackage{myhyphen} \usepackage{color}
\usepackage{changebar}
\usepackage{booktabs}


\DeclareMathOperator*{\argmin}{\arg\,\min}


\lefthyphenmin = 2
\righthyphenmin = 4



\begin{document}


\begin{frontmatter}

\title{Data-Driven Robust Receding Horizon Fault Estimation \thanksref{footnoteinfo}} 
\thanks[footnoteinfo]{This paper was not presented at any IFAC
meeting. Corresponding author Yiming Wan. Tel.: +31152787019;
Fax: +31152786679.}

\author[DCSC]{Yiming Wan}\ead{y.wan@tudelft.nl},    \author[DCSC]{Tamas Keviczky}\ead{t.keviczky@tudelft.nl},               \author[DCSC]{Michel Verhaegen}\ead{m.verhaegen@tudelft.nl},  \author[Linkoping]{Fredrik Gustafsson}\ead{fredrik.gustafsson@liu.se}


\address[DCSC]{Delft Center for Systems and Control, Delft University of Technology, Delft, 2628 CD, The Netherlands}  \address[Linkoping]{Department of Electrical Engineering, Link\"{o}ping University, SE-581 83 Link\"{o}ping, Sweden}             

\begin{keyword}                           Data-driven methods; fault estimation; receding horizon estimation; parameter uncertainty.               \end{keyword}                             

\begin{abstract}                          This paper presents a data-driven receding horizon fault estimation method for additive actuator and sensor faults in unknown linear time-invariant systems, with enhanced robustness to stochastic identification errors. State-of-the-art methods construct fault estimators with identified state-space models or Markov parameters, but they do not compensate for identification errors. Motivated by this limitation, we first propose a receding horizon fault estimator parameterized by predictor Markov parameters. This estimator provides (asymptotically) unbiased fault estimates as long as the subsystem from faults to outputs has no unstable transmission zeros. When the identified Markov parameters are used to construct the above fault estimator, zero-mean stochastic identification errors appear as model uncertainty multiplied with unknown fault signals and online system inputs/outputs (I/O). Based on this fault estimation error analysis, we formulate a mixed-norm problem for the offline robust design that regards online I/O data as unknown. An alternative online mixed-norm problem is also proposed that can further reduce estimation errors when the online I/O data have large amplitudes, at the cost of increased computational burden. Based on a geometrical interpretation of the two proposed mixed-norm problems, systematic methods to tune the user-defined parameters therein are given to achieve desired performance trade-offs. Simulation examples illustrate the benefits of our proposed methods compared to recent literature.
\end{abstract}

\end{frontmatter}

\section{Introduction}
Model-based fault diagnosis techniques for linear dynamic systems have been well established during the past two decades \cite{Blanke2006, ChenPatton1999, Ding2013book, Gust2001}. Recently, the model-based receding horizon approach has received attention because it provides a flexible framework to  enhance robustness of passive fault diagnosis \cite{Wan2013, Zhang2014} and to enable optimal input design in active fault diagnosis \cite{Raim2013a, Raim2013b, Siman2005}. However, an explicit and accurate system model is often unknown in practice. In such situations, a conventional approach first identifies the system model from system I/O data, and then designs the model-based fault diagnosis system under various performance criteria \cite{Simani2003, Patward2005, Manuja2009}. Without explicitly identifying a system model, recent research efforts investigate data-driven approaches to construct a fault diagnosis system utilizing the link between system identification and the model-based fault diagnosis methods \cite{Russell2000, Ding2014book, Ding2014JPC}. These recent data-driven approaches simplify the design procedure by skipping the realization of an explicit system model, while at the same time allow developing systematic methods to address the same fault diagnosis performance criteria as the existing model-based approaches.


Most recent data-driven fault diagnosis approaches for unknown linear dynamic systems can be classified into two categories. The first category, e.g., \cite{QinLi2001} and \cite{Ding2009JPC, Ding2014JPC}, identifies a projection matrix known as parity space/vectors for residual generation, by exploiting the subspace identification method based on principal component analysis (SIM-PCA) \cite{HuangDing2005}.
However, as pointed out in \cite{Dong2012a}, a model reduction step is needed to determine the projection matrix, hence leads to the nonlinear dependence of the generated residuals on the identification errors. Therefore it is difficult to guarantee the robustness of such data-driven methods to the identification errors.


The second category of data-driven fault diagnosis methods, e.g., \cite{Dong2009thesis}, utilizes the Markov parameters (or impulse response parameters) which can be obtained in the first step of the predictor based subspace identification (PBSID) technique \cite{Chiuso2007a, Veen2012}. It constructs residual generators parameterized by the predictor Markov parameters. The main advantage of this method is that the residual signal linearly depends on the identification errors of the predictor Markov parameters. Hence a robust scheme has been developed in \cite{Dong2012a, Dong2012b} to cope with stochastic identification errors. This benefit of robustness compared to the SIM-PCA based method in \cite{Ding2009JPC} is achieved at the cost of increased computational burden in incorporating past I/O data.

Most of the data-driven fault diagnosis literature mentioned above discuss only fault detection and isolation. It is much more involved to estimate/identify the fault signal in the data-driven setting. The work in \cite{Alcala2009, Qin2009} proposed to reconstruct faults by minimizing the reconstructed squared prediction error obtained from PCA. However, this approach did not fully investigate the statistical properties of the calculated fault estimates.
By investigating the link between system-inversion based fault reconstruction and the predictor Markov parameters, the method in \cite{Dong2012c} constructed fault estimators parameterized by the predictor Markov parameters. Its fault estimates are asymptotically unbiased as the estimation horizon length tends to infinity, under the condition that the underlying inverted system is stable.

One drawback of the data-driven fault estimator proposed in \cite{Dong2012c} is that it cannot be directly applied to sensor faults in an unstable open-loop plant because its underlying inverted system is unstable. Another limitation of this method is that it does not compensate for the identification errors. The robustness of fault estimation to the identification errors is critical in two situations: 1) there exist large identification errors due to small number of identification data samples or low signal-to-noise ratio in identification data; 2) multiplication of the erroneous identified matrices with online I/O data of large amplitude cannot be simply ignored.

Motivated by the above two drawbacks of the proposed method in \cite{Dong2012c}, this paper develops data-driven robust fault estimation methods for additive actuator/sensor faults, utilizing the identified Markov parameters. In order to pave the way for data-driven design, we first construct a receding horizon (RH) fault estimator parameterized by the predictor Markov parameters, assuming that the predictor Markov parameters are accurately available. It gives (asymptotically) unbiased fault estimates under the condition that the subsystem from faults to outputs has no unstable transmission zeros. The above condition for unbiasedness generalizes the requirement of stable inversion in \cite{Dong2012c}. An immediate benefit is that our fault estimator can be applied to sensor faults in unstable open-loop plants as long as the above condition for unbiasedness is satisfied, whereas the proposed method in \cite{Dong2012c} cannot.

Our data-driven design parameterizes the above RH fault estimator with predictor Markov parameters identified from closed-loop data. The obtained data-driven fault estimation error is linear with regards to the stochastic identification errors of Markov parameters, although the identification errors appear as multiplicative uncertainty that couples with unknown fault signals as well as online I/O data. In order to enhance robustness to  stochastic identification errors, we propose two mixed-norm fault estimators.
The first one can be designed offline by regarding the online I/O data as unknown. By exploiting online I/O data in its formulated mixed-norm problem, the second robust fault estimator further reduces estimation errors when the online I/O data have large amplitudes, at the cost of increased online computational burden.  Based on a geometric interpretation of the formulated mixed-norm problems, a systematic tuning method for the user-defined parameters therein is provided to achieve the desired trade-offs between estimation bias and variance. 
Our proposed methods can handle sensor and actuator faults either separately or simultaneously. Only the separate scenario is illustrated in detail in this paper. Exact formulas for the simultaneous scenario can be derived in a straightforward manner but are omitted for the sake of brevity.


The rest of this paper starts with the problem formulation and some preliminaries on closed-loop identification of predictor Markov parameters in Section \ref{sect:probformulation}. Section \ref{sect:RHFE_predictor} constructs the predictor-based RH fault estimator, and analyzes its condition for unbiasedness. A data-driven nominal fault estimator is given in Section \ref{sect:RHFE_dd_nominal}. Section \ref{sect:dd_robust} and \ref{sect:dd_robust_onlineopt} propose two mixed-norm fault estimators with enhanced robustness to identification errors. Simulation studies are finally given in Section \ref{sect:sim}.

\section{Preliminaries and problem formulation}\label{sect:probformulation}

\subsection{Notations}
For a matrix , its range and null space is denoted by  and , respectively.  represents the left inverse satisfying , while  represents the generalized inverse satisfying

 represents the  column of . The trace of  is denoted by . Let  represent the Frobenius norm of the matrix . The minimal eigenvalue of a symmetric matrix  is represented by . Let  represent the column vector concatenating the columns of a matrix . The symbol ``'' stands for Kronecker product. Let  denote a block diagonal matrix with  as its diagonal matrices.




\subsection{Problem formulation}\label{sect:system_descrip}
We consider linear discrete-time systems governed by the following state space model:

Here , , and  represent the state, the output measurement, and the known control input at time instant , respectively. The process and measurement noises  and  are white zero-mean Gaussian, with covariance matrices
,
,
.
 is the unknown fault signal to be estimated.
 are constant real matrices, with bounded norms and appropriate dimensions.


The following assumption is standard in Kalman filtering \cite{Kailath2000} and subspace identification \cite{Chiuso2007a, Kata2005}:
\begin{assum}\label{ass:detect_control}
The pair  is assumed detectable; and there are no uncontrollable modes of  on the unit circle, where  is the covariance matrix of .
\end{assum}


Based on Assumption \ref{ass:detect_control}, the system (\ref{eq:sys}) admits the one-step-ahead predictor form given by \cite{Kailath2000}

where  is the steady-state Kalman gain, , , and ,  is the zero-mean innovation process with the covariance matrix . 	


We consider additive sensor or actuator faults in this paper, i.e.,
\begin{itemize}
\item fault of the  sensor: 
	
\item fault of the  actuator: 
	
\item simultaneous faults of the  sensor and  actuator: 
	
\end{itemize}
with  representing the  column of a matrix . 

Denote the predictor Markov parameters by

\vspace*{-0.5cm}


\begin{assum}\label{ass:fault_rank}
	The relative degree of the fault subsystem  is , i.e.,  is the smallest nonnegative integer  such that  and  \cite{Kirt2011}; moreover, 
	 \cite{Dong2012c}.
\end{assum}
Note that  for sensor faults and  for actuator faults.






The essential goals of this paper are to design a fault estimator from identification data without knowing the system matrices in (\ref{eq:sys}), and moreover to robustify the fault estimator against identification errors.

Concerning the identification data, it should be noted that in practice data from faulty conditions may be seldomly available, or if recorded then without a reliable fault description \cite{Ding2014JPC}. Hence we make the assumption as below:
\begin{assum}\label{ass:data}
Only I/O data collected from the fault-free condition are used in our data-driven design.
\end{assum}

In contrast to \cite{Park2000} which assumes the fault signals  evolve according to a random walk model, no assumption is made in this paper about how the fault signals  vary with time.


\subsection{Closed-loop identification of predictor Markov parameters}\label{sect:IDmarkov}
Considering Assumption \ref{ass:data}, we set  in (\ref{eq:sys}) for the identification data collected from the fault-free condition.
Then with , the predictor form (\ref{eq:predictor}) over the time window  can be written into the following data equation \cite{Chiuso2007a, Veen2012}:

where

denotes the sequence of Markov parameters  and  (defined in (\ref{eq:markov_param})) to be identified. 
The detailed definitions of the data matrices ,  and  can be found in \cite{Veen2012}, and  is the sequence of the innovation signal in the identification data. 


The least-squares (LS) estimate of the Markov parameters  is

with .
As standard assumptions for consistent identification from closed-loop data, we assume that 1) the data matrix  has full row rank, and 2) either the controller has at least one-step delay or the plant model has no direct feedthrough () \cite{Chiuso2007a, Veen2012}.


With sufficiently large , the estimation bias
 can be neglected. Then the stochastic identification errors are

Hence according to (\ref{eq:Xi_err}), the identification errors in Markov parameters can also be written as

where  and  represent the estimated Markov parameters in  given by (\ref{eq:LS_id}),  and  are the corresponding blocks of , i.e.,



The innovation covariance can be estimated by \cite{Gust2001, Kata2005}

For the sake of brevity, we shall not distinguish between the estimated innovation covariance  and its true value  in the rest of this paper.

\section{Predictor-based receding horizon fault estimation}\label{sect:RHFE_predictor}
In this section, we will construct an RH fault estimator based on the predictor form of the system (\ref{eq:sys}). Here we consider the predictor form instead of the original system model (\ref{eq:sys}) in order to pave the way for data-driven design. 



Consider a sliding window with a length of  sampling instants. Define stacked data vectors in this window as , , , and , respectively for the signals , , , and ; e.g.,

with . For the predictor form  (\ref{eq:predictor}), let  denote its extended observability matrix with  block elements, and  be the lower triangular block-Toeplitz matrix with  block columns and rows, with  representing , , or :



Given the I/O data over the sliding window ,
the stacked residual signal  in  can be computed by

according to the predictor form (\ref{eq:predictor}).
We can further write down the transitions from unknown initial state, faults and noises to the stacked residual signal  as

With Assumption \ref{ass:fault_rank}, (\ref{eq:res_dyn_batch0}) can be simplified as

where  is the relative degree of the fault subsystem ,  represents the first  block-columns of  defined similar to (\ref{eq:OL_TLu}),  is defined in the same way as in (\ref{eq:ukL}).


With (\ref{eq:res_dyn_batch}), we can formulate the receding horizon fault estimation (RHFE) problem

in the LS sense, with

denoting the covariance matrix of .
It has non-unique solutions because  may not have full column rank. One solution to the problem (\ref{eq:LS_prob}) is

We will show in the following theorem, however, that the last  entries of , i.e.,

with , 
represent an (asymptotically) unbiased estimate of  under certain conditions. The estimation delay  in (\ref{eq:f_hat_tao}) is caused by the relative degree in Assumption \ref{ass:fault_rank}.


\begin{thm}\label{thm:unbias}
Let  and  denote the relative degree and the observability index of the fault subsystem , respectively.
\begin{enumerate}
  \item[(\romannumeral1)] The -delay fault estimate  defined in (\ref{eq:f_hat_tao}) is unbiased for all  if and only if  has no transmission zeros, with 

  \item[(\romannumeral2)] The -delay fault estimate  is asymptotically unbiased for  if and only if all transmission zeros of  are stable.
\end{enumerate}
\end{thm}

The proof is given in Appendix \ref{app:thm_unbias}.

Instead of including the unknown initial state as in the RHFE problem (\ref{eq:LS_prob}), the essential idea of \cite{Dong2012c} is to find a lower triangular block-Toeplitz matrix  such that  and the estimation error caused by the unknown initial state exponentially decays with . The condition for unbiasedness in \cite{Dong2012c} requires that the inverse system related to  is stable. However this has several drawbacks: it does not clarify how the unbiasedness condition is related to the system property of the underlying plant; and moreover, for the case of sensor faults in an open-loop unstable plant, the method in \cite{Dong2012c} cannot find a stable left inverse matrix  for .

On the contrary, Theorem \ref{thm:unbias} clearly states that the condition for unbiasedness is related to the invariant zeros of the fault subsystem in the underlying plant. An immediate benefit is that our proposed RH fault estimator can ensure (asymptotically) unbiased estimates for sensor faults in an open-loop unstable plant, as long as the fault subsystem has no unstable transmission zeros.

\begin{rem}
The unbiasedness condition of the -delay fault estimate stated in Theorem \ref{thm:unbias} has close links with the -delay left inversion in \cite{Massey1968, Gill2007} and the -delay input and initial-state reconstruction in \cite{Kirt2011}. However, the -delay left inversion in \cite{Massey1968, Gill2007} requires the initial state to be known a priori, while the -delay input and initial-state reconstruction in \cite{Kirt2011} requires observability of the pair  to simultaneously reconstruct the initial state with the unknown input. Although it seems that the RHFE problem (\ref{eq:LS_prob}) jointly estimates initial state and faults, we are actually only interested in the fault estimate without unbiased reconstruction of the unknown initial state. This is an intuitive reason why Theorem \ref{thm:unbias} can cope with the unknown initial state in the case that  is detectable.
\end{rem}

\begin{rem}
Theorem \ref{thm:unbias} above generalizes Theorems 1 and 2 in \cite{Wan2014} in two aspects: 1) Theorems 1 and 2 in \cite{Wan2014} are limited to the case , while Theorem \ref{thm:unbias} here applies to general relative degrees; 2) Theorems 1 and 2 in \cite{Wan2014} focus on the fault estimator constructed with the original system (\ref{eq:sys}), while in this work we construct in Theorem \ref{thm:unbias} the fault estimator with the predictor (\ref{eq:predictor}).
\end{rem}


It should be noted that an RHFE problem similar to (\ref{eq:LS_prob}) can also be formulated using the original system (\ref{eq:sys}), see \cite{Wan2014}. Its equivalence to our RHFE problem (\ref{eq:LS_prob}) is shown in the following theorem.


\begin{thm}\label{thm:solution_equivalent}
If both the original system model (\ref{eq:sys}) and its predictor form (\ref{eq:predictor}) are accurately available, the -delay fault estimate , computed by (\ref{eq:fkL_hat}) and (\ref{eq:f_hat_tao}) based on the predictor form (\ref{eq:predictor}), is equivalent to the fault estimate proposed as Equation (15) in \cite{Wan2014} based on the original system model (\ref{eq:sys}).
\end{thm}

The proof of Theorem \ref{thm:solution_equivalent} is given in Appendix \ref{app:equivalence}. 
The above equivalence implies that the predictor gain  does not affect the statistics of the fault estimation error, and the condition of unbiasedness in Theorem \ref{thm:unbias} holds for the RH fault estimation using the original form.


\section{Data-driven nominal receding horizon fault estimator}\label{sect:RHFE_dd_nominal}
In this section, we will parameterize the RH fault estimator introduced in Section \ref{sect:RHFE_predictor} with the predictor Markov parameters, and then provide the nominal data-driven design method without considering identification errors.

In order to construct the LS fault estimator (\ref{eq:fkL_hat}), we first need to construct the block-Toeplitz matrices , , and  from the predictor Markov parameters according to (\ref{eq:OL_TLu}).
Then, we need the extended observability matrix . One possible approach is to identify  from the block-Hankel matrix

through a model reduction step \cite{Veen2012}. But this model reduction step would make the fault estimation error depend nonlinearly on 
the identification errors. In order to avoid this difficulty, we substitute 
 into (\ref{eq:res_dyn_batch}) by exploiting the following property:
 for .
Then (\ref{eq:res_dyn_batch}) can be rewritten as

where  consists of the first  block-columns of  defined in (\ref{eq:OL_TLu}).
By doing so, the fault estimation error becomes linear with regards to the identification errors, as shown later in (\ref{eq:fest_err}).
Based on (\ref{eq:res_batch_markov}), an LS problem similar to (\ref{eq:LS_prob}) can be formulated, and one solution is

Similarly to (\ref{eq:f_hat_tao}), we obtain the fault estimate

with .


\begin{thm}\label{thm:unbias_markov}
The sufficient and necessary condition for unbiased estimation in Theorem \ref{thm:unbias} applies to the fault estimate defined in (\ref{eq:hat_fL_markov})-(\ref{eq:f_hat_tao_aa}).
\end{thm}

The proof is given in Appendix \ref{app:unbias_markov}.

Combining (\ref{eq:resL_compute}), (\ref{eq:hat_fL_markov}), and (\ref{eq:f_hat_tao_aa}) yields the RH fault estimator as below:

where  represents the nominal RH fault estimator based on the residual signal .


Without considering the identification errors, the data-driven design of nominal RH fault estimator can now be summarized in Algorithm \ref{alg:nominal_design}. For the sake of brevity, we do not list the estimated fault Markov parameters  and their estimation errors for simultaneous sensor and actuator faults, because they can be straightforwardly derived similarly to (\ref{eq:hat_Hif_sen}) and (\ref{eq:hat_Hif_act}). Thus all our proposed algorithms in this paper can be directly extended to deal with simultaneous sensor and actuator faults.

\begin{algorithm}
  \caption{Data-driven nominal RH fault estimation}
  \label{alg:nominal_design}
  \begin{algorithmic}
    \State
    \begin{enumerate}
      \item[1)] Collect identification data from the fault-free condition, and form the data matrices  and  with sufficiently large  \cite{Veen2012}.
      \item[2)] Compute the sequence of Markov parameters  and the innovation covariance  via  (\ref{eq:LS_id}) and (\ref{eq:hat_Sigma_e}); extract the identified Markov parameters  and  from  according to (\ref{eq:Xi}); and extract  according to (\ref{eq:senfault_model})-(\ref{eq:markov_param}):
			\begin{itemize}
				\item for  sensor faults:
					
				\item or for  actuator faults:
					
			\end{itemize}
      \item[3)] Select sufficiently large .
          Construct the estimates of  in (\ref{eq:Sigma_eL}), , ,  in (\ref{eq:OL_TLu}),  in (\ref{eq:HLm}), and  in (\ref{eq:res_batch_markov}) as , , , , , and  by using
           and the identified Markov parameters . Form  with the first  block-columns of .
      \item[4)] Compute the nominal fault estimator according to (\ref{eq:RHFEor}) and (\ref{eq:RHFEor_F}).
    \end{enumerate}
  \end{algorithmic}
\end{algorithm}


\section{Data-driven robust receding horizon fault estimation}\label{sect:dd_robust}
The data-driven nominal design in Algorithm \ref{alg:nominal_design} might give biased fault estimates due to errors in the identified Markov parameters. To address this problem, this section proposes an offline robust design which regards the online I/O data as unknown in the design stage.


\subsection{Data-driven robust design}
Since the Markov parameters related to faults are extracted from  or  via (\ref{eq:hat_Hif_act}) or (\ref{eq:hat_Hif_sen}), the identification errors of  can be  expressed as

where

with  and  defined in (\ref{eq:iderr_markov})-(\ref{eq:Muy}).

With (\ref{eq:iderr_markov}) and (\ref{eq:delta_Hif}), the estimated matrices , , ,  and  in Algorithm \ref{alg:nominal_design} can be written as

where  is the block-Hankel matrix constructed with  similarly to  in (\ref{eq:HLm}),  is the block-Toeplitz matrix constructed with  similarly to  in (\ref{eq:OL_TLu}) with  representing , , or ,


and  consists of the first  block-columns of .

Based on (\ref{eq:hatHLm})-(\ref{eq:hatPsiLtau}), we can write down the residual signal  considering identification errors according to (\ref{eq:resL_compute})-(\ref{eq:res_dyn_batch}) and (\ref{eq:res_batch_markov}):


Similarly to  in (\ref{eq:RHFEor}), let the matrix  denote the -delay fault estimator based on the residual , i.e.,

It follows from (\ref{eq:hatrkL}) that the fault estimation error is 

where  is defined in (\ref{eq:f_hat_tao_aa}). It can be seen that  appears as multiplicative uncertainty coupled with the true augmented fault signal  and the online I/O data .

We regard  and  as unknown but energy bounded.
Hence  and  in the first two terms of (\ref{eq:fest_err}) lead to an estimation bias, while the online innovation signal  in the third term causes zero mean, stochastic estimation errors. We would like to reduce the estimation bias by minimizing the matrix 2-norms  (), and at the same time
minimize the Frobenius norm  by using the available innovation covariance . These three objectives are formulated by the following mixed-norm problem:

where the matrix  denotes the -delay fault estimator (\ref{eq:G_rkL}),  denotes mathematical expectation over the identification innovations ,  and  are the user-defined parameters to achieve a trade-off between estimation error variance and bias. Note that the matrix 2-norms  () are affected by the stochastic identification innovations  according to (\ref{eq:fest_err}), hence their mathematical expectations are used in (\ref{eq:offline_mixed_prob_symb}). 
Note also that it is straightforward to prove  holds if and only if  in (\ref{eq:offline_mixed_prob_symb}) holds. Here we use 
 in (\ref{eq:offline_mixed_prob_symb}), because it brings a clear geometrical interpretation for parameter tuning as explained later in Section \ref{sect:tune_geometric}.
With the tedious but straightforward derivations summarized in Appendix \ref{app:computations}, the above problem (\ref{eq:offline_mixed_prob_symb}) can be explicitly written as
\label{eq:offline_mixed_prob_explicit_cost}
\mathcal{G}_{\rm{r,off}} = \argmin\limits_{\mathcal{G}}\; \mathrm{tr} \left( \mathcal{G} \Sigma_{e,L} \mathcal{G}^\mathrm{T} \right)
\label{eq:offline_mixed_prob_explicit_constf}
\mathrm{s.t.}\;
\left[ \begin{array}{cc}
         \mathcal{G} & \mathcal{I}_{n_f}
       \end{array} \right]
\left[ \begin{array}{cc}
         \Pi_f & -\hat {\Upsilon}_{L,\tau} \\
         -\hat {\Upsilon}_{L,\tau}^\mathrm{T} & I_{n_f}
       \end{array}
 \right]
\left[ \begin{array}{c}
         \mathcal{G}^\mathrm{T} \\
         \mathcal{I}_{n_f}^\mathrm{T}
       \end{array}
 \right] \leq \gamma_f^2 I
\label{eq:offline_mixed_prob_explicit_constz}
\mathcal{G} \Pi_z \mathcal{G}^\mathrm{T} \leq \gamma_z^2 I,

with  and  defined in (\ref{eq:pif}) and (\ref{eq:piz}), respectively.
The mixed-norm problem (\ref{eq:offline_mixed_prob_explicit}) can be easily transformed into an equivalent semi-definite programming (SDP) problem that can be solved efficiently \cite{Boyd2004}.
Since the optimization problem (\ref{eq:offline_mixed_prob_explicit}) is determined only by the identification data and does not involve any online I/O data, it can be solved offline to obtain the robust fault estimator denoted as .


\subsection{Parameter tuning using geometric interpretation}
\label{sect:tune_geometric}
Next, we will provide a systematic method to tune the two user-defined parameters  and  by using a geometric interpretation of the mixed-norm problem (\ref{eq:offline_mixed_prob_explicit}).

With some matrix manipulations,
we can see that the constraints (\ref{eq:offline_mixed_prob_explicit_constf}) and (\ref{eq:offline_mixed_prob_explicit_constz}) define two ellipsoids


respectively, with .
Since the objective function (\ref{eq:offline_mixed_prob_explicit_cost})
can be regarded as a measure of the distance from  to
the origin , the optimization problem (\ref{eq:offline_mixed_prob_explicit}) is equivalent to finding the point  in the set
 that is closest to the origin, as shown in Fig. \ref{fig:geom}.


First, we would like to find the region of  and  so that the optimization problem (\ref{eq:offline_mixed_prob_explicit}) is feasible and non-trivial. In the case that the origin , we would have the trivial solution  which makes no sense for fault estimation. Hence
 and  are both required, which implies the region of  as below according to (\ref{eq:elipsoid_f}):


For a given  satisfying (\ref{eq:gammaf_range}), we solve the following optimization problem

whose solution gives the minimal , referred to as , that ensures . Therefore, we should select  to ensure feasibility of the optimization problem (\ref{eq:offline_mixed_prob_explicit}).
The ellipsoid  in Fig. \ref{fig:geom} represents the ellipsoid  with , and its intersection with the ellipsoid  includes only the single point .


By discarding the constraint (\ref{eq:offline_mixed_prob_explicit_constz}) from the problem (\ref{eq:offline_mixed_prob_explicit}) and fixing  at the same given value as in (\ref{eq:offline_prob_gammazmin}), we formulate another problem

Because the optimal solution  gives the shortest distance from the origin to the ellipsoid , and moreover , the solution  must lie at the boundary of the ellipsoid , as shown in Fig. \ref{fig:geom}.
Define . Let the ellipsoid  in Fig. \ref{fig:geom} represent the set  with , and it has the solution   at its boundary.

\begin{table*}[!t]
 \caption{Trade-offs between fault estimation bias and error variance of the robust fault estimator  at time instant  when tuning user-defined parameters  and  in (\ref{eq:offline_mixed_prob_explicit}): ``Constant'', ``'', and ``'' means that the performance criterion in the corresponding column remains constant, monotonically increases, and monotonically decreases with regard to the user-defined parameter specified in the corresponding row, respectively.}
 \label{tab:perf_tradeoff}
 \centering
 \begin{tabular}{lccc}
  \toprule
   User-defined & First bias term & Second bias term & Variance \\
   parameters & 
   & 
   & \\
  \midrule
   & Constant &  &  \\
   & Constant & Constant & Constant \\
   &  &  &  \\
  \bottomrule
 \end{tabular}
\end{table*}

Similarly to the above obtained solution  of the problem (\ref{eq:offline_prob_gammazmax}), the solution  of the problem (\ref{eq:offline_mixed_prob_explicit}) also lies at the boundary of the ellipsoid . This allows the three terms of the fault estimation error in (\ref{eq:fest_err}) to be explained using Fig. \ref{fig:geom}:
\begin{enumerate}
  \item[1)] The bias related to the first term  is determined by the size of the ellipsoid ;
  \item[2)] The bias related to the second term  is determined by the size of the ellipsoid  with  lying on its boundary, i.e., the ellipsoid  with ;
  \item[3)] The fault estimation error variance related to the third term   is represented by the distance from the origin to the optimal solution .
\end{enumerate}



With the above basic geometric interpretation, we can analyze the performance trade-offs of the robust fault estimator  when tuning  and , as shown in Table \ref{tab:perf_tradeoff}.
First, we fix  and tune . In this case, the ellipsoid  is fixed, which makes the first bias term in the first two rows of Table \ref{tab:perf_tradeoff} remain constant.
With the fixed , by increasing  from  towards , the intersection set  becomes larger, and the optimal solution  moves from the point  along the boundary of the ellipsoid  towards the point .
When we further increase  for , the optimal solution  of the problem (\ref{eq:offline_mixed_prob_explicit}) would remain located at the point , because  satisfies both constraints (\ref{eq:offline_mixed_prob_explicit_constf}) and
(\ref{eq:offline_mixed_prob_explicit_constz}) and gives the shortest distance to the origin according to the problem (\ref{eq:offline_prob_gammazmax}).
Therefore, the size of the ellipsoid , which determines the second estimation bias term in the first two rows of Table \ref{tab:perf_tradeoff}, monotonically increases for
 and remains constant for
. The distance from the origin to , which determines the fault estimation error variance in the first two rows of Table \ref{tab:perf_tradeoff}, monotonically decreases for
 and remains constant for . For the third row of Table \ref{tab:perf_tradeoff}, we tune  and select a sufficiently large value of  that ensures the problem (\ref{eq:offline_mixed_prob_explicit}) to be feasible. With  increasing, the size of the ellipsoid , which determines the first bias term in the third row of Table \ref{tab:perf_tradeoff}, monotonically increases. Meanwhile, the optimal solution
, which lies at the boundary of the ellipsoid , moves closer to the origin. Therefore, both the second bias term and the fault estimation error variance in the third row of Table \ref{tab:perf_tradeoff}, which are determined by the size of the ellipsoid  and the distance from the origin to the point , monotonically decrease.


\begin{figure}[!ht]
\begin{center}
\includegraphics[width=2.5in]{fig_offline_design_geometry}
\caption{Geometric interpretation of the mixed-norm problem (\ref{eq:offline_mixed_prob_explicit}): the constraints (\ref{eq:offline_mixed_prob_explicit_constf}) and (\ref{eq:offline_mixed_prob_explicit_constz}) define the ellipsoid  centered at  and the ellipsoid  centered at the origin , respectively.
Lying at the boundary of the ellipsoid , the optimal solution
 gives the shortest distance measured by the objective function (\ref{eq:offline_mixed_prob_explicit_cost}) from the origin to the intersection set .
With , the ellipsoid  becomes  in green which intersects with the ellipsoid  at a single point .
At the boundary of the ellipsoid ,
 gives the shortest distance from the origin to the ellipsoid .
The ellipsoids  in blue and  in red represent the ellipsoids  with  and  lying at the boundary, respectively.}
\label{fig:geom}
\end{center}
\end{figure}




We summarize the data-driven robust design in Algorithm \ref{alg:offline_robust_design}. The nominal design  obtained from Algorithm \ref{alg:nominal_design} can be used as a benchmark for tuning  and  in Step 2 of Algorithm \ref{alg:offline_robust_design}. For example, compared to the nominal design, the robust design achieves smaller averaged worst-case bias if  ().



\begin{algorithm}
  \caption{Data-driven robust RH fault estimation}
  \label{alg:offline_robust_design}
  \begin{algorithmic}
    \State
    \begin{enumerate}
      \item[1)] Complete the steps 1-3 in Algorithm \ref{alg:nominal_design}; compute , , and  according to (\ref{eq:Muy}) and (\ref{eq:Mif}).
      \item[2)] Tune  and  according to the performance trade-offs shown in Table \ref{tab:perf_tradeoff}, where  and  are obtained from the optimization problems (\ref{eq:gammaf_range}) and (\ref{eq:offline_prob_gammazmin}) respectively.
      \item[3)] Solve the problem (\ref{eq:offline_mixed_prob_explicit}) to compute the robust RH fault estimator .
    \end{enumerate}
  \end{algorithmic}
\end{algorithm}



\section{Data-driven robust receding horizon fault estimation with online optimization}\label{sect:dd_robust_onlineopt}
The online I/O data is regarded as unknown in Algorithm \ref{alg:offline_robust_design}. In order to better exploit the available online data, this section proposes an online mixed-norm optimization approach. This can further reduce the estimation errors when the online I/O data have large amplitudes, at the expense of increased computational burden.

\subsection{Online mixed-norm problem}
With the notation

we divide  into  row blocks as in

with . Then the term  in (\ref{eq:fest_err}) can be rewritten as

according to the property of Kronecker product \cite{Brew1978}. Based on (\ref{eq:GEMz}), the estimation error in (\ref{eq:fest_err}) becomes

Then the statistics of , i.e.,
 can be exploited to evaluate the fault estimation error variance.
Therefore, we formulate the following optimization problem similarly to (\ref{eq:offline_mixed_prob_symb}):

with the user-defined parameter . The constraint in the above optimization problem (\ref{eq:online_mixed_prob_symb}) can be explicitly written as (\ref{eq:offline_mixed_prob_explicit_constf}).
The optimization problem (\ref{eq:online_mixed_prob_symb}) has to be solved at each time instant to update the robust fault estimator  because  in the cost function is determined by the online I/O data according to (\ref{eq:bar_beta})-(\ref{eq:GEMz}).


\subsection{Parameter tuning using geometric interpretation}
Since the online mixed-norm problem (\ref{eq:online_mixed_prob_symb}) has the structure similar to that of the offline mixed-norm problem (\ref{eq:offline_mixed_prob_explicit}), the performance trade-offs by tuning  in (\ref{eq:online_mixed_prob_symb}) are also similar to that explained in Table \ref{tab:perf_tradeoff}.


The proposed data-driven robust fault estimation with online optimization is summarized in Algorithm \ref{alg:online_robust_design}.
In order to reduce the computational burden of online optimization, the problem (\ref{eq:online_mixed_prob_symb}) is implemented only if the estimation bias of the offline designed fault estimator is larger than a user-defined threshold , as shown in Step 2 of Algorithm \ref{alg:online_robust_design}.


The offline designed fault estimator  from Algorithm \ref{alg:offline_robust_design} can be used as a benchmark for tuning  in Step 2.2 of Algorithm \ref{alg:online_robust_design}. For example, compared to , the online optimization (\ref{eq:online_mixed_prob_symb}) achieves smaller averaged worst-case bias if .

\begin{algorithm}
  \caption{Data-driven robust RH fault estimation with online optimization}
  \label{alg:online_robust_design}
  \begin{algorithmic}
    \State
    \begin{enumerate}
      \item[1)] Follow Algorithm \ref{alg:offline_robust_design} to compute the offline designed fault estimator .
      \item[2)] If  ( is a user-defined threshold), the online optimization in the following steps is implemented; otherwise, the offline designed estimator  is used.
          \begin{enumerate}
            \item[2.1)] Compute  according to (\ref{eq:bar_beta})-(\ref{eq:GEMz}).
            \item[2.2)] Tune  similarly to Step 2 of Algorithm \ref{alg:offline_robust_design}, with  defined in (\ref{eq:gammaf_range}).
            \item[2.3)] Solve the problem (\ref{eq:online_mixed_prob_symb}) to compute the robust RH fault estimator .
          \end{enumerate}
    \end{enumerate}
  \end{algorithmic}
\end{algorithm}



\section{Simulation studies}\label{sect:sim}
Consider a continuous-time linearized vertical take-off and landing (VTOL) aircraft model that has been studied in \cite{Dong2012a, Dong2012b, Dong2012c, Gust2001}:

With a sampling rate of 0.5 seconds, the discrete-time model (\ref{eq:sys}) is obtained, with  and . The process and measurement noises,  and , are assumed to be zero mean white noises, respectively with covariances of  and .

Since the open-loop plant is unstable, an empirical stabilizing output feedback controller is used \cite{Dong2012c}, i.e.,

where  is the reference signal.



In the identification experiment, the reference signal  is zero-mean white noise with the covariance of , which ensures persistent excitation.
We collect  data samples from the identification experiment. In the identification algorithm, the past horizon is selected as .

The considered fault cases include:
\begin{itemize}
  \item Actuator faults: , ,
  \item Sensor faults: , .
\end{itemize}
The case of simultaneous actuator and sensor faults is not included here, because all the considered algorithms can be applied to the simultaneous scenario in a straightforward way, and their performance comparisons are the same as in the case of separate actuator or sensor faults.

The simulated fault signals in both fault cases are the same:


We will compare the following fault estimation methods:
\begin{itemize}
  \item Alg0: the RH fault estimator using accurate Markov parameters, described in Section \ref{sect:RHFE_dd_nominal}.
  \item DONG: the method proposed by \cite{Dong2012c}.
  \item Alg1: the data-driven nominal RH fault estimator  proposed in Algorithm \ref{alg:nominal_design};
  \item Alg2: the data-driven robust RH fault estimator  proposed in Algorithm \ref{alg:offline_robust_design}; in Step 3 of Algorithm \ref{alg:offline_robust_design}, we select , and
      
  \item Alg3: the data-driven robust RH fault estimator  with online optimization, proposed in Algorithm \ref{alg:online_robust_design}; in Step 2 of Algorithm \ref{alg:online_robust_design}, we select  as the threshold to determine whether or not the online optimization should be implemented;  is set to the same value as in Alg2.
\end{itemize}

We select the estimation horizon length  for the considered five algorithms.

In order to show the necessity of compensating for the identification errors, we make the identification-error-effect term 
in (\ref{eq:fest_err}) significantly large by setting . Fault estimates from the above five algorithms are illustrated in Fig. \ref{fig:plotfest}, and the distributions of their fault estimation errors are shown in Fig. \ref{fig:result_ref1}.
By using accurate Markov parameters, Alg0 achieves unbiased fault estimation in both fault scenarios. Note that DONG cannot be directly applied to sensor faults in the unstable open-loop VTOL model \cite{Dong2012c}, hence it is not included in Fig. \ref{fig:plotfest} and \ref{fig:result_ref1_sensor} for sensor faults.
Because of neglecting the effect of identification errors, both Alg1 and DONG yield estimation biases even larger than the amplitude of true faults. In comparison, Alg2 obtains its robustness to identification error by solving an offline mixed-norm  problem, as shown in Fig. \ref{fig:result_ref1_actuator}. However, the poor performance of Alg2 in our sensor fault case (Fig. \ref{fig:result_ref1_sensor}) shows the limitation of neglecting the online availability of I/O data in the offline mixed-norm problem. 
Compared to Alg2, Alg3 significantly reduces estimation bias, as shown in Fig. \ref{fig:result_ref1_sensor}, by formulating an online mixed-norm problem to exploit online I/O data. This performance improvement is achieved at the cost of higher online computational burden. When implemented with YALMIP \cite{Lofberg2004} in the MATLAB2011b environment, on a computer with a 3.4 GHz processor and 8 GB RAM, the averaged and peak computational time per sample of Alg3 are 1.70s and 2.05s for the estimation horizon length , while those of Alg2 are s and s respectively. We will investigate the computational efficiency of Alg3 for real-time implementation in future work.

\begin{figure}[h]
	\centering
	\includegraphics[width=8.5cm]{plotfest_yita15}
	\caption{True fault signal and fault estimates from different algorithms.}
	\label{fig:plotfest}
\end{figure}



\begin{figure}[h]
\centering
\subfigure[Actuator faults]{
\includegraphics[width=6cm]{2Ddistribution_fault1_yita1}
\label{fig:result_ref1_actuator}
}
\subfigure[Sensor faults]{
\includegraphics[width=6cm]{2Ddistribution_fault0_yita1}
\label{fig:result_ref1_sensor}
}
\caption{Distribution of fault estimation errors when . Circles: 1000 estimation errors by different fault estimation algorithms based on 1000 online I/O data samples. Ellipses: the -contour of the approximated two-dimensional Gaussian distribution of the 1000 estimation errors, i.e., the contour at .}
\label{fig:result_ref1}
\end{figure}


In order to illustrate the performance trade-offs of Alg2, we set  as in (\ref{eq:alg2_gammaz}) and tune  under the condition of different reference signals . Fig. \ref{fig:result_tune} shows how the fault estimation bias, error variance and root mean square error (RMSE) vary with , which can be explained as follows using Table \ref{tab:perf_tradeoff}.
According to the fault estimation error analysis in (\ref{eq:fest_err}), the fault estimation bias is related to both 
and .
For  or , the online I/O data  have small amplitude, thus the total estimation bias is dominated by the bias related to   which monotonically increases with  according to the third row of Table \ref{tab:perf_tradeoff}. This explains the fault estimation bias curves for  and  in Fig. \ref{fig:result_tune}.
For , the online I/O data  have relatively large amplitudes, hence for relatively small values of  the total estimation bias is dominated by the bias related to  which monotonically decreases with , and for relatively large values of  the total estimation bias is dominated by the bias related to  which monotonically increases with , according to the third row of Table \ref{tab:perf_tradeoff}. This explains the fault estimation bias curve for  in Fig. \ref{fig:result_tune}. The monotonic decrease of the fault estimation error variances with  can be directly explained with the third row of Table \ref{tab:perf_tradeoff}.
As the objective function of the optimization problem (\ref{eq:offline_mixed_prob_explicit}), the fault estimation error variance  for different reference signals  is the same because it does not depend on the reference signal . Combining the increase of fault estimation bias and the decrease of fault estimation error variance with , there exist the optimal  such that the RMSE achieves its minimal value, as can be seen in Fig. \ref{fig:result_tune}. It is also shown that the minimal RMSE is achieved at a larger value of  when the amplitude of  increases, because the online I/O data have larger amplitudes with larger , thus the decrease of the bias related to  dominates the fault estimation bias. Based on the above insights, we can anticipate how the estimation performance of Alg2 varies with different  for a fixed , as well as the performance trade-offs of Alg3. Their performance curves are not plotted due to the space limitation.

From the simulation results with different lenghts  of the estimation horizon (omitted for the sake of brevity), it can be seen that the fault estimation bias and variance of Alg0, Alg1, Alg2, and Alg3 decrease with the increasing length  of the estimation horizon. Straightforward proof of this observation can be derived for Alg0 using accurate Markov parameters (following Section 3.4.3 of \cite{Kailath2000}), whereas analytical proof is difficult for Alg1, Alg2, and Alg3 that rely on the identified Markov parameters contaminated with identification errors.

\begin{figure}[h]
\centering
\includegraphics[width=7cm]{result_tune}
\caption{Estimation performance of Alg2 when tuning  under different reference signal }
\label{fig:result_tune}
\end{figure}










\section{Conclusions}
This paper has investigated data-driven fault estimation and its robustness against stochastic identification errors. First, we proposed an RH fault estimator that can be parameterized with the predictor Markov parameters. Its condition for unbiasedness generalizes that of a recently reported data-driven fault estimation method. An immediate benefit is that our proposed method can be applied to sensor faults of an unstable open-loop plant which could not be directly addressed previously. In the formulated RH fault estimator, the identification errors appear as multiplicative model uncertainty coupled with the unknown faults and the online I/O data. Then, two mixed-norm problems were formulated to enhance robustness. One can be solved offline by regarding the online I/O data as unknown signals. The other further reduces estimation errors for larger I/O data by exploiting their online availability in the mixed-norm problem, and it requires online optimization. Based on geometric interpretations of the mixed-norm problems, systematic methods were given to tune the user-defined parameters therein. 
Comparisons using a simulated aircraft model illustrated the advantages and the effectiveness of our proposed method.




\begin{ack}                               The research leading to these results has received funding from the European Union's Seventh Framework Programme (FP7-RECONFIGURE/2007--2013) under grant agreement No. 314544.  \end{ack}

\bibliographystyle{plain}        \bibliography{bib_DD_RHfaultest}           


\appendix
\section{Lemmas for Theorem \ref{thm:unbias}}\label{app:property}

\begin{lem}\label{lem:zero_dyn}
Define , , and  () as the initial state, input and output signal of the fault subsystem , respectively. 
There exists a non-zero initial state  such that  for all , if and only if 
\begin{enumerate}
	\item[(\romannumeral1)] ;
	\item[(\romannumeral2)] the system
		
		is unobservable;
	\item[(\romannumeral3)] the inputs  take the form
		
\end{enumerate}
\end{lem}


In Lemma \ref{lem:zero_dyn},  is ensured because of the condition (\romannumeral1) and the zero Markov matrices  according to Assumption \ref{ass:fault_rank}, while  is ensured by the conditions (\romannumeral2) and (\romannumeral3).
Lemma \ref{lem:zero_dyn} can be proved by slightly modifying Lemmas A.1 and A.2 in \cite{Kirt2011}.


From Lemma \ref{lem:zero_dyn} we can see that perfect reconstruction of system inputs  from system outputs  is impossible if the unobservable input signal (\ref{eq:zero_f}) is non-zero. Hence, next, we will investigate the link between the unobservable input signal (\ref{eq:zero_f}) and the system property of .


By setting , (\ref{eq:zero_f}) becomes 

Then, according to the condition (\romannumeral1) and the unobservability of the system (\ref{eq:zero_init_cond}), there must exist a scalar  and a non-zero  such that \cite{Zhoubook1996}
 
where  defined in (\ref{eq:Htauf}) equals to  because  are zero matrices according to Assumption \ref{ass:fault_rank}.
With (\ref{eq:fe0}) and  in (\ref{eq:inv_zero}), we can rewrite  in (\ref{eq:zero_f}) as

The above analysis indicates that the unobservable inputs  are determined by the invariant zero  of 
, as shown in the following lemma:


\begin{lem}\label{lem:null_space_zero}
Considering the non-zero initial state  in Lemma \ref{lem:zero_dyn}, there are two types of the invariant zeros  of the fault subsystem
 in (\ref{eq:inv_zero}): 1) 
 is an unobservable mode, then (\ref{eq:inv_zero}) implies , thus the input signal  is constantly zero; 2)  is a transmission zero, then , thus the  unobservable input signal  is non-zero. 
\end{lem}


Lemma \ref{lem:null_space_zero} directly extends Lemmas 1 and 2 in \cite{Wan2014} which considers only the case . 


\section{Proof of Theorem \ref{thm:unbias}}\label{app:thm_unbias}

A solution  to the problem (\ref{eq:LS_prob}) satisfies

Let  denote the estimation error.
By substituting (\ref{eq:res_dyn_batch}) into (\ref{eq:LS_sol_cond}), we have

which implies

by taking expectations on both sides. Therefore, the unbiasedness condition of the estimate in (\ref{eq:f_hat_tao}) reduces to the analysis of the linear equation 

since 
. 


The rest of the proof follows the intuitive arguments below.
According to Lemma \ref{lem:zero_dyn}, (\ref{eq:fei}), and the definition of  in (\ref{eq:res_dyn_batch}), there are three scenarios:
\begin{enumerate}
	\item[1)] When  has no invariant zeros, the non-zero initial state  in Lemma \ref{lem:zero_dyn} does not exist according to (\ref{eq:inv_zero}), thus (\ref{eq:fest_err_unbias}) implies , i.e., unbiased fault estimation.
	\item[2)] When  has invariant zeros, (\ref{eq:fest_err_unbias}) implies that for each invariant zero , the expected error of the -delay fault estimate  is 
	
	in the estimation horizon  ().
		\begin{enumerate}
			\item[2.1)] If all the invariant zeros of  correspond to unobservable modes, it follows from the case 1) in Lemma \ref{lem:null_space_zero} that the expected estimation error (\ref{eq:expct_fest_err}) is zero because .
			\item[2.2)] If transmission zeros exist but are all stable, i.e., , it follows from the case 2) in Lemma \ref{lem:null_space_zero} that  and the expected estimation error (\ref{eq:expct_fest_err}) asymptotically reduced to zero as  goes to infinity.
		\end{enumerate}
\end{enumerate}
The scenarios 1) and 2.1) correspond to the case (\romannumeral1) of Theorem \ref{thm:unbias}, and the scenario 2.2) corresponds to the case (\romannumeral2) of Theorem \ref{thm:unbias}.


\section{Proof of Theorem \ref{thm:solution_equivalent}}\label{app:equivalence}
For the original system model (\ref{eq:sys}), the extended output equation in the time window  is

where , , , and  are defined
in the same way as  and  in (\ref{eq:OL_TLu}).
According to (\ref{eq:ext_output_origin}), we can rewrite (\ref{eq:resL_compute}) and (\ref{eq:res_dyn_batch0}) as

by following the relation between the original system model (\ref{eq:sys}) and its predictor form (\ref{eq:predictor}).
Similarly to  in (\ref{eq:res_dyn_batch}),  in (\ref{eq:res_compt_sys}) consists of the first  block-columns of .


Define  and

Comparing (\ref{eq:res_dyn_batch}) with (\ref{eq:res_compt_sys}) leads to

Then by substituting (\ref{eq:appC}) into (\ref{eq:fkL_hat}), the estimate of  becomes

which is actually the LS estimate proposed in \cite{Wan2014} based on the original system model (\ref{eq:sys}).




\section{Proof of Theorem \ref{thm:unbias_markov}}\label{app:unbias_markov}

Split  into two blocks as
,
with  consisting of the first  block-columns of
, and  consisting of the last block-column of
. With these notations, unbiased fault estimation can be proved by showing that 
 because  has full column rank according to Assumption 
\ref{ass:fault_rank}.

According to (\ref{eq:range_HLm}), the following two expressions are equivalent:


\vspace{-0.5cm}

Since the two sufficient conditions for (asymptotic) unbiasedness in Theorem \ref{thm:unbias} imply  and  () for (\ref{eq:D1}), it then follows from the equivalence between (\ref{eq:D1}) and (\ref{eq:D2}) that 
the sufficient conditions in Theorem \ref{thm:unbias} also imply  and  () for (\ref{eq:D2}), or equivalently,  and  (). Therefore we can conclude that the sufficient conditions in Theorem \ref{thm:unbias} imply  (asymptotically) unbiased fault estimation for (\ref{eq:D2}). Similarly, we can prove the necessary condition for the (asymptotically) unbiased fault estimation. 


\section{{Computation of } } \label{app:computations}

By dividing  in (\ref{eq:Mbarf}) into  row blocks as

with , we define  as

 is defined similarly to (\ref{eq:P_psi}), by dividing  in (\ref{eq:hatrkL}) into  row blocks as in (\ref{eq:barMPsi_i}). Then,

with



\end{document}
