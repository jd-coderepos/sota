\documentclass[10pt,twocolumn,letterpaper]{article}
\usepackage{latex8}
\usepackage{subfig}
\usepackage{graphicx}
\usepackage{array}
\usepackage{url}
\usepackage{eso-pic}

\AddToShipoutPicture*{\small \sffamily\raisebox{1.8cm}{\hspace{1.8cm}978-1-4244-4439-7/09/\mmuvuvvvD_{max}r-operatorO(D_{max})D_{max}=2a(a:0)bcd(b:0,a:1)(c:0,a:1)(d:0,a:1)de(e:0,d:1,a:2)(e:0,d:1,c:1,a:2)cda(a:0,b:1,c:1,d:1)dbeD_{max}e(d:0,a:1,c:1,b:2)ea,b,c,dacc(d:0,a:1,b:2)dmin\left(1,\left\lfloor \frac{n_c}{n_b} \right\rfloor \right)n_cn_bO(D_{max})D_{max}D_{max}=2D_{max}$, which yield , at
the cost of greater overhead, a small but noticeable improvement of
the delivery ratio.

The sampling period of the Rollernet traces is 15 seconds. We did not
try to extrapolate the events (link failures, new contact
opportunities, etc..) in the time between multiples of 15 seconds. We
also assume that 15 seconds is enough for a message to traverse any
connected component in Rollernet. Therefore, all our results on delays
when simulating protocols on top over the Rollernet traces will be in
multiples of 15 seconds.


\subsection{Performance}

\begin{figure}[t]
  \centering
  \subfloat[5 copies \label{cfs5}]{\scalebox{0.50}{\includegraphics{routing_d2_nc5}}} \\
  \subfloat[20 copies \label{cfs20}]{\scalebox{0.50}{\includegraphics{routing_d2_nc20}}}
  \caption{Comparison of delivery probabilities}
  \label{cfs}
\end{figure}

Extremely high link instability could mean one of two things. Either
nodes only briefly stay in the vicinity of one another or that nodes
may remain geographically close but that the link fails for other
reasons such as briefly moving out of transmission range or excessive
contention. We measured between each time step and from each node's
point of view, how many of its group members changed (number of new
members + number of excluded members) and how many links between
members of its group changed (either by appearing or
disappearing). The averages for all the nodes are shown in
Fig.~\ref{group_stab}. The composition of a given group appears much
more stable than the links among its members. This supports the idea
that small communities like groups of friends tend to stick together
during the tour and that link failures do not necessarily mean that
two nodes have clearly moved away from each other. Furthermore, the
rate of change of group composition, unlike most other metrics, seems
to smooth the accordion effect. This suggests that these groups are
indeed a good support for our hybrid approach.

To evaluate the performance of HYMAD we replayed the 3000 seconds of
the trace.  Every 15 seconds, during the first 2000 seconds, we
randomly selected 60 pairs of nodes which were instructed to send a
message to each other using Epidemic, HYMAD and Spray-and-Wait. This
averages results over both the connected and disconnected phases of
Rollernet. Figures \ref{cfs5} and \ref{cfs20} were obtained using the
aggregate data from 10 runs of this scenario with respectively 5 and
20 maximum number of copies for HYMAD and Spray-and-Wait. They compare
the cumulative distribution function of the delivery probability for
the three protocols. A few observations can be made:
\begin{itemize}
\item HYMAD clearly outperforms Spray-and-Wait in terms of delay and
  quickly achieves comparable performance with Epidemic.
\item With a low number of copies, HYMAD also outperforms
  Spray-and-Wait in terms of delivery ratio for reasons explained
  hereafter.
\item Predictably, performance increases with the number of
  copies. The maximum number of groups (including singletons) obtained
  at a given time is 29. Therefore, HYMAD with 20 copies will spray
  practically the entire network and therefore quickly and reliably
  reach the destination if in the same connected component as the
  source.
\end{itemize}

Spray-and-Wait's simple forwarding scheme performs very well
\emph{under the assumption of independent and identically distributed
  node mobility}~\cite{spyro_sw}. However this is absolutely
\emph{not} the case in Rollernet where groups of friends tend to stick
together. It is also usually \emph{not} the case in many real-world
situations where underlying social dynamics are often at work.

This can have a impact on performance. For example, when using just 5
copies, Spray-and-Wait simply fails to deliver about 5\% of messages
even after waiting for more than 15 minutes. Using 10 copies, the
average delay with Spray-and-Wait (133 seconds) is nearly three times
that of HYMAD (48 seconds). To further illustrate this point,
Fig~\ref{diffusion} compares the propagation of 10 copies after 15
seconds for HYMAD (Fig.~\ref{swg15}) and Spray-and-wait
(Fig.~\ref{sw15}). The rightmost node is the head of the rollerblading
tour. The bold lines represent intra-group links while the dashed gray
lines represent inter-group links. The nodes holding at least one copy
are represented by a diamond. In HYMAD's case, the destination is a
diamond meaning that our hybrid approach has delivered its message
within 15 seconds. On the other hand, the regular Spray-and-Wait
protocol distributed copies mainly within its own local group. These
nodes remain close to each other thus increase the delay. In this
particular case (Fig.~\ref{sw15}) it will take 525 seconds for a node
with a copy to meet the destination

\begin{figure*}[t]
  \centering
  \includegraphics{legend} \\
  \subfloat[HYMAD: success within 15 seconds. \label{swg15}]{\includegraphics{swg_15s}} \quad
  \subfloat[Spray-and-Wait: the copies stagnate around the source. It will take a total of 525 seconds to hit the destination. \label{sw15}]{\includegraphics{sw_15s}}
  \caption{Regular vs Hybrid Spray and Wait routing in Rollernet. (Partial view of the topology at t=15s)}
  \label{diffusion}
\end{figure*}


\section{Conclusion and further work}
\label{conclusion}
In this paper we identified a new class of dense and highly mobile
networks not well addressed by conventional DTN or MANET approaches. We
proposed a new hybrid approach, HYMAD, that uses nodes' knowledge of
their local group topology to improve the performance of a simple DTN
protocol. In our case we used diameter-constrained groups along with
distance vector for intra-group routing and Spray-and-Wait for
inter-group routing. Simulations of our implementation in a dense and
highly mobile network show significant performance improvements over
regular Spray-and-Wait. Further work includes more comprehensive testing
against other real and synthetic mobility scenarios.

HYMAD is an example of a larger class of hybrid DTN-MANET routing
protocols which can handle a very wide spectrum of networks that
overlaps with those usually handled by either DTN or MANET.  We
believe that the first results that we obtained are encouraging for
further research in this direction. In particular, other more
elaborate DTN/MANET protocol pairs could conceivably be used for intra
and inter-group routing and would be worth exploring.

\section*{Acknowledgments}
This work has been partially supported by the RNRT project Airnet under contract
01205

\small
\bibliographystyle{latex8} 
\bibliography{adhoc_dtn}

\end{document}
