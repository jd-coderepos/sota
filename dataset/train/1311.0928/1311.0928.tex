\documentclass[leqno,12pt]{article}  

\usepackage[margin=1.0in]{geometry}
\sloppy
\usepackage{amsfonts}
\usepackage{amsmath}
\usepackage[noend]{algpseudocode}
\def\R{\mathbb{R}}
\def\L{\mathcal{L}}
\def\C{\mathcal{C}}
\def\A{\mathcal{A}}
\def\orig{\mathbf{0}}
\newtheorem{theorem}{Theorem}
\newtheorem{lemma}{Lemma}
\newtheorem{proposition}{Proposition}
\newtheorem{corollary}{Corollary}
\newenvironment{proof}{\unskip{\bf Proof:}}{\unskip{\hfill }}
\def\qed{\unskip{\hfill }}



\title{The Complexity of Order Type Isomorphism}
\date{}
\author{Greg Aloupis\thanks{Charg\'e de recherches du F.R.S.-FNRS,
    D\'{e}partement d'Informatique, Universit\'{e} Libre de Bruxelles, 
{\tt aloupis.greg@gmail.com}}
 \and 
 John Iacono\thanks{Department of Computer Science and Engineering, Polytechnic Institute of New York University, {\tt jiacono@poly.edu, ozgurozkan@gmail.com}. Research partially supported by NSF grants CCF-0430849,  CCF-1018370.}
  \and 
 Stefan Langerman\thanks{Directeur de Recherches du
   F.R.S.-FNRS,~D\'{e}partement d'Informatique, Universit\'{e} Libre de Bruxelles,
   {\tt stefan.langerman@ulb.ac.be}. Research supported by F.R.S.-FNRS
 and DIMACS.}
  \and
   \"{O}zg\"{u}r \"{O}zkan\footnotemark[2]
 \and 
    Stefanie Wuhrer\thanks{Cluster of Excellence MMCI, Saarland University,
   {\tt swuhrer@mmci.uni-saarland.de} }}




\begin{document}

\maketitle


\begin{abstract}\small\baselineskip=9pt 
The order type of a point set in  maps each -tuple of points
to its orientation (e.g., clockwise or counterclockwise in ). 
Two point sets  and  have the same order type if there
exists a mapping  from  to  for which every -tuple 
 of  and the corresponding
tuple  in  have the same orientation.
In this paper we investigate the complexity of determining whether two
point sets have the same order type.
We provide an  algorithm for this task, thereby improving upon
the  algorithm of Goodman and Pollack (1983). The
algorithm uses only order type queries and also works for abstract
order types (or acyclic oriented matroids).
Our algorithm is optimal, both in the abstract setting and for
realizable points sets if the algorithm only uses order type queries.
\end{abstract}






\section{Introduction}
In the design of geometric algorithms, as well as in their practical
implementation, it is often convenient to encapsulate the geometry of
a given
problem into a small set of elementary geometric predicates. A typical
example, ubiquitous in computational geometry textbooks, is the \emph{left
turn / right turn} determinant

whose sign (, , or ) determines if three points 
are in clockwise or counterclockwise orientation, or collinear, respectively.

The practical motivation for this encapsulation will be obvious to any
programmer: by restricting the use of arithmetic operations to just
one place in the code, it is easier to control the robustness of the
code (e.g. with respect to roundoff errors). It is also easier to
generalize the code should a different geometric space require a
slightly different implementation of the predicate (e.g. solving 
geometric problems on a sphere or in a polygon). This
would require a proper abstraction to generalize the predicate
 to other applications.

The need for a classification or discretization of planar point sets
became evident long before computers were invented. 
In 1882,
Perrin~\cite{Perrin1882} described how a point moving on a line far enough from a
collection of points sees the points under a sequence of 
different radial orders, each produced by swapping two adjacent labels
from the previous ordering. He then showed how this representation
can be used to solve problems without the use of the original
point set. This view of point configurations was revived and
characterized under the name of \emph{allowable sequences} 
by Goodman and Pollack in 1980~\cite{Goodman1980220}. They later showed how the same
allowable sequences can describe pseudoline arrangements~\cite{Goodman1984257}.

The classification of point sets induced by the determinant  above, but
generalized to  dimensions, was discovered around the same time.
Consider a set , let  and for ease of notation let .
The \emph{order type} of  is characterized  by the predicate\footnote{We write  to denote the set of integers .}

This concept appeared independently in various contexts over a span of 15
years, under various names, e.g., 
\emph{-ordered sets}~\cite{Novoa}, \emph{multiplex}~\cite{multiplex},
\emph{chirotope}~\cite{dreiding},
\emph{order type}~\cite{goodman_pollack_83_sorting}, among others inspired by problems from chemistry.
For some of them (e.g., chirotopes or abstract order types), 
the precise algebraic definition above is replaced by a set of axioms that the
predicate  must satisfy. 

In the early 90's Knuth~\cite{DBLP:books/sp/Knuth92} revisited once more the axiomatic system of
chirotopes under the name of \emph{CC-systems}, but this time with a
specific focus on computational aspects, mainly, what predicates and
axioms are necessary in order to compute a convex hull (and later a
Delaunay triangulation), and what running times can be obtained by an
algorithm using only those predicates.

The theory of \emph{oriented matroids}
 appeared in the mid
'70s. Their primary purpose was to provide an abstraction of linear
dependency.  However, through their various equivalent axiomatizations
they have been used to show a translation between virtually all
abstractions mentioned above. 

\paragraph{Isomorphism}
is probably one of the most fundamental problems for any
discrete structure. In graph theory,
determining whether two graphs are isomorphic is one of the few
standard problems in NP not yet known to be NP-complete and not known
to be in P.
In our setting, two (abstract) order types with predicates 
and  are \emph{identical}\footnote{Sometimes, two order types are
  also considered identical if all orientations are reversed, i.e.,
  .} 
if 

or more succinctly .
They are \emph{isomorphic} if there is a permutation  such
that 
 
(X \circ Y)_e = 
  \begin{cases}
    X_e & \text{if } X_e \neq 0 \\
    Y_e & \text{if } X_e = 0
  \end{cases}
  \text{  }\forall e\in E.
  \L\setminus A = \{X\setminus A | X\in\L\} 
\subseteq \{+,-,0\}^{E\setminus A}\L/A = \{X\in\L | A\subseteq X^0\}
\subseteq \{+,-,0\}^{E\setminus A}\chi_A(x_1,\ldots,x_{d+1-|A|}) = \chi(A,x_1,\ldots,x_{d+1-|A|}).
\frac{1}{d} \sum_{i=1}^d S_i[x_i] &=
\frac{1}{d} \sum_{i=1}^d p(x_i,i) \\
&= \frac{1}{d} \sum_{i=1}^d ux_i +\frac{1}{d}\sum_{i=1}^d u_i\\
&=\frac{u}{d} \sum_{i=1}^d x_i + \frac{u}{d}\\
&=u \cdot \frac{1}{d} \left( 1+ \sum_{i=1}^d x_i \right)

\frac{1}{d} \sum_{i\in [1..d]} S'_i[x_i] &=
\frac{1}{d} \left( \sum_{i\in [1..d]\atop i \not \in I(X,Y)} S'_i[x_i] + \sum_{i\in [1..d]\atop i\in I(X,Y)} S'_i[x_i] \right) \\
&= \frac{1}{d} \left( \sum_{i\in [1..d]\atop i \not \in I(X,Y)} p(x_i,i) + \sum_{i\in [1..d]\atop i \in I(X,Y)} p\left(x_i{+}\frac{1}{2d}, i\right)\right) \\
&= \frac{1}{d} \left( \sum_{i\in [1..d]\atop i \not \in I(X,Y)} (ux_i+u_i) + 
\sum_{i\in [1..d]\atop i \in I(X,Y)} \left( u\left(x_i{+}\frac1{2d}\right) + u_i  \right)\right) \\
&= \frac{1}{d} \left( u\sum_{i\in[1..d]} x_i +\sum_{i\in[1..d]} u_i +  u\sum_{i\in [1..d]\atop i \in I(X,Y)}\frac{1}{2d} \right) \\
&= \frac{1}{d} \left( u \sum_{i\in[1..d]} x_i +u + u\frac{|I(X,Y)|}{2d} \right) \\
&=u \cdot \frac{1}{d} \left( 1+\frac{|I(X,Y)|}{2d}+  \sum_{i=1}^d x_i \right)




Since  is just , computing the order type  can be done by simply comparing the scalars  and
. If the sets  and  are different, then they have 
fewer than  elements
in common, and . This means that  and thus the presence or absence of the  term does not affect the comparison. However, if , then the point  is exactly the centroid of 
 and thus the results of  and 
 differ.

\begin{theorem}
 order type comparisons are sometimes needed to determine the order type of  points in  dimensions.
\end{theorem}

\begin{proof}
Let . Note that . Suppose that  fewer than  order type comparisons are performed. Then there is some  and  such that comparison 
was not performed. Let . Now observe that all order type comparisons performed so far in  and  are the same, but 
 and  will have different outcomes.
\end{proof}
\fi

\section*{Acknowledgements}
This work was initiated at the 2011 Mid-Winter
Workshop on Computational Geometry. The authors thank all participants
for providing a stimulating research environment. We also thank the
anonymous referees for numerous helpful comments.
\bibliography{refs}
\bibliographystyle{abbrv}   
\end{document}
