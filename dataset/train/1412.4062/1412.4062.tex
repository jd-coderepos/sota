\documentclass{article}
\usepackage{graphicx}
\usepackage{xcolor}
\usepackage{amsmath, amsthm, amssymb,mathtools}
\usepackage{times}

\newcommand{\bmp}{\begin{minipage}}
\newcommand{\emp}{\end{minipage}}

\newcommand{\rt}{\textcolor{red}}
\newcommand{\Id}{\mbox{\it Id}}
\newcommand{\LR}{\mbox{\it LR}}
\newcommand{\M}{\mbox{\it MinMax}}
\newcommand{\PK}{\mathcal P}
\newcommand{\kM}{\mbox{\it k}}
\newcommand{\oM}{\mbox{\it 1}}
\newcommand{\mint}{\mbox{\it m}_t}
\newcommand{\maxt}{\mbox{\it M}_t}
\newcommand{\Pm}{P^{-1}}
\newcommand{\n}{[n]}
\newcommand{\lrf}[1]{\xleftrightarrow{#1}}
\newcommand{\rf}[1]{\xrightharpoondown{#1}}
\newcommand{\lf}[1]{\xleftharpoondown{#1}}
\newcommand{\Sil}{{\it SilNB}}

\newtheorem{thm}{Theorem}
\newtheorem{cor}{Corollary}
\newtheorem{fait}{Claim}
\newtheorem{rmk}{Remark}
\newtheorem{defin}{Definition}
\newtheorem{ex}{Example}
\newtheorem{pb}{Problem}

\newcommand{\bp}{\begin{pb}\rm}
\newcommand{\ep}{\end{pb}}
\newcommand{\br}{\begin{rmk}\rm}
\newcommand{\er}{\end{rmk}}
\newcommand{\bdefin}{\begin{defin}\rm}
\newcommand{\edefin}{\end{defin} }
\newcommand{\bex}{\begin{ex}\rm}
\newcommand{\eex}{\end{ex}}



\newcommand{\bthm}{\begin{thm}}
\newcommand{\ethm}{\end{thm}}
\newcommand{\bcor}{\begin{cor}}
\newcommand{\ecor}{\end{cor}}
\newcommand{\bfn}{\begin{fait}}
\newcommand{\efn}{\end{fait}}

\usepackage{algorithm}
\usepackage{algorithmic}
\renewcommand{\algorithmicrequire}{\textbf{Input:}}
\renewcommand{\algorithmicensure}{\textbf{Output:}}

\renewcommand\thealgorithm{\arabic{algorithm}}
\makeatletter
\newcommand{\setalglineno}[1]{\setcounter{ALC@line}{\numexpr#1-1}}
\makeatother


\setlength{\textwidth}{14.5cm}
\setlength{\textheight}{22cm}
\hoffset -1.5cm
\voffset -1cm

\renewcommand{\Box}{\rule{1.5mm}{3mm}}

\begin{document}



\begin{center}
{\bf\large Permutation Reconstruction from -Betweenness Constraints}\\




\vspace*{1cm}

Irena Rusu\footnote{Irena.Rusu@univ-nantes.fr}

L.I.N.A., UMR 6241, Universit\'e de Nantes, 2 rue de
la Houssini\` ere,\\

 BP 92208, 44322 Nantes, France
\end{center}



\vspace*{1cm}

\hrule
\vspace{0.3cm}

\noindent{\bf Abstract} 


In this paper, we investigate the  reconstruction of permutations on  from
betweenness constraints involving the minimum and the maximum element located between 
  and , for all .  We propose two
variants of the problem (directed and undirected), and focus first on the directed version,
for which we draw up general features and design a polynomial algorithm 
in a particular case. Then, we investigate
necessary and  sufficient conditions for the uniqueness of the reconstruction
in both directed and undirected versions, using a parameter  whose variation controls the
stringency of the betweenness constraints. We finally point out open problems.
\medskip

\noindent {\bf Keywords:} betweenness, permutation, algorithm, genome, common intervals
\vspace{0.2cm}

\hrule

\section{Introduction}

The {\sc Betweenness} problem is motivated by physical mapping in molecular biology and the design of circuits 
\cite{opatrny1979total}.
In this problem, we are given the set , for some positive integer ,  
and a set of  {\em betweenness constraints} (), each represented as a triple  with  and signifying that  is required to be between  and . The goal is to find a 
permutation on  satisfying a maximum number of betweenness constraints. In \cite{opatrny1979total},
it is shown that the {\sc Betweenness} problem is NP-complete even in the particular case where all the
constraints have to be satisfied.

In this paper we are interested in a problem related to the {\sc Betweenness} problem,  which also
finds its motivations in molecular biology. Given  () permutations on the
same set , representing  genomes given by the sequences of their genes, a {\em common interval} 
of these permutations is a subset of  whose elements are consecutive ({\em i.e.} they form an interval) 
on each of the  permutations. Common intervals thus represent regions of the genomes which have identical 
gene content, but possibly different gene order. Computing common intervals or specific subclasses of
them in linear time (up to the number of output intervals) has been done by case-by-case approaches
until recently, when we  proposed \cite{IR} a common linear framework, whose basis is the notion of
-profile. The -profile of a permutation  forgets the order of the elements in a permutation, 
and keeps only essential betweenness information, defined as, for each , the minimum
and maximum value in the interval delimited by the elements  (included) and  (included) 
on  (with no restriction on the relative positions of  and  on ). When  permutations 
are available, their -profile is defined similarly, by considering for every   the global 
minimum and the global maximum of the  intervals delimited by  and  on the  permutations. 
We show in \cite{IR} that, assuming the permutations have been renumbered such that one of them is
the identity permutation, the -profile of  permutations is all we need to find common intervals,
as well as all the specific subclasses of common intervals defined in the literature, in linear time (up to the number of output intervals).

Hence, the -profile is a simplified representation of a (set of) permutation(s), which is sufficient
to efficiently solve a number of problems related to finding common intervals in permutations. 
Moreover, it may be computed in linear time \cite{IR}. However, it can be easily seen that distinct
(sets of) permutations may have the same -profile, implying that the -profile captures a part,
but not all, of the information in the (set of) permutation(s). 

In this paper, we study the reconstruction of a permutation from a given -profile, and discuss possible
generalizations.  


\section{Definitions and Problems}\label{sect:def}

In the remaining of the paper, permutations are defined on  and are increased with elements  and ,
added respectively at the beginning and the end of each permutation (and assumed to be fixed). This is due to the need 
to make the distinction between a permutation and its reverse order permutation. 

\bdefin \cite{IR}
The {\em -profile of a permutation}  on  is the set of {\em -constraints}



\noindent where  ( respectively) is the minimum (maximum respectively) element  
in the interval delimited on  by the element  (included) and the element  (included). 

\edefin

Note that the relative positions on  ({\em i.e.} which one is on the left of the other) of  on the one hand, and of   on the other hand are not indicated by a -profile.
In the case where the relative positions of  and   are
known for all , we use the term of {\em directed -profile} and the notations
 when  is on the left of ,
respectively   when  is on the left of .

\bex
Let  a permutation on . Then its
-profile is (note that the -constraints sharing an element are concatenated): 




\noindent whereas its directed -profile is:



\noindent Notice that the -profile and the directed -profile of any permutation obtained by 
arbitrarily permuting the elements  are the same, showing that a (directed or not) -profile
may correspond to several distinct permutations. 
\eex
The {\em -profile of a set  of permutations} is defined similarly \cite{IR}, by requiring that  
and  be defined over the union of the intervals delimited by  (included) and  (included)
on the  permutations in . This definition is given here for the sake of completeness,
but is little used in the paper.

We distinguish between the -profile of a (set of) permutation(s) and a {\em -profile}:

\bdefin
A {\em -profile} on  is a set of {\em -constraints} 



\noindent with . 
\edefin

Again, a -profile is {\em directed} when for all , , the relative position of  with respect to 
 is given.
A -profile may be the -profile of some permutation, or of a set of permutations, but may  also be the profile 
of no (set of) permutation(s). We limit this study to one permutation, and therefore formulate the following problem:
\newpage

\noindent {\sc -Betweenness}

\noindent{\bf Input:} A positive integer , a  -profile  on . 

\noindent{\bf Question:} Is there a permutation  on  whose -profile is ? 
\bigskip



The -{\sc Betweenness} problem is obviously related to the {\sc Betweenness} problem, since looking for a permutation
 with -constraints defined by  means satisfying a number of betweenness constraints.  Some differences
exist however, as  also defines non-betweenness constraints. More precisely, each -constraint
 from  may be expressed using the betweenness constraints 
(abbreviated B-constraints):


\noindent along with the non-betweenness constraints (abbreviated NB-constraints):




It is easy to imagine that in the -{\sc Betweenness} problem, the lack of information about
the relative position of   and  on the permutation  (i.e. which one is on the left of the other)
is a major difficulty. The directed version of the problem, in which these relative positions are given, should
possibly be easier.
\bigskip

\noindent{\sc Directed -Betweenness}

\noindent {\bf Input:} A positive integer , a directed -profile  on .

\noindent{\bf Question:} Is there a permutation  on  whose directed -profile is ? 
\bigskip

\br
It is worth noticing here that in a (directed or not) -profile which corresponds to
at least one permutation on , the value  ( respectively) should
only occur in one precise -constraint, namely the one involving  and  ( and
 respectively). Otherwise,  ( respectively) cannot be the leftmost (rightmost, respectively)
value in the permutation. In the subsequent of the paper, it is assumed that this condition
has been verified before further investigations, and assume therefore that  and 
are respectively located in places  and .
\label{rem:places0n+1}
\er

We present below, in Section \ref{sect:GeneralD}, our analysis of the Directed -{\sc Betweenness} problem,
proposing a first algorithmic approach and pointing out the main difficulties for reaching a complete polynomial solution. 
In Section \ref{sect:Particular}, we identify a polynomial particular case for the directed version. 
In Section \ref{sect:Generalizations} we propose to generalize -profiles to -profiles, by introducing 
a parameter  which allows to progressively increase the amount of information contained in a -profile, 
up to a value  which allows to identify each permutation by its -profile. Section \ref{sect:Conclusion} is the
conclusion.

\section{Seeking an algorithm for {\sc Directed -Betweenness}}\label{sect:GeneralD}

\subsection{A na\"{\i}ve approach}

Let  be a directed -profile on . The most intuitive idea for solving 
Directed -{\sc Betweenness} is  to build a simple
directed graph  ({\em i.e.} with no loops or multiple arcs)
whose vertex set  is  and whose arcs  indicate the precedence relationships
between the elements on each permutation corresponding to the given -profile ({\em i.e.}  is on the
left of ). If a permutation exists,  must be a directed acyclic graph (or DAG). 
The -constraints from  directly define arcs using: 1) the relative order between  and ,  
for each  (the corresponding arcs of  are called {\em R-arcs}), 
and 2) the B-constraints (resulting into  {\em B-arcs}). Further arcs may be dynamically obtained by repeatedly invoking:  
3) the transitivity of the precedence relationship 
(resulting into {\em T-arcs}), and 4) the NB-constraints  (resulting into {\em NB-arcs}).  

Algorithm \ref{algo:Arcs} shows these steps. After the construction of the - and -arcs (steps
2-6), either transitivity or NB-constraints may be arbitrarily invoked to add supplementary
arcs as long as possible, performing what we call the {\em NB-transitive closure}
of . This is done by the Build-Closure algorithm (Algorithm \ref{algo:Closure}), called in
step 8 of Algorithm \ref{algo:Arcs}. It is clear that in step 1 of the Build-Closure algorithm
a -arc  may be added iff there is a vertex
  such that  and  are arcs, but  is not an arc. The condition for
adding the -arc  is slightly more complex, as  may be added iff

\begin{algorithm}[t,boxed]
\caption{The Build-Easy-Arcs algorithm}
\begin{algorithmic}[1]
{\small \REQUIRE A directed -profile  over .
\ENSURE  Either the answer "No" (meaning no permutation exists), or the pair 
 where  is the DAG containing all deducible , ,  and -arcs, and
 is the set of silent NB-constraints. \\
\hspace*{-0.6cm}({\sl  Note:} Arcs are added only if they do not create loops, nor multiple arcs with common source and target.)

\medskip

\STATE 
\FOR{each }
\STATE {\bf if}  {\bf then} ;  {\bf else}   ;  {\bf end if}
\STATE add the R-arc  to 
\STATE add the B-arcs  to  {\sl \hfill // according to (\ref{eq:Bc})}
\ENDFOR
\STATE  the set of all NB-constraints  deduced from   {\sl \hfill // according to (\ref{eq:NBc1})}
\STATE  Build-Closure()
\STATE remove from  all NB-constraints  for which a setting is already found
\IF{ is not a DAG}
\STATE output "No"
\ELSE
\STATE output ()
\ENDIF}
\end{algorithmic}
\label{algo:Arcs}
\end{algorithm}

\begin{algorithm}[t,boxed]
\caption{The Build-Closure algorithm}
\begin{algorithmic}[1]
{\small \REQUIRE A simple directed graph  with vertex set , a set  of NB-constraints on .
\ENSURE  The NB-transitive closure of  using the NB-constraints in . \\
\hspace*{-0.6cm}({\sl  Note:} Arcs are added only if they do not create loops, nor multiple arcs with common source and target.)

\medskip

\WHILE{a -arc or an -arc  may be added}
\STATE add  to  
\ENDWHILE
\STATE output()}
\end{algorithmic}
\label{algo:Closure}
\end{algorithm}





\begin{itemize}

\item[] either an NB-constraint  with  exists in ,  and  is an arc,

\item[] or an NB-constraint   with   exists in ,  and  is an arc.
\end{itemize}

\noindent Clearly, this na\"{\i}ve approach for {\sc -Betweenness} attempts to 
exploit all the -constraints. Unfortunately,  for some NB-constraints 
 Algorithm \ref{algo:Arcs} may provide no setting ({\em i.e.} neither the arcs  
and , nor the arcs  and  are present in ), as shown below. 
These constraints are called {\em silent NB-constraints}, and are returned by the algorithm together with , if  is a DAG (step 13).

\bex
Let  be defined on  by the following -constraints:



\noindent Figure \ref{fig:ex} shows the ,  and  arcs used by Algorithm \ref{algo:Arcs} to
build the directed graph deduced from . Vertices  and  are left apart in this figure, since the
constraints they are involved in allow only to place them at the beginning and respectively at
the end of the sought permutations.   The NB-constraints imposed by  (except those with  and )
are  with . When  and , both arcs involved in the NB-constraint are 
already in  (due to -constraints). For  and , all the arcs with  and  are built by
transitivity (although some of them may also be built using the appropriate NB-constraints), during the steps 8 
in Algorithm \ref{algo:Arcs}. For , the NB-constraint cannot be used, since none of the arcs exists
(and no other arc may be created by transitivity). Then we
have  at the end of Algorithm \ref{algo:Arcs}. 
Notice that the pairs  and  have correlated
directions in any setting, that is, either all three arcs have the source ,
or all three arcs have the target . For  and  this is due
to the NB-constraint , whereas for  this is due to the
transitivity ensured by the arcs  and .
\eex


\begin{figure}[t]
\vspace*{-1.5cm}
\begin{center}
\includegraphics[width=15cm]{Fig1prim.pdf}
\end{center}
\vspace*{-3cm}
\caption{{\small Directed acyclic graph  obtained from Algorithm \ref{algo:Arcs} using the directed
-profile . For simplicity reasons, vertices  and  and the arcs incident
to at least one of them are omitted. The pairs of vertices  and  are not arcs, they
show the silent NB-constraint   and the pair , that are related by 
coherent arc directions.  }}
\label{fig:ex}
\end{figure}


Our problem is now this one:

\bigskip
\noindent {\bf (P)} Given  and a set of silent NB-constraints, decide whether a setting is possible
for each silent NB-constraint such that the graph resulting by transitive closure is a DAG.
\bigskip



Unfortunately, the following result shows the difficulty of the problem:

\bfn \cite{guttmann2006variations}
Problem (P) is NP-complete even when the silent NB-constraints involve disjoint triples of vertices.
\label{claim:NPc}
\efn

Notice however that the graph  we obtain at the end of Algorithm \ref{algo:Arcs} may have particular 
features (that we have not identified) making that we are dealing
with a particular case of problem (P). Claim \ref{claim:NPc} shows therefore that our problem is
potentially difficult, but does not prove its hardness.
 


\br
From an algorithmic point of view, we may notice that with the output of Algorithm \ref{algo:Arcs}
we may easily find a parameterized algorithm for -{\sc Betweenness}. Given  and ,
we have  possible settings to test, thus resulting into 
an FPT algorithm with parameter  given by the number of silent NB-constraints. 
\er

\subsection{Further analysis of arc propagation}

With the aim of forcing the setting of some appropriately chosen silent NB-constraint,  
let us now analyze the impact of adding an arbitrary arc  to , where  
 and   are non-adjacent vertices from . Denote   the graph 
obtained from  be adding the arc , and  let  be the NB-transitive closure of 
,  {\em i.e.} the directed graph obtained by performing 
Build-Closure. 

Several definitions are needed before going further. Given an NB-constraint  , the vertex
 of  is called the {\em top} of the NB-constraint, whereas the pair   is called
the {\em basis} of the NB-constraint. An arc  is {\em new} if it is an arc of  but
not of , and is {\em old} if it is an arc of . New arcs are obtained using 
Build-Closure according to a certain linear order, resulting from the arbitrary choices made in step 1. This order is denoted , such that  means that  is created by Build-Closure before .
Then, the following claim is simple:


\bfn
For each new arc , there exists a series of new arcs   such that  ,  ,  for all  with  
and each arc , , is obtained from the preceding one  using one of the following cases:

\begin{enumerate}
\item  and  is either an old arc, or a new arc such that ; in this case  is a new -arc.
\item   and  is the basis of an NB-constraint of  
with top ; in this case  is a new -arc.
\item  and  is either an old arc,  or a new arc such that
; in  this case  is a new -arc.
\item   
and  is the basis of an NB-constraint from  with top ;
in this case  is a new -arc.
\end{enumerate}
\label{claim:imminduce}

\efn  



{\bf Proof.} In order to obtain  , we need to apply either the transitivity
(step 2 in Algorithm \ref{algo:Closure} for a -arc,  which gives cases 1 and 3), or an NB-constraint from 
(again step 2 in Algorithm \ref{algo:Closure}, but for an -arc, which gives cases 2 and 4). 
  
\bigskip



The sequence  is called a {\em setting sequence} for , whereas the index 
of an arc  is called its {\em range} in . From now on, the
case in Claim \ref{claim:imminduce} used to deduce one arc from the preceding one in a setting sequence 
is  indicated between the two arcs.

\bex
For the example in Figure \ref{fig:ex}, if  and , then 
is a setting sequence for  using case 4 followed by case 3 in Claim \ref{claim:imminduce} to go
from one arc to the next one. 
\label{ex:U}
\eex

Now, let  (respectively  
  ) be the subsequence of  
(respectively of ) obtained by replacing  {\em consecutive} 
copies of the same vertex with only one copy of that vertex. Equivalently, if  is 
an arc of , then the next arc is either   (cases 1 and 2 in 
Claim \ref{claim:imminduce}) or  (cases 3 and 4 in Claim
\ref{claim:imminduce}). Of course, we have  and .
 

\bex
Consider ,
and let  be the -profile of . For , apply Algorithm \ref{algo:Arcs} to obtain the
graph  and the set .  Then  - that the reader is invited to build it himself - is partitioned into three sets, respectively made of:
the vertices preceding the pair , the pair  (in this order, and with no intermediate vertex),
and the  vertices following the pair . The set  is , and thus
involves only vertices in the third set, which induces in
 the subgraph  with vertex set  and arcs
.
With , 
we have (see Figure \ref{fig:ex3}a) that  is a setting sequence
for  with  (thus ) and  (thus ).
\label{ex:HU}
\eex

\br
Notice that we could possibly have , for distinct , {\em i.e.} they
correspond to the same vertex of ), if two
arcs with the same endpoint are set in distant steps of the setting process represented by . We could also
possibly have   for some  if, for instance,  (with ) are distinct,
 are  distinct,  is a new arc and  is an old arc 
(making that the vertex  is equal to , and thus by transitivity - or case 3 in Claim \ref{claim:imminduce} - 
one sets ).
\label{rem:several}
\er


\begin{figure}[t]
\vspace*{-1.5cm}
\begin{center}
\includegraphics[width=15cm]{Fig3Prim.pdf}
\end{center}
\vspace*{-4cm}
\caption{{\small Old arcs (thin horizontal arrows) and new arcs (thick vertical arrows) used by 
the setting sequence  in Example \ref{ex:HU}. a) Using notation , , and , . b) Using 
the vertices in , and thus defining the graph . In both cases, the initial arc  is
a dotted arrow.}}
\label{fig:ex3}
\end{figure}


\br
Also note that for every pair of  arcs  and  from , we
have either  and  (when ), or  
and  (when ). It is therefore understood,
here and in the subsequent of the paper, that in case  for some ,
we make a clear difference between the arcs  of  and the arcs  of .
These arcs are incident with the same vertex of  but this vertex is called 
in the first case, and  in the second one.
\label{rem:HU}
\er

\noindent {\bf Example \ref{ex:HU} (cont'd)}. We have
 and , but when we refer to the new arcs containing  we only refer to
the arc  and when we refer to the new arcs containing  we only refer to the arc .
Similarly, when we refer to the new arcs incident with  we refer only to the arc  whereas
when we refer to those incident with  we mean the arcs  and .
\bigskip

In order to represent arc propagation, we need to look closer to the partial subgraph   of  
given by the set of {\em distinct} vertices  used in the
setting sequence , the arcs in  and the arcs used to deduce each arc of  from the previous one, 
using Claim \ref{claim:imminduce}. The graph  is defined as:
\medskip

 



\hspace*{1.85cm} 

\hspace*{1.9cm}

\hspace*{1.9cm}

\medskip

\noindent The graph  is called the {\em setting path} associated with  (or, alternatively, a {\em setting path}
for ). Notice that case 2 (respectively case 4) in Claim \ref{claim:imminduce} may be included in case 1 
(respectively case 3) when the basis is the arc  (arc  respectively). The definition of
 keeps as case 2 (respectively case 4) only the configuration not included in case 1 (respectively case 3).
See Figure \ref{fig:ex3}b).


Claim \ref{claim:imminduce} and the definition of  allow us to have a basis for 
future analysis, but also show us that the
choice of one arc  has effects that are difficult to measure accurately. 
The NP-completeness of the problem {\bf (P)} (see Claim \ref{claim:NPc}) comes from 
this complex arc propagation, which makes that different setting sequences with the
same initial arc may lead to conflicts, {\em i.e.} to circuits.

\bex
In Figure \ref{fig:ex2}, we present a configuration (which is a subgraph of ) showing that not each possible
setting is a correct setting, since imposing the existence of one arc  may induce
circuits in the graph . In this configuration, setting the arc  implies the additional
arcs  and , and thus the construction of a circuit. A -profile inducing such a configuration in the
associated DAG  is the following one (where ):
\medskip

  






\medskip

\noindent In this example, the -constraints in bold define the arcs needed to obtain
the configuration in Figure \ref{fig:ex2}, and some additional arcs. The elements
involved in these -constraints are, in every permutation with this -profile,
on the left of  and  (the minimum and maximum elements), which are neighbors and
in this order on each permutation. The remaining of the elements are on the right of  
and , and are intended to complete the set  without any
participation to the configuration.
\eex

\begin{figure}[t]
\vspace*{-1.5cm}
\begin{center}
\includegraphics[width=15cm]{Fig2.pdf}
\end{center}
\vspace*{-4cm}
\caption{{\small  Configuration where the addition of  the arc  sets
(among other arcs) the arcs  and , thus inducing a circuit.
The types of the arcs are the same as in Figure \ref{fig:ex}, but the pairs depending
on the setting of a silent NB-constraint are missing.}}
\label{fig:ex2}
\end{figure}



In order to find polynomial particular cases, we need to be able to control the
form of the setting paths, and this is what we do in the subsequent. To this end,
notice that:

\br
The vertices  and  belong to no setting path. Indeed, according to Remark \ref{rem:places0n+1}, it is assumed that they are definitely located at places
 and  respectively, and thus their relative positions with respect to any other
element are known. No arc incident to any of them may thus be added, as would be the
case if they belonged to some setting path.
\label{rem:no0n+1} 
\er



\section{Polynomial case for {\sc Directed -Betweenness}}\label{sect:Particular}

Say that a -profile  on  is {\it linear} if the inclusion between sets defines a linear order on the intervals
, , where the notation  denotes the set of integers  with .
We show in this section that the problem {\sc Directed -Betweenness} is polynomial for linear -profiles.

Given , let . In other words,  is the set of values  such that
 is the basis of an NB-constraint with top .

\bfn
Let  be a linear profile on  . Then the inclusion between sets defines a linear order denoted  on
the sets , .
\label{claim:totalNB}
\efn


{\bf Proof.} By contradiction, assume that  and  exist such that  contains
 and  contains . Then . 

In the case where  and , assume w.l.o.g. that . Then 
and thus , a contradiction.  The case where   and 
is similar.

In the case where  and ,
recall that by hypothesis  is linear, and thus either  or vice-versa.
If , then 
and with  we deduce that , a contradiction. If , then 
and with  we deduce that , a contradiction. 


\bigskip

Now, assume Algorithm \ref{algo:Arcs} has been applied for , and let  be its output, assuming  is a DAG.
To finish the algorithm for , we apply Algorithm \ref{algo:linear}. The following claim is easy but very useful.

\bfn
The vertex  chosen in Algorithm \ref{algo:linear} has the following properties:

\begin{itemize}
\item[]  is maximum with respect to the linear order  on the set

.
\item[]  does not belong to a basis, but is a top for all the basis defining constraints from .
\end{itemize}

\label{claim:b1}
\efn

{\bf Proof.} The first affirmation is clear by the choice of  in step 3 of the algorithm and 
Claim \ref{claim:totalNB}. The second affirmation is deduced by contradiction. If 
belonged to a basis  or  with top , then we would have 
since the basis  or  cannot have top  (the vertices of a basis are by definition distinct from
its top). The second part of affirmation  results directly from affirmation . 
\bigskip

In the next claims, we show the correctness of our algorithm.
To this end, each arc  of  (and thus of ) is called a {\em local new arc} 
with respect to ,  in order to make the difference with the arcs from  which are new but 
do not belong to , termed {\em non-local new arcs}. Similarly, a vertex  of  is 
a {\em local top} if there  exists  such that  
 is an NB-constraint used by , i.e. one of the arcs  and 
is deduced from the other in , using case 4.  The pair  is in this case a {\em local basis}. 
The symmetric definitions hold for a vertex  (instead of ). Note that a local basis has a unique local top,
by Remark \ref{rem:HU}.

For any vertex  , we also denote  the minimum  with 
such that  belongs to . 




\begin{algorithm}[t,boxed]
\caption{The Linear-Profile algorithm}
\begin{algorithmic}[1]
\REQUIRE The output  of Algorithm \ref{algo:Arcs} for a directed linear -profile  on . 
\ENSURE  A permutation  with the -profile .

\WHILE{}
\STATE 
\STATE Choose  s.t. 
\STATE Choose  such that . 
\STATE  Build-Closure
\ENDWHILE
\STATE  topologically sort 
\STATE Output()
\end{algorithmic}
\label{algo:linear}
\end{algorithm}

\bfn
Let  be a directed linear profile on   and
let   and  be chosen as in  Algorithm \ref{algo:linear}. Then the following
affirmations hold:

\begin{itemize}

\item[] Let  be a new arc of  and let  be setting sequence 
for  with arc sources   and arc targets .
Then there is no old arc  in , with .
\item[] All arcs  of  are old.
\end{itemize}
\label{claim:before5}
\efn


{\bf Proof.} To prove (a), we assume by contradiction that the affirmation is false,
and choose  and  such that the arc  is the smallest 
with respect to the order . Several cases occur.

\begin{itemize}
\item[{\sl i)}] If  is a local basis (case 2 in the definition of ), 
then  is also a top of it (by Claim \ref{claim:b1}(b)
and thus from the old arc  we deduce the existence of the old arc 
(computed by the call of Build-Closure in step 8 of Algorithm \ref{algo:Arcs}). But then
the choice of  is contradicted, since .

\item[{\sl ii)}] If  is an old arc (case 1 in the definition of , with an old arc), 
then  is also an old arc, computed by the call of Build-Closure in step 8 of Algorithm \ref{algo:Arcs}. As before, the choice of  is contradicted.

\item[{\sl iii)}] If  is a local new arc (case 1 in the definition of , with a local new 
arc), then this arc belongs to  and was built before  since
it must be built before its use. Then there exist  with  and 
 such that  and  are the same arc,
but with different notations due to its multiple use in  (see Remark  
\ref{rem:several}). In particular,  and  are the same vertex of , and
thus  is an old arc of , with . Once again, the choice of
 is contradicted, since .

\item[{\sl iv)}]  Finally, if  is a non-local new  arc (case 1 in the definition of , with a non-local new arc), 
then it was built before . Consequently, 
there exists a setting sequence  for 
with arc sources  and arc targets 
. In this setting sequence, we have that 
 is an old arc, and . Then, 
 ,
contradicting again the choice of  and .
\end{itemize}

To prove , assume by contradiction that some arcs  are  created by 
Build-Closure, and let  be the smallest of them according
to the order . Then in a setting sequence  for  with arc sources  
 and arc targets , we have 
for some  with  and . Then the pair  is
not a basis since  and by Claim \ref{claim:b1}(b),  belongs to no basis. 
Then,  is an arc. This arc cannot be old, since then recalling that
 we have that  is on old arc thus contradicting affirmation (a).
Then  must be a new arc. Now, we have by case 1 in Claim \ref{claim:imminduce}  that . Since  and 
we deduce that , thus contradicting the choice of . 
\bigskip
Say that a setting sequence  for  with arc sources  and arc targets  is {\em canonical} if  has the following properties:

\begin{enumerate}
\item[]  and (if it exists)  are distinct from , , and  is an old arc.
\item[] .  
\item[] , for all  with . 
\end{enumerate} 

\bfn
Let  be a directed linear profile on   and
let   and  be chosen as in  Algorithm \ref{algo:linear}. 
Let  be a new arc of .  Then, for each setting sequence  for  with
arc sources  and arc targets ,
there is a canonical setting sequence  for  with arc sources 
 and arc targets , and (whenever ) 
. 
\label{claim:shortpath}
\efn

{\bf Proof.} The proof is by induction on the range  of  (or, equivalently, of
) in a setting sequence  for . Recall that the arc with range 1 
is .

In the case , we have either  (when cases 1 or 2 in Claim 
\ref{claim:imminduce} are used to obtain the second arc), or  (when cases 3 or 4 
are used).  When  we are already done. When , by Claim \ref{claim:before5}(b)
we know that  is an old arc, and we are done.

In the general case, assume by inductive hypothesis that the claim holds for all
arcs with range less than  in some setting sequence, and that the range of  
(or, equivalently, of ) in  is .  We have two cases.

Case A. The arc preceding  in  is . By inductive hypothesis, for
 there is a canonical setting sequence  and 
(if ) , meaning that  when 
exists, and  when  does not exist. We have two (sub)cases:

\begin{enumerate} 
\item[A.1.]  When  exists, we have that  
(this is concatenation)
is a setting sequence for , in which   is obtained from  
using the
same case of Claim \ref{claim:imminduce} as used in . Notice that the case 3 with a new
arc  cannot appear, since then in any setting sequence  for 
 with arc sources  and arc targets , we have that 
 for some  and , implying that  is an
old arc (as ), a contradiction with Claim~\ref{claim:before5}(a). Then
only case 3 with an old arc, and case 4 may occur. Both cases imply that 
is an old arc, as follows. In case 3 with an old arc , the transitivity
using the old arc  implies indeed the construction of 
in step 8 of Algorithm \ref{algo:Arcs}. If  is a
local basis ({\em i.e.} case 4 is used), we deduce that  is a top for it, by
Claim \ref{claim:b1}(b). Now, since  is an old arc by inductive hypothesis,
we deduce that  is also an old arc obtained from the NB-constraint with top
 and basis  . Thus  is an old arc in all cases. Then
 is a setting sequence for ,
which is canonical if we ensure that  is distinct from all , .
This is guaranteed by Claim \ref{claim:before5}(a).
\item[A.2.] When  does not exist, we have that  is
a canonical setting sequence for . Indeed, as  we know that
 is obtained from  using case 3 or 4 in Claim 
\ref{claim:imminduce}. Moreover, by Claim \ref{claim:b1}(b),  belongs to no basis, 
thus  is an old or new arc. But the latter possibility is forbidden by 
Claim \ref{claim:before5}(b).
\end{enumerate}

Case B. The arc preceding  in  is . By inductive hypothesis, for
 there is a canonical setting sequence  and (if ) , meaning that  when 
exists, and  when  does not exist. We have two (sub)cases:

\begin{enumerate} 
\item[B.1.]  When  exists, we show that the sequence 
 is the sought canonical sequence.
Clearly,  is obtained from  using the setting sequence 
from which  is useless in this case. Also,  is obtained from
 and  by transitivity (case 3 in Claim \ref{claim:imminduce}).
It remains to show that  is deduced from  and .
In ,  is used to deduce  from ,
using either case 1 or case 2 in Claim \ref{claim:imminduce}. If case 1 is used, then
 is an arc (new or old), and it allows to deduce  from 
 using the transitivity. If case 2 is used, then 
is a local basis, thus  is a top of it. The resulting NB-constraint allows to deduce
 from   in this case too. 
\item[B.2] When  does not exist, we have that  and 
is a canonical setting sequence for .


 

\end{enumerate}


\bfn
Let  be a directed linear profile on  . Then 
the NB-transitive closure  obtained in step  5 of Algorithm \ref{algo:linear} 
when  (respectively ) are chosen as in step 3 (respectively step 4) has no circuit.
\label{claim:circuit}
\efn


{\bf Proof.} Assume a circuit , , is created in . Because of
the transitive closure, a shortest such circuit has length 2. Let then  form
a -circuit and
assume that (at least)  is a new arc. Then, according to Claim \ref{claim:shortpath},
there exists a canonical setting path with vertices  and 
  ().  
Consequently  cannot be an old arc, since then
in  either we have directly that  is an old arc (when  and thus 
or the transitivity guarantees the same conclusion when . But this  yields  a contradiction 
with Claim \ref{claim:before5}(a).

We deduce that  is a new arc, implying again the existence of a canonical setting path
with vertices  and   ().
But  and . Consequently  we have either that  (when ) or
 that  is an old arc (when ). In the former case we have a contradiction
with affirmation  in the definition of a canonical setting path since . In the latter case, we have again a contradiction with Claim \ref{claim:before5}(a). 
\bigskip

We are now ready to prove the main theorem:
 
 \begin{thm}
 {\sc Directed -Betweenness} is polynomial for linear -profiles.
 \end{thm}

{\bf Proof.} Given a linear -profile , we first apply Algorithm \ref{algo:Arcs} and,
if it returns a pair , we apply Algorithm \ref{algo:linear}. To  show the correctness
of the algorithm, we show the answer is "No" iff there is no permutation whose -profile is .

If the answer is "No", then Algorithm \ref{algo:Arcs} returns that  is not a DAG, which occurs
iff some -constraints from  cannot be simultaneously satisfied. Thus, there is no permutation 
with -profile .

Now, assume there is no permutation whose -profile is , and suppose by contradiction that
the algorithm returns a permutation . We show that  satisfies all the -constraints in , 
yielding a contradiction with the hypothesis. The permutation  is output at the end of Algorithm \ref{algo:linear},
showing that Algorithm \ref{algo:Arcs} finishes with a DAG . 
Then, in
Algorithm \ref{algo:linear} every execution of the {\bf while} loop in steps 1-6 satisfies at least
one silent NB-constraint and, by Claim \ref{claim:circuit}, creates no circuit. Therefore, 
the pair  obtained at the end of each execution consists again in a DAG  
with , , - and -arcs, and a set  with smaller size than the previous one.
Thus the {\bf while} loop ends when  and yields a DAG  that satisfies
all the constraints imposed by the -profile . Any topological order of , including , is thus
a permutation with -profile . The hypothesis that there is no permutation with
-profile  is thus contradicted.

The execution time of the algorithm is clearly dominated by the  computations of
the NB-transitivity closure in step 5 of Algorithm \ref{algo:linear}. Now, the number
of NB-constraints in  is in  (we have at most one NB-constraint  for each
 and each ) and the NB-transitivity closure is clearly performed in polynomial time,
thus the resulting algorithm is polynomial.  
 

\section{Generalizations}\label{sect:Generalizations}

In this section, we generalize the definition of -profiles so as to allow
them to carry different amounts of information, depending on an integer parameter .

\bdefin
Let  be a positive integer with . The {\em -profile of a permutation}  on  
is the set of {\em -constraints}



\noindent where  ( respectively) is the minimum (maximum respectively) value 
in the interval delimited on  by the element  (included) and the element  (included). 
\edefin

 
Note that -profiles as defined in Section \ref{sect:def} are the -profiles.
A -profile is thus a -profile augmented with longer-range information of the same
nature as the -profile itself,  for pairs  with  at most equal to .
Then all the definitions  related to -profiles generalize to -profiles in an obvious way,
allowing us to state the following variant of the -Betweenness Problem:

\bigskip

\noindent {\sc (directed or not) -  Betweenness}

\noindent{\bf Input:} A positive integer , a (directed or not) -profile  on . 

\noindent{\bf Question:} Is there a permutation  on  whose -profile is ? 
\bigskip

Similarly to the -{\sc Betweenness} problem, the  - {\sc Betweenness} problem provides a -profile,
which imposes B-constraints and NB-constraints for the permutations associated with it, if any. 
The existence of at least one permutation raises the question of its uniqueness,
allowing to know whether the permutation is characterized by its -profile or not.
More formally, we state the two following problems:

\bigskip

\noindent {\sc (Directed or not) -Reconstruction}

\noindent{\bf Input:} A positive integer .

\noindent{\bf Requires:} Find the minimum value of  such that (directed or not) - {\sc Betweenness} 
has at most one solution, for each possible -profile  on .
\bigskip

\noindent {\sc (Directed or not) Unique - Betweenness}

\noindent{\bf Input:} A positive integer , a (directed or not) -profile  on . 

\noindent{\bf Requires:} Decide whether  is the -profile of a unique permutation on ,
or not. In the positive case, find the unique permutation associated with .
\bigskip

Problems - {\sc Betweenness} and  {\sc Unique - Betweenness} are clearly related,
but do not allow easy  deductions in one sense or the other. For instance, even if we have
a solution for the {\sc Directed -Betweenness} in the case of a linear profile 
(see Section \ref{sect:Particular}), we know nothing about the uniqueness of the permutation 
the algorithm outputs (when such a permutation exists).





In the subsequent, we solve the -{\sc Reconstruction} problem in the undirected case,
and give a lower bound for the directed case.  We assume wlog that the 
-profiles we use are compatible with the assumption that 0 and  are respectively 
the leftmost and the rightmost element in the permutations we are dealing with. 
Then we prove the following result:

\begin{thm}
The minimum  in (directed or not) -{\sc Reconstruction} satisfies:

\begin{enumerate}
\item[]   in  -{\sc Reconstruction}.
\item[]  in {\sc Directed -Reconstruction}, for . For , we have .
\end{enumerate}
\label{thm:bounds}
\end{thm}

The proof is based on the following claim. 
\begin{fait}
Let  be a positive integer, and  be a (directed or not) -profile on  . Then: 

\begin{enumerate}
\item In all the permutations whose -profile is  (if any),
the elements  and  have precisely the same positions, denoted  and 
\item If  and , then the sets  and  of elements 
situated  respectively strictly between the positions 0 and  (for ),  and  (for ), 
 and  (for ) are the same over all the permutations with -profile  (if any).
\end{enumerate} 
\label{claim:1n}
\end{fait} 

\noindent{\bf Proof.} Assuming at least one permutation corresponding to  exists, let  be such
a permutation. Denote  the position of  on  and successively consider the B-constraints

 

\noindent The first B-constraint places
 on the left of  iff , the second one places  on the opposite side of  with respect to
 iff  and so on. Each element in  is deterministically placed on the
left or on the right of  depending only on those B-constraints.  As a consequence,  is at the same place 
 in all permutations corresponding to .

A similar reasoning may be done with the element  and the B-constraints: 

 

\noindent We similarly deduce that  is at the same place  in all permutations corresponding to ,
and the sets of elements situated respectively on its left and right are the same in all permutations.

Putting together the previous deductions, whatever the order of  and ,
we have that - on the one hand -  and  are identical in all permutations, and
- on the other hand -  and  are identical in all permutations. The conclusion follows. 
\bigskip


{\bf Proof of Theorem \ref{thm:bounds}.}  We now prove affirmations  and .

\noindent{\sl Proof of affirmation .} For , it is trivial.
When  it is easy to prove,  using Claim \ref{claim:1n}, that the -profile guarantees 
the uniqueness of the associated permutation.
When , assume by contradiction that   and  let  be a permutation on  
whose elements in positions 1 to 4 are , ,  and .
Let  be the -profile of .
According to Claim  \ref{claim:1n}, the elements  and  are situated respectively at positions
3 and 4 in all permutations associated with , and positions 1 and 2 are occupied (whatever the order) 
by the elements  and . Now, in  the 
-constraints involving one of the elements  and   and another
element following  on its right are useless for fixing the places of  and  
since these constraints have the minimum and maximum element  and .
The only possibly useful -constraints are those involving  and , but these 
integers have pairwise difference larger than  except for  and . Now,   and  are 
involved in the -constraint , which does not fix them on the places 1 and 2 of
the permutation. Thus, there are at least two permutations with -profile , a contradiction. 
We thus have .
 
We now show that if , then there is at most one permutation on  whose -profile is 
. This is shown by induction on . 

When  and , Claim \ref{claim:1n} guarantees that, if at least one permutation with the
given -profile exists, then  and  have fixed places, and 
 (respectively ) is located in the  same set among   in all suitable permutations.
If  and  are in different sets, then the uniqueness is guaranteed. Otherwise, either  and  are in a set delimited
by the position of , and then the constraint  allows
to deduce whether  separates  and  or not (thus fixing the positions of 2 and 3), 
or they are in a set delimited by , and then the
constraint  allows to deduce whether  separates  and  or not.
In all cases, all the elements are located at fixed places, thus the permutation associated with the
-profile is unique.

Assume now, by inductive hypothesis, that for all ,  a -profile either has no
associated permutation, or has exactly one. Let now  be a -profile for permutations on 
,  and let  be defined according to  Claim \ref{claim:1n}, assuming at least one
permutation exists. Denote  any of these permutations, extended with  and . 
Let , if ,  and , otherwise. We show
that:




Note that , whose element set is ,  is a subpermutation of  delimited by  and ,  
which are respectively the  
minimum and maximum element in . Now, renumber the elements of   from  
to  according to their increasing values, where  and  is at position . 
Then the resulting permutation is a permutation  on  augmented with  and .

Denote   the -profile of this permutation, and let us show  that  is unique.
For , we show that when  is known, 
 is also known, and then apply inductive hypothesis  to deduce that  (and thus ) fixes 
the places of the elements in .   Whereas for  
we show that there are enough 1-constraints deduced from    to guarantee the uniqueness of .
The case  is trivial.

Let  be a constraint on , which belongs to  if 
({\em i.e.} ) and to  if .   Let  and  
be respectively the labels of  before renumbering. Then the difference between the labels
of  and  in the initial  satisfies:



\noindent Indeed, if  elements of  are between  and , then the total number of elements in 
is, on the one hand,  (the cardinality of ) and, on the other hand, 
(given by  the cardinality of , by  and by the element ,). Then , and it represents
the maximum number of elements that can miss between  and , additionally to the values
separating them in , {\em i.e.} . But then from equation (\ref{eq:this}) we deduce: 



{\sl Case .} From equation (\ref{eq:bh}) we deduce with  that ,
meaning that  is
a constraint from , yielding the constraint 
of  after renumbering. Of course, this affirmation is true since the renumbering
keeps the order between the elements, and thus the (renumbered) minimum and maximum value of each
given interval. Thus  is deducible from  and, by inductive hypothesis, the
permutation  is uniquely determined by .

{\sl Case .} Then, as assumed above,  and thus  equation (\ref{eq:bh}) implies 
 which is larger than . This shows that all the -constraints
on  with  are deducible from constraints in  but the 
-constraints on  with  are not. These latter -constraints 
are obtained when  (according to equation (\ref{eq:bh})), that is, when
. To achieve this with  and the other two elements  in  (w.l.o.g. assume
) we must have either , and thus , (such that ), or  and  (such that
), or  and  (such that ). In all cases, exactly one -constraint is missing
({\em i.e.} not resulting from ) but the uniqueness of the permutation is still guaranteed,
since the two other -constraints are sufficient to fix the elements in a -permutation (including
the endpoints  and ).

{\sl Case .} Using (\ref{eq:bh}) and the information that , we deduce that 
, and thus all the -constraints are available for .
As the theorem is true for permutations on  elements, then we are done.

Affirmation (\ref{eq:three}) is proved.
Similarly, we show that the places of the elements situated on each permutation  between the element 
(in position ) and the element  are fixed. Thus, all the elements of each permutation  are  in
fixed places, and there is only one permutation  with the -profile .  
\medskip

\noindent{\sl Proof of affirmation .} Similarly to the undirected case, in the directed case assume by contradiction that 
 and build  as in the
undirected case, but with  instead of  (thus avoiding  the directed
-constraint  or  , which fixes in the directed case
the positions of   and ). 
Then, with the directed -profile  of , no -constraint exists involving , 
and thus  and  may permute on the positions  and . Therefore, at least two permutations exist
with the -profile , a contradiction.

Thus the uniqueness of the permutation implies . 



\section{Conclusions and Perspectives}\label{sect:Conclusion}

In this paper, we investigated some problems related to the construction of
a permutation from a -profile or, more generally, from some -profile,
with . For the first of these problems, the -{\sc Betweenness} problem,
we noticed the main difficulties of the directed version and gave a polynomial particular case.

The undirected version is even more difficult, due to differences with respect to the
directed version that we  present hereafter. First, as the relative position of  and 
({\em i.e.} the arc of  between  and ) is not directly given by the 
-profile, the B-constraints cannot be directly exploited as in steps 3-5 of 
Algorithm \ref{algo:Arcs}. The construction of those two types of arcs, the -arcs and
the -arcs, must therefore be integrated into the Build-Closure algorithm, 
where the B-constraints must be considered as well as the NB-constraints when seeking  
new arcs to be added to . It may be noticed that, similarly to the case of the NB-constraints,  
any of the B-constraints  has two possible settings, resulting either in the  
set of new  arcs 
, or in the set of new arcs
. When one arc is set, then
the four other arcs are set accordingly. When no arc is set, the B-constraints are silent.
The algorithm obtained from Algorithm \ref{algo:Arcs} by performing the indicated changes
thus outputs either the answer No, or  and two  sets  and  of
silent NB- and silent B-constraints respectively.
We thus arrive at the second main difference between the directed and undirected case. 
Any setting sequence must allow to deduce new arcs also using the B-constraints, 
thus adding  cases to those already in Claim \ref{claim:imminduce}, and yielding the
study of the arc propagation in  even more complicated that in the directed case.

For both versions, and also for the more general - {\sc Betweenness} problem, the 
algorithmic difficulty of the problem is an open problem. The same holds for the
{\sc Directed -Reconstruction} problem. Also, being able to recognize
a -profile allowing to reconstruct exactly one permutation, {\em i.e.} solving
(Directed or not) {\sc Unique} - {\sc Betweenness}, would allow to
identify a subclass of permutations perfectly represented by their -profile.


\bibliographystyle{plain}
\bibliography{ProfPerm}
\end{document}
