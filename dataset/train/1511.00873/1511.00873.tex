\iffalse
Notation:
- u1,u2,u3: the outer face
- v1, v2, ...: vertices of a canonical ordering
- V1, V2, ..., VL: set of the canonical ordering
- k: the running index for vertex-sets of the canonical ordering
- boundaries of k: written as 1<k<L, unless incorrect
- f: the length of a fan
- z1,...,zf: vertices of a fan.  Also vertices of a vertex group.
- i,j,h: running index for some set of vertices
- ell: length of the outer face
\fi




\documentclass[12pt]{article}

\usepackage{fullpage}
\usepackage{latexsym}
\usepackage{epsfig}
\usepackage[usenames,dvipsnames,svgnames,table]{xcolor}
\usepackage{amsmath,amsthm,amssymb}
\usepackage[textsize=footnotesize]{todonotes}




\newtheorem{theorem}{Theorem}
\newtheorem{lemma}{Lemma}
\newtheorem{definition}{Definition}
\newtheorem{corollary}{Corollary}
\newtheorem{conjecture}{Conjecture}
\newtheorem{observation}{Observation}
\newcommand{\smallitem}{\vspace*{-2mm}\itemsep -2pt}
\newcommand{\settab}{-----\=-----\=-----\=-----\=-----\=-----\= \kill}
\newcommand{\comment}[1]{{\color{red} #1}}
\newcommand{\student}[1]{#1}
\newcommand{\postdoc}[1]{#1}

\pagestyle{plain}

\title{The -ordering for 4-connected planar triangulations}
\author{Therese Biedl
\thanks{David R.~Cheriton School of Computer Science, 
University of Waterloo, 
Waterloo, ON N2L 3G1, Canada, {\tt \{biedl,mderka\}@uwaterloo.ca}. Research supported by NSERC. The second author is supported by Vanier CGS. }
\and
Martin Derka
\addtocounter{footnote}{-1}
\footnotemark
}
\date{\today}

\begin{document}
\maketitle

\begin{abstract}
Canonical orderings of planar graphs have frequently been used in graph drawing and other graph algorithms. In this paper we introduce the notion of an
-canonical order, which unifies many of the existing variants of
canonical orderings.  We then show that -canonical ordering
for 4-connected triangulations always exist; to our knowledge
this variant of canonical ordering was not previously known.
We use it to give much simpler proofs of two previously known graph drawing 
results for 4-connected planar triangulations, namely, rectangular duals 
and rectangle-of-influence drawings.
\end{abstract}


\section{Background}
\label{se:intro}


A canonical ordering of a planar graph is a way of building the graph by iteratively attaching vertices to some ``basic graph'' (such as an edge) while preserving some connectivity invariant after each iteration. This concept was introduced in the late 1980's by de Fraysseix, Pach and Pollack~\cite{FPP90}. They used the canonical ordering to show that planar graphs can be drawn on a grid of size . Subsequently, canonical orderings became one of the main tools in graph drawings, e.g.~for drawing graphs in grids of small dimensions (see e.g. \cite{FPP90,CN98}), rectangular duals~\cite{KH97}, and also graph algorithms such as encoding planar graphs \cite{HKL99} or finding -disjoint trees in planar graphs~\cite{NRN97,NN00}. 


\paragraph{Our contribution}
There is now a number of variations of canonical orderings, depending on
the connectivity of the graph and whether it is triangulated or not.  (We
will review these below.)
In this paper, we show the existence yet another canonical ordering, this one  for planar 4-connected triangulations.  It is substantially different from the canonical ordering for such graphs that was presented by Kant and He~\cite{KH97}. 
We call this the -canonical ordering.  More generally, we introduce
the concept of an -canonical ordering, which (roughly speaking) means
that the partial graph must be -connected and the rest-graph must be
-connected; the existing canonical orders all fit into this framework.


We use the -canonical ordering to provide alternate (and, in our opinion, significantly simpler) proofs of two previously known results about 4-connected planar triangulations: they have rectangular duals (Section~\ref{sec:rd}) and rectangle-of-influence drawings (Section~\ref{sec:ri-drawing}). 


\section{Review of existing canonical orderings}

We assume that the reader is familiar with planar graphs (refer e.g.~to \cite{Die12}).  We use the
term {\em triangulation} for a maximal planar simple graph, i.e., a graph in which all faces
are triangles and which has  edges of which none is a multiple edge or a loop.  Such a graph has a unique planar embedding; we further
assume that one face has been fixed as the outer face.
We begin our review of canonical ordering with the
one for triangulations introduced by de Fraysseix et al.~\cite{FPP90}. 
We paraphrase their definition to the following one (which is easily shown to be equivalent):

\begin{definition}[Canonical ordering for triangulations~\cite{FPP90}]
\label{def:triangulated-co}
Let  be a triangulation with outer face . A vertex ordering
 is called a \emph{canonical ordering} if 
\begin{itemize}
\item , , ,
\item For every 
the subgraph  of  induced by vertices  is -connected. 
\end{itemize}
\end{definition}

As we will see later, it will be convenient to define  and so  becomes a partition
of the vertex set.  For any such partition and an index , we use the notation  for the subgraph induced by 
 and we let the {\em complement}
 of  be the subgraph induced by the vertices . 
Note that vertex set  belongs to both  and .
\footnote{Some references instead define  to be the subgraph induced by
.  This complicates stating some of the conditions.}  

One can observe that in a canonical ordering for a triangulation, the complement  is a connected graph for all . 
This holds because any vertex  is not on the outer face and so there must exist some minimal  where  is
not on the outer face of .  Due to the triangular faces,  receives an edge to , and iterating the
argument, hence has a path within  that leads to .

We note here, without giving details, that this canonical ordering
has been generalized to 3-connected planar graphs that are not necessarily
triangulated \cite{Kant96}, and also to non-planar
3-connected graphs (see \cite{Schmidt-ICALP14} and the references therein).
\iffalse
Note that triangulations are 3-connected. Kant~\cite{Kant96} generalized canonical ordering to all planar -connected (not necessarily triangulated) graphs. 
This generalization (see Definition~\ref{def:triconnected-co}) uses the notion of internal 3-connectivity. We say that a graph is \emph{internally -connected} if it is 2-connected and for every 2-cut , one of the connected components of  consists of at most  vertex.  
\todo{We later have internally -connected.  Should unify these definitions.}
\todo{I don't think the current definition of internally -connected is correct.  I think we allow a path of vertices as one component.}
Furthermore, this canonical ordering refers to an ordered partition of vertices rather than ordering of the vertices themselves. 

\begin{definition}[Canonical ordering for -connected planar graphs~\cite{Kant96}]
\label{def:triconnected-co}
Let  be a -connected planar graph with an edge  on the outer-face. Let  be an ordered partition of  and , , the subgraph of  induced by . Let  denote the outer-face of . The partition  is a \emph{canonical ordering} of  if:
\begin{itemize}
\item  and , where  lies on the outer-face of  and is a neighbor of .
\item Each  is a cycle containing edge .
\item Each  is -connected and internally -connected.
\item For any , one of the following holds:
\begin{itemize}
\item  is either a singleton  where  belongs to  and has at least one neighbor in ; or
\item  where each , , has at least one neighbor in , and where  and  each have precisely one neighbor in  and these are the only two neighbors of  in .
\end{itemize}
\end{itemize}
\end{definition}
The canonical ordering guarantees that at each step, the graph  is -connected and its complement in  is connected. 

In view of Definition~\ref{def:triconnected-co}, one can redefine the canonical ordering for triangulations (cf.~Definition~\ref{def:triangulated-co}) as a vertex partition  where every set  is a singleton and the remaining condition hold. This unifies the notion graphs  and their complements  for canonical orderings from both the Definitions~\ref{def:triangulated-co} and~\ref{def:triconnected-co} and in all the canonical orderings that we refer to in this paper. This allows us to define the following:
\fi

In 1997, Kant and He~\cite{KH97} showed that one can define a different canonical ordering for 4-connected triangulations, and used it to construct visibility representations of 4-connected planar graphs. 
Its definition, slightly paraphrased, is as follows:

\begin{definition}[Canonical ordering for 4-connected triangulations~\cite{KH97}]
\label{def:22-ordering}
Let  be a 4-connected triangulation with outer face . A vertex order
 is called a \emph{canonical ordering for 4-connected triangulations} if 
\begin{itemize}
\item , , ,
\item For every , graphs  and  are -connected. 
\end{itemize}
\end{definition}

This canonical ordering was extended to a canonical ordering for all planar 4-connected graphs (not necessarily triangulated) by Nakano, Rahman and Nishizeki~\cite{NRN97}. 
Versions of a canonical order for 4-connected non-planar graphs are also known \cite{CLY05}.

Going one higher in connectivity,
Nagai and Nakano \cite{NN00} introduced a canonical ordering
for 5-connected triangulations.  Here, vertices are added in sets that
are sometimes more than a singleton.  We need a definition.
Let  be a graph where all interior faces are triangles.  A {\em fan} 
of  is a subset of
vertices  that induces a path with
 for all .  We will only apply this
concept if all vertices in the fan belong to the outer face of .
Since interior faces are triangles, it follows that for all 
the third neighbor (i.e., the one not on the outer face) is the
same vertex.   See also Figure~\ref{fig:fan}(right).

\begin{definition}[Canonical ordering for 5-connected triangulations~\cite{NN00}]
Let  be a 5-connected triangulation with outer face . A partition of the vertices  is called a \emph{canonical ordering for 5-connected triangulations} if 
\begin{itemize}
\item ,
\item  consists of all neighbors of  and ,
\item ,
\item  consists of all neighbors of ,
\item For , vertex set  is either a single vertex or a fan,
\item For every , graph  is -connected and
graph  is -connected. 
\end{itemize}
\end{definition}

This canonical ordering was used to find 5 independent spanning trees in 5-connected triangulations~\cite{NN00}. To our knowledge, it has not been generalized to planar 5-connected (not necessarily triangulated) graphs, and not to non-planar 5-connected graphs either.    Since no planar graph is 6-connected, no canonical orderings for higher connectivity can exist for planar graphs.

Note that the three canonical orderings listed here are very similar, with the essence being the
connectivity that is required of the subgraphs and their complements.
In light of this, we 
aim to unify the three definitions with the following:

\begin{definition}[-canonical ordering]
Let  be a triangulation with outer-face .
We say that 
a vertex partition  
is an \emph{-canonical ordering} if 
\begin{itemize}
\item  belongs to  and  belongs to , and
\item for every , graph  is -connected and  is -connected.
\end{itemize}
\end{definition}

Note that this definition is deliberately vague on the exact form that the vertex sets  must have.
In particular, nothing prevents us (yet) from setting  and , which satisfies all conditions.
The existing canonical orderings restrict  to be a singleton or, for -connected triangulations, fans.
Thus the above definition should be viewed as a class of definitions, to be refined further by
stating restrictions on the vertex sets .

Rephrasing the existing canonical orders in the above terms,
the canonical order for triangulations becomes a -canonical ordering with only singletons, the one for 4-connected triangulations becomes a -canonical ordering with only singletons, and the one for 5-connected triangulations becomes a -canonical ordering with only singletons or fans. The reader will notice that the sum of the two numbers equals the connectivity of the graph.  Pushing this further, one may ask whether any -connected
triangulation has an -canonical ordering such that each  has some simple form.
Note that we may assume that , since a reversal of an -canonical ordering gives an -canonical ordering.
We study here -canonical ordering for 4-connected triangulations, under the restriction that each  is a singleton or a fan.
To our knowledge no such ordering was known before.








\section{-canonical orderings}


We have already given the broad idea of a -canonical ordering
earlier.  We re-state it here and give the specific
restrictions imposed  on the vertex sets.  See also`Figure~\ref{fig:fan}.

\begin{figure}[ht]
\hspace*{\fill}
\includegraphics[width=40mm,page=1]{31-canonical.pdf}
\hspace*{\fill}
\includegraphics[width=40mm,page=2]{31-canonical.pdf}
\hspace*{\fill}
\caption{A singleton  and a fan  in a -canonical
ordering.}
\label{fig:fan}
\end{figure}

\begin{definition}
Let  be a 4-connected triangulation with outer-face .
A {\em -canonical order with singletons and fans} is a partition 
such that
\begin{itemize}
\item , where  is the third vertex of the
	interior face adjacent to .
\item .
\item For any , set  is either a singleton or a fan.
\item For any , graph  is -connected 
	and  is connected.
\end{itemize}
\end{definition}

In what follows, we will omit the ``with singletons and fans'', as we
will not study any other version of -canonical orderings.
Our main goal is to show that every 4-connected triangulation
has such a -canonical ordering.  The proof of this proceeds
by induction, and we state the crucial lemma for the induction
step separately first.  We need a few definitions.

A plane graph is called a {\em triangulated disk} if every interior
face is a triangle and the outer-face is a simple cycle.
A triangulated disk is called {\em internally 4-connected} if its
outer-face has no chord, and every triangle is a face.
Observe that a triangle is an internally 4-connected triangulated disk,
and so is any 4-connected triangulation.  Also observe that a subgraph
of an internally 4-connected triangulated disk is again an internally
4-connected triangulated disk if and only if its outer-face is a simple
cycle that has no chord.



\begin{lemma}
\label{lem:find}
Let  be an internally -connected triangulated disk with .  
Let  be an edge on the outer-face. 
Then there exists a vertex set  such that
\begin{itemize}
\item  contains only outer-face vertices, and none of .
\item  is an internally -connected triangulated disk.
\item  is a singleton or a fan.
\end{itemize}
\end{lemma}


\begin{proof}
\footnote{The proof is strongly inspired of the one 
for a -canonical order in 5-connected graphs \cite{NN00}.
Since we demand
less on our -canonical order, we can simplify the exposition
somewhat.}
Enumerate the outer face vertices of  as  in
clockwise order.  Define a {\em 2-leg} to be a path  where
 and  is not on the outer-face.
Vertex  is called a {\em 2-leg-center}.  We always have at least one
2-leg (namely, the one consisting of  and their common 
neighbor at the interior face incident to ; this vertex is 
interior since  has no chord and at least 4 vertices).

We say that a 2-leg-center  {\em dominates} a 2-leg-center  if vertex
 is strictly inside the cycle  formed by 
some 2-leg  with center-vertex .  See also
Figure~\ref{fig:2leg}(left).
The dominance-relationship 
is acyclic since any 2-leg with center-vertex  must enclose strictly 
fewer faces than the 2-leg .  Therefore we must have some
{\em minimal} 2-leg-centers, which are the ones that do not dominate any
other 2-leg-center.

By definition for any 2-leg , we have  and so 
there exists at least one vertex between  and 
on the outer-face.  
We say that a 2-leg  is {\em basic} if the vertices
 all have degree 3, and {\em complex} otherwise.
Note that if  is basic, then  form
a fan and their common neighbor is .

\begin{figure}[ht]
\hspace*{\fill}
\includegraphics[width=40mm,page=1]{2leg.pdf}
\hspace*{\fill}
\includegraphics[width=40mm,page=6]{2leg.pdf}
\hspace*{\fill}
\includegraphics[width=40mm,page=4]{2leg.pdf}
\hspace*{\fill}
\caption{(Left) 2-leg center  dominates both  and .
(Middle)  If all 2-legs containing  are basic, then we can remove a fan.
(Right) If  is complex, then removing  leaves
an internally 4-connected triangulated disk.}
\label{fig:2leg}
\end{figure}

Let  be a minimal 2-leg center.
We have two cases:
\begin{itemize}
\item All 2-legs containing   are basic.

Let  be minimal and  be maximal such that  is adjacent
to  and .  
See also Figure~\ref{fig:2leg}(middle).
Since  is a 2-leg-center, we have .
By case assumption the 2-leg  is basic, so 
 is a fan. 
We verify that  is an internally 4-connected triangulated disk:
\begin{itemize}
\item The outer-face of  consists of the one of  plus .
	By definition of a 2-center  was not on the outer-face, so
	 is a triangulated disk.
\item Since  had no chord, the only possible chord of  would be
	incident to vertex .  But by choice of  and  the only
	neighbors of  on the outer-face of  are  and .  So
	 has no chord.
\end{itemize}


\item Some 2-leg  is complex.

We assume that  has been chosen maximally, i.e., so 
that  is
either not a 2-leg or not complex.
We claim that in this case  is a suitable vertex set.

We first show that  cannot be adjacent
to .  Assume for contradiction that it is, then  is
a triangle and hence a face.  If there were some 
with  and , then this would make
 a complex 2-leg, contradicting the choice of .
So all of  (if any) have degree 3, and they
form a fan with common neighbor .    In particular, edge 
exists, which means triangle  is a face, forcing
.  But then  is basic, not complex.
This is a contradiction, so  is not a neighbor of .

Let  be the neighbors of
 in ccw order.  
See also Figure~\ref{fig:2leg}(right).
None of  can be on the outer-face
of , else  would have a chord.  The outer-face of 
consists of , and so
this is a simple cycle and  is a triangulated disk.   Further, we
can show that it has no chord:
\begin{itemize}
\item If a chord of  connected two vertices in 
,
then it would also be a chord in , which is excluded.
\item If a chord connected two non-consecutive vertices in
, then in  there would be an edge between two
non-consecutive neighbors of , implying a triangle that is not a face.
\item If a chord connected some , , with some 
, , then  would be a 2-leg in .
By minimality of  hence , but this contradicts that  is
not adjacent to .
\item If a chord connected some , , with some 
,  or , then by  it would have to cross 
or , contradicting planarity.
\end{itemize}
So  is an internally 4-connected triangulated disk.  
\end{itemize}
Observe that in both cases

for some , and so  does not contain  or 
as desired.
\end{proof}



\begin{theorem}
Let  be a 4-connected planar triangulation. 
Then  has a -canonical order.
\end{theorem}
\begin{proof}
We choose the vertex set in reverse order.    Let  be
the outer-face and choose ; this satisfies all conditions
since  has at least 3 neighbors.  (We do not at this point know the
correct value of , but simply assign indices backwards and shift
indices at the end so that the vertex sets are numbered .)

Observe that 
is an internally 4-connected triangulated disk, because the neighbors
of  form a simple cycle without chord (else there would be a separating
triangle at ).
Assume now some  have been chosen already such that
the remaining graph  is an
internally 4-connected triangulated disk with  on the outer-face.
If  has at least 4 vertices, then apply Lemma~\ref{lem:find} to find the next 
.  
Graph  is again internally 4-connected, so 
we can continue choosing vertex sets until only 3 vertices, including
 and , are left.
Since the graph is still internally 4-connected, these vertices must
be a triangle, and hence a face of .  So setting  to be the
three vertices of this triangle gives the desired ordering.

To observe that the required connectivity holds, note that any
internally 4-connected graph is 3-connected since it is a
triangulated disk without a chord.  To see that 
is connected, it suffices to show that every vertex except 
has a neighbor in a later vertex set; the set of these edges
then forms a spanning tree in .  The argument for
this is nearly the same as for -orderings.  Clearly 
each of  are adjacent to .  For any vertex ,
vertex  is not on the outer face of , and hence there must
exist some minimal  such that  is on the outer face of
, but not on the outer face of .  Since faces
are triangles, this implies that  is adjacent to some vertex
in .  By the above hence  is connected for
any .
\end{proof}


The proofs of the above results are constructive and lead
to polynomial time algorithms.  With suitable data structures
to keep track of 2-leg-centers, it is not hard to see that 
a -canonical ordering can be found in linear time; we
omit the details.

\section{Applications}
\label{se:appl}


In this section, we demonstrate two uses for the -canonical
ordering in graph drawing.  Both results proved here were known before,
but in our opinion the -canonical ordering significantly simplifies
the proof of these results.

\subsection{Rectangular duals}
\label{sec:rd}


A {\em rectangular dual drawing} (or {\em RD-drawing} for short) of a 
planar graph  consists of
a set of interior-disjoint rectangles assigned to the vertices of  
in such a way that the union of the rectangles forms a rectangle without
holes, and the rectangles assigned to vertices  and  touch in a
non-zero-length line segment if and only if  is an edge.
The following theorem has been proved repeatedly:

\begin{theorem}[\cite{Ung53,Tho84,KH97}]
Let  be a -connected triangulation, and let 
be an edge on the outer-face of .  Then  has a rectangular
dual.
\end{theorem}

Previous proofs on this result usually used the -canonical
ordering (or some equivalent characterization, such as regular
edge labellings).  We give here a different proof using the
-canonical ordering.

\begin{proof}
Let the outer-face be , chosen such that .
Find a -canonical ordering  of .  We
now build the rectangular-dual drawing of  by drawing  for 
.
By construction,  is an edge on the outer-face of , and
we can hence enumerate the outer-face of  as 
with  and .  
We maintain the invariant
that in the RD-drawing of , the rectangles of 
 all attach at the top side of the bounding
box, in this order.

\begin{figure}[ht]
\hspace*{\fill}
\includegraphics[width=40mm,page=1]{RD.pdf}
\hspace*{\fill}
\includegraphics[width=40mm,page=2]{RD.pdf}
\hspace*{\fill}
\includegraphics[width=40mm,page=3]{RD.pdf}
\hspace*{\fill}
\caption{(Left) The invariant.  (Middle and right) 
Adding a singleton and a fan.  }
\label{fig:RD}`
\end{figure}

Such a drawing is easily created for , since  is a triangle and
so  is a path , where  is the third vertex
of the interior face at .
Now assume  is drawn and consider adding either
a singleton or a fan .  Let  and  be the smallest and 
largest index such that  and  are adjacent to a vertex in .  

Extend all rectangles of  and  upward
by one unit.    This leaves a ``gap'' where the rectangles
of  ended. There is at least one such rectangle
since  by properties of the -canonical ordering (else  would not be 3-connected).  
If  is a singleton , then we insert the rectangle for  into
this gap.  If  is a fan , then 
and so the gap consists exactly of the top of .  Split this
range into  pieces and assign rectangles for 
 in this place.  One easily verifies that this represents
all added edges as contacts and satisfies the invariant.
So we have the desired RD-drawing.
\end{proof}


\subsection{Rectangle-of-influence drawings}
\label{sec:ri-drawing}


A planar straight-line drawing of a graph is called a {\em (weak, closed)
rectangle-of-influence drawing} (or {\em RI-drawing} for short) if
for any edge  the {\em rectangle  defined by } is
{\em empty}, i.e., contains no other points of vertices of the graph.  (It
may contain parts of other edges.)  Here,   is the minimum
axis-aligned rectangle that contains the points of  and ; it 
degenerates into a line segment if  or  are on a horizontal or
vertical line.  The following result is known:

\begin{theorem}[\cite{BBM-GD99}]
\label{thm:ri}
Let  be a -connected triangulation and let  be one
edge of the outer-face.  Then  has a (weak, closed) 
rectangle-of-influence drawing.
\end{theorem}

We re-prove this result using the -canonical ordering.  We note
here that the drawing created is exactly the same as in \cite{BBM-GD99};
the difference lies in that we can find the next vertex set to add much
more easily with the -canonical ordering.

\begin{proof}
Let the outer-face be , chosen such that .
Find a -canonical ordering  of .  We
now build the RI-drawing of  by drawing  for .
By construction  is an edge on the outer-face of , and
we can hence enumerate the outer-face of  as 
with  and .  
We maintain the invariant
that in the RI-drawing of 

\begin{figure}[ht]
\hspace*{\fill}
\includegraphics[width=40mm,page=1]{RI.pdf}
\hspace*{\fill}
\includegraphics[width=40mm,page=2]{RI.pdf}
\hspace*{\fill}
\includegraphics[width=40mm,page=3]{RI.pdf}
\hspace*{\fill}
\caption{(Left) The invariant for RI-drawings.  Hatched regions contain no points due to the RI-drawing.  (Middle and right) Adding a singleton
and a fan.  The light gray region contains the new rectangles of influence.}
\label{fig:RI}`
\end{figure}

Such a drawing is easily created for , since  is a triangle and
so  is a path , where  is the third vertex
of the interior face at .
Now assume  is drawn and consider adding either
a singleton or a fan .  Let  be the smallest and  be the
largest index such that  and  are adjacent to a vertex in .
By 3-connectivity of  we have .
If  is a singleton , then 
define 

and   See also
Figure~\ref{fig:RI}(middle).  By  adding this new point satisfies
the invariant.
All rectangles  are empty for , because they do not intersect
the drawing of  except in rectangles  and .
So we have the desired RI-drawing.

If  is a fan , then .
For , define 
and   See also
Figure~\ref{fig:RI}(right).  By  adding these new points satisfies the invariant.
All rectangles  are empty for , because they do not intersect
the drawing of  except in rectangles  and .
So we have the desired RI-drawing.
\end{proof}

\section{Conclusion}
\label{se:concl}


We showed the existence of new canonical order for -connected triangulations. We used this canonical order to give simplified proofs of some previously known graph drawing results for 4-connected triangulations. 
Furthermore, we provided provided a brief survey of canonical  orderings for planar graphs and laid the groundwork for their further investigation.   Of particular interest to us are the following questions:
\begin{itemize}
\item Does every planar -connected triangulation have an -canonical ordering for all  and reasonable restrictions
	on vertex sets  ? 
The missing case is a -canonical ordering for 5-connected triangulations.

\item The -canonical ordering definition naturally generalizes to planar graphs that are not necessarily triangulated.
	For the corresponding -orderings \cite{Kant96} and -orderings \cite{NRN97}
	it suffices to allow adding {\em chains}, i.e., induced paths.  Are there -orderings, -orderings and -orderings 
	for 4-connected/5-connected planar graphs with some simple subgraphs as vertex sets ?  Likewise, exploration of
	-canonical orders for non-planar graphs for  remains completely open.
\end{itemize}
	

\iffalse
\comment{Random things from Therese:
\begin{itemize}
\item The -ordering for non-triangulated is not quite equivalent to the existing orders:  for the
-ordering, it would be ok to have a chain with only two outgoing edges,
while Kant's ordering every vertex must have outgoing edges.

Perhaps this works if we throw ``internally -connected'' and ``internally -connected''
into the mix, but then we have to define those terms precisely.  (What is internally 5-connected?)
\item Can we do -orderings for non-triangulated?  I've worked a
bit on this, and have real trouble with the face near .  Since
that's now not a face, it could turn into a big cycle, and then that's
not 3-connected if we start with this.  We either need to start with a
Halin-graph (ugh), or make some exceptions to the rules near  (also ugh).
\item Text from Martin: We suggested the existence of -canonical ordering for 5-connected planar graphs. We have a proof of such a canonical ordering and it will be contained in a journal publication of this work, which is currently in preparation.  Comment from Therese:
I am not at all convinced that this is really correct until I've seen a fully written-up proof.  In particular, we don't really have 4-connected, we only have internally 4-connected.  This would need extensive discussion in the paper as to why this is ok (is it?). Summary: I don't want this sentence in the paper.  (It's also 
bad style to refer to something that doesn't exist at least in an ArXiV.)
\end{itemize}
}
\fi

\bibliographystyle{alpha}
\bibliography{therese}


\end{document}
