In this section, we extend our results for the case of the static array by allowing 
insertions and deletions of elements in the array, while supporting the selection 
queries. Recall that we are originally given the unsorted list~\A. 
For supporting  and  efficiently, we maintain the newly inserted elements in a
separate data structure, and mark the deleted elements in . These  and  operations are 
occasionally merged to make the array up-to-date. 
Let  denote the current array with length . We want to support the following 
two additional operations:
\begin{itemize}
	\item \ins(), which inserts  into , and;
	\item \del(), which deletes the th (sorted) entry from~.
\end{itemize}

\subsection{Preliminaries}
\label{section:dynamic-preliminaries}

Our solution uses the \emph{dynamic bitvector} data structure of Hon~et~al.~\cite{hon:dynamic-bitvector}. This structure supports the following set of operations on a dynamic bitvector~. 
The  operation tells the number of~ bits up to the th position in~.
The  operation gives the position in~ of the th  bit.
The  operation inserts the bit~ in the th position.
The  operation deletes the bit located in the th position.
The  operation flips the bit in the th position.


Note that one can determine the th bit of~ by computing . (For convenience, we assume that .)
The result of Hon et al.~\cite[Theorem~1]{hon:dynamic-bitvector} can be re-stated as follows, for the case of maintaining a dynamic bit vector (the result of~\cite{hon:dynamic-bitvector} is stated
for a more general case).

\begin{lemma}[\cite{hon:dynamic-bitvector}]
\label{lem:dynamic-bitvector}
Given a bitvector~ of length , there exists a data structure that takes  bits and 
supports  and  in  time, and ,  and  in
 time, for any parameter  such that . The data structure 
assumes access to a precomputed table of size , for any fixed .
\end{lemma}

The elements in the array~ swapped during the queries and  and  operations,
to create new pivots, and the positions of these pivots are maintained
as before using the bitvector~. In addition, we also maintain two  
bitvectors, each of length : (i) an \emph{insert bitvector}~ such 
that  if and only if  is newly inserted, and (ii) a 
\emph{delete bitvector}~ such that if , the th 
element in  has been deleted. If a newly inserted item is deleted, 
it is removed from~ directly. Both  and  are implemented 
as instances of the data structure of Lemma~\ref{lem:dynamic-bitvector}.

We maintain the values of the newly inserted elements in a balanced binary search tree~. The  inorder traversal of the nodes of~ corresponds to the increasing order of their positions in array . 
We support the following operations on this tree are: (i) given an index~, return the element corresponding to the th node in the inorder traversal of~,
and (ii) insert/delete an element at a given inorder position.
By maintaining the subtree sizes of the nodes in~, these operations
 can be performed in  time 
without having to perform any comparisons between the elements.

Our preprocessing steps are the same as in the static case. In addition, the bitvectors~ 
and~ are each initialized to the bitvector of  \zero s. The tree~ is initially empty.

In addition, after performing   and  operations, we merge all the elements 
in  with the array~, modify the bitvector~ appropriately, and 
reset the bitvectors  and  (with all zeroes). This increases the 
amortized cost of the  and  operations by , without requiring any additional
comparisons.

\subsection{Dynamic Online Multiselection}
\label{subsec:internal-dynamic}

We now describe how to support , , , and  
operations.


\paragraph{.} First, we search for the appropriate unsorted interval~ containing~ using a binary search on the original (unsorted) array~.
Now perform  on interval~ (choosing which subinterval 
to expand based on the insertion key~) until 's exact position~ in 
 is determined. The original array~ must have chosen as pivots the 
elements immediately to its left and right (positions~ and  in array~); 
hence, one never needs to consider newly-inserted pivots when choosing subintervals.
Insert~ in sorted order in~ among at position  among 
all the newly-inserted elements.  Calculate , and set 's 
position to . Finally, we update our bitvectors by 
performing  and .
Note that, apart from the  operation, all other operations in the 
insertion procedure do not perform any comparisons between the elements.

\paragraph{.} First compute~. 
If  is newly-inserted (i.e., ), then remove the node 
(element) with inorder number~ from .
Then perform  and . 
If instead  is an older entry, 
simply perform . In other words, 
we mark the position  in  as deleted even though the corresponding 
element may not be in its proper place.\footnote{If a user wants to 
delete an item with value~, one could simply search for it first to 
discover its rank, and then delete it using this function.}


\paragraph{.} If , return the element 
corresponding to the node with inorder number  
in . Otherwise, compute , 
and return ).

\paragraph{.} First, we search for the appropriate 
unsorted interval~ containing~ using a binary search 
on the original (unsorted) array~. Then, perform  
on interval~ until 's exact position~ is found. If  
appears in array~ (which we discover through \search), we 
need to now check whether it has been deleted. We compute 
 and . If , 
return . Otherwise, it is possible that the item has been 
newly-inserted. Compute , which is the number of 
newly-inserted elements that are less than or equal to . 
If , then return ; otherwise, return failure.



We show that the above algorithm achieves the following performance 
(in Appendix~\ref{appendix:dynamic}).

\begin{theorem}[Optimal Online Dynamic Multiselection]
\label{thm:dynamic-internal}
Given a dynamic array~ of  original elements, there exists a dynamic online data structure that can support  , , , and  operations, of which  are , , and , we provide a deterministic online algorithm that uses at most  comparisons.
\end{theorem}

