


\documentclass[11pt]{article}
\usepackage{hyperref}
\usepackage{amsmath}
\usepackage{graphics}
\usepackage{color}
\usepackage{epsfig}
\usepackage{graphicx}\usepackage{amsfonts}\usepackage{amssymb}
\usepackage{setspace}
\usepackage[margin=1in]{geometry}
\usepackage{comment}
\usepackage{listings}
\usepackage{color}
\usepackage{url}







\newtheorem{theorem}{Theorem}[section]
\newtheorem{definition}{Definition}[section]
\newtheorem{claim}{Claim}[section]
\newtheorem{lemma}{Lemma}[section]
\newtheorem{fact}{Fact}[section]
\newtheorem{corollary}{Corollary}[section]
\newtheorem{remark}{Remark}[section]
\newtheorem{proposition}{Proposition}[section]
\newtheorem{example}{Example}[section]
\newcommand{\qed}{\hfill \mbox{\raggedright \rule{2mm}{3mm}}}
\newenvironment{proof}{\noindent{\bf Proof.}}{\qed}
\newenvironment{claimproof}{{\bf Proof of claim.}}{  \rule{1mm}{3mm}}
\newenvironment{sketchproof}{\vspace*{-.1in}\noindent{\bf Proof sketch.}}{  \rule{2mm}{3mm}}
\newcommand{\esect}[2]{\bigskip \centerline{#1 {\sc #2}}}
\newcommand{\esubsect}[2]{\bigskip \\ #1 {\em #2}}
\newcommand{\set}[1]{ \{ #1 \} }
\newcommand{\A}{ }
\newcommand{\D}{\Delta}
\newcommand{\C}{{\cal C}}
\renewcommand{\P}{{\cal P}}
\renewcommand{\L}{\Lambda}
\renewcommand{\l}{\lambda} 
\newcommand{\eps}{\varepsilon}
\newcommand{\bx}{\bar{x}}
\newcommand{\precs}{ }
\newcommand{\tu}{{\cal T}}
\newcommand{\lbfl}{{\sc Lbfl}}
\newcommand{\cfl}{{\sc Cfl}}
\newcommand{\ufl}{{\sc Ufl}}
\newcommand{\opn}{\operatorname}






\date{}

\allowdisplaybreaks


\begin{document}

\title{
Sherali-Adams gaps, flow-cover inequalities and generalized configurations for
 capacity-constrained Facility Location 
}

\author{Stavros G. Kolliopoulos\thanks{Department of Informatics and
Telecommunications, National and Kapodistrian 
University of Athens, Panepistimiopolis Ilissia, Athens
157 84, Greece; (\texttt{sgk@di.uoa.gr}).}   
\and Yannis Moysoglou\thanks{ 
Department of Informatics and
Telecommunications, National and Kapodistrian 
University of Athens, Panepistimiopolis Ilissia, Athens
157 84, Greece; (\texttt{gmoys@di.uoa.gr}). } }


\maketitle

\begin{abstract}
Metric facility  location is a  well-studied problem for  which linear
programming  methods have  been used  with great  success  in deriving
approximation  algorithms.  The capacity-constrained  generalizations,
such  as  capacitated  facility  location (\cfl\/)  and  lower-bounded
facility location  (\lbfl), have proved  notorious as far  as LP-based
approximation  is  concerned:   while  there  are  local-search-based
constant-factor  approximations, there  is no known  linear relaxation
with  constant  integrality gap.  According  to    Williamson and Shmoys 
devising a  relaxation-based approximation for \cfl\ is  among the top
10 open problems in approximation algorithms.

This paper  advances significantly the  state-of-the-art 
on  the effectiveness of
linear programming for capacity-constrained facility location
through  a host of impossibility 
results   
for  both \cfl\  and  \lbfl.   We show  that  the  relaxations
obtained   from  the  natural   LP  at     levels   of  the
Sherali-Adams  hierarchy  have an  unbounded  gap,  partially 
answering an  open
question of  \cite{LiS13, AnBS13}. Here,  denotes the  number of facilities
in the instance.  Building on the ideas for this result, we prove that
the standard \cfl\ relaxation enriched with the generalized flow-cover
valid inequalities  \cite{AardalPW95} has  also an unbounded  gap.  
This disproves a long-standing conjecture of \cite{LeviSS12}. 
We
finally introduce  the family of proper  relaxations which generalizes
to  its  logical extreme  the  classic  star  relaxation and  captures
general  configuration-style  LPs.  We  characterize  the behavior  of
proper  relaxations for  \cfl\ and  \lbfl\ through  a  sharp threshold
phenomenon.


\end{abstract}





 

\section{Introduction}


Facility location is one of the most well-studied problems in combinatorial optimization.
In the  {\em uncapacitated } version \emph{(\ufl)} we are given a set  of facilities and set  of clients. We may open facility  by paying its opening cost  and we may assign client  to facility  by paying the connection cost . We are asked to open a subset 
 of the facilities and  assign each client to an open
facility. The goal is to minimize the total opening and connection
cost. 
A {\em -approximation algorithm,}  outputs in polynomial time a
feasible solution with cost  at most  times the optimum. 
The approximability of general \ufl\ is settled by an -approximation \cite{Hochbaum82} which is asymptotically best
possible,  unless {\sf P = NP}. In  {\em metric} \ufl\ 
the service costs satisfy the following variant of the triangle inequality:
 for any  and 
This very natural special case of \ufl\ is approximable within a
constant-factor, and many improved results have been published over
the years. In those,   LP-based
methods, such as filtering, randomized rounding and the primal-dual method
 have been particularly prominent (see, e.g., 
\cite{ShmoysWbook}). After a long series of papers 
the currently best approximation ratio for 
 metric \ufl\ is  \cite{Li11}, while the best known lower
bound is  unless {\sf P = NP} (\cite{GuhaK99} and Sviridenko \cite{Vygen05}).
In this paper we focus on two generalizations of metric \ufl:  the 
 {\em capacitated facility location (\cfl\/)} and
 the {\em lower-bounded facility location (\lbfl\/)}.


\cfl\/ is the generalization of  metric \ufl\ where every facility 
has a capacity  that specifies the maximum number of clients that
may be assigned to  In {\em uniform} \cfl\ all facilities have the
same capacity  Finding  an approximation algorithm for \cfl\/ that
uses  a linear  programming lower  bound, or  even proving  a constant
integrality  gap for an  efficient LP  relaxation, are  notorious open
problems. Intriguingly,  the following rare phenomenon occurs. 
The natural LP relaxations  have an unbounded
integrality gap and the only known -approximation algorithms are
based  on local  search,  with  the currently  best  ratios being  
\cite{BansalGG12} for  the non-uniform and   \cite{AggarwalLBGGJ12}
for  the uniform  case respectively.  In  the special  case where  all
facility  costs  are equal,  \cfl\  admits  an LP-based  approximation
\cite{LeviSS12}.  Comparing the LP optimum against the solution output
by an LP-based  algorithm establishes a guarantee that  is at least as
strong  as the  one established  a priori  by worst-case  analysis. In
contrast, when a  local search algorithm terminates, it  is not at all
clear what  the lower  bound is.  According  to  Williamson and Shmoys
\cite{ShmoysWbook} devising a  relaxation-based algorithm for \cfl\ is
one of the top  open problems in approximation algorithms.


A lot of effort has been  devoted to 
understanding the quality of relaxations obtained by an iterative 
lift-and-project procedure. Such procedures define hierarchies of 
successively  stronger relaxations, where  valid inequalities are added at 
each level. After at most  levels, where  is the number of 
variables, all valid inequalities have been  added and thus the integer polytope is
expressed. Relevant methods  include those  developed  by Balas et
al. \cite{BalasCC93},  Lov\'{a}sz and Schrijver \cite{LovaszS91} (for
linear and semidefinite programs), 
Sherali and Adams \cite{SheraliA90},    Lasserre  \cite{Lasserre01}
(for semidefinite programs). 
See
\cite{laurent} for a comparative discussion.
  
 The seminal work of Arora et al.  \cite{AroraBLT06}, studied integrality
 gaps  of  families of  relaxations  for  Vertex  Cover,
 including  relaxations    in  the  Lov\'{a}sz-Schrijver  (LS)
 hierarchy.  This  paper  introduced  the  use  of  hierarchies  as  a
 restricted model  of computation  for obtaining LP-based  hardness of
 approximation  results.   Proving  that  the integrality  gap  for  a
 problem  remains  large  after  many  levels of  a  hierarchy  is  an
 unconditional  guarantee   against  the  class   of  relaxation-based
 algorithms obtainable through the specific method.  At the same time,
 if an LP  relaxation maintains a gap of  after  a linear number of
 levels,  one  can  take  this  as evidence  that  polynomially-sized
 relaxations are unlikely to yield approximations better than  (see
 also  \cite{SchoenebeckTT07}).  In fact,  the former  belief is  now a
 theorem   for   maximum    constraint   satisfaction   problems:   
  in terms of approximation,
 LPs of size 
  are exactly as powerful 
 as -level Sherali-Adams relaxations \cite{ChanLRS13}. 


 \lbfl\ is in a sense the opposite problem to \cfl. 
 In an \lbfl\ instance   every facility  comes with a  lower
bound  
which is the minimum number of clients that must be assigned
 to   if we open it. In {\em uniform} \lbfl\ all the lower bounds
have the same value   \lbfl\ is even less well-understood than \cfl. 
 The first approximation algorithm for the uniform case 
  had  a performance guarantee of
 \cite{Svitkina08}, which has been  improved  to  \cite{AhmadianS12}. 
Both use local search.




Apart from some  work of the authors \cite{KolliopoulosM13,KolliopoulosM14b}
there has been no systematic theoretical  study of the power of linear programming
for approximating \cfl.  
In \cite{KolliopoulosM13} we show an unbounded gap for \cfl\ at  levels
of the LS and the semidefinite mixed-LS hierarchies,  being
the number of facilities.  In  
\cite{KolliopoulosM14b} we show that linear relaxations in the classic
variables require at least an exponential number of constraints to
achieve a bounded integrality gap. Note that it is well-known that hierarchies
may produce an exponential number of inequalities already after one
round. 
 For related problems there are  some recent interesting results. 
Improved  approximations  were given  for
-median   \cite{LiS13}      and   capacitated   -center
\cite{CyganHK12,AnBS13},  problems closely  related to
facility location.  For both, the improvements are obtained 
by LP-based techniques
that  include preprocessing  of the  instance in  order to  defeat the
known integrality gap. For -median, the authors of \cite{LiS13}
state that their 
-approximation algorithm   can
be converted  to a  rounding algorithm on an 
 -level LP  in the Sherali-Adams (SA)
lift-and-project hierarchy. 
They propose exploring the  direction of using 
SA  for approximating  \cfl.  In \cite{AnBS13}
the authors raise as an important question  to understand  the
power  of lift-and-project methods  for capacitated  location
problems, including  whether they automatically capture relevant 
preprocessing steps. 



\vspace*{0.3cm}
\noindent
{\bf Our results.}
We give impossibility  results on arguably the most promising 
directions for   strengthening
linear relaxations for \cfl\ and \lbfl\  and in doing so we answer  open
problems from the literature.    Our contribution is threefold. 


First, we show that the LPs  obtained from the
natural relaxations for  \cfl\ and \lbfl\ at  levels of the 
SA hierarchy have an unbounded gap on an  instance
where 
  and  
This result answers the questions of \cite{LiS13} and
\cite{AnBS13} stated above as far as the natural LP is concerned 
and moreover it  is asymptotically tight.
In the  instances we consider clients  have unit demands and
it is well known that in this case the integer polytope and the 
mixed-integer (where fractional client assignments are allowed) 
polytope are the same.  Since SA extends to mixed-integer
programs  as well  \cite{Cornuejols08,BalasCC93},  the mixed-integer
polytope is obtained after at most   levels. Thus at most that many
levels are needed  also by the stronger, full-integer, SA procedure we employ,
which in the lifting stage multiplies 
also with assignment variables. 
From a qualitative aspect,   we  give  the first, to our knowledge, SA
bounds for a relaxation where variables have more than one type of
semantics, namely
the facility opening and the client assignment type. 
Compare this, for example, with the Knapsack and Max~Cut LPs that
contain each one type of variable. 
The lifting of the assignment variables 
raises obstacles in the proof that we managed to overcome as discussed
in Section \ref{SA-result}. 



We use the \emph{local-to-global} method which was implicit in
\cite{AroraBLT06} for 
local-constraint
relaxations   and  was   then  extended   to  the   SA   hierarchy  in
\cite{FernandezdlVKM07}. 
See also \cite{GeorgiouMagen} for an explicit description and
\cite{CharikarMM09} for applications to Max~Cut and other
problems.  
In this approach, the feasibility of a solution for the -level SA
relaxation is established through the design of a set of  appropriate
distributions over  feasible integer solutions for each constraint
such that these global distributions agree with each other locally on relevant
events.  
\begin{comment} -------- too technical 
In this approach, we interpret a linearized
product of a set  of variables, namely , as the probability of
occurrence of the event  with respect to
a distribution over integer solutions.  
If there is an assignment  of
values to the linearized variables
appearing at level  of SA such that for each lifted constraint there is a
distribution over some  integer solutions and the values  of the 
variables    coincide   with    the   probability    of    the   event
 with respect to  that distribution, then
 the projection  of  on the  variables
 is  feasible  for  the  relaxation  obtained  at  level    of  the
 hierarchy. 
\end{comment} 
To prove Theorem~\ref{cfl-SA:theorem} for \cfl\ we 
devise first in Lemma~\ref{assi-sym} an intuitive method to construct an initial 
 set of distributions for a constraint. 
These  initial distributions are  inadequate for 
constraints where all facilities appear as indices. 
An alteration procedure,
explained in Propositions~\ref{level-ratio}--\ref{transf:prop}, 
 produces the final set of distributions. 
Theorem~\ref{cfl-SA:theorem} extends
significantly our earlier result on the  
LS hierarchy for \cfl\ \cite{KolliopoulosM13} to the stronger SA
hierarchy. It turns out that in both cases we can start from the same 
bad instance. 
It should be noted that the methodology in the two proofs is completely
different -- in \cite{KolliopoulosM13} the result was obtained via an 
inductive construction of  protection matrices. 



Our second contribution (cf. Theorem~\ref{effective-cap:theorem})
is that the \emph{effective capacity}
inequalities introduced in \cite{AardalPW95,AardalPW95er} for \cfl\ 
fail to reduce the gap of the classic relaxation to constant. 
These constraints generalize the flow-cover inequalities for
\cfl. Thus  we disprove the long-standing conjecture of \cite{LeviSS12} 
that the addition of the latter to the classic LP 
suffices for  a constant integrality gap. 
Our proof deviates from standard integrality gap constructions 
by applying the local-global method.  
The bad solution fools every inequality   because its part that is
``visible'' to  can be extended to a solution  that is a
convex combination of feasible integer solutions. 
Our  ideas can be extended 
to even more general families
such as the  {\em submodular inequalities} \cite{AardalPW95}, cf. 
Theorem~\ref{thm:submod}
in the Appendix. 
All results  in this paper make no time-complexity assumptions. To our
knowledge no efficient separation algorithm for the effective
capacity inequalities is known.

We finally  introduce    
the family of  proper relaxations
which are  configuration-like linear programs.
The so-called \emph{Configuration LP} was  used by 
Bansal and Sviridenko 
\cite{BansalS06} for the Santa Claus problem and has yielded valuable insights, mostly
for resource allocation  and scheduling problems
(e.g., \cite{Svensson12}).
 The analogue of the Configuration
LP for facility location already exists, it is the {\em star
  relaxation} (see, e.g., \cite{JainMMSV03}).
We take the idea of a star  to its logical  extreme by 
introducing  classes. 
A {\em class} consists of a set with an arbitrary number of facilities and clients
together with an assignment of each client to a facility in the set. 
A {\em proper relaxation} for an instance is defined by a collection
 of classes and a decision variable for every class. 
We allow great freedom in 
defining   
the only requirement   is that the resulting
formulation is symmetric and valid. 
The {\em complexity } of a proper relaxation is the maximum fraction
 of the 
available facilities that are contained in a class of 
In Theorem~\ref{theorem:proper}  we 
characterize the  behavior of proper relaxations  
for \cfl\ and \lbfl\ through a threshold result: 
anything less than maximum complexity results in unboundedness of
the integrality gap, while there are
proper relaxations of maximum complexity with a gap of
.

Our  results disqualify  the  so far most promising approaches 
 for an efficient LP relaxation
for  \cfl. Moreover, we advance drastically the state-of-the-art for the
little understood \lbfl. 
Whether a fundamentally new approach
may succeed for either problem remains as an open question. 

\iffalse 
Our three  results  and their proofs seem additionally to suggest 
 that a bounded-gap relaxation should  not  constrain  the number of variables
 appearing in each  inequality. 
Whether such  non-trivial inequalities,   with large support and an
efficient separation oracle, exist is an open question. 
\fi 

For lack of space, some proofs and all material on \lbfl\ are
 in the Appendix. 







\section{Preliminaries}
\label{sec:prel}


Given an instance  of \cfl\ or \lbfl, we use  to denote  and 
respectively.
We will show our negative results for uniform, integer, capacities and lower
bounds. Each client can be thought of as representing one unit of demand.
 It  is  well-known  that in such a setting  the  splittable  and
unsplittable versions  of the problem are equivalent. 
The following 0-1  IP is the standard  valid formulation of uncapacitated 
facility location with unsplittable unit demands.

\vspace*{-0.35cm}

\vspace*{-0.2cm}

\iffalse 

\fi 

\noindent 
The linear relaxation results  from the above IP by replacing the integrality constraints
 with: 
   
To obtain the standard LP relaxations for 
uniform \cfl\ (and \lbfl) with capacity  (lower bound )
the following constraints are added
respectively: 


\vspace*{-0.2cm}



We will slightly abuse terminology by 
using  the term {\em (LP-classic)} for both LPs. It will be clear from the context to
which problem, \cfl\ or \lbfl,  we refer. 


We proceed to define the Sherali-Adams hierarchy \cite{SheraliA90}. 
Consider a polytope  defined by the linear
constraints 
, . We define the polytope 
 as follows. For every constraint 
of , for every set of variables  such that
 and for every , consider the valid constraint:
.
Linearize the system obtained this way by replacing (i)  with
 for all  and (ii) 
with   for each set . 
is the projection of the resulting linear system onto the singleton
variables. We call  the polytope {\em obtained from  
at level } of the SA
hierarchy. Given a cost vector  the {\em relaxation
obtained from  at level } of SA is  




\begin{comment}  The second well-known LP  is the star relaxation.
A {\em star} is a set consisting of some  clients and
one facility. Let  be a set of stars. For a star   let   be an indicator variable denoting
whether   is picked.   The cost   of star    is equal
to the opening cost of the corresponding facility plus the cost of
connecting the star's clients to it. 
\iffalse     \fi


Defining  as the set of all stars  where  the total number of the clients in  
is at most the
capacity  (at least the bound ),  we get corresponding relaxations for
  \cfl\ (\lbfl).
In the rest of the paper we  slightly abuse terminology by 
using    {\em (LP-star)}  to refer to the
star relaxation for the problem we examine each time (\cfl\  or \lbfl).

It is well known  that for both \cfl\ and \lbfl, (LP-classic) and (LP-star) are
equivalent, therefore (LP-star) can be solved in  polynomial time.
\end{comment}   






\section{Sherali-Adams gap for \cfl }\label{SA-result}

Consider an  instance of metric  \cfl\ with a  total of
 facilities,   with opening cost   which we  call cheap (and
denote the corresponding set by ) and  with opening cost 
which   we  call  costly   (and  denote   by
).  The  capacity    and  we have  a  total  of  
clients. All connection costs are . We will show that the following
bad solution   to the instance\footnote{The reader should notice
  that any similarity with Knapsack is
  superficial. Theorem~\ref{cfl-SA:theorem} is about the {\em \cfl}
    polytope. Moreover, it is easy to embed our instance 
 in a slightly larger one, with a non-trivial metric,  so that 
the projection of the bad \cfl\ solution
to the -variables, is in the integral polytope of the ``underlying''
     knapsack  instance.
}
  survives a  number of SA levels, which is  linear 
in the number  of facilities. On the other hand, it
is known that 
at level  the  relaxation obtained expresses the integral polytope.
Let  .  For all    and  for all  
 and  , and for  all 
and     for     all               and
. Theorem~\ref{cfl-SA:theorem} below
indicates that, as often with hierarchies,  simple valid inequalities are
generated after many rounds.  The reader who is further interested in
the robustness of SA for \cfl\ may consult Section~\ref{sec:robust} in
the Appendix. 

\begin{comment}
The natural \cfl\ relaxation is defined as follows:




We will give the proof via the local-global method introduced in [Delavega-Mathieu], in 
which we interpret a linearized product of a set  variables namely  as the probability 
of occurring all the events   with respect to a distribution over integer solutions.
If there is  an assignment of values to the linearized variables such that in any set of variables appearing in a constraint obtained
after  level of SA (before the projection step) there is a distribution over some integer solutions and the values of the  variables coincide with the probability of the event
 with respect to that distribution, then the initial solution which corresponds to the projection of the "lifted solution" on the  variables survives  rounds.
 
The difficult part of the above method is to device a proper set of distributions. We use a novel
approach: for each such set we define a solution which are consistent on the variables of our
initial bad solution that appear in a lifted constraint and furthermore it has some desired properties
it satisfies some non-valid inequalities. Then we prove that the latter solution survives 
rounds of lifting in the mixed LS hierarchy applied to the initial relaxation with the addition of
those constraints. Thus the solution is a convex combination of integer solutions satisfying 
the particular set of constraints, a fact that we will exploit in proving the consistency of those 
distributions (for "consistency" see lemma below).

We note that proving that a particular solution survives  rounds of mixed LS circumvents
the difficulty of explicitly defining a distribution e.g. via the analysis of a randomized construction
like the one in [Delavega-Mathieu]. For our proof we consider the underlying distributions abstractly - we make use some of their properties.
\end{comment}

The following lemma, which is implicit in previous work
\cite{FernandezdlVKM07,GeorgiouMagen} 
 gives sufficient conditions for a solution to be feasible at  level  of the SA hierarchy.

\begin{lemma}\cite{FernandezdlVKM07,GeorgiouMagen}\label{SA-survival}
Let  be  a feasible solution to the relaxation  and let 
be  the  set  of  variables  appearing  in  a  lifted  constraint
obtained from 
multiplied by  .  Solution  survives  levels of
SA if  for every constraint   and each multiplier   with at most
 distinct variables there is:
\vspace*{-0.2cm}
\begin{itemize}
\item[1] A solution    which agrees with
    on     such that 
is a convex combination  of integer solutions  (and thus  defines a distribution on integer solutions) and
\item[2]  For   any  two  sets     and  ,  let
    be a product appearing
  in both lifted constraints obtained from  and 
  multiplied 
  with  and  respectively.  
Then
  the probability   is
  the  same in  both  distributions   and   associated  with
   and  respectively.
\end{itemize}
  \end{lemma}



\begin{comment}
\subsection{Mixed SA}

As a warm-up, we first consider the mixed SA hierarchy. In this case only the  variables are
lifted.



We add the following non-valid inequality to the above relaxation: .
We call the resulting linear program \emph{enriched}. The enriched LP is a relaxation of the convex
hull of the integer solutions of the instance that satisfy the added equality. Thus if we apply 
rounds of lift-and-project (either LS or SA, mixed or integer) we will get the integral polytope of 
those solutions, which is the intersection of the general integral polytope with the hyperplane . 

Now observe that if we prove that a solution  of the enriched LP survives after  rounds of mixed LS then it is a convex combination of (full) integer solution satisfying .
Thus the corresponding distribution over integer solutions have the following desirable property:
 for every pair  of distinct costly facilities all the events  and  are mutually exclusive, that is .


Now consider the lifted linear program obtained by  levels of the SA procedure, that is the program obtained after the lifting step but before the projection step. Consider
a constraint   for some facility  and a multiplier , consisting only of  variables since we are considering the mixed case. If  there must a costly facility , different from  that do not appear in the multiplied constraint. We construct a solution  by setting 
and letting all other variables the same. We will later prove that  survives  rounds 
of the mixed LS applied to the enriched relaxation and thus it is a convex combination of a set 
 of integer solutions satisfying constraint . The set  gives
the desired distribution for the set of linearized lifted variables . Consider a lifted variable   being the linearization of a product of variables that includes
two variables corresponding to two different costly facilities appearing in , either facility or assignment variables. Then 
since the events corresponding to two distinct costly facility variables are mutual exclusive 
by the properties of the distribution. This observation simplifies our analysis as we have only to assign values to the rest of the lifted variables. But this task is now simple: the remaining cases are polynomials on variables involving at most one costly facility. Since  for  then 
the event  occurs 100\% of the time in the distribution. So if the polynomial has variable
  then the value of  is simply the value of  in bad solution . Likewise if the polynomial has variable   then the value of  is again the value of  in bad solution . If the polynomial includes both  and  , then
observe that conditioning that  we have that  (since event  implies event  by feasibility) so again we set . And finally, if the polynomial includes both  and  , , then  takes the value  which, with no loss of generality as it is later proved in Lemma \ref{assi-sym},  is invariant of the of . 

\end{comment}





First   consider  a  constraint    and a multiplier .   After multiplying by  and
expanding,  we obtain
a   linear  combination   of  monomials
(products). Then, for the   levels
we  consider   there  must  be   some  costly  facility   .  We construct  a solution    by setting
  and letting  all  other
variables the  same as in the  original bad solution .  We say that
facility    \emph{takes the  blame}.  We  will   prove  that
 can be obtained as a convex combination  of a set of
integer   solutions    satisfying   constraint   .  While
 can be obtained as a convex combination  in a variety of
ways, we require that the assignments of clients to the cheap facilities are
indistinguishable in  and the same must be true for the assignments to
costly facilities other than .  In the upcoming definition, we 
use  the product  as an  abbreviation
of the event  

\begin{definition}
 Let  be the facility that takes the blame. We say that a distribution  is \emph{assignment-symmetric}  if the following are true:
\vspace*{-0.1cm}
\begin{itemize}
\item[1] ,  with  is the same if we exchange all occurrences of cheap facility  by cheap facility  (in other words relabeling facilities). Note that we allow repetitions of facilities and clients in the description of the event.

\item[2]   is the same if we exchange all occurrences of client  by client .

\item[3]   is the same if we exchange all occurrences of costly facility  by costly facility , .
\end{itemize}
\end{definition}
\vspace*{-0.1cm}

We can always obtain  from such an assignment-symmetric distribution  as shown
in the following lemma. 

\begin{lemma}\label{assi-sym}
Solution  is a convex combination  of integer solutions which defines an assign\-ment-symmetric distribution.
\end{lemma}

\begin{proof}
We describe  a probabilistic experiment which induces an assignment-symmetric distribution  over integer solutions satisfying .

 Fix           costly           facility          .           Let
  be  the desired number
 of  clients assigned  to facility   in  the integer solutions in
  where facility  is  opened. 
 To simplify the presentation let us assume  that  and the
  values we subsequently define are integers  
(we  discuss in the
 Appendix  how to handle fractional 's).  Let
   be the number  of clients
 assigned  to facility . Likewise,  fix costly facility
 .                         Let
  be the number
 of clients assigned to facility   in each integer solution in
    where  facility     is   opened  and   similarly  let
    be   the  number   of
 clients  assigned  to  facility    in  each  integer
 solution in  where facility  is opened. Observe that all
 the defined 's are less than . The following procedure produces
 the assignment-symmetric distribution .

Pick costly facility  with probability . If  () then consider  bins corresponding to the  cheap facilities  each one having  () slots and  bin corresponding to  having  ()  slots.
Randomly distribute  balls to the slots of the  bins, with exactly one ball in each slot.
Note that the above experiment induces a distribution over feasible integer solutions satisfying  since all the defined bin capacities are less than  and every client is assigned to exactly one opened facility in each outcome and exactly  costly facility is opened. 
Moreover the induced distribution  is assignment-symmetric and
the expected  vector with respect to  is solution . 


\begin{comment}
To handle the case where the 's are not integers, we simply do the following: each time costly facility  () is picked, we set the slots of the corresponding bin to  () with probability  (), otherwise set the slots to (). If the number of slots of  () is set to  () then we pick some  ( ) cheap facilities at random and set their corresponding number of slots to  () and the number of slots of the rest cheap facilities to (). 
Otherwise pick some  ( ) cheap facilities at random and set their corresponding number of slots to  () and the number of slots rest to (). Note than in every case the expected number of slots per facility is as in the previous experiment.
\end{comment}

Clearly,   is the convex combination induced  by  and
 is  assignment-symmetric: the cheap facilities  are always open,
and the costly  are open a fraction  of the time that is  equal to the
value  of  their  corresponding    variable.  The  expected  demand
assigned to  each  is   which is
the  total  demand assigned  to    by  . Since  the
clients  have  the  same  probability  of  being  tossed  in  the  bin
corresponding to , the  expected assignment of each client 
to  is the same as in .  
Similarly we can prove
that the expected assignments to the cheap facilities are as required,
see the Appendix for details.  
\begin{comment} ---moved to Appendix. 
As for  the assignments to  the cheap
facilities, observe that in every outcome of the experiment the demand
not assigned  to costly facilities  is exactly the demand  assigned to
cheap.  Since we  have proved  that  the expected  assignments to  the
costly  facilities are  those of  the  bad solution,  by linearity  of
expectation we get that the  total assignments to all cheap facilities
are  (the total  assignment of each
client add up  to  by the constraints of the  LP). By the symmetric
way the  cheap are handled  in the experiment  we have that  the total
expected demand assigned to  each  is  and
by  the symmetric  way the  clients are  assigned to    through the
experiment we get  that the expected assignment of each   to  is
.
\end{comment} 
\end{proof}

We set the product-variables 
appearing in constraint  multiplied by multiplier  to 
. Constraints  
 
are handled 
in the exact same way; the set of variables appearing
in them is a
subset of those  appearing in the more complex constraints.


The  second and  more challenging  case  is when  constraint   is
 for some client . Let again  be a
multiplier of level . Observe now that all facilities in  appear
in  as indexes of at least the  variables. We select
one facility  not appearing  in  to \emph{take the blame}. Let
  be the corresponding extended  solution that can
be written as  a convex combination/assignment--symmetric distribution
  of integer solutions; the  existence of   is  ensured by
Lemma \ref{assi-sym}.  In this case there  is a major  obstacle to the
agreement  of   the  products  :  conditioning   on  the  event
 the  probability of an event  for some
 is higher  than it would be if we were  to condition on the
event  .  The  same is  true  for  more
complex events involving  assignments to cheap facilities conditioning
on an  assignment of  facility  compared  to the  analogous event
conditioning on  some other costly  facility. This can  be problematic
since facility   takes the  blame in some distributions  but does
not  in some  others and  thus there  is the  danger of  violating the
consistency    required    by    the    2nd   condition    of    Lemma
\ref{SA-survival}. We  overcome this difficulty  by making alterations
to  and constructing a distribution  where the probabilities
of the aforementioned events are the same.

\begin{comment}
**************************

To estimate the probabilities in question we describe a procedure which produces an assignment-symmetric distribution : fix costly facility . Let  be the number of clients assigned to facility  in each integer solution in  where facility  is opened (this is the average number of assigned clients - for simplicity we may assume (pretend) that this number is the same in every such solution and we assume again w.l.o.g. that  is integer). Let  be the  number of clients assigned to facility  in each integer solution in  where facility  is opened. Likewise, fix costly facility . Let  be the number of clients assigned to facility  in each integer solution in  where facility  is opened and similarly let  be the  number of clients assigned to facility  in each integer solution in  where facility  is opened. The following procedure produce the assignment-symmetric distribution : assign an equal amount of measure  to each integer solution that opens all facilities in  plus facility  and
assigns  clients to each cheap facility and  clients to , so that a total fraction  clients have been assigned (total means summing over all clients). Similarly, for each costly facility  assign an equal amount of measure  to each integer solution that opens all facilities in  plus facility  and
assigns  clients to each cheap facility and  clients to , so that a total fraction  clients have been assigned. The following is straightforward by the construction:

\begin{proposition}
The above procedure produces an assignment-symmetric distribution .
\end{proposition}

***************************

\end{comment}

We now  devise the altered  distribution .  We first  display the
intuition in the following example: consider the event  and the event   with    and  .         The        probability        of                 is

and         the          probability         of                  is
.
Note  that    so    is  only  slightly
greater.  We nullify  the  difference between  those probabilities  by
performing  an alteration  step  to distribution    that we  call
\emph{transfusion of probability}. We  pick some measure of an integer
solution    for which    for some  client .  We pick  the same  quantity of
measure of some  integer solution (or of some  set of solutions) 
for which   and
we exchange the values of the assignments  of the solutions.
  Let that quantity be , it is easy to
see  that each  set of  solutions has  enough measure  to  perform the
transfusion. The resulting distribution   now has . In
general, when transfusing probabilistic measure for complex events, we
must be  careful not  to change the  probability of  events involving
only  assignments to cheap facilities, as opposed to the simplified
example above.

Now let   be a product  appearing in constraint   after having
multiplied by multiplier .  We only consider products where exactly
one variable   appears. Recall we chose    so that it
does not appear in ; thus  
we  cannot have   or  more  than one
assignments  of  appearing  in a  product 
  We  may also  assume  that there  is no  
variable in , since if  there is for some  the
probability of  is simply  and if  the we can ignore
the effect of  since it is always true. Likewise we assume that
there is no assignment variable  of another costly facility. We shall
make corrections of the probability of all such events  in a
top-down manner: at step  we  fix the probability of all the events
     where     is a product 
appearing in constraint  multiplied by . In other words, we fix
the probabilities  in decreasing order  of the cardinality of  the set
of variables appearing in .  The following proposition relates the
probability       of            with       that      of
, an event with the additional
requirement that  .

\vspace*{-0.1cm}
\begin{proposition}\label{level-ratio}
Let  and let . Then
in , .
\end{proposition}
\vspace*{-0.1cm}

\begin{comment}
\begin{proof}
Since the considered distribution is assignment-symmetric, event  is equivalent  to the event 
of randomly distributing  balls to the slots of  bins, with at most one ball in each slot, each bin having  identical slots, asking that ball  is tossed in the bin of  and ball  is tossed in bin . Since there are  slots in each bin and the balls are at most , it is easy to see
that .
\end{proof}
\end{comment}

\iffalse---------- expanded version 
\begin{sloppypar}
Consider step  of the above iterative construction of . Let
 and the event
. We wish in  the probability
 to be equal to
 in  for . We bound the ratio 
\end{sloppypar}
----------- end expanded version \fi 
\begin{sloppypar}
Consider step  of the above iterative construction of . Let
 and the event
. We wish in  the probability
 to be equal to
 in  for . We bound the ratio 
\end{sloppypar}



\vspace*{-0.1cm}
\begin{proposition}\label{transf_fraction}
\begin{sloppypar}
Let  and  be defined as above. 
Then  \\
.
\end{sloppypar}
\end{proposition}
\vspace*{-0.1cm}

\begin{comment}
\begin{proof}
Consider again the random experiment of the proof of Proposition \ref{level-ratio}. Recall that
. Note that again both  are  and . In the ball tossing experiment, the probability of success of ball  in the case where the capacities of the cheap bins is  is at most  the probability of success of the same event in the case where the capacities of the cheap bins is  (we are very generous here). So the ratio  using that .
\end{proof}
\end{comment}

The corrections of the  probabilities of events of previous iterations
affect the probabilities of the events of the current iteration of the
procedure  that  constructs  .   We  bound  this  effect  on  the
probability of  an event  of the  current iteration 
by     considering      the     corrections     of      the     events
, with  in the
set of variables appearing in  and , of
the previous iteration and using the union bound.\footnote{Notice that
  any  effect of iteration    on , originates
  from events that are subsets of  and has therefore
  been  accounted for.}   There are  exactly   events needed  to be
taken into consideration for  each such  of the current
step .  The amount of the  effect of the correction of the previous
iteration   is   by    Proposition   \ref{transf_fraction}   at   most

while the measure of the  needed correction for  is at
 least   which by
 Proposition \ref{level-ratio} and by the number of rounds we consider
 is                higher,                in                particular
 . To
 subtract  from    the  rest  of  the  probabilistic
 measure required from  the correction, say a measure  of , we do
 the  following transfusion step:  pick a  measure   of solutions
 from distribution  such that ,  for any
  such that , all the other events of
  are false, and so are all the remaining events
 corresponding to  assignments in . Likewise pick  an equal measure
 of solutions  from   such that ,   with
 , all the other events of 
 are true,  and all the remaining events  corresponding to assignments
 in    are  false.  Now  exchange the  values  of  the  assignments
   and   of the  solutions of  the two  sets. The
 resulting distribution  has the probability  of  fixed
 and moreover,  by the  choice of  the sets of  solutions on  which we
 perform the transfusion step, the  probability of the events fixed in
 previous iterations  was not altered and neither  was the probability
 of events containing only assignments of cheap facilities.
Clearly, the solution  is still obtained in expectation. 
 It remains to show that the transfusion step can be performed, i.e., that
  there is enough  measure   in the  involved sets  of integer
 solutions.

\vspace*{-0.1cm}
\begin{proposition}\label{transf:prop}
The probabilistic transfusion step of the above iterative procedure can always be performed.
\end{proposition}
\vspace*{-0.5cm}
\begin{comment}
\begin{proof}
Consider the measure  in  of the set of integer solution satisfying   
and all events in  being false, namely . Then, by the random experiment of the construction of , this event is equivalent to the event that facility  is picked,  and  the  balls corresponding to the clients of the rest of events
are not tossed in their corresponding bins. Using again that both  are  and , we can bound the probability of the  balls  by that of  Bernoulli trials with probability of success  (we are once again very generous). Then the probability that all events on ball   fail is . Thus measure  is at least  which is constant. On the other hand the measure required by the transfusion step for each event  of iteration  that needs to be fixed is at most . There are  such events of iteration , and summing over all the iterations of our construction we get  which quantity is less than  for the  levels of SA we consider, so we can always pick the required amount of measure. 
\end{proof}
\end{comment}


\begin{theorem}\label{cfl-SA:theorem}
There is  a family of   \cfl\  instances with  facilities and  clients  
such that the  relaxations   obtained   from
(LP-classic)    at  levels of the 
Sherali-Adams  hierarchy have an  integrality gap of 
\end{theorem}

\begin{proof}
For each lifted constraint  multiplied by multiplier  at level
,  the  corresponding distribution    or    is clearly  a
distribution over  integer solutions, so the first  condition of Lemma
\ref{SA-survival} is satisfied. For the second condition, observe that
if an event  involves more than one costly facility, it
has  probability in  all distributions. If an event 
involves only  cheap facilities,  it has the  same probability  in all
distributions    and    since  in  the  construction  of  a
distribution  we took care  not to change the probability of such
events.   An  event     that  involves  more  than  one
assignment of  a costly  facility (but no  other costly) has  in every
distribution  the same probability (which is the same as in every
)  since  in the  construction  of   we  did  not alter  the
probabilities   of   such   events.   And  lastly,   when   an   event
  involves  exactly   one  assignment  of  some  costly
facility , note that in some cases  takes the blame but in
other  cases it  does not,  depending on  . But  due  to the
iterative procedure  of probabilistic transfusion,  the probability of
event   in  a distribution  in which   is  not the
facility that takes the blame is  equal to the probability of the same
event  in the  distributions  that   takes  the  blame. So  Lemma
\ref{SA-survival} holds.  It is easy to see that bad solution has  cost  
 while any feasible solution to the instance has cost .
\end{proof}







\section{Fooling the effective capacity inequalities for \cfl\ }\label{flow-cover}

In this section we show that the (LP-classic) for \cfl\ with the addition
of the effective capacity inequalities proposed in \cite{AardalPW95} has unbounded gap. 

Consider the  general case where  facility  has capacity   and
client   has demand  .  For  a set   of clients,  we denote
their total demand by . Let  be
a set of clients, let  be a set of facilities, and let  be a set of clients for each facility . Given a  facility , we denote the  \emph{effective capacity} of
 with respect to  by .
  is a \emph{cover} with respect to   if   with .   is  called  the \emph{excess
  capacity}. Let .  In the case where  for
all    the  following inequalities  called  \emph{flow-cover}
inequalities were introduced for \cfl\ in \cite{AardalPW95}. 

\vspace*{-0.1cm}
\begin{center}

\end{center}
\vspace*{-0.2cm}

\begin{comment}
For the families of instances that we consider with uniform capacities and unit client demands, the above inequalities are simplified to:
\begin{center}

\end{center}
\end{comment}

If , the following inequalities,
called the {\em effective capacity inequalities} are  valid and strengthen the flow-cover inequalities \cite{AardalPW95}.
\begin{center}
 
\end{center}
\vspace*{-0.3cm}
The proof of the following theorem uses some of the ideas we introduced
earlier for Theorem~\ref{cfl-SA:theorem}. In the appendix we give
Theorem~\ref{thm:submod} which strictly 
generalizes 
Theorem~\ref{effective-cap:theorem} 
to the  so-called submodular inequalities. 


\begin{theorem}\label{effective-cap:theorem}
The integrality gap of the relaxation obtained from (LP-classic)  with
the addition of the effective
capacity inequalities is unbounded, even  for uniform
\cfl\ with unit demands.   
\end{theorem}
\vspace*{-0.2cm}
\begin{proof}
Consider an  instance with    cheap and   costly facilities
and   clients,  Define the bad  solution , similarly to
Section~\ref{SA-result}, s.t. 
 for  every       and      client        
Recall that    We add a
set of  facilities ,  all with  opening
costs, on the same point at  distance  from the rest (an instance of
the so-called \emph{facility location on a line}). In the bad solution
 we  additionally set  and   for all 
and for all clients .

We will prove that in every  cover  with respect to some client set
 and to the  client sets  for each , there must always be a
number of at least  clients whose assignment variables  to some costly and
to some   do not appear  in the constraint. 
This  is because if,
  for each   or,  for
each    then  the  excess capacity    since
 This  contradicts  the requirement that . 
So  there must  be a costly  facility  and  some facility
   such   that   for    the   corresponding   sets   we   have
, and so there is a set  of  clients  whose assignments to those  two facilities do  not  appear in the
constraint. We exchange the values of  and 
for all , leaving everything else the same, and we obtain a
solution  .   We  can  prove  similarly  to  the  proof  of  Lemma
\ref{assi-sym} that  is  a convex combination of integer solutions
and thus solution  satisfies the inequality since the  parts
of  and  visible to that inequality are the same.

We modify  the construction of  Lemma \ref{assi-sym} in  the following
way: facility   is  opened  of  the time but  is active
 of  the time,  when none  of  the costly
facilities  are  opened.  When it  is  not  active,  the capacity  of  its
corresponding bin is . When a costly other than  is opened
the  experiment is  the same  as  in Lemma  \ref{assi-sym}. If  costly
facility  is opened the  capacity of the corresponding bin is
  and the  capacity of
the  cheap  is  .  We randomly  select   some
 clients  that do not belong  to  to be  tossed in the
bin of  we randomly
distribute  the balls corresponding  to the  remaining clients  to the
slots of  the cheap facilities. When   is active,  and thus no
costly facility  is opened, the  capacity of the corresponding  bin is

and the capacity of the cheap is . We select
randomly  some    clients  in    and  we  toss  the
corresponding balls in the bin of .  We randomly toss the
remaining balls to the slots of the bins of the cheap facilities.

Note that  the above experiment  induces a distribution  over feasible
integer solutions since  all the defined bin capacities  are less than
 (this is by  the choice of the size of )  and every client is
assigned to exactly one opened facility in each outcome.  We do not need
this distribution to be assignment-symmetric. Observe that the expected
vector   with  respect   to  the   latter  distribution   is  solution
. Finally, note that we  once again treated the capacities  of
the bins  as
being integral.  For fractional bin capacities (which is
actually  always the case  for the  defined 's)  we can  define the
experiment in a similar way to the proof of Lemma \ref{assi-sym}.  
\end{proof}






\section{Proper Relaxations}\label{sec:firstfamily}

In this section  we present the family of  proper relaxations
and characterize their strength.
Consider a   -  vector on the set of
 variables  of the  classic  relaxation (LP-classic)
such that  for all   The meaning  of
  is the usual one that
 we open facility   Likewise, the meaning of  is
 that we assign client  to facility . We call such a vector a 
 \emph{class}. Note that  the definition is quite general  and a class
 can be defined  from any such , which may or  may not have a
 relationship  to a  feasible  integer solution.  
We  denote the  vector
 corresponding  to a  class   as .  We  associate with
 class     the  {\em cost   of  the  class}   . Let the {\em assignments of class}  be defined as 
  in .
We say that  {\em contains}  facility  if the corresponding entry
  in the vector  equals   
The set of facilities contained in  is denoted by 
  


\iffalse ------------

\begin{definition}  {\bf (Constellation LPs)} \label{def:constell}
Let  be a set of classes defined for an instance 
of
\lbfl. Let   be a variable associated with class 
The {\em  constellation LP with class set}
   is defined as 

 

\end{definition}
We will refer simply to a constellation LP when
 is implied from the context.  
------------------ \fi 

\vspace*{-0.05cm}
\begin{sloppypar}
\begin{definition}  {\bf (Constellation LPs)} \label{def:constell}
Let  be a set of classes defined for an instance 
of \cfl\ or \lbfl. Let   be a variable associated with class 
The {\em  constellation LP with class set}
   denoted LP(),  is defined as .


\begin{comment}

\end{comment}
\end{definition}
\end{sloppypar}
\vspace*{-0.03cm}


\noindent 
We  refer simply to a {\em constellation LP} when
 is implied from the context.  
We define the \emph{projection}  of solution  
of  LP to the  facility opening
and assignment variables  as  and 
.
\iffalse --------------- due to SPACE
We will restrict our attention to  constellation LPs that satisfy a  natural property:   the LP is symmetric
 with respect to the clients and  the  facilities. 
The fact  that all facilities have the  same capacity / lower bound and all
clients have unit demand makes  this  property quite sound. 
For a class    and
 a permutation of the facilities, we denote by 
the class resulting by exchanging  for all  the values  of the   and
 coordinates 
of   with   the value of  the   and  coordinates of
. Similarly,  for  a permutation of the clients, we denote  by  the class resulting  by exchanging for
every   the value of  the  coordinate of   with
 the value of the  coordinate of .


\begin{definition} {\bf (: Symmetry)} \label{def:symmetry}
We  say  that property    holds for the constellation linear program LP()   if  the
following is  true: let  be any  permutation of   and  any permutation of .
 Then, for every  class    
and  are also in 
\end{definition}
\vspace*{-0.6cm}

------------------------------ \fi 
We  restrict our attention to  constellation LPs that satisfy a
symmetry property that is very natural for uniform capacities and unit
demands. 

\vspace*{-0.1cm}
\begin{definition} {\bf (: Symmetry)} \label{def:symmetry}
We  say  that property    holds for the constellation linear program LP()   if  
 for every  class   all classes resulting from
 a  permutation that relabels the facilities and/or the clients of
  are
 also in 
\end{definition}
\vspace*{-0.3cm}



\begin{definition}  {\bf (Proper Relaxations)}  \label{def:proper} 
We call {\em proper relaxation}  for \cfl\ (\lbfl\/)  a constellation LP
 that is valid and satisfies property  
\end{definition}
\vspace*{-0.1cm}


\noindent 
 A simple  example of a constellation LP is the well-known {\em
  (LP-star)}  (see, e.g., \cite{JainMMSV03}) where 
corresponds to the set of all  {\em stars}: 
a facility and a set of at most  (or at least  for \lbfl) clients assigned to
it.
Obviously (LP-star) is a proper relaxation, while (LP-classic) is equivalent to
(LP-star). Therefore proper relaxations generalize the known natural
relaxations for \cfl\ and \lbfl. 
\begin{comment}
Our  result on proper relaxations is that  proper LPs that  are not ``complex'' enough have an unbounded integrality gap while those  that
are sufficiently ``complex'' have an integrality gap of   To that end, we
define  the complexity  of  a  proper LP. 
\end{comment}
In order to  characterize the strength of a proper LP we need  the notion of
complexity.  
\iffalse
Furthermore, for each  such facility   we  denote by
 the  set of clients   for which  there is a facility   so
that  in .
\fi 

\vspace*{-0.2cm}
\begin{definition} {\bf (Complexity of proper relaxations)}  \label{def:complexity}
Given an instance  of \cfl\ (\lbfl\/)
let  be  a 
maximum-cardinality set  of open facilities in an integral feasible
solution. The {\em complexity } of a  proper
relaxation  for  is
defined as the 
   
\end{definition}

 \vspace*{-0.09cm}

The  complexity of  a  proper LP  represents the  maximum
fraction of the  total number of feasibly openable  facilities that is
allowed in a single class. 
A complexity of nearly 
means that there are classes that take each 
into consideration almost the whole instance
at once. Low complexity means that all classes consider
the assignments of a small fraction of the instance at a time.  
\iffalse ===========================
We remark
that    the   proper    LP    with   an    integral   polytope    from
Theorem~\ref{thm:gap1} has
a complexity of   since every class corresponds  by construction to
a feasible integral solution. 
(Clearly not every LP with complexity  has an integrality gap of 
since it might contain weak classes together with the strong
ones.) 
============ \fi 
By increasing the complexity of a proper LP  for a given instance 
 we can produce strictly stronger 
proper relaxations, an example is given in the Appendix. 

\vspace*{-0.1cm}
\begin{theorem}\label{theorem:proper}
Every proper relaxation for uniform \cfl\ (\lbfl) with complexity  has an
unbounded integrality gap. There is a proper  relaxation for
\cfl\ (\lbfl) of  complexity   whose projection to  expresses the integral polytope. 
\end{theorem}



\begin{comment}
\paragraph*{Proof sketch of Theorem~\ref{theorem:proper}.}
We are given an arbitrary proper relaxation  of
complexity  for an instance with  facilities, 
clients and , and the following metric distances: 
put every facility    together with  clients, which we call \emph{exclusive} clients of ,  on a
distinct vertex of an -dimensional regular simplex in  with edge length
. Put facilities  together with their exclusive clients, which are all the  remaining clients, to a point
far away  from the simplex, so  the minimum distance from  a vertex is
. We set all the facility costs to .


A major challenge is that we have no a priori knowledge of  
We use the validity of  and the fact that 
to prove that there  is a class  with some desired properties
that must
belong to  
Using classes that are symmetric to  which also must belong to
  we construct a vector  that is feasible for
  and  whose projection  on the classic  variables is
 the  following  : for  each  facility    its
 exclusive   clients  are   assigned  to   it  with   a   fraction  of
  each, while they  are assigned with a fraction of
  to each other facility  . As for
 facilities , all of  their exclusive clients are assigned with
 a fraction of  to each.  Moreover  for
  and .

The  cost  of  the fractional  solution  we  constructed  is
 due to the assignments  of exclusive clients of
facility   to facilities  with   
As  for the cost  of an arbitrary integral  solution, observe
that since the  exclusive  clients of  are very far from
the  rest of  the facilities,  using   of them  to  satisfy some
demand of  those facilities and help  to open all of  them, incurs a
cost of  On the other hand, if we do not open all of the 
facilities on  the vertices of the  simplex (since they  have in total
  exclusive clients  which is  not enough  to open  all of
them), there  must be  at least  one such facility  not opened  in the
solution, thus its  exclusive clients must be assigned elsewhere,
incurring a cost of  
\end{comment}


\bibliographystyle{plain}

\bibliography{bibliography-ver1}

\appendix










\section{Appendix to Section~\ref{SA-result}}

Omitted part of the proof of Lemma~\ref{assi-sym}.

\begin{proof}
First we explain how to handle fractional bin capacities. 
To handle the case where the  's are not integers, we simply do the
following:  each  time  costly  facility   ()  is
picked, we set the number of slots of the corresponding bin to   ()  with probability
 (), otherwise set the slots to (). If  the  number  of slots  of
  () is  set to   ()  then   we  pick   some    ( )  cheap facilities  at  random and  set their  corresponding
number of  slots to  ()
and the number of slots of the rest of the cheap facilities to
().  Otherwise  pick some  (
)   cheap
facilities at  random and set  their corresponding number of  slots to
  () and the number of
slots of the rest to (). Note
than in every case the expected  number of slots per facility is as in
the previous experiment.  


As for the expected assignments to the cheap
facilities, observe that in every outcome of the experiment the demand
not assigned  to costly facilities  is exactly the demand  assigned to
cheap.  Since we  have proved  that  the expected  assignments to  the
costly  facilities are  those of  the  bad solution,  by linearity  of
expectation we get that the  total assignments to all cheap facilities
are  (the total  assignment of each
client adds up  to  by the constraints of the  LP). By the symmetric
way the  cheap are handled  in the experiment  we have that  the total
expected demand assigned to  each  is  and
by  the symmetric  way the  clients are  assigned to    through the
experiment we get  that the expected assignment of each   to  is
.


\end{proof}


Proof of Proposition \ref{level-ratio}.

\begin{proof}
Since  the  considered  distribution  is  assignment-symmetric,  event
 is  equivalent to  the event of  randomly distributing
 balls to the slots of   bins, with at most one ball in each
slot, each bin having  identical slots, asking that ball 
is  tossed in  the bin  of   and ball   is  tossed  in bin
.  Since there are    slots in  each bin  and the
balls    are   at    most   ,    it   is    easy   to    see   that
.

\end{proof}

Proof of Proposition \ref{transf_fraction}.

\begin{comment}
\begin{proposition}\label{transf_fraction}
Let  and  with . Then 
.
\end{proposition}
\end{comment}


\begin{proof}
Consider  again the  random  experiment of  the  proof of  Proposition
\ref{level-ratio}.     Recall   that,    ignoring    constant   factors,
 and       and
 and since   we can compute the ratio
of the  probability of  success of the  tossing of   balls when
, and thus the capacity of the bins corresponding to cheap
facilities is , to the probability of success of the tossing
of  balls  when  and thus the  capacity of the bins
corresponding to cheap facilities is . When tossing the ball
 given the successful  tossing of balls  with ,
the  probability   of  success  is     and
  respectively, where   is
the  number  of  balls  already   placed  in  some  slot  of  the  bin
corresponding    to    cheap   facility    .    We   have    that
.
\begin{comment}
Note that again both  are  and . In the ball tossing experiment, the probability of success of ball  in the case where the capacities of the cheap bins is  is at most  the probability of success of the same event in the case where the capacities of the cheap bins is  - we are very generous here, in fact the ratio of probability of success is just a little over .
\end{comment}
 So  using that . 
\end{proof}


Proof of Proposition \ref{transf:prop}.


\begin{comment}   \begin{proof}
Consider  the measure   in   of  the set  of  integer solutions
satisfying  and all events
in  
being false,
namely  . Then, by  the random experiment  of the
construction  of ,  this event  is equivalent  to the  event that
facility     is  picked,    and   the    balls
corresponding to the  clients of the rest of events  are not tossed in
their  corresponding bins. Using  again that  both 
are  and ,  we can bound the probability of the
 balls by that of   Bernoulli trials with probability of
success  (we are once  again very generous). Then the probability
that  all events  on ball   fail  is . Thus measure  is at least
 which is  constant. On the other hand the
measure   required   by   the   transfusion  step   for   each   event
 of iteration   that needs to be  fixed is at most
.     There    are
 such events of iteration , and summing over all
the iterations of our  construction we get     which   quantity   is   less   than
  for  the     levels  of  SA  we
consider, so we can always pick the required amount of measure.
\end{proof}
\end{comment} 

\begin{proof}
The intuition behind the proof is that the ``donor'' event that
supplies the required measure is much more likely to occur than the
events that require the transfusion. 

Consider  the measure   in   of  the set  of  integer solutions
satisfying  and all events encountered at any iteration 
being false,
namely  . Then, by the random experiment of the
construction  of ,  this event  is equivalent  to the  event that
facility     is  picked,    and   the    balls
corresponding to the clients of the rest of the events are not tossed in
their  corresponding bins. Using  again that  both 
are  and ,  we can bound the probability of the
 balls by that of   Bernoulli trials with probability of
success  (we are once  again very generous). Then the probability
that all events fail is at least . Thus measure  is at least
 which is  constant. On the other hand the
measure   required   by   the   transfusion  step   for   each   event
 of iteration   that needs to be  fixed is at most
.     There    are
 such events of iteration , and summing over all
the iterations of our  construction we get     which   quantity   is   less   than
  for  the     levels  of  SA  we
consider, so we can always pick the required amount of measure.
\end{proof}



\subsection{SA gap for \lbfl} 

A  similar result  to Theorem~\ref{cfl-SA:theorem}  can be  proved for
\lbfl\/. Consider an instance with  facilities, lower bound 
and a total of  clients. The metric space here is more intriguing
than the one for the \cfl\ case. Consider a regular
-dimensional 
simplex with
edge length   On each of  the  vertices of the simplex  a facility along with
some   clients are  located. All opening  costs are   Clearly
every integer solution has a cost  of at least  since we can open
at most   of the facilities,  and so at least   clients will
have to  be assigned to some facility  other than the one  on the same
vertex.  We call a client   that is located on the same vertex with
facility     \emph{exclusive}   client   of  .   We  denote   by
 the set of clients  that are exclusive to facility .
On the other hand we can  show that the following bad solution  is
feasible at   levels of the  SA hierarchy. 
 For  all  ; 
 for  a client    if
 and
  for   all  other  facilities.  Solution
 incurs a cost of .

\begin{theorem}\label{lbfl-SA:theorem}
There is  a family of  \lbfl\   instances with  facilities and  clients  
such that the  relaxations  obtained from (LP-classic)  at 
levels of  the Sherali-Adams hierarchy have an unbounded integrality
gap. 
\end{theorem}

The proof  is similar to  that of \cfl\  and is thus omitted.  Here the
reader can  find a  sketch of  the necessary changes  to the  proof of
Theorem~\ref{cfl-SA:theorem}.

\vspace*{0.2cm}
\noindent
{\em Sketch of proof of Theorem \ref{lbfl-SA:theorem}.}
Consider a constraint  and
a  multiplier   at  level   and  let   be  the set  of
variables appearing in the  multiplied constraint.  We pick a facility
 not  in   to take the  blame. We construct  a solution
 where we set  and for each  we  set   and  we
distribute the remaining demand that was assigned to  to each
facility from a constant-size set  of facilities  not appearing in
. Solution   can be obtained as a  convex combination of
integer solutions  by constructing  a distribution similarly  to Lemma
\ref{assi-sym}. This time the distribution satisfies 
 that exactly   facilities are opened in
each  outcome of the  experiment. Note that we do not require the 
underlying distribution to be assignment symmetric, 
because facilities have to treat differently their exclusive clients. 
We  set the  
values of  the linearized products appearing in the multiplied constraint
equal to  the probability of the  corresponding events with respect  
to the aforementioned  distribution. No product involving variables of 
 appear in the constraint.  For  constraints 
  and  the  construction 
of the  distribution is the  same. The distributions constructed so far
 are locally consistent as required by Lemma \ref{SA-survival}.

The case where the constraint is  is once
again  more complicated.   We choose  a  facility   and
moreover  to  take the blame and the set
 is  defined as before except  we also require  that  is
not exclusive to any of them. Solution  is constructed like in the
previous case. All products take the value of the corresponding events
in the  distribution except those in which the  unique variable involving
  appears, namely   and  those involving  facilities in
. We perform a transfusion  step so that the probabilities of all
the  events   whose  corresponding  products  appear   in  the  lifted
constraint become consistent with the distributions of the previous
case:  this  time we  need  to fix  the  probabilities  of the  events
involving facility  or some facility .





\subsection{Robustness of the SA gap} \label{sec:robust} 

In  this section  we explain to the  interested reader  how
adding simple valid inequalities does  not affect our arguments on the
SA hierarchy. 

As an example we address the valid inequality 
, where   is the
total  amount  of  demand.  This  is  a  well-known
facet-inducing constraint for our instance, see, e.g., \cite[p. 283]{LeungM89}. 
Of course this inequality is  rendered useless by  slight modifications to
the instance and the bad solution. 
Identifying  ``areas'' of a fractional  solution  where  the demand  exceeds the
available capacity is impossible  without some yet unknown form of preprocessing.
In fact part of the motivation behind   Theorem~\ref{cfl-SA:theorem}
is to demonstrate that the 
SA hierarcy is inadequate for such preprocessing purposes. It therefore 
suffices  to include in the body of the paper 
the simplest possible proof for the theorem. 



We modify  the family of  "bad" instances by  using the same  trick we
used in the proof  of Theorem~\ref{effective-cap:theorem}: we have 
cheap  and   costly  facilities  and   clients,  and the  bad
solution  in which  for every    and
client              and additionally we add a set of  
{\em dummy} facilities
,   all with   opening costs, on  the same
point  at distance   from  the  rest. In  the bad  solution   we
additionally set  and   for all  and for
all clients . The inequality is obviously satisfied.

In the  design of  the locally consistent  distributions, now  we must
give a distribution for the case where the constraint  is the new
one , and verify that the "visible"
part of  the distribution  agrees with the  visible part of  all other
distributions of  the proof.   In this case  there must be  some dummy
facility  not appearing as an  index in the multiplier  of the
constraint    (although   its       variable    does    appear   in
). Additionally  there must be  a costly facility   for which
the assignments  of clients to   do not  appear in  --
this is  ensured by the number  of rounds we consider.   We modify the
solution   to obtain  where the facilities   and
  exchange the  values  of their  corresponding assignments.   We
define  now the  random experiment  similarly  to the  proof of  Lemma
\ref{assi-sym}  with  facility     taking  the  blame.  The  only
difference is that while  is  opened 100\% of the time, it is not
assigned any demand when a  costly facility other than  is opened.
In  the terminology  of  Theorem~\ref{effective-cap:theorem},   is
always open but  it is inactive when some  
is  opened.  It  is  easy to  see  that the  distribution obtained  is
consistent with all the  other distributions defined for this modified
instance, as required by Lemma \ref{SA-survival}.



\section{Appendix to Section~\ref{flow-cover}}

\begin{comment} 
Proof of Theorem \ref{effective-cap:theorem}.

\begin{proof}
Consider an  instance with    cheap and   costly facilities
and   clients, and the bad  solution in which for  every       and      client        .  We add a
set of  facilities ,  all with  opening
costs, on the same point at  distance  from the rest (an instance of
the so-called \emph{facility location on a line}). In the bad solution
 we  additionally set  and   for all 
and for all clients .

We will prove that in every  cover  with respect to some client set
 and to the  client sets  for each , there must always be a
number of at least  clients whose assignment variables  to some costly and
to some   do not appear  in the constraint. This  is because if
either   for each   or  for
each    then  the  excess capacity    since
, which contradicts  the requirement that . So  there must  be a costly  facility  and  some facility
   such   that   for    the   corresponding   sets   we   have
, and so there is a set  of  clients  whose assignment to those  facilities does not  appear in the
constraint. We exchange the values of  and 
for all , leaving everything else the same, and we obtain a
solution  .   We  can  prove  similarly  to  the  proof  of  Lemma
\ref{assi-sym} that  is  a convex combination of integer solutions
and thus solution  satisfies the inequality since the  parts
of  and  visible to that inequality are the same.

We modify  the construction of  Lemma \ref{assi-sym} in  the following
way: facility   is  opened  of  the time but  is active
 of  the time,  when none  of  the costly
facilities  are  opened.  When it  is  not  active,  the capacity  of  its
corresponding bin is . When a costly other than  is opened
the  experiment is  the same  as  in Lemma  \ref{assi-sym}. If  costly
facility  is opened the  capacity of the corresponding bin is
  and the  capacity of
the  cheap  is  .  We randomly  select   some
 clients  that do not belong  to  to be  tossed in the
bin of  we randomly
distribute  the balls corresponding  to the  remaining clients  to the
slots of  the cheap facilities. When   is active,  and thus no
costly facility  is opened, the  capacity of the corresponding  bin is

and the capacity of the cheap is . We select
randomly  some    clients  in    and  we  toss  the
corresponding balls in the bin of .  We randomly toss the
remaining balls to the slots of the bins of the cheap facilities.

Note that  the above experiment  induces a distribution  over feasible
integer solutions since  all the defined bin capacities  are less than
 (this is by  the choice of the size of )  and every client is
assigned to exactly one opened facility in each outcome.  We do not need
this distribution to be assignment-symmetric. Observe that the expected
vector   with  respect   to  the   latter  distribution   is  solution
. Finally, note that we  once again treated the capacities  of
the bins  as
being integral.  For fractional bin capacities (which is
actually  always the case  for the  defined 's)  we can  define the
experiment in a similar way to the proof of Lemma \ref{assi-sym}.  
\end{proof}

\end{comment}





\subsection{How to fool submodular inequalities} 

Here we show that the classic relaxation strengthened by the
submodular inequalities has unbounded gap. The submodular inequalities
introduced in \cite{AardalPW95} are even stronger than the effective
capacity inequalities. We limit our discussion to uniform \cfl\ where
all clients have unit demands. 

Choose a subset  of  clients, and let  be
a subset  of facilities.  For each  facility   choose a subset
.  Consider a 3-level network  with a source  a set of
nodes corresponding to the facilities, a set of nodes corresponding
to the clients and a sink . The source  is connected by an edge of capacity
 to each facility node  That node 
 is connected by an  edge of unit
capacity to each node corresponding to client  .
Each node corresponding to some client is connected by an edge of unit
capacity to the sink .


Define    as the maximum - flow value in  
Define  as the maximum flow when facility  is
closed, i.e., when the
capacity of edge  is set to zero. 
The difference in maximum flow when all facilities in 
are open, and when all facilities except facility  are open, is
called the 
{\em increment}
function and is defined as .

For any choice of   and  for all  the following inequalities, 
 called \emph{the submodular inequalities,} are valid for
 \cfl\ \cite{AardalPW95}. The name reflects the fact that the function
  is submodular. 

\begin{center}

\end{center}

\begin{theorem}  \label{thm:submod} 
The integrality gap of (LP-classic) remains unbounded even after the addition of the submodular inequalities.  
\end{theorem}

\begin{proof}
Consider the  instance and  bad solution   that we  used in
Theorem~\ref{cfl-SA:theorem} 
for the SA
result.  To prove  that   is  feasible for  the classic  relaxation
strengthened  by  the submodular  inequalities  we  take  the idea  of
fooling local  constraints a little further: either  the constraint is
local  enough that  we  can use  the  ideas from  our previous  proofs
(define  that is a convex combination of integer solutions and has
the same  visible part as   with respect to the  constraint), or we
can  define another  instance   and solution   for  which the
inequality in question is true with respect to  and again  has
the same visible part as  with respect to the constraint. Note that
our arguments  include two different  instances as opposed to  all our
other proofs so far.

Consider the submodular inequality  for some   and some
selection of 's.  If not all the costly  facilities appear in the
constraint the  proof is similar  to that of Lemma  \ref{assi-sym}. If
at least   assignment variables
 to cheap facilities do not appear in  we do the
following: we add  one more facility  to  the instance. 
We construct  a solution  for the new instance
 as follows. 
We transfer
the demand  corresponding to the  missing assignments of the  cheap to
  and we set 
Observe that  is valid for  
Now we
can  show that    is  a convex  combination  of integer  solutions
similarly to the proof of Theorem \ref{effective-cap:theorem}, where
the role of  is played by those clients whose assignments were
transferred from the  to  
Facility    will be  active  only  when  no costly  facilities  are
open. Because, in the fractional solution 
 is assigned a total demand of at least  in each outcome of the
random experiment in  which  is active, it will  be assigned at least
one client. 
 By the convex combination produced, 
the inequality is satisfied by  . Thus the same inequality for the
original instance is satisfied by 

Now  consider  the case  where  less  than   assignments  to  cheap
facilities are missing from  We will show that it cannot be the case that
all  variables  of costly facilities appear in  the constraint as
well.  Consider  the quantity    for  some costly  facility
. If   
then  is not empty. We will show that   the set of nodes
 in  has enough incident edges so that the flow
originating from them is equal to  the total client demand  in 
We first give some properties of graph  


\begin{claim}
If less than  assignments to cheap
facilities are missing from  then  and 
 
\end{claim}

\noindent{\em Proof of Claim.} 
To see that  notice that if a cheap
facility  is  missing  from    at least    assignment
variables will
be missing  from   a contradiction. For  the second part  of the
claim, if a client  is missing from  then all the corresponding
 edges that would connect  to a cheap facility cannot be in 
Therefore at least  assignment-to-cheap variables are missing from  a
contradiction. The proof of the claim is complete. 



We return to proving that  has enough incident edges so that the flow
originating from them is equal to  the total client demand  in 
``Assign'' one client  to facility  and for the remaining
  clients do the following: assign each client 
involved in the set of variables  of assignments-to-cheap that are
missing from  to a
cheap facility   such that . There  is always such a
cheap facility   since the missing edges from  the client-nodes in
 to the cheap-facility nodes are less than 
Assign
the remaining clients arbitrarily to the cheap facilities respecting the
capacities,  since all  the  edges  from cheap  to  those clients  are
included in the network. Thus it must be the case that 

for any other costly facility . Since the  variable of
such a  facility   has  coefficient  in the constraint,  it can
take  the   blame  and  the  proof   is  similar  to   that  of  Lemma
\ref{assi-sym}.
\end{proof}


\section{Appendix to Section~\ref{sec:firstfamily}}

\begin{example}\label{proper_str}
An increased complexity allows strictly stronger proper relaxations.
\end{example}

First we show how one can construct any integer solution using classes that open the
same number of facilities.
Consider an integer solution  with opened facilities . We will use the following classes 
in which exactly  facilities are opened:
For any set of   consecutive classes in a cyclic ordering, namely , define a class that opens those facilities and makes the same assignments to them 
as . Then the integer solution is obtained  if for every  we set .
Observe that the latter solution is feasible for the proper relaxation.

We give a toy example showing that by increasing the complexity, we can
get strictly stronger relaxations. Consider an \lbfl\ instance with  facilities  sets 
of  clients each and 2 sets  of  clients each and .  For the star relaxation
(complexity  for this instance)
there is a feasible solution  whose projection to 
 is the following : for facility   and is assigned  integrally, for facility   and is assigned  integrally, for facility   and is assigned each client of  with a fraction of  and each of  with , and similarly for facility   and is assigned
each client of  with a fraction of  and each of  with . Actually
a direct consequence of Theorem \ref{theorem:proper} is that for any proper relaxation of the same complexity as the star relaxation, the above solution is feasible.

Now consider the following proper relaxation: all characteristic vectors 
of integer solutions with at most
 facilities are classes plus all the 
vectors of solutions with  facilities restricted in any  facilities ( parts of integer solutions that open all four facilities).
It is symmetric and valid by the previous discussion and has complexity . 
In any assignment of values to the class variables  that projects to  the following are true:
since classes with less than  facilities are integer solutions, they contain
assignments for all the clients and thus if we were 
to use a non-zero measure of such classes we would make non-zero assignment 
that does not exist in the support of .
 If we use
classes with exactly  facilities, then exactly one of facilities  must be present, 
since no integer solution opens them both with just the clients in . 
So we have to use at least  measure of such classes. 
So each one of facilities 
must be present in more than a unit of classes, which would make the solution infeasible.

\vspace*{0.8cm}

\noindent
{\em Proof of Theorem \ref{theorem:proper}.}

We first prove the easy part, 
that there are proper relaxations for \cfl\ and \lbfl\ with complexity  that
express the integral polytope.
For a given instance let  consist of a class for each
distinct integral solution. The resulting  is clearly
proper. Let  be any feasible solution of  and let
 be the support of  the solution. For every  and 
for every client  there is an  such that  Therefore 

This implies that  is a convex combination of integral
solutions. By the boundedness of the feasible region of
 the  corresponding polytope is integral.  
Clearly not every LP with complexity  has an integrality gap of 
since it might contain weak classes together with  strong
ones.

In the next two subsections, 
we prove the first part of Theorem \ref{theorem:proper} for \lbfl\ and \cfl\ respectively.





\subsection{Proof of Theorem \ref{theorem:proper} for \lbfl\ }
\label{sec:proof_theorem_p1}

Our proof includes the following steps. We define an instance  
and consider any proper relaxation  for  that has complexity
 
Given  we use   the validity  and symmetry properties to show the existence of
a specific set of classes in . Then we use these classes to construct a
desired feasible fractional solution, relying again on symmetry. 
In the last step  we specify  the distances between the clients and  the facilities, so
that the instance is metric and the constructed solution has an  unbounded integrality
gap.



\subsubsection{Existence of a certain type of classes}

Let us fix for the remainder of the section 
an instance  with  facilities, where  is
sufficiently large to ensure  that   where
  is a
constant greater than or equal to  . Let the bound , and let
the number of  clients be . Notice that  there are enough clients
to open  facilities, with  exactly  clients assigned  to each
one that is opened. The  facility costs  and the assignment  costs will  be defined
later.  Recall  that the  space  of  feasible  solutions of  a  proper
relaxation is independent of the costs.

 We  assume that  the  facilities are  numbered
. 
For a solution  we  denote by 
the set  of clients  that are assigned  to facility   in solution
, and  likewise for a  class  we denote  by 
the set of clients that are assigned to facility .  
Consider  an  integral  solution    to  the  instance  where 
facilities   are opened. 
Since our proper  relaxation is valid, it must have   a feasible 
solution   whose projection to  gives the characteristic
vector of .  We prove the existence of a  class  with some desirable
properties, in the support of  

By Definition~\ref{def:constell},
 can only be obtained as a positive combination of classes  such that for
every    facility       we   have    , Otherwise,  if the variables  of a  class 
with  have  non-zero value,
then in  there will be  some client assigned to
some facility with a positive fraction, while the projection of  namely 
does  not include  the
particular  assignment.  
Moreover,  since exactly  clients are  assigned to each
facility in  ,   for every facility 
that   is  contained   in  such   a  class     . To see why this  is true, 
since in   we have  for all    it follows  that for every facility ,
 .  
But  then  we have  that
.  We have already established   that . Then  is  a convex combination
of quantities less than or equal to , so for all such classes 
we have .


Therefore
in the class set of any proper relaxation for  there is 
a class  that assigns exactly  clients to each of  the
facilities in  By the value of  
  The following
lemma has been proved. 


\iffalse     --- old version 
\begin{lemma}
There is a class  that is contained in the class set of the proper
relaxation, that assigns  clients to each of  for some facilities.
\end{lemma}
\fi 
\begin{lemma}   \label{lemma:existence} 
Given the specific instance  any proper relaxation  of complexity
 for  contains in its class set a class 
 that assigns  clients to each of  facilities, for some
integer   
\end{lemma}





\subsubsection{Construction of a bad  solution}  
\label{subsec:badlbfl}

In the present section we will use the class  along with the
symmetric classes to construct a solution to the proper LP with 
the following
property: there are some   facilities   that
are almost integrally opened while the number of distinct  clients assigned to them will be less than . 

Recall that by property  every class that is isomorphic to  is
also a class of our proper relaxation. This means that
every set  of  facilities and every  set of  clients
assigned to those facilities so that each facility is assigned exactly
 clients, defines a class, called {\em admissible,} that belongs to the set of classes
defined of a  proper relaxation for the instance .

Let  us  turn  again  to  the solution    to  provide  some  more
definitions.  For   every  facility     ,  we  choose
arbitrarily a client  assigned to  it by . For each such facility
  we   denote  by     the  set  of   clients   i.e., the set of clients assigned to
 by  after we discard  (we will also call them the
{\em exclusive clients of }). For facilities   the sets
  are identical  and defined to be equal to 
 the union of  with all
the  discarded clients from  the other  facilities. In  the fractional
solution that we will construct below, the clients in 
will be almost integrally assigned to  for .

We  are   ready  to  describe  the  construction   of  the  fractional
solution. We will use a subset  of admissible classes that 
do not contain both  and  .  contains all such classes   
  that assign to each facility   in the class  the set  of clients   plus  one more
client selected  from the sets   for those facilities
 that do not belong  to  (there are at least 
of them). As for facility  (resp. ), if it  is contained in  then
it is  assigned some set  of  clients  out of the total   in
 (resp. ).  
All classes not in   will get a value
of zero in our solution.
We
will distinguish the classes in  into two types: the classes
of {\em type } that contain facility   or  but not both, and classes
of {\em type } that
 contain neither  nor .

We consider  first classes of type  . We give  to each such class   a
very small  quantity of  measure . Let   be  the total
amount of measure used. We call this step .  The
following  lemma shows  that after  , the  partial fractional
solution  induced  by  the  classes  has a  convenient  and  symmetric
structure:

\begin{lemma}  \label{lemma:roundA}
After  ,  each client      is
assigned to   with a fraction of  and is
assigned to each other facility    with a
fraction of . Each client  ()   is   assigned   to    and to  
 with   a   fraction   of .
\end{lemma}

\begin{proof}
Consider a facility  . Since exactly one of facilities   is present in
all the classes of type   and each class contains  facilities,
 is  present in the  classes of    of
the time due to symmetry of the classes. Each time  is present in
a class   that  class  assigns  all  to
.  So  client    is  assigned  to   with  a  fraction  of
. When   is not  present in  class ,
which happens   of the time, then  its exclusive clients
along with  the exclusive  clients of all  the other   facilities
that  are also  not  present in    are used  to  help the  
facilities   reach the bound  of clients (recall
that the number of exclusive clients of each such facility is equal to
).  Each time  this happens, the  facilities  in  need
  additional clients, while  the exclusive  clients of  the 
facilities that are  not present in  are   in total. Due
to symmetry  once again, a  specific client  is
assigned to  one of those   facilities 
of the  time of those cases.  So in total  this happens  of  the
time, so it follows that client  is assigned to a specific facility
    of the
time. The fraction with which   is assigned to  after 
is .

For  the proof  of the  second part  of the  lemma,  consider facilities
. Each one of those is present in the classes of type  an equal 
fraction  of the time. The
only clients that  are assigned to them are  their exclusive clients. Each
class  assigns exactly  out of those  clients. So,
due to symmetry, each client  is present in
   of the time,  so  is assigned to  and 
with a fraction of  to each.     
\end{proof}

Note that  after  each  facility  has  a total
amount   of clients (since it is present
in a class   of the time and  when this happens
it is  given  clients).  Similarly, facilities   after 
have a total amount  each.

Now we can explain the underlying intuition for distinguishing between
the two
types of classes.  The feasible fractional solution 
we  intend to  construct is  the following:  for each
facility  its exclusive clients are assigned to it with
a fraction of  each, while they are assigned with a
fraction of  to  each other facility . As  for facilities  , all  of their exclusive  clients are
assigned with a fraction of   to each.  If  we  project  the solution  to  
 , the  variables will be forced 
to  take   the  values
 for  and . Observe as we give some  amount of  measure to  ,
 the  variables  concerning the
assignments to facilities  tend to their intended values in the
solution we want to construct ``faster'' than the variables concerning the
assignments to the other facilities. This is because, by Lemma~\ref{lemma:roundA}
after  each exclusive client  of  is assigned to each of them with
a fraction of  which is  of  the intended value. At the  same time, every
exclusive  client of  each other  facility is  assigned to  it  with a
fraction of  which is   of the  intended value.  For sufficiently
large  instance  ,  as    tends  to  infinity,  the  assignments
to  and  will reach their intended values while there will
be   some    fraction   of   every    other   client   left    to   be
assigned. Subsequently we have to use classes of type , 
to achieve the opposite effect: the
variables  concerning the  assignments of  the first   facilities
should tend
to their intended values ``faster''  than those of  and  (since
 and   are 
not  present in  any of  the classes  of type  ,  the corresponding
speed will actually be zero).

We  proceed  with  giving  the  details  of  the  usage  of  type  
classes. As before,  we give to each such class  a very small quantity
of  measure .  Let    be the  total  amount of  measure
used. We call this step .

\begin{lemma}   \label{lemma:roundB}
After  ,  each client      is
assigned  to   with a  fraction of    and is
assigned to each other  facility    with a
fraction of .
\end{lemma}


\begin{proof}
The proof  follows closely that of  Lemma~\ref{lemma:roundA}. A  facility   is present in  a class of  type    of the
time (since   this  fraction is less  than ).  Each such
time, every  client  is  assigned to  it (again
this  is due  to the  definition  of classes  of type  ). So  after
,      is  assigned   to      with   a  fraction   of
.  
Also, when   is  present  in a  class, it is assigned exactly one client
which is exclusive to a facility
not in the class. Since in total there are  such candidate clients,
and by symmetry, after round  
each one of them is picked an equal fraction of the time to
be assigned to , we have that
each client  is assigned to a facility for which  is not
exclusive with  a fraction
.    
\end{proof}


\noindent
To  construct  the   aforementioned  fractional  solution ,  set      and     , and  add the  fractional assignments of  the two
rounds. 

It is easy to check that the facility and assignment variables of facilities 
take the value they have in . Same is true for the facility variables for 
and the assignment variables of the clients to the facilities they are exclusive. 
To see that the same goes for the non-exclusive assignments, observe that since
every class assign exactly  clients to its facilities we have that .
So each  takes exactly  demand from non-exclusive clients which  are
 in total. Thus, by symmetry of the construction, each one them is assigned to 
with a fraction of 



\subsubsection{Proof of unbounded integrality gap of the constructed solution}

In the  present subsection, we  manipulate the costs of  instance ,
which we left undefined, so as to create a large integrality gap while
ensuring that the distances form a metric.

Set each facility opening cost to zero. As for the connection costs (distances)
consider the -dimensional Euclidean space . Put
every facility    together with its  exclusive clients on a
distinct vertex of an -dimensional regular simplex with edge length
. Put facilities  together with their exclusive clients to a point
far away  from the simplex, so  the minimum distance from  a vertex is
 Setting  is enough.

Since the distance between a facility and one of its exclusive clients
is  ,  the  cost  of  the fractional  solution  we  constructed  is
. This cost  is due to the assignments  of exclusive clients of
facility   to facilities  with   
As  for the cost  of an arbitrary integral  solution, observe
that since the  exclusive  clients of  are very far from
the  rest of  the facilities,  using   of them  to  satisfy some
demand of  those facilities and help  to open all of  them, incurs a
cost of  On the other hand, if we do not open all of the 
facilities on  the vertices of the  simplex (since they  have in total
  exclusive clients  which is  not enough  to open  all of
them), there  must be  at least  one such facility  not opened  in the
solution, thus its  exclusive clients must be assigned elsewhere,
incurring a cost of  

This concludes the proof of Theorem~\ref{theorem:proper}. 




\subsection{Proof of Theorem \ref{theorem:proper} for \cfl\ } 
\label{sec:proof_theorem_p2}

The proof is similar to that for \lbfl. 
We prove that the relaxation must use 
a specific set of classes and then we use these classes to construct a
desired feasible solution. In the last step we 
 define appropriately  the costs of the instance. 

\subsubsection{Existence of a specific type of classes}

Consider
an instance  with  facilities, where  is
sufficiently large to ensure  that   where
  is a
constant greater than or equal to  . Let the capacity be , and let
the number of  clients be . Notice that in every integer solution of the instance
 we must open  at least  facilities. The  facility costs  and the assignment  costs will  be defined
later.  

 We  assume, like before, that  the  facilities are  numbered
. 
Consider  an  integral  solution    for    where  all the
facilities  are opened, and furthermore 
facilities   are assigned  clients each 
and facility  is assigned one client. 
Since our proper  relaxation is valid, there must be a solution  in the  space of
feasible solutions of the proper relaxation whose  projection is the characteristic 
vector of .  
By Definition~\ref{def:constell},
it is easy to see that  can only be obtained as a 
positive combination of classes  such that for
every    facility       we   have    .  Recall  that  since  the  complexity  of  our
relaxation is , the classes in the support of any solution 
have at most 
facilities. 

Now consider the support  of . We will distinguish the classes  for
which variable  is in the support of  into 2 sets. The first set consists 
of the classes that assign exactly one client to facility ; call them \emph{type A} classes.
The second set  consists  of the classes that do not assign any client to facility 
; call those \emph{type B} classes. By the discussion above those sets form a
partition of the classes in the support of , and moreover they are both non-empty: this is
 by the fact that at most   facilities are in any class, and by the fact
 that in  all  
facilities are opened integrally. Notice also that no  class
of type B can contain facility  even though the definition of a class does not
exclude the possibility that a class contains a facility to which no clients are
assigned. 

We call \emph{density} of  a class  the ratio 
. By the discussion 
above we have that  for all  in the support of . The following holds:

\begin{lemma}
All classes in the support of  have density 
\end{lemma}

\begin{proof}
The amount of demand that a class  contributes to the demand assigned to the set
of the first  facilities by  is 
 We have .
 Observe
that by the projection of  on  and by the fact that for ,
  in , we have . 
Setting  we have from
the above  and . 
The latter together with the fact that  we have that  for all classes
 in the support of .   
\end{proof}

The following corollary is immediate from the above:

\begin{corollary}
There is a type  class in the support of  that has density 
\end{corollary}

So far we have proved that 
in the class set of any proper relaxation for  there is 
a class  of type  with density .
 Let  


\subsubsection{Construction of a bad solution}

Consider the symmetric classes of  for all permutations of the  facilities
and for all permutations of the clients. Those classes are not necessarily in the support of . Take a quantity of measure  and distribute it equally among all 
those classes. Since class  has density  all those symmetric classes
assign on average  clients to each of their facilities. 
Due to symmetry, each facility is in a class  of the time and is assigned  demand. Each client is assigned to
each facility  of the time. We call that step of our construction \emph{round }.

Now consider the symmetric classes of  for all permutations of the first  facilities
and for all permutations of the clients (those classes are well defined since ).
Again distribute a quantity of measure  equally among all 
those classes. Similarly to the previous, each facility is in a class  of the time and is assigned  demand. Each client is assigned to
each facility  of the time. 
We call that step of our construction \emph{round }.

Spending  measure in round  and  
measure in round  we construct a solution  whose projection to  is the 
following :
 for , , and for every client   and 
 for  It is easy to see that  is
a feasible solution for our proper relaxation.

Now simply set all distances to , and define the facility opening costs as 
 and  for  It is easy to see
that the integrality gap of the proper relaxation is . 



\end{document}
