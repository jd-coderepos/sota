\documentclass{stacs_proc}  

\usepackage[latin2]{inputenc}
\usepackage{amsmath,amssymb}
\usepackage{enumerate}
\usepackage[all]{xy}













\newcommand{\isom}{\cong}
\newcommand{\union}{\cup}
\newcommand{\liff}{\leftrightarrow}
\newcommand{\limp}{\rightarrow}
\newcommand{\iffdef}{{\buildrel \rm def \over \iff}}
\newcommand{\eqdef}{{\buildrel \mathrm{def} \over =}}

\newcommand{\Nat}{\mathbb{N}}

\newcommand{\FO}{\mathsf{FO}}
\newcommand{\FOC}{\mathsf{FOC}}
\newcommand{\FOi}{\FO^{\infty}}
\newcommand{\FOm}{\FO^{\mathrm{mod}}}
\newcommand{\FOim}{\FO^{\infty,\mathrm{mod}}}
\newcommand{\FOo}{\FO_{<-\mathsf{inv}}}
\newcommand{\FOoim}{\FO_{<-\mathsf{inv}}^{\infty,\mathrm{mod}}}
\newcommand{\FOQ}{\FO(\mathbf{Q}_1)}
\newcommand{\MSO}{\mathsf{MSO}}
\newcommand{\wMSO}{\mathsf{wMSO}}

\newcommand{\Rat}{\mathsf{Rat}}
\newcommand{\SRat}[1]{\mathsf{S}_{#1}\mathsf{Rat}}
\newcommand{\Reg}{\mathsf{Reg}}
\newcommand{\Rec}{\mathsf{Rec}}

\newcommand{\AS}{{\text{\sc AutStr}}}
\newcommand{\dom}{\mathsf{dom}}
\newcommand{\codom}{\mathsf{co{-}dom}}
\newcommand{\IR}{\mathsf{IR}}
\newcommand{\INR}{\mathsf{INR}}
\newcommand{\AP}{\mathsf{AP}}
\newcommand{\el}{\mathsf{el}}
\newcommand{\pref}{\preceq}
\newcommand{\prefn}{\precneq}

\newcommand{\eqe}{\sim_{\textrm{e}}} 
\newcommand{\blank}{\Box}
\newcommand{\blanktimes}{\boxtimes}

\newcommand{\calA}{\mathcal{A}}
\newcommand{\calB}{\mathcal{B}}
\newcommand{\calC}{\mathcal{C}}
\newcommand{\calD}{\mathcal{D}}
\newcommand{\calE}{\mathcal{E}}
\newcommand{\calF}{\mathcal{F}}
\newcommand{\calG}{\mathcal{G}}
\newcommand{\calH}{\mathcal{H}}
\newcommand{\calI}{\mathcal{I}}
\newcommand{\calJ}{\mathcal{J}}
\newcommand{\calK}{\mathcal{K}}
\newcommand{\calL}{\mathcal{L}}
\newcommand{\calM}{\mathcal{M}}
\newcommand{\calN}{\mathcal{N}}
\newcommand{\calO}{\mathcal{O}}
\newcommand{\calP}{\mathcal{P}}
\newcommand{\calQ}{\mathcal{Q}}
\newcommand{\calR}{\mathcal{R}}
\newcommand{\calS}{\mathcal{S}}
\newcommand{\calT}{\mathcal{T}}
\newcommand{\calU}{\mathcal{U}}
\newcommand{\calV}{\mathcal{V}}
\newcommand{\calW}{\mathcal{W}}
\newcommand{\calX}{\mathcal{X}}
\newcommand{\calY}{\mathcal{Y}}
\newcommand{\calZ}{\mathcal{Z}}

\newcommand{\frakd}{\mathfrak{d}}
\newcommand{\frakA}{\mathfrak{A}}
\newcommand{\frakB}{\mathfrak{B}}
\newcommand{\frakC}{\mathfrak{C}}
\newcommand{\frakD}{\mathfrak{D}}
\newcommand{\frakE}{\mathfrak{E}}
\newcommand{\frakF}{\mathfrak{F}}
\newcommand{\frakG}{\mathfrak{G}}
\newcommand{\frakH}{\mathfrak{H}}
\newcommand{\frakI}{\mathfrak{I}}
\newcommand{\frakJ}{\mathfrak{J}}
\newcommand{\frakK}{\mathfrak{K}}
\newcommand{\frakL}{\mathfrak{L}}
\newcommand{\frakM}{\mathfrak{M}}
\newcommand{\frakN}{\mathfrak{N}}
\newcommand{\frakO}{\mathfrak{O}}
\newcommand{\frakP}{\mathfrak{P}}
\newcommand{\frakQ}{\mathfrak{Q}}
\newcommand{\frakR}{\mathfrak{R}}
\newcommand{\frakS}{\mathfrak{S}}
\newcommand{\frakT}{\mathfrak{T}}
\newcommand{\frakU}{\mathfrak{U}}
\newcommand{\frakV}{\mathfrak{V}}
\newcommand{\frakW}{\mathfrak{W}}
\newcommand{\frakX}{\mathfrak{X}}
\newcommand{\frakY}{\mathfrak{Y}}
\newcommand{\frakZ}{\mathfrak{Z}}









\begin{document}


\title[Cardinality and counting quantifiers on omega-automatic structures]
{Cardinality and counting quantifiers on omega-automatic structures}

\author[mgi]{\L{}.~Kaiser}{\L{}ukasz Kaiser}
\author[uoa]{S. Rubin}{Sasha Rubin}
\author[mgi]{V. B\'ar\'any}{Vince  B\'ar\'any}

\address[mgi]{Mathematische Grundlagen der Informatik, RWTH Aachen}
\email{{kaiser,vbarany}@informatik.rwth-aachen.de} 

\address[uoa]{Department of Computer Science, University of Auckland}
\email{rubin@cs.auckland.ac.nz}

\keywords{ -automatic presentations, -semigroups, -automata}

\bibliographystyle{plain}




\begin{abstract}
We investigate structures that can be represented by omega-automata, so called
omega-automatic structures, and prove that relations defined over such
structures in first-order logic expanded by the first-order quantifiers `there
exist at most  many', 'there exist finitely many' and 'there exist
 modulo  many' are omega-regular. The proof identifies certain algebraic
properties of omega-semigroups.

As a consequence an omega-regular equivalence relation of countable index has
an omega-regular set of representatives. This implies Blumensath's conjecture
that a countable structure with an -automatic presentation can be
represented using automata on finite words. This also complements a very recent
result of Hj\"orth, Khoussainov, Montalban and Nies showing that 
there is an omega-automatic structure which has no injective presentation. 


\end{abstract}










\maketitle

\stacsheading{2008}{385-396}{Bordeaux}
\firstpageno{385}





\vskip-0.3cm
\section{Introduction} \label{sec_intro}



\setlength{\parskip}{0pt}



Automatic structures were introduced in \cite{Hod83} and later again in
\cite{KN95,BG00} along the lines of the B\"uchi-Rabin equivalence of automata and
monadic second-order logic. The idea is to encode elements of a
structure  via words or labelled trees (the codes need not be unique)
and to represent the relations of  via synchronised automata. This way
we reduce the first-order theory of  to the monadic second-order theory
of one or two successors.  In particular, the encoding of relations defined in
 by first order formulas are also regular, and automata for them can be
computed from the original automata. Thus we have the fundamental fact that the
first-order theory of an automatic structure is decidable. 

Depending on the type of elements encoding the structure, the following natural
classes of structures appear: automatic (finite words), -automatic
(infinite words), tree-automatic (finite trees), and -tree automatic
(infinite trees). Besides the obvious inclusions, for instance that automatic
structures are also -automatic, there are still some some outstanding
problems.  For instance, a presentation over finite words or over finite trees 
can be transformed into one where each element has a unique representative. 

Kuske and Lohrey \cite{KL06} point out an -regular
equivalence relation (namely  stating that two infinite words are
position-wise eventually equal) with no -regular set of representatives.
Thus, unlike the finite-word case, injectivity can not generally be achieved by
selecting a regular set of representatives from a given presentation. 
In fact, using topological methods it has recently been shown \cite{HKMNman} 
that there are omega-automatic structures having no injective presentation.
However, we are able to prove that every omega-regular equivalence relation
having only countably many classes does allow to select an omega-regular
set of unique representants. Therefore, every countable omega-automatic structure 
does have an injective presentation.

A related question raised by Blumensath \cite{Blu99} is whether every countable
-automatic structure is also automatic. In Corollary~\ref{coroll_cnt} 
we confirm this by transforming the given presentation into an injective one, 
and then noting that an injective -automatic presentation of a countable 
structure can be ``packed'' into one over finite words.

All these results rest on our main contribution: a characterisation of when
there exist countably many words  satisfying a given formula with parameters
in a given -automatic structure  (with no restriction on the cardinality
of the domain of  or the injectivity of the presentation). The
characterisation is first-order expressible in an -automatic
presentation of an extension of  by .
Hence we obtain an extension of the fundamental fact for -automatic
structures to include cardinality and counting quantifiers such as 'there exists
(un)countably many', 'there exists finitely many', and 'there exists  modulo
 many'.  This generalises results of Kuske and Lohrey \cite{KL06} who achieve
this for structures with \emph{injective} -automatic presentations.
 


\vskip-0.3cm
\section{Preliminaries} \label{sec_prelim}



By countable we mean finite or countably infinite. 
Let  be a finite alphabet. With  and 
we denote the set of finite, respectively -words over .
The length of a word  is denoted by ,
the empty word by , 
and for each  the th symbol of  is written as . 
Similarly  is the factor  and  is defined
by .
Note that we start indexing with  and that for  we
denote by  the concatenation of  number of s, in particular
. 

We consider relations on finite and -words recognised by multi-tape
finite automata operating in a synchronised letter-to-letter fashion.
Formally, \emph{-regular relations} are those accepted by some
finite non-deterministic automaton  with B\"uchi, parity or Muller
acceptance conditions, collectively known as -automata,
and having transitions labelled by -tuples of symbols of .
Equivalently,  is a usual one-tape -automaton over the
alphabet  accepting the \emph{convolution}  of
-words  defined by 
 for all .


Words  have {\em equal ends}, written , if
for almost all , .  This is an important
-regular equivalence relation. We overload notation so that for  we write  to mean for almost all , .
 
\begin{example}\label{ex_1} The non-deterministic B\"uchi automaton depicted in
Fig. \ref{fig_ex1} accepts the equal-ends relation on alphabet . 

\begin{figure}[h]

\caption{An automaton for the equal ends relation .}
\label{fig_ex1}
\end{figure}
\end{example}

In the case of finite words one needs to introduce a padding
end-of-word symbol  
to formally define convolution of words of different length. 
For simplicity, we shall identify each finite word 
with its infinite padding  
where . 
To avoid repeating the definition of automata for finite words,
we say that a -ary relation  
is \emph{regular} (\emph{synchronised rational}) whenever 
it is -regular over .


\vskip-0.3cm
\subsection{Automatic structures}

We now define what it means for a relational structure (we implicitly replace
any structure with its relational counterpart) to have an (-)automatic
presentation. 

\begin{definition}[(-)Automatic presentations] \label{def_as} ~\\
Consider a relational structure  with universe
 and relations .
A tuple of -automata 

together with a surjective naming function 
constitutes an \emph{(-)automatic presentation} of 
if the following criteria are met: 
\begin{enumerate}[(i)]
\item the equivalence, denoted , and defined by  
   is recognised by ,
\item every  has the same arity as ,
\item  is an isomorphism between 
    and .
\end{enumerate}
The presentation is said to be \emph{injective} whenever  is, 
in which case  can be omitted.
\end{definition}

The relation  needs to be a congruence of the structure 
 for item (iii) to make sense. 
In case  only consists of words of the form  where 
, we say that the presentation is {\em automatic}.  
Call a structure {\em automatic} if it has an (-)automatic 
presentation.








The advantage of having an (-)automatic presentation of a structure 
lies in the fact that first-order () formulas can be effectively evaluated
using classical automata constructions. This is expressed by the following 
fundamental theorem.

\begin{theorem} \label{thrm_fo}
   (Cf. \cite{Hod83}, \cite{KN95}, \cite{BG04}.) ~\ 
  s \cdot (t \ast \alpha) = (s \cdot t) \ast \alpha 

  s_0 \cdot \pi(s_1, s_2, \ldots) = \pi(s_0, s_1, s_2, \ldots)
 
  \pi(s_0, s_1, s_2, \ldots) =  
    \pi(s_0 s_1 \cdots s_{k_1}, s_{k_1 +1} s_{k_1 +2} \cdots s_{k_2}, \ldots)

  \forall \vec{z} \left(
    \exists^{\leq \aleph_0} w \, . \, \varphi(w,\vec{z})  
  \longleftrightarrow  
    \exists x_1 \ldots x_C \left( 
      \bigwedge_i \varphi(x_i,\vec{z}) \land 
      \forall x \varphi(x,\vec{z}) \limp 
        \exists y (x \approx y \land \bigvee_i y \eqe x_i) 
    \right)
  \right)

\langle \
\big( \otimes(x_i,\vec{z})[n,m)^\varphi \, \big)_{0 \leq i \leq C}\ , \
\big( \otimes(x_i,x_j)[n,m)^\approx \, \big)_{0 \leq i \leq j \leq C} \
\rangle.

\left<
(s_i)_{1 \leq i \leq C}, 
(t_{(i,j)})_{1 \leq i \leq j \leq C}
\right>

y_2[0,h_2) & := & x_2[0,h_2) \text{ and}\\
y_2[h_{2n},h_{2n+2}) & := & x_2[h_{2n},h_{2n+1}) x_1[h_{2n+1},h_{2n+2}) \text{ for } n > 0.

\otimes(y_1,\vec{z})^\varphi 	& = &	\otimes(x_2,\vec{z})[0,h_2)^\varphi s_1^\omega \\
				& = &	\otimes(x_2,\vec{z})[0,h_2)^\varphi s_2^\omega \\
				& = &	\otimes(x_2,\vec{z})^\varphi

\otimes(y_2,\vec{z})^\varphi 	& = &	\otimes(x_2,\vec{z})[0,h_2)^\varphi (s_2 s_1)^\omega \\
				& = &	\otimes(x_2,\vec{z})[0,h_2)^\varphi s_2^\omega \\
				& = &	\otimes(x_2,\vec{z})^\varphi

\otimes(y_1,y_2)^\approx	
& = & 	\pi_\approx \big( 
		\otimes(x_2,x_2)[0,h_2)^\approx,\, 
		\big( \otimes(x_1,x_2)[h_{2n},h_{2n+1})^\approx,\, 
		 \otimes(x_1,x_1)[h_{2n+1},h_{2n+2})^\approx
		\big)_{n  \in \Nat^+}
		\big)\\
& = & 	\otimes(x_2,x_2)[0,h_1)^\approx \, t_{(2,2)} \ 
		(t_{(1,2)}t_{(1,1)})^{\omega}\\
& = & 	\otimes(x_2,x_2)[0,h_1)^\approx \,  
		t_{(2,2)}t_{(2,2)} \  
		(t_{(1,2)}t_{(1,1)})^{\omega}\\
& = & 	\otimes(x_2,x_2)[0,h_1)^\approx \,  
		t_{(2,2)}t_{(2,2)} \  
		(t_{(1,2)}t_{(2,2)})^{\omega}\\
& = & 	\otimes(x_2,x_2)[0,h_1)^\approx \,  
		t_{(2,2)} \  
		(t_{(2,2)} t_{(1,2)})^{\omega}\\
& = & 	\pi_\approx \big( 
		\otimes(x_2,x_2)[0,h_2)^\approx,\, 
		\big(\otimes(x_2,x_2)[h_{2n},h_{2n+1})^\approx,\, 
		 \otimes(x_1,x_2)[h_{2n+1},h_{2n+2})^\approx
		\big)_{n  \in \Nat^+}
		\big)\\
& = & 	\otimes(y_2,x_2)^\approx	

  t^\uparrow t = r^{\uparrow k} r^k = r^{\uparrow k} = t^{\uparrow} & 
    \quad \text{because } r^{\uparrow} \text{ absorbs } r 

\otimes(y_j,y_j)[0,g_1)^{\approx} 
		& = & \otimes(y_j,y_j)[0,h_2)^{\approx} \otimes(y_j,y_j)[h_2,h_{4k})^\approx\\
		& = & \otimes(y_j,y_j)[0,h_2)^{\approx} r^{4k-2}\\
		& = & \otimes(y_j,y_j)[0,h_2)^{\approx} r^{3k-2}t\\

  \otimes(y_2,y_1)[g_i,g_{i+1})^\approx 	
      \ = \  \otimes(y_2,y_1)[h_{2k(i+1)},h_{2k(i+2)})^\approx  
      \ = \  (r^\downarrow)^k 
      \ = \  t^\downarrow.

\chi_S[0,g_1) & := &y_2[0,g_1) \text{ , and }\\
\chi_S[g_n,g_{n+1}) & := &
\begin{cases} 	
		y_2[g_n,g_{n+1}) & \text{ if } n \in S \\
		y_1[g_n,g_{n+1}) & \text{ otherwise}
\end{cases}

\otimes(\chi_S,\vec{z})^\varphi 	& = & \otimes(y_2,\vec{z})[0,g_1)^\varphi s^\omega\\
				& = & \otimes(y_2,\vec{z})^\varphi 

\otimes(\chi_S,\chi_T)^\approx= 
\begin{cases}
\otimes(x_{\circ\bullet},x_{\bullet\circ})^\approx & \text{ or }\\
\otimes(x_{\bullet\circ},x_{\circ\bullet})^\approx
\end{cases}

p_n := 
		\begin{cases}	
		t^\downarrow & \text{ if } n \in S \setminus T \\
		t^\uparrow   & \text{ if } n \in T \setminus S \\
		t	     & \text{ otherwise}
		\end{cases}

\otimes(\chi_S,\chi_T)^\approx	
& = & 	\pi_\approx \left( 
		p, \, 
		(p_n)_{n \in \Nat}
		\right) 
\  = \ 	p (t^\downarrow t^\uparrow)^\omega \\
& = & 	\otimes(x_{\bullet\circ},x_{\circ\bullet})^\approx

\otimes(x_{\bullet\circ},x_{\circ\bullet\circ\circ})^\approx	
& = & 	p (t^\downarrow t^\uparrow t^\downarrow t)^\omega 
\ = \ 	p (t^\downarrow t^\uparrow t^\downarrow)^\omega 
\ = \  	p (t^\downarrow t^\uparrow)^\omega \\
& = & 	\otimes(x_{\bullet\circ},x_{\circ\bullet})^\approx 

\otimes(x_{\circ\bullet},x_{\circ\bullet\circ\circ})^\approx	
& = & 	p (t t t t^\downarrow)^\omega 
\ = \  	p (t^\downarrow)^\omega \\
& = & 	\otimes(y_2,y_1)^\approx 

\exists x_1 \cdots x_C \, \Psi(x_1,\cdots,x_C,\vec{z})

\exists y_1 \cdots y_C 
  \forall x 
    \exists y 
	\left( 
           \varphi(x,\vec{z})  \limp
	       x \approx y  \land  \bigvee_i \pi(y,y_i,x_i)
	\right)

\bigwedge_i x_i \in A \land (\forall x \in A) (\exists y)\, [x \approx y \land \bigvee_i y \eqe x_i]
0.5 em]

\noindent
{\bf Corollary \ref{coroll_cnt}}
{\it A countable structure is -automatic if and only if it is automatic.
Transforming a presentation of one type into the other can be done effectively.
}\ 
  \frakA \models \exists y \varphi(\vec{b},y) \quad \Rightarrow \quad 
  \frakA_{\textrm{up}} \models \exists y \varphi(\vec{b},y) \ .

By Theorem \ref{thrm_fo}  defines an 
omega-regular relation and, similarly, since the parameters  are 
all ultimately periodic the set defined by  is omega-regular.
Therefore, if it is non-empty, then it also contains an ultimately periodic word,
which is precisely what we needed.
\end{proof}

This proof can be viewed as a model construction akin to a classical
compactness proof.  Indeed, starting with ultimately constant words and
throwing in witnesses for all existential formulas satisfied in  in
each round one constructs an increasing sequence of substructures comprising
ultimately periodic words of increasing period lengths. The union of these is
closed under witnesses by construction. The argument is valid for relational
structures with constants assuming that every constant is represented by an
ultimately periodic word. \\



\noindent {\bf Future work} It remains to be seen whether statements analogous 
to Theorem~\ref{thrm_foc} and Corollary~\ref{coroll_cnt} also hold for automatic 
presentations over infinite trees. \\

\noindent {\bf Acknowledgment} We thank the referees for detailed technical 
remarks and corrections.






\vskip-0.3cm
\begin{thebibliography}{10}

\bibitem{Blu99}
A.~Blumensath.
\newblock Automatic structures.
\newblock {Diploma thesis, RWTH-Aachen}, 1999.

\bibitem{BG00}
A.~Blumensath and E.~Gr{\"a}del.
\newblock {Automatic Structures}.
\newblock In {\em Proceedings of 15th IEEE Symposium on Logic in Computer
  Science LICS 2000}, pages 51--62, 2000.

\bibitem{BG04}
A.~Blumensath and E.~Gr{\"a}del.
\newblock Finite presentations of infinite structures: Automata and
  interpretations.
\newblock {\em Theory of Comp. Sys.}, 37:641 -- 674, 2004.

\bibitem{HKMNman}
G.~Hj{\"o}rth, B.~Khoussainov, A.~Montalban, and A.~Nies.
\newblock Borel structures.
\newblock Manuscript, 2007.

\bibitem{Hod83}
B.R. Hodgson.
\newblock D\'ecidabilit\'e par automate fini.
\newblock {\em Ann. sc. math. Qu\'ebec}, 7(1):39--57, 1983.

\bibitem{KN95}
B.~Khoussainov and A.~Nerode.
\newblock Automatic presentations of structures.
\newblock In {\em LCC '94}, volume 960 of {\em LNCS}, pages 367--392.
  Springer-Verlag, 1995.

\bibitem{KNRS04}
B.~Khoussainov, A.~Nies, S.~Rubin, and F.~Stephan.
\newblock Automatic structures: Richness and limitations.
\newblock In {\em LICS}, pages 44--53. IEEE Comp. Soc., 2004.

\bibitem{KRS04}
B.~Khoussainov, S.~Rubin, and F.~Stephan.
\newblock Definability and regularity in automatic structures.
\newblock In {\em STACS '04}, volume 2996 of {\em LNCS}, pages 440--451, 2004.

\bibitem{KL06}
D.~Kuske and M.~Lohrey.
\newblock First-order and counting theories of -automatic structures.
\newblock In {\em FoSSaCS}, pages 322--336, 2006.

\bibitem{PP95}
D.~Perrin and J.-E. Pin.
\newblock Semigroups and automata on infinite words.
\newblock In J.~Fountain, editor, {\em Semigroups, Formal Languages and
  Groups}, NATO Advanced Study Institute, pages 49--72. Kluwer, 1995.

\end{thebibliography}


\end{document}
