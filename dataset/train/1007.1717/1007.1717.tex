\documentclass[fleqn,12pt,twoside]{article}


\usepackage[headings]{espcrc1}
\readRCS

\ProvidesFile{espcrc1.tex}[\filedate \space v\fileversion
     \space Elsevier 1-column CRC Author Instructions]


\usepackage{graphicx}
\usepackage[figuresright]{rotating}

\newtheorem{theorem}{Theorem}
\newtheorem{algorithm}{Algorithm}
\newtheorem{axiom}{Axiom}
\newtheorem{case}{Case}
\newtheorem{claim}{Claim}
\newtheorem{conclusion}[theorem]{Conclusion}
\newtheorem{condition}[theorem]{Condition}
\newtheorem{conjecture}[theorem]{Conjecture}
\newtheorem{corollary}[theorem]{Corollary}
\newtheorem{criterion}[theorem]{Criterion}
\newtheorem{definition}{Definition}
\newtheorem{example}{Example}
\newtheorem{exercise}[theorem]{Exercise}
\newtheorem{lemma}[theorem]{Lemma}
\newtheorem{notation}[theorem]{Notation}
\newtheorem{problem}{Problem}
\newtheorem{proposition}[theorem]{Proposition}
\newtheorem{remark}[theorem]{Remark}
\newtheorem{solution}[theorem]{Solution}
\newtheorem{summary}[theorem]{Summary}

\newenvironment{dedication}[1][]{\begin{trivlist}
\item[\hskip \labelsep {\bfseries #1}]}{\end{trivlist}}

\newenvironment{proof}[1][Proof.]{\begin{trivlist}
\item[\hskip \labelsep {\bfseries #1}]}{\end{trivlist}}

\newenvironment{acknowledgement}[1][Acknowledgement]{\begin{trivlist}
\item[\hskip \labelsep {\bfseries #1}]}{\end{trivlist}}

\newcommand{\ttbs}{\char'134}
\newcommand{\AmS}{{\protect\the\textfont2
  A\kern-.1667em\lower.5ex\hbox{M}\kern-.125emS}}

\hyphenation{author another created financial paper re-commend-ed Post-Script}

\usepackage{amsmath}
\usepackage{amsfonts}
\title{A note on interval edge-colorings of graphs}

\author{R.R. Kamalian\address[MCSD]{Institute for Informatics and Automation Problems,\\
National Academy of Sciences, 0014, Armenia}\address{Department of Applied Mathematics and Informatics,\\
Russian-Armenian State University, 0051, Armenia}\thanks{email: rrkamalian@yahoo.com.},
        P.A. Petrosyan\addressmark[MCSD]\address{Department of Informatics and Applied Mathematics,\\
Yerevan State University, 0025, Armenia}\thanks {email: pet\_petros@\{ipia.sci.am, ysu.am, yahoo.com\}}}


\runtitle{A note on interval edge-colorings of
graphs}\runauthor{R.R. Kamalian, P.A. Petrosyan}

\begin{document}

\maketitle

\begin{abstract}
An edge-coloring of a graph  with colors  is
called an interval -coloring if for each 
there is at least one edge of  colored by , and the colors of
edges incident to any vertex of  are distinct and form an
interval of integers. In this paper we prove that if a connected
graph  with  vertices admits an interval -coloring, then
. We also show that if  is a connected -regular
graph with  vertices has an interval -coloring and , then this upper bound can be improved to .\\

Keywords: edge-coloring, interval coloring, bipartite graph, regular
graph

\end{abstract}

\bigskip

\section{Introduction}\

All graphs considered in this paper are finite, undirected, and have
no loops or multiple edges. Let  and  denote the sets of
vertices and edges of , respectively. An -biregular
bipartite graph  is a bipartite graph  with the vertices in
one part all having degree  and the vertices in the other part
all having degree . A partial edge-coloring of  is a coloring
of some of the edges of  such that no two adjacent edges receive
the same color. If  is a partial edge-coloring of  and
, then  denotes the set of
colors of colored edges incident to .

An edge-coloring of a graph  with colors  is
called an interval -coloring if for each 
there is at least one edge of  colored by , and the colors of
edges incident to any vertex of  are distinct and form an
interval of integers. A graph  is interval colorable, if there is
 for which  has an interval -coloring. The set of all
interval colorable graphs is denoted by . For a graph
, the greatest value of  for which  has an
interval -coloring is denoted by .

The concept of interval edge-coloring was introduced by Asratian and
Kamalian \cite{b2}. In \cite{b2,b3} they proved the following
theorem.

\begin{theorem}
\label{mytheorem1} If  is a connected triangle-free graph and
, then
\begin{center}
.
\end{center}
\end{theorem}

In particular, from this result it follows that if  is a
connected bipartite graph and , then . It is worth noting that for some families of
bipartite graphs this upper bound can be improved. For example, in
\cite{b1} Asratian and Casselgren proved the following

\begin{theorem}
\label{mytheorem2} If  is a connected -biregular bipartite
graph with  and ,
then
\begin{center}
.
\end{center}
\end{theorem}

For general graphs, Kamalian proved the following

\begin{theorem}
\label{mytheorem3}\cite{b6}. If  is a connected graph and , then
\begin{center}
.
\end{center}
\end{theorem}

The upper bound of Theorem \ref{mytheorem3} was improved in
\cite{b5}.

\begin{theorem}
\label{mytheorem4}\cite{b5}. If  is a connected graph with  and , then
\begin{center}
.
\end{center}
\end{theorem}

On the other hand, in \cite{b7} Petrosyan proved the following
theorem.

\begin{theorem}
\label{mytheorem4} For any , there is a graph 
such that  and
\begin{center}
.
\end{center}
\end{theorem}

For planar graphs, the upper bound of Theorem \ref{mytheorem3} was
improved in \cite{b4}.

\begin{theorem}
\label{mytheorem6}\cite{b4}. If  is a connected planar graph and
, then
\begin{center}
.
\end{center}
\end{theorem}

In this note we give a short proof of Theorem \ref{mytheorem3} based
on Theorem \ref{mytheorem1}. We also derive a new upper bound for
the greatest possible number of colors in interval edge-colorings of
regular graphs.

\bigskip

\section{Main results}\


\begin{proof}[Proof of Theorem \ref{mytheorem3}.] Let  and  be an
interval -coloring of the graph . Define an auxiliary graph
 as follows:
\begin{center}
, where
\end{center}
\begin{center}
, 
and
\end{center}
\begin{center}
.
\end{center}

Clearly,  is a connected bipartite graph with .\\

Define an edge-coloring  of the graph  in the following
way:
\begin{description}
\item[(1)]  for
every edge ,

\item[(2)]  for
.
\end{description}

It is easy to see that  is an edge-coloring of the graph 
with colors  and  for . Now we present an interval
-coloring of the graph . For that we take one edge
 with , and recolor it with color . Clearly, such
a coloring is an interval -coloring of the graph .
Since  is a connected bipartite graph and , by
Theorem \ref{mytheorem1}, we have
\begin{center}
, thus
\end{center}
\begin{center}
.
\end{center}
~
\end{proof}

\begin{theorem}
\label{mytheorem7} If  is a connected -regular graph with
 and , then
\begin{center}
.
\end{center}
\end{theorem}
\begin{proof} In a similar way as in the prove of Theorem
\ref{mytheorem3}, we can construct an auxiliary graph  and to
show that this graph has an interval -coloring. Next,
since  is a connected -regular bipartite graph with  and , by Theorem
\ref{mytheorem2}, we have
\begin{center}
, thus
\end{center}
\begin{center}
.
\end{center}
~
\end{proof}\


\begin{thebibliography}{99}

\bibitem{b1} A.S. Asratian, C.J. Casselgren, On interval edge colorings of
-biregular bipartite graphs, Discrete Mathematics
307 (2006) 1951-1956.

\bibitem{b2} A.S. Asratian, R.R. Kamalian, Interval colorings of edges of a
multigraph (in Russian), Appl. Math. 5 (1987) 25-34.

\bibitem{b3} A.S. Asratian, R.R. Kamalian, Investigation on interval
edge-colorings of graphs, J. Combin. Theory Ser. B 62 (1994) 34-43.

\bibitem{b4} M.A. Axenovich, On interval colorings of planar graphs, Congr.
Numer. 159 (2002) 77-94.

\bibitem{b5} K. Giaro, M. Kubale, M. Malafiejski, Consecutive colorings of
the edges of general graphs, Discrete Math. 236 (2001) 131-143.

\bibitem{b6} R.R. Kamalian, Interval edge-colorings of graphs, Doctoral
Thesis, Novosibirsk, 1990.

\bibitem{b7} P.A. Petrosyan, Interval edge-colorings of complete graphs and
-dimensional cubes, Discrete Mathematics 310 (2010) 1580-1587.

\end{thebibliography}
\end{document}
