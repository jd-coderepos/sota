\documentclass{llncs} 

\usepackage[T1]{fontenc}

\usepackage[dvips]{graphicx}
\usepackage{amsmath}
\usepackage{amssymb}
\usepackage{color}
\usepackage{enumitem}
\usepackage{subfig}
\usepackage{color}
\usepackage{multicol}
\usepackage{pgf,tikz}
\usetikzlibrary{arrows}
\usetikzlibrary{patterns}

\newtheorem{cor}[theorem]{Corollary}
\newtheorem{obs}{Observation}





\usepackage{algpseudocode}





\usepackage{algorithm}


\usepackage{epsfig}
\usepackage{tikz}


\title{Covering the Boundary of a Simple Polygon \\ with Geodesic Unit Disks}
\author{George Rabanca \and Ivo Vigan \thanks{Research supported by NSF grant 1017539}}
\institute{Department of Computer Science, City University of New York, \\The Graduate Center, New York, NY 10016, USA.}


\pagestyle{plain}

\begin{document}

\maketitle



\begin{abstract}          
We consider the problem of covering the boundary of a simple polygon on  vertices using the minimum number of geodesic unit disks. We present an  time -approximation algorithm for finding the centers of the disks, with  denoting the number of centers found by the algorithm.

\end{abstract}

\section{Introduction and Main Results}










For two points  and  in a simple polygon , the \emph{geodesic distance}, denoted by , is the length of the shortest path between  and  inside . A \emph{geodesic unit disk}  centered at a point  is the set of all points in  whose geodesic distance to  is at most .

The \emph{boundary} of , denoted by , consists of all points in  which are either exactly at distance  from  or at distance at most  from  but contained in the polygon boundary . The \emph{interior} of , denoted by ,  consists of all the points of  not contained on the boundary of , i.e.,  ,  as shown in Fig.~\ref{geoDisk}. 

\begin{figure}
\center
\definecolor{uuuuuu}{rgb}{0.27,0.27,0.27}
\definecolor{qqqqff}{rgb}{0.9,0.9,.9}

\definecolor{dddddd}{rgb}{0.7,0.6,0.4}

\begin{tikzpicture}[line cap=round,line join=round,>=triangle 45,x=.7cm,y=.7cm]
\clip(1.5,-0.5) rectangle (6.4,6.1);
\fill[color=qqqqff,fill=qqqqff,fill opacity=1.0] (2.73,3.12) -- (1.9,2.66) -- (2.01,3.18) -- cycle;
\fill[color=qqqqff,fill=qqqqff,fill opacity=1.0] (4,3) -- (2.01,3.18) -- (2.27,4.4) -- cycle;
\fill[color=qqqqff,fill=qqqqff,fill opacity=1.0] (4,3) -- (4.03,2.47) -- (2.73,3.12) -- cycle;
\fill[color=qqqqff,fill=qqqqff,fill opacity=1.0] (4,3) -- (3.71,3.73) -- (3.03,5) -- cycle;
\fill[color=qqqqff,fill=qqqqff,fill opacity=1.0] (4,3) -- (4.46,0.83) -- (4.03,2.47) -- cycle;
\fill[color=qqqqff,fill=qqqqff,fill opacity=1.0] (4,3) -- (4.83,0.94) -- (4.98,3.82) -- cycle;
\fill[color=qqqqff,fill=qqqqff,fill opacity=1.0] (4,3) -- (5.16,4.89) -- (3.71,3.73) -- cycle;

\draw [shift={(4,3)},color=qqqqff,fill=qqqqff,fill opacity=1.0]  (0,0) --  plot[domain=4.92:5.1,variable=\t]({1*2.22*cos(\t r)+0*2.22*sin(\t r)},{0*2.22*cos(\t r)+1*2.22*sin(\t r)}) -- cycle ;
\draw [shift={(4,3)},color=qqqqff,fill=qqqqff,fill opacity=1.0]  (0,0) --  plot[domain=0.7:1.02,variable=\t]({1*2.22*cos(\t r)+0*2.22*sin(\t r)},{0*2.22*cos(\t r)+1*2.22*sin(\t r)}) -- cycle ;
\draw [shift={(4.98,3.82)},color=qqqqff,fill=qqqqff,fill opacity=1.0]  (0,0) --  plot[domain=-0.42:0.7,variable=\t]({1*0.94*cos(\t r)+0*0.94*sin(\t r)},{0*0.94*cos(\t r)+1*0.94*sin(\t r)}) -- cycle ;

\draw [shift={(4,3)},color=qqqqff,fill=qqqqff,fill opacity=1.0]  (0,0) --  plot[domain=2.02:2.46,variable=\t]({1*2.22*cos(\t r)+0*2.22*sin(\t r)},{0*2.22*cos(\t r)+1*2.22*sin(\t r)}) -- cycle ;

\draw [shift={(4.98,3.82)}] plot[domain=-0.42:0.7,variable=\t]({1*0.94*cos(\t r)+0*0.94*sin(\t r)},{0*0.94*cos(\t r)+1*0.94*sin(\t r)});
\draw [shift={(4,3)}] plot[domain=4.92:5.1,variable=\t]({1*2.22*cos(\t r)+0*2.22*sin(\t r)},{0*2.22*cos(\t r)+1*2.22*sin(\t r)});
\draw [shift={(4,3)}] plot[domain=2.02:2.46,variable=\t]({1*2.22*cos(\t r)+0*2.22*sin(\t r)},{0*2.22*cos(\t r)+1*2.22*sin(\t r)});
\draw [shift={(4,3)}] plot[domain=0.7:1.02,variable=\t]({1*2.22*cos(\t r)+0*2.22*sin(\t r)},{0*2.22*cos(\t r)+1*2.22*sin(\t r)});

\draw [shift={(2.73,3.12)},color=qqqqff,fill=qqqqff,fill opacity=1.0]  (0,0) --  plot[domain=3.65:4.79,variable=\t]({1*0.95*cos(\t r)+0*0.95*sin(\t r)},{0*0.95*cos(\t r)+1*0.95*sin(\t r)}) -- cycle ;


\draw [shift={(2.73,3.12)}] plot[domain=3.65:4.79,variable=\t]({1*0.95*cos(\t r)+0*0.95*sin(\t r)},{0*0.95*cos(\t r)+1*0.95*sin(\t r)});






\draw[dotted] (2.88,0.17)-- (1.67,1.56);
\draw[dotted] (1.67,1.56)-- (2.58,5.83);
\draw[dotted] (2.58,5.83)-- (3.71,3.73);
\draw[dotted] (3.71,3.73)-- (6.06,5.61);
\draw[dotted] (6.06,5.61)-- (6.19,3.29);
\draw[dotted] (6.19,3.29)-- (4.98,3.82);
\draw[dotted] (4.98,3.82)-- (4.76,-0.34);
\draw[dotted] (4.76,-0.34)-- (4.03,2.47);
\draw[dotted] (4.03,2.47)-- (2.73,3.12);
\draw[dotted] (2.73,3.12)-- (2.83,1.9);
\draw[dotted] (2.83,1.9)-- (2.88,0.17);






\draw [color=qqqqff] (2.73,3.12)-- (1.9,2.66);
\draw [color=black] (1.9,2.66)-- (2.01,3.18);
\draw [color=qqqqff] (2.01,3.18)-- (2.73,3.12);
\draw [color=qqqqff] (4,3)-- (2.01,3.18);
\draw [color=black] (2.01,3.18)-- (2.27,4.4);
\draw [color=qqqqff] (2.27,4.4)-- (4,3);
\draw [color=qqqqff] (4,3)-- (4.03,2.47);
\draw [color=black] (4.03,2.47)-- (2.73,3.12);
\draw [color=qqqqff] (2.73,3.12)-- (4,3);
\draw [color=qqqqff] (4,3)-- (3.71,3.73);

\draw [color=qqqqff] (3.03,5)-- (4,3);
\draw [color=qqqqff] (4,3)-- (4.46,0.83);
\draw [color=black] (4.46,0.83)-- (4.03,2.47);
\draw [color=qqqqff] (4.03,2.47)-- (4,3);
\draw [color=qqqqff] (4,3)-- (4.83,0.94);
\draw [color=black] (4.83,0.94)-- (4.98,3.82);
\draw [color=qqqqff] (4.98,3.82)-- (4,3);
\draw [color=qqqqff] (4,3)-- (5.16,4.89);
\draw [color=black] (5.16,4.89)-- (3.71,3.73);
\draw [color=qqqqff] (3.71,3.73)-- (4,3);
\draw [color=black] (4.98,3.82) -- (5.84,3.44);
\draw [color=black](2.73,3.12)  -- (2.81,2.17) ;
\draw [color=black] (3.71,3.73)-- (3.03,5);



\begin{scriptsize}
\fill [color=black] (2.88,0.17) circle (1.1pt);
\fill [color=black] (1.67,1.56) circle (1.1pt);
\fill [color=black] (2.58,5.83) circle (1.1pt);
\fill [color=black] (3.71,3.73) circle (1.1pt);
\fill [color=black] (6.06,5.61) circle (1.1pt);
\fill [color=black] (6.19,3.29) circle (1.1pt);
\fill [color=black] (4.98,3.82) circle (1.1pt);
\fill [color=black] (4.76,-0.34) circle (1.1pt);
\fill [color=black] (4.03,2.47) circle (1.1pt);
\fill [color=black] (2.73,3.12) circle (1.1pt);
\fill [color=black] (4,3) circle (1.5pt);
\draw[color=black] (4.08,3.3) node {};
\end{scriptsize}
\end{tikzpicture}
\caption{A polygon (dotted) containing a geodesic disk centered at , whose interior is depicted in gray and its boundary is drawn in black.}
\label{geoDisk}
\end{figure}


A collection of geodesic disks \emph{covers} the polygon boundary , if each point of  is contained in at least one disk. In this paper we present an  time -approximation algorithm which finds a collection of geodesic unit disks covering the boundary of a simple polygon on  vertices, with  denoting the number of disks found by the algorithm. The algorithm then returns the centers of the disks. {We consider the setting where the centers can be placed anywhere inside the polygon, but the algorithm can be easily modified to restrict the centers to lie on .}  Furthermore, the \emph{number} of disks can be computed in time .

  
While it follows from Theorem 7 of \cite{viganPack} that our problem is -hard in polygons with holes, its complexity remains open in simple polygons.

The main motivation for studying this problem comes from sensor networks, where {\emph{Barrier Coverage} problems have been studied extensively} (see for example \cite{conf/algosensors/BeregK09},\cite{5210107},\cite{Chen:2008:MGQ:1374618.1374674},\cite{Kumar:20059},\cite{Liu:2008:SBC:1374618.1374673},\cite{conf/infocom/SaipullaWLW09},\cite{4520201}). In a Barrier Coverage problem the goal is to place few sensors or guards to detect any intruder into a given region. {The algorithm in this paper can be applied to this context}: given a region, bounded by a piecewise linear closed border, such as a fence, place few guards inside the fenced region, such that wherever an intruder cuts through the fence, the closest guard is at most distance one away. Another way of looking at this problem is from an Art Gallery perspective (see for example \cite{ORourke:1987:AGT:40599}), where the polygon represents a gallery and, regardless where on the wall a painting is hanged, the closest guard is at most a distance one away.


\subsection{Related Work}
Several papers (\cite{Funke:2007},\cite{Huang:2003},\cite{1939},\cite{ijdsn/Ko12},\cite{936985},\cite{6226360}) study full coverage of geometric regions with Euclidean disks. For an overview of optimal coverings of squares and triangles with disks see Chapter 1.7  of \cite{unsolv2}. 

In the context of Barrier Coverage, \cite{Cabello:2013:CSP:2462356.2462383} presents a polynomial time algorithm which for two points in the plane and a set of Euclidean disks selects a minimal subset of the disks which separates the two points. Extending the problem to  points, an -approximation algorithm was presented in \cite{Gibson:2011:IPU:2040572.2040580} and -hardness was shown in \cite{DBLP:journals/corr/abs-1303-2779}. The same two point separation problem was studied in \cite{minCellCon} when segments instead of disks are given.

{Covering a simple polygon with a single geodesic disk of minimum radius has been studied in \cite{pollackGeod} and a linear-time algorithm is presented in \cite{heeKapLinear}. {An output sensitive algorithm for computing geodesic disks for a given set of centers and a fixed radius is presented in \cite{geodPac}}. 

\subsection{Paper Organization}
This paper is organized as follows. In Section \ref{algoSec} we present the algorithm and show that it runs in time . In Section \ref{approx} we  {prove that the number of centers placed by the algorithm is at most twice the minimum number of centers needed to cover the polygon boundary}.  In Section \ref{refAna} we show that a simple linear time algorithm achieves an asymptotically optimal approximation ratio when the polygon perimeter is much larger than . All the missing proofs can be found in the Appendix.
\section{The Algorithm and Its Running Time}
\label{algoSec}

Our algorithm makes use of several properties of geodesic Voronoi diagrams which we review below.


\subsection{Geodesic Voronoi diagrams}
\label{vorDia}

A \emph{furthest-site geodesic Voronoi diagram} of  sites in a simple polygon  on  vertices is a decomposition of  into cells such that all points in a cell have the same site \emph{furthest} away from them {(in the geodesic metric)}. As shown in \cite{BorisFurthest}, it has combinatorial complexity  and can be constructed in time . In Section 2.8 of \cite{BorisFurthest} it is shown that these combinatorial and time complexities are with respect to a \emph{refinement} (also called a shortest path partition) of the Voronoi edges. For all points on a refined edge it holds that their shortest paths to each of the two furthest sites are combinatorially equivalent, i.e., they consist of the same sequence of polygon vertices respectively. Furthermore, Section 3.3 of \cite{BorisFurthest} defines for each of the  refined edges, and for each of the two furthest sites  and  defining a Voronoi edge , the \emph{anchor points} ,  which are the last points on the shortest path from ,  respectively to any point on . Those anchors can be computed in total  time and each time we compute a   furthest-site geodesic Voronoi diagram we store the anchors as well as the distance to its site at the refined Voronoi edges. An additional property of this Voronoi diagram is that its edges form a tree, rooted at the \emph{geodesic center} of the  sites, which is defined as the point that minimizes the maximum distance to any of the sites (see Corollary 2.9.3 of \cite{BorisFurthest}). Therefore, the geodesic center of the sites can be obtained within the same time bound.

The second data structure we use is the \emph{closest-site geodesic Voronoi diagram}  { which, for  sites in a simple polygon  on  vertices,}
is a decomposition of  into cells such that all points in a cell have the same site \emph{closest} to them {(in the geodesic metric)}. It has combinatorial complexity  and it can be constructed in time  (see \cite{borisClose}). 

\subsection{The \textsc{ContiguousGreedy} Algorithm}
In this section we describe a greedy -approximation algorithm which finds a collection of geodesic unit disks which cover the boundary of  and returns the set of disk centers. It starts at vertex  of  the vertices  of  and {iteratively extends a contiguous cover  of   (in clockwise order) by the maximum amount that can be covered with a single geodesic disk.} We denote the clockwise endpoint of  by , thus initially  and . \\


We cover segment portions longer than  in time linear in the minimum number of disks needed to cover them. With  denoting the first uncovered vertex in the current iteration, we}  partially cover  by adding  {centers sequentially on }.  By this, we assure that none of  {those disks} contains  and, since each disk contains a boundary portion of length , the disks placed are indeed optimal with respect to the greedy contiguous extension criterion.






\begin{definition}
For a polygonal chain , we denote by  the sum of the lengths of its line segments and we refer to the number of vertices of  by .
\end{definition}

\begin{definition}
For two points , we denote the portion of  in clockwise orientation between  and  by .
\end{definition}



If  does not lie on a long segment, we compute the next endpoint  which extends  in clockwise order by a maximum length boundary portion which can be covered by a single geodesic unit disk. We do this   by finding the first vertex  (in clockwise order) such that  cannot be contained in a single geodesic unit disk. This test is done by calling the \textsc{TestCover} procedure discussed below, which, for a boundary point  and a vertex  tests whether  can be covered with a single geodesic unit disk.  
 {If  is the first vertex in clockwise order after , we find  by first using exponential search with the \textsc{TestCover} predicate with   fixed and  set to  respectively in consecutive steps until \textsc{TestCover} returns  or .} 
This defines an {index}-interval containing {the index}  which can then be found using a simple binary search. 

After finding  and thereby fully determining the sequence of vertices covered in the current iteration, we use the \textsc{AugmentShort} procedure -- discussed below -- to compute the new endpoint  of  as well as the center of the next disk. \\

\vspace{-10px}

{\center
\fbox{\begin{minipage}{\textwidth}
\noindent {\bf \textsc{ContiguousGreedy}}\\
\noindent \\
\noindent \\
\noindent{\bf while}  not covered: 
\begin{itemize}[leftmargin=.5in]
\vspace{-5px}
\item  [1.]  If  is longer than 2 \\
\hspace*{10px}  compute centers on  at steps of ; add them to ; update 
\item [2.] Update  to the first vertex s.t.  cannot be covered by a single disk,
		 using Exponential and Binary Search with predicate \textsc{TestCover} 
\item [3.]  Use \textsc{AugmentShort} to cover the vertices between  and , and a maximal portion of the edge ; add new center to  and update 
\end{itemize}
\vspace{-5px}
\noindent{\bf end while}\\
\noindent{\bf return} 
\end{minipage}}
}


\begin{definition}
For two points  in a simple polygon , we denote the shortest path in  between  and  by . We denote the number of its vertices by .
\end{definition}

\begin{definition}[\cite{Toussaint89computinggeodesic}]
A set  inside a simple polygon  is called \emph{geodesic convex}, if for any two points , the shortest path  is contained in . \end{definition}






\paragraph{\textbf{\textsc{TestCover}.}}
This procedure tests for a boundary point  and a polygon vertex  whether  can be covered with a single geodesic unit disk. Observe that if a geodesic unit disk {can cover} a set of points, then a geodesic unit disk centered at the \emph{geodesic center} of those points obviously  also covers them. Let  denote the sequence of point  and all polygon vertices up to (and including)  in clockwise order. \textsc{TestCover} computes the geodesic center of  and returns true iff it has distance at most one to all points in . \\


\noindent {\em Implementation details.} We compute the geodesic center of  in a smaller polygon  containing . We let  be , with  denoting the concatenation of two polygonal chains sharing two endpoints. Note that  may have touching sides, but it is not self-intersecting. Such polygons are referred to as \emph{weakly simple} polygons (\cite{Dumitrescu2009112}) and the geodesic distance within them is well defined. Since  is the concatenation of a boundary part of  and a shortest path in  it follows that  is geodesic convex in , thus implying that the geodesic center of  in  is the same point as the geodesic center of  in . We find this geodesic center point by computing the furthest-site geodesic Voronoi diagram  of the sites  in , traversing the (oriented) Voronoi edges to the root and thereby obtain the geodesic center of  (see Section \ref{vorDia}). Then, for each site in  we test whether the distance to the geodesic center is at most one. 

\noindent {\em Computational complexity.}  Computing  takes time  after  global pre-processing time, using the algorithm of \cite{Guibas1989126}; concatenating two polygonal chains to construct  takes constant time. Computing  takes  time and the geodesic center can be obtained from  in the same time bound. Computing the distance from the geodesic center to all sites in  can be done in time  (see \cite{lineshortpath}), by building the shortest path tree rooted at the geodesic center. 
Therefore, the procedure has an overall time complexity of . \\


Knowing the first vertex  such that  cannot be covered with a single geodesic unit disk, we compute the center of the next disk and compute the new endpoint  of  the following \textsc{AugmentShort} procedure.

\paragraph{\textbf{\textsc{AugmentShort}.}}
For the new endpoint  of  it needs to hold that  can be covered with one geodesic unit disk, and for any , with  it is not possible to cover  with a single geodesic unit disk. 
Let  denote the clockwise sequence of point  and all vertices up to (and including) . We construct  and denote by  the intersection of the geodesic unit disks centered at the points  in . This intersection is non-empty by construction and the center of the next disk lies in . We denote by  the set of all disk-disk intersection points on  as shown in Fig. \ref{algoPic}. { Lemma~\ref{Aint} below justifies the steps taken to find }.

\begin{figure}
\center
\definecolor{uququq}{rgb}{0.25,0.25,0.25}
\definecolor{sqsqsq}{rgb}{0.13,0.13,0.13}
\definecolor{darkgrey}{rgb}{0.5,0.5,0.5}
\begin{tikzpicture}[line cap=round,line join=round,>=triangle 45,x=1.0cm,y=1.0cm]



\draw [shift={(3,1)},dotted,color=sqsqsq,fill=sqsqsq,fill opacity=0.1]  (0,0) --  
plot[domain=0:0.95,variable=\t]({1*2*cos(\t r)+0*2*sin(\t r)},{0*2*cos(\t r)+1*2*sin(\t r)}) --
plot[domain=2.02:3.14,variable=\t]({1*2*cos(\t r)+0*2*sin(\t r)},{0*2*cos(\t r)+1*2*sin(\t r)}) -- cycle ;

\draw [dotted,color=sqsqsq,fill=sqsqsq,fill opacity=0.1]  (0.4,1) --  
plot[domain=0:1.27,variable=\t]({0.4 + 1*2*cos(\t r)+0*2*sin(\t r)},{1 + 0*2*cos(\t r)+1*2*sin(\t r)}) -- (0, 3) -- cycle ;

\draw [shift={(0,3)},dotted,color=uququq,fill=uququq,fill opacity=0.1]  (0,0) --  
plot[domain=-1.37:-0.09,variable=\t]({1*2*cos(\t r)+0*2*sin(\t r)},{0*2*cos(\t r)+1*2*sin(\t r)}) -- cycle ;

\draw [solid,color=black, fill=darkgrey]  
plot[domain=2.28:3.0,variable=\t]({3 + 1*2*cos(\t r)+0*2*sin(\t r)},{ 1+ 0*2*cos(\t r)+1*2*sin(\t r)}) -- 
plot[domain= -1.0:-0.34,variable=\t]({1*2*cos(\t r)+0*2*sin(\t r)},{ 3+ 0*2*cos(\t r)+1*2*sin(\t r)}) -- 
plot[domain= 0.8:0.86,variable=\t]({0.4 + 1*2*cos(\t r)+0*2*sin(\t r)},{ 1+ 0*2*cos(\t r)+1*2*sin(\t r)});

\draw [fill=white, draw=none] (5.5,2.5) -- (3.2, 1.9)-- (5.55, 1.7) -- cycle;
\draw (5.5, 1)-- (3,1)-- (0.4,1)-- (0,3)-- (5.5,2.5) -- (3.2, 1.9)-- (5.55, 1.7);





\begin{scriptsize}
\draw [fill=black] (3,1) circle (1.4pt);
\draw[color=black] (3.14,.75) node {};
\draw [fill=black] (0.4,1) circle (1.4pt);
\draw[color=black] (0.54,.75) node {};
\draw [fill=black] (0,2.98) circle (1.4pt);
\draw[color=black] (0.14,3.22) node {};
\draw [fill=black] (5.5,2.5) circle (1.4pt);
\draw[color=black] (5.66,2.8) node {};
\draw[color=black] (1.5,2) node {};

\draw[fill=black] (1.7,2.5) circle (1.4pt);
\draw[color=black] (1.4,2.5) node {};
\draw[fill=black] (1.89,2.34) circle (1.4pt);
\draw[color=black] (2.2,2.3) node {};
\draw[fill=black] (1.02,1.26) circle (1.4pt);
\draw[color=black] (.8,1.26) node {};




\end{scriptsize}
\end{tikzpicture}
\caption{Illustration of , i.e., the intersection of the geodesic unit disks centered at points in  as well as the disk-disk intersection points .}
\label{algoPic}
\end{figure}

\begin{lemma}
Given a  simple polygon , let  be the non-empty intersection of a collection of geodesic unit disks in  and let   be a line-segment in , such that for all ,  and .   For any point  and any disk center  furthest away from ,  if and only if either:
\begin{enumerate}
\item[a)]  and , or
\item[b)]  and , 
\end{enumerate}
\label{Aint}
with  denoting the disk-disk intersection points on  and , for a point set  in . 
\end{lemma}

\begin{proof}(Lemma~\ref{Aint})
Let  be the point in  closest to the point  and by  a center farthest from . Notice that since  is geodesic convex,  is unique {and it lies on }.  We prove Lemma \ref{Aint} with the help of the following two observations.

\begin{obs}
If  for some center  and  then .
\label{PinI}
\end{obs}
\begin{proof}
Let  and assume that , thus  is in the interior of all the disks defining , other than . Furthermore,  since  is geodesic convex,  contains a point , with .  Since  and  is on the shortest path from  to , by uniqueness of the shortest path, .  Then, by Lemma~\ref{pollackCon}, , contradicting  that  is the point in  closest to .  \qed
\end{proof}

\begin{obs}
If  then . 
\label{PinA}
\end{obs}
\begin{proof}
Since  there is a unique center  such that .    As shown in Observation \ref{PinI}, if  and  then . Therefore  is contained in  and we denote this point by .

Observe that  is contained in  and is at distance  from , thus .  Clearly,   and since  is the closest point in  to , it follows that .  Now observe that, since  and the distance between  and any other center is less than  (because ),  is the farthest center, i.e., .  Therefore   holds as claimed.  \qed
\end{proof}

We now prove Lemma \ref{Aint} :

"".  We distinguish two cases based on whether  or  .  If , then  as shown in Observation \ref{PinA}. Therefore,  and , and thus condition a) holds.  If  and  condition b) holds.  Otherwise if  and   , let . We can write the distance  as . Since  lies on ,  and since  this implies that . Therefore, distance . 

By the triangle inequality it also holds that . Since  is the closest point in  to ,  by hypothesis. Since  lies on , . Therefore,   and combining this with  from above,  and again condition a) holds.

\noindent ""
\noindent a) Let .  Then .  If , by definition . Since ,  and by the triangle inequality,  which contradicts .

\noindent b) Since  and , obviously .  For  the closest point to  in , assume that .  Since , .  Therefore, by Observation \ref{PinA}, , and thus this intersection is in  contradicting the hypothesis.
\qed
\end{proof}






We use the following steps to determine  on .\\
Step 1)  Find the point  on  closest to , whose distance to its furthest point  in  is exactly  and , if such a point  exists. \\
Step 2) Find the point  on  closest to , whose distance to its closest point in  is exactly .  \\
Step 3) Set  if  does not exist or . In this case we add the point in  closest to  as the new disk center to the set  of centers. Otherwise  and the point  is the new disk center which gets added to .\\ 



Note that since  will be covered in this iteration and  won't be covered, . By continuity of the geodesic distance, there is a point  on , with  and thus by Lemma~\ref{Aint} either  or  exists.\\

In Step 1, to find  if it exists, we construct the (refined) furthest-site geodesic Voronoi diagram of the sites  in  and traverse the Voronoi vertices  on , ordered in the direction from   to  and set . For each such vertex we check in  time whether the distance to (one of) its furthest site(s) is at most , using an  time shortest path query (\cite{Guibas1989126}) after pre-processing  in  time. Once we find the first  with distance at most , if it exists, this determines a sub-segment  on  containing a point  at distance exactly  from its furthest site . Note that since the shortest paths to the furthest site  have the same combinatorial structure for all points on the refined Voronoi edge , we find the point at distance  to  in constant time since we stored the anchor point  at the edge  (see Section 2.1). We check if , by computing  in time  using \cite{lineshortpath} and finding in  time the arc  of  separating  from . We traverse the edges of  and for each edge we test in  time if it intersects . Denoting the intersection point by , we check if , by computing the shortest path tree to the sites in  and test if the distance to all sites is at most  in time . If this intersection is in , we set  to .\\

\vspace{-9pt}
\begin{claim}There can be at most two points on  that have distance exactly  from their respective furthest site; if there are two such points, one of them must be .  
\label{claim}
\end{claim}
\vspace{-3pt}

We prove this claim using the following lemma.
\begin{lemma}[Lemma 1 \cite{pollackGeod}; see also Lemma 2.2.1 \cite{BorisFurthest}]
Given three points  in a simple polygon, for , the distance  is a convex function on , with  .
\label{pollackCon}
\end{lemma}





\begin{proof} [of Claim]
Assume that there are two points  and  on  that are at distance  from their respective furthest sites, with  closer to  than , thus .  Let  be a center furthest away from . Clearly  since both  and  are at distance at most  from any point in . Since  and , by Lemma~\ref{pollackCon} , contradicting the assumption that . \qed
\end{proof}


According to the above claim, the only other candidate for  is . 
Thus, if  we check in  time if the point  is at distance exactly  from its furthest site and if so, we set  to . If  exists  can be covered with one geodesic unit disk, because the point  has distance exactly  to  and lies in .   \\

In Step 2, to find , we first construct the set  of the disk-disk intersection points of ; we do this without explicitly computing . To construct , we look at the furthest-site geodesic Voronoi diagram of the sites  in  constructed in the Step 1. Since any point in  has two points in  at distance , every point in  lies on a Voronoi edge. For every site  we look at the refined edges of  and for such edge  we access its anchor point  as well as the distance from  to the endpoints of , in constant time. We test if there is a point on  having distance  to , again in  time. If such a point exists then this is a disk-disk intersection point and we add it to . Since we need constant time for each refined Voronoi edge,  can be computed in total time .

Having computed , we construct the closest-site geodesic Voronoi diagram of the sites    in . We traverse the Voronoi vertices  on , ordered in the direction from   to  and set . For each such vertex we check whether the distance to (one of) its closest site(s) is at most  again by an  time shortest path distance query. Once we find the first such vertex  on , if it exists, we have determined a sub-segment  on  where  lies. Letting  be the corresponding closest site, by Lemma  \ref{pollackCon}, we find the point in   at distance  from  by computing the intersection point of a geodesic unit disk centered at  with , in time , using the funnel algorithm of \cite{lineshortpath}.

There can be at most two points on  that have distance exactly  from ; if there are two such points, one of them must be .  This can be seen directly from the fact that , and Lemma~\ref{pollackCon}.  We set  to the one closer to . It is easy to see that  is feasible, i.e.,  can be covered with one geodesic unit disk, because  and .\\

In Step 3, , if either  does not exist or , thus   indeed extends  maximally because  is the point on  closest to  having distance exactly  to the closest point in , i.e., to the center of the geodesic unit disk placed in this iteration. Otherwise  and  is the point on  closest to  having distance exactly  to the furthest center in ; any point on  closer to  has distance larger than  from that center and is thus infeasible.\\


\noindent {\em Computational Complexity / Summary.} 
Constructing  takes time  as argued in the \textsc{TestCover} paragraph before.
Step 1 needs  time to construct the geodesic furthest-site Voronoi diagram of  in  and   time to find a sub-segment of the edge  possibly containing , since there are only  Voronoi vertices in total and we spend  on them for finding the sub-segment. The last step is to test if , which takes time  as argued above.

In Step 2, we spend  time to construct the set  and  time to construct the geodesic closest-site Voronoi diagram of the sites . We then traverse edge  in  time to find a sub-segment of the edge  possibly containing , and determine  on this sub-segment in  time.

Thus the overall time spent in \textsc{AugmentShort}  is .

\subsubsection{Total Running Time.}

Let  be the set of all polygons constructed throughout the whole execution of \textsc{ContiguousGreedy}. In each polygon  we spend  time in \textsc{TestCover} and possibly  time in \textsc{AugmentShort} as argued above. Since in each iteration of \textsc{ContiguousGreedy},  is extended to cover at least one new polygon vertex, there are at most  iterations of the main  loop. Furthermore,  covering long segments of  takes total time . Since according to Lemma \ref{log_poly}, , the running time of \textsc{ContiguousGreedy} is . 
\vspace{5pt}
\begin{lemma}
.
\label{log_poly}
\end{lemma}

\begin{proof}
Each polygon of  constructed in the \textsc{ContiguousCover} algorithm has the form , with  an arbitrary point on  and  a vertex of . We call  the -portion and  the -portion of the polygon .  Notice that  every polygon constructed in \textsc{AugmentShort} was also constructed in a \textsc{TestCover} call and thus it suffices to bound the number of polygons constructed in all \textsc{TestCover} calls.

Observe that , since in each iteration,  is extended to cover at least one new vertex, thus there are at most  iteration, and in each iteration we construct  polygons during Exponential and Binary Search.
Observe that if every vertex of  is contained in  polygons of  then . This holds because for each  there is at most one vertex of  which is not a vertex in , namely the point . \\ Since  is covered both in the first and last iteration of the algorithm, we are pessimistically bounding the number of polygons containing  by . To then prove the lemma it is enough to show that every vertex of  except  is contained in  polygons of . For that we fix a vertex , with , and show that  appears in the -portion of  polygons and  appears in the -portion of  polygons of .  

To bound the number of appearances of  on the -portion of a polygon we fix the unique iteration  in which   is first covered.  Since \textsc{TestCover} is used as a predicate in Exponential and Binary search, in iteration  it is called  times and thus  appears in  polygons during this iteration. Observe, that in subsequent iterations, when , vertex  is not part of the -portion of any constructed polygon. For an iteration , let  be the first uncoverable vertex (denoted by  in the algorithm) found in iteration , thus  ; let  be the number of polygons in which  appears on the -portion during this iteration . Also observe that  is the index of the first  vertex of  covered in iteration .
We claim that 

and defining , implies that .

For , inequality (\ref{loginterval}) holds trivially. Otherwise, since  is not covered during this iteration, Exponential Search stops after the first time  appears on the -portion of a constructed polygon. This leaves a search interval of size at most . During Binary Search, there are exactly  search intervals which contain both  and . Since the interval size is halved at each step and all search intervals containing both  and  have size at least ,  inequality (\ref{loginterval}) follows.

So far we have shown that  appears on the -portion of  polygons in  before iteration ,  times during iteration  and does not appear in subsequent iterations.  Therefore, all together,  appears on  -portions of  polygons in .

To bound the number of appearances of  on the -portion of a polygon, let  be the set of polygons containing  on their -portion but not on the -portion.  By Observation~\ref{obsPath} below, any two polygons in  intersect on their -portion because they both contain  on their -portion. Since by construction the -portion of each polygon  ends with a vertex, any two polygons in  have a vertex in common on their -portion.  This is true because the -portion of those polygons are subsequences of  and it is easy to see that there is a vertex  that belongs to the -portion of  all .  Since  appears on  -portions of  polygons, .
\qed
\end{proof}

\begin{obs}
For  four distinct points on , if  contains a polygon vertex not contained in , then . 
\label{obsPath}
\end{obs}



\begin{figure}

\definecolor{cqcqcq}{rgb}{0.75,0.75,0.75}
\center
\begin{tikzpicture}[line cap=round,line join=round,>=triangle 45,x=1.5cm,y=1.5cm]
\clip(0,0) rectangle (8,2.4);
\draw (0.4,0.46)-- (1.44,1.84)-- (1.96,1.13)-- (2.64,1.89)-- (3.29,1.47)-- (3.86,1.74)-- (3.36,1.94)-- (4.92,2.05)-- (5.47,1.41)-- (6.65,0.95);
\draw (0.33,0.25)-- (1.41,0.86)-- (2.41,0.42)-- (2.53,0.7)-- (3.57,1.2)-- (3.26,0.15)-- (4.03,0.25)-- (4.17,0.64)-- (3.83,0.53)-- (3.68,0.73)-- (4.65,1.21)-- (4.97,0.7)-- (6.08,0.67)-- (7.03,0.73);
\draw [dash pattern=on 2pt off 2pt] (0.4,0.46)-- (1.41,0.86)-- (3.57,1.2)-- (4.65,1.21)-- (6.65,0.95);
\fill[color=gray,fill=gray,fill opacity=0.1] (3.57,1.2) -- (4.65,1.21) -- (3.68,0.73) -- (3.83,0.53) -- (4.17,0.64) -- (4.03,0.25) -- (3.26,0.15) -- cycle;


\fill[pattern color=cqcqcq,fill=cqcqcq,pattern=north east lines] (4.65,1.21) -- (4.96,0.67) -- (6.08,0.67) -- (7.05,0.73) -- (6.67,0.97) -- cycle;




\begin{scriptsize}


\draw[color=black] (4.21,2.14) node {};
\draw[color=black] (2.44,1.12) node {};


\fill [color=black] (0.4,0.46) circle (1.5pt);
\draw[color=black] (0.41,0.63) node {};
\fill [color=black] (1.44,1.84) circle (1.5pt);
\fill [color=black] (1.96,1.13) circle (1.5pt);
\fill [color=black] (2.64,1.89) circle (1.5pt);
\fill [color=black] (3.29,1.47) circle (1.5pt);
\fill [color=black] (3.86,1.74) circle (1.5pt);
\fill [color=black] (3.36,1.94) circle (1.5pt);
\fill [color=black] (4.92,2.05) circle (1.5pt);
\fill [color=black] (5.47,1.41) circle (1.5pt);
\fill [color=black] (6.65,0.95) circle (1.5pt);
\draw[color=black] (6.73,1.13) node {};
\fill [color=black] (0.33,0.25) circle (1.5pt);
\fill [color=black] (1.41,0.86) circle (1.5pt);
\fill [color=black] (2.41,0.42) circle (1.5pt);
\fill [color=black] (2.53,0.7) circle (1.5pt);
\fill [color=black] (3.57,1.2) circle (1.5pt);
\draw[color=black] (3.67,1.27) node {};
\fill [color=black] (3.26,0.15) circle (1.5pt);
\fill [color=black] (4.03,0.25) circle (1.5pt);
\fill [color=black] (4.17,0.64) circle (1.5pt);
\draw[color=black] (4.28,0.71) node {};
\fill [color=black] (3.83,0.53) circle (1.5pt);
\fill [color=black] (3.68,0.73) circle (1.5pt);
\fill [color=black] (4.65,1.21) circle (1.5pt);
\draw[color=black] (4.76,1.29) node {};
\fill [color=black] (4.97,0.7) circle (1.5pt);
\fill [color=black] (6.08,0.67) circle (1.5pt);
\draw[color=black] (6.2,0.74) node {};
\fill [color=black] (7.03,0.73) circle (1.5pt);
\draw[color=black] (5.36,0.93) node {};
\draw[color=black] (5.36,0.53) node {};
\draw[color=black] (3.86,0.99) node {};
\end{scriptsize}
\end{tikzpicture}
\caption{Illustration of the proof of Observation~\ref{obsPath}.}
\label{pathPic}
\end{figure}

\begin{proof}

Let  be a vertex contained in   not in  and assume for contradiction that  and  are disjoint. 
Then either  or . W.l.o.g.  assume  . For  the successor vertex of  in , let  be the simple polygon bounded by . If both  are contained in , meaning , since  is a convex vertex in , it holds that , a contradiction. Otherwise, let  be the geodesic convex set bounded by . If both  are contained in , then by geodesic convexity,  and thus . Otherwise  and  as shown in Fig.~\ref{pathPic}. Since in that case  and  is again a geodesic convex set, . Again, since  is a convex vertex in , , and thus  not in , a contradiction.
\qed


\end{proof}







\section{Approximation Ratio}
\label{approx}

Let  denote a set of geodesic unit disks optimally covering . In order to prove the -approximation we prove the existence of a {\em coloring} for   using  distinct colors and introducing at most  monochromatic boundary portions. We then show that  \textsc{ContiguousGreedy} uses at most one disk per monochromatic boundary portion (plus possibly one additional disk for the unique monochromatic boundary portion containing ), which implies the -approximation factor of  \textsc{ContiguousGreedy}. 

{A {\em coloring} of  is a function .  The number of colors used by  is defined as the cardinality of the image of }. A {\em block} is a connected component of  colored with a single color.  We let  denote the subset of the polygon boundary colored with color  and we call each connected component of  a \emph{pocket} of  induced by {color}  (see Fig.~\ref{colorExFig}(b)). 


A coloring of   is called {\em crossing-free} if for any two distinct colors {, it holds that } is contained in a single pocket induced by color .


For a collection  of disks covering , a \emph{disk-coloring} of  w.r.t.  is a function , such that , i.e., a point on   can only be colored with one of the indices of the disks covering it (see Fig.~\ref{colorExFig}(a)).


\definecolor{ffqqqq}{rgb}{1.0,0.0,0.0}
\definecolor{qqwuqq}{rgb}{0.0,0.39215686274509803,0.0}
\definecolor{qqqqff}{rgb}{0.0,0.0,1.0}
\definecolor{cqcqcq}{rgb}{0.7529411764705882,0.7529411764705882,0.7529411764705882}
\definecolor{qqwuqq}{rgb}{0.0,0.39215686274509803,0.0}
\definecolor{ffqqqq}{rgb}{1.0,0.0,0.0}
\definecolor{qqqqff}{rgb}{0.0,0.0,1.0}
\definecolor{orange}{rgb}{1.0,0.5,0.0}
\definecolor{brown}{rgb}{0.5,.5,0.0}
\definecolor{yqyqyq}{rgb}{0.5019607843137255,0.5019607843137255,0.5019607843137255}
\definecolor{cqcqcq}{rgb}{0.7529411764705882,0.7529411764705882,0.7529411764705882}

\vspace{-15pt}

\begin{figure}[]
\center
\begin{tabular}{cc}

\subfloat[] {

\begin{tikzpicture}[line cap=round,line join=round,>=triangle 45,x=1.0cm,y=1.0cm,scale=1.3]
\clip(-0.15,0.2) rectangle (3.0,2);
\draw [dotted,color=qqqqff,fill=qqqqff,fill opacity=0.1] (1.6407502437867678,0.7262099843520922) circle (0.5cm);
\draw [color=qqwuqq,fill=qqwuqq,fill opacity=0.1] (2.3372699835759234,0.7288531726287284) circle (0.5cm);
\draw [dash pattern=on 1pt off 1pt on 1pt off 4pt,color=orange,fill=orange,fill opacity=0.1] (1.6481657819908717,1.4707564648272209) circle (0.5cm);
\draw [color=brown,fill=brown,fill opacity=0.1] (1.1442305039976122,0.7262099843520922) circle (0.5cm);
\draw [dash pattern=on 1pt off 1pt on 1pt off 4pt,color=ffqqqq,fill=ffqqqq,fill opacity=0.1] (0.42379009724472,0.7943597525584467) circle (0.5cm);
\draw [line width=1.2000000000000002pt,dash pattern=on 1pt off 1pt on 1pt off 4pt,color=ffqqqq] (0.5,0.5)-- (0.6967952937271886,0.5032210625048962);
\draw [line width=1.2000000000000002pt,dash pattern=on 1pt off 1pt on 1pt off 4pt,color=ffqqqq] (0.5,0.5)-- (0.0,1.0);
\draw [line width=1.2000000000000002pt,dash pattern=on 1pt off 1pt on 1pt off 4pt,color=ffqqqq] (0.0,1.0)-- (0.7246119238309283,1.00392040437221);
\draw [line width=1.2000000000000002pt,color=brown] (0.6967952937271886,0.5032210625048962)-- (1.1882224255599232,0.5032210625048962);
\draw [line width=1.2000000000000002pt,dotted,color=qqqqff] (1.1882224255599232,0.5032210625048962)-- (1.8882742831707053,0.5032210625048962);
\draw [line width=1.2000000000000002pt,dash pattern=on 1pt off 1pt on 1pt off 4pt,color=orange] (1.4942053567010596,0.9992842993549201)-- (1.499540803585664,1.2735138372423407);
\draw [line width=1.2000000000000002pt,dash pattern=on 1pt off 1pt on 1pt off 4pt,color=orange] (1.499540803585664,1.2735138372423407)-- (1.994904698568374,1.7735138372423407);
\draw [line width=1.2000000000000002pt,dash pattern=on 1pt off 1pt on 1pt off 4pt,color=orange] (1.994904698568374,1.7735138372423407)-- (1.999540803585664,1.1105508197698788);
\draw [line width=1.2000000000000002pt,color=qqwuqq] (1.999540803585664,1.1105508197698788)-- (2.0,1.0);
\draw [line width=1.2000000000000002pt,color=qqwuqq] (2.0,1.0)-- (2.5,1.0);
\draw [line width=1.2000000000000002pt,color=qqwuqq] (2.5,1.0)-- (2.5,0.5);
\draw [line width=1.2000000000000002pt,color=qqwuqq] (2.5,0.5)-- (1.8882742831707053,0.5032210625048962);
\draw [line width=1.2000000000000002pt,color=brown] (1.4942053567010596,0.9992842993549201)-- (0.7246119238309283,1.00392040437221);
\begin{scriptsize}
\draw[color=qqqqff] (1.7,0.4) node {};
\draw[color=qqwuqq] (2.652873888891698,0.7410756695859958) node {};
\draw[color=orange] (1.4436826803576182,1.8148374627642583) node {};
\draw[color=brown] (1,0.4) node {};
\draw[color=ffqqqq] (0.16677676414562995,0.68303449157636) node {};



\end{scriptsize}

\end{tikzpicture}



} &

\subfloat[]{

\begin{tikzpicture}[line cap=round,line join=round,>=triangle 45,x=1.0cm,y=1.0cm,scale=1.3]
\clip(-0.15,-1.55) rectangle (3.0, 0);

\draw [line width=1.2000000000000002pt,dash pattern=on 1pt off 1pt on 1pt off 4pt,color=ffqqqq] (0.5,0.5 - 1.75 )-- (0.6967952937271886,0.5032210625048962   - 1.75 );
\draw [line width=1.2000000000000002pt,dash pattern=on 1pt off 1pt on 1pt off 4pt,color=ffqqqq] (0.5,0.5 - 1.75 )-- (0.0,1.0 - 1.75 );
\draw [line width=1.2000000000000002pt,dash pattern=on 1pt off 1pt on 1pt off 4pt,color=ffqqqq] (0.0,1.0 - 1.75 )-- (0.7246119238309283,1.00392040437221 - 1.75 );

\draw [line width=1.2000000000000002pt,dotted,color=qqqqff] (1.1882224255599232,0.5032210625048962 - 1.75 )-- (1.8882742831707053,0.5032210625048962 - 1.75 );
\draw [line width=1.2000000000000002pt,dash pattern=on 1pt off 1pt on 1pt off 4pt,color=orange] (1.4942053567010596,0.9992842993549201 - 1.75 )-- (1.499540803585664,1.2735138372423407 - 1.75 );
\draw [line width=1.2000000000000002pt,dash pattern=on 1pt off 1pt on 1pt off 4pt,color=orange] (1.499540803585664,1.2735138372423407 - 1.75 )-- (1.994904698568374,1.7735138372423407 - 1.75 );
\draw [line width=1.2000000000000002pt,dash pattern=on 1pt off 1pt on 1pt off 4pt,color=orange] (1.994904698568374,1.7735138372423407 - 1.75 )-- (1.999540803585664,1.1105508197698788 - 1.75 );
\draw [line width=1.2000000000000002pt,color=qqwuqq] (1.999540803585664,1.1105508197698788 - 1.75 )-- (2.0,1.0 - 1.75 );
\draw [line width=1.2000000000000002pt,color=qqwuqq] (2.0,1.0 - 1.75 )-- (2.5,1.0 - 1.75 );
\draw [line width=1.2000000000000002pt,color=qqwuqq] (2.5,1.0 - 1.75 )-- (2.5,0.5 - 1.75 );
\draw [line width=1.2000000000000002pt,color=qqwuqq] (2.5,0.5 - 1.75 )-- (1.8882742831707053,0.5032210625048962 - 1.75 );


\end{tikzpicture}

}
\end{tabular}
\caption{(a) A crossing disk-free coloring of . (b)  The two pockets induced by color . }
\label{colorExFig}
\end{figure}

\vspace{-25pt}




\begin{definition}
For a coloring , two of its colors  and  \emph{cross} each other, if there are two pockets induced by color  containing blocks of color .
\end{definition}

Observe that if two colors  and  cross each other, there are at least two blocks  of color  and two blocks  of color  such that sequence of blocks  occurs in clockwise order on  as shown in Fig. \ref{2pockets}.

\begin{lemma}
In any disk-coloring, if two colors  and  cross each other, one of the following holds: 1) There exists a pocket induced by color  which contains blocks of color  and all these blocks can be re-colored with color , s.t. the resulting coloring is still a disk-coloring. 2) There exists a pocket induced by color  which contains blocks of color  and all these blocks can  be re-colored with color , s.t. the resulting coloring is still a disk-coloring.
\label{crossLem}
\end{lemma}
\begin{proof}

Suppose this is not possible. Since neither  nor  can be colored with , there are  points  and  which lie outside of disk . If we denote the center of  by , it therefore holds that  and  (see Fig. \ref{2pockets}). Analogously, there are two points  and , s.t.  and  can not be colored with color . This again implies that both points lie outside of disk  centered at  and thus   and .


\begin{figure}
\center
\definecolor{ffqqqq}{rgb}{1,0,0}
\definecolor{cqcqcq}{rgb}{0.75,0.75,0.75}
\begin{tikzpicture}[line cap=round,line join=round,>=triangle 45,x=1.0cm,y=1.0cm, scale = 1]
\clip(-1,0.2) rectangle (3,3.9);
\draw [line width=1.2pt,color=red] (0.22,3.22)-- (0.78,3.7)-- (1.6,3.2);
\draw [line width=1.2pt,color=blue] (-0.2,2.68)-- (-0.7,1.92)-- (-0.24,0.56) ;
\draw [line width=1.2pt,color=red] (0.4,0.56)-- (1.82,0.56);
\draw [line width=1.2pt,color=blue]  (2.58,2.96)-- (2.38,2.22)-- (2.66,0.88);
\draw [dotted,color=black] (1.12,2.3)-- (0.64,3.58);
\draw [dotted,color=black] (1.12,2.3)-- (1.64,0.56);
\draw [dotted,color=black] (0.86,1.9)-- (2.64,0.96);
\draw [dotted,color=black] (0.86,1.9)-- (-0.63,1.66);
\begin{scriptsize}
\draw [fill=black] (1.12,2.3) circle (1.4pt);
\draw[color=black] (1.44,2.4) node {};
\draw [fill=black] (0.86,1.9) circle (1.4pt);
\draw[color=black] (0.64,2.14) node {};
\draw [fill=black] (0.64,3.58) circle (1.4pt);
\draw[color=black] (0.45,3.7) node {};
\draw [fill=black] (1.64,0.56) circle (1.4pt);
\draw[color=black] (1.8,0.3) node {};
\draw [fill=black] (2.64,0.96) circle (1.4pt);
\draw[color=black] (2.74,0.66) node {};
\draw [fill=black] (-0.63,1.66) circle (1.4pt);
\draw[color=black] (-0.75,1.3) node {};
\draw [fill=black] (1.32,1.66) circle (1.4pt);
\draw[color=black] (1.22,1.44) node {p};

\draw[color=black] (0.8,0.35) node {};
\draw[color=black] (2.8,2.) node {};
\draw[color=black] (1.5,3.5) node {};
\draw[color=black] (-0.8,2.3) node {};


\end{scriptsize}
\end{tikzpicture}
\caption{Illustration of the four alternating blocks   and the corresponding points  and , ; the disk centers  and , as well as the intersection point  of   and .}
\label{2pockets}
\end{figure}

\begin{lemma}
For any collection of disks covering , there exists a crossing free disk-coloring of .
\label{crossFreeExistsLem}
\end{lemma}
\begin{proof}
Consider the four paths , ,  and . Due to the alternating arrangement of the four blocks   -- and therefore of , on the polygon boundary, one of the paths from  must intersect with one of the paths from .  Assume w.l.o.g. that  intersects  and let  be an intersection point. Again, w.l.o.g., assume that .  Then, by the triangle inequality  contradicting our assumption that .
\qed
\end{proof}


We are now going to prove Lemma~\ref{crossFreeExistsLem}, which states that for any collection of disks covering , there exists a crossing free disk-coloring of .\\

For a given disk-coloring w.r.t. a collection of disks , we let  be the number of pockets induced by color  which contain blocks of color ; observe that . We refer to  as the \emph{crossing number} of the disk-coloring. By definition it holds that the crossing number of a coloring is zero if and only if the coloring is crossing free. We now let  be a  disk-coloring of  w.r.t. disks , having minimum crossing number (over all disk-colorings w.r.t. ). Assume for contradiction that the crossing number of  is not zero and let  and  be two colors of  which cross each other, i.e., . Then, w.l.o.g., according to Lemma \ref{crossLem}, there exists a pocket   induced by color , in which all blocks of color  can be colored with  and the coloring remains a valid disk-coloring w.r.t. . We refer to the resulting disk-coloring as  and by  to the number of pockets induced by color  of  which contain blocks of color  (again in ). Lastly we denote by  the pocket induced by color  in , fully containing  as shown in Figure \ref{PolyDecomp}.


\begin{figure}[ht!]
\center
\begin{tabular}{cc}


\subfloat[]{


\begin{tikzpicture}[line cap=round,line join=round,>=triangle 45,x=1.0cm,y=1.0cm]
\clip(.3,0.25382286090537726) rectangle (6.8,5);
\draw [dotted, rounded corners=0.2cm] (1.18,1.3)-- (0.4,1.58)-- (1.0,2.0);
\draw [dotted, rounded corners=0.2cm] (1.0,3.0)-- (0.5,3.38)-- (1.72,3.6);
\draw [dotted, rounded corners=0.2cm] (2.74,3.6)-- (3.08,4.3)-- (3.5,3.6);
\draw [dotted, rounded corners=0.2cm] (4.22,3.6)-- (4.5,4.32)-- (4.76,3.6);
\draw [dotted, rounded corners=0.2cm] (5.58,3.52)-- (6.28,3.22)-- (5.44,3.0);
\draw [dotted, rounded corners=0.2cm] (5.44,1.96)-- (6.12,1.56)-- (5.06,1.22);

\draw [dotted, rounded corners=0.2cm] (2.18,1.3)-- (2.5,1);
\draw [dotted, rounded corners=0.2cm] (3.76,1.22)-- (3.4,1);

\draw [line width=1.6pt,color=red] (1.18,1.3)-- (2.18,1.3);
\draw [line width=1.6pt,color=red] (3.76,1.22)-- (5.06,1.22);
\draw [line width=1.6pt,color=red] (1.72,3.6)-- (2.74,3.6);
\draw [line width=1.6pt,color=red] (3.5,3.6)-- (4.22,3.6);
\draw [line width=1.6pt,color=red] (4.76,3.6)-- (5.5,3.6);
\draw [line width=1.6pt,color=blue] (1.0,3.0)-- (1.0,2.0);
\draw [line width=1.6pt,color=blue] (5.44,3.0)-- (5.44,1.96);
\draw [dash pattern=on 3pt off 3pt,color=blue, rounded corners=0.4cm] (1.0,3.0)-- (0.19901894358431302,3.2927164927926)-- (1.5,4.087084651455246)-- (2.3861866834341274,4.05180775242541)-- (3.038809315486088,4.774984182537042)-- (3.7443472960828026,4.087084651455246)-- (4.361693029104928,4.916091778656385)-- (5.3318077524254095,3.981253954365739)-- (6.0,4.0)-- (6.3991309738182,3.0439074739189)-- (5.44,3.0);


\begin{scriptsize}
\draw[color=black] (1.4,2.6) node {};
\draw[color=black] (3.1,3.7) node {};
\draw[color=black] (4.47,3.5) node {};
\draw[color=black] (5.1,2.6) node {};
\draw[color=black] (2.1,4.3) node {};
\end{scriptsize}
\end{tikzpicture}
}


\subfloat[]{

\definecolor{blue}{rgb}{0.0,0.0,1.0}
\definecolor{red}{rgb}{1.0,0.0,0.0}
\begin{tikzpicture}[line cap=round,line join=round,>=triangle 45,x=1.0cm,y=1.0cm]
\clip(0.3,0.25382286090537726) rectangle (7,5);
\draw [dotted, rounded corners=0.2cm] (1.18,1.3)-- (0.4,1.58)-- (1.0,2.0);
\draw [dotted, rounded corners=0.2cm] (1.0,3.0)-- (0.5,3.38)-- (1.72,3.6);
\draw [dotted, rounded corners=0.2cm] (2.74,3.6)-- (3.08,4.3)-- (3.5,3.6);
\draw [dotted, rounded corners=0.2cm] (4.22,3.6)-- (4.5,4.32)-- (4.76,3.6);
\draw [dotted, rounded corners=0.2cm] (5.58,3.52)-- (6.28,3.22)-- (5.44,3.0);
\draw [dotted, rounded corners=0.2cm] (5.44,1.96)-- (6.12,1.56)-- (5.06,1.22);

\draw [dotted, rounded corners=0.2cm] (2.18,1.3)-- (2.5,1);
\draw [dotted, rounded corners=0.2cm] (3.76,1.22)-- (3.4,1);

\draw [line width=1.6pt,color=red] (1.18,1.3)-- (2.18,1.3);
\draw [line width=1.6pt,color=red] (3.76,1.22)-- (5.06,1.22);
\draw [line width=1.6pt,color=blue] (1.72,3.6)-- (2.74,3.6);
\draw [line width=1.6pt,color=blue] (3.5,3.6)-- (4.22,3.6);
\draw [line width=1.6pt,color=blue] (4.76,3.6)-- (5.5,3.6);
\draw [line width=1.6pt,color=blue] (1.0,3.0)-- (1.0,2.0);
\draw [line width=1.6pt,color=blue] (5.44,3.0)-- (5.44,1.96);
\draw [dash pattern=on 3pt off 3pt,color=red, rounded corners=0.4cm] (1.18,1.3)-- (0.2525253639975837,1.277456268406644)-- (0.8,2.5)-- (0.19431848059835477,3.366186683434123)-- (1.5518880213785131,4)-- (2.1921637387700317,3.7)-- (3.0458646952920563,4.6)-- (3.802554179482032,3.8)-- (4.617450547071237,4.5)--  (5.323006439098885,3.7)-- (6.8,3.52128817532)-- (6.09202492651837,2.245630522097821)-- (6.809909821775527,1.5277456268406644)-- (5.06,1.22);

\begin{scriptsize}
\draw[color=black] (1.2,3.3) node {};
\draw[color=black] (3.1,3.7) node {};
\draw[color=black] (4.47,3.5) node {};
\draw[color=black] (5.4,3.3) node {};
\draw[color=black] (3.8,4.3) node {};
\end{scriptsize}
\end{tikzpicture}
}

\end{tabular}
\caption{(a) Illustration of  in the disk-coloring . (b)  Illustration of  in the disk-coloring  .}
\label{PolyDecomp}
\end{figure}


We are going to show that  has a smaller crossing number than , thus contradicting the assumption that  is the disk-coloring with minimum crossing number. For this, we extend the definition of  to parts of the polygon boundary: for a contiguous subset  of , we denote by  the number of pockets induced by color  which are fully contained in  and which contain blocks of color .

For the rest of the proof, let  be an arbitrary color of  (and thus also of ). Since every pocket induced by color  in  (and in ) is either contained in  or in , it holds that 

Similarly, it holds that

Furthermore, since  does not differ from  in , it holds that 

and analogously, since , it holds that

Next, we are going to show that 

We are going to prove this by distinguishing two cases:
1) . Since in , by definition  is a single pocket induced by , it follows that  and thus  . Observe that  means that no block of color  was present in  in the  coloring, and this implies that . Furthermore, since a block of color  appears inside  in the coloring , a block of color  appeared inside  in the coloring . Thus it holds that   which together establishes Eq.~(\ref{eqIneq}). 2) . We only need to show that  . To see this, let  be the pockets induced by color  in , which are contained in  (ordered clockwise).  Observe that since in  each block in  which was of color  in  gets colored with color , there are again exactly  such pockets   induced by color  in  which are fully contained in . Next, observe that for any  it holds that . Thus if in  a block of color  is contained in a pocket  then this block was contained in pocket  in . This indeed implies that  proving Eq.~(\ref{eqIneq}) for this second case.


Using Eq.~(\ref{sameOutR}) - (\ref{eqIneq}), we obtain




Furthermore, since color  was chosen arbitrarily, it holds that


Because we colored all blocks of color  in  by color , it follows that  and it thus indeed holds that 
 
contradicting the assumption that  has the smallest crossing number.





\qed


\end{proof}



\begin{lemma}
For a {crossing-free coloring  using  colors}, let   be the set of blocks induced by  . If  then .
\label{indLem}
\end{lemma}
\begin{proof}

We prove the lemma by induction on the number  of colors. For , since  is crossing-free it is easy to see that  and thus the lemma holds.  Assuming the lemma holds for  colors, we show it also holds for  colors.  {For any color  used by , let  be the set of blocks of color }.  If for all , , the lemma trivially holds.  Otherwise fix  to be a color with  and observe that the number of pockets induced by color  is . 



\definecolor{ffqqqq}{rgb}{1.0,0.0,0.0}
\definecolor{qqwuqq}{rgb}{0.0,0.39215686274509803,0.0}
\definecolor{qqqqff}{rgb}{0.0,0.0,1.0}
\definecolor{cqcqcq}{rgb}{0.7529411764705882,0.7529411764705882,0.7529411764705882}
\definecolor{qqwuqq}{rgb}{0.0,0.39215686274509803,0.0}
\definecolor{ffqqqq}{rgb}{1.0,0.0,0.0}
\definecolor{qqqqff}{rgb}{0.0,0.0,1.0}
\definecolor{orange}{rgb}{1.0,0.5,0.0}
\definecolor{brown}{rgb}{0.5,.5,0.0}
\definecolor{yqyqyq}{rgb}{0.5019607843137255,0.5019607843137255,0.5019607843137255}
\definecolor{cqcqcq}{rgb}{0.7529411764705882,0.7529411764705882,0.7529411764705882}

\begin{figure}[ht!]
\center
\begin{tabular}{cc}
\subfloat[]{
\begin{tikzpicture}[line cap=round,line join=round,>=triangle 45,x=1.0cm,y=1.0cm,scale=1.1]
\clip(-0.15,-1.3) rectangle (3.0,2.2);
\draw [dotted,color=qqqqff,fill=qqqqff,fill opacity=0.1] (1.6407502437867678,0.7262099843520922) circle (0.5cm);
\draw [color=qqwuqq,fill=qqwuqq,fill opacity=0.1] (2.3372699835759234,0.7288531726287284) circle (0.5cm);
\draw [dash pattern=on 1pt off 1pt on 1pt off 4pt,color=orange,fill=orange,fill opacity=0.1] (1.6481657819908717,1.4707564648272209) circle (0.5cm);
\draw [color=brown,fill=brown,fill opacity=0.1] (1.1442305039976122,0.7262099843520922) circle (0.5cm);
\draw [dash pattern=on 1pt off 1pt on 1pt off 4pt,color=ffqqqq,fill=ffqqqq,fill opacity=0.1] (0.42379009724472,0.7943597525584467) circle (0.5cm);
\draw [line width=1.2000000000000002pt,dash pattern=on 1pt off 1pt on 1pt off 4pt,color=ffqqqq] (0.5,0.5)-- (0.6967952937271886,0.5032210625048962);
\draw [line width=1.2000000000000002pt,dash pattern=on 1pt off 1pt on 1pt off 4pt,color=ffqqqq] (0.5,0.5)-- (0.0,1.0);
\draw [line width=1.2000000000000002pt,dash pattern=on 1pt off 1pt on 1pt off 4pt,color=ffqqqq] (0.0,1.0)-- (0.7246119238309283,1.00392040437221);
\draw [line width=1.2000000000000002pt,color=brown] (0.6967952937271886,0.5032210625048962)-- (1.1882224255599232,0.5032210625048962);
\draw [line width=1.2000000000000002pt,dotted,color=qqqqff] (1.1882224255599232,0.5032210625048962)-- (1.8882742831707053,0.5032210625048962);
\draw [line width=1.2000000000000002pt,dash pattern=on 1pt off 1pt on 1pt off 4pt,color=orange] (1.4942053567010596,0.9992842993549201)-- (1.499540803585664,1.2735138372423407);
\draw [line width=1.2000000000000002pt,dash pattern=on 1pt off 1pt on 1pt off 4pt,color=orange] (1.499540803585664,1.2735138372423407)-- (1.994904698568374,1.7735138372423407);
\draw [line width=1.2000000000000002pt,dash pattern=on 1pt off 1pt on 1pt off 4pt,color=orange] (1.994904698568374,1.7735138372423407)-- (1.999540803585664,1.1105508197698788);
\draw [line width=1.2000000000000002pt,color=qqwuqq] (1.999540803585664,1.1105508197698788)-- (2.0,1.0);
\draw [line width=1.2000000000000002pt,color=qqwuqq] (2.0,1.0)-- (2.5,1.0);
\draw [line width=1.2000000000000002pt,color=qqwuqq] (2.5,1.0)-- (2.5,0.5);
\draw [line width=1.2000000000000002pt,color=qqwuqq] (2.5,0.5)-- (1.8882742831707053,0.5032210625048962);
\draw [line width=1.2000000000000002pt,color=brown] (1.4942053567010596,0.9992842993549201)-- (0.7246119238309283,1.00392040437221);
\begin{scriptsize}
\draw[color=qqqqff] (1.7,0.4) node {};
\draw[color=qqwuqq] (2.652873888891698,0.7410756695859958) node {};
\draw[color=orange] (1.4436826803576182,1.8148374627642583) node {};
\draw[color=brown] (1,0.4) node {};
\draw[color=ffqqqq] (0.16677676414562995,0.68303449157636) node {};



\end{scriptsize}

\draw [line width=1.2000000000000002pt,dash pattern=on 1pt off 1pt on 1pt off 4pt,color=ffqqqq] (0.5,0.5 - 1.75 )-- (0.6967952937271886,0.5032210625048962   - 1.75 );
\draw [line width=1.2000000000000002pt,dash pattern=on 1pt off 1pt on 1pt off 4pt,color=ffqqqq] (0.5,0.5 - 1.75 )-- (0.0,1.0 - 1.75 );
\draw [line width=1.2000000000000002pt,dash pattern=on 1pt off 1pt on 1pt off 4pt,color=ffqqqq] (0.0,1.0 - 1.75 )-- (0.7246119238309283,1.00392040437221 - 1.75 );

\draw [line width=1.2000000000000002pt,dotted,color=qqqqff] (1.1882224255599232,0.5032210625048962 - 1.75 )-- (1.8882742831707053,0.5032210625048962 - 1.75 );
\draw [line width=1.2000000000000002pt,dash pattern=on 1pt off 1pt on 1pt off 4pt,color=orange] (1.4942053567010596,0.9992842993549201 - 1.75 )-- (1.499540803585664,1.2735138372423407 - 1.75 );
\draw [line width=1.2000000000000002pt,dash pattern=on 1pt off 1pt on 1pt off 4pt,color=orange] (1.499540803585664,1.2735138372423407 - 1.75 )-- (1.994904698568374,1.7735138372423407 - 1.75 );
\draw [line width=1.2000000000000002pt,dash pattern=on 1pt off 1pt on 1pt off 4pt,color=orange] (1.994904698568374,1.7735138372423407 - 1.75 )-- (1.999540803585664,1.1105508197698788 - 1.75 );
\draw [line width=1.2000000000000002pt,color=qqwuqq] (1.999540803585664,1.1105508197698788 - 1.75 )-- (2.0,1.0 - 1.75 );
\draw [line width=1.2000000000000002pt,color=qqwuqq] (2.0,1.0 - 1.75 )-- (2.5,1.0 - 1.75 );
\draw [line width=1.2000000000000002pt,color=qqwuqq] (2.5,1.0 - 1.75 )-- (2.5,0.5 - 1.75 );
\draw [line width=1.2000000000000002pt,color=qqwuqq] (2.5,0.5 - 1.75 )-- (1.8882742831707053,0.5032210625048962 - 1.75 );




\end{tikzpicture}
}
& \subfloat[] {
\begin{tikzpicture}[line cap=round,line join=round,>=triangle 45,x=1.0cm,y=1.0cm,scale=1.1]
\clip(0.0,.25) rectangle (3.0,3.4);
\draw [line width=1.2pt,color=brown] (0.6859854505161606,0.2804467947854956)-- (0.8827807442433493,0.2836678572903919);
\draw [line width=1.2pt,color=brown] (0.6859854505161606,0.2804467947854956)-- (0.18598545051616067,0.7804467947854961);
\draw [line width=1.2pt,color=brown] (0.18598545051616067,0.7804467947854961)-- (0.9105973743470896,0.7843671991577068);
\draw [line width=1.2pt,color=brown] (0.8827807442433493,0.2836678572903919)-- (1.3742078760760807,0.2836678572903919);
\draw [line width=1.2000000000000002pt,dotted,color=qqqqff] (1.3742078760760807,0.2836678572903919)-- (2.0742597336868616,0.2836678572903919);
\draw [line width=1.2000000000000002pt,dash pattern=on 1pt off 1pt on 2pt off 4pt,color=orange] (1.680190807217215,0.7797310941404167)-- (1.6855262541018194,1.0539606320278376);
\draw [line width=1.2000000000000002pt,dash pattern=on 1pt off 1pt on 2pt off 4pt,color=orange] (1.6855262541018194,1.0539606320278376)-- (2.180890149084529,1.5539606320278372);
\draw [line width=1.2000000000000002pt,dash pattern=on 1pt off 1pt on 2pt off 4pt,color=orange] (2.180890149084529,1.5539606320278372)-- (2.1855262541018194,0.8909976145553751);
\draw [line width=1.2000000000000002pt,color=qqwuqq] (2.1855262541018194,0.8909976145553751)-- (2.1859854505161596,0.7804467947854961);
\draw [line width=1.2000000000000002pt,color=qqwuqq] (2.1859854505161596,0.7804467947854961)-- (2.6859854505161604,0.7804467947854961);
\draw [line width=1.2000000000000002pt,color=qqwuqq] (2.6859854505161604,0.7804467947854961)-- (2.6859854505161604,0.2804467947854956);
\draw [line width=1.2000000000000002pt,color=qqwuqq] (2.6859854505161604,0.2804467947854956)-- (2.0742597336868616,0.2836678572903919);
\draw [line width=1.2pt,color=brown] (1.680190807217215,0.7797310941404167)-- (0.9105973743470896,0.7843671991577068);
\draw [line width=1.2000000000000002pt,dash pattern=on 1pt off 1pt on 2pt off 4pt,color=ffqqqq] (0.6473590465507528,2.018634973229065)-- (0.8441543402779409,2.021856035733961);
\draw [line width=1.2000000000000002pt,dash pattern=on 1pt off 1pt on 2pt off 4pt,color=ffqqqq] (0.6473590465507528,2.018634973229065)-- (0.14735904655075305,2.518634973229065);
\draw [line width=1.2000000000000002pt,dash pattern=on 1pt off 1pt on 2pt off 4pt,color=ffqqqq] (0.14735904655075305,2.518634973229065)-- (0.8719709703816804,2.522555377601276);
\draw [line width=1.2pt,color=brown] (0.8441543402779409,2.021856035733961)-- (1.3355814721106687,2.021856035733961);
\draw [line width=1.2pt,color=brown] (1.3355814721106687,2.021856035733961)-- (2.0356333297214486,2.021856035733961);
\draw [line width=1.2pt,color=brown] (1.641564403251802,2.5179192725839856)-- (1.6468998501364065,2.7921488104714065);
\draw [line width=1.2pt,color=brown] (1.6468998501364065,2.7921488104714065)-- (2.142263745119116,3.292148810471407);
\draw [line width=1.2pt,color=brown] (2.142263745119116,3.292148810471407)-- (2.1468998501364065,2.6291857929989444);
\draw [line width=1.2pt,color=brown] (2.1468998501364065,2.6291857929989444)-- (2.1473590465507466,2.518634973229065);
\draw [line width=1.2pt,color=brown] (2.1473590465507466,2.518634973229065)-- (2.6473590465507475,2.518634973229065);
\draw [line width=1.2pt,color=brown] (2.6473590465507475,2.518634973229065)-- (2.6473590465507475,2.018634973229065);
\draw [line width=1.2pt,color=brown] (2.6473590465507475,2.018634973229065)-- (2.0356333297214486,2.021856035733961);
\draw [line width=1.2pt,color=brown] (1.641564403251802,2.5179192725839856)-- (0.8719709703816804,2.522555377601276);
\begin{scriptsize}
\end{scriptsize}
\end{tikzpicture}
}
\end{tabular}
\caption{(a) A disk-coloring example w.r.t. disks ; in the lower part the two pockets induced by color  are shown.  (b) shows the two polygon colorings in the induction step for color  in the proof of Lemma~\ref{indLem}.}
\label{colorExFigProof}
\end{figure}



 Let  be the pockets induced by color .  For each such pocket  we create a new coloring  of , with
 
as illustrated in Fig. \ref{colorExFigProof}(b).

{Since  is crossing-free it is easy to see that  for any pocket , the coloring  is also a crossing-free.  Denoting the number of colors of  by , it holds that , for all }.  Letting  be the {set of blocks} induced by the coloring , by induction hypothesis .  

Observe that each  contains exactly one block not in .  Also, the blocks in  are exactly the blocks not appearing in any of the .  Therefore, since the number of pockets induced by color  equals the number of blocks in {},
it holds that . Thus we obtain



 
\noindent and because each of the colorings  is crossing-free,


where  is the number of colors the coloring {} (and also ) uses for the pocket  .  {The addition of  on the left hand side of (3) attributes for  color }, which was not counted in any of the pockets.
Plugging  into , the lemma follows. 
\qed





\end{proof}








\begin{theorem}
The number of disk centers placed by \textsc{ContiguousGreedy} is at most .
\end{theorem}
\begin{proof}


If  then, by its greedy nature, \textsc{ContiguousGreedy} also uses only one disk. 
If , let  be a crossing free disk-coloring of  w.r.t. , whose existence is guaranteed by Lemma \ref{crossFreeExistsLem}. We let  be the collection of blocks induced by  ordered as they appear on   in clockwise order, with  the block containing . We split  at  into two blocks  and , with  being the portion of  counterclockwise from , and .

Now observe that by the greedy nature, every disk  computed by \textsc{ContiguousGreedy} extends  so that  fully covers at least one new block in the sequence .  Therefore, after computing at most  disks, . By Lemma \ref{indLem}, it holds that  and the theorem follows.  
\qed
\end{proof}


\subsection{Tightness of Analysis}
The analysis for the -approximation ratio of \textsc{ContiguousGreedy} is almost tight, even for convex polygons, as can be seen by a rectangle of length  and height . It can be covered with  many geodesic unit disks (by centering them on the median line at height ). On the other hand, \textsc{ContiguousGreedy} centers disks in steps of  on the boundary, thus after finishing one side of the rectangle, each disk introduced a small uncovered hole on the other side. \textsc{ContiguousGreedy} covers those holes by placing another  disks contiguously on the other side of the polygon, resulting in a total of  disks needed.\\

\vspace{-30pt}

\begin{figure}
\center
\begin{tabular}{cc}
\subfloat[]{
\definecolor{xdxdff}{rgb}{0.49,0.49,1}
\definecolor{black}{rgb}{0,0,1}
\definecolor{cqcqcq}{rgb}{0.75,0.75,0.75}
\begin{tikzpicture}[scale=0.7, line cap=round,line join=round,>=triangle 45,x=1.0cm,y=1.0cm]
\draw (6,5)-- (-1,5)-- (-1,1)-- (6,1) -- (6,5);

\draw [line width=8pt] (0.75,4.8)-- (1.25,4.8);
\draw [line width=8pt] (0.75,1.2)-- (1.25,1.2);


\draw [line width=4pt] (2,4.9)-- (2.5,4.9);
\draw [line width=4pt] (2,1.1)-- (2.5,1.1);




\draw [line width=1.2pt] (3.75,4.95)-- (4.25,4.95);
\draw [line width=1.2pt] (3.75,1.01)-- (4.25,1.01);

\begin{scriptsize}
\node at (1, 5.3) {};
\node at (1, 0.7) {};


\node at (3, 5.3) {};
\node at (3, .7) {};

\node at (4, 5.3) {};
\node at (4, .7) {};

\node at (2.2, 5.3) {};
\node at (2.2, 0.7) {};

\end{scriptsize}

\draw[] (1,2.999) circle(2.01);
\node at (1, 3) {};

\draw[] (2.2,2.999) circle(2.01);
\node at (2.2, 3) {};

\node at (3, 2.999) {};

\draw[] (4,2.999) circle(2.01);
\node at (4, 3) {};


\end{tikzpicture}
}
& \subfloat[] {

\begin{tikzpicture}[line cap=round,line join=round,>=triangle 45,x=1.0cm,y=1.0cm]
\clip(-3.4,-1.5) rectangle (3,1.2);
\draw (-0.8,0)-- (-0.6,0.4)-- (-0.4,0)-- (-0.2,0.4)-- (0,0)-- (0.2,0.4)-- (0.4,0);\draw (0.4,0)-- (.8,0);
\draw (-0.8,0)-- (-1.2,0);
\end{tikzpicture}

}
\end{tabular}
\caption{(a) Illustration of the polygon containing foldings  on the boundary. A global greedy algorithm starts covering the two  foldings on opposite sides by the disk , the two  foldings by a disk  and so on, while  still only uses a constant number of disks to cover . (b) Illustration of a folding. }
\label{globalGreedy}
\end{figure}

Another natural greedy approach is to cover the largest amount of uncovered boundary at each step.  This algorithm results in an approximation ratio of , i.e., it is unbounded with respect to . An example where this greedy rule performs badly is illustrated in Fig.~\ref{globalGreedy}(a). The parts of the boundary denoted by  are dense {\em foldings} as shown in Fig.~\ref{globalGreedy}(b) where the boundary length of  is twice that of , four times that of , and so on.  The global greedy algorithm first covers the two  sections on opposite sides of the boundary (illustrated by  in Fig.~\ref{globalGreedy}(a)), then the two  sections continuing in this way until the two  sections are covered, thereby having used  disks to cover the foldings, (plus some constant number of disks to cover the rest of ). Notice that when the height of the polygon is arbitrary close to , the number of foldings can be made arbitrary large, while  only uses a constant number of disks to cover .\\


It is worth noting that it is crucial that  \textsc{ContiguousGreedy}  computes the maximum extension of the contiguous boundary covered by a single geodesic unit disks in each iteration. Only approximately (even with  precision) extending the contiguously covered part by a single geodesic unit disks results in an approximation factor of at least  (instead of ). To see this, we refer to  Fig.~\ref{counterRef}, where  is the endpoint of the -approximate contiguous greedy extension in the first step and  is the corresponding exact endpoint (obtained from \textsc{ContiguousGreedy}). The -approximate algorithm continues by centering a disks at  which covers the boundary from   up to . At this point, an exact extension could cover the boundary from  up to . However, the approximate algorithm may only cover up to , by, for example, centering the third disk at .  \textsc{ContiguousGreedy} covers up to  using only two disks. Copying the polygon-section between  and , shows that an -approximate algorithm performs at least twice as bad as \textsc{ContiguousGreedy}.


\begin{figure}
\center
\begin{tikzpicture}[line cap=round,line join=round,>=triangle 45,x=1.0cm,y=1.0cm, scale = 1.4]
\clip(-2.5,0.25) rectangle (5.25,3.25);
\draw [line width=3.2pt] (-2.,1.)-- (3.,1.);
\draw [color = white, line width=2.2pt] (-2.,1.)-- (3.,1.);


\draw [line width=3.2pt] (3.,1.)-- (3.,3.);
\draw [color = white, line width=2.2pt] (3.,1.)-- (3.,3.);


\draw [dotted] (0.6,1.)-- (0.6,0.6);
\draw [dotted] (1.,1.)-- (1.,0.6);
\draw [dotted] (3.,1.)-- (3.,0.6);
\draw [dotted] (3.,3.03)-- (3.4,3.03);

\draw [<->, dotted] (-2.,1.4)-- (1.,1.4);
\draw [dotted] (-2.03,1.05)-- (-2.03,1.4);
\draw [dotted] (1,1.)-- (1,1.4);

\draw [line width=3.2pt] (3,1.)-- (5.,1.);
\draw [color = white, line width=2.2pt] (3,1.)-- (5.,1.);

\draw [dotted] (-2.06, 1.04)-- (-2.3, 1.04);
\draw [dotted] (-2.06, 0.97)-- (-2.3, 0.97);

\draw[color=black] (-2.4,1) node {};



\draw [<->, dotted] (3.4,1)-- (3.4,3.03);
\draw [<->, dotted] (0.6,0.6)-- (1.,0.6);
\draw [<->, dotted] (1.,0.6)-- (3.,0.6);



\draw [<->, dotted] (2.4,2.6)-- (2.4,3.03);
\draw[color=black] (2.,2.8) node {};
\draw [dotted] (3.,3.03)-- (2.4,3.03);
\draw [dotted] (3,2.6)-- (2.4,2.6);






\begin{scriptsize}




\draw [fill=black] (1,1.03) circle (1.pt);
\draw[color=black] (1.25157995835,1.165648189885) node {};

\draw [fill=black] (2.4,1.03) circle (1.pt);
\draw[color=black] (2.285993056,1.15648189885) node {};

\fill [white] (2.94,1.03) rectangle (6.017,0.978);

\fill [white] (2.98,1.04) rectangle (3.02,0.978);


\draw [fill=black] (-0.685993056,1.03) circle (1pt);
\draw[color=black] (-0.6885993056,1.15648189885) node {};

\draw [fill=black] (3.03,1.4) circle (1.pt);
\draw[color=black] (3.2085993056,1.45648189885) node {};


\draw [fill=black] (.6,1.03) circle (1.pt);
\draw[color=black] (0.680025181185,1.15648189885) node {};

\draw [fill=black] (2.97,2.6) circle (1.pt);
\draw[color=black] (2.75,2.5) node {};


\draw [fill=black] (4.4,1.03) circle (1.pt);
\draw[color=black] (4.480025181185,1.15648189885) node {};


\draw [fill=black] (5,1.03) circle (1.pt);
\draw[color=black] (5.180025181185,1.15648189885) node {};


\draw[color=black] (.78,0.82) node {};
\draw[color=black] (0,1.28) node {};
\draw[color=black] (2,0.72) node {};
\draw[color=black] (3.8,2) node {};



\end{scriptsize}
\end{tikzpicture}
\caption{Illustration of the -thin polygon where an -approximate contiguous extension algorithm results in an approximation ratio larger than .}
\label{counterRef}
\end{figure}



\section{Covering Large Perimeters}
\label{refAna}

In this section we show that if the polygon perimeter  is significantly larger than , i.e., , with , a simple linear time algorithm achieves an approximation ratio which goes to one as  goes to infinity. For this, we decompose  into long and short portions, based on the length of the corresponding \emph{medial axis}. The medial axis is the set of points in  which have more than one closest point on . It forms a tree whose edges are either line segments or parabolic arcs and it can be computed in linear time \cite{medialAxis}. For a line segment edge, the closest points to the boundary are a subset of two polygon edges; for a parabolic edge, the closest boundary points are a polygon vertex and a subset of a polygon edge.  The idea of the algorithm is to identify long edges of the medial axis (of length at least some constant ), and to cover the corresponding polygon boundary section (referred to as {\em corridors}) almost optimally using only a constant number of disks more than  uses to cover the corridor. It is easy to see that each corridor stemming from a parabolic arc can be covered with at most two more disks than  uses, by centering disks at distance  from each other on the corresponding polygon boundary segment and one disk on the corresponding polygon vertex.  Each corridor consisting of a pair of polygon boundary segments can be covered by greedily centering disks on the corresponding medial axis as long as each disk contains corridor portions of length more than two; if the length becomes two or less, greedily center the disks on  corridor segments in steps of two.  Observe that also in this case, the number of disks needed to cover a corridor is at most two more than  uses and their centers can be computed in time linear in their number. This holds since there is at most one point where the covering changes from centering disks on the medial axis to centering disks on . The rest of the polygon, i.e., the short portions,  can be covered greedily by centering  disks on .

Let  be the set of all disks placed by the algorithm,  the disks covering the corridors and  the  disks covering the short portion of .  
Since the number of edges in the medial axis is  (see \cite{medialAxis}) and the procedure for covering the long corridors uses at most two more disks than  for each corridor, . It therefore holds that . It is easy to see that the disks of  which contain a polygon vertex cover at most an  portion of  implying that .  Therefore, the approximation ratio can be written as
 






\section{Acknowledgments}
\vspace{-8pt}
We thank Alon Efrat for his idea of looking at large perimeter polygons. We further thank the anonymous referees for their review of a previous version of this manuscript.

\vspace{-8pt}

\bibliographystyle{abbrv}	\bibliography{refDrop}




\end{document}