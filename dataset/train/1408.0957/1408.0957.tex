We now show our combined framework.
We assume for each program point, a persistent set and its associated interpolant 
are computed statically, 
i.e., by separate analyses.
In other words, when we are at a program point, we can right away make use of 
the information about its persistent set.




The algorithm is in Fig.~\ref{por:algo:static}. 
The function  has input  and 
assumes the safety property at hand is .
It naturally performs a depth first search of the state space.

\begin{wrapfigure}{r}{0.54\textwidth}\begin{center}
\vspace{-2mm}
\pppreset
\small{
\safetys_0\symstate\pair{\pc}{\unknown}and
}
\end{center}
\vspace{-10pt}
\caption{A Framework for POR and SI (DFS)} 
\label{por:algo:static}
\end{wrapfigure}



\noindent \textbf{Two Base Cases:} The function  handles two base cases. 
One is when the current state is subsumed by some computed 
(and memoed) interpolant .
No further exploration is needed, and  is returned as the interpolant 
(line~\ref{por:algo:base1}). 
The second base case is when the current state is found to be \emph{unsafe}
(line~\ref{por:algo:base2}).

\vspace{2pt}
\noindent \textbf{Combining Interpolants:} 
We make use of the (static) persistent set  computed for the current program point.
We comment further on this in the next section. 


The set of transitions to be considered is denoted by \func{Ts}.
When the current state implies the interpolant  associated with ,
we need to consider only those transitions in . Otherwise, we need to consider
all the schedulable transitions. 
Note that when the persistent set  is employed,
the  interpolant  must contribute to the combined interpolant 
of the current state (line~\ref{por:algo:combine_trace}).
Disabled transitions at the current state will strengthen the interpolant 
as in line~\ref{por:algo:disabled}.
Finally, we recursively follow those transitions which 
are enabled at the current state.
The interpolant of each child state contributes to the interpolant of the current state as
in line~\ref{por:algo:child}. In our framework, interpolants
are propagated back using the precondition operation \wpc,
where  denotes a \emph{safe approximation} of the weakest precondition
wrt. the transition  and the postcondition  \cite{dijkstrawp}.








\begin{theorem}
The algorithm in Fig.~\ref{por:algo:static} is 
\emph{sound}\footnote{Proof outline is in the Appendix.}. 
\qed
\end{theorem}











