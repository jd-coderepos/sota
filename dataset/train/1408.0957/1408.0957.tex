We now show our combined framework.
We assume for each program point, a persistent set and its associated interpolant 
are computed statically, 
i.e., by separate analyses.
In other words, when we are at a program point, we can right away make use of 
the information about its persistent set.




The algorithm is in Fig.~\ref{por:algo:static}. 
The function $\func{Explore}$ has input $s$ and 
assumes the safety property at hand is $\safety$.
It naturally performs a depth first search of the state space.

\begin{wrapfigure}{r}{0.54\textwidth}\begin{center}
\vspace{-2mm}
\pppreset
\small{
$\begin{array}{rl}
\multicolumn{2}{l}{\mbox{Assume safety property $\safety$ and initial state $s_0$}} \\
\ppp &Initially: \func{Explore}(s_0)				\\
\multicolumn{2}{l}{\mbox{\textbf{function}} ~ \func{Explore}(s)} \\
& \mbox{Let $\symstate$ be $\pair{\pc}{\unknown}$} \\
\ppp \label{por:algo:base1} & \If~(\func{memoed}(s, \Intpsymbol))~\Return~\Intpsymbol		\\
\ppp \label{por:algo:base2} & \If~(\symstate \not\models \safety)  
~\texttt{Report Error and TERMINATE}\\
\ppp &  \Intpsymbol ~\Assign~ \safety						\\

\ppp &  \langle T, \Intpsymbol_{trace} \rangle ~\Assign~ \func{Persistent\_Set}(\pc)			\\
\ppp & \If~(\symstate \models \Intpsymbol_{trace}) \\
\ppp & \ptab \func{Ts}~\Assign~T \\
\ppp\label{por:algo:combine_trace} & \ptab \Intpsymbol ~\Assign~\Intpsymbol~\wedge~\Intpsymbol_{trace} \\
\ppp & \Else~\func{Ts}~\Assign~Schedulable(s) \\

\ppp &  \Foreach~t~in~(\func{Ts}~\setminus~Enabled(s))~\Do	\\
\ppp \label{por:algo:disabled}& \ptab \ptab \Intpsymbol ~\Assign~\Intpsymbol~\wedge~
\pre{t}{\false}	\\



\ppp \label{por:algo:enabled-persistent} &  \Foreach~t~in~(\func{Ts}~\cap~Enabled(s))~\Do	 		\\
\ppp \label{por:algo:extend}&  \ptab \transition{\symstate}{\symstate'}{t}	
\ptab \ptab \ptab \ptab  {\mbox{/* Execute t */}}		\\
\ppp \label{por:algo:recursive}&  \ptab \Intpsymbol' ~\Assign~ \func{Explore}(s')			\\
\ppp \label{por:algo:child}&  \ptab \Intpsymbol ~\Assign ~ \Intpsymbol~\wedge~
\pre{t}{\Intpsymbol'}		\\	
\ppp\label{por:algo:return}&  \func{memo}~$and$~\Return~(\Intpsymbol)					\\
\multicolumn{2}{l}{\mbox{\textbf{end function}}}
\end{array}$
}
\end{center}
\vspace{-10pt}
\caption{A Framework for POR and SI (DFS)} 
\label{por:algo:static}
\end{wrapfigure}



\noindent \textbf{Two Base Cases:} The function $\func{Explore}$ handles two base cases. 
One is when the current state is subsumed by some computed 
(and memoed) interpolant $\Intpsymbol$.
No further exploration is needed, and $\Intpsymbol$ is returned as the interpolant 
(line~\ref{por:algo:base1}). 
The second base case is when the current state is found to be \emph{unsafe}
(line~\ref{por:algo:base2}).

\vspace{2pt}
\noindent \textbf{Combining Interpolants:} 
We make use of the (static) persistent set $T$ computed for the current program point.
We comment further on this in the next section. 


The set of transitions to be considered is denoted by \func{Ts}.
When the current state implies the interpolant $\Intpsymbol_{trace}$ associated with $T$,
we need to consider only those transitions in $T$. Otherwise, we need to consider
all the schedulable transitions. 
Note that when the persistent set $T$ is employed,
the  interpolant $\Intpsymbol_{trace}$ must contribute to the combined interpolant 
of the current state (line~\ref{por:algo:combine_trace}).
Disabled transitions at the current state will strengthen the interpolant 
as in line~\ref{por:algo:disabled}.
Finally, we recursively follow those transitions which 
are enabled at the current state.
The interpolant of each child state contributes to the interpolant of the current state as
in line~\ref{por:algo:child}. In our framework, interpolants
are propagated back using the precondition operation \wpc,
where $\pre{t}{\phi}$ denotes a \emph{safe approximation} of the weakest precondition
wrt. the transition $t$ and the postcondition $\phi$ \cite{dijkstrawp}.








\begin{theorem}
The algorithm in Fig.~\ref{por:algo:static} is 
\emph{sound}\footnote{Proof outline is in the Appendix.}. 
\qed
\end{theorem}











