
\documentclass{llncs}
\usepackage[dvips]{graphicx}
\usepackage{pstricks}
\usepackage{psfrag}
\usepackage{amssymb}
\bibliographystyle{splncs}
\begin{document}

\mainmatter

\title{Obtaining a Planar Graph by Vertex Deletion}
\author{D\'aniel Marx \and Ildik\'o Schlotter}

\institute{
Department of Computer Science and Information Theory, \\
Budapest University of Technology and Economics, \\
Budapest, H-1521, Hungary. \\
\email{\{dmarx,ildi\}@cs.bme.hu}\\
}

\maketitle

\begin{abstract}
In the -\textsc{Apex} problem the task is to find at most  vertices
whose deletion makes the given graph planar.
The graphs for which there exists a solution form a minor closed class of graphs,
hence by the deep results of Robertson and Seymour \cite{sey95,sey04}, there is an  time algorithm
for every fixed value of . However, the proof is extremely complicated and the constants hidden by
the big-O notation are huge.
Here we give a much simpler algorithm for this problem with quadratic running time, by
iteratively reducing the input graph and then applying techniques for graphs of bounded treewidth.

\noindent {\bf Keywords:} Planar graph, Apex graph, FPT algorithm, Vertex deletion.
\end{abstract}




\section{Introduction}

Planar graphs are subject of wide research interest in graph theory.
There are many generally hard problems which can be solved in polynomial time when considering planar graphs,
e.g., \textsc{Maximum Clique}, \textsc{Maximum Cut}, and \textsc{Subgraph Isomorphism}
\cite{epp99,had75}.
For problems that remain NP-hard on planar graphs, we often have efficient approximation algorithms.
For example, the problems \textsc{Independent Set, Vertex Cover}, and  \textsc{Dominating Set}
admit an efficient linear time approximation scheme \cite{bak94,lip80}.
The research for efficient algorithms for problems on planar graphs is still very intensive.

Many results on planar graphs can be extended to almost planar graphs,
which can be defined in various ways.
For example, we can consider possible embeddings of a graph in a surface other than the plane.
The genus of a graph is the minimum number of handles that must be added to the
plane to embed the graph without any crossings.
Although determining the genus of a graph is NP-hard \cite{tho89},
the graphs with bounded genus are subjects of wide research.
A similar property of graphs is their crossing number, i.e.,
the minimum possible number of crossings with which the graph can be drawn in the plane.
Determining crossing number is also NP-hard \cite{gar83}.

In \cite{cai96} Cai introduced another notation to capture the distance of a
graph  from a graph class , based on the number of certain elementary modification steps.
He defines the distance of  from  as the minimum number of modifying steps needed to
make  a member of .
Here, modification can mean the deletion or addition of edges or vertices.
In this paper we consider the following question: given a graph  and an integer
, is there a set of at most  vertices in , whose deletion makes  planar?

It was proven by Lewis and Yannakakis in \cite{lew80} that the node-deletion problem is NP-complete
for every non-trivial hereditary graph property decidable in polynomial time.
As planarity is such a property, the problem of finding a maximum induced planar subgraph is NP-complete,
so we cannot hope to find a polynomial-time algorithm that answers the above question.
Therefore, following Cai, we study the problem in the framework of parameterized complexity
developed by Downey and Fellows \cite{dow99}.
This approach deals with problems in which besides the input  an integer  is also
given. The integer  is referred to as the parameter.
In many cases we can solve the problem in time .
Clearly, this is also true for the problem we consider.
Although this is polynomial time for each fixed , these algorithms are practically
too slow for large inputs, even if  is relatively small.
Therefore, the standard goal of parameterized analysis is to take the parameter
out of the exponent in the running time.
A problem is called \emph{fixed-parameter tractable} (FPT) if it can be solved in
time , where  is a polynomial not depending on , and  is an arbitrary function.
An algorithm with such a running time is also called FPT.
For more on fixed-parameter tractability see e.g.  \cite{dow99}, \cite{nie06} or \cite{flu06}.

The standard parameterized version of our problem is the
following: given a graph  and a parameter , the task is to
decide whether deleting at most  vertices from  can result in a planar graph.
Such a set of vertices is sometimes called a set of \emph{apex vertices} or \emph{apices},
so we will denote the class of graphs for which the answer is `yes' by .
We note that Cai \cite{cai96} used the notation  to denote this class.

In the parameterized complexity literature, numerous similar node-deletion problems have been studied.
A classical result of this type by Bodlaender \cite{bod94} and Downey and Fellows \cite{dow92} states that
the \textsc{Feedback Vertex Set} problem, asking whether a graph can be made acyclic by the deletion of at most  vertices,
is FPT. The parameterized complexity of the directed version of this problem has been a long-standing open question,
and it has only been proved recently that it is FPT as well \cite{che07}.
Fixed-parameter tractability has also been proved for the problem of finding  vertices whose deletion results
in a bipartite graph \cite{ree04}, or in a chordal graph \cite{marxx}.
On the negative side, the corresponding node-deletion problem for wheel-free graphs was proved to be W[2]-hard \cite{lok08}.

Considering the graph class ,
we can observe that this family of graphs is closed under taking minors.
The celebrated graph minor theorem by Robertson and Seymour states that such families can be characterized by
a set of excluded minors \cite{sey04}. They also showed that for each graph  it can be tested in cubic time whether
a graph contains  as a minor \cite{sey95}. As a consequence, membership for such graph classes can be decided in cubic time.
In particular, we know that there exists an algorithm with running time  that
can decide whether a graph belongs to .
However, the proof of the graph minor theorem is non-constructive in the following sense.
It proves the existence of an algorithm for the membership test that uses the excluded minor
characterization of the given graph class, but does not provide any algorithm for determining this characterization.
Recently, an algorithm was presented in \cite{adl08} for constructing the set of excluded minors
for a given graph class closed under taking minors,
which yields a way to explicitly construct the algorithm whose existence was proved by Robertson and Seymour.
We remark that it follows also from \cite{fel89} that an algorithm for testing membership in
 can be constructed explicitly.

Although these results provide a general tool that can be applied to our specific problem,
no direct FPT algorithm has been proposed for it so far.
In this paper we present an algorithm which decides membership for  in  time.
Note that the presented algorithm runs in quadratic time, and hence yields a better running time than
any algorithm using the minor testing algorithm that is applied in the above mentioned approaches.
Moreover, if  then our algorithm also returns a solution,
i.e., a set ,  such that  is planar.

The presented algorithm is strongly based on the ideas used by Grohe in \cite{gro04}
for computing crossing number.
Grohe uses the fact that the crossing number of a graph is an upper bound for its genus.
Since the genus of a graph in  cannot be bounded by a function of ,
we need some other ideas.
As in \cite{gro04}, we exploit the fact that in a graph with large treewidth
we can always find a large grid minor \cite{sey94}. Examining the structure of
the graph with such a grid minor, we can reduce our problem to a smaller instance.
Applying this reduction several times, we finally get an instance with
bounded treewidth. Then we make use of Courcelle's Theorem \cite{cou90}, which
states that every graph property that is expressible in monadic second-order logic
can be decided in linear time on graphs of bounded treewidth.

The paper is organized as follows.
Section \ref{notation} summarizes our notation,
Sect. \ref{algo_sketch} outlines the algorithm,
Sect. \ref{phase_I} and \ref{phase_II} describe the two phases of the algorithm.


\section{Notation}
\label{notation}

Graphs in this paper are assumed to be simple, since both loops and multiple edges are
irrelevant in the
-\textsc{Apex}
problem.
The vertex set and edge set of a graph 
are denoted by  and , respectively.
The edges of a graph are unordered pairs of its vertices.
If  is a subgraph of  then  denotes
the graph obtained by deleting  from . For a set of vertices  in ,
we will also use  to denote the graph obtained by deleting  from .

A graph  is a \emph{minor} of a graph  if it can be obtained from a subgraph of 
by contracting some of its edges. Here \emph{contracting an edge}  with endpoints  and  means
deleting , and then identifying vertices  and .

A graph  is a \emph{subdivision} of a graph  if  can be obtained from 
by contracting some of its edges that have at least one endpoint of degree two.
Or, equivalently,  is a subdivision of  if  can be obtained from  by
replacing some of its edges with newly introduced paths
such that the inner vertices of these paths have degree two in .
We refer to these paths in  corresponding to edges of  as \emph{edge-paths}.
A graph  is a \emph{topological minor} of  if  has a subgraph that is a subdivision of .
We say that  and  are \emph{topologically isomorphic} if they both are subdivisions of a graph .

The  \emph{grid} is the graph  where
 and
.
Instead of giving a formal definition for the \emph{hexagonal grid} of radius ,
which we will denote by , we refer to the illustration shown in Fig. \ref{fig_hexa}.
A \emph{cell} of a hexagonal grid is one of its cycles of length .

\begin{figure}[t]
\begin{center}
\includegraphics[scale=0.25]{hexa_grid.eps}
\end{center}
\caption{The hexagonal grids , , and .}
\label{fig_hexa}
\end{figure}

A \emph{tree decomposition} of a graph  is a pair  where
 is a tree,  for all , and the following are true:

\begin{itemize}
\item
for all  there exists a  such that ,
\item
for all  there exists a  such that ,
\item
if  lies on the path connecting  and  in , then
.
\end{itemize}

The \emph{width} of such a tree decomposition is the maximum of  taken over all .
The \emph{treewidth} of a graph , denoted by ,
is the smallest possible width of a tree decomposition of .
For an introduction to treewidth see e.g. \cite{bod97,die00}.

\section{Problem Definition and Overview of the Algorithm}
\label{algo_sketch}
We are looking for the solution of the following problem:

\begin{center}
\fbox{ \parbox{11cm}{
-\textsc{Apex}
problem: \4pt]
\begin{tabular}{ll}
& Input: .\4pt]
\end{tabular}\\
\begin{tabular}{llp{10.3cm}}
& 1. \hspace{-2pt} &
Run algorithm  on , , and . \\
& & If it returns a subgraph  topologically isomorphic to  then go to Step 2.
If it returns a tree decomposition  of , then
output(, , ). Otherwise output(``No solution.'').\4pt]& 3. \hspace{-2pt} &
Let . For all : if  is well-attached then . \\
& & If  or  then output(``No solution.''). \\
& & Otherwise ,  and  go to Step 1. \\textup{partition}(X,Y,Z) := \forall z:
\left( Zz  \rightarrow
\left(
\left( Xz \rightarrow \neg Yz \right) \land \left( \neg Xz \rightarrow Yz \right)
\right)
\right)

& &
\textup{connected}(a,b,Z) := Za \land Zb \land \forall X \forall Y: \\
& & \quad \left(
\left( \textup{partition}(X,Y,Z) \land Xa \land Yb  \right) \rightarrow
\left( \exists c \exists d \exists e: Xc \land Yd \land Ice \land Ide \right)
\right)

& &
\textup{disjoint}(X,Y) := \forall z: (Xz \rightarrow \neg Yz) \\
& &
\textup{almost-disjoint}(X,Y,a) := \forall z: (Xz \rightarrow (\neg Yz \lor (z=a)))

& &
\textup{-top-minor }(v_1,v_2, \dots, v_5, P_{12},P_{13},\dots, P_{45}) :=  \\
& &
\quad \textup{connected}(v_1,v_2,P_{12}) \land \dots \land \textup{connected}(v_4,v_5,P_{45}) \land \\
& &
\quad \textup{almost-disjoint}(P_{12},P_{13},v_1) \land \dots \land \textup{almost-disjoint}(P_{35},P_{45},v_5) \land \\
& &
\quad \textup{disjoint}(P_{12},P_{34}) \land \dots \land \textup{disjoint}(P_{23},P_{45})

& &
\textup{apex}(v_1,v_2, \dots, v_{k'}) :=  \\
& &
\quad \forall x_1 \forall x_2 \dots \forall x_5 \forall X_1 \forall X_2 \dots \forall X_{10}:
(\textup{-top-minor}(x_1, \dots, x_5,X_1, \dots, X_{10}) \\
& &
\quad \quad \rightarrow ((x_1=v_1) \lor \dots \lor (x_5=v_{k'})  \lor X_1 v_1 \lor \dots \lor X_{10} v_{k'})) \land \\
& &
\quad \forall x_1 \forall x_2 \dots \forall x_6 \forall X_1 \forall X_2 \dots \forall X_9:
(\textup{-top-minor}(x_1,\dots, x_6,X_1, \dots, X_9) \\
& &
\quad \quad \rightarrow((x_1=v_1) \lor \dots \lor (x_6=v_{k'})  \lor X_1 v_1 \lor \dots \lor X_{9} v_{k'})) \\

\qed
\end{proof}

Now let us apply Theorem \ref{mso}.
Let  be the algorithm which, given a graph  of bounded treewidth,
decides whether there exist  such that
 is true, and if possible, also produces such variables.
By Theorem \ref{planar_mso}, running  on  either returns a set of vertices
, or reports that this is not possible.
Hence, we can finish algorithm  in the following way:
if  returns  then output(), otherwise
output(``No solution'').

The running time of Phase II is  for some function .

\begin{remark}
Phase II of the algorithm can also be done by applying dynamic programming,
using the tree decomposition  returned by .
This also yields a linear-time algorithm, with a double exponential dependence on  (and hence on ).
Since the proof is quite technical and detailed, we omit it.
\end{remark}

\begin{thebibliography}{99}

\bibitem{adl08}
I. Adler, M. Grohe, S. Kreutzer: Computing excluded minors.
\emph{SODA 2008}, 641--650, 2008.

\bibitem{arn91}
S. Arnborg, A. Proskurowski, D. Seese: Monadic second order logic, tree automata and forbidden minors.
\emph{CSL 1990, LNCS 533}, 1--16, 1991.


\bibitem{bak94}
B.S. Baker: Approximation algorithms for NP-complete problems on planar graphs.
\emph{J. ACM}, 41:153--180, 1994.

\bibitem{bod94}
H.L. Bodlaender: On disjoint cycles.
\emph{International Journal of Foundations of Computer Science}, 5:59--68, 1994.

\bibitem{bod96}
H.L. Bodlaender: A linear-time algorithm for finding tree-decompositions of small treewidth.
\emph{SIAM J. Comput.}, 25:1305--1317, 1996.

\bibitem{bod97}
H.L. Bodlaender: Treewidth: Algorithmic techniques and results.
\emph{MFCS 1997, LNCS 1295}, 19--36, 1997.



\bibitem{cai96}
L. Cai:
Fixed-parameter tractability of graph modification problems for hereditary properties.
\emph{Inform. Process. Lett.}, 58:171--176, 1996.

\bibitem{che07}
J. Chen, Y. Liu, S. Lu:
Directed feedback vertex set problem is FPT.
\emph{Dagstuhl Seminar Proceedings: Structure Theory and FPT Algorithmics for Graphs, Digraphs and Hypergraphs}, 2007.

\bibitem{cou90}
B. Courcelle: \emph{Graph rewriting: an algebraic and logic approach. Handbook of Theoretical Computer Science},
Elsevier, Amsterdam.  2:194--242, 1990.

\bibitem{cou97}
B. Courcelle:
\emph{The expression of graph properties and graph transformations in monadic second-order logic.
Handbook of graph grammars and computing by graph transformations}, World Scientific, New-Jersey,
Vol. 1, Chapter 5, 313--400, 1997.

\bibitem{die00}
R. Diestel: \emph{Graph Theory}, Springer, Berlin, 2000.

\bibitem{dow92}
R.G. Downey, M.R. Fellows: Fixed-parameter tractability and completeness.
\emph{Congressus Numerantium}, 87:161--187, 1992.

\bibitem{dow99}
R.G. Downey, M.R. Fellows:
\emph{Parameterized Complexity}, Springer-Verlag, New York, 1999.

\bibitem{epp99}
D. Eppstein:
Subgraph isomorphism in planar graphs and related problems.
\emph{J. Graph Algorithms Appl.},3:1--27, 1999.

\bibitem{fel89}
M.R. Fellows, M.A. Langston:
On search, decision and the efficiency of polynomial-time algorithms (extended abstract).
\emph{STOC 1989}, 501--512, 1989.

\bibitem{flu02}
J. Flum, M. Frick, M. Grohe:
Query evaluation via tree-decompositions.
\emph{J. ACM}, 49:716--752, 2002.

\bibitem{flu06}
J. Flum, M. Grohe:  \emph{Parameterized Complexity Theory}, Springer-Verlag, Berlin, 2006.

\bibitem{gar83}
M.R. Garey, D.S. Johnson: Crossing Number is NP-Complete.
\emph{SIAM J. Algebraic Discrete Methods}, 4:312--316, 1983.

\bibitem{gro04}
M. Grohe: Computing crossing number in quadratic time.
\emph{J. Comput. System Sci.}, 68:285--302, 2004.

\bibitem{had75}
Hadlock, F.:
Finding a maximum cut of a planar graph in polynomial time.
\emph{SIAM J. Comput.}, 4:221--225, 1975.

\bibitem{hop74}
J.E. Hopcroft, R.E. Tarjan:
Efficient planarity testing.
\emph{J. ACM}, 21:549--568, 1974.

\bibitem{lew80}
J.M. Lewis, M. Yannakakis:
The vertex-deletion problem for hereditary properties is NP-complete.
\emph{J. Comput. System Sci.}, 20:219--230, 1980.

\bibitem{lip80}
R.J. Lipton, R.E. Tarjan:
Applications of a planar separator theorem.
\emph{SIAM J. Comput.}, 9:615--627, 1980.

\bibitem{lok08}
D. Lokshtanov: Wheel-free deletion is W[2]-hard.
\emph{IWPEC 2008, LNCS 5018}, 141--147, 2008.

\bibitem{marxx}
D. Marx: Chordal deletion is fixed-parameter tractable.
\emph{To appear in Algorithmica}.
(Conference version: \emph{WG 2006, LNCS 4271}, 37--48, 2006.)

\bibitem{nie06}
R. Niedermeier: \emph{Invitation to Fixed-Parameter Algorithms}. Oxford University Press, 2006.

\bibitem{per00}
L. Perkovic, B. Reed:
An improved algorithm for finding tree decompositions of small width.
\emph{Internat. J. Found. Comput. Sci.} 11:365--371, 2000.

\bibitem{ree04}
B.A. Reed, K. Smith, A. Vetta: Finding odd cycle transversals.
\emph{Oper. Res. Lett.}, 32(4):299--301, 2004.

\bibitem{sey86}
N. Robertson, P.D. Seymour:
Graph minors. V. Excluding a planar graph.
\emph{J. Combin. Theory Ser. B}, 41(1):92--114, 1986.

\bibitem{sey04}
N. Robertson, P.D. Seymour:
Graph minors. XX. Wagner's conjecture.
\emph{J. Combin. Theory Ser. B}, 92(2):325--357, 2004.

\bibitem{sey95}
N. Robertson, P.D. Seymour:
Graph minors. XIII. The disjoint paths problem.
\emph{J. Combin. Theory Ser. B}, 63(1):65--110, 1995.

\bibitem{sey94}
N. Robertson, P.D. Seymour, R. Thomas:
Quickly excluding a planar graph.
\emph{J. Combin. Theory Ser. B}, 62(2):323--348, 1994.

\bibitem{tho89}
C. Thomassen:
The graph genus problem is NP-complete.
\emph{J. Algorithms}, 10:568--576, 1989.

\end{thebibliography}
\end{document}
