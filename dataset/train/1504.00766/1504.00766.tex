\documentclass[11pt]{article}


\usepackage{latexsym}
\usepackage[dvips]{graphicx}

\usepackage{amsmath,amssymb}

\setlength{\topmargin}{-30pt}
\setlength{\oddsidemargin}{0cm}
\setlength{\evensidemargin}{0cm}
\setlength{\textheight}{23.7cm}
\setlength{\textwidth}{16cm}







\newtheorem{thm}{Theorem}[section]
\newtheorem{df}[thm]{Definitions}
\newtheorem{lm}[thm]{Lemma}
\newtheorem{co}[thm]{Corollary}
\newtheorem{pr}[thm]{Proposition}
\newtheorem{as}[thm]{Assumption}
\newtheorem{ob}[thm]{Observation}
\newtheorem{property}[thm]{Property}
\newcommand{\qed}{\hspace*{\fill} }





\title{Every property is testable on a natural class of scale-free multigraphs --- ver.~2}


\author{
Hiro Ito\thanks{
     School of Informatics and Engineering, 
    The University of Electro-Communications (UEC), 
    Tokyo, Japan; and
    CREST, JST, Tokyo, Japan; 
    {\tt itohiro@uec.ac.jp}
    }
}







\begin{document}

\setcounter{page}{0}

\maketitle



\begin{abstract}
In this paper, we introduce a natural class of multigraphs called hierarchical-scale-free (HSF) multigraphs, and consider constant-time testability on the class. 
We show that a very wide subclass, specifically, that in which the power-law exponent is greater than two, 
of HSF is hyperfinite. 
Based on this result, an algorithm for a deterministic partitioning oracle can 
be constructed.  
We conclude by showing that every property is constant-time testable 
on the above subclass of HSF. 
This algorithm utilizes findings by Newman and Sohler of STOC'11.  
However, their algorithm is based on the bounded-degree model, 
while it is known that actual scale-free networks 
usually include hubs, which have a very large degree.  
HSF is based on scale-free properties and includes such hubs.
This is the first universal result of constant-time testability on  
the general graph model, and it has the potential to be applicable on  a very wide range of scale-free networks. 
\end{abstract}


\newpage
\section{Introduction}




How to handle big data is a very important issue in computer science. 
In the theoretical area, developing efficient algorithms for handling big data 
is an urgent task. 
For this purpose, sublinear-time algorithms look like they could be powerful tools, 
as they are able to read very small parts (constant size) of inputs. 

Property testing is the most well-studied area in sublinear-time algorithms. 
A testing algorithm (or a tester) for a property accepts an input if it has the stipulated property and rejects it if it is far away from having the stipulated property with a high probability (e.g., at least ) by reading 
a constant part of the input. 
A property is said to be testable if there is a tester~\cite{PropertyTestingLNCS10}. 


Property testing of graph properties has been well studied and many fruitful results have been obtained \cite{AlonFNS_Testable_Regularity_SIAMJC09,BSS_MC-testable_STOC08,PropertyTestingLNCS10,GR-STOC97,GGR-JACM98,HKNO_LocalPartition_FOCS09,NS_Testable_SJCOMP13,Levi-Ron_PO_ICALP13}.
Testers on the graphs are separated into three groups according to model: 
the dense-graph model (the adjacent-matrix model), 
the bounded-degree model, 
and 
the general model. 
The dense-graph model is the best clarified: 
In this model, the characteristics of testable properties have been obtained~\cite{AlonFNS_Testable_Regularity_SIAMJC09}.
However, graphs based on actual networks are usually sparse and unfortunately the dense-graph model has been found not to work. 
Studies on the bounded-degree model have been proceeding recently. 
One of the most important findings for this model is that every minor-closed property is testable~\cite{BSS_MC-testable_STOC08}.  
This result can be extended to the surprising result that every property of a hyperfinite graph is testable~\cite{NS_Testable_SJCOMP13}.
However, graphs based on actual models have no degree bounds, i.e., it is known that web-graphs have hubs~\cite{AlbertBarabasi_SF_02,KleinbergLawrence-web-science01}, which have a large degree, and, unfortunately once again, these algorithms do not work for them. 

Typical big-data graph models are scale-free networks, 
which are characterized by the power-law degree distribution.   
Many models have been proposed for scale-free networks \cite{AlbertBarabasi_SF_02,BM_Kronecker_IPL98,Broder_etal-GSinWeb-CN00,CU_SF_Clique_10,Gao_SF_clique_TCS09,KleinbergLawrence-web-science01,StochKronecherG_WAW07,Newman_SF_03,Uno-Watanabe_ScaleFree,WattsStrogatz-Nature98,ZRC_SF_clique_06}. 
Recently, a promising model based on another property of a hierarchical isomorphic structure 
has been presented: 
If we look at a graph in a broad perspective, 
we find a similar structure to local structures. 
Shigezumi, Uno, and Watanabe~\cite{Uno-Watanabe_ScaleFree}
presented a model that is based on the idea of the hierarchical isomorphic 
structure of power-law distribution of isolated cliques.  
An idea of isolated cliques was given by 
Ito and Iwama~\cite{IsoClique_TALG09,IsoClique_ESA05}, 
and the definition is as follows. 
For a nonnegative integer , a {\em -isolated clique}
is a clique such that the number of outgoing edges (edges between the clique and the other vertices) 
is less than , where  is the number of vertices of the clique. 
A 1-isolated clique is sometimes simply called an {\em isolated clique}. 





Based on the model of \cite{Uno-Watanabe_ScaleFree}, we introduce a class of multigraphs, 
hierarchical scale-free multigraphs (HSF, Definitions~\ref{df:HSF})\footnote{
In a preliminary version of this paper, \cite{ScalefreeTest_itohiro15_arXiv}, 
the definition of HSF is different.  
The definition in this paper is far more general (wider) than in the preliminary version. 
}, 
which represents natural scale-free networks. 
We show the following result (Theorem~\ref{th:testable}): 

\begin{quote}
{\em Every property is testable on HSF if the power-law exponents are greater than two}.  
\end{quote}

Given this result, many problems on actual scale-free big networks will prove to be solvable in constant time. 
Although this result is an application of the algorithms of \cite{NS_Testable_SJCOMP13},  
which is a result on bounded-degree graphs, 
HSF is not a class of bounded-degree graphs. 
This is the first result on universal algorithms for the general graph model. 




 













\subsection{Definitions}

In this paper, we consider undirected multigraphs without self-loops.  
We simply call this type of multigraph a ``graph'' in this paper 
and use  to denote it, 
where  is the vertex set and  is the edge (multi)set. 
Sometimes  and  are denoted by  and , respectively. 
Henceforth, we use ``set'' to refer to a multiset for notational simplicity. 
Throughout this paper,  is used to denote the number of vertices of 
a graph, i.e., . 





For a graph  and vertex subsets , 
  denotes the edge set between  and , i.e., 
 . 
  is also simply written as . 
 is denoted by . 
For a vertex , the number of edges incident to  is called the {\em degree} of .  
A singleton set  is often written as  for notational simplicity. 
E.g., the degree of  is represented by .   
The subscript  in the above , , etc., 
may be omitted if it is clear. 

For a vertex ,  denotes the set of vertices adjacent to , 
i.e., . 
Note that  may not be equal to  
as parallel edges may exist. 
For a graph  
and a vertex subset , 
the {\em subgraph induced by } is 
defined as . 




For a vertex subset , 
a {\em contraction} of  is defined as 
an operation to 
(i) replace  with a new vertex , 
(ii) replace each edge  in  () with 
a new edge , and 
(iii) remove all edges between vertices in . 
That is, by contracting , 
a graph  is changed to  such that 

We identify the above  with . 
In other words, we say that  remains in  (as ). 
Note that the graphs are multigraphs, and thus if there are two edges  for ,  and , 
then two parallel edges, both represented by , 
one of which corresponds to  and 
the other of which corresponds to , are added to . 
Also note that none of the graphs considered in this paper contain self-loops, and hence 
an edge  with  is removed by contracting . 




Two graphs  and  
are {\em isomorphic} if there is a one-to-one correspondence 
 
such that  
for all . 
A graph property (or property, for short) is a (possibly infinite) 
family of graphs,  
which is closed under isomorphism.  




\begin{df}[-far and -close]
Let  and  be two graphs 
with  vertices. 
Let  be the number of edges 
that need to be deleted and/or 
inserted from  in order to make it isomorphic to . 
The distance between  and  is defined as\footnote{
The distance defined here may be larger than 1 
as  may occur. 
(In the bounded-degree model 
it is defined as .) 
However, here we consider sparse graphs 
and they have an implicit upper bound of the average (not possibly maximum) degree, 
say , and thus  is bounded by . 
} 
. 
We say that 
 and  are {\em -far} if 
; 
otherwise {\em -close}. 
Let  be a non-empty property. 
The distance between  and  is 
. Otherwise we say that 
 is {\em -far} from  if , 
and {\em -close}. 
\end{df}



\begin{df}[testers]
A {\em testing algorithm} for a property  is an algorithm that, given query access 
to a graph , accepts every graph from  
with a probability of at least , and rejects every graph that is -far 
from  with probability at least . 
Oracles in the general graph model are: for any vertex , the algorithm may ask for the degree , and may ask for the th neighbor of the vertex (for ).\footnote{Although asking whether there is an edge between any two vertices is also allowed in the general graph model, the algorithms we use in this paper do not need to use this query.} 
The number of queries made by an algorithm to the given oracle is called 
the {\em query complexity} of the algorithm. 
If the query complexity of a testing algorithm is a constant, 
independent of  (but it may depend on ), 
then the algorithm is called a {\em tester}. 
A (graph) property is {\em testable} if there is a tester for the property. 
\end{df}







\begin{df}[isolated cliques~\cite{IsoClique_TALG09}]
For a graph  and a real number , 
a vertex subset  is called a {\em -isolated clique} 
if  is a clique (i.e., 
, for all  and ) 
and . 
A 1-isolated clique is sometimes called an {\em isolated clique}.
 is the graph obtained from  by contracting all isolated cliques. 
Two distinct isolated cliques never overlap, 
except in the special case of {\em double-isolated-cliques}, which consists 
of two isolated cliques with size  sharing  vertices.  
A double-isolated-clique  has no edge between  and the 
other part of the graph (i.e., ), and thus we specially define that a double-isolated-clique in  is contracted into  
a vertex in . 
Under this assumption,  is uniquely defined. 
\end{df}













\begin{df}[hyperfinite~\cite{hyperfinite}]
For real numbers  and , 
a graph  consisting of  vertices is {\em -hyperfinite} 
if one can remove at most  edges from  
and obtain a graph whose connected components have size at most . 
For the function RR, 
 is {\em -hyperfinite} 
if it is -hyperfinite for all . 
A family  of graphs is {\em -hyperfinite}  
if all  are -hyperfinite. 
A family  of graphs is {\em hyperfinite} 
if there exists a function  such that  is -hyperfinite. 
\end{df}

Hyperfinite is a large class, as it is known that any minor-closed property is hyperfinite
in a bounded-degree model. From the viewpoint of testing, the importance of hyperfiniteness stems from the following result. 



\begin{thm}[\cite{NS_Testable_SJCOMP13}]
For the bounded-degree model, any property is testable for any class of hyperfinite graphs. 
\end{thm}






This result is very strong, but there is a problem in that the result works on bounded-degree graphs and it is natural to consider that actual scale-free networks do not have a degree bound.  



\subsection{Our contribution and related work}

In this paper, we apply the universal algorithm of \cite{NS_Testable_SJCOMP13} to scale-free networks. 
We formalize two natural classes, 
 and 
that represent scale-free networks\footnote{
 was introduced in the preliminary version of this paper \cite{ScalefreeTest_itohiro15_arXiv}. 
However, the definition in this paper is more general (wider) than in the preliminary version. 
}. 
The latter is a subclass of the former. 






\begin{df}\label{df:DSF}
For positive real numbers  and , 
a class of {\em scale-free graphs (SF)} 

consists of (multi)graphs  for which the following condition holds: 
\begin{itemize}
\item[(i)] Let  be the number of vertices  with . Then: 

\end{itemize}
\end{df}



The above property (i) is generally called a power-law and in many actual scale-free networks, it is said that ~\cite{AlbertBarabasi_SF_02}. 
That is,  is a class of multigraphs that obey the power-law degree distribution. 



We show that this class is -close to a bounded-degree class if  (Lemma~\ref{lm:degreebound}). 

After showing this property, we show the hyperfiniteness of the class. 
Hyperfiniteness seems to be closely related to a high clustering coefficient, where 
the cluster coefficient  of a graph  
is defined as\footnote{
There is another way to define the cluster coefficient: 
. 
Although these two values are different generally, they are close under the assumption of the power-law degree distribution.  
}:


Sometimes  is called the {\em local cluster coefficient} of . 
It is said that  is  for 
many classes that model actual social networks, while 
 for random graphs. 

These three characterizations, 
``high clustering coefficient,'' 
``existence of isolated cliques,'' 
and 
``hyperfiniteness''  
appear to be closely related to each other.  
In fact, it is readily observed that 
if  for a bounded-degree graph  (the degree bound is ), 
then  consists of only (completely) isolated cliques with size at most , and 
 is -hyperfinite! 

Unfortunately, however, 
it is also observed that 
for any , 
there is a class of bounded-degree graphs  
such that  and 
it is not -hyperfinite 
for any pair of constants  and , e.g., 
 
consists of 
 cliques of size , 
and random  edges between 
vertices in different cliques (each vertex has  adjacent vertices in its clique and one adjacent vertex outside the clique).  
To separate this graph into constant-sized connected components, almost all of the edges between cliques (their number is ) must be removed. 

However, we do not need to give up here, as the above model does not look like a natural model of 
scale-free networks, e.g., by contracting each isolated clique, it becomes a mere random graph with  vertices.  
From this fact, the hierarchical structure of a high cluster coefficient looks important. 
The model presented by \cite{Uno-Watanabe_ScaleFree} has such a structure.  
Based on this model, we present the following class of multigraphs: 







\begin{df}[Hierarchical Scale-Free Graphs]\label{df:HSF}
For positive real numbers  and a positive integer , 
a class of {\em hierarchical scale-free graphs (HSF)} 
consists of (multi)graphs  
for which the following conditions hold: 
\begin{itemize}
\item[(i)] 
\item[(ii)] Consider the infinite sequence of graphs 
, , , .   
If , then  includes at least one isolated clique 
 with . 
(Note that if  has no such isolated clique, then .)
\end{itemize}
\end{df}











We show the following results. 


\begin{thm}\label{th:hyperfinite}
For any 
 
with  
and any real number , 
there is a real number  
such that 
 is 
-hyperfinite. 
\end{thm}


We give a global algorithm for obtaining the partition realizing the hyperfiniteness of Theorem~\ref{th:hyperfinite}.  
The algorithm is deterministic, i.e., 
if a graph and the parameter  are fixed, 
then the partition is also fixed. 
The algorithm can be easily revised to a local algorithm and we obtain a deterministic partitioning oracle to get the partition (Lamma~\ref{lm:PO}).  
Note that all known partitioning algorithms
are randomized algorithms.  
By using this partitioning oracle and an argument similar to one used in \cite{NS_Testable_SJCOMP13}, we get the following main theorem. 


\begin{thm}\label{th:testable}
Any property is testable 
for  
with . 
\end{thm}



\paragraph{Related work}
As stated earlier, for the bounded-degree model, Newman and Sohler~\cite{NS_Testable_SJCOMP13} presented a universal tester (which can test any property) 
for hyperfinite graphs.  
In the general graph model, there exist far fewer results than for the bounded-degree graph model and the dense graph model. 
For universal-type sublinear-time algorithms, 
Kusumoto and Yoshida~\cite{Kusumoto-Yoshida-ICALP14} gave 
a testing algorithm with  query complexity for 
forest-isomorphism. In the same paper, they showed an  lower bound for this problem.  

This paper gives a universal tester that can test every property on a natural class of scale-free multigraphs in constant time. 
This is the first result for universal constant-time algorithms on sparse and degree-{\em un}bounded graphs. 






\section{Hyperfiniteness and a Global Partitioning Algorithm}


\subsection{Degree bounding}



For a graph  and a nonnegative integer , 
 is a graph made by deleting all edges incident to each vertex  with  from . 
Note that  is a bounded-degree graph with degree bound . 

\begin{lm}\label{lm:degreebound}
For any  with , 
and any positive real number , 
there is a constant 
 
such that for any graph , 
 is -close to . 
\end{lm}


Before showing a proof of this lemma, we introduce some definitions. 
Riemann zeta function is defined by
. 
This function is known to converge to a constant () 
for any . 
We introduce a generalization of this function by using a positive integer  as 
. 
Note that . 




\begin{lm}\label{lm:converge_zeta}
For any  and , there is an integer 
 
such that . 
\end{lm}

\noindent
{\em Proof}: 
It is clear from the above fact that  converges for every . \qed\\

\noindent
{\em Proof of Lemma~\ref{lm:degreebound}}: 
Let  be an arbitrary positive integer. 
Let  be the number of removed edges to make  from . 
From (\ref{eq:DSF}), 

From the assumption of  and Lemma~\ref{lm:converge_zeta}, 
 if 
. 
Thus by letting 
, 
we have . \qed\\

From here, we denote the above  by  for notational simplicity. 





















\subsection{Hierarchical contraction, structure tree, and coloring}


Let  (, ) 
be a family of subsets of vertices satisfying that 
 for every  and , 
and . 
Then  is called 
a {\em partition} of . 
Below, we explain a global algorithm for obtaining a partition of  realizing the hyperfiniteness 
of a graph in  with , i.e., 
 is bounded by a constant and the number of edges between different  and  is, 
at most, . 
First, we give a base algorithm. \\




\noindent
\textbf{procedure} {\sc HierarchicalContraction}()\\
\textbf{begin}\\
1 \hspace{1em} , \\
2 \hspace{1em} \textbf{while} 
there exists an isolated clique in  \textbf{do}\\
3 \hspace{2em} , \\
4 \hspace{1em} \textbf{enddo}\\
\textbf{end.}\\


We denote  for . 
Let  be the final graph of {\sc HierarchicalContraction}(). 
From the definitions of HSF, 
. 
See Fig.~\ref{fg:fig-hierarch_contractions}~(a)--(c) for 
an example of applying this procedure. 
\begin{figure}[bht]
\begin{center}
\includegraphics[scale=1]{fig-hierarch_contractions.eps}
\end{center}
\caption{An example of {\sc HierarchicalContraction}, the structure tree , and the coloring: 
Here, we assume ; 
the number beside a vertex is ; 
the dotted circles are isolated cliques; 
colored areas are blue or yellow components.}
\label{fg:fig-hierarch_contractions}
\end{figure}


The trail of the contraction can be represented by a rooted tree , 
which is called the {\em structure tree} of ,  
defined as follows.  
(Fig.~\ref{fg:fig-hierarch_contractions}~(d) shows 
an example of the structure tree\footnote{
In this example, we ignore  and the red vertices. Some may feel it curious that if , then  for small , 
and hence many vertices in this example become red. 
This is correct. However,  becomes larger, step by step, and thus 
if  is very large, then  may be larger 
than , by only contracting 
vertices whose degree is at most . 
}.)


, 
where  is the (artificial) root of . 
Each  is a leaf of , and 
a vertex  () is on the level  of , 
i.e., 
 () is the parent of  
if `` is made by contracting a subset (clique or a double-isolated-clique) 
 such that '' or  
`` (i.e.,  is not included in an isolated clique in ).''
The root  is the parent of every vertex in . 
(The reason  is added is only to make  a tree.) 



We introduce a function 
and coloring on the vertices in  
as follows: 
\begin{itemize}
  \item For : 
  \begin{itemize}
    \item , and 
    \item if , then  is colored {\em red}, otherwise uncolored. 
  \end{itemize}
  \item For  (): 
  \begin{itemize}
      \item let  be the set of uncolored children of , 
      \item , and 
      \item if , then  is colored {\em blue}, 
      \item else if  and , then  is colored {\em yellow}, 
      \item otherwise,  is uncolored. 
  \end{itemize}
\end{itemize}









Note that for any two distinct colored vertices , 
. 
For every , we also define a weight function as 
. 
For a blue (resp. yellow) colored vertex , 
 is called a {\em blue (resp. yellow) component}. 


By using these colors, 
we also color the edges in  in the following manner: 
\begin{itemize}
\item For every red vertex , all edges in  are colored {\em red}. 
\item For every blue component , for every edge ,  
if  is not colored red, then  is colored {\em blue}. 
\item For every yellow component , for every edge ,  
if  is not colored either red or blue, 
then  is colored {\em yellow}. 
\end{itemize}







The other edges in  are uncolored. 
The set of red, blue, and yellow edges in  
are represented by , , and , respectively. 
These colors are preserved in 
, , , , 
e.g., if an edge  is red, then the corresponding edge in  is also red. 



\subsection{Proof of Theorem~\ref{th:hyperfinite}}

Before showing the proof of Theorem~\ref{th:hyperfinite}, we prepare some lemmas. 


\begin{lm}\label{lm:high_degree_is_red}
For any  (), 
all edges incident to a vertex with a degree higher than  are red.  
\end{lm}


\noindent
{\em Proof}: 
For , the statement clearly holds from the coloring rule. 
Assume that the statement holds in , and 
does not hold in some . 
Let  be a vertex in  such that  
and a non-red edge is incident to . 
Then  must be made by contracting an 
isolated clique  in , say ,  
such that . 
From the definition of isolated cliques, 
. 
Since  is a clique, every vertex  
has degree at least  in . 
It follows that all edges incident to a vertex in  must be red. 
This contradicts the assumption that a non-red edge is incident to . \qed



\begin{lm}\label{lm:number_of_colored_edges}
, .
\end{lm}




\noindent
{\em Proof}: 
 
is directly obtained from 
Lemma~\ref{lm:degreebound}. 
Let  be a blue vertex 
such that a non-red edge exists in . 
From Lemma~\ref{lm:high_degree_is_red}, 
. 
Thus . 
This means that the average number of blue edges per a vertex is 
less than  . 
Therefore . 
From Lemma~\ref{lm:high_degree_is_red}, 
all edges incident to a vertex with degree higher than  
are red. 
From this it follows that  
the number of non-red edges in  is at most  
. 
Thus the number of yellow edges in  is also at most  . 
By considering , 
we have 
. \qed\\


Let , ,  be the red vertices 
( is the number of red vertices).
Let , ,  be the blue components 
( is the number of blue components). 
Let , ,  be the yellow components 
( is the number of yellow components). 
We consider a family of vertex subsets as 

From the definition of the function  and the coloring. 
 is clearly a partition of . 




Now we can prove Theorem~\ref{th:hyperfinite}. \\



\noindent
{\em Proof of Theorem~\ref{th:hyperfinite}}: 
If , 
then the statement is clear by setting 
. 
Thus, we assume that 
. 
Let  be a graph obtained by deleting all red, blue, and yellow edges from . 
From Lemma~\ref{lm:number_of_colored_edges}, 
the number of deleted edges is less than 


Next, we will show that 
the maximum size of connected components in  is at most 
. 
Assume that there exists a connected component   
consisting of more than  vertices in . 
 includes no vertex  with , since  
from 
Lemma~\ref{lm:high_degree_is_red} 
all edges in  are colored red. 
Moreover, there is no blue component  such that 
 and , 
as otherwise  would be disconnected in  (by deleting blue edges). 

From this it follows that there is a blue or yellow component  including . 
Let  be the (blue or yellow) vertex 
such that . 
If  is a yellow vertex, 
then  
(as otherwise  would be colored blue), and , which is a contradiction.  
Thus  must be a blue vertex. 
Assume that . 
Let  be the set of children of  (in ). 
 consists of an isolated clique or 
a double-isolated-clique in . 
For every vertex ,   
(from Lemma~\ref{lm:high_degree_is_red}). 
Thus . 

Let  be the set of uncolored children of . 
For , . 
Hence, 
 
which is a contradiction. 
Therefore, the maximum size of connected components in  is . 




Thus, we have proved that  is 
()-hyperfinite. 
Here,  is an arbitrary real number in , 
then by defining 
, 
 is 
()-hyperfinite 
for any . \qed\\











\section{Testing Algorithm}



\subsection{Deterministic partitioning oracle}


The global partitioning algorithm of Theorem~\ref{th:hyperfinite} 
can be easily revised to run locally, 
i.e., a ``partitioning oracle'' based on this algorithm can be obtained. 
A partitioning oracle, which calculates a partition realizing hyperfiniteness locally, 
was introduced by Benjamini, et al.~\cite{BSS_MC-testable_STOC08} 
implicitly and by Hassidim, et al.~\cite{HKNO_LocalPartition_FOCS09} explicitly. 
It is a powerful tool for constructing constant-time algorithms 
for sparse graphs. It has been revised by some researchers and 
Levi and Ron's algorithm~\cite{Levi-Ron_PO_ICALP13} is the fastest to date. 
All algorithms for partitioning oracles presented to date have been randomized algorithms. 
Our algorithm, however, does not use any random valuable 
and it runs deterministically. 
That is, we call it a {\em deterministic partitioning oracle}, 
which is rigorously defined as follows\footnote{
However, since Levi and Ron's algorithm~\cite{Levi-Ron_PO_ICALP13} 
looks fast, using it may be better in practice. 
}: 

\begin{df}
 is a deterministic -partitioning oracle for 
a class of graphs , if, 
given query access to a graph , 
it provides query access to a partition  of . 
For a query about , 
 returns 
. 
The partition has the following properties: 
(i)  is a function of , , and . 
(It does not depend on the order of queries to .)
(ii) For every , 
 and 
 induces a connected subgraph of . 
(iii) If , then 
. 
\end{df}





\begin{lm}\label{lm:PO}
There is a deterministic -partitioning oracle  for HSF with  with query complexity  for one query. 
\end{lm}


Before giving a proof of this lemma, we introduce some notation as follows. 
A connected graph  with a specified marked vertex  is called a
{\em rooted graph}, and we sometimes say that  is rooted at .  
A rooted graph  has a radius , 
if every vertex in  has a distance at most  from the root . 
Two rooted graphs are isomorphic if there is a graph isomorphism between these graphs 
that identifies the roots with each other. 
We denote by  the number of all non-isomorphic 
rooted graphs with a maximum degree of  and a maximum
radius of . 
For a graph , integers  and , 
and a vertex , let  be the subgraph  
rooted at  that is induced by all vertices 
of  that are at distance  or less from . 
 is called a {\em -disk} around .  
From these definitions, the number of possible non-isomorphic 
-disks is at most . \\




\noindent
{\em Proof of Lemma~\ref{lm:PO}}: 
The global algorithm of Theorem~\ref{th:hyperfinite} 
can be easily simulated locally. 
To find , 
if , then the algorithm outputs . 
Otherwise, if the algorithm finds a vertex  with  
in the process of the local search, 
 is ignored (the algorithm does not check the neighbors of ). 
Thus, the algorithm behaves as on the bounded-degree model. 
For any vertex , 
. 
Each  
may be included in 
 of . 
Then, the algorithm checks most vertices in 
, 
and thus the query complexity for one call of  is at most 
. \qed\\



\subsection{Abstract of the algorithm}


The method of constructing a testing algorithm based on the partitioning oracle of Lemma~\ref{lm:PO} is almost the same as the method used in \cite{NS_Testable_SJCOMP13}. 
We use a distribution vector, which will be defined in Definition~\ref{df:dist_vector}, 
of rooted subgraphs consisting of at most a constant number of vertices. 



\begin{df}\label{df:dist_vector}
For a graph  
and integers  and , 
let  be the 
distribution vector of all -disks of , 
i.e., 
 
is a vector of dimension . 
Each entry of  
corresponds to some fixed rooted graph , 
and counts the number of 
-disks of  
that are isomorphic to . 
Note that 
 has  different disks, 
thus the sum of entries in 
 
is . 
Let  
be the normalized distribution, 
namely 
. 
\end{df}


For a vector , 
its -norm is 
. 
The -norm is also the length of the vector. 
We say that the two unit-length vectors  and  are 
-close for  
if . 



By using a discussion 
that is the same as in Theorem~3.1 in 
\cite{NS_Testable_SJCOMP13}, 
the following lemma is proved. 

\begin{lm}\label{lm:iff}
There exist functions 
, 
, 
, 
\and 
 
such that for every  the following holds: 
For every  on 
 vertices, 
if 
, 
 then 
  and  are -close. \qed
\end{lm}






A sketch of the algorithm is as follows. 
Let  be a given graph and 
 be a property to test. 
First, we select some (constant) number  
of vertices  () and 
find  given by Theorem~\ref{th:hyperfinite}. 
This is done locally (shown by Lemma~\ref{lm:PO}).
Consider a graph 
. 
Here,  and  
are very close with high provability. 
Next, we calculate  approximately. 
There is a problem in that the number of graphs in  is generally 
infinite. However, to approximate it with a small error is 
adequate for our objective, and thus it is sufficient to compare  with 
a constant number of vectors of . 
(Note that calculating such a set of frequency vectors requires much time. 
However, we can say that there exists such a set.  
This means that the existence of the algorithm is assured.) 
The algorithm accepts  
if the approximate distance of 

is small enough, and otherwise it is rejected. 

The above algorithm is the same as the algorithm 
presented in \cite{NS_Testable_SJCOMP13} 
except for two points: in our model: 
(1)  is not a bounded-degree graph, and 
(2)  is a multigraph. 
However, these differences are trivial. 
For the first difference, it is enough to add an ignoring-large-degree-vertex process, 
i.e., if the algorithm find a vertex  having a degree 
larger than , 
all edges incident to  are ignored.  
By adding this process, 
 is regarded as . 
This modification does not effect the result 
by Lemma~\ref{lm:degreebound}. 
For the second difference, 
the algorithm treats bounded-degree graphs as mentioned above, 
and the number of non-isomorphic multigraphs 
with  vertices and degree upper bound 
 is finite 
(bounded by 
). \\



\noindent
{\em Proof of Theorem~\ref{th:testable}}: 
Obtained from the above discussion. \qed



\section{Summary and future work}

We presented a natural class of multigraphs 
representing scale-free networks, and we showed that a very wide subclass of it is hyperfinite (Theorem~\ref{th:hyperfinite}). 
By using this result, the useful result that every property is testable on the class (Theorem~\ref{th:testable}) is obtained. 


 is a class of multigraphs based on the hierarchical structure of isolated cliques.  
We may relax ``isolated cliques'' to ``isolated dense subgraphs~\cite{IsoClique_TALG09}''
and we may introduce a wider class.  
We consider such classes also to be hyperfinite.  
Finding such classes and proving their hyperfiniteness is important future work.  

 
 




























\section*{Acknowledgements}
We are grateful to Associate Professor Yuichi Yoshida of 
the National Institute of Informatics for his valuable suggestions. 
We also appreciate the fruitful discussions with Professor Osamu Watanabe of the Tokyo Institute of Technology and Associate Professor Yushi Uno of Osaka Prefecture University. 
We also would like to thank the Algorithms on Big Data project (ABD14) of CREST, JST, 
the ELC project (MEXT KAKENHI Grant Number 24106003), and JSPS KAKENHI Grant Numbers 24650006 and 15K11985 through which this work was partially supported. 










\begin{thebibliography}{99}
\bibitem{AlbertBarabasi_SF_02}
R. Albert and A. -L. Barab\'asi: 
Statistical mechanics of complex networks, 
Review of Modern Physics, Vol.~74, 2002, pp.~47--97.
\bibitem{AlonFNS_Testable_Regularity_SIAMJC09}
N. Alon, E. Fischer, I. Newman, and A. Shapira:
A combinatorial characterization of the testable graph properties: it's all about regularity, 
SIAM J. Comput., Vol.~39, No.~1, 2009, pp.~143--167. 
\bibitem{BSS_MC-testable_STOC08}
I. Benjamini, O. Schramm, and A. Shapira: 
Every minor-closed property of sparse
graphs is testable, Proc. STOC 2008, ACM, 2008, pp.~393--402. 
\bibitem{BM_Kronecker_IPL98}
A. Bottreau, Y. M\'etivier: 
Some remarks on the Kronecker product of graphs, 
Information Processing Letters, Vol. 68, Elsevier, 1998, pp.~55--61. 
\bibitem{Broder_etal-GSinWeb-CN00}
A.Z. Broder, S.R. Kumar, F. Maghoul, R. Raghavan, S. Rajagoplalan, 
R.~Stata, A.~Tomkins and J.L.~Wiener: 
Graph structure in the Web, 
Computer Networks, Vol.~33, 2000, pp.309--320. 
\bibitem{CU_SF_Clique_10}
C. Cooper and R. Uehara: 
Scale free properties of random -trees; 
Mathematics in Computer Science, Vol.~3, 2010, pp.~489--496. 
\bibitem{hyperfinite}
G.~Elek: 
-spectral invariants and convergent sequence of finite graphs, 
Journal of Functional Analysis, Vol.~254, No.~10, 2008, pp.~2667--2689.
\bibitem{Gao_SF_clique_TCS09}
Y. Gao: 
The degree distribution of random -trees, 
Theoretical Computer Science, Vol.~410, 2009, pp.~688--695. 
\bibitem{PropertyTestingLNCS10}
O. Goldreich (Ed.): 
Property Testing --- Current Research and Surveys, 
LNSC 6390, 2010. 
\bibitem{GR-STOC97}
O. Goldreich and D. Ron: 
Property testing in bounded
degree graphs: Proc. STOC 1997, 1997, pp.~406--415.
\bibitem{GGR-JACM98}
O. Goldreich, S.~Goldwasser, and D.~Ron: 
Property testing and its connection to learning and approximation: 
Journal of the ACM, Vol.~45, No.~4, July, 1998, pp.~653--750. 
\bibitem{HKNO_LocalPartition_FOCS09}
A.~Hassidim, J.~A.~Kelner, H.~N.~Nguyen, and K.~Onak:
Local graph partitions for approximation and testing, 
Proc. FOCS 2009, IEEE,  pp.~22--31.
\bibitem{ScalefreeTest_itohiro15_arXiv}
H.~Ito, 
Every property is testable on a natural class of scale-free multigraphs, 
arXiv: 1504.00766, Cornell University, April 6, 2015.
\bibitem{IsoClique_TALG09}
H.~Ito and K.~Iwama, 
Enumeration of isolated cliques and pseudo-cliques, 
ACM Transactions on Algorithms, Vol.~5, Issue~4, 
Oct. 2009, Article 40 (pp.~1--13). 
\bibitem{IsoClique_ESA05}
H.~Ito and K.~Iwama, and T.~Osumi: 
Linear-time enumeration of isolated cliques, 
Proc. ESA2005, LNCS, 3669, Springer, 2005, pp. 119--130.
\bibitem{KleinbergLawrence-web-science01}
J. Kleinberg and S. Lawrence: 
The structure of the Web, 
Science, Vol.~294, 2001, pp.~1894--1895.
\bibitem{Kusumoto-Yoshida-ICALP14}
M.~Kusumoto and Y.~Yoshida: 
Testing forest-isomorphizm in the adjacency list model, 
Proc. of ICALP2014~(1), 
LNSC 8572, 2014, pp.~763--774. 
\bibitem{Levi-Ron_PO_ICALP13}
R.~Levi and D.~Ron: 
A quasi-polynomial time partition oracle
for graphs with an excluded minor, 
Proc. ICALP 2013 (1), LNCS, 7965, Springer, 2013, 
pp.~709--720. 
(Journal version: R.~Levi and D.~Ron: 
A quasi-polynomial time partition oracle
for graphs with an excluded minor, 
ACM Transactions on Algorithms, 
Vol.~11, No.~3, 2014, Article~24 (pp.~1--13).)
\bibitem{StochKronecherG_WAW07}
M. Mahdian and Y. Xu: 
Stochastic Kronecker graphs, 
Proc. WAW 2007, LNCS, 4863, Springer, 2007, pp.~179--186. 
\bibitem{NS_Testable_SJCOMP13}
I. Newman and C. Sohler: 
Every property of hyperfinite graphs is testable, 
Proc. STOC 2011, ACM, 2011, pp.~675--784. 
(Journal version: 
I. Newman and C. Sohler: 
Every property of hyperfinite graphs is testable, 
SIAM~J.~Comput., Vol.~42, No.~3, 2013, pp.~1095--1112.)
\bibitem{Newman_SF_03}
M. E. J. Newman: The structure and function of complex networks, 
SIAM Review, Vol.~45, 2003, pp.~167--256.
\bibitem{Uno-Watanabe_ScaleFree}
T. Shigezumi, Y. Uno, and O. Watanabe: 
A new model for a scale-free hierarchical structure of isolated cliques, 
Journal of Graph Algorithms and Applications, Vol.~15, No.~5, 2011, pp.~661--682. 
\bibitem{WattsStrogatz-Nature98}
D.J. Watts and S.H. Strogatz: 
Collective dynamics of 'small-world' networks, 
Nature, Vol.~393, 1998, pp.~440--442.
\bibitem{ZRC_SF_clique_06} 
Z. Zhang, L. Rong, and F. Comellas: 
High-dimensional random apollonian networks, 
Physica A: Statistical Mechanics and its Applications, 
Vol.~364, 2006, 610--618.
\end{thebibliography}






\end{document}
