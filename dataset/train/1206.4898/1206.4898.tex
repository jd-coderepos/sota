\newif\ifconf
\conffalse 

\ifconf
\documentclass[envcountsect]{llncs}
\else
\documentclass[12pt,envcountsect,runningheads]{llncs}
\usepackage{fullpage}
\fi

\usepackage{amssymb,amsmath}
\usepackage{graphicx}
\usepackage{subfigure}
\usepackage{graphics}



\newcommand{\diam}{\mathsf{diam}}
\newcommand{\depth}{\mathsf{depth}}
\newcommand{\trees}{\mathsf{Trees}}
\newcommand{\planar}{\mathsf{Planar}}
\newcommand{\len}{\mathsf{len}}
\newcommand{\genus}{\mathsf{genus}}
\newcommand{\conflict}{\mathsf{conflict}}
\newcommand{\dmax}{d_{\max}}

\DeclareMathOperator*{\argmin}{argmin}
\DeclareMathOperator*{\mymod}{mod}

\renewcommand{\phi}{\varphi}
\newcommand{\eps}{\varepsilon}

\setcounter{tocdepth}{3}

\title{Planarizing an Unknown Surface}
\author{Yury Makarychev\thanks{Yury Makarychev is supported in part by the NSF Career Award CCF-1150062.}
\and Anastasios Sidiropoulos}
\institute{Toyota Technological Institute at Chicago\\
\url{yury@ttic.edu, tasos@ttic.edu}}




\begin{document}
\maketitle

\begin{abstract}
It has been recently shown that any graph of genus  can be stochastically embedded into a distribution over planar graphs, with distortion  [Sidiropoulos, FOCS 2010]. This embedding can be computed in polynomial time, provided that a drawing of the input graph into a genus- surface is given.

We show how to compute the above embedding without having such a drawing.
This implies a general reduction for solving problems on graphs of small genus, even when the drawing into a small genus surface is unknown.
To the best of our knowledge, this is the first result of this type.
\end{abstract}



\section{Introduction}

The genus of a graph is a parameter that quantifies how far it is from being planar.
Informally, a graph has genus , for some
, if it can be drawn without any crossings on the surface of
a sphere with  additional handles (see Section~\ref{sec:prelims}).  For example, a planar graph has genus , and a
graph that can be drawn on a torus has genus at most .

Planar graphs exhibit properties that give rise to improved algorithmic solutions for numerous problems (see, for example \cite{Baker-planar}).
Because of their similarities to planar graphs, graphs
of small genus enjoy similar algorithmic characteristic.
More precisely, algorithms for planar graphs can usually be extended to graphs of bounded genus, with a small loss in efficiency or quality of the solution (e.g.~\cite{CEN09}).

Unfortunately, such extensions typically suffer from two main difficulties.
First, for different problems, one typically needs to develop complicated, and ad-hoc techniques.
Second, a perhaps more challenging issue is that essentially all known algorithms for graphs of small genus require that a drawing of the input graph into a small genus surface is given.
In general, computing a drawing of a graph into a surface of minimum genus is NP-hard \cite{Thomassen89,Thomassen93a}.
Moreover, the currently best-known approximation algorithm for this problem is only a trivial -approximation that follows by bounds on the Euler characteristic.
This has been improved to -approximation for graphs of bounded degree \cite{ChenKK97}.

The first of the above two obstacles has been recently addressed for some problems by Sidiropoulos \cite{sidiropoulos2010optimal}, who showed that any graph of genus  can be embedded into a distribution over planar graphs, with distortion  (see Section~\ref{sec:prelims} for definitions).
This result implies a general reduction for a large class of geometric optimization problems from instances on genus- graphs, to corresponding ones on planar graphs, with a  loss factor in the approximation guarantee. 

Unfortunately, the algorithm from \cite{sidiropoulos2010optimal} can compute the above embedding in polynomial time, only if a drawing of the input graph into a small genus surface is given.
We show how to compute this embedding even when the drawing of the input graph is unknown.
In particular, this implies that the above reduction for solving problems on graphs of small genus, can be performed even on graphs for which we don't have a drawing into a small genus surface.
The statement of our main embedding result follows.

\begin{theorem}[Main result]\label{tim:main}
There exists a polynomial time algorithm which given a graph  of genus , computes a stochastic embedding of  into planar graphs, with distortion .
In particular, the algorithm does not require a drawing of  as part of the input.
\end{theorem}

\iffalse
\begin{theorem}[Planarization of an unknown surface]\label{thm:main}
Any graph  of genus , admits a stochastic embedding into planar graphs, with distortion .
Moreover, given a drawing of  into a genus- surface, the embedding can be computed in polynomial time.
\end{theorem}
\fi

\subsection{Applications}

The main application of our result is a general
reduction from a class of optimization problems on
genus- graphs, to their restriction on planar graphs.
This is the same reduction obtained in \cite{sidiropoulos2010optimal},
only here we don't require a drawing of the input graph.
For completeness, we state precisely the reduction, as given in \cite{sidiropoulos2010optimal} (see also \cite{Bar96}).
Let  be a set,
 a set of non-negative vectors corresponding to all feasible solutions for a minimization problem, and .
Then, we define the \emph{linear minimization problem}  to be the computational problem where we are given a graph , and we are asked to find , minimizing


Observe that this definition captures a very general class of problems.
For example, MST can be encoded by letting  be the set of indicator vectors of the edges of all spanning trees on , and  the all-ones vector.
Similarly, one can easily encode problems such as TSP, Facility-Location, -Server, Bi-Chromatic Matching, etc.

The main Corollary of our embedding result can now be stated as follows.

\begin{corollary}\label{cor:opt}
Let  be a linear minimization problem.  If there exists a polynomial-time -approximation algorithm for  on planar graphs, then there exists a randomized polynomial-time -approximation algorithm for  on graphs of genus , even when the drawing of the input graph is unknown.
\end{corollary}


\subsection{Overview of the Algorithm}
We now give a high-level overview of our algorithm.
Consider a graph .
Let us say that a collection  of shortest paths in  is a \emph{planarizing set of paths}, if the graph  is planar.
It was shown by Sidiropoulos \cite{sidiropoulos2010optimal} that any graph having a planarizing set of paths of size , admits a stochastic embedding into planar graphs, with distortion .
Moreover, given such a set of planarizing paths, the embedding can be computed in polynomial time.
It follows by the work of Eppstein \cite{eppstein2003dynamic}, and Erickson and Whittlesey \cite{erickson2005greedy}, that for any graph  of genus , that there exists a planarizing set of paths, of size .
However, all known algorithms for computing this planarizing set require a drawing of the graph into a surface of genus .
Since we don't know how to compute a drawing of a graph into a minimum-genus surface in polynomial time, all known algorithms are not applicable in our case.

Our main technical contribution is showing how to compute in polynomial time a planarizing set of paths 
of approximately optimal size (up to a  factor) in an arbitrary graph.
For a graph , we say that a collection  of shortest paths having a common endpoint is a \emph{balanced set of paths} if  is a balanced vertex-separator of . That is, removing all paths in  from , leaves a graph where every connected component is at most half the size of .
Our high-level approach is as follows. We find and remove a ``small'' balanced set of paths in . 
Then we compute connected components in the obtained graph.
In each non-planar connected component, we again find and remove a balanced set of paths. We 
repeat this procedure until all components are planar. Finally, we output the planarizing set of paths 
that consists of all paths that we removed from the graph.

In order for this approach to work, we first prove that in a (possibly vertex-weighted) graph  of genus , 
there exists a balanced set  of paths of size .
Next, we show how to compute in polynomial time a balanced set of paths of approximately optimal size in an arbitrary graph .
As outlined above, we then recursively use this as a subroutine to find a set  of planarizing paths.
We begin with a graph  of genus  (for which we don't have a drawing into a genus- surface), and 
inductively build  in steps.
At the first step, we compute a balanced set  of paths in .
We add these paths to .
At every subsequent step , let  be the graph obtained from  after removing all the paths we have computed so far, i.e.~.
Since  has genus , graph  has at most  non-planar connected components.
For every such non-planar component, we compute a balanced set of paths and add it to .
We show that after every step, the size of the largest non-planar component reduces by at least a constant factor.
Therefore, after  steps, we obtain the desired planarizing set of paths.


\subsection{Related Work}

Inspired by Bartal's stochastic embedding of general metrics into
trees \cite{Bar96}, Indyk and Sidiropoulos~\cite{indyk_genus} showed that every metric on a graph of genus  can be stochastically embedded into a planar graph with distortion 
(see Section \ref{sec:prelims} for a formal definition of
stochastic embeddings).
The above bound was later improved by Borradaile, Lee, and Sidiropoulos \cite{BLS09}, who obtained an embedding with distortion .
Subsequently, Sidiropoulos \cite{sidiropoulos2010optimal} gave an embedding with distortion , matching the  lower bound from \cite{BLS09}.
The embeddings from \cite{indyk_genus}, and \cite{sidiropoulos2010optimal} can be computed in polynomial time, provided that the drawing of the graph into a small genus surface is given.
Computing the embedding from \cite{BLS09} requires solving an NP-hard problem, even when the drawing is given.


\subsection{Preliminaries}\label{sec:prelims}
Throughout the paper, we consider graphs
with non-negative edge lengths.
For a tree  with root , and for  we denote by  the unique path in  between  and .

\paragraph{Graphs on surfaces}
Let us recall some notions from topological graph theory (an in-depth
exposition can be found in \cite{MoharT-book}).  A \emph{surface} is a
compact connected 2-dimensional manifold, without boundary.
For a graph  we can
define a one-dimensional simplicial complex  associated with  as
follows: The -cells of  are the vertices of , and for each
edge  of , there is a -cell in  connecting  and
.  A \emph{drawing} of  on a surface  is a continuous injection
.
The \emph{genus} of a surface  is the maximum cardinality of a collection
of simple closed non-intersecting curves  in , such that  is connected.
The genus of a graph  is the minimum , such that  can be drawn into a surface of genus .
Note that a graph of genus  is a planar graph.
We remark that we make no distinction between orientable, and non-orientable genus, since all of our results hold in both settings.

\paragraph{Metric embeddings}
A mapping  between two metric spaces  and 
is {\em non-contracting} if  for all .
If  is any finite metric space, and 
is a family of finite metric spaces, we say that {\em  admits a stochastic -embedding into } if there exists a random metric space  and a random
non-contracting mapping  such that for every ,

The infimal  such that \eqref{eq:expansion} holds is the {\em distortion of
the stochastic embedding.}
A detailed exposition of  results on metric embeddings can be found in \cite{I-survey} and \cite{Matousek-book}.




\section{Path Separators in Embedded Graphs}

For a graph , a real , and a set  we say that  is an \emph{-balanced vertex separator} for  if every connected component of  contains at most  vertices.
It is also called simply balanced vertex separator, when .

For a vertex-weighted graph  with weight function , for every  we use the notation .
Similarly to the unweighted case, we say that a set  is a balanced vertex separator for a weighted graph
  if for every connected component  of  we have . 

\begin{theorem}[Lipton \& Tarjan \cite{lipton1979separator}, Thorup \cite{thorup2004compact}]\label{thm:planar_separators}
Let  be a planar graph, let , and let  be a spanning tree of  with root .
Then, there exist , such that  is a balanced vertex separator for .
Moreover, the vertices  and  can be computed in polynomial time.
\end{theorem}

We will use a slight modification of Theorem \ref{thm:planar_separators}, for the case of weighted graphs.
The proof is a straightforward extension to the one due to Thorup \cite{thorup2004compact}, which is based on the argument of Lipton and Tarjan \cite{lipton1979separator}.

\begin{lemma}\label{lem:weighted_planar_separators}
Let  be a planar graph, let , and let  be a spanning tree of  with root .
Let .
Then, there exist , such that  is a balanced vertex separator for .
Moreover, the vertices  and  can be computed in polynomial time.
\end{lemma}

The next Theorem follows by the work of Eppstein \cite{eppstein2003dynamic}, and Erickson \& Whittlesey \cite{erickson2005greedy}.

\begin{theorem}[Erickson \& Whittlesey \cite{erickson2005greedy}, Eppstein \cite{eppstein2003dynamic}]\label{thm:planarization_spanning}
Let  be a graph of genus , and let  be an embedding of  into a surface  of genus .
Let , and let  be a spanning tree of  with root .
Then, there exist edges , such that
 is planar.
Moreover, the topological space  is homeomorphic to an open disk.
\end{theorem}

We are now ready to prove the main result of this section.

\begin{lemma}[Existence of path separators in embedded graphs]\label{lem:genus_separator_weighted}
Let  be a weighted graph of genus , with weight function .
Let , and let  be a spanning tree of  with root .
Then, there exists , with , such that  is a balanced vertex separator for .
\end{lemma}
\begin{proof}
The case  follows by Lemma \ref{lem:weighted_planar_separators}, so we may assume that .
Fix an embedding  of  into a surface  of genus .
By Theorem \ref{thm:planarization_spanning} there exist , such that the topological space  is homeomorphic to an open disk.
Let

Note that .
Let  be the graph obtained from  by contracting  into a single vertex .
Since  is an open disk, it follows that  is planar.

Let  be the subgraph of  induced by  after contracting .
Since  is a spanning subgraph of , it follows that  is a spanning subgraph of .
Indeed, the set of vertices  spans is a connected subtree of .
Therefore, after contracting , the subgraph  induced by  is still a tree.
Thus,  is a spanning subtree of .
We consider  being rooted at .

Define a weight function  such that for every ,

By Lemma \ref{lem:weighted_planar_separators} it follows that there exist  such that  is a balanced vertex separator for .

Let .
Observe that .
Moreover, for any , we have .
Thus, the set of connected components of  is the same as the set of connected components of .
Let  be a connected component of .
We have

Thus,  is a balanced vertex separator for , as required.
\qed
\end{proof}

\section{Computing Path Separators in Arbitrary Graphs}
Recall the definition of a 
caterpillar decomposition of a tree.
\begin{definition}[Caterpillar decomposition \cite{matousek1999trees,charikar2002dimension}]
A \emph{caterpillar decomposition}
of a rooted tree  is a family of paths , satisfying the following conditions:
\begin{description}
\item{(i)}
Every  is a subpath of a root--leaf path.
\item{(ii)}
For every , we have .
\item{(iii)}
.
\end{description}
\end{definition}
The proof of the following lemma about caterpillar decompositions can be found in \cite{matousek1999trees,charikar2002dimension}.

\begin{lemma}[See \cite{matousek1999trees,charikar2002dimension}]\label{lem:computing_caterpillar}
For every rooted tree , there exists a caterpillar decomposition , such that every root--leaf path  crosses
at most  paths from .
Moreover, this decomposition can be found in polynomial time.
\end{lemma}

We are now ready to prove that main result of this section.

\begin{lemma}[Computing approximate path separators]\label{lem:genus_separator_weighted_approximate}
Let  be a graph, and .
Let , and let  be a spanning tree of  with root .
Suppose that there exists , such that  is a balanced vertex separator for .
Then we can compute in polynomial time a set  
with , such that  is a -balanced vertex separator for .
\end{lemma}
\begin{proof}
We reduce the problem to the problem of finding a vertex separator in an auxiliary graph.
Using Lemma \ref{lem:computing_caterpillar} we construct a caterpillar decomposition  of  such that every root--leaf path  crosses
at most  paths from . We define an auxiliary graph  on the set  as follows:
 and  are connected with an edge in  if there is an edge between sets
 and  in . We assign each  weight equal to the total weight of all 
vertices of . Note that then the total weight of all vertices in  equals .

Observe that for every  the induced graph  is connected if and only if 
the induced graph , where  , is connected. Consequently,
if  are connected components of 
(for some ) then sets  (for ) are connected 
components of  where ; moreover, the weight of each  equals
the weight of .
Therefore,  is a balanced vertex separator in  if and only if 
 is a balanced vertex separator in .

We now prove that there is a balanced vertex separator in  of size
. Let .
First, we show that  is a balanced vertex separator in .
Denote . Observe that . Indeed,
consider . Then  for some . Let  be the path in  that contains . 
Then  intersects  at vertex  and therefore . Hence .
We conclude that . Since  is a balanced vertex separator in ,
set  is also a balanced vertex separator in . Hence  is a balanced vertex separator in .
Now we upper bound the size of . Note that for every , we have
 (by Lemma~\ref{lem:computing_caterpillar}). Thus we have,
. We proved that there is a balanced vertex separator in  of size
. 

We use the algorithm of Feige, Hajiaghayi and Lee~\cite{FHL08} to find 
a  approximation for the optimal balanced vertex separator in . 
We get a -balanced vertex separator  in  of size  at most
.

Finally, we define the set . 
For every path , let  be a leaf of  such that  is a subset of . 
Let . 
Note that  .
Since  is a -balanced separator in , the set
 is a -balanced 
separator in , and therefore  
is a -balanced separator in . 
\qed
\end{proof}

\section{Computing Planarizing Sets of Paths}

\begin{lemma}[Computing a planarizing set of paths]\label{lem:computing_planarizing_paths}
Let  be an -vertex graph of genus .
Let , and let  be a spanning subtree of  with root .
Then, we can compute in polynomial time a set , with , such that the graph  is planar.
\end{lemma}
\begin{proof}
We inductively construct a sequence , for some , where for every , we have .
The resulting desired set will be .

For the basis of the induction, we set .

Let , and suppose that  has already been constructed.
We show how to construct .
Let  be the set of connected components of .
Let also  be the set of non-planar components in .
Note that  is the only component in .
For every component  we define a function  such that for every ,

By Lemma \ref{lem:genus_separator_weighted} it follows that there exists , with , such that  is a balanced vertex separator for .
Therefore, by Lemma \ref{lem:genus_separator_weighted_approximate} we can compute in polynomial time a set , with 

and such that  is an -balanced vertex separator for .
We set

This concludes the inductive construction of the sequence .

We next show that for some , the set  is as required.
Consider some , and let  be a non-planar connected component of .
Observe that there exists a connected component  such that .
Since  is non-planar, it follows that  is also non-planar, and thus .
By the construction, the set  contains the set , where  is 
a -balanced vertex separator for .
It follows that .
Thus, the size of every non-planar connected component in  is at most .
This implies in particular that for , the set  does not contain any non-planar connected components, and therefore the graph  is planar.

It remains to upper bound .
Since  has genus , we have that for every , the set  contains at most  non-planar connected components, i.e.~.
Therefore,

as required.
\qed
\end{proof}

\section{Putting Everything Together}

The next lemma follows by the work of Sidiropoulos \cite{sidiropoulos2010optimal}.

\begin{lemma}[Sidiropoulos \cite{sidiropoulos2010optimal}]\label{lem:optimal}
Let  be a graph, and .
Let  be a collection of shortest paths in , having  as a common end-point.
Suppose that  is planar.
Then,  admits a stochastic embedding into planar graphs, with distortion .
Moreover, if the paths  are given, then we can sample from the stochastic embedding in polynomial time.
\end{lemma}

\begin{theorem}[Kawarabayashi, Mohar \& Reed \cite{DBLP:conf/focs/KawarabayashiMR08}]\label{thm:genus_linear}
There exists an algorithm which given a graph  of genus , computes a drawing of  into a surface of genus , in time .
\end{theorem}


\begin{theorem}[Main result]
There exists a polynomial time algorithm which given a graph  of genus , computes a stochastic embedding of  into planar graphs, with distortion .
In particular, the algorithm does not require a drawing of  as part of the input.
\end{theorem}
\begin{proof}
We can use the algorithm of Kawarabayashi, Mohar \& Reed from Theorem \ref{thm:genus_linear} to test whether  in polynomial time.
If , then the algorithm from Theorem \ref{thm:genus_linear} returns a drawing of  into a surface of genus .
Since we have a drawing of  into a surface of genus , we can use the algorithm of Sidiropoulos \cite{sidiropoulos2010optimal}, to compute the required embedding.

Otherwise, if , we proceed as follows.
Let  be an arbitrary vertex in , and let  be a shortest-path tree in , with root .
By Lemma \ref{lem:computing_planarizing_paths} we can compute a set , with , such that the graph  is planar.
Since for every , the path  has  as an endpoint, it follows that we can use Lemma \ref{lem:optimal} with the collection of paths , to compute in polynomial time a stochastic embedding into planar graphs, with distortion , as required.
\qed
\end{proof}




\bibliography{bibfile}
\bibliographystyle{alpha}

\end{document}
