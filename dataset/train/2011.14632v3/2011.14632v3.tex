



\documentclass{svproc}


\usepackage{url}
\usepackage{biblatex}
\addbibresource{references.bib}

\usepackage[pdftex]{graphicx}
\usepackage{tabularx,booktabs}
\usepackage[toc,page]{appendix}
\usepackage{multirow}
\usepackage{multicol}
\usepackage{graphicx}
\usepackage[dvipsnames]{xcolor}
\usepackage{import}
\usepackage{amsfonts}
\usepackage[algoruled,linesnumbered]{algorithm2e}
\usepackage{subcaption}


\usepackage{amsmath}
\newcommand{\cmt}[1]{\textcolor{red}{#1}}

\newcommand{\argmin}{\mathop{\mathrm{argmin}}}
\newcommand{\argmax}{\mathop{\mathrm{argmax}}}
\renewcommand{\thefootnote}{\fnsymbol{footnote}}

\usepackage{tikz}
\usetikzlibrary{arrows,positioning} 
\tikzset{
>=stealth',
punkt/.style={
           rectangle,
           rounded corners,
           draw=black, very thick,
           text width=8.5em,
           minimum height=2em,
           text centered},
pil/.style={
           ->,
           thick,
           shorten <=2pt,
           shorten >=2pt,}
}


\def\UrlFont{\rmfamily}

\begin{document}
\mainmatter              \title{Optimizing the Neural Architecture of Reinforcement Learning Agents}
\titlerunning{Optimizing the Neural Architecture of Reinforcement Learning Agents
}  \author{N. Mazyavkina\thanks{equal contribution}, S. Moustafa\footnotemark[1], I. Trofimov, E. Burnaev}
\authorrunning{N. Mazyavkina, S. Moustafa, I. Trofimov, E. Burnaev} 

\tocauthor{Nina Mazyavkina, Samir Moustafa, Ilya Trofimov, Evgeny Burnaev}
\institute{Skolkovo Institute of Science and Technology, Moscow, Russia,\\
\email{[n.mazyavkina, samir.mohamed, ilya.trofimov, e.burnaev]@skoltech.ru}}

\maketitle              

\begin{abstract}
Reinforcement learning (RL) enjoyed significant progress over the last years. One of the most important steps forward was the wide application of neural networks. However, architectures of these neural networks are quite simple and typically are constructed manually. In this work, we study recently proposed \textit{neural architecture search (NAS)} methods for optimizing the architecture of RL agents. We create two search spaces for the neural architectures and test two NAS methods: Efficient Neural Architecture Search (ENAS) and Single-Path One-Shot (SPOS). Next, we carry out experiments on the Atari benchmark and conclude that modern NAS methods find architectures of RL agents outperforming a manually selected one. 





\keywords{AutoML, Neural Architecture Search, Reinforcement Learning, Atari} 
\end{abstract}

\section{Introduction}
\import{}{Nina/ninaIntro.tex}

Early NAS methods \cite{zoph2016neural, zoph2018learning} required training of numerous neural architectures. However, even training of one RL agent takes a significant amount of time. 
In this paper, we limit ourselves to fast \textit{one-shot} methods, which perform architecture search in the time not significantly larger than the training time of a single neural network. 
Most of the NAS methods were developed for computer vision applications, the major part -- for the object classification problem. 
At the same time, reinforcement learning is quite a different problem. The performance of an RL agent, that is, average reward, is not differentiable like the cross-entropy of object classification. 
Thus, only few popular NAS methods are suited for RL. In this work, we evaluate ENAS \cite{zoph2018learning} and SPOS \cite{guo2019single}.

The contribution of our paper is the following: we experimentally prove that modern one-shot NAS methods can be successfully applied for optimizing the neural architecture of RL agents. 
The source code is publicly available from \textit{\url{https://github.com/NinaMaz/NAS_RL_torch}}.


\section{Related work}
Early neural architecture search (NAS) approaches
treated this problem as a black-box optimization, that is, search over a discrete domain of architectures. Such methods are quite general but require training of numerous architectures and vast computational resources. One of the first proposed methods of this kind \cite{zoph2016neural, zoph2018learning} used reinforcement learning for the optimization process itself. Architecture creation, layer by layer, was done by an RL agent. Thus, the reward was the performance of the constructed network.
Other works proposed evolutionary optimization \cite{real2019regularized}, bayesian optimization based on Gaussian processes \cite{kandasamy2018neural}, bayesian performance predictors based on architecture features \cite{white2019bananas, shi2019multi}. Black-box optimization enjoy speedup from multi-fidelity methods \cite{trofimov2020multi}. Several benchmarks for NAS were developed \cite{ying2019bench, klyuchnikov2020bench, dong2020bench}.

The later family of methods -- \textit{one-shot NAS} -- gone beyond black-box optimization and utilized the structure a neural network. These methods involve the \textit{supernetwork}, which contain all the architectures from the search space as its subnetworks. Thus, all the architectures share weights of some of the blocks.
The architecture search in the supernetwork is performed simultaneously with the training of networks themselves. 
The one-shot methods are: ENAS \cite{pham2018efficient}, numerous modifications of DARTS \cite{liu2018darts, xu2019pc, chen2019progressive, liang2019darts+, dong2019searching, cai2018proxylessnas}, single path one-shot \cite{guo2019single}, random search with weight sharing \cite{li2019random, bender2019understanding}.

Most of the existing research focuses on problems from computer vision and linguistics.
There are no papers about applications of modern NAS methods to RL to the best of our knowledge.



\section{Reinforcement Learning Methods}
\import{}{Nina/ninaMethods_RL.tex}

\section{Neural Architecture Search Methods}

\begin{figure}
 \centering 
 \includegraphics[width=0.5\textwidth]{./images/single_path.pdf}
  \caption{Supernetwork architecture for a generic design. Each block can be a part of neural network, and each connection between sequence of blocks can be a complete architecture. The \textcolor{blue}{blue} path represent a complete architecture.
  }
  \label{fig:supernet_single}
\end{figure}

\subsection{Adaptation to Reinforcement Learning}
\label{sec:adapation}
Most of the existing research on NAS is dedicated to computer vision (particularly, object classification) or computational linguistics applications. We adapt the existing one-shot NAS methods to reinforcement learning. One-shot methods assume the \textit{supernetwork} which contain all the architectures from the search space as its subgraphs (fig. \ref{fig:supernet_single}). All the architectures share weights of some of the blocks. 
That is, each layer in the supernetwork $i$ has a list of $Block_{i}$ options, only one option can be selected for a particular subnetwork. Such simplification was proposed to reduce the co-adaptation between blocks. The whole search space is $\mathcal{A} = Block_0 \times \ldots \times Block_n$. Some initial or final layers of the supernetwork can be fixed and not contain choice blocks.

One-shot methods typically contain a \textit{fitting} stage. During the \textit{fitting} stage, the subnetwork $\alpha \in \mathcal{A}$ is sampled by some rule and its weights $\Theta(\alpha)$ are updated by a SGD-like step for a batch of data $B$
\begin{equation}
\label{eq:cv-sgd}
\Theta(\alpha) \gets \Theta(\alpha) - \eta \nabla_{\Theta(\alpha)} \sum_{i\in B} \ell(y_i, N(\alpha, \Theta(\alpha), x_i)),
\end{equation}
where $\ell(\cdot)$ is the loss function, $N(\alpha, \theta, x_i)$ is the network of the architecture $\alpha$ having weights $\Theta(\alpha)$.
The \textit{evaluation} of the architecture $\alpha$ typically involves calculation of performance (accuracy) on the validation dataset $D_{val}$
\begin{equation}
\label{eq:cv-acc}
\frac{1}{|D_{val}|}\sum_{i \in D_{val}}[y_i = N(\alpha, \Theta(\alpha),x_i)].
\end{equation}
We adapt one-shot methods to RL in the following way. In our experiments, the neural network $N(\alpha, \Theta(\alpha), x_i)$ corresponds to a policy $\pi_{\Theta(\alpha)}$. Instead of SGD-like step (\ref{eq:cv-sgd}) we do the step of PPO 
\begin{equation}
\Theta(\alpha) \gets \Theta(\alpha) - \eta \nabla_{\Theta(\alpha)} J^{PPO}_{\Theta(\alpha)}.
\end{equation}
The performance of the network is estimated by 
$\mathbb{E}_{t} [ R_{t}[\pi_{\Theta(\alpha)}]$
instead of (\ref{eq:cv-acc}).

\subsection{Efficient Neural Architecture Search (ENAS)}
\import{}{Nina/ninaMethods_ENAS.tex}

\subsection{Single-Path One-Shot with Uniform Sampling (SPOS)}

The method ``Single-path one-shot with uniform sampling'' was proposed in \cite{guo2019single}.
The SPOS method assumes two steps: 1) supernetwork fitting 2) best architecture selection.
The distinctive feature of SPOS is that subnetworks are sampled from the supernetwork uniformly at random.

Thus, weights $\Theta^*$ of the supernetwork are the solution of the following problem 
\[
\Theta^* = \argmin_{\Theta} \mathbb{E}_{\alpha\sim P} [ L_{train}(N(\alpha, \Theta(\alpha)))], 
\]
where $L_{train}(\cdot)$ is the train loss, $P$ - the uniform distribution.
During the architecture selection phase, the best subnetwork $\alpha^*$ is selected by the validation accuracy $Acc_{val}$
\begin{equation}
\label{eq:opt-val}
\alpha^* = \argmax_{\alpha \in \mathcal{A}} Acc_{val}(N(\alpha, \Theta^*(\alpha))).
\end{equation}
This step requires only inference for the validation data. 
In the original paper \cite{guo2019single}, an evolutionary optimization was used to solve (\ref{eq:opt-val}) since the search space was huge. Instead of evolutionary optimization, we do the full search since our search spaces are small.

The adaptation of SPOS to RL is done as described in the Section \ref{sec:adapation}. The subnetwork $N(\alpha, \Theta(\alpha))$ corresponds to a policy $\pi_{\Theta(\alpha)}$. 
Thus, SPOS for optimizing the neural architecture of the RL agent solves the following problem
\begin{align}
\Theta^* = \argmax_{\Theta} \mathbb{E}_{\alpha \sim P} \mathbb{E}_{t} [ R_{t} [\pi_{\Theta(\alpha)}]], \\
\alpha^* = \argmax_{\alpha \in \mathcal{A}} \mathbb{E}_{t} [ R_{t}[\pi_{\Theta^*(\alpha)}].
\end{align}









\section{Experiments}

\subsection{Atari Environment}
\label{sec:atari}


In our experiments, we used the Open AI Gym framework \cite{Brockman2016OpenAIG}, particularly --  \textit{Breakout} and \textit{Freeway} Atari environments.
We chose the \textit{Breakout} because it’s a popular benchmark, having moderate standard deviation of RL agent's reward ($401.2 \pm 26.9$, \cite{Humanlevel2015}). In opposite to the \textit{Breakout}, the \textit{Freeway} environment has very low relative standard deviation of reward ($30.3 \pm 0.7$, \cite{Humanlevel2015}).
The reward of RL agent with random behavior for \textit{Breakout} and \textit{Freeway} is nearly zero so we can make sure that our policy network makes non-stochastic behavior.

We have trained the child networks in the manner described in \cite{espeholt2018impala}, i.e., we have used 8 agents, sharing a policy, trained simultaneously in 8 environments with PPO, in order to collect the trajectories for the policy update. Each of these agents is trained for 128 steps. After that, the controller collects the architecture's rewards. The controller's policy is updated every ten steps using the REINFORCE algorithm. Overall, the number of training steps for one experiment equals 10 million. More implementation details are in Appendix \ref{app:hyperparams}.

It is important to note that the 'scratch' experimental results demonstrate lower reward values than the ones reported in the original PPO paper \cite{schulman2017proximal}. This is due to the fact that in our experiments we have used a smaller number of training timesteps than the classical version of PPO (10M vs. 40M). The reason for this has been that the main aim of our research focuses on investigating whether NAS has a positive effect on RL training process overall, and not on beating the existing ATARI baselines.
\subsection{Search Spaces}
\label{sec:search_sp}

\begin{figure}
\begin{subfigure}{\textwidth}
 \centering 
 \includegraphics[width=\textwidth]{./images/architecture_1.pdf}
  \caption{\textbf{Search space 1:} Block(a): fixed kernel size:{8}, fixed stride:{4};  Block(b) and Block(c): kernel size is chosen between {1, 2, 3, 4, and 5}, and fixed stride: {2, 1} respectively.}
  \label{fig:architecture_1}
\end{subfigure}
\begin{subfigure}{\textwidth}
 \centering 
 \includegraphics[width=\textwidth]{./images/architecture_2.pdf}
  \caption{\textbf{Search space 2:} Block(a): fixed kernel size:{8}, fixed stride:{4};  Block(b)- Block(f): kernel size is chosen between {2 and 5}, with fixed stride: {1} for all of them.}
  \label{fig:architecture_2}
\end{subfigure}
\caption{Search spaces used in experiments.}
\end{figure}

The search spaces that we designed are the extension of the Nature-CNN architecture \cite{Humanlevel2015}, where convolutional layers are followed by
a linear layer which outputs the number of values equal to the size of our action space \cite{Humanlevel2015}. 




In order to facilitate the varying sizes of the layers' parameters and, hence, enable the weight sharing, we use the following techniques in the overall design of the network:
\begin{enumerate}
  \item \textbf{Convolution and max-pooling to the same size}:
  Map the input to one size every time after each convolutional and max-pooling layer. 
\item \textbf{Padding to exact size}:
  Map the output of the last convolutional layer to the target size to be able to fix it during the flattening of the convolutions.
  
\end{enumerate}

In our experiments we have covered two architectural search spaces:

\textbf{Search space 1: 25 architectures}.
The architecture starts with a fixed convolutional layer with input channels equal to 4, and the output channels - to 32, kernel size - 8 and moving stride - 4.
This layer is followed by two convolutional layers with output channels equal to 64 per each and with strides 1 and 2. The choices for kernel sizes are 1, 2, 3, 4, 5. The size of this search space is $5^2$.

\textbf{Search space 2: 32 architectures}.
Instead of two convolutional layers (we do not take into account the first fixed convolutional layer, as we do not vary its kernel size) used in the search space 1, we have used 5 convolutional layers with output channels equal to 64, moving strides equal to 1. The choices for kernels are 2 and 5, which makes the size of this search space equal to $2^5$.

The experimental architecture spaces that we have used can be seen in Figure \ref{fig:architecture_1} and Figure \ref{fig:architecture_2}, and also in Appendix \ref{app:search-spaces}.
The search spaces do not contain very deep architectures, having depth 7 at most.

\subsection{Methodology}
Firstly, we trained all the architectures from scratch (implementation details are in Appendix \ref{app:hyperparams}). For all of those architectures we saved \textit{mean reward} and \textit{total reward} averaged over last 100 episodes. 
These metrics were used as a tabular benchmark, eliminating the need to train the same architecture multiple times.
We used these metrics later to compare the performances of the found architectures by various NAS methods.

Both NAS methods under evaluation (ENAS, SPOS) share the same high-level structure: 
\begin{enumerate}
\item Run neural architecture search method for the given search space $\mathcal{A}$;
\item Select top-K architectures from the search space $\mathcal{A}$ by a \textit{proxy performance};
\item Train these top-K architectures from scratch;
\item Return the best one by the \textit{true performance}. \end{enumerate}

In our experiments, we selected top-3 architectures for the methods under evaluation. We repeated the search 4 times with different seeds and averaged results.

The \textit{proxy performance} is a part of the NAS method, and it is fast to calculate. The calculation time of the proxy performance is negligible to the time of RL agent training from scratch. For ENAS, the proxy performance of an architecture is the probability of sampling this architecture by the controller. For SPOS, the proxy performance of an architecture is calculated from weights of this architecture in the supernetwork. Namely, the proxy performance is the mean reward of an agent with corresponding weights. 

The \textit{true performance} is the total/mean reward of an RL agent trained from scratch. 









\subsection{Random search baseline} 
As a simple baseline, we used the following random search algorithm:
\begin{enumerate}
\item Select K architectures from the search space $\mathcal{A}$ at random;
\item Train these K architectures from scratch;
\item Return the best one by the \textit{true performance}.
\end{enumerate}

As for ENAS and SPOS, we selected top-3 architectures.
The only difference with ENAS/SPOS methods is that architectures are selected at random instead of by \textit{proxy performance}.
We estimated the variance of the random search by repeating it for 1000 times.








\subsection{Experiments with larger search spaces}
We have also tried to expand the search space by increasing the number of consecutive convolutional layers up to 9 and choosing a suitable kernel size and a number of output channels for each of them. However, the results were close to the ones received by random search, which can be caused by the fact that NAS can experience problems with increasingly complex search spaces. For that reason, we do not present the results of these experiments in the paper. 



\begin{table}[t]
\centering 
 \caption{Reward mean and total reward of RL agents with various architectures. Each score is the mean of last 100 episodes.}
 \label{tbl:results_table}
\begin{tabularx}{\textwidth}{@{}lccccc@{}}
\toprule

            & & \multicolumn{2}{c}{Search space 1}  &  \multicolumn{2}{c}{Search space 2} \\ 
\midrule

            & & Breakout & Freeway & Breakout & Freeway \\ 
\midrule


\multirow{2}{*}{Random Search}      & reward mean     & 54.7 $\pm$ 8.3    & 28.4 $\pm$ 1.0  & 33.1 $\pm$ 29.5   & 21.6 $\pm$ 0.2  \\ 
                                    & total reward    & 147.2 $\pm$ 25.5  & \textbf{31.7 $\pm$ 0.8}  & 105.7 $\pm$ 94.2    & \textbf{22.0 $\pm$ 0.2}  \\ 
\addlinespace

 \multirow{1}{*}{Nature-CNN \cite{Humanlevel2015},} & reward mean               & 57.1          & 13.1      & -         & -     \\ 
 \quad reproduced                                   & total reward              & 157.9         & 19.0      & -         & -     \\ 


\addlinespace

\multirow{2}{*}{ENAS}   & reward mean             & \textbf{61.4 $\pm$ 1.8}          & 26.4 $\pm$ 1.1  & 30.7 $\pm$ 21.7      & 21.5 $\pm$ 0.1                     \\
                        & total reward             & \textbf{161.1 $\pm$ 9.8}    & 30.7 $\pm$ 1.3    & 91.4  $\pm$ 64.4   & \textbf{22.0 $\pm$ 0.2}                     \\ 
\addlinespace

\multirow{2}{*}{SPOS}   & reward mean               & 39.7 $\pm$ 18.6         & \textbf{29.6 $\pm$ 0.8}    & \textbf{39.9 $\pm$ 41.0}     & \textbf{21.7 $\pm$ 0.1}     \\ 
                        & total reward              & 144.4 $\pm$ 55.0         & 29.4 $\pm$ 5.0             & \textbf{180.6 $\pm$ 72.5}    & \textbf{22.0 $\pm$ 0.1}              \\ 

\bottomrule
\end{tabularx}
\end{table}

\begin{figure}[t]
\centering
\begin{subfigure}{0.45\linewidth}
    \includegraphics[width=\linewidth]{images/hist_25_Breakout.png}
    \caption{Search Space 1: Breakout}
\end{subfigure}
\hfil
\begin{subfigure}{0.45\linewidth}
    \includegraphics[width=\linewidth]{images/hist_32_Breakout.png}
    \caption{Search Space 2: Breakout}
\end{subfigure}

\begin{subfigure}{0.45\linewidth}
    \includegraphics[width=\linewidth]{images/hist_25_Freeway.png}
    \caption{Search Space 1: Freeway}
\end{subfigure}
\hfil
\begin{subfigure}{0.45\linewidth}
    \includegraphics[width=\linewidth]{images/hist_32_Freeway.png}
    \caption{Search Space 2: Freeway}
\end{subfigure}
\caption{Histograms for the reward mean of RL agents having different architectures from search spaces. The red vertical line depicts the best architecture found by NAS methods.}
\label{fig:histograms}
\end{figure}


\section{Discussion}
Table \ref{tbl:results_table} shows the results of our experiments.
For the Breakout environment, ENAS performs better than random search on the search space 1, while SPOS performs better on the search space 2. For the Freeway environment, there is no clear benefit of NAS methods. Variance of the both of the methods are quite high.
At the same time, SPOS is simpler since it does not contain an auxiliary controller network. 


It is interesting to compare the performances of found architectures with the performance of manually selected Nature-CNN architecture \cite{Humanlevel2015} (see description in the Appendix \ref{app:nature-CNN-architecture}). The Nature-CNN architecture belongs to the search space 1. For the fair comparison, in Table \ref{tbl:results_table}, we report the rewards after training of the Nature-CNN architecture with our pipeline (Section \ref{sec:atari}). The architectures found by NAS methods outperform the manually selected Nature-CNN by a considerable margin.

The existing research on RL methods typically uses the same architectures for different environments.
However, we found that this is not optimal. Different architectures are optimal for different environments, see Appendix \ref{app:found-architectures}.

Figure \ref{fig:histograms} shows histograms of RL agents' mean rewards from different search spaces. The red vertical lines depict the best architecture found by NAS methods. We conclude that NAS methods can find top architectures in both of the search spaces. 



\section{Conclusion}
Traditionally, the progress in RL field came mostly from the development of new methods. Neural architectures of RL agents remained relatively simple when compared to computer vision applications. 

In this paper, we have applied modern neural architecture search methods for optimizing the architecture of RL agents. We have evaluated ENAS \cite{zoph2018learning} and SPOS \cite{guo2019single} methods. Both of them found better architectures than manually picked by experts. We suppose that many RL application can benefit from using better neural architectures.
Testing NAS methods for larger search spaces is an interesting topic for further research.


\subsection*{Acknowledgments}
Authors are thankful to Mikhail Konobeeb.




\nocite{*}
\printbibliography


\newpage

\section{Appendix}
\begin{appendix}
\section{Search Spaces}
\label{app:search-spaces}
Table \ref{search_spaces_table} describes two search spaces that we used in our experiments. An abbreviation \{$1, ..., n$\} means that a parameter can vary from 1 to $n$ in a search space.
\begin{table*}[!h]
\begin{multicols}{2}
    \begin{center}
    \textbf{Search Space 1}
    \begin{tabularx}{0.5\textwidth} {@{}lcccc@{}}
        \hline
        \textbf{Layer} & \textbf{input} & \textbf{output} & \textbf{kernel} & \textbf{stride}\\
        \hline
        Conv-1 & $4$ & $32$ & $8$ & $4$         \\
        Conv-2 & $32$ & $64$ & $1, .., 5$ & $2$ \\
        Conv-3 & $64$ & $64$ & $1, .., 5$ & $1$ \\
        Padding & $-$ & $121*64$ & $-$ & $-$    \\        
        Linear & $121*64$ & $512$ & $-$ & $-$   \\
        \hline

    \end{tabularx}
    \end{center}

    \vfill\eject
    
    \begin{center}
    \textbf{Search Space 2}
    \begin{tabularx}{0.5\textwidth}{@{}lcccc@{}}
        \hline
        \textbf{Layer} & \textbf{input} & \textbf{output} & \textbf{kernel} & \textbf{stride}\\
        \hline
        Conv-1 & $4$ & $32$ & $8$ & $4$         \\
        Conv-2 & $32$ & $64$ & $2, 5$ & $1$     \\
        Conv-3 & $64$ & $64$ & $2, 5$ & $1$     \\
        Conv-4 & $64$ & $64$ & $2, 5$ & $1$     \\
        Conv-5 & $64$ & $64$ & $2, 5$ & $1$     \\
        Conv-6 & $64$ & $64$ & $2, 5$ & $1$     \\
        Padding & $-$ & $121*64$ & $-$ & $-$    \\        
        Linear & $121*64$ & $512$ & $-$ & $-$   \\
        \hline
    
    \end{tabularx}
    \end{center}
\end{multicols}
\caption{Detailed description of the search spaces.}
\label{search_spaces_table}
\end{table*}


\section{Nature CNN architecture}
\label{app:nature-CNN-architecture}
Table \ref{tab:nature-cnn} describes the architecture ``Nature CNN'' \cite{Humanlevel2015} which is a member of the  \textbf{search space 1}.
\begin{table}[!h]
\begin{center}
\begin{tabularx}{0.5\textwidth}{@{}lcccc@{}}
    \hline
    \textbf{Layer} & \textbf{input} & \textbf{output} & \textbf{kernel} & \textbf{stride}\\
    \hline
        Conv-1 & $4$ & $32$ & $8$ & $4$         \\
        Conv-2 & $32$ & $64$ & $4$ & $2$        \\
        Conv-3 & $64$ & $64$ & $3$ & $1$        \\
        Padding & $-$ & $121*64$ & $-$ & $-$    \\
        Linear & $121*64$ & $512$ & $-$ & $-$   \\
    \hline
\end{tabularx}
\caption{Convolution network architecture that used in \cite{Humanlevel2015} that's contain 3 convolution layer followed by flatten and linear layers.}\label{tab:nature-cnn}
\end{center}
\end{table}

\section{Implementation details}
\label{app:hyperparams}
In this section, we present the hyperparameters used for training the RL agents in the ATARI games environment (Table \ref{tab:params_scratch}), as well as the hyperparameters used for ENAS and SPOS (Table \ref{tab:params_enas}).
We use the same set of parameters for both training from scratch experiments, and training the child networks sampled by the NAS controllers.

\begin{table}[!h]
\begin{center}
\begin{tabularx}{0.6\textwidth}{c|c}
    \hline
    \textbf{Hyperparameter's name} & \textbf{Value}\\
    \hline
        \# timesteps & $10$M    \\
        \# runner timesteps & $128$   \\  
        $\epsilon$ (PPO) & $0.1$   \\
        value loss coef. (PPO) & $0.25$ \\
        $\lambda$ (GAE) & $0.95$   \\
        entropy coef. & $0.01$   \\
        learning rate & CosineAnnealing($0.00025$)   \\
        \# parallel env. & $8$  \\
              
    \hline
\end{tabularx}
\caption{The hyperparameters for training PPO agents on ATARI games.}
\label{tab:params_scratch}
\end{center}
\end{table}

\begin{table}[!h]
\begin{center}
\begin{tabularx}{0.6\textwidth}{c|c}
    \hline
    \textbf{Hyperparameter's name} & \textbf{Value}\\
    \hline
        \# child network epochs & $10$    \\
        \# NAS runner timesteps & $3$   \\  
        NAS entropy coef. & $0.0001$   \\
        NAS learning rate & CosineAnnealing($0.001$)   \\
        baseline momentum. & $0.2$  \\
              
    \hline
\end{tabularx}
\caption{The hyperparameters for ENAS training.}
\label{tab:params_enas}
\end{center}
\end{table}


\section{The best architectures}
\label{app:found-architectures}
Tables \ref{tab:best-1}, \ref{tab:best-2}, \ref{tab:best-3}, \ref{tab:best-4} show the best architectures for each game.  The architectures tend to be similar for both of the search spaces except the kernel size for the last layers.
\begin{table}[!h]
\begin{center}
\begin{tabularx}{0.5\textwidth}{@{}lcccc@{}}
    \hline
    \textbf{Layer} & \textbf{input} & \textbf{output} & \textbf{kernel} & \textbf{stride}\\
    \hline
        Conv-1 & $4$ & $32$ & $8$ & $4$         \\
        Conv-2 & $32$ & $64$ & $4$ & $2$        \\
        Conv-3 & $64$ & $64$ & $5$ & $1$        \\
        Padding & $-$ & $121*64$ & $-$ & $-$    \\        
        Linear & $121*64$ & $512$ & $-$ & $-$   \\
    \hline
\end{tabularx}
\caption{The best architecture for Breakout extracted from search space 1 by SPOS using reward mean criteria.}\label{tab:best-1}
\end{center}
\end{table}

\begin{table}[!h]
\begin{center}
\begin{tabularx}{0.5\textwidth}{@{}lcccc@{}}
    \hline
    \textbf{Layer} & \textbf{input} & \textbf{output} & \textbf{kernel} & \textbf{stride}\\
    \hline
        Conv-1 & $4$ & $32$ & $8$ & $4$         \\
        Conv-2 & $32$ & $64$ & $3$ & $2$        \\
        Conv-3 & $64$ & $64$ & $2$ & $1$        \\
        Padding & $-$ & $121*64$ & $-$ & $-$    \\        
        Linear & $121*64$ & $512$ & $-$ & $-$   \\
    \hline
\end{tabularx}
\caption{The best architecture for Freeway extracted from search space 1 by SPOS using reward mean criteria.}\label{tab:best-2}
\end{center}
\end{table}

\begin{table}[!h]
\begin{center}
\begin{tabularx}{0.5\textwidth}{@{}lcccc@{}}
    \hline
    \textbf{Layer} & \textbf{input} & \textbf{output} & \textbf{kernel} & \textbf{stride}\\
    \hline
        Conv-1 & $4$ & $32$ & $8$ & $4$         \\
        Conv-2 & $32$ & $64$ & $5$ & $1$        \\
        Conv-3 & $64$ & $64$ & $5$ & $1$        \\
        Conv-4 & $64$ & $64$ & $5$ & $1$        \\
        Conv-5 & $64$ & $64$ & $2$ & $1$        \\
        Conv-6 & $64$ & $64$ & $2$ & $1$        \\
        Padding & $-$ & $121*64$ & $-$ & $-$    \\        
        Linear & $121*64$ & $512$ & $-$ & $-$   \\
    \hline
\end{tabularx}
\caption{The best architecture for Breakout extracted from search space 2 by SPOS using reward mean criteria.}\label{tab:best-3}
\end{center}
\end{table}

\begin{table}[!h]
\begin{center}
\begin{tabularx}{0.5\textwidth}{@{}lcccc@{}}
    \hline
    \textbf{Layer} & \textbf{input} & \textbf{output} & \textbf{kernel} & \textbf{stride}\\
    \hline
        Conv-1 & $4$ & $32$ & $8$ & $4$         \\
        Conv-2 & $32$ & $64$ & $5$ & $1$        \\
        Conv-3 & $64$ & $64$ & $5$ & $1$        \\
        Conv-4 & $64$ & $64$ & $5$ & $1$        \\
        Conv-5 & $64$ & $64$ & $5$ & $1$        \\
        Conv-6 & $64$ & $64$ & $5$ & $1$        \\
        Padding & $-$ & $121*64$ & $-$ & $-$    \\        
        Linear & $121*64$ & $512$ & $-$ & $-$   \\
    \hline
\end{tabularx}
\caption{The best architecture for Freeway extracted from search space 2 by SPOS using reward mean criteria.\\
}\label{tab:best-4}
\end{center}
\end{table}


\end{appendix}


\end{document}
