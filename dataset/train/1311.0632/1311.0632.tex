\subsection{\textbf{IO(L)} is not an AFL}

Finally, based on the previous results, we prove that,  given $\mathbf{L}$ a separable abstract family of semilinear languages such that $\mathbf{RL} \subseteq \mathbf{L}$, the family $\mathbf{IO(L)}$ is not closed under inverse homomorphism,

We first prove the following lemma, which states that, whenever a chain of IO-substitutions potentially generate a language which is not $a$-linear (\textit{i.e.}~whenever such a chain can  copy words/languages more than once), the derived words must verify a specific pattern.

\begin{lemma}\label{lem:copy-separation}
  Consider a language $L \in \mathbf{IO(L)}$ and:
  \begin{itemize}
    \item a derivation $d=L_0[x_1:=L_1]_{IO} \dots [x_n:=L_n]_{IO} \in \mathcal{D}_L$, fully-effective and in standard form;
    \item a symbol $a \in \Sigma$, a chain $C \in Ch_d^a$, distinct symbols $y_1, y_2 \in C$ and $0 \leq i_1, i_2 \leq n$, $i_1 \neq i_2$ such that, for every $u_1 \in L_{i_1}$ and $u_2 \in L_{i_2}$, $|u_1|_{y_1} > 1$ and $|u_2|_{y_2} > 1$
    \end{itemize}
    Then, for every $w \in L$, there exists $w', w_1, w_2, w_3 \in \Sigma^\ast$ such that $w=w_1aw'aw_2aw'aw_3$.
\end{lemma}
\begin{proof}
First remark, that $y_1 \neq y_2$ and $y_1, y_2 \in C$ implies either $y_1 \in {In}_{d}^{y_2}$ or $y_2 \in {In}_{d}^{y_1}$, by definition of a chain of introducers; let us chose $y_1$ and $y_2$ such that $y_1 \in {In}_{d}^{y_2}$.
We consider a word $u$ in the language derived by $L_0[x_1:=L_1]_{IO} \dots [x_{i_1-1}:=L_{{i_1}-1}]_{IO}$ and a word $u' \in L_{i_1}$.
By hypothesis, $u'$ is of the form $u'_{1}y_1u'_{2}y_1u'_{3}$; because the IO-substitution is not irrelevant, $io_{y_1, u'}(u)$ is of the form $u_1y_1u_2y_1u_3$.

Because there is no deleting IO-substitution, the words in the language derived by $L_{0}[x_{1}:=L_{1}]_{IO} \dots [x_{i_2-1}:=L_{i_2-1}]_{IO}$ must be of the form $w=w_1yw_2yw_3$, where $y_1 {In_d^\ast} y$ and $y {In_d^\ast} y_2$.
Then, because every word in $L_{i_2}$ is of the general form $w'=w'_{1}y_2w'_{2}y_2w'_{3}$, the word $io_{y_2, w'}(w)$ is of the form $w''_1y_2w'_{2}y_2w''_2y_2w'_{2}y_2w''_3$.

Again, because the substitutions are not deleting, we can conclude that the words in $L$ are of the form $u'_1au'au'_2au'au'_3$.
\end{proof}

We now prove our main theorem. The sketch of the proof is similar in many aspects to the proof of the very same non-closure property for IO-macro languages by Fischer.
Indeed, assuming $L_{anp,b} = \{w \in \{a, b\}^\ast \mid |w|_a=nm, \text{ where }n,m > 1\}$ is in $\mathbf{IO(L)}$, and the language $L_{\textit{diff}} = \{b^{p_0}ab^{p_1}a...ab^{p_{nm}} \mid n,m > 1 \text{ and for every }  0 \leq i,j \leq nm, i \neq j \imp p_i \neq p_j\}$, we exhibit a language $L$ such that $L_{\textit{diff}} \subseteq L \subseteq L_{anp,b}$, by removing derivations of Lemma~\ref{lem:copy-separation} in a derivation of $L_{anp,b}$. This means that the IO-substitution is never used in a copying fashion, and therefore, the language derived must be $a$-linear, which is impossible.

\begin{theorem}[Non-closure under inverse homomorphism]
Given an abstract family of semilinear languages $\mathbf{L}$ such that $\mathbf{RL} \subseteq \mathbf{L}$, the family $\mathbf{IO(L)}$ is not closed under inverse homomorphism.
\end{theorem}
\begin{proof}
  Let us consider the language made of a non-prime numbers of $a$
  $$L_{nprime} = \{a^{nm} \mid n,m>1\}$$

  This language is not semilinear since its Parikh image is equal to $\mathrm{Im(F)}$ where $F(x_1, x_2)=4\langle 1 \rangle + x_1\langle 2\rangle + x_2\langle 2\rangle + x_1x_2\langle 1 \rangle$.
  Therefore $L_{nprime}$ does not belong to $\mathbf{L}$, but belongs to $\mathbf{IO(RL)}$: indeed $a^2a^\ast[a:=a^2a^\ast]_{IO} \rightarrow L_{nprime}$ and $a^2a^\ast$ is a regular language. Therefore, if $\mathbf{RL} \subseteq \mathbf{L}$, then $L_{nprime}$ is in $\mathbf{IO(L)}$

  Now, consider the homomorphism $\phi:\{a, b\} \to a^\ast$ such that $\phi(a)=a$ and $\phi(b)=\epsilon$. Then we obtain:
  $$\phi^{-1}(L_{nprime}) = L_{anp,b} = \{w \in \{a, b\}^\ast \mid |w|_a=nm, \text{ where }n,m > 1\}$$
  
  Let us assume $L_{anp, b}$ belongs to $\mathbf{IO(L)}$.
  Then, according to Lemma~\ref{lem:fully-effective-std}, there exists a fully-effective standard derivation $d_{anp,b}=\bigcup_{i \in I}d_i$ for this language,  where for every $i \in I$, $d_i=L_{i0}[x_{i_1}:=L_{i1}]_{IO} \dots [x_{i_{n_i}}:=L_{in_i}]_{IO}$; and for every $0 \leq j \leq n_i$  $L_{ij}$, belongs to $\mathbf{L}$.

  Let us consider the language $L_{\textit{diff}} \subsetneq L_{anp, b}$ defined as:
  $$L_{\textit{diff}} = \{b^{p_0}ab^{p_1}a...ab^{p_{nm}} \mid n,m > 1 \text{ and for every }  0 \leq i,j \leq nm, i \neq j \imp p_i \neq p_j\}$$

We aim at building a language $L$ such that $L_{\textit{diff}} \subseteq L \subseteq L_{anp,b}$. In order to do so, for every $i \in I$, let us consider the congruence $\cong_i$ defined as:
$$w_1 \cong_i w_2 \text{ iff for every }y \in {In_{d_i}^a}, |w_1|>1 \iff |w_2|>1$$

Such a congruence is of finite index. According to the separation lemma and lemma \ref{lem:unions-iol}, we can consider the derivation
$$d'_i = \bigcup_{C_0, \dots, C_{n_i} \in \Sigma^\ast/{\cong_i}} (L_{i0} \cap C_0)[x_1 := (L_{i1} \cap C_1)]_{IO} \dots [x_{n_i} := (L_{in_i} \cap C_{n_i})]_{IO}$$
such that $d_i$ and $d'_i$ derive the same language. Moreover, for every $C_0, \dots, C_{n_i} \in \Sigma^\ast/{\cong_i}$, the derivation $(L_{i0} \cap C_0)[x_1 := (L_{i1} \cap C_1)]_{IO} \dots [x_{n_i} := (L_{in_i} \cap C_{n_i})]_{IO}$ is in standard form, and is fully effective (because $L_{ij} \cap C_j$ is a sublanguage of $L_{ij}$, for every $1 \leq j \leq n_i$).

Now let us consider a derivation $d''_i=(L_{i0} \cap C_0)[x_1 := (L_{i1} \cap C_1)]_{IO} \dots [x_{n_i} := (L_{in_i} \cap C_{n_i})]_{IO}$ for some $C_0, \dots, C_{n_i} \in \Sigma^\ast/{\cong_i}$, such that there exist a chain $ch \in {Ch}_{d''_i}^a$, symbols $y_1, y_2 \in {Ch}_{d''_i}^a$ ($y_1 \neq y_2$), and integers $0 \leq i_1, i_2 \leq n_i$, for which:
\begin{itemize}
  \item for every word $w \in C_{i_1}$, $|w|_{y_1} > 1$;
  \item for every word $w \in C_{i_2}$, $|w|_{y_2} > 1$;
\end{itemize}

Then, according to lemma~\ref{lem:copy-separation}, any word in $L''_i$ where $d''_i \in \mathcal{D}_{L''_i}$ does not belong to $L_{\textit{diff}}$. We can therefore build the language $L$ such that:
$$d = \bigcup_{i \in I'}\bigcup_{C_0 \in \mathcal{C}_{i0}} \dots \bigcup_{C_{n_i} \in \mathcal{C}_{in_i}} (L_{i0} \cap C_0)[x_1 := (L_{i1} \cap C_1)]_{IO} \dots [x_{n_i} := (L_{in_i} \cap C_{n_i})]_{IO}$$
is in $\mathcal{D}_L$, where $d$ results from removing the derivations of languages which intersection with $L_{\textit{diff}}$ is empty.

But, for every $i \in I'$ and every $C_0 \in \mathcal{C}_{i0}, \dots C_{n_i} \in \mathcal{C}_{in_i}$, the derivation $(L_{i0} \cap C_0)[x_1 := (L_{i1} \cap C_1)]_{IO} \dots [x_{n_i} := (L_{in_i} \cap C_{n_i})]_{IO}$ must verify the assumptions of Lemma~\ref{lem:conditions-linearity}; therefore, such a derivation derives a language which is $a$-linear, and $L$ is a finite union of $a$-linear languages, hence an $a$-linear language itself.

But, $\phi(L_{\textit{diff}})=L_{nprime} \subseteq \phi(L) \subseteq \phi(L_{anp, b}) = L_{nprime}$. Therefore, $L_{nprime}$ should be $a$-linear, which is false, and we obtain a contradiction.





















\end{proof}

We already commented the analogy between our demonstration and the one in \cite{fischerphd}. One major difference is that Fischer's proof is strongly related to the formalism generating IO-macro languages. In the present case, we intend to work only on the notions of semilinearity and of the copying power which enrich the original family of semilinear languages \textbf{L}.

%
