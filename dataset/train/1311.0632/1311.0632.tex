\subsection{\textbf{IO(L)} is not an AFL}

Finally, based on the previous results, we prove that,  given  a separable abstract family of semilinear languages such that , the family  is not closed under inverse homomorphism,

We first prove the following lemma, which states that, whenever a chain of IO-substitutions potentially generate a language which is not -linear (\textit{i.e.}~whenever such a chain can  copy words/languages more than once), the derived words must verify a specific pattern.

\begin{lemma}\label{lem:copy-separation}
  Consider a language  and:
  \begin{itemize}
    \item a derivation , fully-effective and in standard form;
    \item a symbol , a chain , distinct symbols  and ,  such that, for every  and ,  and 
    \end{itemize}
    Then, for every , there exists  such that .
\end{lemma}
\begin{proof}
First remark, that  and  implies either  or , by definition of a chain of introducers; let us chose  and  such that .
We consider a word  in the language derived by  and a word .
By hypothesis,  is of the form ; because the IO-substitution is not irrelevant,  is of the form .

Because there is no deleting IO-substitution, the words in the language derived by  must be of the form , where  and .
Then, because every word in  is of the general form , the word  is of the form .

Again, because the substitutions are not deleting, we can conclude that the words in  are of the form .
\end{proof}

We now prove our main theorem. The sketch of the proof is similar in many aspects to the proof of the very same non-closure property for IO-macro languages by Fischer.
Indeed, assuming  is in , and the language , we exhibit a language  such that , by removing derivations of Lemma~\ref{lem:copy-separation} in a derivation of . This means that the IO-substitution is never used in a copying fashion, and therefore, the language derived must be -linear, which is impossible.

\begin{theorem}[Non-closure under inverse homomorphism]
Given an abstract family of semilinear languages  such that , the family  is not closed under inverse homomorphism.
\end{theorem}
\begin{proof}
  Let us consider the language made of a non-prime numbers of 
  

  This language is not semilinear since its Parikh image is equal to  where .
  Therefore  does not belong to , but belongs to : indeed  and  is a regular language. Therefore, if , then  is in 

  Now, consider the homomorphism  such that  and . Then we obtain:
  
  
  Let us assume  belongs to .
  Then, according to Lemma~\ref{lem:fully-effective-std}, there exists a fully-effective standard derivation  for this language,  where for every , ; and for every   , belongs to .

  Let us consider the language  defined as:
  

We aim at building a language  such that . In order to do so, for every , let us consider the congruence  defined as:


Such a congruence is of finite index. According to the separation lemma and lemma \ref{lem:unions-iol}, we can consider the derivation

such that  and  derive the same language. Moreover, for every , the derivation  is in standard form, and is fully effective (because  is a sublanguage of , for every ).

Now let us consider a derivation  for some , such that there exist a chain , symbols  (), and integers , for which:
\begin{itemize}
  \item for every word , ;
  \item for every word , ;
\end{itemize}

Then, according to lemma~\ref{lem:copy-separation}, any word in  where  does not belong to . We can therefore build the language  such that:

is in , where  results from removing the derivations of languages which intersection with  is empty.

But, for every  and every , the derivation  must verify the assumptions of Lemma~\ref{lem:conditions-linearity}; therefore, such a derivation derives a language which is -linear, and  is a finite union of -linear languages, hence an -linear language itself.

But, . Therefore,  should be -linear, which is false, and we obtain a contradiction.





















\end{proof}

We already commented the analogy between our demonstration and the one in \cite{fischerphd}. One major difference is that Fischer's proof is strongly related to the formalism generating IO-macro languages. In the present case, we intend to work only on the notions of semilinearity and of the copying power which enrich the original family of semilinear languages \textbf{L}.

%
