We assume that each node $i$ maintains a sorted list of neighbors $\Ng{i}$ according to their degree $d_i$. 


The algorithms that we adopt in the following use a certain factor basis $B=\{p_1,p_2,\ldots,p_n\}$ where $p_i$s are 
prime numbers \cite{koblitz}. For the sake of notation, let us say that $B|n$ if $n$ factorizes in $B$. Observe that the choice of the factors of the wake-up periods at each node is critical; the Chinese Remainder 
Theorem suggests that period $n_i$ should be chosen free of large factors. 




Nodes choose period $n_i\in [L_i,U_i]$ such that it factorizes according to $B$ using the procedure described 
in Alg.~\ref{alg:algorit0}. Alg.~\ref{alg:algorit0} returns the smallest integer in $[L_i,U_i]$ that can be written 
as $p_1^{q_1}\cdots p_n^{q_n}$, $0\leq q_i\in \Z$, provided that such a number exists. If it does not exist, $L_i$
is returned. We now introduce two heuristics for {\tt Strong WAKE-UP}.


\subsection{{\tt BFS WAKE-UP} Algorithm}\label{sec:equential}


Once nodes have chosen their periods according to Alg.~\ref{alg:algorit0}, the simplest rationale is 
to align their phases pairwise. Alg.~\ref{alg:algorit1} is based on a breadth first 
search (BFS) visit of the network rooted at a given node; as customary for BFS search 
a queue $Q$ is employed.  

Each node $i$ maintains information about nodes in its neighborhood, i.e., the node ID, the phases 
set $A_{i}$ and a special flag $f$ that is used for the BSF procedure, where (i) $f(j)=0$ means $j$ 
unexplored (ii) $f(j)=1$ means node $j$ discovered but not all neighbors of $j$ examined (iii) $f(j)=2$ 
node $j$ explored. By 
default, node $i$ wakes up at the origin of its own period, i.e., at $\alpha_i$. 



The algorithm selects all unexplored nodes and enqueues them at most once, so that it terminates in $N$ steps. 

\begin{algorithm}[t]
\caption{\tt BFS WAKE-UP}
\label{alg:algorit1}
\begin{algorithmic}[1]
\REQUIRE {$L_i \leq U_i$ $\rightarrow Default:$ $A_i=\{\alpha_i\}$} \FORALL {$i \in V$}
\STATE $n_i=\mbox{{\tt PERIOD}}(L_i,U_i)$
\STATE $f(i) \leftarrow 0$  \ENDFOR
\ENSURE {Select: $i\in V$ with largest degree}
\STATE $f(i)=1$  
\STATE $Q\leftarrow \{i\}$      \COMMENT{Enqueue the first node}
\WHILE{$Q \not = \emptyset$ \COMMENT{Visit all the nodes in BFS}}
\STATE $i\leftarrow Q[1]$ \COMMENT{Get the node on top of the queue}
\FORALL {$j \in \Ng{i}$}
\IF{$f(j)=0$ \COMMENT{If it was not explored before}}
\STATE $Q \leftarrow \{j\}$  \COMMENT{Enqueue this node}
\STATE $\alpha_j\leftarrow \alpha_i$  \COMMENT{Align time axis of node $i$ and node $j$}  
\STATE $A_j\leftarrow \{\alpha_j\}$   \COMMENT{Update the phase if $j$} 
\STATE $x_{ij}^k=x_{ji}^k=\alpha_i+k\cdot [n_i,n_j],\quad k\in \mathbb Z$ \COMMENT{Rendezvous times}
\STATE $f(j)\leftarrow 1$ \COMMENT{Visited}          
\ENDIF
\ENDFOR
\STATE    $Q = Q \setminus \{i\}$  \COMMENT{Dequeue $i$}
\STATE    $n_i\leftarrow [n_{i},(n_{i_1},\ldots,n_{i_{d_i}})]$ \COMMENT{Use largest possible period at $i$}
\STATE    $f(i)=2$  \COMMENT{Explored}            
\ENDWHILE 
\end{algorithmic}
\end{algorithm}



Alg.~\ref{alg:algorit1} reports on the pseudocode of the centralized version of the {\tt BFS WAKE-UP} algorithm;  
however, the {\tt BFS WAKE-UP} algorithm can be easily implemented in a distributed fashion \cite{Cormen}. Observe that {\tt BFS WAKE-UP} 
attains phase synchronization and a strictly feasible schedule. 

 Below, we formalize the properties of the algorithm; the proof is reported in App.~\ref{sec:proofs}.
\begin{thm}\label{thm:BFS}
{\tt BFS WAKE-UP} has the following properties:
\begin{enumerate}
\item[  i.] it produces a strictly feasible wake-up schedule; 
\item[ ii.] it has complexity $O(E)$; 
\item[iii.] a distributed implementation requires $O(M)$ messages.
\end{enumerate}
\end{thm}


\begin{rem}
 For implementation purposes, the above algorithm can be very easily coupled to the construction 
of shortest path trees \cite{Cormen}; in such a way customary algorithms such as Dijkstra 
can conveniently provide {\em joint routing and wake-up scheduling} with no need for 
separate message exchange; in turn, the weight of link $ij$ is represented naturally by $[n_i,n_j]$.
\end{rem}

The convergence time of the algorithm is proportional to the diameter of the network~\cite{Cormen};
 the {\tt BFS WAKE-UP} requires a preliminary leader election procedure; further its convergence 
may be slow. Accordingly, we devised an alternative heuristic algorithm, described below, which effectively addresses
such shortcomings by parallelising the computation of the scheduling.


\subsection{{\tt PARALLEL WAKE-UP} Algorithm}\label{sec:parallel}


The design of a parallel algorithm for wake-up synchronization requires careful handling of 
nodes phases; in particular, in order to avoid the increase of the size of the $A_i$s, it is 
possible to operate pairwise adaptation of nodes phases (phase adaptation).

\begin{lem}\label{prop:adjphase}
Let $ij,jk\in E$ and fix $n_i,n_j,n_k$, $a_{kj}$ and $a_{ij}$. A feasible wake-up schedule for $i,j,k$ 
such that $a_{jk}=a_{ji}$ exists if and only if
$(n_j,(n_i,n_k))|(a_{kj} - a_{ij})$. If such condition holds, the schedule is obtained for $a_{jk}=a_{ji} = a_{kj}+n_k \cdot (x_0\cdot e)$,  
where $e, x_0 \in \mathbb{Z}$ are solutions of $a_{kj}-a_{ij}= (n_j,n_k)\cdot e \cdot x_0 + (n_i,n_j)\cdot e
\cdot y_0,$ with $y_0 \in \mathbb{Z}$.
\end{lem}The proof is reported in App.~\ref{sec:proofs}.



\begin{algorithm}[t]
\caption{{\tt PARALLEL WAKE-UP}}\label{alg:algorit2}
\begin{algorithmic}[1]
\REQUIRE {$L_i \leq U_i$}
\FORALL {$i \in V$}
\STATE $n_i=\mbox{{\tt PERIOD}}(L_i,U_i)$ $\rightarrow Default:$ $A_i=\{\alpha_i\}$ 
\STATE $f(i),t(i) \leftarrow 0$ \COMMENT{Fix the initial value to flags $f$ and $t$}
\ENDFOR
\ENSURE {Node $i$ starts a random timer }
\STATE Timer expired at node $i$: send a message to stop neighbors' timers in $\Ng{i}$
\STATE {\bf Initialize:}
\STATE $n_i\leftarrow [n_{i},(n_{i_1},\ldots,n_{i_{d_i}})]$ \COMMENT{Use the largest possible period at $i$}
\IF{$f(i)=0$ \mbox{ \bf and} there exists $j \in \Ng_i$ s.t. $f(j) \neq 0$} \STATE $\alpha_i\leftarrow \alpha_j$ \COMMENT{Assign the neighbour's phase to node $i$}
      \STATE $t(j)\leftarrow 1$  \COMMENT{Node $j$ cannot change its phase anymore}
\ELSIF{ there exist $j,k \in \Ng_i: f(j)=f(k)=2$ \mbox{ \bf and }
  $A_j\cap A_k = \emptyset$ \mbox{ \bf and } conditions of
  Lemma~\ref{prop:adjphase} are satisfied}
  \STATE \mbox{ \bf case }$t(i)=0$ \mbox{ \bf then }$\alpha_i\leftarrow a_{jk} $  \STATE \mbox{ \bf case }$t(i)=1$ \mbox{ \bf then }$A_{i}=A_{i} \cup \{a_{jk}\}$ \STATE    $A_i \leftarrow A_i \setminus \{a_{ij},a_{ik}\}$ \COMMENT{Delete $a_{ij}$ and $a_{ik}$ from $A_i$}
\ENDIF
\STATE {\bf Visit all neighbors:}
\FORALL{$j \in \Ng{i}$  { \bf and}  $f(j)\not=2$} 
\IF{$f(j)=0$ \COMMENT{If it was not explored before}}
  \STATE $\alpha_j\leftarrow \alpha_i$  \COMMENT{Assign the phase to node $j$}     
  \STATE $f(j)=1$
\ELSIF{$A_{i}\cap A_{j} = \emptyset$}  
  \STATE $A_{j}\leftarrow A_{j} \cup \{\alpha_i\}$      \COMMENT{Add the phase to node $j$} 
\ENDIF  
\STATE $x_{ij}^0=\alpha_i+u \frac{\alpha_j-\alpha_i}{(n_i,n_j)}$
\COMMENT{$u$ computed using extended Euclidean algorithm}
\STATE $x_{ij}=x_{ji}=x_{ij}^0+k [n_i,n_j]$, $k=0,1,\ldots$ \COMMENT{Rendezvous times}     
\STATE $f(j)=2$
\ENDFOR
\end{algorithmic}
\end{algorithm}


The algorithm pseudo-code is reported in Alg.~\ref{alg:algorit2}. Each node $i$ maintains information 
about nodes in its neighborhood, i.e., the node ID, the phases set $A_{j}$, the wake-up period $n_j$ 
and two special flags $f$ and $t$. Special flags are used in order to determine $A_i$ and $n_i$. In particular 
(i) $f(j)=0$ means $j$ unexplored (ii) $f(j)=1$ means node $j$ discovered but $n_j$ has not been fixed yet 
and all neighbors of $j$ have been explored (iii) $f=2$ when node $j$ has been explored and therefore 
$\alpha_j$ and $n_j$ have been fixed. Also, $t(j)=0$ means $j$ that can adjust $\alpha_j$ (ii) $t(j)=1$ means 
$i$ cannot change $\alpha_j$ value because some of its neighbors used $\alpha_j$. 

At the start of the algorithm each node $i$ chooses $n_i$ using {\tt PERIOD} and maintains a list of neighbors 
$\Ng{i}$. Then node $i$ is marked {\em unexplored} and {\em available} to change $\alpha_i$ value ($f[i]=t[i]=0$). 
Hence, each node starts a random timer: when the timer expires node $i$ freezes neighbors' timers 
and determines rendezvous times with all nodes in its neighborhood.

If $i$ is unexplored, i.e., $f(i)=0$, node $i$ checks whether it is in range of an already existing 
node $j$ explored. If so, $i$ just aligns $\alpha_i=\alpha_j$; $j$ is then is marked {\em not available} 
to change its time axis anymore, i.e., $t(j)=1$. Otherwise, 
if $i$ is {\em discovered}, i.e., $f(j)=1$, $i$ checks whether it is in range of two already existing 
$j$ and $k$ already explored without a phase in common between them and if the phase can be adjusted then 
$i$ has two options: (i) if $i$ is {\em available} to change its phase ($t(i)=0$), then $i$ just aligns 
its phase to the novel phase, i.e., $\alpha_i=\alpha_j$, (ii) if $i$ is {\em not available} to 
change its phase ($t(i)=1$), then the novel phase is added to $A_i$.

Then, $i$ visits its neighbors for which no adaptation was performed. If $j\in \Ng{i}$ is {\em unexplored}, $\alpha_j=\alpha_i$ and $j$ is marked {\em discovered}, 
$f(j)=1$. If $j$ is {\em explored} and $A_i\cap A_j=\emptyset$, the phase of $i$ is added to $A_j$. Finally, $i$ marks 
itself {\em explored}.

The proof of correctness and the fact that Alg.~\ref{alg:algorit2} has message exchange 
is $O(M)$ is similar to the one for Alg.~\ref{alg:algorit1}:
\begin{thm}
{\tt PARALLEL WAKE-UP}:
\begin{enumerate}
\item[  i.] produces a wake-up schedule; 
\item[iii.] it requires $O(M)$ messages for a distributed implementation.
\end{enumerate}
\end{thm}

In principle, since multiple phases may be added at one node, it is possible that 
the output of the algorithm is a schedule that is not feasible. A simple bound on 
the average duty cycle (DC) is provided by the following
\begin{thm}
Under Alg.~\ref{alg:algorit2} $DC<\frac {d-1}{L}$, where $L=min L_i$ and $d$ is the average node degree
\end{thm}
\proof
The algorithm adds at most $d_i-1$ phases per node: $\frac 1N \sum_{i=1}^N \frac{|A_i|}{n_i}\leq \frac 1{NL}\sum_{i=1}^N (d_i-1)=\frac {d-1}{L}$
\endproof


\begin{rem}
Observe that the above algorithm can be used to adapt the wake-up scheduling dynamically in case one or more nodes 
join the network. In particular, it is possible to combine the two algorithms in order to initialize a 
wake-up schedule network-wide with {\tt BFS WAKE-UP}, and use {\tt PARALLEL WAKE-UP} to add new nodes to existing schedules
 or to repair locally a schedule, should a node fail or loose synchronization.
\end{rem}


\subsection{Example}


We reported in Fig.~\ref{fig:example} a simple example 
illustrating the {\tt BFS WAKE-UP} and of the {\tt Parallel WAKE-UP} 
algorithm on a sample graph with $7$ 
nodes. In that instance $U=20$, $L_i=(2,3,9,7,11,5,2)$, 
$\alpha_i=(1,3,7,9,6,1,0)$ and $B=\{2\}$. Several 
wake-up periods coexist: $4,8,16$.

\begin{figure*}[t]
\centering
\begin{minipage}{4cm}
\includegraphics[width=2cm]{./FIG/topology2}
\end{minipage}
\begin{minipage}{4cm}{\begin{tabular}{|l|c|c|c|c|c|c|}
\hline
\textbf{$i$ } & \textbf{$\alpha_i$} & \textbf{$L_i$}& \textbf{$U_i$}& \textbf{$p_i^{q_i}$}&\textbf{$(\alpha_i,n_i)$}& \textbf{$(\alpha_i,n_i)$}\\
&  &   & &  &\textbf{BFS} & \textbf{PARALLEL}\\
\hline
\hline
$1$ & $1$ &$2$ &$20$ &$2$  & $(1,4)$ & $(3,4)$\\
$2$ & $3$ &$3$ &$20$ &$4$  & $(1,4)$ & $(3,4)$\\
$3$ & $7$  &$9$ &$20$ &$16$ & $(1,16)$ & $(3,16)$\\
$4$ & $9$ &$7$ &$20$ & $8$  & $(1,4)$ & $(\{1,3\},4)$\\
$5$ & $6$  &$11$ &$20$ & $16$ & $(1,16)$ & $(1,16)$\\
$6$ & $1$  &$5$ &$20$ &$8$  & $(1,8)$ & $(1,8)$\\
$7$ & $0$  &$2$ &$20$ &$2$  & $(1,8)$ & $(1,8)$\\
\hline
\end{tabular}
}\end{minipage}
\caption{Example. {\tt BFS WAKE-UP} and {\tt PARALLEL WAKE-UP} on a sample topology. The table reports on the final output of the algorithms.}\label{fig:example}
\end{figure*}

The {\tt BFS WAKE-UP} run generates the list of visited 
nodes $(1,2,3,4,5,6,7)$; observe that every node aligned to 
$\alpha_1=1$. In the {\tt Parallel WAKE-UP} node $2$ and $7$ perform 
 the algorithm operations first, then $1$, $5$, $6$ and finally $4$. 
As we can see from the example, {\tt Parallel WAKE-UP} assigns 
heterogeneous phases; at node $4$ phase adaptation according 
to Thm.~\ref{prop:adjphase} is not possible so that finally 
$A_4=\{1,3\}$.


\subsection{Particular cases}\label{sec:particular}


There exists specific cases when strictly tight wake-up schedules are attained easily.
\begin{lem}
Let for any node $i$ be $B|L_i$, $B|U=U_i$ and $L_i\leq U$, then there exists a strictly tight wake-up schedule; it can be attained
using {\tt BFS WAKE-UP}. 
\end{lem} 
\begin{proof}
Consider any link $ij$ and let $n_i=L_i$ and $n_j=L_j$. The statement follows immediately 
from the Chinese Reminder theorem since $[n_i,n_j]|U$ and one can use 
{\tt BFS WAKE UP} to assign the same phase to all nodes in the network. 
\end{proof}

In particular the following example shows a simple application of the above
case. Consider a tree topology $T$ and assume nodes are labeled in order to respect the partial order induced 
by the distance from the root node $1$. Let $L_0$ be associated to the root node, and $L_k=L_0p^k$ be associated 
to nodes at level $k$. A {\em feasible} schedule can be constructed as follows: root node $0$ wakes at times $x_0=0 \mod L_0$.
Node $j$ wakes up at $x_j=\alpha_0 \mod p^{g(r_j)} L_0 $ where $r_j$ is the level occupied by node $j$ and $g$ is any positive integer 
function. In this case $U=L_0p^{\max g(r_j)}$.



