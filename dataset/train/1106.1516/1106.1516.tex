We assume that each node  maintains a sorted list of neighbors  according to their degree . 


The algorithms that we adopt in the following use a certain factor basis  where s are 
prime numbers \cite{koblitz}. For the sake of notation, let us say that  if  factorizes in . Observe that the choice of the factors of the wake-up periods at each node is critical; the Chinese Remainder 
Theorem suggests that period  should be chosen free of large factors. 




Nodes choose period  such that it factorizes according to  using the procedure described 
in Alg.~\ref{alg:algorit0}. Alg.~\ref{alg:algorit0} returns the smallest integer in  that can be written 
as , , provided that such a number exists. If it does not exist, 
is returned. We now introduce two heuristics for {\tt Strong WAKE-UP}.


\subsection{{\tt BFS WAKE-UP} Algorithm}\label{sec:equential}


Once nodes have chosen their periods according to Alg.~\ref{alg:algorit0}, the simplest rationale is 
to align their phases pairwise. Alg.~\ref{alg:algorit1} is based on a breadth first 
search (BFS) visit of the network rooted at a given node; as customary for BFS search 
a queue  is employed.  

Each node  maintains information about nodes in its neighborhood, i.e., the node ID, the phases 
set  and a special flag  that is used for the BSF procedure, where (i)  means  
unexplored (ii)  means node  discovered but not all neighbors of  examined (iii)  
node  explored. By 
default, node  wakes up at the origin of its own period, i.e., at . 



The algorithm selects all unexplored nodes and enqueues them at most once, so that it terminates in  steps. 

\begin{algorithm}[t]
\caption{\tt BFS WAKE-UP}
\label{alg:algorit1}
\begin{algorithmic}[1]
\REQUIRE {  } \FORALL {}
\STATE 
\STATE   \ENDFOR
\ENSURE {Select:  with largest degree}
\STATE   
\STATE       \COMMENT{Enqueue the first node}
\WHILE{ \COMMENT{Visit all the nodes in BFS}}
\STATE  \COMMENT{Get the node on top of the queue}
\FORALL {}
\IF{ \COMMENT{If it was not explored before}}
\STATE   \COMMENT{Enqueue this node}
\STATE   \COMMENT{Align time axis of node  and node }  
\STATE    \COMMENT{Update the phase if } 
\STATE  \COMMENT{Rendezvous times}
\STATE  \COMMENT{Visited}          
\ENDIF
\ENDFOR
\STATE      \COMMENT{Dequeue }
\STATE     \COMMENT{Use largest possible period at }
\STATE      \COMMENT{Explored}            
\ENDWHILE 
\end{algorithmic}
\end{algorithm}



Alg.~\ref{alg:algorit1} reports on the pseudocode of the centralized version of the {\tt BFS WAKE-UP} algorithm;  
however, the {\tt BFS WAKE-UP} algorithm can be easily implemented in a distributed fashion \cite{Cormen}. Observe that {\tt BFS WAKE-UP} 
attains phase synchronization and a strictly feasible schedule. 

 Below, we formalize the properties of the algorithm; the proof is reported in App.~\ref{sec:proofs}.
\begin{thm}\label{thm:BFS}
{\tt BFS WAKE-UP} has the following properties:
\begin{enumerate}
\item[  i.] it produces a strictly feasible wake-up schedule; 
\item[ ii.] it has complexity ; 
\item[iii.] a distributed implementation requires  messages.
\end{enumerate}
\end{thm}


\begin{rem}
 For implementation purposes, the above algorithm can be very easily coupled to the construction 
of shortest path trees \cite{Cormen}; in such a way customary algorithms such as Dijkstra 
can conveniently provide {\em joint routing and wake-up scheduling} with no need for 
separate message exchange; in turn, the weight of link  is represented naturally by .
\end{rem}

The convergence time of the algorithm is proportional to the diameter of the network~\cite{Cormen};
 the {\tt BFS WAKE-UP} requires a preliminary leader election procedure; further its convergence 
may be slow. Accordingly, we devised an alternative heuristic algorithm, described below, which effectively addresses
such shortcomings by parallelising the computation of the scheduling.


\subsection{{\tt PARALLEL WAKE-UP} Algorithm}\label{sec:parallel}


The design of a parallel algorithm for wake-up synchronization requires careful handling of 
nodes phases; in particular, in order to avoid the increase of the size of the s, it is 
possible to operate pairwise adaptation of nodes phases (phase adaptation).

\begin{lem}\label{prop:adjphase}
Let  and fix ,  and . A feasible wake-up schedule for  
such that  exists if and only if
. If such condition holds, the schedule is obtained for ,  
where  are solutions of  with .
\end{lem}The proof is reported in App.~\ref{sec:proofs}.



\begin{algorithm}[t]
\caption{{\tt PARALLEL WAKE-UP}}\label{alg:algorit2}
\begin{algorithmic}[1]
\REQUIRE {}
\FORALL {}
\STATE    
\STATE  \COMMENT{Fix the initial value to flags  and }
\ENDFOR
\ENSURE {Node  starts a random timer }
\STATE Timer expired at node : send a message to stop neighbors' timers in 
\STATE {\bf Initialize:}
\STATE  \COMMENT{Use the largest possible period at }
\IF{ \mbox{ \bf and} there exists  s.t. } \STATE  \COMMENT{Assign the neighbour's phase to node }
      \STATE   \COMMENT{Node  cannot change its phase anymore}
\ELSIF{ there exist  \mbox{ \bf and }
   \mbox{ \bf and } conditions of
  Lemma~\ref{prop:adjphase} are satisfied}
  \STATE \mbox{ \bf case } \mbox{ \bf then }  \STATE \mbox{ \bf case } \mbox{ \bf then } \STATE     \COMMENT{Delete  and  from }
\ENDIF
\STATE {\bf Visit all neighbors:}
\FORALL{  { \bf and}  } 
\IF{ \COMMENT{If it was not explored before}}
  \STATE   \COMMENT{Assign the phase to node }     
  \STATE 
\ELSIF{}  
  \STATE       \COMMENT{Add the phase to node } 
\ENDIF  
\STATE 
\COMMENT{ computed using extended Euclidean algorithm}
\STATE ,  \COMMENT{Rendezvous times}     
\STATE 
\ENDFOR
\end{algorithmic}
\end{algorithm}


The algorithm pseudo-code is reported in Alg.~\ref{alg:algorit2}. Each node  maintains information 
about nodes in its neighborhood, i.e., the node ID, the phases set , the wake-up period  
and two special flags  and . Special flags are used in order to determine  and . In particular 
(i)  means  unexplored (ii)  means node  discovered but  has not been fixed yet 
and all neighbors of  have been explored (iii)  when node  has been explored and therefore 
 and  have been fixed. Also,  means  that can adjust  (ii)  means 
 cannot change  value because some of its neighbors used . 

At the start of the algorithm each node  chooses  using {\tt PERIOD} and maintains a list of neighbors 
. Then node  is marked {\em unexplored} and {\em available} to change  value (). 
Hence, each node starts a random timer: when the timer expires node  freezes neighbors' timers 
and determines rendezvous times with all nodes in its neighborhood.

If  is unexplored, i.e., , node  checks whether it is in range of an already existing 
node  explored. If so,  just aligns ;  is then is marked {\em not available} 
to change its time axis anymore, i.e., . Otherwise, 
if  is {\em discovered}, i.e., ,  checks whether it is in range of two already existing 
 and  already explored without a phase in common between them and if the phase can be adjusted then 
 has two options: (i) if  is {\em available} to change its phase (), then  just aligns 
its phase to the novel phase, i.e., , (ii) if  is {\em not available} to 
change its phase (), then the novel phase is added to .

Then,  visits its neighbors for which no adaptation was performed. If  is {\em unexplored},  and  is marked {\em discovered}, 
. If  is {\em explored} and , the phase of  is added to . Finally,  marks 
itself {\em explored}.

The proof of correctness and the fact that Alg.~\ref{alg:algorit2} has message exchange 
is  is similar to the one for Alg.~\ref{alg:algorit1}:
\begin{thm}
{\tt PARALLEL WAKE-UP}:
\begin{enumerate}
\item[  i.] produces a wake-up schedule; 
\item[iii.] it requires  messages for a distributed implementation.
\end{enumerate}
\end{thm}

In principle, since multiple phases may be added at one node, it is possible that 
the output of the algorithm is a schedule that is not feasible. A simple bound on 
the average duty cycle (DC) is provided by the following
\begin{thm}
Under Alg.~\ref{alg:algorit2} , where  and  is the average node degree
\end{thm}
\proof
The algorithm adds at most  phases per node: 
\endproof


\begin{rem}
Observe that the above algorithm can be used to adapt the wake-up scheduling dynamically in case one or more nodes 
join the network. In particular, it is possible to combine the two algorithms in order to initialize a 
wake-up schedule network-wide with {\tt BFS WAKE-UP}, and use {\tt PARALLEL WAKE-UP} to add new nodes to existing schedules
 or to repair locally a schedule, should a node fail or loose synchronization.
\end{rem}


\subsection{Example}


We reported in Fig.~\ref{fig:example} a simple example 
illustrating the {\tt BFS WAKE-UP} and of the {\tt Parallel WAKE-UP} 
algorithm on a sample graph with  
nodes. In that instance , , 
 and . Several 
wake-up periods coexist: .

\begin{figure*}[t]
\centering
\begin{minipage}{4cm}
\includegraphics[width=2cm]{./FIG/topology2}
\end{minipage}
\begin{minipage}{4cm}{\begin{tabular}{|l|c|c|c|c|c|c|}
\hline
\textbf{ } & \textbf{} & \textbf{}& \textbf{}& \textbf{}&\textbf{}& \textbf{}\\
&  &   & &  &\textbf{BFS} & \textbf{PARALLEL}\\
\hline
\hline
 &  & & &  &  & \\
 &  & & &  &  & \\
 &   & & & &  & \\
 &  & & &   &  & \\
 &   & & &  &  & \\
 &   & & &  &  & \\
 &   & & &  &  & \\
\hline
\end{tabular}
}\end{minipage}
\caption{Example. {\tt BFS WAKE-UP} and {\tt PARALLEL WAKE-UP} on a sample topology. The table reports on the final output of the algorithms.}\label{fig:example}
\end{figure*}

The {\tt BFS WAKE-UP} run generates the list of visited 
nodes ; observe that every node aligned to 
. In the {\tt Parallel WAKE-UP} node  and  perform 
 the algorithm operations first, then , ,  and finally . 
As we can see from the example, {\tt Parallel WAKE-UP} assigns 
heterogeneous phases; at node  phase adaptation according 
to Thm.~\ref{prop:adjphase} is not possible so that finally 
.


\subsection{Particular cases}\label{sec:particular}


There exists specific cases when strictly tight wake-up schedules are attained easily.
\begin{lem}
Let for any node  be ,  and , then there exists a strictly tight wake-up schedule; it can be attained
using {\tt BFS WAKE-UP}. 
\end{lem} 
\begin{proof}
Consider any link  and let  and . The statement follows immediately 
from the Chinese Reminder theorem since  and one can use 
{\tt BFS WAKE UP} to assign the same phase to all nodes in the network. 
\end{proof}

In particular the following example shows a simple application of the above
case. Consider a tree topology  and assume nodes are labeled in order to respect the partial order induced 
by the distance from the root node . Let  be associated to the root node, and  be associated 
to nodes at level . A {\em feasible} schedule can be constructed as follows: root node  wakes at times .
Node  wakes up at  where  is the level occupied by node  and  is any positive integer 
function. In this case .



