One of the main issues in biometric authentication systems is to protect a biometric template database from compromise. Biometric information is so unique to each user and unchangeable during his or her lifetime. Once biometric template is leaked together with his or her identity, the person will face a severe risk of identity theft. Widely-used template protection systems for biometric authentication systems are tamper-proof hardware-based systems, where biometric template is stored in an ordinary storage as an encrypted form and decrypted only within a tamper-proof hardware  when matching is required. In these systems, even if the database is compromised, biometric information never made public. However,  the drawback of this conventional approach was the requirement of tamper-proof hardware, as it increases the deployment cost especially in high volume matching is required. To overcome this drawback, software-based template protection techniques are proposed recently in many literature\cite{}. Software-based template protection schemes are categorized into 2 approaches\cite{Nagar:2010tg}, feature transformation approach and biometric cryptosystems. Both of them introduces a user-specific key to transform a biometric template into a protected template. 

\subsection{Feature Transformation approach}
Feature transformation approach is first proposed in a paper written by Ratha, Connel and Bolle\cite{Ratha:2001gu} as {\it Cancelable biometrics}. In feature transformation approach, a randomness or key is introduced as a transformation parameter, and each original biometric feature is transformed into a deformed biometric feature. Main advantage of this approach is that it can take benefits from utilizing well-studied high performance algorithms. Thus, the challenge in this approach is to design a transformation function satisfies both (1) that closeness in original biometric feature space should preserve in the transformed feature space and (2) that it is hard to recover the original biometric feature from the transformed feature. On the contrary that feature transformation approach can enjoy the benefit of high-performance algorithms, schemes in this approach tends to have difficulties in theoretical analysis of protection performance such as irreversibility and unlinkability discussed later. Thus, many papers give experimental evidence for security analysis. 


Ratha et al. \cite{Ratha:2001gu} introduced the notion of {\it Cancelable biometrics} and proposed several schemes for fingerprint template protection\cite{Ratha:2007it}. Their approach is to displace fingerprint minutiae at different locations according to a irreversible locally smooth transformation. That is, a small change in a minutiae position before transformation leads to a small change in the minutiae position after transformation, but small correlation in minutia positions before and after transformation. Ratha et al.\cite{Ratha:2007it}  evaluated {\it Accuracy} (Section \ref{sect:Accuracy}) for the recognition performance and {\it Irreversibility}(Section \ref{sect:Irr}) for their schemes. They roughly estimated the complexity of irreversibility by the length of its binary representation.

Teoh et al.'s BioHash\cite{Jin:2004kc} and its subsequent papers\cite{Connie:2005wg,Teoh:2006kr,Teoh:2007dd} proposed distance-preserving transformations for biometric feature vectors multiplied with an randomized orthogonal transformation matrix. The randomized orthogonal matrix woks as a user-specific key, it introduces a low false accept rate. Irreversiblity of BioHash is analyzed in \cite{Jin:2004kc} and \cite{Kuan:2005ut}. In \cite{Jin:2004kc}, irreversibility is discussed based on evidences from {\it recognition performance} (Section \ref{sect:RecPerf}) metrics such as {\it accuracy} (Section \ref{sect:Accuracy}), {\it biometric performance} (Section \ref{sect:BioPerf}) and {\it diversity} (Section \ref{sect:Div}). As argued later, for example, in the real world, a fingerprint left on a glass may be abused by a malicious user, then {\it Diveristy} seems to give the complexity of an adversary to find the correct key. However, this discussion only covers a weak adversary whose attacking strategy is specific. A stronger adversary may take other strategies such as finding the correct key by directly inverting the transformation function utilizing the stolen fingerprint, etc. Likewise, those recognition performance metrics are not suitable for the evaluation of protection performance. In \cite{Kuan:2005ut}, irreversibility is discussed theoretically and experimentally. Their experimental analysis is similar to  \cite{Jin:2004kc}. In their theoretical analysis, irreversibility is defined as the complexity of finding an exact original biometric feature vector from a transformed template and its corresponding key. BioHash is a lossy function, hence it satisfies their notion of irreversibility with some security parameter. However, in the real situation, the adversary usually does not have to find an exact original biometric feature, but enough to find an biometric feature which can be accepted by the biometric authentication system. The latter is trivially easy, given a transformed template and its corresponding key, randomly chosen biometric features will be accepted with probability FAR. Thus, more realistic notion of irreversibility is required.

\subsection{Biometric cryptosystem}
Biometric cryptosysm refers to a series of research motivated by fuzzy commitment and fuzzy vault proposed by Juels and Watenburg\cite{Juels:1999kz} and Juels and Sudan\cite{Juels:2002hd} respectively. Instead of applying sophisticated feature extraction and matching algorithms, they abstracted the metric space of biometrics matching as a hamming distance or a set difference respectively, and make use of error-correcting codes to check if the distance of two biometric features are within a correctable range. Dodis, Reyzin and Smith\cite{Dodis:2004uh} generalized them to secure sketch covering any {\it transitive} metric space, that is, a metric space  has a family of permutations  such that  is distance preserving:  and for any two elements  there exists : . 

None of them conducted experimental analysis both on  {\it recognition performance} (Section \ref{sect:RecPerf}) and {\it protection performance} (Section \ref{sect:ProtectPerf}). Rather, {\it irreversibility} (Section \ref{sect:Irr}) for their {\it un-keyed} schemes are theoretically analyzed. They demonstrated that fuzzy schemes have strong {\it irreversibility} in a practical parameter setting, but introduced impractical assumptions. As shown in this paper, any {\it un-keyed} schemes cannot satisfy {\it irreversibility} in a practical setting for a biometrics application (see Theorem \ref{thm:UNARCH-IRR} in this paper). Those impractical assumptions are considered essential in the analysis. Namely, Juels and Watenburg\cite{Juels:1999kz} assumes uniform distribution on biometric features, and Juels and Sudan\cite{Juels:2002hd} does not assume uniform distribution on elements in a set whereas assumes elements in a set are chosen independently. Dodis, Reyzin and Smith\cite{Dodis:2004uh} evaluated  {\it irreversibility} of secure sketch and fuzzy extractor with a general distribution on biometric feature, hence falls to insecure with a practical parameter setting for biometrics applications.

Sutcu, Li and Memon\cite{Sutcu:2007um} applied a secure sketch\cite{Dodis:2004uh} to a face recognition system, and measured {\it biometric performance} and estimated a lower-bound of {\it irreversibility}. They reported degradation of recognition performance introduced by secure sketch was negligible, but the lower-bound of complexity to break {\it irreversibility} was barely  bits. Arakala, Jeffers and Horadam\cite{Arakala:2007vv} and Chang and Roy\cite{Chang:2007wz} applied to fingerprint recognition system and reported similar results.


\subsection{Related Security Metrics}
As we have seen until now, there are two separate line of research, and there exits a gap in the way of evaluation of  {\it recognition performance} and {\it protection performance} between feature transformation approach and biometric cryptosystem. 
Thus, relations of security statements were ambiguous, and it was not easy to compare the security of proposed schemes. Recently, there are attempts to try to unify the evaluation methods and give metrics applicable to all biometric template protection schemes. 

Nagar, Nandakumar and Jain\cite{Nagar:2010tg} proposed such security metrics. Their security metrics consists of six items: , , , ,  and . The first two items exactly correspond to our proposal, {\it accuracy} and {\it biometric performance}.  , the Intrusion Rate due to Inversion for the Same biometric system, and , the Intrusion Rate due to Inversion for a Different biometric system, are related to our metric of {\it --pseudo-authorized leakage irreversibility} in Definition \ref{def:PALIRR}. Our metric gives the upper-bound of the intrusion probability for all probabilistic polynomial-time inverters, whereas  and  give the intrusion probability for the best possible inverter.  and  can be evaluated experimentally, hence suitable metrics for algorithms in the feature transformation approach. However,  and  should be considered that it gives the lower assurance in {\it irreversibility}, as far as there is no evidence that the best possible inverter used in the evaluation is the best of all probabilistic polynomial-time inverters. Similarly,  , the Cross Match Rates in the Transformed feature domain, and , the Cross Match Rates in the Original feature domain, are related to our {\it diversity} and {\it --unlinkability}, respectively in Definition \ref{def:UNLINK}. 

Simoens, Yang, Zhou, Beato, Busch, Newton and Preneel\cite{SYZBBNP2012} proposed nearly all aspect of requirements normally expected to template protection algorithms, namely from technical performance such as recognition accuracy, throughput and storage requirement, protection performance through operational performance. Based on their proposal,  this paper focuses on the formal definitions of the recognition performance and the protection performance for precise discussions. For recognition performance,  their {\it accuracy}\cite{SYZBBNP2012} and {\it diversity}\cite{SYZBBNP2012} exactly corresponds to our {\it biometric performance} and {\it diversity}. Further, we introduced another {\it accuracy} which corresponds to  in Nagar et al.\cite{Nagar:2010tg} to demonstrate the performance advantage of two-factor template protection algorithms. For protection performance, their {\it irreversibility}\cite{SYZBBNP2012}  and {\it unlinkability}\cite{SYZBBNP2012} exactly corresponds to ours. {\it Irreversibility}\cite{SYZBBNP2012}  is further divided into {\it full-leakage irreversibility}, {\it authorized-leakage irreversibility} and {\it pseudo-aurhorized-leakage irreversibility} depending on the differences of goals for adversary. These three notions of {\it irreversibility} is formally defined and discussed their relations in Section \ref{sect:Irr}. {\it Unlinkability}\cite{SYZBBNP2012} is defined as the false cross match rate () and the false non-cross match rate (). These rate is measured as the performance of a {\it cross-comparator}. Similarly, if one could give an upper-bound of  and  for all probabilistic polynomial-time  {\it cross-comparator}, then {\it unlinkability} can be theoretically evaluated with the high assurance level. On the other hand, if these rates are given experimentally for the best possible  {\it cross-comparator},  {\it unlinkability} is evaluated with lower assurance level. These are discussed in more detail in Section \ref{sect:Unlink}.
