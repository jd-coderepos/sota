\documentclass[11pt]{llncs}
\usepackage{amsmath}
\usepackage{amssymb}
\usepackage{path}
\usepackage{xspace}
\usepackage{graphicx}
\usepackage[font=small, format=hang, labelfont=bf]{caption}
\usepackage[subrefformat=parens, labelfont=default]{subfig}
\usepackage{fullpage}
\usepackage{hyperref}
\usepackage{algorithm}
\usepackage{algorithmic}

\title{Approximating Minimum Manhattan Networks in Higher Dimensions}

\author{
Aparna~Das\inst{1}
\and
Emden~R.~Gansner\inst{2}
\and 
Michael~Kaufmann\inst{3}
\and 
Stephen~Kobourov\inst{1}
\and 
Joachim~Spoerhase\inst{4}
\and 
Alexander~Wolff\inst{4}
}

\authorrunning{A.~Das et al.}

\institute{
  Dept.~of Comp.~Sci., University of Arizona, Tucson, AZ, U.S.A.
\and
AT\&T Labs Research, Florham Park, NJ, U.S.A.
\and
Wilhelm-Schickard-Institut f\"ur Informatik, Universit\"at
  T\"ubingen, Germany 
\and
Institut f\"ur Informatik, Universit\"at W\"urzburg, Germany
}


\newcommand{\R}{\ensuremath{\mathbb{R}}}
\newcommand{\Nalg}{\ensuremath{N^\mathrm{alg}}\xspace}
\newcommand{\NE}{\ensuremath{\mathrm{NE}}\xspace}
\newcommand{\NW}{\ensuremath{\mathrm{NW}}\xspace}
\newcommand{\SE}{\ensuremath{\mathrm{SE}}\xspace}
\newcommand{\SW}{\ensuremath{\mathrm{SW}}\xspace}
\newcommand{\MSN}{\ensuremath{\pi_\mathrm{SN}}\xspace}
\newcommand{\MWE}{\ensuremath{\pi_\mathrm{WE}}\xspace}
\newcommand{\tE}{\ensuremath{t_\mathrm{E}}\xspace}
\newcommand{\tW}{\ensuremath{t_\mathrm{W}}\xspace}
\newcommand{\tS}{\ensuremath{t_\mathrm{S}}\xspace}
\newcommand{\tN}{\ensuremath{t_\mathrm{N}}\xspace}
\newcommand{\Nopt}{\ensuremath{N_\mathrm{opt}}\xspace}
\newcommand{\Nhor}{\ensuremath{N^\mathrm{hor}}\xspace}
\newcommand{\Nver}{\ensuremath{N^\mathrm{ver}}\xspace}
\newcommand{\Mver}{\ensuremath{M^\mathrm{ver}}\xspace}
\newcommand{\cost}{\ensuremath{\mathrm{weight}}\xspace}
\newcommand{\dir}{\ensuremath{\mathrm{dir}}}
\newcommand{\Nopthor}{\ensuremath{\Nopt^\mathrm{hor}}\xspace}
\newcommand{\Noptver}{\ensuremath{\Nopt^\mathrm{ver}}\xspace}
\newcommand{\Nxy}{\ensuremath{N_{xy}}\xspace}
\newcommand{\Txy}{\ensuremath{T_{xy}}\xspace}
\newcommand{\eps}{\ensuremath{\varepsilon}\xspace}
\newcommand{\opt}{\ensuremath{\mathrm{OPT}}\xspace}
\newcommand{\opttwod}{\ensuremath{\opt_\mathrm{2D}}\xspace}
\newcommand{\optxy}{\ensuremath{\opt_{xy}}\xspace}
\newcommand{\optz}{\ensuremath{\opt_z}\xspace}
\newcommand{\optup}{\ensuremath{\opt^\mathrm{up}}\xspace}
\newcommand{\rhorsa}{\ensuremath{\rho_\mathrm{3D-RSA}}\xspace}
\newcommand{\alphatwod}{\ensuremath{\alpha_\mathrm{2D}}\xspace}
\newcommand{\red}{\ensuremath{R}\xspace}
\newcommand{\blue}{\ensuremath{B}\xspace}
\newcommand{\BRP}{\ensuremath{\mathrm{BRP}}\xspace}
\newcommand{\rpatch}{\ensuremath{r_\mathrm{patch}}}
\newcommand{\etal}{et al.}

\graphicspath{{pic/}}

\renewcommand{\topfraction}{.99}
\renewcommand{\bottomfraction}{.99}
\renewcommand{\textfraction}{.01}

\newtheorem{observation}{Observation}
\newenvironment{pf}{\begin{proof}}{\qed\end{proof}}


\begin{document}

\maketitle

\begin{abstract}
  We study the minimum Manhattan network problem,
  which is defined as follows.  Given a set of points called
  \emph{terminals} in~, find a minimum-length network such that
  each pair of terminals is connected by a set of axis-parallel line
  segments whose total length is equal to the pair's Manhattan (that
  is, -) distance.  The problem is NP-hard in 2D and there is no
  PTAS for 3D (unless ). Approximation
  algorithms are known for 2D, but not for 3D.
  
  We present, for any fixed dimension~ and any , an
  -approxi\-ma\-tion algorithm. For 3D, we also give a
  -approximation algorithm for the case that the terminals are
  contained in the union of  parallel planes.
  \vskip0.5em
  \textbf{Keywords:} Approximation Algorithms, Computational Geometry,
  Minimum Manhattan Network
\end{abstract}

\section{Introduction}

In a typical network construction problem, one is given a set of
objects to be interconnected 
such that some constraints regarding the connections are fulfilled. 
Additionally, the network must be of little cost.  
For example, if the objects are points in Euclidean
space and the constraints say that, for some fixed , each pair of
points must be connected by a path whose length is bounded by~
times the Euclidean distance of the points, then the solution is a
so-called \emph{Euclidean -spanner}.  Concerning cost, one usually
requires that the total length of the network is proportional to the
length of a Euclidean minimum spanning tree of the points.  Such low-cost
spanners can be constructed efficiently~\cite{admss-esstl-95}.

In this paper, we are interested 
in constructing 1-spanners, with respect to the
Manhattan (or -) metric.  Rather than requiring that the total
length of the network 
is proportional to the minimum spanning tree of the points, 
our aim is to minimize the total length
(or \emph{weight}) of the network.  Note that the Euclidean 1-spanner
of a set of points is simply the complete graph (if no three points
are collinear) and hence, its weight is completely determined.
Manhattan 1-spanners, in contrast, have many
degrees of freedom and vastly different weights.

More formally, given two points~ and~ in -dimensional
space~, a \emph{Manhattan path} connecting~ and~ (a
-- \emph{M-path}, for short) is a sequence of axis-parallel
line segments connecting~ and~ whose total length equals the
Manhattan distance between~ and~.  
Thus an M-path is a monotone rectilinear path.
For our purposes, a set of axis-parallel line segments is a
\emph{network}.  Given a network~, its \emph{weight}  is the
sum over the lengths of its line segments.
A network~ \emph{Manhattan-connects} (or \emph{M-connects}) two
given points~ and~ if it ``contains'' a -- M-path~. 
Note that we slightly abuse the notation here: we 
mean pointwise containment, that is, we require 
.
Given a set~ of points---called \emph{terminals}---in~, a
network~ is a \emph{Manhattan network} (or \emph{M-network})
for~ if~ M-connects every pair of terminals in~.  
The \emph{minimum Manhattan network problem} (MMN) consists of
finding, for a given set~ of terminals, a minimum-weight M-network.
For examples, see Fig.~\ref{fig:examples}.


M-networks have important applications in several areas such as VLSI
layout and computational biology. For example, Lam
\etal~\cite{lap-pafst-03} used them in gene alignment in order to 
reduce 
the size of the search space of the Viterbi algorithm for pair hidden
Markov models.

\begin{figure}[tb]
  \subfloat[an M-network for \label{sfg:2D-MN}]{\parbox{.2\textwidth}{\centering\includegraphics[page=1]{example-2d}}}
  \hfill
  \subfloat[a minimum M-network for~\label{sfg:2D-MMN}]{\parbox{.19\textwidth}{\hspace*{-2ex}\centering\includegraphics[page=2]{example-2d}}\hspace*{-2ex}}
  \hfill
  \subfloat[a minimum M-network in 3D\label{sfg:3D-MMN}]{\parbox{.24\textwidth}{\hspace*{-5ex}\includegraphics{example-3d}}\hspace*{-5ex}}
  \hfill
  \subfloat[M-paths missing each other\label{sfg:miss}]{\parbox{.2\textwidth}{\includegraphics{paths-miss}}}
  \caption{Examples of M-networks in 2D and 3D.}
  \label{fig:examples}
\end{figure}

\subsection{Previous work} 

The 2D-version of the problem, 2D-MMN, was introduced by
Gudmundsson \etal~\cite{gln-ammn-01}.  They gave an 8- and a
4-approximation algorithm.  Later, the approximation ratio was
improved to~3~\cite{bwws-mmnpa-06,fs-s3amm-08t} and then to~2, which
is currently the best possible.  It was achieved in three different
ways: via linear programming~\cite{cnv-raamm-08}, using
the primal--dual scheme~\cite{n-eprmc-05} and with purely geometric
arguments~\cite{gsz-yaa2a-08}.  The last two algorithms run in  time, given a set of~ points in the plane.  A ratio of~1.5 was
claimed~\cite{su-15amm-05}, but apparently the proof is
incomplete~\cite{fs-s3amm-08t}.  
Chin \etal~\cite{cgs-mmnnp-11} finally
settled the complexity of 2D-MMN by proving it NP-hard.

A little earlier, Mu{\~n}oz \etal~\cite{msu-mmnp3-09}
considered 3D-MMN.  They showed that the 
problem is NP-hard and that it is 
NP-hard to approximate beyond a
factor of 1.00002.  
For the special case of 3D-MMN, where any cuboid spanned by two
terminals contains other terminals or is a rectangle, they gave a
-approximation algorithm, where~ denotes the best approximation
ratio for 2D-MMN.  They posed the design of approximation algorithms
for general 3D-MMN as an open problem.

\subsection{Related problems}

As we observe in Section~\ref{sec:relat-stein-type}, MMN is a special
case of the \emph{directed Steiner forest problem} (DSF).  More
precisely, an instance of MMN can be decomposed into a constant number
of DSF instances.  The input of DSF is an edge-weighted directed
graph~ and a set of vertex pairs.  The goal is to find a
minimum-cost subgraph of~ (not necessarily a forest) that connects
all given vertex pairs.  Recently, Feldman \etal~\cite{fkn-iaadsf-09}
reported, for any , an -approximation
algorithm for DSF, where~ is the number of vertices of the given
graph.  This bound carries over to D-MMN.

An important special case of DSF is the \emph{directed Steiner
  \emph{tree} problem} (DST).  Here, the input instance specifies an
edge-weighted digraph~, a \emph{root} vertex~, and a subset~
of the vertices of~ to which  must connect.  An optimum solution
for DST is a minimum-weight -rooted subtree of~ spanning~.
DST admits an -approximation for any~
\cite{cccdggl-aadsp-98}.

A \emph{geometric} optimization problem that resembles MMN
is the \emph{rectilinear Steiner arborescence problem} (RSA).
Given a set of points in~ with non-negative coordinates, a
rectilinear Steiner arborescence is a spanning tree 
that connects all points with M-paths to the origin.  As in MMN, the
aim is to find a minimum-weight network.  For 2D-RSA, there is a 
polynomial-time approximation scheme (PTAS)
\cite{lr-ptasrsap-00} based on Arora's technique for approximating
geometric optimization problems such as TSP~\cite{a-asnph-03}.   
It is not known whether 2D-MMN admits a PTAS.  
Arora's technique 
does not directly apply here as M-paths between terminals
forbid detours and thus may not respect portals.  




\subsection{Our contribution}  


We first present a -approximation algorithm for the special
case of 3D-MMN where the given terminals are contained in 
planes parallel to the -- plane; see Section~\ref{sec:k-planes}.

Our main result is an -approximation algorithm for
D-MMN, for any .  We first present the algorithm in detail
for three dimensions; see Section~\ref{sec:general-case}.  Since the
algorithm for arbitrary 
dimensions is a straightforward generalization of the algorithm for 3D
but less intuitive, we describe it in the appendix.

Our -approximation algorithm for D-MMN constitutes a
significant improvement 
upon the best known ratio of  for (general) directed
Steiner forest \cite{fkn-iaadsf-09}. We obtain this result by
exploiting the geometric structure of the problem.  To underline the
relevance of our result, we remark that the bound of  is the
best known result also for other directed Steiner-type problems such
as DST~\cite{cccdggl-aadsp-98} or even acyclic
DST~\cite{zelikovsky-adstapprox-97}. 

Our -approximation algorithm for the -planes case relies on
recent work by Soto and Telha~\cite{st-2dorg-11}.
They show that, given a set of red and blue points in the plane, one
can determine efficiently a minimum-cardinality set of points that
together \emph{pierce} all rectangles having a red point in the lower
left corner and a blue point in the upper right corner. Combining this
result with an approximation algorithm for 2D-MMN, 
yields an approximation algorithm for the 2-planes case.  We show how to
generalize this idea to  planes.

\section{Some Basic Observations}\label{sec:basic-observations}

We begin with some notation. Given a point , we
denote the -, - and -coordinate of~ by , , and
, respectively.  Given two points~ and~ in~, let
 be the \emph{rectangle spanned by~ and~}.
If a line segment is parallel to the -, -, or -axis, we say
that it is -, -, or -\emph{aligned}.
In what follows, we consider the
3-dimensional case of the MMN problem, unless otherwise stated.

\subsection{Quadratic Lower Bound for Generating Sets in 3D}
\label{sec:generating}

Intuitively, what makes 3D-MMN more difficult than 2D-MMN is the following: 
in 2D, if the bounding box of terminals~ and~ and the
bounding box of~ and~ cross (as in Fig.~\ref{sfg:2D-MMN}), then
any -- M-path will intersect any -- M-path, which yields
-- and -- M-paths for free (if~ and~ are the lower
left corners of their respective boxes).  A similar statement for 3D
does not hold; M-paths can ``miss'' each other---even if their
bounding cuboids cross; see Fig.~\ref{sfg:miss}.

Let us formalize this observation.  Given a set~ of
terminals, a set~ of pairs of terminals is a \emph{generating
  set}~\cite{kia-iammn-02} if any network that M-connects the pairs
in~ in fact M-connects \emph{all} pairs of terminals.
In~2D, any MMN instance has a generating set of linear
size~\cite{kia-iammn-02}.  Unfortunately this
result does not extend to~3D.  Below, we construct an
instance that requires a generating set of size .
The idea of using linear-size generating sets is 
exploited by several algorithms for
2D-MMN~\cite{cnv-raamm-08,kia-iammn-02}. The following theorem shows that these 
approaches do not easily carry over to~3D.


\begin{theorem}
  There exists an instance of 3D-MMN with  terminals that requires
  a generating set of size .
\end{theorem}

\begin{pf}
  We construct an instance that requires a generating set of size at
  least . The main idea of the construction is to ensure that
   of the terminal pairs must use an edge segment unique to
  that specific pair.  The input consists of two sets  and ,
  each with  terminals, with the following coordinates: for , terminal  is at  and terminal  is at .  Figure~\ref{fig:genpairex} shows the instance for
  .

  \begin{figure}
    \centering
    \includegraphics[scale=.4]{genpairex}
    \caption{The constructed network for .}
    \label{fig:genpairex}
  \end{figure}
  Consider any given generating set
   such that there is a pair , 
   and  that is not in~.
  We now construct a specific network that contains M-paths between all 
  terminal pairs in  but no M-path between .
  
  Consider any pair  such that  and
  .  The
  M-path from  to  has three segments: an -aligned segment 
  from  to , a -aligned segment 
  from  to , and a -aligned segment 
  from  to . To ensure an M-path between
  each generating pair  (similarly between
  ), we add M-paths between each pair of
  consecutive terminals in  (similarly for ) as follows:
  we connect  
by adding a -aligned segment from  to , a
  -aligned segment to , and an -aligned segment to
  ; see Fig.~\ref{fig:genpairex}.

  It is easy to verify that, in this construction, the M-path between
  terminals  and  must use
  the -aligned segment between  and
  .  Since this segment is added only between terminal
  pairs that are present in the generating set~, there is no
  M-path between terminals  and 
  which are not in . Thus, in order to obtain M-paths between all
  pairs of terminals in , we need at least all of the
   pairs in .
\end{pf}

\subsection{Hanan Grid and Directional Subproblems}
\label{sec:hann-grid-direct}

First, we note that any instance of MMN has a solution that is
contained in the \emph{Hanan grid}, the grid induced by the terminals;
see Fig.~\ref{fig:examples}(a).  Gudmundsson \etal~\cite{gln-ammn-01}
showed this for~2D; their proof generalizes to higher dimensions.  In
what follows, we restrict ourselves to finding feasible solutions
that are contained in the Hanan grid.

Second, to simplify our proofs, we consider the \emph{directional}
subproblem of 3D-MMN which consists of connecting all terminal
pairs  such that~ \emph{dominates}~, that is, , , , and .  We call such
terminal pairs \emph{relevant}.  

The idea behind our reduction to the directional subproblem is that
any instance of 3D-MMN can be decomposed into four subproblems of this
type.  One may think of the above-defined directional subproblem as
connecting the terminals which are oriented in a north-east (NE)
configuration in the -- plane (with increasing
-coor\-di\-nates).  Analogous subproblems exist for the
directions~NW, SE, and~SW.  Note that any terminal pair belongs to one
of these four categories (if seen from the terminal with smaller
-coordinate).

The decomposition extends to higher dimensions , by fixing the
relationship between  for one dimension (for example, ), and
enumerating over all possible relationships for the remaining 
dimensions. This decomposes D-MMN into  
subproblems, which is a constant number of subproblems as we consider
 to be a fixed constant.  

This means that any
-approximation algorithm for the directional subproblem leads to
an -approxi\-mation algorithm for the general case.
Thus we can focus on designing algorithms for the directional subproblem.  

\begin{observation} 
  \label{obs:dir-sub} 
  Any instance of MMN can be decomposed into a constant number of
  directional subproblems.  Thus a -approximation algorithm for
  the directional subproblem leads to an -approximation
  algorithm for MMN.
\end{observation}
     

\subsection{Relation to Steiner Problems}
\label{sec:relat-stein-type}

We next show that there is an approximation-preserving reduction
from directional 3D-MMN to the directed Steiner forest (DSF) problem,
which by Observation~\ref{obs:dir-sub}, carries over up to a constant
factor, to general 3D-MMN. 

Let  be a set of  points in .  Let  be the
Hanan grid induced by .  We consider  as an undirected graph
where the length of each edge equals the Euclidean distance between
its endpoints.  We orient each edge in~ so that, for any edge
 in 
the resulting digraph~, the start node  dominates the end node
.  We call  the \emph{oriented Hanan grid} of~.  
Now let
 be a relevant pair of points in , that is, 
dominates~.  Any M-path in  connecting  to  corresponds
to a directed path in~ from~ to~.  The converse also holds:
every directed path in~ corresponds to an M-path in~.

Let  be an instance of directional 3D-MMN and let  be an 
instance of DSF where the input graph is  and where every relevant
terminal pair of  has to be connected.  Then, each feasible solution 
of  contained in  corresponds to a sub-graph~ of~ that
connects every relevant terminal pair, and is therefore a 
feasible solution to .  It is easy to see that  has the same
cost as , 
as  uses the oriented version of each edge of .
Conversely, every feasible solution~ for~ corresponds to a
subgraph  of  that M-connects every relevant terminal pair.
Therefore,  is a feasible solution to  with the same cost as
.  This establishes an efficiently computable one-to-one
correspondence between feasible solutions to  that are contained in
 and feasible solutions to .  Since there is an optimum
solution to~ contained in~~\cite{gln-ammn-01}, this is an
approximation-preserving reduction from directional 3D-MMN to DSF.

By means of the above transformation of the Hanan grid into a digraph,
we also obtain an approximation-preserving reduction from 3D-RSA to
DST.  We use this later in
Section~\ref{sec:general-case} to develop an approximation algorithm
for 3D-MMN.  Let  be an instance of 3D-RSA given by a set  of
terminals with non-negative coordinates that are to be M-connected to
the origin .  We construct an instance  of DSF as above where
 is the set of node pairs to be connected.
Note that any feasible solution to  is, without loss of
generality, a tree.  Hence,  is an instance of DST with root~.
All in all, we have an approximation-preserving reduction from 3D-RSA
to DST. 

\section{The -Plane Case}
\label{sec:k-planes}

In this section we consider 3D-MMN, under the assumption that the
set~ of terminals is contained in the union of  planes
 that are parallel to the -- plane.  Of course,
this assumption always holds for some .  We present a
-approximation algorithm, which outperforms our algorithm for
the general case in Section~\ref{sec:general-case} if .  

Let~\Nopt be some fixed minimum M-network for~, let \Nopthor be the
set of all -aligned and all -aligned segments in~\Nopt, and let
\Noptver be the set of all -aligned segments in~\Nopt.  Let~\opt
denote the weight of~\Nopt.  Clearly, \opt does not depend on the
specific choice of~\Nopt; the weights of~\Nopthor and~\Noptver,
however, may depend on~\Nopt. For , let  be the set of terminals in plane~.  Further, let~\Txy
be the projection of~ onto the -- plane. 

Our algorithm consists of two phases.  Phase~I computes a set~\Nhor of
horizontal (that is, - and -aligned) line segments, phase~II
computes a set~\Nver of vertical (that is, -aligned) line segments.
Finally, the algorithm returns the set . 

Phase~I is simple; we compute a 2-approximate M-network~\Nxy for~\Txy
(using the algorithm of Guo \etal~\cite{gsz-yaa2a-08}) and
project~\Nxy onto each of the planes .  Let~\Nhor be the union of
these projections.  Note that \Nhor M-connects any pair of terminals
that lie in the same plane.
\begin{observation}
  \label{obs:Nhor}
  .
\end{observation}
\begin{pf}
  The projection of~\Nopthor to the -- plane is an M-network
  for~\Txy.  Hence, .  Adding up over the 
  planes yields the claim.
\end{pf}
In Phase~II, we construct a {\em pillar network} by computing a set~\Nver of
vertical line segments, so-called \emph{pillars}, of total cost at
most .  This yields an overall approximation
factor of~ since .

Below we describe Phase~II of our algorithm 
for the directional subproblem 
that runs in direction north-east (NE) in the --
plane (with increasing -coordinates). For this directional subproblem, we
construct a pillar
network~ of weight at most  that, together
with~\Nhor, M-connects all relevant pairs.   
We solve the analogous
subproblems for the directions~NW, SE, and~SW in the same fashion.
Then \Nver is the union of the four partial solutions and has weight
at most , as desired.


Our directional subproblem is closely linked to the
\emph{(directional) bichromatic rectangle piercing problem} (\BRP),
which is defined as follows.  Let~\red and~\blue be sets of red and
blue points in , respectively, and let 
denote the set of axis-aligned rectangles each of which is spanned by
a red point in its SW-corner and a blue point in its NE-corner.  Then
the aim of \BRP is to find a minimum-cardinality set~
such that every rectangle in~ is \emph{pierced},
that is, contains at least one point in~.  The points in~ are
called \emph{piercing points}. 

The problem dual to BRP is the \emph{(directional) bichromatic
  independent set of rectangles problem} (BIS) where the goal is to
find the maximum number of pairwise disjoint rectangles in , given the sets~\red and~\blue.

Recently, Soto and Telha \cite{st-2dorg-11} proved a beautiful
min--max theorem saying that, for , the
minimum number of piercing points always \emph{equals} the maximum
number of independent rectangles.  This enabled them to give efficient
exact algorithms for \BRP and BIS running in 
worst-case time or  expected time, where the
-notation ignores polylogarithmic factors,  is the exponent for fast matrix multiplication, and
 is the input size.

The details of Phase~II appear, for  planes, in Section~\ref{sec:pillar2},
and, for  planes, in Section~\ref{sec:pillark}.
Algorithm~\ref{alg:kplanes} summarizes of our -planes algorithm.

\begin{algorithm}[htpb]
\caption{-Planes Algorithm}
\label{alg:kplanes}
\mbox{Input: Set  of terminals contained in the union of planes , all parallel to the -- plane.}\
    \|\Mver\|=\sum_{j=1}^{k-1}\|M_{j,j+1}\|=\sum_{j=1}^{k-1}|M_{j,j+1}|
    \cdot d_{j,j+1} \ge \sum_{j=1}^{k-1}|P_{i^\star}| \cdot d_{j,j+1} =
    |P_{i^\star}| \cdot d_{1,k} = 
    \|P_{i^\star}\|.   
  
  \label{eqn:weight}
  \cost_z(j,j')\leq \|M_{j,j'}\|+\cost_z(j,i')+\cost_z(i'+1,j')\, ,

  \cost_z(j,j')\leq (j'-j)\|M_{j,j'}\|.

  \sum_{i=1}^cr\left(\frac{n}{c}\right)\opt_i^x\leq
  r\left(\frac{n}{c}\right)\sum_{i=1}^c\|\Nopt\cap C_i^x\|\leq
  r\left(\frac{n}{c}\right)\opt\,.

  r(n)\opt\le 3c^2\opt+\rpatch(n)\opt+3r\left(\frac{n}{c}\right)\opt\,.

    \rho\sum_{C_{ijk}\text{
	relevant}}\optup_{ijk} \quad\le\quad
    \rho\sum_{C_{ijk}\text{ relevant}}\|N'\cap
    C_{ijk}\| \quad\le\quad 4\rho \|N'\|\,. 
  
  \sum_{j=1}^cr\left(\frac{n}{c}\right)\opt_j^i\leq
  r\left(\frac{n}{c}\right)\sum_{j=1}^c\|\Nopt\cap C_j^i\|\leq
  r\left(\frac{n}{c}\right)\opt\,.

  r(n)\opt \leq dc^{d-1}\cdot\opt+d\cdot
  r\left(\frac{n}{c}\right)\opt+r_{\text{patch}}(n)\opt\,. 

    \rho\sum_{C_{j_1,\dots,j_d}\text{
	relevant}}\optup_{j_1,\dots,j_d} \quad\le\quad
    \rho\sum_{C_{j_1,\dots,j_d}\text{ relevant}}\|N'\cap
    C_{j_1,\dots,j_d}\| \quad\le\quad 2^d\rho \|N'\|\,. 
  
  The last inequality follows from the fact that each segment of 
  occurs in at most  cuboids.  The lemma follows since
  .
\end{pf}

\paragraph{Running time.}

Let  denote the running time of the algorithm for 
terminals. The running time is dominated by patching and the recursive
treatment of slabs. Using the DST algorithm of Charikar
\etal~\cite{cccdggl-aadsp-98}, patching cuboid  requires time 
, where  is the number of terminals in
. As each cuboid is patched at most twice and there are 
cuboids, patching requires total time . The algorithm is applied recursively to 
slabs. This yields the recurrence ,
which leads to the claimed running time.


\end{document}
