



\documentclass[11pt]{article}

\usepackage[dvips]{graphicx}
\usepackage{epsfig}
\usepackage{version}

\usepackage{geometry}
\geometry{hmargin=3.5cm,vmargin=2.8cm}

\usepackage{amssymb,amsmath,eepic,graphics,color}

\usepackage{epsfig}


\newenvironment{proof}{\noindent\textit{Proof: }}{{\hfill }}
\newtheorem{lemma}{Lemma}[section]
\newtheorem{theorem}[lemma]{Theorem}
\newtheorem{proposition}[lemma]{Proposition}
\newtheorem{corollary}[lemma]{Corollary}
\newtheorem{reduction}{Rule}
\newtheorem{fact}[lemma]{Fact}
\newtheorem{definition}[lemma]{Definition}
\newtheorem{remark}[lemma]{Remark}
\newtheorem{claim}[lemma]{Claim}

\newtheorem{observation}[lemma]{Observation}

\newcommand{\ssi}{if and ony if}
\newcommand{\eg}{\textit{e.g.}}
\newcommand{\gbc}{\textsc{Not-1-in-3-edge-triangle}}
\newcommand{\tgbc}{\textsc{Tripartite-Not-1-in-3-edge-triangle}}
\newcommand{\hyp}{NP \subseteq coNP / poly}







\begin{document}



\title{\textbf{On the (non-)existence of polynomial kernels for -free edge modification problems}
\thanks{Research supported by the AGAPE project  (ANR-09-BLAN-0159).}}


\author{Sylvain Guillemot \and Christophe Paul \and Anthony Perez \\\\
 Lehrstuhl f\"ur Bioinformatik, Friedrich-Schiller Universit\"at Jena \\
 Universit\'e Montpellier II - CNRS, LIRMM}


\date{}

\maketitle


\begin{abstract}
Given a graph  and an integer , an edge modification problem for a graph property  consists in deciding whether there exists a set of edges  of size at most  such that the graph  satisfies the property . In the  \emph{edge-completion problem}, the set  of edges is constrained to be disjoint from ; in the  \emph{edge-deletion problem},  is a subset of ; no constraint is imposed on  in the  \emph{edge-edition problem}. 
A number of optimization problems can be expressed in terms of graph modification problems which have been extensively studied in the context of parameterized complexity. When parameterized by the size  of the edge set , it has been proved that if  is an hereditary property characterized by a finite set of forbidden induced subgraphs, then the three  edge-modification problems are FPT~\cite{Cai96}. It was then natural to ask~\cite{Cai96} whether these problems also admit a polynomial size kernel. Using recent lower bound techniques, Kratsch and Wahlstr\"om answered this question negatively~\cite{KW09}. However, the problem remains open on many natural graph classes characterized by forbidden induced subgraphs. Kratsch and Wahlstr\"om asked whether the result holds when the forbidden subgraphs are paths or cycles and pointed out that the problem is already open in the case of -free graphs (i.e. cographs). This paper provides positive and negative results in that line of research. We prove that parameterized cograph edge modification problems have cubic vertex kernels whereas polynomial kernels are unlikely to exist for the -free and -free edge-deletion problems for large enough .
\end{abstract}


\section{Introduction}

An edge modification problem aims at changing the edge set of an input graph  in order to get a certain property  satisfied (see~\cite{NSS01} for a recent study). Edge modification problems cover a broad range of graph optimization problems among which completion problems (\eg ~\textsc{minimum fill-in}, \emph{a.k.a}  \textsc{chordal graph completion}~\cite{Ros72,TY84}), edition problems (\eg ~\textsc{cluster editing}~\cite{SST04}) and edge deletion problems (\eg ~\textsc{maximum planar subgraph}~\cite{GJ79}).  In a completion problem, the set  of modified edges is constrained to be disjoint from ; in an edge deletion problem,  has to be a subset of ; and in an edition problem, no restriction applies to . These problems are fundamental in graph theory and play an important role in computational complexity theory (indeed they represent a large number of the earliest NP-Complete problems~\cite{GJ79}). Edge modification problems are also relevant in the context of applications as graphs are often used to model data sets which may contain errors. Adding or deleting an edge thereby corresponds to fixing some false negatives or false positives (see \emph{e.g.}~\cite{SST04} in the context of \textsc{cluster editing}). Different variants of edge modification problems have been studied in the literature such as graph sandwich problems~\cite{GKS95}.
Most of the edge modification problems turns out to be NP-Complete~\cite{NSS01} and approximation algorithms exist for some known graph properties (see \eg~\cite{KS07,ZW08}). But for those who want to compute an exact solution, fixed parameter algorithms~\cite{DF99,FG06,Nie06} are a good alternative to cope with such hard problems. In the last decades, edge modification problems have been extensively studied in the context of fixed parameterized complexity (see~\cite{Cai96,FLR07,HPTV07}). 

A parameterized problem  is \emph{fixed parameter tractable} (FPT for short) with respect to parameter  whenever it can be solved in time , where  is an arbitrary computable function~\cite{DF99,Nie06}. 
In the context of edge modification problems, the size  of the set  of modified edges is a natural parameterization. The generic question is thereby whether a given edge modification problem is FPT for this parameterization. More formally:

\medskip
\noindent
\textsc{Parameterized  edge--modification Problem}\\
\textbf{Input:} An undirected graph .\\
\textbf{Parameter:} An integer .\\
\textbf{Question:} Is there a subset  with
 such that the graph  satisfies .


\medskip
A classical result of parameterized complexity states that a parameterized problem  is FPT if and only if it admits a \emph{kernelization}. 
A \emph{kernelization} of a parameterized problem  is a polynomial time algorithm  that given an instance  computes an equivalent instance  such that the size of  and  are bounded by a computable function  depending only on the parameter . The reduced instance  is called a \emph{kernel} and we say that  admits a \emph{polynomial kernel} if the function  is a polynomial. The equivalence between  the existence of an FPT algorithm and the existence of a kernelization only yields kernels of (at least) exponential size. Determining whether an FPT problem has a polynomial (or even linear) size kernel is thus an important challenge. Indeed, the existence of such polynomial time reduction algorithm (or pre-processing algorithm or \emph{reduction rules}) really speed-up the resolution of the problem, especially if it is interleaved with other techniques~\cite{NR00}. However, recent results proved that not every fixed parameter tractable problem admits a polynomial kernel~\cite{BDFH08}. 

Cai~\cite{Cai96} proved that if  is an hereditary graph property characterized by a finite set of forbidden subgraphs, then the \textsc{parameterized  modification} problems (edge-completion, edge-deletion and edge-edition) are FPT. It was then natural to ask~\cite{Cai96} whether these  edge-modification problems also admit a polynomial size kernel. Using recent lower bound techniques, Kratsch and Wahlstr\"om answered negatively this question~\cite{KW09}. However, the problem remains open on many natural graph classes characterized by forbidden induced subgraphs. Kratsch and Wahlstr\"om asked whether the result holds when the forbidden subgraphs are paths or cycles and pointed out that the problem is already open in the case of -free graphs (i.e. cographs). In this paper, we prove that \textsc{parameterized cograph edge modification} problems have cubic vertex kernels whereas polynomial kernels are unlikely to exist for \textsc{-free} and \textsc{-free edge deletion} problems for large enough .
The NP-Completeness of the cograph edge-deletion and edge-completion problems have been proved in~\cite{EC88}.

\paragraph{Outline of the paper.}
We first establish structural properties of optimal edge-modification sets with respect to modules of the input graph (Section 2). These properties allow us to design general reduction rules (Section 3.1). We then establish cubic kernels using an extra sunflower rule (Section 3.2 and 3.3). Finally, we show it is unlikely that the \textsc{-free} and the \textsc{-free edge-deletion} problems have polynomial kernels (Section 4).









\section{Preliminaries}

\subsection{Notations} 

We only consider finite undirected graphs without loops nor multiple edges. Given a graph , we denote by  the edge of  between the vertices  and  of . We set  and  (subscripts will be used to avoid possible confusion). The neighbourhood of a vertex  is denoted by . If  is a subset of vertices, then  is the subgraph induced by  (i.e. any edge  between vertices  belongs to ). Given a set of pairs of vertices  and a subset ,  denotes the pairs of  with both vertices in . Given two sets  and , we denote by  their symmetric difference.

\subsection{Fixed parameter complexity and kernelization}

We let  denote a finite alphabet and  the set of natural numbers. A \emph{(classical) problem}  is a subset of , and a string  is an \emph{input} of . A \emph{parameterized problem}  over  is a subset of . The second component of an input  of a parameterized problem is called the \emph{parameter}. Given a parameterized problem , one can derive its unparameterized (or classical) version  by , where  is a symbol that does not belong to .

A parameterized problem   is \emph{fixed parameter tractable} (FPT for short) if there is an algorithm which given an instance  decides whether 
in time  where  is an arbitrary computable function (see~\cite{DF99,FG06,Nie06}). 
A \emph{kernelization} of a parameterized problem  is a polynomial time algorithm  which given an instance  outputs an instance  such that 
\begin{enumerate}
\item  and

\item  for some computable function .
\end{enumerate}

The reduced instance  is called a \emph{kernel} and we say that  admits a \emph{polynomial kernel} if the function  is a polynomial. 
It is well known that a parameterized problem  is FPT if and only if it has a kernelization~\cite{Nie06}. But this equivalence only yields (at least) exponential size kernels. Recent results proved that it is unlikely that every fixed parameter tractable problem admits a polynomial kernel~\cite{BDFH08}. These results rely on the notion of \emph{(or-)composition algorithms} for parameterized problems, which together with a polynomial kernel would imply a collapse on the polynomial hierarchy~\cite{BDFH08}.
An \emph{or-composition algorithm} for a parameterized problem  is an algorithm that receives as input a sequence of instances  with  for , runs in time polynomial in  
and outputs an instance  of  such that:
\begin{enumerate}
\item  for some  and
\item  is polynomial in .
\end{enumerate}

A parameterized problem admitting an \emph{or-composition algorithm} is said to be \emph{or-compositional}.

\begin{theorem}\cite{BDFH08,FS08}\label{th:or-comp}
Let  be an or-compositional parameterized problem whose unparameterized version  is NP-complete. The problem  does not admit a polynomial kernel unless .
\end{theorem}

Let  and  be parameterized problems. A \emph{polynomial time and parameter transformation} from  to  is a polynomial time computable function  which given an instance  outputs an instance  such that 
\begin{enumerate}
\item  and

\item  for some polynomial .
\end{enumerate}

\begin{theorem}\cite{BTY09}\label{th:ppt}
Let  and  be parameterized problems and let  and  be their unparameterized versions. Suppose that  is NP-complete and  belongs to NP. If there is a polynomial time and parameter transformation from  to  and if  admits a polynomial kernel, then  also admits a polynomial kernel.
\end{theorem}

\subsection{Modular decomposition and cographs}

A \emph{module} in a graph  is a set of vertices  such that for any  either  or . Clearly if  or , then  is a \emph{trivial} module. A graph without any non-trivial module is called \emph{prime}. 
For two disjoint modules  and , either all the vertices of  are adjacent to all the vertices of  or none of the vertices of  is adjacent to any vertex of . A partition  of the vertex set  whose parts are modules is a \emph{modular partition}. A \emph{quotient graph}  is associated with any modular partition : its vertices are the parts of  and there is an edge between  and  iff  and  are adjacent in .

A module  is \emph{strong} if for any module  distinct from , either  or  or . It is clear from definition that the family of strong modules arranges in an inclusion tree, called the \emph{modular decomposition tree} and denoted . Each node  of  is associated with a quotient graph  whose vertices correspond to the children  of . (see Figure~\ref{fig:md-tree} for an example). 
We say that a node  of  is \emph{parallel} if  has no edge, \emph{series} if  is complete, and \emph{prime} otherwise. 
For a survey on modular decomposition theory, refer to~\cite{HP10}.

\begin{figure}[t]
\centerline{\includegraphics[scale=0.75]{md.pdf}}
\caption{A graph  and its modular decomposition tree . The root of  is prime and its quotient graph is the 5 vertex graph depicted eside. Every other node is either parallel or series.
\label{fig:md-tree}
}
\end{figure}

\begin{definition}
Let   and  be two vertex disjoint graphs. The \emph{series composition} of  and  is the graph . The \emph{parallel composition} of  and  is the graph 
\end{definition}

Parallel and series nodes in the modular decomposition tree respectively correspond to a parallel and series composition of their children.  \\

Cographs are commonly known as -free graphs (a  is an induced path on four vertices). However, they were originally defined as follows:

\begin{definition}[\cite{BLS99}]
\label{def:cographs}
	A graph is a cograph if it can be constructed from single vertex graphs by a sequence of parallel and series composition.
\end{definition}

In particular, this means that the modular decomposition tree of a \emph{cograph} does not contain any prime node. It follows that cographs are also known as the totally decomposable graphs for the modular decomposition.

\section{Polynomial kernels for cograph modification problems}
\label{sec:kernels}

\subsection{Modular decomposition based reduction rules}


Since cographs correspond to -free graphs, cograph edge-modification problems consist in adding or deleting at most  edges to the input graph in order to make it -free. The use of the modular decomposition tree in our algorithms follows from the following observation: 

\begin{observation}\label{obs:p4} [Folklore]
Let  be a module of a graph  and  be four vertices inducing a  of , then .
\end{observation}

This means that given a modular partition  of a graph , any induced  of  is either contained in a part of  or intersects the parts of  in at most one vertex. This observation allows us to show that a cograph edge-modification problem can be solved independently on modules of the partition  and on the quotient graph , as stated by the following results:

\begin{observation} \label{obs:module}
Let  be a non-trivial module of a graph . Let  be an optimal edge-deletion (resp. edge-completion, edge-edition) set of  and  let  be an optimal edge-deletion (resp. edge-completion, edge-edition) set of . Then 

is an optimal edge-deletion (resp. edge-completion, edge-edition) set of .
\end{observation}

\begin{proof}
By Observation~\ref{obs:p4}, it follows that  is -free, thereby  is an edge-deletion set. 
As being a cograph is an hereditary property,  is an edge-deletion set of . Now observe that   since otherwise , which would contradict the optimality of . The same argument holds for edge-completion and edge-edition sets.
 \end{proof}

\begin{lemma} \label{lem:one-module}
Let   be a module of a graph . There exists an optimal edge-deletion (resp. edge-completion, edge-edition) set  such that  is a module of the cograph .
\end{lemma}

\begin{proof}
Let  be an optimal edge-deletion set  and denote . Let  be a vertex of  such that  is minimum.
We argue that the following set of edges is an optimal edge-deletion set:

First observe that by construction  is a module in the graph  and that by the choice of , . Let us prove that  is -free. As  and  are respectively isomorphic to  and , they are -free. So if  contains an induced , its vertices  intersect  and . As  is a module of  it follows by Observation~\ref{obs:p4} that  (say ). It follows by construction of , that  also induces a  in , contradicting the assumption that  is an edge-deletion set. So we proved that  is an edge-deletion set of  which preserves the module  and is not larger than . 
The same proof holds for edge-completion and edge-edition sets.
\end{proof}

\begin{lemma}
\label{lem:modules}
Let  be an arbitrary graph. There exists an optimal edge-deletion (resp. edge-completion, edge-edition) set  such that every module  of  is module of the cograph .
\end{lemma}

\begin{proof}
We prove the statement for edge-deletion sets by induction on the number of modules of a graph. The same proof applies for edge-completion and edge-edition sets. Observe that the result trivially holds if  is a prime graph and follows from Lemma~\ref{lem:one-module} if  contains a unique non-trivial module.  

Let us now assume that the property holds for every graph with at most  non-trivial modules. Let  be a graph with  non-trivial modules and let  be a non-trivial module of  which is minimal for inclusion. By induction hypothesis, the statement holds on  (since it is prime) and on the graph  where  has been contracted to a single vertex  (since it contains at most  non-trivial modules). The conclusion follows from Observation~\ref{obs:module}.
 \end{proof}\\



We now present three reduction rules which apply to the three cograph edge-modification problems we consider. The second reduction rule is not required to obtain a polynomial kernel for each of these problems. However, it will ease the analysis of the structure of a reduced graph.

\begin{reduction} \label{rule1}
Remove the connected components of  which are cographs.
\end{reduction}

\begin{reduction} \label{rule2}
If  is a connected component of , then replace  by .
\end{reduction}

\begin{reduction} \label{rule3}
If  is a non-trivial module of  which is strictly contained in a connected component and is not an independent set of size at most , then return the graph  where  is obtained from  by replacing  by an independent set module of size .
\end{reduction}

Observe that if  is a cograph, adding a disjoint copy to the graph is useless since it will then  be removed by Rule~\ref{rule1}. 

\begin{lemma} \label{lem:rules}
The reduction rules \ref{rule1}, \ref{rule2} and \ref{rule3} are safe and can be carried out in linear time.
\end{lemma}

\begin{proof}
The three rules can be computed in linear time using any linear time modular decomposition algorithm~\cite{HP10}. 
The first rule is trivially safe. The second rule is safe by Lemma~\ref{lem:modules}. The safeness of Rule~\ref{rule3} also follows from Lemma~\ref{lem:modules}: there always exists an optimal solution that updates all or none of the edges between any two disjoint modules. Thereby if a module  has size larger than , none of the edges (or non-edges)  with ,  can be changed in such a solution. Shrinking  into an independent set of size  and adding a disjoint copy of  (to keep track of the edge modification inside the module) is thereby safe.
\end{proof}\\

The analysis of the size of the kernel relies on the following structural property of the modular decomposition tree of an instance reduced under Rule~\ref{rule1}, Rule~\ref{rule2} and Rule~\ref{rule3}.

\begin{observation} \label{obs:reduced-123}
Let  be a graph reduced under Rule~\ref{rule1}, Rule~\ref{rule2} and Rule~\ref{rule3}. If  is a non prime connected component of , then the modules of  are independent sets of size at most .
\end{observation}

\begin{proof}
By Rule~\ref{rule2}, none of the connected components of  results from a series composition. By Rule~\ref{rule3}, a module which is not the union of some connected components of  has size at most  and is an independent set.
\end{proof}\\

Observe that these three reduction rules preserve the parameter. However, Rule~\ref{rule3} increases the number of vertices of the instance. Nevertheless, we will be able to bound the number of vertices of a reduced instance. \\

It remains to show that computing a reduced graph requires polynomial time. Let us mention that it is safe to apply Rule~\ref{rule2} and Rule~\ref{rule3} only on strong modules (in Rule~\ref{rule2},  can be chosen as a strong module).  

\begin{lemma} \label{lem:gen-reduced}
Given a graph , computing a graph reduced under Rule~\ref{rule1}, Rule~\ref{rule2} and Rule~\ref{rule3} requires polynomial time.
\end{lemma}

\begin{proof}
Let us say that a module  of  is \emph{reduced} if it is an independent set of size at most  or the disjoint union of some connected components of  (observe that connected components of  are also modules of ). By Observation~\ref{obs:reduced-123}, if  is reduced under Rule~\ref{rule1}, Rule~\ref{rule2} and Rule~\ref{rule3}, then every module of  is reduced. Notice that if every strong module of  is reduced, then every module of  is reduced.
So to prove the statement, we count the number of strong modules (\textit{i.e.} nodes of the modular decomposition tree ) which are not reduced. 

Let us also remark that if a connected component  is a cograph with at least two vertices, then a series of applications of Rule~\ref{rule2} eventually transforms  in a set of isolated vertices. This means that we can assume that the applications of Rule~\ref{rule1} is postponed to the end of the reduction process. This will ease the argument below.

When Rule~\ref{rule3} is applied, then by definition the number of non-reduced strong modules decreases by one. When Rule~\ref{rule2} is applied, unless  is an independent set of size at most , then the number of non-reduced strong modules also decreases by one. But observe that if  is an independent set of size at most , then its vertices will be removed by Rule~\ref{rule1} as they will become isolated vertices. As the number of strong modules of a graph is bounded by the number of vertices, this proves that a series of at most  applications of Rule~\ref{rule2} and Rule~\ref{rule3} is enough to compute a reduced graph.
\end{proof}

\subsection{Cograph edge-deletion (and edge-completion)}

In addition to the previous reduction rules, we need the classical \emph{sunflower} rule to obtain a polynomial kernel for the parameterized cograph edge-deletion problem.

\begin{reduction} \label{rule:sunflower-del}
If  is an edge of  that belongs to a set  of at least  's such that  is the only common edge of any two distinct 's of , then remove  and decrease  by one.
\end{reduction}

\begin{observation} \label{obs:rule-del}
The reduction rule~\ref{rule:sunflower-del} is safe and can be carried out in polynomial time.
\end{observation}

\begin{proof}
It is clear that the edge  has to be deleted as otherwise at least  edge deletions would be required to break all the 's of the set . Such an edge, if it exists, can be found in polynomial time if one computes the set of all 's of the input graph (which can be done in  time).
\end{proof}\\

To analyse the size of a reduced graph , we study the structure of the cograph  resulting from the removal of an optimal (of size at most ) edge-deletion set . The modular decomposition tree (or cotree) is the appropriate tool for this analysis.

\newpage 

\begin{theorem} \label{th:deletion}
The parameterized cograph edge-deletion problem admits a cubic vertex kernel.
\end{theorem}
\begin{proof}
Let  be a graph reduced under Rule~\ref{rule1}, Rule~\ref{rule2}, Rule~\ref{rule3} and Rule~\ref{rule:sunflower-del} that can be turned into a cograph by deleting at most  edges. Let  be an optimal edge-deletion set and denote by  the cograph resulting from the deletion of  and by  its cotree. We will count the number of leaves of  (or equivalently of vertices of  and ).

Observe that since a set of  edges covers at most  vertices,  contains at most  affected leaves (i.e. leaves corresponding to a vertex incident to a removed edge).
We say that an internal node of the cotree  is \emph{affected} if it is the least common ancestor of two affected leaves. Notice that there are at most  affected nodes.

We first argue that the root of  is a parallel node and is affected. Assume that the root of  is a series node: since no edges are added to , this would imply that  is not reduced under Rule~\ref{rule2}, a contradiction. Moreover, since  is reduced under Rule~\ref{rule1}, none of its connected components is a cograph. It follows that every connected component of  contains a vertex incident to a removed edge, and thus that every subtree attached to the root contains an affected leaf as a descendant. Hence the root of  is an affected node.

\begin{claim}
\label{claim:kernel-del}
Let  be an affected leaf or an affected node different from the root, and  be the least affected ancestor of . The path between  and  has length at most .
\end{claim}

\emph{Proof.} Observe first that the result trivially holds if  is the root of  and  one of its children. In all other cases, let  be the set of leaves descendant of  in . We claim that  contains a leaf  which is incident to a removed edge , with . If  is an affected leaf then this is true by definition. Otherwise, if  is an affected node different from the root, assume by contradiction that all the removed edges in  are of the form  with . In particular, this implies that  is a module of  strictly contained in a connected component. By Observation~\ref{obs:reduced-123}, it follows that  is an independent set and hence contains no edges, a contradiction. 
Let  be the least common ancestor of  and . The node  is a parallel node which is an ancestor of  and  (observe that we may have ). Assume by contradiction that the path between  and  in  contains a sequence of  consecutive non-affected nodes. The type of these nodes is alternatively series and parallel. So we can find a sequence  of consecutive non-affected nodes with  (resp. ) being the father of  (resp. ) and with 's being series nodes and the 's being parallel node. Now each of the 's (resp. ) has a non-affected leaf  (resp. ) which is not a descendant of  (resp. ). Observe that for every  the vertex set  induces a  in . Thereby we found a set of  's in  pairwise intersecting on the edge . It follows that  is not reduced by the Rule~\ref{rule:sunflower-del}: contradiction. This implies that the path between  and  contains at most  non-affected nodes. 
\hfill  \\

Since there are at most  affected nodes and  affected leaves,  contains at most  internal nodes.
As  is reduced, Observation~\ref{obs:reduced-123} implies that each of these  nodes is attached to a set of at most  leaves or a parallel node with  children. It follows that  contains at most  leaves, which correspond to the number of vertices of . 

\medskip
We now conclude with the time complexity needed to compute the kernel. Since the application of Rule~\ref{rule:sunflower-del} decreases the value of the parameter (which is not changed by the other rules), Rule~\ref{rule:sunflower-del} is applied at most  times. It then follows from Lemma~\ref{lem:gen-reduced} that a reduced instance can be computed in polynomial time.
 \end{proof}\\
 
The following corollary simply follows from the observation that the family of cographs is closed under complementation (since the complement
 of a  is a ).
 
 \begin{corollary}
The parameterized cograph edge-completion problem admits a cubic vertex kernel.
\end{corollary}

\subsection{Cograph edge-edition}

The lines of the proof for the cubic kernel of the edge-edition problem are essentially the same as for the edge-deletion problem. But since edges can be added and deleted, the reduction Rule~\ref{rule:sunflower-del} has to be refined in order to avoid that a single edge addition breaks an arbitrary large set of 's.

\begin{reduction} \label{rule:sunflower-edition}
If  is a pair of vertices of  that belongs to a set  of   quadruples  such that for , every  induces a  and for any , , then change  into  and decrease  by one.
\end{reduction}

As for reduction Rule~\ref{rule:sunflower-del}, it is clear that reduction Rule~\ref{rule:sunflower-edition} is safe and can be applied in polynomial time. The kernelization algorithm of cograph edge-edition consists of an exhaustive application of Rules~\ref{rule1}, \ref{rule2}, \ref{rule3} and \ref{rule:sunflower-edition}.

\begin{theorem}
\label{th:edition}
The parameterized cograph edge-edition problem has a cubic vertex kernel.
\end{theorem}

\begin{proof}
Let  be a graph reduced under Rule~\ref{rule1}, Rule~\ref{rule2}, Rule~\ref{rule3} and Rule~\ref{rule:sunflower-edition} that can be turned into a cograph by editing at most  edges. Let  be the cograph obtained by an optimal edge-edition. The cotree of  is denoted by . Unlike in the edge-deletion problem, the root of  is not necessary a parallel node. However it is still true that the root of  is affected. Indeed, assume first that the root of  is a series node. Then it is affected since otherwise  would not be reduced under Rule~\ref{rule2}. Now, assume that the root is a non affected parallel node. This means that at most one of its children contains an affected leaf as descendant, and hence that  is not reduced under Rule~\ref{rule1}: contradiction.

In the following we assume w.l.o.g. that the root of  is a parallel node. We prove that Claim~\ref{claim:kernel-del} still holds in this case.
Let  be an affected leaf or an affected node different from the root, and  be the least affected ancestor of . Observe that the result is trivially true if  is the root of  and  one of its children. In all other cases, let  be the set of leaves descendant of  in . As in the proof of Theorem~\ref{th:deletion}, there must exist an edited edge  with  (otherwise  would be a module of , i.e. an independent set by Observation~\ref{obs:reduced-123} and would thus not be edited by Observation~\ref{obs:module}). 

Now the proof follows the arguments of the proof of Theorem~\ref{th:deletion}, if one can find in  a path of  consecutive non-affected nodes between  and , then  is not reduced under Rule~\ref{rule:sunflower-edition}. Proving that  contains  nodes and thereby  leaves.

\medskip
The fact that a reduced instance can be computed in polynomial time follows from Lemma~\ref{lem:gen-reduced} and the observation that Rule~\ref{rule:sunflower-edition} decreases the value of the parameter and requires polynomial time.
\end{proof}\\

For the deletion (resp. edition) problem there exists a graph reduced under Rule~\ref{rule1}, Rule~\ref{rule2}, Rule~\ref{rule3} and Rule~\ref{rule:sunflower-del} (resp. Rule~\ref{rule:sunflower-edition}) that achieves the cubic bound (see Figure~\ref{fig:lower-bound}).

\begin{figure}[ht]
\centerline{\includegraphics[scale=0.75]{lower-bound.pdf}}
\caption{A reduced graph  with  vertices for which  edge deletions, namely the 's for , are required to obtain a cograph . The cotree  of  is represented. Each parallel node of   which is not the root has  children,  of which are leaves. The root of  has  children.
\label{fig:lower-bound}}
\end{figure}

\section{Kernel lower bounds for -free edge-deletion problems}
\label{sec:lower}

\newcommand{\notone}{\textsc{Not-1-in-3-sat}}

In \cite{KW09}, Kratsch and Wahlstr\"om show that the \notone{} problem has no polynomial kernelization under a complexity-theoretic assumption ().
We observe that their argument still applies to a graph restriction of \notone{} where the constraints arise from the triangles of an input graph. 

\subsection{A graphic version of the {\sc\notone{}} problem}


For a graph , an edge-bicoloring is a function . A \emph{partial edge-bicoloring} of  is an edge-bicoloring of a subset of edges of . An edge colored  (resp. ) is called a -edge (resp. -edge). We say that the edge-bicoloring  \emph{extends} a partial edge-bicoloring  if for every  colored by , then .
The weight of an edge-bicoloring is the number  of -edges. An edge-bicoloring is \emph{valid} if  every triangle of  contains either zero, two or three -edges. We consider the following problem:

\medskip
\noindent
\textsc{Not-1-in-3-edge-triangle}\\
\textbf{Input:} An undirected graph  and a partial edge-bicoloring .\\
\textbf{Parameter:} An integer .\\
\textbf{Question:} Can we extend  to a valid edge-bicoloring  of weight at most ?

\begin{proposition} \label{prop1} 
 \gbc{} is NP-complete and or-compositional.
\end{proposition}

\begin{proof}
The NP-hardness follows from a reduction from \textsc{Vertex Cover}. Let  be an instance of \textsc{Vertex Cover}~\cite{GJ79}, where . We create an instance  of \gbc{} as follows. The graph  is obtained from  by adding a dominating vertex , the partial edge-bicoloring  is such that  for every , and we let . As the triangles of  are monochromatic, the constraints to obtain a valid extension of  are carried by the triangles of the form  with . It is easy to observe that  has a valid edge-bicoloring extension of weight  iff  has a vertex cover of size . As \gbc{} clearly belongs to NP, the NP-completeness follows. 


\medskip
We now show that \gbc{} is or-compositional. The proof closely follows the proof of \cite{KW09} for \notone{}. We first need the following result:

\begin{claim}\label{lemma1} 
Given  an instance  of \gbc{}, and two positive integers  and  such that , we can compute in polynomial time an equivalent instance  of \gbc{} such that .
\end{claim}

\emph{Proof.} To build , we first add to  a set  of  new isolated edges  such that  for all . Then we add to the resulting graph  gadgets as follows: let  (with ) be an arbitrary -edge of ; add the triangles  with . The edges 's are not necessarily distinct. Observe that in any valid edge-bicoloring of  extending , the edge  (for every ) is a -edge while the edge  is a -edge. It follows that  is a positive instance if and only if  is a positive instance as the set  increases the weight by  and the added triangles by .
\hfill  \\

Consider a sequence  of instances of \gbc{}. 
We denote by  the set of -edges of . 
By Claim \ref{lemma1}, we can assume w.l.o.g. that , for . We can also assume that  since otherwise an exact branching algorithm could solve the problem. Moreover, for the sake of the construction, we assume  (duplicating some instance  if necessary).


Intuitively, the graph  of the composed instance  is built on the disjoint union of the 's, .  Then, as a selection gadget, we add a "tree-like graph"  connecting a "root edge"  to edges  for . Finally, for every , the -edges of the graph  are connected via a propagation gadget to the edge  in . The root edge is the unique -edge of . The copies of the 's inherit the -edges of the 's. The idea is that the selection gadget guarantees that at least one of the 's edge gets colored . Then the propagation gadgets attached to that edge  transmit color  to the copies of every -edge of .

Formally, we do the following: (i) we start with a complete binary tree  with  leaves; (ii) to each node  of , we associate an edge  in  as follows: if  is associated to the edge  and if  has two children , we create a new vertex  and we let . The leaves of  are then associated to edges . Now, for every , the propagation gadget  consists of vertex-disjoint graphs  for every edge  of . If  and , then  consists of four triangles , with edges  colored  by  (the other edges remain uncolored). Again the unique -edge of  is the root edge of , in particular the edges of the  are uncolored by . However, the -edge sets of the 's are inherited by . (see Figure~\ref{fig:not-1-in-3-triangle})


\begin{figure}[ht]
\centerline{\includegraphics[scale=0.75]{1-in-3-triangle.pdf}}
\caption{The instance  built from a sequence  with . The unique -edge is . Every "leaf edge"  of  is linked to the copies of the -edges of  via the propagation gadget. The -edges are depicted as dotted lines: they either belong to a propagation gadget or correspond to a -edge of some .
\label{fig:not-1-in-3-triangle}}
\end{figure}


Observe first that every valid edge-bicoloring extending  has to assign color  to at least one edge , for . Then the edges of  and the  non -edges of  are also assigned color . It follows that if we choose , then  is a positive instance if and only if there exists  such that  is a positive instance.
 \end{proof}\\

The following corollary follows from Theorem~\ref{th:or-comp}:

\begin{corollary}
The \gbc{} problem does not admit a polynomial kernel unless .
\end{corollary}

The problem \tgbc{} is the restriction of \gbc{} where the input graph  is 3-colorable. The hardness results obtained for \gbc{} carry over to this restriction:

\begin{lemma} \label{prop2} 
The \tgbc{} problem  does not admit a polynomial kernel unless .
\end{lemma}

\begin{proof}
The proof uses Theorem~\ref{th:ppt}, that is we provide a polynomial parameter-preserving transformation from \gbc{} to \tgbc{}. By Proposition \ref{prop1}, \gbc{} is NP-complete. Observe that \tgbc{} clearly belongs to NP.

Let  be an instance of \gbc{}. We build an instance  of \tgbc{} in the following way. Suppose that , then  has vertex set , and has edge set . The partial edge-bicoloring  is defined as follows:  for ; if the edge  of  is colored, then  for ; the other edges of  are uncolored.

Observe that every valid edge-bicoloring extending  assigns the same color to the six edges of  associated with an edge  of : indeed, given  , if  this holds since , if  this holds since , and otherwise this follows by transitivity. It is then easy to see that solutions of  and solutions of  are in one-to-one correspondence. 
\end{proof}

\subsection{Negative results for -free edge deletion problems}

In this section, we show that unless , the \textsc{-free edge-deletion} and the \textsc{-free edge-deletion} problems have no polynomial kernel for large enough . To that aim, we provide polynomial time and parameter transformations from \tgbc{} to the \textsc{Annotated -free edge-deletion} problem and to the \textsc{Annotated -free edge-deletion} problem. For a graph , the \textsc{Annotated -free edge-deletion} problem is defined as follows:\\

\noindent
\textsc{Annotated -free edge-deletion}\\
\textbf{Input:} An undirected graph  and a subset  of vertices.\\
\textbf{Parameter:} An integer .\\
\textbf{Question:} Is there a subset  such that  is -free?\\

Observe that the \textsc{Annotated -free edge-deletion} problem reduces to the (unannotated) \textsc{-free edge-deletion} problem whenever  is closed under twin addition: it suffices to add for every vertex  a set of  twin vertices. Clearly this transformation also preserves the parameter.  \\


Observe also that we can restrict the \tgbc{} problem to instances  not containing any -edge (\emph{i.e.}  whenever it is defined). The reason is that any uncolored edge  of  can be forced to be assigned color  in every valid edge-bicoloring extending  by adding to   new vertices  such that , , is an uncolored triangle. Clearly if  is a -edge of an edge-bicoloring  extending ,  needs at least  -edges to be valid:  plus one edge per triangle. The same argument was used in~\cite{KW09} for the \notone{} problem.

\newpage

\begin{theorem} \label{prop3} 
The \textsc{-free edge-deletion} problem has no polynomial kernel for any , unless .
\end{theorem}

\begin{proof}
We describe a polynomial time and parameter transformation from the restriction of \tgbc{} without -edges to \textsc{Annotated -free edge-deletion}. The statement then follows from Theorem~\ref{th:ppt} and the fact that \textsc{Annotated -free edge-deletion} reduces to \textsc{-free edge-deletion}.

Let  be an instance of the \tgbc{} problem, where  are disjoint independent sets of . The construction of the instance  of \textsc{Annotated -free edge-deletion} works as follows. First the sets ,  and  are turned into cliques and the -edges of  are removed. In addition to , the graph  contains a set  of new vertices. For each pair  with  an edge of  and  a vertex of , such that  induces a triangle in , we create a path  of length  between  and  in  (the internal vertices of  are added to ). Notice that each triangle of  generates three such paths in . It remains to add some \emph{safety} edges incident to the vertices of . Every two vertices  and  of  that do not belong to the same path are made adjacent.
In every path , we select an internal vertex , called its \emph{centre}, at distance  from . Every centre vertex  is made adjacent to . We denote by  the resulting graph. To complete the description of  we set  and the parameter  where  is the number of -edges of .

\begin{figure}[ht]
\centerline{\includegraphics[scale=0.75]{induced-cycle-ppt.pdf}}
\caption{The graph  built from an instance  of the \tgbc{} problem for . The white and the square vertices form the set  of new vertices. The independent sets ,  and  of  are turned into cliques. The thick dotted edges are the removed -edges of . The non -edges of  are preserved in . 
\label{fig:cycle}}
\end{figure}

\begin{claim}
\label{claim:nokernel-cl}
A subset of vertices  induces a cycle of length  iff  contains a triangle ,  
with  a -edge and ,  uncolored edges, such that  with .
\end{claim}

\emph{Proof.} By construction, if  contains a triangle  with a unique -edge , then  (with ) induces a cycle of length  in  (keep in mind that the -edges of  are removed from ).
Let  be an induced  in . Observe that as ,  and  are turned into cliques, . Thereby  intersects the vertex set . We now argue that there exists a path , with  and , containing the vertices of . Otherwise, since every pair of vertices of  belonging to two distinct paths  and  (with ) are adjacent, we would have  and thus . It follows that  or  belongs to . We prove that they both belong to . Assume , then  uses a safety edge incident to the centre vertex  and half of the internal vertices of  does not belong to . Thereby : contradiction with the hypothesis . Finally as  contains  vertices,  contains an extra vertex and  are not adjacent. As  is adjacent to every vertex of  except ,  and , we have that  as announced and . Now the existence of  witnesses the existence of the triangle  in . As  and ,  is the only -edge of the triangle .
\hfill  \\

We now argue for the correctness of the transformation. Suppose that there exists a set  of allowed edges of size at most  such that  is -free. Define the edge-bicoloring  of  as follows:  if ,  otherwise. As by assumption  does not assign color  to any edge,  extends  and has weight at most .
Besides,  is a valid edge-bicoloring of . Let  with  be a pair such that  induces a triangle in . If we had , , we would obtain that  induces a  in , impossible. Conversely, suppose that  is valid edge-bicoloring of weight at most  of  which extends . Let  be the set of edges such that  but are uncolored by . By construction  is a set of allowed edges of  of size at most . Since  is a valid edge-bicoloring of , Claim~\ref{claim:nokernel-cl} implies that  is -free. 
 \end{proof}\\
 
A slight modification of the above construction yields the following:

\begin{theorem} \label{prop4} 
The \textsc{-free edge-deletion} problem has no polynomial kernel for any , unless .
\end{theorem}

\begin{proof}
Let  be an instance of the \tgbc{} problem not containing any -edge and such that  are disjoint independent sets of . We modify the construction given in Theorem \ref{prop3} to obtain an instance  of \textsc{Annotated -free edge-deletion} problem. The vertex set  of  consists of the union of  and a set  of new vertices. The sets ,  and  are again turned into cliques and the -edges of  are not duplicated in . But for each pair , with  and  such that  is a triangle of , the associated gadget  is no longer a path. Instead,  consist of two paths  and :  is a path of length  containing  as extremity and  is a path of length  containing  as extremity. The vertices of  are added to . As before for every  we add all the edges between vertices of  and . The centre vertex  of  is the vertex of  at distance  from . The centre vertex is made adjacent to every vertex of  except ,  and . To complete the description of  we set  and  where  is the number of -edges of .

The correctness proof of the construction follows the same lines than the proof of Proposition~\ref{prop3}. It now relies on the following claim that characterizes the possible induced 's.

\begin{claim}
\label{claim:nokernel-pl}
A subset of vertices  induces a path of length  iff  contains a triangle ,  
with  a -edge and ,  uncolored edges, such that  with .
\end{claim}

\emph{Proof.} By construction, if  contains a triangle  with a unique -edge , then  (with ) induces a path of length  in  (keep in mind that the -edges of  are removed from ).
Let  be an induced  in . As in the proof of Claim~\ref{claim:nokernel-cl}, observe that  and thereby  intersects the vertex set  and that there exists a unique pair  with  such that  contains . By the choice of the length of  and ,  intersects both  and . It follows that  and  belongs to . Assume that  uses an edge  such that . Then half of the vertices of  does not belong to , which would contradict the hypothesis . Finally as by construction  contains  vertices, we need at least one extra vertex from . Since  is adjacent to all vertices of  except ,  and , that extra vertex can only be . Moreover the chord  cannot exist in , meaning that  is a -edge of .
\end{proof}
 
 \section{Conclusion}
 
In this paper we have shown that the \textsc{parameterized cograph edge modification} problems admit vertex cubic kernels. Moreover, we provide evidence that the \textsc{-free edge-deletion} and the \textsc{-free edge-deletion} problems do not admit polynomial kernels for large enough  (under a complexity-theoretic assumption~\cite{BDFH08}). These problems were left open by Kratsch and Wahlstr\"om in~\cite{KW09}. The value of  being respectively (at least)  and , one remaining question is thus to determine whether the \textsc{-free edge-deletion} and the \textsc{-free edge-deletion} problems admit polynomial kernels for  in the former case, and  in the latter. 

\bibliographystyle{abbrv}
\bibliography{cograph}

\end{document}
