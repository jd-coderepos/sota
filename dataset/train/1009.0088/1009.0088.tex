\documentclass[11pt]{article}
\pdfoutput=1

\newif\ifFull
\Fulltrue
\usepackage{amsmath,amssymb}
\usepackage{graphicx,epstopdf}
\usepackage{url,hyperref}\urlstyle{same}
\usepackage{jeffe}
\usepackage{color}
\usepackage{microtype}
\usepackage {plaatjes}
\usepackage{fullpage}

\usepackage {xspace}

\definecolor {infocolor} {rgb} {0.6,0.6,0.6}
\definecolor {sepia} {rgb} {0.75,0.30,0.15}


\definecolor {remarkcolor} {rgb} {0.60,0.30,0.30}
\newcommand {\namednote} [2] {\note {\textcolor {red}{#1}: \textcolor {blue}{#2}}}
\newcommand {\david} [1] {\namednote {David} {#1}}
\newcommand {\erin} [1] {\namednote {Erin} {#1}}
\newcommand {\maarten} [1] {\namednote {Maarten} {#1}}
\newcommand {\mike} [1] {\namednote {Mike} {#1}}

\newcommand {\fsg} [1] {\ensuremath {G \langle #1 \rangle}}

\newtheorem {observation} {Observation}
\newtheorem {lemma} {Lemma}
\newtheorem {theorem} {Theorem}
\newtheorem {corollary} {Corollary}

\ifFull\else
\renewcommand{\subsection}[1]{\subsubsection{#1.}}
\fi

\newcommand {\plaatjesh} [3]
{
  \begin {figure} [h]
    \centering
    #1
    \hfill
    \mbox {}
    \textsf{\caption {\small #2}
    \label {fig:#3}}
  \end {figure}
}
\newcommand {\drieplaatjesh} [5] []
{
  \plaatjesh
  {
    \subplaatje {#1} {#2}
    \subplaatje {#1} {#3}
    \subplaatje {#1} {#4}
  } {#5} {#2+#3+#4}
}


\title{Drawing Graphs in the Plane \\ with
a Prescribed Outer Face and Polynomial Area}

\author{
  Erin W. Chambers\footnote
  { Dept. of Math and Computer Science, Saint Louis Univ., USA.
    \texttt{echambe5(at)slu.edu}
  }
  \and
  David Eppstein\footnote
  {
    Computer Science Dept., University of California, Irvine, USA.
    \texttt{\{eppstein,goodrich,loffler\}(at)ics.uci.edu}
  }
  \and
  Michael T. Goodrich\footnotemark [2]
  \and
  Maarten L\"offler\footnotemark [2]
}
\date{ }

\begin{document}

\maketitle

\begin{abstract}
We study the classic graph drawing problem of drawing a planar graph
using straight-line edges with a prescribed convex polygon
as the outer face. Unlike previous algorithms for this
problem, which may produce drawings with exponential area,
our method produces drawings with polynomial area.
In addition, we allow for collinear points on the boundary, provided
such vertices do not create overlapping edges.
Thus, we solve an open problem of Duncan {\it et al.}, which,
when combined with their work, implies that we can produce
a planar straight-line drawing of a combinatorially-embedded genus- graph
with the graph's canonical polygonal schema drawn as a convex polygonal
external face.
\end{abstract}


\section{Introduction}
\label{sec:intro}
The study of planar graphs has been a driving force for graph theory,
graph algorithms, and graph drawing.
Our interest, in this paper, is on methods for drawing planar graphs
without edge crossings
using straight line segments for edges, in such a way that all faces are convex
polygons and the outer face is a given shape.
Figure~\ref{fig:example} shows an example.

\drieplaatjesh {intro-input-G} {intro-input-P} {intro-output} {Our problem: given (a) a combinatorially embedded planar graph  and (b)~a polygon  with certain vertices on the outer face of  marked as corresponding to vertices of , find (c)~a straight-line embedding of  that uses  as the shape of its outer face.\label{fig:example}}

\subsection{Related Prior Work}
In seminal work that has been highly influential in the graph drawing
literature, Tutte~\cite{t-crg-60,t-hdg-63}
shows that it is possible to draw any planar graph, , using
non-crossing straight-line edges so that the vertices of the
outer face are drawn on the boundary of a prescribed convex polygon.
This work has influenced a host of subsequent papers, and,
according to Google Scholar, Tutte's 1963 paper
has been directly cited over 600 times.
For instance, along with seminal work on drawing maximal planar
graphs~\cite{f-slrpg-48,s-cm-51,w-bzv-36} (with triangular outer faces),
his work has influenced many researchers
to study methods for
drawing planar graphs using straight-line edges (e.g.,
see~\cite{cgt-cdgtt-96,ck-cgd3c-97,cp-ltadp-95,dpp,d-gdt-10,k-dpguc-96,br-scdpg-06,s-epgg-90}).
Moreover, not only has Tutte's result itself been highly influential,
but because his method is based on a force-directed layout method,
his work has also influenced a considerable amount of work on
force-directed layouts
(e.g., see~\cite{dh-dgnus-96,dett-gd-99,fr-gdfdp-91,ggk-mdafd-04,sm-gdmsm-95}).

Unfortunately,
one of the drawbacks of Tutte's algorithm is that it
can result in drawings with exponential area.
This area blowup is not an inherent requirement for planar
straight-line drawings, however, as
a number of researchers have shown that it is possible to
produce drawings of planar graphs with non-crossing straight-line
edges using polynomial area
(e.g.,~\cite{ck-cgd3c-97,cp-ltadp-95,dpp,k-dpguc-96,s-epgg-90}).
Nevertheless, all of these straight-line drawing algorithms lose a
critical feature of Tutte's drawing algorithm, in that none of them
allow for the vertices of a planar graph's outer
face to be placed on the boundary of a prescribed
convex polygon.
Becker and Hotz~\cite{bh-olpgf-87}, on the other hand,
show how to draw a planar graph with
a prescribed outer face so as to optimize the total weighted edge
length, but, like Tutte's method,
their method may also produce drawings with exponential
area.
Indeed, we know of no such prior result, and, in fact,
Duncan {\it et al.}~\cite{duncan}
pose as an open problem whether there exists an algorithm
that produces straight-line drawings with vertices on the boundary
of a given convex polygon of polynomial area.

One motivation for this problem of prescribing the outer face of a planar drawing
comes from a common way of producing hand-drawn
planar representations of genus- graphs.
Namely, if we are given
a graph, , embedded into a
genus- topological surface,
the surface may be cut along
the edges and vertices of  fundamental cycles in  to form a topological disk (known as a \emph{canonical polygonal schema}), with a boundary that is made up
of  paths (with multiple copies of the vertices on the fundamental cycles).
Moreover, as shown by
Duncan {\it et al.}~\cite{duncan},
 can be cut in this way so that
each of these  paths is chord-free, that is,
so that there are no edges
between two vertices strictly internal to the same path (other than
path edges themselves).

The standard way of drawing this unfolded version of
such an embedding, in the topology
literature (e.g., see~\cite{lpvv-ccpso-01}),
is to draw the disk as a convex polygon with each of its  boundary paths drawn as a straight line
segment: the geometric shape is used to make clear the pattern in which the surface was cut to form a disk.  Fortunately, given Tutte's seminal result, it is
possible to draw any chord-free canonical polygonal schema along the
boundary of a given convex polygon with  edges.
The drawback of using Tutte's algorithm for this purpose is that the resulting drawing may have exponential area.
Thus, we are interested in drawing the unfolded embedding in
polynomial area and in polynomial time.

\subsection{Our Results}
In this paper, we describe an algorithm for drawing a planar graph with a prescribed outer face shape.
The input consists of an embedded planar graph , a partition of the outer face of the embedding into a set  of  chord-free paths, and a -sided polygon ; the output of our algorithm is a drawing of  within  with each path in  drawn along an edge of
.
Given the above-mentioned prior result of Duncan {\it et al.}~\cite{duncan},
for finding chord-free canonical polygonal schemas,
our result implies that we can solve their open problem: any
graph  combinatorially embedded
in a genus- surface has a polynomial-area straight-line planar drawing
of a canonical polygonal schema  for ,
drawn as a -sided convex polygon  with
the vertices of each path in  drawn along
an edge of .

\section {Preliminaries} \label {sec:prelims}

  In this paper, we show how to draw a graph with a given boundary with
  coordinates of polynomial magnitude.
  Before treating the main construction,
  though, we show in this section that
  we can equivalently state the problem
  in terms of the \emph{resolution} of the
  graph. Furthermore, we recall some known results and concepts.

  \subsection {Resolution}

    Instead of drawing a graph with integer coordinates of small total size, we
    will make a drawing with real coordinates that stays within a fixed region
    (inside the input polygon) with a large \emph {resolution}.



    Let  be a graph that is embedded in  with straight line segments as
    edges. We define the \emph {resolution} of  to be the shortest distance
    between either two vertices of the graph, or between a vertex and a
    non-incident edge.
    The \emph {diameter} of  is the largest distance between two vertices of the graph.  
    


    We begin by establishing a relation between resolution and size, which basically says that drawing a graph  with small diameter and large resolution also results in another drawing with integer coordinates and small size.
    Generally it may not be possible to scale a given input polygon such that its coordinates become integers, so we need to do some rounding. We say that two drawings of  are \emph {combinatorially equivalent} if their topology is the same, and any collinear adjacent edges in one drawing are also collinear in the other.    
    We say two drawings are \emph {-equivalent} if the distance between the locations of each vertex of  in the two drawings is at most .

    \begin {lemma} \label {lem:ressize}
      Let  be a graph, and let  be a drawing of  without crossings, with constant resolution, and with diameter . 
      Then there exists another drawing  of  with integer coordinates and diameter , such that a scaled copy of  with diameter  is both combinatorially equivalent and -equivalent to .
    \end {lemma}

    \begin {proof}
      We may assume that the resolution of the initial drawing of  is :
      that is, no vertex is within unit distance of another vertex or edge.
      Scale  by a factor of ,
      forming a drawing, , and consider the integer grid squares
      that contain each vertex of .
      If any set of vertices has its coordinates changed by at most one unit,
      this motion can only bring a vertex
      and an edge closer together by a distance of ,
      less than the resolution (which is now ),
      so this motion cannot introduce crossings or change
      the combinatorial type of the drawing.
      At this stage, we move the vertices of the outer face to their
      nearest integer points (changing their coordinates by at most ),
      but we do not change the positions of the other vertices.

      Next, we scale the drawing again, by a factor of . The vertices of the outer face remain on integer coordinates. The vertices in the interior of the drawing may be moved to any nearby integer point, without changing the combinatorial type of the drawing. It remains to choose integer coordinates for the vertices that lie on the sides of the outer face of . For each such vertex , on side , the integer rounding of the endpoints of  will have caused  to move, but there will still exist a point  whose coordinates (though not necessarily integers) are both within  of . A  box centered on  will consist entirely of points whose coordinates are within  of  (equivalent to being within one unit of  in ), and (because of the way we rounded the endpoints of  prior to the second scaling step) is guaranteed to contain an integer point . By rounding  to , and simultaneously rounding in the same way all the other points of the drawing that belong to the sides of the outer face, we obtain a drawing  that is combinatorially equivalent to , with a combinatorially equivalent outer face, in which all coordinates are integers.
    \end {proof}

    Note that, for a fixed input polygon with non-integer vertex coordinates, this
    perturbation may slightly modify its shape, since it may not be possible to
    find a similar copy of the polygon with integer vertex coordinates.

    Now, let  be a set of points in the plane. We define the \emph {potential
    resolution} of  to be the resolution of the complete graph on .
    Similarly, for a polygon , we define its potential resolution to be the
    potential resolution of its set of vertices.
    Clearly, the resolution of any drawing we can achieve will depend on the
    potential resolution of the input polygon, because the drawing could be
    forced to include any edge of the complete   graph.

    Next, we make an observation about the nature of the potential resolution of
    convex polygons.


    \begin {observation} \label {obs:convex-easier}
      If  is a convex polygon, then the potential resolution of  is the
      minimum over the vertices of  of the distance between that vertex and the
      line through its two neighboring vertices.
    \end {observation}



    Thus, for a convex polygon  to allow for a drawing of polynomial
    area in its interior, we insist that
     has a polynomially-bounded aspect ratio. It cannot be
    arbitrarily thin and still support a polynomial-area drawing in its
    interior.

  \subsection {Alpha Cuts}

    We now describe a useful property of the potential resolution of a convex
    polygon, namely that it can be ``distributed'' any way we want when cutting 
    the polygon into smaller parts.
    This will be made more precise later.
    We first make another observation about convex polygons.
    
    \begin {lemma} \label {lem:move-vertex}
      Let  be a convex polygon,  a vertex of ,  an edge incident to , and  a number. Let  be a copy of  where 
      has been replaced by  by moving  along  over a fraction 
      of the length of . Then the potential resolution of  is at least  times the potential resolution of .
    \end {lemma}

    \begin {proof}
      Let  be three vertices of  in counterclockwise order, and
      consider the distance from  to the line  through  and , as in
      Figure~\ref {fig:ro-original}. By Observation~\ref {obs:convex-easier}, the
      potential resolution of  is the minimum of this distance over all such
      triples of vertices.
      Now consider the situation in . If none of  are equal to ,
      then clearly the distance in  is the same as in .

      \vijfplaatjes {ro-original} {ro-case-br} {ro-case-al} {ro-case-ar}
      {ro-case-arcl} {Cases to consider with respect to resolution with
      respect to a convex polygon.}

      If , then the line  through  and  is still the same as in
      , see Figure~\ref {fig:ro-case-br}. The edge  must be either between  and  or between  and ; in both cases, moving  along 
      changes the distance to  linearly. Therefore, the distance from 
      to  is exactly  times the distance from  to ,
      which is bounded by  times the potential resolution of .

      If , then there are two subcases depending on whether  is between
       and  or between  and its predecessor.
      If  goes between  and the predecessor of , as in Figure~\ref
      {fig:ro-case-al}, then the line  between  and  rotates around
       as  moves over .
      Because  is convex, the distance from  to  as a function of the
      position of  has no local minimum, so when  is at one of the
      endpoints of  the distance is smaller than at any interior point.
      Therefore, the distance from  to  is bounded by the potential
      resolution of .
      If  goes between  and  itself, as in Figure~\ref {fig:ro-case-ar},
      then let  be the point on a fraction  along the edge between 
      and . Then distance from  to the line  through  and  is
      clearly larger than the distance from  to the line  through 
      and , which is exactly  times the distance from  to the
      line  through  and , see Figure~\ref {fig:ro-case-arcl}. So in
      this case the distance is again bounded by  times the
      potential resolution of .

      The case where  is symmetric to the case where .
    \end {proof}
    
    Let  be a convex polygon. We will show that we can cut  into two smaller
    polygons, ``distributing'' its potential resolution in any way we want.

    We define an -cut of  to be a directed line  that splits 
    into two smaller polygons, such that if an edge  of  is intersected by , the length of the piece of  to the left of  is  times
    the length of , and the piece of  to the right of  is 
    times the length of .
    For a given convex polygon and two features of its boundary (either vertices
    or edges), there is a unique -cut that cuts the polygon through those
    two features in order.

    \begin {lemma} \label {lem:alphacut}
      Suppose we are
      given a convex polygon  of resolution , two features (either vertices
      or edges) of , and a fraction .
      Let  be the -cut through the two given features that cuts 
      into a piece  to the left of  and a piece  to the right of .
      Then the potential resolution of  is at least  unless the two
      features are two adjacent edges that meet to the right of .
      Similarly, the potential resolution of  is at least 
      unless the two features are two adjacent edges that meet to the left of .
    \end {lemma}

    \begin {proof}
      We will argue about the potential resolution of ; the argument for  is symmetric.
      We prove this lemma by applying Lemma~\ref {lem:move-vertex} to the
      new vertices of . If both features where  cuts through  are
      vertices, then all vertices of  are also vertices of  and clearly
      the potential resolution can only become better. However, if one or both of
      the features are edges, then  has one or two new vertices that are not
      part of .
      Figure~\ref {fig:alphacut-1+alphacut-2+alphacut-2-adjacent} shows three different cases that can occur.
      To solve this problem, we first alter  to a different
      polygon  that has the new vertices. Let  be the place where 
      enters  and  the closest vertex below  along the boundary to it
      (possibly ), and similarly let  be the place where  exits
       and  the closest vertex below . Now, we create  by moving
       to  and  to . Clearly, both will move a fraction 
      along their edges, so by Lemma~\ref {lem:move-vertex}  has a potential
      resolution of at most  times the potential resolution of .
      Therefore, the potential resolution of  can only be larger.

      The only exception is when ; in this case we cannot move the vertex
      to two new places simultaneously, but we have to create two new vertices.
      Indeed, the result is not true in that case, since the two new vertices
      can be arbitrarily close to each other as  comes arbitrarily
      close to , so the resolution of  cannot be expressed in terms of
      , as can be seen in Figure~\ref {fig:alphacut-2-adjacent}.
    \end {proof}

    \drieplaatjes {alphacut-1} {alphacut-2} {alphacut-2-adjacent}
    {(a) An -cut through a vertex and an edge. (b) An -cut through two non-adjacent
    edges. (c) An -cut through two adjacent edges.}

  \subsection {Combinatorial Embeddings}

    Let  be a plane graph.
    That is, we consider the combinatorial structure of 's
    embedding to be fixed,
    but we are free to move its vertices and edges around.
    Let  be the set of faces of , excluding the outer face.
We make some definitions about faces.
    We say that a subset  induces a subgraph  of 
    that consists of all vertices and edges that are incident to the faces
    in .
    A subset  is said to be \emph {vertex-connected} if 
    is connected; it is said to be \emph {edge-connected} if the dual graph
    induced by the dual vertices of  is connected. In other words, faces
    that share an edge are both edge-connected and vertex-connected, but faces
    that share only a vertex are only vertex-connected.



    We recall a lemma from~\cite{duncan},
    rephrased in terms of the faces of the
    graph:

    \begin{lemma}
    \label{lem:river}
      Given an embedded plane graph  that is fully triangulated except for the
      external face and two edges  and  on that external face,
      it is possible to partition the faces of  into three sets
      
      such that:
      \begin{enumerate}
        \item All vertices of  are in either \fsg {F_1} or \fsg {F_2}.
        \item  is edge-connected and contains the faces incident to 
          and .
        \item  and  are both vertex-connected.
        \item The edge-connected components of  and  all share an edge
          with the outer face of .
      \end{enumerate}
    \end{lemma}


    Intuitively,  is a path of faces that goes from  to  and that
    splits the remaining faces into two sets  and .



\section {Drawing a Graph with a Given Boundary}

  We are now ready to formally state the problem and describe the algorithm to
  solve it.


  \subsection {The Problem}

    Let  be a triangulated planar graph with a given combinatorial embedding,
    and let  be the cycle that bounds the outer face of .
    Let  be a map from a subset of the vertices of  to points in the
    plane, such that these points are in convex position and their order along
    their convex hull is the same as their order along .

    We say that a map  from all vertices of  to points in the plane
    \emph {respects}  when:
    \begin {enumerate}
      \item The vertices mapped by  are also mapped by  to the same points;
      these define a convex polygon .
      \item The remaining vertices of  are mapped to the corresponding edges of .
      \item The remaining vertices of  are mapped to the interior of .
      \item If all edges are drawn as straight line segments, they cause no
      crossings or incidences not present in .
    \end {enumerate}

    An example of a respectful embedding was shown in Figure~\ref
    {fig:intro-input-G+intro-input-P+intro-output}.

    The input to our problem is a pair .
    We will use the notations  and  as above. We will further define 
    to be the set of faces of , excluding the outer face.
    We define , the number of internal faces,
    to be the \emph {size} of the problem.
    We define  to be the \emph {resolution} of the problem, which is the
    potential resolution of  (recall, that is the resolution of the complete
    graph on the corners of ).
Our goal is to compute a mapping  that respects  and such that the
    resolution of the embedded graph is bounded by some function of  and .

    Observe that it will not be possible to do so when there are any edges in 
    between two vertices that have to be on the same edge of . Therefore, we
    call a problem \emph {invalid} if this is the case.
We will show that for any
    valid problem, we can find an embedding with a polynomially bounded resolution.


  \subsection {The Main Idea}

    We want to solve the problem using divide-and-conquer.
    The idea is to divide  into smaller convex polygons, and  into smaller
    sets of faces, and map each subset of faces to one of the smaller regions.
    Then we need to decide which vertices of  are mapped to the new corners
    of the smaller regions, and solve the subproblems.

    A first idea would be
    to find a path in  between two vertices of , and lay that out on a
    straight line, resulting in a split of  into two smaller polygons and solve
    the two subproblems. There are two issues with this approach though. First, the
    vertices on the new straight line have to be placed the same way in the two
    subproblems, which means they are not independent. Second, if there are any
    chords on this path one of the subproblems will become invalid.

    To avoid these issues, we will not split along a single path, but along two
    paths next to each other. The region between these two paths, which we call a
    \emph {river}, has a controlled structure, which means that we can always
    complete the interior independently of how the vertices on the edges were
    placed. Furthermore, if these paths have any chords, we shortcut them along
    the chords and show how to deal with the added complexity of the river.
    Because the river may touch the boundary of  in more places, the problem
    may be decomposed into more than two subproblems.
    Figure~\ref {fig:ex-river} shows an example instance,
    and Figure~\ref {fig:ex-reduced} shows
    a possible decomposition where some vertices on the boundary of  have been
    fixed, and the paths between them are made straight.

    \drieplaatjes {ex-river} {ex-chords} {ex-reduced}
    {(a) A ``river'' (a path in the dual graph that does not reuse any vertices of
    the primal graph) between two edges on . (b) The river banks have chords, and so 
    we include the area behind the chords in the river. (c) We fix the vertices on
    the river boundary that are on , and draw the rest of the river boundary
    straight. This results in three smaller problems, plus the area of the river
    itself.}

    We assume the input is a valid problem with size  and resolution .
    We will keep as an invariant the ratio , and show in
    \ifFull Section~\ref {sec:split} \else the next paragraph \fi
    how to
    subdivide a problem into smaller valid problems with the same (or better)
    ratio, plus an extra region (the river). We then recursively solve the
    independent subproblems, which results in a placement of all vertices
    that are not in the interior of the river. Finally, we show in
    \ifFull Section~\ref {sec:river} \else the paragraph after that \fi
    how to we place the vertices inside the river.

  \subsection {Splitting a Problem} \label {sec:split}

    Let  be a valid problem of size  and resolution ,
    and suppose that  has at least four sides.

    Let  and  be two edges of  that lie on two sides of  that
    are not consecutive. Note that the endpoints of  and  are not
    necessarily fixed yet.
    Now, by Lemma~\ref {lem:river}, there exists a path of faces
    that connects  to , such that the boundary of this path does not
    have any repeated vertices. Let  be the union of the faces of  on
    this path. We call  a \emph {river}; Figure~\ref {fig:ex-river} shows
    an example.
    This river may touch  in other points than  and , so it can
    subdivide the faces of  into any number of edge-connected subsets
    (apart from the river itself). We will assign a separate subproblem to
    each such edge-connected component.

    We would like to straighten the banks of the river, but this may lead to
    invalid subproblems if these banks have any chords. Therefore, we identify
    any chords that the river has
    (note that they can only appear on the outside of the river, since the river forms a dual path),
    and we add the faces of  behind those chords to .
    Similarly,
    if one of the paths touches a side of  more than once, it would create a
    subproblem that would be flattened. To avoid that, we also incorporate such a
    region into the river (even though the straight side that lies alongside 
    is not necessarily a chord).

    Next, we count the numbers of faces in the river, as well as those
    in the parts outside the river. Then we fix the vertices where the river
    touches  by cutting off the subproblems, using -cuts where 
    is the fraction of faces inside the subproblem.
    Now, by Lemma~\ref {lem:alphacut},
    if we have a problem with parameters  and , we will
    construct subproblems with the same (or better) ratio .
    Finally we straighten the new
    banks of the river, so that the subproblems have proper
    convex boundaries. Figure~\ref {fig:ex-river+ex-chords+ex-reduced} shows
    an example.



    \begin {lemma} \label {lem:divide}
      Given a valid problem  where  has at least four sides,
      We can subdivide  and  into disjoint sequences 
      and  such that each  is a valid
      subproblem with ratio , and such that the remainders
       and  have
      the following properties:
      \begin {enumerate}
        \item  and  also have ratio .
        \item The vertices of  that are not vertices of
         form internally 3-connected components that
        share at least two vertices of         .
      \end {enumerate}
    \end {lemma}

    \begin {proof}
      For the first part of the lemma, we need to show that the subproblems
       are valid and have a resolution/size ratio at least as
      good as .
      First, we define the polygons  by applying Lemma~\ref {lem:alphacut}
      to  with  (that is, the fraction of
      faces in  that is in ). The lemma ensures that the new polygons
      have potential resolution at least , so clearly
       as required.
      Second, recall that a subproblem is valid if it does not have any chords
      between two vertices that have to be drawn on the same side of .
      For those sides of  that are part of , we already know there are
      no chords because  was valid. For the new sides, we explicitly
      added all faces behind chords to the river .

      For the second part of the lemma, we need to show that 
      has the right ratio and that the internal vertices of the river form
      3-connected components that share two or more vertices with the boundary
      of the river.
      If we denote  to be the size of the river and  to be the
      potential resolution of the region  in which it is to be drawn, then
      by Lemma~\ref {lem:alphacut}, the potential resolution of  after
      repeatedly slicing off subpolygons is at least , so .
      Finally, since we chose  according to Lemma~\ref {lem:river}, all
      edge-connected pieces of  outside  share an edge with the outer
      face. In particular, this means that the boundary of  does not touch
      itself and that any subgraphs sliced off by chords are 3-connected and
      share exactly two vertices with the boundary of .
      Furthermore, if the boundary of  touches the same side of  multiple
      times, then the edge-connected components of  between that side of 
      and  are also 3-connected and share at least two vertices with the new
      boundary of .
    \end {proof}

    When  has only 3 sides, we cannot choose two edges  and  on
    non-adjacent sides of . However, we can still use the same basic idea; 
    we just have to be careful because of the special case in Lemma~\ref
    {lem:alphacut}. So, let  be a corner of  and let  and 
    be edges of  on the sides incident to . The lemma does not give a
    bound on the resolution of the region on the far side of . So, let
     and  be the edges furthest away from . Since  is a
    triangle, the two vertices of  on the far side on  and 
    are the other two corners of , and they are joined by a side of .
    This means that the region between the river and this side will be
    included into the river, and there will be no subproblem on the far
    side of .

  \subsection {Fixing a River} \label {sec:river}

    \drieplaatjes {ff-input} {ff-bumps} {ff-flat}
    {(a) The interior of a river, after all vertices on its boundary have already
    been fixed by recursive calls to the split algorithm. (b) Because of
    the structure given by the river, we can identify small areas inside the
    river that we draw using the de Fraysseix-Pach-Pollack algorithm. (c) If we
    flatten the triangular drawings enough, every vertex is able to see every
    other one (in fact we need slightly more, namely that no visibility ray comes
    too close to any vertex). Note that in the figure the drawing is not flat
    enough for that, but otherwise the structure would become too hard to see.
    In fact, in this particular case no flattening at all would be required.}

    It remains to show how to place the interior vertices of a river after all vertices on its boundary have been placed recursively.
    Again, we are given a graph  and a polygon  that it has to be drawn in
    ( is the boundary of the river, and  is the part of the graph
    that has to be drawn in it),
    but there are two important differences with the initial problem:
    First, now we know that all vertices of  have already been fixed
    (not only those on the corners but also those on the boundary of ).
    Second, we know that the remaining vertices of  have a very specific
    structure, namely, they  form internally 2-connected components that
    share at least two vertices with the vertices fixed along an edge.
    Furthermore, since all fixed vertices have been placed using the algorithm
    above, they will never be closer than  to each other.
    This means we can draw these components using the algorithm by de Fraysseix,
    Pach and Pollack~\cite{dpp}, and rotate and scale them to fit inside . We
    can then flatten them more such that all remaining edges (between fixed
    vertices on the boundary of  or vertices on the de Fraysseix-Pach-Pollack
    drawings) can be drawn with straight line segments.
    Figure~\ref {fig:ff-input+ff-bumps+ff-flat} shows an example.




    \begin {lemma} \label {lem:internal}
      Given a river placement problem, of size  and resolution , we can lay
      out the graph with a resolution of .
    \end {lemma}

    \begin {proof}
      The polygon  that forms a river and the graph  to be drawn in it
      are formed by subdividing a bigger problem according to Lemma~\ref
      {lem:divide}.
      Then, the vertices on the boundary of  are placed during recursive calls
      to smaller problems. All vertices have to be corners of the polygons of at
      least one such subproblem, and by our invariant all these problems have
      ratio , so the distance between any two fixed vertices cannot be
      smaller than .
      Furthermore, Lemma~\ref {lem:divide} tells us that
      the vertices of  can be grouped into
      a number of subsets  such that each  is
      internally biconnected, and there is a
      sequence of at least two vertices in  that is in , so that have
      been fixed  on the boundary of .
      Note that there will be exactly two if such a component came from a chord on
      the river bank, but there can be more if it came from the river touching a
      side of  multiple times.
      Then we can place these subgraphs using the de Fraysseix-Pach-Pollack
      algorithm, starting from the vertices that are already fixed (at distance
      at least ) and adding the remaining  vertices one by one
      using  edges. This results in a drawing with resolution
      which can be roughly bounded by .
Then, it is sufficient to squeeze them by a factor of  to make sure
      that they do not block any potential edges, and a further factor 2
      to make sure that the tips of the de Fraysseix-Pach-Pollack drawings are in
      fact far enough away from these potential edges, guaranteeing a good
      resolution.
      This means the final resolution of the drawing is .
    \end {proof}


  \subsection {Putting it Together}

    To conclude, Lemmas~\ref {lem:divide} and~\ref {lem:internal} together imply:

    \begin {theorem} \label {thm:fixeddrawing}
      Given a plane graph  with  vertices, a convex polygon  with  corners
      and potential resolution ,  and a map  that maps 
      vertices on the outer face of  to the  corners of ,
      we can draw a  in  respecting  using resolution .
    \end {theorem}

    Note that by Lemma~\ref {lem:ressize}, we can rephrase this in terms of the
    more standard \emph {area} of a drawing when all coordinates are integer.

    \begin {corollary}
      Given a plane graph  with  vertices, a convex polygon  with  corners,
      at integer coordinates and diameter , and a map  that maps 
      vertices on the outer face of  to the  corners of ,
      we can draw the graph  in a scaled copy  of  that has diameter , such that the drawing 
      respects  and uses only integer coordinates for the vertices of .
\end {corollary}

    \begin {proof}
      First of all, if  has only integer coordinate vertices and diameter ,
      then its potential resolution is at least . To see this, consider
      a triangle formed by any three vertices of : this triangle has area at
      least , and in any direction its base is at most  so its height
      must be at least .

      Now, by Theorem~\ref {thm:fixeddrawing}, we can draw  inside  with
      resolution . Then we can blow up the drawing by
      a factor , which results in a polygon of diameter 
      and at least constant resolution. By Lemma~\ref {lem:ressize}, there now also
      exists a drawing of  in a polygon  of diameter 
      in which all vertices are drawn with integer coordinates.
    \end {proof}


\section {Application to Drawing Graphs of Genus }

  As mentioned in the introduction,
  graphs of genus  are often drawn in the plane by drawing their \emph {polygonal schema} in a
  prescribed convex polygon. Using a canonical polygon schema
  allows us to draw this outer face
  as a
  regular -gon that has some pairs of edges identified,
  and vertices on those edges
  duplicated.
  Given previous work by Duncan {\it et al.}~\cite{duncan},
  which gives us a chord-free polygonal schema derived from a graph
   combinatorially-embedded in a genus- surface, we can
  complete a straight-line drawing of  using a given regular -gon as
  its external face, by
  applying Theorem~\ref {thm:fixeddrawing} and rounding the
  coordinates.
  Since a regular -gon with diameter  has potential resolution , this results in a
  drawing with resolution .

Of course, there is a
  slight issue with using a regular -gon:
  not every regular -gon can be embedded with fractional
  coordinates.
  So, such a drawing will not fit exactly on an integer grid no matter how big the
  integers can be.
  Thus, we either have to allow for non-integer coordinates or allow
  for a slight (possibly imperceptible) perturbation of the vertex
  coordinates.

\section{Conclusion and Open Problems}
We have given an algorithm to draw any combinatorially-embedded
planar graph with a prescribed convex shape as its outer face and
polynomial area, with respect to the potential resolution of that
shape. That is, if the given convex shape has a polynomially-bounded
aspect ratio, then we can draw the graph  in its interior using
polynomial area.
We have not made a strenuous attempt to optimize the exponent in this
area bound. So a natural open problem is to determine the upper and
lower bound limits of this function.

In addition, with respect to drawings of genus- graphs using a
canonical polygonal schema, although our construction guarantees that
copies of corresponding
vertices appearing in multiple boundary paths
will be drawn in the same relative order,
it does not guarantee that they will be drawn with the same
inter-path distances.
So another open problem is whether one can extend our algorithm to
draw such paths with matching inter-path distances for corresponding
vertices.

\section*{Acknowledgments}
This research was supported in part by the National Science
Foundation under grant 0830403, and by the
Office of Naval Research under MURI grant N00014-08-1-1015.

\raggedright
\bibliographystyle{abuser}
\bibliography{drawingggg}

\end{document}
