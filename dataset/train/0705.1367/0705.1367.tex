
\documentclass[11pt]{article}

\usepackage{SIGACTNews}
\usepackage{amsmath}
\usepackage{url}



\newcommand{\COLUMNsplit}{\vskip 2em\hrule \vskip 3em}

\newcommand{\COLUMNguest}[2]{\begin{center}{\LARGE\bf #1 \par}\vskip 1.5em{\Large
      \lineskip .75em\begin{tabular}[t]{c}#2
      \end{tabular}\par}\vskip 1.5em\end{center}\par}




\lefthyphenmin=2
\righthyphenmin=3







\newcommand{\sat}{\models}
\newcommand{\rimp}{\Rightarrow}
\newcommand{\riff}{\Leftrightarrow}
\renewcommand{\phi}{\varphi}
\newcommand{\<}{\langle}
\renewcommand{\>}{\rangle}
\renewcommand{\emptyset}{\varnothing}
\newcommand{\etal}{et al.}




\newcommand{\REM}[1]{\textbf{[ \textit{#1} ]}}
\newcommand{\REMP}[1]{\medskip\noindent\textbf{[ \textit{#1} ]}\medskip}
\newcommand{\MREM}[1]{\marginpar{\tiny\raggedright #1}}




\title{SIGACT News Logic Column 18\\text{ is true if  is true \textbf{Broccoli}  is true.}(A \brocc B) \rimp (A \brocc (B \brocc B))  (a \brocccons b) \le (a \brocccons (b \brocccons
b)). \begin{aligned}[c]
    & \forall x\exists y\\
    & \forall z\exists u
   \end{aligned} ~S(x,y,z,u)
 \frac{\{I\land
B\}S\{I\}}{\{I\}\mathsf{while}~B~\mathsf{do}~S\{I\land\neg B\}}
which uses an invariant  preserved by every iteration of the loop.
The problem of constructing such invariants during a proof of an
assertion is examined by Gochet and Gribomont.
The second application of first-order logic they describe is logic
programming, via a rather nice tutorial illustrating the core ideas of
this programming paradigm.

\item[13.] \textbf{Stochastic \emph{versus} Deterministic Features in
Learning Models}\\
Stamatescu gives an overview of the debate on the role of randomness
and stochastic phenomena in scientific inquiry.
Roughly speaking, the two sides of the debate take the view
that randomness should either be treated as  ``statistical noise'', or
be taken into account directly as a process at work in the model. 
This debate is illustrated in the context of learning theory.

\item[14.] \textbf{Praxic Logics}\\
Finkelstein and Baugh contribute the first article on quantum logic in
this collection. 
They mainly argue for a variant semantics for quantum theory, based on
the observation that ``pre-quantum physics describes object but
quantum physics represents actions'' (p. 218). They propose a logic
for quantum states corresponding to this variant semantics. 

\item[15.] \textbf{Reasons From Science for Limiting Classical
Logic}\\
Weingartner, after reviewing some existing logical systems for
quantum mechanics, proposes an alternative quantum logic that can be
viewed as a restriction of first-order predicate logic. 
Very roughly speaking, inference is restricted so that propositional
variables in valid schemas of predicate logic cannot be replaced by
arbitrary formulas, but rather must obey a replacement criterion. 
Similarly, a valid predicate logic formula cannot in general be
``reduced'' to equivalent simpler formulas (e.g.,  to ). 
Weingartner argues that such an approach provides a viable logic
system for quantum mechanics.

\item[16.] \textbf{The Language of Interpretation in Quantum Physics
and Its Logic}\\
It has been widely accepted since the days of the Copenhagen
interpretation of quantum mechanics that there is no good language for
describing what happens at the quantum level. Omn\'es argues that
there \emph{is} a convenient language for expressing interpretation.

\item[17.] \textbf{Why Objectivist Programs in Quantum Theory Do Not
Need an Alternative Logic}\\
Cordero critiques the logical turn in foundational quantum theory by
questioning some assumption of this programme. 
This questioning highlights the extent to which proposed quantum
logics still embody classical intuitions.
He then proceeds to show that these assumptions are mainly dropped
from three of the most developed objectivist approaches to quantum
theory. 

\item[18.] \textbf{Does Quantum Physics Require a New Logic?}\\
Mittelstaedt argues that there is no pluralism of logical systems
corresponding to different fields of experience (for instance,
classical reality versus quantum reality), and that instead there is a
hierarchy of logics, with at its base a ``true'' logic of propositions
about physical systems, which he takes to be a quantum logic.

\item[19.] \textbf{Experimental Approach to Quantum-Logical
Connectives}\\
Stachow devises a process-based semantics for Mittelstaedt's quantum
logic (see 18 above), starting from experiments for elementary
propositions, and developing experiments yielding logical connectives.

\item[20.] \textbf{From Semantics to Syntax: Quantum Logic of
Observables}\\
The original presentation of quantum logic by Birkhoff and von Neumann
is really an algebraic presentation of a quantum logic, in the same
way that Boolean algebras are an algebraic presentation of
propositional logic. 
Vasyukov attempts a syntactical reconstruction of quantum logic
corresponding to an algebraic semantics. 

\item[21.] \textbf{An Unsharp Quantum Logic from Quantum
Computation}\\
Cattaneo, Dalla Chiara, and Giuntini take a first step in  deriving a
quantum logic with a semantics informed by quantum computation. 
Roughly speaking, the quantum equivalent of logic gates (operating on
qbits, the primitive elements of quantum computation, rather than bits)
are taken to provide semantics for the logical connectives. 
The resulting logic seems to be a weak form of quantum logic.

\item[22.] \textbf{Quantum Logic and Quantum Probability}\\
Beltrametti first reviews algebraic models of events in classical and
quantum logic, as well as variations in properties of
probability in classical and quantum systems. He then proposes a
common extension of classical probability theory and quantum
probability theory, by taking convex sets of states as building
blocks. 

\item[23.] \textbf{Operator Algebras and Quantum Logic}\\
R\'edei examines the process of deriving a logic from an algebraic
semantics, in particular when the algebra of events is taken to be a
general class of non-Boolean lattices arising naturally from certain
quantum systems.


\end{itemize}

As a whole, the papers in the collection tend to be short, and not
quite self-contained---the mathematics is often kept short, statements
are not proved, and some literature chasing is perforce
necessary to understand a paper fully.
In particular, had I not gone through Haack's monograph before reading
the collection, many of the subtleties would have over my head. 
There is enormous variation as to the level of technical details
present in each contribution, from the more historical and
philosophical pieces to the more mathematically-oriented ones.
By and large, the more mathematically challenging pieces are those
dealing with set theory and with quantum logics.
This probably fits correctly with the intended audience, philosophers
of science.
It is not clear to what extent this collection will speak to an even
theoretically-minded computer scientist. 

Logic has been called ``the calculus of computer science''.
Weingartner's volume is not calculus for engineers, but calculus for
mathematicians. 
It does not directly impact the daily life of practitioners, but may
contribute to a greater understanding of the foundations. 




\begin{thebibliography}{10}

\bibitem{r:beall06}
J.~C. Beall and G.~Restall.
\newblock {\em Logical Pluralism}.
\newblock Oxford University Press, 2006.

\bibitem{r:birkhoff36}
G.~Birkhoff and J.~von Neumann.
\newblock The logic of quantum mechanics.
\newblock {\em Annals of Mathematics}, 37(4):823--843, 1936.

\bibitem{r:carpenter97}
B.~Carpenter.
\newblock {\em Type-Logical Semantics}.
\newblock MIT Press, 1997.

\bibitem{r:clarke99}
E.~M. Clarke, O.~Grumberg, and D.~Peled.
\newblock {\em Model Checking}.
\newblock MIT Press, 1999.

\bibitem{r:constable86}
R.~L. Constable, S.~F. Allen, H.~M. Bromley, W.~R. Cleaveland, J.~F. Cremer,
  R.~W. Harper, D.~J. Howe, T.~B. Knoblock, N.~P. Mendler, P.~Panangaden, J.~T.
  Sasaki, and S.~F. Smith.
\newblock {\em Implementing Mathematics in the NuPRL Proof Development System}.
\newblock Prentice-Hall, 1986.

\bibitem{r:coquand88}
T.~Coquand and G.~Huet.
\newblock The calculus of constructions.
\newblock {\em Information and Computation}, 76(2--3):95--120, 1988.

\bibitem{r:ebbinghaus95}
H.-D. Ebbinghaus and J.~Flum.
\newblock {\em Finite Model Theory}.
\newblock Springer-Verlag, 1995.

\bibitem{r:felleisen91}
M.~Felleisen.
\newblock On the expressive power of programming languages.
\newblock {\em Science of Computer Programming}, 17:35--75, 1991.

\bibitem{r:girard87}
J.-Y. Girard.
\newblock Linear logic.
\newblock {\em Theoretical Computer Science}, 50:1--102, 1987.

\bibitem{r:girard01}
J.-Y. Girard.
\newblock {Locus Solum}: From the rules of logic to the logic of rules.
\newblock {\em Mathematical Structures in Computer Science}, 11:301--506, 2001.

\bibitem{r:girard88}
J.-Y. Girard, Y.~Lafont, and P.~Taylor.
\newblock {\em Proofs and Types}.
\newblock Cambridge University Press, 1988.

\bibitem{r:griffin90}
T.~Griffin.
\newblock A formulae-as-types notion of control.
\newblock In {\em Proc.~17th Annual ACM Symposium on Principles of Programming
  Languages (POPL'90)}, pages 47--58. ACM Press, 1990.

\bibitem{r:haack74}
S.~Haack.
\newblock {\em Deviant Logic}.
\newblock Cambridge University Press, 1974.

\bibitem{r:halpern03e}
J.~Y. Halpern.
\newblock {\em Reasoning About Uncertainty}.
\newblock MIT Press, 2003.

\bibitem{r:halpern01d}
J.~Y. Halpern, R.~Harper, N.~Immerman, P.~Kolaitis, M.~Y. Vardi, and V.~Vianu.
\newblock On the unusual effectiveness of logic in computer science.
\newblock {\em Bulletin of Symbolic Logic}, 7(2):213--236, 2001.

\bibitem{r:heyting71}
A.~Heyting.
\newblock {\em Intuitionism: An Introduction}.
\newblock North-Holland, third edition, 1971.

\bibitem{r:huth99}
M.~Huth and M.~Ryan.
\newblock {\em Logic in Computer Science: Modelling and Reasoning about
  Systems}.
\newblock Cambridge University Press, 1999.

\bibitem{r:immerman98}
N.~Immerman.
\newblock {\em Descriptive Complexity}.
\newblock Springer-Verlag, 1998.

\bibitem{r:immerman05}
N.~Immerman.
\newblock Progress in descriptive complexity.
\newblock {\em SIGACT News}, 36(4):24--35, 2005.

\bibitem{r:johnstone82}
P.~T. Johnstone.
\newblock {\em Stone Spaces}, volume~3 of {\em Cambridge Studies in Advanced
  Mathematics}.
\newblock Cambridge University Press, 1982.

\bibitem{r:lambek58}
J.~Lambek.
\newblock The mathematics of sentence structure.
\newblock {\em The American Mathematical Monthly}, 65:154--170, 1958.

\bibitem{r:libkin04}
L.~Libkin.
\newblock {\em Elements of Finite Model Theory}.
\newblock Springer-Verlag, 2004.

\bibitem{r:paulson94}
L.~C. Paulson.
\newblock {\em Isabelle, A Generic Theorem Prover}, volume 828 of {\em Lecture
  Notes in Computer Science}.
\newblock Springer-Verlag, 1994.

\bibitem{r:ramsay88}
A.~Ramsay.
\newblock {\em Formal Methods in Artificial Intelligence}, volume~6 of {\em
  Cambridge Tracts in Theoretical Computer Science}.
\newblock Cambridge University Press, 1988.

\bibitem{r:reiter78}
R.~Reiter.
\newblock On reasoning by default.
\newblock In {\em Proc.~Theoretical Issues in Natural Language Processing 2
  (TINLAP-2)}, pages 210--218, 1978.

\bibitem{r:restall00}
G.~Restall.
\newblock {\em An Introduction to Substructural Logics}.
\newblock Routledge, 2000.

\bibitem{r:stone36}
M.~H. Stone.
\newblock The theory of representations for {Boolean} algebras.
\newblock {\em Transactions of the American Mathematical Society},
  40(1):37--111, 1936.

\bibitem{r:weingartner04}
P.~Weingartner, editor.
\newblock {\em Alternative Logics: Do Sciences Need them?}
\newblock Springer-Verlag, 2004.

\end{thebibliography}


\end{document}
