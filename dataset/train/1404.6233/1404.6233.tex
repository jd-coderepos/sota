\documentclass[12pt]{article}

\usepackage{a4wide}
\usepackage[english]{babel}
\usepackage{graphicx}
\usepackage{amsmath}
\usepackage{amssymb}
\usepackage{wrapfig}
\usepackage{xspace}
\usepackage{enumerate}
\usepackage{array}
\usepackage{longtable}
\usepackage{float}

\graphicspath{{./images/}}



\newtheorem{defin}{Definition}
 \newenvironment{definition}{\begin{defin} \sl}{\end{defin}}
\newtheorem{theo}[defin]{Theorem}
 \newenvironment{theorem}{\begin{theo} \sl}{\end{theo}}
\newtheorem{lem}[defin]{Lemma}
 \newenvironment{lemma}{\begin{lem} \sl}{\end{lem}}
\newtheorem{coro}[defin]{Corollary}
 \newenvironment{corollary}{\begin{coro} \sl}{\end{coro}}
\newtheorem{prop}[defin]{Proposition}
 \newenvironment{proposition}{\begin{prop} \sl}{\end{prop}}

\newenvironment{proof}{\emph{Proof.}}{\hfill \\}



\newcommand{\note}[1]{\textbf{NOTE: #1}}

\newcommand{\true}{{\sc true}\xspace}
\newcommand{\false}{{\sc false}\xspace}

\newcommand{\etal}{\emph{et~al.}\xspace}
\newcommand{\keywords}[1]{\noindent\textbf{Keywords}~ #1}

\newcommand{\halfgraph}{half-\ensuremath{\theta_6}-graph\xspace}
\newcommand{\Graph}[1]{\ensuremath{\theta_{(4 k + #1)}}-Graph\xspace}
\newcommand{\graph}[1]{\ensuremath{\theta_{(4 k + #1)}}-graph\xspace}
\newcommand{\canon}[2]{\ensuremath{T_{#1 #2}}}

\newcommand{\const}{\ensuremath{\boldsymbol{c}}\xspace}

\def\Item{\item\abovedisplayskip=0pt\abovedisplayshortskip=0pt~\vspace*{-\baselineskip}}

\title{Towards Tight Bounds on Theta-Graphs}

\author{
Prosenjit Bose\thanks{School of Computer Science, Carleton University. Research supported in part by FQRNT, NSERC, and Carleton University's President's 2010 Doctoral Fellowship. Email: \texttt{jit@scs.carleton.ca, jdecaruf@cg.scs.carleton.ca, morin@scs.carleton.ca, andre@cg.scs.carleton.ca, sander@cg.scs.carleton.ca}.}
\and
\addtocounter{footnote}{-1}
Jean-Lou De Carufel\footnotemark
\and 
\addtocounter{footnote}{-1}
Pat Morin\footnotemark
\and 
\addtocounter{footnote}{-1}
Andr\'e van Renssen\footnotemark
\and
\addtocounter{footnote}{-1}
Sander Verdonschot\footnotemark
}

\date{}

\begin{document}

\maketitle

\begin{abstract}
  We present improved upper and lower bounds on the spanning ratio of -graphs with at least six cones. Given a set of points in the plane, a -graph partitions the plane around each vertex into  disjoint cones, each having aperture , and adds an edge to the `closest' vertex in each cone. We show that for any integer , -graphs with  cones have a spanning ratio of  and we provide a matching lower bound, showing that this spanning ratio tight. 

  Next, we show that for any integer , -graphs with  cones have spanning ratio at most . We also show that -graphs with  and  cones have spanning ratio at most . This is a significant improvement on all families of -graphs for which exact bounds are not known. For example, the spanning ratio of the -graph with 7 cones is decreased from at most 7.5625 to at most 3.5132. These spanning proofs also imply improved upper bounds on the competitiveness of the -routing algorithm. In particular, we show that the -routing algorithm is -competitive on -graphs with  cones and that this ratio is tight.  

  Finally, we present improved lower bounds on the spanning ratio of these graphs. Using these bounds, we provide a partial order on these families of -graphs. In particular, we show that -graphs with  cones have spanning ratio at least , where  is . This is somewhat surprising since, for equal values of , the spanning ratio of -graphs with  cones is greater than that of -graphs with  cones, showing that increasing the number of cones can make the spanning ratio worse. \\

  \keywords{Computational geometry, Spanners, Theta-graphs, Spanning Ratio, Tight bounds}
\end{abstract}


\section{Introduction}
A geometric graph  is a graph whose vertices are points in the plane and whose edges are line segments between pairs of points. A graph  is called plane if no two edges intersect properly. Every edge is weighted by the Euclidean distance between its endpoints. The distance between two vertices  and  in , denoted by , or simply  when  is clear from the context, is defined as the sum of the weights of the edges along the shortest path between  and  in . A subgraph  of  is a -spanner of  (for ) if for each pair of vertices  and , . The smallest value  for which  is a -spanner is the \emph{spanning ratio} or \emph{stretch factor} of . The graph  is referred to as the {\em underlying graph} of . The spanning properties of various geometric graphs have been studied extensively in the literature (see \cite{BS11,NS-GSN-06} for a comprehensive overview of the topic). 

Given a spanner, however, it is important to be able to route, i.e. find a short path, between any two vertices. A routing algorithm is said to be \emph{-competitive} with respect to  if the length of the path returned by the routing algorithm is not more than  times the length of the shortest path in ~\cite{BFRV12}. The smallest value  for which a routing algorithm is -competitive with respect to  is the \emph{routing ratio} of that routing algorithm. 

In this paper, we consider the situation where the underlying graph  is a straightline embedding of the complete graph on a set of  points in the plane with the weight of an edge  being the Euclidean distance~ between  and . A spanner of such a graph is called a \emph{geometric spanner}. We look at a specific type of geometric spanner: -graphs. 

Introduced independently by Clarkson~\cite{Cl87} and Keil~\cite{Keil88}, -graphs are constructed as follows (a more precise definition follows in Section~\ref{sec:Preliminaries}): for each vertex , we partition the plane into  disjoint cones with apex , each having aperture . When  cones are used, we denote the resulting -graph by the -graph. The -graph is constructed by, for each cone with apex , connecting  to the vertex  whose projection onto the bisector of the cone is closest. Ruppert and Seidel~\cite{RS91} showed that the spanning ratio of these graphs is at most , when , i.e. there are at least seven cones. This proof also showed that the \emph{-routing} algorithm (defined in Section~\ref{sec:Preliminaries}) is -competitive on these graphs. 

Recently, Bonichon~\etal~\cite{BGHI10} showed that the -graph has spanning ratio 2. This was done by dividing the cones into two sets, positive and negative cones, such that each positive cone is adjacent to two negative cones and vice versa. It was shown that when edges are added only in the positive cones, in which case the graph is called the \halfgraph, the resulting graph is equivalent to the Delaunay triangulation where the empty region is an equilateral triangle. The spanning ratio of this graph is 2, as shown by Chew~\cite{Chew89}. An alternative, inductive proof of the spanning ratio of the \halfgraph was presented by \mbox{Bose~\etal~\cite{BFRV12}}, along with an optimal local competitive routing algorithm on the half--graph. 

Tight bounds on spanning ratios are notoriously hard to obtain. The standard Delaunay triangulation (where the empty region is a circle) is a good example. Its spanning ratio has been studied for over 20 years and the upper and lower bounds still do not match. Also, even though it was introduced about 25 years ago, the spanning ratio of the -graph has only recently been shown to be finite and tight, making it the first and, until now, only -graph for which tight bounds are known. 

In this paper, we improve on the existing upper bounds on the spanning ratio of all -graphs with at least six cones. First, we generalize the spanning proof of the \halfgraph given by Bose~\etal~\cite{BFRV12} to a large family of -graphs: the \graph{2}, where  is an integer. We show that the \graph{2} has a tight spanning ratio of  (see Section~\ref{subsec:Theta4k+2}). 

We continue by looking at upper bounds on the spanning ratio of the other three families of -graphs: the \graph{3}, the \graph{4}, and the \graph{5}, where  is an integer and at least 1. We show that the \graph{4} has a spanning ratio of at most  (see Section~\ref{subsec:Theta4k+4}). We also show that the \graph{3} and the \graph{5} have spanning ratio at most  (see Section~\ref{subsec:Theta4k+35}). As was the case for Ruppert and Seidel, the structure of these spanning proofs implies that the upper bounds also apply to the competitiveness of -routing on these graphs. These results are summarized in Table~\ref{tab:Summary}. 

Finally, we present improved lower bounds on the spanning ratio of these graphs (see Section~\ref{sec:LowerBounds}) and we provide a partial order on these families (see Section~\ref{sec:Comparison}). In particular, we show that -graphs with  cones have spanning ratio at least . This is somewhat surprising since, for equal values of , the spanning ratio of -graphs with  cones is greater than that of -graphs with  cones, showing that increasing the number of cones can make the spanning ratio worse.

\begin{table}[ht]
  \begin{center}
    \begin{tabular}{| >{\centering\arraybackslash}m{\dimexpr.17\linewidth-2\tabcolsep} || >{\centering\arraybackslash}m{\dimexpr.23\linewidth-2\tabcolsep} | >{\centering\arraybackslash}m{\dimexpr.23\linewidth-2\tabcolsep} | >{\centering\arraybackslash}m{\dimexpr.36\linewidth-2\tabcolsep} |}
    \hline
    & Current Spanning & Current Routing & Previous Spanning \& Routing \\ 
    \hline \hline
    \graph{2} &  & ~\cite{RS91} & \vspace{2ex} ~\cite{RS91} \\ [2ex]
    \hline
    \graph{3} &  &   & \vspace{2ex} ~\cite{RS91} \\ [2ex] 
    \hline
    \graph{4} &  &  & \vspace{2ex} ~\cite{RS91} \\ [2ex]
    \hline
    \graph{5} &  &   & \vspace{2ex} ~\cite{RS91} \\ [2ex] 
    \hline
    \end{tabular}
  \end{center} 
  \caption{An overview of current and previous spanning and routing ratios of -graphs}
  \label{tab:Summary}
\end{table}


\section{Preliminaries}
\label{sec:Preliminaries}
Let a \emph{cone} be the region in the plane between two rays originating from the same vertex (referred to as the apex of the cone). When constructing a -graph, for each vertex  consider the rays originating from~ with the angle between consecutive rays being  (see Figure~\ref{fig:Cones}). Each pair of consecutive rays defines a cone. The cones are oriented such that the bisector of some cone coincides with the vertical halfline through  that lies above . We refer to this cone as  and number the cones in clockwise order around . The cones around the other vertices have the same orientation as the ones around . If the apex is clear from the context, we write  to indicate the -th cone. 

For ease of exposition, we only consider point sets in general position: no two vertices lie on a line parallel to one of the rays that define the cones, no two vertices lie on a line perpendicular to the bisector of one of the cones, and no three points are collinear. 

\begin{figure}[ht]
  \begin{center}
    \includegraphics{Cones}
  \end{center}
  \caption{The cones having apex  in the -graph}
  \label{fig:Cones}
\end{figure}

The -graph is constructed as follows: for each cone  of each vertex~, add an edge from~ to the closest vertex in that cone, where the distance is measured along the bisector of the cone (see Figure~\ref{fig:Projection}). More formally, we add an edge between two vertices  and  if , and for all vertices , , where  and  denote the orthogonal projection of  and  onto the bisector of . Note that our assumptions of general position imply that each vertex adds at most one edge per cone to the graph. 

\begin{figure}[ht]
  \begin{center}
    \includegraphics{Projection}
  \end{center}
  \caption{Three vertices are projected onto the bisector of a cone of . Vertex  is the closest vertex}
  \label{fig:Projection}
\end{figure}

Using the structure of the -graph, \emph{-routing} is defined as follows. Let  be the destination of the routing algorithm and let  be the current vertex. If there exists a direct edge to , follow this edge. Otherwise, follow the edge to the closest vertex in the cone of  that contains . 

Finally, given a vertex  in cone  of a vertex , we define the \emph{canonical triangle} \canon{u}{w} to be the triangle defined by the borders of  and the line through  perpendicular to the bisector of . We use  to denote the midpoint of the side of \canon{u}{w} opposite  and  to denote the smaller unsigned angle between  and  (see Figure~\ref{fig:CanonicalTriangle}). Note that for any pair of vertices  and  in the -graph, there exist two canonical triangles: \canon{u}{w} and \canon{w}{u}. 

\begin{figure}[ht]
  \begin{center}
    \includegraphics{CanonicalTriangle}
  \end{center}
  \caption{The canonical triangle \canon{u}{w}}
  \label{fig:CanonicalTriangle}
\end{figure}


\section{Some Geometric Lemmas}
\label{sec:GeometricLemmas}
First, we prove a few geometric lemmas that are useful when bounding the spanning ratios of the graphs. We start with a nice geometric property of the \graph{2}. 

\begin{lemma}
  \label{lem:Boundary}
  In the \graph{2}, any line perpendicular to the bisector of a cone is parallel to the boundary of some cone. 
\end{lemma}
\begin{proof}
  The angle between the bisector of a cone and the boundary of that cone is . In the \graph{2}, since , the angle between the bisector and the line perpendicular to this bisector is . Thus the angle between the line perpendicular to the bisector and the boundary of the cone is . Since a cone boundary is placed at every multiple of , the line perpendicular to the bisector is parallel to the boundary of some cone. 
\end{proof}

This property helps when bounding the spanning ratio of the \graph{2}. However, before deriving this bound, we prove a few other geometric lemmas. We use  to denote the smaller angle between line segments  and . 

\begin{lemma}
  \label{lem:FourPoints}
  Let , , , and  be four points on a circle such that . It holds that  and . 
\end{lemma}
\begin{proof}
  This situation is illustrated in Figure~\ref{fig:FourPoints}. Without loss of generality, we assume that . Since  and  lie on the same circle and  and  are the angle opposite to the same chord , the inscribed angle theorem implies that . Furthermore, since ,  lies to the right of the perpendicular bisector of . 

  \begin{figure}
    \begin{center}
      \includegraphics{FourPoints}
    \end{center}
    \caption{Illustration of the proof of Lemma~\ref{lem:FourPoints}}
    \label{fig:FourPoints}
  \end{figure}

  First, we show that  by showing that . Let  be the point on the circle when we mirror  along the perpendicular bisector of . Points  and  partition the circle into two arcs. Since ,  lies on the upper arc of the circle. We focus on triangle . The locus of the point  such that the perimeter of  is constant defines an ellipse. This ellipse has major axis  and goes through  and . Since this major axis is horizontal, the ellipse does not intersect the upper arc of the circle. Hence, since  lies on the upper arc of the circle, which is outside of the ellipse, the perimeter of  is greater than that of , completing the first half of the proof. 

  Next, we show that . Using the sine law, we have that   and . Since , we have that . Hence, since , we have that . 
\end{proof}

\begin{lemma}
  \label{lem:ApplyFourPoints} 
  Let ,  and  be three vertices in the \graph{x}, where , such that  and , to the left of . Let  be the intersection of the side of  opposite to  with the left boundary of . Let  denote the cone of  that contains  and let  and  be the upper and lower corner of . If , or  and , then  and .
\end{lemma}
\begin{proof}
  This situation is illustrated in Figure~\ref{fig:ApplyFourPoints}. We perform case distinction on  . 

  \begin{figure}[ht]
    \begin{center}
      \includegraphics{ApplyFourPoints}
    \end{center}
    \caption{The two cases for the situation where we apply Lemma~\ref{lem:FourPoints}: (a) , \mbox{(b) }}
    \label{fig:ApplyFourPoints}
  \end{figure}

  \textit{Case 1:} If  (see Figure~\ref{fig:ApplyFourPoints}a), we need to show that when , we have that  and . Since angles  and  are both angles between the boundary of a cone and the line perpendicular to its bisector, we have that . Thus,  lies on the circle through , , and . Therefore, if we can show that , Lemma~\ref{lem:FourPoints} proves this case. 

  We show  in two steps. Since  and , we have that . Hence, since , we have that . It remains to show that . We note that  and , since . Using that  and , we have the following. 
  

  \textit{Case 2:} If  (see Figure~\ref{fig:ApplyFourPoints}b), we need to show that when , we have that  and . Since angles  and  are both angles between the boundary of a cone and the line perpendicular to its bisector, we have that . Thus, when we reflect  in the line through , the resulting point  lies on the circle through , , and . Therefore, if we can show that , Lemma~\ref{lem:FourPoints} proves this case. 

  We show  in two steps. Since  and , we have that . Hence, since , we have that . It remains to show that . We note that  and . Using that  and , we have the following. 
   
\end{proof}

\begin{lemma}
  \label{lem:CalculationCase}
  Let ,  and  be three vertices in the \graph{x}, such that ,  to the left of , and . Let  be the intersection of the side of  opposite to  with the left boundary of . Let  and  be the corners of  opposite to . Let  and let  be the unsigned angle between  and the bisector of \canon{v}{w}. Let \const be a positive constant. If 
  
  then 
  
\end{lemma}
\begin{proof}
  This situation is illustrated in Figure~\ref{fig:CalculationLemma}. Since the angle between the bisector of a cone and its boundary is , by the sine law, we have the following. 
  

  \begin{figure}[ht]
    \begin{center}
      \includegraphics{CalculationLemma}
    \end{center}
    \caption{Finding a constant \const such that }
    \label{fig:CalculationLemma}
  \end{figure}

  \noindent To show that (\ref{ineq:CalculationCase2}) holds, we first multiply both sides by  and rewrite as follows. 
  
  
  Therefore, to prove that (\ref{ineq:CalculationCase1}) implies (\ref{ineq:CalculationCase2}), we rewrite (\ref{ineq:CalculationCase1}) as follows. 
   

  It remains to show that . Since , we have that \mbox{}. Moreover, we have that , by definition. This implies that , or equivalently, . Thus, we need to show that , or equivalently, . It suffices to show that . This follows from , , and the fact that . 
\end{proof}


\section{Upper Bounds}
In this section, we provide improved upper bounds for the four families of -graphs: the \graph{2}, the \graph{3}, the \graph{4}, and the \graph{5}. We first prove that the \graph{2} has a tight spanning ratio of . Next, we provide a generic framework for the spanning proof for the three other families of -graphs. After providing this framework, we fill in the blanks for the individual families. 


\subsection{Optimal Bounds on the \Graph{2}}
\label{subsec:Theta4k+2}
We start by showing that the \graph{2} has a spanning ratio of . At the end of this section, we also provide a matching lower bound, proving that this spanning ratio is tight. 

\begin{theorem}
  \label{theo:PathLength}
  Let  and  be two vertices in the plane. Let  be the midpoint of the side of \canon{u}{w} opposite  and let  be the unsigned angle between  and . There exists a path connecting  and  in the \graph{2} of length at most  
\end{theorem}
\begin{proof}
  We assume without loss of generality that . We prove the theorem by induction on the area of  (formally, induction on the rank, when ordered by area, of the canonical triangles for all pairs of vertices). Let  and  be the upper left and right corners of  and let  and  be the left and right intersections of the left and right boundaries of  and the boundaries of , the cone of  that contains  (see Figure~\ref{fig:TriangleCases}). Our inductive hypothesis is the following, where  denotes the length of the shortest path from  to  in the \graph{2}:
  \begin{itemize}
    \item If  is empty, then .
    \item If  is empty, then .
    \item If neither  nor  is empty, then . 
  \end{itemize}
  Note that if both  and  are empty, the induction hypothesis implies that . 

  We first show that this induction hypothesis implies the theorem. Basic trigonometry gives us the following equalities: , , , and . Thus, the induction hypothesis gives us that  

  \textbf{Base case:}  has rank 1. Since the triangle is a smallest triangle,  is the closest vertex to  in that cone. Hence, the edge  is part of the \graph{2} and . From the triangle inequality, we have , so the induction hypothesis holds.

  \textbf{Induction step:} We assume that the induction hypothesis holds for all pairs of vertices with canonical triangles of rank up to . Let  be a canonical triangle of rank .

  If  is an edge in the \graph{2}, the induction hypothesis follows from the same argument as in the base case. If there is no edge between  and , let  be the vertex closest to  in , and let  and  be the upper left and right corners of  (see Figure~\ref{fig:TriangleCases}). By definition, , and by the triangle inequality, .

  \begin{figure}[ht]
    \begin{center}
      \includegraphics{TriangleCases}
    \end{center}
    \caption{The three cases of the induction step based on the cone of  that contains , in this case for the -graph}
    \label{fig:TriangleCases}
  \end{figure}

  Without loss of generality, we assume that  lies to the left of . We perform a case analysis based on the cone of  that contains : (a) , (b)  where , (c) . 

  \textbf{Case (a):} Vertex  lies in  (see Figure~\ref{fig:TriangleCases}a). Let  and  be the upper left and right corners of , and let  and  be the left and right intersections of  and the boundaries of . Since  has smaller area than , we apply the inductive hypothesis to . We need to prove all three statements of the inductive hypothesis for .

  \begin{enumerate}
  \item If  is empty, then  is also empty, so by induction . Since , , , and  form a parallelogram, we have:
  
  which proves the first statement of the induction hypothesis.

  \item If  is empty, an analogous argument proves the second statement of the induction hypothesis.

  \item If neither  nor  is empty, by induction we have . Assume, without loss of generality, that the maximum of the right hand side is attained by its second argument  (the other case is similar). Since vertices , , , and  form a parallelogram, we have that:
  
  which proves the third statement of the induction hypothesis. 
  \end{enumerate}

  \textbf{Case (b):} Vertex  lies in  where  (see Figure~\ref{fig:TriangleCases}b). In this case,  lies in . Therefore, the first statement of the induction hypothesis for \canon{u}{w} is vacuously true. It remains to prove the second and third statement of the induction hypothesis. Let  be the intersection of the side of  opposite  and the left boundary of . Since  is smaller than , by induction we have . Since  where , we can apply Lemma~\ref{lem:ApplyFourPoints}. Note that point  in Lemma~\ref{lem:ApplyFourPoints} corresponds to point  in this proof. Hence, we get that . Since  and , , , and  form a parallelogram, we have that , proving the induction hypothesis for \canon{u}{w}.

  \textbf{Case (c):} Vertex  lies in  (see Figure~\ref{fig:TriangleCases}c). Since  lies in , the first statement of the induction hypothesis for \canon{u}{w} is vacuously true. It remains to prove the second and third statement of the induction hypothesis. Let  and  be the upper and lower left corners of \canon{w}{v}, and let  be the intersection of \canon{w}{v} and the lower boundary of , i.e. the cone of  that contains . Note that  is also the right intersection of \canon{u}{v} and \canon{w}{v}. Since  is the closest vertex to , \canon{u}{v} is empty. Hence,  is empty. Since \canon{w}{v} is smaller than \canon{u}{w}, we can apply induction on it. As  is empty, the induction hypothesis for \canon{w}{v} gives . Since  and , , , and  form a parallelogram, we have that , proving the second and third statement of the induction hypothesis for \canon{u}{w}. 
\end{proof}

Since  is increasing for , for , it is maximized when , and we obtain the following corollary: 

\begin{coro}
  \label{cor:SpanningRatio}
  The \graph{2} is a -spanner. 
\end{coro}

The upper bounds given in Theorem~\ref{theo:PathLength} and Corollary~\ref{cor:SpanningRatio} are tight, as shown in Figure~\ref{fig:RatioTight}: we place a vertex~ arbitrarily close to the upper corner of \canon{u}{w} that is furthest from . Likewise, we place a vertex  arbitrarily close to the lower corner of \canon{w}{u} that is furthest from . Both shortest paths between  and  visit either  or , so the path length is arbitrarily close to , showing that the upper bounds are tight. 

\begin{figure}[ht]
  \begin{center}
    \includegraphics{RatioTight}
  \end{center}
  \caption{The lower bound for the \graph{2}}
  \label{fig:RatioTight}
\end{figure}


\subsection{Generic Framework for the Spanning Proof}
In this section, we provide a generic framework for the spanning proof for the three other families of -graphs: the \graph{3}, the \graph{4}, and the \graph{5}. This framework contains those parts of the spanning proof that are identical for all three families. In the subsequent sections, we handle the single case that depends on each specific family and determines their respective spanning ratios. 

\begin{theorem}
  \label{theo:PathLengthGeneric}
  Let  and  be two vertices in the plane. Let  be the midpoint of the side of \canon{u}{w} opposite  and let  be the unsigned angle between  and . There exists a path connecting  and  in the \graph{x} of length at most 
   where  is a function that depends on  and . For the \graph{4}, \const equals  and for the \graph{3} and \graph{5}, \const equals  .
\end{theorem}
\begin{proof}
  We assume without loss of generality that . We prove the theorem by induction on the area of \canon{u}{w} (formally, induction on the rank, when ordered by area, of the canonical triangles for all pairs of vertices). Let  and  be the upper left and right corners of \canon{u}{w}. Our inductive hypothesis is the following, where  denotes the length of the shortest path from  to  in the \graph{x}: . 

  We first show that this induction hypothesis implies the theorem. Basic trigonometry gives us the following equalities: , , , and . Thus the induction hypothesis gives that  

  \textbf{Base case:} \canon{u}{w} has rank 1. Since the triangle is a smallest triangle,  is the closest vertex to  in that cone. Hence, the edge  is part of the \graph{x} and . From the triangle inequality and the fact that , we have , so the induction hypothesis holds.

  \textbf{Induction step:} We assume that the induction hypothesis holds for all pairs of vertices with canonical triangles of rank up to . Let \canon{u}{w} be a canonical triangle of rank .

  If  is an edge in the \graph{x}, the induction hypothesis follows from the same argument as in the base case. If there is no edge between  and , let  be the vertex closest to  in \canon{u}{w}, and let  and  be the upper left and right corners of \canon{u}{v} (see Figure~\ref{fig:TriangleCasesGeneric}). By definition, , and by the triangle inequality, .

  \begin{figure}[ht]
    \begin{center}
      \includegraphics{TriangleCasesGeneric}
    \end{center}
    \caption{The four cases of the induction step based on the cone of  that contains , in this case for the -graph}
    \label{fig:TriangleCasesGeneric}
  \end{figure}

  Without loss of generality, we assume that  lies to the left of . We perform a case analysis based on the cone of  that contains , where  and  are the left and right corners of \canon{v}{w}, opposite to : (a) , (b)  where , or  and , (c)  and , \mbox{(d) }. 

  \textbf{Case (a):} Vertex  lies in  (see Figure~\ref{fig:TriangleCasesGeneric}a). Since \canon{v}{w} has smaller area than \canon{u}{w}, we apply the inductive hypothesis to \canon{v}{w}. Hence we have . Since  lies to the left of , the maximum of the right hand side is attained by its first argument, . Since vertices , , , and  form a parallelogram, and , we have that
  
  which proves the induction hypothesis. 

  \textbf{Case (b):} Vertex  lies in , where , or  and  (see Figure~\ref{fig:TriangleCasesGeneric}b). Let  be the intersection of the side of  opposite  and the left boundary of . Since  is smaller than , by induction we have . Since  where , or  and , we can apply Lemma~\ref{lem:ApplyFourPoints}. Note that point  in Lemma~\ref{lem:ApplyFourPoints} corresponds to point  in this proof. Hence, we get that  and . Since , this implies that  . Since  and , , , and  form a parallelogram, we have that , proving the induction hypothesis for \canon{u}{w}.

  \textbf{Case (c) and (d)} Vertex  lies in  and , or  lies in  (see Figures~\ref{fig:TriangleCasesGeneric}c and d). Let  be the intersection of the side of  opposite  and the left boundary of . Since \canon{v}{w} is smaller than \canon{u}{w}, we can apply induction on it. The actual application of the induction hypothesis varies for the three families of -graphs and, using Lemma~\ref{lem:CalculationCase}, determines the value of \const. Hence, these cases are discussed in the spanning proofs of the three families. 
\end{proof}


\subsection{Upper Bound on the \Graph{4}}
\label{subsec:Theta4k+4}
In this section, we improve the upper bounds on the spanning ratio of the \graph{4}, for any integer . 

\begin{theorem}
  \label{theo:PathLength4k+4}
  Let  and  be two vertices in the plane. Let  be the midpoint of the side of \canon{u}{w} opposite  and let  be the unsigned angle between  and . There exists a path connecting  and  in the \graph{4} of length at most 
   
\end{theorem}
\begin{proof}
  We apply Theorem~\ref{theo:PathLengthGeneric} using . It remains to handle Case (c), where  and , and Case (d), where . 

  Recall that  and  are the left and right corners of \canon{v}{w}, opposite to , and  is the intersection of the side of  opposite  and the left boundary of . Let  be  and let  be the angle between  and the bisector of \canon{v}{w}. Since \canon{v}{w} is smaller than \canon{u}{w}, the induction hypothesis gives an upper bound on . Since  and , , , and  form a parallelogram, we need to show that  for both cases in order to complete the proof. 

  \begin{figure}[ht]
    \begin{center}
      \includegraphics{SpanningProof4k+4}
    \end{center}
    \caption{The remaining cases of the induction step for the \graph{4}: (a)  lies in  and , (b)  lies in }
    \label{fig:SpanningProof4k+4}
  \end{figure}

  \textbf{Case (c):} When  lies in  and , the induction hypothesis for \canon{v}{w} gives  (see Figure~\ref{fig:SpanningProof4k+4}a). We note that . Hence, the inequality follows from Lemma~\ref{lem:CalculationCase} when . Since this function is decreasing in  for , it is maximized when  equals . Hence,  needs to be at least , which can be rewritten to . 

  \textbf{Case (d):} When  lies in ,  lies above the bisector of \canon{v}{w} (see Figure~\ref{fig:SpanningProof4k+4}b) and the induction hypothesis for \canon{v}{w} gives . We note that . Hence, the inequality follows from Lemma~\ref{lem:CalculationCase} when , which is equal to . 
\end{proof}

Since  is increasing for , for , it is maximized when , and we obtain the following corollary: 

\begin{corollary}
  \label{cor:SpanningRatio4k+4}
  The \graph{4} is a -spanner. 
\end{corollary}

Furthermore, we observe that the proof of Theorem~\ref{theo:PathLength4k+4} follows the same path as the -routing algorithm follows: if the direct edge to the destination is part of the graph, it follows this edge, and if it is not, it follows the edge to the closest vertex in the cone that contains the destination. 

\begin{corollary}
  \label{cor:Routing4k+4}
  The -routing algorithm is -competitive on the \\ \graph{4}. 
\end{corollary}


\subsection{Upper Bounds on the \Graph{3} and \Graph{5}}
\label{subsec:Theta4k+35}
In this section, we improve the upper bounds on the spanning ratio of the \graph{3} and the \graph{5}, for any integer . 

\begin{theorem}
  \label{theo:PathLength4k+3}
  Let  and  be two vertices in the plane. Let  be the midpoint of the side of \canon{u}{w} opposite  and let  be the unsigned angle between  and . There exists a path connecting  and  in the \graph{3} of length at most 
   
\end{theorem}
\begin{proof}
  We apply Theorem~\ref{theo:PathLengthGeneric} using . It remains to handle Case (c), where  and , and Case (d), where . 

  Recall that  and  are the left and right corners of \canon{v}{w}, opposite to , and  is the intersection of the side of  opposite  and the left boundary of . Let  be  and let  be the angle between  and the bisector of \canon{v}{w}. Since \canon{v}{w} is smaller than \canon{u}{w}, the induction hypothesis gives an upper bound on . Since  and , , , and  form a parallelogram, we need to show that  for both cases in order to complete the proof. 

  \begin{figure}[ht]
    \begin{center}
      \includegraphics{SpanningProof4k+3}
    \end{center}
    \caption{The remaining cases of the induction step for the \graph{3}: (a)  lies in  and , (b)  lies in }
    \label{fig:SpanningProof4k+3}
  \end{figure}

  \textbf{Case (c):} When  lies in  and , the induction hypothesis for \canon{v}{w} gives  (see Figure~\ref{fig:SpanningProof4k+3}a). We note that . Hence, the inequality follows from Lemma~\ref{lem:CalculationCase} when . Since this function is decreasing in  for , it is maximized when  equals . Hence,  needs to be at least , which is equal to . 

  \textbf{Case (d):} When  lies in ,  lies above the bisector of \canon{v}{w} (see Figure~\ref{fig:SpanningProof4k+3}b) and the induction hypothesis for \canon{v}{w} gives . We note that . Hence, the inequality follows from Lemma~\ref{lem:CalculationCase} when , which is equal to . 
\end{proof}

\begin{theorem}
  \label{theo:PathLength4k+5}
  Let  and  be two vertices in the plane. Let  be the midpoint of the side of \canon{u}{w} opposite  and let  be the unsigned angle between  and . There exists a path connecting  and  in the \graph{5} of length at most 
   
\end{theorem}
\begin{proof}
  We apply Theorem~\ref{theo:PathLengthGeneric} using . It remains to handle Case (c), where  and , and Case (d), where . 

  Recall that  and  are the left and right corners of \canon{v}{w}, opposite to , and  is the intersection of the side of  opposite  and the left boundary of . Let  be  and let  be the angle between  and the bisector of \canon{v}{w}. Since \canon{v}{w} is smaller than \canon{u}{w}, the induction hypothesis gives an upper bound on . Since  and , , , and  form a parallelogram, we need to show that  for both cases in order to complete the proof. 

  \begin{figure}[ht]
    \begin{center}
      \includegraphics{SpanningProof4k+5}
    \end{center}
    \caption{The remaining cases of the induction step for the \graph{5}: (a)  lies in  and , (b)  lies in  and , (c)  lies in  and }
    \label{fig:SpanningProof4k+5}
  \end{figure}

  \textbf{Case (c):} When  lies in  and , the induction hypothesis for \canon{v}{w} gives  (see Figure~\ref{fig:SpanningProof4k+5}a). We note that . Hence, the inequality follows from Lemma~\ref{lem:CalculationCase} when . Since this function is decreasing in  for , it is maximized when  equals . Hence,  needs to be at least , which is less than . 

  \textbf{Case (d):} When  lies in , the induction hypothesis for \canon{v}{w} gives . If  (see Figure~\ref{fig:SpanningProof4k+5}b), the induction hypothesis for \canon{v}{w} gives . We note that . Hence, the inequality follows from Lemma~\ref{lem:CalculationCase} when , which is equal to . 

  If , the induction hypothesis for \canon{v}{w} gives  (see Figure~\ref{fig:SpanningProof4k+5}c). We note that . Hence, the inequality follows from Lemma~\ref{lem:CalculationCase} when . Since this function is decreasing in  for , it is maximized when  equals . Hence,  needs to be at least .
\end{proof}

By looking at two vertices  and  in the \graph{3} and the \graph{5}, we can see that when the angle between  and the bisector of \canon{u}{w} is , the angle between  and the bisector of \canon{w}{u} is . Hence the worst case spanning ratio corresponds to the minimum of the spanning ratio when looking at \canon{u}{w} and the spanning ratio when looking at \canon{w}{u}. 

\begin{theorem}
  \label{theo:SpanningRatio4k+3,5}
  The \graph{3} and \graph{5} are -spanners. 
\end{theorem}
\begin{proof}
  The spanning ratio of the \graph{3} and the \graph{5} is at most 
  

  Since  is increasing for , for , the minimum of these two functions is maximized when the two functions are equal, i.e. when . Thus the \graph{3} and the \graph{5} have spanning ratio at most 
   
\end{proof}

Furthermore, we observe that the proofs of Theorem~\ref{theo:PathLength4k+3} and Theorem~\ref{theo:PathLength4k+5} follow the same path as the -routing algorithm follows. Since in the case of routing, we are forced to consider the canonical triangle with the source as apex, the arguments that decreased the spanning ratio cannot be applied. Hence, we obtain the following corollary. 

\begin{corollary}
  \label{cor:Routing4k+3,5}
  The -routing algorithm is -competitive on the \graph{3} and the \graph{5}. 
\end{corollary}


\section{Lower Bounds}
\label{sec:LowerBounds}
In this section, we provide lower bounds for the \graph{3}, the \graph{4}, and the \graph{5}. For each of the families, we construct a lower bound example by extending the shortest path between two vertices  and . For brevity, we describe only how to extend one of the shortest paths between these vertices. To extend all shortest paths between  and , the same transformation is applied to all equivalent paths or canonical triangles. 

For example, when constructing the lower bound for the \graph{3}, our first step is to ensure that there is no edge between  and . To this end, the proof of Theorem~\ref{theo:LowerBound4k+3} states that we place a vertex  in the corner of \canon{u}{w} that is furthest from . Placing only this single vertex, however, does not prevent the edge  from being present, as  is still the closest vertex in \canon{w}{u}. Hence, we also place a vertex in the corner of \canon{w}{u} that is furthest from . Since these two modifications are essentially the same, but applied to different canonical triangles, we describe only the placement of one of these vertices. The full result of each step is shown in the accompanying figures. 


\subsection{Lower Bounds on the \Graph{3}}
In this section, we construct a lower bound on the spanning ratio of the \graph{3}, for any integer . 

\begin{theorem}
  \label{theo:LowerBound4k+3}
  The worst case spanning ratio of the \graph{3} is at least  
\end{theorem}
\begin{proof}
We construct the lower bound example by extending the shortest path between two vertices  and  in three steps. We describe only how to extend one of the shortest paths between these vertices. To extend all shortest paths, the same modification is performed in each of the analogous cases, as shown in Figure~\ref{fig:LowerBound4k+3}. 

\begin{figure}[ht]
  \begin{center}
    \includegraphics{LowerBound4k+3}
  \end{center}
  \vspace{-1em}
  \caption{The construction of the lower bound for the \graph{3}}
  \label{fig:LowerBound4k+3}
\end{figure}

First, we place  such that the angle between  and the bisector of the cone of  that contains  is . Next, we ensure that there is no edge between  and  by placing a vertex  in the upper corner of \canon{u}{w} that is furthest from  (see Figure~\ref{fig:LowerBound4k+3}a). Next, we place a vertex  in the corner of \canon{v_1}{w} that lies outside \canon{u}{w} (see Figure~\ref{fig:LowerBound4k+3}b). Finally, to ensure that there is no edge between  and , we place a vertex  in \canon{v_2}{w} such that \canon{v_2}{w} and \canon{v_3}{w} have the same orientation (see Figure~\ref{fig:LowerBound4k+3}c). Note that we cannot place  in the lower right corner of \canon{v_2}{w} since this would cause an edge between  and  to be added, creating a shortcut to . 

One of the shortest paths in the resulting graph visits , , , , and . Thus, to obtain a lower bound for the \graph{3}, we compute the length of this path. 

\begin{figure}[ht]
  \begin{center}
    \includegraphics{LowerBoundComputation4k+3}
  \end{center}
  \caption{The lower bound for the \graph{3}}
  \label{fig:LowerBoundComputation4k+3}
\end{figure}

Let  be the midpoint of the side of \canon{u}{w} opposite . By construction, we have that , , , , and  (see Figure~\ref{fig:LowerBoundComputation4k+3}). We can express the various line segments as follows: 



Hence, the total length of the shortest path is , which can be rewritten to  proving the theorem. 
\end{proof}


\subsection{Lower Bound on the \Graph{4}}
The \graph{2} has the nice property that any line perpendicular to the bisector of a cone is parallel to the boundary of a cone (Lemma~\ref{lem:Boundary}). As a result of this, if , , and  are vertices with  in one of the upper corners of \canon{u}{w}, then \canon{w}{v} is completely contained in \canon{u}{w}. The \graph{4} does not have this property. In this section, we show how to exploit this to construct a lower bound for the \graph{4} whose spanning ratio exceeds the worst case spanning ratio of the \graph{2}. 

\begin{theorem}
  The worst case spanning ratio of the \graph{4} is at least  
\end{theorem}
\begin{proof}
We construct the lower bound example by extending the shortest path between two vertices  and  in three steps. We describe only how to extend one of the shortest paths between these vertices. To extend all shortest paths, the same modification is performed in each of the analogous cases, as shown in Figure~\ref{fig:LowerBound4k+4}. 

\begin{figure}[H]
  \begin{center}
    \includegraphics{LowerBound4k+4}
  \end{center}
  \vspace{-1em}
  \caption{The construction of the lower bound for the \graph{4}}
  \label{fig:LowerBound4k+4}
\end{figure}

First, we place  such that the angle between  and the bisector of the cone of  that contains  is . Next, we ensure that there is no edge between  and  by placing a vertex  in the upper corner of \canon{u}{w} that is furthest from  (see Figure~\ref{fig:LowerBound4k+4}a). Next, we place a vertex  in the corner of \canon{v_1}{w} that lies in the same cone of  as  and  (see Figure~\ref{fig:LowerBound4k+4}b). Finally, we place a vertex  in the intersection of the left boundary of \canon{v_2}{w} and the right boundary of \canon{w}{v_2} to ensure that there is no edge between  and  (see Figure~\ref{fig:LowerBound4k+4}c). Note that we cannot place  in the lower right corner of \canon{v_2}{w} since this would cause an edge between  and  to be added, creating a shortcut to . 

One of the shortest paths in the resulting graph visits , , , , and . Thus, to obtain a lower bound for the \graph{4}, we compute the length of this path. 

\begin{figure}[H]
  \begin{center}
    \includegraphics{LowerBoundComputation4k+4}
  \end{center}
  \caption{The lower bound for the \graph{4}}
  \label{fig:LowerBoundComputation4k+4}
\end{figure}

Let  be the midpoint of the side of \canon{u}{w} opposite~. By construction, we have that  (see Figure~\ref{fig:LowerBoundComputation4k+4}). We can express the various line segments as follows: 


Hence, the total length of the shortest path is , which can be rewritten to   
\end{proof}


\subsection{Lower Bounds on the \Graph{5}}
In this section, we give a lower bound on the spanning ratio of the \graph{5}, for any integer . 

\begin{theorem}
  The worst case spanning ratio of the \graph{5} is at least  
\end{theorem}
\begin{proof}
We construct the lower bound example by extending the shortest path between two vertices  and  in two steps. We describe only how to extend one of the shortest paths between these vertices. To extend all shortest paths, the same modification is performed in each of the analogous cases, as shown in Figure~\ref{fig:LowerBound4k+5}. 

\begin{figure}[ht]
  \begin{center}
    \includegraphics{LowerBound4k+5}
  \end{center}
  \caption{The construction of the lower bound for the \graph{5}}
  \label{fig:LowerBound4k+5}
\end{figure}

First, we place  such that the angle between  and the bisector of the cone of  that contains  is . Next, we ensure that there is no edge between  and  by placing a vertex  in the upper corner of \canon{u}{w} that is furthest from  (see Figure~\ref{fig:LowerBound4k+5}a). Finally, we place a vertex  in the corner of \canon{v_1}{w} that lies outside \canon{u}{w}. We also place a vertex  in the corner of \canon{w}{v_1} that lies in the same cone of  as  and  (see Figure~\ref{fig:LowerBound4k+5}b). Note that placing  creates a shortcut between  and , as  is the closest vertex in one of the cones of . 

One of the shortest paths in the resulting graph visits , , and . Thus, to obtain a lower bound for the \graph{5}, we compute the length of this path. 

\begin{figure}[ht]
  \begin{center}
    \includegraphics{LowerBoundComputation4k+5}
  \end{center}
  \caption{The lower bound for the \graph{5}}
  \label{fig:LowerBoundComputation4k+5}
\end{figure}

Let  be the midpoint of the side of \canon{u}{w} opposite . By construction, we have that , , , and  (see Figure~\ref{fig:LowerBoundComputation4k+5}). We can express the various line segments as follows: 


Hence, the total length of the shortest path is , which can be rewritten to    times the length of .
\end{proof}


\section{Comparison}
\label{sec:Comparison}
In this section we prove that the upper and lower bounds of the four families of -graphs admit a partial ordering. We need the following lemma that can be proved by elementary calculus. 

\begin{lemma}
\label{lemma trigonometric inequalities}
  Let  be a real number. Then the following inequalities hold:
  \begin{enumerate}
    \item \label{lemma trigonometric inequalities item sin <}
     with equality if and only if .

    \item \label{lemma trigonometric inequalities item cos >}
     with equality if and only if .

    \item \label{lemma trigonometric inequalities item sin >}
     with equality if and only if .

    \item \label{lemma trigonometric inequalities item cos <}
     with equality if and only if .

    \item \label{lemma trigonometric inequalities item tan >}
     with equality if and only if .

    \item \label{lemma trigonometric inequalities item tan^2 >}
     with equality if and only if .
  \end{enumerate}
\end{lemma}


Using the above properties, we proceed to prove a number of relations between the four families of -graphs. 

\setcounter{equation}{0}
\renewcommand{\theequation}{\alph{equation}}

\begin{lemma}
  \label{proposition inequalities}
  Let  and  denote the upper and lower bound on the -graph: 
  
  Then the following inequalities hold where  is an integer.
  
\end{lemma}

\setcounter{equation}{2}
\renewcommand{\theequation}{\arabic{equation}}

\begin{proof}
We use the same strategy for each inequality. We use the definitions of  and  in combination with Lemma~\ref{lemma trigonometric inequalities}. Notice that the restriction on  in each of these inequalities ensures that we can apply Lemma~\ref{lemma trigonometric inequalities}. We are then left with an algebraic inequality that can be translated into a polynomial inequality, which is easy to verify. 
\begin{enumerate}[(\alph{enumi})]
\Item 

We now explain why~\eqref{proof eqn 1} holds. The inequality  can be simplified to

The largest real root of the polynomial involved in~\eqref{ineq polyn} is negative. Moreover, \eqref{proof eqn 1} holds for . Therefore, \eqref{proof eqn 1} holds for any .

\item The proof is analogous to the one of~\eqref{4k+2 monotonic}.

\item The proof is analogous to the one of~\eqref{4k+2 monotonic}.

\item We let

so that

Using a proof similar to the one of~\eqref{4k+2 monotonic}, we can prove that

Using a proof similar to the one of~\eqref{4k+2 monotonic}, we can prove that , for , thus we can proceed as follows

for .

\item The proof is analogous to the one of~\eqref{4k+2 monotonic}.

\item The proof is analogous to the one of~\eqref{4k+2 monotonic}.

\item The proof is analogous to the one of~\eqref{4k+5 monotonic}.

\item The proof is analogous to the one of~\eqref{4k+5 monotonic}.

\item The proof is analogous to the one of~\eqref{4k+5 monotonic}.
\end{enumerate}
\vspace{-1.5em}
\end{proof}

We note that inequalities~\eqref{4k+2 monotonic}, \eqref{4k+3 monotonic}, \eqref{4k+4 monotonic}, and~\eqref{4k+5 monotonic} imply that the spanning ratio is monotonic within each of the four families. We also note that increasing the number of cones of a -graph by 2 from  to  increases the worst case spanning ratio, thus showing that adding cones can make the spanning ratio worse instead of better. Therefore, the spanning ratio is non-monotonic between families. 

\begin{corollary}
  We have the following partial order on the spanning ratios of the four families (see Figure~\ref{figure partial order}).
\end{corollary}

\begin{figure}[ht]
  \centering
  \includegraphics[scale=1]{PartialOrder.pdf}
  \caption{Partial order on the spanning ratios of the four families}
  \label{figure partial order}
\end{figure}


\section{Tight Routing Bounds}
While improving the upper bounds on the spanning ratio of the \graph{4}, we also improved the upper bound on the routing ratio of the -routing algorithm. In this section we show that this bound of  and the current upper bound of  on the -graph are tight, i.e. we provide matching lower bounds on the routing ratio of the -routing algorithm on these families of graphs. 


\subsection{Tight Routing Bounds for the \Graph{4}}
In this section we show that the upper bound of  on the routing ratio of the -routing algorithm for the \graph{4} is a tight bound. 

\begin{theorem}
  The -routing algorithm is -competitive on the \graph{4} and this bound is tight. 
\end{theorem}
\begin{proof}
  Corollary~\ref{cor:Routing4k+4} showed that the routing ratio is at most , hence it suffices to show that this is also a lower bound. 

  We construct the lower bound example on the competitiveness of the -routing algorithm on the \graph{4} by repeatedly extending the routing path from source  to destination . First, we place  in the right corner of \canon{u}{w}. To ensure that the -routing algorithm does not follow the edge between  and , we place a vertex  in the left corner of \canon{u}{w}. Next, to ensure that the -routing algorithm does not follow the edge between  and , we place a vertex  in the left corner of \canon{v_1}{w}. We repeat this step until we have created a cycle around  (see Figure~\ref{fig:Routing4k+4TightSteps}a). 

  \begin{figure}[ht]
    \centering
    \includegraphics{Routing4k+4TightSteps}
    \caption{Constructing a lower bound example for -routing on the \graph{4}: (a) after constructing the first cycle, (b) after adding , the first vertex of the second cycle, and , the auxiliary vertex needed to maintain the first cycle}
    \label{fig:Routing4k+4TightSteps}
  \end{figure}

  To extend the routing path further, we again place a vertex  in the corner of the current canonical triangle. To ensure that the routing algorithm still routes to  from , we place  slightly outside of \canon{u}{v_1}. However, another problem arises: vertex  is no longer the vertex closest to  in \canon{v_1}{w}, as  is closer. To solve this problem, we also place a vertex  in \canon{v_1}{v_2} such that  lies in \canon{x_1}{w} (see Figure~\ref{fig:Routing4k+4TightSteps}b). By repeating this process four times, we create a second cycle around . 

  To add more cycles around , we repeat the same process as described above: place a vertex in the corner of the current canonical triangle and place an auxiliary vertex to ensure that the previous cycle stays intact. Note that when placing , we also need to ensure that it does not lie in \canon{x_{i-1}}{w}, to prevent shortcuts from being formed. A lower bound example consisting of two cycles is shown in Figure~\ref{fig:Routing4k+4Tight}.

  \begin{figure}[ht]
    \centering
    \includegraphics{Routing4k+4Tight}
    \caption{A lower bound example for -routing on the \graph{4}, consisting of two cycles: the first cycle is coloured orange and the second cycle is coloured blue}
    \label{fig:Routing4k+4Tight}
  \end{figure}

  This way we need to add auxiliary vertices only to the -th cycle, when adding the -th cycle, hence we can add an additional cycle using only a constant number of vertices. Since we can place the vertices arbitrarily close to the corners of the canonical triangles, we ensure that  and that the distance between consecutive vertices  and  is always  times . Hence, when we take  and let the number of vertices approach infinity, we get that the total length of the path is , which can be rewritten to .

\end{proof}


\subsection{Tight Routing Bounds for the -Graph}
In this section we show that the upper bound of  on the routing ratio of the -routing algorithm for the -graph is a tight bound. 

\begin{theorem}
  The -routing algorithm is -competitive on the -graph and this bound is tight. 
\end{theorem}
\begin{proof}
  Ruppert and Seidel~\cite{RS91} showed that the routing ratio is at most , hence it suffices to show that this is also a lower bound. 

  We construct the lower bound example on the competitiveness of the -routing algorithm on the -graph by repeatedly extending the routing path from source  to destination . First, we place  in the right corner of \canon{u}{w}. To ensure that the -routing algorithm does not follow the edge between  and , we place a vertex  in the left corner of \canon{u}{w}. Next, to ensure that the -routing algorithm does not follow the edge between  and , we place a vertex  in the left corner of \canon{v_1}{w}. We repeat this step until we have created a cycle around  (see Figure~\ref{fig:Routing10Tight}). 

  \begin{figure}[ht]
    \centering
    \includegraphics{Routing10Tight}
    \caption{A lower bound example for -routing on the -graph, consisting of two cycles: the first cycle is coloured orange and the second cycle is coloured blue}
    \label{fig:Routing10Tight}
  \end{figure}

  To extend the routing path further, we again place a vertex  in the corner of the current canonical triangle. To ensure that the routing algorithm still routes to  from , we place  slightly outside of \canon{u}{v_1}. However, another problem arises: vertex  is no longer the vertex closest to  in \canon{v_1}{w}, as  is closer. To solve this problem, we also place a vertex  in \canon{v_1}{v_2} such that  lies in \canon{x_1}{w} (see Figure~\ref{fig:Routing10Zoomed}). By repeating this process four times, we create a second cycle around . 

  \begin{figure}[ht]
    \centering
    \includegraphics{Routing10Zoomed}
    \caption{The placement of vertices such that previous cycles stay intact when adding a new cycle}
    \label{fig:Routing10Zoomed}
  \end{figure}

  To add more cycles around , we repeat the same process as described above: place a vertex in the corner of the current canonical triangle and place an auxiliary vertex to ensure that the previous cycle stays intact. Note that when placing , we also need to ensure that it does not lie in \canon{x_{i-1}}{w}, to prevent shortcuts from being formed (see Figure~\ref{fig:Routing10Zoomed}). This means that in general  does not lie arbitrarily close to the corner of \canon{v_i}{v_{i+1}}. 

  This way we need to add auxiliary vertices only to the -th cycle, when adding the -th cycle, hence we can add an additional cycle using only a constant number of vertices. Since we can place the vertices arbitrarily close to the corners of the canonical triangles, we ensure that the distance to  is always  times the distance between  and the previous vertex along the path. Hence, when we take  and let the number of vertices approach infinity, we get that the total length of the path is , which can be rewritten to .
\end{proof}


\section{Conclusion}
We showed that the \graph{2} has a tight spanning ratio of . This is the first time tight spanning ratios have been found for a large family of -graphs. Previously, the only -graph for which tight bounds were known was the -graph. We also gave improved upper bounds on the spanning ratio of the \graph{3}, the \graph{4}, and the \graph{5}. 

We also constructed lower bounds for all four families of -graphs and provided a partial order on these families. In particular, we showed that the \graph{4} has a spanning ratio of at least . This result is somewhat surprising since, for equal values of , the worst case spanning ratio of the \graph{4} is greater than that of the \graph{2}, showing that increasing the number of cones can make the spanning ratio worse. 

There remain a number of open problems, such as finding tight spanning ratios for the \graph{3}, the \graph{4}, and the \graph{5}. Similarly, for the  and -graphs, though upper and lower bounds are known, these are far from tight. It would also be nice if we could improve the routing algorithms for -graphs. At the moment, -routing is the standard routing algorithm for general -graphs, but it is unclear whether this is the best routing algorithm for general -graphs: though we showed that the current bounds on the competitiveness of the -routing algorithm are tight in case of the \graph{4}, this does not imply that there exists no algorithm that can do better on these graphs. As a special case, we note that the -routing algorithm is not -competitive on the -graph, but a better (tight) algorithm is known to exist~\cite{BFRV12}. 


\bibliographystyle{abbrv}
\bibliography{references}

\end{document}
