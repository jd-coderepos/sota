\documentclass{elsart5p}



\usepackage{algorithm}
\usepackage{algpseudocode}

\usepackage{amsmath}

\usepackage{amssymb}
\usepackage{graphicx}

\graphicspath{{graphics/}}

\newcommand{\eat}[1]{}
\newcommand{\sep}{{\bf t\!\!-\!\!sep}}

\begin{document}

\begin{frontmatter}




\title{Local and Global Analysis of Parametric Solid Sweeps}


 \author{Bharat Adsul, Jinesh Machchhar, Milind Sohoni}          
\begin{abstract}
In this work, we propose a detailed computational framework for modelling the
envelope of the swept volume, that is the boundary of the volume obtained by
sweeping an input solid along a trajectory of rigid motions. Our framework 
is adapted to the well-established
industry-standard brep format to enable its implementation in modern CAD
systems. This is achieved via a ``local analysis'', which covers
parametrization and singularities, as well as a ``global theory'' which
tackles face-boundaries, self-intersections and trim curves.  Central to
the local analysis is the ``funnel'' which serves as a natural parameter space
for the basic surfaces constituting the sweep. The trimming problem is
reduced to the problem of surface-surface intersections of these basic surfaces.
Based on the complexity of these intersections, we introduce a
novel classification of sweeps as either decomposable or non-decomposable. Further, 
we construct an {\em invariant} function  on the funnel which 
\eat{allows us to} 
efficiently separates decomposable and non-decomposable sweeps. 
Through a geometric theorem we also show intimate
connections between , local curvatures and
the inverse trajectory used in earlier works as an approach towards trimming.
In contrast to the inverse trajectory approach,  is robust
and is the key to a complete structural understanding, and 
\eat{allows} an efficient computation of both, the singular locus and 
the trim curves, which are central to a stable implementation.  
Several illustrative outputs of a pilot implementation are included.
 
\end{abstract}

\begin{keyword}
Sweeping, boundary representation, parametric curves and surfaces
\end{keyword}
\end{frontmatter}

\section{Introduction} \label{introSec}
This paper is motivated by the need for a robust implementation of solid sweeps in
solid modeling kernels. The solid sweep is of course, the envelope
surface of a solid which is swept in space by a family of rotations and translations.
The uses of sweeps are many, e.g., in the design of scrolls~\cite{scroll}, 
in CNC machining verification~\cite{completeSweep}, 
to detect collisions, and so on. 
See Appendix for an application of solid sweep in designing scrolls,   
where we describe a modeling attempt using an existing kernel and its limitations.
Constant radius blends can be considered as the partial envelope of a sphere 
moving along a specified path.  As with blends, it is expected 
that a deeper mathematical understanding of solid sweep will lead to its rapid 
deployment and use.


A robust implementation of solid sweep poses the following
requirements: (i) allow for input models specified in the
industry-standard brep format, (ii) output the sweep envelope in the brep
format, with effective evaluators,
and finally, (iii) perform body-check, i.e., a check on the orientability,
non-self-intersection, detection of singularities and so on.
Thus there are some ``local'' parts and some ``global'' parts to the
problem.

It is
generally recognized that the harder parts of the local theory is in the
smooth case,
i.e., when faces meet each other smoothly.
For in the non-smooth case, the added complexity in the 
local geometry of the sweep is exactly that of a curve moving in 3-space.
This
is of course well understood, and offered by many kernels as a basic surface
type. As far as we know, the global situation in the non-smooth case, i.e.,
the topological structure of edges and
vertices (i.e., the 1-cage) of the sweep has not been elucidated, but is
also generally assumed
to be simpler than the smooth case.
In fact, much of existing literature has focused on a smooth single-face
solid, as the key problem~\cite{jacobian, sede, trimming}.



In this paper, we focus on the smooth multi-face
solid. In Section~\ref{simpleSec}, we start with the mathematical
structure of the simple sweep (i.e., one without singularities and
self-intersections). By the calculus of curves of contact, we set up a
correspondence between the faces, edges and vertices of
the envelope with those of the swept solid. This sets up the brep structure
of the envelope. Next, we define the funnel as the parametrization space
of a face of the envelope and construct a parametrization. We further elucidate
the structure of the bounding edges/vertices of a face and provide several examples 
of simple sweeps from a pilot implementation.

In Section~\ref{simpleSISec}, we examine the trim structures. The funnel 
of Section~\ref{simpleSec} will remain the ambient parametrization of the faces. 
The correspondence will help us define the trim areas and trim curves which 
must be excised to form the correct envelope. We then define the function  
and use it to define elementary and singular trim curves.

In Section~\ref{decompSec}, we start with the decomposable sweep, i.e., one
which may be partitioned into
a suitable small collection of simple sweeps. The final envelope is obtained
by stable (transversal) boolean operations on this collection. We show that
the trim curves so obtained are elementary. We next define an invariant
 on the funnels, which is robustly and efficiently computable
and we show that  on (all) the funnels characterizes decomposability.
This is an important step in the robust implementation of sweeps.

In Section~\ref{thetaSec}, we prove some of the properties of  such as 
its invariance and show that it is the determinant of the transformation
connecting two 2-frames on the envelope, and is thus an easily computable 
function on the surface. We show that the 
curve on the funnel is also the singular locus for the envelope surface.
Via a geometric theorem, we also show that the function  matches
the one by~\cite{trimming}  for implicitly defined surfaces and using the so-called
inverse trajectory.

In Section~\ref{nonDecompSec}, we define the singular trim curve, i.e., where  may
hit zero. We show that there is a correspondence between singular trim
curves and the curves in the zero-locus of . We also show
that (i) singular trim curves make contact with the  curves, and
(ii) excision at the singular trim curves removes all singularities
of the envelope except at these points of contact. Furthermore, these 
points are easily and robustly computed.

In Section~\ref{conclusionSec} we summarize what has been achieved, viz., that the
decomposability and the zero-locus of  complement to give a complete
understanding of all trim curves. We also discuss some implementation issues
and extensions.

\noindent{\bf Previous work}

We now review existing related work. 
Perhaps the most elaborate proposal for the sweep surface  is the 
sweep envelope differential equations~\cite{sede} approach, where the authors 
(i) assume that surface  being swept is implicitly given by a function , and (ii) derive a 
differential equation whose solution 
is the envelope.  For any point  on the initial curve of contact, 
a Runge-Kutta marching yields a trajectory   such that (i) , 
and (ii) , the curve of contact at time .  
These trajectories presumably serve as the iso-parametric lines .  
Determining whether  is in the trimming set  is solved by using the inverse trajectory 
condition. This is implemented by using 
the second derivative of the function , where  is the inverse trajectory of point . 

On the global front, the building of the envelope  is 
done by selecting a collection of points on the initial curve of 
contact, developing trajectories, testing for membership in  and then 
using the points which pass to construct an approximation to the envelope. 
The drawbacks are clear. Typically, constructing an  which defines  is 
difficult. Furthermore, the choice of  seems to determine many computational
and parametric issues, which is undesirable. The inverse-trajectory check 
remains poorly conditioned, especially when the second derivative of 
the function  w.r.t.  is zero. The structure of the envelope is unknown where 
this derivative is zero.  
A global understanding of  and the nature of
the trim curves is missing.  

In~\cite{classifyPoints}, while classifying points for sweeping solids, the authors give a membership test for a point in the object space to belong inside, outside or on the boundary of the 
swept volume by using inverse trajectory of that point.  
A curve-solid intersection is required to be computed for each point membership query which is computationally expensive, especially when the intersection is non-transversal, as noted by the authors themselves.  Such high degree of computational complexity is prohibitive for a practical implementation.  

In~\cite{planarSwep} the authors work with 2D shapes and 2D motions and quantify singularities using inverse 
trajectories.  This work is based on the computational framework  described in~\cite{classifyPoints}  and involves computing intersections between 2D curves and 2D shapes. 
The authors remark that this work can be extended to the 3-dimensional case involving intersections between 3D curves and 3D solids.  This approach has the same drawback 
as~\cite{classifyPoints}, namely a high computational cost.

In trimming self-intersections in swept volumes~\cite{selfIntersections}, the authors detect self-intersections by computing approximate curves of contact at a few discrete time instances 
which are then checked for intersections.  Approximations are introduced at multiple levels, hence an accurate solution cannot be expected from this method.
\section{Mathematical structure of sweeps} \label{simpleSec}

In this section we formulate the boundary of the volume obtained by sweeping a solid  along 
a given trajectory .  

\subsection{Correspondence and brep structure of envelope}

We will use the boundary representation, also known as brep, which is a popular standard for 
representing a compact and oriented solid  by its boundary . The boundary  
separates the interior of  from the exterior of  and is represented using a set 
of \emph{faces}, \emph{edges} and \emph{vertices}.  See Figure~\ref{coneFig} for the brep of
a solid where different faces are colored differently.  Faces meet in edges and edges meet in vertices.  
The brep consists of two interconnected pieces of information, viz., 
the geometric and the topological.  The geometric information consists of the parametric description of 
the faces and edges while the topological information consists of orientation of the geometric entities and 
adjacency relations between them.


In this paper we consider solids whose boundary is formed by faces meeting smoothly.  In the case 
when the faces do not meet smoothly,  the added complexity in the 
local geometry of the sweep is exactly that of a curve moving in 3-space.
This is of course well understood, and offered by many kernels as a basic 
surface type.  The global geometry and topology for this case will be described in a later paper.

\begin{defn} \label{trajectoryDef}
A {\bf trajectory} in  is specified by a map 

where  is a closed interval of , SO(3)=\{X \mbox{ is a 3 3 real matrix} |X^t \cdot X = I, det(X)=1  \}.    The parameter  represents time.    
\end{defn}

We assume that  is of class  for some , i.e., partial derivatives of order up to  exist and are continuous.  

We make the following key assumption about .

\begin{assum} \label{genericAssum}
The tuple  is in a {\em general position}.
\end{assum}


\begin{defn}  \label{envlDef}
The {\bf action} of  (at time  in ) on  is given 
by .  
The {\bf swept volume}  is the union 
 and the {\bf envelope}  is defined as the 
boundary of the swept volume .  
\end{defn}

Clearly, for each point  of  there must be an  and a  
such that .  This sets up the following correspondence relation.


\begin{defn} \label{corrDef}
The {\bf correspondence}  is the set of tuples  
 
For , we set . 
Similarly, for , we define .
\end{defn}

We will denote the interior of a set  by .
It is clear that . Therefore, we have

\begin{lem} \label{intLem}
If , then for all ,  .
\end{lem}

\begin{figure}
 \centering
 \includegraphics[scale=0.5]{coneCaps}
 \caption{The envelope of a blended cone being swept along a helical trajectory with compounded rotation.}
 \label{coneFig}
\end{figure}


Thus, the points in interior of  do not contribute to  at all and 
.  This sets up the brep 
structure for .
In the sweep example shown in Figure~\ref{coneFig}, 
the correspondence  is illustrated via color coding, i.e.,   
for , the points  and  are shown in the same color.
The general position assumption on  can be formulated as the condition
that the induced brep topology of  remains invariant under a small
perturbation of .

\begin{lem} \label{preImageLem}
Assuming general position of , for any , there are at 
most three distinct tuples  for   which belong to .
\end{lem}
\noindent {\em Proof.} For distinct tuples , it is clear that , for otherwise .
Therefore  and  intersect 
at point .  By Assumption~\ref{genericAssum} 
this intersection is transversal.  Further, by the same assumption, at most  surfaces may intersect in a 
point. \hfill 

\begin{defn} \label{trajXDef}
For a point , define the {\bf trajectory of}  as the map  
given by  and the velocity  as 
.
\end{defn}

For a point , let  be the unit outward normal to  at .  Define the function 
 as 
 
Thus,  is the dot product of the velocity vector with the unit normal at the point .


Proposition~\ref{gLem} gives a necessary condition for a point  to 
contribute a point on  at time , namely, , 
and is a rewording in our notation of the statement in~\cite{sede} that 
{\em the candidate set is the union of the ingress, the egress and the 
grazing set of points}.

\begin{prop} \label{gLem}
For  and , either 
(i)  or 
(ii)  and , or 
(iii)  and .
\end{prop}
For proof, refer the Appendix.
\begin{defn} \label{cocDef}
For a fixed time instant , the set  is
referred to as the {\bf curve of contact} at  and denoted by . 
Observe that . The union of the 
curves of contact is referred to as the {\bf contact set} and denoted by , i.e., 
.
\end{defn}

In the sweep example in Figure~\ref{dumbbellFig}, the curve of contact at  is shown imprinted 
on the solid in red. The curves of contact are referred to as the 
{\em characteristic curves} in~\cite{peternell}.


\begin{defn} \label{projDef}
Define projections  
and  as:
.
\end{defn}

\begin{defn} \label{simpleDef}
A sweep  is said to be {\bf simple} if for all , 
. \end{defn} 

Note that, by Proposition~\ref{gLem}, for any sweep, we have 
. In a simple sweep, we require that .
In other words, every point on the contact-set appears on the envelope, and 
thus, no {\em trimming} of the contact-set is needed in order to obtain the envelope.

\begin{lem} \label{simpleLem}
For a simple sweep, for all ,  is a singleton set.
\end{lem}
\noindent {\em Proof.} We first show that for a simple sweep, for 
, .  Suppose that .
Clearly,  and . 
Hence .  Since  and 
 intersect transversally,  and 
.  It follows by Lemma~\ref{intLem} that 
 and  which
contradicts the fact that  is simple.

Now suppose that there are  tuples  for .  Since  is free from self-intersections 
it follows that  and  which is a contradiction to the fact that 
 is simple.
\hfill 

\subsection{Parametrizations}	\label{paramSec}
Now we describe parametrizations of the various entities
of the induced brep structure of . Here we restrict to the
case of the simple sweep. The more general case is derived from this. 
\eat{As mentioned before, the brep  consists
of faces, edges and vertices. These geometric entities give rise to
corresponding entities on  (see Figure~\ref{coneFig}).}

\subsubsection{Geometry of faces of }\label{keynotation}
Let  be a face of . In general,  gives rise to 
multiple faces of . Below we describe a natural parametrization of
these faces using the parametrization of the surface underlying the
face .

\begin{defn} \label{parSurfDef}
A {\bf smooth/regular parametric surface} in  is a smooth map 
such that at all    
 and  are linearly independent.  Here  and  are called 
the parameters of the surface.
\end{defn}

Let  be the surface underlying the face  of .

\begin{defn} \label{fDef}
Define the function  as 
.
\end{defn}
The domain of function  will be referred to as the parameter space.  Note that  is easily 
and robustly computed.
\begin{defn} \label{LFRDef}
For an interval , we define the following subsets of the parameter space

\end{defn}
The set  will be referred to as the {\bf funnel}.

\begin{figure}
 \centering
 \includegraphics[scale=0.7]{funnel}
 \caption{The funnel and the contact-set.}
 \label{funnelFig}
\end{figure}

By Assumption~\ref{genericAssum} about the general position of  it follows that for all , the gradient 
.  As a consequence,  is a smooth, orientable surface in the parameter space.

\begin{defn} \label{pcocDef}
The set  will be referred to as the {\bf p-curve of contact} at  and 
denoted by .
\end{defn}

We now define the sweep map from the parameter space to the object space.
\begin{defn} \label{sigmaDef}
The {\bf sweep map} is defined as follows.
 
\end{defn}
Note that,  is a smooth map,  and . 
Here and later, by a slight abuse of notation, ,  and  denote the 
appropriate parts of complete ,  and  respectively 
resulting from the face  whose underlying surface is .
The surface patches  and  will be referred to as the 
left and right end-caps respectively.

The funnel, the contact-set,  and  are shown schematically in Figure~\ref{funnelFig}.

The condition  can also be looked upon as the rank deficiency condition~\cite{jacobian} of the 
Jacobian  of the sweep map .  To make this precise, let

where , 
  and 
. Note that if  then 
 is the velocity, also denoted by .   
Observe that  regularity of  ensures 
that  has rank at least 2.  Further, it is easy to show that 
 is a non-zero scalar multiple of the determinant of .
Therefore, the condition  is precisely the rank deficiency 
condition of .

For a simple sweep, by Proposition~\ref{gLem}, Definition~\ref{simpleDef} and Definition~\ref{LFRDef} it 
follows that 
. 
The surface patches  and  can be obtained 
from  using Proposition~\ref{gLem} and Definition~\ref{LFRDef}.  The {\em trim curve}
 in parameter space for  is given by  and that for 
 is given by .  

We now come to the parametrization of .  The non-singularity of  makes 
 an effective parametrization space for .
Since time  is a central parameter of the sweep problem and is important in 
numerous applications, it is useful to have  as one of the parameters of .
For most non-trivial sweeps there is no closed form solution for the parametrization of 
the envelope and we address this problem using the procedural paradigm which is now 
standard in many kernels and is described in the Appendix.
In this approach, a set of evaluators are constructed for the 
curve/surface via numerical procedures  which converge to the solution up to the required tolerance.  
This has the advantage of being computationally efficient as well as accurate.

Clearly, the bounding edges of the multiple faces resulting from the
face  of , are generated by the bounding edges of .

\subsubsection{Geometry of edges of }
\begin{figure}
 \centering
 \includegraphics[scale=0.6]{wireFrCone}
 \caption{The edges of envelope for the sweep example shown in Figure~\ref{coneFig}.}
 \label{wireFrameFig}
\end{figure}


We now briefly describe the computation of edges of .  If  is 
composed of faces meeting smoothly, an edge  of  will, in general, give rise to a 
set of edges in .
We define the restriction of  to the edge  as follows.
\begin{defn}
For an edge , define .
\end{defn}

Let  be the intersection of faces  and  in  and let  denote 
the parameter of .  Since  and  meet smoothly at , at every point 
 of  there is a well-defined normal.  Hence we may define the following function 
on the parameter space .

\begin{defn} \label{feDef}
Define the function  as 
.
\end{defn}

Note that the function  is the restriction of the function  defined in Definition~\ref{fDef} to 
the parameter space curve  corresponding to the edge  so that  
where  is the surface underlying face .  The following Lemma gives a necessary condition for 
a point  to be on  at time .
\begin{lem} \label{feLem}
For  and , either (i)  and , or 
(ii)  and , or (iii) .
\end{lem}
\noindent {\em Proof.}  This follows from Prop.~\ref{gLem} and Definition~\ref{feDef}. 
\hfill 
Figure~\ref{wireFrameFig} shows the edges of the envelope for 
the sweep example shown in Figure~\ref{coneFig}.  The correspondence for one of the edges of the envelope is also marked.


Let  denote the funnel corresponding to the contact set generated by face .
The edge in parameter space which bounds  is given by 
 which we will denote by 
.  Note that  is smooth if 
 at all points
in .


\subsubsection{Geometry of vertices of }

A vertex  on  will, in general, give rise to a set of vertices on .  We further 
restrict the correspondence  to  as .  As  is smooth, there is a well-defined normal at .  Hence we may define the function 
 as .  If  is on the boundary of a face ,  will 
have a set of coordinates in the parameter space of the surface  underlying the face , say , 
so that .  It is easy to see that if  and  then 
either (i)  and , or (ii)  and , or (iii) .

\subsection{Examples of simple sweeps}

Three examples of simple sweeps are shown in Figures~\ref{dumbbellFig}, \ref{ellipBottleFig} 
and \ref{sphereSFig} which 
were generated using a pilot implementation of our algorithm in ACIS 3D Modeler \cite{acis}.
A curve of contact at initial time is shown imprinted on the solid in Figure~\ref{dumbbellFig}.

\begin{figure}
 \centering
 \includegraphics[scale=0.4]{dumbbell1}
 \caption{The envelope(without end-caps) of a dumbbell undergoing translation along -axis and undergoing rotation about -axis.}
 \label{dumbbellFig}
\end{figure}

\begin{figure}
 \centering
 \includegraphics[scale=0.5]{ellipBottle}
 \caption{The envelope(without end-caps) of an elliptical cylinder undergoing a screw motion while rotating about 
its own axis.}
 \label{ellipBottleFig}
\end{figure}
\begin{figure}
 \centering
 \includegraphics[scale=0.55]{sphereS}
 \caption{The envelope(without end-caps) of a sphere sweeping along an 'S' shaped trajectory while rotating 
about -axis}
 \label{sphereSFig}
\end{figure}
\section{The trim structures}	\label{simpleSISec}

Unlike in a simple sweep, all points of  may not belong to the envelope.  We now define the 
subset of  which needs to be excised in order to obtain .
\begin{defn} \label{trimSetDef}
The {\bf trim set} is defined as 
 
\end{defn}

\begin{lem} \label{trimSetLem}
The set  is open in .
\end{lem}
\noindent {\em Proof.}  Consider a point .  Then  for some .  
Hence, there exists an open ball of non-zero radius  
centered at , denote it by , which is itself contained in .  
Let .
Then,  and   is open in . Hence  is open in .
\hfill 

In general, the trim set will span several parts of  corresponding 
to different faces of . For the ease of notation
and presentation, in the rest of this paper, we will analyse the corresponding
trim structures on the {\em funnel} of a fixed face  of .
Thanks to the natural parametrizations (cf. subsection~\ref{paramSec}), 
the migration of these trim structures across different funnels is an easy
implementation detail. In view of this, we carry forward the notation 
developed in subsection~\ref{keynotation} through the rest of this paper.

\begin{defn} \label{trimSetDef} 
The pre-image of  on the funnel under the map  will be 
referred to as the {\bf p-trim set}, denoted by , i.e., 
.
\end{defn}
An immediate corollary of Lemma~\ref{trimSetLem} is:
 is open in .

One can also define similar parametric trim areas on the left and right
caps (cf.  and  from Definition~\ref{LFRDef}) and their counterparts in the object
space. However, for want of space, we assume here that these trim structures are empty. Our analysis can
be extended to also cover the non-empty case.


\begin{defn} \label{trimCurveDef}
The boundary of  will be referred to as the {\bf trim curves} and denoted by . Here  denotes the closure of 
 in . Similarly,
the boundary of the closure  of  in  
will be referred to as the {\bf p-trim curves} and denoted by .
\end{defn}

Note that ,
 and 
.
Therefore the problem of excising the trim set is 
reduced to the problem of computing the trim curves. Further, 
this computation is eventually reduced to {\em guided} parametric 
surface-surface intersections via the parametrization of  
described in subsection~\ref{paramSec}.

For each point  there is a finite set of points  such that  
for all  (cf. Lemma~\ref{preImageLem}). 
Figure~\ref{ptrimCurveAFig} schematically illustrates p-trim curves on .  For every 
point  in the red portion of , there is a point  in the green portion of  
such that .


\begin{figure}
 \centering
 \includegraphics[scale=0.65]{ptrimCurveTypesA}
 \caption{Elementary and singular p-trim curves.}
 \label{ptrimCurveAFig}
\end{figure}

We extend the correspondence of Definition~\ref{corrDef} 
to  as below. Abusing notation, henceforth,  will denote this correspondence. 
\eat{
We extend the correspondence defined in Definition~\ref{corrDef} to . By abuse of notation, we denote this extended relation again by 
 and the following definition of  will be used in  the rest of the paper.
}
\begin{defn} 
Let 
.
As expected, we define  
and  as:
.
Further, as before,
,\!
.
\end{defn}

A crucial observation is that, unlike the earlier correspondence, {\em }.

\begin{defn} \label{lDef}
For , let . Let . Define the function 
 as follows.

Further, we define .
\end{defn}

For ,  is the set of all time instances  (except ) such 
that some point of  coincides with . 
Further, the function  gives the `smallest' 
time  such that some point of  coincides with .

\begin{lem} \label{trimCLem}
Let . Then   iff  contains an interval, and  iff  is a discrete set
of cardinality either two or three.
\end{lem}
\noindent {\em Proof.} Suppose first that .  Let .  Then  and 
 for some .  Let  be an open ball of radius  centered at 
 contained in .  Assume without loss of generality that  and . 
By continuity of the trajectory  it follows that given  there exists  such that 
.  Hence,  for all 
.  In other words, .

Conversely, suppose that  contains an interval , i.e.,  for all 
.  By Assumption~\ref{genericAssum} about the general position of  it 
follows that  for some , i.e.,  and .
We have shown that for ,  iff  contains an interval.  Hence,  is discrete iff . 

As , 
by Lemma~\ref{preImageLem}, it follows that 
at all but finitely many points ,  
 is of cardinality  and at remaining points it is of cardinality . 
\hfill 


We classify trim curves as follows.
\begin{defn} \label{trimCClassifyDef}
A curve  of  is said to be {\bf elementary} if  there exists  such that for all 
, . 
It is said to be {\bf singular} if .
\end{defn}

Figures~\ref{ptrimCurveAFig}(a) and~\ref{ptrimCurveAFig}(b) schematically illustrate elementary and singular p-trim curves on  respectively.  
Further observe that,  in case (a) and 0 in case (b).

Before proceeding further, we introduce the following notation: for 
, .

\begin{lem} \label{transInterLem}
All but finitely many points of elementary trim curves lie on the 
transversal intersections of two surface 
patches  and the remaining points lie on the transversal 
intersection of three surface patches  
where, for ,  are subintervals.
\end{lem}
\noindent {\em Proof.}  Suppose that all curves of  are elementary, i.e.,  
such that for all , .
By Lemma~\ref{trimCLem}, 
all but finitely many points  have two points  and 
 in  such that 
.  Let  
and  .  Then 
.  From Section~\ref{geomThetaSec} we know that 
 and  are tangential to  and  
respectively at .  By Assumption~\ref{genericAssum} about general position of ,  and 
 intersect transversally at .  Hence,  and  intersect 
transversally at .  

 At most finitely many points  have three points  and  in  such 
that .  By an argument similar to above, it can be shown that  lies on the transversal 
intersection of three surface patches  for  corresponding to appropriate 
subintervals .
\hfill 


Figure~\ref{toolFig} shows an example in which a capsule is swept along a helical path while rotating about 
-axis.  The trim curves are elementary.


\begin{figure}
 \centering
 \includegraphics[scale=0.5]{toolHole1}
 \includegraphics[scale=0.5]{toolHole2}
 \caption{(a) The contact set of a capsule moving along a helix while rotating 
about -axis.(b) The contact set restricted to interval  with the trim set excised.}
 \label{toolFig}
\end{figure}

\section{Decomposable sweeps} \label{decompSec}

We now consider sweeps, which though not simple, can be divided into simple sweeps by 
partitioning the sweep interval so that the trim curves can be obtained by transversal intersections 
of the contact sets of the resulting simple sweeps.
Given an interval , we call a partition  of  
into consecutive intervals  to be of width  if .
\begin{defn} \label{decompDef}
We say that the sweep  is {\bf decomposable}
if there exists  such that for all partitions  of  of width , 
each sweep  is simple for  . A sweep
which is not decomposable is called {\bf non-decomposable}.
\end{defn}
Figure~\ref{decompNonDecompFig} schematically illustrates the difference between decomposable 
and non-decomposable sweeps.  The example shown in Figure~\ref{toolFig} is of a decomposable sweep 
in which partitioning the sweep interval  into 2 equal halves will result in 2 simple sweeps. 
\begin{figure}
 \centering
 \includegraphics[scale=0.55]{decompNonDecomp}
 \caption{Contact-sets of decomposable and non-decomposable sweeps.}
 \label{decompNonDecompFig}
\end{figure}

\begin{prop} \label{decompLem}
The sweep  is decomposable iff . Further, if  
then all the p-trim curves are elementary.
\end{prop}
\noindent {\em Proof.} Suppose first that .  Let  be a partition of  of width 
.  We show that  is simple for . 
Let  and  be the envelope and the contact set for  respectively.
\eat{
and  be the correspondence for  ,i.e., 
. Further, we set .}
By Proposition~\ref{gLem}, (modulo end-caps), . 
It needs to be shown that .  
Suppose not.  Let  such that 
 for some .  Then, , i.e.,   for some .  Let  for .  It follows that ,  leading to a contradiction.
Hence,  is decomposable.

Suppose now that  is decomposable with width-parameter  
(cf. Definition~\ref{decompDef}).
Consider a point  and 
let .  
Let  and . Further,
let  and  be the envelope and contact-set for the
sweeps  respectively. Observe that 
 for .
Let .
As  is decomposable with width-parameter ,
both  and  are simple, and hence,
 for . Therefore, 
 belongs to   and .
By Lemma~\ref{simpleLem}, 
 and  are both singleton sets.  
Further,  for 
.  Hence, .  Since for all ,
, we conclude that .

Suppose that .  Since  for all 
 
if follows that all the p-trim curves are elementary.
\hfill 


The above proposition provides a {\em natural} test for decomposability. Further, coupled
with Lemma~\ref{transInterLem}, for a decomposable sweep, the problem of
excising the trim set can be reduced to transversal intersections. However,
note that, the very definition of  is {\em post-facto} as it 
relies on the trim structures.
Besides, it is the infimum value of the 
not necessarily continuous function  and is difficult to compute. Thus, 
the above test of decomposability is not effective.

One of the key contributions of this paper is a novel geometric 
`invariant' function on the funnel which is computed in closed form and 
serves the following objectives.
\begin{enumerate}
\item Quick/efficient and simple detection of decomposability of sweeps, which occur most often in practice.
\item Generation of trim curves for non-decomposable sweeps.
\item Quantification and detection of singularities on the envelope.
\end{enumerate}

For a point , let . 
Recall from subsection~\ref{paramSec} that,  is of rank 2.
As ,  are linearly dependent. 
Recall that  is the velocity of the point  at time  (cf. subsection~\ref{paramSec}).
As  is regular, the set  forms a basis for the tangent space to .
Therefore, we must have  where  and  are well-defined (unique) on the funnel and are themselves continuous functions on the funnel.


\begin{defn} \label{thetaDef}
The function  is defined as follows.

where  and  denote partial derivatives of the function  w.r.t.  and  respectively at , and  and  are as defined before.
\end{defn}

Note that, unlike ,  is easily and robustly computable continuous 
function on the funnel.
Now we are ready to state one of the main theorems of this paper.

\begin{thm} \label{thetaDecompLem}
If for all , , then the sweep is decomposable.  
Further, 
if there exists  such that , then the sweep is 
non-decomposable.
\end{thm}
The proof is given in Section~\ref{proofSec} which highlights
many other surprisingly strong properties of the function .

\begin{defn}
The function  partitions the funnel  into three sets, viz. 
(i) , 
(ii)  and (iii) .  
Further, we define ,  and 
. 
\end{defn}

Figure~\ref{lsiRegionFig} schematically illustrates the sets  and  on the funnel 
and sets  and .

\begin{figure}
 \centering
 \includegraphics[scale=0.7]{lsiRegion}
 \caption{The shaded region on  and  corresponds to  and  respectively.  A 
curve of contact is shown in red.}
 \label{lsiRegionFig}
\end{figure}

Note that, for  in general position, 
either  is a non-empty open set or . 
Whence, the above theorem provides an efficient `open'
test for decomposability, namely, a sweep 
is decomposable iff the open set  is empty. Most kernels
will have an effective procedure for such a test provided  is
effectively computable.

\section{Properties of the invariant }  \label{thetaSec}

In this section we prove some key properties of , namely, its 
invariance under the re-parametrization of the surface being swept and its relation with the notion of 
inverse trajectory used in earlier works. Finally, we use these properties along
with Proposition~\ref{decompLem}, to
prove Theorem~\ref{thetaDecompLem}.


\subsection{Invariance of } \label{thetaInvarSec}
We show that the function  is invariant of the parametrization of  and hence, intrinsic to the sweep. 

\begin{thm} \label{thetaInvarThm}
If  is a re-parametrization of the surface  so that , and , then .
\end{thm}
\noindent {\em Proof.} 
Suppose as before that the boundary  is specified by the parametrized surface .  Let  be a re-parametrization map of  and 
.  Since  is a diffeomorphism,  is an isomorphism at every point in the entire domain of .  Let .  
For convenience of expression, we extend  to define it on the parameter space of the sweep map  so that .  Hence the re-parametrized 
sweep map (for ) is simply .  Recall that ,
 where  is the unit outward normal to  at 
 the point . It is easy to check that  
 can also be expressed as , where 
 is the intrinsic Gauss map,  being the unit sphere  
and  stands for the usual composition of functions.
Thus, 


Similarly, computing with the re-parametrization , and using
the fact that , we have .
\eat{

}
Differentiating w.r.t.  and  we get

where  is the Jacobian of the map .
\eat{. } 

Observe that, from Eq.~\ref{thetaEq}, for  and
, 
where  spans the null-space of  for .  In order to compute  for the re-parametrized sweep we see that  and . 
Now using ,
we get that 

This proves the theorem.
\hfill 

An important corollary of the above theorem is that the function 
on the funnel is a pull-back of an intrinsic 
function, say , on the abstract smooth manifold
.
More precisely, for  with ,
define . Then  remains invariant
under a re-parametrization. Observe that, unlike , in general,
 is not a smooth manifold.


\subsection{Geometric meaning of } \label{geomThetaSec}
For a smooth point  of , let  denote the tangent
space to  at .

We show that the function  arises out of the relation between two 2-frames on . Let  be such that  is a smooth
point of . We first compute a natural 2-frame  in . 
Note that,  being the zero level-set of the function , .  
We set  and note that . It is easy to see that 
 is tangent to the p-curve-of-contact .  
Let .  
Here  is the cross-product in . Clearly, the set  forms a basis of  if 
.  Since , if  then  and  serves as a basis for .
Figure~\ref{funnelFig} illustrates the basis  schematically. 

The set  and can be expressed in terms of 
 as follows

Note that,

Clearly, if  then  is a positive scalar multiple of .  
Again, if , expressing  in terms of 
 we see that .  

The above relation between  and 
 shows that 
if , then for ,  
and  are identitical 
(as subspaces of ), i.e., 
 makes tangential contact with  at .


\subsection{Non-singularity of }  \label{thetaNonSingSec}

We give a sweep example which will demonstrate the non-singularity of the function .  We show 
that on the set , .
Consider a sphere parametrized as ,   
swept along a curvilinear trajectory given by , , .
The unit outward normal at  at time  is given by  and velocity is given by 
.  The envelope function is .  The 
funnel  is given by (i) ,  and (ii) , .  
Hence,  and  can serve as local parameters of .  In component (ii) of the funnel, we see that 
 , hence we will only 
consider component (i).  On ,  where
 and , whence, .  The set 
 is given by , .  
On ,  
and .   

An important consequence of non-singularity of  is that its zero set, i.e.,  
can be computed robustly and easily.

\subsection{Detecting singularities on the envelope} \label{singSec}

Now we characterize the cusp-singular points of . Geometrically, these
are precisely the points where  intersects itself non-transversally.
Note that, the transversal singularities of  are addressed through
decomposability.
We consider the following restriction of  to the funnel:
.
Note that .

\begin{defn} \label{singDef}
The set  is said to have a {\bf cusp-singularity} at a point  if 
 fails to be an immersion at . 
\end{defn}
A basic result about immersion (see \cite{diffTop}) implies that 
if  is an immersion at a point , then there is a neighborhood  of  such
that  is a local diffeomorphism from  onto
its image.

\begin{lem} \label{singLem}
Let  and . The point   is a 
cusp-singularity iff .
\end{lem}
\noindent {\em Proof.}  From subsection~\ref{geomThetaSec},  is a positive multiple of 
the determinant relating frames  and 
 at  .  Since the set  
is always linearly independent, it follows that  is linearly 
dependent iff  fails to be an immersion at  iff .
\hfill 

In other words, the set  is the set of cusp-singular points in .

\subsection{Relation with inverse trajectory}  \label{inverseTrajSec}

We now show the relation of the function  with inverse 
trajectory~\cite{trimming, classifyPoints} used in earlier works.
Given a trajectory  and a fixed point  in object-space, 
the inverse trajectory of  is
the set of points in the object-space which get mapped to  at some time instant by , i.e. 
.  

\begin{defn} \label{invTrajDef}
Given a trajectory , the {\bf inverse trajectory}  is defined as the map  given by .  
Thus, for a fixed point , the inverse trajectory of  is the map  
given by . 
\end{defn}
The range of  is .  We list some of the facts about 
 in the Appendix  which will be used in proving Theorem~\ref{lambdaLem}.

For the inverse trajectory  of a point , let  be 
the projection of  on .  Let  be the signed distance 
of  from . If the point  is in ,  
(the exterior of ) or on the surface , then  is negative, positive 
or zero respectively.  Then we have , where  is the 
projection of  on  along the unit outward pointing normal 
 to  at .  This is illustrated in Figure~\ref{type2LSIFig}.
Thus the following relation holds for .


\begin{figure}
 \centering
 \includegraphics[scale=0.7]{type2LSI}
 \caption{The inverse trajectory of  intersects .}
 \label{type2LSIFig}
\end{figure}

\begin{thm} \label{lambdaLem}
For , 

 where  is the normal curvature of  at  
along velocity ,  is the unit outward normal to  at  and 
.
\end{thm}
See Appendix for the proof.
\eat{
\noindent {\em Proof.}
 Differentiating Eq.~\ref{lambdaEq} with respect to time and denoting derivative w.r.t.  by , we get

At , .  Since , it follows from Eq.~B.5 that .  It is easy to verify that .  Hence, 

From Eq.~\ref{ddotLambdaEq} and Eq.~B.7 it follows that

Since   for all  in some neighbourhood  of , we have that .  
Hence .  
Hence  = .  
Here  is the differential of the Gauss map, i.e. the curvature tensor of 
 at point .  Using this in Eq.~\ref{ddotLambdaTNotEq} and the fact that  ,  we get

Recalling definition of  from Eq.~\ref{thetaEq}

Here  and  
where  is the shape operator (differential of the Gauss map) of  at .  
Also,  and .  Assume without loss of generality that  
and , hence ,  and . Using Eq.~B.3 and the fact that   we get

From Eqs.~\ref{lsi2Eq} and~\ref{lsiRelationEq} and the fact that  we get 
.
\hfill 
}


From Theorem~\ref{lambdaLem}  it is clear that the function  is
intimately connected with the curvature of the solid and that of the trajectory. 
It is easy to see that the function  is identical to the function 
 defined in~\cite{trimming} for implicitly defined solids, albeit, is invariant of the function defining the solid 
as well as the parametrization of the same.

\subsection{Proof of Theorem~\ref{thetaDecompLem}}  \label{proofSec}
\noindent {\em Proof.}  Suppose that for all , .  
For , let  denote the -coordinate of .  
Consider the set of points 
 and .
By the general position assumption,  is a collection of smooth curves in .
For , let  denote the unique point in  such that  and 
.
Further, we define . 
Let .
Consider two cases as follows:

{\bf Case (i)}: , i.e., there exists a sequence  in a curve  of  such that
. Hence there exists  
(closure of )  which is a limit point of .  Since  and  is free from self-intersections, 
we have that . 
Hence, for a small neighborhood  of  in ,
we may parametrize the smooth curve 
 by a map  so that  and, for , 
 and .  
Let .  Note that . Now, 


Hence, 

Since , 
the map  fails to be an immersion at  
and by Lemma~\ref{singLem} we get that , which is a contradiction to the hypothesis.

{\bf Case (ii)}: .
Let  be a partition of  of width .  Let  and  
denote the funnel and the contact set corresponding to subinterval .  Then it is clear that for each ,
 is a diffeomorphism, i.e., for each ,
 for all , . 
We show that the subproblems  are simple for all .  Suppose not, i.e., 
for some , there exists  such that  for some 
.  Hence the trim set  is not empty.  By Lemma~\ref{transInterLem}, for all but finitely 
many points in 
  there are two points  such 
that .  If  and  then 
it follows that  leading to contradiction.
Hence, the subproblems  are simple for all .  
It follows that  is decomposable with width-parameter .


Hence we have proved that if for all ,  then the sweep is decomposable.


Suppose now that there exists  such that .  Let .  
Recall the definition of the function  from Equation~\ref{lambdaEq} and relation 
 from Theorem~\ref{lambdaLem}.  Clearly, if ,  
then  is a local maxima of the function  and the inverse trajectory of  intersects 
.
So, there exists  such that for all , 
there exists  such that .  
Hence, the interval .  Thus  and hence .
By Proposition~\ref{decompLem}, the sweep is non-decomposable.
\hfill 


\section{Trimming non-decomposable sweeps} \label{nonDecompSec}

\begin{figure}
 \centering
 \includegraphics[scale=0.4]{cocSph}
 \includegraphics[scale=0.4]{trimCSph}
 \caption{Example of a non-decomposable sweep: a sphere being swept along a parabola (a) Curves of contact at a few time instances (b) The curve  is shown in red and trim curve is shown in blue.}
 \label{nonDecompSphereFig}
\end{figure}

\begin{figure}
 \centering
 \includegraphics[scale=0.45]{cocEllip}
 \includegraphics[scale=0.45]{trimCEllip}
  \caption{Example of a non-decomposable sweep: an ellipsoid being swept along a circular arc (a) Curves of contact at a few time instances (b) The curve  is shown in red and trim curve is shown in blue.}
 \label{nonDecompEllipseFig}
\end{figure}

\begin{figure}
 \centering
 \includegraphics[scale=0.65]{ptrimCurveTypes}
 \caption{The p-trim curves for decomposable and non-decomposable sweeps shown on .  Here, 
. The point  is a singular trim point.}
 \label{ptrimCurveFig}
\end{figure}

We recall from Section~\ref{simpleSISec}, the classification
of the curves of  as being elementary or singular. In this
section we look at singular p-trim curves, i.e., a curve  of 
where .
Figure~\ref{ptrimCurveFig}(b) schematically illustrates singular p-trim curves.
Figures~\ref{nonDecompSphereFig} and~\ref{nonDecompEllipseFig} show two 
examples of non-decomposable sweeps and the associated singular trim curves.
In Figure~\ref{nonDecompSphereFig} a sphere undergoes curvilinear motion 
along a parabola and in Figure~\ref{nonDecompEllipseFig} an ellipsoid undergoes 
curvilinear motion along a circular arc.  In Figures~\ref{nonDecompSphereFig}(a) 
and~\ref{nonDecompEllipseFig}(a), curves of 
contact at a few time instances are shown. The portions of  where  
and  on  are shown in black and pink respectively.  
By Proposition~\ref{thetaNegLem}, the points where  is negative do not lie on . 
In Figures~\ref{nonDecompSphereFig}(b) and~\ref{nonDecompEllipseFig}(b) such points are excised,  
the curve  is shown in red and the trim curve  is 
shown in blue.  Note that  and  make contact, which 
they must, as we explain in this 
section.  Figure~\ref{cocFig} schematically illustrates the interaction between curves of contact in 
non-decomposable sweeps.  

\begin{prop} \label{singTrimCLem}
If  is a singular p-trim curve and  is a limit-point of
 such that , then .
\end{prop}
\noindent {\em Proof.}  
The proof is similar to Case (i) of proof for Theorem~\ref{thetaDecompLem}.

\eat{For , let  denote the -coordinate of .  By 
Lemma~\ref{trimCLem} it follows that for each  there exists  
such that .  Since  
and  is free from self-intersections, we have that . 
Hence, for a small neighborhood  of  in ,
assuming  is a smooth curve,  we may parametrize 
 by a map  so that  and 
 for .  
Let .  Then, 


Hence, 

Since , 
the map  fails to be an immersion at  
and by Lemma~\ref{singLem} we conclude that .
\hfill .
}

\begin{defn} \label{singTrimPtDef}
A limit point  of a singular p-trim curve  such that  will be called a 
{\bf singular trim point}.
\end{defn}

In Figure~\ref{ptrimCurveFig}(b) a singular trim point  is shown on .


\begin{figure}
 \centering
  \includegraphics[scale=0.45]{coc}
  \caption{A schematic illustrating the interaction between curves of contact in non-decomposable sweeps.}
 \label{cocFig}
\end{figure}

\begin{prop} \label{thetaNegLem}
If  such that  then .
\end{prop}
\noindent {\em Proof.}
Let .
Recall the definition of the function  from Equation~\ref{lambdaEq} and relation 
 from Theorem~\ref{lambdaLem}.  Clearly, if ,  
then  is a local maxima of the function  and the inverse trajectory of  intersects 
 and .  Hence, if  then .
\hfill 

The above two propositions link the curves of  to the curves of .
We see that every curve of  lies inside a curve of  and 
every curve  of  has a curve  of  
which makes contact with it.  We have already seen that  is a collection of curves 
on which  is non-zero.  Thus, the computation of  in modern 
kernels is straightforward. The task before us is now to locate the points of 
. This is enabled by the following function.

\begin{defn}  \label{omegaDef}
Let  be a parametrization of a curve  of .  
Let  and   
at , i.e., .   Define the function 
 as follows.

where  is the cross-product in .
\end{defn}
 Here,  is a measure of the oriented angle between the tangent at  to  
and the kernel  (line) of the Jacobian  restricted to the tangent space .

\begin{prop} \label{omegaProp}
Every singular p-trim curve  makes contact with a curve  of  
so that if
 is a singular trim point of  then .  Furthermore, at 
such points,  where  refers to the derivative of  
along the curve .
\end{prop}
\noindent {\em Proof.} We know from Proposition~\ref{thetaNegLem} that .  
Since  and  form the boundaries of  and  respectively, 
 and a singular p-trim curve  of  meet tangentially at the singular trim point.
Further, by an argument similar to the case (i) of Theorem~\ref{thetaDecompLem}, it can be seen that 
at a singular trim point , 
 is the null-space of the Jacobian .  Since 
, .  
Since the function  measures the oriented angle between  and , 
it follows that .

The derivative  at singular trim points for non-decomposable sweeps shown in 
Figure~\ref{nonDecompSphereFig} and Figure~\ref{nonDecompEllipseFig}.
\hfill 

\eat{
\begin{figure}
 \centering
 \includegraphics[scale=0.5]{plot}
 \caption{The plot of the function  for the sweep example shown in 
Figure~\ref{nonDecompSphereFig}.}
 \label{varphiPlotFig}
\end{figure}
}

Proposition~\ref{omegaProp} confirms that for every singular p-trim curve, we may use the 
function  to locate a singular trim point  in a computationally robust manner.  Thus, 
via  and  we may access every component of .

\begin{prop}
In the generic situation, (i) the singular p-trim curve  has a finite set of singular trim points.  Each of these 
points lie on a curve of . (ii) For all but finitely many non-singular points , the image 
 lies on the transversal intersection of two surface patches  
and the remaining non-singular points lie on intersection of three surface patches 
where each  corresponds to a suitable 
subinterval . 
\end{prop}
\noindent {\em Proof.} 
It follows from Proposition~\ref{singTrimCLem} that the singular trim points lie on . Since at a 
non-singular trim point , , the proof for (ii) is identical to the proof for Lemma~\ref{transInterLem} about elementary trim curves.
\hfill 

Note that the computation of  above is transversal except at the known point , i.e., where . The problem then reduces to a surface-surface intersection which is transversal except at a known point. This information is usually enough for most kernels to compute  robustly.

\eat{
We now describe the tracing of a singular p-trim curve  once a point  in  has been located 
where  is zero.  
Since  and  meet tangentially at , starting 
at  we take small steps in direction  and  
to obtain points  and  respectively which are fed to a Newton-Raphson solver which 
returns points  and  such that ,  and 
.  Let .  
Here,  denotes the -coordinate of .
The point  is on the trim curve and the points  and  are on the p-trim curve . 
Since points  and  are non-singular, these can be fed as starting points to any of the known 
surface-surface intersection algorithms to compute the trim curve. 
}

\begin{figure}
 \centering
 \includegraphics[scale=0.5]{nested}
 \caption{A singular p-trim curve nested inside an elementary p-trim curve}\label{nestedFig}
\end{figure}

Figure~\ref{nestedFig} schematically illustrates a scenario in which a singular p-trim curve is nested 
inside an elementary p-trim curve.  Note that the sweep is non-decomposable and this will be detected 
by the presence of points on  where  is negative.  Further, the region bounded by the 
singular p-trim curve needs to be excised before a surface-surface intersection algorithm can trace the 
elementary trim curves since no neighborhood of  (where  is zero) can be parametrized. 
Our analysis will first successfully identify and excise the region bound by the singular p-trim curve.  
After parametrizing the remaining part, 
the task of excising the regions bound by elementary p-trim curves can be handled by existing kernels.

\section{Discussion} \label{conclusionSec}

This paper develops a mathematical framework for the implementation of the
``generic'' solid sweep in modern solid modelling kernels. This is done via
a complete understanding of singularities and of self-intersections within
the envelope and the notion of decomposability. This in turn is done
through the
important invariant  by which all trim-curves are either stable
surface-surface intersections or are caught by .

We now detail certain implementation issues. Firstly, the use of funnel as
the parametrization space and the so called ``procedural'' framework is now
standard, see e.g., the ACIS kernel. Secondly, the non-generic case in the
sweep, as in blends or surface-surface intersections, will need careful
programming and convergence with existing kernel methods for handling
degeneracy. Next, while we have not tackled the case when the trim curves
intersect the left/right caps, that analysis is not difficult and we skip
it for want of space. Finally, the non-smooth sweep is a step away. The
local geometry is already available. The trim curves and other
combinatorial/topological properties of the smooth and non-smooth case are tackled in
a later paper.

Mathematically, our framework may also extend to more complicated cases
where the curves of contact are not simple. This calls for a more
Morse-theoretic analysis which should yield rich structural insights. The
invariant  is surprisingly strong and needs to be studied
further.
\eat{
In this paper we give a complete characterization of the trim curves using 
decomposability and the zero-locus of the function .  
While the trim curves in decomposable sweeps can be computed by the existing 
surface-surface intersection algorithms, in non-decomposable sweeps 
this approach fails in the neighborhood of singular trim points.  We address this 
problem via the zero-locus of the function .
We give examples of simple, decomposable and non-decomposable to illustrate this. 

This work can be extended by computing the complete brep for the envelope which 
includes orienting the edges and faces of , computing adjacency relations between them, 
and so on.  The case when faces of  do not meet smoothly needs to be addressed.  
Sweeps in which the number of components of curve of contact changes with time also poses an 
interesting problem.
}

\appendix

\section{Application of solid sweep in design of conveyor screws}

\begin{figure}
 \centering
 \includegraphics[scale=0.6]{screw}
 \caption{A conveyor screw for translating cylindrical bottles.}
 \label{screwFig}
\end{figure}

\begin{figure}
 \centering
 \includegraphics[angle=0, scale=0.25]{discreteScrew-lr}
 \caption{A conveyor screw designed via discrete approach using boolean operations.}
 \label{discreteScrewFig}
\end{figure}

In this section we briefly describe an application of solid sweep in the packaging industry 
where complex needs for handling products arise.  A few example scenarios are, orienting 
the products precisely as they pass along the assembly line, separating one stream of products 
into two streams or combing two streams into one, inverting the product as it passes along the line, 
introducing exact spacing between consecutive products, and so on. 
This is often achieved by a conveyor screw which rotates about its own axis and hence 
propels the product ahead which is sitting in its groove.
The surface of this screw is specifically designed for moving the required product 
along the required path.  See~\cite{scroll} for a video 
of conveyor screws which group a set of products together and introduce precise time lag between 
two consecutive products. 

In order to design such a screw for the required object and the required motion profile, 
the rotation of the screw is compounded into the desired motion profile.  The object is then 
swept along the resultant trajectory and the swept volume so obtained is subtracted from a cylinder to 
obtain the conveyor screw.  Figure~\ref{screwFig} shows the surface of a screw designed 
to translate a cylinder.  The conventional method of designing such screws involves sampling 
the trajectory at a finite number of positions, and taking the union of the object positioned at all these 
positions.  The resultant ``discrete" swept volume is then subtracted from the cylinder to obtain 
an approximate screw.  This is shown in Figure~\ref{discreteScrewFig}.  As expected, this approach 
produces a large number of sliver faces and the brep structure of the resulting solid has a high degree 
of complexity.  Further, the solution is neither accurate nor smooth.


\section{Proof for Proposition~8}	\label{gProofSec}
Recall the statement of Proposition~8 that for  and , 
either (i)  and , or 
(ii)  and  or (iii) .


\noindent {\em Proof.} 
Define  and  as 
\eat{  and }
  and  
 respectively.
We define the following objects in  where the fourth dimension is time.
Let  and
.
Note that  is a four dimensional topological manifold and  is a three dimensional
submanifold of . Further, a point  lies in  if  and .  
Further, if ,  
forms the boundary of  where Define the 
projection  is defined as  and 
the projection  is defined as .
Clearly, for a point 
, if  then .  Hence a necessary 
condition for  to be in  is that the line  should be tangent to 
which is a three dimensional manifold which is smooth everywhere except at 
 and at . 
For , \eat{  is spanned by .  Hence }
the outward normal to  at  is given by  and  
the outward normal to  at  is given by .
We now compute the outward normal to  at .
The manifold  is diffeomorphic to , i.e., the cross product of  
which is a 2-dimensional manifold and  which is a 1-dimensional manifold, with the 
diffeomorphism given by , .  
Hence, if  spans   and  spans  
then the tangent space of  at  is spanned by 
 and  is spanned by \\
 .
Hence, the outward normal to  at  is 
.
Consider now three cases as follows.

Case (i): .  At any point  there is a cone of outward normals 
given by  
where  and .  So if the line  is tangent to  at  then 

for some  where  and .  Solving the above for  we get .
Hence .

Case (ii): .  Proof is similar to case (i).

Case (iii): . If the line  is 
tangent to  at  there exist  not all zero such that 

It follows that .  
In other words, .
\hfill 


\section{Some useful facts about the inverse trajectory} \label{invTrajSec}

Recall the inverse trajectory of a fixed point   as .
We will denote the trajectory of  by , .  We now note a few useful facts about . 
We assume without loss of generality that  and .  Denoting the derivative with respect to  by , we have

Since  we have,

Differentiating Eq.~\ref{aSO3Eq} w.r.t.  we get

Using Eq.~\ref{yBarDotEq} and Eq.~\ref{aDotTNotEq} we get

Differentiating Eq.~\ref{yBarDotEq} w.r.t. time we get

Using Equations~\ref{yBarDotDotEq}, \ref{aDotTNotEq} and \ref{aDotDotTNotEq} we get


\section{Proof of Theorem 39} \label{proofthm39Sec}
\noindent {\em Proof.}
Recall the definition of function  as 

 Differentiating Eq.~\ref{lambdaEq} with respect to time and denoting derivative w.r.t.  by , we get

At , .  Since , it follows from Eq.~\ref{yBarDotTNotEq}  that .  It is easy to verify that .  Hence, 

From Eq.~\ref{ddotLambdaEq} and Eq.~\ref{yBarDotDotTNotEq} it follows that

Since   for all  in some neighbourhood  of , we have that .  
Hence .  
Hence  = .  
Here  is the differential of the Gauss map, i.e. the curvature tensor of 
 at point .  Using this in Eq.~\ref{ddotLambdaTNotEq} and the fact that  ,  we get

Recalling that  

Here  and  
where  is the shape operator (differential of the Gauss map) of  at .  
Also,  and .  Assume without loss of generality that  
and , hence ,  and . Using 
Eq.~\ref{aDotTNotEq} and the fact that   we get

From Eqs.~\ref{lsi2Eq} and~\ref{lsiRelationEq} and the fact that  we get 
.
\hfill 

\section{Procedural parametrization of the simple sweep} \label{proceduralSec}

We now describe the parametrization of  assuming that the sweep  is simple. 
We obtain a procedural parametrization of  which is an abstract way of defining curves and surfaces. 
This approach relies on the fact that from the user's point of view, a parametric surface(curve) in  
is a map from () to  and hence is merely a set of programs 
which allow the user to query the key attributes of the surface(curve), e.g. its domain and to evaluate the 
surface(curve) and its derivatives at the given parameter value.  This approach to defining geometry is especially 
useful when closed form formulae are not available for the parametrization map and one must resort to iterative 
numerical methods.  We use the Newton-Raphson(NR) method for this purpose.  As an example, the parametrization 
of the intersection curve of two surfaces is computed procedurally in~\cite{procedural}.  This approach has the 
advantage of being computationally efficient as well as accurate.  For a detailed discussion on the procedural framework, 
see~\cite{sohoni}.

The computational framework is as follows.  Given  and , an approximate funnel is first computed, 
which we will refer to as the seed surface.  Now, when the user wishes to evaluate  or its derivative at some parameter value,  
a NR method will be started with seed obtained from the seed surface.  The NR method will converge, upto the required tolerance, to the required 
point on , or to its derivative, as required.  Here, the precision of the evaluation is only restricted by the finite precision of the computer
 and hence is accurate.  It has the advantage that if a tighter degree of tolerance is required while evaluation of the surface or its derivative, the seed 
surface does not need to be recomputed.  Thus, for the procedural definition of  we need the following:
\begin{enumerate}
\item an NR formulation for computing points on  and its derivatives, which we describe in Section~\ref{NRFormSubSec}
\item Seed surface for seeding the NR procedure, which we describe in Section~\ref{seedSubSec}
\end{enumerate}

Recall that by the non-degeneracy assumption,  is the union of .  This suggests a natural parametrization of  in 
which one of the surface parameters is time .  We will call the other parameter  and denote the seed surface by  which is a map 
from the parameter space of  to the parameter space of , i.e.  and while 
the point  may not belong to , it is close to .  In other words,  is close to .  
We call the image of the seed surface through the sweep map  as the approximate envelope and denote it by , 
i.e. .  We make the following assumption about .
\begin{assum} \label{oneOneAssum}
At every point on the \emph{iso-t} curve of , the normal plane to the \emph{iso-t} curve intersects the \emph{iso-t} curve of  in exactly one point.
\end{assum}
Note that this is not a very strong assumption and holds true in practice even with rather sparse sampling of points for the seed surface.  We now describe the Newton-Raphson formulation for evaluating points on  and its derivatives at a given parameter value.

\subsection{NR formulation for } \label{NRFormSubSec}

Recall that the points on  were characterized by the tangency condition .  
Introducing the parameters  of , we rewrite this equation :

So, given , we have one equation in two unknowns, viz.  and .  is defined as the 
intersection of the plane normal to the iso-(for ) curve of  at  with the iso-(for ) 
curve of  which is nothing but . Recall that  is given by  where  obey Eq.~\ref{envlCondParEq}.  
Henceforth, we will suppress the notation that  and  are functions of  and .  Also, all the evaluations will be 
understood to be done at parameter values .  The tangent to iso- curve of  at   is given by 

Hence,  is the solution of simultaneous system of equations~\ref{envlCondParEq} and~\ref{planeOrthoEq}

Eq.~\ref{envlCondParEq} and Eq.~\ref{planeOrthoEq} give us a system of two equations in two unknowns,  and  and hence can be put into NR 
framework by computing their first order derivatives w.r.t  and .  For any given parameter value , we seed the NR method with the 
point  and solve  Eq.~\ref{envlCondParEq} and Eq.~\ref{planeOrthoEq} for  and compute .


Having computed  we now compute first order derivatives of  assuming that they exist.  In order to compute , we differentiate Eq.~\ref{envlCondParEq} and Eq.~\ref{planeOrthoEq} w.r.t.  to obtain

Eq.~\ref{derPEq1} and Eq.~\ref{derPEq2} give a system of two equations in two unknowns, viz.,  and 
 and can be put into NR framework by computing first order derivatives w.r.t.  
and .  Note that Eq.~\ref{derPEq1} and Eq.~\ref{derPEq2} also involve  and  whose computation we have already described.
After computing  and ,  can be computed as 
.   
can similarly be computed by differentiating Eq.~\ref{envlCondParEq} and Eq.~\ref{planeOrthoEq} w.r.t. .  Higher order derivatives can be computed in a 
similar manner.

\subsection{Computation of seed surface} \label{seedSubSec}

The seed surface is constructed by sampling a few points on the funnel and fitting a tensor 
product B-spline surface through these points.  For this, we first sample a few time instants, 
say,  from the time interval of the sweep.  For each ,  
we sample a few points on the pcurve of contact .  For this, we begin with one point  on  
and compute the tangent to  at , call it . Then  
is used as a seed in Newton-Raphson method to obtain the next point on  and this process is repeated.

While we do not know of any structured way of choosing the number of sampled points, in practice even a small 
number of points suffice to ensure that the Assumption~\ref{oneOneAssum} is valid.


\eat{
\appendix

\section{Proof for Proposition~\ref{gLem}}	\label{gProofSec}
Recall the statement of Proposition~\ref{gLem} that for  and , 
either (i)  and , or 
(ii)  and  or (iii) .
Define  and  as 
\eat{  and }
 and  
 respectively.

\noindent {\em Proof.} We define the following objects in  where the fourth dimension is time.
Let  and
.
Note that  is a four dimensional topological manifold and  is a three dimensional
submanifold of . Further, a point  lies in  if  and .  
Further, if ,  
forms the boundary of  where Define the 
projection  is defined as  and 
the projection  is defined as .
By Lemma~\ref{intLem}, for a point 
, if  then .  Hence a necessary 
condition for  to be in  is that the line  should be tangent to 
which is a three dimensional manifold which is smooth everywhere except at 
 and at . 
For , \eat{  is spanned by .  Hence }
the outward normal to  at  is given by  and  
the outward normal to  at  is given by .
We now compute the outward normal to  at .
The manifold  is diffeomorphic to , i.e., the cross product of  
which is a 2-dimensional manifold and  which is a 1-dimensional manifold, with the 
diffeomorphism given by , .  
Hence, if  spans   and  spans  
then the tangent space of  at  is spanned by 
 and  is spanned by \\
 .
Hence, the outward normal to  at  is 
.
Consider now three cases as follows.

Case (i): .  At any point  there is a cone of outward normals 
given by  
where  and .  So if the line  is tangent to  at  then 

for some  where  and .  Solving the above for  we get .
Hence .

Case (ii): .  Proof is similar to case (i).

Case (iii): . If the line  is 
tangent to  at  there exist  not all zero such that 

It follows that .  
In other words, .
\hfill 
}


\begin{thebibliography}{00}





\bibitem{jacobian}
Abdel-Malek K, Yeh HJ. Geometric representation of the swept volume using Jacobian rank-deficiency conditions. 
Computer-Aided Design 1997;29(6):457-468.

\bibitem{acis}
ACIS 3D Modeler, SPATIAL, 
\verb{www.spatial.com/products/3d_acis_modeling{

\bibitem{sede}
Blackmore D, Leu MC, Wang L. Sweep-envelope differential equation algorithm and its application to NC machining verification. 
Computer-Aided Design 1997;29(9):629-637.

\bibitem{trimming}
Blackmore D, Samulyak R, Leu MC. Trimming swept volumes. 
Computer-Aided Design 1999;31(3):215-223.

\bibitem{errorBounds}
Elber G. Global error bounds and amelioration of sweep surfaces.
Computer-Aided Design 1997;29(6):441-447.

\bibitem{diffTop}
Guillemin V, Pollack A. Differential Topology.
Prentice-Hall, 1974.

\bibitem{classifyPoints}
Huseyin Erdim, Horea T. Ilies. Classifying points for sweeping solids.
Computer-Aided Design 2008;40(9);987-998

\bibitem{planarSwep}
Huseyin Erdim, Horea T. Ilies. Detecting and quantifying envelope singularities in the plane.
Computer-Aided Design 2007;39(10);829-840

\bibitem{procedural}
Markot R, Magedson R. Procedural method for evaluating the intersection curves of two parametric surfaces.
Computer-Aided Design 1990;23(6);395-404

\bibitem{sohoni}
Milind Sohoni. Computer aided geometric design course notes.
\verb{www.cse.iitb.ac.in/~sohoni/336/main.ps{

\bibitem{peternell}
Peternell M, Pottmann H, Steiner T, Zhao H. Swept volumes. 
Computer-Aided Design and Applications 2005;2;599-608

\bibitem{completeSweep}
Seok Won Lee, Andreas Nestler. Complete swept volume generation, Part I: Swept volume of a piecewise C1-continuous cutter at five-axis milling via Gauss map.
Computer-Aided Design 2011;43(4);427-441

\bibitem{completeSweep2}
Seok Won Lee, Andreas Nestler. Complete swept volume generation, Part II: NC simulation of self-penetration via comprehensive analysis of envelope profiles.
Computer-Aided Design 2011;43(4);442-456

\bibitem{selfIntersections}
Xu Z-Q, Ye X-Z, Chen Z-Y, Zhang Y, Zhang S-Y. Trimming self-intersections in swept volume solid modelling. 
Journal of Zhejiang University Science A 2008;9(4):470-480.

\bibitem{scroll}
Kinsley Inc. Timing screw for grouping and turning.
\verb{https://www.youtube.com/watch?v=LooYoMM5DEo{

\end{thebibliography}

\end{document}
