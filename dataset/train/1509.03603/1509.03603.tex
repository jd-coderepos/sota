\documentclass[journal,12pt,draftclsnofoot,onecolumn]{IEEEtran}
\IEEEoverridecommandlockouts
\ifCLASSINFOpdf
\else
\fi
\hyphenation{op-tical net-works semi-conduc-tor}

\newtheorem{theorem}{Theorem}
\usepackage{amsmath,graphicx,cite}
\begin{document}
\begin{titlepage}
\begin{center}
\vspace*{-2\baselineskip}
\begin{minipage}[l]{7cm}
\flushleft
\includegraphics[width=2 in]{njit-logo.jpg}
\end{minipage}
\hfill
\begin{minipage}[r]{7cm}
\flushright
\includegraphics[width=1 in]{ANL_LOGO.jpg}
\end{minipage}

\vfill

\textsc{\LARGE Green Energy Aware Avatar Migration Strategy in Green Cloudlet Networks\12pt]
\LARGE Sep 07, 2015}\
\delta \left( i,k \right)=\sum\limits_{j=1}^{{{n}_{i}}}{{{\eta }_{i}}\left( j,k \right)}														

{{n}_{i}}=\left\lceil \frac{\sum\limits_{k}{\delta \left( i,k \right)}}{\tau } \right\rceil 

			{{P}_{i,j}}={{P}^{s}}+P_{i,j}^{vir}+\alpha \times u_{i,j}^{app}
		
			u_{i,j}^{app}=\sum\limits_{k}{{{\eta }_{i}}\left( j,k \right)\times u_{k}^{app}}
		
			P_{i,j}^{vir}=P_{i,j}^{hyper}+P_{i,j}^{idle}
		
			P_{i,j}^{hyper}=\beta \times \sum\limits_{k}{{{\eta }_{i}}\left( j,k \right)}
		
		P_{i,j}^{idle}=\alpha \times {{u}^{idle}}\times \sum\limits_{k}{{{\eta }_{i}}\left( j,k \right)}
		
		{{P}_{i,j}}={{P}^{s}}+\beta \times \sum\limits_{k}{{{\eta }_{i}}\left( j,k \right)}+\sum\limits_{k}{\left[ {{\eta }_{i}}\left( j,k \right)\times \alpha \left( {{u}^{idle}}+u_{k}^{app} \right) \right]}
		
			{{P}_{i,j}}={{P}^{s}}+\beta \times \sum\limits_{k}{{{\eta }_{i}}\left( j,k \right)}+\sum\limits_{k}{\left[ {{\eta }_{i}}\left( j,k \right)\times \alpha {{u}_{k}} \right]}
		
			{{P}_{i}}=\sum\limits_{j=1}^{{{n}_{i}}}{{{P}_{i,j}}}={{n}_{i}}{{P}^{s}}+\beta \times \sum\limits_{j=1}^{{{n}_{i}}}{\sum\limits_{k}{{{\eta }_{i}}\left( j,k \right)}+\sum\limits_{j=1}^{{{n}_{i}}}{\sum\limits_{k}{\left[ {{\eta }_{i}}\left( j,k \right)\times \alpha {{u}_{k}} \right]}}}
		
			{{P}_{i}}\approx \sum\limits_{k}{\left[ \delta \left( i,k \right)\times \left( \frac{{{P}_{s}}}{\tau }+\beta +\alpha {{u}_{k}} \right) \right]}
		
		{{T}_{i,e}}=\sigma \times {{d}_{i,e}}\times \delta \left( i,k \right)
	
	& \underset{\delta \left( i,k \right)}{\mathop{\min }}\,\text{   }\sum\limits_{i}{{{\rho }_{i}}} \\ 
	& s.t.\text{    }\forall i,\text{  }{{\rho }_{i}}\ge \sum\limits_{k}{\left[ \delta \left( i,k \right)\times \left( \frac{{{P}_{s}}}{\tau }+\beta +\alpha {{u}_{k}} \right) \right]}-{{G}_{i}}, \\ 
	& \forall i,\text{   }{{\rho }_{i}}\ge 0, \\ 
	& \forall k,\text{  }\sigma {{d}_{i,e}}\delta \left( i,k \right)\le \varepsilon , \\ 
	& \forall i,\text{   }\frac{1}{\tau }\text{ }\sum\limits_{k}{\delta \left( i,k \right)}\le {{m}_{i}}, \\ 
	& \forall k,\text{   }\sum\limits_{i}{\delta \left( i,k \right)}=1, 
	
		\underset{\delta \left( i,k \right)}{\mathop{\min }}\,\text{   }\Delta T\sum\limits_{i=1}^{2}{\max \left\{ {{P}_{i}}-Q,0 \right\}}	
	
		s.t.\text{    }\sum\limits_{i=1}^{2}{{{P}_{i}}}=2Q,
	
	where  . Obviously, the optimal solution for minimizing the total on-grid energy consumption of GCN is to assign the total energy demands into the two cloudlets equally, i.e., , which can be considered as the partition problem (i.e., a well know NP-hard problem). So the problem of minimizing the on-grid energy consumption of the GCN is NP-hard.\
	\end{IEEEproof}
	
	To solve GEAR (which is a mixed integer linear programming problem), we use the Branch and Bound search method \cite{23} to find the sub-optimal solution to the problem. Therefore, in each time slot, each Avatar estimates its average CPU utilization for the next time slot by adopting the CPU workload prediction model \cite{24}\cite{25}, acquires the location of its UE, and reports the information to the GCN manager. The GCN manager, i.e., a central controller in GCN, decides the location of all Avatars by solving the above optimization problem. 	
\section{Simulation Result}
\begin{table}[!htb]
	\renewcommand{\arraystretch}{1.3}
	\caption{System Parameter}
	\label{tab:para}
	\centering
	\begin{tabular}{c c}
		\hline
		\hline
		Parameter  &  Value\\
	
		\hline
		
		The length of time slot,    &   15 \\
		
		Capacity of server,      &   16 Avatars\\ 
		
		Power consumption of standby server,  & 80 \\
		
		CPU usage to power mapping coefficient,  &  0.2  \\
		
		Avatar to power mapping coefficient, 		&    0.3 \\
	
		Distance to delay mapping coefficient,  & 3.33 \\ 
	
		SLA,   &  10 \\
	
		\hline
		\hline
	\end{tabular}
\end{table}
We simulate the proposed GEAR strategy in GCN. For comparisons, we select the other Avatar migration strategy, i.e., Follow me AvataR (\emph{FAR}) migration strategy. The idea of FAR is similar to the previous work \cite{7}, which tries to minimize the propagation delay between a UE and its VM in the cloud. Similarly, FAR does not minimize the on-grid energy consumption but minimizes the propagation delay between a UE and its Avatar by selecting the nearest cloudlet as the host of the UE's Avatar (i.e., when the UE moves from one eNB coverage area to the other eNB coverage area, its Avatar also migrates correspondingly). Some system parameters are listed in Table I.\
\begin{figure}[!htb]
	\centering	
\includegraphics[width=0.6\columnwidth]{fig4.jpg}
	\caption{Network topology}
	\label{fig4}
\end{figure}
\begin{figure}[!htb]
	\centering	
\includegraphics[width=0.6\columnwidth]{fig5.jpg}
	\caption{Average solar radiation generated at different time}
	\label{fig5}
\end{figure}
\begin{figure}[!htb]
	\centering	
\includegraphics[width=0.6\columnwidth]{fig6.jpg}
	\caption{On-grid energy consumption at different time}
	\label{fig6}
\end{figure}
\begin{figure}[!htb]
	\centering	
\includegraphics[width=0.6\columnwidth]{fig7.jpg}
	\caption{Total on-grid energy consumption in one day}	
	\label{fig7}
\end{figure}
\begin{figure}[!htb]
	\centering	
\includegraphics[width=0.6\columnwidth]{fig8.jpg}
	\caption{Total on-grid energy consumption over the  number of UEs in GCN}
	\label{fig8}
\end{figure}
\begin{figure}[!htb]
	\centering	
\includegraphics[width=0.6\columnwidth]{fig9.jpg}
	\caption{Total on-grid energy consumption over different values of }	
	\label{fig9}
\end{figure}

To demonstrate the viability of GEAR, we set up a network with the topology shown in Fig. \ref{fig4}, which includes 16 cloudlet-eNB combinations (44) in a square area of 64 . The coverage of each eNB is a square area of 4 . The whole area is divided into 2 parts, i.e., urban and rural areas. Initially, each cloudlet's capacity  in terms of the number of servers is randomly chosen between 10 and 30, and UEs are uniformly distributed in the network. The location of each Avatar is initially chosen to be its nearest cloudlet. Each Avatar's CPU is assigned by one physical core in the server and the CPU utilization of each Avatar is randomly chosen between 10\% and 100\% in each time slot (we assume the OS kernel instances cost 10\% of the Avatar's CPU utilization). Each server can host 16 Avatars at most.\
\subsection{Spatial Dynamics of Energy Demand}
Energy demands of different cloudlets in GCN exhibit spatial dynamics, and so we setup the simulation scenario as follows: UE mobility adopts the modified random waypoint model, i.e., each UE randomly selects a speed between 0 and 10  in every time slot and moves toward its destination, and the locations of UEs' destinations (i.e., the values of x and y coordinates) are randomly selected according to a normal distribution~N(4 ,1.4 ), which implies that UEs more likely move toward the center of each urban area (i.e., based on the characteristics of the normal distribution, UEs more likely select their destinations which are close to the center of the network). For the green energy generation, we use the local daily solar radiation data trace (Millbrook, NY in Jan. 1st, 2015) from National Climatic Data Center \cite{26}, as shown in Fig. \ref{fig5}, where each point indicates the average solar radiation within the current hour. Assume the size of the solar cell equipped in each cloudlet is 5  and the efficiency for converting solar radiation into electricity is 46\% \cite{27}. Also, suppose the green energy generated in different cloudlets is the same in the same time slot. Fig. \ref{fig6} shows the total on-grid energy consumption of two Avatar migration strategies in different time slots. When there is no or little green energy provision in GCN (from 8 a.m. to 9 a.m.), there is no difference between the two live migration strategies since all the cloudlets are powered by on-grid energy. However, when more green energy is generated at each cloudlet, GEAR can save more on-grid energy than FAR, since GEAR can migrate Avatars from the cloudlets with higher energy demand to the cloudlets with lower energy demand so that green energy can be fully utilized. Fig. \ref{fig7} shows the total on-grid energy consumption in GCN in the whole day; note that GEAR saves 35\% on-grid energy as compared to FAR.\

We next examine the effect of the density of UEs by increasing the number of UEs from 600 to 1400. More UEs in the network result in more out-of-balanced energy demand between the urban area and the rural area because all the UEs prefer to go to the urban rather than the rural area (according to the simulation set-up that we mentioned previously). Fig. \ref{fig8} shows the total on-grid energy consumption in one day with respect to different numbers of UEs in the network. We can see that when the number of UEs increases from 600 to 1100, the difference of the total on-grid energy consumption between GEAR and FAR is increasing, i.e., if the energy demand is more unbalanced between areas, GEAR can save more on-grid energy as compared to FAR by balancing the energy gap among the cloudlets. However, as the number of UEs exceeds 1100, the difference of the total on-grid energy consumption between GEAR and FAR remains static because green energy has already been fully utilized by GEAR when the number of UEs reaches 1100, and if the energy demands are still increasing, on-grid energy has to be tapped.\

\subsection{Spatial Dynamics of Green Energy Provision and Energy Demand}
In the real environment, not only the energy demand but also the green energy provision may exhibit spatial dynamics, i.e., the solar cell in different cloudlets may generate different amount of green energy because of the position of the sun, the spatial dynamics of atmospheric conditions, etc. Also, evidence shows that the solar radiation of the rural area is greater than the urban area \cite{28}. Therefore, we setup the simulation scenario as follows: the UE's parameters are the same as in the previous simulation scenario and the hourly average solar radiation in the rural area still follows the data trace as shown in Fig. \ref{fig5}. However, the hourly average solar radiation in the urban area is reduced by κ percentage (the value of κ is selected from 0\% to 30\% in the simulation). Fig. \ref{fig9} shows the total on-grid energy consumption of the network in one day with respect to different values of . Note that as the value of  increases (i.e., hourly green energy generation is getting less in the urban area and the energy gap between the two areas is getting larger), FAR consumes more on-grid energy because the energy gap of the urban area is getting larger, and GEAR can save more on-grid energy as compared to FAR by balancing the energy gap between the two areas.
	
	
\section{Conclusion}
In this paper, we have proposed the GCN architecture to provide seamless and lower latency MCC services to UEs, i.e., UEs can offload tasks to their powerful Avatars with shorter propagation delay. However, owing to the spatial dynamics of energy demand and green energy provisioning, a significant amount of green energy is wasted, thus resulting in more grid energy consumption. Therefore, we have proposed the GEAR strategy to redistribute the energy demand by migrating Avatars among cloudlets according to cloudlets' green energy generation and to guarantee the maximum Avatar propagation delay. Simulation results have demonstrated that GEAR can significantly save on-grid energy as compared to the FAR strategy.\

In the future, we will consider the heterogeneous nature of the cloudlets, i.e., the configurations of the servers in GCN are different, and UEs can choose different configurations of VMs as their Avatars. Also, Avatar migration cost will be considered in the optimal Avatar migration strategy. Moreover, we will also consider the scenario in which a cloudlet and an eNB in the same location can share green energy and both of them can also generate different energy demands in different time slots. It is a big challenge to design an optimal Avatar migration strategy by considering the energy demands of eNBs and cloudlets.

\bibliographystyle{IEEETran}

\begin{thebibliography}{15}

\bibitem{1}
S.~Mahadev, P.~Bahl, R.~Caceres, and N.~Davies, ''{The Case for VM-Based Cloudlets in Mobile Computing},''\emph{IEEE Pervasive Computing}, vol.~8, no.~4, pp.~14--23, Oct.--Dec.~2009.

\bibitem{2}
Latency Considerations in LTE, Sep. 2014. [Online]. Available: http://mavenir.com/files/docdownloads/StokeDocuments/ 130-0029-001LTELatencyConsiderationsFinal.pdf.

\bibitem{3}
C.~Borcea, \emph{et al}., ''{Avatar: Mobile Distributed Computing in the Cloud},'' in \emph{3rd IEEE International Conference on Mobile Cloud Computing, Services, and Engineering (MobileCloud '15)}, San Francisco, CA, Mar.~30-Apr.~3, 2015, pp.~151--157.

\bibitem{4}
A. Basta, \emph{et al}., "A Virtual SDN-Enabled LTE EPC Architecture: A Case Study for S-/P-Gateways Functions," in \emph{et 2013 IEEE SDN for Future Networks and Services (SDN4FNS)}, Trento, Italy, Nov. 11-13, 2013, pp. 1-7.

\bibitem{5}
X. Jin, L. E. Li, L. Vanbever, and J. Rexford, "Softcell: Scalable and flexible cellular core network architecture," in \emph{Proceedings of the 9th ACM conference on Emerging networking experiments and technologies}, Santa Barbara, CA, Dec. 09-12, 2013, pp. 163-174.

\bibitem{6}
W. Liu, J. Cao, X. Qiu, and J. Li, "Improving Performance of Mobile Interactive Data-Streaming Applications with Multiple Cloudlets," in \emph{ 2014 IEEE 6th International Conference on Cloud Computing Technology and Science (CloudCom)}, Singapore, Dec. 15-18, 2014, pp. 46-53.

\bibitem{7}
T. Taleb and A. Ksentini, "Follow me cloud: interworking federated clouds and distributed mobile networks," \emph{IEEE Network}, vol. 27, no. 5, pp.12-19, Sep.-Oct. 2013.

\bibitem{8}
D. Xu, X. Liu, and B. Fan, "Minimizing energy cost for Internet-scale datacenters with dynamic traffic," in \emph{2011 IEEE 19th International Workshop on Quality of Service (IWQoS)}, San Jose, CA, Jun. 6-7, 2011, pp. 1-2.

\bibitem{9}
K. Le, \emph{et al}., "Capping the brown energy consumption of Internet services at low cost," in \emph{2010 International Green Computing Conference}, Chicago, IL, Aug. 15-18, 2010, pp. 3-14.

\bibitem{10}
N. Buchbinder, N. Jain, and I. Menache, "Online job-migration for reducing the electricity bill in the cloud," in \emph{2011 NETWORKING}, Valencia, Spain, May. 9-13, 2011, pp. 172-185. 

\bibitem{11}
A. Qureshi, et al., "Cutting the electric bill for internet-scale systems." in \emph{ACM SIGCOMM Computer Communication Review}, vol. 39, no. 4, pp. 123-134, 2009.

\bibitem{12}
B. Aksanli, J. Venkatesh, T. Rosing, and I. Monga, "A comprehensive approach to reduce the energy cost of network of datacenters," in \emph{2013 IEEE Symposium on Computers and Communications (ISCC'13)}, Split, Croatia, Jul. 7-10, 2013, pp. 275-280.

\bibitem{13}
L. Gkatzikis and I. Koutsopoulos, "Migrate or not? Exploiting dynamic task migration in mobile cloud computing systems," in \emph{IEEE Wireless Communications}, vol. 20, no. 3, pp. 24-32, Jun. 2013.

\bibitem{14}
D. Hatzopoulos, I. Koutsopoulos, G. Koutitas, and W. V. Heddeghem, "Dynamic virtual machine allocation in cloud server facility systems with renewable energy sources," in emph{ Proceedings of IEEE International Conference on Communications (ICC'13)}, Budapest, Hungary, Jun. 9-13, 2013, pp. 4217-4221.

\bibitem{15}
C. Chen, B. He, and X. Tang, "Green-aware workload scheduling in geographically distributed data centers," in emph{ 2012 IEEE 4th International Conference on Cloud Computing Technology and Science (CloudCom)}, Taibei, Taiwan, Dec. 3-6, 2012 , pp. 82-89.

\bibitem{16}
M. Ghamkhari. and H. Mohsenian-Rad, "Energy and performance management of green data centers: A profit maximization approach," \emph{IEEE Transactions on Smart Grid}, vol. 4, no. 2, pp. 1017-1025, 2013. 

\bibitem{17}
A. Kiani, and N. Ansari, "Toward Low-Cost Workload Distribution for Integrated Green Data Centers," \emph{IEEE Communications Letters}, vol. 19, no. 1, pp. 26-29, Jan. 2015.

\bibitem{18}
T. Han and N. Ansari, "On Optimizing Green Energy Utilization for Cellular Networks with Hybrid Energy Supplies," \emph{IEEE Transactions on Wireless Communications}, vol. 12, no. 8, pp. 3872-3882, August 2013.

\bibitem{19}
G. Warkozek, E. Drayer, V. Debusschere, and S. Bacha, "A new approach to model energy consumption of servers in data centers," in \emph{2012 IEEE International Conference on Industrial Technology}, Athens, Greece, Mar. 19-21, 2012, pp. 211-216.

\bibitem{20}
Cisco System, Inc. (2007). “Design Best Practices for Latency Optimization,” Financial Services Technical Decision Maker White Paper. [Online]. Available: https://www.cisco.com/application/pdf/en /us/guest/netsol/ns407/c654/ccmigration09186a008091d542.pdf

\bibitem{21}
Í. Goiri, et al., "GreenHadoop: leveraging green energy in data-processing frameworks," in \emph{Proceedings of the 7th ACM european conference on Computer Systems}, Bern, Switzerland, Apr. 10-13, 2012, pp. 57-70.

\bibitem{22}
T. Han and N. Ansari, "Green-energy Aware and Latency Aware user associations in heterogeneous cellular networks," in \emph{Proceedings of IEEE Global Communications Conference (GLOBECOM'13)}, Atlanta, GA, Dec. 9-13, 2013, pp. 4946-4951.

\bibitem{23}
G. Mitra, "Investigation of some branch and bound strategies for the solution of mixed integer linear programs." \emph{Mathematical Programming}, vol. 4, no. 1, pp. 155-170, 1973.

\bibitem{24}
K. Qazi, Y. Li, and A. Sohn, "Workload Prediction of Virtual Machines for Harnessing Data Center Resources," in \emph{2014 IEEE 7th International Conference on Cloud Computing}, Anchorage, AK, Jun. 27-Jul. 2, 2014, pp. 522-529.

\bibitem{25}
Z. Xiao, W. Song, and Q. Chen, "Dynamic resource allocation using virtual machines for cloud computing environment," \emph{IEEE Transactions on Parallel and Distributed Systems}, vol. 24, no. 6, pp. 1107-1117, 2013.

\bibitem{26}
Daily solar radiation data trace from National Climatic Data Center. [Online]. Available: http://www1.ncdc.noaa.gov/pub/data/uscrn/products/hourly02/2015/CRNH0203-2015-NYMillbrook3W.txt

\bibitem{27}
M. K. Islam, T. Ahammad, E. H. Pathan, A. M. Haque, et al., “Analysis of Maximum Possible Utilization of Solar Radiation on a Solar Photovoltaic Cell with a Proposed Model,” \emph{International Journal of Modeling and Optimization}, vol. 1, no. 1, pp. 66-69, Jan. 2011.

\bibitem{28}
G. Codato, A. P. Oliveira, and J. F. Escobedo, "Comparative study of solar radiation in urban and rural areas," in \emph{Anais do XIII Congresso Brasileiro de Meteorologia}, Fortaleza, Brazil, 2004. 
  
\end{thebibliography}


\end{document}