\section{Experiments}\label{experiments}

In this section, we first describe the details of our data set, then we describe the implementation details of our evaluation and training scheme. Finally we discuss how we select a hypothesis from the set of possible poses created by the MDN when evaluating our method.

\subsection{Dataset}\label{experiments_dataset}

   We evaluate our performance on the Human3.6M dataset \cite{ionescu2013human3}, which contains 3.6 million images attained by a motion capture system with four synchronized cameras in an indoor environment. The images themselves consist of seven professional actors performing daily-life activities such as walking, eating, sitting, smoking, and engaging in a discussion. Both 2D and 3D ground-truth joint locations are available, with the ultimate goal of learning to predict the 3D positions for each joint, given either the raw image, or the 2D ground truth locations of the joints in that image. 
   
   When predicting the 3D joint positions from a single 2D input (i.e., using inputs that are only a single camera view and a single moment in time), the task of determining the 3D joint locations is underdetermined. For example, movement in the depth-axis of the camera and occlusion of joints in a particular camera view create scenarios where many 3D joint positions and poses can be projected onto the same 2D input view. As such, state-of-the-art techniques now often sidestep the difficulties resulting from the ``inverse problem'' nature of the task by making use of the synchronized multi-view images, or the full temporal sequence of images for each action/subject. 
   
   In order to demonstrate how a GNN and MDN combination extends representational capacity beyond either the GNN or MDN model alone, we show results in the case of single-view/single-frame prediction given a single 2D input. We first compare results to the SemGCN model utilizing ground-truth 2D inputs for training and testing (hereafter denoted GT), and second, compare to a multi-modal MDN model using fine-tuned stacked hourglass 2D detections given by \cite{martinez2017simple} as an input (hereafter denoted SH). We then qualitatively show the utility in a model that can entertain and represent multiple hypotheses simultaneously, and discuss further benefits that can be derived from such a representation by incorporating multi-view or temporal information in order to more optimally select between them. 

\subsection{Training and evaluation protocols}\label{experiments_training}

   Following previous work \cite{Zhao_2019_CVPR, pavllo20193d, jahangiri2017generating, Li_2019_CVPR}, we train on subjects 1, 5, 6, 7, and 8 and test on subjects 9 and 11. We use all the images from all four camera views in training and testing, albeit not in a multi-view setup; the different camera views are treated as independent examples. 

   To evaluate our network, we consider two commonly used protocols for evaluating the performance of 3D pose estimation, henceforth referred to as Protocol \#1 and Protocol \#2. Protocol \#1 is the mean per-joint position error (MPJPE) in millimeters between the predicted joint positions and the ground-truth joint positions. Protocol \#2 computes the same error, but after applying a subsequent rigid transformation that further aligns the predictions with the ground truth, and is alternately referred to as P-MPJPE in the literature. 

\subsection{Hypothesis selection}
The MDN outputs are distributions as opposed to most approaches on Human3.6M, which only predict a single scalar. To compare with previous approaches we need to reduce the distributions to a single prediction. In this paper we use the following methods for creating a prediction:

\textbf{Highest}: Simply choose the mean of the kernel $n$ with the highest mixture coefficient. $\hat{\Vec{y}} = \mu_n \mid n = \text{argmax}_j\pi_j$. This essentially represents the guess the network thinks is the most likely to be correct.

\textbf{Mean}: Predict the weighted average of the distribution calculated as 
$\hat{\Vec{y}} = \sum_{j=1}^K \pi_j \mu_j$.
This is useful because of the way error is evaluated. If we know the true distribution of target variables, in order to minimize average MPJPE-error it is beneficial to predict the mean of the distribution, even if the mean itself is not a possible pose. 

\textbf{Oracle}: Since each kernel represents a distinct possible pose, having an \textit{Oracle} to predict which one of these possible poses to choose can improve our results significantly. The \textit{Oracle} selects out of the K kernel means the 3D pose that is closest to the target 3D pose. Previous works treating pose estimation as an inverse problem take this approach \cite{Li_2019_CVPR,jahangiri2017generating, Sharma_2019_ICCV}, and it can be considered as an upper-bound estimate for how well the GraphMDN is performing on this task. While this is not a realistic method for real-world applications (we would never need a prediction mechanism in the first place if we knew the ground truth target), we nevertheless make use of it in order to compare against alternative multi-modal methods.

\subsection{Training details}

In this work we aimed to keep our training hyperparameters and setup as close to SemGCN as possible. However, in order to speed up the training process itself, we increased batch-size from 64 to 256 and replaced the decaying learning rate with the \textit{Super-Convergence} LR schedule described in \cite{smith2017superconvergence} with a peak learning rate of $6 \times 10^{-3}$. This allowed us to train all networks in just 2 epochs instead of 30+ used by \cite{Zhao_2019_CVPR}. Together these changes allow us to reduce training times from 10 hours to less than 15 minutes using the same hardware. We also reproduce the SemGCN results using Super-Convergence and our setup to isolate the effects of the new training setup from the effects of the MDN output. All our models used 5 MDN kernels unless otherwise stated. We train a single model for all actions. Our model is implemented in PyTorch, and we employ ADAM \cite{kingma2014adam} for optimization and initialize our network weights using Xavier initialization \cite{glorot2010understanding}. All models used dropout with a rate of 0.1 unless otherwise mentioned.

\section{Results}
\label{results}



\begin{table}
\scriptsize
\setlength{\tabcolsep}{1pt}
\begin{tabular}{lllllllllllllllll}
\noalign{\smallskip}
\textbf{Protocol   \#1} & Direct. & Discuss & Eating & Greet & Phone & Photo & Posing & Purch. & Sitting & SittingD. & Smoke & Wait & WalkD & Walk & WalkT. & Avg. \\ \hline
SemGCN (reported)\cite{Zhao_2019_CVPR} & 37.8 & 49.4 & 37.6 & 40.9 & 45.1 & 41.4 & 40.1 & 48.3 & 50.1 & 42.2 & 53.5 & 44.3 & 40.5 & 47.3 & 39.0 & 43.8 \\
SemGCN & 34.7 & 41.9 & 35.3 & 38.7 & 41.0 & 53.5 & 41.1 & 37.3 & 44.6 & 53.9 & 40.3 & 41.9 & 41.5 & 33.3 & 36.0 & 41.0 \\
SemGCN (Wide) & 34.4 & 41.6 & 32.9 & 37.5 & 39.2 & 47.5 & 41.0 & 34.1 & 43.6 & 52.3 & 37.6 & 40.7 & 38.6 & 30.5 & 32.9 & 39.0 \\
Ours (Mean) & \textbf{32.2} & \textbf{39.5} & 33.5 & 36.7 & 38.2 & 48.5 & 39.0 & 36.7 & 44.0 & 52.9 & 37.2 & 40.3 & 39.4 & 30.0 & 32.1 & 38.7 \\
Ours (Highest, Wide) & 34.5 & 40.5 & \textbf{33.0} & 35.7 & 37.0 & 44.8 & 39.1 & \textbf{33.0} & 41.2 & 50.2 & 36.6 & 38.6 & 38.2 & 28.4 & 31.8 & 37.5 \\
Ours (Mean, Wide) & 33.9 & 39.9 & \textbf{33.0} & \textbf{35.4} & \textbf{36.8} & \textbf{44.4} & \textbf{38.9} & \textbf{33.0} & \textbf{41.0} & \textbf{50.0} & \textbf{36.4} & \textbf{38.3} & \textbf{37.8} & \textbf{28.2} & \textbf{31.5} & \textbf{37.2} \\ \hline
Ours (Oracle, Wide) & 28.9 & 34.5 & 28.2 & 30.2 & 31.5 & 38.5 & 32.3 & 28.6 & 35.7 & 43.3 & 31.9 & 32.1 & 33.3 & 25.2 & 27.8 & 31.8 \\
 &  &  &  &  &  &  &  &  &  &  &  &  &  &  &  &  \\
\textbf{Protocol \#2} & Direct. & Discuss & Eating & Greet & Phone & Photo & Posing & Purch. & Sitting & SittingD. & Smoke & Wait & WalkD & Walk & WalkT. & Avg. \\ \hline
SemGCN & 26.2 & 32.7 & 28.4 & 31.4 & 31.3 & 40.8 & 31.0 & 28.6 & 35.9 & 43.5 & 32.0 & 32.2 & 33.2 & 26.8 & 29.2 & 32.2 \\
SemGCN (Wide) & 26.4 & 32.4 & 28.0 & 30.2 & 30.9 & 36.9 & 31.5 & 26.9 & 36.1 & 41.6 & 30.8 & 31.6 & 31.3 & 24.3 & 27.1 & 31.1 \\
Ours (Mean) & \textbf{24.6} & \textbf{30.5} & \textbf{26.3} & 29.7 & \textbf{28.8} & 37.0 & 30.1 & 26.8 & \textbf{34.0} & 40.8 & \textbf{29.5} & 31.5 & 31.0 & 24.2 & \textbf{25.9} & 30.0 \\
Ours (Highest, Wide) & 25.8 & 31,4 & 27.5 & 28.7 & 29.1 & 35.2 & 29.8 & 26.0 & 34.2 & \textbf{40.5} & 29.8 & 30.1 & 30.4 & \textbf{22.3} & 26.0 & 29.8 \\
Ours (Mean, Wide) & 25.4 & 30.9 & 27.5 & \textbf{28.6} & 29.1 & \textbf{35.0} & \textbf{29.7} & \textbf{25.9} & 34.1 & \textbf{40.5} & 29.8 & \textbf{29.9} & \textbf{30.2} & 22.5 & \textbf{25.9} & \textbf{29.7} \\ \hline
Ours (Oracle, Wide) & 22.6 & 27.8 & 23.9 & 24.9 & 26.0 & 31.4 & 25.6 & 22.8 & 30.3 & 35.9 & 26.8 & 26.3 & 26.9 & 20.2 & 22.2 & 26.3 \\
 &  &  &  &  &  &  &  &  &  &  &  &  &  &  &  & 
\end{tabular}
\caption{(P) MPJPE in millimeter on Human3.6M under protocol \#1 and \#2 using the ground-truth 2D joint positions as inputs. We can see our GCMDN consistently outperforms SemGCN using comparable hypothesis selection mechanisms (Mean, Highest) and clearly beats it using Oracle which highlights the usefulness of our multimodal approach.}
\label{tab:GT_ours_semgcn}

\end{table}

\begin{table}
\scriptsize
\setlength{\tabcolsep}{1pt}
\begin{tabular}{lcccccccccccccccc}
\textbf{Protocol   \#1} & Direct. & Discuss & Eating & Greet & Phone & Photo & Posing & Purch. & Sitting & SittingD. & Smoke & Wait & WalkD & Walk & WalkT. & Avg. \\ \hline
SemGCN (reported)\cite{Zhao_2019_CVPR} & \textbf{48.2} & 60.8 & \textbf{51.8} & 64.0 & 64.6 & 53.6 & \textbf{51.1} & 67.4 & 88.7 & \textbf{57.7} & 73.2 & 65.6 & \textbf{48.9} & 64.8 & 51.9 & 60.8 \\
SemGCN & 53.7 & 57.9 & 58.0 & 59.1 & 66.0 & 78.1 & 55.7 & 58.1 & 71.7 & 96.1 & 62.8 & 60.6 & 65.4 & 53.3 & 57.3 & 63.6 \\
SemGCN (Wide) & 49.6 & 55.1 & 56.5 & \textbf{55.9} & 62.4 & 74.3 & 51.4 & 54.6 & 69.8 & 92.6 & 59.6 & 56.9 & 60.5 & \textbf{48.3} & 52.1 & 60.0 \\
Ours (Mean) & 51.9 & 56.1 & 55.3 & 58.0 & 63.5 & 75.1 & 53.3 & 56.5 & 69.4 & 92.7 & 60.1 & 58.0 & 65.5 & 49.8 & 53.6 & 61.3 \\
Ours (Mean, Wide) & 49.9 & \textbf{54.9} & 55.2 & 56.0 & \textbf{62.1} & \textbf{73.2} & 51.6 & \textbf{53.2} & \textbf{69.0} & 88.2 & \textbf{58.9} & \textbf{55.8} & 61.0 & 48.6 & \textbf{50.1} & \textbf{59.2} \\
 &  &  &  &  &  &  &  &  &  &  &  &  &  &  &  &  \\
\textbf{Protocol \#2} & Direct. & Discuss & Eating & Greet & Phone & Photo & Posing & Purch. & Sitting & SittingD. & Smoke & Wait & WalkD & Walk & WalkT. & Avg. \\ \hline
SemGCN & 40.7 & 44.7 & 45.6 & 46.5 & 50.6 & 56.4 & 40.4 & 41.8 & 57.3 & 71.1 & 49.8 & 45.6 & 50.1 & 40.8 & 44.4 & 48.4 \\
SemGCN (Wide) & 38.9 & 42.9 & 44.5 & \textbf{44.8} & 48.8 & 54.5 & \textbf{39.0} & 39.9 & 56.0 & 68.4 & 47.9 & 43.1 & 47.0 & \textbf{36.8} & 41.8 & 46.3 \\
Ours (Mean) & 39.7 & 43.4 & \textbf{44.0} & 46.2 & 48.8 & 54.5 & 39.4 & 41.1 & 55.0 & 69.0 & 48.0 & 43.7 & 49.6 & 38.4 & 42.4 & 46.9 \\
Ours (Mean, Wide) & \textbf{38.5} & \textbf{42.6} & 44.1 & 44.9 & \textbf{48.1} & \textbf{53.3} & \textbf{39.0} & \textbf{39.5} & \textbf{54.9} & \textbf{66.2} & \textbf{47.0} & \textbf{42.2} & \textbf{46.8} & \textbf{36.8} & \textbf{39.8} & \textbf{45.6} \\
 &  &  &  &  &  &  &  &  &  &  &  &  &  &  &  & 
\end{tabular}
\label{tab:SH_ours_semgcn}
\caption{(P) MPJPE in millimeter on Human3.6M under protocol \#1 and \#2 using the fine-tuned stacked hourglass 2D inputs. Our method outperforms SemGCN using SH inputs.}
\end{table}

\subsection{Comparison to SemGCN}

In Table 1 we compare our method against SemGCN \cite{Zhao_2019_CVPR} on ground truth (GT) 2D inputs, and in Table 2 we do the same comparisons using Stacked Hourglass (SH) inputs. All methods were trained in 2 epochs using Super-Convergence. Our standard models used a set of hyperparameters matching the SemGCN architecture, with 4 block layers having 128-dimensional hidden layers. Additionally we experimented with a wider network using 3 blocks with a hidden dimension of 512; these results are reported are indicated as (Wide). Regardless of the input type (SH or GT), we can see that our GraphMDN noticeably improves over SemGCN, such that simply replacing SemGCN's output layer with a GraphMDN output reduces average P1-error by up to 5\%. Furthermore, GraphMDN learns a richer representation of the data capable of furnishing predictions about the \textit{set} of poses that reduce to a particular 2D input. It worth noting that our \textit{Oracle} results are substantially better than other published methods using single frame GT inputs, indicating that refinement of hypothesis-selection procedures could yield significant improvement in real-world applications and represents a promising direction for future research.

\subsection{Comparison against multimodal methods}

To evaluate the quality of our learned distributions, we compare our models against previous multimodal approaches of human pose estimation \cite{jahangiri2017generating, Li_2019_CVPR,Sharma_2019_ICCV} using an \textit{Oracle} to choose the best prediction. The results are shown in Table 3. We can see our method produces state of the art results with 5 guesses after only 2 epochs of training. Additionally our MDN scales much better to larger numbers of kernels than \cite{Li_2019_CVPR}, with a significant performance gain when going from 5 to 8 kernels, while \cite{Li_2019_CVPR} does not gain performance beyond 5 kernels. In addition, our method scales well even up to 200 kernels, and is able to outperform \cite{Sharma_2019_ICCV} on a best of 200 predictions under Protocol 1, although the method in \cite{Sharma_2019_ICCV}
still outperforms GraphMDN under Protocol 2. When GraphMDN is trained with 5 kernels, it effectively generates 5 poses as predictions of the underlying 3D pose. Compared to the accuracy of 5 samples (poses) from the latent space generated in \cite{Sharma_2019_ICCV}, GraphMDN clearly outperforms the variational autoencoder approach in \cite{Sharma_2019_ICCV}. When training the model with 200 kernels, we found Super-Convergence to reduce \textit{Oracle} performance; we hypothesize that reduced performance is due to the rapid training procedure reducing kernel diversity. Because of this, our 200 kernel model was trained with 30 epochs using an exponentially decaying learning rate starting at $10^{-3}$ and a dropout rate of 0.5. While some unimodal approaches have attained better results on this task \cite{Iskakov_triangulation, pavllo20193d, Qiu_2019_ICCV}, none of the outperforming methods are comparable as they often utilize either multi-view or time-series data, or 2D joint detectors better than Stacked Hourglass. An obvious source of improvement would be to utilize a more modern 2D joint detector such as CPN\cite{pavllo20193d}, which would clearly improve our results, but this would also improve the results of other multimodal methods and as such is not meaningful when comparing against them.

\begin{table}
\scriptsize
\setlength{\tabcolsep}{1pt}
\begin{tabular}{lllllllllllllllll}
\textbf{Protocol   \#1} & Direct. & Discuss & Eating & Greet & Phone & Photo & Posing & Purch. & Sitting & SittingD. & Smoke & Wait & WalkD & Walk & WalkT. & Avg. \\ \hline
Li et al.\cite{Li_2019_CVPR}(5) & \textbf{43.8} & \textbf{48.6} & 49.1 & 49.8 & 57.6 & \textbf{61.5} & 45.9 & 48.3 & 62 & \textbf{73.4} & 54.8 & 50.6 & 56 & \textbf{43.4} & 45.5 & 52.7 \\
Sharma et al.\cite{Sharma_2019_ICCV}(5)* & - & - & - & - & - & - & - & - & - & - & - & - & - & - & - & 55.4 \\
Ours (Wide, 5) & 44.2 & 48.7 & \textbf{47.2} & \textbf{49.4} & \textbf{55.1} & 62.0 & \textbf{44.8} & \textbf{46.9} & \textbf{59.7} & 76.5 & \textbf{53.2} & \textbf{48.9} & \textbf{53.6} & 44.5 & \textbf{44.4} & \textbf{51.9} \\ \hline
Li et al.\cite{Li_2019_CVPR}(8) & - & - & - & - & - & - & - & - & - & - & - & - & - & - & - & 52.6 \\
Ours (8) & 43.4 & 47.4 & 46.3 & 48.1 & 55.2 & 59.3 & 43.9 & 45.8 & 58.6 & 75.2 & 52.5 & 48.2 & 53.2 & 42.2 & 43.6 & \textbf{50.9} \\ \hline
Sharma et al.\cite{Sharma_2019_ICCV}(200) & \textbf{37.8} & \textbf{43.2} & 43.0 & 44.3 & 51.1 & 57.1 & \textbf{39.7} & 43.0 & 56.3 & \textbf{64.0} & \textbf{48.1} & 45.4 & 50.4 & \textbf{37.9} & \textbf{39.9} & 46.8 \\
Ours (Wide, 200) & 40.0 & \textbf{43.2} & \textbf{41.0} & \textbf{43.4} & \textbf{50.0} & \textbf{53.6} & 40.1 & \textbf{41.4} & \textbf{52.6} & 67.3 & \textbf{48.1} & \textbf{44.2} & \textbf{49.0} & 39.5 & 40.2 & \textbf{46.2} \\
 &  &  &  &  &  &  &  &  &  &  &  &  &  &  &  &  \\
\textbf{Protocol \#2} & Direct. & Discuss & Eating & Greet & Phone & Photo & Posing & Purch. & Sitting & SittingD. & Smoke & Wait & WalkD & Walk & WalkT. & Avg. \\ \hline
Li et al.\cite{Li_2019_CVPR}(5) & 35.5 & 39.8 & 41.3 & 42.3 & 46.0 & 48.9 & 36.9 & 37.3 & 51.0 & 60.6 & 44.9 & 40.2 & 44.1 & \textbf{33.1} & 36.9 & 42.6 \\
Ours (Wide, 5) & \textbf{33.8} & \textbf{38.7} & \textbf{38.3} & \textbf{39.4} & \textbf{43.6} & \textbf{47.8} & \textbf{34.2} & \textbf{35.2} & \textbf{48.5} & \textbf{58.6} & \textbf{42.7} & \textbf{37.8} & \textbf{41.4} & 33.5 & \textbf{34.4} & \textbf{40.5} \\ \hline
Ours (8) & 33.8 & 38.3 & 37.8 & 38.8 & 43.8 & 46.3 & 33.9 & 34.6 & 47.9 & 58.6 & 42.4 & 37.6 & 41.1 & 32.0 & 34.2 & 40.1 \\ \hline
Sharma et al.\cite{Sharma_2019_ICCV}(200) & \textbf{27.6} & \textbf{27.5} & 34.9 & \textbf{32.3} & \textbf{33.3} & 42.7 & \textbf{28.7} & \textbf{28.0} & \textbf{36.1} & \textbf{42.7} & \textbf{36.0} & \textbf{30.7} & \textbf{37.6} & \textbf{24.3} & \textbf{27.1} & \textbf{32.7} \\
Ours (Wide, 200) & 30.8 & 34.7 & \textbf{33.6} & 34.2 & 39.6 & \textbf{42.2} & 31.0 & 31.9 & 42.9 & 53.5 & 38.1 & 34.1 & 38.0 & 29.6 & 31.1 & 36.3 \\
 &  &  &  &  &  &  &  &  &  &  &  &  &  &  &  & 
\end{tabular}
\label{tab:oracle}
\caption{Comparison of state of the art multimodal methods evaluated under the Oracle protocol. All models use the stacked hourglass 2D inputs. * was estimated from figure. The number in brackets is the number of pose predictions considered; i.e. the number of Gaussian kernels (our paper and \cite{Li_2019_CVPR}) or samples generated \cite{Sharma_2019_ICCV}.}
\end{table}

\subsection{Qualitative Analysis of GraphMDNs}

\begin{figure*}[h]
  \begin{center}
\includegraphics[width=1.0\linewidth]{./mdn_qual_figure_pdf.pdf}
  \end{center}
  \caption{Qualitative results on the Human36M data set using ground-truth 2D inputs. The left column, labeled ``Input'', displays the input (2D) skeleton. The next three columns compare the output of our GraphMDN (labeled ``MDN-Estimated Poses'') with the ground truth target value (labeled "Ground truth") at three different camera azimuths (labeled on figure). In each plot of our MDN-Estimated Poses, each skeleton is shaded according to the mixing coefficient of the kernel that generated it -- kernels with low mixing coefficients will be nearly transparent, while kernels with high mixing coefficients will be opaque. From top row to bottom row, the ``Greeting'', ``Purchasing'', and ``Sitting'' actions are displayed.}
  \label{fig:occlusion_figure}
\end{figure*}

Although the \textit{Oracle} predictions are, on their own, not a plausible mechanism for making predictions in the real-world, they nevertheless indicate that representing multiple potential alternatives can \textit{in principle} result in great improvements in performance. Indeed, the \textit{Oracle} result from the GraphMDN model far exceeds the performance of the SemGCN model on its own, and outperforms state-of-the-art \textit{Oracle} predictions from alternative multi-modal modelling techniques. An obvious question, then, is what exactly have the various kernels in our predictions learned to represent, and subsequently, is there a ``fair''  mechanism that would allow selection between the various hypotheses (without requiring foreknowledge of the answer)?


In order to understand how the mixture model functions in our predictions, Figure 2 shows visualizations of the predicted kernels (super-imposed on one another, with shading proportional to their weighting coefficients) selected from frames in which the best performing kernel performs significantly better than the weighted mean of the predicted kernels. When we visualize the 2D inputs used to make the 3D joint predictions, it is immediately clear that there is ambiguity in the depth axis of the camera, as to where exactly some of the joints might be located. In fact, when viewing the 3D predictions from the camera angle used to generate the 2D inputs (azimuth=60; the middle column for each of the rows of Figure 2 the multiple hypotheses are sufficiently indistinguishable that they often simply appear to be a single prediction. Nevertheless, when rotating that view (the columns corresponding to Azimuth values of 0 and 90), it becomes clear that the the multiple-hypotheses all correspond to distinct joint positions along the depth axis of the camera. 


As demonstrated by the feasibility of all hypotheses output by the GraphMDN (given the 2D inputs), the mixture model has learned to meaningfully represent the statistics of the output distribution. Moreover, the fact that the model is capable of providing multiple hypotheses allows for additional information to be applied when deciding which of the possible kernels represents the best predicted 3D joint locations. In this case, additional information can include anything from human-kinematic structural constraints to multi-view information as well as temporal information (for videos). Although unconstrained optimization over this information can be a potentially costly process, selecting the best of N hypotheses can often greatly simplify the solution search space. While the present GraphMDN work does not make use of these constraints in selecting between alternative hypotheses, the visual confirmation that the distribution of 3D-joint hypotheses represent alternatives that cannot be distinguished by a human observer from the 2D image alone, provides further evidence that the combined GNN + MDN approach results in useful representations that would be difficult to achieve with only one of the other techniques.
