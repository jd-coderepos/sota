\documentclass[11pt,letterpaper]{article}

\usepackage{typearea}
\paperwidth 8.5in \paperheight 11in
\typearea{14}

\usepackage{comment}
\usepackage[ruled,linesnumbered,vlined]{algorithm2e}
\renewcommand{\algorithmcfname}{ALGORITHM}

\makeatletter
 \setlength{\parindent}{0pt}
 \addtolength{\partopsep}{-2mm}
 \setlength{\parskip}{5pt plus 1pt}
 \addtolength{\abovedisplayskip}{-3mm}
 \addtolength{\textheight}{20pt}
\makeatother

\usepackage{color,colortbl}


\usepackage{float}
\usepackage{amsthm}
\usepackage{amsmath}
\usepackage{amssymb}
\usepackage{graphicx}
\usepackage{xspace}
\usepackage{nicefrac}
\usepackage[colorinlistoftodos,prependcaption,textsize=tiny]{todonotes}
\usepackage{caption}
\usepackage{url}

\usepackage{thmtools}
\usepackage{thm-restate}
\declaretheorem[name=Lemma]{lemma}

\usepackage{times,xcolor}
\usepackage{amsfonts,algorithmic}

\definecolor{Darkblue}{rgb}{0,0,0.4}
\definecolor{Brown}{cmyk}{0,0.61,1.,0.60}
\definecolor{Purple}{cmyk}{0.45,0.86,0,0}

\usepackage[colorlinks,linkcolor=Darkblue,filecolor=blue,citecolor=blue,urlcolor=Darkblue,pagebackref,pdfstartview=FitH]{hyperref}
\usepackage[nameinlink]{cleveref}


\newtheorem{theorem}{Theorem}
\newtheorem{corollary}{Corollary}
\newtheorem{remark}{Remark}
\newtheorem{claim}{Claim}
\newtheorem{proposition}[lemma]{Proposition}
\newtheorem{definition}{Definition}
\newtheorem{observation}[lemma]{Observation}
\newtheorem{conjecture}{Conjecture}
\newtheorem{fact}{Fact}
\newtheorem{question}{Question}

\newcommand{\exopt}{{\rm exOpt}}
\newcommand{\optl}[1]{{\rm Opt}}
\newcommand{\opts}[1]{{\rm Opt}}


\newcommand{\ddim}{{\rm ddim}}
\newcommand{\dist}{{\rm dist}}
\newcommand{\spandist}{\textrm{span-dist}}
\def\beginsmall#1{\vspace{-\parskip}\begin{#1}\itemsep-\parskip}
	\def\endsmall#1{\end{#1}\vspace{-\parskip}}
\newcommand{\avgdist}{{\rm avgdist}}
\newcommand{\distnorm}{{\rm distnorm}}
\newcommand{\distavg}{{\rm distavg}}
\newcommand{\cost}{{\rm cost}}
\newcommand{\BFS}{{\rm BFS}}
\newcommand{\first}{{\rm first}}
\newcommand{\tail}{{\rm tail}}
\newcommand{\midd}{{\rm middle}}
\newcommand{\lefts}{{\rm left}}
\newcommand{\rights}{{\rm right}}
\newcommand{\on}{{\rm on}}
\newcommand{\adv}{{\rm adv}}
\newcommand{\opt}{{\rm OPT}}
\newcommand{\bS}{{\bar{S}}}
\newcommand{\cut}{{\rm cut}}
\newcommand{\scaler}{{\rm scaler}}
\newcommand{\vol}{{\phi}}
\newcommand{\bvol}{{\phi^{*}}}
\newcommand{\xvol}{{\bar{\phi}}}
\newcommand{\G}{\hat{G}}
\newcommand{\PPhi}{\hat{\Phi}}
\newcommand{\lca}{{\rm lca}}
\newcommand{\prev}{{\rm prev}}
\newcommand{\str}{{\rm avg\ stretch}}
\newcommand{\avgstr}{{\rm avg-stretch}}
\newcommand{\diam}{{\rm diam}}
\newcommand{\rad}{{\rm rad}}
\newcommand{\cen}{{\rm center}}
\newcommand{\A}{\mathcal{A}}
\newcommand{\C}{\mathcal{C}}
\newcommand{\LL}{\mathcal{L}}
\newcommand{\M}{\mathcal{M}}
\newcommand{\N}{\mathcal{N}}
\newcommand{\PP}{\mathcal{P}}
\newcommand{\R}{\mathbb{R}}
\newcommand{\Y}{\mathcal{Y}}
\newcommand{\F}{\mathcal{F}}
\newcommand{\HH}{\mathcal{H}}
\newcommand{\ie}{\emph{i.e. }}
\newcommand{\etal}{\emph{et. al.}}
\newcommand{\hst}{{\rm HST}}
\newcommand{\eps}{\epsilon}
\newcommand{\rank}{{\rm rank}}
\newcommand{\comp}{{\rm comp}}
\newcommand{\TS}{{\rm TS}}
\newcommand{\ind}{{\rm index}}
\newcommand{\E}{{\mathbb{E}}}
\renewcommand{\Re}{\mathbb{R}}
\newcommand{\poly}{{\rm poly}}
\newcommand{\classeqv}[2]{{{#1}^{\stackrel{#2}{\hookleftarrow}}}}
\newcommand{\ynote}[1]{\marginpar{Yair}\textsf{#1 --Yair}}
\newcommand{\mommit}[1]{}
\newcommand{\namedref}[2]{\hyperref[#2]{#1~\ref*{#2}}}
\newcommand{\sectionref}[1]{\namedref{Section}{#1}}
\newcommand{\appendixref}[1]{\namedref{Appendix}{#1}}
\newcommand{\subsectionref}[1]{\namedref{Subsection}{#1}}
\newcommand{\theoremref}[1]{\namedref{Theorem}{#1}}
\newcommand{\questionref}[1]{\namedref{Question}{#1}}
\newcommand{\defref}[1]{\namedref{Definition}{#1}}
\newcommand{\figureref}[1]{\namedref{Figure}{#1}}
\newcommand{\algref}[1]{\namedref{Algorithm}{#1}}
\newcommand{\claimref}[1]{\namedref{Claim}{#1}}
\newcommand{\lemmaref}[1]{\namedref{Lemma}{#1}}
\newcommand{\tableref}[1]{\namedref{Table}{#1}}
\newcommand{\propref}[1]{\namedref{Proposition}{#1}}
\newcommand{\conjref}[1]{\namedref{Conjecture}{#1}}
\newcommand{\obref}[1]{\namedref{Observation}{#1}}
\newcommand{\eqnref}[1]{\namedref{Equation}{#1}}
\newcommand{\factref}[1]{\namedref{Fact}{#1}}
\newcommand{\observationref}[1]{\namedref{Observation}{#1}}
\newcommand{\corollaryref}[1]{\namedref{Corollary}{#1}}
\newcommand{\propertyref}[1]{\namedref{Property}{#1}}

\newcommand {\ignore} [1] {}


\providecommand{\claimname}{Claim}
\providecommand{\conjecturename}{Conjecture}
\providecommand{\corollaryollaryname}{Corollary}
\providecommand{\definitionname}{Definition}
\providecommand{\factname}{Fact}
\providecommand{\lemmaname}{Lemma}
\providecommand{\remarkname}{Remark}
\providecommand{\theoremname}{Theorem}
\providecommand{\theoremname}{Theorem}
\usepackage{authblk}
\usepackage{pdfsync}
\newcommand{\logdiam}{\phi}

\begin{document}
\author[1]{Arnold Filtser}
\author[2]{Shay Solomon}
\affil[1]{Columbia University\thanks{The work was done while the author was affiliated with Ben-Gurion University of the Negev. Work supported in part by Simons Foundation, ISF grant 1817/17, and by BSF Grant 2015813.}. Email: \texttt{arnold273@gmail.com}}
\affil[2]{Tel Aviv University\thanks{Shay Solomon is partially supported by the Israel Science Foundation grant No.1991/19 and by Len Blavatnik and the Blavatnik Family foundation.
		Part of this work was carried out while this author was supported by the Rothschild Postdoctoral Fellowship, the Fulbright Postdoctoral Fellowship and the IBM Herman Goldstine Postdoctoral Fellowship.}. Email: \texttt{solo.shay@gmail.com}}


 

  \title{The Greedy Spanner is Existentially Optimal\footnote{A preliminary version of this paper appeared in PODC 2016 \cite{FS16}.}}
  \maketitle

\begin{abstract}
The greedy spanner is arguably the simplest and most well-studied spanner construction.
Experimental results demonstrate that it is at least as good as any other spanner construction, in terms of both the size and weight parameters.
However, a rigorous proof for this statement has remained elusive.

In this work we fill in the theoretical gap via a surprisingly simple observation: The greedy spanner is \emph{existentially optimal} (or existentially near-optimal)
for several important graph families, in terms of both the size and weight.
Roughly speaking, the greedy spanner is said to be existentially optimal (or near-optimal)
for a graph family  if the worst performance of the greedy spanner over all graphs in 
is just as good (or nearly as good) as the worst performance of an optimal spanner over all graphs in .

Focusing on the weight parameter, the state-of-the-art spanner constructions for both general graphs
(due to Chechik and Wulff-Nilsen [SODA'16]) and doubling metrics (due to Gottlieb [FOCS'15]) are complex. Plugging our observation on these results, we conclude that the greedy spanner achieves near-optimal weight guarantees for both general graphs and doubling metrics, thus resolving two longstanding conjectures in the area.

Further, we observe that approximate-greedy spanners are existentially near-optimal as well.
Consequently, we provide an -time construction of -spanners for doubling metrics with constant lightness and degree.
Our construction improves Gottlieb's construction, whose runtime is  and whose number of edges and degree are unbounded,
and remarkably, it matches the state-of-the-art Euclidean result (due to  Gudmundsson et al.\ [SICOMP'02]) in all the involved parameters
(up to dependencies on  and the dimension).
\end{abstract}
\section{Introduction}
\vspace{3pt}
{\bf 1.1~ Graph Spanners.~}
Given a (connected and undirected) -vertex -edge graph  with positive edge weights and a parameter , a subgraph  of 
() is called a \emph{-spanner} for  if for all , .
(Here  and  denote the distances between  and  in the graphs  and , respectively.)
The parameter  is called the \emph{stretch} of .\footnote{More accurately, the \emph{stretch} of  is the minimum number  such that  is a -spanner for , hence  is in fact an upper bound on the stretch of . However, referring to  as the stretch of  is a standard terminology in the area and is technically more convenient when the focus is existential bounds on spanner properties that depend on the stretch parameter, as in the current work.}
Spanners constitute a fundamental graph structure, and have been extensively and intensively studied since they were introduced \cite{PS89,PU89}.

In many practical applications one is required to construct a spanner that satisfies a number of useful properties, while preserving a small stretch.
First, the spanner  should have a small number of edges.
Second, its \emph{weight}  should be close to the weight of a minimum spanning tree (MST) of the graph .
We henceforth refer to the normalized notion of weight , which is called \emph{lightness};
a \emph{light} spanner is one with small lightness.
Third, its \emph{degree} , defined as the maximum number of edges incident on a vertex, should be small.

Light and sparse spanners are particularly useful for efficient broadcast protocols in the message-passing model of distributed computing \cite{ABP90,ABP91},
where efficiency is measured with respect to both the total communication cost (corresponding to the spanner's size and weight) and the speed of message delivery at all destinations (corresponding to the spanner's stretch).
Additional applications of such spanners in distributed systems include network synchronization and computing global functions \cite{Awerbuch85,PU89,ABP90,ABP91,Peleg00}. Light and sparse spanners were also found useful for various data gathering and dissemination tasks in overlay networks \cite{BKRCV02,VWFME03,KV01}, in wireless and sensor networks \cite{RW04,BDS04,SS10}, for VLSI circuit design \cite{CKRSW91,CKRSW292,CKRSW92,SCRS01},
for routing \cite{WCT02,PU89,PU89b,TZ01}, to compute distance oracles and labels \cite{Peleg00Prox,TZ01b,RTZ05}, and to compute almost shortest paths \cite{Coh98,RZ11,Elkin05,EZ06,FKMSZ05}. Low degree spanners are also very useful in many of these applications.
For example, the degree of the spanner is what determines local memory constraints when using
spanners to construct network synchronizers  and efficient broadcast protocols.
In compact routing schemes, the use of low degree spanners enables the routing tables to be of small size.
More generally, viewing vertices as processors, in many applications the degree of a processor
represents its \emph{load}, hence a low degree spanner guarantees that the load on all the processors in the network will be low.

The \emph{greedy spanner} by Althfer  et al.\ \cite{ADDJS93} is arguably the simplest and most well-studied spanner construction.    			
Althfer  et al.\ showed that for every  weighted -vertex graph  and an integer parameter ,
the \emph{greedy algorithm} (see \algref{fig:greedy_alg}) constructs a -spanner with  edges;
assuming Erd\H{o}s' girth conjecture \cite{erd64}, this size bound is asymptotically tight.
Althfer  et al.\ also showed that the lightness of the greedy spanner is .
Chandra et al.\ \cite{CDNS92} improved the lightness bound, and showed that the greedy spanner for stretch parameter 
(here , ) has lightness .
Two decades later, Elkin, Neiman and the second author \cite{ENS14} improved the analysis of \cite{CDNS92} and showed that the greedy -spanner has lightness .
In a very recent breakthrough, Chechik and Wulff-Nilsen \cite{CW18} improved the lightness bound all the way to .
Assuming Erd\H{o}s' girth conjecture \cite{erd64} and ignoring dependencies on , the bound of \cite{CW18} on the lightness is asymptotically tight,
thus resolving a major open question in this area. However, the result of Chechik and Wulff-Nilsen \cite{CW18}
is not due to a refined analysis of the greedy spanner. Instead, they devised a different construction, which is far more complex, and
bounded the lightness of their own construction. The following question was left open.
\begin{question} \label{question1}
	Is the lightness analysis of \cite{ENS14} for the greedy spanner optimal, or can one refine it to derive a stronger bound?
	In particular, is the spanner of \cite{CW18} lighter than the greedy spanner?
\end{question}
\vspace{2pt}
\noindent{\bf 1.2~ Spanners for Euclidean and Doubling Metrics.~}
Consider a set  of  points in , , and a stretch parameter .
A graph  in which the weight  of each edge 
is equal to the Euclidean distance  between  and  is called a \emph{Euclidean graph}.
We say that the Euclidean graph  is a \emph{-spanner} for  (or equivalently, for the corresponding Euclidean metric ) if for every pair  of distinct points, there exists a path  in 
between  and  whose weight (i.e., the sum of all edge weights in it) is at most .
The path  is said to be a \emph{-spanner path} between  and .
For Euclidean metrics, one usually focuses on the regime , for  being an arbitrarily small parameter.
Euclidean spanners were introduced by Chew \cite{Chew86}, and were subject to intensive ongoing research efforts since then.
We refer to the book ``Geometric Spanner Networks'' \cite{NS07}, which is devoted almost exclusively to Euclidean spanners and their numerous applications.
As with general graphs, it is important to devise Euclidean spanners that achieve small size, lightness and degree.

The \emph{doubling dimension} of a metric space  is the smallest value 
such that every ball  in the metric space can be covered by at most
 balls of half the radius of .
This notion generalizes the Euclidean dimension, since the doubling dimension
of the Euclidean space  is .
A metric space is called \emph{doubling} if its doubling dimension is constant.
Spanners for doubling metrics were also subject of intensive research  \cite{GGN04,CGMZ16,CG09,HM06,Roditty12,GR081,GR082,Smid09,ES15,Sol14}.
The basic line of work in this context is to generalize the known Euclidean spanner results for arbitrary doubling metrics.

Das et al.\ \cite{DHN93} showed that, in low-dimensional Euclidean metrics , the greedy -spanner has constant degree (and so  edges)
and  lightness.
In -point doubling metrics, the greedy -spanner has  edges and lightness  \cite{Smid09}.
As for the degree, there exist -point metric spaces with doubling dimension 1 for which the greedy spanner has a degree of  \cite{HM06,Smid09}.
It has been a major open question to determine whether any doubling metric admits
a -spanner  with sub-logarithmic lightness.
A breakthrough paper of Gottlieb \cite{Got15} answered this fundamental question in the affirmative by devising such a spanner construction with constant lightness.
Again, this result is not due to a refined analysis of the greedy spanner. Instead, Gottlieb devised a different construction,
which is far more complex, and bounded the lightness of his own construction. The following question was left open.
\begin{question} \label{question2}
	Is the lightness analysis of \cite{Smid09} for the greedy spanner optimal, or can one refine it to derive a stronger bound?
	In particular, is the spanner of \cite{Got15} lighter than the greedy spanner?
\end{question}

The relatively high runtime of the greedy spanner is a drawback.
The state-of-the-art implementation of the greedy spanner in both Euclidean and doubling metrics requires time  \cite{BCFMS10} (although there are some heuristics that might be useful in practice \cite{ABBB15,ABBB17}).
Building on \cite{DHN93}, Das and Narasimhan \cite{DN97} devised a much faster algorithm that follows the greedy approach.
The runtime of their ``approximate-greedy'' algorithm is , yet its degree and lightness are both bounded by constants (as with the greedy spanner).
Gudmundsson et al.\ \cite{GLN02} improved the result of \cite{DN97}, implementing the approximate-greedy algorithm within time .
For doubling metrics, however, the only spanner construction with sub-logarithmic lightness is that of   \cite{Got15};
the runtime of Gottlieb's construction is  rather than , and the size and degree of his construction are unbounded.
Hence, there is a big gap in this context between Euclidean and doubling metrics, leading to the following question.
\begin{question} \label{question3}
	Can one compute -spanners with constant lightness in doubling metrics  within time ?
	Furthermore, can one extend the state-of-the-art Euclidean result of \cite{GLN02} to arbitrary doubling metrics?
\end{question}	


There have been numerous experimental studies on Euclidean spanners. (See \cite{FG05,Far08}, and the references therein.)
The conclusion emerging from these experiments is that the greedy Euclidean spanner outperforms the other popular Euclidean spanner constructions, with respect to the
size and lightness bounds. (Specifically, the greedy spanner was found to be  times sparser and  times lighter than any other examined spanner.)
It is reasonable to assume that a similar situation occurs in arbitrary doubling metrics.
\vspace{6pt}
\\
\noindent{\bf 1.3~  Our Contribution.~}
In this work we fill in the theoretical gap by making three important observations.
\begin{enumerate}
	\item Our first observation is surprisingly simple: The greedy spanner is \emph{existentially optimal} with respect to both the size and the lightness,
	for any graph family  that is closed under edge removal.
	
	\vspace{5pt}	
	Applying this observation to the family of general weighted graphs, we conclude that the greedy spanner is just as light as the spanner of \cite{CW18}, thus answering \questionref{question1}.

	\vspace{5pt}
	Moreover, it is known that the greedy spanner 
	can be easily implemented within time  (cf.\ \cite{ES16}),
	and is thus much faster than the complex algorithm of  \cite{CW18}.
	(Although the runtime of the algorithm of \cite{CW18} is not analyzed explicitly therein,
	a naive implementation of that algorithm,
	which involves diameter computations of carefully selected subgraphs, incurs a runtime of .)
	We remark that all faster spanner constructions (e.g. \cite{BS07,ES16,MPVX15,EN17,ADFSW19}) achieve a worse lightness bound than that of the greedy spanner.
	Consequently, the greedy algorithm enjoys the fastest known runtime of any -spanner with  edges and  lightness ,
	or in other words, it is the fastest algorithm for constructing spanners that are near-optimal with respect to all the involved parameters (stretch, size and lightness).
	\vspace{6pt}
	\item The first observation does not hold for doubling metrics.
	Our second observation is that the greedy spanner is existentially \emph{near}-optimal with respect to both the size and the lightness,
	for the family of doubling metrics.
	In particular, it is just as light as the spanner of \cite{Got15}, thus answering \questionref{question2}.
	\vspace{6pt}
	\item Our third observation concerns the optimality of the approximate-greedy algorithm of \cite{DN97,GLN02} in doubling metrics, and is more intricate than the first two observations.
	Informally, it states that the approximate-greedy spanner with stretch parameter  is existentially near-optimal with respect to the lightness,
	for the family of doubling metrics, but when compared to spanners with a slightly smaller stretch parameter .
	This enables us to conclude that the   lightness of the approximate-greedy spanner is close to that of \cite{Got15}.
	In this way we manage to extend the state-of-the-art Euclidean result of \cite{GLN02} to arbitrary doubling metrics, thus answering
	\questionref{question3}.\footnote{The  runtime bound of \cite{GLN02} holds in the traditional algebraic computation-tree model with
		the added power of indirect addressing. Our result applies with respect to the same computation model.}
	
	\vspace{5pt}	
	While our approximate-greedy spanner achieves the same lightness bound as that of Gottlieb's spanner \cite{Got15}, it has several important advantages over it.
First, our algorithm is conceptually much simpler than that of \cite{Got15}.
	Second, our spanner construction has constant degree (ignoring dependencies on  and on the doubling dimension), while the degree in \cite{Got15} is not analyzed, and naively it may be as large as .
	Third, the degree bound of our spanner implies that it has  edges, while the size in \cite{Got15} is not analyzed,
	and it may be significantly larger than the size of our spanner.
Finally, the construction time of \cite{Got15} is , while our construction time is , which is considered the holy grail in the area of Euclidean and doubling spanners.
	
	\vspace{5pt}	
	Our third observation concerning the existential near-optimality of approximate-greedy spanner algorithms
	can be viewed as a \emph{general paradigm} for obtaining \emph{fast} spanner constructions that are both sparse and light.
	Although we apply it in this paper only to the family of doubling metrics, the same paradigm can be applied to other families of graphs, including general graphs, by adjusting it appropriately. In fact, the very recent work of Alstrup et. al. \cite{ADFSW19} follows this paradigm to obtain, for general graphs,
	either a -spanner with  edges and  lightness
	in  time
	or an -spanner with the same size and lightness bounds in  time, where  is a small constant.
\end{enumerate}


To summarize, by introducing and studying a new notion of optimality, \emph{existential (near-)optimality},
this paper provides an extremely simple yet powerful tool in the area of spanners.
We believe that the notion of existential optimality, defined formally below, is of fundamental importance, and we anticipate that it will find more applications, even outside the area of spanners.
\vspace{7pt}
\\
\noindent
{\bf Some definitions concerning existential optimality.~}
Although the meaning of \emph{existential optimality} can be   understood from the context, it is instructive to provide a formal definition.
Fix an arbitrary stretch parameter  and some graph family .
For a graph , let  (respectively, )
denote the optimal size (resp., lightness) of any -spanner for ,
and let 
(resp., )
denote the maximum value  (resp., ) over all graphs  in .
The greedy -spanner is said to be \emph{existentially optimal} with respect to the size (respectively, lightness) if
for any graph , the size (resp., lightness) of the greedy -spanner for  does not exceed   (resp., ).
This does not mean that the size (respectively, lightness) of the greedy  -spanner
for any graph  is bounded by  (resp., ).
It simply means that there \emph{exists} a graph ,
such that the size (resp., lightness) of the greedy -spanner for  is bounded by  (resp., ).
In other words, the maximum size (resp., lightness) of the greedy -spanner over all graphs in 
is equal to the maximum size (resp., lightness) of an optimal -spanner over all graphs in .

{For example, let  be the family of general weighted graphs on  vertices, and let  be an -vertex dense graph of high girth,
	namely, with girth  and  edges, where all edge weights are 1.
	Also, let  be a star on the same vertex set as  rooted at an arbitrary vertex,
	so that all edges of  that belong to  have weight 1 and all edges of  that do not belong to  have weight .
	Finally, let  be the graph containing all edges of  and all edges of  with weight .
	Note that the greedy -spanner for  includes all  edges of the high girth graph ,
	whereas the optimal -spanner (assuming ) consists of the edges of the star , hence is much sparser and lighter.
	(See \figureref{fig:girth} for an illustration.)
	This example, however, does not contradict the existential optimality of the greedy spanner: Although
	the size (respectively, lightness) of the greedy -spanner for  exceeds  (resp., ), it can be shown that it is equal to
	 (resp., ), which, in turn, is bounded by   (resp., ).}
\begin{figure}
	\begin{center}
		\includegraphics[width=0.41\textwidth]{peterson}
		\caption{\small  The graph  in the figure is the Petersen graph on 10 vertices, with girth 5 and 15 edges.
			All edges of  have weight 1, and are colored black.
			The red dashed edges are the edges of the star  of weight .
			The graph  is obtained as the union of the black and red edges in the figure.
			The greedy 3-spanner for  includes all 15 edges of , whereas the optimal 3-spanner for  consists of the 9 edges of .}				
		\label{fig:girth}
	\end{center}
\end{figure}


The meaning of \emph{existential near-optimality} is similar, except that we are allowed to have some \emph{slack},
which may depend on the stretch parameter  as well as on parameters of the graph family of interest .
As mentioned, in our third observation we compare the lightness of the greedy spanner with a certain stretch parameter 
to the optimal lightness of any spanner, but with a slightly smaller stretch parameter .
This is just one example of how the slack parameter can be used.
Another example is to compare the greedy spanner in some graph family  to an optimal spanner, but with respect to a different (closely related) graph family .
In particular, in our second and third observations we compare the lightness of the greedy spanner in metric spaces of bounded doubling dimension
to the optimal lightness of any spanner, but with respect to metric spaces of slightly larger doubling dimension.
It would be interesting to study additional ways of using the slack parameter,
as they may lead to new results in this area.

We remark that light spanners were extensively studied in various graph families such as planar graphs \cite{ADDJS93,Klein05}, apex graphs \cite{GS02},
bounded pathwidth graphs \cite{GH12}, bounded genus graphs \cite{Grigni00,GS02,DHM10}, bounded treewidth graphs \cite{DHM10},
and graphs excluding fixed minors \cite{Grigni00,DHM10,BLW17Minor}.
Since all these graph families are closed under edge removal, our first observation implies that the greedy spanner for them is just as good as any other spanner.
\vspace{6pt}
\\
\noindent{\bf 1.4~  Subsequent work.~}
The preliminary version of this paper appeared in PODC 2016 \cite{FS16}, and it triggered further work in the area.
We focus here on the two most relevant follow-up papers \cite{BLW19,LS19}. 

Borradaile, Le and Wulff{-}Nilsen \cite{BLW19}, improving over  \cite{Got15}, presented a construction of -spanners for doubling metrics with lightness 
; the improvement is in the dependence of the lightness bound on  and  (see Section \ref{sebsec:DoublingPreliminaries} for the details).
Moreover, by building on our Theorem \ref{finish}, Borradaile et al.\  achieved an -time algorithm for constructing -spanners with lightness .

Le and the second author \cite{LS19} studied the greedy spanner in -dimensional Euclidean metrics and determined the exact asymptotic dependencies on  and  for 
both the size and lightness bounds for any  (disregarding polylogarithmic factors of ):  edges and lightness .
Moreover, Le and Solomon \cite{LS19} showed that Steiner points lead to a quadratic improvement in the size of Euclidean spanners.
\vspace{6pt}
\\
\noindent{\bf 1.5~  Organization.~}
In \sectionref{sec:pre} we present the notation that is used throughout the paper, and summarize some statements from previous work that are most relevant to us.
In \sectionref{sec:Greedy_optimal} we show that the greedy spanner is existentially optimal for graph families that are closed under edge removal.
The basic optimality argument of \sectionref{sec:Greedy_optimal} is extended to doubling metrics in \sectionref{sec:doubling}.
Finally, in \sectionref{sec:Fast_alg} we show that the approximate-greedy spanner in doubling metrics is light.

\section{Preliminaries}\label{sec:pre}
All the graphs considered in the paper are connected, undirected and with positive edge weights. Let  be such a graph.
The weight  of a path  is the sum of all edge weights in it, i.e., .
For a pair of vertices , let  denote the distance between  and  in , i.e., the weight of a shortest path between them.
We denote by  the (shortest path) metric space \emph{induced} by ; we will view  as a complete weighted graph  over the vertex set ,
where the weight  of an edge  is given by the graph distance  between its endpoints.
A subgraph  of  (where ) is called a \emph{t-spanner} for  if for all ,  .
The parameter  is called the \emph{stretch} of the spanner .
If  for all edges , then
it also holds that  for all pairs of vertices .
Therefore, to bound the stretch of the spanner, one may restrict the attention to the edges of the graph.
Let  denote the \emph{size} of , and let  denote its \emph{weight}.
The \emph{lightness}  of  is the ratio between the weight of  and the weight of an MST for , i.e., .
(Throughout the paper all logarithms are in base 2.)

We refer the reader to Section 1.3 for some definitions concerning the notion of existential optimality.



The result of  Chechik and Wulff{-}Nilsen \cite{CW18} is summarized in the following theorem.
\begin{theorem}[\cite{CW18}]\label{thm:CW18}
	For every weighted -vertex graph  and parameters  and , there exists a
	-spanner with  edges and lightness . Such a spanner can be constructed in polynomial time.
\end{theorem}	

\subsection{Doubling metrics}\label{sebsec:DoublingPreliminaries}
Most statements in this section will be used in our construction and analysis of spanners for doubling metrics. 

We start with the
standard packing property in doubling metrics (see, e.g.,  \cite{GKL03}).
\begin{lemma} \label{lem:doubling_packing}
	Let  be a metric space  with doubling dimension .
	If  is a subset of points with minimum interpoint distance 
	that is contained in a ball of radius , then
	.
\end{lemma}



The following theorem states that any doubling metric admits a constant degree -spanner. This theorem will be useful in answering Question \ref{question3}, and more specifically for achieving the degree bound required for extending the state-of-the-art Euclidean result of \cite{GLN02} to arbitrary doubling metrics (see \theoremref{finish}).


\begin{theorem}[\cite{CGMZ16,GR08}]\label{thm:doubling_degree}
	For any -point metric  space  with doubling dimension
	 and parameter , there exists a -spanner with degree .
	The runtime of this construction is .
\end{theorem}


The result of Smid \cite{Smid09} is summarized in the following theorem. We provide this theorem only for the sake of comparison with our new result (see \corollaryref{cor:Greedy_G}),
which improves the logarithmic lightness bound provided by this theorem to constant.
\begin{theorem}[\cite{Smid09}]\label{thm:smid09}
	For any -point metric space  with doubling dimension  and any parameter ,  the greedy -spanner has
	  edges and lightness .
\end{theorem}

The result of Gottlieb \cite{Got15} is summarized in the following theorem. We will use this theorem for answering Questions \ref{question2} and \ref{question3}.
\begin{theorem}[\cite{Got15}]\label{thm:Got15}
	For any -point metric space  with doubling dimension  and parameter ,
	there exists a -spanner  with lightness .
	The runtime of this construction is .
\end{theorem}
{\bf Remark.} Recently, Borradaile, Le and Wulff{-}Nilsen \cite{BLW19} showed that the greedy -spanner has lightness   
in doubling metrics, improving over the lightness bound provided by Theorem \ref{thm:Got15}. 

\subsection{The Greedy Spanner and its Basic Properties}
\begin{algorithm}
	\caption{}\label{fig:greedy_alg}
	\begin{algorithmic}[1]
		\STATE .
		\FOR {each edge , in non-decreasing order of weight,}
		\IF {}
		\STATE Add the edge  to .
		\ENDIF
		\ENDFOR
	\end{algorithmic}
\end{algorithm}
The greedy spanner algorithm is presented in \algref{fig:greedy_alg}.
Let  be the output of an arbitrary execution of the greedy algorithm with stretch parameter .
It is immediate that  has stretch at most .
If the edge weights in the graph are distinct, then  is uniquely defined, but this does not hold in general;
nevertheless, by letting  designate an arbitrary such spanner, we may henceforth refer to it as the \emph{greedy -spanner}.
The following observation is immediate (see, .e.g., \cite{ENS14,CW18}).
\begin{observation}
	\label{fct:greedy contains MST}
	 contains all   edges of some MST of , denoted . (Hence  is also an MST of .)
\end{observation}



\section{The Basic Optimality Proof}\label{sec:Greedy_optimal}
In this section we show that the greedy spanner is existentially optimal, with respect to both the size and the lightness, for any graph family that is closed under edge removal.

We start by making the basic observation that the only -spanner of the greedy -spanner is itself.
\begin{lemma}
	\label{obs:spanner_of_greedy}
	Let  be any weighted graph, let  be any stretch parameter, and let  be the greedy -spanner of .
	If  is a -spanner for , then .
\end{lemma}
\begin{proof}
	Assume for contradiction that  is a -spanner for  yet there is an edge .
	Let  be a shortest path in  between the endpoints of . As  is a -spanner of ,
	it holds that .
	Consider the last edge examined by the greedy algorithm among the edges of  and , denoted .
	By the description of the greedy algorithm, we have .
	Consequently, by the time the greedy algorithm examines edge , all the edges of the path  must have already been added to the greedy spanner.
	(See \figureref{fig:lem3} for an illustration.)
	This path connects the endpoints of , and its weight is given by 
	Hence the greedy algorithm will not add edge  to , a contradiction.
	\begin{figure}
		\begin{center}
			\includegraphics[width=0.41\textwidth]{fig3}
			\caption{\small The path  in  between the endpoints of edge  is depicted by a dashed line.
				The path  between the endpoints of edge , all edges of which have been added to  by the time the greedy algorithm examines edge ,
				is colored red.}
			\label{fig:lem3}
		\end{center}
	\end{figure}
\end{proof}



Equipped with \lemmaref{obs:spanner_of_greedy}, we now turn to the basic optimality proof.
\begin{theorem}[Greedy is existentially optimal]\label{thm:main}
	Let  be any family of -vertex graphs that is closed under edge
	removal, and let  be any stretch parameter.
	For every graph ,	the greedy -spanner  of  has at most  edges and lightness at most .
	In other words, the greedy -spanner is existentially optimal for  with respect to both the size and the lightness.
\end{theorem}
\begin{proof}
	Consider an arbitrary graph  in , and let  be the greedy -spanner of .
	Since  is closed under edge removal and  is a subgraph of ,   belongs to .
	Hence, there exist -spanners  and  of  with at most   edges and lightness at most , respectively.
	\lemmaref{obs:spanner_of_greedy} implies that , from which the size bound on  immediately follows. 
	The lightness bound is slightly trickier, as the spanner  is computed on top of the greedy spanner  rather than the original graph .
	Nevertheless, \observationref{fct:greedy contains MST} implies that  and  have the same MST .
	Since the lightness of  is at most  and  is an MST for , it follows that
	
	Using the fact that , we conclude that the lightness of  satisfies
	
\end{proof}

As the family of weighted graphs is closed under edge removal, we can apply \theoremref{thm:main} on it.
Hence the greedy spanner for general graphs has size and lightness at least as good as in \theoremref{thm:CW18}.
\begin{corollary}\label{cor:Greedy_CW}
	For every weighted graph  on  vertices and  edges and parameters  and , the greedy
	-spanner has  edges and lightness .
	(A naive implementation of the greedy algorithm requires  time.)
\end{corollary}	

In \cite{BFN19} it was proved that for any parameter  and any stretch parameter ,
if every  -vertex weighted graph  admits a -spanner with at most  edges and lightness  at most , then for every such graph
there also exists a -spanner with at most  edges and lightness at most  .
Applying \theoremref{thm:main} again, we derive the following result.
\begin{corollary}\label{cor:light_Greedy}
	For every weighted -vertex graph  and parameter ,
	the greedy -spanner has  edges and lightness at most .
\end{corollary}	

As mentioned in the introduction, a plethora of graph families that are closed under edge removal were studied extensively in the spanner literature.
This includes the families of planar graphs, bounded genus graphs, bounded treewidth graphs, graphs excluding fixed minors, and more.
For all these graph families, \theoremref{thm:main} shows that the greedy spanner is existentially optimal.




\section{The Optimality Argument in Doubling Metrics}\label{sec:doubling}
The basic optimality argument of \sectionref{sec:Greedy_optimal} applies to graph families that are closed under edge removal.
Note that metric spaces do not fall into this category.
Nevertheless, for metric spaces, the basic optimality argument suffices: On the one hand, the upper bound for general weighted graphs applies to any metric space,
and on the other hand, the lower bound due to high girth graphs naturally applies to the induced metric spaces (see, e.g., \cite{ADDJS93,RR98}).

In this section we study the optimality of the greedy spanner for \emph{doubling metrics}.
For such metric spaces, one would like to obtain spanners with stretch , where  is arbitrarily close to 0.
We will show that the greedy -spanner is existentially near-optimal in doubling metrics, with respect to both the size and the lightness.
The next observation and subsequent lemma will be used for proving the lightness optimality.
\begin{observation} \label{mst-metric}
	Consider the metric space  induced by an arbitrary weighted graph .
	Then any MST of  is a spanning tree of . (Hence there is a common MST for  and , denoted .)
\end{observation}
\begin{proof}
	Consider an MST  for , and suppose for contradiction that  contains an edge  outside .
	Since  belongs to , any path in  between the endpoints of  consists of at least two edges.
	Consider the (multi) graph obtained from  by replacing edge  with a shortest path in  between the endpoints of .
	It is a spanning subgraph of  of weight , which contains at least  edges (some of which may be multiple edges), and thus at least one cycle. By breaking cycles in this subgraph,
	we obtain a spanning tree of  of weight strictly smaller than , yielding a contradiction to the weight minimality of .
\end{proof}


\begin{lemma} \label{lightopt}
	Let  be any metric space,  be some stretch parameter, and  be the greedy -spanner of .
	For every -spanner  of the metric space  induced by , we have .
\end{lemma}
\begin{proof}
	Let  be a -spanner of , and define  as the subgraph of  obtained from  by replacing each edge  of  with a shortest path in  between
	the endpoints of .
	Clearly, the distances in  are no greater than the respective distances in .
	Since  is a subgraph of , it follows that  is a -spanner for .
	\lemmaref{obs:spanner_of_greedy} implies that .
	Finally, noting that , we have .
\end{proof}


The following lemma will be used  for proving the size optimality.
\begin{lemma}\label{lem:metric_spars}
	Let  be any metric space,  be some stretch parameter, and  be the greedy -spanner of .
	For every -spanner  of the metric space  induced by , we have .
\end{lemma}
\begin{proof}
	Denote by  the weight function of , i.e., for any edge , ,
	and for any path , .
	Similarly, denote by  the weight function of , i.e., for any , ,
	and for any path , .
	For every edge , let  be a shortest path between the endpoints of  in ; by definition, we have .
	We say that edge  \emph{covers} all edges of , and symmetrically, all edges of  are \emph{covered} by .
	(An edge  covers itself.)
	
	For each edge  in , let  be a shortest path
	between the endpoints of  in . Since  is a -spanner for , we have .
	Observe that the edges in  form a  path  in  between the endpoints of .
	(It will be shown next that the path  is not simple.)
	We have
	
	
	
	Next, we argue that the edge  must belong to .
	Indeed, otherwise the edges of
	 contain a simple path in  between the endpoints of  of weight bounded by ,
	implying that the heaviest edge among the edges of this path and  would not be added to the greedy -spanner .
	Consequently, at least one edge  in  must cover .
	
	We define an injection  as follows. For each edge ,  is defined as ; in this case edge  covers itself.
	For each ,  is defined to be an arbitrary edge of  that covers .
	To see that  is injective, suppose for contradiction the existence of two distinct edges  and  in 
	and an edge 	such that . It must hold that  and  are in .
	Assume without loss of generality that .
	Since both  and  are covered by , it follows that .
	On the other hand, by the definition of , the shortest path  in  between the endpoints of  contains the edge .
	Hence the weight of a shortest path in  between the endpoints of  is given by , which contradicts	the fact that  is a -spanner for . It follows that  is injective, from which we conclude that  .
\end{proof}

Let  denote the family of -point metric spaces with doubling dimension bounded by ,
for any  and .
The following observation shows that a small ``stretching'' of any metric space does not change the doubling dimension of the metric space by much.
\begin{observation}\label{lem:doubling_embedding}
	Let  be a -spanner of an arbitrary metric space , for .
	Then the metric space  induced by  belongs to .
\end{observation}
\begin{proof}
	Clearly, any ball of radius  in the ``stretched'' metric space  is contained in the respective ball of the original metric space .
	By definition, this ball can be covered by  balls of radius  in ,
	and so by  balls of radius  in the stretched metric space .
\end{proof}	

The existential near-optimality result for doubling metrics is summarized in the following theorem.
\begin{theorem}[Greedy is near-optimal in doubling metrics]\label{theorem:doubling_light}
	For every metric  and any stretch parameter , the greedy -spanner  of 
	has at most  edges and lightness at most .
\end{theorem}
\begin{proof}
	Let  be an arbitrary metric space in , let  be
	the greedy -spanner for ,	and let  be the metric space induced by .	
	By \observationref{lem:doubling_embedding}, .
	Hence, there exist -spanners  and  for  with at most 
	edges and lightness at most , respectively.
	\lemmaref{lem:metric_spars} implies that  , from which the size bound on  immediately follows.
	As for the lightness bound on , note that  is computed on top of  rather than the original metric space .
	Nevertheless, \observationref{fct:greedy contains MST} and \observationref{mst-metric} imply that  and  have the same MST .
	Since the lightness of  is at most , it follows that
	
	hence .
	By \lemmaref{lightopt}, we have , hence the lightness of  satisfies
	
\end{proof}


By \theoremref{thm:smid09}, the greedy -spanner for -point doubling metrics
has  edges and lightness ,
where the -notation hides a multiplicative term of .
Applying \theoremref{theorem:doubling_light} in conjunction with  \theoremref{thm:Got15}, we reduce the lightness bound of the greedy -spanner to constant.
\begin{corollary}\label{cor:Greedy_G}
	For every metric space   in  and any parameter ,  the greedy -spanner has
	  edges and lightness .
\end{corollary}
\noindent {\bf Remark.}
\corollaryref{cor:Greedy_G} shows that the greedy -spanner in doubling metrics achieves optimal bounds on the size and the lightness, \emph{disregarding dependencies on  and the doubling dimension}.
However, improving these dependencies is a fundamental challenge of practical importance.
By \theoremref{theorem:doubling_light}, any improvement whatsoever in the dependencies on  and the doubling dimension
on either the size or the lightness of \emph{any spanner construction} for doubling metrics -- would trigger a similar improvement to the greedy spanner.
Note that the recent result of Borradaile \etal ~\cite{BLW19} shows that the greedy -spanner has lightness  



\section{The Approximate-Greedy Spanner in Doubling Metrics is Light}\label{sec:Fast_alg}
\corollaryref{cor:Greedy_G} shows that the greedy -spanner in doubling metrics achieves near-optimal bounds on the size and the lightness.
Nevertheless, this spanner has two major disadvantages.
First, as mentioned in the introduction, there exist metric spaces with doubling dimension 1 for which its degree may be unbounded.
(This is in contrast to -dimensional Euclidean metrics, where the greedy -spanner has degree .)
Second, it cannot be constructed within sub-quadratic time in doubling metrics due to a lower bound of \cite{HM06}.
In fact, even in -dimensional Euclidean metrics, the state-of-the-art implementation of the greedy -spanner requires time  \cite{BCFMS10}.

\sloppy Building on \cite{DHN93,DN97}, Gudmundsson et al.\ \cite{GLN02} devised a much faster algorithm that follows the greedy approach,
hereafter Algorithm \texttt{Approximate-Greedy}.
The runtime of this algorithm is , yet the degree and lightness of the approximate-greedy spanner produced by the algorithm are both bounded by , just as with the greedy spanner for Euclidean metrics.
The runtime analysis of Algorithm \texttt{Approximate-Greedy} \cite{GLN02} does not exploit any properties of Euclidean geometry.
Specifically, it relies on the triangle inequality, which applies to arbitrary metric spaces, and on standard packing arguments (cf.\ \lemmaref{lem:doubling_packing}), which apply to arbitrary doubling metrics.
Therefore, the runtime of Algorithm \texttt{Approximate-Greedy} remains  in arbitrary doubling metrics.
Moreover, the degree bound of   applies to arbitrary doubling metrics as well.
(We refer to Chapter 15 in \cite{NS07} for an excellent description of this algorithm and its analysis.)

In this section we show that the approximate-greedy spanner of \cite{GLN02} has constant lightness in arbitrary doubling metrics.
Consequently, Algorithm  \texttt{Approximate-Greedy} provides an -time construction of -spanners in doubling metrics with lightness and degree both bounded by constants.

\subsection{A Rough Sketch of Algorithm \texttt{Approximate-Greedy}}
In this section we provide a  very rough sketch of Algorithm \texttt{Approximate-Greedy}, aiming to highlight the high-level ideas behind it.
This outline is not required for the analysis that is given in \sectionref{bounding}; it is provided here for clarity and completeness.

In metric spaces, the greedy algorithm sorts the  interpoint distances and examines the edges by non-decreasing order of weight.
For each edge that is examined for inclusion in the spanner, the distance between its endpoints in the current spanner is computed.
This is expensive for two reasons: (1) The number of examined interpoint distances is quadratic in .
(2) Computing the \emph{exact} spanner distance between two points is   costly.

Suppose we aim for a stretch of , and let  be an appropriate parameter satisfying .
(Refer to \cite{GLN02,NS07} for the exact constant hiding in the -notation of .)
Instead of examining all  interpoint distances, Algorithm \texttt{Approximate-Greedy} computes a
bounded degree -spanner  for the input metric space ,
and simulates the greedy algorithm with stretch parameter  only on the edges of .
The output of the algorithm is a -spanner  for ,
which is a -spanner for the original metric space  by the ``transitivity'' of spanners.
A spanner  of degree  can be constructed in  time via \theoremref{thm:doubling_degree}.
Since the output -spanner  for  is a subgraph of , its degree will be at most .

The greedy simulation is applied only on the edges of  that are sufficiently ``heavy''.
Formally, let  denote the maximum weight of any edge of the bounded degree spanner , and let  be the set of \emph{light edges} in , namely, of weight at most .
As , we have .
All light edges are taken to the output spanner , and the greedy simulation is applied only on the edges of .
(So the output spanner  will contain \emph{all} edges of  and \emph{some} edges of .)

As mentioned, computing the \emph{exact} distance between two points is costly;
using Dijkstra's algorithm, it requires  time (see, e.g., Section 2.5 and Corollary 2.5.10 in \cite{NS07}).
Since  has  edges, the overall runtime will be .
To speed up the computation time, Algorithm \texttt{Approximate-Greedy} does not compute the exact distance between two points, but rather an approximation of that distance.
This is achieved by maintaining a much simpler and coarser \emph{cluster graph} that approximates the original distances, on which the distance queries are performed.
More specifically, the algorithm partitions the edge set  into  buckets,  for an appropriate parameter ,
such that edge weights within each bucket differ by at most a factor of .
Then it examines the edges of  by going from one bucket to the next, examining edges by non-decreasing order of weight.
Whenever all edges of some bucket have been examined, the cluster graph is updated according to the new edges that were added to the spanner.
The idea is to periodically make the cluster graph simpler and coarser, so that the shortest path computations made on it will be fast.
The bottom-line is that one does not simulate the greedy algorithm (with stretch parameter ) on the edge set , but rather an \emph{approximate version} of it.


\subsection{Bounding the Lightness of the  \texttt{Approximate-Greedy} Spanner} \label{bounding}
As mentioned, the runtime of Algorithm \texttt{Approximate-Greedy} is   in arbitrary doubling metrics.
In what follows let  be the -spanner for   returned by Algorithm \texttt{Approximate-Greedy}.
Since  is a subgraph of the bounded degree spanner , its degree is .

It remains to bound the lightness of .
The lightness argument of \cite{GLN02}, which relies on previous works \cite{DHN93,DN97},
is based on rather deep properties from Euclidean geometry, most notably the \emph{leapfrog property}.
In particular, this argument does not apply to arbitrary doubling metrics.

Instead, we employ the following lemma, which lies at the heart of the lightness analysis of \cite{GLN02}.
While this lemma applies to arbitrary doubling metrics, the way it was used in \cite{GLN02} does not extend to arbitrary doubling metrics.
Specifically, it was used in \cite{GLN02} to show that the edge set  satisfies the leapfrog property.
(Recall that  is the set of light edges in , all of which are taken to the approximate-greedy spanner .)
In Euclidean metrics, it has been proved \cite{DHN93,NS07} that any edge set satisfying the leapfrog property has constant lightness, but this proof does not carry over to arbitrary doubling metrics.
\begin{lemma}[Lemma 17 in \cite{GLN02}]\label{lem:GLN_second_path}
	Let . The weight of the second shortest path between  and  in the approximate-greedy spanner  is greater than .
	(If there are multiple shortest paths between  and , then the weight of the second shortest path equals the weight of the shortest path.)
\end{lemma}
\noindent
{\bf Remark.} The parameter  in the statement of this lemma depends on the stretch parameters of the spanners  and 
that are constructed by Algorithm \texttt{Approximate-Greedy}.
Specifically, recall that the output spanner  is a -spanner for ,
which is, in turn, a -spanner for the input metric space .

We will use the following observation, due to \cite{Smid07}. We include a proof for completeness.
\begin{observation} [Lemma 1.7 in \cite{Smid07}] \label{mstsimple2}
	Let  be an arbitrary weighted graph, and let  be any stretch parameter.
	For any -spanner  of , .
\end{observation}
\begin{proof}
	Consider an MST  for .
	Replace each edge of  by a -spanner path in  between the endpoints of that edge, and then break cycles.
	The resulting structure  is a spanning tree of , hence , and we have
	.
\end{proof}

The following lemma bounds the lightness of .
Its proof is based on the somewhat surprising observation that the lightness of the -spanner  produced by Algorithm \texttt{Approximate-Greedy} is existentially near-optimal with respect to stretch parameter  (rather than ).
We remark that  is not a greedy spanner, but rather an approximate-greedy spanner,
and it is inherently different than the greedy -spanner and the greedy -spanner.
In particular, its weight may be larger than the weights of both these greedy spanners.
Nevertheless, our existential near-optimality argument suffices to derive the required lightness bound.
\begin{lemma}
	The lightness of  is .
\end{lemma}

\begin{proof}					
	Recall that  is a -spanner for , where ,  and let  be the metric space induced by .
	By \observationref{lem:doubling_embedding}, the doubling dimension of  is bounded by .
	Let  be a -spanner of  with lightness , where 
	is the parameter appearing in the statement of \lemmaref{lem:GLN_second_path}, which is optimized as part of Algorithm  \texttt{Approximate-Greedy}.
	As in the proof of \lemmaref{lightopt}, we transform  into a -spanner  of  of weight at most .
	By \observationref{mst-metric} and \observationref{mstsimple2}, the MST weights for all graphs  and  are the same, up to a factor of .
	
	
	We argue that every edge  belongs to .
	Suppose for contradiction that there is an edge  that does not belong to .
	Let  be a shortest path between the endpoints of  in . Since  is a -spanner of , we have .
	Note that this path is contained in . Since  and  is a metric space, the weight of the second shortest path between the endpoints of  in  is at most
	.
	On the other hand, By \lemmaref{lem:GLN_second_path}, the weight of this path is greater than , a contradiction.
	It follows that
	
\end{proof}

Setting  and  (for an appropriate constant ; see \cite{GLN02,NS07}), we conclude:
\begin{theorem}  \label{finish} \sloppy
	For any  metric  space  in  and parameter , Algorithm \texttt{Approximate-Greedy}
	returns a -spanner with lightness  and degree .
	The runtime of Algorithm \texttt{Approximate-Greedy} is .
\end{theorem}
\noindent{\bf Remark.}	\theoremref{finish} should be compared to \theoremref{thm:Got15} due to \cite{Got15}.
Both constructions achieve the same lightness bound, but the degree and number of edges in the spanner construction of \cite{Got15} are unbounded.
Moreover, the runtime of the construction of \cite{Got15} is , whereas that
of \theoremref{finish} is .
By combining the light spanner  of \cite{Got15} with a bounded degree spanner , one can obtain a spanner with constant degree and   lightness.
Specifically, such a spanner  is obtained by replacing each edge of  with a shortest (or approximately shortest) path in  between the endpoints of that edge.
The lightness of the resulting spanner  will not exceed that of  by much, whereas the degree bound  will follow from that of .
There is a major problem with this approach: The runtime needed for computing spanner  may be very high.
Indeed, although there are efficient ways to estimate the weight of an approximately shortest path in  between two points,
we must \emph{compute} the corresponding path in .
In particular, to achieve the degree bound of , one may not use edges outside .
Moreover, even regardless of this computation time, such a path may contain many edges that already belong to the gradually growing spanner .
Deciding which edges of this path should be added to  may be very costly by itself.\\
By the recent result of Borradaile \etal ~\cite{BLW19}, the lightness of the spanner construction provided by   \theoremref{finish} is reduced to .


\section{Acknowledgements}  We are grateful to Michael Elkin and Ofer Neiman for fruitful discussions.
The second-named author thanks L\'{a}szl\'{o} Babai for comments that helped improving the presentation of the paper.

\bibliographystyle{alpha}
\bibliography{bibShay,ENS14,ENS14V2}









\end{document}
