\section{Experiments}
\label{experiment}
We next provide experimental evidence to show that \method works will with various GNN architectures for both node classification and link prediction tasks in Sections~\ref{eval-nc} and  \ref{eval-lp}.
We also show that \method works with different training methods in Section~\ref{eval-train}. 
Finally, we discuss the impact of different \method settings in Sections~\ref{eval-mtl} and \ref{eval-dgnn}.

\subsection{Evaluation Setup}
\paragraph{Datasets} We conducted experiments on seven datasets, including ogbn-products, ogbn-arxiv and ogbn-proteins from ogbn~\cite{hu2020open} and reddit\footnote{http://snap.stanford.edu/graphsage/} for node classification, and ogbl-collab, ogbl-ppa and ogbl-citaiton2 from ogbl~\cite{hu2020open} for link prediction. The detailed statistics are summarized in Table~\ref{tab:dataset}.

\begin{table}[t]
\centering
\begin{threeparttable}
 \caption{Datasets statistics.}
\label{tab:dataset}
 \centering
\begin{tabular}{lcc}
 \toprule 
Datasets & \# Nodes & \# Edges \\
\midrule
\multicolumn{3}{c}{Node Classification} \\
\midrule
ogbn-products & 2,449,029 & 61,859,140 \\
ogbn-arxiv & 169,343 & 1,166,243 \\
ogbn-proteins & 132,524 & 39,561,252 \\
reddit & 232,965 & 114,615,892 \\
\midrule
\multicolumn{3}{c}{Link Prediction} \\
\midrule
ogbl-collab & 235,868 & 1,285,465 \\
ogbl-ppa & 576,289 & 30,326,273 \\
ogbl-ddi & 4,267 & 1,334,889 \\
\bottomrule
\end{tabular}
\end{threeparttable}
\end{table}

\begin{table*}
\centering
\begin{threeparttable}
 \caption{Baseline GNN models used in evaluation.}
\label{tab:models}
\begin{tabular}{ll}
 \toprule 
 GNN Model & Description \\
 \midrule
 \multicolumn{2}{c}{Node classification task} \\
 \midrule
GraphSage & Vanilla GraphSage with neighbor sampling. \\
GraphSage-Cluster & Vanilla GraphSage with cluster based sampling~\cite{chiang2019cluster}. \\
GAT-FLAG & GAT with FLAG~\cite{kong2020flag} enhancement. \\
GAT+BoT & GAT with bag of tricks~\cite{wang2021bag}. \\
AGDN+BoT & AGDN with bag of tricks. \\
AGDN+BoT+self-KD+C\&S & AGDN with bag of tricks, knowledge distillation and correct\&smooth\cite{huang2020combining}. \\
 \midrule
\multicolumn{2}{c}{Link prediction task} \\
 \midrule
SEAL-DGCNN & Vanilla SEAL using DGCNN~\cite{zhang2018end} as the backbone GNN.\\
GCN-full & Vanilla GCN with full graph training. \\
GraphSage-full & Vanilla Sage with full graph training. \\
GraphSage+EdgeAttr & GraphSage with edge attribute.\\
\bottomrule
\end{tabular}
\begin{tablenotes}
\scriptsize
\small
  \item GraphSage+EdgeAttr comes from https://github.com/lustoo/OGB\_link\_prediction 
\end{tablenotes}
\end{threeparttable}
\end{table*}


\begin{table*}[t]
 \centering
\begin{threeparttable}
 \caption{Performance of \method on ogbn-products, ogbn-arxiv, ogbn-proteins and reddit.}
\label{tab:node-class}
\begin{tabular}{lccccc}
 \toprule 
Dataset & & ogbn-products & ogbn-arxiv & ogbn-proteins & reddit \\
Eval Metric & &Accuracy(\%) & Accuracy(\%) & ROC-AUC(\%) & Accuracy(\%) \\
\midrule
\multirow{2}{*}{GraphSage} & Vanilla & 78.270.45   &71.151.66   &75.671.72   &96.190.08 \\
& \method & \textbf{79.880.34}   &  \textbf{71.771.18}   & \textbf{76.300.96}   &96.210.04 \\
\midrule
\multirow{2}{*}{GraphSage-Cluster} & Vanilla & 78.720.63 &   56.571.56  & 67.451.21   & 95.270.09 \\
& \method &   78.910.59 &   56.761.08 & \ \textbf{68.120.96}   &95.340.09 \\
\midrule
\multirow{2}{*}{GAT-NS} & Vanilla & 79.230.16   & 72.101.12   &81.760.17   & 96.120.02 \\
& \method &79.670.09   & 71.881.10   &81.910.21   & 96.450.05 \\
\midrule
\multirow{2}{*}{GAT-FLAG} & Vanilla &80.750.14   &71.561.11  &81.810.15   &95.270.02 \\
& \method &80.990.09   &71.741.10   &81.840.11   &95.680.03 \\

\bottomrule
\end{tabular}
\begin{tablenotes}
\scriptsize
\small
  \item Any score difference between vanilla GNN and \stmethod GNN that is greater than 0.5\% is highlighted with boldface. 
\end{tablenotes}
\end{threeparttable}
\end{table*}

\begin{table}[t]
 \centering
\begin{threeparttable}
 \caption{Performance (as measured by classification accuracy and ROC-AUC for ogbn-arxiv and ogbn-proteins, respectively) of \method combined with bag of tricks on ogbn-arxiv and ogbn-proteins.}
\label{tab:node-class-bot}
\begin{tabular}{lccc}
Dataset & Model &  & Accuracy(\%) \\
\midrule
ogbn-arxiv & \multirow{2}{*}{AGDN+BoT} & Vanilla & 74.030.15\\
ogbn-arxiv & & \method& 74.250.17 \\
\midrule
ogbn-arxiv & AGDN+BoT+ & Vanilla & 74.280.13 \\
ogbn-arxiv & self-KD+C\&S & \method & 74.340.14 \\
\midrule
ogbn-proteins &\multirow{2}{*}{GAT+BoT} & Vanilla & 87.730.18 \\
ogbn-proteins & & \method & 88.090.1 \\
 \toprule 
 \end{tabular}
\end{threeparttable}
\end{table}
\begin{table*}[t]
 \centering
\begin{threeparttable}
 \caption{Performance of \method on ogbl-collab, ogbl-ppa and ogbl-ppi. We use the hit@20, hit@50 and hit@100 as the evaluation metrics.}
\label{tab:link-prediction1}
 \centering
\begin{tabular}{lccccccc}
 \toprule 
& \multirow{2}{*}{Metric (\%)} & \multicolumn{2}{c}{ogbl-collab} & \multicolumn{2}{c}{ogbl-ppa} & \multicolumn{2}{c}{ogbl-ddi} \\
& & Vanilla GNN & \method & Vanilla GNN & \method & Vanilla GNN & \method \\
\midrule
\midrule
SEAL- & hit@20 &45.760.72 &46.190.58   &16.101.85 & \textbf{20.821.76}  & 30.752.12 & \textbf{31.933.00} \\
DGCNN& hit@50 &54.700.49 & 54.820.20  &32.581.42 & \textbf{37.250.98}  & 43.991.11 & 42.393.23 \\
& hit@100 &60.130.32  & 60.700.18  &49.361.24 & \textbf{56.440.99}  & 51.251.60 & 49.633.65 \\
\midrule
\multirow{3}{*}{GCN-full} & hit@10 & 35.941.60 &  36.690.82 & 4.001.46& \textbf{5.640.93}   & 47.82  5.90 & 48.22  7.00 \\
& hit@50 & 49.520.70 & \textbf{51.830.50}  & 14.231.81 & \textbf{18.441.88}  & 79.563.83 &  \textbf{82.564.03} \\
& hit@100 & 55.740.44 & \textbf{57.410.22} & 20.21 1.92  & \textbf{26.780.92}  & 87.581.33 &  \textbf{89.481.68} \\
\midrule
GraphSage- & hit@10 & 32.593.56  & \textbf{36.832.56}  & 3.681.02 & 3.521.24  & 54.279.86 & \textbf{60.754.94} \\
full& hit@50 & 51.660.35 & 52.621.04 & 15.021.69 & 15.551.92 & 82.184.00 & \textbf{84.581.89}\\
& hit@100 & 56.910.72 & \textbf{57.960.56} & 23.561.58 & 24.452.34 & 91.940.64 & 92.580.88\\
\midrule
GraphSage+ & hit@20 & -  & -  & - & -   & 87.064.81 & \textbf{93.281.61}\\
EdgeAttr & hit@50 & - & -  & -  & -  &  97.980.42 & 98.390.21\\
& hit@100 & - & - & - & -  & 98.980.16 & 99.210.08\\
\bottomrule
\end{tabular}
\begin{tablenotes}
\scriptsize
\small
  \item The evaluation metrics used in ogb learderboard are hit@50 for ogbl-collab, hit@100 for ogbl-ppa and hit@20 for ogbl-ddi.
  \item Any hit score difference between vanilla GNN and \method GNN that is greater than 1\% is highlighted with boldface. 
  \item The evaluation metrics used for ogbl-ddi when profiling GCN-full and GraphSage-full are hit@20, hit@50 and hit@100.
\end{tablenotes}
\end{threeparttable}
\end{table*}

We evaluated the effectiveness of \method by applying it to various GNN models including GCN~\cite{kipf2016semi}, Graphsage~\cite{hamilton2017inductive}, Graph Attention Network (GAT)~\cite{velivckovic2017graph}, Adaptive Graph Diffusion Networks (AGDN)~\cite{sun2020adaptive}, and SEAL~\cite{zhang2020revisiting} and their variants. Table~\ref{tab:models} presents all the baseline models. We directly followed the implementation and configuration of each baseline model from the OGB~\cite{hu2020open} leaderboard and added non-linear layer(s) into each GNN layer for \method. Table~\ref{tab:setting} presents the detail configuration of each model.
All models were trained on a single V100 GPU with 32GB memory. We report average performance over 10 runs for all models except SEAL related models. As training SEAL models is very expensive, we took 5 runs instead.


\subsection{Node classification} \label{eval-nc}
Firstly, we analyzed how \method improves the performance of GNN models on node classification tasks. Table~\ref{tab:node-class} presents the overall results.
It can be seen that \stmethod-based models outperform their baseline models in most of the cases. Notably, \method tends to performs well with GraphSage. It improves the test accuracy of GraphSage on ogbn-products and ogbn-arxiv by 1.61 and 0.62 respectively. It also improves the ROC-AUC score of GraphSage on ogbn-proteins by 0.63. 
But as the baseline performance of reddit dataset is quite high, not surprisingly, the overall improvement of \method is not significant.

We further analysis the performance of \method combined with bag of tricks~\cite{wang2021bag} on ogbn-arxiv and ogbn-proteins in Table~\ref{tab:node-class-bot}. It can be seen that \stmethod-based models outperform their vanilla counterparts. \method with AGDN+BoT+self-KD+C\&S even achieves the first place over all the methods with no extension to the input data on the ogbn-arxiv leaderboard as of the time of this submission (The forth place on the entire ogbn-arxiv leaderboard). 
\method with GAT+BoT also achieves the second place on the ogbn-proteins leaderboard with 5.83 times fewer parameters compared with the current leading method RevGNN-Wide.\footnote{\method-GAT+Bot has 11,740,552 parameters while RevGNN-Wide has 68,471,608 parameters.}


\iffalse
\begin{table}
 \caption{Performance of \method on ogbl-citation2. We use the mean reciprocal rank (MRR) as the evaluation metric. Due to GPU memory constrain, we use a hidden size of 128 instead of 256 when evaluating GCN and GraphSage models.}
\label{tab:link-prediction2}
 \centering
\begin{tabular}{lccc}
 \toprule 
 & Vanilla GNN & \method \\
\midrule
SEAL-DGCNN & 0.873 & \\
GCN-full & 0.838 & 0.8391 \\
GraphSage-full & 0.798 & 0.803 \\
\bottomrule
\end{tabular}
\end{table}
\fi
\subsection{Link prediction} \label{eval-lp}
Secondly, we analyzed how \method improves the performance of GNN models on link prediction tasks. Table~\ref{tab:link-prediction1} presents the results on the ogbl-collab, ogbl-ppa and ogbl-ppi datasets. 
As shown in the tables, the performance improvement of \method over SEAL models is significant.
\method improves the hit@20, hit@50 and hit@100 of SEAL-DGCNN by 4.72\%, 4.67\% and 7.08\% respectively on ogbl-ppa. \method with SEAL-DGCNN achieves the first place on the ogbn-ppa leaderboard with an improvement of hit@100 by 5.82\% over the current leading method  MLP+CN\&RA\&AA~\footnote{https://github.com/lustoo/OGB\_link\_prediction}.
Furthermore, \method with GraphSage+EdgeAttr achieves the first place on the ogbl-ddi leaderboard with an improvement of hit@20 by 5.47\% over the current leading method vanilla GraphSage+EdgeAttr. As GraphSage+EdgeAttr only provided the performance on ogbl-ppi, we do not compare its performance on other datasets.
\method also works with GCN and GraphSage on link prediction tasks. As shown in the tables, 
It improves the performance of GCN and GraphSage in all cases. In particular, it improves the hit@20, hit@50 and hit@100 of GCN by 1.64\%, 4.21\% and 6.57\% respectively on ogbl-ppa.



\begin{table}[t]
 \caption{Test accuracy (\%) of GraphSage and GAT with and without \method trained with different training methods on ogbn-products.}
\label{tab:node-train-method}
 \centering
\begin{tabular}{lccc}
 \toprule 
Sampling & \multirow{2}{*}{full-graph} & neighbor & cluster-based \\
Methods & & sampling & sampling\\
\midrule
GraphSage &78.27  &78.70  &78.72	 \\
GraphSage-\method &79.88  &79.11  &78.91 \\
\midrule
GAT &80.75 &79.23 &71.41 \\
GAT-\method &80.99 &79.67 &76.76\\
\bottomrule
\end{tabular}
\end{table}

\begin{table}[t]
 \caption{Test accuracy (\%) of GraphSage and GAT with different number of non-linear layers added into GNN layers on ogbn-products.}
\label{tab:node-class-ml}
 \centering
\begin{tabular}{lrrr}
 \toprule 
Model & \multicolumn{3}{c}{GraphSage}\\
Hidden-size & 128 & 256 & 512 \\
\midrule
baseline & 77.44 & 78.27 & 79.37	 \\
\method-1layer & 77.39 & 79.53 & 79.12  \\
\method-2layer & 78.79 & 79.88 & 79.94  \\
\method-4layer & 78.79 & 79.52 & 79.88  \\
\midrule
Model & \multicolumn{3}{c}{GAT} \\
Hidden-size & 64 & 128 & 256 \\
\midrule
baseline & 68.41  &79.23  &75.26	 \\
\method-1layer & 69.72  &79.67  &77.53	 \\
\method-2layer & 69.86  &78.26  &78.76	 \\
\method-4layer & 69.41  &78.23  &78.61	 \\
\bottomrule
\end{tabular}
\end{table}

\begin{table}
 \caption{The parameter size of each model in Table~\ref{tab:node-class-ml}.}
\label{tab:node-class-ml-size}
 \centering
\begin{tabular}{lrrr}
 \toprule 
Model & \multicolumn{3}{c}{GraphSage}\\
Hidden-size & 128 & 256 & 512 \\
\midrule
baseline & 70,703 & 206,895 & 675,887	 \\
\method-1layer & 87,215 & 272,687 & 938,543  \\
\method-2layer & 103,727 & 338,479 &  1,201,199 \\
\method-4layer & 136,751 & 470,063 & 1,726,511  \\
\midrule
Model & \multicolumn{3}{c}{GAT} \\
Hidden-size & 64 & 128 & 256 \\
\midrule
baseline & 510,056  & 1,543,272  & 5,182,568	 \\
\method-1layer & 514,152  & 1,559,656  & 5,248,104	 \\
\method-2layer & 518,248  & 1,576,040  & 5,313,640 \\
\method-4layer & 526,440  & 1,608,808 & 5,444,712 \\
\bottomrule
\end{tabular}
\end{table}

\begin{table}
 \caption{The single training epoch time of GraphSage with different number of non-linear layers added into GNN layers on ogbn-products.}
\label{tab:node-class-time}
 \centering
\begin{tabular}{lrrr}
 \toprule 
Model & \multicolumn{3}{c}{GraphSage (secs)}\\
Hidden-size & 128 & 256 & 512 \\
\midrule
baseline & 18.710.41 & 20.710.74 & 25.400.35	 \\
\method-1layer & 19.571.04 & 20.980.47 & 29.070.69  \\
\method-2layer & 19.250.87 & 21.360.48 & 30.010.13 \\
\method-4layer & 19.790.72 & 24.410.38 & 32.330.19  \\
\bottomrule
\end{tabular}
\end{table}

\iffalse
\begin{table}
 \caption{The training time cost efficiency per epoch.}
\label{tab:node-class-cost efficiency}
 \centering
\begin{tabular}{lrrr}
 \toprule 
Model & \multicolumn{3}{c}{GraphSage}\\
Hidden-size & 128 & 256 & 512 \\
\midrule
baseline & 2.65e-04 & 1.00e-04 & 3.76e-05	 \\
\method-1layer & 2.24e-04 & 7.70e-05 & 3.10e-05  \\
\method-2layer & 1.86e-04 & 6.31e-05 &  2.50e-05 \\
\method-4layer & 1.45e-04 & 5.19e-05 & 1.87e-05 \\
\bottomrule
\end{tabular}
\end{table}
\fi






\subsection{\method with Different Training Methods} \label{eval-train}
Finally, we presents the effectiveness of using \method with different training methods including full-graph training, neighbor sampling and cluster-based sampling. Table~\ref{tab:node-train-method} presents the results. It can be seen that \method improves the performance of GraphSage and GAT with all kinds of training methods on ogbn-products. 
It is worth mentioning that \method also works with local subgraph sampling method proposed by SEAL~\cite{zhang2018link} as shown in Section~\ref{eval-lp}.



\subsection{Effectiveness of Multiple \method Layers}
\label{eval-mtl}
We studied the effectiveness of adding multiple non-linear layers to GNN layers on ogbn-products using GraphSage and GAT. Table~\ref{tab:node-class-ml} presents the results. The baseline model is a three-layer GNN model. We applied 1, 2 or 4 non-linear layers to each hidden GNN layer denoted as \stmethod-1layer, \stmethod-2layer and \stmethod-4layer respectively. The GAT models use eight attention heads and all heads share the same NGNN layer(s).
Table~\ref{tab:node-class-ml} presents the result. 
As shown in the table, \stmethod-2layer always performed best with different hidden sizes in most of the cases. This reveals that adding non-linear layers can be effective, but the effect may vanish significantly when we continuously add more layers. The reason is straightforward, given that adding more non-linear layers can eventually cause overfitting. 

We also observe that deeper models can achieve better performance with many fewer trainable parameters than wider models. Table~\ref{tab:node-class-ml-size} presents the model parameter size of each model.
As shown in the table, the parameter size of GraphSage with \stmethod-2layer and a hidden size of 256 is 338,479 which is 2 smaller than the parameter size of vanilla GraphSage with a hidden-size of 512, i.e., 675,887. And its performance is much better than vanilla GraphSage with a hidden size of 512. 

Furthermore, we also observe that adding \stmethod layers only slightly increase the model training time. Table~\ref{tab:node-class-time} presents the single training epoch time of GraphSage under different configurations. As shown in the table, the epoch time of GraphSage with \stmethod-2layer and a hidden size of 256 is only 3.1\% longer than that of vanilla GraphSage with the same hidden size. However the corresponding parameter size is 1.63 larger.  



\begin{table}
 \caption{Test accuracy (\%) of GraphSage and GAT when applying \method on different GNN layers on ogbn-product.}
\label{tab:input-hidden}
 \centering
 \begin{tabular}{lrr}
 \toprule 
 & GraphSage & GAT \\
 \midrule
baseline  & 78.27 & 79.23 \\
\method-all & 79.88 & 79.49 \\
\method-input   & 79.81 & 78.87 \\
\method-hidden   & 79.91 & 79.68 \\
\method-output   & 78.60 & 78.45\\
\bottomrule
\end{tabular}
\end{table}

\subsection{Effectiveness of Applying \method to Different GNN Layers} \label{eval-dgnn}
Finally, we studied the effectiveness of applying \method to only the input GNN layer (\stmethod-input), only the hidden GNN layers (\stmethod-hidden), only the output GNN layer (\stmethod-output) and all the GNN layers on the ogbn-products dataset using GraphSage and GAT. 
The baseline model is a three-layer GNN model. The hidden dimension size is 256 and 128 for GraphSage and GAT respectively.
Table~\ref{tab:input-hidden} presents the results. As the table shows, only applying \method to the output GNN layer brings little or no benefit.
While applying \method to hidden and input GNN layers can improve the model performance, especially applying \method to hidden layers.
It demonstrates that the benefit of \method mainly comes from adding additional non-linear layers into the input and hidden GNN layers.









\begin{table*}[t]
 \centering
\begin{threeparttable}
 \caption{Model settings of \method models. The column of \method position presents where we put the non-linear layers. hidden-only means only applying \method to the hidden GNN layers, input-only means only applying \method to the input layer, all-layer means applying \method to all the GNN layers.. The column of \method setting presents how we organize each \method layer. For example, 1-relu+1-sigmoid means \method contains one feedforward neural network with ReLU as its activation function followed by another feedforward neural network with Sigmoid as its activation function and 2-relu means \method contains two feedforward neural network layers with ReLU as the activation function of each layer.}
\label{tab:setting}
 \centering
\begin{tabular}{llccccl}
 \toprule 
Dataset & Model & hidden size & layers & aggregation & \method position & \method setting \\
\midrule
\multicolumn{6}{c}{Node classification tasks} \\
\midrule
ogbn-product & GraphSage & 256 & 3 & mean & hidden-only & 1-relu+1-sigmoid \\
ogbn-product & GraphSage-cluster & 256 & 3 & mean & hidden-only & 1-relu+1-sigmoid\\
ogbn-product & GAT-flag & 256 & 3 & sum & hidden-only & 1-relu+1-sigmoid\\
ogbn-product & GAT-ns & 256 & 3 & sum & hidden-only & 1-relu+1-sigmoid\\
ogbn-arxiv & GraphSage & 256 & 3 & mean & hidden-only & 1-relu+1-sigmoid\\
ogbn-arxiv & GraphSage-cluster & 256 & 3 & mean & hidden-only & 1-relu+1-sigmoid\\
ogbn-arxiv & GAT-flag & 256 & 3 & sum & hidden-only & 1-relu+1-sigmoid\\
ogbn-arxiv & GAT+BoT & 120 & 6 & sum & hidden-only & 2-relu \\
ogbn-arxiv & AGDN+BoT & 256 & 3 & GAT-HA & hidden-only & 1-relu \\
ogbn-arxiv & AGDN+BoT+self-KD+C\&S & 256 & 3 & GAT-HA & hidden-only & 1-relu \\
ogbn-protein & GraphSage & 256 & 3 & mean & hidden-only & 1-relu\\
ogbn-protein & GraphSage-cluster & 256 & 3 & mean & hidden-only & 1-relu\\
ogbn-protein & GAT-flag & 256 & 3 & sum & hidden-only & 1-relu\\
ogbn-protein & GAT-ns & 256 & 3 & sum & hidden-only & 1-relu\\
reddit & GraphSage & 256 & 3 & mean & hidden-only & 1-relu+1-sigmoid\\
reddit & GraphSage-cluster & 256 & 3 & mean & hidden-only & 1-relu+1-sigmoid\\
reddit & GAT-flag & 256 & 3 & sum & hidden-only & 1-relu+1-sigmoid\\
reddit & GAT-ns & 256 & 3 & sum & hidden-only & 1-relu+1-sigmoid\\
\midrule
\multicolumn{6}{c}{Link prediction tasks} \\
\midrule
ogbl-collab & Seal-DGCNN & 256 & 3 & sum & all-layers & 1-relu\\
ogbl-collab & GCN-full & 256 & 3 & mean & hidden-only & 2-relu\\
ogbl-collab & GraphSage-full & 256 & 3 & mean & hidden-only & 2-relu\\
ogbl-ppa & Seal-DGCNN & 32 & 3 & sum & all-layers & 1-relu\\
ogbl-ppa & GCN-full & 256 & 3 & mean & all-layers & 2-relu\\
ogbl-ppa & GraphSage-full & 256 & 3 & mean & all-layers & 2-relu\\
ogbl-ppi & Seal-DGCNN & 32 & 3 & sum & hidden-layers & 1-relu\\
ogbl-ppi & GCN-full & 256 & 2 &  mean & input-only & 1-relu \\
ogbl-ppi & GraphSage-full & 256 & 2 &  mean & input-only & 1-relu \\
ogbl-ppi & GraphSage+EdgeAttr & 512 & 2 & mean & all-layers & 2-relu \\
\bottomrule
\end{tabular}
\end{threeparttable}
\end{table*}

