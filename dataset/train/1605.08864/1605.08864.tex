Let us assume a routing change, e.g., an announcement of a new prefix by an AS, in the network at time $t_{0}=0$. Our goal is to calculate the \textit{BGP convergence time}, i.e., the time needed till all ASes/routers in the network have the final (i.e., shortest, conforming to policies) BGP routes for this prefix.


To this end, using Assumption~\ref{assumption:exponential-lambda}, we can model the dissemination of the BGP updates in the network with the Markov Chain (MC) of Fig.~\ref{fig:mc-nodes}, where each state corresponds to the number of ASes/routers that have the final BGP routes. At time $t_{0}=0$ the system is at state $0$, while the state $N$ denotes the BGP convergence. When an AS in the SDN cluster receives the BGP update, all the nodes in the SDN cluster are informed (through the controller); thus, we have a transition, e.g., from state $i$ to state $k+i$. The transition rates in the MC, as we discuss in detail later, depend on the network topology.


The Markov Chain of Fig.~\ref{fig:mc-nodes} is transient, and the BGP convergence time is the time needed to move from state $0$ to state $N$.

For notation brevity, in the remainder we use the MC of Fig.~\ref{fig:mc-steps}, which is equivalent to the MC of Fig.~\ref{fig:mc-nodes}. Here, the states represent the \textit{number of transitions} in the MC of Fig.~\ref{fig:mc-nodes}. For example, the state/step $1$ corresponds to the state $1$ or $k$ of the MC of Fig.~\ref{fig:mc-nodes}, while the state/step $i$ corresponds to the state $i$ or $k+i-1$ in the MC of Fig.~\ref{fig:mc-nodes}. The states $0$ are equivalent in both MCs, while the state/step $C$ denotes the BGP convergence, and, thus, corresponds to the state $N$ in the MC of Fig.~\ref{fig:mc-nodes}. 

If we denote with $x$ the step at which -for the first time- an AS in the SDN cluster receives the BGP update, then the transitions rates $\lambda_{i}^{'}$ in the MC of Fig.~\ref{fig:mc-steps} are given by
\begin{equation}
\lambda_{i}^{'} = \left\{
\begin{tabular}{ll}
$\lambda_{i,i+1}+\lambda_{i,i+k}$		&$, i\leq x$\\
$\lambda_{k+i-1, k+i}$					&$, i>x$
\end{tabular}
\right.
\end{equation}


\begin{figure}
\subfigure[Markov Chain (number of nodes)]{\includegraphics[width=\linewidth]{./figures/MC_nb_of_nodes1.eps}\label{fig:mc-nodes}}
\subfigure[Markov Chain (number of transitions, or \textit{steps})]{\includegraphics[width=\linewidth]{./figures/MC_steps.eps}\label{fig:mc-steps}}
\caption{Markov Chains where the states correspond to (a) the number of nodes that have updated BGP routes, and (b) the number of transitions, or \textit{steps}, of the BGP update dissemination process.}
\label{fig:markov-chains}
\end{figure}



We now proceed to calculate the rates $\lambda_{i}^{'}$. 
The ASes that have received the BGP updates, will then send BGP updates to some of their neighboring ASes, according to their routing policies. We refer to a neighbor to which the update will be forwarded as a \textit{bgp-eligible} neighbor. 

\begin{definition}\label{def:bgp-degree}
We define as the \texttt{bgp-degree} at step $i$, $D(i)$, the number of the ASes that are bgp-eligible neighbors with \underline{any} of the ASes that have received the BGP updates at step $i$. 
\end{definition}

Although an AS might receive the same BGP update from more than one neighbors, the final BGP route will correspond to only one of the received updates (i.e., shortest path). Hence, to calculate the transition rate $\lambda_{i}^{'}$, we take into account only one BGP connection (corresponding to the shortest path) per bgp-eligible neighbor. Since the BGP update times are exponentially distributed with rate $\lambda$ (see, Assumption~\ref{assumption:exponential-lambda}), it follows that $\lambda_{i}^{'}$ will be given by\footnote{The transition time is the minimum of $D(i)$ i.i.d. exponentially distributed times.}
\begin{equation}\label{eq:lambda-prime}
\lambda_{i}^{'} = \lambda\cdot D(i)
\end{equation}

Knowing the rates $\lambda_{i}^{'}$, we can calculate the transition delays in each step. Adding the delays in each step, we can derive Theorem~\ref{thm:expected-delay-generic}, which gives the BGP convergence time, i.e., the time to move from state $0$ to state $C$. 



\begin{theorem}\label{thm:expected-delay-generic}
The expectation of the BGP convergence time $T$ in a hybrid SDN/BGP inter-domain topology is given by
\begin{equation}\label{eq:theorem-expected-delay-generic}
E[T] = \frac{1}{\lambda}\cdot \sum_{x=0}^{N-k}\sum_{i=1}^{N-k}\frac{1}{\Dix}\cdot P_{sdn}(x)
\end{equation}
where $\Dix$ is the \bgp of the network at step $i$ given that the SDN cluster receives the update at step $x$, and $P_{sdn}(x)$ is the probability that the SDN cluster receives the update at step $x$.
\end{theorem}
\begin{proof}
To calculate $E[T]$, we first apply the conditional expectation
\begin{equation}\label{eq:ET-sum-ETx}
E[T] = \sum_{x=0}^{N-k}E[T|x]\cdot P_{sdn}(x)
\end{equation}
where $E[T|x]$ denotes the expected convergence time, given that the SDN cluster receives the update at step $x$. Under this condition, the bgp-degrees at each step are $D(i|x)$, $i\in[1,N-k]$. Hence, taking also into account \eq{eq:lambda-prime}, it follows that the transition delay from a step $i$ to a step $i+1$, $T_{i,i+1}$, is exponentially distributed with rate $\lambda_{i}^{'} = \lambda\cdot D(i|x)$, and its expectation is given by
\begin{equation}
E[T_{i,i+1}|x] = \frac{1}{\lambda\cdot D(i|x)}
\end{equation}
\underline{Remark:} The state/step $0$ does not correspond to a real state of the system; it is only used for presentation purposes. Thus, we set $T_{0,1}=0$.

As mentioned earlier, the BGP convergence delay is the time needed to move from step $0$ to step $C$, and thus it is given by the sum of the transition delays of all the intermediate steps, i.e., 
\begin{align}\label{eq:ETx-sum-Dix}
E[T|x] 	&= E\left[\sum_{i=1}^{N-k}T_{i,i+1}|x\right] \nonumber\\
		&= \sum_{i=1}^{N-k}E[T_{i,i+1}|x] \nonumber\\
		&= \sum_{i=1}^{N-k}\frac{1}{\lambda\cdot D(i|x)}
\end{align}
Now, the expression of \eq{eq:theorem-expected-delay-generic} follows by substituting \eq{eq:ETx-sum-Dix} to \eq{eq:ET-sum-ETx}.
\end{proof}


In the following sections we calculate the quantities ${\Dix}$ and $P_{sdn}(x)$ for important network topologies.


\subsection{Full-Mesh Network Topology}
We first consider a basic topology: a full-mesh network, where every AS-pair is connected.

\begin{theorem}\label{thm:P-sdn}
The probability that the SDN cluster receives the update at step $x$ is given by
\begin{equation}\label{eq:P-sdn}
P_{sdn}(x) = \frac{k}{N-x}\cdot \prod_{j=0}^{x-1}\left(1-\frac{k}{N-j}\right)
\end{equation}
\end{theorem}
\begin{proof}
The SDN cluster comprises $k$ (out of the total $N$) ASes. Since we consider that the prefix announcement is made by a (random) AS in the network, the probability that the announcing AS is in the SDN cluster (and thus $x=0$) is 
\begin{equation}
P_{sdn}(0) = \frac{k}{N}
\end{equation}
If the announcing AS is not in the SDN cluster, then $x>0$, and thus
\begin{equation}\label{eq:Psdn-x-larger-0}
P_{sdn}(x>0) = 1-P_{sdn}(0) = 1-\frac{k}{N}
\end{equation}
The probability $P_{sdn}(1)$ is given by
\begin{equation}
P_{sdn}(1) = P_{sdn}(1|x>0)\cdot P_{sdn}(x>0)
\end{equation}
where $P_{sdn}(1|x>0)$ denotes the probability that the SDN cluster receives the BGP update at step $1$, given that it has not received it before. If $x>0$, then at step $1$ the remaining ASes  without the update are $N-1$, of which $k$ belong to the SDN cluster. Since the BGP update processes are distributed with the same rate $\lambda$, the probability that the next AS to get the update belongs to the SDN cluster is $\frac{k}{N-1}$. Therefore, and taking into account \eq{eq:Psdn-x-larger-0}, it holds that 
\begin{equation}
P_{sdn}(1) = P_{sdn}(1|x>0)\cdot P_{sdn}(x>0) = \frac{k}{N-1}\cdot \left(1-\frac{k}{N}\right)
\end{equation}
and, respectively,
\begin{align}
P_{sdn}(x>1) &= \left(1-P_{sdn}(1|x>0)\right)\cdot P_{sdn}(x>0) \nonumber\\
			&= \left(1-\frac{k}{N-1}\right)\cdot \left(1-\frac{k}{N}\right)
\end{align}

Proceeding similarly for the next steps $i=2,...,N-k$, it can be shown that
\begin{equation}
P_{sdn}(i|x>i-1) = \frac{k}{N-i}
\end{equation}
and
\begin{align}
P_{sdn}(x>i-1) 	&= \left(1-P_{sdn}(i-1|x>i-2)\right)\cdot ... \cdot P_{sdn}(x>0) \nonumber\\
				& = \left(1-\frac{k}{N-(i-1)}\right)\cdot...\cdot  \left(1-\frac{k}{N}\right) \nonumber\\
				& = \prod_{j=0}^{i-1}\left(1-\frac{k}{N-j}\right)
\end{align}
and, therefore,
\begin{align}
P_{sdn}(i) &= P_{sdn}(i|x>i-1)\cdot P_{sdn}(x>i-1)\nonumber\\
				&= \frac{k}{N-i}\cdot \prod_{j=0}^{i-1}\left(1-\frac{k}{N-j}\right)
\end{align}
which is the expression of \eq{eq:P-sdn}.
\end{proof}


Theorem~\ref{thm:Dix-full-mesh} gives the bgp-degrees $\Dix$ in a mesh network as a function of $n(i|x)$, which is defined as the number of nodes with updated BGP information at step $i$, given that the SDN cluster received the update at step $x$
\begin{equation}
n(i|x) = \left\{
\begin{tabular}{ll}
$i$	& $, i\leq x$ \\
$i+k-1$	& $, i>x$
\end{tabular}
\right.
\end{equation}


\begin{theorem}\label{thm:Dix-full-mesh}
The \bgp $\Dix$, $i\in[1,N-k], x\in[0,N-k]$, in a full-mesh network topology is given by
\begin{equation}
\Dix = N-n(i|x)
\end{equation}
\end{theorem}
\begin{proof}
In a mesh network, since every AS-pair is directly connected, only the BGP messages sent by the announcing AS (i.e., shortest path) need to be considered. In step $i$, the announcing AS has $N-n(i|x)$ neighbors that have not received the BGP updates, and thus, it follows that $D(i|x) = N-n(i|x)$.
\end{proof}



\subsection{Random Graph Network Topologies}

In networks that are not full-meshes, ASes can be connected in many different ways and policies. Since it is not possible to study every single topology, we use two classes of random graphs to capture the effects of routing centralization in non full-mesh networks. In this first approach, we consider unconstrained routing policies.



\subsubsection{Poisson (Erdos-Renyi) Graph}
We first consider the case of a Poisson random graph, where a link between each AS-pair exists with probability $p$. Varying the value of $p$ we can capture different levels of sparseness.

Using similar arguments as in the full-mesh case, it is easy to show that the probabilities $P_{sdn}(x)$ are given by Theorem~\ref{thm:P-sdn}. The \textit{expected} bgp-degrees, which can be used (as an approximation) instead of $\Dix$ in Theorem~\ref{thm:expected-delay-generic}, are given by the following Theorem. 
\begin{theorem}\label{thm:Dix-poisson}
The expectation of the \bgp $\Dix$, $i\in[1,N-k], x\in[0,N-k]$, in a Poisson graph network topology is
\begin{equation}
E[\Dix] = \left(N-n(i|x)\right) \cdot \left(1-(1-p)^{n(i|x)}\right)
\end{equation}
\end{theorem}
\begin{proof}
In a non full-mesh network, some ASes are not directly connected to the announcing AS. Thus, in the calculation of the $\Dix$ we need to consider the bgp-eligible neighbors of \textit{all} the ASes that have received the update.

Let assume that we are at step $i$, and $n(i)$ nodes have received the BGP updates; we denote the set of these nodes as $S_{i}$. A node $m\notin S_{i}$ is a bgp-eligible neighbor with a node $j\in S_{i}$ with probability
\begin{equation}
P(m,j) = p
\end{equation}
since every pair of nodes is connected with probability $p$ (by the definition of a Poisson graph). The probability that $m$ is \textit{not} a bgp-eligible neighbor with \textit{any} of the nodes $j\in S_{i}$, is given by
\begin{align}
1-P(m,S_{i}) = \prod_{j\in S_{i}}(1-P(m,j)) = \prod_{j\in S_{i}}(1-p) = (1-p)^{n(i)}
\end{align}
since $|S_{i}| = n(i)$. It follows easily that the complementary event, i.e., $m$ is a bgp-eligible neighbor with \textit{any} of the nodes $j\in S_{i}$, happens with probability
\begin{align}
P(m,S_{i}) = 1- (1-p)^{n(i)}
\end{align}

There are $N-n(i)$ ASes without the update, with each of them being a bgp-eligible neighbor with any of the nodes $j\in S_{i}$ with (equal) probability $P(m,S_{i})$. Hence, the total number of bgp-eligible neighbors (or, as defined in Def.~\ref{def:bgp-degree}, the \textit{bgp-degree} $D(i)$) is a binomially distributed random variable, whose expectation is given by 
\begin{equation}
E[D(i)] = (N-n(i))\cdot (1-(1-p)^{n(i)})
\end{equation}
\end{proof}


\begin{corollary}
Using the expectation of $D(i|x)$ in Theorem~\ref{thm:expected-delay-generic}, underestimates the BGP convergence time, i.e.,
\begin{equation}
E[T] \geq \frac{1}{\lambda}\cdot \sum_{x=0}^{N-k}\sum_{i=1}^{N-k}\frac{1}{E[\Dix]}\cdot P_{sdn}(x)
\end{equation}
\end{corollary}
\begin{proof}
The bgp-degree $\Dix$ in non full-mesh networks, is a random variable that can take different values, depending on the BGP updates dissemination process (i.e., the exact set of nodes that have received the BGP updates at step $i$, and their links to the rest of the nodes). Thus, we can write for the transition delay 
\begin{equation}\label{eq:transition-delay-generic}
E[T_{i,i+1}|x] = \sum_{y} \frac{1}{y} \cdot P\{\Dix = y\} = E\left[\frac{1}{\Dix}\right]
\end{equation}
Calculating the exact value of the expectation $E\left[\frac{1}{\Dix}\right]$ is difficult, thus, we use a well known approximation in \eq{eq:transition-delay-generic}:
\begin{equation}
E[T_{i,i+1}|x] = E\left[\frac{1}{\Dix}\right] \approx \frac{1}{E[\Dix]}
\end{equation}
where the calculation of $E[\Dix]$ is much easier (see, e.g., proof of Theorem~\ref{thm:Dix-poisson}). Then the BGP convergence delay is approximately given by
\begin{equation}
E[T] \approx \frac{1}{\lambda}\cdot \sum_{x=0}^{N-k}\sum_{i=1}^{N-k}\frac{1}{E[\Dix]}\cdot P_{sdn}(x)
\end{equation}
However, applying Jensen's bound for the expectation of a convex function (here, $f(x) = \frac{1}{x}$) of a random variable (here, $\Dix$) on \eq{eq:transition-delay-generic}, gives
\begin{equation}
E[T_{i,i+1}|x] = E\left[\frac{1}{\Dix}\right] \geq \frac{1}{E[\Dix]}
\end{equation}
which proves the Corollary.
\end{proof}

\subsubsection{Arbitrary Degree Sequence Random Graph}
The structure of networks where the degrees (i.e., the number of connections) of the ASes are largely heterogeneous, e.g., power-law graphs, can be better described with a Configuration-Model Random Graph (CM-RG) rather than a Poisson graph. In the CM-RG model, a random graph is created by connecting randomly the nodes (i.e., ASes), whose degrees are given~\cite{Newman:Networks-book}. Hence, we can use the CM-RG to model a network with \textit{any arbitrary degree sequence} with mean value $\mu_{d}$ and variance $\sigma_{d}^{2}$ (and, $CV_{d} = \frac{\sigma_{d}}{\mu_{d}}$).

If the participation of an AS in the SDN cluster is independent of its degree, then the probabilities $P_{sdn}(x)$ are given by Theorem~\ref{thm:P-sdn}, and the bgp-degrees $\Dix$ are given by the following Result\footnote{We use the notation ``Result'', instead of ``Theorem'', because the provided expression is an approximation.}.
\begin{result}
The expectation of the \bgp $\Dix$, $i\in[1,N-k], x\in[0,N-k]$, in a CM-RG network topology is given by
\begin{equation}\label{eq:Dix-CM}
E[\Dix] = D(1|x)\cdot \prod_{j=1}^{i-1}A(j|x) + \sum_{j=1}^{i-1}\left(\mu_{d}(j|x)-1\right)\cdot \prod_{m=j+1}^{i-1}A(m|x)
\end{equation}
where
\begin{align}\label{eq:CM-D1x}
D(1|x)&= \left\{
\begin{tabular}{ll}
$\mu_{d}$	& $, x>0$ \\
$(N-k)\cdot \mu_{d} \cdot \ln\left(\frac{N}{N-k}\right)$	& $, x=0$
\end{tabular}
\right.\\
\mu_{d}(j|x) 	&= \mu_{d}\cdot \prod_{m=1}^{j-1}\left(1-\frac{CV_{d}^{2}}{N-n(m|x)-1}\right) \label{eq:average-out-degree}\\
A(j|x) 			&= 1-\frac{\mu_{d}(j|x)}{N-n(j|x)-1}
\end{align}
\end{result}
\begin{proof}
The main difference with the Poisson case is that in the CM-RG case, it is more probable that the ASes with the higher degrees will receive the BGP updates faster. For instance, let us assume that the announcing AS, e.g., AS-1, does not belong to the SDN cluster. If we denote with $d_{1},d_{2}$, and $d_{3}$ the degrees of AS-1, AS-2 and AS-3 (where AS-2 and AS-3, have not received yet the BGP update), a property of a CM-RG says that AS-1 is directly connected with AS-2 and AS-3 with probabilities
\begin{equation}
P(1,2) = c\cdot d_{1}\cdot d_{2}~~~~and~~~P(1,3) = c\cdot d_{1}\cdot d_{3}
\end{equation}
respectively, where $c$ a normalizing constant. In other words, the AS with with the higher degree has a higher probability to be directly connected to AS-1. Consequently, ASes with higher degrees have a higher probability to get the BGP update faster.

Now, let us first derive \eq{eq:CM-D1x}. If $x>0$, the announcing AS does not belong to the SDN cluster ($n(1|x>0)=1$), and thus the bgp-degree will be equal to the degree of the announcing AS. Since the average degree of a node is $\mu_{d}$, it follows easily that expectation of the bgp-degree in this case, is given by
\begin{equation}
E[D(1|x>0)] = \mu_{d}
\end{equation}

If $x=0$, the announcing AS belongs to the SDN cluster, and, thus, all the $k$ nodes in the SDN cluster have the BGP updates. In this case, the bgp-degree is the number of all bgp-eligible neighbors of these $k$ nodes. Let us denote with $S_{1}$ the set of nodes in the SDN cluster. Since the fact that a node belongs to the SDN cluster and its degree are independent, the probability that an edge coming out from a node $m\notin S_{1}$ is connected to a node $j\in S_{1}$, is equal to $\frac{k}{N}$. Hence, a node $m\notin S_{1}$, with degree $d_{m}$, is not connected to any of the $k$ nodes in the SDN cluster with probability
\begin{equation}
1-P(m,S_{1}) = \left(1-\frac{k}{N}\right)^{d_{m}}
\end{equation}
and, respectively
\begin{equation}
P(m,S_{1}) = 1-\left(1-\frac{k}{N}\right)^{d_{m}}
\end{equation}
The above equation holds $\forall m\notin S_{1}$; the degrees $d_{m}$ can have different values for each $m$. Since, there are $N-k$ nodes that do not belong to the SDN cluster (and, do not have the BGP updated route), the expected bgp-degree is given by
\begin{equation}
E[D(1|0)] = (N-k)\cdot E\left[1-\left(1-\frac{k}{N}\right)^{d}\right]
\end{equation}
where the expectation is taken over $d$, i.e., over all the degrees $d_{m}, m\notin S_{1}$. To calculate this expectation is difficult, thus we approximate it using a Taylor series approximation, i.e, 
\begin{align}
E\left[1-\left(1-\frac{k}{N}\right)^{d}\right] &= 1-E\left[\left(1-\frac{k}{N}\right)^{d}\right]  \nonumber\\
		&\approx 1-\left(1+E[d]\cdot \ln\left(1-\frac{k}{N}\right)\right) \nonumber\\
		&= - E[d]\cdot \ln\left(1-\frac{k}{N}\right)\nonumber\\
		&= E[d]\cdot \ln\left(\frac{N}{N-k}\right) \nonumber \\
		&= \mu_{d}\cdot \ln\left(\frac{N}{N-k}\right)
\end{align}
which completes the derivation of \eq{eq:CM-D1x}.

To compute the bgp-degrees $D(i|x)$ of the steps $i=2,...,N-k$, we follow a methodology similar to~\cite{pavlos-conf-model}. Let $D(i-1)$ be the bgp-degree at step $i-1$ and $\mu_{d}(i-1)$ the average degree of the nodes that have not received the BGP update by step $i-1$  (i.e., the nodes $m, m\notin S_{i-1}$). The average degree $\mu_{d}(i-1)$ is not equal to $\mu_{d}$, in general; this is due to the fact that nodes with higher degrees receive faster the BGP updates and thus the remaining nodes are nodes with lower degrees (for a more detailed argumentation see~\cite{pavlos-conf-model}). Following similar arguments as in~\cite{pavlos-conf-model}, it can be shown that the average degrees $\mu_{d}(i)$ are approximately given by \eq{eq:average-out-degree}.

We calculate the bgp-degree of the step $i$, based on the quantities $D(i-1)$ and $\mu_{d}(i-1)$, as follows
\begin{equation}\label{eq:Di_Di-1}
D(i) = D(i-1) - 1 + \mu_{d}(i-1)\cdot \left(1- \frac{D(i-1)}{N-n(i-1)}\right)
\end{equation}
The term $D(i-1) - 1$ is the bgp-degree of the previous step minus $1$, which denotes the $i^{th}$ node that received the BGP update (i.e., at the transition between step $i-1$ and $i$). To this quantity, we need to add the number of nodes that are bgp-eligible neighbors of the $i^{th}$ node, but are not bgp-eligible neighbors of any of the nodes $\in S_{i-1}$. The total nodes without the updated BGP routes at step $i-1$ are $N-n(i-1)$; $D(i-1)$ of these nodes are bgp-eligible neighbors of at least one node $\in S_{i-1}$. Hence, the probability that a bgp-eligible neighbor of the $i^{th}$ node is not a bgp-eligible neighbor of any of the nodes $\in S_{i-1}$, is given by $\left(1- \frac{D(i-1)}{N-n(i-1)}\right)$. Since, the $i^{th}$ node has (on average) $\mu_{d}(i-1)$ bgp-eligible neighbors, it follows that the number we need to add to the quantity $(D(i-1) - 1)$ is 
\begin{equation}
\mu_{d}(i-1) \cdot \left(1- \frac{D(i-1)}{N-n(i-1)}\right)
\end{equation} 

Finally, calculating recursively \eq{eq:Di_Di-1}, after some algebraic manipulations, we derive \eq{eq:Dix-CM}. 

\end{proof}