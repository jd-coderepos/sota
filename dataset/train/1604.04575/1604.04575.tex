\begin{abstract}
We present a formalisation in Agda of the theory of concurrent
transitions, residuation, and causal equivalence of traces for the
\piCalculus. Our formalisation employs de Bruijn indices and
dependently-typed syntax, and aligns the ``proved transitions'' proposed
by Boudol and Castellani in the context of CCS with the proof terms
naturally present in Agda's representation of the labelled transition
relation. Our main contributions are proofs of the ``diamond lemma'' for
the residuals of concurrent transitions and a formal definition of
equivalence of traces up to permutation of transitions.

\quad In the \piCalculus transitions represent propagating binders
whenever their actions involve bound names. To accommodate these cases,
we require a more general diamond lemma where the target states of
equivalent traces are no longer identical, but are related by a
\emph{braiding} that rewires the bound and free names to reflect the
particular interleaving of events involving binders. Our approach may be
useful for modelling concurrency in other languages where transitions
carry metadata sensitive to particular interleavings, such as
dynamically allocated memory addresses.
\end{abstract}
