This section presents an overview of how \keyword{SAFE} applications use trust
logic languages (\keyword{slog} and \keyword{slang}) to build and issue
credentials as \keyword{slogsets}, how \keyword{slogsets} can be linked for
credential discovery, and how a set linking supports policy mobility.  It
illustrates with examples from SafeGENI, which is described in a related
technical report~\cite{chase14:safegeni}.  The GENI trust architecture defines
several classes of authority services to manage user identity and authorize
user activity. These services are decentralized: each authority service may
have multiple instances, and the set of instances may change over time.  In
addition, users may delegate various rights to one another using a capability
model. SafeGENI specifies all of these structures using logic.

A notable feature of \keyword{SAFE} is integration of the trust logic with a
scripting layer that manipulates logic content and invokes the proof engine.
\keyword{Slog} is a tractable logic language that generates a proof locally
from a supplied proof context. Who supplies the proof context and how is it
assembled? The \keyword{slang} provides scripting constructs to build and
modify \keyword{slogsets} from templates, link them to form unions, publish/post them
(e.g., as certificates written to SafeSets), pass or fetch them by reference,
and add them to query contexts for trust decisions.

An application includes \keyword{slang} code to construct any logic content it
issues or fetch and cache logic content from other parties, assemble proof
contexts, issue trust queries, and organize its credentials and policies.  Each
participant in a networked system chooses the \keyword{slang} code that it
executes: the participants exchange declarative logic content, but not
scripting code. 

\subsection{SAFE Logic (slog)}
\label{sec:slog}

Slog is a elementary trust logic based on constrained datalog, which is a
subset of first-order logic. Logic statements in \keyword{slog} are written in
datalog augmented with the classic {\bf says} operator of BAN belief
logic~\cite{Burrows:1990} and ABLP logic~\cite{lampson92:authentication}.
Statements are built up from atomic formulas (atoms) and the logical operators
conjunction and implication.  An atom is a predicate symbol applied to a list
of parameters, which may be variables or constants representing principals,
objects, or values. Every atom has a first parameter representing a principal
who {\bf says} it (the {\it speaker}). Consider this slog statement:

\begin{verbatim}
authorize(?Subj) :- 
  Alice: coworker(?Subj).
\end{verbatim}

This statement reads {\it ``self infers authorize(?Subj), for any given subject
represented by a variable ?Subj, if the principal Alice says
coworker(?Subj) is true''}.   The {\tt :-} is datalog syntax for logical
implication: this statement is a {\it rule}.  The text to the left of the {\tt
:-} is the {\it head} of the rule, and the text to the right is the {\it body}.
The head is a single atom whose parameters may include one or more variables
({\tt ?Subj}).  A rule allows the checker to infer that the head is true,
for some substitution of its variables with constants, if the body is true
under that substitution.  The body is a sequence of atoms (called {\it goals})
separated by commas, which indicate conjunction: all of the atoms in the body
must be true for the rule to ``fire''. All variables in the head must also
appear in the body. A {\it fact} is a statement with no body, and therefore no
variables. The checker takes any fact in the context as true.  The predicate in
an atom or fact represents a property, attribute, role, relationship, right,
power, or permission associated with the principals and/or objects named in its
parameters.

Each atom is bound to a speaker. In the example, the atoms in the body are
prefixed with a {\bf says} operator ({\tt :}) naming the principal {\tt Alice}.
If an atom does not name a speaker then the default speaker is {\tt
\IdP''{ // slogset name
    identityProvider(Self). // link to self (geni) ID set
  } // end of slogset definition
end // end of slang function

definit ?Ref := endorse('IdP-ID'), post(?Ref).

\end{lstlisting}

Consider the GENI example in Code Listing~\ref{code:geniRoot}. Geni root
creates a \keyword{slogset} with a name {\tt ``endorse/idp/\IdP} is a
variable passed from \keyword{slang} to \keyword{slog}, which is
interpolated---substituted by its value---at runtime.  A \keyword{slang} rule
tagged with a {\tt def} keyword declares the rule as a function that returns
the value of the last atom on that rule.  The various slang rule types have
additional behaviors to integrate with the application and with SafeSets, and
to extend the scripting primitives. \keyword{Slang} predefines some functions
with prefixes {\tt defenv} for initializing environment variables; {\tt defcon} for
constructing slogsets; {\tt defguard} for entry points to \keyword{slang}
program for access checking incoming requests; and {\tt definit} for
bootstrapping the \keyword{slang} program. Other built-in functions that
implement the SafeSets client API are shown in Table~\ref{tab:slang-func}.
Code Listing~\ref{code:pi} shows how a Project Investigator (PI) relies on
local trust policy and bearer reference provided by the subject to determine
whether the subject is a valid geni user. In this case, it is the subject's
responsibility to provide a reference to a \keyword{slogset}, which contains a
statement that the issued by IdP that the subject is a geni user. The
authorizer augments the proof context constructed from bearer reference with
its own local policy and invokes the slog inference.  This example demonstrates
the flexible and extensible authorization in \keyword{SAFE} in which hybrid
policies are fetch on-demand or a priori at the discretion of the authorizer.
See the technical report for a complete working example of
GENI~\cite{chase14:safegeni}.

\keyword{Slang} also supports {\tt speaksFor} and {\tt speaksForOn} delegation
in a restricted form. Our implementation of {\tt speaksForOn} closely follows
restriction delegation proposed in Snowflake
project~\cite{Howell:2000:FSS:867818, Howell:2000}.  However, unlike
ABLP~\cite{abadi93:accesscalculus} and Snowflake~\cite{Howell:2000:FSS:867818}
projects, we do not view {\tt speaksFor} as a primitive form of delegation.
Recall that \keyword{slog} supports attribute-based delegation with {\bf says}
as the primitive operator. In \keyword{SAFE}, the restricted {\tt speaksFor} is
defined as follows: if a subject {\tt Alice} issues {\tt speaksFor} delegation
capability for {\tt Bob}, then {\tt Alice} grants {\tt Bob} to write to any
\keyword{slogset} owned by {\tt Alice}. Similarly, {\tt speaksForOn} restricts
the capability to a particular \keyword{slogset} named by {\tt Alice}. Now when
{\tt Bob} is issuing statements for {\tt Alice}, it is {\tt Bob's}
responsibility to write to an appropriate \keyword{slogset} under the {\tt
Alice} namespace. The {\tt speaksFor} and {\tt speaksForOn} are stated as
ordinary \keyword{slog} facts but interpreted specially by \keyword{slang} to
achieve the desired functionality. In \keyword{SAFE}, we use {\tt speaksForOn}
delegation for joint ownership among principals in a group and
principal/sub-principal roles, and delegating authority to a set of attributes
(\keyword{slogset}) collectively rather than specifying each attribute
individually as in ABAC~\cite{rt-abac}.

Lastly, \keyword{slang} also provides support for {\tt aggregation}, which
cannot be supported directly in \keyword{slog} without loosing
tractability~\cite{Grohe:2010}.  Aggregation is useful to
implement advanced features in trust logics such as threshold/manifold
structures as used by SPKI/SDSI.






\begin{lstlisting}[float, showstringspaces=false, caption={Project Investigator (PI) relies on local trust policy and bearer reference passed by the {\tt ?Subject} to determine whether {\tt ?Subject} is a valid geni user.}, label=code:pi]

defenv ?Geni :- 'geni-ID'. // hash-of-geni-PK

defcon trustStructure() :-
  spec('trust structure at the authorizer'),
  'policy/localTrust'{
     identityProvider(?X) :- 
       geniRoot(?Geni), 
       ?Geni: identityProvider(?X).
     geniUser(?U) :- 
       identityProvider(?IdP), 
       ?IdP: geniUser(?U).
   }
end

defguard isGeniUser(?Subject, ?BearerRef) :- 
  {
     import('policy/localTrust').
     import(Geni). // substitute env var
     geniUser($Subject)? // subst slang var
  }
end
\end{lstlisting}

\subsection{Set Linking and Support Sets}
\label{sec:linking}

The power of trust logics creates new obstacles to harnessing their power in
practical distributed systems. For example, authorization in decentralized
federated environments involve finding the necessary credentials that satisfy
the local policy for a given access control request. A key obstacle is {\it
credential discovery}: a trust decision may require reasoning from statements
drawn from various sources, requiring a method to discover and retrieve them.
In general, credential discovery is the process of finding the chain of
credentials that delegates the authority from the source to the requester.
Credential discovery is different from the certificate path discovery in X.509
certificates~\cite{Elley01:cert-path-disc} since credentials in trust
management systems generally have more complex meanings than simply binding
names to public keys. For example, a credential chain is often a DAG, rather
than a linear path as in X.509.

Most previous work in trust management assumes that authorizer has already
gathered all the potentially relevant credentials before a request is made and
does not consider credential discovery problem
further~\cite{blaze96:trust, blaze03:keynote, Becker10:secpal}.
Even if the authorizer gathers all the credentials a priori, a crucial issue is
that tailoring the credentials per query request rather than supplying all the
available credentials to the proof context---since the cost of inference
depends on the size of the proof context.

To make the credential discovery possible and efficient, we propose {\it set
linking} to build trust chains by linking relevant logic sets in advance.  A
further advantage of our approach is that it naturally supports caching of
context sets for future decisions.

The construction procedure is distributed across the participants who issue and
receive credentials---\keyword{slogsets} containing endorsements and delegations. Each
participant collects and stores the received credentials by using
meta-predicate {\tt link} to cross reference them into credential sets that it maintains.
The issuer of a credential uses {\tt link} in the set constructor to link the
new set to any of its own credential sets that support its authority to issue
the credential. Code Listing~\ref{code:geniRoot} shows {\tt link} predicate is
used as a reference to Geni root's ID set. If all issuers follow this convention
then by induction the transitive closure of any given credential contains the
totality of upstream credentials that an authorizer needs to validate it---the
credential's {\it support set}. In this way, set linking naturally forms
delegation chains in the credential graph. The authorizer uses its local
checker to validate that these chains lead back to one or more trust anchors
(e.g., {\tt geniRoot}) according to its policies.

The bearer reference link (id) provided by the subject ({\tt ?BearerRef} in
Code Listing~\ref{code:pi}) makes the authorizer to inspect the user
endorsements that she received. Linked support sets make it easy for an
authorizer to obtain all credentials necessary for an authorization decision by
``pulling'' a credential set token passed as an argument in a request, fetching
the closure of the linked subsets recursively, caching them, and adding them to
the proof context. Each participant is free to organize its credentials as it
sees fit, possibly across multiple sets. What is important is that each issuer
links sufficient support into each endorsement or delegation, and that each
requester passes sufficient support to justify each request. The linked sets
may contain a superset of what is required: the authorizer's \keyword{slog}
engine searches the context for relevant content. We emphasize that in
practice, a server finds frequently linked supporting credentials in its cache,
and does not fetch or validate them again on each request. 

In this way, set linking organizes credentials and policies into a DAG that
facilitates discovery and assembly of proof contexts. We note that fetch is
cycle-safe: it ignores any cycles, which do not affect the contents of the
closure.  The DAG is collaboratively editable: each node in the DAG is
controlled by its owners, and changes to a set by its owners are visible in
other sets that link to it. The sets are, in essence, materialized views for
standard queries, in which the subset owners control what statements to include
in the views.

Further, set linking naturally supports {\it policy mobility}: the guard
policies can be defined once by the trust anchors and the authorizers can use
them wherever applicable by simply fetching from SafeSets. 


\subsection{End-to-End Example: SafeNS}
\label{sec:safedns}

We implemented a secure name service---SafeNS---in \keyword{SAFE}. Using
SafeNS, we emulate the DNSSec resolver in \keyword{SAFE}.
Figure~\ref{fig:safedns} illustrative the end-to-end workflow of credential
discovery, set linking, context building and pruning using SafeNS as an
example. 

Given a name service request by the client, the browser/client-agent augments
the request with a bootstrapped reference to the root ({\tt ICAANN-ID})
\keyword{slogset} and passes to the SafeNS resolver. The SafeNS resolver uses
the bearer reference to initiate the credential discovery process using a
\keyword{slang} library function {\tt fetchSRN()}. The resolver fetches the
root set referenced by {\tt ICAANN-ID} and matches the common name for the root
({\tt cn(.)}) with the first name token {\tt `.'} by issuing a \keyword{slog}
query. If the \keyword{slog} query returns true---i.e., the safe resource name
(SRN) binds/matches with the \keyword{slogset} local name given by the {\tt
srn()} predicate---then the search continues further following the {\tt link}
predicates until a closure is reached. The \keyword{slang} runtime builds a
tailored context based on the SRN, and once the search completes,
\keyword{slang} invokes \keyword{slog} with the relevant proof context. The
\keyword{slog} process validates the proof based on local trust anchors ({\tt
ICAANN-ID}) and policies, and certifies the response. The workflow also
illustrates the use of {\tt speaksForOn} issued by the principal {\tt duke}
delegating ownership to the principal {\tt cs} on a particular set named {\tt
cs}.

SafeNS resolver curtails the proof context at each stage of the credential
discovery by using a constrained function {\tt fetchSRN()} rather than {\tt
fetch()} (see Table~\ref{tab:slang-func}). {\tt fetch()} fetches the
transitive closure name service endorsements starting from root to the proof
context, which is prohibitive if not curtailed to the relevant context. It is
important to note that {\tt fetchSRN()} is implemented as a \keyword{slang}
library function (in 30 lines of code) rather than a native implementation. The
context resolver is programmable: for example, the authorizer can add custom
rules to accept content only from authoritative servers located in the US
region.
