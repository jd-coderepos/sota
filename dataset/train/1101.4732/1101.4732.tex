\documentclass[submission,copyright,creativecommons]{eptcs}
\providecommand{\event}{FIT 2010} \usepackage{breakurl}             
\usepackage{amsthm}

\usepackage{amssymb}
\usepackage[english]{babel}
\usepackage{url}
\usepackage{amsfonts}
\usepackage{stmaryrd}
\usepackage{epsfig}

\usepackage{mathtools}
\usepackage{centernot}

\newcommand{\nota}[1]{}

\newcommand{\abst}[2]{[\![#1]\!]_{#2}}

\newcommand{\eg}{e.g.}
\newcommand{\ie}{i.e.}
\newcommand{\wrt}{w.r.t.}
\newcommand{\ws}{{\sc ws}}
\newcommand{\wscl}{{\sc wscl}}
\newcommand{\opaque}{\square}
\newcommand{\red}{\mapsto}
\newcommand{\bydef}{\stackrel{\scriptstyle{\it def}}{=}}
\newcommand{\denote}[1]{\llbracket #1\rrbracket}
\newcommand{\xml}[1]{\tt{#1}}
\newcommand{\subst}[2]{\{^{#1}/_{#2}\}}
 
\newcommand{\mathaxiom}[2]{
     \begin{array}{l@{\quad}l}{\mbox{\small ({\sc #1})} }
      &{#2}\end{array}}

\newcommand{\mathaxiomtwo}[2]{
   \begin{array}{l}{\mbox{\small ({\sc #1})} }\-.5mm]
    \quad \irule{#2}{#3}\end{array}}

\newcommand{\irule}[2]{\frac{\textstyle\rule[-1.3ex]{0cm}{3ex}#1}{\textstyle\rule[-.5ex]{0cm}{3ex}#2}}


\def \anmathrule #1#2{
    \irule{#1}{#2}}

\newcommand{\comp}{\bowtie}
\newcommand{\outp}[2]{\overline{#1}\langle{#2}\rangle}
\newcommand{\inp}[2]{#1(#2)}
\newcommand{\finp}[2]{{#1}\langle{#2}\rangle}
\newcommand{\ifte}[3]{{\tt if}\ #1\ {\tt then }\ #2\ {\tt else }\ #3}
\newcommand{\fn}{\mathit{fn}}
\newcommand{\bn}{\mathit{bn}}
\newcommand{\vn}{\mathit{vn}}
\newcommand{\inn}{\mathit{in}}
\newcommand{\onn}{\mathit{on}}
\newcommand{\ppar}{|\!|}

\newcommand{\n}{\mathit{n}}
\newcommand{\tuple}[1]{\widetilde{#1}}


\newcommand{\orbar}{\,\mathrel{\big|}\,}
\newcommand{\as}{\,\mathrel{::=}\,}

\newcommand{\act}{\lambda}
\newcommand{\tr}[1]{\stackrel{#1}{\rightarrowfill}}
\newcommand{\tra}[1]{\stackrel{#1}{\rightarrowtail}}
\newcommand{\ltr}[1]{\stackrel{#1}{\longrightarrow}}
\newcommand{\dtr}[1]{\stackrel{#1}{\Longrightarrow}}
\newcommand{\true}{\mathit{true}}
\newcommand{\false}{\mathit{false}}
\newcommand{\Ev}{\mathit{Ev}}

\makeatletter
\def \rightarrowfill{\m@th\mathord{\smash-}\mkern-6mu\cleaders\hbox{}\hfill
  \mkern-6mu\mathord\rightarrow}
\makeatother

\makeatletter
\def \Rightarrowfill{\m@th\mathord{\smash-}\mkern-6mu\cleaders\hbox{}\hfill
  \mkern-6mu\mathord\Rightarrow}
\makeatother

\newcommand{\subj}{\mathit{subj}}
\newcommand{\obj}{\mathit{obj}}
\newcommand{\dom}{\mathit{dom}}
\newcommand{\cod}{\mathit{cod}}



\newcommand{\co}[1]{\overline{#1}}

\newcommand{\nt}[1]{\centernot{\xmapsto{\ #1\ }}}


\title{Contracts for Abstract Processes in Service Composition\thanks{
 Research supported
by the EU FET-GC2 IST-2004-16004 Integrated Project {\sc Sensoria}}}
 

\author{Maria Grazia Buscemi        
\institute{IMT Lucca Institute for Advanced Studies, Italy} 
        \email{m.buscemi@imtlucca.it}
 \and 
 Hern\'an Melgratti
\institute{FCEyN, University of Buenos Aires, Argentina}
\institute{CONICET}
\email{hmelgra@dc.uba.ar}
}

\def\titlerunning{Contracts for Abstract Processes}
\def\authorrunning{M.G. Buscemi \& H. Melgratti}
\newtheorem{definition1}{Definition}
\newtheorem{proposition}{Proposition}
\newtheorem{theorem}{Theorem}
\newtheorem{lemma}{Lemma}
\newtheorem{example}{Example}

\begin{document}

\maketitle

\begin{abstract}
Contracts are a well-established approach for describing and analyzing behavioral aspects of web service compositions. The theory of contracts comes equipped with a notion of compatibility between clients and servers that ensures that every possible interaction between compatible clients and servers will complete successfully. It is generally agreed that real applications often require the ability of exposing  just  partial descriptions of their behaviors, which are usually known as abstract processes.  We propose a formal characterization of abstraction as an extension of the usual symbolic bisimulation and we recover the notion of abstraction in the context of contracts. 
\end{abstract}

\section{Introduction}
Service Oriented Computing is a paradigm that builds upon the notion of 
services as interoperable elements that can be dynamically discovered through a 
public description
of their interface, which includes their behavior or \emph{contract}.
Session types~\cite{Honda93,DezaniECCOP06,GayHole} and contracts~\cite{LaneveP07,CastagnaGP08,CGP09:TCWS,BZ:TUTCCCC} 
provide a framework for checking whether a client is compliant with a service
and whether a process can be ``safely'' replaced with another one. 
Both contracts and session types statically ensure the successful completion 
of every possible interaction between compatible clients and services. 


In a previous work~\cite{BM:APOL} we have addressed an issue related to 
contracts
by developing a formal theory of \emph{abstract processes} in 
orchestration languages.
An {\em orchestrator} describes the execution flow of a single party in 
a composite
service. The execution of an orchestrator takes control of service 
invocation, handles
service answers and data flow among the different parties in the 
composition. Since
orchestrators are descriptions at implementation level and may contain 
sensitive information
that should be kept private to each party, orchestration comes equipped 
with the notion of
abstract process, which enables the interaction of parties while hiding 
private information.
Essentially, abstract processes are partial descriptions intended to 
expose the protocols
followed by the actual, concrete processes. Typically, abstract 
processes are used for slicing
the interactions of a concrete process over a fixed set of ports. 
As a sample scenario, consider an organization that sells goods that are produced
by another company. The process that handles order requests can be written as follows.

The process  starts by accepting an order as a message on port .  
Then, the received order is forwarded to the actual producer  to obtain a 
quotation. Finally, the client request is answered by sending the production cost 
incremented by a . An abstract process of  should at the same time hide 
the sensitive details of the organization and give enough information to the client for 
allowing interaction. For instance, the following abstract process (where 
 stands for a silent, hidden action) shows the interaction of  with a client.

Another feature of abstract processes is to hide particular values and 
internal decisions made by concrete processes. Consider, e.g., the following 
process for authorizing loans.
\begin{small}

\end{small}
 Suppose also that the bank does not want to publicly declare its policy, under which 
 a loan is approved only when the requested amount is at most 50 times the solicitor's salary. 
 This can be achieved by providing an abstract process where some values are
 {\it opaque} (noted with ), i.e., not specified. An abstract process of  can be as below.

Note that the conditional process in  has to be thought of as an
internal, non-deterministic choice in which the bank may decide either
to approve or to refuse the application. In other words, the client cannot
infer from    the actual decision that the bank will take. In general, we 
require an abstraction to provide enough information to decide whether a client and a 
service are compliant, i.e., whether their interaction will allow them to complete their 
execution or not.


In~\cite{BM:APOL} we have characterized the valid abstractions of a 
concrete orchestration and we have shown that valid abstractions preserve
compliance. More precisely, we have formally defined suitable abstractions of concrete processes
as a relation among abstract and concrete processes, called \emph{simulation-based abstraction} relation,
which is an extension of the usual symbolic bisimulation \cite{HL:SB,BD:SSPC}. 

A main goal of the present paper is to investigate the relation between simulation-based abstraction 
and contracts. In particular, we aim at recovering the notion of abstraction in the context of 
the theory of contracts developed in~\cite{CGP09:TCWS}. 
Contracts are types describing the external, visible behavior of a service. Contracts come 
equipped with a notion of service compatibility that characterizes all the valid clients of a service, i.e., 
the clients that terminate any possible interaction with the service. 
In this sense, contracts can be used to statically ensure that the composition of two services
is safe. Contract compatibility induces a preorder relation () among contracts that characterizes 
the safely replacement of services. For instance,  considering two contracts  and , if 
 we know that any valid client of  is also a valid client of ,
hence  can be substituted by  in any context. 
A contract for the service  corresponding to the selling company example introduced above can be 
written as follows. 


Note that  describes the interactions of  with both the client and the producer. 
Hence, we would like to use the idea of abstraction in the context of contracts to obtain
slices of the behaviour of a service and to reason about the interactions of a 
service with a particular partner, i.e., we would like to use  to conclude that 
any client behaving as  is compliant with the role 
client of the service . 

More in detail, given a contract   and a role, defined in terms of a set of visible actions , 
the abstraction  of  can be thought as the contract that hides all the actions
in  that do not
appear in . For instance, the abstraction of  for the role client will be as follows

Another key property of the abstraction type is to turn guarded choices into internal choices, if some guards are hidden. For instance, consider the process . The type of  is . If we hide , the abstraction type of  is

The main technical contributions of this work are the following. Firstly, we define abstraction as a function  over contracts and show that our definition 
behaves well with respect to safe replacement, i.e.,   can be substituted by  
whenever  can be substituted by . Technically speaking, this fact amounts to proving that 
 implies ,  when taking 
 as the strong subcontract preorder.

Then, we show that contract abstraction can be used on top of contracts to reason about
 slices of a concrete service. That is, given a suitable type system for assigning 
contracts to concrete services, the type of a particular slice of a concrete 
service can be defined simply as the abstraction of the original contract. Formally, we 
show that any consistent type system enriched with a typing rule that assigns 
any slice of a concrete service with 
the corresponding contract abstraction is a consistent type system. This result allows us 
to use abstraction over contracts to reason about slices of a concrete service, e.g., we 
can use  to safely reason about the interactions
of  with a client. 
 
Finally,\   we show that contract abstraction matches simulation-based abstraction. Consider the 
simu-lation-based abstraction  
of  that characterizes a particular slice of . Assume that  has contract  and 
consider any compliant client  of  (namely, the type of  is compliant with ). Our 
results ensure that 
 is also compliant with the slice of  described by . 


\section{Abstract processes}
\label{sec:concrete-processes}

In this section we recall the language of abstract processes proposed
in~\cite{BM:APOL} along with a notion of abstraction relation over processes, 
which is a generalisation of \cite{BD:SSPC}.
First, we introduce the language of {\em abstract processes}, 
which is a version of value-passing {\sc  ccs}~\cite{Mil:CC} with 
input guarded choices and conditional statements but without recursion 
plus the possibility of having {\em opaque} definitions.
An opaque element is meant to hide the precise value of an element: for 
instance, an opaque assignment to a data variable hides the assigned value. 
We assume the set of data values to be finite so that the present 
version of the calculus can be encoded into the fragment without 
value-passing. We refer the interested reader to \cite{Mil:CC} for a more 
detailed treatment. 

\paragraph{Syntax}
\label{sec:synt-inform-semant-concrete}
We assume an infinite denumerable set of names 
that is partitioned into a set of port names ,
a set of finite data variables ,
and a finite set of data constants .
We write the special name  to denote an opaque element, and 
we assume . We let  range over 
,  range over , 
 range over , and 
 range over . We let  
range over . 
We write  for a tuple of names. {\em Substitutions}, 
ranged over by , are partial maps from 
 onto . 
Domain and co-domain of , noted  and , are 
defined as usual.  By  we denote  if , and  otherwise. 

The set of {\em abstract processes}  is given by the following grammar:

As usual,  stands for the inert process,  for the 
parallel composition of processes, 
 for the process that performs a silent action and then behaves 
like ,  for the process that sends the message 
 over the port  and then becomes . The process

denotes an external choice in which some process  is chosen when the corresponding guard  is enabled. The conditional process 
behaves either as  if  and  are syntactically equivalent, or
as  otherwise. Opaque names can appear either as subjects of 
input and output prefixes, values of output prefixes, or parts of 
conditions in  processes, but not as a bound variables. 
A conditional statement becomes an \emph{internal} choice when 
at least one value in the condition is opaque; similarly, a guarded choice  
becomes an internal choice when the subject of the input guard is the opaque 
name. 

We let  range over abstract processes and we simply write 
process to denote an abstract process. By \emph{concrete} processes we 
denote processes not containing opaque names. Note that in 
, the data 
variables  are bound, for all . We use the standard notions 
of {\it free} and {\it bound} names of processes, noted respectively as 
 and , and -conversion on bound names. We assume 
that the sets of free and bound names are disjoint 
and that the bound names of a process are all distinct from each other. 
As usual, a process  is {\em closed} if . We also adopt the usual convention of omitting trailing 's. 

\subsection{Symbolic semantics}
For the purpose of this paper we only recall the \emph{symbolic} labeled 
transition relation over processes, 
while we report in Appendix \ref{app:non-symb} the non-symbolic semantics along with a proof that the two semantics are \
equivalent.

We define {\em structural congruence}, , as the least
congruence over processes that is closed with respect to -conversion 
and such that the set of process is a monoid with respect to parallel 
composition  (being  the neutral element). 

We let {\em symbolic actions}  range over the {\em silent move}, 
{\em input} and {\em free output} and we let {\em conditions}  range 
over a language of Boolean formulas:

As usual, for ,  and  
denote the subject and the object of  respectively.  
The notions of {\it free} names , {\it bound} names ,
and -conversion over actions and conditions are as expected,
considering that the occurrences of the names 's are bound in
 and that conditions have no bound names.
For  a process or an action,  denotes the expression obtained
by replacing in  each data variable  with , 
possibly -converting to avoid name capturing.
By  we mean the condition obtained by simultaneously
replacing in  each data variable  with .  A
condition  is {\em ground} if  does not contain data variables.
The {\em evaluation}  of a ground condition  into the set
 is defined by extending in the expected
homomorphical way the following clauses:

A substitution  {\em respects} , written , if  is ground and . A condition 
is \emph{consistent} if there is a substitution  such that
.  A condition  {\em logically entails} a
condition , written , if, for every ,
 implies .  
For instance,  
and .  For  a symbolic 
action and  a substitution such that every data variable in 
 belongs to , we write  to 
denote the following action:
 1mm]
              \outp x {a_1,\ldots,a_k} & \mbox{if  and  for 
                }\
By  we denote the following condition:
  
For  a condition and  a finite set of
conditions,  is a {\em -decomposition} if .  For instance,  is a
-decomposition.

The symbolic labeled transition relation  over abstract 
processes is the least relation satisfying the inference rules in
Table~\ref{table:lts}. 
Intuitively, the condition  in the label  of a transition 
collects the Boolean constraints on the free data variables of the source 
process necessary for action  to take place. For instance, the rules 
for prefixes say that each prefix can be consumed unconditionally, while 
rules ({\sc s-if}) and ({\sc  s-else}) make the equalities or inequalities of 
the conditional statements explicit. For instance, the process , after a first step, can make a transition
under condition that variable  is equal to :

As another example, consider the process . 
By rule ({\sc s-choice-3}), a possible move for  is , 
where the input guard is executed. Another possibility is , where 
 makes an internal choice. 


\begin{table}[t]
 5pt]
\mathrule{s-par}
					 {P\tr {M, \act} P' \quad \bn(\act) \cap \fn(Q) = \emptyset}{P\;|\;Q\tr{M,\act}P'\;|\;Q}
                                         \hfill
                                         \mathrule{s-if}
                                         {P\tr {M, \act} P' \quad m=n \wedge M~\mbox{consistent}}
                                         {\ifte{m=n}{P}{Q} \tr {m=n \wedge M,\act} P'}
                                         \15pt]
                                           \mathruletwo{s-choice-1}
  {P \tr {M, \act} P' \quad  \opaque\in\{m,n\}}
  {\ifte{m=n}{P}{Q} \tr {M, \act} P'}
  \qquad \qquad 
  \mathruletwo{s-choice-2}
  {Q \tr {M, \act} Q' \quad  \opaque\in\{m,n\}}
  {\ifte{m=n}{P}{Q} \tr {M, \act} Q'}
  \
\caption{Symbolic LTS for processes}
\label{table:lts}
\end{table}







\paragraph{Non-symbolic semantics.} \
The following definition corresponds to the original semantics proposed in~\cite{BM:APOL}. (Details are in Appendix \ref{app:non-symb}.)

\begin{definition1}[Non-symbolic semantics] \label{def:alt-non-symb-sem}
Let ,  and  be closed terms.  iff  , ,  and .
\end{definition1}



\subsection{Simulation-based abstraction}


\begin{definition1}[visible names] Given a set of visible names  and
a symbolic action , the set of visible received names of , written
, is defined as follows:
   1mm]
              \emptyset & \mbox{otherwise}\\ 
              \end{array}}
      \right .
 
   P \equiv \ifte {v=\opaque} {\outp y v} {\outp z v} \quad 
   Q \equiv \ifte {v=a} {\outp y v} {\outp z v}. 
 
   \begin{array}{l@{\ ::=\ }ll}
     \alpha & a \ |\ \co{a} & a\in\mathcal{N}\\
     \sigma & 0 \ |\ \alpha.\sigma \ |\ \sigma\oplus\sigma \ | \ \sigma+\sigma
   \end{array}

   \begin{array}{l@{\qquad}l@{\qquad}l@{\qquad}l}
      0 \nt{\alpha}
      &
      \anmathrule{\alpha\neq \beta}{\beta.\sigma \nt{\alpha}}
      &
      \anmathrule{\sigma\nt{\alpha}\quad \rho\nt{\alpha}}{\sigma\oplus\rho\nt{\alpha}}
      &
      \anmathrule{\sigma\nt{\alpha}\quad \rho\nt{\alpha}}{\sigma+\rho\nt{\alpha}}\\
   \end{array}

  \begin{array}{l@{\;\;\;}l@{\;\;\;}l@{\;\;\;}l@{\;\;\;}l}
     \alpha.\sigma \xmapsto{\alpha} \sigma
     &
     \anmathrule{\sigma\xmapsto{\alpha} \sigma' \quad \rho\xmapsto{\alpha}\rho'}
                {\sigma+\rho\xmapsto{\alpha} \sigma'\oplus\rho'}
     &
     \anmathrule{\sigma\xmapsto{\alpha} \sigma' \quad \rho\nt{\alpha}}
                {\sigma+\rho\xmapsto{\alpha} \sigma'}
     &
     \anmathrule{\sigma\xmapsto{\alpha} \sigma' \quad \rho\xmapsto{\alpha}\rho'}
                {\sigma\oplus\rho\xmapsto{\alpha} \sigma'\oplus\rho'}
     &
     \anmathrule{\sigma\xmapsto{\alpha} \sigma' \quad \rho\nt{\alpha}}
                {\sigma\oplus\rho\xmapsto{\alpha} \sigma'}
  \end{array}

\begin{array} {l}
   0\Downarrow \emptyset
   \quad
   \alpha.\sigma\Downarrow \{\alpha\}
   \quad
   \anmathrule {\sigma\Downarrow R \quad \rho\Downarrow S}
               {\sigma+\rho \Downarrow R \cup S} 
   \quad
   \anmathrule {\sigma\Downarrow R}
               {\sigma\oplus\rho \Downarrow R} 
   \quad
   \anmathrule {\rho\Downarrow R}
               {\sigma\oplus\rho \Downarrow R} 
\end{array}

  \anmathrule{P\xrightarrow{\tau}P'}{P\ppar Q \rightarrow P'\ppar Q}
  \quad
  \anmathrule{Q\xrightarrow{\tau}Q'}{P\ppar Q \rightarrow P\ppar Q'}
  \quad
  \anmathrule{P\xrightarrow{\alpha}P'\quad Q\xrightarrow{\co{\alpha}}Q'}{P\ppar Q \rightarrow P'\ppar Q'}

  \anmathrule{P\xrightarrow{\alpha} P'\quad f\xmapsto{\alpha} f'}
             {f[P]\xrightarrow{\alpha} f'[P']}
  \qquad
  \anmathrule{P\xrightarrow{\tau} P}
             {f[P]\xrightarrow{\tau} f'[P']}
 \mathrule{TypeFilter} {\vdash P:\sigma}{\vdash f[P]:f(\sigma)} \mathcal{A}_V[P]   \anmathrule{P\xrightarrow{\alpha} P'\quad   \alpha\in V}
               {\mathcal{A}_V[P]\xrightarrow{\alpha} \mathcal{A}_V[P']}
\qquad
  \anmathrule{P\xrightarrow{\alpha} P'\quad   \alpha\not\in V}
             {\mathcal{A}_V[P]\xrightarrow{\tau} \mathcal{A}_V[P']}

  \begin{array}{r@{\ =\ }l@{\hspace{2cm}}l}
    \mathcal{A}_V(0) & 0 \\
    \mathcal{A}_V(\alpha.\sigma) & \alpha.\mathcal{A}_V(\sigma) & \mbox{if }\ \alpha\in V\\
    \mathcal{A}_V(\alpha.\sigma) & \mathcal{A}_V(\sigma) & \mbox{if }\ \alpha\notin V\\
    \mathcal{A}_V(\Sigma_{i\in I} \alpha_i.\sigma_i) & \Sigma_{j\in J} \alpha_j.\mathcal{A}_V(\sigma_j) \oplus
                                 \bigoplus_{k \in K} \mathcal{A}_V(\sigma_k) \\
                               \multicolumn{3}{r}{ \mbox{with }\ J= \{i \in I| \alpha_i \in V\}\ \mbox{and}\
                                  K= \{i \in I| \alpha_i \not\in V\}}\\
    \mathcal{A}_V(\bigoplus_{i\in I} \alpha_i.\sigma_i ) & \bigoplus_{i \in I} \mathcal{A}_V(\alpha_i.\sigma_i)    
  \end{array} 

   \sigma = {\it request}.\overline{\it askadvice}.({\it negative}.\overline{\it refused} + {\it positive}.\overline{\it approved}) 

   \mathcal{A}_{\{ {\it request,refused, approved}\}}(\sigma) = {\it request}.(\overline{\it refused} \oplus \overline{\it approved}) 
\mathtt{Alc}(\sigma,\alpha,V) = \{\sigma'\ |\ \sigma\xmapsto{\beta_1}\sigma_1\ldots\xmapsto{\beta_n}\sigma_n\xmapsto{\alpha}\sigma'\ \mbox{and}\ \beta_1,\ldots,\beta_n\not\in V\}  \neq\emptyset.
   \mathrule{TypeAbstraction}{\vdash P : \sigma}{\vdash\mathcal{A}_V[P] : \mathcal{A}_V(\sigma)}

\begin{array}{l}
 \mathaxiom{Nil}{\vdash {0}: 0}\\
  \\
  \mathrule{Tau}
  {\vdash P : \sigma}
  {\vdash \tau.P : \sigma}\\
  \\
  \mathrule{Pref}
  {\vdash P : \sigma \qquad \act \neq \tau}
  {\vdash \act.P : \act.\sigma}\\
  \\
  \mathrule{Sum}
  {\act_i \neq\tau \qquad \vdash P_i:\sigma_i \qquad \vdash Q_j:\rho_j}
  {\vdash \Sigma_{i\in I} \act_i.P_i + \Sigma_{j \in J} \tau.Q_j: \Sigma_{i \in I} \act_i.\sigma_i \oplus 
   \bigoplus_{j\in J} \rho_j}\\ 
  \\
  \mathrule{Par}
  {\vdash P_i | Q:\sigma_i\ {\it for\  all\ } P \ltr{\act_i} P_i   \qquad \vdash P | Q_j:\rho_j\     
  {\it for\  all\ }  Q \ltr{\beta_j} Q_j}
  {\vdash P | Q : (\Sigma_{\act_i \neq \tau} \act_i.\sigma_i + \Sigma_{\beta_j \neq\tau} 
   \beta_j.\rho_j)
   \oplus  \bigoplus_{\act_i = \tau} \sigma_i \oplus \bigoplus_{\beta_j = \tau} \rho_j }
  \qquad
  {\mbox{where:}\Bigg\{ 
    \begin{array}{l}
      P \ltr{\act_i} P_i\\
      Q \ltr{\beta_j} Q_j
    \end{array}}\\
   \\ 
   \mathrule{cond1}
  {\vdash P : \sigma \qquad \vdash Q : \rho \qquad \opaque\in\{m,n\}}
  {\vdash\ifte{m=n}PQ : \sigma \oplus \rho}\\
  \\
  \mathrule{cond2}
  {\vdash P : \sigma \qquad \qquad \opaque\not\in\{m,n\} \qquad \qquad m = n}
  {\vdash\ifte{m=n}PQ : \sigma}\\
  \\
  \mathrule{cond3}
  {\vdash Q : \rho \qquad \qquad \opaque\not\in\{m,n\} \qquad \qquad m \neq n}
  {\vdash\ifte{m=n}PQ : \rho}\\
  \\  
 \end{array}

 \begin{array}{l}
   \mathaxiom{tau}
   {\tau.P \tra {\tau} P}
 \quad
  \mathaxiom{out}
  {\outp{x}{\tuple{a}}.P \tra {\outp{x}{\tuple{a}}} P}
 \quad
  \mathaxiom{in}
  {\inp{x_1}{\tuple{v_1}}.P_1+ \ldots+\inp{x_n}{\tuple{v_n}}.P_n
   \tra {\finp{x_i}{\tuple{a}}} P_i\{\tuple{a}/\tuple{v_i}\}}
 \5pt]
 \mathrule{par}
 				 {P\tra {\alpha} P'}{P\;|\;Q\tra {\alpha}P'\;|\;Q}
 \hfill
 \mathrule{str}
 	{P\equiv Q\quad Q\tra{\alpha} Q'\quad Q'\equiv P'}
         {P\tra{\alpha} P'}
\10pt]
  \mathaxiom{choice-3}
  {\inp{x_1}{\tuple{v_1}}.P_1+
  \ldots+\inp{\opaque}{\tuple{v_i}}.P_i+\ldots+\inp{x_n}{\tuple{v_n}}.P_n
  \tra {\tau} P_i}
 \end{array}
 \mathcal{S}^V_M = \{(P\{a/x\},Q\{a/x\}) | P \mathcal{R}^{V\cup\{x\}}_M Q\}\mathtt{Alc}(\rho,\alpha,V) = \{\rho'\ |\ \rho\xmapsto{\beta_1}\rho_1\ldots\xmapsto{\beta_n}\rho_n\xmapsto{\alpha}\rho'\ \mbox{and}\ \beta_1,\ldots,\beta_n\not\in V\}.\mathtt{Alc}(\rho,\alpha,V) = \{\rho'\ |\ \rho\xmapsto{\beta_1}\rho_1\ldots\xmapsto{\beta_n}\rho_n\xmapsto{\alpha}\rho'\ \mbox{and}\ \beta_1,\ldots,\beta_n\not\in V\}.\mathtt{Alc}(\rho,\alpha,V) = \{\rho'\ |\ \rho\xmapsto{\beta_1}\rho_1\ldots\xmapsto{\beta_n}\rho_n\xmapsto{\alpha}\rho'\ \mbox{and}\ \beta_1,\ldots,\beta_n\not\in V\}. 
 Since , for any  in  there exists 
 a  s.t. , i.e., . 
  By proposition~\ref{redabstraccion},
 . Moreover, , i.e. ,
 by Proposition~\ref{prop:abstofvisiblered}.
 It remains to show that . This case follows as for Proposition~\ref{prop:abstractionsubcontracthiddenreduction}.
\end{enumerate}

\nota{\paragraph{Proof of Proposition \ref{prop:enrichedconsistency}.}
Let . As regards consistency condition (1), assume , then either  or
 with . When , consistency
ensures that  and  . By Proposition~\ref{prop:abstractionsubcontract}, . If  then by consistency  and  . By Proposition~\ref{prop:abstractionsubcontracthiddenreduction}, .   As regards
consistency condition (2), assume that  and . By consistency, 
, . By Proposition~\ref{prop:abstractionsubcontractvisiblereduction}, 
. As regards consistency
condition (3), assume that  diverges. Then either  diverges or  has an infinite 
derivation  with  or .
If  diverges, then . Therefore, . 
Otherwise, assume  has an infinite 
derivation  with  or . 
By consistency, this implies that there exists an infinite derivation of for the contract  but this is not possible, since we are considering finite contracts.  Finally, as regards consistency condition (4),
assume that . Then,   and   for all .  We derive  
where . Moreover  since   for all . 
By proposition~\ref{seqofhiddenreductions}, . Since,  implies , 
implies . Hence, .
}
\paragraph{Proof of Theorem \ref{thm:consistency}.}
We prove by structural induction on   that all conditions in Definition~\ref{def:consistency} are satisfied.
First of all, note that the language of orchestrators we rely on does not diverge, hence consistency condition (3) is trivially satisfied in all cases
    \begin{itemize}   
      \item : Conditions (1), (2) hold trivially since  has no reductions. As far as condition(4) is concerned, note that  and  implies  for any .  

      \item . If  then . The only possible type for  (derived by using rule {\sc (tau)}) is  with  and clearly  and therefore condition (1) holds. Moreover, conditions (2) and (4) trivially hold. Let . Then, condition (1) trivially hold. As regards to condition (2), note that  with  and
      . Consequently,  and condition (2) holds. As condition (4) is concerned, note that  implies .

      \item . From typing rule {\sc (sum)} we have that  with . Condition (1): If  then there exists some  such that   and  with . Note that  for some . Consequently, 
      . Condition (2): If  with , then there exists some  such that   and  and . Consequently,  for some . Hence, . As far as condition (4) is concerned, note that  implies .Then  then  implies .   
         
      \item . Condition (1), if   then either  or . If  then  is an internal choice containing a subterm   where . Consequently, . The case  is analogous. For condition (2), note that either  or  and . The proof follows as for condition (1). As regards to condition (3), note that neither  nor . Hence, 
 where  and 
       
    Therefore  implies .
    
    \item . There are two cases  and . Assume 
    . By rule {\sc(cond1)},  with  and . As far as condition (1) is concerned,   when either  or .
    Let   with  and  by inductive hypothesis. Therefore,
    . The case  follows analogously. 
    For condition (2), the proof follows analogously to condition 1. In respect to condition (3), note that  implies 
    that either  or . By inductive hypothesis, we know that  implies
     and  implies
    . Hence, . It is easy to see that 
     if and only if  or  .
    The cases for 
     follows analogously by noting that  is either  or  depending on whether  or  hold.
    \end{itemize}

\paragraph{Proof of Proposition \ref{prop:abstractionandreductions}.} 
We only prove the first case above; the second case is similar. 
From  we have that  with ,  being  defined as the expected counterpart of .
By Definition~\ref{def:alt-non-symb-sem},  there exist  and  
such that  and  and .  and  are closed, hence
. Since,  there exists a -decomposition 
such that ,  with , ,
and there exists some simulation-based abstraction relation   such that 
.   Since  and  is a -decomposition, 
there exists at least one  such that  (and hence ). 
By Definition~\ref{def:alt-non-symb-sem},
. There are two cases:
\begin{itemize}
   \item  or : In both cases, 
    and  are closed. Hence,   
   for any substitution . Since, , we have that  .
   
   It remains to show that  with . Since , we have .
   Also note that  and  are closed because  and  are closed. Hence, ,      and .
\item :  Since  is an input action, we have that  is an input action and 
   both  and  have the same subject  that belongs to . Consequently,
   . Then, , and consequently
   . Since,  is an input action whose subject is in , . Consequently,
   .
   It remains to show that  with . We know that . Since
    is an input action , hence . By Proposition~\ref{prop:visible-cut},
   . Since  and  are closed,  holds by Proposition~\ref{prop:condition-cut}.
\end{itemize}

\end{document}
