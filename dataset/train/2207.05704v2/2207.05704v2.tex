\documentclass[10pt,twocolumn,letterpaper]{article}

\usepackage{wacv}
\usepackage{times}
\usepackage{epsfig}
\usepackage{graphicx}
\usepackage{amsmath}
\usepackage{amssymb}

\makeatletter
\@namedef{ver@everyshi.sty}{}
\makeatother
\usepackage{tikz}


\usepackage{multirow}
\usepackage{booktabs}
\usepackage{colortbl}

\usepackage{pifont}
\newcommand{\xmark}{\ding{55}}
\newcommand{\cmark}{\ding{51}}
\newcommand{\skipping}[1]{}

\usepackage[accsupp]{axessibility}  

\newcommand*\circled[1]{\tikz[baseline=(char.base)]{\node[shape=circle,draw,inner sep=0.5pt] (char) {#1};}}

\pagenumbering{gobble}



\def\wacvPaperID{345} 

\wacvfinalcopy 

\ifwacvfinal
\def\assignedStartPage{9876} \fi






\ifwacvfinal
\usepackage[breaklinks=true,bookmarks=false]{hyperref}
\else
\usepackage[pagebackref=true,breaklinks=true,colorlinks,bookmarks=false]{hyperref}
\fi

\ifwacvfinal
\setcounter{page}{\assignedStartPage}
\else
\pagestyle{empty}
\fi

\def\httilde{\mbox{\tt\raisebox{-.5ex}{\symbol{126}}}}

\definecolor{ao}{rgb}{0.0, 0.5, 0.0}
\definecolor{applegreen}{rgb}{0.55, 0.71, 0.0}

\begin{document}

\title{M-FUSE: Multi-frame Fusion for Scene Flow Estimation}

\author{Lukas Mehl \hspace{15mm} Azin Jahedi \hspace{15mm} Jenny Schmalfuss \hspace{15mm} Andr\'{e}s Bruhn\\
Institute for Visualization and Interactive Systems, University of Stuttgart\\
{\tt\small \{lukas.mehl,azin.jahedi,jenny.schmalfuss,andres.bruhn\}@vis.uni-stuttgart.de}
}

\maketitle


\begin{abstract}
Recently, neural network for scene flow estimation show impressive results on automotive data such as the KITTI benchmark. 
However, despite of using sophisticated rigidity assumptions and parametrizations, such networks are typically limited to only two frame pairs which does not allow them to exploit temporal information.
In our paper we address this shortcoming by proposing a novel multi-frame approach that considers an additional preceding stereo pair.
To this end, we proceed in two steps: 
Firstly, building upon the recent RAFT-3D approach, we develop an improved two-frame baseline by incorporating an advanced stereo method.
Secondly, and even more importantly, exploiting the specific modeling concepts of RAFT-3D, we propose a U-Net architecture that performs a fusion of forward and backward flow estimates and hence allows to integrate temporal information on demand. 
Experiments on the KITTI benchmark do not only show that the advantages of the improved baseline
and the temporal fusion approach complement each other, they also demonstrate that the computed scene flow is highly accurate. 
More precisely, our approach ranks second overall and first for the even more challenging foreground objects, in total outperforming the original RAFT-3D method by more than 16\%.
Code is available at \url{https://github.com/cv-stuttgart/M-FUSE}.\end{abstract}



\section{Introduction}

Estimating the 3D motion field of objects in the 3D world
from stereo or RGBD image sequences, 
the so-called scene flow, is one of the fundamental tasks in computer vision. Its fields of application range from robotics and automotive scenarios \cite{Menze2015_KITTI} over markerless motion capture for virtual and augmented reality
\cite{Valgaerts2012_PerformanceCapture} to action recognition and intention prediction~\cite{Wang2017_ActionReg}. 

Early works go back to the seminal approach of Vedula \etal \cite{Vedula1999_SceneFlow} in the late nineties and since then variational methods have been among the leading techniques to solve this task; see \eg \cite{Devernay2007_SceneFlow,Vogel2015_PRSM,Wedel2011_SceneFlow}.
Only recently, four years after 
their first application to scene flow estimation~\cite{Mayer2016_SceneFlow},
neural networks have been able take the lead in dedicated benchmarks such as KITTI \cite{Menze2015_KITTI}; see \eg the approaches in  \cite{Liu2022_camliflow,Teed2021_RAFT3D,Yang2020_opticalexpansion,Yang2021_SegmentRigid}.
This comparably~late success of neural networks, however,
is not surprising:
Scene flow estimation has more degrees of freedom than other correspondence problems that only work in 2D or 1D, such as optical flow and stereo,
hence solving this task requires more sophisticated ideas and more complex network architectures. 

One way to deal with these additional degrees of freedom
is to use semantic information.
This information can be given in terms of object models \cite{Menze2015_KITTI} or instance segmentations \cite{Behl2017_ISF,Ma2019_DRISF,Yang2021_SegmentRigid}.
Another way is to rely on point-wise \cite{Teed2021_RAFT3D} or segment-wise rigidity priors \cite{Golyanik2017_3DV,Ma2019_DRISF,Menze2015_KITTI,Vogel2015_PRSM}, or to explicitly learn segmenting rigid motions \cite{Yang2021_SegmentRigid}.
In combination with semantic information such rigidity estimates allow to assign rigid motions to all independently moving objects and to the background \cite{Ma2019_DRISF,Yang2021_SegmentRigid}.
And finally, it is also possible to reduce the difficulty of the problem.
This can either be done by decoupling stereo and 3D motion estimation \cite{Badki2021_binaryTTC,Liu2022_camliflow,Teed2021_RAFT3D,Wedel2011_SceneFlow,Yang2020_opticalexpansion},
which also enables the use of dedicated state-of-the-art algorithms for stereo, or by directly relying on RGBD footage \cite{Golyanik2017_3DV,Gottfried_RGBDSceneFlow,Qiao2018_RGBDSceneFlow,Quiroga2014_RGBDSceneFlow}, \eg by using time-of-flight cameras or LiDAR \cite{Liu2022_camliflow}. 



In view of the aforementioned progress in neural networks for scene flow estimation, it is remarkable that currently leading methods \cite{Badki2021_binaryTTC,Li2021_acosf,Liu2022_camliflow,Ma2019_DRISF,Teed2021_RAFT3D,Yang2020_opticalexpansion,Yang2021_SegmentRigid} 
do not exploit potentially valuable temporal information to further improve the results.
In fact, while differing in the actual inputs --
monocular images
\cite{Badki2021_binaryTTC,Yang2020_opticalexpansion}, stereo pairs \cite{Li2021_acosf,Liu2022_camliflow,Ma2019_DRISF,Yang2020_opticalexpansion,Yang2021_SegmentRigid,Teed2021_RAFT3D}, RGBD images \cite{Teed2021_RAFT3D}
or LiDAR point clouds \cite{Liu2022_camliflow} 
-- all leading networks are restricted to the standard two-frame setting.
In this context, it is also surprising that the best multi-frame scene flow method on the KITTI benchmark is still a classical variational method which dates back to 2015 \cite{Vogel2015_PRSM}. 
This illustrates
that developing suitable multi-frame extensions of existing network 
architectures is indeed a difficult task. 


\skipping{
The previously discussed observation is also reflected in
recent works on 
multi-frame scene flow networks \cite{Schuster2021_DTF,Hur2021_MonoSceneFlow}. 
While on the KITTI benchmark,
moderate improvements of 2\%-4\%\footnote{We considered from both papers the best overall results in terms of the standard SF-all outlier measure:
\cite{Schuster2021_DTF}: DTF-SENSE (9.18) vs.\ SENSE (9.55),
\cite{Hur2021_MonoSceneFlow}: Multi-Mono-SF-ft (33.09) vs.\ Self-Mono-SF-ft (33.88).}have been reported by those works, the underlying concepts were either not integrated into state-of-the-art
baselines \cite{Schuster2021_DTF} or they were developed for the even more challenging self-supervised monocular setting \cite{Hur2021_MonoSceneFlow}, 
not allowing the results to be competitive with those of leading supervised methods.
Moreover, in the meantime, the accuracy of currently leading two-frame networks improved by a factor two \cite{Liu2022_camliflow,Yang2021_SegmentRigid,Teed2020_RAFT} compared to the baseline in \cite{Schuster2021_DTF}. Therefore, it remains unclear, if multi-frame extensions are still capable of providing similar improvements in performance as observed for previous baselines with  significantly lower accuracy.
}

The latter observation is also reflected in the recent literature on
multi-frame scene flow networks \cite{Schuster2021_DTF,Hur2021_MonoSceneFlow}. 
On the one hand, on the KITTI benchmark,
only slight improvements of 2\%-4\% have been reported compared to the underlying two-frame baselines\footnote{We considered from both papers the best overall results in terms of the standard SF-all outlier measure:
\cite{Schuster2021_DTF}: DTF-SENSE (9.18) vs.\ SENSE (9.55),
\cite{Hur2021_MonoSceneFlow}: Multi-Mono-SF-ft (33.09) vs.\ Self-Mono-SF-ft (33.88).}.
Evidently,  for recent multi-frame architectures, the often much larger training gains  
do not generalize well to the actual test data.
On the other hand, the proposed multi-frame concepts were either not incorporated into state-of-the-art
baselines \cite{Schuster2021_DTF} or they were developed for the even more challenging self-supervised monocular setting~\cite{Hur2021_MonoSceneFlow}.
This in turn gives an explanation for the relatively poor overall performance of recent multi-frame methods 
compared to currently leading supervised two-frame approaches.
And finally, as of today, the accuracy of leading two-frame approaches in general has improved by a factor two compared to the baseline in \cite{Schuster2021_DTF}; see \eg \cite{Liu2022_camliflow,Yang2021_SegmentRigid,Teed2020_RAFT}. This in turn raises the question
if suitable multi-frame extensions can be developed at all, if the underlying baseline already provides a sufficiently high accuracy.














\skipping{
In view of the aforementioned progress in scene flow estimation,
it is surprising that currently leading neural networks are limited to two time frames
\cite{Badki2021_binaryTTC,Li2021_acosf,Ma2019_DRISF,Teed2021_RAFT3D,Yang2020_opticalexpansion,Yang2021_SegmentRigid,Liu2022_camliflow}.
However, although temporal information is likely to to improve the results, developing suitable network architectures is a difficult task; see \eg \cite{Hur2021_MonoSceneFlow}. 
Hence, to benefit from \mbox{recent} advances in the two-frame case, the trend in current multi-frame networks goes towards extending existing architectures by suitable temporal concepts. Such concepts include the propagation of motion object labels over time \cite{Neoral2017_SceneFlowTemporal}, the integration of a trainable motion inverter that allows to generate predictions from the past \cite{Schuster2021_DTF} as well as the use of double cost volumes combined with  convolutional LSTMs to propagate the hidden state over time \cite{Hur2021_MonoSceneFlow}.
}






\skipping{
In this context, Maurer and Bruhn 
\cite{Maurer2018_ProFlow} showed recently in the field of optical flow estimation that so called {\em online prediction models} can be a valuable tool not only to exploit temporal information 
but also to exploit information from the actual image sequence at 
{\em run time}. The basic idea of such models is to train an individual small prediction network at run time {\em on the fly for each frame of each sequence} to forecast forward from backward flow, and, based on this prediction, to replace estimates at previously identified unreliable locations, \eg at occlusions. 
Not surprisingly, such a strategy has shown good generalization capabilities across different scenarios due to its inherent adaptivity by training the network on the actual data under consideration \cite{RVC2018}. Furthermore, the prediction network can be used in addition to any baseline method and it is simple to train, since the architecture is comparably lightweight.
}










\medskip
\noindent
\textbf{Contributions.}
In our paper, we show that multi-frame ideas are still valuable in the context of recent high-accuracy networks.
Building upon the RAFT-3D method~\cite{Teed2021_RAFT3D}, we present a novel multi-frame approach that allows to leverage the performance of current two-frame techniques.
In this context we make the following contributions:
(i) We propose a multi-frame architecture that particularly exploits the advantages of the underlying RAFT-3D architecture by combining a  based prediction step with a U-Net based fusion architecture. In this context, we also improve the underlying two-frame baseline by substituting the employed stereo approach.
(ii) Performing ablation studies and further experiments based on fourfold cross validation, we illustrate the benefits of the different architectural components of our method. In this way, we identify a fusion strategy that generalizes well to the test data.
(iii) With improvements of  for the baseline and  for the overall approach, we report much larger performance gains than existing multi-frame networks from the literature. These gains also lead to highly competitive results, eventually ranking second in the KITTI scene flow benchmark.



\section{Related work}


\paragraph{Multi-frame scene flow.}
Regarding the use of multiple time frames for scene flow estimation,
one can mainly distinguish three types of methods.
Like our approach, most of them rely on a three frame setting that has proven to be a good compromise between available temporal information and efficiency for both optical flow \cite{Maurer2018_ProFlow,Maurer2018,Ren2019_FlowTemporalFusion,Wulff2017_MRFlow} and scene flow \cite{Hur2021_MonoSceneFlow,Neoral2017_SceneFlowTemporal,Schuster2021_DTF,Schuster2020_SceneFlowFields,Taniai2017_MultiFrameSceneFlow}. 

(i) On the one hand, there are approaches that explicitly model multi-frame scene flow in terms of an {\em energy minimization} framework. Such approaches are the method of Vogel \etal \cite{Vogel2015_PRSM} that, based on piece-wise rigidity assumption, enforces a consistent piece-wise planar segmentation over time, the method of Golyanik \etal \cite{Golyanik2017_3DV} that follows a similar idea but relies on RGBD data instead of stereo sequences, the method of Taniai \etal \cite{Taniai2017_MultiFrameSceneFlow} which fuses estimates from optical flow and multi-frame time stereo, and the method of Neoral and \v{S}ochman \cite{Neoral2017_SceneFlowTemporal} that extends the two-frame scene flow approach of Menze and Geiger \cite{Menze2015_KITTI} by additionally propagating object labels over time.

(ii) On the other hand, there are {\em sparse-to-dense} methods that speed up the computation of energy-based methods by considering sparse matching strategies followed by a robust interpolation step. Such a method is the approach of Schuster \etal \cite{Schuster2020_SceneFlowFields}, which performs a sparse multi-frame matching relying on the assumption that the 3D motion in terms of the scene flow is constant over time. 

(iii) And finally, also {\em neural networks}
gained recently popularity in the context of multi-frame scene flow. 
Such methods include another approach of Schuster \etal \cite{Schuster2021_DTF} that predicts the forward from the backward flow based on a small learned motion inverter and subsequently fuses both flows using a convex fusion step, and the self-supervised monocular approach of Hur and Roth \cite{Hur2021_MonoSceneFlow} that uses a convolutional LSTM to encourage consistency over time.



While our multi-frame method is also based on a neural network that fuses flow
estimates,
its underlying strategy differs significantly 
from the one in
\cite{Schuster2021_DTF}.
On the one hand, our method not only relies on a much more advanced baseline, \ie RAFT-3D. Its entire architecture is also specifically tailored towards this baseline; \eg our method exploits both the local  parametrization as well as the valuable inputs of RAFT-3D's recurrent unit when adaptively integrating temporal information.
On the other hand, instead of learning a motion model via a small motion inverter that is naturally limited in its generalization capabilities and subsequently restricting the fusion to a convex combination, our method predicts the motion using a SE(3)-based extrapolation and then considers a more generalized U-Net based fusion step. 
While the SE(3)-based prediction holds in many scenarios, the generalized fusion step allows to implicitly learn possibly required corrections of this prediction. 















































\begin{figure*}
    \centering
\tikzset{overviewstyle/.style={anchor=north west, text=white, inner sep=0.8, text opacity=1, fill=black, opacity=0.4}}
    \begin{tikzpicture}
    \draw (0, 0) node[inner sep=0,anchor=north west] (img) {
	\includegraphics[width=\textwidth]{images/Method_v9}
	};
	\draw (0.30, -0.54) node[overviewstyle] {\footnotesize };
	\draw (0.11, -0.96) node[overviewstyle] {\footnotesize };
	\draw (0.30, -1.87) node[overviewstyle] {\footnotesize };
	\draw (0.11, -2.28) node[overviewstyle] {\footnotesize };
	\draw (0.30, -3.19) node[overviewstyle] {\footnotesize };
	\draw (0.11, -3.61) node[overviewstyle] {\footnotesize };
	\draw (6.09, -0.02) node[anchor=north west, inner sep=0] {\footnotesize };
	\draw (6.09, -2.68) node[anchor=north west, inner sep=0] {\footnotesize };
	\draw (10.4, -2.16) node[anchor=north west, inner sep=0] {\footnotesize };
    \draw (3.95, -0.82) node[scale=0.8] {\footnotesize \circled{1}};
    \draw (3.95, -3.48) node[scale=0.8] {\footnotesize \circled{1}};
    \draw (5.83, -1.18) node[scale=0.78] {\footnotesize \circled{2}};
    \draw (5.83, -1.50) node[scale=0.78] {\footnotesize \circled{2}};
    \draw (5.85, -3.84) node[scale=0.78] {\footnotesize \circled{2}};
    \draw (5.85, -4.16) node[scale=0.78] {\footnotesize \circled{2}};
    \draw (7.35, -0.99) node[scale=0.8] {\footnotesize \circled{3}};
    \draw (7.35, -3.65) node[scale=0.8] {\footnotesize \circled{3}};
    \draw (8.04, -0.54) node[scale=0.78] {\footnotesize \circled{4}};
    \draw (8.04, -3.20) node[scale=0.78] {\footnotesize \circled{4}};
    \draw (9.22, -3.53) node[scale=0.8] {\footnotesize \circled{5}};
    \draw (10.28, -0.25) node[scale=0.8] {\footnotesize \circled{6}};
    \draw (10.28, -2.30) node[scale=0.8] {\footnotesize \circled{6}};
    \draw (10.28, -2.91) node[scale=0.8] {\footnotesize \circled{6}};
    \draw (14.51, -1.85) node[scale=0.8] {\footnotesize \circled{7}};
\end{tikzpicture}
    \caption{Overview of our M-FUSE approach (see Sec.\ 3.2). We employ two shared instances of our baseline model to predict forward () and backward () scene flow as well as additional features used in our fusion U-Net to predict the final flow estimate.}
    \label{fig:overview}
\end{figure*}
\begin{figure*}
    \centering
    {
    \setlength\tabcolsep{1pt}
    \begin{tabular}{cccccc}
     &  &  & \emph{embVec} & \emph{corrCost} & \emph{dispRes}
    \\
    \multirow{2}{*}{
    \includegraphics[width=0.16\textwidth]{images/inputs/disp1.png}
    }\hspace{-0.8mm}
    &
    \includegraphics[width=0.16\textwidth]{images/inputs/dispch_FW}
    &
    \includegraphics[width=0.16\textwidth]{images/inputs/flow_FW}
    &
    \includegraphics[width=0.16\textwidth]{images/inputs/ae_FW_pca}
    &
    \includegraphics[width=0.16\textwidth]{images/inputs/corr_FW}
    &
    \includegraphics[width=0.16\textwidth]{images/inputs/dzhs_FW}
    \
    \mathcal{W}(D^{t+1}, u_\text{fw},v_\text{fw}) &- d'_\text{fw} \;,
    \\
    \mathcal{W}(D^{t-1}, u_\text{bw},v_\text{bw}) &- d'_\text{bw} \;,

\mathcal{L}_\text{fuse} \!=\! \sum_\mathbf{x} \left( \alpha \cdot | d' \!-\! d'_\text{gt} | + | u \!-\! u_\text{gt} | + | v \!-\! v_\text{gt} | + \epsilon \right)^\gamma.
-1.9mm]
\begin{tikzpicture}
\draw[fill=none,draw=none] (0, 0) rectangle (0.4,1.03) node[pos=.5, inner sep=0] {\small \emph{D2}};
\end{tikzpicture}
&
\begin{tikzpicture}
\draw (0, 0) node[imgstyle] (img) {
\includegraphics[trim={0 0 0 2.85cm},clip,width=0.241\textwidth]{images/improvement_vis/00_disp.png}
};
\draw (img.north west) node[labelstyle] {Baseline: 1.49\;\; M-FUSE: 1.71\;\; -14.8\%};
\end{tikzpicture}
&
\begin{tikzpicture}
\draw (0, 0) node[imgstyle] (img) {
\includegraphics[trim={0 0 0 2.85cm},clip,width=0.241\textwidth]{images/improvement_vis/03_disp.png}
};
\draw (img.north west) node[labelstyle] {Baseline: 2.14\;\; M-FUSE: 2.28\;\; -6.5\%};
\end{tikzpicture}
&
\begin{tikzpicture}
\draw (0, 0) node[imgstyle] (img) {
\includegraphics[trim={0 0 0 2.85cm},clip,width=0.241\textwidth]{images/improvement_vis/16_disp.png}
};
\draw (img.north west) node[labelstyle] {Baseline: 6.18\;\; M-FUSE: 4.78\;\; +22.7\%};
\end{tikzpicture}
&
\begin{tikzpicture}
\draw (0, 0) node[imgstyle] (img) {
\includegraphics[trim={0 0 0 2.85cm},clip,width=0.241\textwidth]{images/improvement_vis/18_disp.png}
};
\draw (img.north west) node[labelstyle] {Baseline: 47.81\,\; M-FUSE: 30.93\,\; +35.3\%};
\end{tikzpicture}
\-1.9mm]
\begin{tikzpicture}
\draw[fill=none,draw=none] (0, 0) rectangle (0.4,1.03) node[pos=.5, inner sep=0] {\small \emph{SF}};
\end{tikzpicture}
&
\begin{tikzpicture}
\draw (0, 0) node[imgstyle] (img) {
\includegraphics[trim={0 0 0 2.85cm},clip,width=0.241\textwidth]{images/improvement_vis/00_sceneflow.png}
};
\draw (img.north west) node[labelstyle] {Baseline: 7.55\;\; M-FUSE: 5.96\;\; +21.1\%};
\end{tikzpicture}
&
\begin{tikzpicture}
\draw (0, 0) node[imgstyle] (img) {
\includegraphics[trim={0 0 0 2.85cm},clip,width=0.241\textwidth]{images/improvement_vis/03_sceneflow.png}
};
\draw (img.north west) node[labelstyle] {Baseline: 6.64\;\; M-FUSE: 4.50\;\; +32.2\%};
\end{tikzpicture}
&
\begin{tikzpicture}
\draw (0, 0) node[imgstyle] (img) {
\includegraphics[trim={0 0 0 2.85cm},clip,width=0.241\textwidth]{images/improvement_vis/16_sceneflow.png}
};
\draw (img.north west) node[labelstyle] {Baseline: 9.70\;\; M-FUSE: 7.21\;\; +25.7\%};
\end{tikzpicture}
&
\begin{tikzpicture}
\draw (0, 0) node[imgstyle] (img) {
\includegraphics[trim={0 0 0 2.85cm},clip,width=0.241\textwidth]{images/improvement_vis/18_sceneflow.png}
};
\draw (img.north west) node[labelstyle] {Baseline: 52.77\,\; M-FUSE: 46.16\,\; +12.5\%};
\end{tikzpicture}
\end{tabular}
\caption{Qualitative evaluation of multi-frame improvements for four sequences of the KITTI benchmark (M-FUSE vs.\ baseline). \emph{From top to bottom:} reference frame, change in the outlier errors \emph{D2}, \emph{Fl} and \emph{SF}. \emph{Grey:} Both methods are inliers, \emph{blue:} M-FUSE is inlier and two-frame baseline is outlier, \emph{red:} two-frame baseline is inlier and M-FUSE is outlier, \emph{yellow:} both methods are outliers.}
\label{fig:multframe_improv}
}
\end{figure*}



\begin{figure*}
\newcommand*\imgtrimtop{3cm}
\centering{
\tikzset{labelstyle/.style={anchor=north west, text=white, inner sep=2, text opacity=1, scale=0.7, yshift=-1, xshift=1, fill=black, opacity=0.6}}
\tikzset{imgstyle/.style={inner sep=0,anchor=north west,outer sep=0,draw=none,line width=0}}
\setlength\tabcolsep{1pt}
\begin{tabular}{ccccc}
\begin{tikzpicture}
\draw (0, 0) node[imgstyle] (img) {
\includegraphics[trim={0 0 0 \imgtrimtop},clip,width=0.197\textwidth]{images/additional_images/RigidMaskISF/result_disp_img_0/000004_10.png}};
\draw (img.north west) node[labelstyle] {RigidMask+ISF};
\end{tikzpicture} &
\begin{tikzpicture}
\draw (0, 0) node[imgstyle] (img) {
\includegraphics[trim={0 0 0 \imgtrimtop},clip,width=0.197\textwidth]{images/additional_images/RigidMaskISF/errors_disp_img_0/000004_10.png}};
\draw (img.north west) node[labelstyle] {D2: 2.80};
\end{tikzpicture} &
\includegraphics[trim={0 0 0 \imgtrimtop},clip,width=0.197\textwidth]{images/additional_images/RigidMaskISF/result_flow_img/000004_10.png} &
\begin{tikzpicture}
\draw (0, 0) node[imgstyle] (img) {
\includegraphics[trim={0 0 0 \imgtrimtop},clip,width=0.197\textwidth]{images/additional_images/RigidMaskISF/errors_flow_img/000004_10.png}};
\draw (img.north west) node[labelstyle] {Fl: 3.96};
\end{tikzpicture} &
\begin{tikzpicture}
\draw (0, 0) node[imgstyle] (img) {
\includegraphics[trim={0 0 0 \imgtrimtop},clip,width=0.197\textwidth]{images/additional_images/RigidMaskISF/errors_scene_flow_img/000004_10.png}};
\draw (img.north west) node[labelstyle] {SF: 4.02};
\end{tikzpicture}
\-0.6mm]
\begin{tikzpicture}
\draw (0, 0) node[imgstyle] (img) {
\includegraphics[trim={0 0 0 \imgtrimtop},clip,width=0.197\textwidth]{images/additional_images/RAFT3D/result_disp_img_0/000004_10.png}};
\draw (img.north west) node[labelstyle] {RAFT-3D};
\end{tikzpicture} &
\begin{tikzpicture}
\draw (0, 0) node[imgstyle] (img) {
\includegraphics[trim={0 0 0 \imgtrimtop},clip,width=0.197\textwidth]{images/additional_images/RAFT3D/errors_disp_img_0/000004_10.png}};
\draw (img.north west) node[labelstyle] {D2: 2.61};
\end{tikzpicture} &
\includegraphics[trim={0 0 0 \imgtrimtop},clip,width=0.197\textwidth]{images/additional_images/RAFT3D/result_flow_img/000004_10.png} &
\begin{tikzpicture}
\draw (0, 0) node[imgstyle] (img) {
\includegraphics[trim={0 0 0 \imgtrimtop},clip,width=0.197\textwidth]{images/additional_images/RAFT3D/errors_flow_img/000004_10.png}};
\draw (img.north west) node[labelstyle] {Fl: 4.95};
\end{tikzpicture} &
\begin{tikzpicture}
\draw (0, 0) node[imgstyle] (img) {
\includegraphics[trim={0 0 0 \imgtrimtop},clip,width=0.197\textwidth]{images/additional_images/RAFT3D/errors_scene_flow_img/000004_10.png}};
\draw (img.north west) node[labelstyle] {SF: 5.54};
\end{tikzpicture}
\-0.6mm]
\begin{tikzpicture}
\draw (0, 0) node[imgstyle] (img) {
\includegraphics[trim={0 0 0 \imgtrimtop},clip,width=0.197\textwidth]{images/additional_images/CamLiFlow/result_disp_img_0/000007_10.png}};
\draw (img.north west) node[labelstyle] {CamLiFlow};
\end{tikzpicture} &
\begin{tikzpicture}
\draw (0, 0) node[imgstyle] (img) {
\includegraphics[trim={0 0 0 \imgtrimtop},clip,width=0.197\textwidth]{images/additional_images/CamLiFlow/errors_disp_img_0/000007_10.png}};
\draw (img.north west) node[labelstyle] {D2: 1.00};
\end{tikzpicture} &
\includegraphics[trim={0 0 0 \imgtrimtop},clip,width=0.197\textwidth]{images/additional_images/CamLiFlow/result_flow_img/000007_10.png} &
\begin{tikzpicture}
\draw (0, 0) node[imgstyle] (img) {
\includegraphics[trim={0 0 0 \imgtrimtop},clip,width=0.197\textwidth]{images/additional_images/CamLiFlow/errors_flow_img/000007_10.png}};
\draw (img.north west) node[labelstyle] {Fl: 3.03};
\end{tikzpicture} &
\begin{tikzpicture}
\draw (0, 0) node[imgstyle] (img) {
\includegraphics[trim={0 0 0 \imgtrimtop},clip,width=0.197\textwidth]{images/additional_images/CamLiFlow/errors_scene_flow_img/000007_10.png}};
\draw (img.north west) node[labelstyle] {SF: 3.42};
\end{tikzpicture}
\-0.6mm]
\begin{tikzpicture}
\draw (0, 0) node[imgstyle] (img) {
\includegraphics[trim={0 0 0 \imgtrimtop},clip,width=0.197\textwidth]{images/additional_images/MFUSE/result_disp_img_0/000007_10.png}};
\draw (img.north west) node[labelstyle] {M-FUSE};
\end{tikzpicture} &
\begin{tikzpicture}
\draw (0, 0) node[imgstyle] (img) {
\includegraphics[trim={0 0 0 \imgtrimtop},clip,width=0.197\textwidth]{images/additional_images/MFUSE/errors_disp_img_0/000007_10.png}};
\draw (img.north west) node[labelstyle] {D2: 0.99};
\end{tikzpicture} &
\includegraphics[trim={0 0 0 \imgtrimtop},clip,width=0.197\textwidth]{images/additional_images/MFUSE/result_flow_img/000007_10.png} &
\begin{tikzpicture}
\draw (0, 0) node[imgstyle] (img) {
\includegraphics[trim={0 0 0 \imgtrimtop},clip,width=0.197\textwidth]{images/additional_images/MFUSE/errors_flow_img/000007_10.png}};
\draw (img.north west) node[labelstyle] {Fl: 2.02};
\end{tikzpicture} &
\begin{tikzpicture}
\draw (0, 0) node[imgstyle] (img) {
\includegraphics[trim={0 0 0 \imgtrimtop},clip,width=0.197\textwidth]{images/additional_images/MFUSE/errors_scene_flow_img/000007_10.png}};
\draw (img.north west) node[labelstyle] {SF: 2.60};
\end{tikzpicture}
\end{tabular}
}
\caption{Qualitative comparison of our method, the original RAFT-3D, as well as the two top-performing approaches from the literature for two scenes using the visualizations provided by the KITTI benchmark~\cite{Menze2015_KITTI}. \emph{From left to right:} Target disparity visualization, corresponding \emph{D2} error plot, optical flow visualization, corresponding \emph{Fl} error plot, combined \emph{SF} error plot.}
\label{fig:addQualitative}
\end{figure*}

\subsection{Benchmark results}
In our first experiment, we compare the accuracy of our multi-frame scene flow method to that of other recent scene flow approaches from the literature. 
To this end, we computed the scene flow for the KITTI {\em test} split
both with our novel M-FUSE approach as well as with its underlying two-frame baseline and submitted the corresponding flow fields to the official benchmark~\cite{Menze2015_KITTI}.
To this end, we can not only show total improvements but also investigate the influence of the improved stereo method we employ.
Table~\ref{tab:KITTI_results} shows the obtained results together with the results of the ten top-ranked published scene flow methods.
Thereby, it lists the standard outlier rates 
\emph{D1} and \emph{D2} for the disparities at time  and , the optical flow error \emph{Fl} and the scene flow error \emph{SF}.
These errors are evaluated on \emph{all} pixels, as well as separately for static \emph{background} (bg) objects only moving due to camera motion and for dynamic \emph{foreground} (fg) objects that move independently;
see~\cite{Menze2015_KITTI} for details.
Additionally, for each outlier rate, the table shows relative improvements of the baseline and our method with respect to RAFT-3D as well as the relative improvements of our multi-frame approach compared to the two-frame baseline.

As one can see, already our baseline (RAFT-3D, with LEA\-Stereo) shows significantly improved results compared to the original RAFT-3D approach (with GANet).
In this context, the total gain of 9.9\% can be mainly attributed to strong improvements in the background region.
Our full \mbox{M-FUSE} approach then improves these results even further, outperforming RAFT-3D by 16.3\%.
Thereby, it also shows strong gains in the foreground, which are due to the consideration of multi-frame information (see M-FUSE vs.\ Baseline).
As a result, on the KITTI benchmark our method ranks second for all pixels, and first for foreground regions.

In Figure~\ref{fig:multframe_improv} we analyze the multi-frame improvements for four exemplary KITTI sequences.
In accordance with the numbers in Table~\ref{tab:KITTI_results}, we observe that (i) multi-frame improvements are strongest for the optical flow error compared to the disparity error and (ii) the improvements are most prominent on the individually moving foreground objects.
Figure~\ref{fig:addQualitative} visually shows improvements for background regions (\emph{top}) and foreground objects (\emph{bottom}).

\subsection{Ablations}

We ablate our model architecture in Table~\ref{tab:ablations}.
For these and all following experiments, we perform 4-fold cross validation on the KITTI \emph{train} split for more reliable evaluations with only limited data available.
Note that we omit the error measure for \emph{D1} in the tables since it is identical.

\vspace{1cm}
\medskip
\noindent
\textbf{Feature aggregation.}
Our U-Net computes additive increments for previous layers at the same resolution, which leads to a residual structure.
We compare this approach to the 
strategy presented in~\cite{Ronneberger2015_UNet},
where feature maps are concatenated and not summed up before a convolution.
Using additive residual connections slightly improves results.

\medskip
\noindent
\textbf{Fusion module depth.}
In a second study, we ablate the depth of our fusion U-Net by comparing variants with two, three and four levels.
While a two-level U-Net still gives on-par results in the \emph{D2} error, our three-level U-Net 
outperforms both the other networks 
in the \emph{Fl} and \emph{SF} errors.



\begin{table}
\caption{Ablation study. We show 4-fold cross validation results on KITTI \emph{train} in terms of the D2, Fl and SF errors~\cite{Menze2015_KITTI} as well as the number of parameters in millions.}
\label{tab:ablations}
\begin{center}
{\setlength\tabcolsep{2.7pt}
\begin{tabular}{lcccc}
\toprule
& D2 & Fl & SF & \#param\!
\\
\midrule
\midrule
two-frame & 1.81 & 3.67 & 4.07
\\
\midrule
\midrule
\multicolumn{4}{l}{\emph{Feature aggregation}}
\\
\midrule
concat. & 2.08 & 3.42 & 3.99 & 2.56
\\
add (ours) & \textbf{1.99} & \textbf{3.21} & \textbf{3.82} & 2.38
\\
\midrule
\midrule
\multicolumn{4}{l}{\emph{Fusion module depth}}
\\
\midrule
2 levels & \textbf{1.99} & 3.40 & 4.02 & 0.53
\\
3 levels (ours) & \textbf{1.99} & \textbf{3.21} & \textbf{3.82} & 2.38
\\
4 levels & 2.06 & 3.34 & 4.02 & 9.79
\\
\midrule
\midrule
\multicolumn{4}{l}{\emph{Additional fusion inputs}}
\\
\midrule
none & 2.72 & 3.33 & 4.62 & 2.36
\\
corrCost,dispRes & \textbf{1.87} & 3.36 & 3.83 & 2.36
\\
corrCost,embVec & 2.29 & 3.71 & 4.58 & 2.38
\\
dispRes,embVec & 1.99 & 3.43 & 3.97 & 2.38
\\
corrCost,dispRes,embVec (ours) & 1.99 & \textbf{3.21} & \textbf{3.82} & 2.38
\\
corrCost,dispRes,embVec,  & 2.10 & 3.27 & 3.96 & 2.38
\\
\bottomrule
\end{tabular}
}
\end{center}
\end{table}

\medskip
\noindent
\textbf{Additional fusion inputs.}
Our fusion network makes use of forward and backward scene flow estimates.
In a larger set of experiments we determined which additional inputs to our fusion module are useful.
To this end, we compare our set of additional inputs (correlation costs, disparity residuals and rigid motion embedding vectors) to omitting all of them (none) and to omitting each of them individually, to assess their individual contribution.
The results show that omitting all additional features significantly worsens results, which indicates that valuable information is contained in our set of features.
Further, we see that omitting the rigid motion embedding vectors gives inconclusive results compared to our method, with superior \emph{D2} results but a worse \emph{Fl} error.
The disparity residuals seem most essential:
When removed, the resulting quality lowers significantly for all measures.
Further, removing correlation costs has a slight negative impact.
Additionally, we investigated if adding the reference frame  to the set of inputs \cite{Ren2019_FlowTemporalFusion} is helpful.
However, this did not improve results any further, presumably because the correlation cost already provides sufficient information.

We show additional ablations with less conclusive results in the supplementary material.

\subsection{Scene flow parametrization}
Next, we investigate the influence of the underlying scene flow parametrization in our fusion module.
Table~\ref{tab:sf_param} compares the image-space parametrizations of optical flow and target disparity  against optical flow and the change in disparity .
Additionally, we investigate the world-space 3D motion vector parametrization .

Evidently, the image-space parametrizations outperform the 3D vector parametrization by a large margin in the optical flow error \emph{Fl}.
We argue that this is due to the error measures that are employed in scene flow estimation, which also work in image space.
Among the image-space parametrizations, using the change in disparity instead of the target disparity yields better results.
Presumably, predicting the disparity change (motion) instead of the target disparity (structure) bears a greater resemblance to the optical flow, which renders this strategy superior for the joint prediction.
\begin{table}
\caption{Influence of the scene flow parametrization.}
\label{tab:sf_param}
\begin{center}
\begin{tabular}{lccc}
\toprule
& D2 & Fl & SF
\\
\midrule
\midrule
 & 2.06 & 3.49 & 4.13
\\
 (ours) & \textbf{1.99} & \textbf{3.21} & \textbf{3.82}
\\
 & 2.04 & 8.50 & 8.79
\\
\bottomrule
\end{tabular}
\end{center}
\end{table}

\subsection{Comparison of multi-frame strategies}
\begin{table}
\setlength\tabcolsep{4pt}
\caption{Comparison of multi-frame strategies}
\label{tab:comparisonStrategies}
\begin{center}
\begin{tabular}{lccc}
\toprule
& D2 & Fl & SF
\\
\midrule
\midrule
two-frame & \textbf{1.81} & 3.67 & 4.07
\\
\midrule
warm-start (inv. backward) & 2.59 & 5.29 & 5.73
\\
warm-start (fw-warped prev.) & 2.23 & 4.48 & 4.88
\\
\midrule
learned inv + mask fusion & 2.06 & 3.96 & 4.39
\\
\midrule
specialized U-Net (bw-warped prev.) & 2.10 & 3.78 & 4.26
\\
specialized U-Net (inv. backward) & 2.01 & 3.59 & 4.05
\\
\midrule
M-FUSE & 1.99 & \textbf{3.21} & \textbf{3.82}
\\
\bottomrule
\end{tabular}
\end{center}
\end{table}

Finally, we compare our approach to three multi-frame strategies available in the literature:
Warm-starting the method, a learned inversion with mask-based fusion and a specialized U-Net with additional inputs; see Table~\ref{tab:comparisonStrategies}.

First, the warm-start initialization strategy has been shown to be highly successful in recent recurrent networks~\cite{Teed2020_RAFT}.
We considered two variants for our baseline approach:
For one, we used the matrix-inverted backward flow as an initialization, in contrast to the identity matrix initialization from~\cite{Teed2021_RAFT3D}.
For the other, we initialized with the previous forward scene flow that is forward-warped in the corresponding Lie algebra~\cite{Teed2021_Lie} using the estimated optical flow.
In Table~\ref{tab:comparisonStrategies}, both approaches perform considerably worse than the two-frame baseline,
even though forward-warping yields better results than inverted backward flow.
This is in line with previous studies, where warm-start on the KITTI dataset did not yield improvements~\cite{Teed2020_RAFT}.



Second, we considered a recent strategy that relies on a learned backward-to-forward inverter~\cite{Maurer2018_ProFlow,Schuster2021_DTF} followed by a predicted fusion mask that linearly combines forward and backward estimates~\cite{Schuster2021_DTF}.
We reimplemented the inversion and fusion module from~\cite{Schuster2021_DTF} and pretrained the former, before using these modules in our method.
For comparability, we adapted the modules to our three-channel prediction case, keeping  fixed.
In Table~\ref{tab:comparisonStrategies}, this strategy clearly yields worse results than our approach.
We attribute this to the simplistic structure of the motion model and the restrictive convex combination of flow inputs.



Third, we investigated a strategy that employs the specialized fusion U-Net from FlowNet2~\cite{Ilg2017_Flownet2} for fusing optical flow estimates~\cite{Ren2019_FlowTemporalFusion} guided by a brightness constancy map and the reference image.
To this end, we extended this fusion module to the scene flow setting and embedded it in our approach.
For a fair comparison, we also added disparity residuals to its fusion inputs.
We evaluated two variants, one with backward-warping the previous flow estimate as in~\cite{Ren2019_FlowTemporalFusion}, and one with inverted backward flow, as in our approach.
While only the approach with inversion is able to reach results on-par with the two-frame baseline, both cannot keep up with the results achieved by our method.


\subsection{Timing and parameter counts}


\begin{table}
\setlength\tabcolsep{3.5pt}
\caption{Parameter Count and Timing}
\label{tab:paramcount}
\begin{center}
\begin{tabular}{lcccccc}
\toprule
& \multicolumn{2}{c}{stereo} & \multicolumn{2}{c}{scene flow} & \multicolumn{2}{c}{total}
\\
\midrule
RAFT-3D\;\; & 6.6M & 4.0s\;\; & 44.9M & 0.4s\;\; & 51.4M & 4.4s
\\
Baseline & 1.8M & 0.8s\;\; & 44.9M & 0.4s\;\; & 46.7M & 1.2s
\\
M-FUSE & 1.8M & 1.2s\;\; & 47.2M & 1.3s\;\; & 49.0M & 2.5s
\\
\bottomrule
\end{tabular}
\end{center}
\end{table}



Table \ref{tab:paramcount} shows that our method takes a total of around 2.5s per frame for inference on a NVIDIA GeForce RTX 2080 Ti with 51.4M parameters.
The runtime is composed of 1.2s for stereo (30.4s LEAStereo) and 1.3s for scene flow (20.4s RAFT-3D baseline + 0.5s fusion).
While runtime and parameter count is increased compared to the two-frame baseline, our method is still faster and more parameter-efficient than the original RAFT-3D approach due to the fast and lightweight LEAStereo method~\cite{Cheng2020_LEAStereo}.





\section{Conclusion}
We proposed a novel multi-frame scene flow approach that leverages the performance of recent high accuracy two-frame methods. To this end, we developed an improved RAFT-3D baseline and embedded it into a U-Net-based fusion approach that adaptively integrates temporal information by combining an -based extrapolation of the backward flow with the jointly estimated forward flow. 
The achieved results clearly demonstrate that our strategy of explicitly tailoring our architecture towards the underlying baseline pays off.
With more than 16\% improvements compared to the original RAFT-3D approach, they show significantly larger improvements than other multi-frame networks in the literature. Moreover, in absolute accuracy  
our method ranks second in the public KITTI benchmark, clearly outperforming all other multi-frame approaches. 



\medskip
\noindent
\textbf{Acknowledgements.}
Funded by the Deutsche Forschungsgemeinschaft (DFG, German Research Foundation) -- Project-ID 251654672 -- TRR 161 (B04).
A.J.\ and J.S.\ acknowledge support from the International Max Planck Research School for Intelligent Systems (IMPRS-IS).


{\small
\bibliographystyle{ieee_fullname}
\bibliography{references}
}

\clearpage
\appendix

In this supplementary material, we provide a visualization of our fusion U-Net, additional ablations and additional qualitative results.

\section{Architecture of the U-Net}
Figure~\ref{fig:UNet} shows the architecture of our 3-level U-Net with residual connections.
\begin{figure}[h!]
    \centering
    \includegraphics[width=0.8\linewidth]{images/UNet_v2}
    \caption{Architecture of our fusion U-Net.}
    \label{fig:UNet}
\end{figure}


\section{Additional ablations}
In addition to the ablations in the main paper, we conducted two more experiments as shown in Table~\ref{tab:addablations}.

\medskip
\noindent
\textbf{In-between convolutions.}
As can be seen in Figure~\ref{fig:UNet}, in every depth level for the contracting as well as the expanding part one additional in-between convolutional layer is used to process information.
Thus, we performed an ablation over several options: completely omitting this layer (none), having one (1 conv) or two (2 convs) convolutions, or using a residual block~\cite{He2016_ResNet} (resblock).
The results for none, one or two convolutional layers are inconclusive, with no significant best option.
As a compromise, we chose one convolutional layer for our method since it is most similar to other U-Nets in the literature.
Finally, despite being most closely related to the two convolutions, the residual block slightly decreases the quality compared to all other cases.



\medskip
\noindent
\textbf{Image features.}
Finally, we compare two options to encode image-related features guiding our fusion module.
The first option is to utilize the learned correlation cost from our baseline, which is upsampled from 1/8th of the resolution.
The second option is a full-resolution brightness constancy error map~\cite{Ren2019_FlowTemporalFusion} as the  distance between the warped and original image.
As one can see, the learned correlation features outperform the brightness constancy maps slightly -- although the former are upsampled from lower resolution.



\begin{table}
\caption{Additional ablations. We show 4-fold cross validation results on KITTI \emph{train} in terms of the D2, Fl and SF errors as well as the number of parameters in millions.}
\label{tab:addablations}
\begin{center}
{\setlength\tabcolsep{2.7pt}
\begin{tabular}{lcccc}
\toprule
& D2 & Fl & SF & \#param\!
\\
\midrule
\midrule
two-frame & 1.81 & 3.67 & 4.07
\\
\midrule
\midrule
\multicolumn{4}{l}{\emph{In-between convs}}
\\
\midrule
none & \textbf{1.99} & 3.33 & 3.89 & 1.42
\\
1 conv (ours) & \textbf{1.99} & 3.21 & \textbf{3.82} & 2.38
\\
2 convs & 2.06 & \textbf{3.19} & 3.84 & 3.34
\\
resblock & 2.08 & 3.47 & 3.99 & 3.34
\\
\midrule
\midrule
\multicolumn{4}{l}{\emph{Image features}}
\\
\midrule
corrCost (ours) & \textbf{1.99} & \textbf{3.21} & \textbf{3.82} & 2.38
\\
BCE & 2.00 & 3.33 & 3.96 & 2.38
\\
\bottomrule
\end{tabular}
}
\end{center}
\end{table}


\section{Additional qualitative results}
We show additional visual results from the KITTI benchmark in Figures~\ref{fig:begin}--\ref{fig:end}.


\begin{figure*}
\centering{
\tikzset{labelstyle/.style={anchor=north west, text=white, inner sep=2, text opacity=1, scale=0.7, yshift=-1, xshift=1, fill=black, opacity=0.6}}
\tikzset{imgstyle/.style={inner sep=0,anchor=north west,outer sep=0,draw=none,line width=0}}
\setlength\tabcolsep{1pt}
\begin{tabular}{ccccc}
\begin{tikzpicture}
\draw (0, 0) node[imgstyle] (img) {
\includegraphics[trim={0 0 0 2.85cm},clip,width=0.197\textwidth]{images/additional_images/RigidMaskISF/result_disp_img_0/000002_10.png}};
\draw (img.north west) node[labelstyle] {RigidMask+ISF};
\end{tikzpicture} &
\begin{tikzpicture}
\draw (0, 0) node[imgstyle] (img) {
\includegraphics[trim={0 0 0 2.85cm},clip,width=0.197\textwidth]{images/additional_images/RigidMaskISF/errors_disp_img_0/000002_10.png}};
\draw (img.north west) node[labelstyle] {D2: 2.26};
\end{tikzpicture} &
\includegraphics[trim={0 0 0 2.85cm},clip,width=0.197\textwidth]{images/additional_images/RigidMaskISF/result_flow_img/000002_10.png} &
\begin{tikzpicture}
\draw (0, 0) node[imgstyle] (img) {
\includegraphics[trim={0 0 0 2.85cm},clip,width=0.197\textwidth]{images/additional_images/RigidMaskISF/errors_flow_img/000002_10.png}};
\draw (img.north west) node[labelstyle] {Fl: 3.24};
\end{tikzpicture} &
\begin{tikzpicture}
\draw (0, 0) node[imgstyle] (img) {
\includegraphics[trim={0 0 0 2.85cm},clip,width=0.197\textwidth]{images/additional_images/RigidMaskISF/errors_scene_flow_img/000002_10.png}};
\draw (img.north west) node[labelstyle] {SF: 3.66};
\end{tikzpicture}
\-0.6mm]
\begin{tikzpicture}
\draw (0, 0) node[imgstyle] (img) {
\includegraphics[trim={0 0 0 2.85cm},clip,width=0.197\textwidth]{images/additional_images/RAFT3D/result_disp_img_0/000002_10.png}};
\draw (img.north west) node[labelstyle] {RAFT-3D};
\end{tikzpicture} &
\begin{tikzpicture}
\draw (0, 0) node[imgstyle] (img) {
\includegraphics[trim={0 0 0 2.85cm},clip,width=0.197\textwidth]{images/additional_images/RAFT3D/errors_disp_img_0/000002_10.png}};
\draw (img.north west) node[labelstyle] {D2: 2.65};
\end{tikzpicture} &
\includegraphics[trim={0 0 0 2.85cm},clip,width=0.197\textwidth]{images/additional_images/RAFT3D/result_flow_img/000002_10.png} &
\begin{tikzpicture}
\draw (0, 0) node[imgstyle] (img) {
\includegraphics[trim={0 0 0 2.85cm},clip,width=0.197\textwidth]{images/additional_images/RAFT3D/errors_flow_img/000002_10.png}};
\draw (img.north west) node[labelstyle] {Fl: 3.69};
\end{tikzpicture} &
\begin{tikzpicture}
\draw (0, 0) node[imgstyle] (img) {
\includegraphics[trim={0 0 0 2.85cm},clip,width=0.197\textwidth]{images/additional_images/RAFT3D/errors_scene_flow_img/000002_10.png}};
\draw (img.north west) node[labelstyle] {SF: 4.67};
\end{tikzpicture}
\-0.6mm]
\begin{tikzpicture}
\draw (0, 0) node[imgstyle] (img) {
\includegraphics[trim={0 0 0 2.85cm},clip,width=0.197\textwidth]{images/additional_images/CamLiFlow/result_disp_img_0/000003_10.png}};
\draw (img.north west) node[labelstyle] {CamLiFlow};
\end{tikzpicture} &
\begin{tikzpicture}
\draw (0, 0) node[imgstyle] (img) {
\includegraphics[trim={0 0 0 2.85cm},clip,width=0.197\textwidth]{images/additional_images/CamLiFlow/errors_disp_img_0/000003_10.png}};
\draw (img.north west) node[labelstyle] {D2: 1.99};
\end{tikzpicture} &
\includegraphics[trim={0 0 0 2.85cm},clip,width=0.197\textwidth]{images/additional_images/CamLiFlow/result_flow_img/000003_10.png} &
\begin{tikzpicture}
\draw (0, 0) node[imgstyle] (img) {
\includegraphics[trim={0 0 0 2.85cm},clip,width=0.197\textwidth]{images/additional_images/CamLiFlow/errors_flow_img/000003_10.png}};
\draw (img.north west) node[labelstyle] {Fl: 3.07};
\end{tikzpicture} &
\begin{tikzpicture}
\draw (0, 0) node[imgstyle] (img) {
\includegraphics[trim={0 0 0 2.85cm},clip,width=0.197\textwidth]{images/additional_images/CamLiFlow/errors_scene_flow_img/000003_10.png}};
\draw (img.north west) node[labelstyle] {SF: 3.25};
\end{tikzpicture}
\-0.6mm]
\begin{tikzpicture}
\draw (0, 0) node[imgstyle] (img) {
\includegraphics[trim={0 0 0 2.85cm},clip,width=0.197\textwidth]{images/additional_images/MFUSE/result_disp_img_0/000003_10.png}};
\draw (img.north west) node[labelstyle] {M-FUSE};
\end{tikzpicture} &
\begin{tikzpicture}
\draw (0, 0) node[imgstyle] (img) {
\includegraphics[trim={0 0 0 2.85cm},clip,width=0.197\textwidth]{images/additional_images/MFUSE/errors_disp_img_0/000003_10.png}};
\draw (img.north west) node[labelstyle] {D2: 2.28};
\end{tikzpicture} &
\includegraphics[trim={0 0 0 2.85cm},clip,width=0.197\textwidth]{images/additional_images/MFUSE/result_flow_img/000003_10.png} &
\begin{tikzpicture}
\draw (0, 0) node[imgstyle] (img) {
\includegraphics[trim={0 0 0 2.85cm},clip,width=0.197\textwidth]{images/additional_images/MFUSE/errors_flow_img/000003_10.png}};
\draw (img.north west) node[labelstyle] {Fl: 4.23};
\end{tikzpicture} &
\begin{tikzpicture}
\draw (0, 0) node[imgstyle] (img) {
\includegraphics[trim={0 0 0 2.85cm},clip,width=0.197\textwidth]{images/additional_images/MFUSE/errors_scene_flow_img/000003_10.png}};
\draw (img.north west) node[labelstyle] {SF: 4.50};
\end{tikzpicture}
\\
\begin{tikzpicture}
\draw (0, 0) node[imgstyle] (img) {
\includegraphics[trim={0 0 0 2.85cm},clip,width=0.197\textwidth]{images/additional_images/RigidMaskISF/result_disp_img_0/000005_10.png}};
\draw (img.north west) node[labelstyle] {RigidMask+ISF};
\end{tikzpicture} &
\begin{tikzpicture}
\draw (0, 0) node[imgstyle] (img) {
\includegraphics[trim={0 0 0 2.85cm},clip,width=0.197\textwidth]{images/additional_images/RigidMaskISF/errors_disp_img_0/000005_10.png}};
\draw (img.north west) node[labelstyle] {D2: 2.51};
\end{tikzpicture} &
\includegraphics[trim={0 0 0 2.85cm},clip,width=0.197\textwidth]{images/additional_images/RigidMaskISF/result_flow_img/000005_10.png} &
\begin{tikzpicture}
\draw (0, 0) node[imgstyle] (img) {
\includegraphics[trim={0 0 0 2.85cm},clip,width=0.197\textwidth]{images/additional_images/RigidMaskISF/errors_flow_img/000005_10.png}};
\draw (img.north west) node[labelstyle] {Fl: 1.64};
\end{tikzpicture} &
\begin{tikzpicture}
\draw (0, 0) node[imgstyle] (img) {
\includegraphics[trim={0 0 0 2.85cm},clip,width=0.197\textwidth]{images/additional_images/RigidMaskISF/errors_scene_flow_img/000005_10.png}};
\draw (img.north west) node[labelstyle] {SF: 2.60};
\end{tikzpicture}
\-0.6mm]
\begin{tikzpicture}
\draw (0, 0) node[imgstyle] (img) {
\includegraphics[trim={0 0 0 2.85cm},clip,width=0.197\textwidth]{images/additional_images/RAFT3D/result_disp_img_0/000005_10.png}};
\draw (img.north west) node[labelstyle] {RAFT-3D};
\end{tikzpicture} &
\begin{tikzpicture}
\draw (0, 0) node[imgstyle] (img) {
\includegraphics[trim={0 0 0 2.85cm},clip,width=0.197\textwidth]{images/additional_images/RAFT3D/errors_disp_img_0/000005_10.png}};
\draw (img.north west) node[labelstyle] {D2: 2.68};
\end{tikzpicture} &
\includegraphics[trim={0 0 0 2.85cm},clip,width=0.197\textwidth]{images/additional_images/RAFT3D/result_flow_img/000005_10.png} &
\begin{tikzpicture}
\draw (0, 0) node[imgstyle] (img) {
\includegraphics[trim={0 0 0 2.85cm},clip,width=0.197\textwidth]{images/additional_images/RAFT3D/errors_flow_img/000005_10.png}};
\draw (img.north west) node[labelstyle] {Fl: 1.59};
\end{tikzpicture} &
\begin{tikzpicture}
\draw (0, 0) node[imgstyle] (img) {
\includegraphics[trim={0 0 0 2.85cm},clip,width=0.197\textwidth]{images/additional_images/RAFT3D/errors_scene_flow_img/000005_10.png}};
\draw (img.north west) node[labelstyle] {SF: 3.00};
\end{tikzpicture}
\-0.6mm]
\begin{tikzpicture}
\draw (0, 0) node[imgstyle] (img) {
\includegraphics[trim={0 0 0 2.85cm},clip,width=0.197\textwidth]{images/additional_images/CamLiFlow/result_disp_img_0/000000_10.png}};
\draw (img.north west) node[labelstyle] {CamLiFlow};
\end{tikzpicture} &
\begin{tikzpicture}
\draw (0, 0) node[imgstyle] (img) {
\includegraphics[trim={0 0 0 2.85cm},clip,width=0.197\textwidth]{images/additional_images/CamLiFlow/errors_disp_img_0/000000_10.png}};
\draw (img.north west) node[labelstyle] {D2: 1.59};
\end{tikzpicture} &
\includegraphics[trim={0 0 0 2.85cm},clip,width=0.197\textwidth]{images/additional_images/CamLiFlow/result_flow_img/000000_10.png} &
\begin{tikzpicture}
\draw (0, 0) node[imgstyle] (img) {
\includegraphics[trim={0 0 0 2.85cm},clip,width=0.197\textwidth]{images/additional_images/CamLiFlow/errors_flow_img/000000_10.png}};
\draw (img.north west) node[labelstyle] {Fl: 2.99};
\end{tikzpicture} &
\begin{tikzpicture}
\draw (0, 0) node[imgstyle] (img) {
\includegraphics[trim={0 0 0 2.85cm},clip,width=0.197\textwidth]{images/additional_images/CamLiFlow/errors_scene_flow_img/000000_10.png}};
\draw (img.north west) node[labelstyle] {SF: 3.53};
\end{tikzpicture}
\-0.6mm]
\begin{tikzpicture}
\draw (0, 0) node[imgstyle] (img) {
\includegraphics[trim={0 0 0 2.85cm},clip,width=0.197\textwidth]{images/additional_images/MFUSE/result_disp_img_0/000000_10.png}};
\draw (img.north west) node[labelstyle] {M-FUSE};
\end{tikzpicture} &
\begin{tikzpicture}
\draw (0, 0) node[imgstyle] (img) {
\includegraphics[trim={0 0 0 2.85cm},clip,width=0.197\textwidth]{images/additional_images/MFUSE/errors_disp_img_0/000000_10.png}};
\draw (img.north west) node[labelstyle] {D2: 1.71};
\end{tikzpicture} &
\includegraphics[trim={0 0 0 2.85cm},clip,width=0.197\textwidth]{images/additional_images/MFUSE/result_flow_img/000000_10.png} &
\begin{tikzpicture}
\draw (0, 0) node[imgstyle] (img) {
\includegraphics[trim={0 0 0 2.85cm},clip,width=0.197\textwidth]{images/additional_images/MFUSE/errors_flow_img/000000_10.png}};
\draw (img.north west) node[labelstyle] {Fl: 5.29};
\end{tikzpicture} &
\begin{tikzpicture}
\draw (0, 0) node[imgstyle] (img) {
\includegraphics[trim={0 0 0 2.85cm},clip,width=0.197\textwidth]{images/additional_images/MFUSE/errors_scene_flow_img/000000_10.png}};
\draw (img.north west) node[labelstyle] {SF: 5.96};
\end{tikzpicture}
\\
\begin{tikzpicture}
\draw (0, 0) node[imgstyle] (img) {
\includegraphics[trim={0 0 0 2.85cm},clip,width=0.197\textwidth]{images/additional_images/RigidMaskISF/result_disp_img_0/000001_10.png}};
\draw (img.north west) node[labelstyle] {RigidMask+ISF};
\end{tikzpicture} &
\begin{tikzpicture}
\draw (0, 0) node[imgstyle] (img) {
\includegraphics[trim={0 0 0 2.85cm},clip,width=0.197\textwidth]{images/additional_images/RigidMaskISF/errors_disp_img_0/000001_10.png}};
\draw (img.north west) node[labelstyle] {D2: 1.65};
\end{tikzpicture} &
\includegraphics[trim={0 0 0 2.85cm},clip,width=0.197\textwidth]{images/additional_images/RigidMaskISF/result_flow_img/000001_10.png} &
\begin{tikzpicture}
\draw (0, 0) node[imgstyle] (img) {
\includegraphics[trim={0 0 0 2.85cm},clip,width=0.197\textwidth]{images/additional_images/RigidMaskISF/errors_flow_img/000001_10.png}};
\draw (img.north west) node[labelstyle] {Fl: 2.51};
\end{tikzpicture} &
\begin{tikzpicture}
\draw (0, 0) node[imgstyle] (img) {
\includegraphics[trim={0 0 0 2.85cm},clip,width=0.197\textwidth]{images/additional_images/RigidMaskISF/errors_scene_flow_img/000001_10.png}};
\draw (img.north west) node[labelstyle] {SF: 3.09};
\end{tikzpicture}
\-0.6mm]
\begin{tikzpicture}
\draw (0, 0) node[imgstyle] (img) {
\includegraphics[trim={0 0 0 2.85cm},clip,width=0.197\textwidth]{images/additional_images/RAFT3D/result_disp_img_0/000001_10.png}};
\draw (img.north west) node[labelstyle] {RAFT-3D};
\end{tikzpicture} &
\begin{tikzpicture}
\draw (0, 0) node[imgstyle] (img) {
\includegraphics[trim={0 0 0 2.85cm},clip,width=0.197\textwidth]{images/additional_images/RAFT3D/errors_disp_img_0/000001_10.png}};
\draw (img.north west) node[labelstyle] {D2: 2.94};
\end{tikzpicture} &
\includegraphics[trim={0 0 0 2.85cm},clip,width=0.197\textwidth]{images/additional_images/RAFT3D/result_flow_img/000001_10.png} &
\begin{tikzpicture}
\draw (0, 0) node[imgstyle] (img) {
\includegraphics[trim={0 0 0 2.85cm},clip,width=0.197\textwidth]{images/additional_images/RAFT3D/errors_flow_img/000001_10.png}};
\draw (img.north west) node[labelstyle] {Fl: 3.84};
\end{tikzpicture} &
\begin{tikzpicture}
\draw (0, 0) node[imgstyle] (img) {
\includegraphics[trim={0 0 0 2.85cm},clip,width=0.197\textwidth]{images/additional_images/RAFT3D/errors_scene_flow_img/000001_10.png}};
\draw (img.north west) node[labelstyle] {SF: 4.88};
\end{tikzpicture}
\-0.6mm]
\end{tabular}
}
\caption{Qualitative comparison of our method, the original RAFT-3D, as well as the two top-performing approaches from the literature using the visualizations provided by the KITTI benchmark. \emph{From left to right:} Target disparity visualization, corresponding \emph{D2} error plot, optical flow visualization, corresponding \emph{Fl} error plot, combined \emph{SF} error plot.}
\label{fig:begin}
\end{figure*}





\begin{figure*}
\centering{
\tikzset{labelstyle/.style={anchor=north west, text=white, inner sep=2, text opacity=1, scale=0.7, yshift=-1, xshift=1, fill=black, opacity=0.6}}
\tikzset{imgstyle/.style={inner sep=0,anchor=north west,outer sep=0,draw=none,line width=0}}
\setlength\tabcolsep{1pt}
\begin{tabular}{ccccc}
\begin{tikzpicture}
\draw (0, 0) node[imgstyle] (img) {
\includegraphics[trim={0 0 0 2.85cm},clip,width=0.197\textwidth]{images/additional_images/RigidMaskISF/result_disp_img_0/000006_10.png}};
\draw (img.north west) node[labelstyle] {RigidMask+ISF};
\end{tikzpicture} &
\begin{tikzpicture}
\draw (0, 0) node[imgstyle] (img) {
\includegraphics[trim={0 0 0 2.85cm},clip,width=0.197\textwidth]{images/additional_images/RigidMaskISF/errors_disp_img_0/000006_10.png}};
\draw (img.north west) node[labelstyle] {D2: 3.96};
\end{tikzpicture} &
\includegraphics[trim={0 0 0 2.85cm},clip,width=0.197\textwidth]{images/additional_images/RigidMaskISF/result_flow_img/000006_10.png} &
\begin{tikzpicture}
\draw (0, 0) node[imgstyle] (img) {
\includegraphics[trim={0 0 0 2.85cm},clip,width=0.197\textwidth]{images/additional_images/RigidMaskISF/errors_flow_img/000006_10.png}};
\draw (img.north west) node[labelstyle] {Fl: 3.24};
\end{tikzpicture} &
\begin{tikzpicture}
\draw (0, 0) node[imgstyle] (img) {
\includegraphics[trim={0 0 0 2.85cm},clip,width=0.197\textwidth]{images/additional_images/RigidMaskISF/errors_scene_flow_img/000006_10.png}};
\draw (img.north west) node[labelstyle] {SF: 4.41};
\end{tikzpicture}
\-0.6mm]
\begin{tikzpicture}
\draw (0, 0) node[imgstyle] (img) {
\includegraphics[trim={0 0 0 2.85cm},clip,width=0.197\textwidth]{images/additional_images/RAFT3D/result_disp_img_0/000006_10.png}};
\draw (img.north west) node[labelstyle] {RAFT-3D};
\end{tikzpicture} &
\begin{tikzpicture}
\draw (0, 0) node[imgstyle] (img) {
\includegraphics[trim={0 0 0 2.85cm},clip,width=0.197\textwidth]{images/additional_images/RAFT3D/errors_disp_img_0/000006_10.png}};
\draw (img.north west) node[labelstyle] {D2: 4.18};
\end{tikzpicture} &
\includegraphics[trim={0 0 0 2.85cm},clip,width=0.197\textwidth]{images/additional_images/RAFT3D/result_flow_img/000006_10.png} &
\begin{tikzpicture}
\draw (0, 0) node[imgstyle] (img) {
\includegraphics[trim={0 0 0 2.85cm},clip,width=0.197\textwidth]{images/additional_images/RAFT3D/errors_flow_img/000006_10.png}};
\draw (img.north west) node[labelstyle] {Fl: 3.18};
\end{tikzpicture} &
\begin{tikzpicture}
\draw (0, 0) node[imgstyle] (img) {
\includegraphics[trim={0 0 0 2.85cm},clip,width=0.197\textwidth]{images/additional_images/RAFT3D/errors_scene_flow_img/000006_10.png}};
\draw (img.north west) node[labelstyle] {SF: 4.50};
\end{tikzpicture}
\-0.6mm]
\begin{tikzpicture}
\draw (0, 0) node[imgstyle] (img) {
\includegraphics[trim={0 0 0 2.85cm},clip,width=0.197\textwidth]{images/additional_images/CamLiFlow/result_disp_img_0/000008_10.png}};
\draw (img.north west) node[labelstyle] {CamLiFlow};
\end{tikzpicture} &
\begin{tikzpicture}
\draw (0, 0) node[imgstyle] (img) {
\includegraphics[trim={0 0 0 2.85cm},clip,width=0.197\textwidth]{images/additional_images/CamLiFlow/errors_disp_img_0/000008_10.png}};
\draw (img.north west) node[labelstyle] {D2: 0.80};
\end{tikzpicture} &
\includegraphics[trim={0 0 0 2.85cm},clip,width=0.197\textwidth]{images/additional_images/CamLiFlow/result_flow_img/000008_10.png} &
\begin{tikzpicture}
\draw (0, 0) node[imgstyle] (img) {
\includegraphics[trim={0 0 0 2.85cm},clip,width=0.197\textwidth]{images/additional_images/CamLiFlow/errors_flow_img/000008_10.png}};
\draw (img.north west) node[labelstyle] {Fl: 1.24};
\end{tikzpicture} &
\begin{tikzpicture}
\draw (0, 0) node[imgstyle] (img) {
\includegraphics[trim={0 0 0 2.85cm},clip,width=0.197\textwidth]{images/additional_images/CamLiFlow/errors_scene_flow_img/000008_10.png}};
\draw (img.north west) node[labelstyle] {SF: 1.55};
\end{tikzpicture}
\-0.6mm]
\begin{tikzpicture}
\draw (0, 0) node[imgstyle] (img) {
\includegraphics[trim={0 0 0 2.85cm},clip,width=0.197\textwidth]{images/additional_images/MFUSE/result_disp_img_0/000008_10.png}};
\draw (img.north west) node[labelstyle] {M-FUSE};
\end{tikzpicture} &
\begin{tikzpicture}
\draw (0, 0) node[imgstyle] (img) {
\includegraphics[trim={0 0 0 2.85cm},clip,width=0.197\textwidth]{images/additional_images/MFUSE/errors_disp_img_0/000008_10.png}};
\draw (img.north west) node[labelstyle] {D2: 0.63};
\end{tikzpicture} &
\includegraphics[trim={0 0 0 2.85cm},clip,width=0.197\textwidth]{images/additional_images/MFUSE/result_flow_img/000008_10.png} &
\begin{tikzpicture}
\draw (0, 0) node[imgstyle] (img) {
\includegraphics[trim={0 0 0 2.85cm},clip,width=0.197\textwidth]{images/additional_images/MFUSE/errors_flow_img/000008_10.png}};
\draw (img.north west) node[labelstyle] {Fl: 2.33};
\end{tikzpicture} &
\begin{tikzpicture}
\draw (0, 0) node[imgstyle] (img) {
\includegraphics[trim={0 0 0 2.85cm},clip,width=0.197\textwidth]{images/additional_images/MFUSE/errors_scene_flow_img/000008_10.png}};
\draw (img.north west) node[labelstyle] {SF: 2.66};
\end{tikzpicture}
\\
\begin{tikzpicture}
\draw (0, 0) node[imgstyle] (img) {
\includegraphics[trim={0 0 0 2.85cm},clip,width=0.197\textwidth]{images/additional_images/RigidMaskISF/result_disp_img_0/000009_10.png}};
\draw (img.north west) node[labelstyle] {RigidMask+ISF};
\end{tikzpicture} &
\begin{tikzpicture}
\draw (0, 0) node[imgstyle] (img) {
\includegraphics[trim={0 0 0 2.85cm},clip,width=0.197\textwidth]{images/additional_images/RigidMaskISF/errors_disp_img_0/000009_10.png}};
\draw (img.north west) node[labelstyle] {D2: 0.81};
\end{tikzpicture} &
\includegraphics[trim={0 0 0 2.85cm},clip,width=0.197\textwidth]{images/additional_images/RigidMaskISF/result_flow_img/000009_10.png} &
\begin{tikzpicture}
\draw (0, 0) node[imgstyle] (img) {
\includegraphics[trim={0 0 0 2.85cm},clip,width=0.197\textwidth]{images/additional_images/RigidMaskISF/errors_flow_img/000009_10.png}};
\draw (img.north west) node[labelstyle] {Fl: 2.21};
\end{tikzpicture} &
\begin{tikzpicture}
\draw (0, 0) node[imgstyle] (img) {
\includegraphics[trim={0 0 0 2.85cm},clip,width=0.197\textwidth]{images/additional_images/RigidMaskISF/errors_scene_flow_img/000009_10.png}};
\draw (img.north west) node[labelstyle] {SF: 2.63};
\end{tikzpicture}
\-0.6mm]
\begin{tikzpicture}
\draw (0, 0) node[imgstyle] (img) {
\includegraphics[trim={0 0 0 2.85cm},clip,width=0.197\textwidth]{images/additional_images/RAFT3D/result_disp_img_0/000009_10.png}};
\draw (img.north west) node[labelstyle] {RAFT-3D};
\end{tikzpicture} &
\begin{tikzpicture}
\draw (0, 0) node[imgstyle] (img) {
\includegraphics[trim={0 0 0 2.85cm},clip,width=0.197\textwidth]{images/additional_images/RAFT3D/errors_disp_img_0/000009_10.png}};
\draw (img.north west) node[labelstyle] {D2: 0.77};
\end{tikzpicture} &
\includegraphics[trim={0 0 0 2.85cm},clip,width=0.197\textwidth]{images/additional_images/RAFT3D/result_flow_img/000009_10.png} &
\begin{tikzpicture}
\draw (0, 0) node[imgstyle] (img) {
\includegraphics[trim={0 0 0 2.85cm},clip,width=0.197\textwidth]{images/additional_images/RAFT3D/errors_flow_img/000009_10.png}};
\draw (img.north west) node[labelstyle] {Fl: 5.16};
\end{tikzpicture} &
\begin{tikzpicture}
\draw (0, 0) node[imgstyle] (img) {
\includegraphics[trim={0 0 0 2.85cm},clip,width=0.197\textwidth]{images/additional_images/RAFT3D/errors_scene_flow_img/000009_10.png}};
\draw (img.north west) node[labelstyle] {SF: 5.86};
\end{tikzpicture}
\-0.6mm]
\begin{tikzpicture}
\draw (0, 0) node[imgstyle] (img) {
\includegraphics[trim={0 0 0 2.85cm},clip,width=0.197\textwidth]{images/additional_images/CamLiFlow/result_disp_img_0/000010_10.png}};
\draw (img.north west) node[labelstyle] {CamLiFlow};
\end{tikzpicture} &
\begin{tikzpicture}
\draw (0, 0) node[imgstyle] (img) {
\includegraphics[trim={0 0 0 2.85cm},clip,width=0.197\textwidth]{images/additional_images/CamLiFlow/errors_disp_img_0/000010_10.png}};
\draw (img.north west) node[labelstyle] {D2: 2.30};
\end{tikzpicture} &
\includegraphics[trim={0 0 0 2.85cm},clip,width=0.197\textwidth]{images/additional_images/CamLiFlow/result_flow_img/000010_10.png} &
\begin{tikzpicture}
\draw (0, 0) node[imgstyle] (img) {
\includegraphics[trim={0 0 0 2.85cm},clip,width=0.197\textwidth]{images/additional_images/CamLiFlow/errors_flow_img/000010_10.png}};
\draw (img.north west) node[labelstyle] {Fl: 2.24};
\end{tikzpicture} &
\begin{tikzpicture}
\draw (0, 0) node[imgstyle] (img) {
\includegraphics[trim={0 0 0 2.85cm},clip,width=0.197\textwidth]{images/additional_images/CamLiFlow/errors_scene_flow_img/000010_10.png}};
\draw (img.north west) node[labelstyle] {SF: 3.39};
\end{tikzpicture}
\-0.6mm]
\begin{tikzpicture}
\draw (0, 0) node[imgstyle] (img) {
\includegraphics[trim={0 0 0 2.85cm},clip,width=0.197\textwidth]{images/additional_images/MFUSE/result_disp_img_0/000010_10.png}};
\draw (img.north west) node[labelstyle] {M-FUSE};
\end{tikzpicture} &
\begin{tikzpicture}
\draw (0, 0) node[imgstyle] (img) {
\includegraphics[trim={0 0 0 2.85cm},clip,width=0.197\textwidth]{images/additional_images/MFUSE/errors_disp_img_0/000010_10.png}};
\draw (img.north west) node[labelstyle] {D2: 1.82};
\end{tikzpicture} &
\includegraphics[trim={0 0 0 2.85cm},clip,width=0.197\textwidth]{images/additional_images/MFUSE/result_flow_img/000010_10.png} &
\begin{tikzpicture}
\draw (0, 0) node[imgstyle] (img) {
\includegraphics[trim={0 0 0 2.85cm},clip,width=0.197\textwidth]{images/additional_images/MFUSE/errors_flow_img/000010_10.png}};
\draw (img.north west) node[labelstyle] {Fl: 1.73};
\end{tikzpicture} &
\begin{tikzpicture}
\draw (0, 0) node[imgstyle] (img) {
\includegraphics[trim={0 0 0 2.85cm},clip,width=0.197\textwidth]{images/additional_images/MFUSE/errors_scene_flow_img/000010_10.png}};
\draw (img.north west) node[labelstyle] {SF: 2.66};
\end{tikzpicture}
\\
\begin{tikzpicture}
\draw (0, 0) node[imgstyle] (img) {
\includegraphics[trim={0 0 0 2.85cm},clip,width=0.197\textwidth]{images/additional_images/RigidMaskISF/result_disp_img_0/000011_10.png}};
\draw (img.north west) node[labelstyle] {RigidMask+ISF};
\end{tikzpicture} &
\begin{tikzpicture}
\draw (0, 0) node[imgstyle] (img) {
\includegraphics[trim={0 0 0 2.85cm},clip,width=0.197\textwidth]{images/additional_images/RigidMaskISF/errors_disp_img_0/000011_10.png}};
\draw (img.north west) node[labelstyle] {D2: 1.06};
\end{tikzpicture} &
\includegraphics[trim={0 0 0 2.85cm},clip,width=0.197\textwidth]{images/additional_images/RigidMaskISF/result_flow_img/000011_10.png} &
\begin{tikzpicture}
\draw (0, 0) node[imgstyle] (img) {
\includegraphics[trim={0 0 0 2.85cm},clip,width=0.197\textwidth]{images/additional_images/RigidMaskISF/errors_flow_img/000011_10.png}};
\draw (img.north west) node[labelstyle] {Fl: 1.50};
\end{tikzpicture} &
\begin{tikzpicture}
\draw (0, 0) node[imgstyle] (img) {
\includegraphics[trim={0 0 0 2.85cm},clip,width=0.197\textwidth]{images/additional_images/RigidMaskISF/errors_scene_flow_img/000011_10.png}};
\draw (img.north west) node[labelstyle] {SF: 1.63};
\end{tikzpicture}
\-0.6mm]
\begin{tikzpicture}
\draw (0, 0) node[imgstyle] (img) {
\includegraphics[trim={0 0 0 2.85cm},clip,width=0.197\textwidth]{images/additional_images/RAFT3D/result_disp_img_0/000011_10.png}};
\draw (img.north west) node[labelstyle] {RAFT-3D};
\end{tikzpicture} &
\begin{tikzpicture}
\draw (0, 0) node[imgstyle] (img) {
\includegraphics[trim={0 0 0 2.85cm},clip,width=0.197\textwidth]{images/additional_images/RAFT3D/errors_disp_img_0/000011_10.png}};
\draw (img.north west) node[labelstyle] {D2: 1.25};
\end{tikzpicture} &
\includegraphics[trim={0 0 0 2.85cm},clip,width=0.197\textwidth]{images/additional_images/RAFT3D/result_flow_img/000011_10.png} &
\begin{tikzpicture}
\draw (0, 0) node[imgstyle] (img) {
\includegraphics[trim={0 0 0 2.85cm},clip,width=0.197\textwidth]{images/additional_images/RAFT3D/errors_flow_img/000011_10.png}};
\draw (img.north west) node[labelstyle] {Fl: 2.52};
\end{tikzpicture} &
\begin{tikzpicture}
\draw (0, 0) node[imgstyle] (img) {
\includegraphics[trim={0 0 0 2.85cm},clip,width=0.197\textwidth]{images/additional_images/RAFT3D/errors_scene_flow_img/000011_10.png}};
\draw (img.north west) node[labelstyle] {SF: 2.82};
\end{tikzpicture}
\-0.6mm]
\end{tabular}
}
\caption{Qualitative comparison of our method, the original RAFT-3D, as well as the two top-performing approaches from the literature using the visualizations provided by the KITTI benchmark. \emph{From left to right:} Target disparity visualization, corresponding \emph{D2} error plot, optical flow visualization, corresponding \emph{Fl} error plot, combined \emph{SF} error plot.}
\end{figure*}






\begin{figure*}
\centering{
\tikzset{labelstyle/.style={anchor=north west, text=white, inner sep=2, text opacity=1, scale=0.7, yshift=-1, xshift=1, fill=black, opacity=0.6}}
\tikzset{imgstyle/.style={inner sep=0,anchor=north west,outer sep=0,draw=none,line width=0}}
\setlength\tabcolsep{1pt}
\begin{tabular}{ccccc}
\begin{tikzpicture}
\draw (0, 0) node[imgstyle] (img) {
\includegraphics[trim={0 0 0 2.85cm},clip,width=0.197\textwidth]{images/additional_images/RigidMaskISF/result_disp_img_0/000012_10.png}};
\draw (img.north west) node[labelstyle] {RigidMask+ISF};
\end{tikzpicture} &
\begin{tikzpicture}
\draw (0, 0) node[imgstyle] (img) {
\includegraphics[trim={0 0 0 2.85cm},clip,width=0.197\textwidth]{images/additional_images/RigidMaskISF/errors_disp_img_0/000012_10.png}};
\draw (img.north west) node[labelstyle] {D2: 0.68};
\end{tikzpicture} &
\includegraphics[trim={0 0 0 2.85cm},clip,width=0.197\textwidth]{images/additional_images/RigidMaskISF/result_flow_img/000012_10.png} &
\begin{tikzpicture}
\draw (0, 0) node[imgstyle] (img) {
\includegraphics[trim={0 0 0 2.85cm},clip,width=0.197\textwidth]{images/additional_images/RigidMaskISF/errors_flow_img/000012_10.png}};
\draw (img.north west) node[labelstyle] {Fl: 1.02};
\end{tikzpicture} &
\begin{tikzpicture}
\draw (0, 0) node[imgstyle] (img) {
\includegraphics[trim={0 0 0 2.85cm},clip,width=0.197\textwidth]{images/additional_images/RigidMaskISF/errors_scene_flow_img/000012_10.png}};
\draw (img.north west) node[labelstyle] {SF: 1.10};
\end{tikzpicture}
\-0.6mm]
\begin{tikzpicture}
\draw (0, 0) node[imgstyle] (img) {
\includegraphics[trim={0 0 0 2.85cm},clip,width=0.197\textwidth]{images/additional_images/RAFT3D/result_disp_img_0/000012_10.png}};
\draw (img.north west) node[labelstyle] {RAFT-3D};
\end{tikzpicture} &
\begin{tikzpicture}
\draw (0, 0) node[imgstyle] (img) {
\includegraphics[trim={0 0 0 2.85cm},clip,width=0.197\textwidth]{images/additional_images/RAFT3D/errors_disp_img_0/000012_10.png}};
\draw (img.north west) node[labelstyle] {D2: 0.74};
\end{tikzpicture} &
\includegraphics[trim={0 0 0 2.85cm},clip,width=0.197\textwidth]{images/additional_images/RAFT3D/result_flow_img/000012_10.png} &
\begin{tikzpicture}
\draw (0, 0) node[imgstyle] (img) {
\includegraphics[trim={0 0 0 2.85cm},clip,width=0.197\textwidth]{images/additional_images/RAFT3D/errors_flow_img/000012_10.png}};
\draw (img.north west) node[labelstyle] {Fl: 1.05};
\end{tikzpicture} &
\begin{tikzpicture}
\draw (0, 0) node[imgstyle] (img) {
\includegraphics[trim={0 0 0 2.85cm},clip,width=0.197\textwidth]{images/additional_images/RAFT3D/errors_scene_flow_img/000012_10.png}};
\draw (img.north west) node[labelstyle] {SF: 1.22};
\end{tikzpicture}
\-0.6mm]
\begin{tikzpicture}
\draw (0, 0) node[imgstyle] (img) {
\includegraphics[trim={0 0 0 2.85cm},clip,width=0.197\textwidth]{images/additional_images/CamLiFlow/result_disp_img_0/000013_10.png}};
\draw (img.north west) node[labelstyle] {CamLiFlow};
\end{tikzpicture} &
\begin{tikzpicture}
\draw (0, 0) node[imgstyle] (img) {
\includegraphics[trim={0 0 0 2.85cm},clip,width=0.197\textwidth]{images/additional_images/CamLiFlow/errors_disp_img_0/000013_10.png}};
\draw (img.north west) node[labelstyle] {D2: 0.96};
\end{tikzpicture} &
\includegraphics[trim={0 0 0 2.85cm},clip,width=0.197\textwidth]{images/additional_images/CamLiFlow/result_flow_img/000013_10.png} &
\begin{tikzpicture}
\draw (0, 0) node[imgstyle] (img) {
\includegraphics[trim={0 0 0 2.85cm},clip,width=0.197\textwidth]{images/additional_images/CamLiFlow/errors_flow_img/000013_10.png}};
\draw (img.north west) node[labelstyle] {Fl: 1.74};
\end{tikzpicture} &
\begin{tikzpicture}
\draw (0, 0) node[imgstyle] (img) {
\includegraphics[trim={0 0 0 2.85cm},clip,width=0.197\textwidth]{images/additional_images/CamLiFlow/errors_scene_flow_img/000013_10.png}};
\draw (img.north west) node[labelstyle] {SF: 1.79};
\end{tikzpicture}
\-0.6mm]
\begin{tikzpicture}
\draw (0, 0) node[imgstyle] (img) {
\includegraphics[trim={0 0 0 2.85cm},clip,width=0.197\textwidth]{images/additional_images/MFUSE/result_disp_img_0/000013_10.png}};
\draw (img.north west) node[labelstyle] {M-FUSE};
\end{tikzpicture} &
\begin{tikzpicture}
\draw (0, 0) node[imgstyle] (img) {
\includegraphics[trim={0 0 0 2.85cm},clip,width=0.197\textwidth]{images/additional_images/MFUSE/errors_disp_img_0/000013_10.png}};
\draw (img.north west) node[labelstyle] {D2: 0.99};
\end{tikzpicture} &
\includegraphics[trim={0 0 0 2.85cm},clip,width=0.197\textwidth]{images/additional_images/MFUSE/result_flow_img/000013_10.png} &
\begin{tikzpicture}
\draw (0, 0) node[imgstyle] (img) {
\includegraphics[trim={0 0 0 2.85cm},clip,width=0.197\textwidth]{images/additional_images/MFUSE/errors_flow_img/000013_10.png}};
\draw (img.north west) node[labelstyle] {Fl: 1.55};
\end{tikzpicture} &
\begin{tikzpicture}
\draw (0, 0) node[imgstyle] (img) {
\includegraphics[trim={0 0 0 2.85cm},clip,width=0.197\textwidth]{images/additional_images/MFUSE/errors_scene_flow_img/000013_10.png}};
\draw (img.north west) node[labelstyle] {SF: 1.63};
\end{tikzpicture}
\\
\begin{tikzpicture}
\draw (0, 0) node[imgstyle] (img) {
\includegraphics[trim={0 0 0 2.85cm},clip,width=0.197\textwidth]{images/additional_images/RigidMaskISF/result_disp_img_0/000014_10.png}};
\draw (img.north west) node[labelstyle] {RigidMask+ISF};
\end{tikzpicture} &
\begin{tikzpicture}
\draw (0, 0) node[imgstyle] (img) {
\includegraphics[trim={0 0 0 2.85cm},clip,width=0.197\textwidth]{images/additional_images/RigidMaskISF/errors_disp_img_0/000014_10.png}};
\draw (img.north west) node[labelstyle] {D2: 1.64};
\end{tikzpicture} &
\includegraphics[trim={0 0 0 2.85cm},clip,width=0.197\textwidth]{images/additional_images/RigidMaskISF/result_flow_img/000014_10.png} &
\begin{tikzpicture}
\draw (0, 0) node[imgstyle] (img) {
\includegraphics[trim={0 0 0 2.85cm},clip,width=0.197\textwidth]{images/additional_images/RigidMaskISF/errors_flow_img/000014_10.png}};
\draw (img.north west) node[labelstyle] {Fl: 2.38};
\end{tikzpicture} &
\begin{tikzpicture}
\draw (0, 0) node[imgstyle] (img) {
\includegraphics[trim={0 0 0 2.85cm},clip,width=0.197\textwidth]{images/additional_images/RigidMaskISF/errors_scene_flow_img/000014_10.png}};
\draw (img.north west) node[labelstyle] {SF: 2.49};
\end{tikzpicture}
\-0.6mm]
\begin{tikzpicture}
\draw (0, 0) node[imgstyle] (img) {
\includegraphics[trim={0 0 0 2.85cm},clip,width=0.197\textwidth]{images/additional_images/RAFT3D/result_disp_img_0/000014_10.png}};
\draw (img.north west) node[labelstyle] {RAFT-3D};
\end{tikzpicture} &
\begin{tikzpicture}
\draw (0, 0) node[imgstyle] (img) {
\includegraphics[trim={0 0 0 2.85cm},clip,width=0.197\textwidth]{images/additional_images/RAFT3D/errors_disp_img_0/000014_10.png}};
\draw (img.north west) node[labelstyle] {D2: 1.60};
\end{tikzpicture} &
\includegraphics[trim={0 0 0 2.85cm},clip,width=0.197\textwidth]{images/additional_images/RAFT3D/result_flow_img/000014_10.png} &
\begin{tikzpicture}
\draw (0, 0) node[imgstyle] (img) {
\includegraphics[trim={0 0 0 2.85cm},clip,width=0.197\textwidth]{images/additional_images/RAFT3D/errors_flow_img/000014_10.png}};
\draw (img.north west) node[labelstyle] {Fl: 2.21};
\end{tikzpicture} &
\begin{tikzpicture}
\draw (0, 0) node[imgstyle] (img) {
\includegraphics[trim={0 0 0 2.85cm},clip,width=0.197\textwidth]{images/additional_images/RAFT3D/errors_scene_flow_img/000014_10.png}};
\draw (img.north west) node[labelstyle] {SF: 2.40};
\end{tikzpicture}
\-0.6mm]
\begin{tikzpicture}
\draw (0, 0) node[imgstyle] (img) {
\includegraphics[trim={0 0 0 2.85cm},clip,width=0.197\textwidth]{images/additional_images/CamLiFlow/result_disp_img_0/000015_10.png}};
\draw (img.north west) node[labelstyle] {CamLiFlow};
\end{tikzpicture} &
\begin{tikzpicture}
\draw (0, 0) node[imgstyle] (img) {
\includegraphics[trim={0 0 0 2.85cm},clip,width=0.197\textwidth]{images/additional_images/CamLiFlow/errors_disp_img_0/000015_10.png}};
\draw (img.north west) node[labelstyle] {D2: 3.69};
\end{tikzpicture} &
\includegraphics[trim={0 0 0 2.85cm},clip,width=0.197\textwidth]{images/additional_images/CamLiFlow/result_flow_img/000015_10.png} &
\begin{tikzpicture}
\draw (0, 0) node[imgstyle] (img) {
\includegraphics[trim={0 0 0 2.85cm},clip,width=0.197\textwidth]{images/additional_images/CamLiFlow/errors_flow_img/000015_10.png}};
\draw (img.north west) node[labelstyle] {Fl: 5.22};
\end{tikzpicture} &
\begin{tikzpicture}
\draw (0, 0) node[imgstyle] (img) {
\includegraphics[trim={0 0 0 2.85cm},clip,width=0.197\textwidth]{images/additional_images/CamLiFlow/errors_scene_flow_img/000015_10.png}};
\draw (img.north west) node[labelstyle] {SF: 5.73};
\end{tikzpicture}
\-0.6mm]
\begin{tikzpicture}
\draw (0, 0) node[imgstyle] (img) {
\includegraphics[trim={0 0 0 2.85cm},clip,width=0.197\textwidth]{images/additional_images/MFUSE/result_disp_img_0/000015_10.png}};
\draw (img.north west) node[labelstyle] {M-FUSE};
\end{tikzpicture} &
\begin{tikzpicture}
\draw (0, 0) node[imgstyle] (img) {
\includegraphics[trim={0 0 0 2.85cm},clip,width=0.197\textwidth]{images/additional_images/MFUSE/errors_disp_img_0/000015_10.png}};
\draw (img.north west) node[labelstyle] {D2: 3.52};
\end{tikzpicture} &
\includegraphics[trim={0 0 0 2.85cm},clip,width=0.197\textwidth]{images/additional_images/MFUSE/result_flow_img/000015_10.png} &
\begin{tikzpicture}
\draw (0, 0) node[imgstyle] (img) {
\includegraphics[trim={0 0 0 2.85cm},clip,width=0.197\textwidth]{images/additional_images/MFUSE/errors_flow_img/000015_10.png}};
\draw (img.north west) node[labelstyle] {Fl: 4.96};
\end{tikzpicture} &
\begin{tikzpicture}
\draw (0, 0) node[imgstyle] (img) {
\includegraphics[trim={0 0 0 2.85cm},clip,width=0.197\textwidth]{images/additional_images/MFUSE/errors_scene_flow_img/000015_10.png}};
\draw (img.north west) node[labelstyle] {SF: 5.40};
\end{tikzpicture}
\\
\begin{tikzpicture}
\draw (0, 0) node[imgstyle] (img) {
\includegraphics[trim={0 0 0 2.85cm},clip,width=0.197\textwidth]{images/additional_images/RigidMaskISF/result_disp_img_0/000016_10.png}};
\draw (img.north west) node[labelstyle] {RigidMask+ISF};
\end{tikzpicture} &
\begin{tikzpicture}
\draw (0, 0) node[imgstyle] (img) {
\includegraphics[trim={0 0 0 2.85cm},clip,width=0.197\textwidth]{images/additional_images/RigidMaskISF/errors_disp_img_0/000016_10.png}};
\draw (img.north west) node[labelstyle] {D2: 5.90};
\end{tikzpicture} &
\includegraphics[trim={0 0 0 2.85cm},clip,width=0.197\textwidth]{images/additional_images/RigidMaskISF/result_flow_img/000016_10.png} &
\begin{tikzpicture}
\draw (0, 0) node[imgstyle] (img) {
\includegraphics[trim={0 0 0 2.85cm},clip,width=0.197\textwidth]{images/additional_images/RigidMaskISF/errors_flow_img/000016_10.png}};
\draw (img.north west) node[labelstyle] {Fl: 6.48};
\end{tikzpicture} &
\begin{tikzpicture}
\draw (0, 0) node[imgstyle] (img) {
\includegraphics[trim={0 0 0 2.85cm},clip,width=0.197\textwidth]{images/additional_images/RigidMaskISF/errors_scene_flow_img/000016_10.png}};
\draw (img.north west) node[labelstyle] {SF: 8.01};
\end{tikzpicture}
\-0.6mm]
\begin{tikzpicture}
\draw (0, 0) node[imgstyle] (img) {
\includegraphics[trim={0 0 0 2.85cm},clip,width=0.197\textwidth]{images/additional_images/RAFT3D/result_disp_img_0/000016_10.png}};
\draw (img.north west) node[labelstyle] {RAFT-3D};
\end{tikzpicture} &
\begin{tikzpicture}
\draw (0, 0) node[imgstyle] (img) {
\includegraphics[trim={0 0 0 2.85cm},clip,width=0.197\textwidth]{images/additional_images/RAFT3D/errors_disp_img_0/000016_10.png}};
\draw (img.north west) node[labelstyle] {D2: 5.66};
\end{tikzpicture} &
\includegraphics[trim={0 0 0 2.85cm},clip,width=0.197\textwidth]{images/additional_images/RAFT3D/result_flow_img/000016_10.png} &
\begin{tikzpicture}
\draw (0, 0) node[imgstyle] (img) {
\includegraphics[trim={0 0 0 2.85cm},clip,width=0.197\textwidth]{images/additional_images/RAFT3D/errors_flow_img/000016_10.png}};
\draw (img.north west) node[labelstyle] {Fl: 7.23};
\end{tikzpicture} &
\begin{tikzpicture}
\draw (0, 0) node[imgstyle] (img) {
\includegraphics[trim={0 0 0 2.85cm},clip,width=0.197\textwidth]{images/additional_images/RAFT3D/errors_scene_flow_img/000016_10.png}};
\draw (img.north west) node[labelstyle] {SF: 8.14};
\end{tikzpicture}
\-0.6mm]
\end{tabular}
}
\caption{Qualitative comparison of our method, the original RAFT-3D, as well as the two top-performing approaches from the literature using the visualizations provided by the KITTI benchmark. \emph{From left to right:} Target disparity visualization, corresponding \emph{D2} error plot, optical flow visualization, corresponding \emph{Fl} error plot, combined \emph{SF} error plot.}
\end{figure*}




\begin{figure*}
\centering{
\tikzset{labelstyle/.style={anchor=north west, text=white, inner sep=2, text opacity=1, scale=0.7, yshift=-1, xshift=1, fill=black, opacity=0.6}}
\tikzset{imgstyle/.style={inner sep=0,anchor=north west,outer sep=0,draw=none,line width=0}}
\setlength\tabcolsep{1pt}
\begin{tabular}{ccccc}
\begin{tikzpicture}
\draw (0, 0) node[imgstyle] (img) {
\includegraphics[trim={0 0 0 2.85cm},clip,width=0.197\textwidth]{images/additional_images/RigidMaskISF/result_disp_img_0/000017_10.png}};
\draw (img.north west) node[labelstyle] {RigidMask+ISF};
\end{tikzpicture} &
\begin{tikzpicture}
\draw (0, 0) node[imgstyle] (img) {
\includegraphics[trim={0 0 0 2.85cm},clip,width=0.197\textwidth]{images/additional_images/RigidMaskISF/errors_disp_img_0/000017_10.png}};
\draw (img.north west) node[labelstyle] {D2: 1.65};
\end{tikzpicture} &
\includegraphics[trim={0 0 0 2.85cm},clip,width=0.197\textwidth]{images/additional_images/RigidMaskISF/result_flow_img/000017_10.png} &
\begin{tikzpicture}
\draw (0, 0) node[imgstyle] (img) {
\includegraphics[trim={0 0 0 2.85cm},clip,width=0.197\textwidth]{images/additional_images/RigidMaskISF/errors_flow_img/000017_10.png}};
\draw (img.north west) node[labelstyle] {Fl: 2.87};
\end{tikzpicture} &
\begin{tikzpicture}
\draw (0, 0) node[imgstyle] (img) {
\includegraphics[trim={0 0 0 2.85cm},clip,width=0.197\textwidth]{images/additional_images/RigidMaskISF/errors_scene_flow_img/000017_10.png}};
\draw (img.north west) node[labelstyle] {SF: 3.36};
\end{tikzpicture}
\-0.6mm]
\begin{tikzpicture}
\draw (0, 0) node[imgstyle] (img) {
\includegraphics[trim={0 0 0 2.85cm},clip,width=0.197\textwidth]{images/additional_images/RAFT3D/result_disp_img_0/000017_10.png}};
\draw (img.north west) node[labelstyle] {RAFT-3D};
\end{tikzpicture} &
\begin{tikzpicture}
\draw (0, 0) node[imgstyle] (img) {
\includegraphics[trim={0 0 0 2.85cm},clip,width=0.197\textwidth]{images/additional_images/RAFT3D/errors_disp_img_0/000017_10.png}};
\draw (img.north west) node[labelstyle] {D2: 1.31};
\end{tikzpicture} &
\includegraphics[trim={0 0 0 2.85cm},clip,width=0.197\textwidth]{images/additional_images/RAFT3D/result_flow_img/000017_10.png} &
\begin{tikzpicture}
\draw (0, 0) node[imgstyle] (img) {
\includegraphics[trim={0 0 0 2.85cm},clip,width=0.197\textwidth]{images/additional_images/RAFT3D/errors_flow_img/000017_10.png}};
\draw (img.north west) node[labelstyle] {Fl: 2.92};
\end{tikzpicture} &
\begin{tikzpicture}
\draw (0, 0) node[imgstyle] (img) {
\includegraphics[trim={0 0 0 2.85cm},clip,width=0.197\textwidth]{images/additional_images/RAFT3D/errors_scene_flow_img/000017_10.png}};
\draw (img.north west) node[labelstyle] {SF: 3.30};
\end{tikzpicture}
\-0.6mm]
\begin{tikzpicture}
\draw (0, 0) node[imgstyle] (img) {
\includegraphics[trim={0 0 0 2.85cm},clip,width=0.197\textwidth]{images/additional_images/CamLiFlow/result_disp_img_0/000018_10.png}};
\draw (img.north west) node[labelstyle] {CamLiFlow};
\end{tikzpicture} &
\begin{tikzpicture}
\draw (0, 0) node[imgstyle] (img) {
\includegraphics[trim={0 0 0 2.85cm},clip,width=0.197\textwidth]{images/additional_images/CamLiFlow/errors_disp_img_0/000018_10.png}};
\draw (img.north west) node[labelstyle] {D2: 33.24};
\end{tikzpicture} &
\includegraphics[trim={0 0 0 2.85cm},clip,width=0.197\textwidth]{images/additional_images/CamLiFlow/result_flow_img/000018_10.png} &
\begin{tikzpicture}
\draw (0, 0) node[imgstyle] (img) {
\includegraphics[trim={0 0 0 2.85cm},clip,width=0.197\textwidth]{images/additional_images/CamLiFlow/errors_flow_img/000018_10.png}};
\draw (img.north west) node[labelstyle] {Fl: 46.64};
\end{tikzpicture} &
\begin{tikzpicture}
\draw (0, 0) node[imgstyle] (img) {
\includegraphics[trim={0 0 0 2.85cm},clip,width=0.197\textwidth]{images/additional_images/CamLiFlow/errors_scene_flow_img/000018_10.png}};
\draw (img.north west) node[labelstyle] {SF: 49.97};
\end{tikzpicture}
\-0.6mm]
\begin{tikzpicture}
\draw (0, 0) node[imgstyle] (img) {
\includegraphics[trim={0 0 0 2.85cm},clip,width=0.197\textwidth]{images/additional_images/MFUSE/result_disp_img_0/000018_10.png}};
\draw (img.north west) node[labelstyle] {M-FUSE};
\end{tikzpicture} &
\begin{tikzpicture}
\draw (0, 0) node[imgstyle] (img) {
\includegraphics[trim={0 0 0 2.85cm},clip,width=0.197\textwidth]{images/additional_images/MFUSE/errors_disp_img_0/000018_10.png}};
\draw (img.north west) node[labelstyle] {D2: 30.93};
\end{tikzpicture} &
\includegraphics[trim={0 0 0 2.85cm},clip,width=0.197\textwidth]{images/additional_images/MFUSE/result_flow_img/000018_10.png} &
\begin{tikzpicture}
\draw (0, 0) node[imgstyle] (img) {
\includegraphics[trim={0 0 0 2.85cm},clip,width=0.197\textwidth]{images/additional_images/MFUSE/errors_flow_img/000018_10.png}};
\draw (img.north west) node[labelstyle] {Fl: 42.31};
\end{tikzpicture} &
\begin{tikzpicture}
\draw (0, 0) node[imgstyle] (img) {
\includegraphics[trim={0 0 0 2.85cm},clip,width=0.197\textwidth]{images/additional_images/MFUSE/errors_scene_flow_img/000018_10.png}};
\draw (img.north west) node[labelstyle] {SF: 46.16};
\end{tikzpicture}
\\
\begin{tikzpicture}
\draw (0, 0) node[imgstyle] (img) {
\includegraphics[trim={0 0 0 2.85cm},clip,width=0.197\textwidth]{images/additional_images/RigidMaskISF/result_disp_img_0/000019_10.png}};
\draw (img.north west) node[labelstyle] {RigidMask+ISF};
\end{tikzpicture} &
\begin{tikzpicture}
\draw (0, 0) node[imgstyle] (img) {
\includegraphics[trim={0 0 0 2.85cm},clip,width=0.197\textwidth]{images/additional_images/RigidMaskISF/errors_disp_img_0/000019_10.png}};
\draw (img.north west) node[labelstyle] {D2: 1.36};
\end{tikzpicture} &
\includegraphics[trim={0 0 0 2.85cm},clip,width=0.197\textwidth]{images/additional_images/RigidMaskISF/result_flow_img/000019_10.png} &
\begin{tikzpicture}
\draw (0, 0) node[imgstyle] (img) {
\includegraphics[trim={0 0 0 2.85cm},clip,width=0.197\textwidth]{images/additional_images/RigidMaskISF/errors_flow_img/000019_10.png}};
\draw (img.north west) node[labelstyle] {Fl: 1.94};
\end{tikzpicture} &
\begin{tikzpicture}
\draw (0, 0) node[imgstyle] (img) {
\includegraphics[trim={0 0 0 2.85cm},clip,width=0.197\textwidth]{images/additional_images/RigidMaskISF/errors_scene_flow_img/000019_10.png}};
\draw (img.north west) node[labelstyle] {SF: 2.34};
\end{tikzpicture}
\-0.6mm]
\begin{tikzpicture}
\draw (0, 0) node[imgstyle] (img) {
\includegraphics[trim={0 0 0 2.85cm},clip,width=0.197\textwidth]{images/additional_images/RAFT3D/result_disp_img_0/000019_10.png}};
\draw (img.north west) node[labelstyle] {RAFT-3D};
\end{tikzpicture} &
\begin{tikzpicture}
\draw (0, 0) node[imgstyle] (img) {
\includegraphics[trim={0 0 0 2.85cm},clip,width=0.197\textwidth]{images/additional_images/RAFT3D/errors_disp_img_0/000019_10.png}};
\draw (img.north west) node[labelstyle] {D2: 1.31};
\end{tikzpicture} &
\includegraphics[trim={0 0 0 2.85cm},clip,width=0.197\textwidth]{images/additional_images/RAFT3D/result_flow_img/000019_10.png} &
\begin{tikzpicture}
\draw (0, 0) node[imgstyle] (img) {
\includegraphics[trim={0 0 0 2.85cm},clip,width=0.197\textwidth]{images/additional_images/RAFT3D/errors_flow_img/000019_10.png}};
\draw (img.north west) node[labelstyle] {Fl: 1.97};
\end{tikzpicture} &
\begin{tikzpicture}
\draw (0, 0) node[imgstyle] (img) {
\includegraphics[trim={0 0 0 2.85cm},clip,width=0.197\textwidth]{images/additional_images/RAFT3D/errors_scene_flow_img/000019_10.png}};
\draw (img.north west) node[labelstyle] {SF: 2.21};
\end{tikzpicture}
\-0.6mm]
\begin{tikzpicture}
\draw (0, 0) node[imgstyle] (img) {
\includegraphics[trim={0 0 0 2.85cm},clip,width=0.197\textwidth]{images/additional_images/MFUSE/result_disp_img_0/000019_10.png}};
\draw (img.north west) node[labelstyle] {M-FUSE};
\end{tikzpicture} &
\begin{tikzpicture}
\draw (0, 0) node[imgstyle] (img) {
\includegraphics[trim={0 0 0 2.85cm},clip,width=0.197\textwidth]{images/additional_images/MFUSE/errors_disp_img_0/000019_10.png}};
\draw (img.north west) node[labelstyle] {D2: 1.11};
\end{tikzpicture} &
\includegraphics[trim={0 0 0 2.85cm},clip,width=0.197\textwidth]{images/additional_images/MFUSE/result_flow_img/000019_10.png} &
\begin{tikzpicture}
\draw (0, 0) node[imgstyle] (img) {
\includegraphics[trim={0 0 0 2.85cm},clip,width=0.197\textwidth]{images/additional_images/MFUSE/errors_flow_img/000019_10.png}};
\draw (img.north west) node[labelstyle] {Fl: 1.91};
\end{tikzpicture} &
\begin{tikzpicture}
\draw (0, 0) node[imgstyle] (img) {
\includegraphics[trim={0 0 0 2.85cm},clip,width=0.197\textwidth]{images/additional_images/MFUSE/errors_scene_flow_img/000019_10.png}};
\draw (img.north west) node[labelstyle] {SF: 2.04};
\end{tikzpicture}
\end{tabular}
}
\caption{Qualitative comparison of our method, the original RAFT-3D, as well as the two top-performing approaches from the literature using the visualizations provided by the KITTI benchmark. \emph{From left to right:} Target disparity visualization, corresponding \emph{D2} error plot, optical flow visualization, corresponding \emph{Fl} error plot, combined \emph{SF} error plot.}
\label{fig:end}
\end{figure*}

\end{document}
