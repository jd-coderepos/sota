\chapter{The linear combination method}\label{ch:LCmethod}
In this chapter we introduce the linear combination method, or the LC method for short. We present in \SCref{LCprelim} some preliminary lemmas. Then we describe in \SCref{convLC} how to perform a conversion step. In \SCref{equivalent} we give definitions and results about equivalence of DAEs and address how equivalence is related to the LC method.

For simplicity, throughout this report, we consider only the second type of SA's failures described in \SCref{idfailure}: ``singular'' means identically singular but not structurally singular.
Based on this assumption, symbolic cancellations are not considered an issue that makes the \Sigmeth fail. 


\section{Preliminary lemmas}\label{sc:LCprelim}


\begin{lemma}\label{le:griewank}
{\bf\em (Griewank's Lemma)}\cite[Lemma 5.1]{nedialkov2007solving}
Let  be a function of , 's and derivatives of them . Denote , where . If ,
then

Hence

\end{lemma}



\begin{lemma}\label{le:sigred0}
Let  and  be  signature matrices. 
Assume  is finite,   are valid offsets for , 
and  for all . 
If a HVT in  contains a position  where , then .
\end{lemma}
\begin{proof}
Let  be a HVT in . Then

\renewcommand{\qed}{}
\end{proof}

\begin{corollary}\label{co:sigred}
For a row index , let

Then .
\end{corollary}

\begin{proof}
Since  for all , the intersection of a HVT in  with positions in row  is a position  with . By \LEref{sigred0},  
.
\end{proof}

This lemma shows that, if we replace a row  in  with a row with entries less than  for each column , then the value of this signature matrix decreases.

\section{Conversion step}\label{sc:convLC}









Given a SWP DAE of the form \rf{maineq}, assume that we apply the -method and obtain a singular system Jacobian . We seek a reformulation of this DAE so that the system Jacobian  of the new DAE may be generically nonsingular. We denote by  and  the signature matrices of the original DAE and this new DAE, respectively. Denote by  the valid offsets for .

We describe below how to perform a conversion step using a linear combination (LC) of equations.  We call this conversion technique the {\em LC conversion method}, or simply the {\em LC method}.
The main result from this conversion is that, under certain conditions, we can obtain an equation in a row, say , such that  occurs in this row of order  for all . Hence by \COref{sigred}, }.

We assume . Let  be a nonzero vector function from the null space of .
Here,  and  are considered as functions of , 's and appropriate derivatives of them.

Denote by  the set of indices for which the th component of  is not identically zero

and let

Since  is nonzero and  is identically singular,  has at least two elements. Otherwise  has a row of identical zeros and is structurally singular.

\begin{remark}
We consider  in its simplest form in the sense that its elements do not have a common factor comprising , 's, or/and derivatives of them. For instance, in \EXref{lenaintro}, we do not use  though , but use .

Also, we do not consider  with any fractions. For example, we use  instead of 
. 
\end{remark}

The {\em sufficient condition} for applying the LC method is the following: for a nonzero ,

 If this condition is satisfied, then we can perform a conversion step. We explain this ``sufficiency" in \RMref{suffLC}.

Denote by  the set of indices  such that the th component of  is not an identical zero and :

From \rf{thdef}, there exists at least one  such that , so .

We choose an  and replace  by 

We refer to \rf{replacing} as a {\em conversion step} using the LC method
and to the resulting DAE as a {\em converted} DAE. Critical for the success of the LC method is the following lemma.
\begin{theorem}\label{le:LC}
For a SWP DAE with identically singular , let  be a nonzero -vector such that .
If 

and we replace  by  in \rf{replacing}, 
then the converted DAE has  with .
\end{theorem}

First, we illustrate with an example how to perform a conversion step, and then we prove this lemma. Since  is fixed during a conversion step, for brevity we write , , and  as , , and , respectively\footnote{This set  is not to be confused with the constant  in the pendulum-related DAEs.}.

\begin{example}\label{ex:FGxy}
Consider

Here , and similarly we write , , and . 



Because of singular , the SA fails. It reports structural index 2, but the differentiation index is 3. If we take , then .

We illustrate \rrf{idef}{LCsetK}:

Then we check if condition \rf{LCcond} holds:

Hence  for all .

Using \rf{replacing} gives

Now, with ,

That is,  for all . 


For each , assuming , we can replace  by . We show in the following the three possible converted DAEs, each with  and generically nonsingular .

\begin{itemize}
\item :


When  and , the determinant is nonzero and the SA succeeds.

\item :


Similarly, the SA succeeds when  and .

\item :


In this case, SA's success requires only .
\end{itemize}

\end{example}

Using the LC method, we obtain three converted DAEs from \rf{FGxy}: \rf{FGxyk=1}, \rf{FGxyk=2}, and \rf{FGxyk=4}. 
However, it is not guaranteed that all converted DAEs and the original DAE have exactly the same solution sets.
We will cover this equivalence issue in \SCref{equivalent}.

Now we prove \LEref{LC}.
\begin{proof}
We show first that 


Since ,  for . By \rf{cidj}, . Applying Griewank's Lemma \rf{griewank2} to \rf{sysjac}, with , gives

Then for all ,

This shows that  does not truly depend on , that is,

By \COref{sigred}, }.
\end{proof}

\begin{remark}\label{rm:nameLC}
We name this method ``LC'' because of the following considerations. The vector  from the null space of  that satisfies \rf{LCcond} does not comprise the leading derivatives  for all  in equation  for all . We consider here each  as a ``constant'' in 

and  as a ``linear combination'' of equations .
\end{remark}

\begin{remark}\label{rm:suffLC}
If  for some , then  is not guaranteed . In this case, we cannot swap the sum and the differentiation operator in \rf{notdepend}. Therefore, we cannot prove . Then, in , it can happen that

giving  instead of strictly  in \rf{lcstrictless}.

However, if  holds for  from a particular set, we can still achieve . We leave this investigation for future work, consider the condition \rf{LCcond} sufficient for now, and require it to be satisfied for the LC method.
\end{remark}


If  is a constant vector, then  for every , and the condition \rf{LCcond} is automatically satisfied. In this case we do not need to check it. We illustrate this in the next example.
\begin{example}\label{ex:xyzt}
Consider


For , . Using \rrf{idef}{LCsetK} gives

Since  is a constant vector, condition \rf{LCcond} is satisfied. We replace  by


The converted DAE is


If  for a suitable  depending on the machine precision, then  is computably nonsingular.
\end{example}

\section{Equivalent DAEs}\label{sc:equivalent}
First, we give a definition for equivalent DAEs.
\begin{definition}\label{df:equivalent}
Let  and  denote two DAEs. They are {\em equivalent} (on some interval for )
if a solution of  is a solution to 
and vice versa.
\end{definition}

In the following context, we denote by  the original DAE with equations , , and singular Jacobian . After a conversion step using the LC method, we obtain a (converted) DAE, denoted by , with equations , , and Jacobian , which may be singular.

\begin{theorem}\label{th:equivLC}
After a conversion step using the LC method, DAEs  and  are equivalent on some real time interval  for , if  for all .
\end{theorem}

\begin{proof}
Let a solution of , over some interval , be a vector-valued function 

that satisfies \rf{maineq} for all .

We denote the vector used in the LC method by , where its th component is of the form


If  is defined at  for all , then

 vanish at , and thus   is a solution to .

Conversely, assume that  is a solution of  on . If  is defined at  for all  and
, then 

vanish at , and thus  is a solution to .

By \DFref{equivalent},  and  are equivalent.
\end{proof}

\begin{remark}\label{rm:goodchoiceLC}
We can see from \rf{LCsolution2} that, if we have a choice for , it is desirable to choose it such that  is identically nonzero, e.g., a nonzero constant,  , or . In this case,  and  are {\em always} equivalent---we do not need to check the {\em equivalence condition}  when we solve .
\end{remark}

\begin{example}
In \EXref{FGxy}, case  [resp. ]  requires  [resp. ] to recover the original DAE \rf{FGxy} from \rf{FGxyk=1} [resp. from \rf{FGxyk=2}].  However, for case ,   is a nonzero constant for any . Therefore this choice is more desirable than the other two.
\end{example}



Below we define an ill-posed DAE using the structural posedness defined in the \daesa papers \cite{NedialkovPryce2012a,NedialkovPryce2012b}.
\begin{definition}\label{df:illposed}
A DAE is ill posed if it has an equivalent DAE that is structurally ill-posed (SIP); otherwise it is well posed.
\end{definition}

\begin{example}
Consider problem \rf{pendmess1}. Using , we reduce  to the trivial . This is just performing a simple substitution, and is not applying the LC method.
The signature matrix

 does not have a finite HVT, so the resulting DAE is SIP. Hence, by \DFref{illposed} , the original SWP DAE \rf{pendmess1}  is ill posed.
\end{example}

\begin{corollary}\label{co:illposed}
If a structurally well-posed DAE can be converted, by the LC method, to an equivalent DAE that is structurally ill-posed, then the original DAE is ill posed.
\end{corollary}

\begin{proof}
This follows from \THref{equivLC} and \DFref{illposed}.
\end{proof}

\begin{example}\label{ex:pendmess2}
Consider the following SWP DAE



For , . Using \rrf{idef}{LCsetK} gives

Since  is a constant vector, condition \rf{LCcond} is satisfied. We replace  by

The signature matrix of the resulting problem is exactly \rf{SigIllposed}. Hence, by \COref{illposed}, \rf{pendmess2} is ill posed.
\end{example}



If the Jacobian of the converted DAE is still singular, we may be able to apply the LC method iteratively, provided condition \rf{LCcond} is satisfied on each iteration. Since after each conversion step we reduce the value of the signature matrix by at least 1, the number of iterations does not exceed , where  is for the original DAE. We use \EXref{messpend} to show how we can iterate with the LC method.
\begin{example}\label{ex:messpend}
We construct the following (artificial) \modpenda DAE from \pend \rf{pend}:


Here, a superscript denotes an iteration number, not a power. We show how to recover the simple pendulum problem.

We find a vector in : . Then 

We replace the second equation  by

The converted DAE is



Although }, matrix  is still singular. If , then . This gives 

We replace the third equation  by

The converted DAE is



We have , but  is still singular. We find  such that . Then 

Replacing the first equation  by

we recover  from \rf{messpendLC}. This is exactly the DAE \pend\rf{pend}, with  and ; cf. \EXref{simplepend}.

Since each  in every conversion iteration is a constant vector, each  we pick is a nonzero constant. By \RMref{goodchoiceLC}, the original DAE \rf{messpendLC} and \pend are {\em always} equivalent. Hence, we can solve \rf{messpendLC} by simply solving \pend.
\end{example}

%
