\pdfoutput=1  \documentclass[]{article}  
\usepackage{url,float}
\usepackage{graphicx}
\usepackage{amsfonts}
\usepackage{amssymb}
\usepackage{latexsym}




\newcommand{\hide}[1]{}

\newcommand{\ABox}{
\raisebox{3pt}{\framebox[6pt]{\rule{6pt}{0pt}}}
}
\newenvironment{proof}{{\bf Proof:}}{\hfill\ABox}


\newtheorem{theorem}{{\bf Theorem}}
\newtheorem{corollary}[theorem]{Corollary}
\newtheorem{lemma}[theorem]{Lemma}
\newtheorem{conjecture}[theorem]{Conjecture}
\newtheorem{definition}[theorem]{Definition}

\newcommand{\lemlab}[1]{\label{lemma:#1}}
\newcommand{\thmlab}[1]{\label{thm:#1}}
\newcommand{\eqlab}[1]{\label{eq:#1}}
\newcommand{\corlab}[1]{\label{cor:#1}}
\newcommand{\deflab}[1]{\label{def:#1}}
\newcommand{\tablab}[1]{\label{tab:#1}}
\newcommand{\figlab}[1]{\label{fig:#1}}
\newcommand{\seclab}[1]{\label{sec:#1}}
\newcommand{\chaplab}[1]{\label{chap:#1}}

\newcommand{\lemref}[1]{\ref{lemma:#1}}
\newcommand{\thmref}[1]{\ref{thm:#1}}
\newcommand{\corref}[1]{\ref{cor:#1}}
\newcommand{\defref}[1]{\ref{def:#1}}
\newcommand{\chapref}[1]{\ref{chap:#1}}
\newcommand{\secref}[1]{\ref{sec:#1}}
\newcommand{\eqref}[1]{\ref{eq:#1}}
\newcommand{\figref}[1]{\ref{fig:#1}}
\newcommand{\tabref}[1]{\ref{tab:#1}}




{\makeatletter
 \gdef\xxxmark{\expandafter\ifx\csname @mpargs\endcsname\relax \expandafter\ifx\csname @captype\endcsname\relax \marginpar{xxx}\else
       xxx \fi
   \else
     xxx \fi}
 \gdef\xxx{\@ifnextchar[\xxx@lab\xxx@nolab}
 \long\gdef\xxx@lab[#1]#2{{\bf [\xxxmark #2 ---{\sc #1}]}}
 \long\gdef\xxx@nolab#1{{\bf [\xxxmark #1]}}
\gdef\turnoffxxx{\long\gdef\xxx@lab[##1]##2{}\long\gdef\xxx@nolab##1{}}}


\def\P{{\mathcal P}}
\def\Q{{\mathcal Q}}
\def\D{{\mathcal D}}
\def\G{{\Gamma}}
\def\a{{\alpha}}
\def\b{{\beta}}
\def\g{{\gamma}}
\def\l{{\lambda}}
\def\r{{\rho}}
\def\s{{\sigma}}
\def\d{{\delta}}
\def\t{{\theta}}



\newcommand{\squeezelist}{\setlength{\itemsep}{0pt}}




\title{Some Properties of Yao  Subgraphs}


\author{Joseph O'Rourke\thanks{Department of Computer Science, Smith College, Northampton, MA
      01063, USA.
      \protect\url{orourke@cs.smith.edu}.}
}

\begin{document}
\maketitle

\begin{abstract}
The Yao graph for , , 
is naturally partitioned into four subgraphs, one per quadrant.
We show that the subgraphs for one quadrant differ from the subgraphs for
two adjacent quadrants in three properties:
planarity, connectedness, and whether the directed graphs are spanners.
\end{abstract}


\section{Introduction}
The Yao graph is defined for an integer parameter ;
here we study only , and call
 the directed Yao graph, and  the undirected version.
For a set of points ,  connects each point
to its closest neighbor in each of the four quadrants surrounding it,
defined as in
Figure~\figref{QuadrantsDefinition}.
Ties are broken arbitrarily.
\begin{figure}[htbp]
\centering
\includegraphics[width=0.5\linewidth]{Figures/QuadrantsDef}
\caption{Definition of quadrants. Solid lines are closed, dotted lines are open.}
\figlab{QuadrantsDefinition}
\end{figure}
The undirected graph  simply ignores the direction.

The question of whether  is a spanner was raised in
\cite{sppyg-dmp-09}.
A -spanner has the property that the path between  and  in the
graph is no longer than , for a constant .
In this note, we do not further motivate the study of ,
but rather investigate some properties of subgraphs of , which
may ultimately have some bearing on whether it is a spanner.

We make two ``general position'' assumptions:
\begin{enumerate}
\item No two pair of points determine the same distance
(so there are no ties).
\item
No two points share a vertical or horizontal coordinate.
\end{enumerate}
These assumptions simplify the presentation.
In this note, we will not explore whether the assumptions can be removed while retaining
all the results.

\paragraph{Notation.}
 is the circular quadrant whose origin is at  and which reaches out to .
Often the subscript  will be dropped, as it is determined by  and .
 is the unbounded quadrant with corner at .
Thus,
.
 is the closed rectangle with opposite corners  and .


We focus on two adjacent quadrants,  and .
Let  be the  graph restricted to the 
quadrants in the list .
See Figure~\figref{Y40} for examples.
\begin{figure}[htbp]
\centering
\includegraphics[height=0.9\textheight]{Figures/Y40}
\caption{, , and , for the same -point set.}
\figlab{Y40}
\end{figure}


Our results are summarized in Table~\tabref{Results}.
\begin{table}[htbp]
\begin{center}
\begin{tabular}{| l || c | c |}
\hline
\emph{Property} &  &  
\\ \hline \hline
Planarity & planar & not planar
\\ \hline
Connectedness & not connected & connected 
\\ \hline
Undirected spanner & not a spanner & not a spanner 
\\ \hline
Directed spanner & spanner & not a spanner
\\ \hline
\end{tabular}
\end{center}
\tablab{Results}
\caption{Summary of Results}
\end{table}

\section{Planarity}
It is known that  is a planar forest, in general disconnected;
see Figure~\figref{Y40}(a,b).
This is folklore,\footnote{
Mirela Damian [private communication, Feb. 2009].
}
but we offer a proof of planarity.


\begin{lemma}
No two edges of  properly cross.
\end{lemma}
\begin{proof}
Let both  and  be in , and suppose  and  properly cross.
see Figure~\figref{Q0noncrossing}.
\begin{figure}[htbp]
\centering
\includegraphics[width=\linewidth]{Figures/Q0noncrossing}
\caption{ and  may not cross.}
\figlab{Q0noncrossing}
\end{figure}
The quadrants  and  must be empty of points.
We consider three cases, depending on the location of  w.r.t. .
\begin{enumerate}
\item .
Then  crosses  from below.  We analyze just this case in detail.
Because , the circular boundary of  must cut ,
say at .  Consider two further cases
\begin{enumerate}
\item The slope of the arc of  at  is shallower than the slope of the
arc of  at ; see Figure~\figref{Q0noncrossing}(a).
Then .
\item The slope at  is equal to or steeper than that at .
Then, because  is strictly below , the radius  is greater than
.
But then  cannot be in .
\end{enumerate}
\item .
Then  could cross  from below, Figure~\figref{Q0noncrossing}(b),
or from above, Figure~\figref{Q0noncrossing}(c).
In both cases, a quadrant that must be empty is not.
\item .
This case is the same as the first case, with the roles of  and  interchanged.
\end{enumerate}
\end{proof}

In contrast,  may be nonplanar.
Figure~\figref{NonPlanar01}(a) shows two crossing edges;
(b) shows the full graph .
\begin{figure}[htbp]
\centering
\includegraphics[width=0.75\linewidth]{Figures/NonPlanar01}
\caption{ can be nonplanar.}
\figlab{NonPlanar01}
\end{figure}


As should be evident from Figure~\figref{Y40}(c),
crossing edges are rare, requiring precise placement of four points.
Although it would be difficult to quantify, a ``typical''  graph
is planar.

\section{Connectedness}
We can see in Figure~\figref{Y40}(a,b)
that
 is, in general, disconnected.
In contrast,  is connected.
See again Figure~\figref{Y40}(c).

\begin{lemma}
 is a connected graph.
\end{lemma}
\begin{proof}
We choose  w.l.o.g.
So we are concerned with upward -connections, in  and .
The proof is by induction on the number of points  in the set .
The basis of the induction is trivial, for an  point set is connected.
Let  have  points, and let  be the point with the lowest
-coordinate.  
By Assumption~(2),  is unique.  

Delete this from , reducing to a point set  with .  
Then the set of points 
satisfies the induction hypothesis, and so is connected into a graph . 
See Figure~\figref{Connected01a}.
Put back point .
Because all the quadrants determining edges  are  or , they lie at or above , the -coordinate of the lowest point
in ,
.
Thus  cannot lie in any quadrant, and so adding  to  does not break any edge
of .\footnote{
    Note that if the induction instead removed the topmost point from , this claim
    would no longer hold.}
Finally,  itself must have at least one outgoing edge upward, for  and  cover
the half-plane above , which contains at least one point of .
\end{proof}



\begin{figure}[htbp]
\centering
\includegraphics[width=0.5\linewidth]{Figures/Connected01a}
\caption{ must be connected.}
\figlab{Connected{01}a}
\end{figure}


\section{Undirected Spanners}
It is clear that  is not a spanner, because it may be
disconnected.  Points on a negatively sloped line result in a completely
disconnected graph of isolated points.
Neither is  a spanner.  Points uniformly spaced on two lines forming a `' shape
both have directed paths up to the apex in , 
but the leftmost and rightmost lowest
points can be arbitrarily far apart in the graph.

\section{Directed Spanners}
We turn then to directed versions of these questions.
Call a directed graph a \emph{directed spanner} if every directed path
is no more than  times the path's end-to-end Euclidean distance,
for  a constant.


\begin{lemma}
 is a directed spanner:
no directed path is more than  times the end-to-end Euclidean distance.
\end{lemma}
\begin{proof}
Let  and  be the endpoints of the path.
Then the path is an -monotone path remaining inside
.  Therefore its length is at most half the perimeter of this
rectangle, which is at most  times the diagonal length.
\end{proof}

\begin{lemma}
 is not a directed spanner:
directed paths can be arbitrarily long:
more than any constant  times the end-to-end Euclidean distance.
\end{lemma}
\begin{proof}
Consider the path  in Figure~\figref{Q01longpath}(a).
\begin{figure}[htbp]
\centering
\includegraphics[width=0.99\linewidth]{Figures/Q01longpath}
\caption{An arbitrarily long path in  .}
\figlab{Q01longpath}
\end{figure}
It is clear that this path can be made arbitrarily long with respect to ,
by lowering the vertical coordinates of  and .
Now we show how to avoid any other directed connection between  and .

Let the other outgoing edge from  go to  as shown.
We now direct paths from  and from  that do not connect.
The idea is depicted in Figure~\figref{Q01longpath}(b).
We create a series of nearly vertical paths from , and from .
Above , two points are placed at ,
.
The two outgoing edges from  will terminate on these.
Then above those we place two more points at .
Now we get both upward and diagonal connections among the four points,
with one ``diagonal'' being horizontal.\footnote{
	The definition in Figure~\protect\figref{QuadrantsDefinition} shows that
	  will connect horizontally to .
	 }
The point is that all the outgoing edges are
accounted for.

Repeating this construction, we can make a nearly vertical tower of points,
connected by vertical paths, but otherwise insulated from one another.
So the only path from  to  is .
\end{proof}

\section{Future Work}
The obvious next step is to examine properties of three
quadrants, , before finally tackling  itself.




\bibliographystyle{alpha}
\bibliography{/home/orourke/bib/geom/geom}
\end{document}
