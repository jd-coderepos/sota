\documentclass{sig-alternate-2013}

\usepackage{cite}



\usepackage{subcaption}
\usepackage{graphicx}
\usepackage{amsmath,amssymb,amscd}
\usepackage{soul}
\usepackage{color}
\usepackage{balance}
\usepackage{paralist}
\usepackage{url}



\setlength{\paperheight}{11in}
\setlength{\paperwidth}{8.5in}
\usepackage[
  pass,]{geometry}



\newcommand{\pk}{pk^{ABE}}
\newcommand{\mk}{mk^{ABE}}

\newcommand{\G}{\mathbb{G}_1}
\newcommand{\GG}{\mathbb{G}_2}
\newcommand{\Z}{\mathbb{Z}}
\newcommand{\tree}{\mathcal{T}} 
\newcommand{\andraben}{\textsc{AndrABEn}}





\newfont{\mycrnotice}{ptmr8t at 7pt}
\newfont{\myconfname}{ptmri8t at 7pt}
\let\crnotice\mycrnotice \let\confname\myconfname 






\permission{Permission to make digital or hard copies of all or part of this work for personal or classroom use is granted without fee provided that copies are not made or distributed for profit or commercial advantage and that copies bear this notice and the full citation on the first page. Copyrights for components of this work owned by others than ACM must be honored. Abstracting with credit is permitted. To copy otherwise, or republish, to post on servers or to redistribute to lists, requires prior specific permission and/or a fee. Request permissions from Permissions@acm.org.}
\conferenceinfo{IoT-Sys 2015,}{May 18, 2015, Florence, Italy.}
\copyrightetc{Copyright \copyright~2015 ACM \the\acmcopyr}
\crdata{978-1-4503-3502-7/15/05\ ...\(A \wedge B)\vee CABCtruefalseDD\pk\mkM\gamma\pkEA\mk\pkDE\gammaDA\pkM\gammaA\pk\mkMA\pkE\gamma\mk\pkDEAD\gamma\pkM\gammaAAAy^2 = x^3 + xF_qq=3~mod~4k = 2qrP\in E(F_q)rq<\!2<\!250\%13.514.5\approx45\approx90\approx110\approx\approx\approx\approx\approx\approx$172~s, respectively. 

Overall, we can conclude that, compared with the approach discussed in~\cite{ABE_icc_2014}, \andraben~provides significantly better performance, in terms of execution time, memory and CPU usage, and energy consumption.

\section{Conclusion}\label{sec:conclusion} 
With the increasing use of cloud environment and smart devices connected to the Internet of Things, exchanged data confidentiality and access control to the stored data become a challenging issue. Attribute-Based Encryption is one of the best solutions that can be used to satisfy users privacy concerns~\cite{ABE_icc_2014}. However, its performance on resource constraint devices is a challenging issue, and still represents a big concern for researchers willing to use ABE to develop novel privacy-preserving and access control solutions for such devices.

In this paper, we studied the feasibility of applying ABE on smartphone devices and presented \andraben, an implementation of ABE in C language.
We also provided a comparative analysis with a similar research study~\cite{ABE_icc_2014} in which the authors proposed a Java-based implementation of ABE for Android smartphone.
Based on the results of our thorough experiments, we conclude that using ABE on Android smartphones and similar devices is feasible. 
The evidence that we bring in this paper will be a reference for applicability of ABE in resource-constrained devices.

\balance
\bibliographystyle{abbrv}
\bibliography{ref}


\end{document}
