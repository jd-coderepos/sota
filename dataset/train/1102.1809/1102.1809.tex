\documentclass{article}
\usepackage{amsmath,amssymb,enumerate,theorem}

\newcommand{\email}[1]{{\textit{Email:} \texttt{#1}}}
\newcommand{\homepage}[1]{{\textit{Web:} \texttt{#1}}}
\newcommand{\tmem}[1]{{\em #1\/}}
\newcommand{\tmmathbf}[1]{\ensuremath{\boldsymbol{#1}}}
\newcommand{\tmop}[1]{\ensuremath{\operatorname{#1}}}
\newcommand{\tmtexttt}[1]{{\ttfamily{#1}}}
\newenvironment{enumeratenumeric}{\begin{enumerate}[1.] }{\end{enumerate}}
\newenvironment{itemizedot}{\begin{itemize} \renewcommand{\labelitemi}{}\renewcommand{\labelitemii}{}\renewcommand{\labelitemiii}{}\renewcommand{\labelitemiv}{}}{\end{itemize}}
\newenvironment{proof}{\noindent\textbf{Proof\ }}{\hspace*{\fill}\medskip}
\newtheorem{corollary}{Corollary}
\newtheorem{definition}{Definition}
\newtheorem{lemma}{Lemma}
{\theorembodyfont{\rmfamily\small}\newtheorem{problem}{Problem}}
\newtheorem{proposition}{Proposition}
\newtheorem{theorem}{Theorem}


\begin{document}

\title{Generalized companion matrix for approximate GCD}\author{Paola
Boito \thanks{\email{paola.boito@unilim.fr};
\homepage{http://www.unilim.fr/pages\_perso/paola.boito/index\_en.html}}\\
XLIM-DMI UMR 6172 Universit\'e de Limoges - CNRS \\  \and Olivier
Ruatta\thanks{\email{olivier.ruatta@unilim.fr}}\\
XLIM-DMI UMR 6172 Universit\'e de Limoges - CNRS}\maketitle



{\small{We study a variant of the univariate approximate GCD problem, where
the coefficients of one polynomial are known exactly, whereas the
coefficients of the second polynomial may be perturbed. Our approach
relies on the properties of the matrix which describes the operator of
multiplication by in the quotient ring . In
particular, the structure of the null space of the multiplication matrix
contains all the essential information about GCD. Moreover, the
multiplication matrix exhibits a displacement structure that allows us to
design a fast algorithm for approximate GCD computation with quadratic
complexity w.r.t. polynomial degrees.}}

\section{Introduction\label{intro}}


The approximate polynomial greatest common divisor (denoted as AGCD) is a
central object of symbolic-numeric computation. The main difficulty of the
problem comes from the fact that is no universal notion of AGCD. One can find
different approaches and different notions for AGCD. We will not give a review
of all the existing work on this subject, but we will recall one of the most
popular approaches to show how our work brings a different point of view on
the problem.

The main approach to the computation of an AGCD consists in considering two
univariate polynomials whose coefficients are known with uncertainty. This
uncertainty can be the result of the fact that the polynomials have floating
point coefficients coming from previous computation (and so are subject to
round-off errors). The most frequently adopted formulation is related to
semi-algebraic optimization : given  and  two
approximate polynomials, find two polynomials  and  such that  and  are small (lower than a given
tolerance for instance) and such that the degree of  is maximal.
That is, one looks for the most singular system close to the input . An -gcd is obtained if the conditions  and  are
satisfied. One can try to compute the tolerance on the perturbation of the
input polynomial thanks to direct computation (for instance from a jump on
singular values of particular matrix for instance). This last approach has
received a great interest following the work of Zeng using Sylvester like
matrices ({\cite{Zeng}}). \

Here, we consider a slightly different problem. One of the polynomials, say
, is known exactly (it is the result of an exact model) and the second one,
say , is an approximate polynomial (result of measures or previous
approximation for instance). This case occurs in applications such as model
checking (to compare results of an exact model and measures). There are many
other instances of such a problem, such as simplification of fractions when
one of the polynomial is known exactly but the other one is not.

We give a example of such a situation. When modeling an electromagnetic
filter, one might want to parametrize its behavior with respect to the
frequency. But one may need to do so even if there are singularities and to do
so one may use Pad\'e approximations of the electromagnetic signal at each
point as a function of the frequency. In some cases of interest, one can know
all the singularities and so compute an exact polynomial called
characteristic. Pad\'e approximations are computed independently for each
point by a numerical process and denominators may have a non trivial gcd with
the ``characteristic'' polynomial. The denominators are not known exactly. So,
in order to identify unwanted common factors in denominators one has to
compute approximate gcds between an exact and non exact polynomials.

This AGCD problem can also be interpreted as an optimization problem. Given
 exactly and  approximately, compute a polynomial  close to
 such that  has a maximal degree gcd with . Our approach
takes advantage of the asymmetry of the problem and of the structure of the
quotient algebra  (more accurately, of the
displacement rank of the multiplication operator in this algebra). So, we
address the following problem :



\begin{problem}
  Let  a given polynomial and  another
  polynomial. Find  close to  (in a sense that will be
  explained) such that  and  have a gcd of maximal
  degree. \ 
\end{problem}

This may be also an interesting approach when one has two polynomials, one
known with high confidence and another with worse accuracy. This approach may
take advantage of this asymmetry which would not be possible for classical
framework based on Sylvester or B\'ezout matrices.



In this paper, we propose an approach and an algorithm to address this
problem. The proposed algorithm is ``fast'' since the exponent of its
complexity is better than the classical linear algebra exponent in the degree
of the input polynomials.

Organisation of the paper: The second section is devoted to some basic result
on algebra needed after, the third section gives an algebraic method for gcd
based on linear algebra, the fourth section recalls the Barnett formula
allowing to compute the multiplication matrix without division, the fifth
gives the displacement rank structure of the multiplication matrix, the sixth
describes the final algorithm and experiments before finishing with
conclusions and perpectives.

\section{Euclidian structure and quotient algebra\label{quot}}



In this section, we recall basic algebraic results allowing to understand the
principle of our approach. All material in this section can be found (even in
the non reduced case and in the multivariate setting) in {\cite{R}}. \

Assume that  is an algebraically closed field (here we think
about ). Let  and  and assume
that  and
that  for all  in . Let
 and  be the natural projection. For , we
define , the  Lagrange polynomial
associated to . Clearly, since  we have , for all . Let
 be
the usual dual space of . For all , we
define  by
 for all . The
following lemma is obvious form the definition of the polynomials  that
for  and , we have . This implies that the set  is a
basis of . A well known fact is that the set
 form a basis
 dual of the basis  of
. As a corollary, we have the Lagrange interpolation formula :
Each  can be written . A funny
consequence is that if we choose  as a basis of
, for all , the remainder  of the
euclidian division of  by  is given by  in the basis , i.e. . In other word, divide  by 
is equivalent to evaluate  at the roots of .

The general philosophy of this last proposition will allows us to make a lot
of proof in a very simple way. For example, it is very easy to see the
different operation in  using this representation. Let  and
 be to elements in , then we have  and 
in . This allows us to avoid the use of the section . In
fact, the Lagrange polynomials  reveal a deeper structure on
the algebra  : The polynomials  are the
idempotents of , i.e. .

Thanks to this description of the quotient algebra, it is easy to derive
algorithms for both polynomial solving and gcd computation even though the
problems are of very different nature.

Remark that we have expressed everything in the monomial basis since it is the
most widely used basis to express polynomials but we could use other bases. A
particular basis is the Chebyshev basis where all results are exactly the same
since it is a graduated basis.

\section{An algebraic algorithm for gcd computation\label{exact}}



To first give an idea on how to exploit the section above in order to design
algorithm for gcd, We recall a classical method for polynomial solving (see
{\cite{C}} for instance). Proofs are given for the sake of completeness and
because very similar ideas will lead us to the AGCD computation.

\subsection{Roots via eigenvalues}



Let 
be a polynomial of degree . Then we consider the matrix of the
multiplication by  in . Its matrix in the monomial
basis  is the following:

well known as the Froebenius companion matrix associated to .

\begin{proposition}
  Let be polynomial of degree  with  distinct
  roots , then the eigenvalues of
   are the roots of , i.e. . \ 
\end{proposition}

\begin{proof}
  It follows directly from the fact that  is the matrix of
  the multiplication by  in \ . But here we propose
  to give a direct proof by induction. In fact, we prove by induction that the
  characteristic polynomial of  is  itself (up to a
  sign and a scalar factor ), i.e.:
  
  since we have:
  
  and by assumption  and
  finally we have the wanted result. Remark that this proof allows to avoid
  the condition that all the roots of  are distinct.
\end{proof}

Then, to compute the roots of  one can compute the eigenvalues of is
Froebenius companion matrix. This is the object of the method proposed
(reintroduced) by Edelman and Murakami {\cite{EM}} and revisited by Fortune
{\cite{F}} and many others trying to use the displacement structure of the
companion matrix. In fact, often, the author realized that the monomial basis
of the quotient algebra is not the most suitable one and proposed to express
the matrix of the same linear application but in other basis. In the case of
the Chebyshev basis this algorithm was already known by Barnett {\cite{B}} and
Cardinal later {\cite{C}}.

In the next section, we will also take advantage of the structure of the
quotient algebra to design an algorithm for gcd computation mainly using
linear algebra (eigenvalues are used in theory and never computed).

\subsection{Structure of quotient and gcd}



Let  such that they are both monic.
As above, we denote  and . We
denote denote  the set of roots of  and
we assume that  is squarefree, i.e.  if . We define  where  denote the remainder of
 by division by . We denote  the matrix
of  in the monomial basis  of
 but other bases can be used. A matrix representing the map
 is called an extended companion matrix.

\begin{proposition}
  The eigenvalues of  are .
\end{proposition}

\begin{proof}
  It is a direct corollary of the proposition \ref{mainprop} since if we write
  the matrix of this linear map in the Lagrange basis associated to
   is
  
  and gives the wanted result.
\end{proof}

Trivially, we have:

\begin{corollary}
  We have .
\end{corollary}

The column of index  of  is the column vector of the coefficients of
.



Let  be a basis of  and let  be the corresponding polynomials. First remark that
 is an ideal of .

\begin{lemma}
  The ideal  is a principal ideal. 
\end{lemma}

\begin{proof}
  Let us define
  
  For all  it is clear that
   and then  divide . Furthermore  since in the Lagrange basis
  
  This shows that .
\end{proof}

To compute , we built the matrix with columns formed by  and we make a triangulation operating only on the columns. This way we
obtain the polynomial of minimal degree linear combination of  and it is easily seen that this  up to a
multiplicative scalar factor.

\begin{lemma}
  The first column of a column echelon form of the matrix  built from a
  basis of  is the generator of , i.e. it is the vector of the coefficients of  up to a scalar
  multiplication.
\end{lemma}

\begin{proof}
  Since the columns of a column echelon form of the matrix  are linearly
  independent, they form a basis of  as
  -vector space. So  is a linear combination of the
  polynomials associated to those columns. The polynomial associated to the
  column echelon form of  have all different degree (because it is an
  echelon form) and so  is a linear combination of those polynomial.
  Because  as the lowest degree possible, it is a scalar multiple of
  the polynomial associated to the first column. 
\end{proof}

\begin{proposition}
  
\end{proposition}

\begin{proof}
  By construction, we have  and so  divide . We also have  since the roots of  are the root of  where  vanishes. Since  we have the wanted result.
\end{proof}

In all this section, we did not care if the polynomials are known in monomial
or Chebyshev basis for instance. In fact, in order to have an algebraic
algorithm, we only need to be able to perform euclidian division and this is
always the case if the polynomial basis is graduated (as for monomial,
Chebyshev, most of the orthogonal bases).



\section{Bezoutian and Barnett's formula\label{barnett}}



A classical matricial formulation of resultant is given by the B\'ezout
matrix. In this part, we recall the construction of the B\'ezout matrix and a
special factorization of the multiplication matrix expressed in the monomial
basis. This factorization is called Barnett formula (see {\cite{B}}). The
Barnett's formula allows to build the classical extended companion matrix
without using euclidian division and only stable numerical computations.
Furthermore, this factorization reveals that the extended companion matrix has
a special rank structure and we will use this fact later to design a fast
algorithm to compute AGCD.

\begin{definition}
  Let  and  of degree  and  respectively (with
  ), we denote . The
  B\'ezout matrix associated with  and  is .
\end{definition}

Remark that since  the matrix
 is symmetric. The polynomials  are univariate
polynomials of degree at most . One particular case of interest is when
. In this case the B\'ezout matrix has a Hankel structure, i.e.
. In this case we denote  for  which are called the
Horner polynomials.

\begin{proposition}
  Let , the polynomial  has degree  and since they have different
  degree, they form a basis of . Furthermore,
  .
\end{proposition}

\begin{corollary}
  The matrix  is the basis conversion from the Horner basis  to the monomial basis  of
  .
\end{corollary}

This leads us to the following theorem, known as Barnett formula (see
{\cite{B}}):

\begin{theorem}
  Let  be the multiplication matrix associated to  in  in the monomial basis, we have:
  
\end{theorem}

\begin{proof}
  We have  and so  in . So, for each
  , we have . This last equality means that  is the
  matrix of the multiplication by  in . The
  result follows directly from this fact.
\end{proof}

The Barnett's formula reveals the rank structure of the multiplication matrix.
Furthermore, this formula is already known if we choose Chebyshev basis
instead of monomial basis to express the polynomials and the matrices have
exactly the same nature. \



\section{Structured matrices and asymptotically fast algorithms}



In this section, we briefly recall some basics on displacement structured
matrices and related algorithms.

\subsection{Displacement structure}

Given an integer  and a complex number  with , define the circulant matrix

Next, define the {\tmem{Toeplitz-like displacement operator}} as the linear
operator

A matrix  is said to be {\tmem{Toeplitz-like}}
if  is a small rank matrix (where ``small'' means small with
respect to the matrix size). The number  is
called the {\tmem{displacement rank}} of . If  is Toeplitz-like, then
there exist (non-unique) {\tmem{displacement generators}}  and  such
that

Toeplitz matrices and their inverses are examples of Toeplitz-like matrices.
Another useful example is the multiplication matrix , which has
Toeplitz-like displacement rank equal to , regardless of its size.

A similar definition holds for {\tmem{Cauchy-like}} structure; here the
relevant displacement operator is

where  and  are diagonal matrices of appropriate size with disjoint
spectra. See {\cite{KS}} for a detailed description of displacement structure.

\subsection{Fast solution of displacement structured linear systems}

Gaussian elimination with partial pivoting (GEPP) is a well-known and reliable
algorithm that computes the solution of a linear system. Its arithmetic
complexity for an  matrix is asymptotically .
But if the system matrix exhibits displacement structure, it is possible to
apply a variant of GEPP with complexity . The main idea
consists in operating on displacement generators rather than on the whole
matrix; see {\cite{GKO}} for details.

Strictly speaking, the GKO algorithm performs GEPP (or, equivalently,
computes the PLU factorization) for Cauchy-like matrices. However, several
authors have pointed out (see {\cite{GKO}}, {\cite{Hei}}, {\cite{Pan}}) that
Toeplitz-like matrices can be stably and cheaply transformed into Cauchy-like
matrices; the same is true for displacement generators.

Consider, for instance, the case  and let  be an  Toeplitz-like matrix with generators  and . Denote by  the
matrix  and let
 be the Fourier matrix of size . Then the matrix is Cauchy-like, of the same displacement rank as , with respect to the
displacement operator defined by  and . Its Cauchy-like generators can
be computed as  and .

Generalization to the case of  rectangular matrices is possible.
In this case, the parameter  should be chosen so that the spectra
of  and  are well separated (see {\cite{AR}} and {\cite{BB}}).

We also point out that the GKO algorithm can be adapted to pivoting techniques
other than partial pivoting ({\cite{Gu}}, {\cite{Ste}}). This is especially
useful in case of instability due to internal growth of generator entries. A
Matlab implementation of the GKO algorithm that takes into account several
pivoting strategies is found in the package DRSolve described in {\cite{AR}}.
In our implementation, we use the pivoting strategy proposed in {\cite{Gu}}.

\section{A structured approach to AGCD computation}

We propose here an algorithm that exploits the algebraic and displacement
structure of the multiplication matrix to compute the AGCD of two given
polynomials with real coefficients (as defined in section \ref{intro}).

\subsection{Rank estimation}

It has been pointed out in Section \ref{exact} that the rank deficiency of the
multiplication matrix equals the AGCD degree. Here we use the structured
pivoted LU decomposition to estimate the approximate rank of the
multiplication matrix. Recall that  has a Toeplitz-like structure with
displacement rank 2; it can then be transformed into a Cauchy-like matrix
 as described in Section 5.2. Fast pivoted Gauss elimination yields
a factorization , where  is a square, nonsingular,
lower triangular matrix with diagonal entries equal to 1,  is upper
triangular and  are permutation matrices. Inspection of the diagonal
entries (or of the row norms) of  allows to estimate the approximate rank
of  and, therefore, of .

\subsection{Minimization of a quadratic functional}



Let us suppose that:
\begin{itemizedot}
  \item the polynomial  is exactly known,
  
  \item the polynomial  is approximately known
  and may be perturbed, so that we consider its coefficients as variables,
  
  \item the AGCD degree is known. 
\end{itemizedot}
Then we can reformulate the problem of AGCD computation as the minimization
of a quadratic functional. Indeed, recall that the cofactor  with
respect to  is defined by the ``shortest'' vector (i.e., the vector
with the maximum number of trailing zeros) that belongs to the null space of
 We assume  to be monic; we denote its degree as  and we have

Also observe that the entries of  are linear functions of the
coefficients of . Then the equation  can be rewritten as
=0, where the functional  is defined as

For a preliminary study of the problem, we have chosen to solve the equation
=0 by means of Newton's method, applied so as to exploit
structure. Denote by  the
vector of unknowns; then each Newton step has the form

In particular, notice that the Jacobian matrix associated with 
is an  Toeplitz-like matrix of displacement rank .
This property allows to compute a solution of the linear system  in a fast way; therefore, the arithmetic
complexity of each iteration is quadratic w.r.t. the degree of the input
polynomials.

We propose in the future to take into consideration other optimization methods
in the quasi-Newton family, such as BFGS.

\subsection{Computation of displacement generators}



In order to perform fast factorization of the multiplication matrix , we
need to compute Toeplitz-like displacement generators. It turns out that the
range of  is spanned by the first and last column of the
displaced matrix, and the columns of indices from  to  are multiples
of the first one. Therefore, it suffices to compute a few rows and columns of
 in order to obtain displacement generators. This can be done in a fast
and stable way by using Barnett's formula. \ If we denote as  the th
vector of the canonical basis of , then the computation of
the -th column of can be seen as



{\noindent}that is, it consists in solving a triangular Hankel linear system
and computing a matrix-vector product. For row computation, recall that the
Bezoutian is a symmetric matrix; we have analogously:

,

{\noindent}so that the computation of a row of  amounts to performing a
matrix-vector product and solving a Hankel triangular system.

A similar approach holds for computation of displacement generators of the
Jacobian matrix  associated with the functional .

\subsection{Description of the algorithm} \

Input: coefficients of polynomials  and .

{\noindent}Output: a perturbed polynomial  such that  and
 have a nontrivial common factor.
\begin{enumeratenumeric}
  \item Estimate the approximate rank  of  by computing a fast pivoted
  LU decomposition of the associated Cauchy-like matrix.
  
  \item Again by using fast LU, compute a vector  in the approximate null space of
  .
  
  \item Apply structured Newton with initial guess  and compute
  polynomials  and  such that  and  have a
  common factor of degree  and  is the monic cofactor
  for .
\end{enumeratenumeric}

\subsection{Numerical experiments and computational issues}



We have written a preliminary implementation of the proposed method in Matlab
(available at the URL
\tmtexttt{http://www.unilim.fr/pages\_perso/paola.boito/MMgcd.m}).

The results of a few numerical experiments are shown below. The polynomials
 and  are monic and have random coefficients uniformly distributed over
. They have an exact GCD of prescribed degree. A
perturbation is then added to . The perturbation vector has random entries
uniformly distributed over  and its norm is of the order of
magnitude of . We show:
\begin{itemizedot}
  \item the residual 
  
  \item the 2-norm distance between the exact and the computed cofactor ,
  
  \item the 2-norm distance between the exact and the computed perturbed
  polynomial  (which is expected to be roughly of the same order of
  magnitude as ).
\end{itemizedot}
In the following table we have taken 1e-5.



\begin{tabular}{|l|l|l|l|}
  \hline
   &  &  & \\
  \hline
  8, 7, 3 & 1.02e-15 & 1.19e-15 & 1.40e-5\\
  \hline
  15, 14, 5 & 1.51e-15 & 2.26e-15 & 1.35e-4\\
  \hline
  22, 22, 7 & 2.07e-13 & 1.40e-13 & 2.20e-4\\
  \hline
  36, 36, 11 & 1.19e-12 & 5.07e-14 & 0.0012\\
  \hline
\end{tabular}





Here are results for 1e-8:



\begin{tabular}{|c|c|c|c|}
  \hline
   &  &  & \\
  \hline
  8, 7, 3 & 5.49e-15 & 1.63e-15 & 5.85e-8\\
  \hline
  28, 27, 13 & 7.90e-14 & 8.98e-14 & 6.50e-7\\
  \hline
  38, 37, 13 & 4.88e-12 & 4.26e-12 & 2.30e-5\\
  \hline
  58, 57, 23 & 2.03e-12 & 4.40e-12 & 2.54e-4\\
  \hline
\end{tabular}







There are several issues in our approach that deserve further investigation.
Let us mention in particular:
\begin{itemizedot}
  \item The choice of a threshold (or a more refined technique) for estimating
  approximate rank.
  
  \item Normalization of polynomials: here we mostly work with monic
  polynomials, but other normalizations may be considered.
  
  \item The structured implementation of the optimization step (minimizing
  ). We have used for now a heuristic structured version of
  the Gauss-Newton algorithm. Observe that each step of classical Gauss-Newton
  applied to our problem has the form , where  is the vector containing the coefficients of
  the -th iterate polynomials  is the least-norm solution to the underdetermined system . Computing this least-norm
  solution in a structured and fast way is a difficult point that will require
  more work. Our implementation gives a solution which is not, in general, the
  least-norm one, even though it is typically quite close. Further work will
  also include a study of other possible optimization methods that lend
  themselves well to a structured approach.
  
  
\end{itemizedot}

\section{Conclusions}



We have proposed and implemented a fast structured matrix-based approach to a
variant of the AGCD problem, namely, the problem of computing an approximate
greatest common divisor of two univariate polynomials, one of which is known
to be exact. To our knowledge, this variant has been so far neglected in the
existing literature. It may be also interesting when one polynomial is known
with high accuracy and the other is not.

Our approach is based on the structure of the multiplication matrix and on
the subsequent reformulation of the problem as the minimization of a suitably
defined functional. Our choice of the multiplication matrix  over other
resultant matrices (e.g., Sylvester, B\'ezout...) is motivated by
\begin{itemizedot}
  \item the smaller size of , with respect e.g. to the Sylvester matrix,
  
  \item the strong link between the null space of  and the gcd, and in
  particular the fact that the null space of  immediately yields a gcd
  cofactor,
  
  \item the displacement structure of ,
  
  \item the possibility of computing selected rows and columns of  in a
  stable and cheap way, thanks to Barnett's formula.
\end{itemizedot}
This is, however, a preliminary study. Further work will include
generalizations of the proposed problem and a more thorough analysis of the
optimization part of the algorithm. Furthermore, this approach can be
generalized in several intersting way:
\begin{itemizedot}
  \item using better bases then the monomial one,
  
  \item it can be extended to some multivariate setting to compute the
  co-factor of a polynomial  in  when  define a complete intersection since
  Barnett formula still holds,
  
  \item to compute the AGCD of  with  where  is known
  with accuracy but  are inaccurate, one can take  as a
  linear combination of  with our method and succeed with a
  high probability.
\end{itemizedot}


\begin{thebibliography}{}
  {\bibitem{AR} A. Aric\`o, G. Rodriguez. {\tmem{A fast solver for linear
  systems with displacement structure}}. Numer. Algorithms, 2010. DOI:
  10.1007/s11075-010-9421-x.}
  
  {\bibitem{B} S. Barnett, {\tmem{Polynomials and linear control systems}},
  Monographs and Textbooks in Pure and Applied Mathematics, 77, Marcel Dekker,
  Inc., New York, 1983. xi+452 pp. ISBN: 0-8247-1898-4.}
  
  {\bibitem{BB} D. A. Bini, P. Boito, {\tmem{A fast algorithm for approximate
  polynomial gcd based on structured matrix computations}}, in Numerical
  Methods for Structured Matrices and Applications: Georg Heinig memorial
  volume, Operator Theory: Advances and Applications, vol. 199, Birkh\"auser
  (2010), 155-173.}
  
  {\bibitem{C} J. P. Cardinal, {\tmem{On two iterative methods for
  approximating roots of a polynomial}}, In J. Renegar, M. Shub and S. Smale
  editors, Proc. SIAM-AMS Summer Seminar on Math. of Numerical Analysis, Vol.
  32 of Lecture Notes in Applied Math., AMS Press (1996), 165-188.}
  
  {\bibitem{EM} A. Edelman, H. Murakami, \ {\tmem{Polynomial roots from
  companion matrix eigenvalues}}, Math. Comp. 64 (1995), no. 210, 763-776.}
  
  {\bibitem{F} S. Fortune, {\tmem{An iterated eigenvalue algorithm for
  approximating roots of univariate polynomials}}, Journal of Symbolic
  Computation, 33 (2002), no. 5, 627-646.}
  
  {\bibitem{GKO} I. Gohberg, T. Kailath, and V. Olshevsky, {\tmem{Fast
  Gaussian elimination with partial pivoting for matrices with displacement
  structure,}} Math. Comp. 64 (212), 1557-1576 (1995).}
  
  {\bibitem{Gu} M. Gu, {\tmem{Stable and Efficient Algorithms for Structured
  Systems of Linear Equations}}, SIAM J. Matrix Anal. Appl. 19, 279-306
  (1998).}
  
  {\bibitem{Hei} G. Heinig, {\tmem{Inversion of generalized Cauchy matrices
  and other classes of structured matrices}}, Linear Algebra in Signal
  Processing, IMA volumes in Mathematics and its Applications 69, 95-114
  (1994).}
  
  {\bibitem{KS} T. Kailath, A. H. Sayed, {\tmem{Displacement Structure: Theory
  and Applications}}, SIAM Review 37(3), 297-386 (1995). }
  
  {\bibitem{R} B. Mourrain, O. Ruatta, {\tmem{Relations between roots and
  coefficients, interpolation and application to system solving}}, Journal of
  Symbolic Computation, 33 (2002), no. 5, 679-699.}
  
  {\bibitem{Pan} V. Y. Pan, {\tmem{On computations with dense structured
  matrices}}, Math. of Comput. 55(191), 179-190 (1990).}
  
  {\bibitem{Ste} M. Stewart, {\tmem{Stable Pivoting for the Fast Factorization
  of Cauchy-Like Matrices}}, preprint (1997).}
  
  {\bibitem{Zeng} Z. Zeng, {\tmem{The approximate GCD of inexact polynomials
  Part I: a univariate algorithm}}, http://www.neiu.edu/\~{ }zzeng/uvgcd.pdf.}
  
  
\end{thebibliography}



\end{document}
