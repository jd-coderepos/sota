\documentclass{amsart}[10pt]

\sloppy 

\usepackage{amscd,amssymb,latexsym,url,verbatim,graphicx,color}
\usepackage{tikz,tikz-cd}
\usetikzlibrary{decorations.pathmorphing}

\usepackage{cases,amsmath}



\usepackage{txfonts}

\usepackage{tikz,tikz-cd}

\usepackage[dvips]{epsfig}

\title {Topology of the immediate snapshot complexes}

\author{Dmitry N. Kozlov}

\address{Department of Mathematics, University of Bremen, 28334
  Bremen, Federal Republic of Germany}

\email{dfk@math.uni-bremen.de}

\keywords{collapses, distributed computing, combinatorial 
algebraic topology, immediate snapshot, protocol complexes}
\newtheorem{theorem}{Theorem}[section]
\newtheorem{df}[theorem]{Definition}
\newtheorem{thm}[theorem]{Theorem} \newtheorem{lemma}[theorem]{Lemma}
\newtheorem{prop}[theorem]{Proposition}
\newtheorem{lm}[theorem]{Lemma}
\newtheorem{crl}[theorem]{Corollary}
\newtheorem{observ}[theorem]{Observation}
\newtheorem{openq}[theorem]{Open Question}
\newtheorem{rem}[theorem]{Remark}
\newtheorem{dcc}[theorem]{Distributed Computing Context}
\newtheorem{conj}[theorem]{Conjecture} \newcommand{\nin}{\noindent}
\newcommand{\pr}{\nin{\bf Proof.} }

\newcommand{\act}{\text{\rm act}\,}
\newcommand{\bd}{{\text{\rm bd}\,}}
\newcommand{\bo}{\partial}
\newcommand{\bss}{\text{\bf SS}}
\newcommand{\btop}{\text{\bf Top}}
\newcommand{\cb}{{\mathcal B}}
\newcommand{\cf}{{\mathcal F}}
\newcommand{\Ch}{\mych}
\newcommand{\chd}[1]{\mych(\Delta^{#1})}
\newcommand{\chdn}{\mych(\Delta^n)}
\newcommand{\ci}{{\mathcal I}}
\newcommand{\ck}{{\mathcal K}}
\newcommand{\cl}{{\mathcal L}}
\newcommand{\cn}{{\mathcal N}}
\newcommand{\cp}{{\mathcal P}}
\newcommand{\cs}{{\mathcal S}}
\newcommand{\codim}{\text{\rm codim}\,}
\newcommand{\csn}{\cs_{\zz_+}}
\newcommand{\da}{\Delta}
\newcommand{\dar}{\downarrow}
\newcommand{\dl}{\textrm{dl}}
\newcommand{\es}{\emptyset}
\newcommand{\fat}{\text{\bf{1}}}
\newcommand{\hra}{\hookrightarrow}
\newcommand{\inte}{\text{\rm int}\,}
\newcommand{\is}{\chi(\da^n)}
\newcommand{\last}{\text{\rm last}\,}
\newcommand{\lk}{\textrm{lk}}
\newcommand{\mqed}{\quad\Box}
\newcommand{\mych}{\chi}
\newcommand{\pass}{\text{\rm pass}\,}
\newcommand{\pnt}{round counter }
\newcommand{\pnts}{round counters }
\newcommand{\ra}{\rightarrow}
\newcommand{\rr}{{\mathbb R}}
\newcommand{\st}{\text{\rm{st}}}
\newcommand{\sm}{\setminus}
\newcommand{\smax}{\text{\rm smax}\,}
\newcommand{\sch}{\textrm{Sch}}
\newcommand{\sd}{{\text{\rm sd}\,}}
\newcommand{\supp}{\text{\rm supp}\,}
\newcommand{\tbe}{{\tt to be extended}}
\newcommand{\ti}{\tilde}
\newcommand{\tr}{{\bar r}}
\newcommand{\trc}{{\text{\rm Tr}}}
\newcommand{\tq}{{\bar q}}
\newcommand{\view}{\textrm{\rm View}^n}
\newcommand{\wti}{\widetilde}
\newcommand{\zz}{{\mathbb Z}}
\newcommand{\ab}{\allowbreak}

\numberwithin{equation}{section}
\numberwithin{figure}{section}
\numberwithin{table}{section}



\def\pstexInput#1{
\input{#1.pstex_t}  
}
\begin{document}

\begin{abstract}
The immediate snapshot complexes were introduced as combinatorial
models for the protocol complexes in the context of theoretical
distributed computing. In the previous work we have developed a~formal
language of witness structures in order to define and to analyze
these complexes.

In this paper, we study topology of immediate snapshot complexes. It
is known that these complexes are always pure and that they are
pseudomanifolds. Here we prove two further independent topological
properties. First, we show that immediate snapshot complexes are
collapsible. Second, we show that these complexes are homeomorphic to
closed balls. Specifically, given any immediate snapshot complex
, we show that there exists a~homeomorphism
, such that  is
a subcomplex of , whenever  is a simplex in the
simplicial complex .
\end{abstract}

\maketitle

\section{Witness structures and immediate snapshot protocol complexes}

\subsection{Modeling protocol complexes for the immediate snapshot 
read/write distributed protocols} 

\nin A crucial ingredient in the topological approach to theoretical
distributed computing, see~Herlihy et al, \cite{HKR}, is associating
a~simplicial complex, called the {\it protocol complex}, to every
distributed protocol, once the computational model is fixed. In this
paper, we study topology of standard full-information protocol
complexes in one of the central models of computation.

Let us fix the computational model to be the immediate snapshot
read/write model, which was originally introduced by Borowsky and
Gafni in~\cite{BG}. Roughly, this means that the processes can write
their values to the assigned memory registers, and they can read the
entire memory in one atomic step (snapshot read). The execution of the
protocol must have a layer structure, where in each layer a group of
processes becomes active, the processes in this group atomically write
their values to the memory, after this they atomically read the entire
memory.  Importantly, there are no further restrictions on how these
layers get activated during the protocol execution.

In our previous work, \cite{k1}, we introduced combinatorial models
for the protocol complexes for the standard protocols in that chosen
computational model, called {\it immediate snapshot complexes}. For
this, we needed to define new combinatorial structures, called {\it
  witness structures}, and study their structure theory, including
various operations, such as {\it ghosting}. We have proved that the
immediate snapshot complexes provide the correct model for these
protocol complexes, and started to study their topology. 

The standard protocols are naturally enumerated by finite sequences of
nonnegative integers, which we called {\it round counters}, denoted
. Accordingly, the immediate snapshot complexes themselves were
denoted . In \cite{k1} it was proved that the complexes
 are always pseudomanifolds with boundary, and the
combinatorics of the boundary subcomplex was described.

In this paper, we improve our understanding of topology of 
significantly. We refine the notion of canonical subcomplex
decomposition of  from \cite{k1}, and give a~complete
combinatorial description of the incidence relations in this
stratification. This gives us a~good approach to understanding the
inner structure of . In particular, it is straightforward to
prove the contractibility of  by pairing the combinatorial
description of this incidence structure with the standard result in
combinatorial topology, called the {\it Nerve Lemma},
see~\cite{book}. As a~first topological property we show a~stronger
result: namely, that the complexes  are always
collapsible. The collapsing sequence is also explicitly described.

It takes much more effort to derive the second topological property
of , namely the fact that each such complex is homeomorphic to
a~closed ball of dimension . This is the content of the
Corollary~\ref{crl:main}, which is an immediate consequence of our
main Theorem~\ref{thm:main}. Specifically, we prove that, for every
, there exists a~homeomorphism , such that  is a subcomplex of ,
whenever  is a simplex in the simplicial complex
.

The work presented here is the rigorous workout of the second part of
the preprint \cite{kfull}. The detailed expansion of the first part
of~\cite{kfull} has already appeared in \cite{k1}, where we laid the
combinatorial groundwork for the topological results of this paper. We
spend the rest of this section reminding the notations of~\cite{k1}
and results proved there. Our presentation here is quite condensed and
the reader is referred to~\cite{k1} for further details. We remark
that topology of protocol complexes for related computational models
has been studied by many authors, see
e.g.,~\cite{Ha04,HKR,HS,subd,view}. Furthermore, we recommend Attiya
and Welch, \cite{AW}, for an~in-depth background on theoretical
distributed computing.

Fundamentally, this paper can be viewed a~stand-alone article, written
in a~rigorous mathematical fashion, making it possible, in principle,
to be read independently. However, we strongly recommend that the
reader consults \cite{k1}, before starting reading this
one. Furthermore, in order both to facilitate researchers who are
mainly interested in distributed computing, as well as topologists
interested in more distributed computing background, we shall comment
throughout the text, explaining the distributed computing intuition
behind the mathematical concepts.

\subsection{Round counters} 

\nin To start with, we review some of the standard terminology which
we will use. We let  denote the set of nonnegative integers
.  For a natural number  we shall use  to
denote the set , with a convention that .
For a~finite subset , such that , we let
 denote the {\it second} largest element, i.e., . Finally, for a set  and an element
, we set

\nin Furthermore, whenever  is a~family of topological
spaces, we set .  Also, when no confusion
arises, we identify one-element sets with that element, and write,
e.g.,  instead of .


Next, we proceed to the combinatorial enumeration of all standard
protocols, together with relation terminology. This is accomplished by
the introduction of the so-called {\it round counters}.

\begin{df}
Given a function , we consider
the set
 
This set is called the {\bf support set} of .

\nin A {\bf round counter} is a function
 with a~finite support set.
\end{df}

Obviously, a~round counter can be thought of as an infinite sequence
, where, for all , either
 is a nonnegative integer, or , such that only
finitely many entries of  are nonnegative integers. We shall
frequently use a~short-hand notation  to denote
the round counter given by


\begin{df}
Given a round counter , the number  is
called the {\bf cardinality} of , and is denoted . The
sets
 
are called the {\bf active} and the {\bf passive} sets of~.
\end{df}

\begin{dcc}
Since we consider full-information protocols only, they can be
described by specifying the number of rounds each process executes the
write-read sequence. Mathematically, these protocols are indexed by
round counters. Given a~round counter , the set 
indexes the participating processes, and is required to be finite. The
symbol  means that the process does not participate.
Accordingly, the set  indexes the passive processes, i.e.,
those, which formally take part in the execution, but which do not
actually perform any active steps, while the set  indexes the
processes which execute at least one step.
\end{dcc}

The following special class of round counters is important for our
study.

\begin{df}
For an arbitrary pair of disjoint finite sets 
we define a~round counter  given by


Furthermore, for an arbitrary round counter , we set
.
\end{df}

We note that . In the paper we shall also
use the short-hand notation .

We define two operations on the round counters. To start with, assume
 is a~\pnt and we have a~subset . We let
 denote the \pnt defined by

We say that the \pnt  is obtained from  by the {\it
  deletion} of~. Note that ,
, and . 

Furthermore, we have . Finally, we note for future reference that for
 we have
 

For the second operation, assume  is a~\pnt and we have a~subset
. We let  denote the \pnt defined by

We say that the \pnt  is obtained from  by the {\it
  execution} of~. Note that ,
, and .
However, in general we have .

\begin{dcc} 
The replacement of  with  yields a new protocol, where
all processes from  have been banned from participation. The
replacement of  with  corresponds to letting processes
from  execute one round, and then running the remaining protocol
with new inputs.
\end{dcc}

For an arbitrary round pointer  and sets ,
 we set

In the special case, when , the identity \eqref{eq:rsa1}
specializes to

When , we shall frequently use the short-hand notation 
instead of , in other words, . 
Again, for future reference, we note that for , we have



\subsection{Witness structures and the ghosting operation} 

\nin Next, we describe the basic terminology which we will need to
define the immediate snapshot complexes.

\begin{df}\label{df:ws}
A~{\bf witness prestructure} is a~finite sequence of pairs of finite
subsets of , denoted , with
, satisfying the following conditions:
\begin{enumerate}
\item[(P1)] , for all ;
\item[(P2)] , for all , ;
\item[(P3)] , for all , .
\end{enumerate}

\noindent
A witness prestructure is called {\bf stable} if in addition the
following condition is satisfied:
\begin{enumerate}
\item[(S)] if , then .
\end{enumerate}

\noindent
A {\bf witness structure} is a witness prestructure satisfying the
following strengthening of condition (S):
\begin{enumerate}
\item[(W)] the subsets  are all nonempty.
\end{enumerate}
\end{df}

\begin{df}
We define the following data associated to an arbitrary witness
prestructure :
\begin{itemize}
\item the set  is called the {\bf support} of  and is
denoted by ;
\item the {\bf ghost set} of  is the set
  ;
\item the {\bf active set} of  is the complement of the ghost set

\item the {\bf dimension} of  is
  
\end{itemize}
\end{df}
\noindent
For brevity of some formulas, we set .

\begin{df}\label{df:trc}
For a~prestructure  and an arbitrary , we set
 
and call it the {\bf trace} of~. Furthermore, for all
, we set 

\end{df}

\nin When the choice of  is unambiguous, we shall simply write
 and . The following definition provides an
alternative approach to witness structures using traces.

\begin{df}\label{df:trws}
A~{\bf witness prestructure} is a~pair of finite subsets
 together with a~family  of finite subsets of , satisfying the following
condition:
\begin{itemize}
\item[(T)] , for all .
\end{itemize}

\noindent
A witness prestructure is called {\bf stable} if it satisfies
an~additional condition:
\begin{itemize}
\item[(TS)] if , then , for all ,
else 

\end{itemize}

Set . A~stable witness prestructure is
called {\bf witness structure} if the following strengthening of
Condition (TS) is satisfied:
\begin{enumerate}
\item[(TW)] {\it for all  either there exists 
  such that , or there exists  such that
  .}
\end{enumerate}
\end{df}

We shall call the form of the presentation of the witness prestructure
as a~triple  its {\it trace form}.

\begin{dcc}
The witness structure is a mathematical object modelling the
information which the processes have during the execution of the
full-information protocol. Let us explain the distributed computing
intuition behind this notation.

The set  indexes all processes which are participating in
the protocol. The processes indexed by the set  are of two
different types.  Those, whose view is included in , and
those, who have only been passively witnessed by others. The processes
of the first type are indexed by the set , the other ones
are indexed by the union . The set  indexes
those processes from  which have not be witnessed by
anybody in this particular execution.

The fact, that  is to be interpreted as ``the active
participation of process  in round  has been witnessed''. This
can happen in two ways, either  itself is active in this execution,
or  is being passively witnessed and this is not the last occurence
of . The fact that  means that process  has been
passively witnessed and this is the last occurence of . 

We refer the reader to \cite[Section 6]{k1}, where connection between
witness structures and witness posets is explained.
\end{dcc}

Next, we proceed to describe various operations in witness structures
and prestructures. To start with, any stable witness prestructure can
be turned into a witness structure, which is called its {\it canonical
  form}.

\begin{df}\label{df:cform}
Assume  is an arbitrary stable
witness prestructure. Set .  Pick , such that
.  We define
the witness structure , which is called the {\bf
  canonical form} of~, by setting

\end{df}


Furthermore, any witness prestructure can be made stable using the
following operation.

\begin{df} \label{df:st}
Let  be a~witness prestructure,
set

The {\bf stabilization} of  is the witness prestructure
 whose trace form is .

More generally, assume , and set

The {\bf stabilization} of  {\bf modulo}  is the witness
prestructure  whose trace form is .
\end{df}

Combining stabilization modulo a~set with taking the canonical form
yields a~new operation, called {\it ghosting}, which will be of utter
importance for the combinatorial description of the incidence
structure in the immediate snapshot complexes.

\begin{df}\label{df:go}
For an arbitrary witness structure , and an~arbitrary
, we define
. We say that 
is obtained from  {\bf by ghosting~}.
\end{df}

\begin{dcc}\label{dcc:4}
The operation of ghosting the set of processes  corresponds to
excluding their views from the knowledge that the witness structure
encodes. Clearly, the occurences of processes from  will not vanish
from the witness structure  altogether, but these
processes will cease being active, and whatever we will see of them
will just be the residual information passively witnessed by other
processes.
\end{dcc}

The main property of ghosting which one needs for proving the
well-definedness of the immediate snapshot complexes is that it
behaves well with respect to iterations.

\begin{prop} 
\label{prop:gg}
Assume  is a witness structure, and , 
such that . Then we have
, expressed 
functorially we have .
\end{prop}

\subsection{Immediate snapshot complexes} 

\nin We have now introduced sufficient terminology in order to
describe our main objects of study.

\begin{df}\label{df:ptr}
Assume  is a~round counter. We define an abstract simplicial
complex , called the {\bf immediate snapshot complex}
associated to the round counter~, as follows. The vertices of
 are indexed by all witness structures , satisfying these three
conditions:
\begin{enumerate}
\item ;
\item ;
\item , for all .
\end{enumerate}
\nin We say that such a vertex has {\bf color} . In general, the
simplices of  are indexed by all witness structures
, satisfying:
\begin{enumerate}
\item ;
\item , for all ;
\item , for all .
\end{enumerate}

\nin The empty witness structure  indexes the empty
simplex of . When convenient, we identify the simplices of
 with the witness structures which index them.  

Let  be a~non-empty witness structure satisfying the
conditions above. The set of vertices  of the simplex 
is given by .
\end{df}

Taking boundaries of simplices in  corresponds to ghosting of
the witness structures. This is only natural taking into account the
intuition from the distributed computing context~\ref{dcc:4}.

\begin{prop}
\label{prop:b}
Assume  is the round counter, and assume  and  are
simplices of . Then  if and only if there
exists , such that .
\end{prop}

The first property of the simplicial complexes , which is
quite easy to see, is that these complexes are pure of dimension
. Furthermore, zero values in the round counter have
a~simple topological interpretation.

\begin{prop}\label{prop:p4}{\rm (\cite[Proposition 4.4]{k1}).}
Assume  and . Let  denote the
truncated round counter . Consider a~cone over
, which we denote , where  is the apex
of the cone. Then we have
 
where  denotes the simplicial isomorphism.
\end{prop}

For brevity, we set . It turns
out that the standard chromatic subdivision of an~-simplex,
see~\cite{subd}, is a~special case of the immediate snapshot complex.

\begin{prop} {\rm (\cite[Proposition 4.10]{k1}).}
The immediate snapshot complex  and the standard chromatic
subdivision of an -simplex  are isomorphic as
simplicial complexes. Explicitly, the isomorphism can be given by

where .
\end{prop}

Recall the following property of pure simplicial complexes,
strengthening the usual connectivity.

\begin{df}
Let  be a pure simplicial complex of dimension . Two
-simplices of  are said to be {\bf strongly connected} if there
is a~sequence of -simplices so that each pair of consecutive
simplices has a~common -dimensional face. The complex  is
said to be {\bf strongly connected} if any two -simplices of 
are strongly connected.
\end{df}

Clearly, being strongly connected is an equivalence relation on the
set of all -simplices.

\begin{prop}\label{prop:strc} {\rm (\cite[Proposition 5.6]{k1}).}
For an arbitrary round counter , the simplicial complex 
is strongly connected.
\end{prop}

The next definition describes a~weak simplicial analog of being
a~manifold.

\begin{df}
We say that a~strongly connected pure simplicial complex  is a~{\bf
  pseudomanifold} if each -simplex of  is a~face of
precisely one or two -simplices of . The -simplices of
 which are faces of precisely one -simplex of  form
a~simplicial subcomplex of , called the {\bf boundary} of , and
denoted .
\end{df}

It was shown in \cite{k1}, that immediate snapshot complexes are
always pseudomanifolds with boundary.

\begin{thm}\label{prop:pseudo} {\rm (\cite[Proposition 5.9]{k1}).}
For an arbitrary round counter , the simplicial complex 
is a~pseudomanifold, where  is the subcomplex
consisting of all simplices ,
satisfying .
\end{thm}

\section{A canonical decomposition of the immediate snapshot complexes}

\subsection{Definition and examples}


\nin The canonical decomposition of the immediate snapshot complexes has
been introduced in \cite{k1}. In order to better understand the
topology of these complexes, we need to generalize that definition and
look at finer strata.

\begin{df}
Assume  is a~round counter.
\begin{itemize}
\item For every subset , let  denote the set
  of all simplices , such
  that .
\item For every pair of subsets , let
   denote the set of all simplices
  , such that  and
  . Furthermore, set 
\end{itemize} 
\end{df}

We shall also use the following short-hand notation:
. This case has been considered in \cite{k1},
where it was shown that  is a~simplicial subcomplex of 
for an~arbitrary~.

\begin{dcc}
The subcomplexes  correspond to the subset of executions which
start with the processes indexed by the set  executing
simultaneously.  This explains, from the point of view of distributed
computing, why the protocol complex decomposes into these strata.
\end{dcc}

On the other extreme, clearly  for all .  When
, we shall use the convention .  Note,
that in general the sets  need not be closed under taking
boundary.

\begin{prop}
The sets  are closed under taking boundary, hence form
simplicial subcomplexes of .
\end{prop}
\pr Let  be a~simplex in
, and assume . By Proposition~\ref{prop:b}
there exists , such that
. By Proposition~\ref{prop:gg} it is enough to
consider the case , so assume , and let .

By definition of  we have either  or
.  Consider first the case , so
. Since , we have .

Now, assume . This means  and
. Again  implies
.  \qed

\vspace{5pt}

\noindent
In particular,  and  are simplicial subcomplexes of
, for all . When we are dealing with several round
counters, in order to avoid confusion, we shall add  to the
notations, and write , , ,
. We shall also let  denote the inclusion
map 




\subsection{The strata of the canonical decomposition as immediate snapshot complexes}


\nin The first important property of the simplicial complexes
 is that they themselves can be interpreted as immediate
snapshot complexes. Here, and in the rest of the paper, we shall use
 to denote simplicial isomorphisms.

\begin{prop}\label{prop:strata}
Assume , then there exists a~simplicial
isomorphism
 
\end{prop}
\pr We start by considering the case . Pick an~arbitrary
simplex  belonging to .  
By the construction of , we either have , or
. If , then set

else , in which case we set


Reversely, assume  is a~simplex of
. Note, that in any case, we have . If , we set

else , and we set


It is immediate that  and  preserve supports,
, , and hence also the dimension. Furthermore, we can see
what happens with the cardinalities of the traces. For all elements
 which do not belong to~, the cardinalities of their traces are
preserved. For all elements in , the map  decreases the
cardinality of the trace, whereas, the map  increases it. It
follows that  and  are well-defined as
dimension-preserving maps between sets of simplices.

To see that  preserves boundaries, pick a~top-dimensional
simplex  in  and ghost the
set~. Assume first . In this case not all
elements in  are ghosted. Assume now that . This
implies that  is well-defined as a~simplicial map.  Finally,
a~direct verification shows that the maps  and  are
inverses of each other, hence they are simplicial isomorphisms.

Let us now consider the case when  is arbitrary. The simplicial
complex  is a~subcomplex of  consisting of all simplices
 satisfying the additional condition . The
image  consists of all
 in  satisfying
. The map , taking  to , is obviously a~simplicial isomorphism,
hence the composition
 is
a~simplicial isomorphism as well.  \qed

\vskip5pt

\nin Note that, in particular,

\nin The statement of Proposition~\ref{prop:strata} for the example
, is shown on Figure~\ref{fig:f211b}.

The next proposition is a~first of several results, which claim
commutativity of a~certain diagram. All these results have alternative
intuitive meaning. For example, the commutativity of the
diagram~\eqref{cd:xs} can be interpreted as saying that the relation of
the stratum  to  is the same as the
relation of the stratum  to .

\begin{prop}\label{prop:cxs}
Assume  is an arbitrary round counter, and ,
such that , then the following diagram commutes

where  denotes the strata inclusion map.
\end{prop}
\pr To start with, note that , so the diagram \eqref{cd:xs} is
well-defined. To see that it is commutative, pick an arbitrary
. We know that either  and , or . On one
hand, we have

On the other hand, we have
0.6cm]
\begin{array}{|c|c|c|c|c|}
\hline
W_0\sm(S\cup A) & W_1             & W_2 & \dots & W_t \\ \hline
G_0\cup S       & G_1\sm(S\cup A) & G_2 & \dots & G_t \\ 
\hline
\end{array},
\textrm{ if } A\cup S \subseteq G_1.
\end{cases}\label{eq:xaa}
X_{A,A}(\tr)=\bigcup_{\es\neq S\subseteq\act\tr\sm A}X_{S\cup A,A}(\tr)=
\bigcup_{A\subset T\subseteq\act\tr}X_{T,A}(\tr).
\begin{cases}R_1=S\\A\subseteq G_1\end{cases}\Rightarrow
\begin{cases}R_1=T\\B\subseteq G_1\end{cases}
\Rightarrow\sigma\in Y_{T,B}.\tau:=\begin{array}{|c|c|c|c|c|}
\hline
\supp\tr & S\sm A & p_1 & \dots & p_t \\ \hline
\es      & A      & \es & \dots & \es \\ 
\hline
\end{array},\tau:=\begin{array}{|c|c|c|c|c|}
\hline
\supp\tr & p_1 & p_2 & \dots & p_t \\ \hline
\es      & S   & \es & \dots & \es \\ 
\hline
\end{array}, \label{eq:xx}
X_{S,A}\cap X_{T,B}=(Z_S\cap Z_T)\cup(Z_S\cap Y_{T,B})\cup(Y_{S,A}\cap Z_T)\cup
(Y_{S,A}\cap Y_{T,B})\\
=\begin{cases}
Z_{S\cup T}\cup Y_{T,S\cup B}\cup Y_{S,T\cup A}\cup Y_{S,A\cup B}, & \text{ if } S=T; \\
Z_{S\cup T}\cup Y_{T,S\cup B}\cup Y_{S,T\cup A}, & \text{ otherwise}.
\end{cases}
X_S\cap X_T=\begin{cases}
X_{T,S}, & \text{ if } S\subset T, \\
Z_{S\cup T}, & \text{ otherwise,}
\end{cases}
\label{eq:xz}
X_S\cap Z_T=Z_{S\cup T}.
X_{S,\es}\cap X_{T,T}=\begin{cases}
X_{T,S\cup T}, & \text{ if } S\subset T \\
Z_{S\cup T}, & \text{ otherwise}
\end{cases}=\begin{cases}
Z_T, & \text{ if } S\subset T \\
Z_{S\cup T}, & \text{ otherwise}
\end{cases}=Z_{S\cup T}.\mqed
 X_{S_1}\cap\dots\cap X_{S_t}=X_{S_1,\es}\cap X_{S_2,\es}\cap\dots 
\cap X_{S_t,\es}= X_{S_1,S_2}\cap X_{S_3,\es}\cap\dots\cap
X_{S_t,\es}\\ =X_{S_1,S_2\cup S_3}\cap X_{S_4,\es}\cap\dots\cap
X_{S_t,\es}=\dots=X_{S_1,S_2\cup\dots\cup S_t}.

Z_{S_1\cup S_2}\cap X_{S_3}\cap\dots\cap X_{S_t}=Z_{S_1\cup S_2\cup
  S_3}\cap X_{S_4}\cap\dots\cap X_{S_t} =X_{S_1\cup S_2\cup\dots\cup
  S_t},
\beta_V(\tr):B_V(\tr)\hookrightarrow P(\tr).\delta_V(\tr):((W_0,G_0),\dots,(W_t,G_t))\mapsto((W_0,G_0\sm V),\dots,(W_t,G_t))X_{S,A,V}(\tr):=X_{S,A}(\tr)\cap B_V(\tr).\label{eq:bar}
\begin{tikzcd}P(\tr)\arrow[hookleftarrow]{r}{\alpha}\arrow[hookleftarrow]{d}{\beta} 
&X_{S,A}(\tr)\arrow[squiggly]{r}{\gamma}\arrow[hookleftarrow]{d}{j} 
&P(\tr_{S,A})\arrow[hookleftarrow]{d}{\beta} \\
B_V(\tr)\arrow[hookleftarrow]{r}{i}
\arrow[squiggly]{d}{\delta} 
&X_{S,A,V}(\tr)\arrow[squiggly]{r}{\varphi}\arrow[squiggly]{d}{\psi} 
&B_V(\tr_{S,A})\arrow[squiggly]{d}{\delta} \\
P(\tr\sm V)\arrow[hookleftarrow]{r}{\alpha} 
&X_{S,A}(\tr\sm V)\arrow[squiggly]{r}{\gamma} 
&P(\bar r_{S\cup V,A\cup V}),
\end{tikzcd}
\label{cd:b1}
\begin{tikzcd}[column sep=large]
X_{S,B}(\tr)\arrow[hookleftarrow]{r}{i}\arrow[squiggly]{d}[swap]{\gamma_{S,B}(\tr)}
& X_{S,A}(\tr)\arrow[squiggly]{rd}{\gamma_{S,A}(\tr)} \\
P(\tr_{S,B})\arrow[hookleftarrow]{r}{\beta_{A\sm B}(\tr_{S,B})}
& B_{A\sm B}(\tr_{S,B})\arrow[squiggly]{r}{\delta_{A\sm B}(\tr_{S,B})}
& P(\tr_{S,A})
\end{tikzcd}
(\gamma_{S,B}(\tr)\circ i)(\sigma)=
\begin{cases}
\begin{array}{|c|c|c|c|}
\hline
W_0\sm G_1       &  W_2 & \dots & W_t \\ \hline
G_0\cup G_1\sm B &  G_2 & \dots & G_t \\ 
\hline
\end{array}, 
\textrm{ if } W_1\cup G_1=S,\,\,A\subseteq G_1;\
On the other hand, we have 
0.6cm]
\begin{array}{|c|c|c|c|c|}
\hline
W_0\sm S  & W_1      & W_2 & \dots & W_t \\ \hline
G_0\cup S\sm A & G_1\sm S & G_2 & \dots & G_t \\ 
\hline
\end{array},
\textrm{ if } S \subseteq G_1.
\end{cases}
(\gamma_{S,B}(\tr)\circ i)(\sigma)=(\beta_{A\sm B}(\tr_{S,B})\circ
\delta_{A\sm B}(\tr_{S,B})^{-1}\circ\gamma_{S,A}(\tr))(\sigma).(S,(B_1,\dots,B_t)(C_1,\dots,C_t))\mapsto
\begin{array}{|c|c|c|c|}
\hline
W_0            & C_1        & \dots & C_t \\ \hline
(A\cup B)\sm W_0 & B_1\sm C_1 & \dots & B_t\sm C_t \\ 
\hline
\end{array},\tau_A:P(\chi_A)\,{\underset{\cong}\longrightarrow}\,\Delta^A,\label{cd:tau3}
\begin{tikzcd}[column sep=1.5cm]
P(\chi_A)\arrow[hookleftarrow]{r}{\beta_{A\sm C}(\chi_A)}
\arrow{d}{\cong}[swap]{\tau_A} 
&B_{A\sm C}(\chi_A)\arrow[squiggly]{r}{\delta_{A\sm C}(\chi_A)} 
&P(\chi_C)\arrow{ld}{\cong}[swap]{\tau_C}\\
\da^A\arrow[hookleftarrow]{r}{i}&\da^C
\end{tikzcd}
\tau_{A,B}:P(\chi_{A,B})\underset{\cong}\longrightarrow\Delta^{A\cup B}.\tau(\chi_{A,B},\chi_{C,D}):=\tau_{C,D}^{-1}\circ\tau_{A,B},
\begin{tikzcd}[column sep=2cm]
B_S(\chi_{A,B})\arrow[squiggly]{r}{\tau(\chi_{A,B},\chi_{C,D})}
\arrow[hookrightarrow]{d}{\beta_S(\chi_{A,B})}  
&B_S(\chi_{C,D})\arrow[hookrightarrow]{d}{\beta_S(\chi_{C,D})}  \\ 
P(\chi_{A,B}) \arrow{r}{\tau(\chi_{A,B},\chi_{C,D})}[swap]{\cong}
&P(\chi_{C,D}) 
\end{tikzcd}
\tau(\chi_{A_1,B_1},\chi_{A_2,B_2})\circ\tau(\chi_{A_2,B_2},\chi_{A_3,B_3})=
\tau(\chi_{A_1,B_1},\chi_{A_3,B_3}).\beta_V(\chi_{C.D}):B_V(\chi_{C,D})\underset\cong\longrightarrow P(\chi_{C\sm A,D\sm A}).X_S(\chi_{C.D}):\gamma_S(\chi_{C,D})\underset\cong\longrightarrow P(\chi_{C\sm S,D\cup S}). \label{cd:tau}
\begin{tikzcd}[column sep=1.5cm]
P(\chi_{A,B})\arrow[hookleftarrow]{r}{\beta_V(\chi_{A,B})}
\arrow{d}{\cong}[swap]{\tau(\chi_{A,B},\chi_{C,D})}
& B_V(\chi_{A,B})\arrow[squiggly]{r}{\delta_V(\chi_{A,B})}
& P(\chi_{A,B}\sm V)\arrow{d}{\tau(\chi_{A,B}\sm V,\chi_{C,D}\sm V)}[swap]{\cong} \\   
P(\chi_{C,D})\arrow[hookleftarrow]{r}{\beta_V(\chi_{C,D})}
& B_V(\chi_{C,D})\arrow[squiggly]{r}{\delta_V(\chi_{C,D})}
& P(\chi_{C,D}\sm V)
\end{tikzcd}
\inte P(\tr):=\bigcup_{\sigma\in P(\tr),\,\,\sigma\notin\partial P(\tr)}\inte\sigma,\partial X_{S,A}(\tr):=\gamma_{S,A}(\tr)^{-1}(\partial P(\tr_{S,A})),\quad
\inte X_{S,A}(\tr):=\gamma_{S,A}(\tr)^{-1}(\inte P(\tr_{S,A})).\sigma=((W_0,V),\allowbreak (S\sm A,A),\dots,(W_t,G_t)).\rho(\sigma)=((W_0\sm G_1,(G_0\cup G_1)\sm(A\cup V)),(W_2,G_2),
\dots,(W_t,G_t)).\rho(\sigma)=((W_0\sm S,(G_0\cup S)\sm(A\cup V)),(W_1,G_1\sm S),
(W_2,G_2),\dots,(W_t,G_t)).\label{eq:intt}
\inte\tau\subseteq\inte X_1\cup\dots\cup\inte X_i\cup
\inte Y_1\cup\dots\cup\inte Y_i.
 \label{eq:c2}
\textrm{ if } \varphi(x)\in B_i, \textrm{ then } x\in A_i,
\label{eq:jhc}
\varphi_i(A\cap A_i)=\varphi(A)\cap B_i.
 \label{eq:c3}
\textrm{ for all } I\subseteq[t]: \varphi:A_I\rightarrow B_I
\textrm{ is a~bijection. }
 \label{cd:ij1}
\begin{tikzcd}
A_J\arrow{r}{\varphi_J}[swap]{\cong}\arrow[hookrightarrow]{d} 
&B_J\arrow[hookrightarrow]{d} \\
A_I \arrow{r}{\varphi_I}[swap]{\cong} &B_I
\end{tikzcd}
\begin{tikzcd}
A_i\arrow[hookleftarrow]{r}\arrow{d}{\varphi_{\{i\}}}[swap]{\cong} 
&A_{\{i,j\}}\arrow[hookrightarrow]{r}\arrow{d}{\varphi_{\{i,j\}}}[swap]{\cong} 
&A_j\arrow{d}{\varphi_{\{j\}}}[swap]{\cong} \\
B_i\arrow[hookleftarrow]{r} 
&B_{\{i,j\}}\arrow[hookrightarrow]{r} &B_j
\end{tikzcd}\Phi(\tr):P(\tr)\stackrel\cong\longrightarrow P(\chi(\tr)),\label{cd:b}
\begin{tikzcd}[column sep=1.1cm]
P(\tr\sm V)\arrow[leftsquigarrow]{r}{\delta_V(\tr)} 
\arrow{d}{\cong}[swap]{\Phi(\tr\sm V)}
&B_V(\tr)\arrow[hookrightarrow]{r}{\beta_V(\tr)} 
& P(\tr)\arrow{d}{\Phi(\tr)}[swap]{\cong} \\ 
P(\chi(\tr\sm V))\arrow[leftsquigarrow]{r}{\delta_V(\chi(\tr))} 
& B_V(\chi(\tr))\arrow[hookrightarrow]{r}{\beta_V(\chi(\tr))} 
&P(\chi(\tr)) 
\end{tikzcd}
\label{cd:b2}
\begin{tikzcd}[column sep=1.1cm]
X_S(\tr)\arrow[squiggly]{r}{\gamma_S(\tr)}\arrow[hookrightarrow]{rd}{\alpha_S(\tr)}
&P(\tr_S)\arrow{r}{\Phi(\tr_S)}[swap]{\cong}
&P(\chi(\tr_S))\arrow{r}{\tau}[swap]{\cong}
&P(\chi(\tr)_S)\arrow[leftsquigarrow]{r}{\gamma_S(\chi(\tr))}
&X_S(\chi(\tr))\\
&P(\tr)\arrow{rr}{\Phi(\tr)}[swap]{\cong}
& & P(\chi(\tr))\arrow[hookleftarrow]{ru}{\alpha_S(\chi(\tr))}
\end{tikzcd}
\varphi_{S,A}(\tr):X_{S,A}(\tr)\longrightarrow X_{S,A}(\chi(\tr)), \label{cd:phi}
\begin{tikzcd}[column sep=0.9cm]
\varphi_{S,A}(\tr):X_{S,A}(\tr)\arrow[squiggly]{r}{\gamma_{S,A}(\tr)} 
&P(\tr_{S,A}) \arrow{r}{\Phi(\tr_{S,A})}[swap]{\cong} 
&P(\chi(\tr_{S,A}))\arrow{r}{\tau}[swap]{\cong} 
&P(\chi(\tr)_{S,A})\arrow[leftsquigarrow]{r}{\gamma_{S,A}(\chi(\tr))} 
&X_{S,A}(\chi(\tr)),
\end{tikzcd}
\label{cd:m1}
\begin{tikzcd}
X_{S,A}(\tr)\arrow{r}{\varphi_{S,A}(\tr)}[swap]{\cong}
\arrow[hookrightarrow]{d}{i} 
&X_{S,A}(\chi(\tr))\arrow[hookrightarrow]{d}{j}  \\ 
X_{T,B}(\tr)\arrow{r}{\varphi_{T,B}(\tr)}[swap]{\cong} 
&X_{T,B}(\chi(\tr)),  
\end{tikzcd}

\begin{tikzcd}[column sep=2cm]
\rho:X_S(\chi(\tr))\arrow[squiggly]{r}{\gamma_S(\chi(\tr))}
&P(\chi(\tr)_S)\arrow{r}{\tau(\chi(\tr_S),\chi(\tr)_S)^{-1}}
&P(\chi(\tr_S))
\end{tikzcd}

\begin{tikzcd}[column sep=2cm]
\nu:X_S(\chi(\tr\sm A))\arrow[squiggly]{r}{\gamma_S(\chi(\tr\sm A))}
&P(\chi(\tr)_{S,A})\arrow{r}{\tau(\chi(\tr_{S,A}),\chi(\tr)_{S,A})^{-1}}
&P(\chi(\tr_{S,A})),
\end{tikzcd}

\label{eq:s1}
\begin{tikzcd}[column sep=0.6cm]
B_V(\tr)\cap X_S(\tr)\arrow[hookrightarrow]{r}
&B_V(\tr)\arrow[hookrightarrow]{r}{\beta}
&P(\tr)\arrow{r}{\Phi}[swap]{\cong}&P(\chi(\tr))
\end{tikzcd}

\label{eq:s2}
\begin{tikzcd}[column sep=0.5cm]
X_{S,\es,V}(\tr)\arrow[hookrightarrow]{r}
&B_V(\tr)\arrow[squiggly]{r}{\delta}
&P(\tr\sm V)\arrow{r}{\Phi}[swap]{\cong}
&P(\chi(\tr\sm V))\arrow[leftsquigarrow]{r}{\delta}
&B_V(\chi(\tr))\arrow[hookrightarrow]{r}{\beta}
&P(\chi(\tr))
\end{tikzcd}

It follows by a~simple diagram chase that the commutativities in the
diagram on Figure~\ref{cd:mt2} which we have shown imply the equality
of these two maps. This is true for all , such that
 and . On the other hand, the
subcomplexes , where , , cover .  As a~matter of fact, the simplicial
isomorphisms  and  show that they induce
a~stratification which is isomorphic to the stratification of
 by . The fact that they cover 
completely implies that the maps \eqref{eq:s1} and \eqref{eq:s2}
remain the same after the first term is skipped, which is the same as
to say that \eqref{cd:b} commutes.  This concludes the proof.  \qed

\begin{crl}\label{crl:main}
For an arbitrary round counter  the immediate snapshot complex
  is homeomorphic to the closed ball of dimension~.
\end{crl}

\begin{comment} 
\section{Topological properties}

In all following conjectures there are points a, b, c.  Point a refers
to , point b to , point c to the general
.

\begin{conj}
 Topological space  is contractible.
\end{conj}

\begin{conj}
  Topological space  is homeomorphic to an -ball.
\end{conj}

\begin{conj}
 Simplicial complex  is collapsible.
\end{conj}


In total, these are 9 conjectures, whereas for c we need to define the formal
framework of protocol families.

\section{Geometric properties}

\begin{conj}
 Simplicial complex  has a geometric realization which is 
a simplicial subdivision of the standard simplex .
\end{conj}

\begin{conj}
 We can give explicit description of this geometric realization
including algorithms for each point of a simplex to decide
in which simplex in the subdivision it is.
\end{conj}

\end{comment}




\begin{thebibliography}{CR12a,}


\bibitem[AW]{AW} H.\ Attiya, J.\ Welch, {\it Distributed Computing:
  Fundamentals, Simulations, and Advanced Topics}, Wiley Series on
  Parallel and Distributed Computing, 2nd Edition, Wiley-Interscience,
  2004. 432~pp.

\bibitem[BG]{BG} E.\ Borowsky, E.\ Gafni, {\it Immediate atomic
  snapshots and fast renaming}, Proc.\ 12th annual ACM symposium on
  Principles of distributed computing (PODC'93), ACM, New York, NY,
  (1993), 41--51.

\bibitem[Co73]{coh} M.\ Cohen, {\em A course in simple-homotopy
  theory}, Graduate Texts in Mathematics, Vol. 10, Springer-Verlag,
  New York-Berlin, 1973.
	
\bibitem[Hat02]{Hat} A.\ Hatcher, {\it Algebraic topology}, Cambridge
University Press, Cambridge, 2002.

\bibitem[Ha04]{Ha04} J.\ Havlicek, {\em A Note on the Homotopy Type of Wait-Free 
Atomic Snapshot Protocol Complexes}, SIAM J. Computing {\bf 33} Issue 5, 
(2004), 1215--1222. 

\bibitem[Her]{Herl} M.\ Herlihy, personal communication, 2013.

\bibitem[HKR]{HKR} M.\ Herlihy, D.N.\ Kozlov, S.\ Rajsbaum, {\it
  Distributed Computing through Combinatorial Topology}, Elsevier, 2014, 336 pp.

\bibitem[HS]{HS} M.\ Herlihy, N.\ Shavit, {\it The topological
  structure of asynchronous computability}, J.\ ACM {\bf 46} (1999),
  no.\ 6, 858--923.

\bibitem[Ko07]{book} D.N.\ Kozlov, {\em Combinatorial
Algebraic Topology}, Algorithms and Computation in Mathematics {\bf
21}, Springer-Verlag Berlin Heidelberg, 2008, XX, 390 pp.\ 115 illus.

\bibitem[Ko12]{subd} D.N.\ Kozlov, {\em Chromatic subdivision of 
a~simplicial complex}, Homology, Homotopy and Applications {\bf
  14}(2) (2012), 197--209.

\bibitem[Ko13]{view} D.N.\ Kozlov, {\em Topology of the view complex},
  preprint, 10 pages, submitted for publication, \newline {\tt
    arXiv:1311.7283 [cs.DC]}

\bibitem[Ko14a]{kfull} D.N.\ Kozlov, {\em Standard protocol complexes
  for the immediate snapshot read/write model}, preprint,
  34~pages, \newline {\tt arXiv:1402.4707 [cs.DC]}

\bibitem[Ko14b]{k1} D.N.\ Kozlov, {\em Witness structures and
  immediate snapshot complexes}, preprint, 26 pages, \newline {\tt
  arXiv:1404.4250 [cs.DC]}

\bibitem[Mun]{Mun} J.R.\ Munkres, {\em Topology: a first course},
  Prentice-Hall, Inc., Englewood Cliffs, N.J., 1975, xvi+413 pp.

\end{thebibliography}

\end{document}
